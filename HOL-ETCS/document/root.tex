\documentclass[11pt,a4paper]{article}
\usepackage[T1]{fontenc}
\usepackage{amsmath, amssymb}
\usepackage{isabelle,isabellesym}

% this should be the last package used
\usepackage{pdfsetup}

% urls in roman style, theory text in math-similar italics
\urlstyle{rm}
\isabellestyle{it}

\newcommand\sslash{\mathbin{/\mkern-5.5mu/}}%


\begin{document}

\title{The Elementary Theory of the Category of Sets}
\author{James Baxter \and Dustin Bryant}
\maketitle

\begin{abstract}
  Category theory presents a formulation of mathematical structures in
  terms of common properties of those structures. 
  A particular formulation of interest is the Elementary Theory of the
  Category of Sets (ETCS), which is an axiomatization of set theory in
  category theory terms. 
  This axiomatization provides an unusual view of sets, where the
  functions between sets are regarded as more important than the
  elements of the sets. 
  We formalise an axiomatization of ETCS on top of HOL, following the
  presentation given by Halvorson~\cite{Halvorson2019}. 
  We also build some other set theoretic results on top of the
  axiomatization, including Cantor's diagonalization theorem and
  mathematical induction. 
  We additionally define a system of quantified predicate logic within the ETCS
  axiomatization.
\end{abstract}

\tableofcontents

% include generated text of all theories
%
\begin{isabellebody}%
\setisabellecontext{Cfunc}%
%
\isadelimdocument
%
\endisadelimdocument
%
\isatagdocument
%
\isamarkupsection{Basic Types and Operators for the Category of Sets%
}
\isamarkuptrue%
%
\endisatagdocument
{\isafolddocument}%
%
\isadelimdocument
%
\endisadelimdocument
%
\isadelimtheory
%
\endisadelimtheory
%
\isatagtheory
\isacommand{theory}\isamarkupfalse%
\ Cfunc\isanewline
\ \ \isakeyword{imports}\ Main\ {\isachardoublequoteopen}HOL{\isacharminus}{\kern0pt}Eisbach{\isachardot}{\kern0pt}Eisbach{\isachardoublequoteclose}\isanewline
\isakeyword{begin}%
\endisatagtheory
{\isafoldtheory}%
%
\isadelimtheory
\isanewline
%
\endisadelimtheory
\isanewline
\isacommand{typedecl}\isamarkupfalse%
\ cset\isanewline
\isacommand{typedecl}\isamarkupfalse%
\ cfunc%
\begin{isamarkuptext}%
We declare \isa{cset} and \isa{cfunc} as types to represent the sets and functions within
  ETCS, as distinct from HOL sets and functions.
  The "c" prefix here is intended to stand for "category", and emphasises that these are
  category-theoretic objects.%
\end{isamarkuptext}\isamarkuptrue%
%
\begin{isamarkuptext}%
The axiomatization below corresponds to Axiom 1 (Sets Is a Category) in Halvorson.%
\end{isamarkuptext}\isamarkuptrue%
\isacommand{axiomatization}\isamarkupfalse%
\isanewline
\ \ domain\ {\isacharcolon}{\kern0pt}{\isacharcolon}{\kern0pt}\ {\isachardoublequoteopen}cfunc\ {\isasymRightarrow}\ cset{\isachardoublequoteclose}\ \isakeyword{and}\isanewline
\ \ codomain\ {\isacharcolon}{\kern0pt}{\isacharcolon}{\kern0pt}\ {\isachardoublequoteopen}cfunc\ {\isasymRightarrow}\ cset{\isachardoublequoteclose}\ \isakeyword{and}\isanewline
\ \ comp\ {\isacharcolon}{\kern0pt}{\isacharcolon}{\kern0pt}\ {\isachardoublequoteopen}cfunc\ {\isasymRightarrow}\ cfunc\ {\isasymRightarrow}\ cfunc{\isachardoublequoteclose}\ {\isacharparenleft}{\kern0pt}\isakeyword{infixr}\ {\isachardoublequoteopen}{\isasymcirc}\isactrlsub c{\isachardoublequoteclose}\ {\isadigit{5}}{\isadigit{5}}{\isacharparenright}{\kern0pt}\ \isakeyword{and}\isanewline
\ \ id\ {\isacharcolon}{\kern0pt}{\isacharcolon}{\kern0pt}\ {\isachardoublequoteopen}cset\ {\isasymRightarrow}\ cfunc{\isachardoublequoteclose}\ {\isacharparenleft}{\kern0pt}{\isachardoublequoteopen}id\isactrlsub c{\isachardoublequoteclose}{\isacharparenright}{\kern0pt}\ \isanewline
\isakeyword{where}\isanewline
\ \ domain{\isacharunderscore}{\kern0pt}comp{\isacharcolon}{\kern0pt}\ {\isachardoublequoteopen}domain\ g\ {\isacharequal}{\kern0pt}\ codomain\ f\ {\isasymLongrightarrow}\ domain\ {\isacharparenleft}{\kern0pt}g\ {\isasymcirc}\isactrlsub c\ f{\isacharparenright}{\kern0pt}\ {\isacharequal}{\kern0pt}\ domain\ f{\isachardoublequoteclose}\ \isakeyword{and}\isanewline
\ \ codomain{\isacharunderscore}{\kern0pt}comp{\isacharcolon}{\kern0pt}\ {\isachardoublequoteopen}domain\ g\ {\isacharequal}{\kern0pt}\ codomain\ f\ {\isasymLongrightarrow}\ codomain\ {\isacharparenleft}{\kern0pt}g\ {\isasymcirc}\isactrlsub c\ f{\isacharparenright}{\kern0pt}\ {\isacharequal}{\kern0pt}\ codomain\ g{\isachardoublequoteclose}\ \isakeyword{and}\isanewline
\ \ comp{\isacharunderscore}{\kern0pt}associative{\isacharcolon}{\kern0pt}\ {\isachardoublequoteopen}domain\ h\ {\isacharequal}{\kern0pt}\ codomain\ g\ {\isasymLongrightarrow}\ domain\ g\ {\isacharequal}{\kern0pt}\ codomain\ f\ {\isasymLongrightarrow}\ h\ {\isasymcirc}\isactrlsub c\ {\isacharparenleft}{\kern0pt}g\ {\isasymcirc}\isactrlsub c\ f{\isacharparenright}{\kern0pt}\ {\isacharequal}{\kern0pt}\ {\isacharparenleft}{\kern0pt}h\ {\isasymcirc}\isactrlsub c\ g{\isacharparenright}{\kern0pt}\ {\isasymcirc}\isactrlsub c\ f{\isachardoublequoteclose}\ \isakeyword{and}\isanewline
\ \ id{\isacharunderscore}{\kern0pt}domain{\isacharcolon}{\kern0pt}\ {\isachardoublequoteopen}domain\ {\isacharparenleft}{\kern0pt}id\ X{\isacharparenright}{\kern0pt}\ {\isacharequal}{\kern0pt}\ X{\isachardoublequoteclose}\ \isakeyword{and}\isanewline
\ \ id{\isacharunderscore}{\kern0pt}codomain{\isacharcolon}{\kern0pt}\ {\isachardoublequoteopen}codomain\ {\isacharparenleft}{\kern0pt}id\ X{\isacharparenright}{\kern0pt}\ {\isacharequal}{\kern0pt}\ X{\isachardoublequoteclose}\ \isakeyword{and}\isanewline
\ \ id{\isacharunderscore}{\kern0pt}right{\isacharunderscore}{\kern0pt}unit{\isacharcolon}{\kern0pt}\ {\isachardoublequoteopen}f\ {\isasymcirc}\isactrlsub c\ id\ {\isacharparenleft}{\kern0pt}domain\ f{\isacharparenright}{\kern0pt}\ {\isacharequal}{\kern0pt}\ f{\isachardoublequoteclose}\ \isakeyword{and}\isanewline
\ \ id{\isacharunderscore}{\kern0pt}left{\isacharunderscore}{\kern0pt}unit{\isacharcolon}{\kern0pt}\ {\isachardoublequoteopen}id\ {\isacharparenleft}{\kern0pt}codomain\ f{\isacharparenright}{\kern0pt}\ {\isasymcirc}\isactrlsub c\ f\ {\isacharequal}{\kern0pt}\ f{\isachardoublequoteclose}%
\begin{isamarkuptext}%
We define a neater way of stating types and lift the type axioms into lemmas using it.%
\end{isamarkuptext}\isamarkuptrue%
\isacommand{definition}\isamarkupfalse%
\ cfunc{\isacharunderscore}{\kern0pt}type\ {\isacharcolon}{\kern0pt}{\isacharcolon}{\kern0pt}\ {\isachardoublequoteopen}cfunc\ {\isasymRightarrow}\ cset\ {\isasymRightarrow}\ cset\ {\isasymRightarrow}\ bool{\isachardoublequoteclose}\ {\isacharparenleft}{\kern0pt}{\isachardoublequoteopen}{\isacharunderscore}{\kern0pt}\ {\isacharcolon}{\kern0pt}\ {\isacharunderscore}{\kern0pt}\ {\isasymrightarrow}\ {\isacharunderscore}{\kern0pt}{\isachardoublequoteclose}\ {\isacharbrackleft}{\kern0pt}{\isadigit{5}}{\isadigit{0}}{\isacharcomma}{\kern0pt}\ {\isadigit{5}}{\isadigit{0}}{\isacharcomma}{\kern0pt}\ {\isadigit{5}}{\isadigit{0}}{\isacharbrackright}{\kern0pt}{\isadigit{5}}{\isadigit{0}}{\isacharparenright}{\kern0pt}\ \isakeyword{where}\isanewline
\ \ {\isachardoublequoteopen}{\isacharparenleft}{\kern0pt}f\ {\isacharcolon}{\kern0pt}\ X\ {\isasymrightarrow}\ Y{\isacharparenright}{\kern0pt}\ {\isasymlongleftrightarrow}\ {\isacharparenleft}{\kern0pt}domain\ f\ {\isacharequal}{\kern0pt}\ X\ {\isasymand}\ codomain\ f\ {\isacharequal}{\kern0pt}\ Y{\isacharparenright}{\kern0pt}{\isachardoublequoteclose}\isanewline
\isanewline
\isacommand{lemma}\isamarkupfalse%
\ comp{\isacharunderscore}{\kern0pt}type{\isacharcolon}{\kern0pt}\isanewline
\ \ {\isachardoublequoteopen}f\ {\isacharcolon}{\kern0pt}\ X\ {\isasymrightarrow}\ Y\ {\isasymLongrightarrow}\ g\ {\isacharcolon}{\kern0pt}\ Y\ {\isasymrightarrow}\ Z\ {\isasymLongrightarrow}\ g\ {\isasymcirc}\isactrlsub c\ f\ {\isacharcolon}{\kern0pt}\ X\ {\isasymrightarrow}\ Z{\isachardoublequoteclose}\isanewline
%
\isadelimproof
\ \ %
\endisadelimproof
%
\isatagproof
\isacommand{by}\isamarkupfalse%
\ {\isacharparenleft}{\kern0pt}simp\ add{\isacharcolon}{\kern0pt}\ cfunc{\isacharunderscore}{\kern0pt}type{\isacharunderscore}{\kern0pt}def\ codomain{\isacharunderscore}{\kern0pt}comp\ domain{\isacharunderscore}{\kern0pt}comp{\isacharparenright}{\kern0pt}%
\endisatagproof
{\isafoldproof}%
%
\isadelimproof
\isanewline
%
\endisadelimproof
\isanewline
\isacommand{lemma}\isamarkupfalse%
\ comp{\isacharunderscore}{\kern0pt}associative{\isadigit{2}}{\isacharcolon}{\kern0pt}\isanewline
\ \ {\isachardoublequoteopen}f\ {\isacharcolon}{\kern0pt}\ X\ {\isasymrightarrow}\ Y\ {\isasymLongrightarrow}\ g\ {\isacharcolon}{\kern0pt}\ Y\ {\isasymrightarrow}\ Z\ {\isasymLongrightarrow}\ h\ {\isacharcolon}{\kern0pt}\ Z\ {\isasymrightarrow}\ W\ {\isasymLongrightarrow}\ h\ {\isasymcirc}\isactrlsub c\ {\isacharparenleft}{\kern0pt}g\ {\isasymcirc}\isactrlsub c\ f{\isacharparenright}{\kern0pt}\ {\isacharequal}{\kern0pt}\ {\isacharparenleft}{\kern0pt}h\ {\isasymcirc}\isactrlsub c\ g{\isacharparenright}{\kern0pt}\ {\isasymcirc}\isactrlsub c\ f{\isachardoublequoteclose}\isanewline
%
\isadelimproof
\ \ %
\endisadelimproof
%
\isatagproof
\isacommand{by}\isamarkupfalse%
\ {\isacharparenleft}{\kern0pt}simp\ add{\isacharcolon}{\kern0pt}\ cfunc{\isacharunderscore}{\kern0pt}type{\isacharunderscore}{\kern0pt}def\ comp{\isacharunderscore}{\kern0pt}associative{\isacharparenright}{\kern0pt}%
\endisatagproof
{\isafoldproof}%
%
\isadelimproof
\isanewline
%
\endisadelimproof
\isanewline
\isacommand{lemma}\isamarkupfalse%
\ id{\isacharunderscore}{\kern0pt}type{\isacharcolon}{\kern0pt}\ {\isachardoublequoteopen}id\ X\ {\isacharcolon}{\kern0pt}\ X\ {\isasymrightarrow}\ X{\isachardoublequoteclose}\isanewline
%
\isadelimproof
\ \ %
\endisadelimproof
%
\isatagproof
\isacommand{unfolding}\isamarkupfalse%
\ cfunc{\isacharunderscore}{\kern0pt}type{\isacharunderscore}{\kern0pt}def\ \isacommand{using}\isamarkupfalse%
\ id{\isacharunderscore}{\kern0pt}domain\ id{\isacharunderscore}{\kern0pt}codomain\ \isacommand{by}\isamarkupfalse%
\ auto%
\endisatagproof
{\isafoldproof}%
%
\isadelimproof
\isanewline
%
\endisadelimproof
\isanewline
\isacommand{lemma}\isamarkupfalse%
\ id{\isacharunderscore}{\kern0pt}right{\isacharunderscore}{\kern0pt}unit{\isadigit{2}}{\isacharcolon}{\kern0pt}\ {\isachardoublequoteopen}f\ {\isacharcolon}{\kern0pt}\ X\ {\isasymrightarrow}\ Y\ {\isasymLongrightarrow}\ f\ {\isasymcirc}\isactrlsub c\ id\ X\ {\isacharequal}{\kern0pt}\ f{\isachardoublequoteclose}\isanewline
%
\isadelimproof
\ \ %
\endisadelimproof
%
\isatagproof
\isacommand{unfolding}\isamarkupfalse%
\ cfunc{\isacharunderscore}{\kern0pt}type{\isacharunderscore}{\kern0pt}def\ \isacommand{using}\isamarkupfalse%
\ id{\isacharunderscore}{\kern0pt}right{\isacharunderscore}{\kern0pt}unit\ \isacommand{by}\isamarkupfalse%
\ auto%
\endisatagproof
{\isafoldproof}%
%
\isadelimproof
\isanewline
%
\endisadelimproof
\isanewline
\isacommand{lemma}\isamarkupfalse%
\ id{\isacharunderscore}{\kern0pt}left{\isacharunderscore}{\kern0pt}unit{\isadigit{2}}{\isacharcolon}{\kern0pt}\ {\isachardoublequoteopen}f\ {\isacharcolon}{\kern0pt}\ X\ {\isasymrightarrow}\ Y\ {\isasymLongrightarrow}\ id\ Y\ {\isasymcirc}\isactrlsub c\ f\ {\isacharequal}{\kern0pt}\ f{\isachardoublequoteclose}\isanewline
%
\isadelimproof
\ \ %
\endisadelimproof
%
\isatagproof
\isacommand{unfolding}\isamarkupfalse%
\ cfunc{\isacharunderscore}{\kern0pt}type{\isacharunderscore}{\kern0pt}def\ \isacommand{using}\isamarkupfalse%
\ id{\isacharunderscore}{\kern0pt}left{\isacharunderscore}{\kern0pt}unit\ \isacommand{by}\isamarkupfalse%
\ auto%
\endisatagproof
{\isafoldproof}%
%
\isadelimproof
%
\endisadelimproof
%
\isadelimdocument
%
\endisadelimdocument
%
\isatagdocument
%
\isamarkupsubsection{Tactics for Applying Typing Rules%
}
\isamarkuptrue%
%
\endisatagdocument
{\isafolddocument}%
%
\isadelimdocument
%
\endisadelimdocument
%
\begin{isamarkuptext}%
ETCS lemmas often have assumptions on its ETCS type, which can often be cumbersome to prove.
  To simplify proofs involving ETCS types, we provide proof methods that apply type rules in a
  structured way to prove facts about ETCS function types.
  The type rules state the types of the basic constants and operators of ETCS and are declared as
  a named set of theorems called $type\_rule$.%
\end{isamarkuptext}\isamarkuptrue%
\isacommand{named{\isacharunderscore}{\kern0pt}theorems}\isamarkupfalse%
\ type{\isacharunderscore}{\kern0pt}rule\isanewline
\isanewline
\isacommand{declare}\isamarkupfalse%
\ id{\isacharunderscore}{\kern0pt}type{\isacharbrackleft}{\kern0pt}type{\isacharunderscore}{\kern0pt}rule{\isacharbrackright}{\kern0pt}\isanewline
\isacommand{declare}\isamarkupfalse%
\ comp{\isacharunderscore}{\kern0pt}type{\isacharbrackleft}{\kern0pt}type{\isacharunderscore}{\kern0pt}rule{\isacharbrackright}{\kern0pt}\isanewline
%
\isadelimML
\isanewline
%
\endisadelimML
%
\isatagML
\isacommand{ML{\isacharunderscore}{\kern0pt}file}\isamarkupfalse%
\ {\isacartoucheopen}typecheck{\isachardot}{\kern0pt}ml{\isacartoucheclose}%
\endisatagML
{\isafoldML}%
%
\isadelimML
%
\endisadelimML
%
\isadelimdocument
%
\endisadelimdocument
%
\isatagdocument
%
\isamarkupsubsubsection{typecheck\_cfuncs: Tactic to Construct Type Facts%
}
\isamarkuptrue%
%
\endisatagdocument
{\isafolddocument}%
%
\isadelimdocument
%
\endisadelimdocument
%
\isadelimML
%
\endisadelimML
%
\isatagML
\isacommand{method{\isacharunderscore}{\kern0pt}setup}\isamarkupfalse%
\ typecheck{\isacharunderscore}{\kern0pt}cfuncs\ {\isacharequal}{\kern0pt}\isanewline
\ \ {\isacartoucheopen}Scan{\isachardot}{\kern0pt}option\ {\isacharparenleft}{\kern0pt}{\isacharparenleft}{\kern0pt}Scan{\isachardot}{\kern0pt}lift\ {\isacharparenleft}{\kern0pt}Args{\isachardot}{\kern0pt}{\isachardollar}{\kern0pt}{\isachardollar}{\kern0pt}{\isachardollar}{\kern0pt}\ {\isachardoublequote}{\kern0pt}type{\isacharunderscore}{\kern0pt}rule{\isachardoublequote}{\kern0pt}\ {\isacharminus}{\kern0pt}{\isacharminus}{\kern0pt}\ Args{\isachardot}{\kern0pt}colon{\isacharparenright}{\kern0pt}{\isacharparenright}{\kern0pt}\ {\isacharbar}{\kern0pt}{\isacharminus}{\kern0pt}{\isacharminus}{\kern0pt}\ Attrib{\isachardot}{\kern0pt}thms{\isacharparenright}{\kern0pt}\isanewline
\ \ \ \ \ {\isachargreater}{\kern0pt}{\isachargreater}{\kern0pt}\ typecheck{\isacharunderscore}{\kern0pt}cfuncs{\isacharunderscore}{\kern0pt}method{\isacartoucheclose}\isanewline
\ \ {\isachardoublequoteopen}Check\ types\ of\ cfuncs\ in\ current\ goal\ and\ add\ as\ assumptions\ of\ the\ current\ goal{\isachardoublequoteclose}\isanewline
\isanewline
\isacommand{method{\isacharunderscore}{\kern0pt}setup}\isamarkupfalse%
\ typecheck{\isacharunderscore}{\kern0pt}cfuncs{\isacharunderscore}{\kern0pt}all\ {\isacharequal}{\kern0pt}\isanewline
\ \ {\isacartoucheopen}Scan{\isachardot}{\kern0pt}option\ {\isacharparenleft}{\kern0pt}{\isacharparenleft}{\kern0pt}Scan{\isachardot}{\kern0pt}lift\ {\isacharparenleft}{\kern0pt}Args{\isachardot}{\kern0pt}{\isachardollar}{\kern0pt}{\isachardollar}{\kern0pt}{\isachardollar}{\kern0pt}\ {\isachardoublequote}{\kern0pt}type{\isacharunderscore}{\kern0pt}rule{\isachardoublequote}{\kern0pt}\ {\isacharminus}{\kern0pt}{\isacharminus}{\kern0pt}\ Args{\isachardot}{\kern0pt}colon{\isacharparenright}{\kern0pt}{\isacharparenright}{\kern0pt}\ {\isacharbar}{\kern0pt}{\isacharminus}{\kern0pt}{\isacharminus}{\kern0pt}\ Attrib{\isachardot}{\kern0pt}thms{\isacharparenright}{\kern0pt}\isanewline
\ \ \ \ \ {\isachargreater}{\kern0pt}{\isachargreater}{\kern0pt}\ typecheck{\isacharunderscore}{\kern0pt}cfuncs{\isacharunderscore}{\kern0pt}all{\isacharunderscore}{\kern0pt}method{\isacartoucheclose}\isanewline
\ \ {\isachardoublequoteopen}Check\ types\ of\ cfuncs\ in\ all\ subgoals\ and\ add\ as\ assumptions\ of\ the\ current\ goal{\isachardoublequoteclose}\isanewline
\isanewline
\isacommand{method{\isacharunderscore}{\kern0pt}setup}\isamarkupfalse%
\ typecheck{\isacharunderscore}{\kern0pt}cfuncs{\isacharunderscore}{\kern0pt}prems\ {\isacharequal}{\kern0pt}\isanewline
\ \ {\isacartoucheopen}Scan{\isachardot}{\kern0pt}option\ {\isacharparenleft}{\kern0pt}{\isacharparenleft}{\kern0pt}Scan{\isachardot}{\kern0pt}lift\ {\isacharparenleft}{\kern0pt}Args{\isachardot}{\kern0pt}{\isachardollar}{\kern0pt}{\isachardollar}{\kern0pt}{\isachardollar}{\kern0pt}\ {\isachardoublequote}{\kern0pt}type{\isacharunderscore}{\kern0pt}rule{\isachardoublequote}{\kern0pt}\ {\isacharminus}{\kern0pt}{\isacharminus}{\kern0pt}\ Args{\isachardot}{\kern0pt}colon{\isacharparenright}{\kern0pt}{\isacharparenright}{\kern0pt}\ {\isacharbar}{\kern0pt}{\isacharminus}{\kern0pt}{\isacharminus}{\kern0pt}\ Attrib{\isachardot}{\kern0pt}thms{\isacharparenright}{\kern0pt}\isanewline
\ \ \ \ \ {\isachargreater}{\kern0pt}{\isachargreater}{\kern0pt}\ typecheck{\isacharunderscore}{\kern0pt}cfuncs{\isacharunderscore}{\kern0pt}prems{\isacharunderscore}{\kern0pt}method{\isacartoucheclose}\isanewline
\ \ {\isachardoublequoteopen}Check\ types\ of\ cfuncs\ in\ assumptions\ of\ the\ current\ goal\ and\ add\ as\ assumptions\ of\ the\ current\ goal{\isachardoublequoteclose}%
\endisatagML
{\isafoldML}%
%
\isadelimML
%
\endisadelimML
%
\isadelimdocument
%
\endisadelimdocument
%
\isatagdocument
%
\isamarkupsubsubsection{etcs\_rule: Tactic to Apply Rules with ETCS Typechecking%
}
\isamarkuptrue%
%
\endisatagdocument
{\isafolddocument}%
%
\isadelimdocument
%
\endisadelimdocument
%
\isadelimML
%
\endisadelimML
%
\isatagML
\isacommand{method{\isacharunderscore}{\kern0pt}setup}\isamarkupfalse%
\ etcs{\isacharunderscore}{\kern0pt}rule\ {\isacharequal}{\kern0pt}\ \isanewline
\ \ {\isacartoucheopen}Scan{\isachardot}{\kern0pt}repeats\ {\isacharparenleft}{\kern0pt}Scan{\isachardot}{\kern0pt}unless\ {\isacharparenleft}{\kern0pt}Scan{\isachardot}{\kern0pt}lift\ {\isacharparenleft}{\kern0pt}Args{\isachardot}{\kern0pt}{\isachardollar}{\kern0pt}{\isachardollar}{\kern0pt}{\isachardollar}{\kern0pt}\ {\isachardoublequote}{\kern0pt}type{\isacharunderscore}{\kern0pt}rule{\isachardoublequote}{\kern0pt}\ {\isacharminus}{\kern0pt}{\isacharminus}{\kern0pt}\ Args{\isachardot}{\kern0pt}colon{\isacharparenright}{\kern0pt}{\isacharparenright}{\kern0pt}\ Attrib{\isachardot}{\kern0pt}multi{\isacharunderscore}{\kern0pt}thm{\isacharparenright}{\kern0pt}\isanewline
\ \ \ \ {\isacharminus}{\kern0pt}{\isacharminus}{\kern0pt}\ Scan{\isachardot}{\kern0pt}option\ {\isacharparenleft}{\kern0pt}{\isacharparenleft}{\kern0pt}Scan{\isachardot}{\kern0pt}lift\ {\isacharparenleft}{\kern0pt}Args{\isachardot}{\kern0pt}{\isachardollar}{\kern0pt}{\isachardollar}{\kern0pt}{\isachardollar}{\kern0pt}\ {\isachardoublequote}{\kern0pt}type{\isacharunderscore}{\kern0pt}rule{\isachardoublequote}{\kern0pt}\ {\isacharminus}{\kern0pt}{\isacharminus}{\kern0pt}\ Args{\isachardot}{\kern0pt}colon{\isacharparenright}{\kern0pt}{\isacharparenright}{\kern0pt}\ {\isacharbar}{\kern0pt}{\isacharminus}{\kern0pt}{\isacharminus}{\kern0pt}\ Attrib{\isachardot}{\kern0pt}thms{\isacharparenright}{\kern0pt}\isanewline
\ \ \ \ \ {\isachargreater}{\kern0pt}{\isachargreater}{\kern0pt}\ ETCS{\isacharunderscore}{\kern0pt}resolve{\isacharunderscore}{\kern0pt}method{\isacartoucheclose}\isanewline
\ \ {\isachardoublequoteopen}apply\ rule\ with\ ETCS\ type\ checking{\isachardoublequoteclose}%
\endisatagML
{\isafoldML}%
%
\isadelimML
%
\endisadelimML
%
\isadelimdocument
%
\endisadelimdocument
%
\isatagdocument
%
\isamarkupsubsubsection{etcs\_subst: Tactic to Apply Substitutions with ETCS Typechecking%
}
\isamarkuptrue%
%
\endisatagdocument
{\isafolddocument}%
%
\isadelimdocument
%
\endisadelimdocument
%
\isadelimML
%
\endisadelimML
%
\isatagML
\isacommand{method{\isacharunderscore}{\kern0pt}setup}\isamarkupfalse%
\ etcs{\isacharunderscore}{\kern0pt}subst\ {\isacharequal}{\kern0pt}\ \isanewline
\ \ {\isacartoucheopen}Scan{\isachardot}{\kern0pt}repeats\ {\isacharparenleft}{\kern0pt}Scan{\isachardot}{\kern0pt}unless\ {\isacharparenleft}{\kern0pt}Scan{\isachardot}{\kern0pt}lift\ {\isacharparenleft}{\kern0pt}Args{\isachardot}{\kern0pt}{\isachardollar}{\kern0pt}{\isachardollar}{\kern0pt}{\isachardollar}{\kern0pt}\ {\isachardoublequote}{\kern0pt}type{\isacharunderscore}{\kern0pt}rule{\isachardoublequote}{\kern0pt}\ {\isacharminus}{\kern0pt}{\isacharminus}{\kern0pt}\ Args{\isachardot}{\kern0pt}colon{\isacharparenright}{\kern0pt}{\isacharparenright}{\kern0pt}\ Attrib{\isachardot}{\kern0pt}multi{\isacharunderscore}{\kern0pt}thm{\isacharparenright}{\kern0pt}\isanewline
\ \ \ \ {\isacharminus}{\kern0pt}{\isacharminus}{\kern0pt}\ Scan{\isachardot}{\kern0pt}option\ {\isacharparenleft}{\kern0pt}{\isacharparenleft}{\kern0pt}Scan{\isachardot}{\kern0pt}lift\ {\isacharparenleft}{\kern0pt}Args{\isachardot}{\kern0pt}{\isachardollar}{\kern0pt}{\isachardollar}{\kern0pt}{\isachardollar}{\kern0pt}\ {\isachardoublequote}{\kern0pt}type{\isacharunderscore}{\kern0pt}rule{\isachardoublequote}{\kern0pt}\ {\isacharminus}{\kern0pt}{\isacharminus}{\kern0pt}\ Args{\isachardot}{\kern0pt}colon{\isacharparenright}{\kern0pt}{\isacharparenright}{\kern0pt}\ {\isacharbar}{\kern0pt}{\isacharminus}{\kern0pt}{\isacharminus}{\kern0pt}\ Attrib{\isachardot}{\kern0pt}thms{\isacharparenright}{\kern0pt}\isanewline
\ \ \ \ \ {\isachargreater}{\kern0pt}{\isachargreater}{\kern0pt}\ ETCS{\isacharunderscore}{\kern0pt}subst{\isacharunderscore}{\kern0pt}method{\isacartoucheclose}\ \isanewline
\ \ {\isachardoublequoteopen}apply\ substitution\ with\ ETCS\ type\ checking{\isachardoublequoteclose}%
\endisatagML
{\isafoldML}%
%
\isadelimML
\isanewline
%
\endisadelimML
\isanewline
\isacommand{method}\isamarkupfalse%
\ etcs{\isacharunderscore}{\kern0pt}assocl\ \isakeyword{declares}\ type{\isacharunderscore}{\kern0pt}rule\ {\isacharequal}{\kern0pt}\ {\isacharparenleft}{\kern0pt}etcs{\isacharunderscore}{\kern0pt}subst\ comp{\isacharunderscore}{\kern0pt}associative{\isadigit{2}}{\isacharparenright}{\kern0pt}{\isacharplus}{\kern0pt}\isanewline
\isacommand{method}\isamarkupfalse%
\ etcs{\isacharunderscore}{\kern0pt}assocr\ \isakeyword{declares}\ type{\isacharunderscore}{\kern0pt}rule\ {\isacharequal}{\kern0pt}\ {\isacharparenleft}{\kern0pt}etcs{\isacharunderscore}{\kern0pt}subst\ sym{\isacharbrackleft}{\kern0pt}OF\ comp{\isacharunderscore}{\kern0pt}associative{\isadigit{2}}{\isacharbrackright}{\kern0pt}{\isacharparenright}{\kern0pt}{\isacharplus}{\kern0pt}\isanewline
%
\isadelimML
\isanewline
%
\endisadelimML
%
\isatagML
\isacommand{method{\isacharunderscore}{\kern0pt}setup}\isamarkupfalse%
\ etcs{\isacharunderscore}{\kern0pt}subst{\isacharunderscore}{\kern0pt}asm\ {\isacharequal}{\kern0pt}\ \isanewline
\ \ {\isacartoucheopen}Runtime{\isachardot}{\kern0pt}exn{\isacharunderscore}{\kern0pt}trace\ {\isacharparenleft}{\kern0pt}fn\ {\isacharunderscore}{\kern0pt}\ {\isacharequal}{\kern0pt}{\isachargreater}{\kern0pt}\ Scan{\isachardot}{\kern0pt}repeats\ {\isacharparenleft}{\kern0pt}Scan{\isachardot}{\kern0pt}unless\ {\isacharparenleft}{\kern0pt}Scan{\isachardot}{\kern0pt}lift\ {\isacharparenleft}{\kern0pt}Args{\isachardot}{\kern0pt}{\isachardollar}{\kern0pt}{\isachardollar}{\kern0pt}{\isachardollar}{\kern0pt}\ {\isachardoublequote}{\kern0pt}type{\isacharunderscore}{\kern0pt}rule{\isachardoublequote}{\kern0pt}\ {\isacharminus}{\kern0pt}{\isacharminus}{\kern0pt}\ Args{\isachardot}{\kern0pt}colon{\isacharparenright}{\kern0pt}{\isacharparenright}{\kern0pt}\ Attrib{\isachardot}{\kern0pt}multi{\isacharunderscore}{\kern0pt}thm{\isacharparenright}{\kern0pt}\isanewline
\ \ \ \ {\isacharminus}{\kern0pt}{\isacharminus}{\kern0pt}\ Scan{\isachardot}{\kern0pt}option\ {\isacharparenleft}{\kern0pt}{\isacharparenleft}{\kern0pt}Scan{\isachardot}{\kern0pt}lift\ {\isacharparenleft}{\kern0pt}Args{\isachardot}{\kern0pt}{\isachardollar}{\kern0pt}{\isachardollar}{\kern0pt}{\isachardollar}{\kern0pt}\ {\isachardoublequote}{\kern0pt}type{\isacharunderscore}{\kern0pt}rule{\isachardoublequote}{\kern0pt}\ {\isacharminus}{\kern0pt}{\isacharminus}{\kern0pt}\ Args{\isachardot}{\kern0pt}colon{\isacharparenright}{\kern0pt}{\isacharparenright}{\kern0pt}\ {\isacharbar}{\kern0pt}{\isacharminus}{\kern0pt}{\isacharminus}{\kern0pt}\ Attrib{\isachardot}{\kern0pt}thms{\isacharparenright}{\kern0pt}\isanewline
\ \ \ \ \ {\isachargreater}{\kern0pt}{\isachargreater}{\kern0pt}\ ETCS{\isacharunderscore}{\kern0pt}subst{\isacharunderscore}{\kern0pt}asm{\isacharunderscore}{\kern0pt}method{\isacharparenright}{\kern0pt}{\isacartoucheclose}\ \isanewline
\ \ {\isachardoublequoteopen}apply\ substitution\ to\ assumptions\ of\ the\ goal{\isacharcomma}{\kern0pt}\ with\ ETCS\ type\ checking{\isachardoublequoteclose}%
\endisatagML
{\isafoldML}%
%
\isadelimML
\isanewline
%
\endisadelimML
\isanewline
\isacommand{method}\isamarkupfalse%
\ etcs{\isacharunderscore}{\kern0pt}assocl{\isacharunderscore}{\kern0pt}asm\ \isakeyword{declares}\ type{\isacharunderscore}{\kern0pt}rule\ {\isacharequal}{\kern0pt}\ {\isacharparenleft}{\kern0pt}etcs{\isacharunderscore}{\kern0pt}subst{\isacharunderscore}{\kern0pt}asm\ comp{\isacharunderscore}{\kern0pt}associative{\isadigit{2}}{\isacharparenright}{\kern0pt}{\isacharplus}{\kern0pt}\isanewline
\isacommand{method}\isamarkupfalse%
\ etcs{\isacharunderscore}{\kern0pt}assocr{\isacharunderscore}{\kern0pt}asm\ \isakeyword{declares}\ type{\isacharunderscore}{\kern0pt}rule\ {\isacharequal}{\kern0pt}\ {\isacharparenleft}{\kern0pt}etcs{\isacharunderscore}{\kern0pt}subst{\isacharunderscore}{\kern0pt}asm\ sym{\isacharbrackleft}{\kern0pt}OF\ comp{\isacharunderscore}{\kern0pt}associative{\isadigit{2}}{\isacharbrackright}{\kern0pt}{\isacharparenright}{\kern0pt}{\isacharplus}{\kern0pt}%
\isadelimdocument
%
\endisadelimdocument
%
\isatagdocument
%
\isamarkupsubsubsection{etcs\_erule: Tactic to Apply Elimination Rules with ETCS Typechecking%
}
\isamarkuptrue%
%
\endisatagdocument
{\isafolddocument}%
%
\isadelimdocument
%
\endisadelimdocument
%
\isadelimML
%
\endisadelimML
%
\isatagML
\isacommand{method{\isacharunderscore}{\kern0pt}setup}\isamarkupfalse%
\ etcs{\isacharunderscore}{\kern0pt}erule\ {\isacharequal}{\kern0pt}\ \isanewline
\ \ {\isacartoucheopen}Scan{\isachardot}{\kern0pt}repeats\ {\isacharparenleft}{\kern0pt}Scan{\isachardot}{\kern0pt}unless\ {\isacharparenleft}{\kern0pt}Scan{\isachardot}{\kern0pt}lift\ {\isacharparenleft}{\kern0pt}Args{\isachardot}{\kern0pt}{\isachardollar}{\kern0pt}{\isachardollar}{\kern0pt}{\isachardollar}{\kern0pt}\ {\isachardoublequote}{\kern0pt}type{\isacharunderscore}{\kern0pt}rule{\isachardoublequote}{\kern0pt}\ {\isacharminus}{\kern0pt}{\isacharminus}{\kern0pt}\ Args{\isachardot}{\kern0pt}colon{\isacharparenright}{\kern0pt}{\isacharparenright}{\kern0pt}\ Attrib{\isachardot}{\kern0pt}multi{\isacharunderscore}{\kern0pt}thm{\isacharparenright}{\kern0pt}\isanewline
\ \ \ \ {\isacharminus}{\kern0pt}{\isacharminus}{\kern0pt}\ Scan{\isachardot}{\kern0pt}option\ {\isacharparenleft}{\kern0pt}{\isacharparenleft}{\kern0pt}Scan{\isachardot}{\kern0pt}lift\ {\isacharparenleft}{\kern0pt}Args{\isachardot}{\kern0pt}{\isachardollar}{\kern0pt}{\isachardollar}{\kern0pt}{\isachardollar}{\kern0pt}\ {\isachardoublequote}{\kern0pt}type{\isacharunderscore}{\kern0pt}rule{\isachardoublequote}{\kern0pt}\ {\isacharminus}{\kern0pt}{\isacharminus}{\kern0pt}\ Args{\isachardot}{\kern0pt}colon{\isacharparenright}{\kern0pt}{\isacharparenright}{\kern0pt}\ {\isacharbar}{\kern0pt}{\isacharminus}{\kern0pt}{\isacharminus}{\kern0pt}\ Attrib{\isachardot}{\kern0pt}thms{\isacharparenright}{\kern0pt}\isanewline
\ \ \ \ \ {\isachargreater}{\kern0pt}{\isachargreater}{\kern0pt}\ ETCS{\isacharunderscore}{\kern0pt}eresolve{\isacharunderscore}{\kern0pt}method{\isacartoucheclose}\isanewline
\ \ {\isachardoublequoteopen}apply\ erule\ with\ ETCS\ type\ checking{\isachardoublequoteclose}%
\endisatagML
{\isafoldML}%
%
\isadelimML
%
\endisadelimML
%
\isadelimdocument
%
\endisadelimdocument
%
\isatagdocument
%
\isamarkupsubsection{Monomorphisms, Epimorphisms and Isomorphisms%
}
\isamarkuptrue%
%
\isamarkupsubsubsection{Monomorphisms%
}
\isamarkuptrue%
%
\endisatagdocument
{\isafolddocument}%
%
\isadelimdocument
%
\endisadelimdocument
\isacommand{definition}\isamarkupfalse%
\ monomorphism\ {\isacharcolon}{\kern0pt}{\isacharcolon}{\kern0pt}\ {\isachardoublequoteopen}cfunc\ {\isasymRightarrow}\ bool{\isachardoublequoteclose}\ \isakeyword{where}\isanewline
\ \ {\isachardoublequoteopen}monomorphism\ f\ {\isasymlongleftrightarrow}\ {\isacharparenleft}{\kern0pt}{\isasymforall}\ g\ h{\isachardot}{\kern0pt}\ \isanewline
\ \ \ \ {\isacharparenleft}{\kern0pt}codomain\ g\ {\isacharequal}{\kern0pt}\ domain\ f\ {\isasymand}\ codomain\ h\ {\isacharequal}{\kern0pt}\ domain\ f{\isacharparenright}{\kern0pt}\ {\isasymlongrightarrow}\ {\isacharparenleft}{\kern0pt}f\ {\isasymcirc}\isactrlsub c\ g\ {\isacharequal}{\kern0pt}\ f\ {\isasymcirc}\isactrlsub c\ h\ {\isasymlongrightarrow}\ g\ {\isacharequal}{\kern0pt}\ h{\isacharparenright}{\kern0pt}{\isacharparenright}{\kern0pt}{\isachardoublequoteclose}\isanewline
\isanewline
\isacommand{lemma}\isamarkupfalse%
\ monomorphism{\isacharunderscore}{\kern0pt}def{\isadigit{2}}{\isacharcolon}{\kern0pt}\isanewline
\ \ {\isachardoublequoteopen}monomorphism\ f\ {\isasymlongleftrightarrow}\ {\isacharparenleft}{\kern0pt}{\isasymforall}\ g\ h\ A\ X\ Y{\isachardot}{\kern0pt}\ g\ {\isacharcolon}{\kern0pt}\ A\ {\isasymrightarrow}\ X\ {\isasymand}\ h\ {\isacharcolon}{\kern0pt}\ A\ {\isasymrightarrow}\ X\ {\isasymand}\ f\ {\isacharcolon}{\kern0pt}\ X\ {\isasymrightarrow}\ Y\ {\isasymlongrightarrow}\ {\isacharparenleft}{\kern0pt}f\ {\isasymcirc}\isactrlsub c\ g\ {\isacharequal}{\kern0pt}\ f\ {\isasymcirc}\isactrlsub c\ h\ {\isasymlongrightarrow}\ g\ {\isacharequal}{\kern0pt}\ h{\isacharparenright}{\kern0pt}{\isacharparenright}{\kern0pt}{\isachardoublequoteclose}\isanewline
%
\isadelimproof
\ \ %
\endisadelimproof
%
\isatagproof
\isacommand{unfolding}\isamarkupfalse%
\ monomorphism{\isacharunderscore}{\kern0pt}def\ \isacommand{by}\isamarkupfalse%
\ {\isacharparenleft}{\kern0pt}smt\ cfunc{\isacharunderscore}{\kern0pt}type{\isacharunderscore}{\kern0pt}def\ domain{\isacharunderscore}{\kern0pt}comp{\isacharparenright}{\kern0pt}%
\endisatagproof
{\isafoldproof}%
%
\isadelimproof
\isanewline
%
\endisadelimproof
\isanewline
\isacommand{lemma}\isamarkupfalse%
\ monomorphism{\isacharunderscore}{\kern0pt}def{\isadigit{3}}{\isacharcolon}{\kern0pt}\isanewline
\ \ \isakeyword{assumes}\ {\isachardoublequoteopen}f\ {\isacharcolon}{\kern0pt}\ X\ {\isasymrightarrow}\ Y{\isachardoublequoteclose}\isanewline
\ \ \isakeyword{shows}\ {\isachardoublequoteopen}monomorphism\ f\ {\isasymlongleftrightarrow}\ {\isacharparenleft}{\kern0pt}{\isasymforall}\ g\ h\ A{\isachardot}{\kern0pt}\ g\ {\isacharcolon}{\kern0pt}\ A\ {\isasymrightarrow}\ X\ {\isasymand}\ h\ {\isacharcolon}{\kern0pt}\ A\ {\isasymrightarrow}\ X\ {\isasymlongrightarrow}\ {\isacharparenleft}{\kern0pt}f\ {\isasymcirc}\isactrlsub c\ g\ {\isacharequal}{\kern0pt}\ f\ {\isasymcirc}\isactrlsub c\ h\ {\isasymlongrightarrow}\ g\ {\isacharequal}{\kern0pt}\ h{\isacharparenright}{\kern0pt}{\isacharparenright}{\kern0pt}{\isachardoublequoteclose}\isanewline
%
\isadelimproof
\ \ %
\endisadelimproof
%
\isatagproof
\isacommand{unfolding}\isamarkupfalse%
\ monomorphism{\isacharunderscore}{\kern0pt}def{\isadigit{2}}\ \isacommand{using}\isamarkupfalse%
\ assms\ cfunc{\isacharunderscore}{\kern0pt}type{\isacharunderscore}{\kern0pt}def\ \isacommand{by}\isamarkupfalse%
\ auto%
\endisatagproof
{\isafoldproof}%
%
\isadelimproof
%
\endisadelimproof
%
\begin{isamarkuptext}%
The lemma below corresponds to Exercise 2.1.7a in Halvorson.%
\end{isamarkuptext}\isamarkuptrue%
\isacommand{lemma}\isamarkupfalse%
\ comp{\isacharunderscore}{\kern0pt}monic{\isacharunderscore}{\kern0pt}imp{\isacharunderscore}{\kern0pt}monic{\isacharcolon}{\kern0pt}\isanewline
\ \ \isakeyword{assumes}\ {\isachardoublequoteopen}domain\ g\ {\isacharequal}{\kern0pt}\ codomain\ f{\isachardoublequoteclose}\isanewline
\ \ \isakeyword{shows}\ {\isachardoublequoteopen}monomorphism\ {\isacharparenleft}{\kern0pt}g\ {\isasymcirc}\isactrlsub c\ f{\isacharparenright}{\kern0pt}\ {\isasymLongrightarrow}\ monomorphism\ f{\isachardoublequoteclose}\isanewline
%
\isadelimproof
\ \ %
\endisadelimproof
%
\isatagproof
\isacommand{unfolding}\isamarkupfalse%
\ monomorphism{\isacharunderscore}{\kern0pt}def\isanewline
\isacommand{proof}\isamarkupfalse%
\ clarify\isanewline
\ \ \isacommand{fix}\isamarkupfalse%
\ s\ t\isanewline
\ \ \isacommand{assume}\isamarkupfalse%
\ gf{\isacharunderscore}{\kern0pt}monic{\isacharcolon}{\kern0pt}\ {\isachardoublequoteopen}{\isasymforall}s{\isachardot}{\kern0pt}\ {\isasymforall}t{\isachardot}{\kern0pt}\ \isanewline
\ \ \ \ codomain\ s\ {\isacharequal}{\kern0pt}\ domain\ {\isacharparenleft}{\kern0pt}g\ {\isasymcirc}\isactrlsub c\ f{\isacharparenright}{\kern0pt}\ {\isasymand}\ codomain\ t\ {\isacharequal}{\kern0pt}\ domain\ {\isacharparenleft}{\kern0pt}g\ {\isasymcirc}\isactrlsub c\ f{\isacharparenright}{\kern0pt}\ {\isasymlongrightarrow}\isanewline
\ \ \ \ \ \ \ \ \ \ {\isacharparenleft}{\kern0pt}g\ {\isasymcirc}\isactrlsub c\ f{\isacharparenright}{\kern0pt}\ {\isasymcirc}\isactrlsub c\ s\ {\isacharequal}{\kern0pt}\ {\isacharparenleft}{\kern0pt}g\ {\isasymcirc}\isactrlsub c\ f{\isacharparenright}{\kern0pt}\ {\isasymcirc}\isactrlsub c\ t\ {\isasymlongrightarrow}\ s\ {\isacharequal}{\kern0pt}\ t{\isachardoublequoteclose}\isanewline
\ \ \isacommand{assume}\isamarkupfalse%
\ codomain{\isacharunderscore}{\kern0pt}s{\isacharcolon}{\kern0pt}\ {\isachardoublequoteopen}codomain\ s\ {\isacharequal}{\kern0pt}\ domain\ f{\isachardoublequoteclose}\isanewline
\ \ \isacommand{assume}\isamarkupfalse%
\ codomain{\isacharunderscore}{\kern0pt}t{\isacharcolon}{\kern0pt}\ {\isachardoublequoteopen}codomain\ t\ {\isacharequal}{\kern0pt}\ domain\ f{\isachardoublequoteclose}\isanewline
\ \ \isacommand{assume}\isamarkupfalse%
\ {\isachardoublequoteopen}f\ {\isasymcirc}\isactrlsub c\ s\ {\isacharequal}{\kern0pt}\ f\ {\isasymcirc}\isactrlsub c\ t{\isachardoublequoteclose}\isanewline
\isanewline
\ \ \isacommand{then}\isamarkupfalse%
\ \isacommand{have}\isamarkupfalse%
\ {\isachardoublequoteopen}{\isacharparenleft}{\kern0pt}g\ {\isasymcirc}\isactrlsub c\ f{\isacharparenright}{\kern0pt}\ {\isasymcirc}\isactrlsub c\ s\ {\isacharequal}{\kern0pt}\ {\isacharparenleft}{\kern0pt}g\ {\isasymcirc}\isactrlsub c\ f{\isacharparenright}{\kern0pt}\ {\isasymcirc}\isactrlsub c\ t{\isachardoublequoteclose}\isanewline
\ \ \ \ \isacommand{by}\isamarkupfalse%
\ {\isacharparenleft}{\kern0pt}metis\ assms\ codomain{\isacharunderscore}{\kern0pt}s\ codomain{\isacharunderscore}{\kern0pt}t\ comp{\isacharunderscore}{\kern0pt}associative{\isacharparenright}{\kern0pt}\isanewline
\ \ \isacommand{then}\isamarkupfalse%
\ \isacommand{show}\isamarkupfalse%
\ {\isachardoublequoteopen}s\ {\isacharequal}{\kern0pt}\ t{\isachardoublequoteclose}\isanewline
\ \ \ \ \isacommand{using}\isamarkupfalse%
\ gf{\isacharunderscore}{\kern0pt}monic\ codomain{\isacharunderscore}{\kern0pt}s\ codomain{\isacharunderscore}{\kern0pt}t\ domain{\isacharunderscore}{\kern0pt}comp\ \isacommand{by}\isamarkupfalse%
\ {\isacharparenleft}{\kern0pt}simp\ add{\isacharcolon}{\kern0pt}\ assms{\isacharparenright}{\kern0pt}\isanewline
\isacommand{qed}\isamarkupfalse%
%
\endisatagproof
{\isafoldproof}%
%
\isadelimproof
\isanewline
%
\endisadelimproof
\isanewline
\isacommand{lemma}\isamarkupfalse%
\ comp{\isacharunderscore}{\kern0pt}monic{\isacharunderscore}{\kern0pt}imp{\isacharunderscore}{\kern0pt}monic{\isacharprime}{\kern0pt}{\isacharcolon}{\kern0pt}\isanewline
\ \ \isakeyword{assumes}\ {\isachardoublequoteopen}f\ {\isacharcolon}{\kern0pt}\ X\ {\isasymrightarrow}\ Y{\isachardoublequoteclose}\ {\isachardoublequoteopen}g\ {\isacharcolon}{\kern0pt}\ Y\ {\isasymrightarrow}\ Z{\isachardoublequoteclose}\isanewline
\ \ \isakeyword{shows}\ {\isachardoublequoteopen}monomorphism\ {\isacharparenleft}{\kern0pt}g\ {\isasymcirc}\isactrlsub c\ f{\isacharparenright}{\kern0pt}\ {\isasymLongrightarrow}\ monomorphism\ f{\isachardoublequoteclose}\isanewline
%
\isadelimproof
\ \ %
\endisadelimproof
%
\isatagproof
\isacommand{by}\isamarkupfalse%
\ {\isacharparenleft}{\kern0pt}metis\ assms\ cfunc{\isacharunderscore}{\kern0pt}type{\isacharunderscore}{\kern0pt}def\ comp{\isacharunderscore}{\kern0pt}monic{\isacharunderscore}{\kern0pt}imp{\isacharunderscore}{\kern0pt}monic{\isacharparenright}{\kern0pt}%
\endisatagproof
{\isafoldproof}%
%
\isadelimproof
%
\endisadelimproof
%
\begin{isamarkuptext}%
The lemma below corresponds to Exercise 2.1.7c in Halvorson.%
\end{isamarkuptext}\isamarkuptrue%
\isacommand{lemma}\isamarkupfalse%
\ composition{\isacharunderscore}{\kern0pt}of{\isacharunderscore}{\kern0pt}monic{\isacharunderscore}{\kern0pt}pair{\isacharunderscore}{\kern0pt}is{\isacharunderscore}{\kern0pt}monic{\isacharcolon}{\kern0pt}\isanewline
\ \ \isakeyword{assumes}\ {\isachardoublequoteopen}codomain\ f\ {\isacharequal}{\kern0pt}\ domain\ g{\isachardoublequoteclose}\isanewline
\ \ \isakeyword{shows}\ {\isachardoublequoteopen}monomorphism\ f\ {\isasymLongrightarrow}\ monomorphism\ g\ {\isasymLongrightarrow}\ monomorphism\ {\isacharparenleft}{\kern0pt}g\ {\isasymcirc}\isactrlsub c\ f{\isacharparenright}{\kern0pt}{\isachardoublequoteclose}\isanewline
%
\isadelimproof
\ \ %
\endisadelimproof
%
\isatagproof
\isacommand{unfolding}\isamarkupfalse%
\ monomorphism{\isacharunderscore}{\kern0pt}def\isanewline
\isacommand{proof}\isamarkupfalse%
\ clarify\isanewline
\ \ \isacommand{fix}\isamarkupfalse%
\ h\ k\isanewline
\ \ \isacommand{assume}\isamarkupfalse%
\ f{\isacharunderscore}{\kern0pt}mono{\isacharcolon}{\kern0pt}\ {\isachardoublequoteopen}{\isasymforall}s\ t{\isachardot}{\kern0pt}\ \isanewline
\ \ \ \ codomain\ s\ {\isacharequal}{\kern0pt}\ domain\ f\ {\isasymand}\ codomain\ t\ {\isacharequal}{\kern0pt}\ domain\ f\ {\isasymlongrightarrow}\ f\ {\isasymcirc}\isactrlsub c\ s\ {\isacharequal}{\kern0pt}\ f\ {\isasymcirc}\isactrlsub c\ t\ {\isasymlongrightarrow}\ s\ {\isacharequal}{\kern0pt}\ t{\isachardoublequoteclose}\isanewline
\ \ \isacommand{assume}\isamarkupfalse%
\ g{\isacharunderscore}{\kern0pt}mono{\isacharcolon}{\kern0pt}\ {\isachardoublequoteopen}{\isasymforall}s{\isachardot}{\kern0pt}\ {\isasymforall}t{\isachardot}{\kern0pt}\ \isanewline
\ \ \ \ codomain\ s\ {\isacharequal}{\kern0pt}\ domain\ g\ {\isasymand}\ codomain\ t\ {\isacharequal}{\kern0pt}\ domain\ g\ {\isasymlongrightarrow}\ g\ {\isasymcirc}\isactrlsub c\ s\ {\isacharequal}{\kern0pt}\ g\ {\isasymcirc}\isactrlsub c\ t\ {\isasymlongrightarrow}\ s\ {\isacharequal}{\kern0pt}\ t{\isachardoublequoteclose}\isanewline
\ \ \isacommand{assume}\isamarkupfalse%
\ codomain{\isacharunderscore}{\kern0pt}k{\isacharcolon}{\kern0pt}\ {\isachardoublequoteopen}codomain\ k\ {\isacharequal}{\kern0pt}\ domain\ {\isacharparenleft}{\kern0pt}g\ {\isasymcirc}\isactrlsub c\ f{\isacharparenright}{\kern0pt}{\isachardoublequoteclose}\isanewline
\ \ \isacommand{assume}\isamarkupfalse%
\ codomain{\isacharunderscore}{\kern0pt}h{\isacharcolon}{\kern0pt}\ {\isachardoublequoteopen}codomain\ h\ {\isacharequal}{\kern0pt}\ domain\ {\isacharparenleft}{\kern0pt}g\ {\isasymcirc}\isactrlsub c\ f{\isacharparenright}{\kern0pt}{\isachardoublequoteclose}\isanewline
\ \ \isacommand{assume}\isamarkupfalse%
\ gfh{\isacharunderscore}{\kern0pt}eq{\isacharunderscore}{\kern0pt}gfk{\isacharcolon}{\kern0pt}\ {\isachardoublequoteopen}{\isacharparenleft}{\kern0pt}g\ {\isasymcirc}\isactrlsub c\ f{\isacharparenright}{\kern0pt}\ {\isasymcirc}\isactrlsub c\ k\ {\isacharequal}{\kern0pt}\ {\isacharparenleft}{\kern0pt}g\ {\isasymcirc}\isactrlsub c\ f{\isacharparenright}{\kern0pt}\ {\isasymcirc}\isactrlsub c\ h{\isachardoublequoteclose}\isanewline
\ \isanewline
\ \ \isacommand{have}\isamarkupfalse%
\ {\isachardoublequoteopen}g\ {\isasymcirc}\isactrlsub c\ {\isacharparenleft}{\kern0pt}f\ {\isasymcirc}\isactrlsub c\ h{\isacharparenright}{\kern0pt}\ {\isacharequal}{\kern0pt}\ {\isacharparenleft}{\kern0pt}g\ \ {\isasymcirc}\isactrlsub c\ f{\isacharparenright}{\kern0pt}\ \ {\isasymcirc}\isactrlsub c\ h{\isachardoublequoteclose}\isanewline
\ \ \ \ \isacommand{by}\isamarkupfalse%
\ {\isacharparenleft}{\kern0pt}simp\ add{\isacharcolon}{\kern0pt}\ assms\ codomain{\isacharunderscore}{\kern0pt}h\ comp{\isacharunderscore}{\kern0pt}associative\ domain{\isacharunderscore}{\kern0pt}comp{\isacharparenright}{\kern0pt}\isanewline
\ \ \isacommand{also}\isamarkupfalse%
\ \isacommand{have}\isamarkupfalse%
\ {\isachardoublequoteopen}{\isachardot}{\kern0pt}{\isachardot}{\kern0pt}{\isachardot}{\kern0pt}\ {\isacharequal}{\kern0pt}\ {\isacharparenleft}{\kern0pt}g\ {\isasymcirc}\isactrlsub c\ f{\isacharparenright}{\kern0pt}\ {\isasymcirc}\isactrlsub c\ k{\isachardoublequoteclose}\isanewline
\ \ \ \ \isacommand{by}\isamarkupfalse%
\ {\isacharparenleft}{\kern0pt}simp\ add{\isacharcolon}{\kern0pt}\ gfh{\isacharunderscore}{\kern0pt}eq{\isacharunderscore}{\kern0pt}gfk{\isacharparenright}{\kern0pt}\isanewline
\ \ \isacommand{also}\isamarkupfalse%
\ \isacommand{have}\isamarkupfalse%
\ {\isachardoublequoteopen}{\isachardot}{\kern0pt}{\isachardot}{\kern0pt}{\isachardot}{\kern0pt}\ {\isacharequal}{\kern0pt}\ g\ {\isasymcirc}\isactrlsub c\ {\isacharparenleft}{\kern0pt}f\ {\isasymcirc}\isactrlsub c\ k{\isacharparenright}{\kern0pt}{\isachardoublequoteclose}\isanewline
\ \ \ \ \isacommand{by}\isamarkupfalse%
\ {\isacharparenleft}{\kern0pt}simp\ add{\isacharcolon}{\kern0pt}\ assms\ codomain{\isacharunderscore}{\kern0pt}k\ comp{\isacharunderscore}{\kern0pt}associative\ domain{\isacharunderscore}{\kern0pt}comp{\isacharparenright}{\kern0pt}\isanewline
\ \ \isacommand{then}\isamarkupfalse%
\ \isacommand{have}\isamarkupfalse%
\ {\isachardoublequoteopen}f\ {\isasymcirc}\isactrlsub c\ h\ {\isacharequal}{\kern0pt}\ f\ {\isasymcirc}\isactrlsub c\ k{\isachardoublequoteclose}\isanewline
\ \ \ \ \isacommand{using}\isamarkupfalse%
\ assms\ calculation\ cfunc{\isacharunderscore}{\kern0pt}type{\isacharunderscore}{\kern0pt}def\ codomain{\isacharunderscore}{\kern0pt}h\ codomain{\isacharunderscore}{\kern0pt}k\ comp{\isacharunderscore}{\kern0pt}type\ domain{\isacharunderscore}{\kern0pt}comp\ g{\isacharunderscore}{\kern0pt}mono\ \isacommand{by}\isamarkupfalse%
\ auto\isanewline
\ \ \isacommand{then}\isamarkupfalse%
\ \isacommand{show}\isamarkupfalse%
\ {\isachardoublequoteopen}k\ {\isacharequal}{\kern0pt}\ h{\isachardoublequoteclose}\isanewline
\ \ \ \ \isacommand{by}\isamarkupfalse%
\ {\isacharparenleft}{\kern0pt}simp\ add{\isacharcolon}{\kern0pt}\ codomain{\isacharunderscore}{\kern0pt}h\ codomain{\isacharunderscore}{\kern0pt}k\ domain{\isacharunderscore}{\kern0pt}comp\ f{\isacharunderscore}{\kern0pt}mono\ assms{\isacharparenright}{\kern0pt}\isanewline
\isacommand{qed}\isamarkupfalse%
%
\endisatagproof
{\isafoldproof}%
%
\isadelimproof
%
\endisadelimproof
%
\isadelimdocument
%
\endisadelimdocument
%
\isatagdocument
%
\isamarkupsubsubsection{Epimorphisms%
}
\isamarkuptrue%
%
\endisatagdocument
{\isafolddocument}%
%
\isadelimdocument
%
\endisadelimdocument
\isacommand{definition}\isamarkupfalse%
\ epimorphism\ {\isacharcolon}{\kern0pt}{\isacharcolon}{\kern0pt}\ {\isachardoublequoteopen}cfunc\ {\isasymRightarrow}\ bool{\isachardoublequoteclose}\ \isakeyword{where}\isanewline
\ \ {\isachardoublequoteopen}epimorphism\ f\ {\isasymlongleftrightarrow}\ {\isacharparenleft}{\kern0pt}{\isasymforall}\ g\ h{\isachardot}{\kern0pt}\ \isanewline
\ \ \ \ {\isacharparenleft}{\kern0pt}domain\ g\ {\isacharequal}{\kern0pt}\ codomain\ f\ {\isasymand}\ domain\ h\ {\isacharequal}{\kern0pt}\ codomain\ f{\isacharparenright}{\kern0pt}\ {\isasymlongrightarrow}\ {\isacharparenleft}{\kern0pt}g\ {\isasymcirc}\isactrlsub c\ f\ {\isacharequal}{\kern0pt}\ h\ {\isasymcirc}\isactrlsub c\ f\ {\isasymlongrightarrow}\ g\ {\isacharequal}{\kern0pt}\ h{\isacharparenright}{\kern0pt}{\isacharparenright}{\kern0pt}{\isachardoublequoteclose}\isanewline
\isanewline
\isacommand{lemma}\isamarkupfalse%
\ epimorphism{\isacharunderscore}{\kern0pt}def{\isadigit{2}}{\isacharcolon}{\kern0pt}\isanewline
\ \ {\isachardoublequoteopen}epimorphism\ f\ {\isasymlongleftrightarrow}\ {\isacharparenleft}{\kern0pt}{\isasymforall}\ g\ h\ A\ X\ Y{\isachardot}{\kern0pt}\ f\ {\isacharcolon}{\kern0pt}\ X\ {\isasymrightarrow}\ Y\ {\isasymand}\ g\ {\isacharcolon}{\kern0pt}\ Y\ {\isasymrightarrow}\ A\ {\isasymand}\ h\ {\isacharcolon}{\kern0pt}\ Y\ {\isasymrightarrow}\ A\ {\isasymlongrightarrow}\ {\isacharparenleft}{\kern0pt}g\ {\isasymcirc}\isactrlsub c\ f\ {\isacharequal}{\kern0pt}\ h\ {\isasymcirc}\isactrlsub c\ f\ {\isasymlongrightarrow}\ g\ {\isacharequal}{\kern0pt}\ h{\isacharparenright}{\kern0pt}{\isacharparenright}{\kern0pt}{\isachardoublequoteclose}\isanewline
%
\isadelimproof
\ \ %
\endisadelimproof
%
\isatagproof
\isacommand{unfolding}\isamarkupfalse%
\ epimorphism{\isacharunderscore}{\kern0pt}def\ \isacommand{by}\isamarkupfalse%
\ {\isacharparenleft}{\kern0pt}smt\ cfunc{\isacharunderscore}{\kern0pt}type{\isacharunderscore}{\kern0pt}def\ codomain{\isacharunderscore}{\kern0pt}comp{\isacharparenright}{\kern0pt}%
\endisatagproof
{\isafoldproof}%
%
\isadelimproof
\ \isanewline
%
\endisadelimproof
\isanewline
\isacommand{lemma}\isamarkupfalse%
\ epimorphism{\isacharunderscore}{\kern0pt}def{\isadigit{3}}{\isacharcolon}{\kern0pt}\isanewline
\ \ \isakeyword{assumes}\ {\isachardoublequoteopen}f\ {\isacharcolon}{\kern0pt}\ X\ {\isasymrightarrow}\ Y{\isachardoublequoteclose}\isanewline
\ \ \isakeyword{shows}\ {\isachardoublequoteopen}epimorphism\ f\ {\isasymlongleftrightarrow}\ {\isacharparenleft}{\kern0pt}{\isasymforall}\ g\ h\ A{\isachardot}{\kern0pt}\ g\ {\isacharcolon}{\kern0pt}\ Y\ {\isasymrightarrow}\ A\ {\isasymand}\ h\ {\isacharcolon}{\kern0pt}\ Y\ {\isasymrightarrow}\ A\ {\isasymlongrightarrow}\ {\isacharparenleft}{\kern0pt}g\ {\isasymcirc}\isactrlsub c\ f\ {\isacharequal}{\kern0pt}\ h\ {\isasymcirc}\isactrlsub c\ f\ {\isasymlongrightarrow}\ g\ {\isacharequal}{\kern0pt}\ h{\isacharparenright}{\kern0pt}{\isacharparenright}{\kern0pt}{\isachardoublequoteclose}\isanewline
%
\isadelimproof
\ \ %
\endisadelimproof
%
\isatagproof
\isacommand{unfolding}\isamarkupfalse%
\ epimorphism{\isacharunderscore}{\kern0pt}def{\isadigit{2}}\ \isacommand{using}\isamarkupfalse%
\ assms\ cfunc{\isacharunderscore}{\kern0pt}type{\isacharunderscore}{\kern0pt}def\ \isacommand{by}\isamarkupfalse%
\ auto%
\endisatagproof
{\isafoldproof}%
%
\isadelimproof
%
\endisadelimproof
%
\begin{isamarkuptext}%
The lemma below corresponds to Exercise 2.1.7b in Halvorson.%
\end{isamarkuptext}\isamarkuptrue%
\isacommand{lemma}\isamarkupfalse%
\ comp{\isacharunderscore}{\kern0pt}epi{\isacharunderscore}{\kern0pt}imp{\isacharunderscore}{\kern0pt}epi{\isacharcolon}{\kern0pt}\isanewline
\ \ \isakeyword{assumes}\ {\isachardoublequoteopen}domain\ g\ {\isacharequal}{\kern0pt}\ codomain\ f{\isachardoublequoteclose}\isanewline
\ \ \isakeyword{shows}\ {\isachardoublequoteopen}epimorphism\ {\isacharparenleft}{\kern0pt}g\ {\isasymcirc}\isactrlsub c\ f{\isacharparenright}{\kern0pt}\ {\isasymLongrightarrow}\ epimorphism\ g{\isachardoublequoteclose}\isanewline
%
\isadelimproof
\ \ %
\endisadelimproof
%
\isatagproof
\isacommand{unfolding}\isamarkupfalse%
\ epimorphism{\isacharunderscore}{\kern0pt}def\isanewline
\isacommand{proof}\isamarkupfalse%
\ clarify\isanewline
\ \ \isacommand{fix}\isamarkupfalse%
\ s\ t\isanewline
\ \ \isacommand{assume}\isamarkupfalse%
\ gf{\isacharunderscore}{\kern0pt}epi{\isacharcolon}{\kern0pt}\ {\isachardoublequoteopen}{\isasymforall}s{\isachardot}{\kern0pt}\ {\isasymforall}t{\isachardot}{\kern0pt}\isanewline
\ \ \ \ domain\ s\ {\isacharequal}{\kern0pt}\ codomain\ {\isacharparenleft}{\kern0pt}g\ {\isasymcirc}\isactrlsub c\ f{\isacharparenright}{\kern0pt}\ {\isasymand}\ domain\ t\ {\isacharequal}{\kern0pt}\ codomain\ {\isacharparenleft}{\kern0pt}g\ {\isasymcirc}\isactrlsub c\ f{\isacharparenright}{\kern0pt}\ {\isasymlongrightarrow}\isanewline
\ \ \ \ \ \ \ \ \ \ s\ {\isasymcirc}\isactrlsub c\ g\ {\isasymcirc}\isactrlsub c\ f\ {\isacharequal}{\kern0pt}\ t\ {\isasymcirc}\isactrlsub c\ g\ {\isasymcirc}\isactrlsub c\ f\ {\isasymlongrightarrow}\ s\ {\isacharequal}{\kern0pt}\ t{\isachardoublequoteclose}\isanewline
\ \ \isacommand{assume}\isamarkupfalse%
\ domain{\isacharunderscore}{\kern0pt}s{\isacharcolon}{\kern0pt}\ {\isachardoublequoteopen}domain\ s\ {\isacharequal}{\kern0pt}\ codomain\ g{\isachardoublequoteclose}\isanewline
\ \ \isacommand{assume}\isamarkupfalse%
\ domain{\isacharunderscore}{\kern0pt}t{\isacharcolon}{\kern0pt}\ {\isachardoublequoteopen}domain\ t\ {\isacharequal}{\kern0pt}\ codomain\ g{\isachardoublequoteclose}\isanewline
\ \ \isacommand{assume}\isamarkupfalse%
\ sf{\isacharunderscore}{\kern0pt}eq{\isacharunderscore}{\kern0pt}tf{\isacharcolon}{\kern0pt}\ {\isachardoublequoteopen}s\ {\isasymcirc}\isactrlsub c\ g\ {\isacharequal}{\kern0pt}\ t\ {\isasymcirc}\isactrlsub c\ g{\isachardoublequoteclose}\isanewline
\isanewline
\ \ \isacommand{from}\isamarkupfalse%
\ sf{\isacharunderscore}{\kern0pt}eq{\isacharunderscore}{\kern0pt}tf\ \isacommand{have}\isamarkupfalse%
\ {\isachardoublequoteopen}s\ {\isasymcirc}\isactrlsub c\ {\isacharparenleft}{\kern0pt}g\ {\isasymcirc}\isactrlsub c\ f{\isacharparenright}{\kern0pt}\ {\isacharequal}{\kern0pt}\ t\ {\isasymcirc}\isactrlsub c\ {\isacharparenleft}{\kern0pt}g\ {\isasymcirc}\isactrlsub c\ f{\isacharparenright}{\kern0pt}{\isachardoublequoteclose}\isanewline
\ \ \ \ \isacommand{by}\isamarkupfalse%
\ {\isacharparenleft}{\kern0pt}simp\ add{\isacharcolon}{\kern0pt}\ assms\ comp{\isacharunderscore}{\kern0pt}associative\ domain{\isacharunderscore}{\kern0pt}s\ domain{\isacharunderscore}{\kern0pt}t{\isacharparenright}{\kern0pt}\isanewline
\ \ \isacommand{then}\isamarkupfalse%
\ \isacommand{show}\isamarkupfalse%
\ {\isachardoublequoteopen}s\ {\isacharequal}{\kern0pt}\ t{\isachardoublequoteclose}\isanewline
\ \ \ \ \isacommand{using}\isamarkupfalse%
\ gf{\isacharunderscore}{\kern0pt}epi\ codomain{\isacharunderscore}{\kern0pt}comp\ domain{\isacharunderscore}{\kern0pt}s\ domain{\isacharunderscore}{\kern0pt}t\ \isacommand{by}\isamarkupfalse%
\ {\isacharparenleft}{\kern0pt}simp\ add{\isacharcolon}{\kern0pt}\ assms{\isacharparenright}{\kern0pt}\isanewline
\isacommand{qed}\isamarkupfalse%
%
\endisatagproof
{\isafoldproof}%
%
\isadelimproof
%
\endisadelimproof
%
\begin{isamarkuptext}%
The lemma below corresponds to Exercise 2.1.7d in Halvorson.%
\end{isamarkuptext}\isamarkuptrue%
\isacommand{lemma}\isamarkupfalse%
\ composition{\isacharunderscore}{\kern0pt}of{\isacharunderscore}{\kern0pt}epi{\isacharunderscore}{\kern0pt}pair{\isacharunderscore}{\kern0pt}is{\isacharunderscore}{\kern0pt}epi{\isacharcolon}{\kern0pt}\isanewline
\isakeyword{assumes}\ {\isachardoublequoteopen}codomain\ f\ {\isacharequal}{\kern0pt}\ domain\ g{\isachardoublequoteclose}\isanewline
\ \ \isakeyword{shows}\ {\isachardoublequoteopen}epimorphism\ f\ {\isasymLongrightarrow}\ epimorphism\ g\ {\isasymLongrightarrow}\ epimorphism\ {\isacharparenleft}{\kern0pt}g\ {\isasymcirc}\isactrlsub c\ f{\isacharparenright}{\kern0pt}{\isachardoublequoteclose}\isanewline
%
\isadelimproof
\ \ %
\endisadelimproof
%
\isatagproof
\isacommand{unfolding}\isamarkupfalse%
\ epimorphism{\isacharunderscore}{\kern0pt}def\isanewline
\isacommand{proof}\isamarkupfalse%
\ clarify\isanewline
\ \ \isacommand{fix}\isamarkupfalse%
\ h\ k\isanewline
\ \ \isacommand{assume}\isamarkupfalse%
\ f{\isacharunderscore}{\kern0pt}epi\ {\isacharcolon}{\kern0pt}{\isachardoublequoteopen}{\isasymforall}\ s\ h{\isachardot}{\kern0pt}\isanewline
\ \ \ \ {\isacharparenleft}{\kern0pt}domain\ s\ {\isacharequal}{\kern0pt}\ codomain\ f\ {\isasymand}\ domain\ h\ {\isacharequal}{\kern0pt}\ codomain\ f{\isacharparenright}{\kern0pt}\ {\isasymlongrightarrow}\ {\isacharparenleft}{\kern0pt}s\ {\isasymcirc}\isactrlsub c\ f\ {\isacharequal}{\kern0pt}\ h\ {\isasymcirc}\isactrlsub c\ f\ {\isasymlongrightarrow}\ s\ {\isacharequal}{\kern0pt}\ h{\isacharparenright}{\kern0pt}{\isachardoublequoteclose}\isanewline
\ \ \isacommand{assume}\isamarkupfalse%
\ g{\isacharunderscore}{\kern0pt}epi\ {\isacharcolon}{\kern0pt}{\isachardoublequoteopen}{\isasymforall}\ s\ h{\isachardot}{\kern0pt}\isanewline
\ \ \ \ {\isacharparenleft}{\kern0pt}domain\ s\ {\isacharequal}{\kern0pt}\ codomain\ g\ {\isasymand}\ domain\ h\ {\isacharequal}{\kern0pt}\ codomain\ g{\isacharparenright}{\kern0pt}\ {\isasymlongrightarrow}\ {\isacharparenleft}{\kern0pt}s\ {\isasymcirc}\isactrlsub c\ g\ {\isacharequal}{\kern0pt}\ h\ {\isasymcirc}\isactrlsub c\ g\ {\isasymlongrightarrow}\ s\ {\isacharequal}{\kern0pt}\ h{\isacharparenright}{\kern0pt}{\isachardoublequoteclose}\isanewline
\ \ \isacommand{assume}\isamarkupfalse%
\ domain{\isacharunderscore}{\kern0pt}k{\isacharcolon}{\kern0pt}\ {\isachardoublequoteopen}domain\ k\ {\isacharequal}{\kern0pt}\ codomain\ {\isacharparenleft}{\kern0pt}g\ {\isasymcirc}\isactrlsub c\ f{\isacharparenright}{\kern0pt}{\isachardoublequoteclose}\isanewline
\ \ \isacommand{assume}\isamarkupfalse%
\ domain{\isacharunderscore}{\kern0pt}h{\isacharcolon}{\kern0pt}\ {\isachardoublequoteopen}domain\ h\ {\isacharequal}{\kern0pt}\ codomain\ {\isacharparenleft}{\kern0pt}g\ {\isasymcirc}\isactrlsub c\ f{\isacharparenright}{\kern0pt}{\isachardoublequoteclose}\isanewline
\ \ \isacommand{assume}\isamarkupfalse%
\ hgf{\isacharunderscore}{\kern0pt}eq{\isacharunderscore}{\kern0pt}kgf{\isacharcolon}{\kern0pt}\ {\isachardoublequoteopen}h\ {\isasymcirc}\isactrlsub c\ {\isacharparenleft}{\kern0pt}g\ {\isasymcirc}\isactrlsub c\ f{\isacharparenright}{\kern0pt}\ {\isacharequal}{\kern0pt}\ k\ {\isasymcirc}\isactrlsub c\ {\isacharparenleft}{\kern0pt}g\ {\isasymcirc}\isactrlsub c\ f{\isacharparenright}{\kern0pt}{\isachardoublequoteclose}\isanewline
\ \ \isanewline
\ \ \isacommand{have}\isamarkupfalse%
\ {\isachardoublequoteopen}{\isacharparenleft}{\kern0pt}h\ {\isasymcirc}\isactrlsub c\ g{\isacharparenright}{\kern0pt}\ {\isasymcirc}\isactrlsub c\ f\ {\isacharequal}{\kern0pt}\ h\ {\isasymcirc}\isactrlsub c\ {\isacharparenleft}{\kern0pt}g\ {\isasymcirc}\isactrlsub c\ f{\isacharparenright}{\kern0pt}{\isachardoublequoteclose}\isanewline
\ \ \ \ \isacommand{by}\isamarkupfalse%
\ {\isacharparenleft}{\kern0pt}simp\ add{\isacharcolon}{\kern0pt}\ assms\ codomain{\isacharunderscore}{\kern0pt}comp\ comp{\isacharunderscore}{\kern0pt}associative\ domain{\isacharunderscore}{\kern0pt}h{\isacharparenright}{\kern0pt}\isanewline
\ \ \isacommand{also}\isamarkupfalse%
\ \isacommand{have}\isamarkupfalse%
\ {\isachardoublequoteopen}{\isachardot}{\kern0pt}{\isachardot}{\kern0pt}{\isachardot}{\kern0pt}\ {\isacharequal}{\kern0pt}\ k\ {\isasymcirc}\isactrlsub c\ {\isacharparenleft}{\kern0pt}g\ {\isasymcirc}\isactrlsub c\ f{\isacharparenright}{\kern0pt}{\isachardoublequoteclose}\isanewline
\ \ \ \ \isacommand{by}\isamarkupfalse%
\ {\isacharparenleft}{\kern0pt}simp\ add{\isacharcolon}{\kern0pt}\ hgf{\isacharunderscore}{\kern0pt}eq{\isacharunderscore}{\kern0pt}kgf{\isacharparenright}{\kern0pt}\isanewline
\ \ \isacommand{also}\isamarkupfalse%
\ \isacommand{have}\isamarkupfalse%
\ {\isachardoublequoteopen}{\isachardot}{\kern0pt}{\isachardot}{\kern0pt}{\isachardot}{\kern0pt}\ {\isacharequal}{\kern0pt}{\isacharparenleft}{\kern0pt}k\ {\isasymcirc}\isactrlsub c\ g{\isacharparenright}{\kern0pt}\ {\isasymcirc}\isactrlsub c\ f\ {\isachardoublequoteclose}\isanewline
\ \ \ \ \isacommand{by}\isamarkupfalse%
\ {\isacharparenleft}{\kern0pt}simp\ add{\isacharcolon}{\kern0pt}\ assms\ codomain{\isacharunderscore}{\kern0pt}comp\ comp{\isacharunderscore}{\kern0pt}associative\ domain{\isacharunderscore}{\kern0pt}k{\isacharparenright}{\kern0pt}\isanewline
\ \isanewline
\ \ \isacommand{then}\isamarkupfalse%
\ \isacommand{have}\isamarkupfalse%
\ {\isachardoublequoteopen}h\ {\isasymcirc}\isactrlsub c\ g\ {\isacharequal}{\kern0pt}\ k\ {\isasymcirc}\isactrlsub c\ g{\isachardoublequoteclose}\isanewline
\ \ \ \ \isacommand{by}\isamarkupfalse%
\ {\isacharparenleft}{\kern0pt}simp\ add{\isacharcolon}{\kern0pt}\ assms\ calculation\ codomain{\isacharunderscore}{\kern0pt}comp\ domain{\isacharunderscore}{\kern0pt}comp\ domain{\isacharunderscore}{\kern0pt}h\ domain{\isacharunderscore}{\kern0pt}k\ f{\isacharunderscore}{\kern0pt}epi{\isacharparenright}{\kern0pt}\isanewline
\ \ \isacommand{then}\isamarkupfalse%
\ \isacommand{show}\isamarkupfalse%
\ {\isachardoublequoteopen}h\ {\isacharequal}{\kern0pt}\ k{\isachardoublequoteclose}\isanewline
\ \ \ \ \isacommand{by}\isamarkupfalse%
\ {\isacharparenleft}{\kern0pt}simp\ add{\isacharcolon}{\kern0pt}\ codomain{\isacharunderscore}{\kern0pt}comp\ domain{\isacharunderscore}{\kern0pt}h\ domain{\isacharunderscore}{\kern0pt}k\ g{\isacharunderscore}{\kern0pt}epi\ assms{\isacharparenright}{\kern0pt}\isanewline
\isacommand{qed}\isamarkupfalse%
%
\endisatagproof
{\isafoldproof}%
%
\isadelimproof
%
\endisadelimproof
%
\isadelimdocument
%
\endisadelimdocument
%
\isatagdocument
%
\isamarkupsubsubsection{Isomorphisms%
}
\isamarkuptrue%
%
\endisatagdocument
{\isafolddocument}%
%
\isadelimdocument
%
\endisadelimdocument
\isacommand{definition}\isamarkupfalse%
\ isomorphism\ {\isacharcolon}{\kern0pt}{\isacharcolon}{\kern0pt}\ {\isachardoublequoteopen}cfunc\ {\isasymRightarrow}\ bool{\isachardoublequoteclose}\ \isakeyword{where}\isanewline
\ \ {\isachardoublequoteopen}isomorphism\ f\ {\isasymlongleftrightarrow}\ {\isacharparenleft}{\kern0pt}{\isasymexists}\ g{\isachardot}{\kern0pt}\ domain\ g\ {\isacharequal}{\kern0pt}\ codomain\ f\ {\isasymand}\ codomain\ g\ {\isacharequal}{\kern0pt}\ domain\ f\ {\isasymand}\ \isanewline
\ \ \ \ g\ {\isasymcirc}\isactrlsub c\ f\ {\isacharequal}{\kern0pt}\ id{\isacharparenleft}{\kern0pt}domain\ f{\isacharparenright}{\kern0pt}\ {\isasymand}\ f\ {\isasymcirc}\isactrlsub c\ g\ {\isacharequal}{\kern0pt}\ id{\isacharparenleft}{\kern0pt}domain\ g{\isacharparenright}{\kern0pt}{\isacharparenright}{\kern0pt}{\isachardoublequoteclose}\isanewline
\isanewline
\isacommand{lemma}\isamarkupfalse%
\ isomorphism{\isacharunderscore}{\kern0pt}def{\isadigit{2}}{\isacharcolon}{\kern0pt}\isanewline
\ \ {\isachardoublequoteopen}isomorphism\ f\ {\isasymlongleftrightarrow}\ {\isacharparenleft}{\kern0pt}{\isasymexists}\ g\ X\ Y{\isachardot}{\kern0pt}\ f\ {\isacharcolon}{\kern0pt}\ X\ {\isasymrightarrow}\ Y\ {\isasymand}\ g\ {\isacharcolon}{\kern0pt}\ Y\ {\isasymrightarrow}\ X\ {\isasymand}\ g\ {\isasymcirc}\isactrlsub c\ f\ {\isacharequal}{\kern0pt}\ id\ X\ {\isasymand}\ f\ {\isasymcirc}\isactrlsub c\ g\ {\isacharequal}{\kern0pt}\ id\ Y{\isacharparenright}{\kern0pt}{\isachardoublequoteclose}\isanewline
%
\isadelimproof
\ \ %
\endisadelimproof
%
\isatagproof
\isacommand{unfolding}\isamarkupfalse%
\ isomorphism{\isacharunderscore}{\kern0pt}def\ cfunc{\isacharunderscore}{\kern0pt}type{\isacharunderscore}{\kern0pt}def\ \isacommand{by}\isamarkupfalse%
\ auto%
\endisatagproof
{\isafoldproof}%
%
\isadelimproof
\isanewline
%
\endisadelimproof
\isanewline
\isacommand{lemma}\isamarkupfalse%
\ isomorphism{\isacharunderscore}{\kern0pt}def{\isadigit{3}}{\isacharcolon}{\kern0pt}\isanewline
\ \ \isakeyword{assumes}\ {\isachardoublequoteopen}f\ {\isacharcolon}{\kern0pt}\ X\ {\isasymrightarrow}\ Y{\isachardoublequoteclose}\isanewline
\ \ \isakeyword{shows}\ {\isachardoublequoteopen}isomorphism\ f\ {\isasymlongleftrightarrow}\ {\isacharparenleft}{\kern0pt}{\isasymexists}\ g{\isachardot}{\kern0pt}\ g\ {\isacharcolon}{\kern0pt}\ Y\ {\isasymrightarrow}\ X\ {\isasymand}\ g\ {\isasymcirc}\isactrlsub c\ f\ {\isacharequal}{\kern0pt}\ id\ X\ {\isasymand}\ f\ {\isasymcirc}\isactrlsub c\ g\ {\isacharequal}{\kern0pt}\ id\ Y{\isacharparenright}{\kern0pt}{\isachardoublequoteclose}\isanewline
%
\isadelimproof
\ \ %
\endisadelimproof
%
\isatagproof
\isacommand{using}\isamarkupfalse%
\ assms\ \isacommand{unfolding}\isamarkupfalse%
\ isomorphism{\isacharunderscore}{\kern0pt}def{\isadigit{2}}\ cfunc{\isacharunderscore}{\kern0pt}type{\isacharunderscore}{\kern0pt}def\ \isacommand{by}\isamarkupfalse%
\ auto%
\endisatagproof
{\isafoldproof}%
%
\isadelimproof
\isanewline
%
\endisadelimproof
\isanewline
\isacommand{definition}\isamarkupfalse%
\ inverse\ {\isacharcolon}{\kern0pt}{\isacharcolon}{\kern0pt}\ {\isachardoublequoteopen}cfunc\ {\isasymRightarrow}\ cfunc{\isachardoublequoteclose}\ {\isacharparenleft}{\kern0pt}{\isachardoublequoteopen}{\isacharunderscore}{\kern0pt}\isactrlbold {\isasyminverse}{\isachardoublequoteclose}\ {\isacharbrackleft}{\kern0pt}{\isadigit{1}}{\isadigit{0}}{\isadigit{0}}{\isadigit{0}}{\isacharbrackright}{\kern0pt}\ {\isadigit{9}}{\isadigit{9}}{\isadigit{9}}{\isacharparenright}{\kern0pt}\ \isakeyword{where}\isanewline
\ \ {\isachardoublequoteopen}inverse\ f\ {\isacharequal}{\kern0pt}\ {\isacharparenleft}{\kern0pt}THE\ g{\isachardot}{\kern0pt}\ g\ {\isacharcolon}{\kern0pt}\ codomain\ f\ {\isasymrightarrow}\ domain\ f\ {\isasymand}\ g\ {\isasymcirc}\isactrlsub c\ f\ {\isacharequal}{\kern0pt}\ id{\isacharparenleft}{\kern0pt}domain\ f{\isacharparenright}{\kern0pt}\ {\isasymand}\ f\ {\isasymcirc}\isactrlsub c\ g\ {\isacharequal}{\kern0pt}\ id{\isacharparenleft}{\kern0pt}codomain\ f{\isacharparenright}{\kern0pt}{\isacharparenright}{\kern0pt}{\isachardoublequoteclose}\isanewline
\isanewline
\isacommand{lemma}\isamarkupfalse%
\ inverse{\isacharunderscore}{\kern0pt}def{\isadigit{2}}{\isacharcolon}{\kern0pt}\isanewline
\ \ \isakeyword{assumes}\ {\isachardoublequoteopen}isomorphism\ f{\isachardoublequoteclose}\isanewline
\ \ \isakeyword{shows}\ {\isachardoublequoteopen}f\isactrlbold {\isasyminverse}\ {\isacharcolon}{\kern0pt}\ codomain\ f\ {\isasymrightarrow}\ domain\ f\ {\isasymand}\ f\isactrlbold {\isasyminverse}\ {\isasymcirc}\isactrlsub c\ f\ {\isacharequal}{\kern0pt}\ id{\isacharparenleft}{\kern0pt}domain\ f{\isacharparenright}{\kern0pt}\ {\isasymand}\ f\ {\isasymcirc}\isactrlsub c\ f\isactrlbold {\isasyminverse}\ {\isacharequal}{\kern0pt}\ id{\isacharparenleft}{\kern0pt}codomain\ f{\isacharparenright}{\kern0pt}{\isachardoublequoteclose}\isanewline
%
\isadelimproof
\ \ %
\endisadelimproof
%
\isatagproof
\isacommand{unfolding}\isamarkupfalse%
\ inverse{\isacharunderscore}{\kern0pt}def\isanewline
\isacommand{proof}\isamarkupfalse%
\ {\isacharparenleft}{\kern0pt}rule\ theI{\isacharprime}{\kern0pt}{\isacharcomma}{\kern0pt}\ safe{\isacharparenright}{\kern0pt}\isanewline
\ \ \isacommand{show}\isamarkupfalse%
\ {\isachardoublequoteopen}{\isasymexists}g{\isachardot}{\kern0pt}\ g\ {\isacharcolon}{\kern0pt}\ codomain\ f\ {\isasymrightarrow}\ domain\ f\ {\isasymand}\ g\ {\isasymcirc}\isactrlsub c\ f\ {\isacharequal}{\kern0pt}\ id\isactrlsub c\ {\isacharparenleft}{\kern0pt}domain\ f{\isacharparenright}{\kern0pt}\ {\isasymand}\ f\ {\isasymcirc}\isactrlsub c\ g\ {\isacharequal}{\kern0pt}\ id\isactrlsub c\ {\isacharparenleft}{\kern0pt}codomain\ f{\isacharparenright}{\kern0pt}{\isachardoublequoteclose}\isanewline
\ \ \ \ \isacommand{using}\isamarkupfalse%
\ assms\ \isacommand{unfolding}\isamarkupfalse%
\ isomorphism{\isacharunderscore}{\kern0pt}def\ cfunc{\isacharunderscore}{\kern0pt}type{\isacharunderscore}{\kern0pt}def\ \isacommand{by}\isamarkupfalse%
\ auto\isanewline
\isacommand{next}\isamarkupfalse%
\isanewline
\ \ \isacommand{fix}\isamarkupfalse%
\ g{\isadigit{1}}\ g{\isadigit{2}}\isanewline
\ \ \isacommand{assume}\isamarkupfalse%
\ g{\isadigit{1}}{\isacharunderscore}{\kern0pt}f{\isacharcolon}{\kern0pt}\ {\isachardoublequoteopen}g{\isadigit{1}}\ {\isasymcirc}\isactrlsub c\ f\ {\isacharequal}{\kern0pt}\ id\isactrlsub c\ {\isacharparenleft}{\kern0pt}domain\ f{\isacharparenright}{\kern0pt}{\isachardoublequoteclose}\ \isakeyword{and}\ f{\isacharunderscore}{\kern0pt}g{\isadigit{1}}{\isacharcolon}{\kern0pt}\ {\isachardoublequoteopen}f\ {\isasymcirc}\isactrlsub c\ g{\isadigit{1}}\ {\isacharequal}{\kern0pt}\ id\isactrlsub c\ {\isacharparenleft}{\kern0pt}codomain\ f{\isacharparenright}{\kern0pt}{\isachardoublequoteclose}\isanewline
\ \ \isacommand{assume}\isamarkupfalse%
\ g{\isadigit{2}}{\isacharunderscore}{\kern0pt}f{\isacharcolon}{\kern0pt}\ {\isachardoublequoteopen}g{\isadigit{2}}\ {\isasymcirc}\isactrlsub c\ f\ {\isacharequal}{\kern0pt}\ id\isactrlsub c\ {\isacharparenleft}{\kern0pt}domain\ f{\isacharparenright}{\kern0pt}{\isachardoublequoteclose}\ \isakeyword{and}\ f{\isacharunderscore}{\kern0pt}g{\isadigit{2}}{\isacharcolon}{\kern0pt}\ {\isachardoublequoteopen}f\ {\isasymcirc}\isactrlsub c\ g{\isadigit{2}}\ {\isacharequal}{\kern0pt}\ id\isactrlsub c\ {\isacharparenleft}{\kern0pt}codomain\ f{\isacharparenright}{\kern0pt}{\isachardoublequoteclose}\isanewline
\ \ \isacommand{assume}\isamarkupfalse%
\ {\isachardoublequoteopen}g{\isadigit{1}}\ {\isacharcolon}{\kern0pt}\ codomain\ f\ {\isasymrightarrow}\ domain\ f{\isachardoublequoteclose}\ {\isachardoublequoteopen}g{\isadigit{2}}\ {\isacharcolon}{\kern0pt}\ codomain\ f\ {\isasymrightarrow}\ domain\ f{\isachardoublequoteclose}\isanewline
\ \ \isacommand{then}\isamarkupfalse%
\ \isacommand{have}\isamarkupfalse%
\ {\isachardoublequoteopen}codomain\ g{\isadigit{1}}\ {\isacharequal}{\kern0pt}\ domain\ f{\isachardoublequoteclose}\ {\isachardoublequoteopen}domain\ g{\isadigit{2}}\ {\isacharequal}{\kern0pt}\ codomain\ f{\isachardoublequoteclose}\isanewline
\ \ \ \ \isacommand{unfolding}\isamarkupfalse%
\ cfunc{\isacharunderscore}{\kern0pt}type{\isacharunderscore}{\kern0pt}def\ \isacommand{by}\isamarkupfalse%
\ auto\isanewline
\ \ \isacommand{then}\isamarkupfalse%
\ \isacommand{show}\isamarkupfalse%
\ {\isachardoublequoteopen}g{\isadigit{1}}\ {\isacharequal}{\kern0pt}\ g{\isadigit{2}}{\isachardoublequoteclose}\isanewline
\ \ \ \ \isacommand{by}\isamarkupfalse%
\ {\isacharparenleft}{\kern0pt}metis\ comp{\isacharunderscore}{\kern0pt}associative\ f{\isacharunderscore}{\kern0pt}g{\isadigit{1}}\ g{\isadigit{2}}{\isacharunderscore}{\kern0pt}f\ id{\isacharunderscore}{\kern0pt}left{\isacharunderscore}{\kern0pt}unit\ id{\isacharunderscore}{\kern0pt}right{\isacharunderscore}{\kern0pt}unit{\isacharparenright}{\kern0pt}\isanewline
\isacommand{qed}\isamarkupfalse%
%
\endisatagproof
{\isafoldproof}%
%
\isadelimproof
\isanewline
%
\endisadelimproof
\isanewline
\isacommand{lemma}\isamarkupfalse%
\ inverse{\isacharunderscore}{\kern0pt}type{\isacharbrackleft}{\kern0pt}type{\isacharunderscore}{\kern0pt}rule{\isacharbrackright}{\kern0pt}{\isacharcolon}{\kern0pt}\isanewline
\ \ \isakeyword{assumes}\ {\isachardoublequoteopen}isomorphism\ f{\isachardoublequoteclose}\ {\isachardoublequoteopen}f\ {\isacharcolon}{\kern0pt}\ X\ {\isasymrightarrow}\ Y{\isachardoublequoteclose}\isanewline
\ \ \isakeyword{shows}\ {\isachardoublequoteopen}f\isactrlbold {\isasyminverse}\ {\isacharcolon}{\kern0pt}\ Y\ {\isasymrightarrow}\ X{\isachardoublequoteclose}\isanewline
%
\isadelimproof
\ \ %
\endisadelimproof
%
\isatagproof
\isacommand{using}\isamarkupfalse%
\ assms\ inverse{\isacharunderscore}{\kern0pt}def{\isadigit{2}}\ \isacommand{unfolding}\isamarkupfalse%
\ cfunc{\isacharunderscore}{\kern0pt}type{\isacharunderscore}{\kern0pt}def\ \isacommand{by}\isamarkupfalse%
\ auto%
\endisatagproof
{\isafoldproof}%
%
\isadelimproof
\isanewline
%
\endisadelimproof
\isanewline
\isacommand{lemma}\isamarkupfalse%
\ inv{\isacharunderscore}{\kern0pt}left{\isacharcolon}{\kern0pt}\isanewline
\ \ \isakeyword{assumes}\ {\isachardoublequoteopen}isomorphism\ f{\isachardoublequoteclose}\ {\isachardoublequoteopen}f\ {\isacharcolon}{\kern0pt}\ X\ {\isasymrightarrow}\ Y{\isachardoublequoteclose}\isanewline
\ \ \isakeyword{shows}\ {\isachardoublequoteopen}f\isactrlbold {\isasyminverse}\ {\isasymcirc}\isactrlsub c\ f\ {\isacharequal}{\kern0pt}\ id\ X{\isachardoublequoteclose}\isanewline
%
\isadelimproof
\ \ %
\endisadelimproof
%
\isatagproof
\isacommand{using}\isamarkupfalse%
\ assms\ inverse{\isacharunderscore}{\kern0pt}def{\isadigit{2}}\ \isacommand{unfolding}\isamarkupfalse%
\ cfunc{\isacharunderscore}{\kern0pt}type{\isacharunderscore}{\kern0pt}def\ \isacommand{by}\isamarkupfalse%
\ auto%
\endisatagproof
{\isafoldproof}%
%
\isadelimproof
\isanewline
%
\endisadelimproof
\isanewline
\isacommand{lemma}\isamarkupfalse%
\ inv{\isacharunderscore}{\kern0pt}right{\isacharcolon}{\kern0pt}\isanewline
\ \ \isakeyword{assumes}\ {\isachardoublequoteopen}isomorphism\ f{\isachardoublequoteclose}\ {\isachardoublequoteopen}f\ {\isacharcolon}{\kern0pt}\ X\ {\isasymrightarrow}\ Y{\isachardoublequoteclose}\isanewline
\ \ \isakeyword{shows}\ {\isachardoublequoteopen}f\ {\isasymcirc}\isactrlsub c\ f\isactrlbold {\isasyminverse}\ {\isacharequal}{\kern0pt}\ id\ Y{\isachardoublequoteclose}\isanewline
%
\isadelimproof
\ \ %
\endisadelimproof
%
\isatagproof
\isacommand{using}\isamarkupfalse%
\ assms\ inverse{\isacharunderscore}{\kern0pt}def{\isadigit{2}}\ \isacommand{unfolding}\isamarkupfalse%
\ cfunc{\isacharunderscore}{\kern0pt}type{\isacharunderscore}{\kern0pt}def\ \isacommand{by}\isamarkupfalse%
\ auto%
\endisatagproof
{\isafoldproof}%
%
\isadelimproof
\isanewline
%
\endisadelimproof
\isanewline
\isacommand{lemma}\isamarkupfalse%
\ inv{\isacharunderscore}{\kern0pt}iso{\isacharcolon}{\kern0pt}\isanewline
\ \ \isakeyword{assumes}\ {\isachardoublequoteopen}isomorphism\ f{\isachardoublequoteclose}\isanewline
\ \ \isakeyword{shows}\ {\isachardoublequoteopen}isomorphism{\isacharparenleft}{\kern0pt}f\isactrlbold {\isasyminverse}{\isacharparenright}{\kern0pt}{\isachardoublequoteclose}\isanewline
%
\isadelimproof
\ \ %
\endisadelimproof
%
\isatagproof
\isacommand{using}\isamarkupfalse%
\ assms\ inverse{\isacharunderscore}{\kern0pt}def{\isadigit{2}}\ \isacommand{unfolding}\isamarkupfalse%
\ isomorphism{\isacharunderscore}{\kern0pt}def\ cfunc{\isacharunderscore}{\kern0pt}type{\isacharunderscore}{\kern0pt}def\ \isacommand{by}\isamarkupfalse%
\ {\isacharparenleft}{\kern0pt}intro\ exI{\isacharbrackleft}{\kern0pt}\isakeyword{where}\ x{\isacharequal}{\kern0pt}f{\isacharbrackright}{\kern0pt}{\isacharcomma}{\kern0pt}\ auto{\isacharparenright}{\kern0pt}%
\endisatagproof
{\isafoldproof}%
%
\isadelimproof
\isanewline
%
\endisadelimproof
\isanewline
\isacommand{lemma}\isamarkupfalse%
\ inv{\isacharunderscore}{\kern0pt}idempotent{\isacharcolon}{\kern0pt}\isanewline
\ \ \isakeyword{assumes}\ {\isachardoublequoteopen}isomorphism\ f{\isachardoublequoteclose}\isanewline
\ \ \isakeyword{shows}\ {\isachardoublequoteopen}{\isacharparenleft}{\kern0pt}f\isactrlbold {\isasyminverse}{\isacharparenright}{\kern0pt}\isactrlbold {\isasyminverse}\ {\isacharequal}{\kern0pt}\ f{\isachardoublequoteclose}\isanewline
%
\isadelimproof
\ \ %
\endisadelimproof
%
\isatagproof
\isacommand{by}\isamarkupfalse%
\ {\isacharparenleft}{\kern0pt}smt\ assms\ cfunc{\isacharunderscore}{\kern0pt}type{\isacharunderscore}{\kern0pt}def\ comp{\isacharunderscore}{\kern0pt}associative\ id{\isacharunderscore}{\kern0pt}left{\isacharunderscore}{\kern0pt}unit\ inv{\isacharunderscore}{\kern0pt}iso\ inverse{\isacharunderscore}{\kern0pt}def{\isadigit{2}}{\isacharparenright}{\kern0pt}%
\endisatagproof
{\isafoldproof}%
%
\isadelimproof
\isanewline
%
\endisadelimproof
\isanewline
\isacommand{definition}\isamarkupfalse%
\ is{\isacharunderscore}{\kern0pt}isomorphic\ {\isacharcolon}{\kern0pt}{\isacharcolon}{\kern0pt}\ {\isachardoublequoteopen}cset\ {\isasymRightarrow}\ cset\ {\isasymRightarrow}\ bool{\isachardoublequoteclose}\ {\isacharparenleft}{\kern0pt}\isakeyword{infix}\ {\isachardoublequoteopen}{\isasymcong}{\isachardoublequoteclose}\ {\isadigit{5}}{\isadigit{0}}{\isacharparenright}{\kern0pt}\ \ \isakeyword{where}\isanewline
\ \ {\isachardoublequoteopen}X\ {\isasymcong}\ Y\ {\isasymlongleftrightarrow}\ {\isacharparenleft}{\kern0pt}{\isasymexists}\ f{\isachardot}{\kern0pt}\ f\ {\isacharcolon}{\kern0pt}\ X\ {\isasymrightarrow}\ Y\ {\isasymand}\ isomorphism\ f{\isacharparenright}{\kern0pt}{\isachardoublequoteclose}\isanewline
\isanewline
\isacommand{lemma}\isamarkupfalse%
\ id{\isacharunderscore}{\kern0pt}isomorphism{\isacharcolon}{\kern0pt}\ {\isachardoublequoteopen}isomorphism\ {\isacharparenleft}{\kern0pt}id\ X{\isacharparenright}{\kern0pt}{\isachardoublequoteclose}\isanewline
%
\isadelimproof
\ \ %
\endisadelimproof
%
\isatagproof
\isacommand{unfolding}\isamarkupfalse%
\ isomorphism{\isacharunderscore}{\kern0pt}def\isanewline
\ \ \isacommand{by}\isamarkupfalse%
\ {\isacharparenleft}{\kern0pt}intro\ exI{\isacharbrackleft}{\kern0pt}\isakeyword{where}\ x{\isacharequal}{\kern0pt}\ {\isachardoublequoteopen}id\ X{\isachardoublequoteclose}{\isacharbrackright}{\kern0pt}{\isacharcomma}{\kern0pt}\ auto\ simp\ add{\isacharcolon}{\kern0pt}\ id{\isacharunderscore}{\kern0pt}codomain\ id{\isacharunderscore}{\kern0pt}domain{\isacharcomma}{\kern0pt}\ metis\ id{\isacharunderscore}{\kern0pt}domain\ id{\isacharunderscore}{\kern0pt}right{\isacharunderscore}{\kern0pt}unit{\isacharparenright}{\kern0pt}%
\endisatagproof
{\isafoldproof}%
%
\isadelimproof
\isanewline
%
\endisadelimproof
\isanewline
\isacommand{lemma}\isamarkupfalse%
\ isomorphic{\isacharunderscore}{\kern0pt}is{\isacharunderscore}{\kern0pt}reflexive{\isacharcolon}{\kern0pt}\ {\isachardoublequoteopen}X\ {\isasymcong}\ X{\isachardoublequoteclose}\isanewline
%
\isadelimproof
\ \ %
\endisadelimproof
%
\isatagproof
\isacommand{unfolding}\isamarkupfalse%
\ is{\isacharunderscore}{\kern0pt}isomorphic{\isacharunderscore}{\kern0pt}def\isanewline
\ \ \isacommand{by}\isamarkupfalse%
\ {\isacharparenleft}{\kern0pt}intro\ exI{\isacharbrackleft}{\kern0pt}\isakeyword{where}\ x{\isacharequal}{\kern0pt}\ {\isachardoublequoteopen}id\ X{\isachardoublequoteclose}{\isacharbrackright}{\kern0pt}{\isacharcomma}{\kern0pt}\ auto\ simp\ add{\isacharcolon}{\kern0pt}\ id{\isacharunderscore}{\kern0pt}domain\ id{\isacharunderscore}{\kern0pt}codomain\ id{\isacharunderscore}{\kern0pt}isomorphism\ id{\isacharunderscore}{\kern0pt}type{\isacharparenright}{\kern0pt}%
\endisatagproof
{\isafoldproof}%
%
\isadelimproof
\isanewline
%
\endisadelimproof
\isanewline
\isacommand{lemma}\isamarkupfalse%
\ isomorphic{\isacharunderscore}{\kern0pt}is{\isacharunderscore}{\kern0pt}symmetric{\isacharcolon}{\kern0pt}\ {\isachardoublequoteopen}X\ {\isasymcong}\ Y\ {\isasymlongrightarrow}\ Y\ {\isasymcong}\ X{\isachardoublequoteclose}\isanewline
%
\isadelimproof
\ \ %
\endisadelimproof
%
\isatagproof
\isacommand{unfolding}\isamarkupfalse%
\ is{\isacharunderscore}{\kern0pt}isomorphic{\isacharunderscore}{\kern0pt}def\ isomorphism{\isacharunderscore}{\kern0pt}def\ \isanewline
\ \ \isacommand{by}\isamarkupfalse%
\ {\isacharparenleft}{\kern0pt}auto{\isacharcomma}{\kern0pt}\ metis\ cfunc{\isacharunderscore}{\kern0pt}type{\isacharunderscore}{\kern0pt}def{\isacharparenright}{\kern0pt}%
\endisatagproof
{\isafoldproof}%
%
\isadelimproof
\isanewline
%
\endisadelimproof
\isanewline
\isacommand{lemma}\isamarkupfalse%
\ isomorphism{\isacharunderscore}{\kern0pt}comp{\isacharcolon}{\kern0pt}\ \isanewline
\ \ {\isachardoublequoteopen}domain\ f\ {\isacharequal}{\kern0pt}\ codomain\ g\ {\isasymLongrightarrow}\ isomorphism\ f\ {\isasymLongrightarrow}\ isomorphism\ g\ {\isasymLongrightarrow}\ isomorphism\ {\isacharparenleft}{\kern0pt}f\ {\isasymcirc}\isactrlsub c\ g{\isacharparenright}{\kern0pt}{\isachardoublequoteclose}\isanewline
%
\isadelimproof
\ \ %
\endisadelimproof
%
\isatagproof
\isacommand{unfolding}\isamarkupfalse%
\ isomorphism{\isacharunderscore}{\kern0pt}def\ \isacommand{by}\isamarkupfalse%
\ {\isacharparenleft}{\kern0pt}auto{\isacharcomma}{\kern0pt}\ smt\ codomain{\isacharunderscore}{\kern0pt}comp\ comp{\isacharunderscore}{\kern0pt}associative\ domain{\isacharunderscore}{\kern0pt}comp\ id{\isacharunderscore}{\kern0pt}right{\isacharunderscore}{\kern0pt}unit{\isacharparenright}{\kern0pt}%
\endisatagproof
{\isafoldproof}%
%
\isadelimproof
\isanewline
%
\endisadelimproof
\isanewline
\isacommand{lemma}\isamarkupfalse%
\ isomorphism{\isacharunderscore}{\kern0pt}comp{\isacharprime}{\kern0pt}{\isacharcolon}{\kern0pt}\ \isanewline
\ \ \isakeyword{assumes}\ {\isachardoublequoteopen}f\ {\isacharcolon}{\kern0pt}\ Y\ {\isasymrightarrow}\ Z{\isachardoublequoteclose}\ {\isachardoublequoteopen}g\ {\isacharcolon}{\kern0pt}\ X\ {\isasymrightarrow}\ Y{\isachardoublequoteclose}\isanewline
\ \ \isakeyword{shows}\ {\isachardoublequoteopen}isomorphism\ f\ {\isasymLongrightarrow}\ isomorphism\ g\ {\isasymLongrightarrow}\ isomorphism\ {\isacharparenleft}{\kern0pt}f\ {\isasymcirc}\isactrlsub c\ g{\isacharparenright}{\kern0pt}{\isachardoublequoteclose}\isanewline
%
\isadelimproof
\ \ %
\endisadelimproof
%
\isatagproof
\isacommand{using}\isamarkupfalse%
\ assms\ cfunc{\isacharunderscore}{\kern0pt}type{\isacharunderscore}{\kern0pt}def\ isomorphism{\isacharunderscore}{\kern0pt}comp\ \isacommand{by}\isamarkupfalse%
\ auto%
\endisatagproof
{\isafoldproof}%
%
\isadelimproof
\isanewline
%
\endisadelimproof
\isanewline
\isacommand{lemma}\isamarkupfalse%
\ isomorphic{\isacharunderscore}{\kern0pt}is{\isacharunderscore}{\kern0pt}transitive{\isacharcolon}{\kern0pt}\ {\isachardoublequoteopen}{\isacharparenleft}{\kern0pt}X\ {\isasymcong}\ Y\ {\isasymand}\ Y\ {\isasymcong}\ Z{\isacharparenright}{\kern0pt}\ {\isasymlongrightarrow}\ X\ {\isasymcong}\ Z{\isachardoublequoteclose}\isanewline
%
\isadelimproof
\ \ %
\endisadelimproof
%
\isatagproof
\isacommand{unfolding}\isamarkupfalse%
\ is{\isacharunderscore}{\kern0pt}isomorphic{\isacharunderscore}{\kern0pt}def\ \isacommand{by}\isamarkupfalse%
\ {\isacharparenleft}{\kern0pt}auto{\isacharcomma}{\kern0pt}\ metis\ cfunc{\isacharunderscore}{\kern0pt}type{\isacharunderscore}{\kern0pt}def\ comp{\isacharunderscore}{\kern0pt}type\ isomorphism{\isacharunderscore}{\kern0pt}comp{\isacharparenright}{\kern0pt}%
\endisatagproof
{\isafoldproof}%
%
\isadelimproof
\isanewline
%
\endisadelimproof
\isanewline
\isacommand{lemma}\isamarkupfalse%
\ is{\isacharunderscore}{\kern0pt}isomorphic{\isacharunderscore}{\kern0pt}equiv{\isacharcolon}{\kern0pt}\isanewline
\ \ {\isachardoublequoteopen}equiv\ UNIV\ {\isacharbraceleft}{\kern0pt}{\isacharparenleft}{\kern0pt}X{\isacharcomma}{\kern0pt}\ Y{\isacharparenright}{\kern0pt}{\isachardot}{\kern0pt}\ X\ {\isasymcong}\ Y{\isacharbraceright}{\kern0pt}{\isachardoublequoteclose}\isanewline
%
\isadelimproof
\ \ %
\endisadelimproof
%
\isatagproof
\isacommand{unfolding}\isamarkupfalse%
\ equiv{\isacharunderscore}{\kern0pt}def\isanewline
\isacommand{proof}\isamarkupfalse%
\ safe\isanewline
\ \ \isacommand{show}\isamarkupfalse%
\ {\isachardoublequoteopen}refl\ {\isacharbraceleft}{\kern0pt}{\isacharparenleft}{\kern0pt}x{\isacharcomma}{\kern0pt}\ y{\isacharparenright}{\kern0pt}{\isachardot}{\kern0pt}\ x\ {\isasymcong}\ y{\isacharbraceright}{\kern0pt}{\isachardoublequoteclose}\isanewline
\ \ \ \ \isacommand{unfolding}\isamarkupfalse%
\ refl{\isacharunderscore}{\kern0pt}on{\isacharunderscore}{\kern0pt}def\ \isacommand{using}\isamarkupfalse%
\ isomorphic{\isacharunderscore}{\kern0pt}is{\isacharunderscore}{\kern0pt}reflexive\ \isacommand{by}\isamarkupfalse%
\ auto\isanewline
\isacommand{next}\isamarkupfalse%
\isanewline
\ \ \isacommand{show}\isamarkupfalse%
\ {\isachardoublequoteopen}sym\ {\isacharbraceleft}{\kern0pt}{\isacharparenleft}{\kern0pt}x{\isacharcomma}{\kern0pt}\ y{\isacharparenright}{\kern0pt}{\isachardot}{\kern0pt}\ x\ {\isasymcong}\ y{\isacharbraceright}{\kern0pt}{\isachardoublequoteclose}\isanewline
\ \ \ \ \isacommand{unfolding}\isamarkupfalse%
\ sym{\isacharunderscore}{\kern0pt}def\ \isacommand{using}\isamarkupfalse%
\ isomorphic{\isacharunderscore}{\kern0pt}is{\isacharunderscore}{\kern0pt}symmetric\ \isacommand{by}\isamarkupfalse%
\ blast\isanewline
\isacommand{next}\isamarkupfalse%
\isanewline
\ \ \isacommand{show}\isamarkupfalse%
\ {\isachardoublequoteopen}trans\ {\isacharbraceleft}{\kern0pt}{\isacharparenleft}{\kern0pt}x{\isacharcomma}{\kern0pt}\ y{\isacharparenright}{\kern0pt}{\isachardot}{\kern0pt}\ x\ {\isasymcong}\ y{\isacharbraceright}{\kern0pt}{\isachardoublequoteclose}\isanewline
\ \ \ \ \isacommand{unfolding}\isamarkupfalse%
\ trans{\isacharunderscore}{\kern0pt}def\ \isacommand{using}\isamarkupfalse%
\ isomorphic{\isacharunderscore}{\kern0pt}is{\isacharunderscore}{\kern0pt}transitive\ \isacommand{by}\isamarkupfalse%
\ blast\isanewline
\isacommand{qed}\isamarkupfalse%
%
\endisatagproof
{\isafoldproof}%
%
\isadelimproof
%
\endisadelimproof
%
\begin{isamarkuptext}%
The lemma below corresponds to Exercise 2.1.7e in Halvorson.%
\end{isamarkuptext}\isamarkuptrue%
\isacommand{lemma}\isamarkupfalse%
\ iso{\isacharunderscore}{\kern0pt}imp{\isacharunderscore}{\kern0pt}epi{\isacharunderscore}{\kern0pt}and{\isacharunderscore}{\kern0pt}monic{\isacharcolon}{\kern0pt}\isanewline
\ \ {\isachardoublequoteopen}isomorphism\ f\ {\isasymLongrightarrow}\ epimorphism\ f\ {\isasymand}\ monomorphism\ f{\isachardoublequoteclose}\isanewline
%
\isadelimproof
\ \ %
\endisadelimproof
%
\isatagproof
\isacommand{unfolding}\isamarkupfalse%
\ isomorphism{\isacharunderscore}{\kern0pt}def\ epimorphism{\isacharunderscore}{\kern0pt}def\ monomorphism{\isacharunderscore}{\kern0pt}def\isanewline
\isacommand{proof}\isamarkupfalse%
\ safe\isanewline
\ \ \isacommand{fix}\isamarkupfalse%
\ g\ s\ t\isanewline
\ \ \isacommand{assume}\isamarkupfalse%
\ domain{\isacharunderscore}{\kern0pt}g{\isacharcolon}{\kern0pt}\ {\isachardoublequoteopen}domain\ g\ {\isacharequal}{\kern0pt}\ codomain\ f{\isachardoublequoteclose}\isanewline
\ \ \isacommand{assume}\isamarkupfalse%
\ codomain{\isacharunderscore}{\kern0pt}g{\isacharcolon}{\kern0pt}\ {\isachardoublequoteopen}codomain\ g\ {\isacharequal}{\kern0pt}\ domain\ f{\isachardoublequoteclose}\isanewline
\ \ \isacommand{assume}\isamarkupfalse%
\ gf{\isacharunderscore}{\kern0pt}id{\isacharcolon}{\kern0pt}\ {\isachardoublequoteopen}g\ {\isasymcirc}\isactrlsub c\ f\ {\isacharequal}{\kern0pt}\ id\ {\isacharparenleft}{\kern0pt}domain\ f{\isacharparenright}{\kern0pt}{\isachardoublequoteclose}\isanewline
\ \ \isacommand{assume}\isamarkupfalse%
\ fg{\isacharunderscore}{\kern0pt}id{\isacharcolon}{\kern0pt}\ {\isachardoublequoteopen}f\ {\isasymcirc}\isactrlsub c\ g\ {\isacharequal}{\kern0pt}\ id\ {\isacharparenleft}{\kern0pt}domain\ g{\isacharparenright}{\kern0pt}{\isachardoublequoteclose}\isanewline
\ \ \isacommand{assume}\isamarkupfalse%
\ domain{\isacharunderscore}{\kern0pt}s{\isacharcolon}{\kern0pt}\ {\isachardoublequoteopen}domain\ s\ {\isacharequal}{\kern0pt}\ codomain\ f{\isachardoublequoteclose}\isanewline
\ \ \isacommand{assume}\isamarkupfalse%
\ domain{\isacharunderscore}{\kern0pt}t{\isacharcolon}{\kern0pt}\ {\isachardoublequoteopen}domain\ t\ {\isacharequal}{\kern0pt}\ codomain\ f{\isachardoublequoteclose}\isanewline
\ \ \isacommand{assume}\isamarkupfalse%
\ sf{\isacharunderscore}{\kern0pt}eq{\isacharunderscore}{\kern0pt}tf{\isacharcolon}{\kern0pt}\ {\isachardoublequoteopen}s\ {\isasymcirc}\isactrlsub c\ f\ {\isacharequal}{\kern0pt}\ t\ {\isasymcirc}\isactrlsub c\ f{\isachardoublequoteclose}\isanewline
\isanewline
\ \ \isacommand{have}\isamarkupfalse%
\ {\isachardoublequoteopen}s\ {\isacharequal}{\kern0pt}\ s\ {\isasymcirc}\isactrlsub c\ id{\isacharparenleft}{\kern0pt}codomain{\isacharparenleft}{\kern0pt}f{\isacharparenright}{\kern0pt}{\isacharparenright}{\kern0pt}{\isachardoublequoteclose}\isanewline
\ \ \ \ \isacommand{by}\isamarkupfalse%
\ {\isacharparenleft}{\kern0pt}metis\ domain{\isacharunderscore}{\kern0pt}s\ id{\isacharunderscore}{\kern0pt}right{\isacharunderscore}{\kern0pt}unit{\isacharparenright}{\kern0pt}\isanewline
\ \ \isacommand{also}\isamarkupfalse%
\ \isacommand{have}\isamarkupfalse%
\ {\isachardoublequoteopen}{\isachardot}{\kern0pt}{\isachardot}{\kern0pt}{\isachardot}{\kern0pt}\ {\isacharequal}{\kern0pt}\ s\ {\isasymcirc}\isactrlsub c\ {\isacharparenleft}{\kern0pt}f\ {\isasymcirc}\isactrlsub c\ g{\isacharparenright}{\kern0pt}{\isachardoublequoteclose}\isanewline
\ \ \ \ \isacommand{by}\isamarkupfalse%
\ {\isacharparenleft}{\kern0pt}simp\ add{\isacharcolon}{\kern0pt}\ domain{\isacharunderscore}{\kern0pt}g\ fg{\isacharunderscore}{\kern0pt}id{\isacharparenright}{\kern0pt}\isanewline
\ \ \isacommand{also}\isamarkupfalse%
\ \isacommand{have}\isamarkupfalse%
\ {\isachardoublequoteopen}{\isachardot}{\kern0pt}{\isachardot}{\kern0pt}{\isachardot}{\kern0pt}\ {\isacharequal}{\kern0pt}\ {\isacharparenleft}{\kern0pt}s\ {\isasymcirc}\isactrlsub c\ f{\isacharparenright}{\kern0pt}\ {\isasymcirc}\isactrlsub c\ g{\isachardoublequoteclose}\isanewline
\ \ \ \ \isacommand{by}\isamarkupfalse%
\ {\isacharparenleft}{\kern0pt}simp\ add{\isacharcolon}{\kern0pt}\ codomain{\isacharunderscore}{\kern0pt}g\ comp{\isacharunderscore}{\kern0pt}associative\ domain{\isacharunderscore}{\kern0pt}s{\isacharparenright}{\kern0pt}\isanewline
\ \ \isacommand{also}\isamarkupfalse%
\ \isacommand{have}\isamarkupfalse%
\ {\isachardoublequoteopen}{\isachardot}{\kern0pt}{\isachardot}{\kern0pt}{\isachardot}{\kern0pt}\ {\isacharequal}{\kern0pt}\ {\isacharparenleft}{\kern0pt}t\ {\isasymcirc}\isactrlsub c\ f{\isacharparenright}{\kern0pt}\ {\isasymcirc}\isactrlsub c\ g{\isachardoublequoteclose}\isanewline
\ \ \ \ \isacommand{by}\isamarkupfalse%
\ {\isacharparenleft}{\kern0pt}simp\ add{\isacharcolon}{\kern0pt}\ sf{\isacharunderscore}{\kern0pt}eq{\isacharunderscore}{\kern0pt}tf{\isacharparenright}{\kern0pt}\isanewline
\ \ \isacommand{also}\isamarkupfalse%
\ \isacommand{have}\isamarkupfalse%
\ {\isachardoublequoteopen}{\isachardot}{\kern0pt}{\isachardot}{\kern0pt}{\isachardot}{\kern0pt}\ {\isacharequal}{\kern0pt}\ t\ {\isasymcirc}\isactrlsub c\ {\isacharparenleft}{\kern0pt}f\ {\isasymcirc}\isactrlsub c\ g{\isacharparenright}{\kern0pt}{\isachardoublequoteclose}\isanewline
\ \ \ \ \isacommand{by}\isamarkupfalse%
\ {\isacharparenleft}{\kern0pt}simp\ add{\isacharcolon}{\kern0pt}\ codomain{\isacharunderscore}{\kern0pt}g\ comp{\isacharunderscore}{\kern0pt}associative\ domain{\isacharunderscore}{\kern0pt}t{\isacharparenright}{\kern0pt}\isanewline
\ \ \isacommand{also}\isamarkupfalse%
\ \isacommand{have}\isamarkupfalse%
\ {\isachardoublequoteopen}{\isachardot}{\kern0pt}{\isachardot}{\kern0pt}{\isachardot}{\kern0pt}\ {\isacharequal}{\kern0pt}\ t\ {\isasymcirc}\isactrlsub c\ id{\isacharparenleft}{\kern0pt}codomain\ f{\isacharparenright}{\kern0pt}{\isachardoublequoteclose}\isanewline
\ \ \ \ \isacommand{by}\isamarkupfalse%
\ {\isacharparenleft}{\kern0pt}simp\ add{\isacharcolon}{\kern0pt}\ domain{\isacharunderscore}{\kern0pt}g\ fg{\isacharunderscore}{\kern0pt}id{\isacharparenright}{\kern0pt}\isanewline
\ \ \isacommand{also}\isamarkupfalse%
\ \isacommand{have}\isamarkupfalse%
\ {\isachardoublequoteopen}{\isachardot}{\kern0pt}{\isachardot}{\kern0pt}{\isachardot}{\kern0pt}\ {\isacharequal}{\kern0pt}\ t{\isachardoublequoteclose}\isanewline
\ \ \ \ \isacommand{by}\isamarkupfalse%
\ {\isacharparenleft}{\kern0pt}metis\ domain{\isacharunderscore}{\kern0pt}t\ id{\isacharunderscore}{\kern0pt}right{\isacharunderscore}{\kern0pt}unit{\isacharparenright}{\kern0pt}\ \ \ \ \isanewline
\ \ \isacommand{then}\isamarkupfalse%
\ \isacommand{show}\isamarkupfalse%
\ {\isachardoublequoteopen}s\ {\isacharequal}{\kern0pt}\ t{\isachardoublequoteclose}\isanewline
\ \ \ \ \isacommand{using}\isamarkupfalse%
\ calculation\ \isacommand{by}\isamarkupfalse%
\ auto\isanewline
\isacommand{next}\isamarkupfalse%
\isanewline
\ \ \isacommand{fix}\isamarkupfalse%
\ g\ h\ k\isanewline
\ \ \isacommand{assume}\isamarkupfalse%
\ domain{\isacharunderscore}{\kern0pt}g{\isacharcolon}{\kern0pt}\ {\isachardoublequoteopen}domain\ g\ {\isacharequal}{\kern0pt}\ codomain\ f{\isachardoublequoteclose}\isanewline
\ \ \isacommand{assume}\isamarkupfalse%
\ codomain{\isacharunderscore}{\kern0pt}g{\isacharcolon}{\kern0pt}\ {\isachardoublequoteopen}codomain\ g\ {\isacharequal}{\kern0pt}\ domain\ f{\isachardoublequoteclose}\isanewline
\ \ \isacommand{assume}\isamarkupfalse%
\ gf{\isacharunderscore}{\kern0pt}id{\isacharcolon}{\kern0pt}\ {\isachardoublequoteopen}g\ {\isasymcirc}\isactrlsub c\ f\ {\isacharequal}{\kern0pt}\ id\ {\isacharparenleft}{\kern0pt}domain\ f{\isacharparenright}{\kern0pt}{\isachardoublequoteclose}\isanewline
\ \ \isacommand{assume}\isamarkupfalse%
\ fg{\isacharunderscore}{\kern0pt}id{\isacharcolon}{\kern0pt}\ {\isachardoublequoteopen}f\ {\isasymcirc}\isactrlsub c\ g\ {\isacharequal}{\kern0pt}\ id\ {\isacharparenleft}{\kern0pt}domain\ g{\isacharparenright}{\kern0pt}{\isachardoublequoteclose}\isanewline
\ \ \isacommand{assume}\isamarkupfalse%
\ codomain{\isacharunderscore}{\kern0pt}h{\isacharcolon}{\kern0pt}\ {\isachardoublequoteopen}codomain\ h\ {\isacharequal}{\kern0pt}\ domain\ f{\isachardoublequoteclose}\isanewline
\ \ \isacommand{assume}\isamarkupfalse%
\ codomain{\isacharunderscore}{\kern0pt}k{\isacharcolon}{\kern0pt}\ {\isachardoublequoteopen}codomain\ k\ {\isacharequal}{\kern0pt}\ domain\ f{\isachardoublequoteclose}\isanewline
\ \ \isacommand{assume}\isamarkupfalse%
\ fk{\isacharunderscore}{\kern0pt}eq{\isacharunderscore}{\kern0pt}fh{\isacharcolon}{\kern0pt}\ {\isachardoublequoteopen}f\ {\isasymcirc}\isactrlsub c\ k\ {\isacharequal}{\kern0pt}\ f\ {\isasymcirc}\isactrlsub c\ h{\isachardoublequoteclose}\isanewline
\isanewline
\ \ \isacommand{have}\isamarkupfalse%
\ {\isachardoublequoteopen}h\ {\isacharequal}{\kern0pt}\ id{\isacharparenleft}{\kern0pt}domain\ f{\isacharparenright}{\kern0pt}\ {\isasymcirc}\isactrlsub c\ h{\isachardoublequoteclose}\isanewline
\ \ \ \ \isacommand{by}\isamarkupfalse%
\ {\isacharparenleft}{\kern0pt}metis\ codomain{\isacharunderscore}{\kern0pt}h\ id{\isacharunderscore}{\kern0pt}left{\isacharunderscore}{\kern0pt}unit{\isacharparenright}{\kern0pt}\isanewline
\ \ \isacommand{also}\isamarkupfalse%
\ \isacommand{have}\isamarkupfalse%
\ {\isachardoublequoteopen}{\isachardot}{\kern0pt}{\isachardot}{\kern0pt}{\isachardot}{\kern0pt}\ {\isacharequal}{\kern0pt}\ {\isacharparenleft}{\kern0pt}g\ {\isasymcirc}\isactrlsub c\ f{\isacharparenright}{\kern0pt}\ {\isasymcirc}\isactrlsub c\ h{\isachardoublequoteclose}\isanewline
\ \ \ \ \isacommand{using}\isamarkupfalse%
\ gf{\isacharunderscore}{\kern0pt}id\ \isacommand{by}\isamarkupfalse%
\ auto\isanewline
\ \ \isacommand{also}\isamarkupfalse%
\ \isacommand{have}\isamarkupfalse%
\ {\isachardoublequoteopen}{\isachardot}{\kern0pt}{\isachardot}{\kern0pt}{\isachardot}{\kern0pt}\ {\isacharequal}{\kern0pt}\ g\ {\isasymcirc}\isactrlsub c\ {\isacharparenleft}{\kern0pt}f\ {\isasymcirc}\isactrlsub c\ h{\isacharparenright}{\kern0pt}{\isachardoublequoteclose}\isanewline
\ \ \ \ \isacommand{by}\isamarkupfalse%
\ {\isacharparenleft}{\kern0pt}simp\ add{\isacharcolon}{\kern0pt}\ codomain{\isacharunderscore}{\kern0pt}h\ comp{\isacharunderscore}{\kern0pt}associative\ domain{\isacharunderscore}{\kern0pt}g{\isacharparenright}{\kern0pt}\isanewline
\ \ \isacommand{also}\isamarkupfalse%
\ \isacommand{have}\isamarkupfalse%
\ {\isachardoublequoteopen}{\isachardot}{\kern0pt}{\isachardot}{\kern0pt}{\isachardot}{\kern0pt}\ {\isacharequal}{\kern0pt}\ g\ {\isasymcirc}\isactrlsub c\ {\isacharparenleft}{\kern0pt}f\ {\isasymcirc}\isactrlsub c\ k{\isacharparenright}{\kern0pt}{\isachardoublequoteclose}\isanewline
\ \ \ \ \isacommand{by}\isamarkupfalse%
\ {\isacharparenleft}{\kern0pt}simp\ add{\isacharcolon}{\kern0pt}\ fk{\isacharunderscore}{\kern0pt}eq{\isacharunderscore}{\kern0pt}fh{\isacharparenright}{\kern0pt}\isanewline
\ \ \isacommand{also}\isamarkupfalse%
\ \isacommand{have}\isamarkupfalse%
\ {\isachardoublequoteopen}{\isachardot}{\kern0pt}{\isachardot}{\kern0pt}{\isachardot}{\kern0pt}\ {\isacharequal}{\kern0pt}\ {\isacharparenleft}{\kern0pt}g\ {\isasymcirc}\isactrlsub c\ f{\isacharparenright}{\kern0pt}\ {\isasymcirc}\isactrlsub c\ k{\isachardoublequoteclose}\isanewline
\ \ \ \ \isacommand{by}\isamarkupfalse%
\ {\isacharparenleft}{\kern0pt}simp\ add{\isacharcolon}{\kern0pt}\ codomain{\isacharunderscore}{\kern0pt}k\ comp{\isacharunderscore}{\kern0pt}associative\ domain{\isacharunderscore}{\kern0pt}g{\isacharparenright}{\kern0pt}\isanewline
\ \ \isacommand{also}\isamarkupfalse%
\ \isacommand{have}\isamarkupfalse%
\ {\isachardoublequoteopen}{\isachardot}{\kern0pt}{\isachardot}{\kern0pt}{\isachardot}{\kern0pt}\ {\isacharequal}{\kern0pt}\ id{\isacharparenleft}{\kern0pt}domain\ f{\isacharparenright}{\kern0pt}\ {\isasymcirc}\isactrlsub c\ k{\isachardoublequoteclose}\isanewline
\ \ \ \ \isacommand{by}\isamarkupfalse%
\ {\isacharparenleft}{\kern0pt}simp\ add{\isacharcolon}{\kern0pt}\ gf{\isacharunderscore}{\kern0pt}id{\isacharparenright}{\kern0pt}\isanewline
\ \ \isacommand{also}\isamarkupfalse%
\ \isacommand{have}\isamarkupfalse%
\ {\isachardoublequoteopen}{\isachardot}{\kern0pt}{\isachardot}{\kern0pt}{\isachardot}{\kern0pt}\ {\isacharequal}{\kern0pt}\ k{\isachardoublequoteclose}\isanewline
\ \ \ \ \isacommand{by}\isamarkupfalse%
\ {\isacharparenleft}{\kern0pt}metis\ codomain{\isacharunderscore}{\kern0pt}k\ id{\isacharunderscore}{\kern0pt}left{\isacharunderscore}{\kern0pt}unit{\isacharparenright}{\kern0pt}\isanewline
\ \ \isacommand{then}\isamarkupfalse%
\ \isacommand{show}\isamarkupfalse%
\ {\isachardoublequoteopen}k\ {\isacharequal}{\kern0pt}\ h{\isachardoublequoteclose}\isanewline
\ \ \ \ \isacommand{using}\isamarkupfalse%
\ calculation\ \isacommand{by}\isamarkupfalse%
\ auto\isanewline
\isacommand{qed}\isamarkupfalse%
%
\endisatagproof
{\isafoldproof}%
%
\isadelimproof
\isanewline
%
\endisadelimproof
\isanewline
\isacommand{lemma}\isamarkupfalse%
\ isomorphism{\isacharunderscore}{\kern0pt}sandwich{\isacharcolon}{\kern0pt}\ \isanewline
\ \ \isakeyword{assumes}\ f{\isacharunderscore}{\kern0pt}type{\isacharcolon}{\kern0pt}\ {\isachardoublequoteopen}f\ {\isacharcolon}{\kern0pt}\ A\ {\isasymrightarrow}\ B{\isachardoublequoteclose}\ \isakeyword{and}\ g{\isacharunderscore}{\kern0pt}type{\isacharcolon}{\kern0pt}\ {\isachardoublequoteopen}g\ {\isacharcolon}{\kern0pt}\ B\ {\isasymrightarrow}\ C{\isachardoublequoteclose}\ \isakeyword{and}\ h{\isacharunderscore}{\kern0pt}type{\isacharcolon}{\kern0pt}\ {\isachardoublequoteopen}h{\isacharcolon}{\kern0pt}\ C\ {\isasymrightarrow}\ D{\isachardoublequoteclose}\isanewline
\ \ \isakeyword{assumes}\ f{\isacharunderscore}{\kern0pt}iso{\isacharcolon}{\kern0pt}\ {\isachardoublequoteopen}isomorphism\ f{\isachardoublequoteclose}\isanewline
\ \ \isakeyword{assumes}\ h{\isacharunderscore}{\kern0pt}iso{\isacharcolon}{\kern0pt}\ {\isachardoublequoteopen}isomorphism\ h{\isachardoublequoteclose}\isanewline
\ \ \isakeyword{assumes}\ hgf{\isacharunderscore}{\kern0pt}iso{\isacharcolon}{\kern0pt}\ {\isachardoublequoteopen}isomorphism{\isacharparenleft}{\kern0pt}h\ {\isasymcirc}\isactrlsub c\ g\ {\isasymcirc}\isactrlsub c\ f{\isacharparenright}{\kern0pt}{\isachardoublequoteclose}\isanewline
\ \ \isakeyword{shows}\ {\isachardoublequoteopen}isomorphism\ g{\isachardoublequoteclose}\isanewline
%
\isadelimproof
%
\endisadelimproof
%
\isatagproof
\isacommand{proof}\isamarkupfalse%
\ {\isacharminus}{\kern0pt}\isanewline
\ \ \isacommand{have}\isamarkupfalse%
\ {\isachardoublequoteopen}isomorphism{\isacharparenleft}{\kern0pt}h\isactrlbold {\isasyminverse}\ {\isasymcirc}\isactrlsub c\ {\isacharparenleft}{\kern0pt}h\ {\isasymcirc}\isactrlsub c\ g\ {\isasymcirc}\isactrlsub c\ f{\isacharparenright}{\kern0pt}\ {\isasymcirc}\isactrlsub c\ f\isactrlbold {\isasyminverse}{\isacharparenright}{\kern0pt}{\isachardoublequoteclose}\isanewline
\ \ \ \ \isacommand{using}\isamarkupfalse%
\ assms\ \isacommand{by}\isamarkupfalse%
\ {\isacharparenleft}{\kern0pt}typecheck{\isacharunderscore}{\kern0pt}cfuncs{\isacharcomma}{\kern0pt}\ simp\ add{\isacharcolon}{\kern0pt}\ f{\isacharunderscore}{\kern0pt}iso\ h{\isacharunderscore}{\kern0pt}iso\ hgf{\isacharunderscore}{\kern0pt}iso\ inv{\isacharunderscore}{\kern0pt}iso\ isomorphism{\isacharunderscore}{\kern0pt}comp{\isacharprime}{\kern0pt}{\isacharparenright}{\kern0pt}\isanewline
\ \ \isacommand{then}\isamarkupfalse%
\ \isacommand{show}\isamarkupfalse%
\ {\isachardoublequoteopen}isomorphism\ g{\isachardoublequoteclose}\isanewline
\ \ \ \ \isacommand{using}\isamarkupfalse%
\ assms\ \isacommand{by}\isamarkupfalse%
\ {\isacharparenleft}{\kern0pt}typecheck{\isacharunderscore}{\kern0pt}cfuncs{\isacharunderscore}{\kern0pt}prems{\isacharcomma}{\kern0pt}\ smt\ comp{\isacharunderscore}{\kern0pt}associative{\isadigit{2}}\ id{\isacharunderscore}{\kern0pt}left{\isacharunderscore}{\kern0pt}unit{\isadigit{2}}\ id{\isacharunderscore}{\kern0pt}right{\isacharunderscore}{\kern0pt}unit{\isadigit{2}}\ inv{\isacharunderscore}{\kern0pt}left\ inv{\isacharunderscore}{\kern0pt}right{\isacharparenright}{\kern0pt}\isanewline
\isacommand{qed}\isamarkupfalse%
%
\endisatagproof
{\isafoldproof}%
%
\isadelimproof
\isanewline
%
\endisadelimproof
%
\isadelimtheory
\isanewline
%
\endisadelimtheory
%
\isatagtheory
\isacommand{end}\isamarkupfalse%
%
\endisatagtheory
{\isafoldtheory}%
%
\isadelimtheory
%
\endisadelimtheory
%
\end{isabellebody}%
\endinput
%:%file=~/ETCS/Category_Set/Cfunc.thy%:%
%:%11=1%:%
%:%27=3%:%
%:%28=3%:%
%:%29=4%:%
%:%30=5%:%
%:%35=5%:%
%:%38=6%:%
%:%39=7%:%
%:%40=7%:%
%:%41=8%:%
%:%42=8%:%
%:%44=10%:%
%:%45=11%:%
%:%46=12%:%
%:%47=13%:%
%:%51=15%:%
%:%53=16%:%
%:%54=16%:%
%:%55=17%:%
%:%56=18%:%
%:%57=19%:%
%:%58=20%:%
%:%59=21%:%
%:%60=22%:%
%:%61=23%:%
%:%62=24%:%
%:%63=25%:%
%:%64=26%:%
%:%65=27%:%
%:%66=28%:%
%:%68=31%:%
%:%70=32%:%
%:%71=32%:%
%:%72=33%:%
%:%73=34%:%
%:%74=35%:%
%:%75=35%:%
%:%76=36%:%
%:%79=37%:%
%:%83=37%:%
%:%84=37%:%
%:%89=37%:%
%:%92=38%:%
%:%93=39%:%
%:%94=39%:%
%:%95=40%:%
%:%98=41%:%
%:%102=41%:%
%:%103=41%:%
%:%108=41%:%
%:%111=42%:%
%:%112=43%:%
%:%113=43%:%
%:%116=44%:%
%:%120=44%:%
%:%121=44%:%
%:%122=44%:%
%:%123=44%:%
%:%128=44%:%
%:%131=45%:%
%:%132=46%:%
%:%133=46%:%
%:%136=47%:%
%:%140=47%:%
%:%141=47%:%
%:%142=47%:%
%:%143=47%:%
%:%148=47%:%
%:%151=48%:%
%:%152=49%:%
%:%153=49%:%
%:%156=50%:%
%:%160=50%:%
%:%161=50%:%
%:%162=50%:%
%:%163=50%:%
%:%177=52%:%
%:%189=54%:%
%:%190=55%:%
%:%191=56%:%
%:%192=57%:%
%:%193=58%:%
%:%195=60%:%
%:%196=60%:%
%:%197=61%:%
%:%198=62%:%
%:%199=62%:%
%:%200=63%:%
%:%201=63%:%
%:%204=64%:%
%:%209=65%:%
%:%210=65%:%
%:%224=67%:%
%:%240=69%:%
%:%241=69%:%
%:%242=70%:%
%:%243=71%:%
%:%244=72%:%
%:%245=73%:%
%:%246=74%:%
%:%247=74%:%
%:%248=75%:%
%:%249=76%:%
%:%250=77%:%
%:%251=78%:%
%:%252=79%:%
%:%253=79%:%
%:%254=80%:%
%:%255=81%:%
%:%256=82%:%
%:%270=84%:%
%:%286=86%:%
%:%287=86%:%
%:%288=87%:%
%:%290=89%:%
%:%291=90%:%
%:%305=92%:%
%:%321=94%:%
%:%322=94%:%
%:%323=95%:%
%:%325=97%:%
%:%326=98%:%
%:%331=98%:%
%:%334=99%:%
%:%335=100%:%
%:%336=100%:%
%:%337=101%:%
%:%338=101%:%
%:%341=102%:%
%:%346=103%:%
%:%347=103%:%
%:%348=104%:%
%:%350=106%:%
%:%351=107%:%
%:%356=107%:%
%:%359=108%:%
%:%360=109%:%
%:%361=109%:%
%:%362=110%:%
%:%363=110%:%
%:%370=112%:%
%:%386=114%:%
%:%387=114%:%
%:%388=115%:%
%:%390=117%:%
%:%391=118%:%
%:%405=120%:%
%:%409=122%:%
%:%419=124%:%
%:%420=124%:%
%:%421=125%:%
%:%422=126%:%
%:%423=127%:%
%:%424=128%:%
%:%425=128%:%
%:%426=129%:%
%:%429=130%:%
%:%433=130%:%
%:%434=130%:%
%:%435=130%:%
%:%440=130%:%
%:%443=131%:%
%:%444=132%:%
%:%445=132%:%
%:%446=133%:%
%:%447=134%:%
%:%450=135%:%
%:%454=135%:%
%:%455=135%:%
%:%456=135%:%
%:%457=135%:%
%:%466=137%:%
%:%468=138%:%
%:%469=138%:%
%:%470=139%:%
%:%471=140%:%
%:%474=141%:%
%:%478=141%:%
%:%479=141%:%
%:%480=142%:%
%:%481=142%:%
%:%482=143%:%
%:%483=143%:%
%:%484=144%:%
%:%485=144%:%
%:%487=146%:%
%:%488=147%:%
%:%489=147%:%
%:%490=148%:%
%:%491=148%:%
%:%492=149%:%
%:%493=149%:%
%:%494=150%:%
%:%495=151%:%
%:%496=151%:%
%:%497=151%:%
%:%498=152%:%
%:%499=152%:%
%:%500=153%:%
%:%501=153%:%
%:%502=153%:%
%:%503=154%:%
%:%504=154%:%
%:%505=154%:%
%:%506=155%:%
%:%512=155%:%
%:%515=156%:%
%:%516=157%:%
%:%517=157%:%
%:%518=158%:%
%:%519=159%:%
%:%522=160%:%
%:%526=160%:%
%:%527=160%:%
%:%536=162%:%
%:%538=163%:%
%:%539=163%:%
%:%540=164%:%
%:%541=165%:%
%:%544=166%:%
%:%548=166%:%
%:%549=166%:%
%:%550=167%:%
%:%551=167%:%
%:%552=168%:%
%:%553=168%:%
%:%554=169%:%
%:%555=169%:%
%:%556=170%:%
%:%557=171%:%
%:%558=171%:%
%:%559=172%:%
%:%560=173%:%
%:%561=173%:%
%:%562=174%:%
%:%563=174%:%
%:%564=175%:%
%:%565=175%:%
%:%566=176%:%
%:%567=177%:%
%:%568=177%:%
%:%569=178%:%
%:%570=178%:%
%:%571=179%:%
%:%572=179%:%
%:%573=179%:%
%:%574=180%:%
%:%575=180%:%
%:%576=181%:%
%:%577=181%:%
%:%578=181%:%
%:%579=182%:%
%:%580=182%:%
%:%581=183%:%
%:%582=183%:%
%:%583=183%:%
%:%584=184%:%
%:%585=184%:%
%:%586=184%:%
%:%587=185%:%
%:%588=185%:%
%:%589=185%:%
%:%590=186%:%
%:%591=186%:%
%:%592=187%:%
%:%607=189%:%
%:%617=191%:%
%:%618=191%:%
%:%619=192%:%
%:%620=193%:%
%:%621=194%:%
%:%622=195%:%
%:%623=195%:%
%:%624=196%:%
%:%627=197%:%
%:%631=197%:%
%:%632=197%:%
%:%633=197%:%
%:%638=197%:%
%:%641=198%:%
%:%642=199%:%
%:%643=199%:%
%:%644=200%:%
%:%645=201%:%
%:%648=202%:%
%:%652=202%:%
%:%653=202%:%
%:%654=202%:%
%:%655=202%:%
%:%664=204%:%
%:%666=205%:%
%:%667=205%:%
%:%668=206%:%
%:%669=207%:%
%:%672=208%:%
%:%676=208%:%
%:%677=208%:%
%:%678=209%:%
%:%679=209%:%
%:%680=210%:%
%:%681=210%:%
%:%682=211%:%
%:%683=211%:%
%:%685=213%:%
%:%686=214%:%
%:%687=214%:%
%:%688=215%:%
%:%689=215%:%
%:%690=216%:%
%:%691=216%:%
%:%692=217%:%
%:%693=218%:%
%:%694=218%:%
%:%695=218%:%
%:%696=219%:%
%:%697=219%:%
%:%698=220%:%
%:%699=220%:%
%:%700=220%:%
%:%701=221%:%
%:%702=221%:%
%:%703=221%:%
%:%704=222%:%
%:%714=224%:%
%:%716=225%:%
%:%717=225%:%
%:%718=226%:%
%:%719=227%:%
%:%722=228%:%
%:%726=228%:%
%:%727=228%:%
%:%728=229%:%
%:%729=229%:%
%:%730=230%:%
%:%731=230%:%
%:%732=231%:%
%:%733=231%:%
%:%734=232%:%
%:%735=233%:%
%:%736=233%:%
%:%737=234%:%
%:%738=235%:%
%:%739=235%:%
%:%740=236%:%
%:%741=236%:%
%:%742=237%:%
%:%743=237%:%
%:%744=238%:%
%:%745=239%:%
%:%746=239%:%
%:%747=240%:%
%:%748=240%:%
%:%749=241%:%
%:%750=241%:%
%:%751=241%:%
%:%752=242%:%
%:%753=242%:%
%:%754=243%:%
%:%755=243%:%
%:%756=243%:%
%:%757=244%:%
%:%758=244%:%
%:%759=245%:%
%:%760=246%:%
%:%761=246%:%
%:%762=246%:%
%:%763=247%:%
%:%764=247%:%
%:%765=248%:%
%:%766=248%:%
%:%767=248%:%
%:%768=249%:%
%:%769=249%:%
%:%770=250%:%
%:%785=252%:%
%:%795=254%:%
%:%796=254%:%
%:%797=255%:%
%:%798=256%:%
%:%799=257%:%
%:%800=258%:%
%:%801=258%:%
%:%802=259%:%
%:%805=260%:%
%:%809=260%:%
%:%810=260%:%
%:%811=260%:%
%:%816=260%:%
%:%819=261%:%
%:%820=262%:%
%:%821=262%:%
%:%822=263%:%
%:%823=264%:%
%:%826=265%:%
%:%830=265%:%
%:%831=265%:%
%:%832=265%:%
%:%833=265%:%
%:%838=265%:%
%:%841=266%:%
%:%842=267%:%
%:%843=267%:%
%:%844=268%:%
%:%845=269%:%
%:%846=270%:%
%:%847=270%:%
%:%848=271%:%
%:%849=272%:%
%:%852=273%:%
%:%856=273%:%
%:%857=273%:%
%:%858=274%:%
%:%859=274%:%
%:%860=275%:%
%:%861=275%:%
%:%862=276%:%
%:%863=276%:%
%:%864=276%:%
%:%865=276%:%
%:%866=277%:%
%:%867=277%:%
%:%868=278%:%
%:%869=278%:%
%:%870=279%:%
%:%871=279%:%
%:%872=280%:%
%:%873=280%:%
%:%874=281%:%
%:%875=281%:%
%:%876=282%:%
%:%877=282%:%
%:%878=282%:%
%:%879=283%:%
%:%880=283%:%
%:%881=283%:%
%:%882=284%:%
%:%883=284%:%
%:%884=284%:%
%:%885=285%:%
%:%886=285%:%
%:%887=286%:%
%:%893=286%:%
%:%896=287%:%
%:%897=288%:%
%:%898=288%:%
%:%899=289%:%
%:%900=290%:%
%:%903=291%:%
%:%907=291%:%
%:%908=291%:%
%:%909=291%:%
%:%910=291%:%
%:%915=291%:%
%:%918=292%:%
%:%919=293%:%
%:%920=293%:%
%:%921=294%:%
%:%922=295%:%
%:%925=296%:%
%:%929=296%:%
%:%930=296%:%
%:%931=296%:%
%:%932=296%:%
%:%937=296%:%
%:%940=297%:%
%:%941=298%:%
%:%942=298%:%
%:%943=299%:%
%:%944=300%:%
%:%947=301%:%
%:%951=301%:%
%:%952=301%:%
%:%953=301%:%
%:%954=301%:%
%:%959=301%:%
%:%962=302%:%
%:%963=303%:%
%:%964=303%:%
%:%965=304%:%
%:%966=305%:%
%:%969=306%:%
%:%973=306%:%
%:%974=306%:%
%:%975=306%:%
%:%976=306%:%
%:%981=306%:%
%:%984=307%:%
%:%985=308%:%
%:%986=308%:%
%:%987=309%:%
%:%988=310%:%
%:%991=311%:%
%:%995=311%:%
%:%996=311%:%
%:%1001=311%:%
%:%1004=312%:%
%:%1005=313%:%
%:%1006=313%:%
%:%1007=314%:%
%:%1008=315%:%
%:%1009=316%:%
%:%1010=316%:%
%:%1013=317%:%
%:%1017=317%:%
%:%1018=317%:%
%:%1019=318%:%
%:%1020=318%:%
%:%1025=318%:%
%:%1028=319%:%
%:%1029=320%:%
%:%1030=320%:%
%:%1033=321%:%
%:%1037=321%:%
%:%1038=321%:%
%:%1039=322%:%
%:%1040=322%:%
%:%1045=322%:%
%:%1048=323%:%
%:%1049=324%:%
%:%1050=324%:%
%:%1053=325%:%
%:%1057=325%:%
%:%1058=325%:%
%:%1059=326%:%
%:%1060=326%:%
%:%1065=326%:%
%:%1068=327%:%
%:%1069=328%:%
%:%1070=328%:%
%:%1071=329%:%
%:%1074=330%:%
%:%1078=330%:%
%:%1079=330%:%
%:%1080=330%:%
%:%1085=330%:%
%:%1088=331%:%
%:%1089=332%:%
%:%1090=332%:%
%:%1091=333%:%
%:%1092=334%:%
%:%1095=335%:%
%:%1099=335%:%
%:%1100=335%:%
%:%1101=335%:%
%:%1106=335%:%
%:%1109=336%:%
%:%1110=337%:%
%:%1111=337%:%
%:%1114=338%:%
%:%1118=338%:%
%:%1119=338%:%
%:%1120=338%:%
%:%1125=338%:%
%:%1128=339%:%
%:%1129=340%:%
%:%1130=340%:%
%:%1131=341%:%
%:%1134=342%:%
%:%1138=342%:%
%:%1139=342%:%
%:%1140=343%:%
%:%1141=343%:%
%:%1142=344%:%
%:%1143=344%:%
%:%1144=345%:%
%:%1145=345%:%
%:%1146=345%:%
%:%1147=345%:%
%:%1148=346%:%
%:%1149=346%:%
%:%1150=347%:%
%:%1151=347%:%
%:%1152=348%:%
%:%1153=348%:%
%:%1154=348%:%
%:%1155=348%:%
%:%1156=349%:%
%:%1157=349%:%
%:%1158=350%:%
%:%1159=350%:%
%:%1160=351%:%
%:%1161=351%:%
%:%1162=351%:%
%:%1163=351%:%
%:%1164=352%:%
%:%1174=354%:%
%:%1176=355%:%
%:%1177=355%:%
%:%1178=356%:%
%:%1181=357%:%
%:%1185=357%:%
%:%1186=357%:%
%:%1187=358%:%
%:%1188=358%:%
%:%1189=359%:%
%:%1190=359%:%
%:%1191=360%:%
%:%1192=360%:%
%:%1193=361%:%
%:%1194=361%:%
%:%1195=362%:%
%:%1196=362%:%
%:%1197=363%:%
%:%1198=363%:%
%:%1199=364%:%
%:%1200=364%:%
%:%1201=365%:%
%:%1202=365%:%
%:%1203=366%:%
%:%1204=366%:%
%:%1205=367%:%
%:%1206=368%:%
%:%1207=368%:%
%:%1208=369%:%
%:%1209=369%:%
%:%1210=370%:%
%:%1211=370%:%
%:%1212=370%:%
%:%1213=371%:%
%:%1214=371%:%
%:%1215=372%:%
%:%1216=372%:%
%:%1217=372%:%
%:%1218=373%:%
%:%1219=373%:%
%:%1220=374%:%
%:%1221=374%:%
%:%1222=374%:%
%:%1223=375%:%
%:%1224=375%:%
%:%1225=376%:%
%:%1226=376%:%
%:%1227=376%:%
%:%1228=377%:%
%:%1229=377%:%
%:%1230=378%:%
%:%1231=378%:%
%:%1232=378%:%
%:%1233=379%:%
%:%1234=379%:%
%:%1235=380%:%
%:%1236=380%:%
%:%1237=380%:%
%:%1238=381%:%
%:%1239=381%:%
%:%1240=382%:%
%:%1241=382%:%
%:%1242=382%:%
%:%1243=383%:%
%:%1244=383%:%
%:%1245=383%:%
%:%1246=384%:%
%:%1247=384%:%
%:%1248=385%:%
%:%1249=385%:%
%:%1250=386%:%
%:%1251=386%:%
%:%1252=387%:%
%:%1253=387%:%
%:%1254=388%:%
%:%1255=388%:%
%:%1256=389%:%
%:%1257=389%:%
%:%1258=390%:%
%:%1259=390%:%
%:%1260=391%:%
%:%1261=391%:%
%:%1262=392%:%
%:%1263=392%:%
%:%1264=393%:%
%:%1265=394%:%
%:%1266=394%:%
%:%1267=395%:%
%:%1268=395%:%
%:%1269=396%:%
%:%1270=396%:%
%:%1271=396%:%
%:%1272=397%:%
%:%1273=397%:%
%:%1274=397%:%
%:%1275=398%:%
%:%1276=398%:%
%:%1277=398%:%
%:%1278=399%:%
%:%1279=399%:%
%:%1280=400%:%
%:%1281=400%:%
%:%1282=400%:%
%:%1283=401%:%
%:%1284=401%:%
%:%1285=402%:%
%:%1286=402%:%
%:%1287=402%:%
%:%1288=403%:%
%:%1289=403%:%
%:%1290=404%:%
%:%1291=404%:%
%:%1292=404%:%
%:%1293=405%:%
%:%1294=405%:%
%:%1295=406%:%
%:%1296=406%:%
%:%1297=406%:%
%:%1298=407%:%
%:%1299=407%:%
%:%1300=408%:%
%:%1301=408%:%
%:%1302=408%:%
%:%1303=409%:%
%:%1304=409%:%
%:%1305=409%:%
%:%1306=410%:%
%:%1312=410%:%
%:%1315=411%:%
%:%1316=412%:%
%:%1317=412%:%
%:%1318=413%:%
%:%1319=414%:%
%:%1320=415%:%
%:%1321=416%:%
%:%1322=417%:%
%:%1329=418%:%
%:%1330=418%:%
%:%1331=419%:%
%:%1332=419%:%
%:%1333=420%:%
%:%1334=420%:%
%:%1335=420%:%
%:%1336=421%:%
%:%1337=421%:%
%:%1338=421%:%
%:%1339=422%:%
%:%1340=422%:%
%:%1341=422%:%
%:%1342=423%:%
%:%1348=423%:%
%:%1353=424%:%
%:%1358=425%:%

%
\begin{isabellebody}%
\setisabellecontext{Product}%
%
\isadelimdocument
%
\endisadelimdocument
%
\isatagdocument
%
\isamarkupsection{Cartesian Products of Sets%
}
\isamarkuptrue%
%
\endisatagdocument
{\isafolddocument}%
%
\isadelimdocument
%
\endisadelimdocument
%
\isadelimtheory
%
\endisadelimtheory
%
\isatagtheory
\isacommand{theory}\isamarkupfalse%
\ Product\isanewline
\ \ \isakeyword{imports}\ Cfunc\isanewline
\isakeyword{begin}%
\endisatagtheory
{\isafoldtheory}%
%
\isadelimtheory
%
\endisadelimtheory
%
\begin{isamarkuptext}%
The axiomatization below corresponds to Axiom 2 (Cartesian Products) in Halvorson.%
\end{isamarkuptext}\isamarkuptrue%
\isacommand{axiomatization}\isamarkupfalse%
\isanewline
\ \ cart{\isacharunderscore}{\kern0pt}prod\ {\isacharcolon}{\kern0pt}{\isacharcolon}{\kern0pt}\ {\isachardoublequoteopen}cset\ {\isasymRightarrow}\ cset\ {\isasymRightarrow}\ cset{\isachardoublequoteclose}\ {\isacharparenleft}{\kern0pt}\isakeyword{infixr}\ {\isachardoublequoteopen}{\isasymtimes}\isactrlsub c{\isachardoublequoteclose}\ {\isadigit{6}}{\isadigit{5}}{\isacharparenright}{\kern0pt}\ \isakeyword{and}\isanewline
\ \ left{\isacharunderscore}{\kern0pt}cart{\isacharunderscore}{\kern0pt}proj\ {\isacharcolon}{\kern0pt}{\isacharcolon}{\kern0pt}\ {\isachardoublequoteopen}cset\ {\isasymRightarrow}\ cset\ {\isasymRightarrow}\ cfunc{\isachardoublequoteclose}\ \isakeyword{and}\isanewline
\ \ right{\isacharunderscore}{\kern0pt}cart{\isacharunderscore}{\kern0pt}proj\ {\isacharcolon}{\kern0pt}{\isacharcolon}{\kern0pt}\ {\isachardoublequoteopen}cset\ {\isasymRightarrow}\ cset\ {\isasymRightarrow}\ cfunc{\isachardoublequoteclose}\ \isakeyword{and}\isanewline
\ \ cfunc{\isacharunderscore}{\kern0pt}prod\ {\isacharcolon}{\kern0pt}{\isacharcolon}{\kern0pt}\ {\isachardoublequoteopen}cfunc\ {\isasymRightarrow}\ cfunc\ {\isasymRightarrow}\ cfunc{\isachardoublequoteclose}\ {\isacharparenleft}{\kern0pt}{\isachardoublequoteopen}{\isasymlangle}{\isacharunderscore}{\kern0pt}{\isacharcomma}{\kern0pt}{\isacharunderscore}{\kern0pt}{\isasymrangle}{\isachardoublequoteclose}{\isacharparenright}{\kern0pt}\isanewline
\isakeyword{where}\isanewline
\ \ left{\isacharunderscore}{\kern0pt}cart{\isacharunderscore}{\kern0pt}proj{\isacharunderscore}{\kern0pt}type{\isacharbrackleft}{\kern0pt}type{\isacharunderscore}{\kern0pt}rule{\isacharbrackright}{\kern0pt}{\isacharcolon}{\kern0pt}\ {\isachardoublequoteopen}left{\isacharunderscore}{\kern0pt}cart{\isacharunderscore}{\kern0pt}proj\ X\ Y\ {\isacharcolon}{\kern0pt}\ X\ {\isasymtimes}\isactrlsub c\ Y\ {\isasymrightarrow}\ X{\isachardoublequoteclose}\ \isakeyword{and}\isanewline
\ \ right{\isacharunderscore}{\kern0pt}cart{\isacharunderscore}{\kern0pt}proj{\isacharunderscore}{\kern0pt}type{\isacharbrackleft}{\kern0pt}type{\isacharunderscore}{\kern0pt}rule{\isacharbrackright}{\kern0pt}{\isacharcolon}{\kern0pt}\ {\isachardoublequoteopen}right{\isacharunderscore}{\kern0pt}cart{\isacharunderscore}{\kern0pt}proj\ X\ Y\ {\isacharcolon}{\kern0pt}\ X\ {\isasymtimes}\isactrlsub c\ Y\ {\isasymrightarrow}\ Y{\isachardoublequoteclose}\ \isakeyword{and}\isanewline
\ \ cfunc{\isacharunderscore}{\kern0pt}prod{\isacharunderscore}{\kern0pt}type{\isacharbrackleft}{\kern0pt}type{\isacharunderscore}{\kern0pt}rule{\isacharbrackright}{\kern0pt}{\isacharcolon}{\kern0pt}\ {\isachardoublequoteopen}f\ {\isacharcolon}{\kern0pt}\ Z\ {\isasymrightarrow}\ X\ {\isasymLongrightarrow}\ g\ {\isacharcolon}{\kern0pt}\ Z\ {\isasymrightarrow}\ Y\ {\isasymLongrightarrow}\ {\isasymlangle}f{\isacharcomma}{\kern0pt}g{\isasymrangle}\ {\isacharcolon}{\kern0pt}\ Z\ {\isasymrightarrow}\ X\ {\isasymtimes}\isactrlsub c\ Y{\isachardoublequoteclose}\ \isakeyword{and}\isanewline
\ \ left{\isacharunderscore}{\kern0pt}cart{\isacharunderscore}{\kern0pt}proj{\isacharunderscore}{\kern0pt}cfunc{\isacharunderscore}{\kern0pt}prod{\isacharcolon}{\kern0pt}\ {\isachardoublequoteopen}f\ {\isacharcolon}{\kern0pt}\ Z\ {\isasymrightarrow}\ X\ {\isasymLongrightarrow}\ g\ {\isacharcolon}{\kern0pt}\ Z\ {\isasymrightarrow}\ Y\ {\isasymLongrightarrow}\ left{\isacharunderscore}{\kern0pt}cart{\isacharunderscore}{\kern0pt}proj\ X\ Y\ {\isasymcirc}\isactrlsub c\ {\isasymlangle}f{\isacharcomma}{\kern0pt}g{\isasymrangle}\ {\isacharequal}{\kern0pt}\ f{\isachardoublequoteclose}\ \isakeyword{and}\isanewline
\ \ right{\isacharunderscore}{\kern0pt}cart{\isacharunderscore}{\kern0pt}proj{\isacharunderscore}{\kern0pt}cfunc{\isacharunderscore}{\kern0pt}prod{\isacharcolon}{\kern0pt}\ {\isachardoublequoteopen}f\ {\isacharcolon}{\kern0pt}\ Z\ {\isasymrightarrow}\ X\ {\isasymLongrightarrow}\ g\ {\isacharcolon}{\kern0pt}\ Z\ {\isasymrightarrow}\ Y\ {\isasymLongrightarrow}\ right{\isacharunderscore}{\kern0pt}cart{\isacharunderscore}{\kern0pt}proj\ X\ Y\ {\isasymcirc}\isactrlsub c\ {\isasymlangle}f{\isacharcomma}{\kern0pt}g{\isasymrangle}\ {\isacharequal}{\kern0pt}\ g{\isachardoublequoteclose}\ \isakeyword{and}\isanewline
\ \ cfunc{\isacharunderscore}{\kern0pt}prod{\isacharunderscore}{\kern0pt}unique{\isacharcolon}{\kern0pt}\ {\isachardoublequoteopen}f\ {\isacharcolon}{\kern0pt}\ Z\ {\isasymrightarrow}\ X\ {\isasymLongrightarrow}\ g\ {\isacharcolon}{\kern0pt}\ Z\ {\isasymrightarrow}\ Y\ {\isasymLongrightarrow}\ h\ {\isacharcolon}{\kern0pt}\ Z\ {\isasymrightarrow}\ X\ {\isasymtimes}\isactrlsub c\ Y\ {\isasymLongrightarrow}\ \isanewline
\ \ \ \ left{\isacharunderscore}{\kern0pt}cart{\isacharunderscore}{\kern0pt}proj\ X\ Y\ {\isasymcirc}\isactrlsub c\ h\ {\isacharequal}{\kern0pt}\ f\ {\isasymLongrightarrow}\ right{\isacharunderscore}{\kern0pt}cart{\isacharunderscore}{\kern0pt}proj\ X\ Y\ {\isasymcirc}\isactrlsub c\ h\ {\isacharequal}{\kern0pt}\ g\ {\isasymLongrightarrow}\ h\ {\isacharequal}{\kern0pt}\ {\isasymlangle}f{\isacharcomma}{\kern0pt}g{\isasymrangle}{\isachardoublequoteclose}\isanewline
\isanewline
\isacommand{definition}\isamarkupfalse%
\ is{\isacharunderscore}{\kern0pt}cart{\isacharunderscore}{\kern0pt}prod\ {\isacharcolon}{\kern0pt}{\isacharcolon}{\kern0pt}\ {\isachardoublequoteopen}cset\ {\isasymRightarrow}\ cfunc\ {\isasymRightarrow}\ cfunc\ {\isasymRightarrow}\ cset\ {\isasymRightarrow}\ cset\ {\isasymRightarrow}\ bool{\isachardoublequoteclose}\ \isakeyword{where}\isanewline
\ \ {\isachardoublequoteopen}is{\isacharunderscore}{\kern0pt}cart{\isacharunderscore}{\kern0pt}prod\ W\ {\isasympi}\isactrlsub {\isadigit{0}}\ {\isasympi}\isactrlsub {\isadigit{1}}\ X\ Y\ {\isasymlongleftrightarrow}\ \isanewline
\ \ \ \ {\isacharparenleft}{\kern0pt}{\isasympi}\isactrlsub {\isadigit{0}}\ {\isacharcolon}{\kern0pt}\ W\ {\isasymrightarrow}\ X\ {\isasymand}\ {\isasympi}\isactrlsub {\isadigit{1}}\ {\isacharcolon}{\kern0pt}\ W\ {\isasymrightarrow}\ Y\ {\isasymand}\isanewline
\ \ \ \ {\isacharparenleft}{\kern0pt}{\isasymforall}\ f\ g\ Z{\isachardot}{\kern0pt}\ {\isacharparenleft}{\kern0pt}f\ {\isacharcolon}{\kern0pt}\ Z\ {\isasymrightarrow}\ X\ {\isasymand}\ g\ {\isacharcolon}{\kern0pt}\ Z\ {\isasymrightarrow}\ Y{\isacharparenright}{\kern0pt}\ {\isasymlongrightarrow}\ \isanewline
\ \ \ \ \ \ {\isacharparenleft}{\kern0pt}{\isasymexists}\ h{\isachardot}{\kern0pt}\ h\ {\isacharcolon}{\kern0pt}\ Z\ {\isasymrightarrow}\ W\ {\isasymand}\ {\isasympi}\isactrlsub {\isadigit{0}}\ {\isasymcirc}\isactrlsub c\ h\ {\isacharequal}{\kern0pt}\ f\ {\isasymand}\ {\isasympi}\isactrlsub {\isadigit{1}}\ {\isasymcirc}\isactrlsub c\ h\ {\isacharequal}{\kern0pt}\ g\ {\isasymand}\isanewline
\ \ \ \ \ \ \ \ {\isacharparenleft}{\kern0pt}{\isasymforall}\ h{\isadigit{2}}{\isachardot}{\kern0pt}\ {\isacharparenleft}{\kern0pt}h{\isadigit{2}}\ {\isacharcolon}{\kern0pt}\ Z\ {\isasymrightarrow}\ W\ {\isasymand}\ {\isasympi}\isactrlsub {\isadigit{0}}\ {\isasymcirc}\isactrlsub c\ h{\isadigit{2}}\ {\isacharequal}{\kern0pt}\ f\ {\isasymand}\ {\isasympi}\isactrlsub {\isadigit{1}}\ {\isasymcirc}\isactrlsub c\ h{\isadigit{2}}\ {\isacharequal}{\kern0pt}\ g{\isacharparenright}{\kern0pt}\ {\isasymlongrightarrow}\ h{\isadigit{2}}\ {\isacharequal}{\kern0pt}\ h{\isacharparenright}{\kern0pt}{\isacharparenright}{\kern0pt}{\isacharparenright}{\kern0pt}{\isacharparenright}{\kern0pt}{\isachardoublequoteclose}\isanewline
\isanewline
\isacommand{lemma}\isamarkupfalse%
\ is{\isacharunderscore}{\kern0pt}cart{\isacharunderscore}{\kern0pt}prod{\isacharunderscore}{\kern0pt}def{\isadigit{2}}{\isacharcolon}{\kern0pt}\isanewline
\ \ \isakeyword{assumes}\ {\isachardoublequoteopen}{\isasympi}\isactrlsub {\isadigit{0}}\ {\isacharcolon}{\kern0pt}\ W\ {\isasymrightarrow}\ X{\isachardoublequoteclose}\ {\isachardoublequoteopen}{\isasympi}\isactrlsub {\isadigit{1}}\ {\isacharcolon}{\kern0pt}\ W\ {\isasymrightarrow}\ Y{\isachardoublequoteclose}\isanewline
\ \ \isakeyword{shows}\ {\isachardoublequoteopen}is{\isacharunderscore}{\kern0pt}cart{\isacharunderscore}{\kern0pt}prod\ W\ {\isasympi}\isactrlsub {\isadigit{0}}\ {\isasympi}\isactrlsub {\isadigit{1}}\ X\ Y\ {\isasymlongleftrightarrow}\ \isanewline
\ \ \ \ {\isacharparenleft}{\kern0pt}{\isasymforall}\ f\ g\ Z{\isachardot}{\kern0pt}\ {\isacharparenleft}{\kern0pt}f\ {\isacharcolon}{\kern0pt}\ Z\ {\isasymrightarrow}\ X\ {\isasymand}\ g\ {\isacharcolon}{\kern0pt}\ Z\ {\isasymrightarrow}\ Y{\isacharparenright}{\kern0pt}\ {\isasymlongrightarrow}\ \isanewline
\ \ \ \ \ \ {\isacharparenleft}{\kern0pt}{\isasymexists}\ h{\isachardot}{\kern0pt}\ h\ {\isacharcolon}{\kern0pt}\ Z\ {\isasymrightarrow}\ W\ {\isasymand}\ {\isasympi}\isactrlsub {\isadigit{0}}\ {\isasymcirc}\isactrlsub c\ h\ {\isacharequal}{\kern0pt}\ f\ {\isasymand}\ {\isasympi}\isactrlsub {\isadigit{1}}\ {\isasymcirc}\isactrlsub c\ h\ {\isacharequal}{\kern0pt}\ g\ {\isasymand}\isanewline
\ \ \ \ \ \ \ \ {\isacharparenleft}{\kern0pt}{\isasymforall}\ h{\isadigit{2}}{\isachardot}{\kern0pt}\ {\isacharparenleft}{\kern0pt}h{\isadigit{2}}\ {\isacharcolon}{\kern0pt}\ Z\ {\isasymrightarrow}\ W\ {\isasymand}\ {\isasympi}\isactrlsub {\isadigit{0}}\ {\isasymcirc}\isactrlsub c\ h{\isadigit{2}}\ {\isacharequal}{\kern0pt}\ f\ {\isasymand}\ {\isasympi}\isactrlsub {\isadigit{1}}\ {\isasymcirc}\isactrlsub c\ h{\isadigit{2}}\ {\isacharequal}{\kern0pt}\ g{\isacharparenright}{\kern0pt}\ {\isasymlongrightarrow}\ h{\isadigit{2}}\ {\isacharequal}{\kern0pt}\ h{\isacharparenright}{\kern0pt}{\isacharparenright}{\kern0pt}{\isacharparenright}{\kern0pt}{\isachardoublequoteclose}\isanewline
%
\isadelimproof
\ \ %
\endisadelimproof
%
\isatagproof
\isacommand{unfolding}\isamarkupfalse%
\ is{\isacharunderscore}{\kern0pt}cart{\isacharunderscore}{\kern0pt}prod{\isacharunderscore}{\kern0pt}def\ \isacommand{using}\isamarkupfalse%
\ assms\ \isacommand{by}\isamarkupfalse%
\ auto%
\endisatagproof
{\isafoldproof}%
%
\isadelimproof
\isanewline
%
\endisadelimproof
\isanewline
\isacommand{abbreviation}\isamarkupfalse%
\ is{\isacharunderscore}{\kern0pt}cart{\isacharunderscore}{\kern0pt}prod{\isacharunderscore}{\kern0pt}triple\ {\isacharcolon}{\kern0pt}{\isacharcolon}{\kern0pt}\ {\isachardoublequoteopen}cset\ {\isasymtimes}\ cfunc\ {\isasymtimes}\ cfunc\ {\isasymRightarrow}\ cset\ {\isasymRightarrow}\ cset\ {\isasymRightarrow}\ bool{\isachardoublequoteclose}\ \isakeyword{where}\isanewline
\ \ {\isachardoublequoteopen}is{\isacharunderscore}{\kern0pt}cart{\isacharunderscore}{\kern0pt}prod{\isacharunderscore}{\kern0pt}triple\ W{\isasympi}\ X\ Y\ {\isasymequiv}\ is{\isacharunderscore}{\kern0pt}cart{\isacharunderscore}{\kern0pt}prod\ {\isacharparenleft}{\kern0pt}fst\ W{\isasympi}{\isacharparenright}{\kern0pt}\ {\isacharparenleft}{\kern0pt}fst\ {\isacharparenleft}{\kern0pt}snd\ W{\isasympi}{\isacharparenright}{\kern0pt}{\isacharparenright}{\kern0pt}\ {\isacharparenleft}{\kern0pt}snd\ {\isacharparenleft}{\kern0pt}snd\ W{\isasympi}{\isacharparenright}{\kern0pt}{\isacharparenright}{\kern0pt}\ X\ Y{\isachardoublequoteclose}\isanewline
\isanewline
\isacommand{lemma}\isamarkupfalse%
\ canonical{\isacharunderscore}{\kern0pt}cart{\isacharunderscore}{\kern0pt}prod{\isacharunderscore}{\kern0pt}is{\isacharunderscore}{\kern0pt}cart{\isacharunderscore}{\kern0pt}prod{\isacharcolon}{\kern0pt}\isanewline
\ {\isachardoublequoteopen}is{\isacharunderscore}{\kern0pt}cart{\isacharunderscore}{\kern0pt}prod\ {\isacharparenleft}{\kern0pt}X\ {\isasymtimes}\isactrlsub c\ Y{\isacharparenright}{\kern0pt}\ {\isacharparenleft}{\kern0pt}left{\isacharunderscore}{\kern0pt}cart{\isacharunderscore}{\kern0pt}proj\ X\ Y{\isacharparenright}{\kern0pt}\ {\isacharparenleft}{\kern0pt}right{\isacharunderscore}{\kern0pt}cart{\isacharunderscore}{\kern0pt}proj\ X\ Y{\isacharparenright}{\kern0pt}\ X\ Y{\isachardoublequoteclose}\isanewline
%
\isadelimproof
\ \ %
\endisadelimproof
%
\isatagproof
\isacommand{unfolding}\isamarkupfalse%
\ is{\isacharunderscore}{\kern0pt}cart{\isacharunderscore}{\kern0pt}prod{\isacharunderscore}{\kern0pt}def\isanewline
\isacommand{proof}\isamarkupfalse%
\ {\isacharparenleft}{\kern0pt}typecheck{\isacharunderscore}{\kern0pt}cfuncs{\isacharparenright}{\kern0pt}\isanewline
\ \ \isacommand{fix}\isamarkupfalse%
\ f\ g\ Z\isanewline
\ \ \isacommand{assume}\isamarkupfalse%
\ f{\isacharunderscore}{\kern0pt}type{\isacharcolon}{\kern0pt}\ {\isachardoublequoteopen}f{\isacharcolon}{\kern0pt}\ Z\ {\isasymrightarrow}\ X{\isachardoublequoteclose}\isanewline
\ \ \isacommand{assume}\isamarkupfalse%
\ g{\isacharunderscore}{\kern0pt}type{\isacharcolon}{\kern0pt}\ {\isachardoublequoteopen}g{\isacharcolon}{\kern0pt}\ Z\ {\isasymrightarrow}\ Y{\isachardoublequoteclose}\isanewline
\ \ \isacommand{show}\isamarkupfalse%
\ {\isachardoublequoteopen}{\isasymexists}h{\isachardot}{\kern0pt}\ h\ {\isacharcolon}{\kern0pt}\ Z\ {\isasymrightarrow}\ X\ {\isasymtimes}\isactrlsub c\ Y\ {\isasymand}\isanewline
\ \ \ \ \ \ \ \ \ \ \ left{\isacharunderscore}{\kern0pt}cart{\isacharunderscore}{\kern0pt}proj\ X\ Y\ {\isasymcirc}\isactrlsub c\ h\ {\isacharequal}{\kern0pt}\ f\ {\isasymand}\isanewline
\ \ \ \ \ \ \ \ \ \ \ right{\isacharunderscore}{\kern0pt}cart{\isacharunderscore}{\kern0pt}proj\ X\ Y\ {\isasymcirc}\isactrlsub c\ h\ {\isacharequal}{\kern0pt}\ g\ {\isasymand}\isanewline
\ \ \ \ \ \ \ \ \ \ \ {\isacharparenleft}{\kern0pt}{\isasymforall}h{\isadigit{2}}{\isachardot}{\kern0pt}\ h{\isadigit{2}}\ {\isacharcolon}{\kern0pt}\ Z\ {\isasymrightarrow}\ X\ {\isasymtimes}\isactrlsub c\ Y\ {\isasymand}\isanewline
\ \ \ \ \ \ \ \ \ \ \ \ \ \ \ \ \ left{\isacharunderscore}{\kern0pt}cart{\isacharunderscore}{\kern0pt}proj\ X\ Y\ {\isasymcirc}\isactrlsub c\ h{\isadigit{2}}\ {\isacharequal}{\kern0pt}\ f\ {\isasymand}\ right{\isacharunderscore}{\kern0pt}cart{\isacharunderscore}{\kern0pt}proj\ X\ Y\ {\isasymcirc}\isactrlsub c\ h{\isadigit{2}}\ {\isacharequal}{\kern0pt}\ g\ {\isasymlongrightarrow}\isanewline
\ \ \ \ \ \ \ \ \ \ \ \ \ \ \ \ \ h{\isadigit{2}}\ {\isacharequal}{\kern0pt}\ h{\isacharparenright}{\kern0pt}{\isachardoublequoteclose}\isanewline
\ \ \ \ \ \ \ \isacommand{using}\isamarkupfalse%
\ f{\isacharunderscore}{\kern0pt}type\ g{\isacharunderscore}{\kern0pt}type\ left{\isacharunderscore}{\kern0pt}cart{\isacharunderscore}{\kern0pt}proj{\isacharunderscore}{\kern0pt}cfunc{\isacharunderscore}{\kern0pt}prod\ right{\isacharunderscore}{\kern0pt}cart{\isacharunderscore}{\kern0pt}proj{\isacharunderscore}{\kern0pt}cfunc{\isacharunderscore}{\kern0pt}prod\ cfunc{\isacharunderscore}{\kern0pt}prod{\isacharunderscore}{\kern0pt}unique\isanewline
\ \ \ \ \isacommand{by}\isamarkupfalse%
\ {\isacharparenleft}{\kern0pt}rule{\isacharunderscore}{\kern0pt}tac\ x{\isacharequal}{\kern0pt}{\isachardoublequoteopen}{\isasymlangle}f{\isacharcomma}{\kern0pt}g{\isasymrangle}{\isachardoublequoteclose}\ \isakeyword{in}\ exI{\isacharcomma}{\kern0pt}\ simp\ add{\isacharcolon}{\kern0pt}\ cfunc{\isacharunderscore}{\kern0pt}prod{\isacharunderscore}{\kern0pt}type{\isacharparenright}{\kern0pt}\isanewline
\isacommand{qed}\isamarkupfalse%
%
\endisatagproof
{\isafoldproof}%
%
\isadelimproof
%
\endisadelimproof
%
\begin{isamarkuptext}%
The lemma below corresponds to Proposition 2.1.8 in Halvorson.%
\end{isamarkuptext}\isamarkuptrue%
\isacommand{lemma}\isamarkupfalse%
\ cart{\isacharunderscore}{\kern0pt}prods{\isacharunderscore}{\kern0pt}isomorphic{\isacharcolon}{\kern0pt}\isanewline
\ \ \isakeyword{assumes}\ W{\isacharunderscore}{\kern0pt}cart{\isacharunderscore}{\kern0pt}prod{\isacharcolon}{\kern0pt}\ \ {\isachardoublequoteopen}is{\isacharunderscore}{\kern0pt}cart{\isacharunderscore}{\kern0pt}prod{\isacharunderscore}{\kern0pt}triple\ {\isacharparenleft}{\kern0pt}W{\isacharcomma}{\kern0pt}\ {\isasympi}\isactrlsub {\isadigit{0}}{\isacharcomma}{\kern0pt}\ {\isasympi}\isactrlsub {\isadigit{1}}{\isacharparenright}{\kern0pt}\ X\ Y{\isachardoublequoteclose}\isanewline
\ \ \isakeyword{assumes}\ W{\isacharprime}{\kern0pt}{\isacharunderscore}{\kern0pt}cart{\isacharunderscore}{\kern0pt}prod{\isacharcolon}{\kern0pt}\ {\isachardoublequoteopen}is{\isacharunderscore}{\kern0pt}cart{\isacharunderscore}{\kern0pt}prod{\isacharunderscore}{\kern0pt}triple\ {\isacharparenleft}{\kern0pt}W{\isacharprime}{\kern0pt}{\isacharcomma}{\kern0pt}\ {\isasympi}{\isacharprime}{\kern0pt}\isactrlsub {\isadigit{0}}{\isacharcomma}{\kern0pt}\ {\isasympi}{\isacharprime}{\kern0pt}\isactrlsub {\isadigit{1}}{\isacharparenright}{\kern0pt}\ X\ Y{\isachardoublequoteclose}\isanewline
\ \ \isakeyword{shows}\ {\isachardoublequoteopen}{\isasymexists}\ f{\isachardot}{\kern0pt}\ f\ {\isacharcolon}{\kern0pt}\ W\ {\isasymrightarrow}\ W{\isacharprime}{\kern0pt}\ {\isasymand}\ isomorphism\ f\ {\isasymand}\ {\isasympi}{\isacharprime}{\kern0pt}\isactrlsub {\isadigit{0}}\ {\isasymcirc}\isactrlsub c\ f\ {\isacharequal}{\kern0pt}\ {\isasympi}\isactrlsub {\isadigit{0}}\ {\isasymand}\ {\isasympi}{\isacharprime}{\kern0pt}\isactrlsub {\isadigit{1}}\ {\isasymcirc}\isactrlsub c\ f\ {\isacharequal}{\kern0pt}\ {\isasympi}\isactrlsub {\isadigit{1}}{\isachardoublequoteclose}\isanewline
%
\isadelimproof
%
\endisadelimproof
%
\isatagproof
\isacommand{proof}\isamarkupfalse%
\ {\isacharminus}{\kern0pt}\isanewline
\ \ \isacommand{obtain}\isamarkupfalse%
\ f\ \isakeyword{where}\ f{\isacharunderscore}{\kern0pt}def{\isacharcolon}{\kern0pt}\ {\isachardoublequoteopen}f\ {\isacharcolon}{\kern0pt}\ W\ {\isasymrightarrow}\ W{\isacharprime}{\kern0pt}\ {\isasymand}\ {\isasympi}{\isacharprime}{\kern0pt}\isactrlsub {\isadigit{0}}\ {\isasymcirc}\isactrlsub c\ f\ {\isacharequal}{\kern0pt}\ {\isasympi}\isactrlsub {\isadigit{0}}\ {\isasymand}\ {\isasympi}{\isacharprime}{\kern0pt}\isactrlsub {\isadigit{1}}\ {\isasymcirc}\isactrlsub c\ f\ {\isacharequal}{\kern0pt}\ {\isasympi}\isactrlsub {\isadigit{1}}{\isachardoublequoteclose}\isanewline
\ \ \ \ \isacommand{using}\isamarkupfalse%
\ W{\isacharprime}{\kern0pt}{\isacharunderscore}{\kern0pt}cart{\isacharunderscore}{\kern0pt}prod\ W{\isacharunderscore}{\kern0pt}cart{\isacharunderscore}{\kern0pt}prod\ \isacommand{unfolding}\isamarkupfalse%
\ is{\isacharunderscore}{\kern0pt}cart{\isacharunderscore}{\kern0pt}prod{\isacharunderscore}{\kern0pt}def\ \isacommand{by}\isamarkupfalse%
\ {\isacharparenleft}{\kern0pt}metis\ fstI\ sndI{\isacharparenright}{\kern0pt}\isanewline
\isanewline
\ \ \isacommand{obtain}\isamarkupfalse%
\ g\ \isakeyword{where}\ g{\isacharunderscore}{\kern0pt}def{\isacharcolon}{\kern0pt}\ {\isachardoublequoteopen}g\ {\isacharcolon}{\kern0pt}\ W{\isacharprime}{\kern0pt}\ {\isasymrightarrow}\ W\ {\isasymand}\ {\isasympi}\isactrlsub {\isadigit{0}}\ {\isasymcirc}\isactrlsub c\ g\ {\isacharequal}{\kern0pt}\ {\isasympi}{\isacharprime}{\kern0pt}\isactrlsub {\isadigit{0}}\ {\isasymand}\ {\isasympi}\isactrlsub {\isadigit{1}}\ {\isasymcirc}\isactrlsub c\ g\ {\isacharequal}{\kern0pt}\ {\isasympi}{\isacharprime}{\kern0pt}\isactrlsub {\isadigit{1}}{\isachardoublequoteclose}\isanewline
\ \ \ \ \ \ \isacommand{using}\isamarkupfalse%
\ W{\isacharprime}{\kern0pt}{\isacharunderscore}{\kern0pt}cart{\isacharunderscore}{\kern0pt}prod\ W{\isacharunderscore}{\kern0pt}cart{\isacharunderscore}{\kern0pt}prod\ \isacommand{unfolding}\isamarkupfalse%
\ is{\isacharunderscore}{\kern0pt}cart{\isacharunderscore}{\kern0pt}prod{\isacharunderscore}{\kern0pt}def\ \isacommand{by}\isamarkupfalse%
\ {\isacharparenleft}{\kern0pt}metis\ fstI\ sndI{\isacharparenright}{\kern0pt}\isanewline
\isanewline
\ \ \isacommand{have}\isamarkupfalse%
\ fg{\isadigit{0}}{\isacharcolon}{\kern0pt}\ {\isachardoublequoteopen}{\isasympi}{\isacharprime}{\kern0pt}\isactrlsub {\isadigit{0}}\ {\isasymcirc}\isactrlsub c\ {\isacharparenleft}{\kern0pt}f\ {\isasymcirc}\isactrlsub c\ g{\isacharparenright}{\kern0pt}\ {\isacharequal}{\kern0pt}\ {\isasympi}{\isacharprime}{\kern0pt}\isactrlsub {\isadigit{0}}{\isachardoublequoteclose}\isanewline
\ \ \ \ \isacommand{using}\isamarkupfalse%
\ W{\isacharprime}{\kern0pt}{\isacharunderscore}{\kern0pt}cart{\isacharunderscore}{\kern0pt}prod\ comp{\isacharunderscore}{\kern0pt}associative{\isadigit{2}}\ f{\isacharunderscore}{\kern0pt}def\ g{\isacharunderscore}{\kern0pt}def\ is{\isacharunderscore}{\kern0pt}cart{\isacharunderscore}{\kern0pt}prod{\isacharunderscore}{\kern0pt}def\ \isacommand{by}\isamarkupfalse%
\ auto\isanewline
\ \ \isacommand{have}\isamarkupfalse%
\ fg{\isadigit{1}}{\isacharcolon}{\kern0pt}\ {\isachardoublequoteopen}{\isasympi}{\isacharprime}{\kern0pt}\isactrlsub {\isadigit{1}}\ {\isasymcirc}\isactrlsub c\ {\isacharparenleft}{\kern0pt}f\ {\isasymcirc}\isactrlsub c\ g{\isacharparenright}{\kern0pt}\ {\isacharequal}{\kern0pt}\ {\isasympi}{\isacharprime}{\kern0pt}\isactrlsub {\isadigit{1}}{\isachardoublequoteclose}\isanewline
\ \ \ \ \isacommand{using}\isamarkupfalse%
\ W{\isacharprime}{\kern0pt}{\isacharunderscore}{\kern0pt}cart{\isacharunderscore}{\kern0pt}prod\ comp{\isacharunderscore}{\kern0pt}associative{\isadigit{2}}\ f{\isacharunderscore}{\kern0pt}def\ g{\isacharunderscore}{\kern0pt}def\ is{\isacharunderscore}{\kern0pt}cart{\isacharunderscore}{\kern0pt}prod{\isacharunderscore}{\kern0pt}def\ \isacommand{by}\isamarkupfalse%
\ auto\isanewline
\isanewline
\ \ \isacommand{obtain}\isamarkupfalse%
\ idW{\isacharprime}{\kern0pt}\ \isakeyword{where}\ {\isachardoublequoteopen}idW{\isacharprime}{\kern0pt}\ {\isacharcolon}{\kern0pt}\ W{\isacharprime}{\kern0pt}\ {\isasymrightarrow}\ W{\isacharprime}{\kern0pt}\ {\isasymand}\ {\isacharparenleft}{\kern0pt}{\isasymforall}\ h{\isadigit{2}}{\isachardot}{\kern0pt}\ {\isacharparenleft}{\kern0pt}h{\isadigit{2}}\ {\isacharcolon}{\kern0pt}\ W{\isacharprime}{\kern0pt}\ {\isasymrightarrow}\ W{\isacharprime}{\kern0pt}\ {\isasymand}\ {\isasympi}{\isacharprime}{\kern0pt}\isactrlsub {\isadigit{0}}\ {\isasymcirc}\isactrlsub c\ h{\isadigit{2}}\ {\isacharequal}{\kern0pt}\ {\isasympi}{\isacharprime}{\kern0pt}\isactrlsub {\isadigit{0}}\ {\isasymand}\ {\isasympi}{\isacharprime}{\kern0pt}\isactrlsub {\isadigit{1}}\ {\isasymcirc}\isactrlsub c\ h{\isadigit{2}}\ {\isacharequal}{\kern0pt}\ {\isasympi}{\isacharprime}{\kern0pt}\isactrlsub {\isadigit{1}}{\isacharparenright}{\kern0pt}\ {\isasymlongrightarrow}\ h{\isadigit{2}}\ {\isacharequal}{\kern0pt}\ idW{\isacharprime}{\kern0pt}{\isacharparenright}{\kern0pt}{\isachardoublequoteclose}\isanewline
\ \ \ \ \isacommand{using}\isamarkupfalse%
\ W{\isacharprime}{\kern0pt}{\isacharunderscore}{\kern0pt}cart{\isacharunderscore}{\kern0pt}prod\ \isacommand{unfolding}\isamarkupfalse%
\ is{\isacharunderscore}{\kern0pt}cart{\isacharunderscore}{\kern0pt}prod{\isacharunderscore}{\kern0pt}def\ \isacommand{by}\isamarkupfalse%
\ {\isacharparenleft}{\kern0pt}metis\ fst{\isacharunderscore}{\kern0pt}conv\ snd{\isacharunderscore}{\kern0pt}conv{\isacharparenright}{\kern0pt}\isanewline
\ \ \isacommand{then}\isamarkupfalse%
\ \isacommand{have}\isamarkupfalse%
\ fg{\isacharcolon}{\kern0pt}\ {\isachardoublequoteopen}f\ {\isasymcirc}\isactrlsub c\ g\ {\isacharequal}{\kern0pt}\ id\ W{\isacharprime}{\kern0pt}{\isachardoublequoteclose}\isanewline
\ \ \isacommand{proof}\isamarkupfalse%
\ clarify\isanewline
\ \ \ \ \isacommand{assume}\isamarkupfalse%
\ idW{\isacharprime}{\kern0pt}{\isacharunderscore}{\kern0pt}unique{\isacharcolon}{\kern0pt}\ {\isachardoublequoteopen}{\isasymforall}h{\isadigit{2}}{\isachardot}{\kern0pt}\ h{\isadigit{2}}\ {\isacharcolon}{\kern0pt}\ W{\isacharprime}{\kern0pt}\ {\isasymrightarrow}\ W{\isacharprime}{\kern0pt}\ {\isasymand}\ {\isasympi}{\isacharprime}{\kern0pt}\isactrlsub {\isadigit{0}}\ {\isasymcirc}\isactrlsub c\ h{\isadigit{2}}\ {\isacharequal}{\kern0pt}\ {\isasympi}{\isacharprime}{\kern0pt}\isactrlsub {\isadigit{0}}\ {\isasymand}\ {\isasympi}{\isacharprime}{\kern0pt}\isactrlsub {\isadigit{1}}\ {\isasymcirc}\isactrlsub c\ h{\isadigit{2}}\ {\isacharequal}{\kern0pt}\ {\isasympi}{\isacharprime}{\kern0pt}\isactrlsub {\isadigit{1}}\ {\isasymlongrightarrow}\ h{\isadigit{2}}\ {\isacharequal}{\kern0pt}\ idW{\isacharprime}{\kern0pt}{\isachardoublequoteclose}\isanewline
\ \ \ \ \isacommand{have}\isamarkupfalse%
\ {\isadigit{1}}{\isacharcolon}{\kern0pt}\ {\isachardoublequoteopen}f\ {\isasymcirc}\isactrlsub c\ g\ {\isacharequal}{\kern0pt}\ idW{\isacharprime}{\kern0pt}{\isachardoublequoteclose}\isanewline
\ \ \ \ \ \ \isacommand{using}\isamarkupfalse%
\ comp{\isacharunderscore}{\kern0pt}type\ f{\isacharunderscore}{\kern0pt}def\ fg{\isadigit{0}}\ fg{\isadigit{1}}\ g{\isacharunderscore}{\kern0pt}def\ idW{\isacharprime}{\kern0pt}{\isacharunderscore}{\kern0pt}unique\ \isacommand{by}\isamarkupfalse%
\ blast\isanewline
\ \ \ \ \isacommand{have}\isamarkupfalse%
\ {\isadigit{2}}{\isacharcolon}{\kern0pt}\ {\isachardoublequoteopen}id\ W{\isacharprime}{\kern0pt}\ {\isacharequal}{\kern0pt}\ idW{\isacharprime}{\kern0pt}{\isachardoublequoteclose}\isanewline
\ \ \ \ \ \ \isacommand{using}\isamarkupfalse%
\ W{\isacharprime}{\kern0pt}{\isacharunderscore}{\kern0pt}cart{\isacharunderscore}{\kern0pt}prod\ idW{\isacharprime}{\kern0pt}{\isacharunderscore}{\kern0pt}unique\ id{\isacharunderscore}{\kern0pt}right{\isacharunderscore}{\kern0pt}unit{\isadigit{2}}\ id{\isacharunderscore}{\kern0pt}type\ is{\isacharunderscore}{\kern0pt}cart{\isacharunderscore}{\kern0pt}prod{\isacharunderscore}{\kern0pt}def\ \isacommand{by}\isamarkupfalse%
\ auto\isanewline
\ \ \ \ \isacommand{from}\isamarkupfalse%
\ {\isadigit{1}}\ {\isadigit{2}}\ \isacommand{show}\isamarkupfalse%
\ {\isachardoublequoteopen}f\ {\isasymcirc}\isactrlsub c\ g\ {\isacharequal}{\kern0pt}\ id\ W{\isacharprime}{\kern0pt}{\isachardoublequoteclose}\isanewline
\ \ \ \ \ \ \isacommand{by}\isamarkupfalse%
\ auto\isanewline
\ \ \isacommand{qed}\isamarkupfalse%
\isanewline
\isanewline
\ \ \isacommand{have}\isamarkupfalse%
\ gf{\isadigit{0}}{\isacharcolon}{\kern0pt}\ {\isachardoublequoteopen}{\isasympi}\isactrlsub {\isadigit{0}}\ {\isasymcirc}\isactrlsub c\ {\isacharparenleft}{\kern0pt}g\ {\isasymcirc}\isactrlsub c\ f{\isacharparenright}{\kern0pt}\ {\isacharequal}{\kern0pt}\ {\isasympi}\isactrlsub {\isadigit{0}}{\isachardoublequoteclose}\isanewline
\ \ \ \ \isacommand{using}\isamarkupfalse%
\ W{\isacharunderscore}{\kern0pt}cart{\isacharunderscore}{\kern0pt}prod\ comp{\isacharunderscore}{\kern0pt}associative{\isadigit{2}}\ f{\isacharunderscore}{\kern0pt}def\ g{\isacharunderscore}{\kern0pt}def\ is{\isacharunderscore}{\kern0pt}cart{\isacharunderscore}{\kern0pt}prod{\isacharunderscore}{\kern0pt}def\ \isacommand{by}\isamarkupfalse%
\ auto\isanewline
\ \ \isacommand{have}\isamarkupfalse%
\ gf{\isadigit{1}}{\isacharcolon}{\kern0pt}\ {\isachardoublequoteopen}{\isasympi}\isactrlsub {\isadigit{1}}\ {\isasymcirc}\isactrlsub c\ {\isacharparenleft}{\kern0pt}g\ {\isasymcirc}\isactrlsub c\ f{\isacharparenright}{\kern0pt}\ {\isacharequal}{\kern0pt}\ {\isasympi}\isactrlsub {\isadigit{1}}{\isachardoublequoteclose}\isanewline
\ \ \ \ \isacommand{using}\isamarkupfalse%
\ W{\isacharunderscore}{\kern0pt}cart{\isacharunderscore}{\kern0pt}prod\ comp{\isacharunderscore}{\kern0pt}associative{\isadigit{2}}\ f{\isacharunderscore}{\kern0pt}def\ g{\isacharunderscore}{\kern0pt}def\ is{\isacharunderscore}{\kern0pt}cart{\isacharunderscore}{\kern0pt}prod{\isacharunderscore}{\kern0pt}def\ \isacommand{by}\isamarkupfalse%
\ auto\isanewline
\isanewline
\ \ \isacommand{obtain}\isamarkupfalse%
\ idW\ \isakeyword{where}\ {\isachardoublequoteopen}idW\ {\isacharcolon}{\kern0pt}\ W\ {\isasymrightarrow}\ W\ {\isasymand}\ {\isacharparenleft}{\kern0pt}{\isasymforall}\ h{\isadigit{2}}{\isachardot}{\kern0pt}\ {\isacharparenleft}{\kern0pt}h{\isadigit{2}}\ {\isacharcolon}{\kern0pt}\ W\ {\isasymrightarrow}\ W\ {\isasymand}\ {\isasympi}\isactrlsub {\isadigit{0}}\ {\isasymcirc}\isactrlsub c\ h{\isadigit{2}}\ {\isacharequal}{\kern0pt}\ {\isasympi}\isactrlsub {\isadigit{0}}\ {\isasymand}\ {\isasympi}\isactrlsub {\isadigit{1}}\ {\isasymcirc}\isactrlsub c\ h{\isadigit{2}}\ {\isacharequal}{\kern0pt}\ {\isasympi}\isactrlsub {\isadigit{1}}{\isacharparenright}{\kern0pt}\ {\isasymlongrightarrow}\ h{\isadigit{2}}\ {\isacharequal}{\kern0pt}\ idW{\isacharparenright}{\kern0pt}{\isachardoublequoteclose}\isanewline
\ \ \ \ \isacommand{using}\isamarkupfalse%
\ W{\isacharunderscore}{\kern0pt}cart{\isacharunderscore}{\kern0pt}prod\ \isacommand{unfolding}\isamarkupfalse%
\ is{\isacharunderscore}{\kern0pt}cart{\isacharunderscore}{\kern0pt}prod{\isacharunderscore}{\kern0pt}def\ \isacommand{by}\isamarkupfalse%
\ {\isacharparenleft}{\kern0pt}metis\ fst{\isacharunderscore}{\kern0pt}conv\ snd{\isacharunderscore}{\kern0pt}conv{\isacharparenright}{\kern0pt}\isanewline
\ \ \isacommand{then}\isamarkupfalse%
\ \isacommand{have}\isamarkupfalse%
\ gf{\isacharcolon}{\kern0pt}\ {\isachardoublequoteopen}g\ {\isasymcirc}\isactrlsub c\ f\ {\isacharequal}{\kern0pt}\ id\ W{\isachardoublequoteclose}\isanewline
\ \ \isacommand{proof}\isamarkupfalse%
\ clarify\isanewline
\ \ \ \ \isacommand{assume}\isamarkupfalse%
\ idW{\isacharunderscore}{\kern0pt}unique{\isacharcolon}{\kern0pt}\ {\isachardoublequoteopen}{\isasymforall}h{\isadigit{2}}{\isachardot}{\kern0pt}\ h{\isadigit{2}}\ {\isacharcolon}{\kern0pt}\ W\ {\isasymrightarrow}\ W\ {\isasymand}\ {\isasympi}\isactrlsub {\isadigit{0}}\ {\isasymcirc}\isactrlsub c\ h{\isadigit{2}}\ {\isacharequal}{\kern0pt}\ {\isasympi}\isactrlsub {\isadigit{0}}\ {\isasymand}\ {\isasympi}\isactrlsub {\isadigit{1}}\ {\isasymcirc}\isactrlsub c\ h{\isadigit{2}}\ {\isacharequal}{\kern0pt}\ {\isasympi}\isactrlsub {\isadigit{1}}\ {\isasymlongrightarrow}\ h{\isadigit{2}}\ {\isacharequal}{\kern0pt}\ idW{\isachardoublequoteclose}\isanewline
\ \ \ \ \isacommand{have}\isamarkupfalse%
\ {\isadigit{1}}{\isacharcolon}{\kern0pt}\ {\isachardoublequoteopen}g\ {\isasymcirc}\isactrlsub c\ f\ {\isacharequal}{\kern0pt}\ idW{\isachardoublequoteclose}\isanewline
\ \ \ \ \ \ \isacommand{using}\isamarkupfalse%
\ idW{\isacharunderscore}{\kern0pt}unique\ cfunc{\isacharunderscore}{\kern0pt}type{\isacharunderscore}{\kern0pt}def\ codomain{\isacharunderscore}{\kern0pt}comp\ domain{\isacharunderscore}{\kern0pt}comp\ f{\isacharunderscore}{\kern0pt}def\ gf{\isadigit{0}}\ gf{\isadigit{1}}\ g{\isacharunderscore}{\kern0pt}def\ \isacommand{by}\isamarkupfalse%
\ {\isacharparenleft}{\kern0pt}erule{\isacharunderscore}{\kern0pt}tac\ x{\isacharequal}{\kern0pt}{\isachardoublequoteopen}g\ {\isasymcirc}\isactrlsub c\ f{\isachardoublequoteclose}\ \isakeyword{in}\ allE{\isacharcomma}{\kern0pt}\ auto{\isacharparenright}{\kern0pt}\isanewline
\ \ \ \ \isacommand{have}\isamarkupfalse%
\ {\isadigit{2}}{\isacharcolon}{\kern0pt}\ {\isachardoublequoteopen}id\ W\ {\isacharequal}{\kern0pt}\ idW{\isachardoublequoteclose}\isanewline
\ \ \ \ \ \ \isacommand{using}\isamarkupfalse%
\ idW{\isacharunderscore}{\kern0pt}unique\ W{\isacharunderscore}{\kern0pt}cart{\isacharunderscore}{\kern0pt}prod\ id{\isacharunderscore}{\kern0pt}right{\isacharunderscore}{\kern0pt}unit{\isadigit{2}}\ id{\isacharunderscore}{\kern0pt}type\ is{\isacharunderscore}{\kern0pt}cart{\isacharunderscore}{\kern0pt}prod{\isacharunderscore}{\kern0pt}def\ \isacommand{by}\isamarkupfalse%
\ {\isacharparenleft}{\kern0pt}erule{\isacharunderscore}{\kern0pt}tac\ x{\isacharequal}{\kern0pt}{\isachardoublequoteopen}id\ W{\isachardoublequoteclose}\ \isakeyword{in}\ allE{\isacharcomma}{\kern0pt}\ auto{\isacharparenright}{\kern0pt}\isanewline
\ \ \ \ \isacommand{from}\isamarkupfalse%
\ {\isadigit{1}}\ {\isadigit{2}}\ \isacommand{show}\isamarkupfalse%
\ {\isachardoublequoteopen}g\ {\isasymcirc}\isactrlsub c\ f\ {\isacharequal}{\kern0pt}\ id\ W{\isachardoublequoteclose}\isanewline
\ \ \ \ \ \ \isacommand{by}\isamarkupfalse%
\ auto\isanewline
\ \ \isacommand{qed}\isamarkupfalse%
\isanewline
\isanewline
\ \ \isacommand{have}\isamarkupfalse%
\ f{\isacharunderscore}{\kern0pt}iso{\isacharcolon}{\kern0pt}\ {\isachardoublequoteopen}isomorphism\ f{\isachardoublequoteclose}\isanewline
\ \ \ \ \isacommand{using}\isamarkupfalse%
\ f{\isacharunderscore}{\kern0pt}def\ fg\ g{\isacharunderscore}{\kern0pt}def\ gf\ isomorphism{\isacharunderscore}{\kern0pt}def{\isadigit{3}}\ \isacommand{by}\isamarkupfalse%
\ blast\isanewline
\ \ \isacommand{from}\isamarkupfalse%
\ f{\isacharunderscore}{\kern0pt}iso\ f{\isacharunderscore}{\kern0pt}def\ \isacommand{show}\isamarkupfalse%
\ {\isachardoublequoteopen}{\isasymexists}f{\isachardot}{\kern0pt}\ f\ {\isacharcolon}{\kern0pt}\ W\ {\isasymrightarrow}\ W{\isacharprime}{\kern0pt}\ {\isasymand}\ isomorphism\ f\ {\isasymand}\ {\isasympi}{\isacharprime}{\kern0pt}\isactrlsub {\isadigit{0}}\ {\isasymcirc}\isactrlsub c\ f\ {\isacharequal}{\kern0pt}\ {\isasympi}\isactrlsub {\isadigit{0}}\ {\isasymand}\ {\isasympi}{\isacharprime}{\kern0pt}\isactrlsub {\isadigit{1}}\ {\isasymcirc}\isactrlsub c\ f\ {\isacharequal}{\kern0pt}\ {\isasympi}\isactrlsub {\isadigit{1}}{\isachardoublequoteclose}\isanewline
\ \ \ \ \isacommand{by}\isamarkupfalse%
\ auto\isanewline
\isacommand{qed}\isamarkupfalse%
%
\endisatagproof
{\isafoldproof}%
%
\isadelimproof
\isanewline
%
\endisadelimproof
\isanewline
\isacommand{lemma}\isamarkupfalse%
\ product{\isacharunderscore}{\kern0pt}commutes{\isacharcolon}{\kern0pt}\isanewline
\ \ {\isachardoublequoteopen}A\ {\isasymtimes}\isactrlsub c\ B\ {\isasymcong}\ B\ {\isasymtimes}\isactrlsub c\ A{\isachardoublequoteclose}\isanewline
%
\isadelimproof
%
\endisadelimproof
%
\isatagproof
\isacommand{proof}\isamarkupfalse%
\ {\isacharminus}{\kern0pt}\isanewline
\ \ \isacommand{have}\isamarkupfalse%
\ id{\isacharunderscore}{\kern0pt}AB{\isacharcolon}{\kern0pt}\ {\isachardoublequoteopen}{\isasymlangle}right{\isacharunderscore}{\kern0pt}cart{\isacharunderscore}{\kern0pt}proj\ B\ A{\isacharcomma}{\kern0pt}\ left{\isacharunderscore}{\kern0pt}cart{\isacharunderscore}{\kern0pt}proj\ B\ A{\isasymrangle}\ {\isasymcirc}\isactrlsub c\ {\isasymlangle}right{\isacharunderscore}{\kern0pt}cart{\isacharunderscore}{\kern0pt}proj\ A\ B{\isacharcomma}{\kern0pt}\ left{\isacharunderscore}{\kern0pt}cart{\isacharunderscore}{\kern0pt}proj\ A\ B{\isasymrangle}\ {\isacharequal}{\kern0pt}\ id{\isacharparenleft}{\kern0pt}A\ {\isasymtimes}\isactrlsub c\ B{\isacharparenright}{\kern0pt}{\isachardoublequoteclose}\isanewline
\ \ \ \ \isacommand{by}\isamarkupfalse%
\ {\isacharparenleft}{\kern0pt}typecheck{\isacharunderscore}{\kern0pt}cfuncs{\isacharcomma}{\kern0pt}\ smt\ {\isacharparenleft}{\kern0pt}z{\isadigit{3}}{\isacharparenright}{\kern0pt}\ cfunc{\isacharunderscore}{\kern0pt}prod{\isacharunderscore}{\kern0pt}unique\ comp{\isacharunderscore}{\kern0pt}associative{\isadigit{2}}\ id{\isacharunderscore}{\kern0pt}right{\isacharunderscore}{\kern0pt}unit{\isadigit{2}}\ left{\isacharunderscore}{\kern0pt}cart{\isacharunderscore}{\kern0pt}proj{\isacharunderscore}{\kern0pt}cfunc{\isacharunderscore}{\kern0pt}prod\ right{\isacharunderscore}{\kern0pt}cart{\isacharunderscore}{\kern0pt}proj{\isacharunderscore}{\kern0pt}cfunc{\isacharunderscore}{\kern0pt}prod{\isacharparenright}{\kern0pt}\isanewline
\ \ \isacommand{have}\isamarkupfalse%
\ id{\isacharunderscore}{\kern0pt}BA{\isacharcolon}{\kern0pt}\ {\isachardoublequoteopen}{\isasymlangle}right{\isacharunderscore}{\kern0pt}cart{\isacharunderscore}{\kern0pt}proj\ A\ B{\isacharcomma}{\kern0pt}\ left{\isacharunderscore}{\kern0pt}cart{\isacharunderscore}{\kern0pt}proj\ A\ B{\isasymrangle}\ {\isasymcirc}\isactrlsub c\ {\isasymlangle}right{\isacharunderscore}{\kern0pt}cart{\isacharunderscore}{\kern0pt}proj\ B\ A{\isacharcomma}{\kern0pt}\ left{\isacharunderscore}{\kern0pt}cart{\isacharunderscore}{\kern0pt}proj\ B\ A{\isasymrangle}\ {\isacharequal}{\kern0pt}\ id{\isacharparenleft}{\kern0pt}B\ {\isasymtimes}\isactrlsub c\ A{\isacharparenright}{\kern0pt}{\isachardoublequoteclose}\isanewline
\ \ \ \ \isacommand{by}\isamarkupfalse%
\ {\isacharparenleft}{\kern0pt}typecheck{\isacharunderscore}{\kern0pt}cfuncs{\isacharcomma}{\kern0pt}\ smt\ {\isacharparenleft}{\kern0pt}z{\isadigit{3}}{\isacharparenright}{\kern0pt}\ cfunc{\isacharunderscore}{\kern0pt}prod{\isacharunderscore}{\kern0pt}unique\ comp{\isacharunderscore}{\kern0pt}associative{\isadigit{2}}\ id{\isacharunderscore}{\kern0pt}right{\isacharunderscore}{\kern0pt}unit{\isadigit{2}}\ left{\isacharunderscore}{\kern0pt}cart{\isacharunderscore}{\kern0pt}proj{\isacharunderscore}{\kern0pt}cfunc{\isacharunderscore}{\kern0pt}prod\ right{\isacharunderscore}{\kern0pt}cart{\isacharunderscore}{\kern0pt}proj{\isacharunderscore}{\kern0pt}cfunc{\isacharunderscore}{\kern0pt}prod{\isacharparenright}{\kern0pt}\isanewline
\ \ \isacommand{show}\isamarkupfalse%
\ {\isachardoublequoteopen}A\ {\isasymtimes}\isactrlsub c\ B\ {\isasymcong}\ B\ {\isasymtimes}\isactrlsub c\ A{\isachardoublequoteclose}\isanewline
\ \ \ \ \isacommand{by}\isamarkupfalse%
\ {\isacharparenleft}{\kern0pt}smt\ {\isacharparenleft}{\kern0pt}verit{\isacharcomma}{\kern0pt}\ ccfv{\isacharunderscore}{\kern0pt}threshold{\isacharparenright}{\kern0pt}\ canonical{\isacharunderscore}{\kern0pt}cart{\isacharunderscore}{\kern0pt}prod{\isacharunderscore}{\kern0pt}is{\isacharunderscore}{\kern0pt}cart{\isacharunderscore}{\kern0pt}prod\ cfunc{\isacharunderscore}{\kern0pt}prod{\isacharunderscore}{\kern0pt}unique\ cfunc{\isacharunderscore}{\kern0pt}type{\isacharunderscore}{\kern0pt}def\ id{\isacharunderscore}{\kern0pt}AB\ id{\isacharunderscore}{\kern0pt}BA\ is{\isacharunderscore}{\kern0pt}cart{\isacharunderscore}{\kern0pt}prod{\isacharunderscore}{\kern0pt}def\ is{\isacharunderscore}{\kern0pt}isomorphic{\isacharunderscore}{\kern0pt}def\ isomorphism{\isacharunderscore}{\kern0pt}def{\isacharparenright}{\kern0pt}\isanewline
\isacommand{qed}\isamarkupfalse%
%
\endisatagproof
{\isafoldproof}%
%
\isadelimproof
\isanewline
%
\endisadelimproof
\isanewline
\isacommand{lemma}\isamarkupfalse%
\ cart{\isacharunderscore}{\kern0pt}prod{\isacharunderscore}{\kern0pt}eq{\isacharcolon}{\kern0pt}\isanewline
\ \ \isakeyword{assumes}\ {\isachardoublequoteopen}a\ {\isacharcolon}{\kern0pt}\ Z\ {\isasymrightarrow}\ X\ {\isasymtimes}\isactrlsub c\ Y{\isachardoublequoteclose}\ {\isachardoublequoteopen}b\ {\isacharcolon}{\kern0pt}\ Z\ {\isasymrightarrow}\ \ X\ {\isasymtimes}\isactrlsub c\ Y{\isachardoublequoteclose}\isanewline
\ \ \isakeyword{shows}\ {\isachardoublequoteopen}a\ {\isacharequal}{\kern0pt}\ b\ {\isasymlongleftrightarrow}\ \isanewline
\ \ \ \ {\isacharparenleft}{\kern0pt}left{\isacharunderscore}{\kern0pt}cart{\isacharunderscore}{\kern0pt}proj\ X\ Y\ {\isasymcirc}\isactrlsub c\ a\ {\isacharequal}{\kern0pt}\ left{\isacharunderscore}{\kern0pt}cart{\isacharunderscore}{\kern0pt}proj\ X\ Y\ {\isasymcirc}\isactrlsub c\ b\ \isanewline
\ \ \ \ \ \ {\isasymand}\ right{\isacharunderscore}{\kern0pt}cart{\isacharunderscore}{\kern0pt}proj\ X\ Y\ {\isasymcirc}\isactrlsub c\ a\ {\isacharequal}{\kern0pt}\ right{\isacharunderscore}{\kern0pt}cart{\isacharunderscore}{\kern0pt}proj\ X\ Y\ {\isasymcirc}\isactrlsub c\ b{\isacharparenright}{\kern0pt}{\isachardoublequoteclose}\isanewline
%
\isadelimproof
\ \ %
\endisadelimproof
%
\isatagproof
\isacommand{by}\isamarkupfalse%
\ {\isacharparenleft}{\kern0pt}metis\ {\isacharparenleft}{\kern0pt}full{\isacharunderscore}{\kern0pt}types{\isacharparenright}{\kern0pt}\ assms\ cfunc{\isacharunderscore}{\kern0pt}prod{\isacharunderscore}{\kern0pt}unique\ comp{\isacharunderscore}{\kern0pt}type\ left{\isacharunderscore}{\kern0pt}cart{\isacharunderscore}{\kern0pt}proj{\isacharunderscore}{\kern0pt}type\ right{\isacharunderscore}{\kern0pt}cart{\isacharunderscore}{\kern0pt}proj{\isacharunderscore}{\kern0pt}type{\isacharparenright}{\kern0pt}%
\endisatagproof
{\isafoldproof}%
%
\isadelimproof
\isanewline
%
\endisadelimproof
\isanewline
\isacommand{lemma}\isamarkupfalse%
\ cart{\isacharunderscore}{\kern0pt}prod{\isacharunderscore}{\kern0pt}eqI{\isacharcolon}{\kern0pt}\isanewline
\ \ \isakeyword{assumes}\ {\isachardoublequoteopen}a\ {\isacharcolon}{\kern0pt}\ Z\ {\isasymrightarrow}\ X\ {\isasymtimes}\isactrlsub c\ Y{\isachardoublequoteclose}\ {\isachardoublequoteopen}b\ {\isacharcolon}{\kern0pt}\ Z\ {\isasymrightarrow}\ \ X\ {\isasymtimes}\isactrlsub c\ Y{\isachardoublequoteclose}\isanewline
\ \ \isakeyword{assumes}\ {\isachardoublequoteopen}{\isacharparenleft}{\kern0pt}left{\isacharunderscore}{\kern0pt}cart{\isacharunderscore}{\kern0pt}proj\ X\ Y\ {\isasymcirc}\isactrlsub c\ a\ {\isacharequal}{\kern0pt}\ left{\isacharunderscore}{\kern0pt}cart{\isacharunderscore}{\kern0pt}proj\ X\ Y\ {\isasymcirc}\isactrlsub c\ b\ \isanewline
\ \ \ \ \ \ {\isasymand}\ right{\isacharunderscore}{\kern0pt}cart{\isacharunderscore}{\kern0pt}proj\ X\ Y\ {\isasymcirc}\isactrlsub c\ a\ {\isacharequal}{\kern0pt}\ right{\isacharunderscore}{\kern0pt}cart{\isacharunderscore}{\kern0pt}proj\ X\ Y\ {\isasymcirc}\isactrlsub c\ b{\isacharparenright}{\kern0pt}{\isachardoublequoteclose}\isanewline
\ \ \isakeyword{shows}\ {\isachardoublequoteopen}a\ {\isacharequal}{\kern0pt}\ b{\isachardoublequoteclose}\isanewline
%
\isadelimproof
\ \ %
\endisadelimproof
%
\isatagproof
\isacommand{using}\isamarkupfalse%
\ assms\ cart{\isacharunderscore}{\kern0pt}prod{\isacharunderscore}{\kern0pt}eq\ \isacommand{by}\isamarkupfalse%
\ blast%
\endisatagproof
{\isafoldproof}%
%
\isadelimproof
\isanewline
%
\endisadelimproof
\isanewline
\isacommand{lemma}\isamarkupfalse%
\ cart{\isacharunderscore}{\kern0pt}prod{\isacharunderscore}{\kern0pt}eq{\isadigit{2}}{\isacharcolon}{\kern0pt}\isanewline
\ \ \isakeyword{assumes}\ {\isachardoublequoteopen}a\ {\isacharcolon}{\kern0pt}\ Z\ {\isasymrightarrow}\ X{\isachardoublequoteclose}\ {\isachardoublequoteopen}b\ {\isacharcolon}{\kern0pt}\ Z\ {\isasymrightarrow}\ Y{\isachardoublequoteclose}\ {\isachardoublequoteopen}c\ {\isacharcolon}{\kern0pt}\ Z\ {\isasymrightarrow}\ \ X{\isachardoublequoteclose}\ {\isachardoublequoteopen}d\ {\isacharcolon}{\kern0pt}\ Z\ {\isasymrightarrow}\ \ Y{\isachardoublequoteclose}\isanewline
\ \ \isakeyword{shows}\ {\isachardoublequoteopen}{\isasymlangle}a{\isacharcomma}{\kern0pt}\ b{\isasymrangle}\ {\isacharequal}{\kern0pt}\ {\isasymlangle}c{\isacharcomma}{\kern0pt}d{\isasymrangle}\ {\isasymlongleftrightarrow}\ {\isacharparenleft}{\kern0pt}a\ {\isacharequal}{\kern0pt}\ c\ {\isasymand}\ b\ {\isacharequal}{\kern0pt}\ d{\isacharparenright}{\kern0pt}{\isachardoublequoteclose}\isanewline
%
\isadelimproof
\ \ %
\endisadelimproof
%
\isatagproof
\isacommand{by}\isamarkupfalse%
\ {\isacharparenleft}{\kern0pt}metis\ assms\ left{\isacharunderscore}{\kern0pt}cart{\isacharunderscore}{\kern0pt}proj{\isacharunderscore}{\kern0pt}cfunc{\isacharunderscore}{\kern0pt}prod\ right{\isacharunderscore}{\kern0pt}cart{\isacharunderscore}{\kern0pt}proj{\isacharunderscore}{\kern0pt}cfunc{\isacharunderscore}{\kern0pt}prod{\isacharparenright}{\kern0pt}%
\endisatagproof
{\isafoldproof}%
%
\isadelimproof
\isanewline
%
\endisadelimproof
\isanewline
\isacommand{lemma}\isamarkupfalse%
\ cart{\isacharunderscore}{\kern0pt}prod{\isacharunderscore}{\kern0pt}decomp{\isacharcolon}{\kern0pt}\isanewline
\ \ \isakeyword{assumes}\ {\isachardoublequoteopen}a\ {\isacharcolon}{\kern0pt}\ A\ {\isasymrightarrow}\ X\ {\isasymtimes}\isactrlsub c\ Y{\isachardoublequoteclose}\isanewline
\ \ \isakeyword{shows}\ {\isachardoublequoteopen}{\isasymexists}\ x\ y{\isachardot}{\kern0pt}\ a\ {\isacharequal}{\kern0pt}\ {\isasymlangle}x{\isacharcomma}{\kern0pt}\ y{\isasymrangle}\ {\isasymand}\ x\ {\isacharcolon}{\kern0pt}\ A\ {\isasymrightarrow}\ X\ {\isasymand}\ y\ {\isacharcolon}{\kern0pt}\ A\ {\isasymrightarrow}\ Y{\isachardoublequoteclose}\isanewline
%
\isadelimproof
%
\endisadelimproof
%
\isatagproof
\isacommand{proof}\isamarkupfalse%
\ {\isacharparenleft}{\kern0pt}rule{\isacharunderscore}{\kern0pt}tac\ x{\isacharequal}{\kern0pt}{\isachardoublequoteopen}left{\isacharunderscore}{\kern0pt}cart{\isacharunderscore}{\kern0pt}proj\ X\ Y\ {\isasymcirc}\isactrlsub c\ a{\isachardoublequoteclose}\ \isakeyword{in}\ exI{\isacharcomma}{\kern0pt}\ rule{\isacharunderscore}{\kern0pt}tac\ x{\isacharequal}{\kern0pt}{\isachardoublequoteopen}right{\isacharunderscore}{\kern0pt}cart{\isacharunderscore}{\kern0pt}proj\ X\ Y\ {\isasymcirc}\isactrlsub c\ a{\isachardoublequoteclose}\ \isakeyword{in}\ exI{\isacharcomma}{\kern0pt}\ safe{\isacharparenright}{\kern0pt}\isanewline
\ \ \isacommand{show}\isamarkupfalse%
\ {\isachardoublequoteopen}a\ {\isacharequal}{\kern0pt}\ {\isasymlangle}left{\isacharunderscore}{\kern0pt}cart{\isacharunderscore}{\kern0pt}proj\ X\ Y\ {\isasymcirc}\isactrlsub c\ a{\isacharcomma}{\kern0pt}right{\isacharunderscore}{\kern0pt}cart{\isacharunderscore}{\kern0pt}proj\ X\ Y\ {\isasymcirc}\isactrlsub c\ a{\isasymrangle}{\isachardoublequoteclose}\isanewline
\ \ \ \ \isacommand{using}\isamarkupfalse%
\ assms\ \isacommand{by}\isamarkupfalse%
\ {\isacharparenleft}{\kern0pt}typecheck{\isacharunderscore}{\kern0pt}cfuncs{\isacharcomma}{\kern0pt}\ simp\ add{\isacharcolon}{\kern0pt}\ cfunc{\isacharunderscore}{\kern0pt}prod{\isacharunderscore}{\kern0pt}unique{\isacharparenright}{\kern0pt}\isanewline
\ \ \isacommand{show}\isamarkupfalse%
\ {\isachardoublequoteopen}left{\isacharunderscore}{\kern0pt}cart{\isacharunderscore}{\kern0pt}proj\ X\ Y\ {\isasymcirc}\isactrlsub c\ a\ {\isacharcolon}{\kern0pt}\ A\ {\isasymrightarrow}\ \ X{\isachardoublequoteclose}\isanewline
\ \ \ \ \isacommand{using}\isamarkupfalse%
\ assms\ \isacommand{by}\isamarkupfalse%
\ typecheck{\isacharunderscore}{\kern0pt}cfuncs\isanewline
\ \ \isacommand{show}\isamarkupfalse%
\ {\isachardoublequoteopen}right{\isacharunderscore}{\kern0pt}cart{\isacharunderscore}{\kern0pt}proj\ X\ Y\ {\isasymcirc}\isactrlsub c\ a\ {\isacharcolon}{\kern0pt}\ A\ {\isasymrightarrow}\ Y{\isachardoublequoteclose}\isanewline
\ \ \ \ \isacommand{using}\isamarkupfalse%
\ assms\ \isacommand{by}\isamarkupfalse%
\ typecheck{\isacharunderscore}{\kern0pt}cfuncs\isanewline
\isacommand{qed}\isamarkupfalse%
%
\endisatagproof
{\isafoldproof}%
%
\isadelimproof
%
\endisadelimproof
%
\isadelimdocument
%
\endisadelimdocument
%
\isatagdocument
%
\isamarkupsubsection{Diagonal Functions%
}
\isamarkuptrue%
%
\endisatagdocument
{\isafolddocument}%
%
\isadelimdocument
%
\endisadelimdocument
%
\begin{isamarkuptext}%
The definition below corresponds to Definition 2.1.9 in Halvorson.%
\end{isamarkuptext}\isamarkuptrue%
\isacommand{definition}\isamarkupfalse%
\ diagonal\ {\isacharcolon}{\kern0pt}{\isacharcolon}{\kern0pt}\ {\isachardoublequoteopen}cset\ {\isasymRightarrow}\ cfunc{\isachardoublequoteclose}\ \isakeyword{where}\isanewline
\ \ {\isachardoublequoteopen}diagonal\ X\ {\isacharequal}{\kern0pt}\ {\isasymlangle}id\ X{\isacharcomma}{\kern0pt}id\ X{\isasymrangle}{\isachardoublequoteclose}\isanewline
\isanewline
\isacommand{lemma}\isamarkupfalse%
\ diagonal{\isacharunderscore}{\kern0pt}type{\isacharbrackleft}{\kern0pt}type{\isacharunderscore}{\kern0pt}rule{\isacharbrackright}{\kern0pt}{\isacharcolon}{\kern0pt}\isanewline
\ \ {\isachardoublequoteopen}diagonal\ X\ {\isacharcolon}{\kern0pt}\ X\ {\isasymrightarrow}\ X\ {\isasymtimes}\isactrlsub c\ X{\isachardoublequoteclose}\isanewline
%
\isadelimproof
\ \ %
\endisadelimproof
%
\isatagproof
\isacommand{unfolding}\isamarkupfalse%
\ diagonal{\isacharunderscore}{\kern0pt}def\ \isacommand{by}\isamarkupfalse%
\ {\isacharparenleft}{\kern0pt}simp\ add{\isacharcolon}{\kern0pt}\ cfunc{\isacharunderscore}{\kern0pt}prod{\isacharunderscore}{\kern0pt}type\ id{\isacharunderscore}{\kern0pt}type{\isacharparenright}{\kern0pt}%
\endisatagproof
{\isafoldproof}%
%
\isadelimproof
\isanewline
%
\endisadelimproof
\isanewline
\isacommand{lemma}\isamarkupfalse%
\ diag{\isacharunderscore}{\kern0pt}mono{\isacharcolon}{\kern0pt}\isanewline
{\isachardoublequoteopen}monomorphism{\isacharparenleft}{\kern0pt}diagonal\ X{\isacharparenright}{\kern0pt}{\isachardoublequoteclose}\isanewline
%
\isadelimproof
%
\endisadelimproof
%
\isatagproof
\isacommand{proof}\isamarkupfalse%
\ {\isacharminus}{\kern0pt}\ \isanewline
\ \ \isacommand{have}\isamarkupfalse%
\ {\isachardoublequoteopen}left{\isacharunderscore}{\kern0pt}cart{\isacharunderscore}{\kern0pt}proj\ X\ X\ {\isasymcirc}\isactrlsub c\ diagonal\ X\ {\isacharequal}{\kern0pt}\ id\ X{\isachardoublequoteclose}\isanewline
\ \ \ \ \isacommand{by}\isamarkupfalse%
\ {\isacharparenleft}{\kern0pt}metis\ diagonal{\isacharunderscore}{\kern0pt}def\ id{\isacharunderscore}{\kern0pt}type\ left{\isacharunderscore}{\kern0pt}cart{\isacharunderscore}{\kern0pt}proj{\isacharunderscore}{\kern0pt}cfunc{\isacharunderscore}{\kern0pt}prod{\isacharparenright}{\kern0pt}\isanewline
\ \ \isacommand{then}\isamarkupfalse%
\ \isacommand{show}\isamarkupfalse%
\ {\isachardoublequoteopen}monomorphism{\isacharparenleft}{\kern0pt}diagonal\ X{\isacharparenright}{\kern0pt}{\isachardoublequoteclose}\isanewline
\ \ \ \ \isacommand{by}\isamarkupfalse%
\ {\isacharparenleft}{\kern0pt}metis\ cfunc{\isacharunderscore}{\kern0pt}type{\isacharunderscore}{\kern0pt}def\ comp{\isacharunderscore}{\kern0pt}monic{\isacharunderscore}{\kern0pt}imp{\isacharunderscore}{\kern0pt}monic\ diagonal{\isacharunderscore}{\kern0pt}type\ id{\isacharunderscore}{\kern0pt}isomorphism\ iso{\isacharunderscore}{\kern0pt}imp{\isacharunderscore}{\kern0pt}epi{\isacharunderscore}{\kern0pt}and{\isacharunderscore}{\kern0pt}monic\ left{\isacharunderscore}{\kern0pt}cart{\isacharunderscore}{\kern0pt}proj{\isacharunderscore}{\kern0pt}type{\isacharparenright}{\kern0pt}\isanewline
\isacommand{qed}\isamarkupfalse%
%
\endisatagproof
{\isafoldproof}%
%
\isadelimproof
%
\endisadelimproof
%
\isadelimdocument
%
\endisadelimdocument
%
\isatagdocument
%
\isamarkupsubsection{Products of Functions%
}
\isamarkuptrue%
%
\endisatagdocument
{\isafolddocument}%
%
\isadelimdocument
%
\endisadelimdocument
%
\begin{isamarkuptext}%
The definition below corresponds to Definition 2.1.10 in Halvorson.%
\end{isamarkuptext}\isamarkuptrue%
\isacommand{definition}\isamarkupfalse%
\ cfunc{\isacharunderscore}{\kern0pt}cross{\isacharunderscore}{\kern0pt}prod\ {\isacharcolon}{\kern0pt}{\isacharcolon}{\kern0pt}\ {\isachardoublequoteopen}cfunc\ {\isasymRightarrow}\ cfunc\ {\isasymRightarrow}\ cfunc{\isachardoublequoteclose}\ {\isacharparenleft}{\kern0pt}\isakeyword{infixr}\ {\isachardoublequoteopen}{\isasymtimes}\isactrlsub f{\isachardoublequoteclose}\ {\isadigit{5}}{\isadigit{5}}{\isacharparenright}{\kern0pt}\ \isakeyword{where}\isanewline
\ \ {\isachardoublequoteopen}f\ {\isasymtimes}\isactrlsub f\ g\ {\isacharequal}{\kern0pt}\ {\isasymlangle}f\ {\isasymcirc}\isactrlsub c\ left{\isacharunderscore}{\kern0pt}cart{\isacharunderscore}{\kern0pt}proj\ {\isacharparenleft}{\kern0pt}domain\ f{\isacharparenright}{\kern0pt}\ {\isacharparenleft}{\kern0pt}domain\ g{\isacharparenright}{\kern0pt}{\isacharcomma}{\kern0pt}\ g\ {\isasymcirc}\isactrlsub c\ right{\isacharunderscore}{\kern0pt}cart{\isacharunderscore}{\kern0pt}proj\ {\isacharparenleft}{\kern0pt}domain\ f{\isacharparenright}{\kern0pt}\ {\isacharparenleft}{\kern0pt}domain\ g{\isacharparenright}{\kern0pt}{\isasymrangle}{\isachardoublequoteclose}\isanewline
\isanewline
\isacommand{lemma}\isamarkupfalse%
\ cfunc{\isacharunderscore}{\kern0pt}cross{\isacharunderscore}{\kern0pt}prod{\isacharunderscore}{\kern0pt}def{\isadigit{2}}{\isacharcolon}{\kern0pt}\ \isanewline
\ \ \isakeyword{assumes}\ {\isachardoublequoteopen}f\ {\isacharcolon}{\kern0pt}\ X\ {\isasymrightarrow}\ Y{\isachardoublequoteclose}\ {\isachardoublequoteopen}g\ {\isacharcolon}{\kern0pt}\ V{\isasymrightarrow}\ W{\isachardoublequoteclose}\isanewline
\ \ \isakeyword{shows}\ {\isachardoublequoteopen}f\ {\isasymtimes}\isactrlsub f\ g\ {\isacharequal}{\kern0pt}\ {\isasymlangle}f\ {\isasymcirc}\isactrlsub c\ left{\isacharunderscore}{\kern0pt}cart{\isacharunderscore}{\kern0pt}proj\ X\ V{\isacharcomma}{\kern0pt}\ g\ {\isasymcirc}\isactrlsub c\ right{\isacharunderscore}{\kern0pt}cart{\isacharunderscore}{\kern0pt}proj\ X\ V{\isasymrangle}{\isachardoublequoteclose}\isanewline
%
\isadelimproof
\ \ %
\endisadelimproof
%
\isatagproof
\isacommand{using}\isamarkupfalse%
\ assms\ cfunc{\isacharunderscore}{\kern0pt}cross{\isacharunderscore}{\kern0pt}prod{\isacharunderscore}{\kern0pt}def\ cfunc{\isacharunderscore}{\kern0pt}type{\isacharunderscore}{\kern0pt}def\ \isacommand{by}\isamarkupfalse%
\ auto%
\endisatagproof
{\isafoldproof}%
%
\isadelimproof
\isanewline
%
\endisadelimproof
\ \ \ \ \isanewline
\isacommand{lemma}\isamarkupfalse%
\ cfunc{\isacharunderscore}{\kern0pt}cross{\isacharunderscore}{\kern0pt}prod{\isacharunderscore}{\kern0pt}type{\isacharbrackleft}{\kern0pt}type{\isacharunderscore}{\kern0pt}rule{\isacharbrackright}{\kern0pt}{\isacharcolon}{\kern0pt}\isanewline
\ \ {\isachardoublequoteopen}f\ {\isacharcolon}{\kern0pt}\ W\ {\isasymrightarrow}\ Y\ {\isasymLongrightarrow}\ g\ {\isacharcolon}{\kern0pt}\ X\ {\isasymrightarrow}\ Z\ {\isasymLongrightarrow}\ f\ {\isasymtimes}\isactrlsub f\ g\ {\isacharcolon}{\kern0pt}\ W\ {\isasymtimes}\isactrlsub c\ X\ {\isasymrightarrow}\ Y\ {\isasymtimes}\isactrlsub c\ Z{\isachardoublequoteclose}\isanewline
%
\isadelimproof
\ \ %
\endisadelimproof
%
\isatagproof
\isacommand{unfolding}\isamarkupfalse%
\ cfunc{\isacharunderscore}{\kern0pt}cross{\isacharunderscore}{\kern0pt}prod{\isacharunderscore}{\kern0pt}def\isanewline
\ \ \isacommand{using}\isamarkupfalse%
\ cfunc{\isacharunderscore}{\kern0pt}prod{\isacharunderscore}{\kern0pt}type\ cfunc{\isacharunderscore}{\kern0pt}type{\isacharunderscore}{\kern0pt}def\ comp{\isacharunderscore}{\kern0pt}type\ left{\isacharunderscore}{\kern0pt}cart{\isacharunderscore}{\kern0pt}proj{\isacharunderscore}{\kern0pt}type\ right{\isacharunderscore}{\kern0pt}cart{\isacharunderscore}{\kern0pt}proj{\isacharunderscore}{\kern0pt}type\ \isacommand{by}\isamarkupfalse%
\ auto%
\endisatagproof
{\isafoldproof}%
%
\isadelimproof
\isanewline
%
\endisadelimproof
\isanewline
\isacommand{lemma}\isamarkupfalse%
\ left{\isacharunderscore}{\kern0pt}cart{\isacharunderscore}{\kern0pt}proj{\isacharunderscore}{\kern0pt}cfunc{\isacharunderscore}{\kern0pt}cross{\isacharunderscore}{\kern0pt}prod{\isacharcolon}{\kern0pt}\isanewline
\ \ {\isachardoublequoteopen}f\ {\isacharcolon}{\kern0pt}\ W\ {\isasymrightarrow}\ Y\ {\isasymLongrightarrow}\ g\ {\isacharcolon}{\kern0pt}\ X\ {\isasymrightarrow}\ Z\ {\isasymLongrightarrow}\ left{\isacharunderscore}{\kern0pt}cart{\isacharunderscore}{\kern0pt}proj\ Y\ Z\ {\isasymcirc}\isactrlsub c\ f\ {\isasymtimes}\isactrlsub f\ g\ {\isacharequal}{\kern0pt}\ f\ {\isasymcirc}\isactrlsub c\ left{\isacharunderscore}{\kern0pt}cart{\isacharunderscore}{\kern0pt}proj\ W\ X{\isachardoublequoteclose}\isanewline
%
\isadelimproof
\ \ %
\endisadelimproof
%
\isatagproof
\isacommand{unfolding}\isamarkupfalse%
\ cfunc{\isacharunderscore}{\kern0pt}cross{\isacharunderscore}{\kern0pt}prod{\isacharunderscore}{\kern0pt}def\isanewline
\ \ \isacommand{using}\isamarkupfalse%
\ cfunc{\isacharunderscore}{\kern0pt}type{\isacharunderscore}{\kern0pt}def\ comp{\isacharunderscore}{\kern0pt}type\ left{\isacharunderscore}{\kern0pt}cart{\isacharunderscore}{\kern0pt}proj{\isacharunderscore}{\kern0pt}cfunc{\isacharunderscore}{\kern0pt}prod\ left{\isacharunderscore}{\kern0pt}cart{\isacharunderscore}{\kern0pt}proj{\isacharunderscore}{\kern0pt}type\ right{\isacharunderscore}{\kern0pt}cart{\isacharunderscore}{\kern0pt}proj{\isacharunderscore}{\kern0pt}type\ \isacommand{by}\isamarkupfalse%
\ {\isacharparenleft}{\kern0pt}smt\ {\isacharparenleft}{\kern0pt}verit{\isacharparenright}{\kern0pt}{\isacharparenright}{\kern0pt}%
\endisatagproof
{\isafoldproof}%
%
\isadelimproof
\isanewline
%
\endisadelimproof
\isanewline
\isacommand{lemma}\isamarkupfalse%
\ right{\isacharunderscore}{\kern0pt}cart{\isacharunderscore}{\kern0pt}proj{\isacharunderscore}{\kern0pt}cfunc{\isacharunderscore}{\kern0pt}cross{\isacharunderscore}{\kern0pt}prod{\isacharcolon}{\kern0pt}\isanewline
\ \ {\isachardoublequoteopen}f\ {\isacharcolon}{\kern0pt}\ W\ {\isasymrightarrow}\ Y\ {\isasymLongrightarrow}\ g\ {\isacharcolon}{\kern0pt}\ X\ {\isasymrightarrow}\ Z\ {\isasymLongrightarrow}\ right{\isacharunderscore}{\kern0pt}cart{\isacharunderscore}{\kern0pt}proj\ Y\ Z\ {\isasymcirc}\isactrlsub c\ f\ {\isasymtimes}\isactrlsub f\ g\ {\isacharequal}{\kern0pt}\ g\ {\isasymcirc}\isactrlsub c\ right{\isacharunderscore}{\kern0pt}cart{\isacharunderscore}{\kern0pt}proj\ W\ X{\isachardoublequoteclose}\isanewline
%
\isadelimproof
\ \ %
\endisadelimproof
%
\isatagproof
\isacommand{unfolding}\isamarkupfalse%
\ cfunc{\isacharunderscore}{\kern0pt}cross{\isacharunderscore}{\kern0pt}prod{\isacharunderscore}{\kern0pt}def\isanewline
\ \ \isacommand{using}\isamarkupfalse%
\ cfunc{\isacharunderscore}{\kern0pt}type{\isacharunderscore}{\kern0pt}def\ comp{\isacharunderscore}{\kern0pt}type\ right{\isacharunderscore}{\kern0pt}cart{\isacharunderscore}{\kern0pt}proj{\isacharunderscore}{\kern0pt}cfunc{\isacharunderscore}{\kern0pt}prod\ left{\isacharunderscore}{\kern0pt}cart{\isacharunderscore}{\kern0pt}proj{\isacharunderscore}{\kern0pt}type\ right{\isacharunderscore}{\kern0pt}cart{\isacharunderscore}{\kern0pt}proj{\isacharunderscore}{\kern0pt}type\ \isacommand{by}\isamarkupfalse%
\ {\isacharparenleft}{\kern0pt}smt\ {\isacharparenleft}{\kern0pt}verit{\isacharparenright}{\kern0pt}{\isacharparenright}{\kern0pt}%
\endisatagproof
{\isafoldproof}%
%
\isadelimproof
\isanewline
%
\endisadelimproof
\isanewline
\isacommand{lemma}\isamarkupfalse%
\ cfunc{\isacharunderscore}{\kern0pt}cross{\isacharunderscore}{\kern0pt}prod{\isacharunderscore}{\kern0pt}unique{\isacharcolon}{\kern0pt}\ {\isachardoublequoteopen}f\ {\isacharcolon}{\kern0pt}\ W\ {\isasymrightarrow}\ Y\ {\isasymLongrightarrow}\ g\ {\isacharcolon}{\kern0pt}\ X\ {\isasymrightarrow}\ Z\ {\isasymLongrightarrow}\ h\ {\isacharcolon}{\kern0pt}\ W\ {\isasymtimes}\isactrlsub c\ X\ {\isasymrightarrow}\ Y\ {\isasymtimes}\isactrlsub c\ Z\ {\isasymLongrightarrow}\isanewline
\ \ \ \ left{\isacharunderscore}{\kern0pt}cart{\isacharunderscore}{\kern0pt}proj\ Y\ Z\ {\isasymcirc}\isactrlsub c\ h\ {\isacharequal}{\kern0pt}\ f\ {\isasymcirc}\isactrlsub c\ left{\isacharunderscore}{\kern0pt}cart{\isacharunderscore}{\kern0pt}proj\ W\ X\ {\isasymLongrightarrow}\isanewline
\ \ \ \ right{\isacharunderscore}{\kern0pt}cart{\isacharunderscore}{\kern0pt}proj\ Y\ Z\ {\isasymcirc}\isactrlsub c\ h\ {\isacharequal}{\kern0pt}\ g\ {\isasymcirc}\isactrlsub c\ right{\isacharunderscore}{\kern0pt}cart{\isacharunderscore}{\kern0pt}proj\ W\ X\ {\isasymLongrightarrow}\ h\ {\isacharequal}{\kern0pt}\ f\ {\isasymtimes}\isactrlsub f\ g{\isachardoublequoteclose}\isanewline
%
\isadelimproof
\ \ %
\endisadelimproof
%
\isatagproof
\isacommand{unfolding}\isamarkupfalse%
\ cfunc{\isacharunderscore}{\kern0pt}cross{\isacharunderscore}{\kern0pt}prod{\isacharunderscore}{\kern0pt}def\isanewline
\ \ \isacommand{using}\isamarkupfalse%
\ cfunc{\isacharunderscore}{\kern0pt}prod{\isacharunderscore}{\kern0pt}unique\ cfunc{\isacharunderscore}{\kern0pt}type{\isacharunderscore}{\kern0pt}def\ comp{\isacharunderscore}{\kern0pt}type\ left{\isacharunderscore}{\kern0pt}cart{\isacharunderscore}{\kern0pt}proj{\isacharunderscore}{\kern0pt}type\ right{\isacharunderscore}{\kern0pt}cart{\isacharunderscore}{\kern0pt}proj{\isacharunderscore}{\kern0pt}type\ \isacommand{by}\isamarkupfalse%
\ auto%
\endisatagproof
{\isafoldproof}%
%
\isadelimproof
%
\endisadelimproof
%
\begin{isamarkuptext}%
The lemma below corresponds to Proposition 2.1.11 in Halvorson.%
\end{isamarkuptext}\isamarkuptrue%
\isacommand{lemma}\isamarkupfalse%
\ identity{\isacharunderscore}{\kern0pt}distributes{\isacharunderscore}{\kern0pt}across{\isacharunderscore}{\kern0pt}composition{\isacharcolon}{\kern0pt}\isanewline
\ \ \isakeyword{assumes}\ f{\isacharunderscore}{\kern0pt}type{\isacharcolon}{\kern0pt}\ {\isachardoublequoteopen}f\ {\isacharcolon}{\kern0pt}\ A\ {\isasymrightarrow}\ B{\isachardoublequoteclose}\ \isakeyword{and}\ g{\isacharunderscore}{\kern0pt}type{\isacharcolon}{\kern0pt}\ {\isachardoublequoteopen}g\ {\isacharcolon}{\kern0pt}\ B\ {\isasymrightarrow}\ C{\isachardoublequoteclose}\isanewline
\ \ \isakeyword{shows}\ {\isachardoublequoteopen}id\ X\ {\isasymtimes}\isactrlsub f\ {\isacharparenleft}{\kern0pt}g\ \ {\isasymcirc}\isactrlsub c\ f{\isacharparenright}{\kern0pt}\ {\isacharequal}{\kern0pt}\ {\isacharparenleft}{\kern0pt}id\ X\ {\isasymtimes}\isactrlsub f\ g{\isacharparenright}{\kern0pt}\ {\isasymcirc}\isactrlsub c\ {\isacharparenleft}{\kern0pt}id\ X\ {\isasymtimes}\isactrlsub f\ f{\isacharparenright}{\kern0pt}{\isachardoublequoteclose}\isanewline
%
\isadelimproof
%
\endisadelimproof
%
\isatagproof
\isacommand{proof}\isamarkupfalse%
\ {\isacharminus}{\kern0pt}\isanewline
\ \ \isacommand{from}\isamarkupfalse%
\ cfunc{\isacharunderscore}{\kern0pt}cross{\isacharunderscore}{\kern0pt}prod{\isacharunderscore}{\kern0pt}unique\isanewline
\ \ \isacommand{have}\isamarkupfalse%
\ uniqueness{\isacharcolon}{\kern0pt}\ {\isachardoublequoteopen}{\isasymforall}h{\isachardot}{\kern0pt}\ h\ {\isacharcolon}{\kern0pt}\ X\ {\isasymtimes}\isactrlsub c\ A\ {\isasymrightarrow}\ X\ {\isasymtimes}\isactrlsub c\ C\ {\isasymand}\isanewline
\ \ \ \ left{\isacharunderscore}{\kern0pt}cart{\isacharunderscore}{\kern0pt}proj\ X\ C\ {\isasymcirc}\isactrlsub c\ h\ {\isacharequal}{\kern0pt}\ id\isactrlsub c\ X\ {\isasymcirc}\isactrlsub c\ left{\isacharunderscore}{\kern0pt}cart{\isacharunderscore}{\kern0pt}proj\ X\ A\ {\isasymand}\isanewline
\ \ \ \ right{\isacharunderscore}{\kern0pt}cart{\isacharunderscore}{\kern0pt}proj\ X\ C\ {\isasymcirc}\isactrlsub c\ h\ {\isacharequal}{\kern0pt}\ {\isacharparenleft}{\kern0pt}g\ {\isasymcirc}\isactrlsub c\ f{\isacharparenright}{\kern0pt}\ {\isasymcirc}\isactrlsub c\ right{\isacharunderscore}{\kern0pt}cart{\isacharunderscore}{\kern0pt}proj\ X\ A\ {\isasymlongrightarrow}\isanewline
\ \ \ \ h\ {\isacharequal}{\kern0pt}\ id\isactrlsub c\ X\ {\isasymtimes}\isactrlsub f\ {\isacharparenleft}{\kern0pt}g\ {\isasymcirc}\isactrlsub c\ f{\isacharparenright}{\kern0pt}{\isachardoublequoteclose}\isanewline
\ \ \ \ \isacommand{by}\isamarkupfalse%
\ {\isacharparenleft}{\kern0pt}meson\ comp{\isacharunderscore}{\kern0pt}type\ f{\isacharunderscore}{\kern0pt}type\ g{\isacharunderscore}{\kern0pt}type\ id{\isacharunderscore}{\kern0pt}type{\isacharparenright}{\kern0pt}\isanewline
\isanewline
\ \ \isacommand{have}\isamarkupfalse%
\ left{\isacharunderscore}{\kern0pt}eq{\isacharcolon}{\kern0pt}\ {\isachardoublequoteopen}left{\isacharunderscore}{\kern0pt}cart{\isacharunderscore}{\kern0pt}proj\ X\ C\ {\isasymcirc}\isactrlsub c\ {\isacharparenleft}{\kern0pt}id\isactrlsub c\ X\ {\isasymtimes}\isactrlsub f\ g{\isacharparenright}{\kern0pt}\ {\isasymcirc}\isactrlsub c\ {\isacharparenleft}{\kern0pt}id\isactrlsub c\ X\ {\isasymtimes}\isactrlsub f\ f{\isacharparenright}{\kern0pt}\ {\isacharequal}{\kern0pt}\ id\isactrlsub c\ X\ {\isasymcirc}\isactrlsub c\ left{\isacharunderscore}{\kern0pt}cart{\isacharunderscore}{\kern0pt}proj\ X\ A{\isachardoublequoteclose}\isanewline
\ \ \ \ \isacommand{using}\isamarkupfalse%
\ assms\ \isacommand{by}\isamarkupfalse%
\ {\isacharparenleft}{\kern0pt}typecheck{\isacharunderscore}{\kern0pt}cfuncs{\isacharcomma}{\kern0pt}\ smt\ comp{\isacharunderscore}{\kern0pt}associative{\isadigit{2}}\ id{\isacharunderscore}{\kern0pt}left{\isacharunderscore}{\kern0pt}unit{\isadigit{2}}\ left{\isacharunderscore}{\kern0pt}cart{\isacharunderscore}{\kern0pt}proj{\isacharunderscore}{\kern0pt}cfunc{\isacharunderscore}{\kern0pt}cross{\isacharunderscore}{\kern0pt}prod\ left{\isacharunderscore}{\kern0pt}cart{\isacharunderscore}{\kern0pt}proj{\isacharunderscore}{\kern0pt}type{\isacharparenright}{\kern0pt}\isanewline
\ \ \isacommand{have}\isamarkupfalse%
\ right{\isacharunderscore}{\kern0pt}eq{\isacharcolon}{\kern0pt}\ {\isachardoublequoteopen}right{\isacharunderscore}{\kern0pt}cart{\isacharunderscore}{\kern0pt}proj\ X\ C\ {\isasymcirc}\isactrlsub c\ {\isacharparenleft}{\kern0pt}id\isactrlsub c\ X\ {\isasymtimes}\isactrlsub f\ g{\isacharparenright}{\kern0pt}\ {\isasymcirc}\isactrlsub c\ {\isacharparenleft}{\kern0pt}id\isactrlsub c\ X\ {\isasymtimes}\isactrlsub f\ f{\isacharparenright}{\kern0pt}\ {\isacharequal}{\kern0pt}\ {\isacharparenleft}{\kern0pt}g\ {\isasymcirc}\isactrlsub c\ f{\isacharparenright}{\kern0pt}\ {\isasymcirc}\isactrlsub c\ right{\isacharunderscore}{\kern0pt}cart{\isacharunderscore}{\kern0pt}proj\ X\ A{\isachardoublequoteclose}\isanewline
\ \ \ \ \isacommand{using}\isamarkupfalse%
\ assms\ \isacommand{by}\isamarkupfalse%
{\isacharparenleft}{\kern0pt}typecheck{\isacharunderscore}{\kern0pt}cfuncs{\isacharcomma}{\kern0pt}\ smt\ comp{\isacharunderscore}{\kern0pt}associative{\isadigit{2}}\ right{\isacharunderscore}{\kern0pt}cart{\isacharunderscore}{\kern0pt}proj{\isacharunderscore}{\kern0pt}cfunc{\isacharunderscore}{\kern0pt}cross{\isacharunderscore}{\kern0pt}prod\ right{\isacharunderscore}{\kern0pt}cart{\isacharunderscore}{\kern0pt}proj{\isacharunderscore}{\kern0pt}type{\isacharparenright}{\kern0pt}\isanewline
\ \ \isacommand{show}\isamarkupfalse%
\ {\isachardoublequoteopen}id\isactrlsub c\ X\ {\isasymtimes}\isactrlsub f\ g\ {\isasymcirc}\isactrlsub c\ f\ {\isacharequal}{\kern0pt}\ {\isacharparenleft}{\kern0pt}id\isactrlsub c\ X\ {\isasymtimes}\isactrlsub f\ g{\isacharparenright}{\kern0pt}\ {\isasymcirc}\isactrlsub c\ id\isactrlsub c\ X\ {\isasymtimes}\isactrlsub f\ f{\isachardoublequoteclose}\isanewline
\ \ \ \ \isacommand{using}\isamarkupfalse%
\ assms\ left{\isacharunderscore}{\kern0pt}eq\ right{\isacharunderscore}{\kern0pt}eq\ uniqueness\ \isacommand{by}\isamarkupfalse%
\ {\isacharparenleft}{\kern0pt}typecheck{\isacharunderscore}{\kern0pt}cfuncs{\isacharcomma}{\kern0pt}\ auto{\isacharparenright}{\kern0pt}\isanewline
\isacommand{qed}\isamarkupfalse%
%
\endisatagproof
{\isafoldproof}%
%
\isadelimproof
\isanewline
%
\endisadelimproof
\isanewline
\isacommand{lemma}\isamarkupfalse%
\ cfunc{\isacharunderscore}{\kern0pt}cross{\isacharunderscore}{\kern0pt}prod{\isacharunderscore}{\kern0pt}comp{\isacharunderscore}{\kern0pt}cfunc{\isacharunderscore}{\kern0pt}prod{\isacharcolon}{\kern0pt}\isanewline
\ \ \isakeyword{assumes}\ a{\isacharunderscore}{\kern0pt}type{\isacharcolon}{\kern0pt}\ {\isachardoublequoteopen}a\ {\isacharcolon}{\kern0pt}\ A\ {\isasymrightarrow}\ W{\isachardoublequoteclose}\ \isakeyword{and}\ b{\isacharunderscore}{\kern0pt}type{\isacharcolon}{\kern0pt}\ {\isachardoublequoteopen}b\ {\isacharcolon}{\kern0pt}\ A\ {\isasymrightarrow}\ X{\isachardoublequoteclose}\isanewline
\ \ \isakeyword{assumes}\ f{\isacharunderscore}{\kern0pt}type{\isacharcolon}{\kern0pt}\ {\isachardoublequoteopen}f\ {\isacharcolon}{\kern0pt}\ W\ {\isasymrightarrow}\ Y{\isachardoublequoteclose}\ \isakeyword{and}\ g{\isacharunderscore}{\kern0pt}type{\isacharcolon}{\kern0pt}\ {\isachardoublequoteopen}g\ {\isacharcolon}{\kern0pt}\ X\ {\isasymrightarrow}\ Z{\isachardoublequoteclose}\isanewline
\ \ \isakeyword{shows}\ {\isachardoublequoteopen}{\isacharparenleft}{\kern0pt}f\ {\isasymtimes}\isactrlsub f\ g{\isacharparenright}{\kern0pt}\ {\isasymcirc}\isactrlsub c\ {\isasymlangle}a{\isacharcomma}{\kern0pt}\ b{\isasymrangle}\ {\isacharequal}{\kern0pt}\ {\isasymlangle}f\ {\isasymcirc}\isactrlsub c\ a{\isacharcomma}{\kern0pt}\ g\ {\isasymcirc}\isactrlsub c\ b{\isasymrangle}{\isachardoublequoteclose}\isanewline
%
\isadelimproof
%
\endisadelimproof
%
\isatagproof
\isacommand{proof}\isamarkupfalse%
\ {\isacharminus}{\kern0pt}\isanewline
\ \ \isacommand{from}\isamarkupfalse%
\ cfunc{\isacharunderscore}{\kern0pt}prod{\isacharunderscore}{\kern0pt}unique\ \isacommand{have}\isamarkupfalse%
\ uniqueness{\isacharcolon}{\kern0pt}\isanewline
\ \ \ \ {\isachardoublequoteopen}{\isasymforall}h{\isachardot}{\kern0pt}\ h\ {\isacharcolon}{\kern0pt}\ A\ {\isasymrightarrow}\ Y\ {\isasymtimes}\isactrlsub c\ Z\ {\isasymand}\ left{\isacharunderscore}{\kern0pt}cart{\isacharunderscore}{\kern0pt}proj\ Y\ Z\ {\isasymcirc}\isactrlsub c\ h\ {\isacharequal}{\kern0pt}\ f\ {\isasymcirc}\isactrlsub c\ a\ {\isasymand}\ right{\isacharunderscore}{\kern0pt}cart{\isacharunderscore}{\kern0pt}proj\ Y\ Z\ {\isasymcirc}\isactrlsub c\ h\ {\isacharequal}{\kern0pt}\ g\ {\isasymcirc}\isactrlsub c\ b\ {\isasymlongrightarrow}\ \isanewline
\ \ \ \ \ \ h\ {\isacharequal}{\kern0pt}\ {\isasymlangle}f\ {\isasymcirc}\isactrlsub c\ a{\isacharcomma}{\kern0pt}\ g\ {\isasymcirc}\isactrlsub c\ b{\isasymrangle}{\isachardoublequoteclose}\isanewline
\ \ \ \ \isacommand{using}\isamarkupfalse%
\ assms\ comp{\isacharunderscore}{\kern0pt}type\ \isacommand{by}\isamarkupfalse%
\ blast\isanewline
\isanewline
\ \ \isacommand{have}\isamarkupfalse%
\ {\isachardoublequoteopen}left{\isacharunderscore}{\kern0pt}cart{\isacharunderscore}{\kern0pt}proj\ Y\ Z\ {\isasymcirc}\isactrlsub c\ {\isacharparenleft}{\kern0pt}f\ {\isasymtimes}\isactrlsub f\ g{\isacharparenright}{\kern0pt}\ {\isasymcirc}\isactrlsub c\ {\isasymlangle}a{\isacharcomma}{\kern0pt}\ b{\isasymrangle}\ {\isacharequal}{\kern0pt}\ f\ {\isasymcirc}\isactrlsub c\ left{\isacharunderscore}{\kern0pt}cart{\isacharunderscore}{\kern0pt}proj\ W\ X\ {\isasymcirc}\isactrlsub c\ {\isasymlangle}a{\isacharcomma}{\kern0pt}\ b{\isasymrangle}{\isachardoublequoteclose}\isanewline
\ \ \ \ \isacommand{using}\isamarkupfalse%
\ assms\ \isacommand{by}\isamarkupfalse%
\ {\isacharparenleft}{\kern0pt}typecheck{\isacharunderscore}{\kern0pt}cfuncs{\isacharcomma}{\kern0pt}\ simp\ add{\isacharcolon}{\kern0pt}\ comp{\isacharunderscore}{\kern0pt}associative{\isadigit{2}}\ left{\isacharunderscore}{\kern0pt}cart{\isacharunderscore}{\kern0pt}proj{\isacharunderscore}{\kern0pt}cfunc{\isacharunderscore}{\kern0pt}cross{\isacharunderscore}{\kern0pt}prod{\isacharparenright}{\kern0pt}\isanewline
\ \ \isacommand{then}\isamarkupfalse%
\ \isacommand{have}\isamarkupfalse%
\ left{\isacharunderscore}{\kern0pt}eq{\isacharcolon}{\kern0pt}\ {\isachardoublequoteopen}left{\isacharunderscore}{\kern0pt}cart{\isacharunderscore}{\kern0pt}proj\ Y\ Z\ {\isasymcirc}\isactrlsub c\ {\isacharparenleft}{\kern0pt}f\ {\isasymtimes}\isactrlsub f\ g{\isacharparenright}{\kern0pt}\ {\isasymcirc}\isactrlsub c\ {\isasymlangle}a{\isacharcomma}{\kern0pt}\ b{\isasymrangle}\ {\isacharequal}{\kern0pt}\ f\ {\isasymcirc}\isactrlsub c\ a{\isachardoublequoteclose}\isanewline
\ \ \ \ \isacommand{using}\isamarkupfalse%
\ a{\isacharunderscore}{\kern0pt}type\ b{\isacharunderscore}{\kern0pt}type\ left{\isacharunderscore}{\kern0pt}cart{\isacharunderscore}{\kern0pt}proj{\isacharunderscore}{\kern0pt}cfunc{\isacharunderscore}{\kern0pt}prod\ \isacommand{by}\isamarkupfalse%
\ auto\isanewline
\ \ \isanewline
\ \ \isacommand{have}\isamarkupfalse%
\ {\isachardoublequoteopen}right{\isacharunderscore}{\kern0pt}cart{\isacharunderscore}{\kern0pt}proj\ Y\ Z\ {\isasymcirc}\isactrlsub c\ {\isacharparenleft}{\kern0pt}f\ {\isasymtimes}\isactrlsub f\ g{\isacharparenright}{\kern0pt}\ {\isasymcirc}\isactrlsub c\ {\isasymlangle}a{\isacharcomma}{\kern0pt}\ b{\isasymrangle}\ {\isacharequal}{\kern0pt}\ g\ {\isasymcirc}\isactrlsub c\ right{\isacharunderscore}{\kern0pt}cart{\isacharunderscore}{\kern0pt}proj\ W\ X\ {\isasymcirc}\isactrlsub c\ {\isasymlangle}a{\isacharcomma}{\kern0pt}\ b{\isasymrangle}{\isachardoublequoteclose}\isanewline
\ \ \ \ \isacommand{using}\isamarkupfalse%
\ assms\ \isacommand{by}\isamarkupfalse%
\ {\isacharparenleft}{\kern0pt}typecheck{\isacharunderscore}{\kern0pt}cfuncs{\isacharcomma}{\kern0pt}\ simp\ add{\isacharcolon}{\kern0pt}\ comp{\isacharunderscore}{\kern0pt}associative{\isadigit{2}}\ right{\isacharunderscore}{\kern0pt}cart{\isacharunderscore}{\kern0pt}proj{\isacharunderscore}{\kern0pt}cfunc{\isacharunderscore}{\kern0pt}cross{\isacharunderscore}{\kern0pt}prod{\isacharparenright}{\kern0pt}\isanewline
\ \ \isacommand{then}\isamarkupfalse%
\ \isacommand{have}\isamarkupfalse%
\ right{\isacharunderscore}{\kern0pt}eq{\isacharcolon}{\kern0pt}\ {\isachardoublequoteopen}right{\isacharunderscore}{\kern0pt}cart{\isacharunderscore}{\kern0pt}proj\ Y\ Z\ {\isasymcirc}\isactrlsub c\ {\isacharparenleft}{\kern0pt}f\ {\isasymtimes}\isactrlsub f\ g{\isacharparenright}{\kern0pt}\ {\isasymcirc}\isactrlsub c\ {\isasymlangle}a{\isacharcomma}{\kern0pt}\ b{\isasymrangle}\ {\isacharequal}{\kern0pt}\ g\ {\isasymcirc}\isactrlsub c\ b{\isachardoublequoteclose}\isanewline
\ \ \ \ \isacommand{using}\isamarkupfalse%
\ a{\isacharunderscore}{\kern0pt}type\ b{\isacharunderscore}{\kern0pt}type\ right{\isacharunderscore}{\kern0pt}cart{\isacharunderscore}{\kern0pt}proj{\isacharunderscore}{\kern0pt}cfunc{\isacharunderscore}{\kern0pt}prod\ \isacommand{by}\isamarkupfalse%
\ auto\isanewline
\isanewline
\ \ \isacommand{show}\isamarkupfalse%
\ {\isachardoublequoteopen}{\isacharparenleft}{\kern0pt}f\ {\isasymtimes}\isactrlsub f\ g{\isacharparenright}{\kern0pt}\ {\isasymcirc}\isactrlsub c\ {\isasymlangle}a{\isacharcomma}{\kern0pt}b{\isasymrangle}\ {\isacharequal}{\kern0pt}\ {\isasymlangle}f\ {\isasymcirc}\isactrlsub c\ a{\isacharcomma}{\kern0pt}g\ {\isasymcirc}\isactrlsub c\ b{\isasymrangle}{\isachardoublequoteclose}\isanewline
\ \ \ \ \isacommand{using}\isamarkupfalse%
\ uniqueness\ left{\isacharunderscore}{\kern0pt}eq\ right{\isacharunderscore}{\kern0pt}eq\ assms\ \isacommand{by}\isamarkupfalse%
\ {\isacharparenleft}{\kern0pt}erule{\isacharunderscore}{\kern0pt}tac\ x{\isacharequal}{\kern0pt}{\isachardoublequoteopen}f\ {\isasymtimes}\isactrlsub f\ g\ {\isasymcirc}\isactrlsub c\ {\isasymlangle}a{\isacharcomma}{\kern0pt}b{\isasymrangle}{\isachardoublequoteclose}\ \isakeyword{in}\ allE{\isacharcomma}{\kern0pt}\isanewline
\ \ \ \ \ \ \ \ \ \ \ \ \ \ \ \ \ \ \ \ \ \ meson\ cfunc{\isacharunderscore}{\kern0pt}cross{\isacharunderscore}{\kern0pt}prod{\isacharunderscore}{\kern0pt}type\ cfunc{\isacharunderscore}{\kern0pt}prod{\isacharunderscore}{\kern0pt}type\ comp{\isacharunderscore}{\kern0pt}type\ uniqueness{\isacharparenright}{\kern0pt}\isanewline
\isacommand{qed}\isamarkupfalse%
%
\endisatagproof
{\isafoldproof}%
%
\isadelimproof
\isanewline
%
\endisadelimproof
\isanewline
\isacommand{lemma}\isamarkupfalse%
\ cfunc{\isacharunderscore}{\kern0pt}prod{\isacharunderscore}{\kern0pt}comp{\isacharcolon}{\kern0pt}\isanewline
\ \ \isakeyword{assumes}\ f{\isacharunderscore}{\kern0pt}type{\isacharcolon}{\kern0pt}\ {\isachardoublequoteopen}f\ {\isacharcolon}{\kern0pt}\ X\ {\isasymrightarrow}\ Y{\isachardoublequoteclose}\isanewline
\ \ \isakeyword{assumes}\ a{\isacharunderscore}{\kern0pt}type{\isacharcolon}{\kern0pt}\ {\isachardoublequoteopen}a\ {\isacharcolon}{\kern0pt}\ Y\ {\isasymrightarrow}\ A{\isachardoublequoteclose}\ \isakeyword{and}\ b{\isacharunderscore}{\kern0pt}type{\isacharcolon}{\kern0pt}\ {\isachardoublequoteopen}b\ {\isacharcolon}{\kern0pt}\ Y\ {\isasymrightarrow}\ B{\isachardoublequoteclose}\isanewline
\ \ \isakeyword{shows}\ {\isachardoublequoteopen}{\isasymlangle}a{\isacharcomma}{\kern0pt}\ b{\isasymrangle}\ {\isasymcirc}\isactrlsub c\ f\ {\isacharequal}{\kern0pt}\ {\isasymlangle}a\ {\isasymcirc}\isactrlsub c\ f{\isacharcomma}{\kern0pt}\ b\ {\isasymcirc}\isactrlsub c\ f{\isasymrangle}{\isachardoublequoteclose}\isanewline
%
\isadelimproof
%
\endisadelimproof
%
\isatagproof
\isacommand{proof}\isamarkupfalse%
\ {\isacharminus}{\kern0pt}\isanewline
\ \ \isacommand{have}\isamarkupfalse%
\ same{\isacharunderscore}{\kern0pt}left{\isacharunderscore}{\kern0pt}proj{\isacharcolon}{\kern0pt}\ {\isachardoublequoteopen}left{\isacharunderscore}{\kern0pt}cart{\isacharunderscore}{\kern0pt}proj\ A\ B\ {\isasymcirc}\isactrlsub c\ {\isasymlangle}a{\isacharcomma}{\kern0pt}\ b{\isasymrangle}\ {\isasymcirc}\isactrlsub c\ f\ {\isacharequal}{\kern0pt}\ a\ {\isasymcirc}\isactrlsub c\ f{\isachardoublequoteclose}\isanewline
\ \ \ \ \isacommand{using}\isamarkupfalse%
\ assms\ \isacommand{by}\isamarkupfalse%
\ {\isacharparenleft}{\kern0pt}typecheck{\isacharunderscore}{\kern0pt}cfuncs{\isacharcomma}{\kern0pt}\ simp\ add{\isacharcolon}{\kern0pt}\ comp{\isacharunderscore}{\kern0pt}associative{\isadigit{2}}\ left{\isacharunderscore}{\kern0pt}cart{\isacharunderscore}{\kern0pt}proj{\isacharunderscore}{\kern0pt}cfunc{\isacharunderscore}{\kern0pt}prod{\isacharparenright}{\kern0pt}\isanewline
\ \ \isacommand{have}\isamarkupfalse%
\ same{\isacharunderscore}{\kern0pt}right{\isacharunderscore}{\kern0pt}proj{\isacharcolon}{\kern0pt}\ {\isachardoublequoteopen}right{\isacharunderscore}{\kern0pt}cart{\isacharunderscore}{\kern0pt}proj\ A\ B\ {\isasymcirc}\isactrlsub c\ {\isasymlangle}a{\isacharcomma}{\kern0pt}\ b{\isasymrangle}\ {\isasymcirc}\isactrlsub c\ f\ {\isacharequal}{\kern0pt}\ b\ {\isasymcirc}\isactrlsub c\ f{\isachardoublequoteclose}\isanewline
\ \ \ \ \isacommand{using}\isamarkupfalse%
\ assms\ comp{\isacharunderscore}{\kern0pt}associative{\isadigit{2}}\ right{\isacharunderscore}{\kern0pt}cart{\isacharunderscore}{\kern0pt}proj{\isacharunderscore}{\kern0pt}cfunc{\isacharunderscore}{\kern0pt}prod\ \isacommand{by}\isamarkupfalse%
\ {\isacharparenleft}{\kern0pt}typecheck{\isacharunderscore}{\kern0pt}cfuncs{\isacharcomma}{\kern0pt}\ auto{\isacharparenright}{\kern0pt}\isanewline
\ \ \isacommand{show}\isamarkupfalse%
\ {\isachardoublequoteopen}{\isasymlangle}a{\isacharcomma}{\kern0pt}b{\isasymrangle}\ {\isasymcirc}\isactrlsub c\ f\ {\isacharequal}{\kern0pt}\ {\isasymlangle}a\ {\isasymcirc}\isactrlsub c\ f{\isacharcomma}{\kern0pt}\ b\ {\isasymcirc}\isactrlsub c\ f{\isasymrangle}{\isachardoublequoteclose}\isanewline
\ \ \ \ \isacommand{by}\isamarkupfalse%
\ {\isacharparenleft}{\kern0pt}typecheck{\isacharunderscore}{\kern0pt}cfuncs{\isacharcomma}{\kern0pt}\ metis\ a{\isacharunderscore}{\kern0pt}type\ b{\isacharunderscore}{\kern0pt}type\ cfunc{\isacharunderscore}{\kern0pt}prod{\isacharunderscore}{\kern0pt}unique\ f{\isacharunderscore}{\kern0pt}type\ same{\isacharunderscore}{\kern0pt}left{\isacharunderscore}{\kern0pt}proj\ same{\isacharunderscore}{\kern0pt}right{\isacharunderscore}{\kern0pt}proj{\isacharparenright}{\kern0pt}\isanewline
\isacommand{qed}\isamarkupfalse%
%
\endisatagproof
{\isafoldproof}%
%
\isadelimproof
%
\endisadelimproof
%
\begin{isamarkuptext}%
The lemma below corresponds to Exercise 2.1.12 in Halvorson.%
\end{isamarkuptext}\isamarkuptrue%
\isacommand{lemma}\isamarkupfalse%
\ id{\isacharunderscore}{\kern0pt}cross{\isacharunderscore}{\kern0pt}prod{\isacharcolon}{\kern0pt}\ {\isachardoublequoteopen}id{\isacharparenleft}{\kern0pt}X{\isacharparenright}{\kern0pt}\ {\isasymtimes}\isactrlsub f\ id{\isacharparenleft}{\kern0pt}Y{\isacharparenright}{\kern0pt}\ {\isacharequal}{\kern0pt}\ id{\isacharparenleft}{\kern0pt}X\ {\isasymtimes}\isactrlsub c\ Y{\isacharparenright}{\kern0pt}{\isachardoublequoteclose}\isanewline
%
\isadelimproof
\ \ %
\endisadelimproof
%
\isatagproof
\isacommand{by}\isamarkupfalse%
\ {\isacharparenleft}{\kern0pt}typecheck{\isacharunderscore}{\kern0pt}cfuncs{\isacharcomma}{\kern0pt}\ smt\ {\isacharparenleft}{\kern0pt}z{\isadigit{3}}{\isacharparenright}{\kern0pt}\ cfunc{\isacharunderscore}{\kern0pt}cross{\isacharunderscore}{\kern0pt}prod{\isacharunderscore}{\kern0pt}unique\ id{\isacharunderscore}{\kern0pt}left{\isacharunderscore}{\kern0pt}unit{\isadigit{2}}\ id{\isacharunderscore}{\kern0pt}right{\isacharunderscore}{\kern0pt}unit{\isadigit{2}}\ left{\isacharunderscore}{\kern0pt}cart{\isacharunderscore}{\kern0pt}proj{\isacharunderscore}{\kern0pt}type\ right{\isacharunderscore}{\kern0pt}cart{\isacharunderscore}{\kern0pt}proj{\isacharunderscore}{\kern0pt}type{\isacharparenright}{\kern0pt}%
\endisatagproof
{\isafoldproof}%
%
\isadelimproof
%
\endisadelimproof
%
\begin{isamarkuptext}%
The lemma below corresponds to Exercise 2.1.14 in Halvorson.%
\end{isamarkuptext}\isamarkuptrue%
\isacommand{lemma}\isamarkupfalse%
\ cfunc{\isacharunderscore}{\kern0pt}cross{\isacharunderscore}{\kern0pt}prod{\isacharunderscore}{\kern0pt}comp{\isacharunderscore}{\kern0pt}diagonal{\isacharcolon}{\kern0pt}\isanewline
\ \ \isakeyword{assumes}\ {\isachardoublequoteopen}f{\isacharcolon}{\kern0pt}\ X\ {\isasymrightarrow}\ Y{\isachardoublequoteclose}\ \isanewline
\ \ \isakeyword{shows}\ {\isachardoublequoteopen}{\isacharparenleft}{\kern0pt}f\ {\isasymtimes}\isactrlsub f\ f{\isacharparenright}{\kern0pt}\ {\isasymcirc}\isactrlsub c\ diagonal{\isacharparenleft}{\kern0pt}X{\isacharparenright}{\kern0pt}\ {\isacharequal}{\kern0pt}\ diagonal{\isacharparenleft}{\kern0pt}Y{\isacharparenright}{\kern0pt}\ {\isasymcirc}\isactrlsub c\ f{\isachardoublequoteclose}\isanewline
%
\isadelimproof
\ \ %
\endisadelimproof
%
\isatagproof
\isacommand{unfolding}\isamarkupfalse%
\ diagonal{\isacharunderscore}{\kern0pt}def\isanewline
\isacommand{proof}\isamarkupfalse%
\ {\isacharminus}{\kern0pt}\isanewline
\ \ \isacommand{have}\isamarkupfalse%
\ {\isachardoublequoteopen}{\isacharparenleft}{\kern0pt}f\ {\isasymtimes}\isactrlsub f\ f{\isacharparenright}{\kern0pt}\ {\isasymcirc}\isactrlsub c\ {\isasymlangle}id\isactrlsub c\ X{\isacharcomma}{\kern0pt}\ id\isactrlsub c\ X{\isasymrangle}\ {\isacharequal}{\kern0pt}\ {\isasymlangle}f\ {\isasymcirc}\isactrlsub c\ id\isactrlsub c\ X{\isacharcomma}{\kern0pt}\ f\ {\isasymcirc}\isactrlsub c\ id\isactrlsub c\ X{\isasymrangle}{\isachardoublequoteclose}\isanewline
\ \ \ \ \isacommand{using}\isamarkupfalse%
\ assms\ cfunc{\isacharunderscore}{\kern0pt}cross{\isacharunderscore}{\kern0pt}prod{\isacharunderscore}{\kern0pt}comp{\isacharunderscore}{\kern0pt}cfunc{\isacharunderscore}{\kern0pt}prod\ id{\isacharunderscore}{\kern0pt}type\ \isacommand{by}\isamarkupfalse%
\ blast\isanewline
\ \ \isacommand{also}\isamarkupfalse%
\ \isacommand{have}\isamarkupfalse%
\ {\isachardoublequoteopen}{\isachardot}{\kern0pt}{\isachardot}{\kern0pt}{\isachardot}{\kern0pt}\ {\isacharequal}{\kern0pt}\ {\isasymlangle}f{\isacharcomma}{\kern0pt}\ f{\isasymrangle}{\isachardoublequoteclose}\isanewline
\ \ \ \ \isacommand{using}\isamarkupfalse%
\ assms\ cfunc{\isacharunderscore}{\kern0pt}type{\isacharunderscore}{\kern0pt}def\ id{\isacharunderscore}{\kern0pt}right{\isacharunderscore}{\kern0pt}unit\ \isacommand{by}\isamarkupfalse%
\ auto\isanewline
\ \ \isacommand{also}\isamarkupfalse%
\ \isacommand{have}\isamarkupfalse%
\ {\isachardoublequoteopen}{\isachardot}{\kern0pt}{\isachardot}{\kern0pt}{\isachardot}{\kern0pt}\ {\isacharequal}{\kern0pt}\ {\isasymlangle}id\isactrlsub c\ Y\ {\isasymcirc}\isactrlsub c\ f{\isacharcomma}{\kern0pt}\ id\isactrlsub c\ Y\ {\isasymcirc}\isactrlsub c\ f{\isasymrangle}{\isachardoublequoteclose}\isanewline
\ \ \ \ \isacommand{using}\isamarkupfalse%
\ assms\ cfunc{\isacharunderscore}{\kern0pt}type{\isacharunderscore}{\kern0pt}def\ id{\isacharunderscore}{\kern0pt}left{\isacharunderscore}{\kern0pt}unit\ \isacommand{by}\isamarkupfalse%
\ auto\isanewline
\ \ \isacommand{also}\isamarkupfalse%
\ \isacommand{have}\isamarkupfalse%
\ {\isachardoublequoteopen}{\isachardot}{\kern0pt}{\isachardot}{\kern0pt}{\isachardot}{\kern0pt}\ {\isacharequal}{\kern0pt}\ {\isasymlangle}id\isactrlsub c\ Y{\isacharcomma}{\kern0pt}\ id\isactrlsub c\ Y{\isasymrangle}\ {\isasymcirc}\isactrlsub c\ f{\isachardoublequoteclose}\isanewline
\ \ \ \ \isacommand{using}\isamarkupfalse%
\ assms\ cfunc{\isacharunderscore}{\kern0pt}prod{\isacharunderscore}{\kern0pt}comp\ id{\isacharunderscore}{\kern0pt}type\ \isacommand{by}\isamarkupfalse%
\ fastforce\isanewline
\ \ \isacommand{then}\isamarkupfalse%
\ \isacommand{show}\isamarkupfalse%
\ {\isachardoublequoteopen}{\isacharparenleft}{\kern0pt}f\ {\isasymtimes}\isactrlsub f\ f{\isacharparenright}{\kern0pt}\ {\isasymcirc}\isactrlsub c\ {\isasymlangle}id\isactrlsub c\ X{\isacharcomma}{\kern0pt}id\isactrlsub c\ X{\isasymrangle}\ {\isacharequal}{\kern0pt}\ {\isasymlangle}id\isactrlsub c\ Y{\isacharcomma}{\kern0pt}id\isactrlsub c\ Y{\isasymrangle}\ {\isasymcirc}\isactrlsub c\ f{\isachardoublequoteclose}\isanewline
\ \ \ \ \isacommand{using}\isamarkupfalse%
\ calculation\ \isacommand{by}\isamarkupfalse%
\ auto\isanewline
\isacommand{qed}\isamarkupfalse%
%
\endisatagproof
{\isafoldproof}%
%
\isadelimproof
\isanewline
%
\endisadelimproof
\isanewline
\isacommand{lemma}\isamarkupfalse%
\ cfunc{\isacharunderscore}{\kern0pt}cross{\isacharunderscore}{\kern0pt}prod{\isacharunderscore}{\kern0pt}comp{\isacharunderscore}{\kern0pt}cfunc{\isacharunderscore}{\kern0pt}cross{\isacharunderscore}{\kern0pt}prod{\isacharcolon}{\kern0pt}\isanewline
\ \ \isakeyword{assumes}\ {\isachardoublequoteopen}a\ {\isacharcolon}{\kern0pt}\ A\ {\isasymrightarrow}\ X{\isachardoublequoteclose}\ {\isachardoublequoteopen}b\ {\isacharcolon}{\kern0pt}\ B\ {\isasymrightarrow}\ Y{\isachardoublequoteclose}\ {\isachardoublequoteopen}x\ {\isacharcolon}{\kern0pt}\ X\ {\isasymrightarrow}\ Z{\isachardoublequoteclose}\ {\isachardoublequoteopen}y\ {\isacharcolon}{\kern0pt}\ Y\ {\isasymrightarrow}\ W{\isachardoublequoteclose}\isanewline
\ \ \isakeyword{shows}\ {\isachardoublequoteopen}{\isacharparenleft}{\kern0pt}x\ {\isasymtimes}\isactrlsub f\ y{\isacharparenright}{\kern0pt}\ {\isasymcirc}\isactrlsub c\ {\isacharparenleft}{\kern0pt}a\ {\isasymtimes}\isactrlsub f\ b{\isacharparenright}{\kern0pt}\ {\isacharequal}{\kern0pt}\ {\isacharparenleft}{\kern0pt}x\ {\isasymcirc}\isactrlsub c\ a{\isacharparenright}{\kern0pt}\ {\isasymtimes}\isactrlsub f\ {\isacharparenleft}{\kern0pt}y\ {\isasymcirc}\isactrlsub c\ b{\isacharparenright}{\kern0pt}{\isachardoublequoteclose}\isanewline
%
\isadelimproof
%
\endisadelimproof
%
\isatagproof
\isacommand{proof}\isamarkupfalse%
\ {\isacharminus}{\kern0pt}\isanewline
\ \ \isacommand{have}\isamarkupfalse%
\ {\isachardoublequoteopen}{\isacharparenleft}{\kern0pt}x\ {\isasymtimes}\isactrlsub f\ y{\isacharparenright}{\kern0pt}\ {\isasymcirc}\isactrlsub c\ {\isasymlangle}a\ {\isasymcirc}\isactrlsub c\ left{\isacharunderscore}{\kern0pt}cart{\isacharunderscore}{\kern0pt}proj\ A\ B\ {\isacharcomma}{\kern0pt}\ b\ {\isasymcirc}\isactrlsub c\ right{\isacharunderscore}{\kern0pt}cart{\isacharunderscore}{\kern0pt}proj\ A\ B{\isasymrangle}\isanewline
\ \ \ \ \ \ {\isacharequal}{\kern0pt}\ {\isasymlangle}x\ {\isasymcirc}\isactrlsub c\ a\ {\isasymcirc}\isactrlsub c\ left{\isacharunderscore}{\kern0pt}cart{\isacharunderscore}{\kern0pt}proj\ A\ B{\isacharcomma}{\kern0pt}\ y\ {\isasymcirc}\isactrlsub c\ b\ {\isasymcirc}\isactrlsub c\ right{\isacharunderscore}{\kern0pt}cart{\isacharunderscore}{\kern0pt}proj\ A\ B{\isasymrangle}{\isachardoublequoteclose}\isanewline
\ \ \ \ \isacommand{by}\isamarkupfalse%
\ {\isacharparenleft}{\kern0pt}meson\ assms\ cfunc{\isacharunderscore}{\kern0pt}cross{\isacharunderscore}{\kern0pt}prod{\isacharunderscore}{\kern0pt}comp{\isacharunderscore}{\kern0pt}cfunc{\isacharunderscore}{\kern0pt}prod\ comp{\isacharunderscore}{\kern0pt}type\ left{\isacharunderscore}{\kern0pt}cart{\isacharunderscore}{\kern0pt}proj{\isacharunderscore}{\kern0pt}type\ right{\isacharunderscore}{\kern0pt}cart{\isacharunderscore}{\kern0pt}proj{\isacharunderscore}{\kern0pt}type{\isacharparenright}{\kern0pt}\isanewline
\ \ \isacommand{then}\isamarkupfalse%
\ \isacommand{show}\isamarkupfalse%
\ {\isachardoublequoteopen}{\isacharparenleft}{\kern0pt}x\ {\isasymtimes}\isactrlsub f\ y{\isacharparenright}{\kern0pt}\ {\isasymcirc}\isactrlsub c\ a\ {\isasymtimes}\isactrlsub f\ b\ {\isacharequal}{\kern0pt}\ {\isacharparenleft}{\kern0pt}x\ {\isasymcirc}\isactrlsub c\ a{\isacharparenright}{\kern0pt}\ {\isasymtimes}\isactrlsub f\ y\ {\isasymcirc}\isactrlsub c\ b{\isachardoublequoteclose}\isanewline
\ \ \ \ \isacommand{by}\isamarkupfalse%
\ {\isacharparenleft}{\kern0pt}typecheck{\isacharunderscore}{\kern0pt}cfuncs{\isacharcomma}{\kern0pt}smt\ {\isacharparenleft}{\kern0pt}z{\isadigit{3}}{\isacharparenright}{\kern0pt}\ assms\ cfunc{\isacharunderscore}{\kern0pt}cross{\isacharunderscore}{\kern0pt}prod{\isacharunderscore}{\kern0pt}def{\isadigit{2}}\ comp{\isacharunderscore}{\kern0pt}associative{\isadigit{2}}\ left{\isacharunderscore}{\kern0pt}cart{\isacharunderscore}{\kern0pt}proj{\isacharunderscore}{\kern0pt}type\ right{\isacharunderscore}{\kern0pt}cart{\isacharunderscore}{\kern0pt}proj{\isacharunderscore}{\kern0pt}type{\isacharparenright}{\kern0pt}\isanewline
\isacommand{qed}\isamarkupfalse%
%
\endisatagproof
{\isafoldproof}%
%
\isadelimproof
\isanewline
%
\endisadelimproof
\isanewline
\isacommand{lemma}\isamarkupfalse%
\ cfunc{\isacharunderscore}{\kern0pt}cross{\isacharunderscore}{\kern0pt}prod{\isacharunderscore}{\kern0pt}mono{\isacharcolon}{\kern0pt}\isanewline
\ \ \isakeyword{assumes}\ type{\isacharunderscore}{\kern0pt}assms{\isacharcolon}{\kern0pt}\ {\isachardoublequoteopen}f\ {\isacharcolon}{\kern0pt}\ X\ {\isasymrightarrow}\ Y{\isachardoublequoteclose}\ {\isachardoublequoteopen}g\ {\isacharcolon}{\kern0pt}\ Z\ {\isasymrightarrow}\ W{\isachardoublequoteclose}\isanewline
\ \ \isakeyword{assumes}\ f{\isacharunderscore}{\kern0pt}mono{\isacharcolon}{\kern0pt}\ {\isachardoublequoteopen}monomorphism\ f{\isachardoublequoteclose}\ \isakeyword{and}\ g{\isacharunderscore}{\kern0pt}mono{\isacharcolon}{\kern0pt}\ {\isachardoublequoteopen}monomorphism\ g{\isachardoublequoteclose}\isanewline
\ \ \isakeyword{shows}\ {\isachardoublequoteopen}monomorphism\ {\isacharparenleft}{\kern0pt}f\ {\isasymtimes}\isactrlsub f\ g{\isacharparenright}{\kern0pt}{\isachardoublequoteclose}\isanewline
%
\isadelimproof
\ \ %
\endisadelimproof
%
\isatagproof
\isacommand{using}\isamarkupfalse%
\ type{\isacharunderscore}{\kern0pt}assms\isanewline
\isacommand{proof}\isamarkupfalse%
\ {\isacharparenleft}{\kern0pt}typecheck{\isacharunderscore}{\kern0pt}cfuncs{\isacharcomma}{\kern0pt}\ unfold\ monomorphism{\isacharunderscore}{\kern0pt}def{\isadigit{3}}{\isacharcomma}{\kern0pt}\ clarify{\isacharparenright}{\kern0pt}\isanewline
\ \ \isacommand{fix}\isamarkupfalse%
\ x\ y\ A\isanewline
\ \ \isacommand{assume}\isamarkupfalse%
\ x{\isacharunderscore}{\kern0pt}type{\isacharcolon}{\kern0pt}\ {\isachardoublequoteopen}x\ {\isacharcolon}{\kern0pt}\ A\ {\isasymrightarrow}\ X\ {\isasymtimes}\isactrlsub c\ Z{\isachardoublequoteclose}\isanewline
\ \ \isacommand{assume}\isamarkupfalse%
\ y{\isacharunderscore}{\kern0pt}type{\isacharcolon}{\kern0pt}\ {\isachardoublequoteopen}y\ {\isacharcolon}{\kern0pt}\ A\ {\isasymrightarrow}\ X\ {\isasymtimes}\isactrlsub c\ Z{\isachardoublequoteclose}\isanewline
\isanewline
\ \ \isacommand{obtain}\isamarkupfalse%
\ x{\isadigit{1}}\ x{\isadigit{2}}\ \isakeyword{where}\ x{\isacharunderscore}{\kern0pt}expand{\isacharcolon}{\kern0pt}\ {\isachardoublequoteopen}x\ {\isacharequal}{\kern0pt}\ {\isasymlangle}x{\isadigit{1}}{\isacharcomma}{\kern0pt}\ x{\isadigit{2}}{\isasymrangle}{\isachardoublequoteclose}\ \isakeyword{and}\ x{\isadigit{1}}{\isacharunderscore}{\kern0pt}x{\isadigit{2}}{\isacharunderscore}{\kern0pt}type{\isacharcolon}{\kern0pt}\ {\isachardoublequoteopen}x{\isadigit{1}}\ {\isacharcolon}{\kern0pt}\ A\ {\isasymrightarrow}\ X{\isachardoublequoteclose}\ {\isachardoublequoteopen}x{\isadigit{2}}\ {\isacharcolon}{\kern0pt}\ A\ {\isasymrightarrow}\ Z{\isachardoublequoteclose}\isanewline
\ \ \ \ \isacommand{using}\isamarkupfalse%
\ cfunc{\isacharunderscore}{\kern0pt}prod{\isacharunderscore}{\kern0pt}unique\ comp{\isacharunderscore}{\kern0pt}type\ left{\isacharunderscore}{\kern0pt}cart{\isacharunderscore}{\kern0pt}proj{\isacharunderscore}{\kern0pt}type\ right{\isacharunderscore}{\kern0pt}cart{\isacharunderscore}{\kern0pt}proj{\isacharunderscore}{\kern0pt}type\ x{\isacharunderscore}{\kern0pt}type\ \isacommand{by}\isamarkupfalse%
\ blast\isanewline
\ \ \isacommand{obtain}\isamarkupfalse%
\ y{\isadigit{1}}\ y{\isadigit{2}}\ \isakeyword{where}\ y{\isacharunderscore}{\kern0pt}expand{\isacharcolon}{\kern0pt}\ {\isachardoublequoteopen}y\ {\isacharequal}{\kern0pt}\ {\isasymlangle}y{\isadigit{1}}{\isacharcomma}{\kern0pt}\ y{\isadigit{2}}{\isasymrangle}{\isachardoublequoteclose}\ \isakeyword{and}\ y{\isadigit{1}}{\isacharunderscore}{\kern0pt}y{\isadigit{2}}{\isacharunderscore}{\kern0pt}type{\isacharcolon}{\kern0pt}\ {\isachardoublequoteopen}y{\isadigit{1}}\ {\isacharcolon}{\kern0pt}\ A\ {\isasymrightarrow}\ X{\isachardoublequoteclose}\ {\isachardoublequoteopen}y{\isadigit{2}}\ {\isacharcolon}{\kern0pt}\ A\ {\isasymrightarrow}\ Z{\isachardoublequoteclose}\isanewline
\ \ \ \ \isacommand{using}\isamarkupfalse%
\ cfunc{\isacharunderscore}{\kern0pt}prod{\isacharunderscore}{\kern0pt}unique\ comp{\isacharunderscore}{\kern0pt}type\ left{\isacharunderscore}{\kern0pt}cart{\isacharunderscore}{\kern0pt}proj{\isacharunderscore}{\kern0pt}type\ right{\isacharunderscore}{\kern0pt}cart{\isacharunderscore}{\kern0pt}proj{\isacharunderscore}{\kern0pt}type\ y{\isacharunderscore}{\kern0pt}type\ \isacommand{by}\isamarkupfalse%
\ blast\isanewline
\isanewline
\ \ \isacommand{assume}\isamarkupfalse%
\ {\isachardoublequoteopen}{\isacharparenleft}{\kern0pt}f\ {\isasymtimes}\isactrlsub f\ g{\isacharparenright}{\kern0pt}\ {\isasymcirc}\isactrlsub c\ x\ {\isacharequal}{\kern0pt}\ {\isacharparenleft}{\kern0pt}f\ {\isasymtimes}\isactrlsub f\ g{\isacharparenright}{\kern0pt}\ {\isasymcirc}\isactrlsub c\ y{\isachardoublequoteclose}\isanewline
\ \ \isacommand{then}\isamarkupfalse%
\ \isacommand{have}\isamarkupfalse%
\ {\isachardoublequoteopen}{\isacharparenleft}{\kern0pt}f\ {\isasymtimes}\isactrlsub f\ g{\isacharparenright}{\kern0pt}\ {\isasymcirc}\isactrlsub c\ {\isasymlangle}x{\isadigit{1}}{\isacharcomma}{\kern0pt}\ x{\isadigit{2}}{\isasymrangle}\ {\isacharequal}{\kern0pt}\ {\isacharparenleft}{\kern0pt}f\ {\isasymtimes}\isactrlsub f\ g{\isacharparenright}{\kern0pt}\ {\isasymcirc}\isactrlsub c\ {\isasymlangle}y{\isadigit{1}}{\isacharcomma}{\kern0pt}\ y{\isadigit{2}}{\isasymrangle}{\isachardoublequoteclose}\isanewline
\ \ \ \ \isacommand{using}\isamarkupfalse%
\ x{\isacharunderscore}{\kern0pt}expand\ y{\isacharunderscore}{\kern0pt}expand\ \isacommand{by}\isamarkupfalse%
\ blast\isanewline
\ \ \isacommand{then}\isamarkupfalse%
\ \isacommand{have}\isamarkupfalse%
\ {\isachardoublequoteopen}{\isasymlangle}f\ {\isasymcirc}\isactrlsub c\ x{\isadigit{1}}{\isacharcomma}{\kern0pt}\ g\ {\isasymcirc}\isactrlsub c\ x{\isadigit{2}}{\isasymrangle}\ {\isacharequal}{\kern0pt}\ {\isasymlangle}f\ {\isasymcirc}\isactrlsub c\ y{\isadigit{1}}{\isacharcomma}{\kern0pt}\ g\ {\isasymcirc}\isactrlsub c\ y{\isadigit{2}}{\isasymrangle}{\isachardoublequoteclose}\isanewline
\ \ \ \ \isacommand{using}\isamarkupfalse%
\ cfunc{\isacharunderscore}{\kern0pt}cross{\isacharunderscore}{\kern0pt}prod{\isacharunderscore}{\kern0pt}comp{\isacharunderscore}{\kern0pt}cfunc{\isacharunderscore}{\kern0pt}prod\ type{\isacharunderscore}{\kern0pt}assms\ x{\isadigit{1}}{\isacharunderscore}{\kern0pt}x{\isadigit{2}}{\isacharunderscore}{\kern0pt}type\ y{\isadigit{1}}{\isacharunderscore}{\kern0pt}y{\isadigit{2}}{\isacharunderscore}{\kern0pt}type\ \isacommand{by}\isamarkupfalse%
\ auto\isanewline
\ \ \isacommand{then}\isamarkupfalse%
\ \isacommand{have}\isamarkupfalse%
\ {\isachardoublequoteopen}f\ {\isasymcirc}\isactrlsub c\ x{\isadigit{1}}\ {\isacharequal}{\kern0pt}\ f\ {\isasymcirc}\isactrlsub c\ y{\isadigit{1}}\ {\isasymand}\ g\ {\isasymcirc}\isactrlsub c\ x{\isadigit{2}}\ {\isacharequal}{\kern0pt}\ g\ {\isasymcirc}\isactrlsub c\ y{\isadigit{2}}{\isachardoublequoteclose}\isanewline
\ \ \ \ \isacommand{by}\isamarkupfalse%
\ {\isacharparenleft}{\kern0pt}meson\ cart{\isacharunderscore}{\kern0pt}prod{\isacharunderscore}{\kern0pt}eq{\isadigit{2}}\ comp{\isacharunderscore}{\kern0pt}type\ type{\isacharunderscore}{\kern0pt}assms\ x{\isadigit{1}}{\isacharunderscore}{\kern0pt}x{\isadigit{2}}{\isacharunderscore}{\kern0pt}type\ y{\isadigit{1}}{\isacharunderscore}{\kern0pt}y{\isadigit{2}}{\isacharunderscore}{\kern0pt}type{\isacharparenright}{\kern0pt}\isanewline
\ \ \isacommand{then}\isamarkupfalse%
\ \isacommand{have}\isamarkupfalse%
\ {\isachardoublequoteopen}x{\isadigit{1}}\ {\isacharequal}{\kern0pt}\ y{\isadigit{1}}\ {\isasymand}\ x{\isadigit{2}}\ {\isacharequal}{\kern0pt}\ y{\isadigit{2}}{\isachardoublequoteclose}\isanewline
\ \ \ \ \isacommand{using}\isamarkupfalse%
\ cfunc{\isacharunderscore}{\kern0pt}type{\isacharunderscore}{\kern0pt}def\ f{\isacharunderscore}{\kern0pt}mono\ g{\isacharunderscore}{\kern0pt}mono\ monomorphism{\isacharunderscore}{\kern0pt}def\ type{\isacharunderscore}{\kern0pt}assms\ x{\isadigit{1}}{\isacharunderscore}{\kern0pt}x{\isadigit{2}}{\isacharunderscore}{\kern0pt}type\ y{\isadigit{1}}{\isacharunderscore}{\kern0pt}y{\isadigit{2}}{\isacharunderscore}{\kern0pt}type\ \isacommand{by}\isamarkupfalse%
\ auto\isanewline
\ \ \isacommand{then}\isamarkupfalse%
\ \isacommand{have}\isamarkupfalse%
\ {\isachardoublequoteopen}{\isasymlangle}x{\isadigit{1}}{\isacharcomma}{\kern0pt}\ x{\isadigit{2}}{\isasymrangle}\ {\isacharequal}{\kern0pt}\ {\isasymlangle}y{\isadigit{1}}{\isacharcomma}{\kern0pt}\ y{\isadigit{2}}{\isasymrangle}{\isachardoublequoteclose}\isanewline
\ \ \ \ \isacommand{by}\isamarkupfalse%
\ blast\isanewline
\ \ \isacommand{then}\isamarkupfalse%
\ \isacommand{show}\isamarkupfalse%
\ {\isachardoublequoteopen}x\ {\isacharequal}{\kern0pt}\ y{\isachardoublequoteclose}\isanewline
\ \ \ \ \isacommand{by}\isamarkupfalse%
\ {\isacharparenleft}{\kern0pt}simp\ add{\isacharcolon}{\kern0pt}\ x{\isacharunderscore}{\kern0pt}expand\ y{\isacharunderscore}{\kern0pt}expand{\isacharparenright}{\kern0pt}\isanewline
\isacommand{qed}\isamarkupfalse%
%
\endisatagproof
{\isafoldproof}%
%
\isadelimproof
%
\endisadelimproof
%
\isadelimdocument
%
\endisadelimdocument
%
\isatagdocument
%
\isamarkupsubsection{Useful Cartesian Product Permuting Functions%
}
\isamarkuptrue%
%
\isamarkupsubsubsection{Swapping a Cartesian Product%
}
\isamarkuptrue%
%
\endisatagdocument
{\isafolddocument}%
%
\isadelimdocument
%
\endisadelimdocument
\isacommand{definition}\isamarkupfalse%
\ swap\ {\isacharcolon}{\kern0pt}{\isacharcolon}{\kern0pt}\ {\isachardoublequoteopen}cset\ {\isasymRightarrow}\ cset\ {\isasymRightarrow}\ cfunc{\isachardoublequoteclose}\ \isakeyword{where}\isanewline
\ \ {\isachardoublequoteopen}swap\ X\ Y\ {\isacharequal}{\kern0pt}\ {\isasymlangle}right{\isacharunderscore}{\kern0pt}cart{\isacharunderscore}{\kern0pt}proj\ X\ Y{\isacharcomma}{\kern0pt}\ left{\isacharunderscore}{\kern0pt}cart{\isacharunderscore}{\kern0pt}proj\ X\ Y{\isasymrangle}{\isachardoublequoteclose}\isanewline
\isanewline
\isacommand{lemma}\isamarkupfalse%
\ swap{\isacharunderscore}{\kern0pt}type{\isacharbrackleft}{\kern0pt}type{\isacharunderscore}{\kern0pt}rule{\isacharbrackright}{\kern0pt}{\isacharcolon}{\kern0pt}\ {\isachardoublequoteopen}swap\ X\ Y\ {\isacharcolon}{\kern0pt}\ X\ {\isasymtimes}\isactrlsub c\ Y\ {\isasymrightarrow}\ Y\ {\isasymtimes}\isactrlsub c\ X{\isachardoublequoteclose}\isanewline
%
\isadelimproof
\ \ %
\endisadelimproof
%
\isatagproof
\isacommand{unfolding}\isamarkupfalse%
\ swap{\isacharunderscore}{\kern0pt}def\ \isacommand{by}\isamarkupfalse%
\ {\isacharparenleft}{\kern0pt}simp\ add{\isacharcolon}{\kern0pt}\ cfunc{\isacharunderscore}{\kern0pt}prod{\isacharunderscore}{\kern0pt}type\ left{\isacharunderscore}{\kern0pt}cart{\isacharunderscore}{\kern0pt}proj{\isacharunderscore}{\kern0pt}type\ right{\isacharunderscore}{\kern0pt}cart{\isacharunderscore}{\kern0pt}proj{\isacharunderscore}{\kern0pt}type{\isacharparenright}{\kern0pt}%
\endisatagproof
{\isafoldproof}%
%
\isadelimproof
\isanewline
%
\endisadelimproof
\isanewline
\isacommand{lemma}\isamarkupfalse%
\ swap{\isacharunderscore}{\kern0pt}ap{\isacharcolon}{\kern0pt}\isanewline
\ \ \isakeyword{assumes}\ {\isachardoublequoteopen}x\ {\isacharcolon}{\kern0pt}\ A\ {\isasymrightarrow}\ X{\isachardoublequoteclose}\ {\isachardoublequoteopen}y\ {\isacharcolon}{\kern0pt}\ A\ {\isasymrightarrow}\ Y{\isachardoublequoteclose}\isanewline
\ \ \isakeyword{shows}\ {\isachardoublequoteopen}swap\ X\ Y\ {\isasymcirc}\isactrlsub c\ {\isasymlangle}x{\isacharcomma}{\kern0pt}\ y{\isasymrangle}\ {\isacharequal}{\kern0pt}\ {\isasymlangle}y{\isacharcomma}{\kern0pt}\ x{\isasymrangle}{\isachardoublequoteclose}\isanewline
%
\isadelimproof
%
\endisadelimproof
%
\isatagproof
\isacommand{proof}\isamarkupfalse%
\ {\isacharminus}{\kern0pt}\isanewline
\ \ \isacommand{have}\isamarkupfalse%
\ {\isachardoublequoteopen}swap\ X\ Y\ {\isasymcirc}\isactrlsub c\ {\isasymlangle}x{\isacharcomma}{\kern0pt}\ y{\isasymrangle}\ {\isacharequal}{\kern0pt}\ {\isasymlangle}right{\isacharunderscore}{\kern0pt}cart{\isacharunderscore}{\kern0pt}proj\ X\ Y{\isacharcomma}{\kern0pt}left{\isacharunderscore}{\kern0pt}cart{\isacharunderscore}{\kern0pt}proj\ X\ Y{\isasymrangle}\ {\isasymcirc}\isactrlsub c\ {\isasymlangle}x{\isacharcomma}{\kern0pt}y{\isasymrangle}{\isachardoublequoteclose}\isanewline
\ \ \ \ \isacommand{unfolding}\isamarkupfalse%
\ swap{\isacharunderscore}{\kern0pt}def\ \isacommand{by}\isamarkupfalse%
\ auto\isanewline
\ \ \isacommand{also}\isamarkupfalse%
\ \isacommand{have}\isamarkupfalse%
\ {\isachardoublequoteopen}{\isachardot}{\kern0pt}{\isachardot}{\kern0pt}{\isachardot}{\kern0pt}\ {\isacharequal}{\kern0pt}\ {\isasymlangle}right{\isacharunderscore}{\kern0pt}cart{\isacharunderscore}{\kern0pt}proj\ X\ Y\ {\isasymcirc}\isactrlsub c\ {\isasymlangle}x{\isacharcomma}{\kern0pt}y{\isasymrangle}{\isacharcomma}{\kern0pt}\ left{\isacharunderscore}{\kern0pt}cart{\isacharunderscore}{\kern0pt}proj\ X\ Y\ {\isasymcirc}\isactrlsub c\ {\isasymlangle}x{\isacharcomma}{\kern0pt}y{\isasymrangle}{\isasymrangle}{\isachardoublequoteclose}\isanewline
\ \ \ \ \isacommand{by}\isamarkupfalse%
\ {\isacharparenleft}{\kern0pt}meson\ assms\ cfunc{\isacharunderscore}{\kern0pt}prod{\isacharunderscore}{\kern0pt}comp\ cfunc{\isacharunderscore}{\kern0pt}prod{\isacharunderscore}{\kern0pt}type\ left{\isacharunderscore}{\kern0pt}cart{\isacharunderscore}{\kern0pt}proj{\isacharunderscore}{\kern0pt}type\ right{\isacharunderscore}{\kern0pt}cart{\isacharunderscore}{\kern0pt}proj{\isacharunderscore}{\kern0pt}type{\isacharparenright}{\kern0pt}\isanewline
\ \ \isacommand{also}\isamarkupfalse%
\ \isacommand{have}\isamarkupfalse%
\ {\isachardoublequoteopen}{\isachardot}{\kern0pt}{\isachardot}{\kern0pt}{\isachardot}{\kern0pt}\ {\isacharequal}{\kern0pt}\ {\isasymlangle}y{\isacharcomma}{\kern0pt}\ x{\isasymrangle}{\isachardoublequoteclose}\isanewline
\ \ \ \ \isacommand{using}\isamarkupfalse%
\ assms\ left{\isacharunderscore}{\kern0pt}cart{\isacharunderscore}{\kern0pt}proj{\isacharunderscore}{\kern0pt}cfunc{\isacharunderscore}{\kern0pt}prod\ right{\isacharunderscore}{\kern0pt}cart{\isacharunderscore}{\kern0pt}proj{\isacharunderscore}{\kern0pt}cfunc{\isacharunderscore}{\kern0pt}prod\ \isacommand{by}\isamarkupfalse%
\ auto\isanewline
\ \ \isacommand{then}\isamarkupfalse%
\ \isacommand{show}\isamarkupfalse%
\ {\isacharquery}{\kern0pt}thesis\isanewline
\ \ \ \ \isacommand{using}\isamarkupfalse%
\ calculation\ \isacommand{by}\isamarkupfalse%
\ auto\isanewline
\isacommand{qed}\isamarkupfalse%
%
\endisatagproof
{\isafoldproof}%
%
\isadelimproof
\isanewline
%
\endisadelimproof
\isanewline
\isacommand{lemma}\isamarkupfalse%
\ swap{\isacharunderscore}{\kern0pt}cross{\isacharunderscore}{\kern0pt}prod{\isacharcolon}{\kern0pt}\isanewline
\ \ \isakeyword{assumes}\ {\isachardoublequoteopen}x\ {\isacharcolon}{\kern0pt}\ A\ {\isasymrightarrow}\ X{\isachardoublequoteclose}\ {\isachardoublequoteopen}y\ {\isacharcolon}{\kern0pt}\ B\ {\isasymrightarrow}\ Y{\isachardoublequoteclose}\isanewline
\ \ \isakeyword{shows}\ {\isachardoublequoteopen}swap\ X\ Y\ {\isasymcirc}\isactrlsub c\ {\isacharparenleft}{\kern0pt}x\ {\isasymtimes}\isactrlsub f\ y{\isacharparenright}{\kern0pt}\ {\isacharequal}{\kern0pt}\ {\isacharparenleft}{\kern0pt}y\ {\isasymtimes}\isactrlsub f\ x{\isacharparenright}{\kern0pt}\ {\isasymcirc}\isactrlsub c\ swap\ A\ B{\isachardoublequoteclose}\isanewline
%
\isadelimproof
%
\endisadelimproof
%
\isatagproof
\isacommand{proof}\isamarkupfalse%
\ {\isacharminus}{\kern0pt}\isanewline
\ \ \isacommand{have}\isamarkupfalse%
\ {\isachardoublequoteopen}swap\ X\ Y\ {\isasymcirc}\isactrlsub c\ {\isacharparenleft}{\kern0pt}x\ {\isasymtimes}\isactrlsub f\ y{\isacharparenright}{\kern0pt}\ {\isacharequal}{\kern0pt}\ swap\ X\ Y\ {\isasymcirc}\isactrlsub c\ {\isasymlangle}x\ {\isasymcirc}\isactrlsub c\ left{\isacharunderscore}{\kern0pt}cart{\isacharunderscore}{\kern0pt}proj\ A\ B{\isacharcomma}{\kern0pt}\ y\ {\isasymcirc}\isactrlsub c\ right{\isacharunderscore}{\kern0pt}cart{\isacharunderscore}{\kern0pt}proj\ A\ B{\isasymrangle}{\isachardoublequoteclose}\isanewline
\ \ \ \ \isacommand{using}\isamarkupfalse%
\ assms\ \isacommand{unfolding}\isamarkupfalse%
\ cfunc{\isacharunderscore}{\kern0pt}cross{\isacharunderscore}{\kern0pt}prod{\isacharunderscore}{\kern0pt}def\ cfunc{\isacharunderscore}{\kern0pt}type{\isacharunderscore}{\kern0pt}def\ \isacommand{by}\isamarkupfalse%
\ auto\isanewline
\ \ \isacommand{also}\isamarkupfalse%
\ \isacommand{have}\isamarkupfalse%
\ {\isachardoublequoteopen}{\isachardot}{\kern0pt}{\isachardot}{\kern0pt}{\isachardot}{\kern0pt}\ {\isacharequal}{\kern0pt}\ {\isasymlangle}y\ {\isasymcirc}\isactrlsub c\ right{\isacharunderscore}{\kern0pt}cart{\isacharunderscore}{\kern0pt}proj\ A\ B{\isacharcomma}{\kern0pt}\ x\ {\isasymcirc}\isactrlsub c\ left{\isacharunderscore}{\kern0pt}cart{\isacharunderscore}{\kern0pt}proj\ A\ B{\isasymrangle}{\isachardoublequoteclose}\isanewline
\ \ \ \ \isacommand{by}\isamarkupfalse%
\ {\isacharparenleft}{\kern0pt}meson\ assms\ comp{\isacharunderscore}{\kern0pt}type\ left{\isacharunderscore}{\kern0pt}cart{\isacharunderscore}{\kern0pt}proj{\isacharunderscore}{\kern0pt}type\ right{\isacharunderscore}{\kern0pt}cart{\isacharunderscore}{\kern0pt}proj{\isacharunderscore}{\kern0pt}type\ swap{\isacharunderscore}{\kern0pt}ap{\isacharparenright}{\kern0pt}\isanewline
\ \ \isacommand{also}\isamarkupfalse%
\ \isacommand{have}\isamarkupfalse%
\ {\isachardoublequoteopen}{\isachardot}{\kern0pt}{\isachardot}{\kern0pt}{\isachardot}{\kern0pt}\ {\isacharequal}{\kern0pt}\ {\isacharparenleft}{\kern0pt}y\ {\isasymtimes}\isactrlsub f\ x{\isacharparenright}{\kern0pt}\ {\isasymcirc}\isactrlsub c\ {\isasymlangle}right{\isacharunderscore}{\kern0pt}cart{\isacharunderscore}{\kern0pt}proj\ A\ B{\isacharcomma}{\kern0pt}\ left{\isacharunderscore}{\kern0pt}cart{\isacharunderscore}{\kern0pt}proj\ A\ B{\isasymrangle}{\isachardoublequoteclose}\isanewline
\ \ \ \ \isacommand{using}\isamarkupfalse%
\ assms\ \isacommand{by}\isamarkupfalse%
\ {\isacharparenleft}{\kern0pt}typecheck{\isacharunderscore}{\kern0pt}cfuncs{\isacharcomma}{\kern0pt}\ simp\ add{\isacharcolon}{\kern0pt}\ cfunc{\isacharunderscore}{\kern0pt}cross{\isacharunderscore}{\kern0pt}prod{\isacharunderscore}{\kern0pt}comp{\isacharunderscore}{\kern0pt}cfunc{\isacharunderscore}{\kern0pt}prod{\isacharparenright}{\kern0pt}\isanewline
\ \ \isacommand{also}\isamarkupfalse%
\ \isacommand{have}\isamarkupfalse%
\ {\isachardoublequoteopen}{\isachardot}{\kern0pt}{\isachardot}{\kern0pt}{\isachardot}{\kern0pt}\ {\isacharequal}{\kern0pt}\ {\isacharparenleft}{\kern0pt}y\ {\isasymtimes}\isactrlsub f\ x{\isacharparenright}{\kern0pt}\ {\isasymcirc}\isactrlsub c\ swap\ A\ B{\isachardoublequoteclose}\isanewline
\ \ \ \ \isacommand{unfolding}\isamarkupfalse%
\ swap{\isacharunderscore}{\kern0pt}def\ \isacommand{by}\isamarkupfalse%
\ auto\isanewline
\ \ \isacommand{then}\isamarkupfalse%
\ \isacommand{show}\isamarkupfalse%
\ {\isacharquery}{\kern0pt}thesis\isanewline
\ \ \ \ \isacommand{using}\isamarkupfalse%
\ calculation\ \isacommand{by}\isamarkupfalse%
\ auto\isanewline
\isacommand{qed}\isamarkupfalse%
%
\endisatagproof
{\isafoldproof}%
%
\isadelimproof
\isanewline
%
\endisadelimproof
\isanewline
\isacommand{lemma}\isamarkupfalse%
\ swap{\isacharunderscore}{\kern0pt}idempotent{\isacharcolon}{\kern0pt}\isanewline
\ \ {\isachardoublequoteopen}swap\ Y\ X\ {\isasymcirc}\isactrlsub c\ swap\ X\ Y\ {\isacharequal}{\kern0pt}\ id\ {\isacharparenleft}{\kern0pt}X\ {\isasymtimes}\isactrlsub c\ Y{\isacharparenright}{\kern0pt}{\isachardoublequoteclose}\isanewline
%
\isadelimproof
\ \ %
\endisadelimproof
%
\isatagproof
\isacommand{by}\isamarkupfalse%
\ {\isacharparenleft}{\kern0pt}metis\ swap{\isacharunderscore}{\kern0pt}def\ cfunc{\isacharunderscore}{\kern0pt}prod{\isacharunderscore}{\kern0pt}unique\ id{\isacharunderscore}{\kern0pt}right{\isacharunderscore}{\kern0pt}unit{\isadigit{2}}\ id{\isacharunderscore}{\kern0pt}type\ left{\isacharunderscore}{\kern0pt}cart{\isacharunderscore}{\kern0pt}proj{\isacharunderscore}{\kern0pt}type\isanewline
\ \ \ \ \ \ right{\isacharunderscore}{\kern0pt}cart{\isacharunderscore}{\kern0pt}proj{\isacharunderscore}{\kern0pt}type\ swap{\isacharunderscore}{\kern0pt}ap{\isacharparenright}{\kern0pt}%
\endisatagproof
{\isafoldproof}%
%
\isadelimproof
\isanewline
%
\endisadelimproof
\isanewline
\isacommand{lemma}\isamarkupfalse%
\ swap{\isacharunderscore}{\kern0pt}mono{\isacharcolon}{\kern0pt}\isanewline
\ \ {\isachardoublequoteopen}monomorphism{\isacharparenleft}{\kern0pt}swap\ X\ Y{\isacharparenright}{\kern0pt}{\isachardoublequoteclose}\isanewline
%
\isadelimproof
\ \ %
\endisadelimproof
%
\isatagproof
\isacommand{by}\isamarkupfalse%
\ {\isacharparenleft}{\kern0pt}metis\ cfunc{\isacharunderscore}{\kern0pt}type{\isacharunderscore}{\kern0pt}def\ iso{\isacharunderscore}{\kern0pt}imp{\isacharunderscore}{\kern0pt}epi{\isacharunderscore}{\kern0pt}and{\isacharunderscore}{\kern0pt}monic\ isomorphism{\isacharunderscore}{\kern0pt}def\ swap{\isacharunderscore}{\kern0pt}idempotent\ swap{\isacharunderscore}{\kern0pt}type{\isacharparenright}{\kern0pt}%
\endisatagproof
{\isafoldproof}%
%
\isadelimproof
%
\endisadelimproof
%
\isadelimdocument
%
\endisadelimdocument
%
\isatagdocument
%
\isamarkupsubsubsection{Permuting a Cartesian Product to Associate to the Right%
}
\isamarkuptrue%
%
\endisatagdocument
{\isafolddocument}%
%
\isadelimdocument
%
\endisadelimdocument
\isacommand{definition}\isamarkupfalse%
\ associate{\isacharunderscore}{\kern0pt}right\ {\isacharcolon}{\kern0pt}{\isacharcolon}{\kern0pt}\ {\isachardoublequoteopen}cset\ {\isasymRightarrow}\ cset\ {\isasymRightarrow}\ cset\ {\isasymRightarrow}\ cfunc{\isachardoublequoteclose}\ \isakeyword{where}\isanewline
\ \ {\isachardoublequoteopen}associate{\isacharunderscore}{\kern0pt}right\ X\ Y\ Z\ {\isacharequal}{\kern0pt}\isanewline
\ \ \ \ {\isasymlangle}\isanewline
\ \ \ \ \ \ left{\isacharunderscore}{\kern0pt}cart{\isacharunderscore}{\kern0pt}proj\ X\ Y\ {\isasymcirc}\isactrlsub c\ left{\isacharunderscore}{\kern0pt}cart{\isacharunderscore}{\kern0pt}proj\ {\isacharparenleft}{\kern0pt}X\ {\isasymtimes}\isactrlsub c\ Y{\isacharparenright}{\kern0pt}\ Z{\isacharcomma}{\kern0pt}\ \isanewline
\ \ \ \ \ \ {\isasymlangle}\isanewline
\ \ \ \ \ \ \ \ right{\isacharunderscore}{\kern0pt}cart{\isacharunderscore}{\kern0pt}proj\ X\ Y\ {\isasymcirc}\isactrlsub c\ left{\isacharunderscore}{\kern0pt}cart{\isacharunderscore}{\kern0pt}proj\ {\isacharparenleft}{\kern0pt}X\ {\isasymtimes}\isactrlsub c\ Y{\isacharparenright}{\kern0pt}\ Z{\isacharcomma}{\kern0pt}\isanewline
\ \ \ \ \ \ \ \ right{\isacharunderscore}{\kern0pt}cart{\isacharunderscore}{\kern0pt}proj\ {\isacharparenleft}{\kern0pt}X\ {\isasymtimes}\isactrlsub c\ Y{\isacharparenright}{\kern0pt}\ Z\isanewline
\ \ \ \ \ \ {\isasymrangle}\isanewline
\ \ \ \ {\isasymrangle}{\isachardoublequoteclose}\isanewline
\isanewline
\isacommand{lemma}\isamarkupfalse%
\ associate{\isacharunderscore}{\kern0pt}right{\isacharunderscore}{\kern0pt}type{\isacharbrackleft}{\kern0pt}type{\isacharunderscore}{\kern0pt}rule{\isacharbrackright}{\kern0pt}{\isacharcolon}{\kern0pt}\ {\isachardoublequoteopen}associate{\isacharunderscore}{\kern0pt}right\ X\ Y\ Z\ {\isacharcolon}{\kern0pt}\ {\isacharparenleft}{\kern0pt}X\ {\isasymtimes}\isactrlsub c\ Y{\isacharparenright}{\kern0pt}\ {\isasymtimes}\isactrlsub c\ Z\ {\isasymrightarrow}\ X\ {\isasymtimes}\isactrlsub c\ {\isacharparenleft}{\kern0pt}Y\ {\isasymtimes}\isactrlsub c\ Z{\isacharparenright}{\kern0pt}{\isachardoublequoteclose}\isanewline
%
\isadelimproof
\ \ %
\endisadelimproof
%
\isatagproof
\isacommand{unfolding}\isamarkupfalse%
\ associate{\isacharunderscore}{\kern0pt}right{\isacharunderscore}{\kern0pt}def\ \isacommand{by}\isamarkupfalse%
\ {\isacharparenleft}{\kern0pt}meson\ cfunc{\isacharunderscore}{\kern0pt}prod{\isacharunderscore}{\kern0pt}type\ comp{\isacharunderscore}{\kern0pt}type\ left{\isacharunderscore}{\kern0pt}cart{\isacharunderscore}{\kern0pt}proj{\isacharunderscore}{\kern0pt}type\ right{\isacharunderscore}{\kern0pt}cart{\isacharunderscore}{\kern0pt}proj{\isacharunderscore}{\kern0pt}type{\isacharparenright}{\kern0pt}%
\endisatagproof
{\isafoldproof}%
%
\isadelimproof
\isanewline
%
\endisadelimproof
\isanewline
\isacommand{lemma}\isamarkupfalse%
\ associate{\isacharunderscore}{\kern0pt}right{\isacharunderscore}{\kern0pt}ap{\isacharcolon}{\kern0pt}\isanewline
\ \ \isakeyword{assumes}\ {\isachardoublequoteopen}x\ {\isacharcolon}{\kern0pt}\ A\ {\isasymrightarrow}\ X{\isachardoublequoteclose}\ {\isachardoublequoteopen}y\ {\isacharcolon}{\kern0pt}\ A\ {\isasymrightarrow}\ Y{\isachardoublequoteclose}\ {\isachardoublequoteopen}z\ {\isacharcolon}{\kern0pt}\ A\ {\isasymrightarrow}\ Z{\isachardoublequoteclose}\isanewline
\ \ \isakeyword{shows}\ {\isachardoublequoteopen}associate{\isacharunderscore}{\kern0pt}right\ X\ Y\ Z\ {\isasymcirc}\isactrlsub c\ {\isasymlangle}{\isasymlangle}x{\isacharcomma}{\kern0pt}\ y{\isasymrangle}{\isacharcomma}{\kern0pt}\ z{\isasymrangle}\ {\isacharequal}{\kern0pt}\ {\isasymlangle}x{\isacharcomma}{\kern0pt}\ {\isasymlangle}y{\isacharcomma}{\kern0pt}\ z{\isasymrangle}{\isasymrangle}{\isachardoublequoteclose}\isanewline
%
\isadelimproof
%
\endisadelimproof
%
\isatagproof
\isacommand{proof}\isamarkupfalse%
\ {\isacharminus}{\kern0pt}\isanewline
\ \ \isacommand{have}\isamarkupfalse%
\ {\isachardoublequoteopen}associate{\isacharunderscore}{\kern0pt}right\ X\ Y\ Z\ {\isasymcirc}\isactrlsub c\ {\isasymlangle}{\isasymlangle}x{\isacharcomma}{\kern0pt}\ y{\isasymrangle}{\isacharcomma}{\kern0pt}\ z{\isasymrangle}\ {\isacharequal}{\kern0pt}\ {\isasymlangle}{\isacharparenleft}{\kern0pt}left{\isacharunderscore}{\kern0pt}cart{\isacharunderscore}{\kern0pt}proj\ X\ Y\ {\isasymcirc}\isactrlsub c\ \ left{\isacharunderscore}{\kern0pt}cart{\isacharunderscore}{\kern0pt}proj\ {\isacharparenleft}{\kern0pt}X\ {\isasymtimes}\isactrlsub c\ Y{\isacharparenright}{\kern0pt}\ Z{\isacharparenright}{\kern0pt}\ {\isasymcirc}\isactrlsub c\ {\isasymlangle}{\isasymlangle}x{\isacharcomma}{\kern0pt}y{\isasymrangle}{\isacharcomma}{\kern0pt}z{\isasymrangle}{\isacharcomma}{\kern0pt}\ {\isasymlangle}right{\isacharunderscore}{\kern0pt}cart{\isacharunderscore}{\kern0pt}proj\ X\ Y\ {\isasymcirc}\isactrlsub c\ left{\isacharunderscore}{\kern0pt}cart{\isacharunderscore}{\kern0pt}proj\ {\isacharparenleft}{\kern0pt}X\ {\isasymtimes}\isactrlsub c\ Y{\isacharparenright}{\kern0pt}\ Z{\isacharcomma}{\kern0pt}right{\isacharunderscore}{\kern0pt}cart{\isacharunderscore}{\kern0pt}proj\ {\isacharparenleft}{\kern0pt}X\ {\isasymtimes}\isactrlsub c\ Y{\isacharparenright}{\kern0pt}\ Z{\isasymrangle}\ {\isasymcirc}\isactrlsub c\ {\isasymlangle}{\isasymlangle}x{\isacharcomma}{\kern0pt}y{\isasymrangle}{\isacharcomma}{\kern0pt}z{\isasymrangle}{\isasymrangle}{\isachardoublequoteclose}\isanewline
\ \ \ \ \isacommand{by}\isamarkupfalse%
\ {\isacharparenleft}{\kern0pt}typecheck{\isacharunderscore}{\kern0pt}cfuncs{\isacharcomma}{\kern0pt}\ metis\ assms\ associate{\isacharunderscore}{\kern0pt}right{\isacharunderscore}{\kern0pt}def\ cfunc{\isacharunderscore}{\kern0pt}prod{\isacharunderscore}{\kern0pt}comp{\isacharparenright}{\kern0pt}\isanewline
\ \ \isacommand{also}\isamarkupfalse%
\ \isacommand{have}\isamarkupfalse%
\ {\isachardoublequoteopen}{\isachardot}{\kern0pt}{\isachardot}{\kern0pt}{\isachardot}{\kern0pt}\ {\isacharequal}{\kern0pt}\ {\isasymlangle}{\isacharparenleft}{\kern0pt}left{\isacharunderscore}{\kern0pt}cart{\isacharunderscore}{\kern0pt}proj\ X\ Y\ {\isasymcirc}\isactrlsub c\ left{\isacharunderscore}{\kern0pt}cart{\isacharunderscore}{\kern0pt}proj\ {\isacharparenleft}{\kern0pt}X\ {\isasymtimes}\isactrlsub c\ Y{\isacharparenright}{\kern0pt}\ Z{\isacharparenright}{\kern0pt}\ {\isasymcirc}\isactrlsub c\ {\isasymlangle}{\isasymlangle}x{\isacharcomma}{\kern0pt}y{\isasymrangle}{\isacharcomma}{\kern0pt}z{\isasymrangle}{\isacharcomma}{\kern0pt}\ {\isasymlangle}{\isacharparenleft}{\kern0pt}right{\isacharunderscore}{\kern0pt}cart{\isacharunderscore}{\kern0pt}proj\ X\ Y\ {\isasymcirc}\isactrlsub c\ left{\isacharunderscore}{\kern0pt}cart{\isacharunderscore}{\kern0pt}proj\ {\isacharparenleft}{\kern0pt}X\ {\isasymtimes}\isactrlsub c\ Y{\isacharparenright}{\kern0pt}\ Z{\isacharparenright}{\kern0pt}\ {\isasymcirc}\isactrlsub c\ {\isasymlangle}{\isasymlangle}x{\isacharcomma}{\kern0pt}y{\isasymrangle}{\isacharcomma}{\kern0pt}z{\isasymrangle}{\isacharcomma}{\kern0pt}\ right{\isacharunderscore}{\kern0pt}cart{\isacharunderscore}{\kern0pt}proj\ {\isacharparenleft}{\kern0pt}X\ {\isasymtimes}\isactrlsub c\ Y{\isacharparenright}{\kern0pt}\ Z\ {\isasymcirc}\isactrlsub c\ {\isasymlangle}{\isasymlangle}x{\isacharcomma}{\kern0pt}y{\isasymrangle}{\isacharcomma}{\kern0pt}z{\isasymrangle}{\isasymrangle}{\isasymrangle}{\isachardoublequoteclose}\isanewline
\ \ \ \ \isacommand{by}\isamarkupfalse%
\ {\isacharparenleft}{\kern0pt}typecheck{\isacharunderscore}{\kern0pt}cfuncs{\isacharcomma}{\kern0pt}\ metis\ assms\ calculation\ cfunc{\isacharunderscore}{\kern0pt}prod{\isacharunderscore}{\kern0pt}comp\ cfunc{\isacharunderscore}{\kern0pt}prod{\isacharunderscore}{\kern0pt}type\ right{\isacharunderscore}{\kern0pt}cart{\isacharunderscore}{\kern0pt}proj{\isacharunderscore}{\kern0pt}type{\isacharparenright}{\kern0pt}\isanewline
\ \ \isacommand{also}\isamarkupfalse%
\ \isacommand{have}\isamarkupfalse%
\ {\isachardoublequoteopen}{\isachardot}{\kern0pt}{\isachardot}{\kern0pt}{\isachardot}{\kern0pt}\ {\isacharequal}{\kern0pt}\ {\isasymlangle}left{\isacharunderscore}{\kern0pt}cart{\isacharunderscore}{\kern0pt}proj\ X\ Y\ {\isasymcirc}\isactrlsub c\ {\isasymlangle}x{\isacharcomma}{\kern0pt}y{\isasymrangle}{\isacharcomma}{\kern0pt}\ {\isasymlangle}right{\isacharunderscore}{\kern0pt}cart{\isacharunderscore}{\kern0pt}proj\ X\ Y\ {\isasymcirc}\isactrlsub c\ {\isasymlangle}x{\isacharcomma}{\kern0pt}y{\isasymrangle}{\isacharcomma}{\kern0pt}\ z{\isasymrangle}{\isasymrangle}{\isachardoublequoteclose}\isanewline
\ \ \ \ \isacommand{using}\isamarkupfalse%
\ assms\ \isacommand{by}\isamarkupfalse%
\ {\isacharparenleft}{\kern0pt}typecheck{\isacharunderscore}{\kern0pt}cfuncs{\isacharcomma}{\kern0pt}\ smt\ comp{\isacharunderscore}{\kern0pt}associative{\isadigit{2}}\ left{\isacharunderscore}{\kern0pt}cart{\isacharunderscore}{\kern0pt}proj{\isacharunderscore}{\kern0pt}cfunc{\isacharunderscore}{\kern0pt}prod\ right{\isacharunderscore}{\kern0pt}cart{\isacharunderscore}{\kern0pt}proj{\isacharunderscore}{\kern0pt}cfunc{\isacharunderscore}{\kern0pt}prod{\isacharparenright}{\kern0pt}\isanewline
\ \ \isacommand{also}\isamarkupfalse%
\ \isacommand{have}\isamarkupfalse%
\ {\isachardoublequoteopen}{\isachardot}{\kern0pt}{\isachardot}{\kern0pt}{\isachardot}{\kern0pt}\ {\isacharequal}{\kern0pt}{\isasymlangle}x{\isacharcomma}{\kern0pt}\ {\isasymlangle}y{\isacharcomma}{\kern0pt}\ z{\isasymrangle}{\isasymrangle}{\isachardoublequoteclose}\isanewline
\ \ \ \ \isacommand{using}\isamarkupfalse%
\ assms\ left{\isacharunderscore}{\kern0pt}cart{\isacharunderscore}{\kern0pt}proj{\isacharunderscore}{\kern0pt}cfunc{\isacharunderscore}{\kern0pt}prod\ right{\isacharunderscore}{\kern0pt}cart{\isacharunderscore}{\kern0pt}proj{\isacharunderscore}{\kern0pt}cfunc{\isacharunderscore}{\kern0pt}prod\ \isacommand{by}\isamarkupfalse%
\ auto\isanewline
\ \ \isacommand{then}\isamarkupfalse%
\ \isacommand{show}\isamarkupfalse%
\ {\isacharquery}{\kern0pt}thesis\isanewline
\ \ \ \ \isacommand{using}\isamarkupfalse%
\ calculation\ \isacommand{by}\isamarkupfalse%
\ auto\isanewline
\isacommand{qed}\isamarkupfalse%
%
\endisatagproof
{\isafoldproof}%
%
\isadelimproof
\isanewline
%
\endisadelimproof
\isanewline
\isacommand{lemma}\isamarkupfalse%
\ associate{\isacharunderscore}{\kern0pt}right{\isacharunderscore}{\kern0pt}crossprod{\isacharunderscore}{\kern0pt}ap{\isacharcolon}{\kern0pt}\isanewline
\ \ \isakeyword{assumes}\ {\isachardoublequoteopen}x\ {\isacharcolon}{\kern0pt}\ A\ {\isasymrightarrow}\ X{\isachardoublequoteclose}\ {\isachardoublequoteopen}y\ {\isacharcolon}{\kern0pt}\ B\ {\isasymrightarrow}\ Y{\isachardoublequoteclose}\ {\isachardoublequoteopen}z\ {\isacharcolon}{\kern0pt}\ C\ {\isasymrightarrow}\ Z{\isachardoublequoteclose}\isanewline
\ \ \isakeyword{shows}\ {\isachardoublequoteopen}associate{\isacharunderscore}{\kern0pt}right\ X\ Y\ Z\ {\isasymcirc}\isactrlsub c\ {\isacharparenleft}{\kern0pt}{\isacharparenleft}{\kern0pt}x\ {\isasymtimes}\isactrlsub f\ y{\isacharparenright}{\kern0pt}\ {\isasymtimes}\isactrlsub f\ z{\isacharparenright}{\kern0pt}\ {\isacharequal}{\kern0pt}\ {\isacharparenleft}{\kern0pt}x\ {\isasymtimes}\isactrlsub f\ {\isacharparenleft}{\kern0pt}y{\isasymtimes}\isactrlsub f\ z{\isacharparenright}{\kern0pt}{\isacharparenright}{\kern0pt}\ {\isasymcirc}\isactrlsub c\ \ associate{\isacharunderscore}{\kern0pt}right\ A\ B\ C{\isachardoublequoteclose}\isanewline
%
\isadelimproof
%
\endisadelimproof
%
\isatagproof
\isacommand{proof}\isamarkupfalse%
{\isacharminus}{\kern0pt}\isanewline
\ \ \isacommand{have}\isamarkupfalse%
\ {\isachardoublequoteopen}associate{\isacharunderscore}{\kern0pt}right\ X\ Y\ Z\ {\isasymcirc}\isactrlsub c\ {\isacharparenleft}{\kern0pt}{\isacharparenleft}{\kern0pt}x\ {\isasymtimes}\isactrlsub f\ y{\isacharparenright}{\kern0pt}\ {\isasymtimes}\isactrlsub f\ z{\isacharparenright}{\kern0pt}\ {\isacharequal}{\kern0pt}\isanewline
\ \ \ \ \ \ \ \ associate{\isacharunderscore}{\kern0pt}right\ X\ Y\ Z\ {\isasymcirc}\isactrlsub c\ {\isasymlangle}{\isasymlangle}x\ \ {\isasymcirc}\isactrlsub c\ left{\isacharunderscore}{\kern0pt}cart{\isacharunderscore}{\kern0pt}proj\ A\ B{\isacharcomma}{\kern0pt}\ y\ {\isasymcirc}\isactrlsub c\ right{\isacharunderscore}{\kern0pt}cart{\isacharunderscore}{\kern0pt}proj\ A\ B{\isasymrangle}\ {\isasymcirc}\isactrlsub c\ left{\isacharunderscore}{\kern0pt}cart{\isacharunderscore}{\kern0pt}proj\ {\isacharparenleft}{\kern0pt}A{\isasymtimes}\isactrlsub cB{\isacharparenright}{\kern0pt}\ C{\isacharcomma}{\kern0pt}\ z\ {\isasymcirc}\isactrlsub c\ right{\isacharunderscore}{\kern0pt}cart{\isacharunderscore}{\kern0pt}proj\ {\isacharparenleft}{\kern0pt}A\ {\isasymtimes}\isactrlsub c\ B{\isacharparenright}{\kern0pt}\ C{\isasymrangle}{\isachardoublequoteclose}\isanewline
\ \ \ \ \isacommand{using}\isamarkupfalse%
\ assms\ \isacommand{by}\isamarkupfalse%
{\isacharparenleft}{\kern0pt}unfold\ cfunc{\isacharunderscore}{\kern0pt}cross{\isacharunderscore}{\kern0pt}prod{\isacharunderscore}{\kern0pt}def{\isadigit{2}}{\isacharcomma}{\kern0pt}\ typecheck{\isacharunderscore}{\kern0pt}cfuncs{\isacharcomma}{\kern0pt}\ unfold\ cfunc{\isacharunderscore}{\kern0pt}cross{\isacharunderscore}{\kern0pt}prod{\isacharunderscore}{\kern0pt}def{\isadigit{2}}{\isacharcomma}{\kern0pt}\ auto{\isacharparenright}{\kern0pt}\ \isanewline
\ \ \isacommand{also}\isamarkupfalse%
\ \isacommand{have}\isamarkupfalse%
\ {\isachardoublequoteopen}{\isachardot}{\kern0pt}{\isachardot}{\kern0pt}{\isachardot}{\kern0pt}\ {\isacharequal}{\kern0pt}\ associate{\isacharunderscore}{\kern0pt}right\ X\ Y\ Z\ {\isasymcirc}\isactrlsub c\ {\isasymlangle}{\isasymlangle}x\ \ {\isasymcirc}\isactrlsub c\ left{\isacharunderscore}{\kern0pt}cart{\isacharunderscore}{\kern0pt}proj\ A\ B\ {\isasymcirc}\isactrlsub c\ left{\isacharunderscore}{\kern0pt}cart{\isacharunderscore}{\kern0pt}proj\ {\isacharparenleft}{\kern0pt}A{\isasymtimes}\isactrlsub cB{\isacharparenright}{\kern0pt}\ C{\isacharcomma}{\kern0pt}\ y\ \ {\isasymcirc}\isactrlsub c\ right{\isacharunderscore}{\kern0pt}cart{\isacharunderscore}{\kern0pt}proj\ A\ B\ {\isasymcirc}\isactrlsub c\ left{\isacharunderscore}{\kern0pt}cart{\isacharunderscore}{\kern0pt}proj\ {\isacharparenleft}{\kern0pt}A{\isasymtimes}\isactrlsub cB{\isacharparenright}{\kern0pt}\ C{\isasymrangle}{\isacharcomma}{\kern0pt}\ z\ {\isasymcirc}\isactrlsub c\ right{\isacharunderscore}{\kern0pt}cart{\isacharunderscore}{\kern0pt}proj\ {\isacharparenleft}{\kern0pt}A\ {\isasymtimes}\isactrlsub c\ B{\isacharparenright}{\kern0pt}\ C{\isasymrangle}{\isachardoublequoteclose}\isanewline
\ \ \ \ \isacommand{using}\isamarkupfalse%
\ assms\ \ cfunc{\isacharunderscore}{\kern0pt}prod{\isacharunderscore}{\kern0pt}comp\ comp{\isacharunderscore}{\kern0pt}associative{\isadigit{2}}\ \isacommand{by}\isamarkupfalse%
\ {\isacharparenleft}{\kern0pt}typecheck{\isacharunderscore}{\kern0pt}cfuncs{\isacharcomma}{\kern0pt}\ auto{\isacharparenright}{\kern0pt}\isanewline
\ \ \isacommand{also}\isamarkupfalse%
\ \isacommand{have}\isamarkupfalse%
\ {\isachardoublequoteopen}{\isachardot}{\kern0pt}{\isachardot}{\kern0pt}{\isachardot}{\kern0pt}\ {\isacharequal}{\kern0pt}\ {\isasymlangle}x\ \ {\isasymcirc}\isactrlsub c\ left{\isacharunderscore}{\kern0pt}cart{\isacharunderscore}{\kern0pt}proj\ A\ B\ {\isasymcirc}\isactrlsub c\ left{\isacharunderscore}{\kern0pt}cart{\isacharunderscore}{\kern0pt}proj\ {\isacharparenleft}{\kern0pt}A{\isasymtimes}\isactrlsub cB{\isacharparenright}{\kern0pt}\ C{\isacharcomma}{\kern0pt}\ {\isasymlangle}y\ {\isasymcirc}\isactrlsub c\ right{\isacharunderscore}{\kern0pt}cart{\isacharunderscore}{\kern0pt}proj\ A\ B\ {\isasymcirc}\isactrlsub c\ left{\isacharunderscore}{\kern0pt}cart{\isacharunderscore}{\kern0pt}proj\ {\isacharparenleft}{\kern0pt}A{\isasymtimes}\isactrlsub cB{\isacharparenright}{\kern0pt}\ C{\isacharcomma}{\kern0pt}\ z\ {\isasymcirc}\isactrlsub c\ right{\isacharunderscore}{\kern0pt}cart{\isacharunderscore}{\kern0pt}proj\ {\isacharparenleft}{\kern0pt}A\ {\isasymtimes}\isactrlsub c\ B{\isacharparenright}{\kern0pt}\ C{\isasymrangle}{\isasymrangle}{\isachardoublequoteclose}\isanewline
\ \ \ \ \isacommand{using}\isamarkupfalse%
\ assms\ \isacommand{by}\isamarkupfalse%
\ {\isacharparenleft}{\kern0pt}typecheck{\isacharunderscore}{\kern0pt}cfuncs{\isacharcomma}{\kern0pt}\ simp\ add{\isacharcolon}{\kern0pt}\ associate{\isacharunderscore}{\kern0pt}right{\isacharunderscore}{\kern0pt}ap{\isacharparenright}{\kern0pt}\isanewline
\ \ \isacommand{also}\isamarkupfalse%
\ \isacommand{have}\isamarkupfalse%
\ {\isachardoublequoteopen}{\isachardot}{\kern0pt}{\isachardot}{\kern0pt}{\isachardot}{\kern0pt}\ {\isacharequal}{\kern0pt}\ {\isasymlangle}x\ \ {\isasymcirc}\isactrlsub c\ left{\isacharunderscore}{\kern0pt}cart{\isacharunderscore}{\kern0pt}proj\ A\ B\ {\isasymcirc}\isactrlsub c\ left{\isacharunderscore}{\kern0pt}cart{\isacharunderscore}{\kern0pt}proj\ {\isacharparenleft}{\kern0pt}A{\isasymtimes}\isactrlsub cB{\isacharparenright}{\kern0pt}\ C{\isacharcomma}{\kern0pt}\ {\isacharparenleft}{\kern0pt}y\ {\isasymtimes}\isactrlsub f\ z{\isacharparenright}{\kern0pt}{\isasymcirc}\isactrlsub c\ {\isasymlangle}right{\isacharunderscore}{\kern0pt}cart{\isacharunderscore}{\kern0pt}proj\ A\ B\ {\isasymcirc}\isactrlsub c\ left{\isacharunderscore}{\kern0pt}cart{\isacharunderscore}{\kern0pt}proj\ {\isacharparenleft}{\kern0pt}A{\isasymtimes}\isactrlsub cB{\isacharparenright}{\kern0pt}\ C{\isacharcomma}{\kern0pt}right{\isacharunderscore}{\kern0pt}cart{\isacharunderscore}{\kern0pt}proj\ {\isacharparenleft}{\kern0pt}A\ {\isasymtimes}\isactrlsub c\ B{\isacharparenright}{\kern0pt}\ C{\isasymrangle}{\isasymrangle}{\isachardoublequoteclose}\isanewline
\ \ \ \ \isacommand{using}\isamarkupfalse%
\ assms\ \isacommand{by}\isamarkupfalse%
\ {\isacharparenleft}{\kern0pt}typecheck{\isacharunderscore}{\kern0pt}cfuncs{\isacharcomma}{\kern0pt}\ simp\ add{\isacharcolon}{\kern0pt}\ cfunc{\isacharunderscore}{\kern0pt}cross{\isacharunderscore}{\kern0pt}prod{\isacharunderscore}{\kern0pt}comp{\isacharunderscore}{\kern0pt}cfunc{\isacharunderscore}{\kern0pt}prod{\isacharparenright}{\kern0pt}\isanewline
\ \ \isacommand{also}\isamarkupfalse%
\ \isacommand{have}\isamarkupfalse%
\ {\isachardoublequoteopen}{\isachardot}{\kern0pt}{\isachardot}{\kern0pt}{\isachardot}{\kern0pt}\ {\isacharequal}{\kern0pt}\ {\isacharparenleft}{\kern0pt}x\ {\isasymtimes}\isactrlsub f\ {\isacharparenleft}{\kern0pt}y{\isasymtimes}\isactrlsub f\ z{\isacharparenright}{\kern0pt}{\isacharparenright}{\kern0pt}\ {\isasymcirc}\isactrlsub c\ {\isasymlangle}left{\isacharunderscore}{\kern0pt}cart{\isacharunderscore}{\kern0pt}proj\ A\ B\ {\isasymcirc}\isactrlsub c\ left{\isacharunderscore}{\kern0pt}cart{\isacharunderscore}{\kern0pt}proj\ {\isacharparenleft}{\kern0pt}A{\isasymtimes}\isactrlsub cB{\isacharparenright}{\kern0pt}\ C{\isacharcomma}{\kern0pt}{\isasymlangle}right{\isacharunderscore}{\kern0pt}cart{\isacharunderscore}{\kern0pt}proj\ A\ B\ {\isasymcirc}\isactrlsub c\ left{\isacharunderscore}{\kern0pt}cart{\isacharunderscore}{\kern0pt}proj\ {\isacharparenleft}{\kern0pt}A{\isasymtimes}\isactrlsub cB{\isacharparenright}{\kern0pt}\ C{\isacharcomma}{\kern0pt}right{\isacharunderscore}{\kern0pt}cart{\isacharunderscore}{\kern0pt}proj\ {\isacharparenleft}{\kern0pt}A\ {\isasymtimes}\isactrlsub c\ B{\isacharparenright}{\kern0pt}\ C{\isasymrangle}{\isasymrangle}{\isachardoublequoteclose}\isanewline
\ \ \ \ \isacommand{using}\isamarkupfalse%
\ assms\ \isacommand{by}\isamarkupfalse%
\ {\isacharparenleft}{\kern0pt}typecheck{\isacharunderscore}{\kern0pt}cfuncs{\isacharcomma}{\kern0pt}\ simp\ add{\isacharcolon}{\kern0pt}\ cfunc{\isacharunderscore}{\kern0pt}cross{\isacharunderscore}{\kern0pt}prod{\isacharunderscore}{\kern0pt}comp{\isacharunderscore}{\kern0pt}cfunc{\isacharunderscore}{\kern0pt}prod{\isacharparenright}{\kern0pt}\isanewline
\ \ \isacommand{also}\isamarkupfalse%
\ \isacommand{have}\isamarkupfalse%
\ {\isachardoublequoteopen}{\isachardot}{\kern0pt}{\isachardot}{\kern0pt}{\isachardot}{\kern0pt}\ {\isacharequal}{\kern0pt}\ {\isacharparenleft}{\kern0pt}x\ {\isasymtimes}\isactrlsub f\ {\isacharparenleft}{\kern0pt}y{\isasymtimes}\isactrlsub f\ z{\isacharparenright}{\kern0pt}{\isacharparenright}{\kern0pt}\ {\isasymcirc}\isactrlsub c\ \ associate{\isacharunderscore}{\kern0pt}right\ A\ B\ C{\isachardoublequoteclose}\ \ \ \isanewline
\ \ \ \ \isacommand{unfolding}\isamarkupfalse%
\ associate{\isacharunderscore}{\kern0pt}right{\isacharunderscore}{\kern0pt}def\ \isacommand{by}\isamarkupfalse%
\ auto\isanewline
\ \ \isacommand{then}\isamarkupfalse%
\ \isacommand{show}\isamarkupfalse%
\ {\isacharquery}{\kern0pt}thesis\ \isacommand{using}\isamarkupfalse%
\ calculation\ \isacommand{by}\isamarkupfalse%
\ auto\isanewline
\isacommand{qed}\isamarkupfalse%
%
\endisatagproof
{\isafoldproof}%
%
\isadelimproof
%
\endisadelimproof
%
\isadelimdocument
%
\endisadelimdocument
%
\isatagdocument
%
\isamarkupsubsubsection{Permuting a Cartesian Product to Associate to the Left%
}
\isamarkuptrue%
%
\endisatagdocument
{\isafolddocument}%
%
\isadelimdocument
%
\endisadelimdocument
\isacommand{definition}\isamarkupfalse%
\ associate{\isacharunderscore}{\kern0pt}left\ {\isacharcolon}{\kern0pt}{\isacharcolon}{\kern0pt}\ {\isachardoublequoteopen}cset\ {\isasymRightarrow}\ cset\ {\isasymRightarrow}\ cset\ {\isasymRightarrow}\ cfunc{\isachardoublequoteclose}\ \isakeyword{where}\isanewline
\ \ {\isachardoublequoteopen}associate{\isacharunderscore}{\kern0pt}left\ X\ Y\ Z\ {\isacharequal}{\kern0pt}\isanewline
\ \ \ \ {\isasymlangle}\isanewline
\ \ \ \ \ \ {\isasymlangle}\isanewline
\ \ \ \ \ \ \ \ left{\isacharunderscore}{\kern0pt}cart{\isacharunderscore}{\kern0pt}proj\ X\ {\isacharparenleft}{\kern0pt}Y\ {\isasymtimes}\isactrlsub c\ Z{\isacharparenright}{\kern0pt}{\isacharcomma}{\kern0pt}\isanewline
\ \ \ \ \ \ \ \ left{\isacharunderscore}{\kern0pt}cart{\isacharunderscore}{\kern0pt}proj\ Y\ Z\ {\isasymcirc}\isactrlsub c\ right{\isacharunderscore}{\kern0pt}cart{\isacharunderscore}{\kern0pt}proj\ X\ {\isacharparenleft}{\kern0pt}Y\ {\isasymtimes}\isactrlsub c\ Z{\isacharparenright}{\kern0pt}\isanewline
\ \ \ \ \ \ {\isasymrangle}{\isacharcomma}{\kern0pt}\isanewline
\ \ \ \ \ \ right{\isacharunderscore}{\kern0pt}cart{\isacharunderscore}{\kern0pt}proj\ Y\ Z\ {\isasymcirc}\isactrlsub c\ right{\isacharunderscore}{\kern0pt}cart{\isacharunderscore}{\kern0pt}proj\ X\ {\isacharparenleft}{\kern0pt}Y\ {\isasymtimes}\isactrlsub c\ Z{\isacharparenright}{\kern0pt}\isanewline
\ \ \ \ {\isasymrangle}{\isachardoublequoteclose}\isanewline
\isanewline
\isacommand{lemma}\isamarkupfalse%
\ associate{\isacharunderscore}{\kern0pt}left{\isacharunderscore}{\kern0pt}type{\isacharbrackleft}{\kern0pt}type{\isacharunderscore}{\kern0pt}rule{\isacharbrackright}{\kern0pt}{\isacharcolon}{\kern0pt}\ {\isachardoublequoteopen}associate{\isacharunderscore}{\kern0pt}left\ X\ Y\ Z\ {\isacharcolon}{\kern0pt}\ X\ {\isasymtimes}\isactrlsub c\ {\isacharparenleft}{\kern0pt}Y\ {\isasymtimes}\isactrlsub c\ Z{\isacharparenright}{\kern0pt}\ {\isasymrightarrow}\ {\isacharparenleft}{\kern0pt}X\ {\isasymtimes}\isactrlsub c\ Y{\isacharparenright}{\kern0pt}\ {\isasymtimes}\isactrlsub c\ Z{\isachardoublequoteclose}\isanewline
%
\isadelimproof
\ \ %
\endisadelimproof
%
\isatagproof
\isacommand{unfolding}\isamarkupfalse%
\ associate{\isacharunderscore}{\kern0pt}left{\isacharunderscore}{\kern0pt}def\isanewline
\ \ \isacommand{by}\isamarkupfalse%
\ {\isacharparenleft}{\kern0pt}meson\ cfunc{\isacharunderscore}{\kern0pt}prod{\isacharunderscore}{\kern0pt}type\ comp{\isacharunderscore}{\kern0pt}type\ left{\isacharunderscore}{\kern0pt}cart{\isacharunderscore}{\kern0pt}proj{\isacharunderscore}{\kern0pt}type\ right{\isacharunderscore}{\kern0pt}cart{\isacharunderscore}{\kern0pt}proj{\isacharunderscore}{\kern0pt}type{\isacharparenright}{\kern0pt}%
\endisatagproof
{\isafoldproof}%
%
\isadelimproof
\isanewline
%
\endisadelimproof
\isanewline
\isacommand{lemma}\isamarkupfalse%
\ associate{\isacharunderscore}{\kern0pt}left{\isacharunderscore}{\kern0pt}ap{\isacharcolon}{\kern0pt}\isanewline
\ \ \isakeyword{assumes}\ {\isachardoublequoteopen}x\ {\isacharcolon}{\kern0pt}\ A\ {\isasymrightarrow}\ X{\isachardoublequoteclose}\ {\isachardoublequoteopen}y\ {\isacharcolon}{\kern0pt}\ A\ {\isasymrightarrow}\ Y{\isachardoublequoteclose}\ {\isachardoublequoteopen}z\ {\isacharcolon}{\kern0pt}\ A\ {\isasymrightarrow}\ Z{\isachardoublequoteclose}\isanewline
\ \ \isakeyword{shows}\ {\isachardoublequoteopen}associate{\isacharunderscore}{\kern0pt}left\ X\ Y\ Z\ {\isasymcirc}\isactrlsub c\ {\isasymlangle}x{\isacharcomma}{\kern0pt}\ {\isasymlangle}y{\isacharcomma}{\kern0pt}\ z{\isasymrangle}{\isasymrangle}\ {\isacharequal}{\kern0pt}\ {\isasymlangle}{\isasymlangle}x{\isacharcomma}{\kern0pt}\ y{\isasymrangle}{\isacharcomma}{\kern0pt}\ z{\isasymrangle}{\isachardoublequoteclose}\isanewline
%
\isadelimproof
%
\endisadelimproof
%
\isatagproof
\isacommand{proof}\isamarkupfalse%
\ {\isacharminus}{\kern0pt}\isanewline
\ \ \isacommand{have}\isamarkupfalse%
\ {\isachardoublequoteopen}associate{\isacharunderscore}{\kern0pt}left\ X\ Y\ Z\ {\isasymcirc}\isactrlsub c\ {\isasymlangle}x{\isacharcomma}{\kern0pt}\ {\isasymlangle}y{\isacharcomma}{\kern0pt}\ z{\isasymrangle}{\isasymrangle}\ \ {\isacharequal}{\kern0pt}\ {\isasymlangle}{\isasymlangle}left{\isacharunderscore}{\kern0pt}cart{\isacharunderscore}{\kern0pt}proj\ X\ {\isacharparenleft}{\kern0pt}Y\ {\isasymtimes}\isactrlsub c\ Z{\isacharparenright}{\kern0pt}{\isacharcomma}{\kern0pt}\isanewline
\ \ \ \ \ \ \ \ left{\isacharunderscore}{\kern0pt}cart{\isacharunderscore}{\kern0pt}proj\ Y\ Z\ {\isasymcirc}\isactrlsub c\ right{\isacharunderscore}{\kern0pt}cart{\isacharunderscore}{\kern0pt}proj\ X\ {\isacharparenleft}{\kern0pt}Y\ {\isasymtimes}\isactrlsub c\ Z{\isacharparenright}{\kern0pt}{\isasymrangle}\ {\isasymcirc}\isactrlsub c\ {\isasymlangle}x{\isacharcomma}{\kern0pt}\ {\isasymlangle}y{\isacharcomma}{\kern0pt}\ z{\isasymrangle}{\isasymrangle}{\isacharcomma}{\kern0pt}\isanewline
\ \ \ \ \ \ \ \ right{\isacharunderscore}{\kern0pt}cart{\isacharunderscore}{\kern0pt}proj\ Y\ Z\ {\isasymcirc}\isactrlsub c\ right{\isacharunderscore}{\kern0pt}cart{\isacharunderscore}{\kern0pt}proj\ X\ {\isacharparenleft}{\kern0pt}Y\ {\isasymtimes}\isactrlsub c\ Z{\isacharparenright}{\kern0pt}\ \ {\isasymcirc}\isactrlsub c\ {\isasymlangle}x{\isacharcomma}{\kern0pt}\ {\isasymlangle}y{\isacharcomma}{\kern0pt}\ z{\isasymrangle}{\isasymrangle}{\isasymrangle}{\isachardoublequoteclose}\isanewline
\ \ \ \ \isacommand{using}\isamarkupfalse%
\ assms\ associate{\isacharunderscore}{\kern0pt}left{\isacharunderscore}{\kern0pt}def\ cfunc{\isacharunderscore}{\kern0pt}prod{\isacharunderscore}{\kern0pt}comp\ cfunc{\isacharunderscore}{\kern0pt}type{\isacharunderscore}{\kern0pt}def\ comp{\isacharunderscore}{\kern0pt}associative\ comp{\isacharunderscore}{\kern0pt}type\ \isacommand{by}\isamarkupfalse%
\ {\isacharparenleft}{\kern0pt}typecheck{\isacharunderscore}{\kern0pt}cfuncs{\isacharcomma}{\kern0pt}\ auto{\isacharparenright}{\kern0pt}\isanewline
\ \ \isacommand{also}\isamarkupfalse%
\ \isacommand{have}\isamarkupfalse%
\ {\isachardoublequoteopen}{\isachardot}{\kern0pt}{\isachardot}{\kern0pt}{\isachardot}{\kern0pt}\ {\isacharequal}{\kern0pt}\ {\isasymlangle}\ {\isasymlangle}left{\isacharunderscore}{\kern0pt}cart{\isacharunderscore}{\kern0pt}proj\ X\ {\isacharparenleft}{\kern0pt}Y\ {\isasymtimes}\isactrlsub c\ Z{\isacharparenright}{\kern0pt}\ {\isasymcirc}\isactrlsub c\ {\isasymlangle}x{\isacharcomma}{\kern0pt}\ {\isasymlangle}y{\isacharcomma}{\kern0pt}\ z{\isasymrangle}{\isasymrangle}{\isacharcomma}{\kern0pt}\isanewline
\ \ \ \ \ \ \ \ left{\isacharunderscore}{\kern0pt}cart{\isacharunderscore}{\kern0pt}proj\ Y\ Z\ {\isasymcirc}\isactrlsub c\ right{\isacharunderscore}{\kern0pt}cart{\isacharunderscore}{\kern0pt}proj\ X\ {\isacharparenleft}{\kern0pt}Y\ {\isasymtimes}\isactrlsub c\ Z{\isacharparenright}{\kern0pt}\ {\isasymcirc}\isactrlsub c\ {\isasymlangle}x{\isacharcomma}{\kern0pt}\ {\isasymlangle}y{\isacharcomma}{\kern0pt}\ z{\isasymrangle}{\isasymrangle}{\isasymrangle}{\isacharcomma}{\kern0pt}\isanewline
\ \ \ \ \ \ \ \ right{\isacharunderscore}{\kern0pt}cart{\isacharunderscore}{\kern0pt}proj\ Y\ Z\ {\isasymcirc}\isactrlsub c\ right{\isacharunderscore}{\kern0pt}cart{\isacharunderscore}{\kern0pt}proj\ X\ {\isacharparenleft}{\kern0pt}Y\ {\isasymtimes}\isactrlsub c\ Z{\isacharparenright}{\kern0pt}\ \ {\isasymcirc}\isactrlsub c\ {\isasymlangle}x{\isacharcomma}{\kern0pt}\ {\isasymlangle}y{\isacharcomma}{\kern0pt}\ z{\isasymrangle}{\isasymrangle}{\isasymrangle}{\isachardoublequoteclose}\isanewline
\ \ \ \ \isacommand{using}\isamarkupfalse%
\ assms\ \isacommand{by}\isamarkupfalse%
\ {\isacharparenleft}{\kern0pt}typecheck{\isacharunderscore}{\kern0pt}cfuncs{\isacharcomma}{\kern0pt}\ simp\ add{\isacharcolon}{\kern0pt}\ cfunc{\isacharunderscore}{\kern0pt}prod{\isacharunderscore}{\kern0pt}comp\ comp{\isacharunderscore}{\kern0pt}associative{\isadigit{2}}{\isacharparenright}{\kern0pt}\isanewline
\ \ \isacommand{also}\isamarkupfalse%
\ \isacommand{have}\isamarkupfalse%
\ {\isachardoublequoteopen}{\isachardot}{\kern0pt}{\isachardot}{\kern0pt}{\isachardot}{\kern0pt}\ {\isacharequal}{\kern0pt}\ {\isasymlangle}{\isasymlangle}x{\isacharcomma}{\kern0pt}\ left{\isacharunderscore}{\kern0pt}cart{\isacharunderscore}{\kern0pt}proj\ Y\ Z\ {\isasymcirc}\isactrlsub c\ {\isasymlangle}y{\isacharcomma}{\kern0pt}\ z{\isasymrangle}{\isasymrangle}{\isacharcomma}{\kern0pt}\ right{\isacharunderscore}{\kern0pt}cart{\isacharunderscore}{\kern0pt}proj\ Y\ Z\ {\isasymcirc}\isactrlsub c\ {\isasymlangle}y{\isacharcomma}{\kern0pt}\ z{\isasymrangle}{\isasymrangle}{\isachardoublequoteclose}\isanewline
\ \ \ \ \isacommand{using}\isamarkupfalse%
\ assms\ left{\isacharunderscore}{\kern0pt}cart{\isacharunderscore}{\kern0pt}proj{\isacharunderscore}{\kern0pt}cfunc{\isacharunderscore}{\kern0pt}prod\ right{\isacharunderscore}{\kern0pt}cart{\isacharunderscore}{\kern0pt}proj{\isacharunderscore}{\kern0pt}cfunc{\isacharunderscore}{\kern0pt}prod\ \isacommand{by}\isamarkupfalse%
\ {\isacharparenleft}{\kern0pt}typecheck{\isacharunderscore}{\kern0pt}cfuncs{\isacharcomma}{\kern0pt}\ auto{\isacharparenright}{\kern0pt}\isanewline
\ \ \isacommand{also}\isamarkupfalse%
\ \isacommand{have}\isamarkupfalse%
\ {\isachardoublequoteopen}{\isachardot}{\kern0pt}{\isachardot}{\kern0pt}{\isachardot}{\kern0pt}\ {\isacharequal}{\kern0pt}\ {\isasymlangle}{\isasymlangle}x{\isacharcomma}{\kern0pt}\ y{\isasymrangle}{\isacharcomma}{\kern0pt}\ z{\isasymrangle}{\isachardoublequoteclose}\isanewline
\ \ \ \ \isacommand{using}\isamarkupfalse%
\ assms\ left{\isacharunderscore}{\kern0pt}cart{\isacharunderscore}{\kern0pt}proj{\isacharunderscore}{\kern0pt}cfunc{\isacharunderscore}{\kern0pt}prod\ right{\isacharunderscore}{\kern0pt}cart{\isacharunderscore}{\kern0pt}proj{\isacharunderscore}{\kern0pt}cfunc{\isacharunderscore}{\kern0pt}prod\ \isacommand{by}\isamarkupfalse%
\ auto\isanewline
\ \ \isacommand{then}\isamarkupfalse%
\ \isacommand{show}\isamarkupfalse%
\ {\isacharquery}{\kern0pt}thesis\isanewline
\ \ \ \ \isacommand{using}\isamarkupfalse%
\ calculation\ \isacommand{by}\isamarkupfalse%
\ auto\isanewline
\isacommand{qed}\isamarkupfalse%
%
\endisatagproof
{\isafoldproof}%
%
\isadelimproof
\isanewline
%
\endisadelimproof
\isanewline
\isacommand{lemma}\isamarkupfalse%
\ right{\isacharunderscore}{\kern0pt}left{\isacharcolon}{\kern0pt}\ \isanewline
\ {\isachardoublequoteopen}associate{\isacharunderscore}{\kern0pt}right\ A\ B\ C\ {\isasymcirc}\isactrlsub c\ associate{\isacharunderscore}{\kern0pt}left\ A\ B\ C\ {\isacharequal}{\kern0pt}\ id\ {\isacharparenleft}{\kern0pt}A\ {\isasymtimes}\isactrlsub c\ {\isacharparenleft}{\kern0pt}B\ {\isasymtimes}\isactrlsub c\ C{\isacharparenright}{\kern0pt}{\isacharparenright}{\kern0pt}{\isachardoublequoteclose}\isanewline
%
\isadelimproof
\ \ %
\endisadelimproof
%
\isatagproof
\isacommand{by}\isamarkupfalse%
\ {\isacharparenleft}{\kern0pt}typecheck{\isacharunderscore}{\kern0pt}cfuncs{\isacharcomma}{\kern0pt}\ smt\ {\isacharparenleft}{\kern0pt}verit{\isacharcomma}{\kern0pt}\ ccfv{\isacharunderscore}{\kern0pt}threshold{\isacharparenright}{\kern0pt}\ associate{\isacharunderscore}{\kern0pt}left{\isacharunderscore}{\kern0pt}def\ associate{\isacharunderscore}{\kern0pt}right{\isacharunderscore}{\kern0pt}ap\ cfunc{\isacharunderscore}{\kern0pt}prod{\isacharunderscore}{\kern0pt}unique\ comp{\isacharunderscore}{\kern0pt}type\ id{\isacharunderscore}{\kern0pt}right{\isacharunderscore}{\kern0pt}unit{\isadigit{2}}\ left{\isacharunderscore}{\kern0pt}cart{\isacharunderscore}{\kern0pt}proj{\isacharunderscore}{\kern0pt}type\ right{\isacharunderscore}{\kern0pt}cart{\isacharunderscore}{\kern0pt}proj{\isacharunderscore}{\kern0pt}type{\isacharparenright}{\kern0pt}%
\endisatagproof
{\isafoldproof}%
%
\isadelimproof
\isanewline
%
\endisadelimproof
\isanewline
\isacommand{lemma}\isamarkupfalse%
\ left{\isacharunderscore}{\kern0pt}right{\isacharcolon}{\kern0pt}\ \isanewline
\ {\isachardoublequoteopen}associate{\isacharunderscore}{\kern0pt}left\ A\ B\ C\ {\isasymcirc}\isactrlsub c\ associate{\isacharunderscore}{\kern0pt}right\ A\ B\ C\ {\isacharequal}{\kern0pt}\ id\ {\isacharparenleft}{\kern0pt}{\isacharparenleft}{\kern0pt}A\ {\isasymtimes}\isactrlsub c\ B{\isacharparenright}{\kern0pt}\ {\isasymtimes}\isactrlsub c\ C{\isacharparenright}{\kern0pt}{\isachardoublequoteclose}\isanewline
%
\isadelimproof
\ \ \ \ %
\endisadelimproof
%
\isatagproof
\isacommand{by}\isamarkupfalse%
\ {\isacharparenleft}{\kern0pt}smt\ associate{\isacharunderscore}{\kern0pt}left{\isacharunderscore}{\kern0pt}ap\ associate{\isacharunderscore}{\kern0pt}right{\isacharunderscore}{\kern0pt}def\ cfunc{\isacharunderscore}{\kern0pt}cross{\isacharunderscore}{\kern0pt}prod{\isacharunderscore}{\kern0pt}def\ cfunc{\isacharunderscore}{\kern0pt}prod{\isacharunderscore}{\kern0pt}unique\ comp{\isacharunderscore}{\kern0pt}type\ id{\isacharunderscore}{\kern0pt}cross{\isacharunderscore}{\kern0pt}prod\ id{\isacharunderscore}{\kern0pt}domain\ id{\isacharunderscore}{\kern0pt}left{\isacharunderscore}{\kern0pt}unit{\isadigit{2}}\ left{\isacharunderscore}{\kern0pt}cart{\isacharunderscore}{\kern0pt}proj{\isacharunderscore}{\kern0pt}type\ right{\isacharunderscore}{\kern0pt}cart{\isacharunderscore}{\kern0pt}proj{\isacharunderscore}{\kern0pt}type{\isacharparenright}{\kern0pt}%
\endisatagproof
{\isafoldproof}%
%
\isadelimproof
\isanewline
%
\endisadelimproof
\isanewline
\isacommand{lemma}\isamarkupfalse%
\ product{\isacharunderscore}{\kern0pt}associates{\isacharcolon}{\kern0pt}\isanewline
\ \ {\isachardoublequoteopen}A\ {\isasymtimes}\isactrlsub c\ {\isacharparenleft}{\kern0pt}B\ {\isasymtimes}\isactrlsub c\ C{\isacharparenright}{\kern0pt}\ \ {\isasymcong}\ {\isacharparenleft}{\kern0pt}A\ {\isasymtimes}\isactrlsub c\ B{\isacharparenright}{\kern0pt}\ {\isasymtimes}\isactrlsub c\ C{\isachardoublequoteclose}\isanewline
%
\isadelimproof
\ \ \ \ %
\endisadelimproof
%
\isatagproof
\isacommand{by}\isamarkupfalse%
\ {\isacharparenleft}{\kern0pt}metis\ associate{\isacharunderscore}{\kern0pt}left{\isacharunderscore}{\kern0pt}type\ associate{\isacharunderscore}{\kern0pt}right{\isacharunderscore}{\kern0pt}type\ cfunc{\isacharunderscore}{\kern0pt}type{\isacharunderscore}{\kern0pt}def\ is{\isacharunderscore}{\kern0pt}isomorphic{\isacharunderscore}{\kern0pt}def\ isomorphism{\isacharunderscore}{\kern0pt}def\ left{\isacharunderscore}{\kern0pt}right\ right{\isacharunderscore}{\kern0pt}left{\isacharparenright}{\kern0pt}%
\endisatagproof
{\isafoldproof}%
%
\isadelimproof
\ \isanewline
%
\endisadelimproof
\isanewline
\isacommand{lemma}\isamarkupfalse%
\ associate{\isacharunderscore}{\kern0pt}left{\isacharunderscore}{\kern0pt}crossprod{\isacharunderscore}{\kern0pt}ap{\isacharcolon}{\kern0pt}\isanewline
\ \ \isakeyword{assumes}\ {\isachardoublequoteopen}x\ {\isacharcolon}{\kern0pt}\ A\ {\isasymrightarrow}\ X{\isachardoublequoteclose}\ {\isachardoublequoteopen}y\ {\isacharcolon}{\kern0pt}\ B\ {\isasymrightarrow}\ Y{\isachardoublequoteclose}\ {\isachardoublequoteopen}z\ {\isacharcolon}{\kern0pt}\ C\ {\isasymrightarrow}\ Z{\isachardoublequoteclose}\isanewline
\ \ \isakeyword{shows}\ {\isachardoublequoteopen}associate{\isacharunderscore}{\kern0pt}left\ X\ Y\ Z\ {\isasymcirc}\isactrlsub c\ {\isacharparenleft}{\kern0pt}x\ {\isasymtimes}\isactrlsub f\ {\isacharparenleft}{\kern0pt}y{\isasymtimes}\isactrlsub f\ z{\isacharparenright}{\kern0pt}{\isacharparenright}{\kern0pt}\ {\isacharequal}{\kern0pt}\ {\isacharparenleft}{\kern0pt}{\isacharparenleft}{\kern0pt}x\ {\isasymtimes}\isactrlsub f\ y{\isacharparenright}{\kern0pt}{\isasymtimes}\isactrlsub f\ z{\isacharparenright}{\kern0pt}\ {\isasymcirc}\isactrlsub c\ \ associate{\isacharunderscore}{\kern0pt}left\ A\ B\ C{\isachardoublequoteclose}\isanewline
%
\isadelimproof
%
\endisadelimproof
%
\isatagproof
\isacommand{proof}\isamarkupfalse%
{\isacharminus}{\kern0pt}\isanewline
\ \ \isacommand{have}\isamarkupfalse%
\ {\isachardoublequoteopen}associate{\isacharunderscore}{\kern0pt}left\ X\ Y\ Z\ {\isasymcirc}\isactrlsub c\ {\isacharparenleft}{\kern0pt}x\ {\isasymtimes}\isactrlsub f\ {\isacharparenleft}{\kern0pt}y{\isasymtimes}\isactrlsub f\ z{\isacharparenright}{\kern0pt}{\isacharparenright}{\kern0pt}\ {\isacharequal}{\kern0pt}\isanewline
\ \ \ \ \ \ \ \ associate{\isacharunderscore}{\kern0pt}left\ X\ Y\ Z\ {\isasymcirc}\isactrlsub c\ {\isasymlangle}x\ {\isasymcirc}\isactrlsub c\ left{\isacharunderscore}{\kern0pt}cart{\isacharunderscore}{\kern0pt}proj\ A\ {\isacharparenleft}{\kern0pt}B{\isasymtimes}\isactrlsub cC{\isacharparenright}{\kern0pt}{\isacharcomma}{\kern0pt}\ {\isasymlangle}y\ {\isasymcirc}\isactrlsub c\ left{\isacharunderscore}{\kern0pt}cart{\isacharunderscore}{\kern0pt}proj\ B\ C{\isacharcomma}{\kern0pt}\ z\ {\isasymcirc}\isactrlsub c\ right{\isacharunderscore}{\kern0pt}cart{\isacharunderscore}{\kern0pt}proj\ B\ C{\isasymrangle}\ {\isasymcirc}\isactrlsub c\ right{\isacharunderscore}{\kern0pt}cart{\isacharunderscore}{\kern0pt}proj\ A\ {\isacharparenleft}{\kern0pt}B{\isasymtimes}\isactrlsub cC{\isacharparenright}{\kern0pt}{\isasymrangle}{\isachardoublequoteclose}\isanewline
\ \ \ \ \isacommand{using}\isamarkupfalse%
\ assms\ \isacommand{by}\isamarkupfalse%
{\isacharparenleft}{\kern0pt}unfold\ cfunc{\isacharunderscore}{\kern0pt}cross{\isacharunderscore}{\kern0pt}prod{\isacharunderscore}{\kern0pt}def{\isadigit{2}}{\isacharcomma}{\kern0pt}\ typecheck{\isacharunderscore}{\kern0pt}cfuncs{\isacharcomma}{\kern0pt}\ unfold\ cfunc{\isacharunderscore}{\kern0pt}cross{\isacharunderscore}{\kern0pt}prod{\isacharunderscore}{\kern0pt}def{\isadigit{2}}{\isacharcomma}{\kern0pt}\ auto{\isacharparenright}{\kern0pt}\ \isanewline
\ \ \isacommand{also}\isamarkupfalse%
\ \isacommand{have}\isamarkupfalse%
\ {\isachardoublequoteopen}{\isachardot}{\kern0pt}{\isachardot}{\kern0pt}{\isachardot}{\kern0pt}\ {\isacharequal}{\kern0pt}\ associate{\isacharunderscore}{\kern0pt}left\ X\ Y\ Z\ {\isasymcirc}\isactrlsub c\ {\isasymlangle}x\ {\isasymcirc}\isactrlsub c\ left{\isacharunderscore}{\kern0pt}cart{\isacharunderscore}{\kern0pt}proj\ A\ {\isacharparenleft}{\kern0pt}B{\isasymtimes}\isactrlsub cC{\isacharparenright}{\kern0pt}{\isacharcomma}{\kern0pt}\ {\isasymlangle}y\ {\isasymcirc}\isactrlsub c\ left{\isacharunderscore}{\kern0pt}cart{\isacharunderscore}{\kern0pt}proj\ B\ C\ {\isasymcirc}\isactrlsub c\ right{\isacharunderscore}{\kern0pt}cart{\isacharunderscore}{\kern0pt}proj\ A\ {\isacharparenleft}{\kern0pt}B{\isasymtimes}\isactrlsub cC{\isacharparenright}{\kern0pt}{\isacharcomma}{\kern0pt}\ z\ {\isasymcirc}\isactrlsub c\ right{\isacharunderscore}{\kern0pt}cart{\isacharunderscore}{\kern0pt}proj\ B\ C\ {\isasymcirc}\isactrlsub c\ right{\isacharunderscore}{\kern0pt}cart{\isacharunderscore}{\kern0pt}proj\ A\ {\isacharparenleft}{\kern0pt}B{\isasymtimes}\isactrlsub cC{\isacharparenright}{\kern0pt}{\isasymrangle}{\isasymrangle}{\isachardoublequoteclose}\isanewline
\ \ \ \ \isacommand{using}\isamarkupfalse%
\ assms\ \ cfunc{\isacharunderscore}{\kern0pt}prod{\isacharunderscore}{\kern0pt}comp\ comp{\isacharunderscore}{\kern0pt}associative{\isadigit{2}}\ \isacommand{by}\isamarkupfalse%
\ {\isacharparenleft}{\kern0pt}typecheck{\isacharunderscore}{\kern0pt}cfuncs{\isacharcomma}{\kern0pt}\ auto{\isacharparenright}{\kern0pt}\isanewline
\ \ \isacommand{also}\isamarkupfalse%
\ \isacommand{have}\isamarkupfalse%
\ {\isachardoublequoteopen}{\isachardot}{\kern0pt}{\isachardot}{\kern0pt}{\isachardot}{\kern0pt}\ {\isacharequal}{\kern0pt}\ {\isasymlangle}{\isasymlangle}x\ {\isasymcirc}\isactrlsub c\ left{\isacharunderscore}{\kern0pt}cart{\isacharunderscore}{\kern0pt}proj\ A\ {\isacharparenleft}{\kern0pt}B{\isasymtimes}\isactrlsub cC{\isacharparenright}{\kern0pt}{\isacharcomma}{\kern0pt}\ y\ {\isasymcirc}\isactrlsub c\ left{\isacharunderscore}{\kern0pt}cart{\isacharunderscore}{\kern0pt}proj\ B\ C\ {\isasymcirc}\isactrlsub c\ right{\isacharunderscore}{\kern0pt}cart{\isacharunderscore}{\kern0pt}proj\ A\ {\isacharparenleft}{\kern0pt}B{\isasymtimes}\isactrlsub cC{\isacharparenright}{\kern0pt}{\isasymrangle}{\isacharcomma}{\kern0pt}z\ {\isasymcirc}\isactrlsub c\ right{\isacharunderscore}{\kern0pt}cart{\isacharunderscore}{\kern0pt}proj\ B\ C\ {\isasymcirc}\isactrlsub c\ right{\isacharunderscore}{\kern0pt}cart{\isacharunderscore}{\kern0pt}proj\ A\ {\isacharparenleft}{\kern0pt}B{\isasymtimes}\isactrlsub cC{\isacharparenright}{\kern0pt}{\isasymrangle}{\isachardoublequoteclose}\isanewline
\ \ \ \ \isacommand{using}\isamarkupfalse%
\ assms\ \isacommand{by}\isamarkupfalse%
\ {\isacharparenleft}{\kern0pt}typecheck{\isacharunderscore}{\kern0pt}cfuncs{\isacharcomma}{\kern0pt}\ simp\ add{\isacharcolon}{\kern0pt}\ associate{\isacharunderscore}{\kern0pt}left{\isacharunderscore}{\kern0pt}ap{\isacharparenright}{\kern0pt}\isanewline
\ \ \isacommand{also}\isamarkupfalse%
\ \isacommand{have}\isamarkupfalse%
\ {\isachardoublequoteopen}{\isachardot}{\kern0pt}{\isachardot}{\kern0pt}{\isachardot}{\kern0pt}\ {\isacharequal}{\kern0pt}\ {\isasymlangle}{\isacharparenleft}{\kern0pt}x\ {\isasymtimes}\isactrlsub f\ y{\isacharparenright}{\kern0pt}{\isasymcirc}\isactrlsub c\ {\isasymlangle}\ left{\isacharunderscore}{\kern0pt}cart{\isacharunderscore}{\kern0pt}proj\ A\ {\isacharparenleft}{\kern0pt}B{\isasymtimes}\isactrlsub cC{\isacharparenright}{\kern0pt}{\isacharcomma}{\kern0pt}\ left{\isacharunderscore}{\kern0pt}cart{\isacharunderscore}{\kern0pt}proj\ B\ C\ {\isasymcirc}\isactrlsub c\ right{\isacharunderscore}{\kern0pt}cart{\isacharunderscore}{\kern0pt}proj\ A\ {\isacharparenleft}{\kern0pt}B{\isasymtimes}\isactrlsub cC{\isacharparenright}{\kern0pt}{\isasymrangle}{\isacharcomma}{\kern0pt}z\ {\isasymcirc}\isactrlsub c\ right{\isacharunderscore}{\kern0pt}cart{\isacharunderscore}{\kern0pt}proj\ B\ C\ {\isasymcirc}\isactrlsub c\ right{\isacharunderscore}{\kern0pt}cart{\isacharunderscore}{\kern0pt}proj\ A\ {\isacharparenleft}{\kern0pt}B{\isasymtimes}\isactrlsub cC{\isacharparenright}{\kern0pt}{\isasymrangle}{\isachardoublequoteclose}\isanewline
\ \ \ \ \isacommand{using}\isamarkupfalse%
\ assms\ \isacommand{by}\isamarkupfalse%
\ {\isacharparenleft}{\kern0pt}typecheck{\isacharunderscore}{\kern0pt}cfuncs{\isacharcomma}{\kern0pt}\ simp\ add{\isacharcolon}{\kern0pt}\ cfunc{\isacharunderscore}{\kern0pt}cross{\isacharunderscore}{\kern0pt}prod{\isacharunderscore}{\kern0pt}comp{\isacharunderscore}{\kern0pt}cfunc{\isacharunderscore}{\kern0pt}prod{\isacharparenright}{\kern0pt}\isanewline
\ \ \isacommand{also}\isamarkupfalse%
\ \isacommand{have}\isamarkupfalse%
\ {\isachardoublequoteopen}{\isachardot}{\kern0pt}{\isachardot}{\kern0pt}{\isachardot}{\kern0pt}\ {\isacharequal}{\kern0pt}\ {\isacharparenleft}{\kern0pt}{\isacharparenleft}{\kern0pt}x\ {\isasymtimes}\isactrlsub f\ y{\isacharparenright}{\kern0pt}{\isasymtimes}\isactrlsub f\ z{\isacharparenright}{\kern0pt}\ {\isasymcirc}\isactrlsub c\ {\isasymlangle}{\isasymlangle}left{\isacharunderscore}{\kern0pt}cart{\isacharunderscore}{\kern0pt}proj\ A\ {\isacharparenleft}{\kern0pt}B{\isasymtimes}\isactrlsub cC{\isacharparenright}{\kern0pt}{\isacharcomma}{\kern0pt}\ left{\isacharunderscore}{\kern0pt}cart{\isacharunderscore}{\kern0pt}proj\ B\ C\ {\isasymcirc}\isactrlsub c\ right{\isacharunderscore}{\kern0pt}cart{\isacharunderscore}{\kern0pt}proj\ A\ {\isacharparenleft}{\kern0pt}B{\isasymtimes}\isactrlsub cC{\isacharparenright}{\kern0pt}{\isasymrangle}{\isacharcomma}{\kern0pt}right{\isacharunderscore}{\kern0pt}cart{\isacharunderscore}{\kern0pt}proj\ B\ C\ {\isasymcirc}\isactrlsub c\ right{\isacharunderscore}{\kern0pt}cart{\isacharunderscore}{\kern0pt}proj\ A\ {\isacharparenleft}{\kern0pt}B{\isasymtimes}\isactrlsub cC{\isacharparenright}{\kern0pt}{\isasymrangle}{\isachardoublequoteclose}\isanewline
\ \ \ \ \isacommand{using}\isamarkupfalse%
\ assms\ \isacommand{by}\isamarkupfalse%
\ {\isacharparenleft}{\kern0pt}typecheck{\isacharunderscore}{\kern0pt}cfuncs{\isacharcomma}{\kern0pt}\ simp\ add{\isacharcolon}{\kern0pt}\ cfunc{\isacharunderscore}{\kern0pt}cross{\isacharunderscore}{\kern0pt}prod{\isacharunderscore}{\kern0pt}comp{\isacharunderscore}{\kern0pt}cfunc{\isacharunderscore}{\kern0pt}prod{\isacharparenright}{\kern0pt}\isanewline
\ \ \isacommand{also}\isamarkupfalse%
\ \isacommand{have}\isamarkupfalse%
\ {\isachardoublequoteopen}{\isachardot}{\kern0pt}{\isachardot}{\kern0pt}{\isachardot}{\kern0pt}\ {\isacharequal}{\kern0pt}\ {\isacharparenleft}{\kern0pt}{\isacharparenleft}{\kern0pt}x\ {\isasymtimes}\isactrlsub f\ y{\isacharparenright}{\kern0pt}{\isasymtimes}\isactrlsub f\ z{\isacharparenright}{\kern0pt}\ {\isasymcirc}\isactrlsub c\ associate{\isacharunderscore}{\kern0pt}left\ A\ B\ C{\isachardoublequoteclose}\ \ \ \isanewline
\ \ \ \ \isacommand{unfolding}\isamarkupfalse%
\ associate{\isacharunderscore}{\kern0pt}left{\isacharunderscore}{\kern0pt}def\ \isacommand{by}\isamarkupfalse%
\ auto\isanewline
\ \ \isacommand{then}\isamarkupfalse%
\ \isacommand{show}\isamarkupfalse%
\ {\isacharquery}{\kern0pt}thesis\ \isacommand{using}\isamarkupfalse%
\ calculation\ \isacommand{by}\isamarkupfalse%
\ auto\isanewline
\isacommand{qed}\isamarkupfalse%
%
\endisatagproof
{\isafoldproof}%
%
\isadelimproof
%
\endisadelimproof
%
\isadelimdocument
%
\endisadelimdocument
%
\isatagdocument
%
\isamarkupsubsubsection{Distributing over a Cartesian Product from the Right%
}
\isamarkuptrue%
%
\endisatagdocument
{\isafolddocument}%
%
\isadelimdocument
%
\endisadelimdocument
\isacommand{definition}\isamarkupfalse%
\ distribute{\isacharunderscore}{\kern0pt}right{\isacharunderscore}{\kern0pt}left\ {\isacharcolon}{\kern0pt}{\isacharcolon}{\kern0pt}\ {\isachardoublequoteopen}cset\ {\isasymRightarrow}\ cset\ {\isasymRightarrow}\ cset\ {\isasymRightarrow}\ cfunc{\isachardoublequoteclose}\ \isakeyword{where}\isanewline
\ \ {\isachardoublequoteopen}distribute{\isacharunderscore}{\kern0pt}right{\isacharunderscore}{\kern0pt}left\ X\ Y\ Z\ {\isacharequal}{\kern0pt}\ \isanewline
\ \ \ \ {\isasymlangle}left{\isacharunderscore}{\kern0pt}cart{\isacharunderscore}{\kern0pt}proj\ X\ Y\ {\isasymcirc}\isactrlsub c\ left{\isacharunderscore}{\kern0pt}cart{\isacharunderscore}{\kern0pt}proj\ {\isacharparenleft}{\kern0pt}X\ {\isasymtimes}\isactrlsub c\ Y{\isacharparenright}{\kern0pt}\ Z{\isacharcomma}{\kern0pt}\ right{\isacharunderscore}{\kern0pt}cart{\isacharunderscore}{\kern0pt}proj\ {\isacharparenleft}{\kern0pt}X\ {\isasymtimes}\isactrlsub c\ Y{\isacharparenright}{\kern0pt}\ Z{\isasymrangle}{\isachardoublequoteclose}\isanewline
\isanewline
\isacommand{lemma}\isamarkupfalse%
\ distribute{\isacharunderscore}{\kern0pt}right{\isacharunderscore}{\kern0pt}left{\isacharunderscore}{\kern0pt}type{\isacharbrackleft}{\kern0pt}type{\isacharunderscore}{\kern0pt}rule{\isacharbrackright}{\kern0pt}{\isacharcolon}{\kern0pt}\isanewline
\ \ {\isachardoublequoteopen}distribute{\isacharunderscore}{\kern0pt}right{\isacharunderscore}{\kern0pt}left\ X\ Y\ Z\ {\isacharcolon}{\kern0pt}\ {\isacharparenleft}{\kern0pt}X\ {\isasymtimes}\isactrlsub c\ Y{\isacharparenright}{\kern0pt}\ {\isasymtimes}\isactrlsub c\ Z\ {\isasymrightarrow}\ X\ {\isasymtimes}\isactrlsub c\ Z{\isachardoublequoteclose}\isanewline
%
\isadelimproof
\ \ %
\endisadelimproof
%
\isatagproof
\isacommand{unfolding}\isamarkupfalse%
\ distribute{\isacharunderscore}{\kern0pt}right{\isacharunderscore}{\kern0pt}left{\isacharunderscore}{\kern0pt}def\isanewline
\ \ \isacommand{using}\isamarkupfalse%
\ cfunc{\isacharunderscore}{\kern0pt}prod{\isacharunderscore}{\kern0pt}type\ comp{\isacharunderscore}{\kern0pt}type\ left{\isacharunderscore}{\kern0pt}cart{\isacharunderscore}{\kern0pt}proj{\isacharunderscore}{\kern0pt}type\ right{\isacharunderscore}{\kern0pt}cart{\isacharunderscore}{\kern0pt}proj{\isacharunderscore}{\kern0pt}type\ \isacommand{by}\isamarkupfalse%
\ blast%
\endisatagproof
{\isafoldproof}%
%
\isadelimproof
\isanewline
%
\endisadelimproof
\isanewline
\isacommand{lemma}\isamarkupfalse%
\ distribute{\isacharunderscore}{\kern0pt}right{\isacharunderscore}{\kern0pt}left{\isacharunderscore}{\kern0pt}ap{\isacharcolon}{\kern0pt}\ \isanewline
\ \ \isakeyword{assumes}\ {\isachardoublequoteopen}x\ {\isacharcolon}{\kern0pt}\ A\ {\isasymrightarrow}\ X{\isachardoublequoteclose}\ {\isachardoublequoteopen}y\ {\isacharcolon}{\kern0pt}\ A\ {\isasymrightarrow}\ Y{\isachardoublequoteclose}\ {\isachardoublequoteopen}z\ {\isacharcolon}{\kern0pt}\ A\ {\isasymrightarrow}\ Z{\isachardoublequoteclose}\isanewline
\ \ \isakeyword{shows}\ {\isachardoublequoteopen}distribute{\isacharunderscore}{\kern0pt}right{\isacharunderscore}{\kern0pt}left\ X\ Y\ Z\ {\isasymcirc}\isactrlsub c\ {\isasymlangle}{\isasymlangle}x{\isacharcomma}{\kern0pt}\ y{\isasymrangle}{\isacharcomma}{\kern0pt}\ z{\isasymrangle}\ {\isacharequal}{\kern0pt}\ {\isasymlangle}x{\isacharcomma}{\kern0pt}\ z{\isasymrangle}{\isachardoublequoteclose}\isanewline
%
\isadelimproof
\ \ %
\endisadelimproof
%
\isatagproof
\isacommand{unfolding}\isamarkupfalse%
\ distribute{\isacharunderscore}{\kern0pt}right{\isacharunderscore}{\kern0pt}left{\isacharunderscore}{\kern0pt}def\ \isanewline
\ \ \isacommand{by}\isamarkupfalse%
\ {\isacharparenleft}{\kern0pt}typecheck{\isacharunderscore}{\kern0pt}cfuncs{\isacharcomma}{\kern0pt}\ smt\ {\isacharparenleft}{\kern0pt}verit{\isacharcomma}{\kern0pt}\ best{\isacharparenright}{\kern0pt}\ assms\ cfunc{\isacharunderscore}{\kern0pt}prod{\isacharunderscore}{\kern0pt}comp\ comp{\isacharunderscore}{\kern0pt}associative{\isadigit{2}}\ left{\isacharunderscore}{\kern0pt}cart{\isacharunderscore}{\kern0pt}proj{\isacharunderscore}{\kern0pt}cfunc{\isacharunderscore}{\kern0pt}prod\ right{\isacharunderscore}{\kern0pt}cart{\isacharunderscore}{\kern0pt}proj{\isacharunderscore}{\kern0pt}cfunc{\isacharunderscore}{\kern0pt}prod{\isacharparenright}{\kern0pt}%
\endisatagproof
{\isafoldproof}%
%
\isadelimproof
\isanewline
%
\endisadelimproof
\isanewline
\isacommand{definition}\isamarkupfalse%
\ distribute{\isacharunderscore}{\kern0pt}right{\isacharunderscore}{\kern0pt}right\ {\isacharcolon}{\kern0pt}{\isacharcolon}{\kern0pt}\ {\isachardoublequoteopen}cset\ {\isasymRightarrow}\ cset\ {\isasymRightarrow}\ cset\ {\isasymRightarrow}\ cfunc{\isachardoublequoteclose}\ \isakeyword{where}\isanewline
\ \ {\isachardoublequoteopen}distribute{\isacharunderscore}{\kern0pt}right{\isacharunderscore}{\kern0pt}right\ X\ Y\ Z\ {\isacharequal}{\kern0pt}\ \isanewline
\ \ \ \ {\isasymlangle}right{\isacharunderscore}{\kern0pt}cart{\isacharunderscore}{\kern0pt}proj\ X\ Y\ {\isasymcirc}\isactrlsub c\ left{\isacharunderscore}{\kern0pt}cart{\isacharunderscore}{\kern0pt}proj\ {\isacharparenleft}{\kern0pt}X\ {\isasymtimes}\isactrlsub c\ Y{\isacharparenright}{\kern0pt}\ Z{\isacharcomma}{\kern0pt}\ right{\isacharunderscore}{\kern0pt}cart{\isacharunderscore}{\kern0pt}proj\ {\isacharparenleft}{\kern0pt}X\ {\isasymtimes}\isactrlsub c\ Y{\isacharparenright}{\kern0pt}\ Z{\isasymrangle}{\isachardoublequoteclose}\isanewline
\isanewline
\isacommand{lemma}\isamarkupfalse%
\ distribute{\isacharunderscore}{\kern0pt}right{\isacharunderscore}{\kern0pt}right{\isacharunderscore}{\kern0pt}type{\isacharbrackleft}{\kern0pt}type{\isacharunderscore}{\kern0pt}rule{\isacharbrackright}{\kern0pt}{\isacharcolon}{\kern0pt}\isanewline
\ \ {\isachardoublequoteopen}distribute{\isacharunderscore}{\kern0pt}right{\isacharunderscore}{\kern0pt}right\ X\ Y\ Z\ {\isacharcolon}{\kern0pt}\ {\isacharparenleft}{\kern0pt}X\ {\isasymtimes}\isactrlsub c\ Y{\isacharparenright}{\kern0pt}\ {\isasymtimes}\isactrlsub c\ Z\ {\isasymrightarrow}\ Y\ {\isasymtimes}\isactrlsub c\ Z{\isachardoublequoteclose}\isanewline
%
\isadelimproof
\ \ %
\endisadelimproof
%
\isatagproof
\isacommand{unfolding}\isamarkupfalse%
\ distribute{\isacharunderscore}{\kern0pt}right{\isacharunderscore}{\kern0pt}right{\isacharunderscore}{\kern0pt}def\isanewline
\ \ \isacommand{using}\isamarkupfalse%
\ cfunc{\isacharunderscore}{\kern0pt}prod{\isacharunderscore}{\kern0pt}type\ comp{\isacharunderscore}{\kern0pt}type\ left{\isacharunderscore}{\kern0pt}cart{\isacharunderscore}{\kern0pt}proj{\isacharunderscore}{\kern0pt}type\ right{\isacharunderscore}{\kern0pt}cart{\isacharunderscore}{\kern0pt}proj{\isacharunderscore}{\kern0pt}type\ \isacommand{by}\isamarkupfalse%
\ blast%
\endisatagproof
{\isafoldproof}%
%
\isadelimproof
\isanewline
%
\endisadelimproof
\isanewline
\isacommand{lemma}\isamarkupfalse%
\ distribute{\isacharunderscore}{\kern0pt}right{\isacharunderscore}{\kern0pt}right{\isacharunderscore}{\kern0pt}ap{\isacharcolon}{\kern0pt}\ \isanewline
\ \ \isakeyword{assumes}\ {\isachardoublequoteopen}x\ {\isacharcolon}{\kern0pt}\ A\ {\isasymrightarrow}\ X{\isachardoublequoteclose}\ {\isachardoublequoteopen}y\ {\isacharcolon}{\kern0pt}\ A\ {\isasymrightarrow}\ Y{\isachardoublequoteclose}\ {\isachardoublequoteopen}z\ {\isacharcolon}{\kern0pt}\ A\ {\isasymrightarrow}\ Z{\isachardoublequoteclose}\isanewline
\ \ \isakeyword{shows}\ {\isachardoublequoteopen}distribute{\isacharunderscore}{\kern0pt}right{\isacharunderscore}{\kern0pt}right\ X\ Y\ Z\ {\isasymcirc}\isactrlsub c\ {\isasymlangle}{\isasymlangle}x{\isacharcomma}{\kern0pt}\ y{\isasymrangle}{\isacharcomma}{\kern0pt}\ z{\isasymrangle}\ {\isacharequal}{\kern0pt}\ {\isasymlangle}y{\isacharcomma}{\kern0pt}\ z{\isasymrangle}{\isachardoublequoteclose}\isanewline
%
\isadelimproof
\ \ %
\endisadelimproof
%
\isatagproof
\isacommand{unfolding}\isamarkupfalse%
\ distribute{\isacharunderscore}{\kern0pt}right{\isacharunderscore}{\kern0pt}right{\isacharunderscore}{\kern0pt}def\ \ \isanewline
\ \ \isacommand{by}\isamarkupfalse%
\ {\isacharparenleft}{\kern0pt}typecheck{\isacharunderscore}{\kern0pt}cfuncs{\isacharcomma}{\kern0pt}\ smt\ {\isacharparenleft}{\kern0pt}z{\isadigit{3}}{\isacharparenright}{\kern0pt}\ assms\ cfunc{\isacharunderscore}{\kern0pt}prod{\isacharunderscore}{\kern0pt}comp\ comp{\isacharunderscore}{\kern0pt}associative{\isadigit{2}}\ left{\isacharunderscore}{\kern0pt}cart{\isacharunderscore}{\kern0pt}proj{\isacharunderscore}{\kern0pt}cfunc{\isacharunderscore}{\kern0pt}prod\ right{\isacharunderscore}{\kern0pt}cart{\isacharunderscore}{\kern0pt}proj{\isacharunderscore}{\kern0pt}cfunc{\isacharunderscore}{\kern0pt}prod{\isacharparenright}{\kern0pt}%
\endisatagproof
{\isafoldproof}%
%
\isadelimproof
\isanewline
%
\endisadelimproof
\isanewline
\isacommand{definition}\isamarkupfalse%
\ distribute{\isacharunderscore}{\kern0pt}right\ {\isacharcolon}{\kern0pt}{\isacharcolon}{\kern0pt}\ {\isachardoublequoteopen}cset\ {\isasymRightarrow}\ cset\ {\isasymRightarrow}\ cset\ {\isasymRightarrow}\ cfunc{\isachardoublequoteclose}\ \isakeyword{where}\isanewline
\ \ {\isachardoublequoteopen}distribute{\isacharunderscore}{\kern0pt}right\ X\ Y\ Z\ {\isacharequal}{\kern0pt}\ {\isasymlangle}distribute{\isacharunderscore}{\kern0pt}right{\isacharunderscore}{\kern0pt}left\ X\ Y\ Z{\isacharcomma}{\kern0pt}\ distribute{\isacharunderscore}{\kern0pt}right{\isacharunderscore}{\kern0pt}right\ X\ Y\ Z{\isasymrangle}{\isachardoublequoteclose}\isanewline
\isanewline
\isacommand{lemma}\isamarkupfalse%
\ distribute{\isacharunderscore}{\kern0pt}right{\isacharunderscore}{\kern0pt}type{\isacharbrackleft}{\kern0pt}type{\isacharunderscore}{\kern0pt}rule{\isacharbrackright}{\kern0pt}{\isacharcolon}{\kern0pt}\isanewline
\ \ {\isachardoublequoteopen}distribute{\isacharunderscore}{\kern0pt}right\ X\ Y\ Z\ {\isacharcolon}{\kern0pt}\ {\isacharparenleft}{\kern0pt}X\ {\isasymtimes}\isactrlsub c\ Y{\isacharparenright}{\kern0pt}\ {\isasymtimes}\isactrlsub c\ Z\ {\isasymrightarrow}\ {\isacharparenleft}{\kern0pt}X\ {\isasymtimes}\isactrlsub c\ Z{\isacharparenright}{\kern0pt}\ {\isasymtimes}\isactrlsub c\ {\isacharparenleft}{\kern0pt}Y\ {\isasymtimes}\isactrlsub c\ Z{\isacharparenright}{\kern0pt}{\isachardoublequoteclose}\isanewline
%
\isadelimproof
\ \ %
\endisadelimproof
%
\isatagproof
\isacommand{unfolding}\isamarkupfalse%
\ distribute{\isacharunderscore}{\kern0pt}right{\isacharunderscore}{\kern0pt}def\isanewline
\ \ \isacommand{by}\isamarkupfalse%
\ {\isacharparenleft}{\kern0pt}simp\ add{\isacharcolon}{\kern0pt}\ cfunc{\isacharunderscore}{\kern0pt}prod{\isacharunderscore}{\kern0pt}type\ distribute{\isacharunderscore}{\kern0pt}right{\isacharunderscore}{\kern0pt}left{\isacharunderscore}{\kern0pt}type\ distribute{\isacharunderscore}{\kern0pt}right{\isacharunderscore}{\kern0pt}right{\isacharunderscore}{\kern0pt}type{\isacharparenright}{\kern0pt}%
\endisatagproof
{\isafoldproof}%
%
\isadelimproof
\isanewline
%
\endisadelimproof
\isanewline
\isacommand{lemma}\isamarkupfalse%
\ distribute{\isacharunderscore}{\kern0pt}right{\isacharunderscore}{\kern0pt}ap{\isacharcolon}{\kern0pt}\ \isanewline
\ \ \isakeyword{assumes}\ {\isachardoublequoteopen}x\ {\isacharcolon}{\kern0pt}\ A\ {\isasymrightarrow}\ X{\isachardoublequoteclose}\ {\isachardoublequoteopen}y\ {\isacharcolon}{\kern0pt}\ A\ {\isasymrightarrow}\ Y{\isachardoublequoteclose}\ {\isachardoublequoteopen}z\ {\isacharcolon}{\kern0pt}\ A\ {\isasymrightarrow}\ Z{\isachardoublequoteclose}\isanewline
\ \ \isakeyword{shows}\ {\isachardoublequoteopen}distribute{\isacharunderscore}{\kern0pt}right\ X\ Y\ Z\ {\isasymcirc}\isactrlsub c\ {\isasymlangle}{\isasymlangle}x{\isacharcomma}{\kern0pt}\ y{\isasymrangle}{\isacharcomma}{\kern0pt}\ z{\isasymrangle}\ {\isacharequal}{\kern0pt}\ {\isasymlangle}{\isasymlangle}x{\isacharcomma}{\kern0pt}\ z{\isasymrangle}{\isacharcomma}{\kern0pt}\ {\isasymlangle}y{\isacharcomma}{\kern0pt}\ z{\isasymrangle}{\isasymrangle}{\isachardoublequoteclose}\isanewline
%
\isadelimproof
\ \ %
\endisadelimproof
%
\isatagproof
\isacommand{using}\isamarkupfalse%
\ assms\ \isacommand{unfolding}\isamarkupfalse%
\ distribute{\isacharunderscore}{\kern0pt}right{\isacharunderscore}{\kern0pt}def\ \ \isanewline
\ \ \isacommand{by}\isamarkupfalse%
\ {\isacharparenleft}{\kern0pt}typecheck{\isacharunderscore}{\kern0pt}cfuncs{\isacharcomma}{\kern0pt}\ simp\ add{\isacharcolon}{\kern0pt}\ cfunc{\isacharunderscore}{\kern0pt}prod{\isacharunderscore}{\kern0pt}comp\ distribute{\isacharunderscore}{\kern0pt}right{\isacharunderscore}{\kern0pt}left{\isacharunderscore}{\kern0pt}ap\ distribute{\isacharunderscore}{\kern0pt}right{\isacharunderscore}{\kern0pt}right{\isacharunderscore}{\kern0pt}ap{\isacharparenright}{\kern0pt}%
\endisatagproof
{\isafoldproof}%
%
\isadelimproof
\isanewline
%
\endisadelimproof
\isanewline
\isacommand{lemma}\isamarkupfalse%
\ distribute{\isacharunderscore}{\kern0pt}right{\isacharunderscore}{\kern0pt}mono{\isacharcolon}{\kern0pt}\isanewline
\ \ {\isachardoublequoteopen}monomorphism\ {\isacharparenleft}{\kern0pt}distribute{\isacharunderscore}{\kern0pt}right\ X\ Y\ Z{\isacharparenright}{\kern0pt}{\isachardoublequoteclose}\isanewline
%
\isadelimproof
%
\endisadelimproof
%
\isatagproof
\isacommand{proof}\isamarkupfalse%
\ {\isacharparenleft}{\kern0pt}typecheck{\isacharunderscore}{\kern0pt}cfuncs{\isacharcomma}{\kern0pt}\ unfold\ monomorphism{\isacharunderscore}{\kern0pt}def{\isadigit{3}}{\isacharcomma}{\kern0pt}\ clarify{\isacharparenright}{\kern0pt}\isanewline
\ \ \isacommand{fix}\isamarkupfalse%
\ g\ h\ A\isanewline
\ \ \isacommand{assume}\isamarkupfalse%
\ {\isachardoublequoteopen}g\ {\isacharcolon}{\kern0pt}\ A\ {\isasymrightarrow}\ {\isacharparenleft}{\kern0pt}X\ {\isasymtimes}\isactrlsub c\ Y{\isacharparenright}{\kern0pt}\ {\isasymtimes}\isactrlsub c\ Z{\isachardoublequoteclose}\isanewline
\ \ \isacommand{then}\isamarkupfalse%
\ \isacommand{obtain}\isamarkupfalse%
\ g{\isadigit{1}}\ g{\isadigit{2}}\ g{\isadigit{3}}\ \isakeyword{where}\ g{\isacharunderscore}{\kern0pt}expand{\isacharcolon}{\kern0pt}\ {\isachardoublequoteopen}g\ {\isacharequal}{\kern0pt}\ {\isasymlangle}{\isasymlangle}g{\isadigit{1}}{\isacharcomma}{\kern0pt}\ g{\isadigit{2}}{\isasymrangle}{\isacharcomma}{\kern0pt}\ g{\isadigit{3}}{\isasymrangle}{\isachardoublequoteclose}\isanewline
\ \ \ \ \ \ \isakeyword{and}\ g{\isadigit{1}}{\isacharunderscore}{\kern0pt}g{\isadigit{2}}{\isacharunderscore}{\kern0pt}g{\isadigit{3}}{\isacharunderscore}{\kern0pt}types{\isacharcolon}{\kern0pt}\ {\isachardoublequoteopen}g{\isadigit{1}}\ {\isacharcolon}{\kern0pt}\ A\ {\isasymrightarrow}\ X{\isachardoublequoteclose}\ {\isachardoublequoteopen}g{\isadigit{2}}\ {\isacharcolon}{\kern0pt}\ A\ {\isasymrightarrow}\ Y{\isachardoublequoteclose}\ {\isachardoublequoteopen}g{\isadigit{3}}\ {\isacharcolon}{\kern0pt}\ A\ {\isasymrightarrow}\ Z{\isachardoublequoteclose}\isanewline
\ \ \ \ \isacommand{using}\isamarkupfalse%
\ cart{\isacharunderscore}{\kern0pt}prod{\isacharunderscore}{\kern0pt}decomp\ \isacommand{by}\isamarkupfalse%
\ blast\ \isanewline
\ \ \isacommand{assume}\isamarkupfalse%
\ {\isachardoublequoteopen}h\ {\isacharcolon}{\kern0pt}\ A\ {\isasymrightarrow}\ {\isacharparenleft}{\kern0pt}X\ {\isasymtimes}\isactrlsub c\ Y{\isacharparenright}{\kern0pt}\ {\isasymtimes}\isactrlsub c\ Z{\isachardoublequoteclose}\isanewline
\ \ \isacommand{then}\isamarkupfalse%
\ \isacommand{obtain}\isamarkupfalse%
\ h{\isadigit{1}}\ h{\isadigit{2}}\ h{\isadigit{3}}\ \isakeyword{where}\ h{\isacharunderscore}{\kern0pt}expand{\isacharcolon}{\kern0pt}\ {\isachardoublequoteopen}h\ {\isacharequal}{\kern0pt}\ {\isasymlangle}{\isasymlangle}h{\isadigit{1}}{\isacharcomma}{\kern0pt}\ h{\isadigit{2}}{\isasymrangle}{\isacharcomma}{\kern0pt}\ h{\isadigit{3}}{\isasymrangle}{\isachardoublequoteclose}\isanewline
\ \ \ \ \ \ \isakeyword{and}\ h{\isadigit{1}}{\isacharunderscore}{\kern0pt}h{\isadigit{2}}{\isacharunderscore}{\kern0pt}h{\isadigit{3}}{\isacharunderscore}{\kern0pt}types{\isacharcolon}{\kern0pt}\ {\isachardoublequoteopen}h{\isadigit{1}}\ {\isacharcolon}{\kern0pt}\ A\ {\isasymrightarrow}\ X{\isachardoublequoteclose}\ {\isachardoublequoteopen}h{\isadigit{2}}\ {\isacharcolon}{\kern0pt}\ A\ {\isasymrightarrow}\ Y{\isachardoublequoteclose}\ {\isachardoublequoteopen}h{\isadigit{3}}\ {\isacharcolon}{\kern0pt}\ A\ {\isasymrightarrow}\ Z{\isachardoublequoteclose}\isanewline
\ \ \ \ \isacommand{using}\isamarkupfalse%
\ cart{\isacharunderscore}{\kern0pt}prod{\isacharunderscore}{\kern0pt}decomp\ \isacommand{by}\isamarkupfalse%
\ blast\ \isanewline
\isanewline
\ \ \isacommand{assume}\isamarkupfalse%
\ {\isachardoublequoteopen}distribute{\isacharunderscore}{\kern0pt}right\ X\ Y\ Z\ {\isasymcirc}\isactrlsub c\ g\ {\isacharequal}{\kern0pt}\ distribute{\isacharunderscore}{\kern0pt}right\ X\ Y\ Z\ {\isasymcirc}\isactrlsub c\ h{\isachardoublequoteclose}\isanewline
\ \ \isacommand{then}\isamarkupfalse%
\ \isacommand{have}\isamarkupfalse%
\ {\isachardoublequoteopen}distribute{\isacharunderscore}{\kern0pt}right\ X\ Y\ Z\ {\isasymcirc}\isactrlsub c\ {\isasymlangle}{\isasymlangle}g{\isadigit{1}}{\isacharcomma}{\kern0pt}\ g{\isadigit{2}}{\isasymrangle}{\isacharcomma}{\kern0pt}\ g{\isadigit{3}}{\isasymrangle}\ {\isacharequal}{\kern0pt}\ distribute{\isacharunderscore}{\kern0pt}right\ X\ Y\ Z\ {\isasymcirc}\isactrlsub c\ {\isasymlangle}{\isasymlangle}h{\isadigit{1}}{\isacharcomma}{\kern0pt}\ h{\isadigit{2}}{\isasymrangle}{\isacharcomma}{\kern0pt}\ h{\isadigit{3}}{\isasymrangle}{\isachardoublequoteclose}\isanewline
\ \ \ \ \isacommand{using}\isamarkupfalse%
\ g{\isacharunderscore}{\kern0pt}expand\ h{\isacharunderscore}{\kern0pt}expand\ \isacommand{by}\isamarkupfalse%
\ auto\isanewline
\ \ \isacommand{then}\isamarkupfalse%
\ \isacommand{have}\isamarkupfalse%
\ {\isachardoublequoteopen}{\isasymlangle}{\isasymlangle}g{\isadigit{1}}{\isacharcomma}{\kern0pt}\ g{\isadigit{3}}{\isasymrangle}{\isacharcomma}{\kern0pt}\ {\isasymlangle}g{\isadigit{2}}{\isacharcomma}{\kern0pt}\ g{\isadigit{3}}{\isasymrangle}{\isasymrangle}\ {\isacharequal}{\kern0pt}\ {\isasymlangle}{\isasymlangle}h{\isadigit{1}}{\isacharcomma}{\kern0pt}\ h{\isadigit{3}}{\isasymrangle}{\isacharcomma}{\kern0pt}\ {\isasymlangle}h{\isadigit{2}}{\isacharcomma}{\kern0pt}\ h{\isadigit{3}}{\isasymrangle}{\isasymrangle}{\isachardoublequoteclose}\isanewline
\ \ \ \ \isacommand{using}\isamarkupfalse%
\ distribute{\isacharunderscore}{\kern0pt}right{\isacharunderscore}{\kern0pt}ap\ g{\isadigit{1}}{\isacharunderscore}{\kern0pt}g{\isadigit{2}}{\isacharunderscore}{\kern0pt}g{\isadigit{3}}{\isacharunderscore}{\kern0pt}types\ h{\isadigit{1}}{\isacharunderscore}{\kern0pt}h{\isadigit{2}}{\isacharunderscore}{\kern0pt}h{\isadigit{3}}{\isacharunderscore}{\kern0pt}types\ \isacommand{by}\isamarkupfalse%
\ auto\isanewline
\ \ \isacommand{then}\isamarkupfalse%
\ \isacommand{have}\isamarkupfalse%
\ {\isachardoublequoteopen}{\isasymlangle}g{\isadigit{1}}{\isacharcomma}{\kern0pt}\ g{\isadigit{3}}{\isasymrangle}\ {\isacharequal}{\kern0pt}\ {\isasymlangle}h{\isadigit{1}}{\isacharcomma}{\kern0pt}\ h{\isadigit{3}}{\isasymrangle}\ {\isasymand}\ {\isasymlangle}g{\isadigit{2}}{\isacharcomma}{\kern0pt}\ g{\isadigit{3}}{\isasymrangle}\ {\isacharequal}{\kern0pt}\ {\isasymlangle}h{\isadigit{2}}{\isacharcomma}{\kern0pt}\ h{\isadigit{3}}{\isasymrangle}{\isachardoublequoteclose}\isanewline
\ \ \ \ \isacommand{using}\isamarkupfalse%
\ g{\isadigit{1}}{\isacharunderscore}{\kern0pt}g{\isadigit{2}}{\isacharunderscore}{\kern0pt}g{\isadigit{3}}{\isacharunderscore}{\kern0pt}types\ h{\isadigit{1}}{\isacharunderscore}{\kern0pt}h{\isadigit{2}}{\isacharunderscore}{\kern0pt}h{\isadigit{3}}{\isacharunderscore}{\kern0pt}types\ cart{\isacharunderscore}{\kern0pt}prod{\isacharunderscore}{\kern0pt}eq{\isadigit{2}}\ \isacommand{by}\isamarkupfalse%
\ {\isacharparenleft}{\kern0pt}typecheck{\isacharunderscore}{\kern0pt}cfuncs{\isacharcomma}{\kern0pt}\ auto{\isacharparenright}{\kern0pt}\isanewline
\ \ \isacommand{then}\isamarkupfalse%
\ \isacommand{have}\isamarkupfalse%
\ {\isachardoublequoteopen}g{\isadigit{1}}\ {\isacharequal}{\kern0pt}\ h{\isadigit{1}}\ {\isasymand}\ g{\isadigit{2}}\ {\isacharequal}{\kern0pt}\ h{\isadigit{2}}\ {\isasymand}\ g{\isadigit{3}}\ {\isacharequal}{\kern0pt}\ h{\isadigit{3}}{\isachardoublequoteclose}\isanewline
\ \ \ \ \isacommand{using}\isamarkupfalse%
\ g{\isadigit{1}}{\isacharunderscore}{\kern0pt}g{\isadigit{2}}{\isacharunderscore}{\kern0pt}g{\isadigit{3}}{\isacharunderscore}{\kern0pt}types\ h{\isadigit{1}}{\isacharunderscore}{\kern0pt}h{\isadigit{2}}{\isacharunderscore}{\kern0pt}h{\isadigit{3}}{\isacharunderscore}{\kern0pt}types\ cart{\isacharunderscore}{\kern0pt}prod{\isacharunderscore}{\kern0pt}eq{\isadigit{2}}\ \isacommand{by}\isamarkupfalse%
\ auto\isanewline
\ \ \isacommand{then}\isamarkupfalse%
\ \isacommand{have}\isamarkupfalse%
\ {\isachardoublequoteopen}{\isasymlangle}{\isasymlangle}g{\isadigit{1}}{\isacharcomma}{\kern0pt}\ g{\isadigit{2}}{\isasymrangle}{\isacharcomma}{\kern0pt}\ g{\isadigit{3}}{\isasymrangle}\ {\isacharequal}{\kern0pt}\ {\isasymlangle}{\isasymlangle}h{\isadigit{1}}{\isacharcomma}{\kern0pt}\ h{\isadigit{2}}{\isasymrangle}{\isacharcomma}{\kern0pt}\ h{\isadigit{3}}{\isasymrangle}{\isachardoublequoteclose}\isanewline
\ \ \ \ \isacommand{by}\isamarkupfalse%
\ simp\isanewline
\ \ \isacommand{then}\isamarkupfalse%
\ \isacommand{show}\isamarkupfalse%
\ {\isachardoublequoteopen}g\ {\isacharequal}{\kern0pt}\ h{\isachardoublequoteclose}\isanewline
\ \ \ \ \isacommand{by}\isamarkupfalse%
\ {\isacharparenleft}{\kern0pt}simp\ add{\isacharcolon}{\kern0pt}\ g{\isacharunderscore}{\kern0pt}expand\ h{\isacharunderscore}{\kern0pt}expand{\isacharparenright}{\kern0pt}\isanewline
\isacommand{qed}\isamarkupfalse%
%
\endisatagproof
{\isafoldproof}%
%
\isadelimproof
%
\endisadelimproof
%
\isadelimdocument
%
\endisadelimdocument
%
\isatagdocument
%
\isamarkupsubsubsection{Distributing over a Cartesian Product from the Left%
}
\isamarkuptrue%
%
\endisatagdocument
{\isafolddocument}%
%
\isadelimdocument
%
\endisadelimdocument
\isacommand{definition}\isamarkupfalse%
\ distribute{\isacharunderscore}{\kern0pt}left{\isacharunderscore}{\kern0pt}left\ {\isacharcolon}{\kern0pt}{\isacharcolon}{\kern0pt}\ {\isachardoublequoteopen}cset\ {\isasymRightarrow}\ cset\ {\isasymRightarrow}\ cset\ {\isasymRightarrow}\ cfunc{\isachardoublequoteclose}\ \isakeyword{where}\isanewline
\ \ {\isachardoublequoteopen}distribute{\isacharunderscore}{\kern0pt}left{\isacharunderscore}{\kern0pt}left\ X\ Y\ Z\ {\isacharequal}{\kern0pt}\ \isanewline
\ \ \ \ {\isasymlangle}left{\isacharunderscore}{\kern0pt}cart{\isacharunderscore}{\kern0pt}proj\ X\ {\isacharparenleft}{\kern0pt}Y\ {\isasymtimes}\isactrlsub c\ Z{\isacharparenright}{\kern0pt}{\isacharcomma}{\kern0pt}\ left{\isacharunderscore}{\kern0pt}cart{\isacharunderscore}{\kern0pt}proj\ Y\ Z\ {\isasymcirc}\isactrlsub c\ right{\isacharunderscore}{\kern0pt}cart{\isacharunderscore}{\kern0pt}proj\ X\ {\isacharparenleft}{\kern0pt}Y\ {\isasymtimes}\isactrlsub c\ Z{\isacharparenright}{\kern0pt}{\isasymrangle}{\isachardoublequoteclose}\isanewline
\isanewline
\isacommand{lemma}\isamarkupfalse%
\ distribute{\isacharunderscore}{\kern0pt}left{\isacharunderscore}{\kern0pt}left{\isacharunderscore}{\kern0pt}type{\isacharbrackleft}{\kern0pt}type{\isacharunderscore}{\kern0pt}rule{\isacharbrackright}{\kern0pt}{\isacharcolon}{\kern0pt}\isanewline
\ \ {\isachardoublequoteopen}distribute{\isacharunderscore}{\kern0pt}left{\isacharunderscore}{\kern0pt}left\ X\ Y\ Z\ {\isacharcolon}{\kern0pt}\ X\ {\isasymtimes}\isactrlsub c\ {\isacharparenleft}{\kern0pt}Y\ {\isasymtimes}\isactrlsub c\ Z{\isacharparenright}{\kern0pt}\ {\isasymrightarrow}\ X\ {\isasymtimes}\isactrlsub c\ Y{\isachardoublequoteclose}\isanewline
%
\isadelimproof
\ \ %
\endisadelimproof
%
\isatagproof
\isacommand{unfolding}\isamarkupfalse%
\ distribute{\isacharunderscore}{\kern0pt}left{\isacharunderscore}{\kern0pt}left{\isacharunderscore}{\kern0pt}def\isanewline
\ \ \isacommand{using}\isamarkupfalse%
\ cfunc{\isacharunderscore}{\kern0pt}prod{\isacharunderscore}{\kern0pt}type\ comp{\isacharunderscore}{\kern0pt}type\ left{\isacharunderscore}{\kern0pt}cart{\isacharunderscore}{\kern0pt}proj{\isacharunderscore}{\kern0pt}type\ right{\isacharunderscore}{\kern0pt}cart{\isacharunderscore}{\kern0pt}proj{\isacharunderscore}{\kern0pt}type\ \isacommand{by}\isamarkupfalse%
\ blast%
\endisatagproof
{\isafoldproof}%
%
\isadelimproof
\isanewline
%
\endisadelimproof
\isanewline
\isacommand{lemma}\isamarkupfalse%
\ distribute{\isacharunderscore}{\kern0pt}left{\isacharunderscore}{\kern0pt}left{\isacharunderscore}{\kern0pt}ap{\isacharcolon}{\kern0pt}\ \isanewline
\ \ \isakeyword{assumes}\ {\isachardoublequoteopen}x\ {\isacharcolon}{\kern0pt}\ A\ {\isasymrightarrow}\ X{\isachardoublequoteclose}\ {\isachardoublequoteopen}y\ {\isacharcolon}{\kern0pt}\ A\ {\isasymrightarrow}\ Y{\isachardoublequoteclose}\ {\isachardoublequoteopen}z\ {\isacharcolon}{\kern0pt}\ A\ {\isasymrightarrow}\ Z{\isachardoublequoteclose}\isanewline
\ \ \isakeyword{shows}\ {\isachardoublequoteopen}distribute{\isacharunderscore}{\kern0pt}left{\isacharunderscore}{\kern0pt}left\ X\ Y\ Z\ {\isasymcirc}\isactrlsub c\ {\isasymlangle}x{\isacharcomma}{\kern0pt}\ {\isasymlangle}y{\isacharcomma}{\kern0pt}\ z{\isasymrangle}{\isasymrangle}\ {\isacharequal}{\kern0pt}\ {\isasymlangle}x{\isacharcomma}{\kern0pt}\ y{\isasymrangle}{\isachardoublequoteclose}\isanewline
%
\isadelimproof
\ \ %
\endisadelimproof
%
\isatagproof
\isacommand{using}\isamarkupfalse%
\ assms\ distribute{\isacharunderscore}{\kern0pt}left{\isacharunderscore}{\kern0pt}left{\isacharunderscore}{\kern0pt}def\ \ \isanewline
\ \ \isacommand{by}\isamarkupfalse%
\ {\isacharparenleft}{\kern0pt}typecheck{\isacharunderscore}{\kern0pt}cfuncs{\isacharcomma}{\kern0pt}\ smt\ {\isacharparenleft}{\kern0pt}z{\isadigit{3}}{\isacharparenright}{\kern0pt}\ associate{\isacharunderscore}{\kern0pt}left{\isacharunderscore}{\kern0pt}ap\ associate{\isacharunderscore}{\kern0pt}left{\isacharunderscore}{\kern0pt}def\ cart{\isacharunderscore}{\kern0pt}prod{\isacharunderscore}{\kern0pt}decomp\ cart{\isacharunderscore}{\kern0pt}prod{\isacharunderscore}{\kern0pt}eq{\isadigit{2}}\ cfunc{\isacharunderscore}{\kern0pt}prod{\isacharunderscore}{\kern0pt}comp\ comp{\isacharunderscore}{\kern0pt}associative{\isadigit{2}}\ distribute{\isacharunderscore}{\kern0pt}left{\isacharunderscore}{\kern0pt}left{\isacharunderscore}{\kern0pt}def\ right{\isacharunderscore}{\kern0pt}cart{\isacharunderscore}{\kern0pt}proj{\isacharunderscore}{\kern0pt}cfunc{\isacharunderscore}{\kern0pt}prod\ right{\isacharunderscore}{\kern0pt}cart{\isacharunderscore}{\kern0pt}proj{\isacharunderscore}{\kern0pt}type{\isacharparenright}{\kern0pt}%
\endisatagproof
{\isafoldproof}%
%
\isadelimproof
\isanewline
%
\endisadelimproof
\isanewline
\isacommand{definition}\isamarkupfalse%
\ distribute{\isacharunderscore}{\kern0pt}left{\isacharunderscore}{\kern0pt}right\ {\isacharcolon}{\kern0pt}{\isacharcolon}{\kern0pt}\ {\isachardoublequoteopen}cset\ {\isasymRightarrow}\ cset\ {\isasymRightarrow}\ cset\ {\isasymRightarrow}\ cfunc{\isachardoublequoteclose}\ \isakeyword{where}\isanewline
\ \ {\isachardoublequoteopen}distribute{\isacharunderscore}{\kern0pt}left{\isacharunderscore}{\kern0pt}right\ X\ Y\ Z\ {\isacharequal}{\kern0pt}\ \isanewline
\ \ \ \ {\isasymlangle}left{\isacharunderscore}{\kern0pt}cart{\isacharunderscore}{\kern0pt}proj\ X\ {\isacharparenleft}{\kern0pt}Y\ {\isasymtimes}\isactrlsub c\ Z{\isacharparenright}{\kern0pt}{\isacharcomma}{\kern0pt}\ right{\isacharunderscore}{\kern0pt}cart{\isacharunderscore}{\kern0pt}proj\ Y\ Z\ {\isasymcirc}\isactrlsub c\ right{\isacharunderscore}{\kern0pt}cart{\isacharunderscore}{\kern0pt}proj\ X\ {\isacharparenleft}{\kern0pt}Y\ {\isasymtimes}\isactrlsub c\ Z{\isacharparenright}{\kern0pt}{\isasymrangle}{\isachardoublequoteclose}\isanewline
\isanewline
\isacommand{lemma}\isamarkupfalse%
\ distribute{\isacharunderscore}{\kern0pt}left{\isacharunderscore}{\kern0pt}right{\isacharunderscore}{\kern0pt}type{\isacharbrackleft}{\kern0pt}type{\isacharunderscore}{\kern0pt}rule{\isacharbrackright}{\kern0pt}{\isacharcolon}{\kern0pt}\isanewline
\ \ {\isachardoublequoteopen}distribute{\isacharunderscore}{\kern0pt}left{\isacharunderscore}{\kern0pt}right\ X\ Y\ Z\ {\isacharcolon}{\kern0pt}\ X\ {\isasymtimes}\isactrlsub c\ {\isacharparenleft}{\kern0pt}Y\ {\isasymtimes}\isactrlsub c\ Z{\isacharparenright}{\kern0pt}\ {\isasymrightarrow}\ X\ {\isasymtimes}\isactrlsub c\ Z{\isachardoublequoteclose}\isanewline
%
\isadelimproof
\ \ %
\endisadelimproof
%
\isatagproof
\isacommand{unfolding}\isamarkupfalse%
\ distribute{\isacharunderscore}{\kern0pt}left{\isacharunderscore}{\kern0pt}right{\isacharunderscore}{\kern0pt}def\isanewline
\ \ \isacommand{using}\isamarkupfalse%
\ cfunc{\isacharunderscore}{\kern0pt}prod{\isacharunderscore}{\kern0pt}type\ comp{\isacharunderscore}{\kern0pt}type\ left{\isacharunderscore}{\kern0pt}cart{\isacharunderscore}{\kern0pt}proj{\isacharunderscore}{\kern0pt}type\ right{\isacharunderscore}{\kern0pt}cart{\isacharunderscore}{\kern0pt}proj{\isacharunderscore}{\kern0pt}type\ \isacommand{by}\isamarkupfalse%
\ blast%
\endisatagproof
{\isafoldproof}%
%
\isadelimproof
\isanewline
%
\endisadelimproof
\isanewline
\isacommand{lemma}\isamarkupfalse%
\ distribute{\isacharunderscore}{\kern0pt}left{\isacharunderscore}{\kern0pt}right{\isacharunderscore}{\kern0pt}ap{\isacharcolon}{\kern0pt}\ \isanewline
\ \ \isakeyword{assumes}\ {\isachardoublequoteopen}x\ {\isacharcolon}{\kern0pt}\ A\ {\isasymrightarrow}\ X{\isachardoublequoteclose}\ {\isachardoublequoteopen}y\ {\isacharcolon}{\kern0pt}\ A\ {\isasymrightarrow}\ Y{\isachardoublequoteclose}\ {\isachardoublequoteopen}z\ {\isacharcolon}{\kern0pt}\ A\ {\isasymrightarrow}\ Z{\isachardoublequoteclose}\isanewline
\ \ \isakeyword{shows}\ {\isachardoublequoteopen}distribute{\isacharunderscore}{\kern0pt}left{\isacharunderscore}{\kern0pt}right\ X\ Y\ Z\ {\isasymcirc}\isactrlsub c\ {\isasymlangle}x{\isacharcomma}{\kern0pt}\ {\isasymlangle}y{\isacharcomma}{\kern0pt}\ z{\isasymrangle}{\isasymrangle}\ {\isacharequal}{\kern0pt}\ {\isasymlangle}x{\isacharcomma}{\kern0pt}\ z{\isasymrangle}{\isachardoublequoteclose}\isanewline
%
\isadelimproof
\ \ %
\endisadelimproof
%
\isatagproof
\isacommand{using}\isamarkupfalse%
\ assms\ \isacommand{unfolding}\isamarkupfalse%
\ distribute{\isacharunderscore}{\kern0pt}left{\isacharunderscore}{\kern0pt}right{\isacharunderscore}{\kern0pt}def\ \ \isanewline
\ \ \isacommand{by}\isamarkupfalse%
\ {\isacharparenleft}{\kern0pt}typecheck{\isacharunderscore}{\kern0pt}cfuncs{\isacharcomma}{\kern0pt}\ smt\ {\isacharparenleft}{\kern0pt}z{\isadigit{3}}{\isacharparenright}{\kern0pt}\ cfunc{\isacharunderscore}{\kern0pt}prod{\isacharunderscore}{\kern0pt}comp\ comp{\isacharunderscore}{\kern0pt}associative{\isadigit{2}}\ left{\isacharunderscore}{\kern0pt}cart{\isacharunderscore}{\kern0pt}proj{\isacharunderscore}{\kern0pt}cfunc{\isacharunderscore}{\kern0pt}prod\ right{\isacharunderscore}{\kern0pt}cart{\isacharunderscore}{\kern0pt}proj{\isacharunderscore}{\kern0pt}cfunc{\isacharunderscore}{\kern0pt}prod{\isacharparenright}{\kern0pt}%
\endisatagproof
{\isafoldproof}%
%
\isadelimproof
\isanewline
%
\endisadelimproof
\isanewline
\isacommand{definition}\isamarkupfalse%
\ distribute{\isacharunderscore}{\kern0pt}left\ {\isacharcolon}{\kern0pt}{\isacharcolon}{\kern0pt}\ {\isachardoublequoteopen}cset\ {\isasymRightarrow}\ cset\ {\isasymRightarrow}\ cset\ {\isasymRightarrow}\ cfunc{\isachardoublequoteclose}\ \isakeyword{where}\isanewline
\ \ {\isachardoublequoteopen}distribute{\isacharunderscore}{\kern0pt}left\ X\ Y\ Z\ {\isacharequal}{\kern0pt}\ {\isasymlangle}distribute{\isacharunderscore}{\kern0pt}left{\isacharunderscore}{\kern0pt}left\ X\ Y\ Z{\isacharcomma}{\kern0pt}\ distribute{\isacharunderscore}{\kern0pt}left{\isacharunderscore}{\kern0pt}right\ X\ Y\ Z{\isasymrangle}{\isachardoublequoteclose}\isanewline
\isanewline
\isacommand{lemma}\isamarkupfalse%
\ distribute{\isacharunderscore}{\kern0pt}left{\isacharunderscore}{\kern0pt}type{\isacharbrackleft}{\kern0pt}type{\isacharunderscore}{\kern0pt}rule{\isacharbrackright}{\kern0pt}{\isacharcolon}{\kern0pt}\isanewline
\ \ {\isachardoublequoteopen}distribute{\isacharunderscore}{\kern0pt}left\ X\ Y\ Z\ {\isacharcolon}{\kern0pt}\ X\ {\isasymtimes}\isactrlsub c\ {\isacharparenleft}{\kern0pt}Y\ {\isasymtimes}\isactrlsub c\ Z{\isacharparenright}{\kern0pt}\ {\isasymrightarrow}\ {\isacharparenleft}{\kern0pt}X\ {\isasymtimes}\isactrlsub c\ Y{\isacharparenright}{\kern0pt}\ {\isasymtimes}\isactrlsub c\ {\isacharparenleft}{\kern0pt}X\ {\isasymtimes}\isactrlsub c\ Z{\isacharparenright}{\kern0pt}{\isachardoublequoteclose}\isanewline
%
\isadelimproof
\ \ %
\endisadelimproof
%
\isatagproof
\isacommand{unfolding}\isamarkupfalse%
\ distribute{\isacharunderscore}{\kern0pt}left{\isacharunderscore}{\kern0pt}def\isanewline
\ \ \isacommand{by}\isamarkupfalse%
\ {\isacharparenleft}{\kern0pt}simp\ add{\isacharcolon}{\kern0pt}\ cfunc{\isacharunderscore}{\kern0pt}prod{\isacharunderscore}{\kern0pt}type\ distribute{\isacharunderscore}{\kern0pt}left{\isacharunderscore}{\kern0pt}left{\isacharunderscore}{\kern0pt}type\ distribute{\isacharunderscore}{\kern0pt}left{\isacharunderscore}{\kern0pt}right{\isacharunderscore}{\kern0pt}type{\isacharparenright}{\kern0pt}%
\endisatagproof
{\isafoldproof}%
%
\isadelimproof
\isanewline
%
\endisadelimproof
\isanewline
\isacommand{lemma}\isamarkupfalse%
\ distribute{\isacharunderscore}{\kern0pt}left{\isacharunderscore}{\kern0pt}ap{\isacharcolon}{\kern0pt}\ \isanewline
\ \ \isakeyword{assumes}\ {\isachardoublequoteopen}x\ {\isacharcolon}{\kern0pt}\ A\ {\isasymrightarrow}\ X{\isachardoublequoteclose}\ {\isachardoublequoteopen}y\ {\isacharcolon}{\kern0pt}\ A\ {\isasymrightarrow}\ Y{\isachardoublequoteclose}\ {\isachardoublequoteopen}z\ {\isacharcolon}{\kern0pt}\ A\ {\isasymrightarrow}\ Z{\isachardoublequoteclose}\isanewline
\ \ \isakeyword{shows}\ {\isachardoublequoteopen}distribute{\isacharunderscore}{\kern0pt}left\ X\ Y\ Z\ {\isasymcirc}\isactrlsub c\ {\isasymlangle}x{\isacharcomma}{\kern0pt}\ {\isasymlangle}y{\isacharcomma}{\kern0pt}\ z{\isasymrangle}{\isasymrangle}\ {\isacharequal}{\kern0pt}\ {\isasymlangle}{\isasymlangle}x{\isacharcomma}{\kern0pt}\ y{\isasymrangle}{\isacharcomma}{\kern0pt}\ {\isasymlangle}x{\isacharcomma}{\kern0pt}\ z{\isasymrangle}{\isasymrangle}{\isachardoublequoteclose}\isanewline
%
\isadelimproof
\ \ %
\endisadelimproof
%
\isatagproof
\isacommand{using}\isamarkupfalse%
\ assms\ \isacommand{unfolding}\isamarkupfalse%
\ distribute{\isacharunderscore}{\kern0pt}left{\isacharunderscore}{\kern0pt}def\ \isanewline
\ \ \isacommand{by}\isamarkupfalse%
\ {\isacharparenleft}{\kern0pt}typecheck{\isacharunderscore}{\kern0pt}cfuncs{\isacharcomma}{\kern0pt}\ simp\ add{\isacharcolon}{\kern0pt}\ cfunc{\isacharunderscore}{\kern0pt}prod{\isacharunderscore}{\kern0pt}comp\ distribute{\isacharunderscore}{\kern0pt}left{\isacharunderscore}{\kern0pt}left{\isacharunderscore}{\kern0pt}ap\ distribute{\isacharunderscore}{\kern0pt}left{\isacharunderscore}{\kern0pt}right{\isacharunderscore}{\kern0pt}ap{\isacharparenright}{\kern0pt}%
\endisatagproof
{\isafoldproof}%
%
\isadelimproof
\isanewline
%
\endisadelimproof
\isanewline
\isacommand{lemma}\isamarkupfalse%
\ distribute{\isacharunderscore}{\kern0pt}left{\isacharunderscore}{\kern0pt}mono{\isacharcolon}{\kern0pt}\isanewline
\ \ {\isachardoublequoteopen}monomorphism\ {\isacharparenleft}{\kern0pt}distribute{\isacharunderscore}{\kern0pt}left\ X\ Y\ Z{\isacharparenright}{\kern0pt}{\isachardoublequoteclose}\isanewline
%
\isadelimproof
%
\endisadelimproof
%
\isatagproof
\isacommand{proof}\isamarkupfalse%
\ {\isacharparenleft}{\kern0pt}typecheck{\isacharunderscore}{\kern0pt}cfuncs{\isacharcomma}{\kern0pt}\ unfold\ monomorphism{\isacharunderscore}{\kern0pt}def{\isadigit{3}}{\isacharcomma}{\kern0pt}\ clarify{\isacharparenright}{\kern0pt}\isanewline
\ \ \isacommand{fix}\isamarkupfalse%
\ g\ h\ A\isanewline
\ \ \isacommand{assume}\isamarkupfalse%
\ g{\isacharunderscore}{\kern0pt}type{\isacharcolon}{\kern0pt}\ {\isachardoublequoteopen}g\ {\isacharcolon}{\kern0pt}\ A\ {\isasymrightarrow}\ X\ {\isasymtimes}\isactrlsub c\ {\isacharparenleft}{\kern0pt}Y\ {\isasymtimes}\isactrlsub c\ Z{\isacharparenright}{\kern0pt}{\isachardoublequoteclose}\isanewline
\ \ \isacommand{then}\isamarkupfalse%
\ \isacommand{obtain}\isamarkupfalse%
\ g{\isadigit{1}}\ g{\isadigit{2}}\ g{\isadigit{3}}\ \isakeyword{where}\ g{\isacharunderscore}{\kern0pt}expand{\isacharcolon}{\kern0pt}\ {\isachardoublequoteopen}g\ {\isacharequal}{\kern0pt}\ {\isasymlangle}g{\isadigit{1}}{\isacharcomma}{\kern0pt}\ {\isasymlangle}g{\isadigit{2}}{\isacharcomma}{\kern0pt}\ g{\isadigit{3}}{\isasymrangle}{\isasymrangle}{\isachardoublequoteclose}\isanewline
\ \ \ \ \ \ \isakeyword{and}\ g{\isadigit{1}}{\isacharunderscore}{\kern0pt}g{\isadigit{2}}{\isacharunderscore}{\kern0pt}g{\isadigit{3}}{\isacharunderscore}{\kern0pt}types{\isacharcolon}{\kern0pt}\ {\isachardoublequoteopen}g{\isadigit{1}}\ {\isacharcolon}{\kern0pt}\ A\ {\isasymrightarrow}\ X{\isachardoublequoteclose}\ {\isachardoublequoteopen}g{\isadigit{2}}\ {\isacharcolon}{\kern0pt}\ A\ {\isasymrightarrow}\ Y{\isachardoublequoteclose}\ {\isachardoublequoteopen}g{\isadigit{3}}\ {\isacharcolon}{\kern0pt}\ A\ {\isasymrightarrow}\ Z{\isachardoublequoteclose}\isanewline
\ \ \ \ \isacommand{using}\isamarkupfalse%
\ cart{\isacharunderscore}{\kern0pt}prod{\isacharunderscore}{\kern0pt}decomp\ \isacommand{by}\isamarkupfalse%
\ blast\ \isanewline
\ \ \isacommand{assume}\isamarkupfalse%
\ h{\isacharunderscore}{\kern0pt}type{\isacharcolon}{\kern0pt}\ {\isachardoublequoteopen}h\ {\isacharcolon}{\kern0pt}\ A\ {\isasymrightarrow}\ X\ {\isasymtimes}\isactrlsub c\ {\isacharparenleft}{\kern0pt}Y\ {\isasymtimes}\isactrlsub c\ Z{\isacharparenright}{\kern0pt}{\isachardoublequoteclose}\isanewline
\ \ \isacommand{then}\isamarkupfalse%
\ \isacommand{obtain}\isamarkupfalse%
\ h{\isadigit{1}}\ h{\isadigit{2}}\ h{\isadigit{3}}\ \isakeyword{where}\ h{\isacharunderscore}{\kern0pt}expand{\isacharcolon}{\kern0pt}\ {\isachardoublequoteopen}h\ {\isacharequal}{\kern0pt}\ {\isasymlangle}h{\isadigit{1}}{\isacharcomma}{\kern0pt}\ {\isasymlangle}h{\isadigit{2}}{\isacharcomma}{\kern0pt}\ h{\isadigit{3}}{\isasymrangle}{\isasymrangle}{\isachardoublequoteclose}\isanewline
\ \ \ \ \ \ \isakeyword{and}\ h{\isadigit{1}}{\isacharunderscore}{\kern0pt}h{\isadigit{2}}{\isacharunderscore}{\kern0pt}h{\isadigit{3}}{\isacharunderscore}{\kern0pt}types{\isacharcolon}{\kern0pt}\ {\isachardoublequoteopen}h{\isadigit{1}}\ {\isacharcolon}{\kern0pt}\ A\ {\isasymrightarrow}\ X{\isachardoublequoteclose}\ {\isachardoublequoteopen}h{\isadigit{2}}\ {\isacharcolon}{\kern0pt}\ A\ {\isasymrightarrow}\ Y{\isachardoublequoteclose}\ {\isachardoublequoteopen}h{\isadigit{3}}\ {\isacharcolon}{\kern0pt}\ A\ {\isasymrightarrow}\ Z{\isachardoublequoteclose}\isanewline
\ \ \ \ \isacommand{using}\isamarkupfalse%
\ cart{\isacharunderscore}{\kern0pt}prod{\isacharunderscore}{\kern0pt}decomp\ \isacommand{by}\isamarkupfalse%
\ blast\ \isanewline
\isanewline
\ \ \isacommand{assume}\isamarkupfalse%
\ {\isachardoublequoteopen}distribute{\isacharunderscore}{\kern0pt}left\ X\ Y\ Z\ {\isasymcirc}\isactrlsub c\ g\ {\isacharequal}{\kern0pt}\ distribute{\isacharunderscore}{\kern0pt}left\ X\ Y\ Z\ {\isasymcirc}\isactrlsub c\ h{\isachardoublequoteclose}\isanewline
\ \ \isacommand{then}\isamarkupfalse%
\ \isacommand{have}\isamarkupfalse%
\ {\isachardoublequoteopen}distribute{\isacharunderscore}{\kern0pt}left\ X\ Y\ Z\ {\isasymcirc}\isactrlsub c\ {\isasymlangle}g{\isadigit{1}}{\isacharcomma}{\kern0pt}\ {\isasymlangle}g{\isadigit{2}}{\isacharcomma}{\kern0pt}\ g{\isadigit{3}}{\isasymrangle}{\isasymrangle}\ {\isacharequal}{\kern0pt}\ distribute{\isacharunderscore}{\kern0pt}left\ X\ Y\ Z\ {\isasymcirc}\isactrlsub c\ {\isasymlangle}h{\isadigit{1}}{\isacharcomma}{\kern0pt}\ {\isasymlangle}h{\isadigit{2}}{\isacharcomma}{\kern0pt}\ h{\isadigit{3}}{\isasymrangle}{\isasymrangle}{\isachardoublequoteclose}\isanewline
\ \ \ \ \isacommand{using}\isamarkupfalse%
\ g{\isacharunderscore}{\kern0pt}expand\ h{\isacharunderscore}{\kern0pt}expand\ \isacommand{by}\isamarkupfalse%
\ auto\isanewline
\ \ \isacommand{then}\isamarkupfalse%
\ \isacommand{have}\isamarkupfalse%
\ {\isachardoublequoteopen}{\isasymlangle}{\isasymlangle}g{\isadigit{1}}{\isacharcomma}{\kern0pt}\ g{\isadigit{2}}{\isasymrangle}{\isacharcomma}{\kern0pt}\ {\isasymlangle}g{\isadigit{1}}{\isacharcomma}{\kern0pt}\ g{\isadigit{3}}{\isasymrangle}{\isasymrangle}\ {\isacharequal}{\kern0pt}\ {\isasymlangle}{\isasymlangle}h{\isadigit{1}}{\isacharcomma}{\kern0pt}\ h{\isadigit{2}}{\isasymrangle}{\isacharcomma}{\kern0pt}\ {\isasymlangle}h{\isadigit{1}}{\isacharcomma}{\kern0pt}\ h{\isadigit{3}}{\isasymrangle}{\isasymrangle}{\isachardoublequoteclose}\isanewline
\ \ \ \ \isacommand{using}\isamarkupfalse%
\ distribute{\isacharunderscore}{\kern0pt}left{\isacharunderscore}{\kern0pt}ap\ g{\isadigit{1}}{\isacharunderscore}{\kern0pt}g{\isadigit{2}}{\isacharunderscore}{\kern0pt}g{\isadigit{3}}{\isacharunderscore}{\kern0pt}types\ h{\isadigit{1}}{\isacharunderscore}{\kern0pt}h{\isadigit{2}}{\isacharunderscore}{\kern0pt}h{\isadigit{3}}{\isacharunderscore}{\kern0pt}types\ \isacommand{by}\isamarkupfalse%
\ auto\isanewline
\ \ \isacommand{then}\isamarkupfalse%
\ \isacommand{have}\isamarkupfalse%
\ {\isachardoublequoteopen}{\isasymlangle}g{\isadigit{1}}{\isacharcomma}{\kern0pt}\ g{\isadigit{2}}{\isasymrangle}\ {\isacharequal}{\kern0pt}\ {\isasymlangle}h{\isadigit{1}}{\isacharcomma}{\kern0pt}\ h{\isadigit{2}}{\isasymrangle}\ {\isasymand}\ {\isasymlangle}g{\isadigit{1}}{\isacharcomma}{\kern0pt}\ g{\isadigit{3}}{\isasymrangle}\ {\isacharequal}{\kern0pt}\ {\isasymlangle}h{\isadigit{1}}{\isacharcomma}{\kern0pt}\ h{\isadigit{3}}{\isasymrangle}{\isachardoublequoteclose}\isanewline
\ \ \ \ \isacommand{using}\isamarkupfalse%
\ g{\isadigit{1}}{\isacharunderscore}{\kern0pt}g{\isadigit{2}}{\isacharunderscore}{\kern0pt}g{\isadigit{3}}{\isacharunderscore}{\kern0pt}types\ h{\isadigit{1}}{\isacharunderscore}{\kern0pt}h{\isadigit{2}}{\isacharunderscore}{\kern0pt}h{\isadigit{3}}{\isacharunderscore}{\kern0pt}types\ cart{\isacharunderscore}{\kern0pt}prod{\isacharunderscore}{\kern0pt}eq{\isadigit{2}}\ \isacommand{by}\isamarkupfalse%
\ {\isacharparenleft}{\kern0pt}typecheck{\isacharunderscore}{\kern0pt}cfuncs{\isacharcomma}{\kern0pt}\ auto{\isacharparenright}{\kern0pt}\isanewline
\ \ \isacommand{then}\isamarkupfalse%
\ \isacommand{have}\isamarkupfalse%
\ {\isachardoublequoteopen}g{\isadigit{1}}\ {\isacharequal}{\kern0pt}\ h{\isadigit{1}}\ {\isasymand}\ g{\isadigit{2}}\ {\isacharequal}{\kern0pt}\ h{\isadigit{2}}\ {\isasymand}\ g{\isadigit{3}}\ {\isacharequal}{\kern0pt}\ h{\isadigit{3}}{\isachardoublequoteclose}\isanewline
\ \ \ \ \isacommand{using}\isamarkupfalse%
\ g{\isadigit{1}}{\isacharunderscore}{\kern0pt}g{\isadigit{2}}{\isacharunderscore}{\kern0pt}g{\isadigit{3}}{\isacharunderscore}{\kern0pt}types\ h{\isadigit{1}}{\isacharunderscore}{\kern0pt}h{\isadigit{2}}{\isacharunderscore}{\kern0pt}h{\isadigit{3}}{\isacharunderscore}{\kern0pt}types\ cart{\isacharunderscore}{\kern0pt}prod{\isacharunderscore}{\kern0pt}eq{\isadigit{2}}\ \isacommand{by}\isamarkupfalse%
\ auto\isanewline
\ \ \isacommand{then}\isamarkupfalse%
\ \isacommand{have}\isamarkupfalse%
\ {\isachardoublequoteopen}{\isasymlangle}g{\isadigit{1}}{\isacharcomma}{\kern0pt}\ {\isasymlangle}g{\isadigit{2}}{\isacharcomma}{\kern0pt}\ g{\isadigit{3}}{\isasymrangle}{\isasymrangle}\ {\isacharequal}{\kern0pt}\ {\isasymlangle}h{\isadigit{1}}{\isacharcomma}{\kern0pt}\ {\isasymlangle}h{\isadigit{2}}{\isacharcomma}{\kern0pt}\ h{\isadigit{3}}{\isasymrangle}{\isasymrangle}{\isachardoublequoteclose}\isanewline
\ \ \ \ \isacommand{by}\isamarkupfalse%
\ simp\isanewline
\ \ \isacommand{then}\isamarkupfalse%
\ \isacommand{show}\isamarkupfalse%
\ {\isachardoublequoteopen}g\ {\isacharequal}{\kern0pt}\ h{\isachardoublequoteclose}\isanewline
\ \ \ \ \isacommand{by}\isamarkupfalse%
\ {\isacharparenleft}{\kern0pt}simp\ add{\isacharcolon}{\kern0pt}\ g{\isacharunderscore}{\kern0pt}expand\ h{\isacharunderscore}{\kern0pt}expand{\isacharparenright}{\kern0pt}\isanewline
\isacommand{qed}\isamarkupfalse%
%
\endisatagproof
{\isafoldproof}%
%
\isadelimproof
%
\endisadelimproof
%
\isadelimdocument
%
\endisadelimdocument
%
\isatagdocument
%
\isamarkupsubsubsection{Selecting Pairs from a Pair of Pairs%
}
\isamarkuptrue%
%
\endisatagdocument
{\isafolddocument}%
%
\isadelimdocument
%
\endisadelimdocument
\isacommand{definition}\isamarkupfalse%
\ outers\ {\isacharcolon}{\kern0pt}{\isacharcolon}{\kern0pt}\ {\isachardoublequoteopen}cset\ {\isasymRightarrow}\ cset\ {\isasymRightarrow}\ cset\ {\isasymRightarrow}\ cset\ {\isasymRightarrow}\ cfunc{\isachardoublequoteclose}\ \isakeyword{where}\isanewline
\ \ {\isachardoublequoteopen}outers\ A\ B\ C\ D\ {\isacharequal}{\kern0pt}\ {\isasymlangle}\isanewline
\ \ \ \ \ \ left{\isacharunderscore}{\kern0pt}cart{\isacharunderscore}{\kern0pt}proj\ A\ B\ {\isasymcirc}\isactrlsub c\ left{\isacharunderscore}{\kern0pt}cart{\isacharunderscore}{\kern0pt}proj\ {\isacharparenleft}{\kern0pt}A\ {\isasymtimes}\isactrlsub c\ B{\isacharparenright}{\kern0pt}\ {\isacharparenleft}{\kern0pt}C\ {\isasymtimes}\isactrlsub c\ D{\isacharparenright}{\kern0pt}{\isacharcomma}{\kern0pt}\isanewline
\ \ \ \ \ \ right{\isacharunderscore}{\kern0pt}cart{\isacharunderscore}{\kern0pt}proj\ C\ D\ {\isasymcirc}\isactrlsub c\ right{\isacharunderscore}{\kern0pt}cart{\isacharunderscore}{\kern0pt}proj\ {\isacharparenleft}{\kern0pt}A\ {\isasymtimes}\isactrlsub c\ B{\isacharparenright}{\kern0pt}\ {\isacharparenleft}{\kern0pt}C\ {\isasymtimes}\isactrlsub c\ D{\isacharparenright}{\kern0pt}\isanewline
\ \ \ \ {\isasymrangle}{\isachardoublequoteclose}\isanewline
\isanewline
\isacommand{lemma}\isamarkupfalse%
\ outers{\isacharunderscore}{\kern0pt}type{\isacharbrackleft}{\kern0pt}type{\isacharunderscore}{\kern0pt}rule{\isacharbrackright}{\kern0pt}{\isacharcolon}{\kern0pt}\ {\isachardoublequoteopen}outers\ A\ B\ C\ D\ {\isacharcolon}{\kern0pt}\ {\isacharparenleft}{\kern0pt}A\ {\isasymtimes}\isactrlsub c\ B{\isacharparenright}{\kern0pt}\ {\isasymtimes}\isactrlsub c\ {\isacharparenleft}{\kern0pt}C\ {\isasymtimes}\isactrlsub c\ D{\isacharparenright}{\kern0pt}\ {\isasymrightarrow}\ {\isacharparenleft}{\kern0pt}A\ {\isasymtimes}\isactrlsub c\ D{\isacharparenright}{\kern0pt}{\isachardoublequoteclose}\isanewline
%
\isadelimproof
\ \ %
\endisadelimproof
%
\isatagproof
\isacommand{unfolding}\isamarkupfalse%
\ outers{\isacharunderscore}{\kern0pt}def\ \isacommand{by}\isamarkupfalse%
\ typecheck{\isacharunderscore}{\kern0pt}cfuncs%
\endisatagproof
{\isafoldproof}%
%
\isadelimproof
\isanewline
%
\endisadelimproof
\isanewline
\isacommand{lemma}\isamarkupfalse%
\ outers{\isacharunderscore}{\kern0pt}apply{\isacharcolon}{\kern0pt}\isanewline
\ \ \isakeyword{assumes}\ {\isachardoublequoteopen}a\ {\isacharcolon}{\kern0pt}\ Z\ {\isasymrightarrow}\ A{\isachardoublequoteclose}\ {\isachardoublequoteopen}b\ {\isacharcolon}{\kern0pt}\ Z\ {\isasymrightarrow}\ B{\isachardoublequoteclose}\ {\isachardoublequoteopen}c\ {\isacharcolon}{\kern0pt}\ Z\ {\isasymrightarrow}\ C{\isachardoublequoteclose}\ {\isachardoublequoteopen}d\ {\isacharcolon}{\kern0pt}\ Z\ {\isasymrightarrow}\ D{\isachardoublequoteclose}\isanewline
\ \ \isakeyword{shows}\ {\isachardoublequoteopen}outers\ A\ B\ C\ D\ {\isasymcirc}\isactrlsub c\ {\isasymlangle}{\isasymlangle}a{\isacharcomma}{\kern0pt}b{\isasymrangle}{\isacharcomma}{\kern0pt}\ {\isasymlangle}c{\isacharcomma}{\kern0pt}d{\isasymrangle}{\isasymrangle}\ {\isacharequal}{\kern0pt}\ {\isasymlangle}a{\isacharcomma}{\kern0pt}d{\isasymrangle}{\isachardoublequoteclose}\isanewline
%
\isadelimproof
%
\endisadelimproof
%
\isatagproof
\isacommand{proof}\isamarkupfalse%
\ {\isacharminus}{\kern0pt}\isanewline
\ \ \isacommand{have}\isamarkupfalse%
\ {\isachardoublequoteopen}outers\ A\ B\ C\ D\ {\isasymcirc}\isactrlsub c\ {\isasymlangle}{\isasymlangle}a{\isacharcomma}{\kern0pt}b{\isasymrangle}{\isacharcomma}{\kern0pt}\ {\isasymlangle}c{\isacharcomma}{\kern0pt}d{\isasymrangle}{\isasymrangle}\ {\isacharequal}{\kern0pt}\ {\isasymlangle}\isanewline
\ \ \ \ \ \ left{\isacharunderscore}{\kern0pt}cart{\isacharunderscore}{\kern0pt}proj\ A\ B\ {\isasymcirc}\isactrlsub c\ left{\isacharunderscore}{\kern0pt}cart{\isacharunderscore}{\kern0pt}proj\ {\isacharparenleft}{\kern0pt}A\ {\isasymtimes}\isactrlsub c\ B{\isacharparenright}{\kern0pt}\ {\isacharparenleft}{\kern0pt}C\ {\isasymtimes}\isactrlsub c\ D{\isacharparenright}{\kern0pt}\ {\isasymcirc}\isactrlsub c\ {\isasymlangle}{\isasymlangle}a{\isacharcomma}{\kern0pt}b{\isasymrangle}{\isacharcomma}{\kern0pt}\ {\isasymlangle}c{\isacharcomma}{\kern0pt}\ d{\isasymrangle}{\isasymrangle}{\isacharcomma}{\kern0pt}\isanewline
\ \ \ \ \ \ right{\isacharunderscore}{\kern0pt}cart{\isacharunderscore}{\kern0pt}proj\ C\ D\ {\isasymcirc}\isactrlsub c\ right{\isacharunderscore}{\kern0pt}cart{\isacharunderscore}{\kern0pt}proj\ {\isacharparenleft}{\kern0pt}A\ {\isasymtimes}\isactrlsub c\ B{\isacharparenright}{\kern0pt}\ {\isacharparenleft}{\kern0pt}C\ {\isasymtimes}\isactrlsub c\ D{\isacharparenright}{\kern0pt}\ {\isasymcirc}\isactrlsub c\ {\isasymlangle}{\isasymlangle}a{\isacharcomma}{\kern0pt}b{\isasymrangle}{\isacharcomma}{\kern0pt}\ {\isasymlangle}c{\isacharcomma}{\kern0pt}\ d{\isasymrangle}{\isasymrangle}\isanewline
\ \ \ \ {\isasymrangle}{\isachardoublequoteclose}\isanewline
\ \ \ \ \isacommand{unfolding}\isamarkupfalse%
\ outers{\isacharunderscore}{\kern0pt}def\ \ \isacommand{using}\isamarkupfalse%
\ assms\ \isacommand{by}\isamarkupfalse%
\ {\isacharparenleft}{\kern0pt}typecheck{\isacharunderscore}{\kern0pt}cfuncs{\isacharcomma}{\kern0pt}\ simp\ add{\isacharcolon}{\kern0pt}\ cfunc{\isacharunderscore}{\kern0pt}prod{\isacharunderscore}{\kern0pt}comp\ comp{\isacharunderscore}{\kern0pt}associative{\isadigit{2}}{\isacharparenright}{\kern0pt}\isanewline
\ \ \isacommand{also}\isamarkupfalse%
\ \isacommand{have}\isamarkupfalse%
\ {\isachardoublequoteopen}{\isachardot}{\kern0pt}{\isachardot}{\kern0pt}{\isachardot}{\kern0pt}\ {\isacharequal}{\kern0pt}\ {\isasymlangle}left{\isacharunderscore}{\kern0pt}cart{\isacharunderscore}{\kern0pt}proj\ A\ B\ {\isasymcirc}\isactrlsub c\ {\isasymlangle}a{\isacharcomma}{\kern0pt}b{\isasymrangle}{\isacharcomma}{\kern0pt}\ right{\isacharunderscore}{\kern0pt}cart{\isacharunderscore}{\kern0pt}proj\ C\ D\ {\isasymcirc}\isactrlsub c\ {\isasymlangle}c{\isacharcomma}{\kern0pt}d{\isasymrangle}{\isasymrangle}{\isachardoublequoteclose}\isanewline
\ \ \ \ \isacommand{using}\isamarkupfalse%
\ assms\ \isacommand{by}\isamarkupfalse%
\ {\isacharparenleft}{\kern0pt}typecheck{\isacharunderscore}{\kern0pt}cfuncs{\isacharcomma}{\kern0pt}\ simp\ add{\isacharcolon}{\kern0pt}\ left{\isacharunderscore}{\kern0pt}cart{\isacharunderscore}{\kern0pt}proj{\isacharunderscore}{\kern0pt}cfunc{\isacharunderscore}{\kern0pt}prod\ right{\isacharunderscore}{\kern0pt}cart{\isacharunderscore}{\kern0pt}proj{\isacharunderscore}{\kern0pt}cfunc{\isacharunderscore}{\kern0pt}prod{\isacharparenright}{\kern0pt}\isanewline
\ \ \isacommand{also}\isamarkupfalse%
\ \isacommand{have}\isamarkupfalse%
\ {\isachardoublequoteopen}{\isachardot}{\kern0pt}{\isachardot}{\kern0pt}{\isachardot}{\kern0pt}\ {\isacharequal}{\kern0pt}\ {\isasymlangle}a{\isacharcomma}{\kern0pt}\ d{\isasymrangle}{\isachardoublequoteclose}\isanewline
\ \ \ \ \isacommand{using}\isamarkupfalse%
\ assms\ \isacommand{by}\isamarkupfalse%
\ {\isacharparenleft}{\kern0pt}typecheck{\isacharunderscore}{\kern0pt}cfuncs{\isacharcomma}{\kern0pt}\ simp\ add{\isacharcolon}{\kern0pt}\ left{\isacharunderscore}{\kern0pt}cart{\isacharunderscore}{\kern0pt}proj{\isacharunderscore}{\kern0pt}cfunc{\isacharunderscore}{\kern0pt}prod\ right{\isacharunderscore}{\kern0pt}cart{\isacharunderscore}{\kern0pt}proj{\isacharunderscore}{\kern0pt}cfunc{\isacharunderscore}{\kern0pt}prod{\isacharparenright}{\kern0pt}\isanewline
\ \ \isacommand{then}\isamarkupfalse%
\ \isacommand{show}\isamarkupfalse%
\ {\isacharquery}{\kern0pt}thesis\isanewline
\ \ \ \ \isacommand{using}\isamarkupfalse%
\ calculation\ \isacommand{by}\isamarkupfalse%
\ auto\isanewline
\isacommand{qed}\isamarkupfalse%
%
\endisatagproof
{\isafoldproof}%
%
\isadelimproof
\isanewline
%
\endisadelimproof
\isanewline
\isacommand{definition}\isamarkupfalse%
\ inners\ {\isacharcolon}{\kern0pt}{\isacharcolon}{\kern0pt}\ {\isachardoublequoteopen}cset\ {\isasymRightarrow}\ cset\ {\isasymRightarrow}\ cset\ {\isasymRightarrow}\ cset\ {\isasymRightarrow}\ cfunc{\isachardoublequoteclose}\ \isakeyword{where}\isanewline
\ \ {\isachardoublequoteopen}inners\ A\ B\ C\ D\ {\isacharequal}{\kern0pt}\ {\isasymlangle}\isanewline
\ \ \ \ \ \ right{\isacharunderscore}{\kern0pt}cart{\isacharunderscore}{\kern0pt}proj\ A\ B\ {\isasymcirc}\isactrlsub c\ left{\isacharunderscore}{\kern0pt}cart{\isacharunderscore}{\kern0pt}proj\ {\isacharparenleft}{\kern0pt}A\ {\isasymtimes}\isactrlsub c\ B{\isacharparenright}{\kern0pt}\ {\isacharparenleft}{\kern0pt}C\ {\isasymtimes}\isactrlsub c\ D{\isacharparenright}{\kern0pt}{\isacharcomma}{\kern0pt}\isanewline
\ \ \ \ \ \ left{\isacharunderscore}{\kern0pt}cart{\isacharunderscore}{\kern0pt}proj\ C\ D\ {\isasymcirc}\isactrlsub c\ right{\isacharunderscore}{\kern0pt}cart{\isacharunderscore}{\kern0pt}proj\ {\isacharparenleft}{\kern0pt}A\ {\isasymtimes}\isactrlsub c\ B{\isacharparenright}{\kern0pt}\ {\isacharparenleft}{\kern0pt}C\ {\isasymtimes}\isactrlsub c\ D{\isacharparenright}{\kern0pt}\isanewline
\ \ \ \ {\isasymrangle}{\isachardoublequoteclose}\isanewline
\isanewline
\isacommand{lemma}\isamarkupfalse%
\ inners{\isacharunderscore}{\kern0pt}type{\isacharbrackleft}{\kern0pt}type{\isacharunderscore}{\kern0pt}rule{\isacharbrackright}{\kern0pt}{\isacharcolon}{\kern0pt}\ {\isachardoublequoteopen}inners\ A\ B\ C\ D\ {\isacharcolon}{\kern0pt}\ {\isacharparenleft}{\kern0pt}A\ {\isasymtimes}\isactrlsub c\ B{\isacharparenright}{\kern0pt}\ {\isasymtimes}\isactrlsub c\ {\isacharparenleft}{\kern0pt}C\ {\isasymtimes}\isactrlsub c\ D{\isacharparenright}{\kern0pt}\ {\isasymrightarrow}\ {\isacharparenleft}{\kern0pt}B\ {\isasymtimes}\isactrlsub c\ C{\isacharparenright}{\kern0pt}{\isachardoublequoteclose}\isanewline
%
\isadelimproof
\ \ %
\endisadelimproof
%
\isatagproof
\isacommand{unfolding}\isamarkupfalse%
\ inners{\isacharunderscore}{\kern0pt}def\ \isacommand{by}\isamarkupfalse%
\ typecheck{\isacharunderscore}{\kern0pt}cfuncs%
\endisatagproof
{\isafoldproof}%
%
\isadelimproof
\isanewline
%
\endisadelimproof
\ \ \ \ \isanewline
\isacommand{lemma}\isamarkupfalse%
\ inners{\isacharunderscore}{\kern0pt}apply{\isacharcolon}{\kern0pt}\isanewline
\ \ \isakeyword{assumes}\ {\isachardoublequoteopen}a\ {\isacharcolon}{\kern0pt}\ Z\ {\isasymrightarrow}\ A{\isachardoublequoteclose}\ {\isachardoublequoteopen}b\ {\isacharcolon}{\kern0pt}\ Z\ {\isasymrightarrow}\ B{\isachardoublequoteclose}\ {\isachardoublequoteopen}c\ {\isacharcolon}{\kern0pt}\ Z\ {\isasymrightarrow}\ C{\isachardoublequoteclose}\ {\isachardoublequoteopen}d\ {\isacharcolon}{\kern0pt}\ Z\ {\isasymrightarrow}\ D{\isachardoublequoteclose}\isanewline
\ \ \isakeyword{shows}\ {\isachardoublequoteopen}inners\ A\ B\ C\ D\ {\isasymcirc}\isactrlsub c\ {\isasymlangle}{\isasymlangle}a{\isacharcomma}{\kern0pt}b{\isasymrangle}{\isacharcomma}{\kern0pt}\ {\isasymlangle}c{\isacharcomma}{\kern0pt}\ d{\isasymrangle}{\isasymrangle}\ {\isacharequal}{\kern0pt}\ {\isasymlangle}b{\isacharcomma}{\kern0pt}c{\isasymrangle}{\isachardoublequoteclose}\isanewline
%
\isadelimproof
%
\endisadelimproof
%
\isatagproof
\isacommand{proof}\isamarkupfalse%
\ {\isacharminus}{\kern0pt}\isanewline
\ \ \isacommand{have}\isamarkupfalse%
\ {\isachardoublequoteopen}inners\ A\ B\ C\ D\ {\isasymcirc}\isactrlsub c\ {\isasymlangle}{\isasymlangle}a{\isacharcomma}{\kern0pt}b{\isasymrangle}{\isacharcomma}{\kern0pt}\ {\isasymlangle}c{\isacharcomma}{\kern0pt}\ d{\isasymrangle}{\isasymrangle}\ {\isacharequal}{\kern0pt}\ {\isasymlangle}\isanewline
\ \ \ \ \ \ right{\isacharunderscore}{\kern0pt}cart{\isacharunderscore}{\kern0pt}proj\ A\ B\ {\isasymcirc}\isactrlsub c\ left{\isacharunderscore}{\kern0pt}cart{\isacharunderscore}{\kern0pt}proj\ {\isacharparenleft}{\kern0pt}A\ {\isasymtimes}\isactrlsub c\ B{\isacharparenright}{\kern0pt}\ {\isacharparenleft}{\kern0pt}C\ {\isasymtimes}\isactrlsub c\ D{\isacharparenright}{\kern0pt}\ {\isasymcirc}\isactrlsub c\ {\isasymlangle}{\isasymlangle}a{\isacharcomma}{\kern0pt}b{\isasymrangle}{\isacharcomma}{\kern0pt}\ {\isasymlangle}c{\isacharcomma}{\kern0pt}\ d{\isasymrangle}{\isasymrangle}{\isacharcomma}{\kern0pt}\isanewline
\ \ \ \ \ \ left{\isacharunderscore}{\kern0pt}cart{\isacharunderscore}{\kern0pt}proj\ C\ D\ {\isasymcirc}\isactrlsub c\ right{\isacharunderscore}{\kern0pt}cart{\isacharunderscore}{\kern0pt}proj\ {\isacharparenleft}{\kern0pt}A\ {\isasymtimes}\isactrlsub c\ B{\isacharparenright}{\kern0pt}\ {\isacharparenleft}{\kern0pt}C\ {\isasymtimes}\isactrlsub c\ D{\isacharparenright}{\kern0pt}\ {\isasymcirc}\isactrlsub c\ {\isasymlangle}{\isasymlangle}a{\isacharcomma}{\kern0pt}b{\isasymrangle}{\isacharcomma}{\kern0pt}\ {\isasymlangle}c{\isacharcomma}{\kern0pt}\ d{\isasymrangle}{\isasymrangle}{\isasymrangle}{\isachardoublequoteclose}\isanewline
\ \ \ \ \isacommand{unfolding}\isamarkupfalse%
\ inners{\isacharunderscore}{\kern0pt}def\ \isacommand{using}\isamarkupfalse%
\ assms\ \isacommand{by}\isamarkupfalse%
\ {\isacharparenleft}{\kern0pt}typecheck{\isacharunderscore}{\kern0pt}cfuncs{\isacharcomma}{\kern0pt}\ simp\ add{\isacharcolon}{\kern0pt}\ cfunc{\isacharunderscore}{\kern0pt}prod{\isacharunderscore}{\kern0pt}comp\ comp{\isacharunderscore}{\kern0pt}associative{\isadigit{2}}{\isacharparenright}{\kern0pt}\isanewline
\ \ \isacommand{also}\isamarkupfalse%
\ \isacommand{have}\isamarkupfalse%
\ {\isachardoublequoteopen}{\isachardot}{\kern0pt}{\isachardot}{\kern0pt}{\isachardot}{\kern0pt}\ {\isacharequal}{\kern0pt}\ {\isasymlangle}right{\isacharunderscore}{\kern0pt}cart{\isacharunderscore}{\kern0pt}proj\ A\ B\ {\isasymcirc}\isactrlsub c\ {\isasymlangle}a{\isacharcomma}{\kern0pt}b{\isasymrangle}{\isacharcomma}{\kern0pt}\ left{\isacharunderscore}{\kern0pt}cart{\isacharunderscore}{\kern0pt}proj\ C\ D\ {\isasymcirc}\isactrlsub c\ {\isasymlangle}c{\isacharcomma}{\kern0pt}d{\isasymrangle}{\isasymrangle}{\isachardoublequoteclose}\isanewline
\ \ \ \ \isacommand{using}\isamarkupfalse%
\ assms\ \isacommand{by}\isamarkupfalse%
\ {\isacharparenleft}{\kern0pt}typecheck{\isacharunderscore}{\kern0pt}cfuncs{\isacharcomma}{\kern0pt}\ simp\ add{\isacharcolon}{\kern0pt}\ left{\isacharunderscore}{\kern0pt}cart{\isacharunderscore}{\kern0pt}proj{\isacharunderscore}{\kern0pt}cfunc{\isacharunderscore}{\kern0pt}prod\ right{\isacharunderscore}{\kern0pt}cart{\isacharunderscore}{\kern0pt}proj{\isacharunderscore}{\kern0pt}cfunc{\isacharunderscore}{\kern0pt}prod{\isacharparenright}{\kern0pt}\isanewline
\ \ \isacommand{also}\isamarkupfalse%
\ \isacommand{have}\isamarkupfalse%
\ {\isachardoublequoteopen}{\isachardot}{\kern0pt}{\isachardot}{\kern0pt}{\isachardot}{\kern0pt}\ {\isacharequal}{\kern0pt}\ {\isasymlangle}b{\isacharcomma}{\kern0pt}\ c{\isasymrangle}{\isachardoublequoteclose}\isanewline
\ \ \ \ \isacommand{using}\isamarkupfalse%
\ assms\ \isacommand{by}\isamarkupfalse%
\ {\isacharparenleft}{\kern0pt}typecheck{\isacharunderscore}{\kern0pt}cfuncs{\isacharcomma}{\kern0pt}\ simp\ add{\isacharcolon}{\kern0pt}\ left{\isacharunderscore}{\kern0pt}cart{\isacharunderscore}{\kern0pt}proj{\isacharunderscore}{\kern0pt}cfunc{\isacharunderscore}{\kern0pt}prod\ right{\isacharunderscore}{\kern0pt}cart{\isacharunderscore}{\kern0pt}proj{\isacharunderscore}{\kern0pt}cfunc{\isacharunderscore}{\kern0pt}prod{\isacharparenright}{\kern0pt}\isanewline
\ \ \isacommand{then}\isamarkupfalse%
\ \isacommand{show}\isamarkupfalse%
\ {\isacharquery}{\kern0pt}thesis\isanewline
\ \ \ \ \isacommand{using}\isamarkupfalse%
\ calculation\ \isacommand{by}\isamarkupfalse%
\ auto\isanewline
\isacommand{qed}\isamarkupfalse%
%
\endisatagproof
{\isafoldproof}%
%
\isadelimproof
\isanewline
%
\endisadelimproof
\isanewline
\isacommand{definition}\isamarkupfalse%
\ lefts\ {\isacharcolon}{\kern0pt}{\isacharcolon}{\kern0pt}\ {\isachardoublequoteopen}cset\ {\isasymRightarrow}\ cset\ {\isasymRightarrow}\ cset\ {\isasymRightarrow}\ cset\ {\isasymRightarrow}\ cfunc{\isachardoublequoteclose}\ \isakeyword{where}\isanewline
\ \ {\isachardoublequoteopen}lefts\ A\ B\ C\ D\ {\isacharequal}{\kern0pt}\ {\isasymlangle}\isanewline
\ \ \ \ \ \ left{\isacharunderscore}{\kern0pt}cart{\isacharunderscore}{\kern0pt}proj\ A\ B\ {\isasymcirc}\isactrlsub c\ left{\isacharunderscore}{\kern0pt}cart{\isacharunderscore}{\kern0pt}proj\ {\isacharparenleft}{\kern0pt}A\ {\isasymtimes}\isactrlsub c\ B{\isacharparenright}{\kern0pt}\ {\isacharparenleft}{\kern0pt}C\ {\isasymtimes}\isactrlsub c\ D{\isacharparenright}{\kern0pt}{\isacharcomma}{\kern0pt}\isanewline
\ \ \ \ \ \ left{\isacharunderscore}{\kern0pt}cart{\isacharunderscore}{\kern0pt}proj\ C\ D\ {\isasymcirc}\isactrlsub c\ right{\isacharunderscore}{\kern0pt}cart{\isacharunderscore}{\kern0pt}proj\ {\isacharparenleft}{\kern0pt}A\ {\isasymtimes}\isactrlsub c\ B{\isacharparenright}{\kern0pt}\ {\isacharparenleft}{\kern0pt}C\ {\isasymtimes}\isactrlsub c\ D{\isacharparenright}{\kern0pt}\isanewline
\ \ \ \ {\isasymrangle}{\isachardoublequoteclose}\isanewline
\isanewline
\isacommand{lemma}\isamarkupfalse%
\ lefts{\isacharunderscore}{\kern0pt}type{\isacharbrackleft}{\kern0pt}type{\isacharunderscore}{\kern0pt}rule{\isacharbrackright}{\kern0pt}{\isacharcolon}{\kern0pt}\ {\isachardoublequoteopen}lefts\ A\ B\ C\ D\ {\isacharcolon}{\kern0pt}\ {\isacharparenleft}{\kern0pt}A\ {\isasymtimes}\isactrlsub c\ B{\isacharparenright}{\kern0pt}\ {\isasymtimes}\isactrlsub c\ {\isacharparenleft}{\kern0pt}C\ {\isasymtimes}\isactrlsub c\ D{\isacharparenright}{\kern0pt}\ {\isasymrightarrow}\ {\isacharparenleft}{\kern0pt}A\ {\isasymtimes}\isactrlsub c\ C{\isacharparenright}{\kern0pt}{\isachardoublequoteclose}\isanewline
%
\isadelimproof
\ \ %
\endisadelimproof
%
\isatagproof
\isacommand{unfolding}\isamarkupfalse%
\ lefts{\isacharunderscore}{\kern0pt}def\ \isacommand{by}\isamarkupfalse%
\ typecheck{\isacharunderscore}{\kern0pt}cfuncs%
\endisatagproof
{\isafoldproof}%
%
\isadelimproof
\isanewline
%
\endisadelimproof
\isanewline
\isacommand{lemma}\isamarkupfalse%
\ lefts{\isacharunderscore}{\kern0pt}apply{\isacharcolon}{\kern0pt}\isanewline
\ \ \isakeyword{assumes}\ {\isachardoublequoteopen}a\ {\isacharcolon}{\kern0pt}\ Z\ {\isasymrightarrow}\ A{\isachardoublequoteclose}\ {\isachardoublequoteopen}b\ {\isacharcolon}{\kern0pt}\ Z\ {\isasymrightarrow}\ B{\isachardoublequoteclose}\ {\isachardoublequoteopen}c\ {\isacharcolon}{\kern0pt}\ Z\ {\isasymrightarrow}\ C{\isachardoublequoteclose}\ {\isachardoublequoteopen}d\ {\isacharcolon}{\kern0pt}\ Z\ {\isasymrightarrow}\ D{\isachardoublequoteclose}\isanewline
\ \ \isakeyword{shows}\ {\isachardoublequoteopen}lefts\ A\ B\ C\ D\ {\isasymcirc}\isactrlsub c\ {\isasymlangle}{\isasymlangle}a{\isacharcomma}{\kern0pt}b{\isasymrangle}{\isacharcomma}{\kern0pt}\ {\isasymlangle}c{\isacharcomma}{\kern0pt}\ d{\isasymrangle}{\isasymrangle}\ {\isacharequal}{\kern0pt}\ {\isasymlangle}a{\isacharcomma}{\kern0pt}c{\isasymrangle}{\isachardoublequoteclose}\isanewline
%
\isadelimproof
%
\endisadelimproof
%
\isatagproof
\isacommand{proof}\isamarkupfalse%
\ {\isacharminus}{\kern0pt}\isanewline
\ \ \isacommand{have}\isamarkupfalse%
\ {\isachardoublequoteopen}lefts\ A\ B\ C\ D\ {\isasymcirc}\isactrlsub c\ {\isasymlangle}{\isasymlangle}a{\isacharcomma}{\kern0pt}b{\isasymrangle}{\isacharcomma}{\kern0pt}\ {\isasymlangle}c{\isacharcomma}{\kern0pt}\ d{\isasymrangle}{\isasymrangle}\ {\isacharequal}{\kern0pt}\ {\isasymlangle}left{\isacharunderscore}{\kern0pt}cart{\isacharunderscore}{\kern0pt}proj\ A\ B\ {\isasymcirc}\isactrlsub c\ left{\isacharunderscore}{\kern0pt}cart{\isacharunderscore}{\kern0pt}proj\ {\isacharparenleft}{\kern0pt}A\ {\isasymtimes}\isactrlsub c\ B{\isacharparenright}{\kern0pt}\ {\isacharparenleft}{\kern0pt}C\ {\isasymtimes}\isactrlsub c\ D{\isacharparenright}{\kern0pt}\ {\isasymcirc}\isactrlsub c\ {\isasymlangle}{\isasymlangle}a{\isacharcomma}{\kern0pt}b{\isasymrangle}{\isacharcomma}{\kern0pt}\ {\isasymlangle}c{\isacharcomma}{\kern0pt}\ d{\isasymrangle}{\isasymrangle}{\isacharcomma}{\kern0pt}\ left{\isacharunderscore}{\kern0pt}cart{\isacharunderscore}{\kern0pt}proj\ C\ D\ {\isasymcirc}\isactrlsub c\ right{\isacharunderscore}{\kern0pt}cart{\isacharunderscore}{\kern0pt}proj\ {\isacharparenleft}{\kern0pt}A\ {\isasymtimes}\isactrlsub c\ B{\isacharparenright}{\kern0pt}\ {\isacharparenleft}{\kern0pt}C\ {\isasymtimes}\isactrlsub c\ D{\isacharparenright}{\kern0pt}\ {\isasymcirc}\isactrlsub c\ {\isasymlangle}{\isasymlangle}a{\isacharcomma}{\kern0pt}b{\isasymrangle}{\isacharcomma}{\kern0pt}\ {\isasymlangle}c{\isacharcomma}{\kern0pt}\ d{\isasymrangle}{\isasymrangle}{\isasymrangle}{\isachardoublequoteclose}\isanewline
\ \ \ \ \isacommand{unfolding}\isamarkupfalse%
\ lefts{\isacharunderscore}{\kern0pt}def\ \isacommand{using}\isamarkupfalse%
\ assms\ \isacommand{by}\isamarkupfalse%
\ {\isacharparenleft}{\kern0pt}typecheck{\isacharunderscore}{\kern0pt}cfuncs{\isacharcomma}{\kern0pt}\ simp\ add{\isacharcolon}{\kern0pt}\ cfunc{\isacharunderscore}{\kern0pt}prod{\isacharunderscore}{\kern0pt}comp\ comp{\isacharunderscore}{\kern0pt}associative{\isadigit{2}}{\isacharparenright}{\kern0pt}\isanewline
\ \ \isacommand{also}\isamarkupfalse%
\ \isacommand{have}\isamarkupfalse%
\ {\isachardoublequoteopen}{\isachardot}{\kern0pt}{\isachardot}{\kern0pt}{\isachardot}{\kern0pt}\ {\isacharequal}{\kern0pt}\ {\isasymlangle}left{\isacharunderscore}{\kern0pt}cart{\isacharunderscore}{\kern0pt}proj\ A\ B\ {\isasymcirc}\isactrlsub c\ {\isasymlangle}a{\isacharcomma}{\kern0pt}b{\isasymrangle}{\isacharcomma}{\kern0pt}\ left{\isacharunderscore}{\kern0pt}cart{\isacharunderscore}{\kern0pt}proj\ C\ D\ {\isasymcirc}\isactrlsub c\ {\isasymlangle}c{\isacharcomma}{\kern0pt}d{\isasymrangle}{\isasymrangle}{\isachardoublequoteclose}\isanewline
\ \ \ \ \isacommand{using}\isamarkupfalse%
\ assms\ \isacommand{by}\isamarkupfalse%
\ {\isacharparenleft}{\kern0pt}typecheck{\isacharunderscore}{\kern0pt}cfuncs{\isacharcomma}{\kern0pt}\ simp\ add{\isacharcolon}{\kern0pt}\ left{\isacharunderscore}{\kern0pt}cart{\isacharunderscore}{\kern0pt}proj{\isacharunderscore}{\kern0pt}cfunc{\isacharunderscore}{\kern0pt}prod\ right{\isacharunderscore}{\kern0pt}cart{\isacharunderscore}{\kern0pt}proj{\isacharunderscore}{\kern0pt}cfunc{\isacharunderscore}{\kern0pt}prod{\isacharparenright}{\kern0pt}\isanewline
\ \ \isacommand{also}\isamarkupfalse%
\ \isacommand{have}\isamarkupfalse%
\ {\isachardoublequoteopen}{\isachardot}{\kern0pt}{\isachardot}{\kern0pt}{\isachardot}{\kern0pt}\ {\isacharequal}{\kern0pt}\ {\isasymlangle}a{\isacharcomma}{\kern0pt}\ c{\isasymrangle}{\isachardoublequoteclose}\isanewline
\ \ \ \ \isacommand{using}\isamarkupfalse%
\ assms\ \isacommand{by}\isamarkupfalse%
\ {\isacharparenleft}{\kern0pt}typecheck{\isacharunderscore}{\kern0pt}cfuncs{\isacharcomma}{\kern0pt}\ simp\ add{\isacharcolon}{\kern0pt}\ left{\isacharunderscore}{\kern0pt}cart{\isacharunderscore}{\kern0pt}proj{\isacharunderscore}{\kern0pt}cfunc{\isacharunderscore}{\kern0pt}prod{\isacharparenright}{\kern0pt}\isanewline
\ \ \isacommand{then}\isamarkupfalse%
\ \isacommand{show}\isamarkupfalse%
\ {\isacharquery}{\kern0pt}thesis\isanewline
\ \ \ \ \isacommand{using}\isamarkupfalse%
\ calculation\ \isacommand{by}\isamarkupfalse%
\ auto\isanewline
\isacommand{qed}\isamarkupfalse%
%
\endisatagproof
{\isafoldproof}%
%
\isadelimproof
\isanewline
%
\endisadelimproof
\isanewline
\isacommand{definition}\isamarkupfalse%
\ rights\ {\isacharcolon}{\kern0pt}{\isacharcolon}{\kern0pt}\ {\isachardoublequoteopen}cset\ {\isasymRightarrow}\ cset\ {\isasymRightarrow}\ cset\ {\isasymRightarrow}\ cset\ {\isasymRightarrow}\ cfunc{\isachardoublequoteclose}\ \isakeyword{where}\isanewline
\ \ {\isachardoublequoteopen}rights\ A\ B\ C\ D\ {\isacharequal}{\kern0pt}\ {\isasymlangle}\isanewline
\ \ \ \ \ \ right{\isacharunderscore}{\kern0pt}cart{\isacharunderscore}{\kern0pt}proj\ A\ B\ {\isasymcirc}\isactrlsub c\ left{\isacharunderscore}{\kern0pt}cart{\isacharunderscore}{\kern0pt}proj\ {\isacharparenleft}{\kern0pt}A\ {\isasymtimes}\isactrlsub c\ B{\isacharparenright}{\kern0pt}\ {\isacharparenleft}{\kern0pt}C\ {\isasymtimes}\isactrlsub c\ D{\isacharparenright}{\kern0pt}{\isacharcomma}{\kern0pt}\isanewline
\ \ \ \ \ \ right{\isacharunderscore}{\kern0pt}cart{\isacharunderscore}{\kern0pt}proj\ C\ D\ {\isasymcirc}\isactrlsub c\ right{\isacharunderscore}{\kern0pt}cart{\isacharunderscore}{\kern0pt}proj\ {\isacharparenleft}{\kern0pt}A\ {\isasymtimes}\isactrlsub c\ B{\isacharparenright}{\kern0pt}\ {\isacharparenleft}{\kern0pt}C\ {\isasymtimes}\isactrlsub c\ D{\isacharparenright}{\kern0pt}\isanewline
\ \ \ \ {\isasymrangle}{\isachardoublequoteclose}\isanewline
\isanewline
\isacommand{lemma}\isamarkupfalse%
\ rights{\isacharunderscore}{\kern0pt}type{\isacharbrackleft}{\kern0pt}type{\isacharunderscore}{\kern0pt}rule{\isacharbrackright}{\kern0pt}{\isacharcolon}{\kern0pt}\ {\isachardoublequoteopen}rights\ A\ B\ C\ D\ {\isacharcolon}{\kern0pt}\ {\isacharparenleft}{\kern0pt}A\ {\isasymtimes}\isactrlsub c\ B{\isacharparenright}{\kern0pt}\ {\isasymtimes}\isactrlsub c\ {\isacharparenleft}{\kern0pt}C\ {\isasymtimes}\isactrlsub c\ D{\isacharparenright}{\kern0pt}\ {\isasymrightarrow}\ {\isacharparenleft}{\kern0pt}B\ {\isasymtimes}\isactrlsub c\ D{\isacharparenright}{\kern0pt}{\isachardoublequoteclose}\isanewline
%
\isadelimproof
\ \ %
\endisadelimproof
%
\isatagproof
\isacommand{unfolding}\isamarkupfalse%
\ rights{\isacharunderscore}{\kern0pt}def\ \isacommand{by}\isamarkupfalse%
\ typecheck{\isacharunderscore}{\kern0pt}cfuncs%
\endisatagproof
{\isafoldproof}%
%
\isadelimproof
\isanewline
%
\endisadelimproof
\isanewline
\isacommand{lemma}\isamarkupfalse%
\ rights{\isacharunderscore}{\kern0pt}apply{\isacharcolon}{\kern0pt}\isanewline
\ \ \isakeyword{assumes}\ {\isachardoublequoteopen}a\ {\isacharcolon}{\kern0pt}\ Z\ {\isasymrightarrow}\ A{\isachardoublequoteclose}\ {\isachardoublequoteopen}b\ {\isacharcolon}{\kern0pt}\ Z\ {\isasymrightarrow}\ B{\isachardoublequoteclose}\ {\isachardoublequoteopen}c\ {\isacharcolon}{\kern0pt}\ Z\ {\isasymrightarrow}\ C{\isachardoublequoteclose}\ {\isachardoublequoteopen}d\ {\isacharcolon}{\kern0pt}\ Z\ {\isasymrightarrow}\ D{\isachardoublequoteclose}\isanewline
\ \ \isakeyword{shows}\ {\isachardoublequoteopen}rights\ A\ B\ C\ D\ {\isasymcirc}\isactrlsub c\ {\isasymlangle}{\isasymlangle}a{\isacharcomma}{\kern0pt}b{\isasymrangle}{\isacharcomma}{\kern0pt}\ {\isasymlangle}c{\isacharcomma}{\kern0pt}\ d{\isasymrangle}{\isasymrangle}\ {\isacharequal}{\kern0pt}\ {\isasymlangle}b{\isacharcomma}{\kern0pt}d{\isasymrangle}{\isachardoublequoteclose}\isanewline
%
\isadelimproof
%
\endisadelimproof
%
\isatagproof
\isacommand{proof}\isamarkupfalse%
\ {\isacharminus}{\kern0pt}\isanewline
\ \ \isacommand{have}\isamarkupfalse%
\ {\isachardoublequoteopen}rights\ A\ B\ C\ D\ {\isasymcirc}\isactrlsub c\ {\isasymlangle}{\isasymlangle}a{\isacharcomma}{\kern0pt}b{\isasymrangle}{\isacharcomma}{\kern0pt}\ {\isasymlangle}c{\isacharcomma}{\kern0pt}\ d{\isasymrangle}{\isasymrangle}\ {\isacharequal}{\kern0pt}\ {\isasymlangle}right{\isacharunderscore}{\kern0pt}cart{\isacharunderscore}{\kern0pt}proj\ A\ B\ {\isasymcirc}\isactrlsub c\ left{\isacharunderscore}{\kern0pt}cart{\isacharunderscore}{\kern0pt}proj\ {\isacharparenleft}{\kern0pt}A\ {\isasymtimes}\isactrlsub c\ B{\isacharparenright}{\kern0pt}\ {\isacharparenleft}{\kern0pt}C\ {\isasymtimes}\isactrlsub c\ D{\isacharparenright}{\kern0pt}\ {\isasymcirc}\isactrlsub c\ {\isasymlangle}{\isasymlangle}a{\isacharcomma}{\kern0pt}b{\isasymrangle}{\isacharcomma}{\kern0pt}\ {\isasymlangle}c{\isacharcomma}{\kern0pt}\ d{\isasymrangle}{\isasymrangle}{\isacharcomma}{\kern0pt}\ right{\isacharunderscore}{\kern0pt}cart{\isacharunderscore}{\kern0pt}proj\ C\ D\ {\isasymcirc}\isactrlsub c\ right{\isacharunderscore}{\kern0pt}cart{\isacharunderscore}{\kern0pt}proj\ {\isacharparenleft}{\kern0pt}A\ {\isasymtimes}\isactrlsub c\ B{\isacharparenright}{\kern0pt}\ {\isacharparenleft}{\kern0pt}C\ {\isasymtimes}\isactrlsub c\ D{\isacharparenright}{\kern0pt}\ {\isasymcirc}\isactrlsub c\ {\isasymlangle}{\isasymlangle}a{\isacharcomma}{\kern0pt}b{\isasymrangle}{\isacharcomma}{\kern0pt}\ {\isasymlangle}c{\isacharcomma}{\kern0pt}\ d{\isasymrangle}{\isasymrangle}{\isasymrangle}{\isachardoublequoteclose}\isanewline
\ \ \ \ \isacommand{unfolding}\isamarkupfalse%
\ rights{\isacharunderscore}{\kern0pt}def\ \isacommand{using}\isamarkupfalse%
\ assms\ \isacommand{by}\isamarkupfalse%
\ {\isacharparenleft}{\kern0pt}typecheck{\isacharunderscore}{\kern0pt}cfuncs{\isacharcomma}{\kern0pt}\ simp\ add{\isacharcolon}{\kern0pt}\ cfunc{\isacharunderscore}{\kern0pt}prod{\isacharunderscore}{\kern0pt}comp\ comp{\isacharunderscore}{\kern0pt}associative{\isadigit{2}}{\isacharparenright}{\kern0pt}\isanewline
\ \ \isacommand{also}\isamarkupfalse%
\ \isacommand{have}\isamarkupfalse%
\ {\isachardoublequoteopen}{\isachardot}{\kern0pt}{\isachardot}{\kern0pt}{\isachardot}{\kern0pt}\ {\isacharequal}{\kern0pt}\ {\isasymlangle}right{\isacharunderscore}{\kern0pt}cart{\isacharunderscore}{\kern0pt}proj\ A\ B\ {\isasymcirc}\isactrlsub c\ {\isasymlangle}a{\isacharcomma}{\kern0pt}b{\isasymrangle}{\isacharcomma}{\kern0pt}\ right{\isacharunderscore}{\kern0pt}cart{\isacharunderscore}{\kern0pt}proj\ C\ D\ {\isasymcirc}\isactrlsub c\ {\isasymlangle}c{\isacharcomma}{\kern0pt}d{\isasymrangle}{\isasymrangle}{\isachardoublequoteclose}\isanewline
\ \ \ \ \isacommand{using}\isamarkupfalse%
\ assms\ \isacommand{by}\isamarkupfalse%
\ {\isacharparenleft}{\kern0pt}typecheck{\isacharunderscore}{\kern0pt}cfuncs{\isacharcomma}{\kern0pt}\ simp\ add{\isacharcolon}{\kern0pt}\ left{\isacharunderscore}{\kern0pt}cart{\isacharunderscore}{\kern0pt}proj{\isacharunderscore}{\kern0pt}cfunc{\isacharunderscore}{\kern0pt}prod\ right{\isacharunderscore}{\kern0pt}cart{\isacharunderscore}{\kern0pt}proj{\isacharunderscore}{\kern0pt}cfunc{\isacharunderscore}{\kern0pt}prod{\isacharparenright}{\kern0pt}\isanewline
\ \ \isacommand{also}\isamarkupfalse%
\ \isacommand{have}\isamarkupfalse%
\ {\isachardoublequoteopen}{\isachardot}{\kern0pt}{\isachardot}{\kern0pt}{\isachardot}{\kern0pt}\ {\isacharequal}{\kern0pt}\ {\isasymlangle}b{\isacharcomma}{\kern0pt}\ d{\isasymrangle}{\isachardoublequoteclose}\isanewline
\ \ \ \ \isacommand{using}\isamarkupfalse%
\ assms\ \isacommand{by}\isamarkupfalse%
\ {\isacharparenleft}{\kern0pt}typecheck{\isacharunderscore}{\kern0pt}cfuncs{\isacharcomma}{\kern0pt}\ simp\ add{\isacharcolon}{\kern0pt}\ right{\isacharunderscore}{\kern0pt}cart{\isacharunderscore}{\kern0pt}proj{\isacharunderscore}{\kern0pt}cfunc{\isacharunderscore}{\kern0pt}prod{\isacharparenright}{\kern0pt}\isanewline
\ \ \isacommand{then}\isamarkupfalse%
\ \isacommand{show}\isamarkupfalse%
\ {\isacharquery}{\kern0pt}thesis\isanewline
\ \ \ \ \isacommand{using}\isamarkupfalse%
\ calculation\ \isacommand{by}\isamarkupfalse%
\ auto\isanewline
\isacommand{qed}\isamarkupfalse%
%
\endisatagproof
{\isafoldproof}%
%
\isadelimproof
\isanewline
%
\endisadelimproof
%
\isadelimtheory
\isanewline
%
\endisadelimtheory
%
\isatagtheory
\isacommand{end}\isamarkupfalse%
%
\endisatagtheory
{\isafoldtheory}%
%
\isadelimtheory
%
\endisadelimtheory
%
\end{isabellebody}%
\endinput
%:%file=~/ETCS/HOL-ETCS/Product.thy%:%
%:%11=1%:%
%:%27=3%:%
%:%28=3%:%
%:%29=4%:%
%:%30=5%:%
%:%39=7%:%
%:%41=8%:%
%:%42=8%:%
%:%43=9%:%
%:%44=10%:%
%:%45=11%:%
%:%46=12%:%
%:%47=13%:%
%:%48=14%:%
%:%49=15%:%
%:%50=16%:%
%:%51=17%:%
%:%52=18%:%
%:%53=19%:%
%:%54=20%:%
%:%55=21%:%
%:%56=22%:%
%:%57=22%:%
%:%58=23%:%
%:%62=27%:%
%:%63=28%:%
%:%64=29%:%
%:%65=29%:%
%:%66=30%:%
%:%67=31%:%
%:%70=34%:%
%:%73=35%:%
%:%77=35%:%
%:%78=35%:%
%:%79=35%:%
%:%80=35%:%
%:%85=35%:%
%:%88=36%:%
%:%89=37%:%
%:%90=37%:%
%:%91=38%:%
%:%92=39%:%
%:%93=40%:%
%:%94=40%:%
%:%95=41%:%
%:%98=42%:%
%:%102=42%:%
%:%103=42%:%
%:%104=43%:%
%:%105=43%:%
%:%106=44%:%
%:%107=44%:%
%:%108=45%:%
%:%109=45%:%
%:%110=46%:%
%:%111=46%:%
%:%112=47%:%
%:%113=47%:%
%:%118=52%:%
%:%119=53%:%
%:%120=53%:%
%:%121=54%:%
%:%122=54%:%
%:%123=55%:%
%:%133=57%:%
%:%135=58%:%
%:%136=58%:%
%:%137=59%:%
%:%138=60%:%
%:%139=61%:%
%:%146=62%:%
%:%147=62%:%
%:%148=63%:%
%:%149=63%:%
%:%150=64%:%
%:%151=64%:%
%:%152=64%:%
%:%153=64%:%
%:%154=65%:%
%:%155=66%:%
%:%156=66%:%
%:%157=67%:%
%:%158=67%:%
%:%159=67%:%
%:%160=67%:%
%:%161=68%:%
%:%162=69%:%
%:%163=69%:%
%:%164=70%:%
%:%165=70%:%
%:%166=70%:%
%:%167=71%:%
%:%168=71%:%
%:%169=72%:%
%:%170=72%:%
%:%171=72%:%
%:%172=73%:%
%:%173=74%:%
%:%174=74%:%
%:%175=75%:%
%:%176=75%:%
%:%177=75%:%
%:%178=75%:%
%:%179=76%:%
%:%180=76%:%
%:%181=76%:%
%:%182=77%:%
%:%183=77%:%
%:%184=78%:%
%:%185=78%:%
%:%186=79%:%
%:%187=79%:%
%:%188=80%:%
%:%189=80%:%
%:%190=80%:%
%:%191=81%:%
%:%192=81%:%
%:%193=82%:%
%:%194=82%:%
%:%195=82%:%
%:%196=83%:%
%:%197=83%:%
%:%198=83%:%
%:%199=84%:%
%:%200=84%:%
%:%201=85%:%
%:%202=85%:%
%:%203=86%:%
%:%204=87%:%
%:%205=87%:%
%:%206=88%:%
%:%207=88%:%
%:%208=88%:%
%:%209=89%:%
%:%210=89%:%
%:%211=90%:%
%:%212=90%:%
%:%213=90%:%
%:%214=91%:%
%:%215=92%:%
%:%216=92%:%
%:%217=93%:%
%:%218=93%:%
%:%219=93%:%
%:%220=93%:%
%:%221=94%:%
%:%222=94%:%
%:%223=94%:%
%:%224=95%:%
%:%225=95%:%
%:%226=96%:%
%:%227=96%:%
%:%228=97%:%
%:%229=97%:%
%:%230=98%:%
%:%231=98%:%
%:%232=98%:%
%:%233=99%:%
%:%234=99%:%
%:%235=100%:%
%:%236=100%:%
%:%237=100%:%
%:%238=101%:%
%:%239=101%:%
%:%240=101%:%
%:%241=102%:%
%:%242=102%:%
%:%243=103%:%
%:%244=103%:%
%:%245=104%:%
%:%246=105%:%
%:%247=105%:%
%:%248=106%:%
%:%249=106%:%
%:%250=106%:%
%:%251=107%:%
%:%252=107%:%
%:%253=107%:%
%:%254=108%:%
%:%255=108%:%
%:%256=109%:%
%:%262=109%:%
%:%265=110%:%
%:%266=111%:%
%:%267=111%:%
%:%268=112%:%
%:%275=113%:%
%:%276=113%:%
%:%277=114%:%
%:%278=114%:%
%:%279=115%:%
%:%280=115%:%
%:%281=116%:%
%:%282=116%:%
%:%283=117%:%
%:%284=117%:%
%:%285=118%:%
%:%286=118%:%
%:%287=119%:%
%:%288=119%:%
%:%289=120%:%
%:%295=120%:%
%:%298=121%:%
%:%299=122%:%
%:%300=122%:%
%:%301=123%:%
%:%302=124%:%
%:%304=126%:%
%:%307=127%:%
%:%311=127%:%
%:%312=127%:%
%:%317=127%:%
%:%320=128%:%
%:%321=129%:%
%:%322=129%:%
%:%323=130%:%
%:%324=131%:%
%:%325=132%:%
%:%326=133%:%
%:%329=134%:%
%:%333=134%:%
%:%334=134%:%
%:%335=134%:%
%:%340=134%:%
%:%343=135%:%
%:%344=136%:%
%:%345=136%:%
%:%346=137%:%
%:%347=138%:%
%:%350=139%:%
%:%354=139%:%
%:%355=139%:%
%:%360=139%:%
%:%363=140%:%
%:%364=141%:%
%:%365=141%:%
%:%366=142%:%
%:%367=143%:%
%:%374=144%:%
%:%375=144%:%
%:%376=145%:%
%:%377=145%:%
%:%378=146%:%
%:%379=146%:%
%:%380=146%:%
%:%381=147%:%
%:%382=147%:%
%:%383=148%:%
%:%384=148%:%
%:%385=148%:%
%:%386=149%:%
%:%387=149%:%
%:%388=150%:%
%:%389=150%:%
%:%390=150%:%
%:%391=151%:%
%:%406=153%:%
%:%418=155%:%
%:%420=156%:%
%:%421=156%:%
%:%422=157%:%
%:%423=158%:%
%:%424=159%:%
%:%425=159%:%
%:%426=160%:%
%:%429=161%:%
%:%433=161%:%
%:%434=161%:%
%:%435=161%:%
%:%440=161%:%
%:%443=162%:%
%:%444=163%:%
%:%445=163%:%
%:%446=164%:%
%:%453=165%:%
%:%454=165%:%
%:%455=166%:%
%:%456=166%:%
%:%457=167%:%
%:%458=167%:%
%:%459=168%:%
%:%460=168%:%
%:%461=168%:%
%:%462=169%:%
%:%463=169%:%
%:%464=170%:%
%:%479=172%:%
%:%491=174%:%
%:%493=175%:%
%:%494=175%:%
%:%495=176%:%
%:%496=177%:%
%:%497=178%:%
%:%498=178%:%
%:%499=179%:%
%:%500=180%:%
%:%503=181%:%
%:%507=181%:%
%:%508=181%:%
%:%509=181%:%
%:%514=181%:%
%:%517=182%:%
%:%518=183%:%
%:%519=183%:%
%:%520=184%:%
%:%523=185%:%
%:%527=185%:%
%:%528=185%:%
%:%529=186%:%
%:%530=186%:%
%:%531=186%:%
%:%536=186%:%
%:%539=187%:%
%:%540=188%:%
%:%541=188%:%
%:%542=189%:%
%:%545=190%:%
%:%549=190%:%
%:%550=190%:%
%:%551=191%:%
%:%552=191%:%
%:%553=191%:%
%:%558=191%:%
%:%561=192%:%
%:%562=193%:%
%:%563=193%:%
%:%564=194%:%
%:%567=195%:%
%:%571=195%:%
%:%572=195%:%
%:%573=196%:%
%:%574=196%:%
%:%575=196%:%
%:%580=196%:%
%:%583=197%:%
%:%584=198%:%
%:%585=198%:%
%:%587=200%:%
%:%590=201%:%
%:%594=201%:%
%:%595=201%:%
%:%596=202%:%
%:%597=202%:%
%:%598=202%:%
%:%607=204%:%
%:%609=205%:%
%:%610=205%:%
%:%611=206%:%
%:%612=207%:%
%:%619=208%:%
%:%620=208%:%
%:%621=209%:%
%:%622=209%:%
%:%623=210%:%
%:%624=210%:%
%:%627=213%:%
%:%628=214%:%
%:%629=214%:%
%:%630=215%:%
%:%631=216%:%
%:%632=216%:%
%:%633=217%:%
%:%634=217%:%
%:%635=217%:%
%:%636=218%:%
%:%637=218%:%
%:%638=219%:%
%:%639=219%:%
%:%640=219%:%
%:%641=220%:%
%:%642=220%:%
%:%643=221%:%
%:%644=221%:%
%:%645=221%:%
%:%646=222%:%
%:%652=222%:%
%:%655=223%:%
%:%656=224%:%
%:%657=224%:%
%:%658=225%:%
%:%659=226%:%
%:%660=227%:%
%:%667=228%:%
%:%668=228%:%
%:%669=229%:%
%:%670=229%:%
%:%671=229%:%
%:%672=230%:%
%:%673=231%:%
%:%674=232%:%
%:%675=232%:%
%:%676=232%:%
%:%677=233%:%
%:%678=234%:%
%:%679=234%:%
%:%680=235%:%
%:%681=235%:%
%:%682=235%:%
%:%683=236%:%
%:%684=236%:%
%:%685=236%:%
%:%686=237%:%
%:%687=237%:%
%:%688=237%:%
%:%689=238%:%
%:%690=239%:%
%:%691=239%:%
%:%692=240%:%
%:%693=240%:%
%:%694=240%:%
%:%695=241%:%
%:%696=241%:%
%:%697=241%:%
%:%698=242%:%
%:%699=242%:%
%:%700=242%:%
%:%701=243%:%
%:%702=244%:%
%:%703=244%:%
%:%704=245%:%
%:%705=245%:%
%:%706=245%:%
%:%707=246%:%
%:%708=247%:%
%:%714=247%:%
%:%717=248%:%
%:%718=249%:%
%:%719=249%:%
%:%720=250%:%
%:%721=251%:%
%:%722=252%:%
%:%729=253%:%
%:%730=253%:%
%:%731=254%:%
%:%732=254%:%
%:%733=255%:%
%:%734=255%:%
%:%735=255%:%
%:%736=256%:%
%:%737=256%:%
%:%738=257%:%
%:%739=257%:%
%:%740=257%:%
%:%741=258%:%
%:%742=258%:%
%:%743=259%:%
%:%744=259%:%
%:%745=260%:%
%:%755=262%:%
%:%757=263%:%
%:%758=263%:%
%:%761=264%:%
%:%765=264%:%
%:%766=264%:%
%:%775=266%:%
%:%777=267%:%
%:%778=267%:%
%:%779=268%:%
%:%780=269%:%
%:%783=270%:%
%:%787=270%:%
%:%788=270%:%
%:%789=271%:%
%:%790=271%:%
%:%791=272%:%
%:%792=272%:%
%:%793=273%:%
%:%794=273%:%
%:%795=273%:%
%:%796=274%:%
%:%797=274%:%
%:%798=274%:%
%:%799=275%:%
%:%800=275%:%
%:%801=275%:%
%:%802=276%:%
%:%803=276%:%
%:%804=276%:%
%:%805=277%:%
%:%806=277%:%
%:%807=277%:%
%:%808=278%:%
%:%809=278%:%
%:%810=278%:%
%:%811=279%:%
%:%812=279%:%
%:%813=279%:%
%:%814=280%:%
%:%815=280%:%
%:%816=280%:%
%:%817=281%:%
%:%818=281%:%
%:%819=281%:%
%:%820=282%:%
%:%826=282%:%
%:%829=283%:%
%:%830=284%:%
%:%831=284%:%
%:%832=285%:%
%:%833=286%:%
%:%840=287%:%
%:%841=287%:%
%:%842=288%:%
%:%843=288%:%
%:%844=289%:%
%:%845=290%:%
%:%846=290%:%
%:%847=291%:%
%:%848=291%:%
%:%849=291%:%
%:%850=292%:%
%:%851=292%:%
%:%852=293%:%
%:%858=293%:%
%:%861=294%:%
%:%862=295%:%
%:%863=295%:%
%:%864=296%:%
%:%865=297%:%
%:%866=298%:%
%:%869=299%:%
%:%873=299%:%
%:%874=299%:%
%:%875=300%:%
%:%876=300%:%
%:%877=301%:%
%:%878=301%:%
%:%879=302%:%
%:%880=302%:%
%:%881=303%:%
%:%882=303%:%
%:%883=304%:%
%:%884=305%:%
%:%885=305%:%
%:%886=306%:%
%:%887=306%:%
%:%888=306%:%
%:%889=307%:%
%:%890=307%:%
%:%891=308%:%
%:%892=308%:%
%:%893=308%:%
%:%894=309%:%
%:%895=310%:%
%:%896=310%:%
%:%897=311%:%
%:%898=311%:%
%:%899=311%:%
%:%900=312%:%
%:%901=312%:%
%:%902=312%:%
%:%903=313%:%
%:%904=313%:%
%:%905=313%:%
%:%906=314%:%
%:%907=314%:%
%:%908=314%:%
%:%909=315%:%
%:%910=315%:%
%:%911=315%:%
%:%912=316%:%
%:%913=316%:%
%:%914=317%:%
%:%915=317%:%
%:%916=317%:%
%:%917=318%:%
%:%918=318%:%
%:%919=318%:%
%:%920=319%:%
%:%921=319%:%
%:%922=319%:%
%:%923=320%:%
%:%924=320%:%
%:%925=321%:%
%:%926=321%:%
%:%927=321%:%
%:%928=322%:%
%:%929=322%:%
%:%930=323%:%
%:%945=325%:%
%:%949=327%:%
%:%959=329%:%
%:%960=329%:%
%:%961=330%:%
%:%962=331%:%
%:%963=332%:%
%:%964=332%:%
%:%967=333%:%
%:%971=333%:%
%:%972=333%:%
%:%973=333%:%
%:%978=333%:%
%:%981=334%:%
%:%982=335%:%
%:%983=335%:%
%:%984=336%:%
%:%985=337%:%
%:%992=338%:%
%:%993=338%:%
%:%994=339%:%
%:%995=339%:%
%:%996=340%:%
%:%997=340%:%
%:%998=340%:%
%:%999=341%:%
%:%1000=341%:%
%:%1001=341%:%
%:%1002=342%:%
%:%1003=342%:%
%:%1004=343%:%
%:%1005=343%:%
%:%1006=343%:%
%:%1007=344%:%
%:%1008=344%:%
%:%1009=344%:%
%:%1010=345%:%
%:%1011=345%:%
%:%1012=345%:%
%:%1013=346%:%
%:%1014=346%:%
%:%1015=346%:%
%:%1016=347%:%
%:%1022=347%:%
%:%1025=348%:%
%:%1026=349%:%
%:%1027=349%:%
%:%1028=350%:%
%:%1029=351%:%
%:%1036=352%:%
%:%1037=352%:%
%:%1038=353%:%
%:%1039=353%:%
%:%1040=354%:%
%:%1041=354%:%
%:%1042=354%:%
%:%1043=354%:%
%:%1044=355%:%
%:%1045=355%:%
%:%1046=355%:%
%:%1047=356%:%
%:%1048=356%:%
%:%1049=357%:%
%:%1050=357%:%
%:%1051=357%:%
%:%1052=358%:%
%:%1053=358%:%
%:%1054=358%:%
%:%1055=359%:%
%:%1056=359%:%
%:%1057=359%:%
%:%1058=360%:%
%:%1059=360%:%
%:%1060=360%:%
%:%1061=361%:%
%:%1062=361%:%
%:%1063=361%:%
%:%1064=362%:%
%:%1065=362%:%
%:%1066=362%:%
%:%1067=363%:%
%:%1073=363%:%
%:%1076=364%:%
%:%1077=365%:%
%:%1078=365%:%
%:%1079=366%:%
%:%1082=367%:%
%:%1086=367%:%
%:%1087=367%:%
%:%1088=368%:%
%:%1093=368%:%
%:%1096=369%:%
%:%1097=370%:%
%:%1098=370%:%
%:%1099=371%:%
%:%1102=372%:%
%:%1106=372%:%
%:%1107=372%:%
%:%1121=374%:%
%:%1131=376%:%
%:%1132=376%:%
%:%1133=377%:%
%:%1140=384%:%
%:%1141=385%:%
%:%1142=386%:%
%:%1143=386%:%
%:%1146=387%:%
%:%1150=387%:%
%:%1151=387%:%
%:%1152=387%:%
%:%1157=387%:%
%:%1160=388%:%
%:%1161=389%:%
%:%1162=389%:%
%:%1163=390%:%
%:%1164=391%:%
%:%1171=392%:%
%:%1172=392%:%
%:%1173=393%:%
%:%1174=393%:%
%:%1175=394%:%
%:%1176=394%:%
%:%1177=395%:%
%:%1178=395%:%
%:%1179=395%:%
%:%1180=396%:%
%:%1181=396%:%
%:%1182=397%:%
%:%1183=397%:%
%:%1184=397%:%
%:%1185=398%:%
%:%1186=398%:%
%:%1187=398%:%
%:%1188=399%:%
%:%1189=399%:%
%:%1190=399%:%
%:%1191=400%:%
%:%1192=400%:%
%:%1193=400%:%
%:%1194=401%:%
%:%1195=401%:%
%:%1196=401%:%
%:%1197=402%:%
%:%1198=402%:%
%:%1199=402%:%
%:%1200=403%:%
%:%1206=403%:%
%:%1209=404%:%
%:%1210=405%:%
%:%1211=405%:%
%:%1212=406%:%
%:%1213=407%:%
%:%1220=408%:%
%:%1221=408%:%
%:%1222=409%:%
%:%1223=409%:%
%:%1224=410%:%
%:%1225=411%:%
%:%1226=411%:%
%:%1227=411%:%
%:%1228=412%:%
%:%1229=412%:%
%:%1230=412%:%
%:%1231=413%:%
%:%1232=413%:%
%:%1233=413%:%
%:%1234=414%:%
%:%1235=414%:%
%:%1236=414%:%
%:%1237=415%:%
%:%1238=415%:%
%:%1239=415%:%
%:%1240=416%:%
%:%1241=416%:%
%:%1242=416%:%
%:%1243=417%:%
%:%1244=417%:%
%:%1245=417%:%
%:%1246=418%:%
%:%1247=418%:%
%:%1248=418%:%
%:%1249=419%:%
%:%1250=419%:%
%:%1251=419%:%
%:%1252=420%:%
%:%1253=420%:%
%:%1254=420%:%
%:%1255=421%:%
%:%1256=421%:%
%:%1257=421%:%
%:%1258=422%:%
%:%1259=422%:%
%:%1260=422%:%
%:%1261=422%:%
%:%1262=422%:%
%:%1263=423%:%
%:%1278=425%:%
%:%1288=427%:%
%:%1289=427%:%
%:%1290=428%:%
%:%1297=435%:%
%:%1298=436%:%
%:%1299=437%:%
%:%1300=437%:%
%:%1303=438%:%
%:%1307=438%:%
%:%1308=438%:%
%:%1309=439%:%
%:%1310=439%:%
%:%1315=439%:%
%:%1318=440%:%
%:%1319=441%:%
%:%1320=441%:%
%:%1321=442%:%
%:%1322=443%:%
%:%1329=444%:%
%:%1330=444%:%
%:%1331=445%:%
%:%1332=445%:%
%:%1334=447%:%
%:%1335=448%:%
%:%1336=448%:%
%:%1337=448%:%
%:%1338=449%:%
%:%1339=449%:%
%:%1340=449%:%
%:%1342=451%:%
%:%1343=452%:%
%:%1344=452%:%
%:%1345=452%:%
%:%1346=453%:%
%:%1347=453%:%
%:%1348=453%:%
%:%1349=454%:%
%:%1350=454%:%
%:%1351=454%:%
%:%1352=455%:%
%:%1353=455%:%
%:%1354=455%:%
%:%1355=456%:%
%:%1356=456%:%
%:%1357=456%:%
%:%1358=457%:%
%:%1359=457%:%
%:%1360=457%:%
%:%1361=458%:%
%:%1362=458%:%
%:%1363=458%:%
%:%1364=459%:%
%:%1370=459%:%
%:%1373=460%:%
%:%1374=461%:%
%:%1375=461%:%
%:%1376=462%:%
%:%1379=463%:%
%:%1383=463%:%
%:%1384=463%:%
%:%1389=463%:%
%:%1392=464%:%
%:%1393=465%:%
%:%1394=465%:%
%:%1395=466%:%
%:%1398=467%:%
%:%1402=467%:%
%:%1403=467%:%
%:%1408=467%:%
%:%1411=468%:%
%:%1412=469%:%
%:%1413=469%:%
%:%1414=470%:%
%:%1417=471%:%
%:%1421=471%:%
%:%1422=471%:%
%:%1427=471%:%
%:%1430=472%:%
%:%1431=473%:%
%:%1432=473%:%
%:%1433=474%:%
%:%1434=475%:%
%:%1441=476%:%
%:%1442=476%:%
%:%1443=477%:%
%:%1444=477%:%
%:%1445=478%:%
%:%1446=479%:%
%:%1447=479%:%
%:%1448=479%:%
%:%1449=480%:%
%:%1450=480%:%
%:%1451=480%:%
%:%1452=481%:%
%:%1453=481%:%
%:%1454=481%:%
%:%1455=482%:%
%:%1456=482%:%
%:%1457=482%:%
%:%1458=483%:%
%:%1459=483%:%
%:%1460=483%:%
%:%1461=484%:%
%:%1462=484%:%
%:%1463=484%:%
%:%1464=485%:%
%:%1465=485%:%
%:%1466=485%:%
%:%1467=486%:%
%:%1468=486%:%
%:%1469=486%:%
%:%1470=487%:%
%:%1471=487%:%
%:%1472=487%:%
%:%1473=488%:%
%:%1474=488%:%
%:%1475=488%:%
%:%1476=489%:%
%:%1477=489%:%
%:%1478=489%:%
%:%1479=490%:%
%:%1480=490%:%
%:%1481=490%:%
%:%1482=490%:%
%:%1483=490%:%
%:%1484=491%:%
%:%1499=493%:%
%:%1509=495%:%
%:%1510=495%:%
%:%1511=496%:%
%:%1512=497%:%
%:%1513=498%:%
%:%1514=499%:%
%:%1515=499%:%
%:%1516=500%:%
%:%1519=501%:%
%:%1523=501%:%
%:%1524=501%:%
%:%1525=502%:%
%:%1526=502%:%
%:%1527=502%:%
%:%1532=502%:%
%:%1535=503%:%
%:%1536=504%:%
%:%1537=504%:%
%:%1538=505%:%
%:%1539=506%:%
%:%1542=507%:%
%:%1546=507%:%
%:%1547=507%:%
%:%1548=508%:%
%:%1549=508%:%
%:%1554=508%:%
%:%1557=509%:%
%:%1558=510%:%
%:%1559=510%:%
%:%1560=511%:%
%:%1561=512%:%
%:%1562=513%:%
%:%1563=514%:%
%:%1564=514%:%
%:%1565=515%:%
%:%1568=516%:%
%:%1572=516%:%
%:%1573=516%:%
%:%1574=517%:%
%:%1575=517%:%
%:%1576=517%:%
%:%1581=517%:%
%:%1584=518%:%
%:%1585=519%:%
%:%1586=519%:%
%:%1587=520%:%
%:%1588=521%:%
%:%1591=522%:%
%:%1595=522%:%
%:%1596=522%:%
%:%1597=523%:%
%:%1598=523%:%
%:%1603=523%:%
%:%1606=524%:%
%:%1607=525%:%
%:%1608=525%:%
%:%1609=526%:%
%:%1610=527%:%
%:%1611=528%:%
%:%1612=528%:%
%:%1613=529%:%
%:%1616=530%:%
%:%1620=530%:%
%:%1621=530%:%
%:%1622=531%:%
%:%1623=531%:%
%:%1628=531%:%
%:%1631=532%:%
%:%1632=533%:%
%:%1633=533%:%
%:%1634=534%:%
%:%1635=535%:%
%:%1638=536%:%
%:%1642=536%:%
%:%1643=536%:%
%:%1644=536%:%
%:%1645=537%:%
%:%1646=537%:%
%:%1651=537%:%
%:%1654=538%:%
%:%1655=539%:%
%:%1656=539%:%
%:%1657=540%:%
%:%1664=541%:%
%:%1665=541%:%
%:%1666=542%:%
%:%1667=542%:%
%:%1668=543%:%
%:%1669=543%:%
%:%1670=544%:%
%:%1671=544%:%
%:%1672=544%:%
%:%1673=545%:%
%:%1674=546%:%
%:%1675=546%:%
%:%1676=546%:%
%:%1677=547%:%
%:%1678=547%:%
%:%1679=548%:%
%:%1680=548%:%
%:%1681=548%:%
%:%1682=549%:%
%:%1683=550%:%
%:%1684=550%:%
%:%1685=550%:%
%:%1686=551%:%
%:%1687=552%:%
%:%1688=552%:%
%:%1689=553%:%
%:%1690=553%:%
%:%1691=553%:%
%:%1692=554%:%
%:%1693=554%:%
%:%1694=554%:%
%:%1695=555%:%
%:%1696=555%:%
%:%1697=555%:%
%:%1698=556%:%
%:%1699=556%:%
%:%1700=556%:%
%:%1701=557%:%
%:%1702=557%:%
%:%1703=557%:%
%:%1704=558%:%
%:%1705=558%:%
%:%1706=558%:%
%:%1707=559%:%
%:%1708=559%:%
%:%1709=559%:%
%:%1710=560%:%
%:%1711=560%:%
%:%1712=560%:%
%:%1713=561%:%
%:%1714=561%:%
%:%1715=561%:%
%:%1716=562%:%
%:%1717=562%:%
%:%1718=563%:%
%:%1719=563%:%
%:%1720=563%:%
%:%1721=564%:%
%:%1722=564%:%
%:%1723=565%:%
%:%1738=567%:%
%:%1748=569%:%
%:%1749=569%:%
%:%1750=570%:%
%:%1751=571%:%
%:%1752=572%:%
%:%1753=573%:%
%:%1754=573%:%
%:%1755=574%:%
%:%1758=575%:%
%:%1762=575%:%
%:%1763=575%:%
%:%1764=576%:%
%:%1765=576%:%
%:%1766=576%:%
%:%1771=576%:%
%:%1774=577%:%
%:%1775=578%:%
%:%1776=578%:%
%:%1777=579%:%
%:%1778=580%:%
%:%1781=581%:%
%:%1785=581%:%
%:%1786=581%:%
%:%1787=582%:%
%:%1788=582%:%
%:%1793=582%:%
%:%1796=583%:%
%:%1797=584%:%
%:%1798=584%:%
%:%1799=585%:%
%:%1800=586%:%
%:%1801=587%:%
%:%1802=588%:%
%:%1803=588%:%
%:%1804=589%:%
%:%1807=590%:%
%:%1811=590%:%
%:%1812=590%:%
%:%1813=591%:%
%:%1814=591%:%
%:%1815=591%:%
%:%1820=591%:%
%:%1823=592%:%
%:%1824=593%:%
%:%1825=593%:%
%:%1826=594%:%
%:%1827=595%:%
%:%1830=596%:%
%:%1834=596%:%
%:%1835=596%:%
%:%1836=596%:%
%:%1837=597%:%
%:%1838=597%:%
%:%1843=597%:%
%:%1846=598%:%
%:%1847=599%:%
%:%1848=599%:%
%:%1849=600%:%
%:%1850=601%:%
%:%1851=602%:%
%:%1852=602%:%
%:%1853=603%:%
%:%1856=604%:%
%:%1860=604%:%
%:%1861=604%:%
%:%1862=605%:%
%:%1863=605%:%
%:%1868=605%:%
%:%1871=606%:%
%:%1872=607%:%
%:%1873=607%:%
%:%1874=608%:%
%:%1875=609%:%
%:%1878=610%:%
%:%1882=610%:%
%:%1883=610%:%
%:%1884=610%:%
%:%1885=611%:%
%:%1886=611%:%
%:%1891=611%:%
%:%1894=612%:%
%:%1895=613%:%
%:%1896=613%:%
%:%1897=614%:%
%:%1904=615%:%
%:%1905=615%:%
%:%1906=616%:%
%:%1907=616%:%
%:%1908=617%:%
%:%1909=617%:%
%:%1910=618%:%
%:%1911=618%:%
%:%1912=618%:%
%:%1913=619%:%
%:%1914=620%:%
%:%1915=620%:%
%:%1916=620%:%
%:%1917=621%:%
%:%1918=621%:%
%:%1919=622%:%
%:%1920=622%:%
%:%1921=622%:%
%:%1922=623%:%
%:%1923=624%:%
%:%1924=624%:%
%:%1925=624%:%
%:%1926=625%:%
%:%1927=626%:%
%:%1928=626%:%
%:%1929=627%:%
%:%1930=627%:%
%:%1931=627%:%
%:%1932=628%:%
%:%1933=628%:%
%:%1934=628%:%
%:%1935=629%:%
%:%1936=629%:%
%:%1937=629%:%
%:%1938=630%:%
%:%1939=630%:%
%:%1940=630%:%
%:%1941=631%:%
%:%1942=631%:%
%:%1943=631%:%
%:%1944=632%:%
%:%1945=632%:%
%:%1946=632%:%
%:%1947=633%:%
%:%1948=633%:%
%:%1949=633%:%
%:%1950=634%:%
%:%1951=634%:%
%:%1952=634%:%
%:%1953=635%:%
%:%1954=635%:%
%:%1955=635%:%
%:%1956=636%:%
%:%1957=636%:%
%:%1958=637%:%
%:%1959=637%:%
%:%1960=637%:%
%:%1961=638%:%
%:%1962=638%:%
%:%1963=639%:%
%:%1978=641%:%
%:%1988=643%:%
%:%1989=643%:%
%:%1990=644%:%
%:%1993=647%:%
%:%1994=648%:%
%:%1995=649%:%
%:%1996=649%:%
%:%1999=650%:%
%:%2003=650%:%
%:%2004=650%:%
%:%2005=650%:%
%:%2010=650%:%
%:%2013=651%:%
%:%2014=652%:%
%:%2015=652%:%
%:%2016=653%:%
%:%2017=654%:%
%:%2024=655%:%
%:%2025=655%:%
%:%2026=656%:%
%:%2027=656%:%
%:%2030=659%:%
%:%2031=660%:%
%:%2032=660%:%
%:%2033=660%:%
%:%2034=660%:%
%:%2035=661%:%
%:%2036=661%:%
%:%2037=661%:%
%:%2038=662%:%
%:%2039=662%:%
%:%2040=662%:%
%:%2041=663%:%
%:%2042=663%:%
%:%2043=663%:%
%:%2044=664%:%
%:%2045=664%:%
%:%2046=664%:%
%:%2047=665%:%
%:%2048=665%:%
%:%2049=665%:%
%:%2050=666%:%
%:%2051=666%:%
%:%2052=666%:%
%:%2053=667%:%
%:%2059=667%:%
%:%2062=668%:%
%:%2063=669%:%
%:%2064=669%:%
%:%2065=670%:%
%:%2068=673%:%
%:%2069=674%:%
%:%2070=675%:%
%:%2071=675%:%
%:%2074=676%:%
%:%2078=676%:%
%:%2079=676%:%
%:%2080=676%:%
%:%2085=676%:%
%:%2088=677%:%
%:%2089=678%:%
%:%2090=678%:%
%:%2091=679%:%
%:%2092=680%:%
%:%2099=681%:%
%:%2100=681%:%
%:%2101=682%:%
%:%2102=682%:%
%:%2104=684%:%
%:%2105=685%:%
%:%2106=685%:%
%:%2107=685%:%
%:%2108=685%:%
%:%2109=686%:%
%:%2110=686%:%
%:%2111=686%:%
%:%2112=687%:%
%:%2113=687%:%
%:%2114=687%:%
%:%2115=688%:%
%:%2116=688%:%
%:%2117=688%:%
%:%2118=689%:%
%:%2119=689%:%
%:%2120=689%:%
%:%2121=690%:%
%:%2122=690%:%
%:%2123=690%:%
%:%2124=691%:%
%:%2125=691%:%
%:%2126=691%:%
%:%2127=692%:%
%:%2133=692%:%
%:%2136=693%:%
%:%2137=694%:%
%:%2138=694%:%
%:%2139=695%:%
%:%2142=698%:%
%:%2143=699%:%
%:%2144=700%:%
%:%2145=700%:%
%:%2148=701%:%
%:%2152=701%:%
%:%2153=701%:%
%:%2154=701%:%
%:%2159=701%:%
%:%2162=702%:%
%:%2163=703%:%
%:%2164=703%:%
%:%2165=704%:%
%:%2166=705%:%
%:%2173=706%:%
%:%2174=706%:%
%:%2175=707%:%
%:%2176=707%:%
%:%2177=708%:%
%:%2178=708%:%
%:%2179=708%:%
%:%2180=708%:%
%:%2181=709%:%
%:%2182=709%:%
%:%2183=709%:%
%:%2184=710%:%
%:%2185=710%:%
%:%2186=710%:%
%:%2187=711%:%
%:%2188=711%:%
%:%2189=711%:%
%:%2190=712%:%
%:%2191=712%:%
%:%2192=712%:%
%:%2193=713%:%
%:%2194=713%:%
%:%2195=713%:%
%:%2196=714%:%
%:%2197=714%:%
%:%2198=714%:%
%:%2199=715%:%
%:%2205=715%:%
%:%2208=716%:%
%:%2209=717%:%
%:%2210=717%:%
%:%2211=718%:%
%:%2214=721%:%
%:%2215=722%:%
%:%2216=723%:%
%:%2217=723%:%
%:%2220=724%:%
%:%2224=724%:%
%:%2225=724%:%
%:%2226=724%:%
%:%2231=724%:%
%:%2234=725%:%
%:%2235=726%:%
%:%2236=726%:%
%:%2237=727%:%
%:%2238=728%:%
%:%2245=729%:%
%:%2246=729%:%
%:%2247=730%:%
%:%2248=730%:%
%:%2249=731%:%
%:%2250=731%:%
%:%2251=731%:%
%:%2252=731%:%
%:%2253=732%:%
%:%2254=732%:%
%:%2255=732%:%
%:%2256=733%:%
%:%2257=733%:%
%:%2258=733%:%
%:%2259=734%:%
%:%2260=734%:%
%:%2261=734%:%
%:%2262=735%:%
%:%2263=735%:%
%:%2264=735%:%
%:%2265=736%:%
%:%2266=736%:%
%:%2267=736%:%
%:%2268=737%:%
%:%2269=737%:%
%:%2270=737%:%
%:%2271=738%:%
%:%2277=738%:%
%:%2282=739%:%
%:%2287=740%:%

%
\begin{isabellebody}%
\setisabellecontext{Terminal}%
%
\isadelimdocument
%
\endisadelimdocument
%
\isatagdocument
%
\isamarkupsection{Terminal Objects and Elements%
}
\isamarkuptrue%
%
\endisatagdocument
{\isafolddocument}%
%
\isadelimdocument
%
\endisadelimdocument
%
\isadelimtheory
%
\endisadelimtheory
%
\isatagtheory
\isacommand{theory}\isamarkupfalse%
\ Terminal\isanewline
\ \ \isakeyword{imports}\ Cfunc\ Product\isanewline
\isakeyword{begin}%
\endisatagtheory
{\isafoldtheory}%
%
\isadelimtheory
%
\endisadelimtheory
%
\begin{isamarkuptext}%
The axiomatization below corresponds to Axiom 3 (Terminal Object) in Halvorson.%
\end{isamarkuptext}\isamarkuptrue%
\isacommand{axiomatization}\isamarkupfalse%
\isanewline
\ \ terminal{\isacharunderscore}{\kern0pt}func\ {\isacharcolon}{\kern0pt}{\isacharcolon}{\kern0pt}\ {\isachardoublequoteopen}cset\ {\isasymRightarrow}\ cfunc{\isachardoublequoteclose}\ {\isacharparenleft}{\kern0pt}{\isachardoublequoteopen}{\isasymbeta}\isactrlbsub {\isacharunderscore}{\kern0pt}\isactrlesub {\isachardoublequoteclose}\ {\isadigit{1}}{\isadigit{0}}{\isadigit{0}}{\isacharparenright}{\kern0pt}\ \isakeyword{and}\isanewline
\ \ one{\isacharunderscore}{\kern0pt}set\ {\isacharcolon}{\kern0pt}{\isacharcolon}{\kern0pt}\ {\isachardoublequoteopen}cset{\isachardoublequoteclose}\ {\isacharparenleft}{\kern0pt}{\isachardoublequoteopen}{\isasymone}{\isachardoublequoteclose}{\isacharparenright}{\kern0pt}\isanewline
\isakeyword{where}\isanewline
\ \ terminal{\isacharunderscore}{\kern0pt}func{\isacharunderscore}{\kern0pt}type{\isacharbrackleft}{\kern0pt}type{\isacharunderscore}{\kern0pt}rule{\isacharbrackright}{\kern0pt}{\isacharcolon}{\kern0pt}\ {\isachardoublequoteopen}{\isasymbeta}\isactrlbsub X\isactrlesub \ {\isacharcolon}{\kern0pt}\ X\ {\isasymrightarrow}\ {\isasymone}{\isachardoublequoteclose}\ \isakeyword{and}\isanewline
\ \ terminal{\isacharunderscore}{\kern0pt}func{\isacharunderscore}{\kern0pt}unique{\isacharcolon}{\kern0pt}\ {\isachardoublequoteopen}h\ {\isacharcolon}{\kern0pt}\ \ X\ {\isasymrightarrow}\ {\isasymone}\ {\isasymLongrightarrow}\ h\ {\isacharequal}{\kern0pt}\ {\isasymbeta}\isactrlbsub X\isactrlesub {\isachardoublequoteclose}\ \isakeyword{and}\isanewline
\ \ one{\isacharunderscore}{\kern0pt}separator{\isacharcolon}{\kern0pt}\ {\isachardoublequoteopen}f\ {\isacharcolon}{\kern0pt}\ X\ {\isasymrightarrow}\ Y\ {\isasymLongrightarrow}\ g\ {\isacharcolon}{\kern0pt}\ X\ {\isasymrightarrow}\ Y\ {\isasymLongrightarrow}\ {\isacharparenleft}{\kern0pt}{\isasymAnd}\ x{\isachardot}{\kern0pt}\ x\ {\isacharcolon}{\kern0pt}\ {\isasymone}\ {\isasymrightarrow}\ X\ {\isasymLongrightarrow}\ f\ {\isasymcirc}\isactrlsub c\ x\ {\isacharequal}{\kern0pt}\ g\ {\isasymcirc}\isactrlsub c\ x{\isacharparenright}{\kern0pt}\ {\isasymLongrightarrow}\ f\ {\isacharequal}{\kern0pt}\ g{\isachardoublequoteclose}\isanewline
\isanewline
\isacommand{lemma}\isamarkupfalse%
\ one{\isacharunderscore}{\kern0pt}separator{\isacharunderscore}{\kern0pt}contrapos{\isacharcolon}{\kern0pt}\isanewline
\ \ \isakeyword{assumes}\ {\isachardoublequoteopen}f\ {\isacharcolon}{\kern0pt}\ X\ {\isasymrightarrow}\ Y{\isachardoublequoteclose}\ {\isachardoublequoteopen}g\ {\isacharcolon}{\kern0pt}\ X\ {\isasymrightarrow}\ Y{\isachardoublequoteclose}\isanewline
\ \ \isakeyword{shows}\ {\isachardoublequoteopen}f\ {\isasymnoteq}\ g\ {\isasymLongrightarrow}\ {\isasymexists}\ x{\isachardot}{\kern0pt}\ x\ {\isacharcolon}{\kern0pt}\ {\isasymone}\ {\isasymrightarrow}\ X\ {\isasymand}\ f\ {\isasymcirc}\isactrlsub c\ x\ {\isasymnoteq}\ g\ {\isasymcirc}\isactrlsub c\ x{\isachardoublequoteclose}\isanewline
%
\isadelimproof
\ \ %
\endisadelimproof
%
\isatagproof
\isacommand{using}\isamarkupfalse%
\ assms\ one{\isacharunderscore}{\kern0pt}separator\ \isacommand{by}\isamarkupfalse%
\ {\isacharparenleft}{\kern0pt}typecheck{\isacharunderscore}{\kern0pt}cfuncs{\isacharcomma}{\kern0pt}\ blast{\isacharparenright}{\kern0pt}%
\endisatagproof
{\isafoldproof}%
%
\isadelimproof
\isanewline
%
\endisadelimproof
\isanewline
\isacommand{lemma}\isamarkupfalse%
\ terminal{\isacharunderscore}{\kern0pt}func{\isacharunderscore}{\kern0pt}comp{\isacharcolon}{\kern0pt}\isanewline
\ \ {\isachardoublequoteopen}x\ {\isacharcolon}{\kern0pt}\ X\ {\isasymrightarrow}\ Y\ {\isasymLongrightarrow}\ {\isasymbeta}\isactrlbsub Y\isactrlesub \ {\isasymcirc}\isactrlsub c\ x\ {\isacharequal}{\kern0pt}\ {\isasymbeta}\isactrlbsub X\isactrlesub {\isachardoublequoteclose}\isanewline
%
\isadelimproof
\ \ %
\endisadelimproof
%
\isatagproof
\isacommand{by}\isamarkupfalse%
\ {\isacharparenleft}{\kern0pt}simp\ add{\isacharcolon}{\kern0pt}\ comp{\isacharunderscore}{\kern0pt}type\ terminal{\isacharunderscore}{\kern0pt}func{\isacharunderscore}{\kern0pt}type\ terminal{\isacharunderscore}{\kern0pt}func{\isacharunderscore}{\kern0pt}unique{\isacharparenright}{\kern0pt}%
\endisatagproof
{\isafoldproof}%
%
\isadelimproof
\isanewline
%
\endisadelimproof
\isanewline
\isacommand{lemma}\isamarkupfalse%
\ terminal{\isacharunderscore}{\kern0pt}func{\isacharunderscore}{\kern0pt}comp{\isacharunderscore}{\kern0pt}elem{\isacharcolon}{\kern0pt}\isanewline
\ \ {\isachardoublequoteopen}x\ {\isacharcolon}{\kern0pt}\ {\isasymone}\ {\isasymrightarrow}\ X\ {\isasymLongrightarrow}\ {\isasymbeta}\isactrlbsub X\isactrlesub \ {\isasymcirc}\isactrlsub c\ x\ {\isacharequal}{\kern0pt}\ id\ {\isasymone}{\isachardoublequoteclose}\isanewline
%
\isadelimproof
\ \ %
\endisadelimproof
%
\isatagproof
\isacommand{by}\isamarkupfalse%
\ {\isacharparenleft}{\kern0pt}metis\ id{\isacharunderscore}{\kern0pt}type\ terminal{\isacharunderscore}{\kern0pt}func{\isacharunderscore}{\kern0pt}comp\ terminal{\isacharunderscore}{\kern0pt}func{\isacharunderscore}{\kern0pt}unique{\isacharparenright}{\kern0pt}%
\endisatagproof
{\isafoldproof}%
%
\isadelimproof
%
\endisadelimproof
%
\isadelimdocument
%
\endisadelimdocument
%
\isatagdocument
%
\isamarkupsubsection{Set Membership and Emptiness%
}
\isamarkuptrue%
%
\endisatagdocument
{\isafolddocument}%
%
\isadelimdocument
%
\endisadelimdocument
%
\begin{isamarkuptext}%
The abbreviation below captures Definition 2.1.16 in Halvorson.%
\end{isamarkuptext}\isamarkuptrue%
\isacommand{abbreviation}\isamarkupfalse%
\ member\ {\isacharcolon}{\kern0pt}{\isacharcolon}{\kern0pt}\ {\isachardoublequoteopen}cfunc\ {\isasymRightarrow}\ cset\ {\isasymRightarrow}\ bool{\isachardoublequoteclose}\ {\isacharparenleft}{\kern0pt}\isakeyword{infix}\ {\isachardoublequoteopen}{\isasymin}\isactrlsub c{\isachardoublequoteclose}\ {\isadigit{5}}{\isadigit{0}}{\isacharparenright}{\kern0pt}\ \isakeyword{where}\isanewline
\ \ {\isachardoublequoteopen}x\ {\isasymin}\isactrlsub c\ X\ {\isasymequiv}\ {\isacharparenleft}{\kern0pt}x\ {\isacharcolon}{\kern0pt}\ {\isasymone}\ {\isasymrightarrow}\ X{\isacharparenright}{\kern0pt}{\isachardoublequoteclose}\isanewline
\isanewline
\isacommand{definition}\isamarkupfalse%
\ nonempty\ {\isacharcolon}{\kern0pt}{\isacharcolon}{\kern0pt}\ {\isachardoublequoteopen}cset\ {\isasymRightarrow}\ bool{\isachardoublequoteclose}\ \isakeyword{where}\isanewline
\ \ {\isachardoublequoteopen}nonempty\ X\ {\isasymequiv}\ {\isacharparenleft}{\kern0pt}{\isasymexists}x{\isachardot}{\kern0pt}\ x\ {\isasymin}\isactrlsub c\ X{\isacharparenright}{\kern0pt}{\isachardoublequoteclose}\isanewline
\isanewline
\isacommand{definition}\isamarkupfalse%
\ is{\isacharunderscore}{\kern0pt}empty\ {\isacharcolon}{\kern0pt}{\isacharcolon}{\kern0pt}\ {\isachardoublequoteopen}cset\ {\isasymRightarrow}\ bool{\isachardoublequoteclose}\ \isakeyword{where}\isanewline
\ \ {\isachardoublequoteopen}is{\isacharunderscore}{\kern0pt}empty\ X\ {\isasymequiv}\ {\isasymnot}{\isacharparenleft}{\kern0pt}{\isasymexists}x{\isachardot}{\kern0pt}\ x\ {\isasymin}\isactrlsub c\ X{\isacharparenright}{\kern0pt}{\isachardoublequoteclose}%
\begin{isamarkuptext}%
The lemma below corresponds to Exercise 2.1.18 in Halvorson.%
\end{isamarkuptext}\isamarkuptrue%
\isacommand{lemma}\isamarkupfalse%
\ element{\isacharunderscore}{\kern0pt}monomorphism{\isacharcolon}{\kern0pt}\isanewline
\ \ {\isachardoublequoteopen}x\ {\isasymin}\isactrlsub c\ X\ {\isasymLongrightarrow}\ monomorphism\ x{\isachardoublequoteclose}\isanewline
%
\isadelimproof
\ \ %
\endisadelimproof
%
\isatagproof
\isacommand{unfolding}\isamarkupfalse%
\ monomorphism{\isacharunderscore}{\kern0pt}def\isanewline
\ \ \isacommand{by}\isamarkupfalse%
\ {\isacharparenleft}{\kern0pt}metis\ cfunc{\isacharunderscore}{\kern0pt}type{\isacharunderscore}{\kern0pt}def\ domain{\isacharunderscore}{\kern0pt}comp\ terminal{\isacharunderscore}{\kern0pt}func{\isacharunderscore}{\kern0pt}unique{\isacharparenright}{\kern0pt}%
\endisatagproof
{\isafoldproof}%
%
\isadelimproof
\isanewline
%
\endisadelimproof
\isanewline
\isacommand{lemma}\isamarkupfalse%
\ one{\isacharunderscore}{\kern0pt}unique{\isacharunderscore}{\kern0pt}element{\isacharcolon}{\kern0pt}\isanewline
\ \ {\isachardoublequoteopen}{\isasymexists}{\isacharbang}{\kern0pt}\ x{\isachardot}{\kern0pt}\ x\ {\isasymin}\isactrlsub c\ {\isasymone}{\isachardoublequoteclose}\isanewline
%
\isadelimproof
\ \ %
\endisadelimproof
%
\isatagproof
\isacommand{using}\isamarkupfalse%
\ terminal{\isacharunderscore}{\kern0pt}func{\isacharunderscore}{\kern0pt}type\ terminal{\isacharunderscore}{\kern0pt}func{\isacharunderscore}{\kern0pt}unique\ \isacommand{by}\isamarkupfalse%
\ blast%
\endisatagproof
{\isafoldproof}%
%
\isadelimproof
\isanewline
%
\endisadelimproof
\isanewline
\isacommand{lemma}\isamarkupfalse%
\ prod{\isacharunderscore}{\kern0pt}with{\isacharunderscore}{\kern0pt}empty{\isacharunderscore}{\kern0pt}is{\isacharunderscore}{\kern0pt}empty{\isadigit{1}}{\isacharcolon}{\kern0pt}\isanewline
\ \ \isakeyword{assumes}\ {\isachardoublequoteopen}is{\isacharunderscore}{\kern0pt}empty\ {\isacharparenleft}{\kern0pt}A{\isacharparenright}{\kern0pt}{\isachardoublequoteclose}\isanewline
\ \ \isakeyword{shows}\ {\isachardoublequoteopen}is{\isacharunderscore}{\kern0pt}empty{\isacharparenleft}{\kern0pt}A\ {\isasymtimes}\isactrlsub c\ B{\isacharparenright}{\kern0pt}{\isachardoublequoteclose}\isanewline
%
\isadelimproof
\ \ %
\endisadelimproof
%
\isatagproof
\isacommand{by}\isamarkupfalse%
\ {\isacharparenleft}{\kern0pt}meson\ assms\ comp{\isacharunderscore}{\kern0pt}type\ left{\isacharunderscore}{\kern0pt}cart{\isacharunderscore}{\kern0pt}proj{\isacharunderscore}{\kern0pt}type\ is{\isacharunderscore}{\kern0pt}empty{\isacharunderscore}{\kern0pt}def{\isacharparenright}{\kern0pt}%
\endisatagproof
{\isafoldproof}%
%
\isadelimproof
\isanewline
%
\endisadelimproof
\isanewline
\isacommand{lemma}\isamarkupfalse%
\ prod{\isacharunderscore}{\kern0pt}with{\isacharunderscore}{\kern0pt}empty{\isacharunderscore}{\kern0pt}is{\isacharunderscore}{\kern0pt}empty{\isadigit{2}}{\isacharcolon}{\kern0pt}\isanewline
\ \ \isakeyword{assumes}\ {\isachardoublequoteopen}is{\isacharunderscore}{\kern0pt}empty\ {\isacharparenleft}{\kern0pt}B{\isacharparenright}{\kern0pt}{\isachardoublequoteclose}\isanewline
\ \ \isakeyword{shows}\ {\isachardoublequoteopen}is{\isacharunderscore}{\kern0pt}empty\ {\isacharparenleft}{\kern0pt}A\ {\isasymtimes}\isactrlsub c\ B{\isacharparenright}{\kern0pt}{\isachardoublequoteclose}\isanewline
%
\isadelimproof
\ \ %
\endisadelimproof
%
\isatagproof
\isacommand{using}\isamarkupfalse%
\ assms\ cart{\isacharunderscore}{\kern0pt}prod{\isacharunderscore}{\kern0pt}decomp\ is{\isacharunderscore}{\kern0pt}empty{\isacharunderscore}{\kern0pt}def\ \isacommand{by}\isamarkupfalse%
\ blast%
\endisatagproof
{\isafoldproof}%
%
\isadelimproof
%
\endisadelimproof
%
\isadelimdocument
%
\endisadelimdocument
%
\isatagdocument
%
\isamarkupsubsection{Terminal Objects (sets with one element)%
}
\isamarkuptrue%
%
\endisatagdocument
{\isafolddocument}%
%
\isadelimdocument
%
\endisadelimdocument
\isacommand{definition}\isamarkupfalse%
\ terminal{\isacharunderscore}{\kern0pt}object\ {\isacharcolon}{\kern0pt}{\isacharcolon}{\kern0pt}\ {\isachardoublequoteopen}cset\ {\isasymRightarrow}\ bool{\isachardoublequoteclose}\ \isakeyword{where}\isanewline
\ \ {\isachardoublequoteopen}terminal{\isacharunderscore}{\kern0pt}object\ X\ {\isasymlongleftrightarrow}\ {\isacharparenleft}{\kern0pt}{\isasymforall}\ Y{\isachardot}{\kern0pt}\ {\isasymexists}{\isacharbang}{\kern0pt}\ f{\isachardot}{\kern0pt}\ f\ {\isacharcolon}{\kern0pt}\ Y\ {\isasymrightarrow}\ X{\isacharparenright}{\kern0pt}{\isachardoublequoteclose}\isanewline
\isanewline
\isacommand{lemma}\isamarkupfalse%
\ one{\isacharunderscore}{\kern0pt}terminal{\isacharunderscore}{\kern0pt}object{\isacharcolon}{\kern0pt}\ {\isachardoublequoteopen}terminal{\isacharunderscore}{\kern0pt}object{\isacharparenleft}{\kern0pt}{\isasymone}{\isacharparenright}{\kern0pt}{\isachardoublequoteclose}\isanewline
%
\isadelimproof
\ \ %
\endisadelimproof
%
\isatagproof
\isacommand{unfolding}\isamarkupfalse%
\ terminal{\isacharunderscore}{\kern0pt}object{\isacharunderscore}{\kern0pt}def\ \isacommand{using}\isamarkupfalse%
\ terminal{\isacharunderscore}{\kern0pt}func{\isacharunderscore}{\kern0pt}type\ terminal{\isacharunderscore}{\kern0pt}func{\isacharunderscore}{\kern0pt}unique\ \isacommand{by}\isamarkupfalse%
\ blast%
\endisatagproof
{\isafoldproof}%
%
\isadelimproof
%
\endisadelimproof
%
\begin{isamarkuptext}%
The lemma below is a generalisation of \isa{{\isacharquery}{\kern0pt}x\ {\isasymin}\isactrlsub c\ {\isacharquery}{\kern0pt}X\ {\isasymLongrightarrow}\ monomorphism\ {\isacharquery}{\kern0pt}x}%
\end{isamarkuptext}\isamarkuptrue%
\isacommand{lemma}\isamarkupfalse%
\ terminal{\isacharunderscore}{\kern0pt}el{\isacharunderscore}{\kern0pt}monomorphism{\isacharcolon}{\kern0pt}\isanewline
\ \ \isakeyword{assumes}\ {\isachardoublequoteopen}x\ {\isacharcolon}{\kern0pt}\ T\ {\isasymrightarrow}\ X{\isachardoublequoteclose}\isanewline
\ \ \isakeyword{assumes}\ {\isachardoublequoteopen}terminal{\isacharunderscore}{\kern0pt}object\ T{\isachardoublequoteclose}\isanewline
\ \ \isakeyword{shows}\ {\isachardoublequoteopen}monomorphism\ x{\isachardoublequoteclose}\isanewline
%
\isadelimproof
\ \ %
\endisadelimproof
%
\isatagproof
\isacommand{unfolding}\isamarkupfalse%
\ monomorphism{\isacharunderscore}{\kern0pt}def\isanewline
\ \ \isacommand{by}\isamarkupfalse%
\ {\isacharparenleft}{\kern0pt}metis\ assms\ cfunc{\isacharunderscore}{\kern0pt}type{\isacharunderscore}{\kern0pt}def\ domain{\isacharunderscore}{\kern0pt}comp\ terminal{\isacharunderscore}{\kern0pt}object{\isacharunderscore}{\kern0pt}def{\isacharparenright}{\kern0pt}%
\endisatagproof
{\isafoldproof}%
%
\isadelimproof
%
\endisadelimproof
%
\begin{isamarkuptext}%
The lemma below corresponds to Exercise 2.1.15 in Halvorson.%
\end{isamarkuptext}\isamarkuptrue%
\isacommand{lemma}\isamarkupfalse%
\ terminal{\isacharunderscore}{\kern0pt}objects{\isacharunderscore}{\kern0pt}isomorphic{\isacharcolon}{\kern0pt}\isanewline
\ \ \isakeyword{assumes}\ {\isachardoublequoteopen}terminal{\isacharunderscore}{\kern0pt}object\ X{\isachardoublequoteclose}\ {\isachardoublequoteopen}terminal{\isacharunderscore}{\kern0pt}object\ Y{\isachardoublequoteclose}\isanewline
\ \ \isakeyword{shows}\ {\isachardoublequoteopen}X\ {\isasymcong}\ Y{\isachardoublequoteclose}\isanewline
%
\isadelimproof
\ \ %
\endisadelimproof
%
\isatagproof
\isacommand{unfolding}\isamarkupfalse%
\ is{\isacharunderscore}{\kern0pt}isomorphic{\isacharunderscore}{\kern0pt}def\isanewline
\isacommand{proof}\isamarkupfalse%
\ {\isacharminus}{\kern0pt}\isanewline
\ \ \isacommand{obtain}\isamarkupfalse%
\ f\ \isakeyword{where}\ f{\isacharunderscore}{\kern0pt}type{\isacharcolon}{\kern0pt}\ {\isachardoublequoteopen}f\ {\isacharcolon}{\kern0pt}\ X\ {\isasymrightarrow}\ Y{\isachardoublequoteclose}\ \isakeyword{and}\ f{\isacharunderscore}{\kern0pt}unique{\isacharcolon}{\kern0pt}\ {\isachardoublequoteopen}{\isasymforall}g{\isachardot}{\kern0pt}\ g\ {\isacharcolon}{\kern0pt}\ X\ {\isasymrightarrow}\ Y\ {\isasymlongrightarrow}\ f\ {\isacharequal}{\kern0pt}\ g{\isachardoublequoteclose}\isanewline
\ \ \ \ \isacommand{using}\isamarkupfalse%
\ assms{\isacharparenleft}{\kern0pt}{\isadigit{2}}{\isacharparenright}{\kern0pt}\ terminal{\isacharunderscore}{\kern0pt}object{\isacharunderscore}{\kern0pt}def\ \isacommand{by}\isamarkupfalse%
\ force\isanewline
\isanewline
\ \ \isacommand{obtain}\isamarkupfalse%
\ g\ \isakeyword{where}\ g{\isacharunderscore}{\kern0pt}type{\isacharcolon}{\kern0pt}\ {\isachardoublequoteopen}g\ {\isacharcolon}{\kern0pt}\ Y\ {\isasymrightarrow}\ X{\isachardoublequoteclose}\ \isakeyword{and}\ g{\isacharunderscore}{\kern0pt}unique{\isacharcolon}{\kern0pt}\ {\isachardoublequoteopen}{\isasymforall}f{\isachardot}{\kern0pt}\ f\ {\isacharcolon}{\kern0pt}\ Y\ {\isasymrightarrow}\ X\ {\isasymlongrightarrow}\ g\ {\isacharequal}{\kern0pt}\ f{\isachardoublequoteclose}\isanewline
\ \ \ \ \isacommand{using}\isamarkupfalse%
\ assms{\isacharparenleft}{\kern0pt}{\isadigit{1}}{\isacharparenright}{\kern0pt}\ terminal{\isacharunderscore}{\kern0pt}object{\isacharunderscore}{\kern0pt}def\ \isacommand{by}\isamarkupfalse%
\ force\isanewline
\isanewline
\ \ \isacommand{have}\isamarkupfalse%
\ g{\isacharunderscore}{\kern0pt}f{\isacharunderscore}{\kern0pt}is{\isacharunderscore}{\kern0pt}id{\isacharcolon}{\kern0pt}\ {\isachardoublequoteopen}g\ {\isasymcirc}\isactrlsub c\ f\ {\isacharequal}{\kern0pt}\ id\ X{\isachardoublequoteclose}\isanewline
\ \ \ \ \isacommand{using}\isamarkupfalse%
\ assms{\isacharparenleft}{\kern0pt}{\isadigit{1}}{\isacharparenright}{\kern0pt}\ comp{\isacharunderscore}{\kern0pt}type\ f{\isacharunderscore}{\kern0pt}type\ g{\isacharunderscore}{\kern0pt}type\ id{\isacharunderscore}{\kern0pt}type\ terminal{\isacharunderscore}{\kern0pt}object{\isacharunderscore}{\kern0pt}def\ \isacommand{by}\isamarkupfalse%
\ blast\isanewline
\isanewline
\ \ \isacommand{have}\isamarkupfalse%
\ f{\isacharunderscore}{\kern0pt}g{\isacharunderscore}{\kern0pt}is{\isacharunderscore}{\kern0pt}id{\isacharcolon}{\kern0pt}\ {\isachardoublequoteopen}f\ {\isasymcirc}\isactrlsub c\ g\ {\isacharequal}{\kern0pt}\ id\ Y{\isachardoublequoteclose}\isanewline
\ \ \ \ \isacommand{using}\isamarkupfalse%
\ assms{\isacharparenleft}{\kern0pt}{\isadigit{2}}{\isacharparenright}{\kern0pt}\ comp{\isacharunderscore}{\kern0pt}type\ f{\isacharunderscore}{\kern0pt}type\ g{\isacharunderscore}{\kern0pt}type\ id{\isacharunderscore}{\kern0pt}type\ terminal{\isacharunderscore}{\kern0pt}object{\isacharunderscore}{\kern0pt}def\ \isacommand{by}\isamarkupfalse%
\ blast\isanewline
\isanewline
\ \ \isacommand{have}\isamarkupfalse%
\ f{\isacharunderscore}{\kern0pt}isomorphism{\isacharcolon}{\kern0pt}\ {\isachardoublequoteopen}isomorphism\ f{\isachardoublequoteclose}\isanewline
\ \ \ \ \isacommand{unfolding}\isamarkupfalse%
\ isomorphism{\isacharunderscore}{\kern0pt}def\isanewline
\ \ \ \ \isacommand{using}\isamarkupfalse%
\ cfunc{\isacharunderscore}{\kern0pt}type{\isacharunderscore}{\kern0pt}def\ f{\isacharunderscore}{\kern0pt}type\ g{\isacharunderscore}{\kern0pt}type\ g{\isacharunderscore}{\kern0pt}f{\isacharunderscore}{\kern0pt}is{\isacharunderscore}{\kern0pt}id\ f{\isacharunderscore}{\kern0pt}g{\isacharunderscore}{\kern0pt}is{\isacharunderscore}{\kern0pt}id\isanewline
\ \ \ \ \isacommand{by}\isamarkupfalse%
\ {\isacharparenleft}{\kern0pt}rule{\isacharunderscore}{\kern0pt}tac\ x{\isacharequal}{\kern0pt}g\ \isakeyword{in}\ exI{\isacharcomma}{\kern0pt}\ auto{\isacharparenright}{\kern0pt}\isanewline
\isanewline
\ \ \isacommand{show}\isamarkupfalse%
\ {\isachardoublequoteopen}{\isasymexists}f{\isachardot}{\kern0pt}\ f\ {\isacharcolon}{\kern0pt}\ X\ {\isasymrightarrow}\ Y\ {\isasymand}\ isomorphism\ f{\isachardoublequoteclose}\isanewline
\ \ \ \ \isacommand{using}\isamarkupfalse%
\ f{\isacharunderscore}{\kern0pt}isomorphism\ f{\isacharunderscore}{\kern0pt}type\ \isacommand{by}\isamarkupfalse%
\ auto\isanewline
\isacommand{qed}\isamarkupfalse%
%
\endisatagproof
{\isafoldproof}%
%
\isadelimproof
%
\endisadelimproof
%
\begin{isamarkuptext}%
The two lemmas below show the converse to Exercise 2.1.15 in Halvorson.%
\end{isamarkuptext}\isamarkuptrue%
\isacommand{lemma}\isamarkupfalse%
\ iso{\isacharunderscore}{\kern0pt}to{\isadigit{1}}{\isacharunderscore}{\kern0pt}is{\isacharunderscore}{\kern0pt}term{\isacharcolon}{\kern0pt}\isanewline
\ \ \isakeyword{assumes}\ {\isachardoublequoteopen}X\ {\isasymcong}\ {\isasymone}{\isachardoublequoteclose}\isanewline
\ \ \isakeyword{shows}\ {\isachardoublequoteopen}terminal{\isacharunderscore}{\kern0pt}object\ X{\isachardoublequoteclose}\isanewline
%
\isadelimproof
\ \ %
\endisadelimproof
%
\isatagproof
\isacommand{unfolding}\isamarkupfalse%
\ terminal{\isacharunderscore}{\kern0pt}object{\isacharunderscore}{\kern0pt}def\isanewline
\isacommand{proof}\isamarkupfalse%
\ \isanewline
\ \ \isacommand{fix}\isamarkupfalse%
\ Y\isanewline
\ \ \isacommand{obtain}\isamarkupfalse%
\ x\ \isakeyword{where}\ x{\isacharunderscore}{\kern0pt}type{\isacharbrackleft}{\kern0pt}type{\isacharunderscore}{\kern0pt}rule{\isacharbrackright}{\kern0pt}{\isacharcolon}{\kern0pt}\ {\isachardoublequoteopen}x\ {\isacharcolon}{\kern0pt}\ {\isasymone}\ {\isasymrightarrow}\ X{\isachardoublequoteclose}\ \isakeyword{and}\ x{\isacharunderscore}{\kern0pt}unique{\isacharcolon}{\kern0pt}\ {\isachardoublequoteopen}{\isasymforall}\ y{\isachardot}{\kern0pt}\ y\ {\isacharcolon}{\kern0pt}\ {\isasymone}\ {\isasymrightarrow}\ X\ {\isasymlongrightarrow}\ x\ {\isacharequal}{\kern0pt}\ y{\isachardoublequoteclose}\isanewline
\ \ \ \ \isacommand{by}\isamarkupfalse%
\ {\isacharparenleft}{\kern0pt}smt\ assms\ is{\isacharunderscore}{\kern0pt}isomorphic{\isacharunderscore}{\kern0pt}def\ iso{\isacharunderscore}{\kern0pt}imp{\isacharunderscore}{\kern0pt}epi{\isacharunderscore}{\kern0pt}and{\isacharunderscore}{\kern0pt}monic\ isomorphic{\isacharunderscore}{\kern0pt}is{\isacharunderscore}{\kern0pt}symmetric\ monomorphism{\isacharunderscore}{\kern0pt}def{\isadigit{2}}\ terminal{\isacharunderscore}{\kern0pt}func{\isacharunderscore}{\kern0pt}comp\ terminal{\isacharunderscore}{\kern0pt}func{\isacharunderscore}{\kern0pt}unique{\isacharparenright}{\kern0pt}\isanewline
\ \ \isacommand{show}\isamarkupfalse%
\ \ {\isachardoublequoteopen}{\isasymexists}{\isacharbang}{\kern0pt}f{\isachardot}{\kern0pt}\ f\ {\isacharcolon}{\kern0pt}\ Y\ {\isasymrightarrow}\ X{\isachardoublequoteclose}\isanewline
\ \ \isacommand{proof}\isamarkupfalse%
\ {\isacharparenleft}{\kern0pt}rule{\isacharunderscore}{\kern0pt}tac\ a{\isacharequal}{\kern0pt}{\isachardoublequoteopen}x\ {\isasymcirc}\isactrlsub c\ {\isasymbeta}\isactrlbsub Y\isactrlesub {\isachardoublequoteclose}\ \isakeyword{in}\ ex{\isadigit{1}}I{\isacharparenright}{\kern0pt}\isanewline
\ \ \ \ \isacommand{show}\isamarkupfalse%
\ {\isachardoublequoteopen}x\ {\isasymcirc}\isactrlsub c\ {\isasymbeta}\isactrlbsub Y\isactrlesub \ {\isacharcolon}{\kern0pt}\ Y\ {\isasymrightarrow}\ X{\isachardoublequoteclose}\isanewline
\ \ \ \ \ \ \isacommand{by}\isamarkupfalse%
\ typecheck{\isacharunderscore}{\kern0pt}cfuncs\isanewline
\ \ \isacommand{next}\isamarkupfalse%
\isanewline
\ \ \ \ \isacommand{fix}\isamarkupfalse%
\ xa\isanewline
\ \ \ \ \isacommand{assume}\isamarkupfalse%
\ xa{\isacharunderscore}{\kern0pt}type{\isacharcolon}{\kern0pt}\ {\isachardoublequoteopen}xa\ {\isacharcolon}{\kern0pt}\ Y\ {\isasymrightarrow}\ X{\isachardoublequoteclose}\isanewline
\ \ \ \ \isacommand{show}\isamarkupfalse%
\ {\isachardoublequoteopen}xa\ {\isacharequal}{\kern0pt}\ x\ {\isasymcirc}\isactrlsub c\ {\isasymbeta}\isactrlbsub Y\isactrlesub {\isachardoublequoteclose}\isanewline
\ \ \ \ \isacommand{proof}\isamarkupfalse%
\ {\isacharparenleft}{\kern0pt}rule\ ccontr{\isacharparenright}{\kern0pt}\isanewline
\ \ \ \ \ \ \isacommand{assume}\isamarkupfalse%
\ {\isachardoublequoteopen}xa\ {\isasymnoteq}\ x\ {\isasymcirc}\isactrlsub c\ {\isasymbeta}\isactrlbsub Y\isactrlesub {\isachardoublequoteclose}\isanewline
\ \ \ \ \ \ \isacommand{then}\isamarkupfalse%
\ \isacommand{obtain}\isamarkupfalse%
\ y\ \isakeyword{where}\ elems{\isacharunderscore}{\kern0pt}neq{\isacharcolon}{\kern0pt}\ {\isachardoublequoteopen}xa\ {\isasymcirc}\isactrlsub c\ y\ {\isasymnoteq}\ {\isacharparenleft}{\kern0pt}x\ {\isasymcirc}\isactrlsub c\ {\isasymbeta}\isactrlbsub Y\isactrlesub {\isacharparenright}{\kern0pt}\ {\isasymcirc}\isactrlsub c\ y{\isachardoublequoteclose}\ \isakeyword{and}\ y{\isacharunderscore}{\kern0pt}type{\isacharcolon}{\kern0pt}\ {\isachardoublequoteopen}y\ {\isacharcolon}{\kern0pt}\ {\isasymone}\ {\isasymrightarrow}\ Y{\isachardoublequoteclose}\isanewline
\ \ \ \ \ \ \ \ \isacommand{using}\isamarkupfalse%
\ one{\isacharunderscore}{\kern0pt}separator{\isacharunderscore}{\kern0pt}contrapos\ comp{\isacharunderscore}{\kern0pt}type\ terminal{\isacharunderscore}{\kern0pt}func{\isacharunderscore}{\kern0pt}type\ x{\isacharunderscore}{\kern0pt}type\ xa{\isacharunderscore}{\kern0pt}type\ \isacommand{by}\isamarkupfalse%
\ blast\isanewline
\ \ \ \ \ \ \isacommand{then}\isamarkupfalse%
\ \isacommand{show}\isamarkupfalse%
\ False\isanewline
\ \ \ \ \ \ \ \ \isacommand{by}\isamarkupfalse%
\ {\isacharparenleft}{\kern0pt}smt\ {\isacharparenleft}{\kern0pt}z{\isadigit{3}}{\isacharparenright}{\kern0pt}\ comp{\isacharunderscore}{\kern0pt}type\ elems{\isacharunderscore}{\kern0pt}neq\ terminal{\isacharunderscore}{\kern0pt}func{\isacharunderscore}{\kern0pt}type\ x{\isacharunderscore}{\kern0pt}unique\ xa{\isacharunderscore}{\kern0pt}type\ y{\isacharunderscore}{\kern0pt}type{\isacharparenright}{\kern0pt}\ \ \ \ \ \isanewline
\ \ \ \ \isacommand{qed}\isamarkupfalse%
\isanewline
\ \ \isacommand{qed}\isamarkupfalse%
\isanewline
\isacommand{qed}\isamarkupfalse%
%
\endisatagproof
{\isafoldproof}%
%
\isadelimproof
\isanewline
%
\endisadelimproof
\isanewline
\isacommand{lemma}\isamarkupfalse%
\ iso{\isacharunderscore}{\kern0pt}to{\isacharunderscore}{\kern0pt}term{\isacharunderscore}{\kern0pt}is{\isacharunderscore}{\kern0pt}term{\isacharcolon}{\kern0pt}\isanewline
\ \ \isakeyword{assumes}\ {\isachardoublequoteopen}X\ {\isasymcong}\ Y{\isachardoublequoteclose}\isanewline
\ \ \isakeyword{assumes}\ {\isachardoublequoteopen}terminal{\isacharunderscore}{\kern0pt}object\ Y{\isachardoublequoteclose}\isanewline
\ \ \isakeyword{shows}\ {\isachardoublequoteopen}terminal{\isacharunderscore}{\kern0pt}object\ X{\isachardoublequoteclose}\isanewline
%
\isadelimproof
\ \ %
\endisadelimproof
%
\isatagproof
\isacommand{by}\isamarkupfalse%
\ {\isacharparenleft}{\kern0pt}meson\ assms\ iso{\isacharunderscore}{\kern0pt}to{\isadigit{1}}{\isacharunderscore}{\kern0pt}is{\isacharunderscore}{\kern0pt}term\ isomorphic{\isacharunderscore}{\kern0pt}is{\isacharunderscore}{\kern0pt}transitive\ one{\isacharunderscore}{\kern0pt}terminal{\isacharunderscore}{\kern0pt}object\ terminal{\isacharunderscore}{\kern0pt}objects{\isacharunderscore}{\kern0pt}isomorphic{\isacharparenright}{\kern0pt}%
\endisatagproof
{\isafoldproof}%
%
\isadelimproof
%
\endisadelimproof
%
\begin{isamarkuptext}%
The lemma below corresponds to Proposition 2.1.19 in Halvorson.%
\end{isamarkuptext}\isamarkuptrue%
\isacommand{lemma}\isamarkupfalse%
\ single{\isacharunderscore}{\kern0pt}elem{\isacharunderscore}{\kern0pt}iso{\isacharunderscore}{\kern0pt}one{\isacharcolon}{\kern0pt}\isanewline
\ \ {\isachardoublequoteopen}{\isacharparenleft}{\kern0pt}{\isasymexists}{\isacharbang}{\kern0pt}\ x{\isachardot}{\kern0pt}\ x\ {\isasymin}\isactrlsub c\ X{\isacharparenright}{\kern0pt}\ {\isasymlongleftrightarrow}\ X\ {\isasymcong}\ {\isasymone}{\isachardoublequoteclose}\isanewline
%
\isadelimproof
%
\endisadelimproof
%
\isatagproof
\isacommand{proof}\isamarkupfalse%
\isanewline
\ \ \isacommand{assume}\isamarkupfalse%
\ X{\isacharunderscore}{\kern0pt}iso{\isacharunderscore}{\kern0pt}one{\isacharcolon}{\kern0pt}\ {\isachardoublequoteopen}X\ {\isasymcong}\ {\isasymone}{\isachardoublequoteclose}\isanewline
\ \ \isacommand{then}\isamarkupfalse%
\ \isacommand{have}\isamarkupfalse%
\ {\isachardoublequoteopen}{\isasymone}\ {\isasymcong}\ X{\isachardoublequoteclose}\isanewline
\ \ \ \ \isacommand{by}\isamarkupfalse%
\ {\isacharparenleft}{\kern0pt}simp\ add{\isacharcolon}{\kern0pt}\ isomorphic{\isacharunderscore}{\kern0pt}is{\isacharunderscore}{\kern0pt}symmetric{\isacharparenright}{\kern0pt}\isanewline
\ \ \isacommand{then}\isamarkupfalse%
\ \isacommand{obtain}\isamarkupfalse%
\ f\ \isakeyword{where}\ f{\isacharunderscore}{\kern0pt}type{\isacharcolon}{\kern0pt}\ {\isachardoublequoteopen}f\ {\isacharcolon}{\kern0pt}\ {\isasymone}\ {\isasymrightarrow}\ X{\isachardoublequoteclose}\ \isakeyword{and}\ f{\isacharunderscore}{\kern0pt}iso{\isacharcolon}{\kern0pt}\ {\isachardoublequoteopen}isomorphism\ f{\isachardoublequoteclose}\isanewline
\ \ \ \ \isacommand{using}\isamarkupfalse%
\ is{\isacharunderscore}{\kern0pt}isomorphic{\isacharunderscore}{\kern0pt}def\ \isacommand{by}\isamarkupfalse%
\ blast\isanewline
\ \ \isacommand{show}\isamarkupfalse%
\ {\isachardoublequoteopen}{\isasymexists}{\isacharbang}{\kern0pt}x{\isachardot}{\kern0pt}\ x\ {\isasymin}\isactrlsub c\ X{\isachardoublequoteclose}\isanewline
\ \ \isacommand{proof}\isamarkupfalse%
{\isacharparenleft}{\kern0pt}safe{\isacharparenright}{\kern0pt}\isanewline
\ \ \ \ \isacommand{show}\isamarkupfalse%
\ {\isachardoublequoteopen}{\isasymexists}x{\isachardot}{\kern0pt}\ x\ {\isasymin}\isactrlsub c\ X{\isachardoublequoteclose}\isanewline
\ \ \ \ \ \ \isacommand{by}\isamarkupfalse%
\ {\isacharparenleft}{\kern0pt}meson\ f{\isacharunderscore}{\kern0pt}type{\isacharparenright}{\kern0pt}\isanewline
\ \ \isacommand{next}\isamarkupfalse%
\ \ \isanewline
\ \ \ \ \isacommand{fix}\isamarkupfalse%
\ x\ y\isanewline
\ \ \ \ \isacommand{assume}\isamarkupfalse%
\ x{\isacharunderscore}{\kern0pt}type{\isacharbrackleft}{\kern0pt}type{\isacharunderscore}{\kern0pt}rule{\isacharbrackright}{\kern0pt}{\isacharcolon}{\kern0pt}\ {\isachardoublequoteopen}x\ {\isasymin}\isactrlsub c\ X{\isachardoublequoteclose}\isanewline
\ \ \ \ \isacommand{assume}\isamarkupfalse%
\ y{\isacharunderscore}{\kern0pt}type{\isacharbrackleft}{\kern0pt}type{\isacharunderscore}{\kern0pt}rule{\isacharbrackright}{\kern0pt}{\isacharcolon}{\kern0pt}\ {\isachardoublequoteopen}y\ {\isasymin}\isactrlsub c\ X{\isachardoublequoteclose}\isanewline
\ \ \ \ \isacommand{have}\isamarkupfalse%
\ {\isasymbeta}x{\isacharunderscore}{\kern0pt}eq{\isacharunderscore}{\kern0pt}{\isasymbeta}y{\isacharcolon}{\kern0pt}\ {\isachardoublequoteopen}{\isasymbeta}\isactrlbsub X\isactrlesub \ {\isasymcirc}\isactrlsub c\ x\ {\isacharequal}{\kern0pt}\ {\isasymbeta}\isactrlbsub X\isactrlesub \ {\isasymcirc}\isactrlsub c\ y{\isachardoublequoteclose}\isanewline
\ \ \ \ \ \ \isacommand{using}\isamarkupfalse%
\ one{\isacharunderscore}{\kern0pt}unique{\isacharunderscore}{\kern0pt}element\ \isacommand{by}\isamarkupfalse%
\ {\isacharparenleft}{\kern0pt}typecheck{\isacharunderscore}{\kern0pt}cfuncs{\isacharcomma}{\kern0pt}\ blast{\isacharparenright}{\kern0pt}\ \ \ \ \ \ \isanewline
\ \ \ \ \isacommand{have}\isamarkupfalse%
\ {\isachardoublequoteopen}isomorphism\ {\isacharparenleft}{\kern0pt}{\isasymbeta}\isactrlbsub X\isactrlesub {\isacharparenright}{\kern0pt}{\isachardoublequoteclose}\isanewline
\ \ \ \ \ \ \isacommand{using}\isamarkupfalse%
\ X{\isacharunderscore}{\kern0pt}iso{\isacharunderscore}{\kern0pt}one\ is{\isacharunderscore}{\kern0pt}isomorphic{\isacharunderscore}{\kern0pt}def\ terminal{\isacharunderscore}{\kern0pt}func{\isacharunderscore}{\kern0pt}unique\ \isacommand{by}\isamarkupfalse%
\ blast\isanewline
\ \ \ \ \isacommand{then}\isamarkupfalse%
\ \isacommand{have}\isamarkupfalse%
\ {\isachardoublequoteopen}monomorphism\ {\isacharparenleft}{\kern0pt}{\isasymbeta}\isactrlbsub X\isactrlesub {\isacharparenright}{\kern0pt}{\isachardoublequoteclose}\isanewline
\ \ \ \ \ \ \isacommand{by}\isamarkupfalse%
\ {\isacharparenleft}{\kern0pt}simp\ add{\isacharcolon}{\kern0pt}\ iso{\isacharunderscore}{\kern0pt}imp{\isacharunderscore}{\kern0pt}epi{\isacharunderscore}{\kern0pt}and{\isacharunderscore}{\kern0pt}monic{\isacharparenright}{\kern0pt}\isanewline
\ \ \ \ \isacommand{then}\isamarkupfalse%
\ \isacommand{show}\isamarkupfalse%
\ {\isachardoublequoteopen}x{\isacharequal}{\kern0pt}\ y{\isachardoublequoteclose}\isanewline
\ \ \ \ \ \ \isacommand{using}\isamarkupfalse%
\ {\isasymbeta}x{\isacharunderscore}{\kern0pt}eq{\isacharunderscore}{\kern0pt}{\isasymbeta}y\ \ monomorphism{\isacharunderscore}{\kern0pt}def{\isadigit{2}}\ terminal{\isacharunderscore}{\kern0pt}func{\isacharunderscore}{\kern0pt}type\ \isacommand{by}\isamarkupfalse%
\ {\isacharparenleft}{\kern0pt}typecheck{\isacharunderscore}{\kern0pt}cfuncs{\isacharcomma}{\kern0pt}\ blast{\isacharparenright}{\kern0pt}\ \ \ \ \ \ \isanewline
\ \ \isacommand{qed}\isamarkupfalse%
\isanewline
\isacommand{next}\isamarkupfalse%
\isanewline
\ \ \isacommand{assume}\isamarkupfalse%
\ {\isachardoublequoteopen}{\isasymexists}{\isacharbang}{\kern0pt}x{\isachardot}{\kern0pt}\ x\ {\isasymin}\isactrlsub c\ X{\isachardoublequoteclose}\isanewline
\ \ \isacommand{then}\isamarkupfalse%
\ \isacommand{obtain}\isamarkupfalse%
\ x\ \isakeyword{where}\ x{\isacharunderscore}{\kern0pt}type{\isacharcolon}{\kern0pt}\ {\isachardoublequoteopen}x\ {\isacharcolon}{\kern0pt}\ {\isasymone}\ {\isasymrightarrow}\ X{\isachardoublequoteclose}\ \isakeyword{and}\ x{\isacharunderscore}{\kern0pt}unique{\isacharcolon}{\kern0pt}\ {\isachardoublequoteopen}{\isasymforall}\ y{\isachardot}{\kern0pt}\ y\ {\isacharcolon}{\kern0pt}\ {\isasymone}\ {\isasymrightarrow}\ X\ {\isasymlongrightarrow}\ x\ {\isacharequal}{\kern0pt}\ y{\isachardoublequoteclose}\isanewline
\ \ \ \ \isacommand{by}\isamarkupfalse%
\ blast\isanewline
\ \ \isacommand{have}\isamarkupfalse%
\ {\isachardoublequoteopen}terminal{\isacharunderscore}{\kern0pt}object\ X{\isachardoublequoteclose}\isanewline
\ \ \ \ \isacommand{unfolding}\isamarkupfalse%
\ terminal{\isacharunderscore}{\kern0pt}object{\isacharunderscore}{\kern0pt}def\ \ \isanewline
\ \ \isacommand{proof}\isamarkupfalse%
\ \isanewline
\ \ \ \ \isacommand{fix}\isamarkupfalse%
\ Y\isanewline
\ \ \ \ \isacommand{show}\isamarkupfalse%
\ {\isachardoublequoteopen}{\isasymexists}{\isacharbang}{\kern0pt}f{\isachardot}{\kern0pt}\ f\ {\isacharcolon}{\kern0pt}\ Y\ {\isasymrightarrow}\ X{\isachardoublequoteclose}\isanewline
\ \ \ \ \isacommand{proof}\isamarkupfalse%
\ {\isacharparenleft}{\kern0pt}rule{\isacharunderscore}{\kern0pt}tac\ a{\isacharequal}{\kern0pt}{\isachardoublequoteopen}x\ {\isasymcirc}\isactrlsub c\ {\isasymbeta}\isactrlbsub Y\isactrlesub {\isachardoublequoteclose}\ \isakeyword{in}\ ex{\isadigit{1}}I{\isacharparenright}{\kern0pt}\isanewline
\ \ \ \ \ \ \isacommand{show}\isamarkupfalse%
\ {\isachardoublequoteopen}x\ {\isasymcirc}\isactrlsub c\ {\isasymbeta}\isactrlbsub Y\isactrlesub \ {\isacharcolon}{\kern0pt}\ Y\ {\isasymrightarrow}\ X{\isachardoublequoteclose}\isanewline
\ \ \ \ \ \ \ \ \isacommand{using}\isamarkupfalse%
\ comp{\isacharunderscore}{\kern0pt}type\ terminal{\isacharunderscore}{\kern0pt}func{\isacharunderscore}{\kern0pt}type\ x{\isacharunderscore}{\kern0pt}type\ \isacommand{by}\isamarkupfalse%
\ blast\isanewline
\ \ \ \ \isacommand{next}\isamarkupfalse%
\isanewline
\ \ \ \ \ \ \isacommand{fix}\isamarkupfalse%
\ xa\isanewline
\ \ \ \ \ \ \isacommand{assume}\isamarkupfalse%
\ xa{\isacharunderscore}{\kern0pt}type{\isacharcolon}{\kern0pt}\ {\isachardoublequoteopen}xa\ {\isacharcolon}{\kern0pt}\ Y\ {\isasymrightarrow}\ X{\isachardoublequoteclose}\isanewline
\ \ \ \ \ \ \isacommand{show}\isamarkupfalse%
\ {\isachardoublequoteopen}xa\ {\isacharequal}{\kern0pt}\ x\ {\isasymcirc}\isactrlsub c\ {\isasymbeta}\isactrlbsub Y\isactrlesub {\isachardoublequoteclose}\isanewline
\ \ \ \ \ \ \isacommand{proof}\isamarkupfalse%
\ {\isacharparenleft}{\kern0pt}rule\ ccontr{\isacharparenright}{\kern0pt}\isanewline
\ \ \ \ \ \ \ \ \isacommand{assume}\isamarkupfalse%
\ {\isachardoublequoteopen}xa\ {\isasymnoteq}\ x\ {\isasymcirc}\isactrlsub c\ {\isasymbeta}\isactrlbsub Y\isactrlesub {\isachardoublequoteclose}\isanewline
\ \ \ \ \ \ \ \ \isacommand{then}\isamarkupfalse%
\ \isacommand{obtain}\isamarkupfalse%
\ y\ \isakeyword{where}\ elems{\isacharunderscore}{\kern0pt}neq{\isacharcolon}{\kern0pt}\ {\isachardoublequoteopen}xa\ {\isasymcirc}\isactrlsub c\ y\ {\isasymnoteq}\ {\isacharparenleft}{\kern0pt}x\ {\isasymcirc}\isactrlsub c\ {\isasymbeta}\isactrlbsub Y\isactrlesub {\isacharparenright}{\kern0pt}\ {\isasymcirc}\isactrlsub c\ y{\isachardoublequoteclose}\ \isakeyword{and}\ y{\isacharunderscore}{\kern0pt}type{\isacharcolon}{\kern0pt}\ {\isachardoublequoteopen}y\ {\isacharcolon}{\kern0pt}\ {\isasymone}\ {\isasymrightarrow}\ Y{\isachardoublequoteclose}\isanewline
\ \ \ \ \ \ \ \ \ \ \isacommand{using}\isamarkupfalse%
\ one{\isacharunderscore}{\kern0pt}separator{\isacharunderscore}{\kern0pt}contrapos{\isacharbrackleft}{\kern0pt}\isakeyword{where}\ f{\isacharequal}{\kern0pt}xa{\isacharcomma}{\kern0pt}\ \isakeyword{where}\ g{\isacharequal}{\kern0pt}{\isachardoublequoteopen}x\ {\isasymcirc}\isactrlsub c\ {\isasymbeta}\isactrlbsub Y\isactrlesub {\isachardoublequoteclose}{\isacharcomma}{\kern0pt}\ \isakeyword{where}\ X{\isacharequal}{\kern0pt}Y{\isacharcomma}{\kern0pt}\ \isakeyword{where}\ Y{\isacharequal}{\kern0pt}X{\isacharbrackright}{\kern0pt}\isanewline
\ \ \ \ \ \ \ \ \ \ \isacommand{using}\isamarkupfalse%
\ comp{\isacharunderscore}{\kern0pt}type\ terminal{\isacharunderscore}{\kern0pt}func{\isacharunderscore}{\kern0pt}type\ x{\isacharunderscore}{\kern0pt}type\ xa{\isacharunderscore}{\kern0pt}type\ \isacommand{by}\isamarkupfalse%
\ blast\isanewline
\ \ \ \ \ \ \ \ \isacommand{have}\isamarkupfalse%
\ elem{\isadigit{1}}{\isacharcolon}{\kern0pt}\ {\isachardoublequoteopen}xa\ {\isasymcirc}\isactrlsub c\ y\ {\isasymin}\isactrlsub c\ X{\isachardoublequoteclose}\isanewline
\ \ \ \ \ \ \ \ \ \ \isacommand{using}\isamarkupfalse%
\ comp{\isacharunderscore}{\kern0pt}type\ xa{\isacharunderscore}{\kern0pt}type\ y{\isacharunderscore}{\kern0pt}type\ \isacommand{by}\isamarkupfalse%
\ auto\isanewline
\ \ \ \ \ \ \ \ \isacommand{have}\isamarkupfalse%
\ elem{\isadigit{2}}{\isacharcolon}{\kern0pt}\ {\isachardoublequoteopen}{\isacharparenleft}{\kern0pt}x\ {\isasymcirc}\isactrlsub c\ {\isasymbeta}\isactrlbsub Y\isactrlesub {\isacharparenright}{\kern0pt}\ {\isasymcirc}\isactrlsub c\ y\ {\isasymin}\isactrlsub c\ X{\isachardoublequoteclose}\isanewline
\ \ \ \ \ \ \ \ \ \ \isacommand{using}\isamarkupfalse%
\ comp{\isacharunderscore}{\kern0pt}type\ terminal{\isacharunderscore}{\kern0pt}func{\isacharunderscore}{\kern0pt}type\ x{\isacharunderscore}{\kern0pt}type\ y{\isacharunderscore}{\kern0pt}type\ \isacommand{by}\isamarkupfalse%
\ blast\isanewline
\ \ \ \ \ \ \ \ \isacommand{show}\isamarkupfalse%
\ False\isanewline
\ \ \ \ \ \ \ \ \ \ \isacommand{using}\isamarkupfalse%
\ elem{\isadigit{1}}\ elem{\isadigit{2}}\ elems{\isacharunderscore}{\kern0pt}neq\ x{\isacharunderscore}{\kern0pt}unique\ \isacommand{by}\isamarkupfalse%
\ blast\isanewline
\ \ \ \ \ \ \isacommand{qed}\isamarkupfalse%
\isanewline
\ \ \ \ \isacommand{qed}\isamarkupfalse%
\isanewline
\ \ \isacommand{qed}\isamarkupfalse%
\isanewline
\ \ \isacommand{then}\isamarkupfalse%
\ \isacommand{show}\isamarkupfalse%
\ {\isachardoublequoteopen}X\ {\isasymcong}\ {\isasymone}{\isachardoublequoteclose}\isanewline
\ \ \ \ \isacommand{by}\isamarkupfalse%
\ {\isacharparenleft}{\kern0pt}simp\ add{\isacharcolon}{\kern0pt}\ one{\isacharunderscore}{\kern0pt}terminal{\isacharunderscore}{\kern0pt}object\ terminal{\isacharunderscore}{\kern0pt}objects{\isacharunderscore}{\kern0pt}isomorphic{\isacharparenright}{\kern0pt}\isanewline
\isacommand{qed}\isamarkupfalse%
%
\endisatagproof
{\isafoldproof}%
%
\isadelimproof
%
\endisadelimproof
%
\isadelimdocument
%
\endisadelimdocument
%
\isatagdocument
%
\isamarkupsubsection{Injectivity%
}
\isamarkuptrue%
%
\endisatagdocument
{\isafolddocument}%
%
\isadelimdocument
%
\endisadelimdocument
%
\begin{isamarkuptext}%
The definition below corresponds to Definition 2.1.24 in Halvorson.%
\end{isamarkuptext}\isamarkuptrue%
\isacommand{definition}\isamarkupfalse%
\ injective\ {\isacharcolon}{\kern0pt}{\isacharcolon}{\kern0pt}\ {\isachardoublequoteopen}cfunc\ {\isasymRightarrow}\ bool{\isachardoublequoteclose}\ \isakeyword{where}\isanewline
\ {\isachardoublequoteopen}injective\ f\ \ {\isasymlongleftrightarrow}\ {\isacharparenleft}{\kern0pt}{\isasymforall}\ x\ y{\isachardot}{\kern0pt}\ {\isacharparenleft}{\kern0pt}x\ {\isasymin}\isactrlsub c\ domain\ f\ {\isasymand}\ y\ {\isasymin}\isactrlsub c\ domain\ f\ {\isasymand}\ f\ {\isasymcirc}\isactrlsub c\ x\ {\isacharequal}{\kern0pt}\ f\ {\isasymcirc}\isactrlsub c\ y{\isacharparenright}{\kern0pt}\ {\isasymlongrightarrow}\ x\ {\isacharequal}{\kern0pt}\ y{\isacharparenright}{\kern0pt}{\isachardoublequoteclose}\isanewline
\isanewline
\isacommand{lemma}\isamarkupfalse%
\ injective{\isacharunderscore}{\kern0pt}def{\isadigit{2}}{\isacharcolon}{\kern0pt}\isanewline
\ \ \isakeyword{assumes}\ {\isachardoublequoteopen}f\ {\isacharcolon}{\kern0pt}\ X\ {\isasymrightarrow}\ Y{\isachardoublequoteclose}\isanewline
\ \ \isakeyword{shows}\ {\isachardoublequoteopen}injective\ f\ \ {\isasymlongleftrightarrow}\ {\isacharparenleft}{\kern0pt}{\isasymforall}\ x\ y{\isachardot}{\kern0pt}\ {\isacharparenleft}{\kern0pt}x\ {\isasymin}\isactrlsub c\ X\ {\isasymand}\ y\ {\isasymin}\isactrlsub c\ X\ {\isasymand}\ f\ {\isasymcirc}\isactrlsub c\ x\ {\isacharequal}{\kern0pt}\ f\ {\isasymcirc}\isactrlsub c\ y{\isacharparenright}{\kern0pt}\ {\isasymlongrightarrow}\ x\ {\isacharequal}{\kern0pt}\ y{\isacharparenright}{\kern0pt}{\isachardoublequoteclose}\isanewline
%
\isadelimproof
\ \ %
\endisadelimproof
%
\isatagproof
\isacommand{using}\isamarkupfalse%
\ assms\ cfunc{\isacharunderscore}{\kern0pt}type{\isacharunderscore}{\kern0pt}def\ injective{\isacharunderscore}{\kern0pt}def\ \isacommand{by}\isamarkupfalse%
\ force%
\endisatagproof
{\isafoldproof}%
%
\isadelimproof
%
\endisadelimproof
%
\begin{isamarkuptext}%
The lemma below corresponds to Exercise 2.1.26 in Halvorson.%
\end{isamarkuptext}\isamarkuptrue%
\isacommand{lemma}\isamarkupfalse%
\ monomorphism{\isacharunderscore}{\kern0pt}imp{\isacharunderscore}{\kern0pt}injective{\isacharcolon}{\kern0pt}\isanewline
\ \ {\isachardoublequoteopen}monomorphism\ f\ {\isasymLongrightarrow}\ injective\ f{\isachardoublequoteclose}\isanewline
%
\isadelimproof
\ \ %
\endisadelimproof
%
\isatagproof
\isacommand{by}\isamarkupfalse%
\ {\isacharparenleft}{\kern0pt}simp\ add{\isacharcolon}{\kern0pt}\ cfunc{\isacharunderscore}{\kern0pt}type{\isacharunderscore}{\kern0pt}def\ injective{\isacharunderscore}{\kern0pt}def\ monomorphism{\isacharunderscore}{\kern0pt}def{\isacharparenright}{\kern0pt}%
\endisatagproof
{\isafoldproof}%
%
\isadelimproof
%
\endisadelimproof
%
\begin{isamarkuptext}%
The lemma below corresponds to Proposition 2.1.27 in Halvorson.%
\end{isamarkuptext}\isamarkuptrue%
\isacommand{lemma}\isamarkupfalse%
\ injective{\isacharunderscore}{\kern0pt}imp{\isacharunderscore}{\kern0pt}monomorphism{\isacharcolon}{\kern0pt}\isanewline
\ \ {\isachardoublequoteopen}injective\ f\ {\isasymLongrightarrow}\ monomorphism\ f{\isachardoublequoteclose}\isanewline
%
\isadelimproof
\ \ %
\endisadelimproof
%
\isatagproof
\isacommand{unfolding}\isamarkupfalse%
\ monomorphism{\isacharunderscore}{\kern0pt}def\ injective{\isacharunderscore}{\kern0pt}def\isanewline
\isacommand{proof}\isamarkupfalse%
\ clarify\isanewline
\ \ \isacommand{fix}\isamarkupfalse%
\ g\ h\isanewline
\ \ \isacommand{assume}\isamarkupfalse%
\ f{\isacharunderscore}{\kern0pt}inj{\isacharcolon}{\kern0pt}\ {\isachardoublequoteopen}{\isasymforall}x\ y{\isachardot}{\kern0pt}\ x\ {\isasymin}\isactrlsub c\ domain\ f\ {\isasymand}\ y\ {\isasymin}\isactrlsub c\ domain\ f\ {\isasymand}\ f\ {\isasymcirc}\isactrlsub c\ x\ {\isacharequal}{\kern0pt}\ f\ {\isasymcirc}\isactrlsub c\ y\ {\isasymlongrightarrow}\ x\ {\isacharequal}{\kern0pt}\ y{\isachardoublequoteclose}\isanewline
\ \ \isacommand{assume}\isamarkupfalse%
\ cd{\isacharunderscore}{\kern0pt}g{\isacharunderscore}{\kern0pt}eq{\isacharunderscore}{\kern0pt}d{\isacharunderscore}{\kern0pt}f{\isacharcolon}{\kern0pt}\ {\isachardoublequoteopen}codomain\ g\ {\isacharequal}{\kern0pt}\ domain\ f{\isachardoublequoteclose}\isanewline
\ \ \isacommand{assume}\isamarkupfalse%
\ cd{\isacharunderscore}{\kern0pt}h{\isacharunderscore}{\kern0pt}eq{\isacharunderscore}{\kern0pt}d{\isacharunderscore}{\kern0pt}f{\isacharcolon}{\kern0pt}\ {\isachardoublequoteopen}codomain\ h\ {\isacharequal}{\kern0pt}\ domain\ f{\isachardoublequoteclose}\isanewline
\ \ \isacommand{assume}\isamarkupfalse%
\ fg{\isacharunderscore}{\kern0pt}eq{\isacharunderscore}{\kern0pt}fh{\isacharcolon}{\kern0pt}\ {\isachardoublequoteopen}f\ {\isasymcirc}\isactrlsub c\ g\ {\isacharequal}{\kern0pt}\ f\ {\isasymcirc}\isactrlsub c\ h{\isachardoublequoteclose}\isanewline
\isanewline
\ \ \isacommand{obtain}\isamarkupfalse%
\ X\ Y\ \isakeyword{where}\ f{\isacharunderscore}{\kern0pt}type{\isacharcolon}{\kern0pt}\ {\isachardoublequoteopen}f\ {\isacharcolon}{\kern0pt}\ X\ {\isasymrightarrow}\ Y{\isachardoublequoteclose}\isanewline
\ \ \ \ \isacommand{using}\isamarkupfalse%
\ cfunc{\isacharunderscore}{\kern0pt}type{\isacharunderscore}{\kern0pt}def\ \isacommand{by}\isamarkupfalse%
\ auto\ \ \ \ \isanewline
\ \ \isacommand{obtain}\isamarkupfalse%
\ A\ \isakeyword{where}\ g{\isacharunderscore}{\kern0pt}type{\isacharcolon}{\kern0pt}\ {\isachardoublequoteopen}g\ {\isacharcolon}{\kern0pt}\ A\ {\isasymrightarrow}\ X{\isachardoublequoteclose}\ \isakeyword{and}\ h{\isacharunderscore}{\kern0pt}type{\isacharcolon}{\kern0pt}\ {\isachardoublequoteopen}h\ {\isacharcolon}{\kern0pt}\ A\ {\isasymrightarrow}\ X{\isachardoublequoteclose}\isanewline
\ \ \ \ \isacommand{by}\isamarkupfalse%
\ {\isacharparenleft}{\kern0pt}metis\ cd{\isacharunderscore}{\kern0pt}g{\isacharunderscore}{\kern0pt}eq{\isacharunderscore}{\kern0pt}d{\isacharunderscore}{\kern0pt}f\ cd{\isacharunderscore}{\kern0pt}h{\isacharunderscore}{\kern0pt}eq{\isacharunderscore}{\kern0pt}d{\isacharunderscore}{\kern0pt}f\ cfunc{\isacharunderscore}{\kern0pt}type{\isacharunderscore}{\kern0pt}def\ domain{\isacharunderscore}{\kern0pt}comp\ f{\isacharunderscore}{\kern0pt}type\ fg{\isacharunderscore}{\kern0pt}eq{\isacharunderscore}{\kern0pt}fh{\isacharparenright}{\kern0pt}\isanewline
\isanewline
\ \ \isacommand{have}\isamarkupfalse%
\ {\isachardoublequoteopen}{\isasymforall}x{\isachardot}{\kern0pt}\ x\ {\isasymin}\isactrlsub c\ A\ {\isasymlongrightarrow}\ g\ {\isasymcirc}\isactrlsub c\ x\ {\isacharequal}{\kern0pt}\ h\ {\isasymcirc}\isactrlsub c\ x{\isachardoublequoteclose}\isanewline
\ \ \isacommand{proof}\isamarkupfalse%
\ clarify\isanewline
\ \ \ \ \isacommand{fix}\isamarkupfalse%
\ x\isanewline
\ \ \ \ \isacommand{assume}\isamarkupfalse%
\ x{\isacharunderscore}{\kern0pt}in{\isacharunderscore}{\kern0pt}A{\isacharcolon}{\kern0pt}\ {\isachardoublequoteopen}x\ {\isasymin}\isactrlsub c\ A{\isachardoublequoteclose}\isanewline
\isanewline
\ \ \ \ \isacommand{have}\isamarkupfalse%
\ {\isachardoublequoteopen}f\ {\isasymcirc}\isactrlsub c\ g\ {\isasymcirc}\isactrlsub c\ x\ {\isacharequal}{\kern0pt}\ f\ {\isasymcirc}\isactrlsub c\ h\ {\isasymcirc}\isactrlsub c\ x{\isachardoublequoteclose}\isanewline
\ \ \ \ \ \ \isacommand{using}\isamarkupfalse%
\ g{\isacharunderscore}{\kern0pt}type\ h{\isacharunderscore}{\kern0pt}type\ x{\isacharunderscore}{\kern0pt}in{\isacharunderscore}{\kern0pt}A\ f{\isacharunderscore}{\kern0pt}type\ comp{\isacharunderscore}{\kern0pt}associative{\isadigit{2}}\ fg{\isacharunderscore}{\kern0pt}eq{\isacharunderscore}{\kern0pt}fh\ \isacommand{by}\isamarkupfalse%
\ {\isacharparenleft}{\kern0pt}typecheck{\isacharunderscore}{\kern0pt}cfuncs{\isacharcomma}{\kern0pt}\ auto{\isacharparenright}{\kern0pt}\isanewline
\ \ \ \ \isacommand{then}\isamarkupfalse%
\ \isacommand{show}\isamarkupfalse%
\ {\isachardoublequoteopen}g\ {\isasymcirc}\isactrlsub c\ x\ {\isacharequal}{\kern0pt}\ h\ {\isasymcirc}\isactrlsub c\ x{\isachardoublequoteclose}\isanewline
\ \ \ \ \ \ \isacommand{using}\isamarkupfalse%
\ cd{\isacharunderscore}{\kern0pt}h{\isacharunderscore}{\kern0pt}eq{\isacharunderscore}{\kern0pt}d{\isacharunderscore}{\kern0pt}f\ cfunc{\isacharunderscore}{\kern0pt}type{\isacharunderscore}{\kern0pt}def\ comp{\isacharunderscore}{\kern0pt}type\ f{\isacharunderscore}{\kern0pt}inj\ g{\isacharunderscore}{\kern0pt}type\ h{\isacharunderscore}{\kern0pt}type\ x{\isacharunderscore}{\kern0pt}in{\isacharunderscore}{\kern0pt}A\ \isacommand{by}\isamarkupfalse%
\ presburger\isanewline
\ \ \isacommand{qed}\isamarkupfalse%
\isanewline
\ \ \isacommand{then}\isamarkupfalse%
\ \isacommand{show}\isamarkupfalse%
\ {\isachardoublequoteopen}g\ {\isacharequal}{\kern0pt}\ h{\isachardoublequoteclose}\isanewline
\ \ \ \ \isacommand{using}\isamarkupfalse%
\ g{\isacharunderscore}{\kern0pt}type\ h{\isacharunderscore}{\kern0pt}type\ one{\isacharunderscore}{\kern0pt}separator\ \isacommand{by}\isamarkupfalse%
\ auto\isanewline
\isacommand{qed}\isamarkupfalse%
%
\endisatagproof
{\isafoldproof}%
%
\isadelimproof
\isanewline
%
\endisadelimproof
\isanewline
\isacommand{lemma}\isamarkupfalse%
\ cfunc{\isacharunderscore}{\kern0pt}cross{\isacharunderscore}{\kern0pt}prod{\isacharunderscore}{\kern0pt}inj{\isacharcolon}{\kern0pt}\isanewline
\ \ \isakeyword{assumes}\ type{\isacharunderscore}{\kern0pt}assms{\isacharcolon}{\kern0pt}\ {\isachardoublequoteopen}f\ {\isacharcolon}{\kern0pt}\ X\ {\isasymrightarrow}\ Y{\isachardoublequoteclose}\ {\isachardoublequoteopen}g\ {\isacharcolon}{\kern0pt}\ Z\ {\isasymrightarrow}\ W{\isachardoublequoteclose}\isanewline
\ \ \isakeyword{assumes}\ {\isachardoublequoteopen}injective\ f\ {\isasymand}\ injective\ g{\isachardoublequoteclose}\isanewline
\ \ \isakeyword{shows}\ {\isachardoublequoteopen}injective\ {\isacharparenleft}{\kern0pt}f\ {\isasymtimes}\isactrlsub f\ g{\isacharparenright}{\kern0pt}{\isachardoublequoteclose}\isanewline
%
\isadelimproof
\ \ %
\endisadelimproof
%
\isatagproof
\isacommand{by}\isamarkupfalse%
\ {\isacharparenleft}{\kern0pt}typecheck{\isacharunderscore}{\kern0pt}cfuncs{\isacharcomma}{\kern0pt}\ metis\ assms\ cfunc{\isacharunderscore}{\kern0pt}cross{\isacharunderscore}{\kern0pt}prod{\isacharunderscore}{\kern0pt}mono\ injective{\isacharunderscore}{\kern0pt}imp{\isacharunderscore}{\kern0pt}monomorphism\ monomorphism{\isacharunderscore}{\kern0pt}imp{\isacharunderscore}{\kern0pt}injective{\isacharparenright}{\kern0pt}%
\endisatagproof
{\isafoldproof}%
%
\isadelimproof
\isanewline
%
\endisadelimproof
\isanewline
\isacommand{lemma}\isamarkupfalse%
\ cfunc{\isacharunderscore}{\kern0pt}cross{\isacharunderscore}{\kern0pt}prod{\isacharunderscore}{\kern0pt}mono{\isacharunderscore}{\kern0pt}converse{\isacharcolon}{\kern0pt}\isanewline
\ \ \isakeyword{assumes}\ type{\isacharunderscore}{\kern0pt}assms{\isacharcolon}{\kern0pt}\ {\isachardoublequoteopen}f\ {\isacharcolon}{\kern0pt}\ X\ {\isasymrightarrow}\ Y{\isachardoublequoteclose}\ {\isachardoublequoteopen}g\ {\isacharcolon}{\kern0pt}\ Z\ {\isasymrightarrow}\ W{\isachardoublequoteclose}\isanewline
\ \ \isakeyword{assumes}\ fg{\isacharunderscore}{\kern0pt}inject{\isacharcolon}{\kern0pt}\ {\isachardoublequoteopen}injective\ {\isacharparenleft}{\kern0pt}f\ {\isasymtimes}\isactrlsub f\ g{\isacharparenright}{\kern0pt}{\isachardoublequoteclose}\isanewline
\ \ \isakeyword{assumes}\ nonempty{\isacharcolon}{\kern0pt}\ {\isachardoublequoteopen}nonempty\ X{\isachardoublequoteclose}\ {\isachardoublequoteopen}nonempty\ Z{\isachardoublequoteclose}\isanewline
\ \ \isakeyword{shows}\ {\isachardoublequoteopen}injective\ f\ {\isasymand}\ injective\ g{\isachardoublequoteclose}\isanewline
%
\isadelimproof
\ \ %
\endisadelimproof
%
\isatagproof
\isacommand{unfolding}\isamarkupfalse%
\ injective{\isacharunderscore}{\kern0pt}def\isanewline
\isacommand{proof}\isamarkupfalse%
\ safe\isanewline
\ \ \isacommand{fix}\isamarkupfalse%
\ x\ y\ \isanewline
\ \ \isacommand{assume}\isamarkupfalse%
\ x{\isacharunderscore}{\kern0pt}type{\isacharcolon}{\kern0pt}\ {\isachardoublequoteopen}x\ {\isasymin}\isactrlsub c\ domain\ f{\isachardoublequoteclose}\isanewline
\ \ \isacommand{assume}\isamarkupfalse%
\ y{\isacharunderscore}{\kern0pt}type{\isacharcolon}{\kern0pt}\ {\isachardoublequoteopen}y\ {\isasymin}\isactrlsub c\ domain\ f{\isachardoublequoteclose}\isanewline
\ \ \isacommand{assume}\isamarkupfalse%
\ equals{\isacharcolon}{\kern0pt}\ {\isachardoublequoteopen}f\ {\isasymcirc}\isactrlsub c\ x\ {\isacharequal}{\kern0pt}\ f\ {\isasymcirc}\isactrlsub c\ y{\isachardoublequoteclose}\isanewline
\ \ \isacommand{have}\isamarkupfalse%
\ fg{\isacharunderscore}{\kern0pt}type{\isacharcolon}{\kern0pt}\ {\isachardoublequoteopen}f\ {\isasymtimes}\isactrlsub f\ g\ {\isacharcolon}{\kern0pt}\ X\ {\isasymtimes}\isactrlsub c\ Z\ {\isasymrightarrow}\ Y\ {\isasymtimes}\isactrlsub c\ W{\isachardoublequoteclose}\isanewline
\ \ \ \ \isacommand{using}\isamarkupfalse%
\ assms\ \isacommand{by}\isamarkupfalse%
\ typecheck{\isacharunderscore}{\kern0pt}cfuncs\isanewline
\ \ \isacommand{have}\isamarkupfalse%
\ x{\isacharunderscore}{\kern0pt}type{\isadigit{2}}{\isacharcolon}{\kern0pt}\ {\isachardoublequoteopen}x\ {\isasymin}\isactrlsub c\ X{\isachardoublequoteclose}\isanewline
\ \ \ \ \isacommand{using}\isamarkupfalse%
\ cfunc{\isacharunderscore}{\kern0pt}type{\isacharunderscore}{\kern0pt}def\ type{\isacharunderscore}{\kern0pt}assms{\isacharparenleft}{\kern0pt}{\isadigit{1}}{\isacharparenright}{\kern0pt}\ x{\isacharunderscore}{\kern0pt}type\ \isacommand{by}\isamarkupfalse%
\ auto\isanewline
\ \ \isacommand{have}\isamarkupfalse%
\ y{\isacharunderscore}{\kern0pt}type{\isadigit{2}}{\isacharcolon}{\kern0pt}\ {\isachardoublequoteopen}y\ {\isasymin}\isactrlsub c\ X{\isachardoublequoteclose}\isanewline
\ \ \ \ \isacommand{using}\isamarkupfalse%
\ cfunc{\isacharunderscore}{\kern0pt}type{\isacharunderscore}{\kern0pt}def\ type{\isacharunderscore}{\kern0pt}assms{\isacharparenleft}{\kern0pt}{\isadigit{1}}{\isacharparenright}{\kern0pt}\ y{\isacharunderscore}{\kern0pt}type\ \isacommand{by}\isamarkupfalse%
\ auto\isanewline
\ \ \isacommand{show}\isamarkupfalse%
\ {\isachardoublequoteopen}x\ {\isacharequal}{\kern0pt}\ y{\isachardoublequoteclose}\isanewline
\ \ \isacommand{proof}\isamarkupfalse%
\ {\isacharminus}{\kern0pt}\ \isanewline
\ \ \ \ \isacommand{obtain}\isamarkupfalse%
\ b\ \isakeyword{where}\ b{\isacharunderscore}{\kern0pt}def{\isacharcolon}{\kern0pt}\ {\isachardoublequoteopen}b\ {\isasymin}\isactrlsub c\ Z{\isachardoublequoteclose}\isanewline
\ \ \ \ \ \ \isacommand{using}\isamarkupfalse%
\ nonempty{\isacharparenleft}{\kern0pt}{\isadigit{2}}{\isacharparenright}{\kern0pt}\ nonempty{\isacharunderscore}{\kern0pt}def\ \isacommand{by}\isamarkupfalse%
\ blast\isanewline
\isanewline
\ \ \ \ \isacommand{have}\isamarkupfalse%
\ xb{\isacharunderscore}{\kern0pt}type{\isacharcolon}{\kern0pt}\ {\isachardoublequoteopen}{\isasymlangle}x{\isacharcomma}{\kern0pt}b{\isasymrangle}\ {\isasymin}\isactrlsub c\ X\ {\isasymtimes}\isactrlsub c\ Z{\isachardoublequoteclose}\isanewline
\ \ \ \ \ \ \isacommand{by}\isamarkupfalse%
\ {\isacharparenleft}{\kern0pt}simp\ add{\isacharcolon}{\kern0pt}\ b{\isacharunderscore}{\kern0pt}def\ cfunc{\isacharunderscore}{\kern0pt}prod{\isacharunderscore}{\kern0pt}type\ x{\isacharunderscore}{\kern0pt}type{\isadigit{2}}{\isacharparenright}{\kern0pt}\isanewline
\ \ \ \ \isacommand{have}\isamarkupfalse%
\ yb{\isacharunderscore}{\kern0pt}type{\isacharcolon}{\kern0pt}\ {\isachardoublequoteopen}{\isasymlangle}y{\isacharcomma}{\kern0pt}b{\isasymrangle}\ {\isasymin}\isactrlsub c\ X\ {\isasymtimes}\isactrlsub c\ Z{\isachardoublequoteclose}\isanewline
\ \ \ \ \ \ \isacommand{by}\isamarkupfalse%
\ {\isacharparenleft}{\kern0pt}simp\ add{\isacharcolon}{\kern0pt}\ b{\isacharunderscore}{\kern0pt}def\ cfunc{\isacharunderscore}{\kern0pt}prod{\isacharunderscore}{\kern0pt}type\ y{\isacharunderscore}{\kern0pt}type{\isadigit{2}}{\isacharparenright}{\kern0pt}\isanewline
\ \ \ \ \isacommand{have}\isamarkupfalse%
\ {\isachardoublequoteopen}{\isacharparenleft}{\kern0pt}f\ {\isasymtimes}\isactrlsub f\ g{\isacharparenright}{\kern0pt}\ {\isasymcirc}\isactrlsub c\ {\isasymlangle}x{\isacharcomma}{\kern0pt}b{\isasymrangle}\ {\isacharequal}{\kern0pt}\ {\isasymlangle}f\ {\isasymcirc}\isactrlsub c\ x{\isacharcomma}{\kern0pt}g\ {\isasymcirc}\isactrlsub c\ b{\isasymrangle}{\isachardoublequoteclose}\isanewline
\ \ \ \ \ \ \isacommand{using}\isamarkupfalse%
\ b{\isacharunderscore}{\kern0pt}def\ cfunc{\isacharunderscore}{\kern0pt}cross{\isacharunderscore}{\kern0pt}prod{\isacharunderscore}{\kern0pt}comp{\isacharunderscore}{\kern0pt}cfunc{\isacharunderscore}{\kern0pt}prod\ type{\isacharunderscore}{\kern0pt}assms\ x{\isacharunderscore}{\kern0pt}type{\isadigit{2}}\ \isacommand{by}\isamarkupfalse%
\ blast\isanewline
\ \ \ \ \isacommand{also}\isamarkupfalse%
\ \isacommand{have}\isamarkupfalse%
\ {\isachardoublequoteopen}{\isachardot}{\kern0pt}{\isachardot}{\kern0pt}{\isachardot}{\kern0pt}\ {\isacharequal}{\kern0pt}\ {\isasymlangle}f\ {\isasymcirc}\isactrlsub c\ y{\isacharcomma}{\kern0pt}g\ {\isasymcirc}\isactrlsub c\ b{\isasymrangle}{\isachardoublequoteclose}\isanewline
\ \ \ \ \ \ \isacommand{by}\isamarkupfalse%
\ {\isacharparenleft}{\kern0pt}simp\ add{\isacharcolon}{\kern0pt}\ equals{\isacharparenright}{\kern0pt}\isanewline
\ \ \ \ \isacommand{also}\isamarkupfalse%
\ \isacommand{have}\isamarkupfalse%
\ {\isachardoublequoteopen}{\isachardot}{\kern0pt}{\isachardot}{\kern0pt}{\isachardot}{\kern0pt}\ {\isacharequal}{\kern0pt}\ {\isacharparenleft}{\kern0pt}f\ {\isasymtimes}\isactrlsub f\ g{\isacharparenright}{\kern0pt}\ {\isasymcirc}\isactrlsub c\ {\isasymlangle}y{\isacharcomma}{\kern0pt}b{\isasymrangle}{\isachardoublequoteclose}\isanewline
\ \ \ \ \ \ \isacommand{using}\isamarkupfalse%
\ b{\isacharunderscore}{\kern0pt}def\ cfunc{\isacharunderscore}{\kern0pt}cross{\isacharunderscore}{\kern0pt}prod{\isacharunderscore}{\kern0pt}comp{\isacharunderscore}{\kern0pt}cfunc{\isacharunderscore}{\kern0pt}prod\ type{\isacharunderscore}{\kern0pt}assms\ y{\isacharunderscore}{\kern0pt}type{\isadigit{2}}\ \isacommand{by}\isamarkupfalse%
\ auto\isanewline
\ \ \ \ \isacommand{then}\isamarkupfalse%
\ \isacommand{have}\isamarkupfalse%
\ {\isachardoublequoteopen}{\isasymlangle}x{\isacharcomma}{\kern0pt}b{\isasymrangle}\ {\isacharequal}{\kern0pt}\ {\isasymlangle}y{\isacharcomma}{\kern0pt}b{\isasymrangle}{\isachardoublequoteclose}\isanewline
\ \ \ \ \ \ \isacommand{by}\isamarkupfalse%
\ {\isacharparenleft}{\kern0pt}metis\ calculation\ cfunc{\isacharunderscore}{\kern0pt}type{\isacharunderscore}{\kern0pt}def\ fg{\isacharunderscore}{\kern0pt}inject\ fg{\isacharunderscore}{\kern0pt}type\ injective{\isacharunderscore}{\kern0pt}def\ xb{\isacharunderscore}{\kern0pt}type\ yb{\isacharunderscore}{\kern0pt}type{\isacharparenright}{\kern0pt}\isanewline
\ \ \ \ \isacommand{then}\isamarkupfalse%
\ \isacommand{show}\isamarkupfalse%
\ {\isachardoublequoteopen}x\ {\isacharequal}{\kern0pt}\ y{\isachardoublequoteclose}\isanewline
\ \ \ \ \ \ \isacommand{using}\isamarkupfalse%
\ b{\isacharunderscore}{\kern0pt}def\ cart{\isacharunderscore}{\kern0pt}prod{\isacharunderscore}{\kern0pt}eq{\isadigit{2}}\ x{\isacharunderscore}{\kern0pt}type{\isadigit{2}}\ y{\isacharunderscore}{\kern0pt}type{\isadigit{2}}\ \isacommand{by}\isamarkupfalse%
\ auto\isanewline
\ \ \isacommand{qed}\isamarkupfalse%
\isanewline
\isacommand{next}\isamarkupfalse%
\isanewline
\ \ \isacommand{fix}\isamarkupfalse%
\ x\ y\ \isanewline
\ \ \isacommand{assume}\isamarkupfalse%
\ x{\isacharunderscore}{\kern0pt}type{\isacharcolon}{\kern0pt}\ {\isachardoublequoteopen}x\ {\isasymin}\isactrlsub c\ domain\ g{\isachardoublequoteclose}\isanewline
\ \ \isacommand{assume}\isamarkupfalse%
\ y{\isacharunderscore}{\kern0pt}type{\isacharcolon}{\kern0pt}\ {\isachardoublequoteopen}y\ {\isasymin}\isactrlsub c\ domain\ g{\isachardoublequoteclose}\isanewline
\ \ \isacommand{assume}\isamarkupfalse%
\ equals{\isacharcolon}{\kern0pt}\ {\isachardoublequoteopen}g\ {\isasymcirc}\isactrlsub c\ x\ {\isacharequal}{\kern0pt}\ g\ {\isasymcirc}\isactrlsub c\ y{\isachardoublequoteclose}\isanewline
\ \ \isacommand{have}\isamarkupfalse%
\ fg{\isacharunderscore}{\kern0pt}type{\isacharcolon}{\kern0pt}\ {\isachardoublequoteopen}f\ {\isasymtimes}\isactrlsub f\ g\ {\isacharcolon}{\kern0pt}\ X\ {\isasymtimes}\isactrlsub c\ Z\ {\isasymrightarrow}\ Y\ {\isasymtimes}\isactrlsub c\ W{\isachardoublequoteclose}\isanewline
\ \ \ \ \isacommand{using}\isamarkupfalse%
\ assms\ \isacommand{by}\isamarkupfalse%
\ typecheck{\isacharunderscore}{\kern0pt}cfuncs\isanewline
\ \ \isacommand{have}\isamarkupfalse%
\ x{\isacharunderscore}{\kern0pt}type{\isadigit{2}}{\isacharcolon}{\kern0pt}\ {\isachardoublequoteopen}x\ {\isasymin}\isactrlsub c\ Z{\isachardoublequoteclose}\isanewline
\ \ \ \ \isacommand{using}\isamarkupfalse%
\ cfunc{\isacharunderscore}{\kern0pt}type{\isacharunderscore}{\kern0pt}def\ type{\isacharunderscore}{\kern0pt}assms{\isacharparenleft}{\kern0pt}{\isadigit{2}}{\isacharparenright}{\kern0pt}\ x{\isacharunderscore}{\kern0pt}type\ \isacommand{by}\isamarkupfalse%
\ auto\isanewline
\ \ \isacommand{have}\isamarkupfalse%
\ y{\isacharunderscore}{\kern0pt}type{\isadigit{2}}{\isacharcolon}{\kern0pt}\ {\isachardoublequoteopen}y\ {\isasymin}\isactrlsub c\ Z{\isachardoublequoteclose}\isanewline
\ \ \ \ \isacommand{using}\isamarkupfalse%
\ cfunc{\isacharunderscore}{\kern0pt}type{\isacharunderscore}{\kern0pt}def\ type{\isacharunderscore}{\kern0pt}assms{\isacharparenleft}{\kern0pt}{\isadigit{2}}{\isacharparenright}{\kern0pt}\ y{\isacharunderscore}{\kern0pt}type\ \isacommand{by}\isamarkupfalse%
\ auto\isanewline
\ \ \isacommand{show}\isamarkupfalse%
\ {\isachardoublequoteopen}x\ {\isacharequal}{\kern0pt}\ y{\isachardoublequoteclose}\isanewline
\ \ \isacommand{proof}\isamarkupfalse%
\ {\isacharminus}{\kern0pt}\ \isanewline
\ \ \ \ \isacommand{obtain}\isamarkupfalse%
\ b\ \isakeyword{where}\ b{\isacharunderscore}{\kern0pt}def{\isacharcolon}{\kern0pt}\ {\isachardoublequoteopen}b\ {\isasymin}\isactrlsub c\ X{\isachardoublequoteclose}\isanewline
\ \ \ \ \ \ \isacommand{using}\isamarkupfalse%
\ nonempty{\isacharparenleft}{\kern0pt}{\isadigit{1}}{\isacharparenright}{\kern0pt}\ nonempty{\isacharunderscore}{\kern0pt}def\ \isacommand{by}\isamarkupfalse%
\ blast\isanewline
\ \ \ \ \isacommand{have}\isamarkupfalse%
\ xb{\isacharunderscore}{\kern0pt}type{\isacharcolon}{\kern0pt}\ {\isachardoublequoteopen}{\isasymlangle}b{\isacharcomma}{\kern0pt}x{\isasymrangle}\ {\isasymin}\isactrlsub c\ X\ {\isasymtimes}\isactrlsub c\ Z{\isachardoublequoteclose}\isanewline
\ \ \ \ \ \ \isacommand{by}\isamarkupfalse%
\ {\isacharparenleft}{\kern0pt}simp\ add{\isacharcolon}{\kern0pt}\ b{\isacharunderscore}{\kern0pt}def\ cfunc{\isacharunderscore}{\kern0pt}prod{\isacharunderscore}{\kern0pt}type\ x{\isacharunderscore}{\kern0pt}type{\isadigit{2}}{\isacharparenright}{\kern0pt}\isanewline
\ \ \ \ \isacommand{have}\isamarkupfalse%
\ yb{\isacharunderscore}{\kern0pt}type{\isacharcolon}{\kern0pt}\ {\isachardoublequoteopen}{\isasymlangle}b{\isacharcomma}{\kern0pt}y{\isasymrangle}\ {\isasymin}\isactrlsub c\ X\ {\isasymtimes}\isactrlsub c\ Z{\isachardoublequoteclose}\isanewline
\ \ \ \ \ \ \isacommand{by}\isamarkupfalse%
\ {\isacharparenleft}{\kern0pt}simp\ add{\isacharcolon}{\kern0pt}\ b{\isacharunderscore}{\kern0pt}def\ cfunc{\isacharunderscore}{\kern0pt}prod{\isacharunderscore}{\kern0pt}type\ y{\isacharunderscore}{\kern0pt}type{\isadigit{2}}{\isacharparenright}{\kern0pt}\isanewline
\ \ \ \ \isacommand{have}\isamarkupfalse%
\ {\isachardoublequoteopen}{\isacharparenleft}{\kern0pt}f\ {\isasymtimes}\isactrlsub f\ g{\isacharparenright}{\kern0pt}\ {\isasymcirc}\isactrlsub c\ {\isasymlangle}b{\isacharcomma}{\kern0pt}x{\isasymrangle}\ {\isacharequal}{\kern0pt}\ {\isasymlangle}f\ {\isasymcirc}\isactrlsub c\ b{\isacharcomma}{\kern0pt}g\ {\isasymcirc}\isactrlsub c\ x{\isasymrangle}{\isachardoublequoteclose}\isanewline
\ \ \ \ \ \ \isacommand{using}\isamarkupfalse%
\ b{\isacharunderscore}{\kern0pt}def\ cfunc{\isacharunderscore}{\kern0pt}cross{\isacharunderscore}{\kern0pt}prod{\isacharunderscore}{\kern0pt}comp{\isacharunderscore}{\kern0pt}cfunc{\isacharunderscore}{\kern0pt}prod\ type{\isacharunderscore}{\kern0pt}assms{\isacharparenleft}{\kern0pt}{\isadigit{1}}{\isacharparenright}{\kern0pt}\ type{\isacharunderscore}{\kern0pt}assms{\isacharparenleft}{\kern0pt}{\isadigit{2}}{\isacharparenright}{\kern0pt}\ x{\isacharunderscore}{\kern0pt}type{\isadigit{2}}\ \isacommand{by}\isamarkupfalse%
\ blast\isanewline
\ \ \ \ \isacommand{also}\isamarkupfalse%
\ \isacommand{have}\isamarkupfalse%
\ {\isachardoublequoteopen}{\isachardot}{\kern0pt}{\isachardot}{\kern0pt}{\isachardot}{\kern0pt}\ {\isacharequal}{\kern0pt}\ {\isasymlangle}f\ {\isasymcirc}\isactrlsub c\ b{\isacharcomma}{\kern0pt}g\ {\isasymcirc}\isactrlsub c\ x{\isasymrangle}{\isachardoublequoteclose}\isanewline
\ \ \ \ \ \ \isacommand{by}\isamarkupfalse%
\ {\isacharparenleft}{\kern0pt}simp\ add{\isacharcolon}{\kern0pt}\ equals{\isacharparenright}{\kern0pt}\isanewline
\ \ \ \ \isacommand{also}\isamarkupfalse%
\ \isacommand{have}\isamarkupfalse%
\ {\isachardoublequoteopen}{\isachardot}{\kern0pt}{\isachardot}{\kern0pt}{\isachardot}{\kern0pt}\ {\isacharequal}{\kern0pt}\ {\isacharparenleft}{\kern0pt}f\ {\isasymtimes}\isactrlsub f\ g{\isacharparenright}{\kern0pt}\ {\isasymcirc}\isactrlsub c\ {\isasymlangle}b{\isacharcomma}{\kern0pt}y{\isasymrangle}{\isachardoublequoteclose}\isanewline
\ \ \ \ \ \ \isacommand{using}\isamarkupfalse%
\ b{\isacharunderscore}{\kern0pt}def\ cfunc{\isacharunderscore}{\kern0pt}cross{\isacharunderscore}{\kern0pt}prod{\isacharunderscore}{\kern0pt}comp{\isacharunderscore}{\kern0pt}cfunc{\isacharunderscore}{\kern0pt}prod\ equals\ type{\isacharunderscore}{\kern0pt}assms{\isacharparenleft}{\kern0pt}{\isadigit{1}}{\isacharparenright}{\kern0pt}\ type{\isacharunderscore}{\kern0pt}assms{\isacharparenleft}{\kern0pt}{\isadigit{2}}{\isacharparenright}{\kern0pt}\ y{\isacharunderscore}{\kern0pt}type{\isadigit{2}}\ \isacommand{by}\isamarkupfalse%
\ auto\isanewline
\ \ \ \ \isacommand{then}\isamarkupfalse%
\ \isacommand{have}\isamarkupfalse%
\ {\isachardoublequoteopen}{\isasymlangle}b{\isacharcomma}{\kern0pt}x{\isasymrangle}\ {\isacharequal}{\kern0pt}\ {\isasymlangle}b{\isacharcomma}{\kern0pt}y{\isasymrangle}{\isachardoublequoteclose}\isanewline
\ \ \ \ \ \ \isacommand{by}\isamarkupfalse%
\ {\isacharparenleft}{\kern0pt}metis\ {\isacartoucheopen}{\isacharparenleft}{\kern0pt}f\ {\isasymtimes}\isactrlsub f\ g{\isacharparenright}{\kern0pt}\ {\isasymcirc}\isactrlsub c\ {\isasymlangle}b{\isacharcomma}{\kern0pt}x{\isasymrangle}\ {\isacharequal}{\kern0pt}\ {\isasymlangle}f\ {\isasymcirc}\isactrlsub c\ b{\isacharcomma}{\kern0pt}g\ {\isasymcirc}\isactrlsub c\ x{\isasymrangle}{\isacartoucheclose}\ cfunc{\isacharunderscore}{\kern0pt}type{\isacharunderscore}{\kern0pt}def\ fg{\isacharunderscore}{\kern0pt}inject\ fg{\isacharunderscore}{\kern0pt}type\ injective{\isacharunderscore}{\kern0pt}def\ xb{\isacharunderscore}{\kern0pt}type\ yb{\isacharunderscore}{\kern0pt}type{\isacharparenright}{\kern0pt}\isanewline
\ \ \ \ \isacommand{then}\isamarkupfalse%
\ \isacommand{show}\isamarkupfalse%
\ {\isachardoublequoteopen}x\ {\isacharequal}{\kern0pt}\ y{\isachardoublequoteclose}\isanewline
\ \ \ \ \ \ \isacommand{using}\isamarkupfalse%
\ b{\isacharunderscore}{\kern0pt}def\ cart{\isacharunderscore}{\kern0pt}prod{\isacharunderscore}{\kern0pt}eq{\isadigit{2}}\ x{\isacharunderscore}{\kern0pt}type{\isadigit{2}}\ y{\isacharunderscore}{\kern0pt}type{\isadigit{2}}\ \isacommand{by}\isamarkupfalse%
\ blast\isanewline
\ \ \isacommand{qed}\isamarkupfalse%
\isanewline
\isacommand{qed}\isamarkupfalse%
%
\endisatagproof
{\isafoldproof}%
%
\isadelimproof
%
\endisadelimproof
%
\begin{isamarkuptext}%
The next lemma shows that unless both domains are nonempty we gain no new information. 
That is, it will be the case that $f \times g$ is injective, and we cannot infer from this that $f$ or $g$ are
injective since $f \times g$ will be injective no matter what.%
\end{isamarkuptext}\isamarkuptrue%
\isacommand{lemma}\isamarkupfalse%
\ the{\isacharunderscore}{\kern0pt}nonempty{\isacharunderscore}{\kern0pt}assumption{\isacharunderscore}{\kern0pt}above{\isacharunderscore}{\kern0pt}is{\isacharunderscore}{\kern0pt}always{\isacharunderscore}{\kern0pt}required{\isacharcolon}{\kern0pt}\isanewline
\ \ \isakeyword{assumes}\ {\isachardoublequoteopen}f\ {\isacharcolon}{\kern0pt}\ X\ {\isasymrightarrow}\ Y{\isachardoublequoteclose}\ {\isachardoublequoteopen}g\ {\isacharcolon}{\kern0pt}\ Z\ {\isasymrightarrow}\ W{\isachardoublequoteclose}\isanewline
\ \ \isakeyword{assumes}\ {\isachardoublequoteopen}{\isasymnot}{\isacharparenleft}{\kern0pt}nonempty\ X{\isacharparenright}{\kern0pt}\ {\isasymor}\ {\isasymnot}{\isacharparenleft}{\kern0pt}nonempty\ Z{\isacharparenright}{\kern0pt}{\isachardoublequoteclose}\isanewline
\ \ \isakeyword{shows}\ {\isachardoublequoteopen}injective\ {\isacharparenleft}{\kern0pt}f\ {\isasymtimes}\isactrlsub f\ g{\isacharparenright}{\kern0pt}{\isachardoublequoteclose}\isanewline
%
\isadelimproof
\ \ %
\endisadelimproof
%
\isatagproof
\isacommand{unfolding}\isamarkupfalse%
\ injective{\isacharunderscore}{\kern0pt}def\ \isanewline
\isacommand{proof}\isamarkupfalse%
{\isacharparenleft}{\kern0pt}cases\ {\isachardoublequoteopen}nonempty{\isacharparenleft}{\kern0pt}X{\isacharparenright}{\kern0pt}{\isachardoublequoteclose}{\isacharcomma}{\kern0pt}\ safe{\isacharparenright}{\kern0pt}\isanewline
\ \ \isacommand{fix}\isamarkupfalse%
\ x\ y\isanewline
\ \ \isacommand{assume}\isamarkupfalse%
\ nonempty{\isacharcolon}{\kern0pt}\ \ {\isachardoublequoteopen}nonempty\ X{\isachardoublequoteclose}\isanewline
\ \ \isacommand{assume}\isamarkupfalse%
\ x{\isacharunderscore}{\kern0pt}type{\isacharcolon}{\kern0pt}\ {\isachardoublequoteopen}x\ \ {\isasymin}\isactrlsub c\ domain\ {\isacharparenleft}{\kern0pt}f\ {\isasymtimes}\isactrlsub f\ g{\isacharparenright}{\kern0pt}{\isachardoublequoteclose}\isanewline
\ \ \isacommand{assume}\isamarkupfalse%
\ {\isachardoublequoteopen}y\ {\isasymin}\isactrlsub c\ domain\ {\isacharparenleft}{\kern0pt}f\ {\isasymtimes}\isactrlsub f\ g{\isacharparenright}{\kern0pt}{\isachardoublequoteclose}\isanewline
\ \ \isacommand{then}\isamarkupfalse%
\ \isacommand{have}\isamarkupfalse%
\ {\isachardoublequoteopen}{\isasymnot}{\isacharparenleft}{\kern0pt}nonempty\ Z{\isacharparenright}{\kern0pt}{\isachardoublequoteclose}\isanewline
\ \ \ \ \isacommand{using}\isamarkupfalse%
\ nonempty\ assms{\isacharparenleft}{\kern0pt}{\isadigit{3}}{\isacharparenright}{\kern0pt}\ \isacommand{by}\isamarkupfalse%
\ blast\isanewline
\ \ \isacommand{have}\isamarkupfalse%
\ fg{\isacharunderscore}{\kern0pt}type{\isacharcolon}{\kern0pt}\ {\isachardoublequoteopen}f\ {\isasymtimes}\isactrlsub f\ g\ {\isacharcolon}{\kern0pt}\ X\ {\isasymtimes}\isactrlsub c\ Z\ {\isasymrightarrow}\ Y\ {\isasymtimes}\isactrlsub c\ W{\isachardoublequoteclose}\isanewline
\ \ \ \ \isacommand{by}\isamarkupfalse%
\ {\isacharparenleft}{\kern0pt}typecheck{\isacharunderscore}{\kern0pt}cfuncs{\isacharcomma}{\kern0pt}\ simp\ add{\isacharcolon}{\kern0pt}\ assms{\isacharparenleft}{\kern0pt}{\isadigit{1}}{\isacharcomma}{\kern0pt}{\isadigit{2}}{\isacharparenright}{\kern0pt}{\isacharparenright}{\kern0pt}\isanewline
\ \ \isacommand{then}\isamarkupfalse%
\ \isacommand{have}\isamarkupfalse%
\ {\isachardoublequoteopen}x\ \ {\isasymin}\isactrlsub c\ X\ {\isasymtimes}\isactrlsub c\ Z{\isachardoublequoteclose}\isanewline
\ \ \ \ \isacommand{using}\isamarkupfalse%
\ x{\isacharunderscore}{\kern0pt}type\ cfunc{\isacharunderscore}{\kern0pt}type{\isacharunderscore}{\kern0pt}def\ \isacommand{by}\isamarkupfalse%
\ auto\isanewline
\ \ \isacommand{then}\isamarkupfalse%
\ \isacommand{have}\isamarkupfalse%
\ {\isachardoublequoteopen}{\isasymexists}z{\isachardot}{\kern0pt}\ z{\isasymin}\isactrlsub c\ Z{\isachardoublequoteclose}\isanewline
\ \ \ \ \isacommand{using}\isamarkupfalse%
\ cart{\isacharunderscore}{\kern0pt}prod{\isacharunderscore}{\kern0pt}decomp\ \isacommand{by}\isamarkupfalse%
\ blast\isanewline
\ \ \isacommand{then}\isamarkupfalse%
\ \isacommand{have}\isamarkupfalse%
\ False\isanewline
\ \ \ \ \isacommand{using}\isamarkupfalse%
\ assms{\isacharparenleft}{\kern0pt}{\isadigit{3}}{\isacharparenright}{\kern0pt}\ nonempty\ nonempty{\isacharunderscore}{\kern0pt}def\ \isacommand{by}\isamarkupfalse%
\ blast\isanewline
\ \ \isacommand{then}\isamarkupfalse%
\ \isacommand{show}\isamarkupfalse%
\ {\isachardoublequoteopen}x{\isacharequal}{\kern0pt}y{\isachardoublequoteclose}\isanewline
\ \ \ \ \isacommand{by}\isamarkupfalse%
\ auto\isanewline
\isacommand{next}\isamarkupfalse%
\isanewline
\ \ \isacommand{fix}\isamarkupfalse%
\ x\ y\isanewline
\ \ \isacommand{assume}\isamarkupfalse%
\ X{\isacharunderscore}{\kern0pt}is{\isacharunderscore}{\kern0pt}empty{\isacharcolon}{\kern0pt}\ {\isachardoublequoteopen}{\isasymnot}\ nonempty\ X{\isachardoublequoteclose}\isanewline
\ \ \isacommand{assume}\isamarkupfalse%
\ x{\isacharunderscore}{\kern0pt}type{\isacharcolon}{\kern0pt}\ {\isachardoublequoteopen}x\ \ {\isasymin}\isactrlsub c\ domain\ {\isacharparenleft}{\kern0pt}f\ {\isasymtimes}\isactrlsub f\ g{\isacharparenright}{\kern0pt}{\isachardoublequoteclose}\isanewline
\ \ \isacommand{assume}\isamarkupfalse%
\ {\isachardoublequoteopen}y\ {\isasymin}\isactrlsub c\ domain{\isacharparenleft}{\kern0pt}f\ {\isasymtimes}\isactrlsub f\ g{\isacharparenright}{\kern0pt}{\isachardoublequoteclose}\isanewline
\ \ \isacommand{have}\isamarkupfalse%
\ fg{\isacharunderscore}{\kern0pt}type{\isacharcolon}{\kern0pt}\ {\isachardoublequoteopen}f\ {\isasymtimes}\isactrlsub f\ g\ {\isacharcolon}{\kern0pt}\ X\ {\isasymtimes}\isactrlsub c\ Z\ \ {\isasymrightarrow}\ Y\ {\isasymtimes}\isactrlsub c\ W{\isachardoublequoteclose}\isanewline
\ \ \ \ \isacommand{by}\isamarkupfalse%
\ {\isacharparenleft}{\kern0pt}typecheck{\isacharunderscore}{\kern0pt}cfuncs{\isacharcomma}{\kern0pt}\ simp\ add{\isacharcolon}{\kern0pt}\ assms{\isacharparenleft}{\kern0pt}{\isadigit{1}}{\isacharcomma}{\kern0pt}{\isadigit{2}}{\isacharparenright}{\kern0pt}{\isacharparenright}{\kern0pt}\isanewline
\ \ \isacommand{then}\isamarkupfalse%
\ \isacommand{have}\isamarkupfalse%
\ {\isachardoublequoteopen}x\ \ {\isasymin}\isactrlsub c\ X\ {\isasymtimes}\isactrlsub c\ Z{\isachardoublequoteclose}\isanewline
\ \ \ \ \isacommand{using}\isamarkupfalse%
\ x{\isacharunderscore}{\kern0pt}type\ cfunc{\isacharunderscore}{\kern0pt}type{\isacharunderscore}{\kern0pt}def\ \isacommand{by}\isamarkupfalse%
\ auto\isanewline
\ \ \isacommand{then}\isamarkupfalse%
\ \isacommand{have}\isamarkupfalse%
\ {\isachardoublequoteopen}{\isasymexists}z{\isachardot}{\kern0pt}\ z{\isasymin}\isactrlsub c\ X{\isachardoublequoteclose}\isanewline
\ \ \ \ \isacommand{using}\isamarkupfalse%
\ cart{\isacharunderscore}{\kern0pt}prod{\isacharunderscore}{\kern0pt}decomp\ \isacommand{by}\isamarkupfalse%
\ blast\isanewline
\ \ \isacommand{then}\isamarkupfalse%
\ \isacommand{have}\isamarkupfalse%
\ False\isanewline
\ \ \ \ \isacommand{using}\isamarkupfalse%
\ assms{\isacharparenleft}{\kern0pt}{\isadigit{3}}{\isacharparenright}{\kern0pt}\ X{\isacharunderscore}{\kern0pt}is{\isacharunderscore}{\kern0pt}empty\ nonempty{\isacharunderscore}{\kern0pt}def\ \isacommand{by}\isamarkupfalse%
\ blast\isanewline
\ \ \isacommand{then}\isamarkupfalse%
\ \isacommand{show}\isamarkupfalse%
\ {\isachardoublequoteopen}x{\isacharequal}{\kern0pt}y{\isachardoublequoteclose}\isanewline
\ \ \ \ \isacommand{by}\isamarkupfalse%
\ auto\isanewline
\isacommand{qed}\isamarkupfalse%
%
\endisatagproof
{\isafoldproof}%
%
\isadelimproof
%
\endisadelimproof
%
\isadelimdocument
%
\endisadelimdocument
%
\isatagdocument
%
\isamarkupsubsection{Surjectivity%
}
\isamarkuptrue%
%
\endisatagdocument
{\isafolddocument}%
%
\isadelimdocument
%
\endisadelimdocument
%
\begin{isamarkuptext}%
The definition below corresponds to Definition 2.1.28 in Halvorson.%
\end{isamarkuptext}\isamarkuptrue%
\isacommand{definition}\isamarkupfalse%
\ surjective\ {\isacharcolon}{\kern0pt}{\isacharcolon}{\kern0pt}\ {\isachardoublequoteopen}cfunc\ {\isasymRightarrow}\ bool{\isachardoublequoteclose}\ \isakeyword{where}\isanewline
\ {\isachardoublequoteopen}surjective\ f\ \ {\isasymlongleftrightarrow}\ {\isacharparenleft}{\kern0pt}{\isasymforall}y{\isachardot}{\kern0pt}\ y\ {\isasymin}\isactrlsub c\ codomain\ f\ {\isasymlongrightarrow}\ {\isacharparenleft}{\kern0pt}{\isasymexists}x{\isachardot}{\kern0pt}\ x\ {\isasymin}\isactrlsub c\ domain\ f\ {\isasymand}\ f\ {\isasymcirc}\isactrlsub c\ x\ {\isacharequal}{\kern0pt}\ y{\isacharparenright}{\kern0pt}{\isacharparenright}{\kern0pt}{\isachardoublequoteclose}\isanewline
\isanewline
\isacommand{lemma}\isamarkupfalse%
\ surjective{\isacharunderscore}{\kern0pt}def{\isadigit{2}}{\isacharcolon}{\kern0pt}\isanewline
\ \ \isakeyword{assumes}\ {\isachardoublequoteopen}f\ {\isacharcolon}{\kern0pt}\ X\ {\isasymrightarrow}\ Y{\isachardoublequoteclose}\isanewline
\ \ \isakeyword{shows}\ {\isachardoublequoteopen}surjective\ f\ \ {\isasymlongleftrightarrow}\ {\isacharparenleft}{\kern0pt}{\isasymforall}y{\isachardot}{\kern0pt}\ y\ {\isasymin}\isactrlsub c\ Y\ {\isasymlongrightarrow}\ {\isacharparenleft}{\kern0pt}{\isasymexists}x{\isachardot}{\kern0pt}\ x\ {\isasymin}\isactrlsub c\ X\ {\isasymand}\ f\ {\isasymcirc}\isactrlsub c\ x\ {\isacharequal}{\kern0pt}\ y{\isacharparenright}{\kern0pt}{\isacharparenright}{\kern0pt}{\isachardoublequoteclose}\isanewline
%
\isadelimproof
\ \ %
\endisadelimproof
%
\isatagproof
\isacommand{using}\isamarkupfalse%
\ assms\ \isacommand{unfolding}\isamarkupfalse%
\ surjective{\isacharunderscore}{\kern0pt}def\ cfunc{\isacharunderscore}{\kern0pt}type{\isacharunderscore}{\kern0pt}def\ \isacommand{by}\isamarkupfalse%
\ auto%
\endisatagproof
{\isafoldproof}%
%
\isadelimproof
%
\endisadelimproof
%
\begin{isamarkuptext}%
The lemma below corresponds to Exercise 2.1.30 in Halvorson.%
\end{isamarkuptext}\isamarkuptrue%
\isacommand{lemma}\isamarkupfalse%
\ surjective{\isacharunderscore}{\kern0pt}is{\isacharunderscore}{\kern0pt}epimorphism{\isacharcolon}{\kern0pt}\isanewline
\ \ {\isachardoublequoteopen}surjective\ f\ {\isasymLongrightarrow}\ epimorphism\ f{\isachardoublequoteclose}\isanewline
%
\isadelimproof
\ \ %
\endisadelimproof
%
\isatagproof
\isacommand{unfolding}\isamarkupfalse%
\ surjective{\isacharunderscore}{\kern0pt}def\ epimorphism{\isacharunderscore}{\kern0pt}def\isanewline
\isacommand{proof}\isamarkupfalse%
\ {\isacharparenleft}{\kern0pt}cases\ {\isachardoublequoteopen}nonempty\ {\isacharparenleft}{\kern0pt}codomain\ f{\isacharparenright}{\kern0pt}{\isachardoublequoteclose}{\isacharcomma}{\kern0pt}\ safe{\isacharparenright}{\kern0pt}\isanewline
\ \ \isacommand{fix}\isamarkupfalse%
\ g\ h\isanewline
\ \ \isacommand{assume}\isamarkupfalse%
\ f{\isacharunderscore}{\kern0pt}surj{\isacharcolon}{\kern0pt}\ {\isachardoublequoteopen}{\isasymforall}y{\isachardot}{\kern0pt}\ y\ {\isasymin}\isactrlsub c\ codomain\ f\ {\isasymlongrightarrow}\ {\isacharparenleft}{\kern0pt}{\isasymexists}x{\isachardot}{\kern0pt}\ x\ {\isasymin}\isactrlsub c\ domain\ f\ {\isasymand}\ f\ {\isasymcirc}\isactrlsub c\ x\ {\isacharequal}{\kern0pt}\ y{\isacharparenright}{\kern0pt}{\isachardoublequoteclose}\isanewline
\ \ \isacommand{assume}\isamarkupfalse%
\ d{\isacharunderscore}{\kern0pt}g{\isacharunderscore}{\kern0pt}eq{\isacharunderscore}{\kern0pt}cd{\isacharunderscore}{\kern0pt}f{\isacharcolon}{\kern0pt}\ {\isachardoublequoteopen}domain\ g\ {\isacharequal}{\kern0pt}\ codomain\ f{\isachardoublequoteclose}\isanewline
\ \ \isacommand{assume}\isamarkupfalse%
\ d{\isacharunderscore}{\kern0pt}h{\isacharunderscore}{\kern0pt}eq{\isacharunderscore}{\kern0pt}cd{\isacharunderscore}{\kern0pt}f{\isacharcolon}{\kern0pt}\ {\isachardoublequoteopen}domain\ h\ {\isacharequal}{\kern0pt}\ codomain\ f{\isachardoublequoteclose}\isanewline
\ \ \isacommand{assume}\isamarkupfalse%
\ gf{\isacharunderscore}{\kern0pt}eq{\isacharunderscore}{\kern0pt}hf{\isacharcolon}{\kern0pt}\ {\isachardoublequoteopen}g\ {\isasymcirc}\isactrlsub c\ f\ {\isacharequal}{\kern0pt}\ h\ {\isasymcirc}\isactrlsub c\ f{\isachardoublequoteclose}\isanewline
\ \ \isacommand{assume}\isamarkupfalse%
\ nonempty{\isacharcolon}{\kern0pt}\ {\isachardoublequoteopen}nonempty\ {\isacharparenleft}{\kern0pt}codomain\ f{\isacharparenright}{\kern0pt}{\isachardoublequoteclose}\isanewline
\isanewline
\ \ \isacommand{obtain}\isamarkupfalse%
\ X\ Y\ \isakeyword{where}\ f{\isacharunderscore}{\kern0pt}type{\isacharcolon}{\kern0pt}\ {\isachardoublequoteopen}f\ {\isacharcolon}{\kern0pt}\ X\ {\isasymrightarrow}\ Y{\isachardoublequoteclose}\isanewline
\ \ \ \ \isacommand{using}\isamarkupfalse%
\ nonempty\ cfunc{\isacharunderscore}{\kern0pt}type{\isacharunderscore}{\kern0pt}def\ f{\isacharunderscore}{\kern0pt}surj\ nonempty{\isacharunderscore}{\kern0pt}def\ \isacommand{by}\isamarkupfalse%
\ auto\isanewline
\ \ \isacommand{obtain}\isamarkupfalse%
\ A\ \isakeyword{where}\ g{\isacharunderscore}{\kern0pt}type{\isacharcolon}{\kern0pt}\ {\isachardoublequoteopen}g\ {\isacharcolon}{\kern0pt}\ Y\ {\isasymrightarrow}\ A{\isachardoublequoteclose}\ \isakeyword{and}\ h{\isacharunderscore}{\kern0pt}type{\isacharcolon}{\kern0pt}\ {\isachardoublequoteopen}h\ {\isacharcolon}{\kern0pt}\ Y\ {\isasymrightarrow}\ A{\isachardoublequoteclose}\isanewline
\ \ \ \ \isacommand{by}\isamarkupfalse%
\ {\isacharparenleft}{\kern0pt}metis\ cfunc{\isacharunderscore}{\kern0pt}type{\isacharunderscore}{\kern0pt}def\ codomain{\isacharunderscore}{\kern0pt}comp\ d{\isacharunderscore}{\kern0pt}g{\isacharunderscore}{\kern0pt}eq{\isacharunderscore}{\kern0pt}cd{\isacharunderscore}{\kern0pt}f\ d{\isacharunderscore}{\kern0pt}h{\isacharunderscore}{\kern0pt}eq{\isacharunderscore}{\kern0pt}cd{\isacharunderscore}{\kern0pt}f\ f{\isacharunderscore}{\kern0pt}type\ gf{\isacharunderscore}{\kern0pt}eq{\isacharunderscore}{\kern0pt}hf{\isacharparenright}{\kern0pt}\isanewline
\ \ \isacommand{show}\isamarkupfalse%
\ {\isachardoublequoteopen}g\ {\isacharequal}{\kern0pt}\ h{\isachardoublequoteclose}\isanewline
\ \ \isacommand{proof}\isamarkupfalse%
\ {\isacharparenleft}{\kern0pt}rule\ ccontr{\isacharparenright}{\kern0pt}\isanewline
\ \ \ \ \isacommand{assume}\isamarkupfalse%
\ {\isachardoublequoteopen}g\ {\isasymnoteq}\ h{\isachardoublequoteclose}\isanewline
\ \ \ \ \isacommand{then}\isamarkupfalse%
\ \isacommand{obtain}\isamarkupfalse%
\ y\ \isakeyword{where}\ y{\isacharunderscore}{\kern0pt}in{\isacharunderscore}{\kern0pt}X{\isacharcolon}{\kern0pt}\ {\isachardoublequoteopen}y\ {\isasymin}\isactrlsub c\ Y{\isachardoublequoteclose}\ \isakeyword{and}\ gy{\isacharunderscore}{\kern0pt}neq{\isacharunderscore}{\kern0pt}hy{\isacharcolon}{\kern0pt}\ {\isachardoublequoteopen}g\ {\isasymcirc}\isactrlsub c\ y\ {\isasymnoteq}\ h\ {\isasymcirc}\isactrlsub c\ y{\isachardoublequoteclose}\isanewline
\ \ \ \ \ \ \isacommand{using}\isamarkupfalse%
\ g{\isacharunderscore}{\kern0pt}type\ h{\isacharunderscore}{\kern0pt}type\ one{\isacharunderscore}{\kern0pt}separator\ \isacommand{by}\isamarkupfalse%
\ blast\isanewline
\ \ \ \ \isacommand{then}\isamarkupfalse%
\ \isacommand{obtain}\isamarkupfalse%
\ x\ \isakeyword{where}\ {\isachardoublequoteopen}x\ {\isasymin}\isactrlsub c\ X{\isachardoublequoteclose}\ \isakeyword{and}\ {\isachardoublequoteopen}f\ {\isasymcirc}\isactrlsub c\ x\ {\isacharequal}{\kern0pt}\ y{\isachardoublequoteclose}\isanewline
\ \ \ \ \ \ \isacommand{using}\isamarkupfalse%
\ cfunc{\isacharunderscore}{\kern0pt}type{\isacharunderscore}{\kern0pt}def\ f{\isacharunderscore}{\kern0pt}surj\ f{\isacharunderscore}{\kern0pt}type\ \isacommand{by}\isamarkupfalse%
\ auto\isanewline
\ \ \ \ \isacommand{then}\isamarkupfalse%
\ \isacommand{have}\isamarkupfalse%
\ {\isachardoublequoteopen}g\ {\isasymcirc}\isactrlsub c\ f\ {\isasymnoteq}\ h\ {\isasymcirc}\isactrlsub c\ f{\isachardoublequoteclose}\isanewline
\ \ \ \ \ \ \isacommand{using}\isamarkupfalse%
\ comp{\isacharunderscore}{\kern0pt}associative{\isadigit{2}}\ f{\isacharunderscore}{\kern0pt}type\ g{\isacharunderscore}{\kern0pt}type\ gy{\isacharunderscore}{\kern0pt}neq{\isacharunderscore}{\kern0pt}hy\ h{\isacharunderscore}{\kern0pt}type\ \isacommand{by}\isamarkupfalse%
\ auto\isanewline
\ \ \ \ \isacommand{then}\isamarkupfalse%
\ \isacommand{show}\isamarkupfalse%
\ False\isanewline
\ \ \ \ \ \ \isacommand{using}\isamarkupfalse%
\ gf{\isacharunderscore}{\kern0pt}eq{\isacharunderscore}{\kern0pt}hf\ \isacommand{by}\isamarkupfalse%
\ auto\isanewline
\ \ \isacommand{qed}\isamarkupfalse%
\isanewline
\isacommand{next}\isamarkupfalse%
\isanewline
\ \ \isacommand{fix}\isamarkupfalse%
\ g\ h\isanewline
\ \ \isacommand{assume}\isamarkupfalse%
\ empty{\isacharcolon}{\kern0pt}\ {\isachardoublequoteopen}{\isasymnot}\ nonempty\ {\isacharparenleft}{\kern0pt}codomain\ f{\isacharparenright}{\kern0pt}{\isachardoublequoteclose}\isanewline
\ \ \isacommand{assume}\isamarkupfalse%
\ {\isachardoublequoteopen}domain\ g\ {\isacharequal}{\kern0pt}\ codomain\ f{\isachardoublequoteclose}\ {\isachardoublequoteopen}domain\ h\ {\isacharequal}{\kern0pt}\ codomain\ f{\isachardoublequoteclose}\isanewline
\ \ \isacommand{then}\isamarkupfalse%
\ \isacommand{show}\isamarkupfalse%
\ {\isachardoublequoteopen}g\ {\isasymcirc}\isactrlsub c\ f\ {\isacharequal}{\kern0pt}\ h\ {\isasymcirc}\isactrlsub c\ f\ {\isasymLongrightarrow}\ g\ {\isacharequal}{\kern0pt}\ h{\isachardoublequoteclose}\isanewline
\ \ \ \ \isacommand{by}\isamarkupfalse%
\ {\isacharparenleft}{\kern0pt}metis\ empty\ cfunc{\isacharunderscore}{\kern0pt}type{\isacharunderscore}{\kern0pt}def\ codomain{\isacharunderscore}{\kern0pt}comp\ nonempty{\isacharunderscore}{\kern0pt}def\ one{\isacharunderscore}{\kern0pt}separator{\isacharparenright}{\kern0pt}\isanewline
\isacommand{qed}\isamarkupfalse%
%
\endisatagproof
{\isafoldproof}%
%
\isadelimproof
%
\endisadelimproof
%
\begin{isamarkuptext}%
The lemma below corresponds to Proposition 2.2.10 in Halvorson.%
\end{isamarkuptext}\isamarkuptrue%
\isacommand{lemma}\isamarkupfalse%
\ cfunc{\isacharunderscore}{\kern0pt}cross{\isacharunderscore}{\kern0pt}prod{\isacharunderscore}{\kern0pt}surj{\isacharcolon}{\kern0pt}\isanewline
\ \ \isakeyword{assumes}\ type{\isacharunderscore}{\kern0pt}assms{\isacharcolon}{\kern0pt}\ {\isachardoublequoteopen}f\ {\isacharcolon}{\kern0pt}\ A\ {\isasymrightarrow}\ C{\isachardoublequoteclose}\ {\isachardoublequoteopen}g\ {\isacharcolon}{\kern0pt}\ B\ {\isasymrightarrow}\ D{\isachardoublequoteclose}\isanewline
\ \ \isakeyword{assumes}\ f{\isacharunderscore}{\kern0pt}surj{\isacharcolon}{\kern0pt}\ {\isachardoublequoteopen}surjective\ f{\isachardoublequoteclose}\ \isakeyword{and}\ g{\isacharunderscore}{\kern0pt}surj{\isacharcolon}{\kern0pt}\ {\isachardoublequoteopen}surjective\ g{\isachardoublequoteclose}\isanewline
\ \ \isakeyword{shows}\ {\isachardoublequoteopen}surjective\ {\isacharparenleft}{\kern0pt}f\ {\isasymtimes}\isactrlsub f\ g{\isacharparenright}{\kern0pt}{\isachardoublequoteclose}\isanewline
%
\isadelimproof
\ \ %
\endisadelimproof
%
\isatagproof
\isacommand{unfolding}\isamarkupfalse%
\ surjective{\isacharunderscore}{\kern0pt}def\isanewline
\isacommand{proof}\isamarkupfalse%
{\isacharparenleft}{\kern0pt}clarify{\isacharparenright}{\kern0pt}\isanewline
\ \ \isacommand{fix}\isamarkupfalse%
\ y\isanewline
\ \ \isacommand{assume}\isamarkupfalse%
\ y{\isacharunderscore}{\kern0pt}type{\isacharcolon}{\kern0pt}\ {\isachardoublequoteopen}y\ {\isasymin}\isactrlsub c\ codomain\ {\isacharparenleft}{\kern0pt}f\ {\isasymtimes}\isactrlsub f\ g{\isacharparenright}{\kern0pt}{\isachardoublequoteclose}\isanewline
\ \ \isacommand{have}\isamarkupfalse%
\ fg{\isacharunderscore}{\kern0pt}type{\isacharcolon}{\kern0pt}\ {\isachardoublequoteopen}f\ {\isasymtimes}\isactrlsub f\ g{\isacharcolon}{\kern0pt}\ A\ {\isasymtimes}\isactrlsub c\ \ B\ {\isasymrightarrow}\ C\ {\isasymtimes}\isactrlsub c\ D{\isachardoublequoteclose}\isanewline
\ \ \ \ \isacommand{using}\isamarkupfalse%
\ assms\ \ \isacommand{by}\isamarkupfalse%
\ typecheck{\isacharunderscore}{\kern0pt}cfuncs\ \ \ \ \isanewline
\ \ \isacommand{then}\isamarkupfalse%
\ \isacommand{have}\isamarkupfalse%
\ {\isachardoublequoteopen}y\ {\isasymin}\isactrlsub c\ C\ {\isasymtimes}\isactrlsub c\ D{\isachardoublequoteclose}\isanewline
\ \ \ \ \isacommand{using}\isamarkupfalse%
\ cfunc{\isacharunderscore}{\kern0pt}type{\isacharunderscore}{\kern0pt}def\ y{\isacharunderscore}{\kern0pt}type\ \isacommand{by}\isamarkupfalse%
\ auto\isanewline
\ \ \isacommand{then}\isamarkupfalse%
\ \isacommand{have}\isamarkupfalse%
\ {\isachardoublequoteopen}{\isasymexists}\ c\ d{\isachardot}{\kern0pt}\ c\ {\isasymin}\isactrlsub c\ C\ {\isasymand}\ d\ {\isasymin}\isactrlsub c\ D\ {\isasymand}\ y\ {\isacharequal}{\kern0pt}\ {\isasymlangle}c{\isacharcomma}{\kern0pt}d{\isasymrangle}{\isachardoublequoteclose}\isanewline
\ \ \ \ \isacommand{using}\isamarkupfalse%
\ cart{\isacharunderscore}{\kern0pt}prod{\isacharunderscore}{\kern0pt}decomp\ \isacommand{by}\isamarkupfalse%
\ blast\isanewline
\ \ \isacommand{then}\isamarkupfalse%
\ \isacommand{obtain}\isamarkupfalse%
\ c\ d\ \isakeyword{where}\ y{\isacharunderscore}{\kern0pt}def{\isacharcolon}{\kern0pt}\ {\isachardoublequoteopen}c\ {\isasymin}\isactrlsub c\ C\ {\isasymand}\ d\ {\isasymin}\isactrlsub c\ D\ {\isasymand}\ y\ {\isacharequal}{\kern0pt}\ {\isasymlangle}c{\isacharcomma}{\kern0pt}d{\isasymrangle}{\isachardoublequoteclose}\isanewline
\ \ \ \ \isacommand{by}\isamarkupfalse%
\ blast\isanewline
\ \ \isacommand{then}\isamarkupfalse%
\ \isacommand{have}\isamarkupfalse%
\ {\isachardoublequoteopen}{\isasymexists}\ a\ b{\isachardot}{\kern0pt}\ a\ {\isasymin}\isactrlsub c\ A\ {\isasymand}\ b\ {\isasymin}\isactrlsub c\ B\ {\isasymand}\ f\ {\isasymcirc}\isactrlsub c\ a\ {\isacharequal}{\kern0pt}\ c\ {\isasymand}\ g\ {\isasymcirc}\isactrlsub c\ b\ {\isacharequal}{\kern0pt}\ d{\isachardoublequoteclose}\isanewline
\ \ \ \ \isacommand{by}\isamarkupfalse%
\ {\isacharparenleft}{\kern0pt}metis\ cfunc{\isacharunderscore}{\kern0pt}type{\isacharunderscore}{\kern0pt}def\ f{\isacharunderscore}{\kern0pt}surj\ g{\isacharunderscore}{\kern0pt}surj\ surjective{\isacharunderscore}{\kern0pt}def\ type{\isacharunderscore}{\kern0pt}assms{\isacharparenright}{\kern0pt}\isanewline
\ \ \isacommand{then}\isamarkupfalse%
\ \isacommand{obtain}\isamarkupfalse%
\ a\ b\ \isakeyword{where}\ ab{\isacharunderscore}{\kern0pt}def{\isacharcolon}{\kern0pt}\ {\isachardoublequoteopen}a\ {\isasymin}\isactrlsub c\ A\ {\isasymand}\ b\ {\isasymin}\isactrlsub c\ B\ {\isasymand}\ f\ {\isasymcirc}\isactrlsub c\ a\ {\isacharequal}{\kern0pt}\ c\ {\isasymand}\ g\ {\isasymcirc}\isactrlsub c\ b\ {\isacharequal}{\kern0pt}\ d{\isachardoublequoteclose}\isanewline
\ \ \ \ \isacommand{by}\isamarkupfalse%
\ blast\isanewline
\ \ \isacommand{then}\isamarkupfalse%
\ \isacommand{obtain}\isamarkupfalse%
\ x\ \isakeyword{where}\ x{\isacharunderscore}{\kern0pt}def{\isacharcolon}{\kern0pt}\ {\isachardoublequoteopen}x\ {\isacharequal}{\kern0pt}\ {\isasymlangle}a{\isacharcomma}{\kern0pt}b{\isasymrangle}{\isachardoublequoteclose}\isanewline
\ \ \ \ \isacommand{by}\isamarkupfalse%
\ auto\isanewline
\ \ \isacommand{have}\isamarkupfalse%
\ x{\isacharunderscore}{\kern0pt}type{\isacharcolon}{\kern0pt}\ {\isachardoublequoteopen}x\ {\isasymin}\isactrlsub c\ domain\ {\isacharparenleft}{\kern0pt}f\ {\isasymtimes}\isactrlsub f\ g{\isacharparenright}{\kern0pt}{\isachardoublequoteclose}\isanewline
\ \ \ \ \isacommand{using}\isamarkupfalse%
\ ab{\isacharunderscore}{\kern0pt}def\ cfunc{\isacharunderscore}{\kern0pt}prod{\isacharunderscore}{\kern0pt}type\ cfunc{\isacharunderscore}{\kern0pt}type{\isacharunderscore}{\kern0pt}def\ fg{\isacharunderscore}{\kern0pt}type\ x{\isacharunderscore}{\kern0pt}def\ \isacommand{by}\isamarkupfalse%
\ auto\isanewline
\ \ \isacommand{have}\isamarkupfalse%
\ {\isachardoublequoteopen}{\isacharparenleft}{\kern0pt}f\ {\isasymtimes}\isactrlsub f\ g{\isacharparenright}{\kern0pt}\ {\isasymcirc}\isactrlsub c\ x\ {\isacharequal}{\kern0pt}\ y{\isachardoublequoteclose}\isanewline
\ \ \ \ \isacommand{using}\isamarkupfalse%
\ ab{\isacharunderscore}{\kern0pt}def\ cfunc{\isacharunderscore}{\kern0pt}cross{\isacharunderscore}{\kern0pt}prod{\isacharunderscore}{\kern0pt}comp{\isacharunderscore}{\kern0pt}cfunc{\isacharunderscore}{\kern0pt}prod\ type{\isacharunderscore}{\kern0pt}assms{\isacharparenleft}{\kern0pt}{\isadigit{1}}{\isacharparenright}{\kern0pt}\ type{\isacharunderscore}{\kern0pt}assms{\isacharparenleft}{\kern0pt}{\isadigit{2}}{\isacharparenright}{\kern0pt}\ x{\isacharunderscore}{\kern0pt}def\ y{\isacharunderscore}{\kern0pt}def\ \isacommand{by}\isamarkupfalse%
\ blast\isanewline
\ \ \isacommand{then}\isamarkupfalse%
\ \isacommand{show}\isamarkupfalse%
\ {\isachardoublequoteopen}{\isasymexists}x{\isachardot}{\kern0pt}\ x\ {\isasymin}\isactrlsub c\ domain\ {\isacharparenleft}{\kern0pt}f\ {\isasymtimes}\isactrlsub f\ g{\isacharparenright}{\kern0pt}\ {\isasymand}\ {\isacharparenleft}{\kern0pt}f\ {\isasymtimes}\isactrlsub f\ g{\isacharparenright}{\kern0pt}\ {\isasymcirc}\isactrlsub c\ x\ {\isacharequal}{\kern0pt}\ y{\isachardoublequoteclose}\isanewline
\ \ \ \ \isacommand{using}\isamarkupfalse%
\ x{\isacharunderscore}{\kern0pt}type\ \isacommand{by}\isamarkupfalse%
\ blast\isanewline
\isacommand{qed}\isamarkupfalse%
%
\endisatagproof
{\isafoldproof}%
%
\isadelimproof
\isanewline
%
\endisadelimproof
\isanewline
\isacommand{lemma}\isamarkupfalse%
\ cfunc{\isacharunderscore}{\kern0pt}cross{\isacharunderscore}{\kern0pt}prod{\isacharunderscore}{\kern0pt}surj{\isacharunderscore}{\kern0pt}converse{\isacharcolon}{\kern0pt}\isanewline
\ \ \isakeyword{assumes}\ type{\isacharunderscore}{\kern0pt}assms{\isacharcolon}{\kern0pt}\ {\isachardoublequoteopen}f\ {\isacharcolon}{\kern0pt}\ A\ {\isasymrightarrow}\ C{\isachardoublequoteclose}\ {\isachardoublequoteopen}g\ {\isacharcolon}{\kern0pt}\ B\ {\isasymrightarrow}\ D{\isachardoublequoteclose}\isanewline
\ \ \isakeyword{assumes}\ nonempty{\isacharcolon}{\kern0pt}\ {\isachardoublequoteopen}nonempty\ C\ {\isasymand}\ nonempty\ D{\isachardoublequoteclose}\isanewline
\ \ \isakeyword{assumes}\ {\isachardoublequoteopen}surjective\ {\isacharparenleft}{\kern0pt}f\ {\isasymtimes}\isactrlsub f\ g{\isacharparenright}{\kern0pt}{\isachardoublequoteclose}\isanewline
\ \ \isakeyword{shows}\ {\isachardoublequoteopen}surjective\ f\ {\isasymand}\ surjective\ g{\isachardoublequoteclose}\isanewline
%
\isadelimproof
\ \ %
\endisadelimproof
%
\isatagproof
\isacommand{unfolding}\isamarkupfalse%
\ surjective{\isacharunderscore}{\kern0pt}def\isanewline
\isacommand{proof}\isamarkupfalse%
{\isacharparenleft}{\kern0pt}safe{\isacharparenright}{\kern0pt}\isanewline
\ \ \isacommand{fix}\isamarkupfalse%
\ c\ \isanewline
\ \ \isacommand{assume}\isamarkupfalse%
\ c{\isacharunderscore}{\kern0pt}type{\isacharbrackleft}{\kern0pt}type{\isacharunderscore}{\kern0pt}rule{\isacharbrackright}{\kern0pt}{\isacharcolon}{\kern0pt}\ {\isachardoublequoteopen}c\ {\isasymin}\isactrlsub c\ codomain\ f{\isachardoublequoteclose}\isanewline
\ \ \isacommand{then}\isamarkupfalse%
\ \isacommand{have}\isamarkupfalse%
\ c{\isacharunderscore}{\kern0pt}type{\isadigit{2}}{\isacharcolon}{\kern0pt}\ \ {\isachardoublequoteopen}c\ {\isasymin}\isactrlsub c\ C{\isachardoublequoteclose}\isanewline
\ \ \ \ \isacommand{using}\isamarkupfalse%
\ cfunc{\isacharunderscore}{\kern0pt}type{\isacharunderscore}{\kern0pt}def\ type{\isacharunderscore}{\kern0pt}assms{\isacharparenleft}{\kern0pt}{\isadigit{1}}{\isacharparenright}{\kern0pt}\ \isacommand{by}\isamarkupfalse%
\ auto\isanewline
\ \ \isacommand{obtain}\isamarkupfalse%
\ d\ \isakeyword{where}\ d{\isacharunderscore}{\kern0pt}type{\isacharbrackleft}{\kern0pt}type{\isacharunderscore}{\kern0pt}rule{\isacharbrackright}{\kern0pt}{\isacharcolon}{\kern0pt}\ {\isachardoublequoteopen}d\ \ {\isasymin}\isactrlsub c\ D{\isachardoublequoteclose}\ \isanewline
\ \ \ \ \isacommand{using}\isamarkupfalse%
\ nonempty\ nonempty{\isacharunderscore}{\kern0pt}def\ \isacommand{by}\isamarkupfalse%
\ blast\isanewline
\ \ \isacommand{then}\isamarkupfalse%
\ \isacommand{obtain}\isamarkupfalse%
\ ab\ \isakeyword{where}\ ab{\isacharunderscore}{\kern0pt}type{\isacharbrackleft}{\kern0pt}type{\isacharunderscore}{\kern0pt}rule{\isacharbrackright}{\kern0pt}{\isacharcolon}{\kern0pt}\ {\isachardoublequoteopen}ab\ {\isasymin}\isactrlsub c\ A\ {\isasymtimes}\isactrlsub c\ B{\isachardoublequoteclose}\ \isakeyword{and}\ ab{\isacharunderscore}{\kern0pt}def{\isacharcolon}{\kern0pt}\ {\isachardoublequoteopen}{\isacharparenleft}{\kern0pt}f\ {\isasymtimes}\isactrlsub f\ g{\isacharparenright}{\kern0pt}\ {\isasymcirc}\isactrlsub c\ ab\ {\isacharequal}{\kern0pt}\ {\isasymlangle}c{\isacharcomma}{\kern0pt}\ d{\isasymrangle}{\isachardoublequoteclose}\isanewline
\ \ \ \ \isacommand{using}\isamarkupfalse%
\ assms\ \isacommand{by}\isamarkupfalse%
\ {\isacharparenleft}{\kern0pt}typecheck{\isacharunderscore}{\kern0pt}cfuncs{\isacharcomma}{\kern0pt}\ metis\ assms{\isacharparenleft}{\kern0pt}{\isadigit{4}}{\isacharparenright}{\kern0pt}\ cfunc{\isacharunderscore}{\kern0pt}type{\isacharunderscore}{\kern0pt}def\ surjective{\isacharunderscore}{\kern0pt}def{\isadigit{2}}{\isacharparenright}{\kern0pt}\isanewline
\ \ \isacommand{then}\isamarkupfalse%
\ \isacommand{obtain}\isamarkupfalse%
\ a\ b\ \isakeyword{where}\ a{\isacharunderscore}{\kern0pt}type{\isacharbrackleft}{\kern0pt}type{\isacharunderscore}{\kern0pt}rule{\isacharbrackright}{\kern0pt}{\isacharcolon}{\kern0pt}\ {\isachardoublequoteopen}a\ {\isasymin}\isactrlsub c\ A{\isachardoublequoteclose}\ \isakeyword{and}\ b{\isacharunderscore}{\kern0pt}type{\isacharbrackleft}{\kern0pt}type{\isacharunderscore}{\kern0pt}rule{\isacharbrackright}{\kern0pt}{\isacharcolon}{\kern0pt}\ {\isachardoublequoteopen}b\ {\isasymin}\isactrlsub c\ B{\isachardoublequoteclose}\ \isakeyword{and}\ ab{\isacharunderscore}{\kern0pt}def{\isadigit{2}}{\isacharcolon}{\kern0pt}\ {\isachardoublequoteopen}ab\ {\isacharequal}{\kern0pt}\ {\isasymlangle}a{\isacharcomma}{\kern0pt}b{\isasymrangle}{\isachardoublequoteclose}\isanewline
\ \ \ \ \isacommand{using}\isamarkupfalse%
\ cart{\isacharunderscore}{\kern0pt}prod{\isacharunderscore}{\kern0pt}decomp\ \isacommand{by}\isamarkupfalse%
\ blast\isanewline
\ \ \isacommand{have}\isamarkupfalse%
\ \ {\isachardoublequoteopen}a\ {\isasymin}\isactrlsub c\ domain\ f\ {\isasymand}\ f\ {\isasymcirc}\isactrlsub c\ a\ {\isacharequal}{\kern0pt}\ c{\isachardoublequoteclose}\isanewline
\ \ \ \ \isacommand{using}\isamarkupfalse%
\ ab{\isacharunderscore}{\kern0pt}def\ ab{\isacharunderscore}{\kern0pt}def{\isadigit{2}}\ b{\isacharunderscore}{\kern0pt}type\ cfunc{\isacharunderscore}{\kern0pt}cross{\isacharunderscore}{\kern0pt}prod{\isacharunderscore}{\kern0pt}comp{\isacharunderscore}{\kern0pt}cfunc{\isacharunderscore}{\kern0pt}prod\ cfunc{\isacharunderscore}{\kern0pt}type{\isacharunderscore}{\kern0pt}def\isanewline
\ \ \ \ \ \ \ \ \ \ comp{\isacharunderscore}{\kern0pt}type\ d{\isacharunderscore}{\kern0pt}type\ cart{\isacharunderscore}{\kern0pt}prod{\isacharunderscore}{\kern0pt}eq{\isadigit{2}}\ type{\isacharunderscore}{\kern0pt}assms\ \isacommand{by}\isamarkupfalse%
\ {\isacharparenleft}{\kern0pt}typecheck{\isacharunderscore}{\kern0pt}cfuncs{\isacharcomma}{\kern0pt}\ auto{\isacharparenright}{\kern0pt}\isanewline
\ \ \isacommand{then}\isamarkupfalse%
\ \isacommand{show}\isamarkupfalse%
\ {\isachardoublequoteopen}{\isasymexists}x{\isachardot}{\kern0pt}\ x\ {\isasymin}\isactrlsub c\ domain\ f\ {\isasymand}\ f\ {\isasymcirc}\isactrlsub c\ x\ {\isacharequal}{\kern0pt}\ c{\isachardoublequoteclose}\isanewline
\ \ \ \ \isacommand{by}\isamarkupfalse%
\ blast\isanewline
\isacommand{next}\isamarkupfalse%
\isanewline
\ \ \isacommand{fix}\isamarkupfalse%
\ d\ \isanewline
\ \ \isacommand{assume}\isamarkupfalse%
\ d{\isacharunderscore}{\kern0pt}type{\isacharbrackleft}{\kern0pt}type{\isacharunderscore}{\kern0pt}rule{\isacharbrackright}{\kern0pt}{\isacharcolon}{\kern0pt}\ {\isachardoublequoteopen}d\ {\isasymin}\isactrlsub c\ codomain\ g{\isachardoublequoteclose}\isanewline
\ \ \isacommand{then}\isamarkupfalse%
\ \isacommand{have}\isamarkupfalse%
\ y{\isacharunderscore}{\kern0pt}type{\isadigit{2}}{\isacharcolon}{\kern0pt}\ \ {\isachardoublequoteopen}d\ {\isasymin}\isactrlsub c\ D{\isachardoublequoteclose}\isanewline
\ \ \ \ \isacommand{using}\isamarkupfalse%
\ cfunc{\isacharunderscore}{\kern0pt}type{\isacharunderscore}{\kern0pt}def\ type{\isacharunderscore}{\kern0pt}assms{\isacharparenleft}{\kern0pt}{\isadigit{2}}{\isacharparenright}{\kern0pt}\ \isacommand{by}\isamarkupfalse%
\ auto\isanewline
\ \ \isacommand{obtain}\isamarkupfalse%
\ c\ \isakeyword{where}\ d{\isacharunderscore}{\kern0pt}type{\isacharbrackleft}{\kern0pt}type{\isacharunderscore}{\kern0pt}rule{\isacharbrackright}{\kern0pt}{\isacharcolon}{\kern0pt}\ {\isachardoublequoteopen}c\ \ {\isasymin}\isactrlsub c\ C{\isachardoublequoteclose}\ \isanewline
\ \ \ \ \isacommand{using}\isamarkupfalse%
\ nonempty\ nonempty{\isacharunderscore}{\kern0pt}def\ \isacommand{by}\isamarkupfalse%
\ blast\isanewline
\ \ \isacommand{then}\isamarkupfalse%
\ \isacommand{obtain}\isamarkupfalse%
\ ab\ \isakeyword{where}\ ab{\isacharunderscore}{\kern0pt}type{\isacharbrackleft}{\kern0pt}type{\isacharunderscore}{\kern0pt}rule{\isacharbrackright}{\kern0pt}{\isacharcolon}{\kern0pt}\ {\isachardoublequoteopen}ab\ {\isasymin}\isactrlsub c\ A\ {\isasymtimes}\isactrlsub c\ B{\isachardoublequoteclose}\ \isakeyword{and}\ ab{\isacharunderscore}{\kern0pt}def{\isacharcolon}{\kern0pt}\ {\isachardoublequoteopen}{\isacharparenleft}{\kern0pt}f\ {\isasymtimes}\isactrlsub f\ g{\isacharparenright}{\kern0pt}\ {\isasymcirc}\isactrlsub c\ ab\ {\isacharequal}{\kern0pt}\ {\isasymlangle}c{\isacharcomma}{\kern0pt}\ d{\isasymrangle}{\isachardoublequoteclose}\isanewline
\ \ \ \ \isacommand{using}\isamarkupfalse%
\ assms\ \isacommand{by}\isamarkupfalse%
\ {\isacharparenleft}{\kern0pt}typecheck{\isacharunderscore}{\kern0pt}cfuncs{\isacharcomma}{\kern0pt}\ metis\ assms{\isacharparenleft}{\kern0pt}{\isadigit{4}}{\isacharparenright}{\kern0pt}\ cfunc{\isacharunderscore}{\kern0pt}type{\isacharunderscore}{\kern0pt}def\ surjective{\isacharunderscore}{\kern0pt}def{\isadigit{2}}{\isacharparenright}{\kern0pt}\isanewline
\ \ \isacommand{then}\isamarkupfalse%
\ \isacommand{obtain}\isamarkupfalse%
\ a\ b\ \isakeyword{where}\ a{\isacharunderscore}{\kern0pt}type{\isacharbrackleft}{\kern0pt}type{\isacharunderscore}{\kern0pt}rule{\isacharbrackright}{\kern0pt}{\isacharcolon}{\kern0pt}\ {\isachardoublequoteopen}a\ {\isasymin}\isactrlsub c\ A{\isachardoublequoteclose}\ \isakeyword{and}\ b{\isacharunderscore}{\kern0pt}type{\isacharbrackleft}{\kern0pt}type{\isacharunderscore}{\kern0pt}rule{\isacharbrackright}{\kern0pt}{\isacharcolon}{\kern0pt}\ {\isachardoublequoteopen}b\ {\isasymin}\isactrlsub c\ B{\isachardoublequoteclose}\ \isakeyword{and}\ ab{\isacharunderscore}{\kern0pt}def{\isadigit{2}}{\isacharcolon}{\kern0pt}\ {\isachardoublequoteopen}ab\ {\isacharequal}{\kern0pt}\ {\isasymlangle}a{\isacharcomma}{\kern0pt}b{\isasymrangle}{\isachardoublequoteclose}\isanewline
\ \ \ \ \isacommand{using}\isamarkupfalse%
\ cart{\isacharunderscore}{\kern0pt}prod{\isacharunderscore}{\kern0pt}decomp\ \isacommand{by}\isamarkupfalse%
\ blast\isanewline
\ \ \isacommand{then}\isamarkupfalse%
\ \isacommand{obtain}\isamarkupfalse%
\ a\ b\ \isakeyword{where}\ a{\isacharunderscore}{\kern0pt}type{\isacharbrackleft}{\kern0pt}type{\isacharunderscore}{\kern0pt}rule{\isacharbrackright}{\kern0pt}{\isacharcolon}{\kern0pt}\ {\isachardoublequoteopen}a\ {\isasymin}\isactrlsub c\ A{\isachardoublequoteclose}\ \isakeyword{and}\ b{\isacharunderscore}{\kern0pt}type{\isacharbrackleft}{\kern0pt}type{\isacharunderscore}{\kern0pt}rule{\isacharbrackright}{\kern0pt}{\isacharcolon}{\kern0pt}\ {\isachardoublequoteopen}b\ {\isasymin}\isactrlsub c\ B{\isachardoublequoteclose}\ \isakeyword{and}\ ab{\isacharunderscore}{\kern0pt}def{\isadigit{2}}{\isacharcolon}{\kern0pt}\ {\isachardoublequoteopen}ab\ {\isacharequal}{\kern0pt}\ {\isasymlangle}a{\isacharcomma}{\kern0pt}b{\isasymrangle}{\isachardoublequoteclose}\isanewline
\ \ \ \ \isacommand{using}\isamarkupfalse%
\ cart{\isacharunderscore}{\kern0pt}prod{\isacharunderscore}{\kern0pt}decomp\ \isacommand{by}\isamarkupfalse%
\ blast\isanewline
\ \ \isacommand{have}\isamarkupfalse%
\ \ {\isachardoublequoteopen}b\ {\isasymin}\isactrlsub c\ domain\ g\ {\isasymand}\ g\ {\isasymcirc}\isactrlsub c\ b\ {\isacharequal}{\kern0pt}\ d{\isachardoublequoteclose}\isanewline
\ \ \ \ \isacommand{using}\isamarkupfalse%
\ a{\isacharunderscore}{\kern0pt}type\ ab{\isacharunderscore}{\kern0pt}def\ ab{\isacharunderscore}{\kern0pt}def{\isadigit{2}}\ cfunc{\isacharunderscore}{\kern0pt}cross{\isacharunderscore}{\kern0pt}prod{\isacharunderscore}{\kern0pt}comp{\isacharunderscore}{\kern0pt}cfunc{\isacharunderscore}{\kern0pt}prod\ cfunc{\isacharunderscore}{\kern0pt}type{\isacharunderscore}{\kern0pt}def\ comp{\isacharunderscore}{\kern0pt}type\ d{\isacharunderscore}{\kern0pt}type\ cart{\isacharunderscore}{\kern0pt}prod{\isacharunderscore}{\kern0pt}eq{\isadigit{2}}\ type{\isacharunderscore}{\kern0pt}assms\ \isacommand{by}\isamarkupfalse%
{\isacharparenleft}{\kern0pt}typecheck{\isacharunderscore}{\kern0pt}cfuncs{\isacharcomma}{\kern0pt}\ force{\isacharparenright}{\kern0pt}\isanewline
\ \ \isacommand{then}\isamarkupfalse%
\ \isacommand{show}\isamarkupfalse%
\ {\isachardoublequoteopen}{\isasymexists}x{\isachardot}{\kern0pt}\ x\ {\isasymin}\isactrlsub c\ domain\ g\ {\isasymand}\ g\ {\isasymcirc}\isactrlsub c\ x\ {\isacharequal}{\kern0pt}\ d{\isachardoublequoteclose}\isanewline
\ \ \ \ \isacommand{by}\isamarkupfalse%
\ blast\isanewline
\isacommand{qed}\isamarkupfalse%
%
\endisatagproof
{\isafoldproof}%
%
\isadelimproof
%
\endisadelimproof
%
\isadelimdocument
%
\endisadelimdocument
%
\isatagdocument
%
\isamarkupsubsection{Interactions of Cartesian Products with Terminal Objects%
}
\isamarkuptrue%
%
\endisatagdocument
{\isafolddocument}%
%
\isadelimdocument
%
\endisadelimdocument
\isacommand{lemma}\isamarkupfalse%
\ diag{\isacharunderscore}{\kern0pt}on{\isacharunderscore}{\kern0pt}elements{\isacharcolon}{\kern0pt}\isanewline
\ \ \isakeyword{assumes}\ {\isachardoublequoteopen}x\ {\isasymin}\isactrlsub c\ X{\isachardoublequoteclose}\isanewline
\ \ \isakeyword{shows}\ {\isachardoublequoteopen}diagonal\ X\ {\isasymcirc}\isactrlsub c\ x\ {\isacharequal}{\kern0pt}\ {\isasymlangle}x{\isacharcomma}{\kern0pt}x{\isasymrangle}{\isachardoublequoteclose}\isanewline
%
\isadelimproof
\ \ %
\endisadelimproof
%
\isatagproof
\isacommand{using}\isamarkupfalse%
\ assms\ cfunc{\isacharunderscore}{\kern0pt}prod{\isacharunderscore}{\kern0pt}comp\ cfunc{\isacharunderscore}{\kern0pt}type{\isacharunderscore}{\kern0pt}def\ diagonal{\isacharunderscore}{\kern0pt}def\ id{\isacharunderscore}{\kern0pt}left{\isacharunderscore}{\kern0pt}unit\ id{\isacharunderscore}{\kern0pt}type\ \isacommand{by}\isamarkupfalse%
\ auto%
\endisatagproof
{\isafoldproof}%
%
\isadelimproof
\isanewline
%
\endisadelimproof
\isanewline
\isacommand{lemma}\isamarkupfalse%
\ one{\isacharunderscore}{\kern0pt}cross{\isacharunderscore}{\kern0pt}one{\isacharunderscore}{\kern0pt}unique{\isacharunderscore}{\kern0pt}element{\isacharcolon}{\kern0pt}\isanewline
\ \ {\isachardoublequoteopen}{\isasymexists}{\isacharbang}{\kern0pt}\ x{\isachardot}{\kern0pt}\ x\ {\isasymin}\isactrlsub c\ {\isasymone}\ {\isasymtimes}\isactrlsub c\ {\isasymone}{\isachardoublequoteclose}\isanewline
%
\isadelimproof
%
\endisadelimproof
%
\isatagproof
\isacommand{proof}\isamarkupfalse%
\ {\isacharparenleft}{\kern0pt}rule{\isacharunderscore}{\kern0pt}tac\ a{\isacharequal}{\kern0pt}{\isachardoublequoteopen}diagonal\ {\isasymone}{\isachardoublequoteclose}\ \isakeyword{in}\ ex{\isadigit{1}}I{\isacharparenright}{\kern0pt}\isanewline
\ \ \isacommand{show}\isamarkupfalse%
\ {\isachardoublequoteopen}diagonal\ {\isasymone}\ {\isasymin}\isactrlsub c\ {\isasymone}\ {\isasymtimes}\isactrlsub c\ {\isasymone}{\isachardoublequoteclose}\isanewline
\ \ \ \ \isacommand{by}\isamarkupfalse%
\ {\isacharparenleft}{\kern0pt}simp\ add{\isacharcolon}{\kern0pt}\ cfunc{\isacharunderscore}{\kern0pt}prod{\isacharunderscore}{\kern0pt}type\ diagonal{\isacharunderscore}{\kern0pt}def\ id{\isacharunderscore}{\kern0pt}type{\isacharparenright}{\kern0pt}\isanewline
\isacommand{next}\isamarkupfalse%
\isanewline
\ \ \isacommand{fix}\isamarkupfalse%
\ x\isanewline
\ \ \isacommand{assume}\isamarkupfalse%
\ x{\isacharunderscore}{\kern0pt}type{\isacharcolon}{\kern0pt}\ {\isachardoublequoteopen}x\ {\isasymin}\isactrlsub c\ {\isasymone}\ {\isasymtimes}\isactrlsub c\ {\isasymone}{\isachardoublequoteclose}\isanewline
\ \ \isanewline
\ \ \isacommand{have}\isamarkupfalse%
\ left{\isacharunderscore}{\kern0pt}eq{\isacharcolon}{\kern0pt}\ {\isachardoublequoteopen}left{\isacharunderscore}{\kern0pt}cart{\isacharunderscore}{\kern0pt}proj\ {\isasymone}\ {\isasymone}\ {\isasymcirc}\isactrlsub c\ x\ {\isacharequal}{\kern0pt}\ id\ {\isasymone}{\isachardoublequoteclose}\isanewline
\ \ \ \ \isacommand{using}\isamarkupfalse%
\ x{\isacharunderscore}{\kern0pt}type\ one{\isacharunderscore}{\kern0pt}unique{\isacharunderscore}{\kern0pt}element\ \isacommand{by}\isamarkupfalse%
\ {\isacharparenleft}{\kern0pt}typecheck{\isacharunderscore}{\kern0pt}cfuncs{\isacharcomma}{\kern0pt}\ blast{\isacharparenright}{\kern0pt}\isanewline
\ \ \isacommand{have}\isamarkupfalse%
\ right{\isacharunderscore}{\kern0pt}eq{\isacharcolon}{\kern0pt}\ {\isachardoublequoteopen}right{\isacharunderscore}{\kern0pt}cart{\isacharunderscore}{\kern0pt}proj\ {\isasymone}\ {\isasymone}\ {\isasymcirc}\isactrlsub c\ x\ {\isacharequal}{\kern0pt}\ id\ {\isasymone}{\isachardoublequoteclose}\isanewline
\ \ \ \ \isacommand{using}\isamarkupfalse%
\ x{\isacharunderscore}{\kern0pt}type\ one{\isacharunderscore}{\kern0pt}unique{\isacharunderscore}{\kern0pt}element\ \isacommand{by}\isamarkupfalse%
\ {\isacharparenleft}{\kern0pt}typecheck{\isacharunderscore}{\kern0pt}cfuncs{\isacharcomma}{\kern0pt}\ blast{\isacharparenright}{\kern0pt}\isanewline
\isanewline
\ \ \isacommand{then}\isamarkupfalse%
\ \isacommand{show}\isamarkupfalse%
\ {\isachardoublequoteopen}x\ {\isacharequal}{\kern0pt}\ diagonal\ {\isasymone}{\isachardoublequoteclose}\isanewline
\ \ \ \ \isacommand{unfolding}\isamarkupfalse%
\ diagonal{\isacharunderscore}{\kern0pt}def\ \isacommand{using}\isamarkupfalse%
\ cfunc{\isacharunderscore}{\kern0pt}prod{\isacharunderscore}{\kern0pt}unique\ id{\isacharunderscore}{\kern0pt}type\ left{\isacharunderscore}{\kern0pt}eq\ x{\isacharunderscore}{\kern0pt}type\ \isacommand{by}\isamarkupfalse%
\ blast\isanewline
\isacommand{qed}\isamarkupfalse%
%
\endisatagproof
{\isafoldproof}%
%
\isadelimproof
%
\endisadelimproof
%
\begin{isamarkuptext}%
The lemma below corresponds to Proposition 2.1.20 in Halvorson.%
\end{isamarkuptext}\isamarkuptrue%
\isacommand{lemma}\isamarkupfalse%
\ X{\isacharunderscore}{\kern0pt}is{\isacharunderscore}{\kern0pt}cart{\isacharunderscore}{\kern0pt}prod{\isadigit{1}}{\isacharcolon}{\kern0pt}\isanewline
\ \ {\isachardoublequoteopen}is{\isacharunderscore}{\kern0pt}cart{\isacharunderscore}{\kern0pt}prod\ X\ {\isacharparenleft}{\kern0pt}id\ X{\isacharparenright}{\kern0pt}\ {\isacharparenleft}{\kern0pt}{\isasymbeta}\isactrlbsub X\isactrlesub {\isacharparenright}{\kern0pt}\ X\ {\isasymone}{\isachardoublequoteclose}\isanewline
%
\isadelimproof
\ \ %
\endisadelimproof
%
\isatagproof
\isacommand{unfolding}\isamarkupfalse%
\ is{\isacharunderscore}{\kern0pt}cart{\isacharunderscore}{\kern0pt}prod{\isacharunderscore}{\kern0pt}def\isanewline
\isacommand{proof}\isamarkupfalse%
\ safe\isanewline
\ \ \isacommand{show}\isamarkupfalse%
\ {\isachardoublequoteopen}id\isactrlsub c\ X\ {\isacharcolon}{\kern0pt}\ X\ {\isasymrightarrow}\ X{\isachardoublequoteclose}\isanewline
\ \ \ \ \isacommand{by}\isamarkupfalse%
\ typecheck{\isacharunderscore}{\kern0pt}cfuncs\isanewline
\isacommand{next}\isamarkupfalse%
\isanewline
\ \ \isacommand{show}\isamarkupfalse%
\ {\isachardoublequoteopen}{\isasymbeta}\isactrlbsub X\isactrlesub \ {\isacharcolon}{\kern0pt}\ X\ {\isasymrightarrow}\ {\isasymone}{\isachardoublequoteclose}\isanewline
\ \ \ \ \isacommand{by}\isamarkupfalse%
\ typecheck{\isacharunderscore}{\kern0pt}cfuncs\isanewline
\isacommand{next}\isamarkupfalse%
\isanewline
\ \ \isacommand{fix}\isamarkupfalse%
\ f\ g\ Y\isanewline
\ \ \isacommand{assume}\isamarkupfalse%
\ f{\isacharunderscore}{\kern0pt}type{\isacharcolon}{\kern0pt}\ {\isachardoublequoteopen}f\ {\isacharcolon}{\kern0pt}\ Y\ {\isasymrightarrow}\ X{\isachardoublequoteclose}\ \isakeyword{and}\ g{\isacharunderscore}{\kern0pt}type{\isacharcolon}{\kern0pt}\ {\isachardoublequoteopen}g\ {\isacharcolon}{\kern0pt}\ Y\ {\isasymrightarrow}\ {\isasymone}{\isachardoublequoteclose}\isanewline
\ \ \isacommand{then}\isamarkupfalse%
\ \isacommand{show}\isamarkupfalse%
\ {\isachardoublequoteopen}{\isasymexists}h{\isachardot}{\kern0pt}\ h\ {\isacharcolon}{\kern0pt}\ Y\ {\isasymrightarrow}\ X\ {\isasymand}\isanewline
\ \ \ \ \ \ \ \ \ \ \ id\isactrlsub c\ X\ {\isasymcirc}\isactrlsub c\ h\ {\isacharequal}{\kern0pt}\ f\ {\isasymand}\ {\isasymbeta}\isactrlbsub X\isactrlesub \ {\isasymcirc}\isactrlsub c\ h\ {\isacharequal}{\kern0pt}\ g\ {\isasymand}\ {\isacharparenleft}{\kern0pt}{\isasymforall}h{\isadigit{2}}{\isachardot}{\kern0pt}\ h{\isadigit{2}}\ {\isacharcolon}{\kern0pt}\ Y\ {\isasymrightarrow}\ X\ {\isasymand}\ id\isactrlsub c\ X\ {\isasymcirc}\isactrlsub c\ h{\isadigit{2}}\ {\isacharequal}{\kern0pt}\ f\ {\isasymand}\ {\isasymbeta}\isactrlbsub X\isactrlesub \ {\isasymcirc}\isactrlsub c\ h{\isadigit{2}}\ {\isacharequal}{\kern0pt}\ g\ {\isasymlongrightarrow}\ h{\isadigit{2}}\ {\isacharequal}{\kern0pt}\ h{\isacharparenright}{\kern0pt}{\isachardoublequoteclose}\isanewline
\ \ \isacommand{proof}\isamarkupfalse%
\ {\isacharparenleft}{\kern0pt}rule{\isacharunderscore}{\kern0pt}tac\ x{\isacharequal}{\kern0pt}f\ \isakeyword{in}\ exI{\isacharcomma}{\kern0pt}\ safe{\isacharparenright}{\kern0pt}\isanewline
\ \ \ \ \isacommand{show}\isamarkupfalse%
\ {\isachardoublequoteopen}id\ X\ {\isasymcirc}\isactrlsub c\ f\ {\isacharequal}{\kern0pt}\ f{\isachardoublequoteclose}\isanewline
\ \ \ \ \ \ \isacommand{using}\isamarkupfalse%
\ cfunc{\isacharunderscore}{\kern0pt}type{\isacharunderscore}{\kern0pt}def\ f{\isacharunderscore}{\kern0pt}type\ id{\isacharunderscore}{\kern0pt}left{\isacharunderscore}{\kern0pt}unit\ \isacommand{by}\isamarkupfalse%
\ auto\isanewline
\ \ \ \ \isacommand{show}\isamarkupfalse%
\ {\isachardoublequoteopen}{\isasymbeta}\isactrlbsub X\isactrlesub \ {\isasymcirc}\isactrlsub c\ f\ {\isacharequal}{\kern0pt}\ g{\isachardoublequoteclose}\isanewline
\ \ \ \ \ \ \isacommand{by}\isamarkupfalse%
\ {\isacharparenleft}{\kern0pt}metis\ comp{\isacharunderscore}{\kern0pt}type\ f{\isacharunderscore}{\kern0pt}type\ g{\isacharunderscore}{\kern0pt}type\ terminal{\isacharunderscore}{\kern0pt}func{\isacharunderscore}{\kern0pt}type\ terminal{\isacharunderscore}{\kern0pt}func{\isacharunderscore}{\kern0pt}unique{\isacharparenright}{\kern0pt}\isanewline
\ \ \ \ \isacommand{show}\isamarkupfalse%
\ {\isachardoublequoteopen}{\isasymAnd}h{\isadigit{2}}{\isachardot}{\kern0pt}\ h{\isadigit{2}}\ {\isacharcolon}{\kern0pt}\ Y\ {\isasymrightarrow}\ X\ {\isasymLongrightarrow}\ h{\isadigit{2}}\ {\isacharequal}{\kern0pt}\ id\isactrlsub c\ X\ {\isasymcirc}\isactrlsub c\ h{\isadigit{2}}{\isachardoublequoteclose}\isanewline
\ \ \ \ \ \ \isacommand{using}\isamarkupfalse%
\ cfunc{\isacharunderscore}{\kern0pt}type{\isacharunderscore}{\kern0pt}def\ id{\isacharunderscore}{\kern0pt}left{\isacharunderscore}{\kern0pt}unit\ \isacommand{by}\isamarkupfalse%
\ auto\isanewline
\ \ \isacommand{qed}\isamarkupfalse%
\isanewline
\isacommand{qed}\isamarkupfalse%
%
\endisatagproof
{\isafoldproof}%
%
\isadelimproof
\isanewline
%
\endisadelimproof
\isanewline
\isacommand{lemma}\isamarkupfalse%
\ X{\isacharunderscore}{\kern0pt}is{\isacharunderscore}{\kern0pt}cart{\isacharunderscore}{\kern0pt}prod{\isadigit{2}}{\isacharcolon}{\kern0pt}\isanewline
\ \ {\isachardoublequoteopen}is{\isacharunderscore}{\kern0pt}cart{\isacharunderscore}{\kern0pt}prod\ X\ {\isacharparenleft}{\kern0pt}{\isasymbeta}\isactrlbsub X\isactrlesub {\isacharparenright}{\kern0pt}\ {\isacharparenleft}{\kern0pt}id\ X{\isacharparenright}{\kern0pt}\ {\isasymone}\ X{\isachardoublequoteclose}\isanewline
%
\isadelimproof
\ \ %
\endisadelimproof
%
\isatagproof
\isacommand{unfolding}\isamarkupfalse%
\ is{\isacharunderscore}{\kern0pt}cart{\isacharunderscore}{\kern0pt}prod{\isacharunderscore}{\kern0pt}def\isanewline
\isacommand{proof}\isamarkupfalse%
\ safe\isanewline
\ \ \isacommand{show}\isamarkupfalse%
\ {\isachardoublequoteopen}id\isactrlsub c\ X\ {\isacharcolon}{\kern0pt}\ X\ {\isasymrightarrow}\ X{\isachardoublequoteclose}\isanewline
\ \ \ \ \isacommand{by}\isamarkupfalse%
\ typecheck{\isacharunderscore}{\kern0pt}cfuncs\isanewline
\isacommand{next}\isamarkupfalse%
\isanewline
\ \ \isacommand{show}\isamarkupfalse%
\ {\isachardoublequoteopen}{\isasymbeta}\isactrlbsub X\isactrlesub \ {\isacharcolon}{\kern0pt}\ X\ {\isasymrightarrow}\ {\isasymone}{\isachardoublequoteclose}\isanewline
\ \ \ \ \isacommand{by}\isamarkupfalse%
\ typecheck{\isacharunderscore}{\kern0pt}cfuncs\isanewline
\isacommand{next}\isamarkupfalse%
\isanewline
\ \ \isacommand{fix}\isamarkupfalse%
\ f\ g\ Z\isanewline
\ \ \isacommand{assume}\isamarkupfalse%
\ f{\isacharunderscore}{\kern0pt}type{\isacharcolon}{\kern0pt}\ {\isachardoublequoteopen}f\ {\isacharcolon}{\kern0pt}\ Z\ {\isasymrightarrow}\ {\isasymone}{\isachardoublequoteclose}\ \isakeyword{and}\ g{\isacharunderscore}{\kern0pt}type{\isacharcolon}{\kern0pt}\ {\isachardoublequoteopen}g\ {\isacharcolon}{\kern0pt}\ Z\ {\isasymrightarrow}\ X{\isachardoublequoteclose}\isanewline
\ \ \isacommand{then}\isamarkupfalse%
\ \isacommand{show}\isamarkupfalse%
\ {\isachardoublequoteopen}{\isasymexists}h{\isachardot}{\kern0pt}\ h\ {\isacharcolon}{\kern0pt}\ Z\ {\isasymrightarrow}\ X\ {\isasymand}\isanewline
\ \ \ \ \ \ \ \ \ \ \ {\isasymbeta}\isactrlbsub X\isactrlesub \ {\isasymcirc}\isactrlsub c\ h\ {\isacharequal}{\kern0pt}\ f\ {\isasymand}\ id\isactrlsub c\ X\ {\isasymcirc}\isactrlsub c\ h\ {\isacharequal}{\kern0pt}\ g\ {\isasymand}\ {\isacharparenleft}{\kern0pt}{\isasymforall}h{\isadigit{2}}{\isachardot}{\kern0pt}\ h{\isadigit{2}}\ {\isacharcolon}{\kern0pt}\ Z\ {\isasymrightarrow}\ X\ {\isasymand}\ {\isasymbeta}\isactrlbsub X\isactrlesub \ {\isasymcirc}\isactrlsub c\ h{\isadigit{2}}\ {\isacharequal}{\kern0pt}\ f\ {\isasymand}\ id\isactrlsub c\ X\ {\isasymcirc}\isactrlsub c\ h{\isadigit{2}}\ {\isacharequal}{\kern0pt}\ g\ {\isasymlongrightarrow}\ h{\isadigit{2}}\ {\isacharequal}{\kern0pt}\ h{\isacharparenright}{\kern0pt}{\isachardoublequoteclose}\isanewline
\ \ \isacommand{proof}\isamarkupfalse%
\ {\isacharparenleft}{\kern0pt}rule{\isacharunderscore}{\kern0pt}tac\ x{\isacharequal}{\kern0pt}g\ \isakeyword{in}\ exI{\isacharcomma}{\kern0pt}\ safe{\isacharparenright}{\kern0pt}\isanewline
\ \ \ \ \isacommand{show}\isamarkupfalse%
\ {\isachardoublequoteopen}id\isactrlsub c\ X\ {\isasymcirc}\isactrlsub c\ g\ {\isacharequal}{\kern0pt}\ g{\isachardoublequoteclose}\isanewline
\ \ \ \ \ \ \isacommand{using}\isamarkupfalse%
\ cfunc{\isacharunderscore}{\kern0pt}type{\isacharunderscore}{\kern0pt}def\ g{\isacharunderscore}{\kern0pt}type\ id{\isacharunderscore}{\kern0pt}left{\isacharunderscore}{\kern0pt}unit\ \isacommand{by}\isamarkupfalse%
\ auto\isanewline
\ \ \ \ \isacommand{show}\isamarkupfalse%
\ {\isachardoublequoteopen}{\isasymbeta}\isactrlbsub X\isactrlesub \ {\isasymcirc}\isactrlsub c\ g\ {\isacharequal}{\kern0pt}\ f{\isachardoublequoteclose}\isanewline
\ \ \ \ \ \ \isacommand{by}\isamarkupfalse%
\ {\isacharparenleft}{\kern0pt}metis\ comp{\isacharunderscore}{\kern0pt}type\ f{\isacharunderscore}{\kern0pt}type\ g{\isacharunderscore}{\kern0pt}type\ terminal{\isacharunderscore}{\kern0pt}func{\isacharunderscore}{\kern0pt}type\ terminal{\isacharunderscore}{\kern0pt}func{\isacharunderscore}{\kern0pt}unique{\isacharparenright}{\kern0pt}\isanewline
\ \ \ \ \isacommand{show}\isamarkupfalse%
\ {\isachardoublequoteopen}{\isasymAnd}h{\isadigit{2}}{\isachardot}{\kern0pt}\ h{\isadigit{2}}\ {\isacharcolon}{\kern0pt}\ Z\ {\isasymrightarrow}\ X\ {\isasymLongrightarrow}\ h{\isadigit{2}}\ {\isacharequal}{\kern0pt}\ id\isactrlsub c\ X\ {\isasymcirc}\isactrlsub c\ h{\isadigit{2}}{\isachardoublequoteclose}\isanewline
\ \ \ \ \ \ \isacommand{using}\isamarkupfalse%
\ cfunc{\isacharunderscore}{\kern0pt}type{\isacharunderscore}{\kern0pt}def\ id{\isacharunderscore}{\kern0pt}left{\isacharunderscore}{\kern0pt}unit\ \isacommand{by}\isamarkupfalse%
\ auto\isanewline
\ \ \isacommand{qed}\isamarkupfalse%
\isanewline
\isacommand{qed}\isamarkupfalse%
%
\endisatagproof
{\isafoldproof}%
%
\isadelimproof
\isanewline
%
\endisadelimproof
\isanewline
\isacommand{lemma}\isamarkupfalse%
\ A{\isacharunderscore}{\kern0pt}x{\isacharunderscore}{\kern0pt}one{\isacharunderscore}{\kern0pt}iso{\isacharunderscore}{\kern0pt}A{\isacharcolon}{\kern0pt}\isanewline
\ \ {\isachardoublequoteopen}X\ {\isasymtimes}\isactrlsub c\ {\isasymone}\ {\isasymcong}\ X{\isachardoublequoteclose}\isanewline
%
\isadelimproof
\ \ %
\endisadelimproof
%
\isatagproof
\isacommand{by}\isamarkupfalse%
\ {\isacharparenleft}{\kern0pt}metis\ X{\isacharunderscore}{\kern0pt}is{\isacharunderscore}{\kern0pt}cart{\isacharunderscore}{\kern0pt}prod{\isadigit{1}}\ canonical{\isacharunderscore}{\kern0pt}cart{\isacharunderscore}{\kern0pt}prod{\isacharunderscore}{\kern0pt}is{\isacharunderscore}{\kern0pt}cart{\isacharunderscore}{\kern0pt}prod\ cart{\isacharunderscore}{\kern0pt}prods{\isacharunderscore}{\kern0pt}isomorphic\ fst{\isacharunderscore}{\kern0pt}conv\ is{\isacharunderscore}{\kern0pt}isomorphic{\isacharunderscore}{\kern0pt}def\ snd{\isacharunderscore}{\kern0pt}conv{\isacharparenright}{\kern0pt}%
\endisatagproof
{\isafoldproof}%
%
\isadelimproof
\isanewline
%
\endisadelimproof
\isanewline
\isacommand{lemma}\isamarkupfalse%
\ one{\isacharunderscore}{\kern0pt}x{\isacharunderscore}{\kern0pt}A{\isacharunderscore}{\kern0pt}iso{\isacharunderscore}{\kern0pt}A{\isacharcolon}{\kern0pt}\isanewline
\ \ {\isachardoublequoteopen}{\isasymone}\ {\isasymtimes}\isactrlsub c\ X\ {\isasymcong}\ X{\isachardoublequoteclose}\isanewline
%
\isadelimproof
\ \ %
\endisadelimproof
%
\isatagproof
\isacommand{by}\isamarkupfalse%
\ {\isacharparenleft}{\kern0pt}meson\ A{\isacharunderscore}{\kern0pt}x{\isacharunderscore}{\kern0pt}one{\isacharunderscore}{\kern0pt}iso{\isacharunderscore}{\kern0pt}A\ isomorphic{\isacharunderscore}{\kern0pt}is{\isacharunderscore}{\kern0pt}transitive\ product{\isacharunderscore}{\kern0pt}commutes{\isacharparenright}{\kern0pt}%
\endisatagproof
{\isafoldproof}%
%
\isadelimproof
%
\endisadelimproof
%
\begin{isamarkuptext}%
The following four lemmas provide some concrete examples of the above isomorphisms%
\end{isamarkuptext}\isamarkuptrue%
\isacommand{lemma}\isamarkupfalse%
\ left{\isacharunderscore}{\kern0pt}cart{\isacharunderscore}{\kern0pt}proj{\isacharunderscore}{\kern0pt}one{\isacharunderscore}{\kern0pt}left{\isacharunderscore}{\kern0pt}inverse{\isacharcolon}{\kern0pt}\isanewline
\ \ {\isachardoublequoteopen}{\isasymlangle}id\ X{\isacharcomma}{\kern0pt}{\isasymbeta}\isactrlbsub X\isactrlesub {\isasymrangle}\ {\isasymcirc}\isactrlsub c\ left{\isacharunderscore}{\kern0pt}cart{\isacharunderscore}{\kern0pt}proj\ X\ {\isasymone}\ {\isacharequal}{\kern0pt}\ id\ {\isacharparenleft}{\kern0pt}X\ {\isasymtimes}\isactrlsub c\ {\isasymone}{\isacharparenright}{\kern0pt}{\isachardoublequoteclose}\isanewline
%
\isadelimproof
\ \ %
\endisadelimproof
%
\isatagproof
\isacommand{by}\isamarkupfalse%
\ {\isacharparenleft}{\kern0pt}typecheck{\isacharunderscore}{\kern0pt}cfuncs{\isacharcomma}{\kern0pt}\ smt\ {\isacharparenleft}{\kern0pt}z{\isadigit{3}}{\isacharparenright}{\kern0pt}\ cfunc{\isacharunderscore}{\kern0pt}prod{\isacharunderscore}{\kern0pt}comp\ cfunc{\isacharunderscore}{\kern0pt}prod{\isacharunderscore}{\kern0pt}unique\ id{\isacharunderscore}{\kern0pt}left{\isacharunderscore}{\kern0pt}unit{\isadigit{2}}\ id{\isacharunderscore}{\kern0pt}right{\isacharunderscore}{\kern0pt}unit{\isadigit{2}}\ right{\isacharunderscore}{\kern0pt}cart{\isacharunderscore}{\kern0pt}proj{\isacharunderscore}{\kern0pt}type\ terminal{\isacharunderscore}{\kern0pt}func{\isacharunderscore}{\kern0pt}comp\ terminal{\isacharunderscore}{\kern0pt}func{\isacharunderscore}{\kern0pt}unique{\isacharparenright}{\kern0pt}%
\endisatagproof
{\isafoldproof}%
%
\isadelimproof
\isanewline
%
\endisadelimproof
\isanewline
\isacommand{lemma}\isamarkupfalse%
\ left{\isacharunderscore}{\kern0pt}cart{\isacharunderscore}{\kern0pt}proj{\isacharunderscore}{\kern0pt}one{\isacharunderscore}{\kern0pt}right{\isacharunderscore}{\kern0pt}inverse{\isacharcolon}{\kern0pt}\isanewline
\ \ {\isachardoublequoteopen}left{\isacharunderscore}{\kern0pt}cart{\isacharunderscore}{\kern0pt}proj\ X\ {\isasymone}\ {\isasymcirc}\isactrlsub c\ {\isasymlangle}id\ X{\isacharcomma}{\kern0pt}{\isasymbeta}\isactrlbsub X\isactrlesub {\isasymrangle}\ {\isacharequal}{\kern0pt}\ id\ X{\isachardoublequoteclose}\isanewline
%
\isadelimproof
\ \ %
\endisadelimproof
%
\isatagproof
\isacommand{using}\isamarkupfalse%
\ left{\isacharunderscore}{\kern0pt}cart{\isacharunderscore}{\kern0pt}proj{\isacharunderscore}{\kern0pt}cfunc{\isacharunderscore}{\kern0pt}prod\ \isacommand{by}\isamarkupfalse%
\ {\isacharparenleft}{\kern0pt}typecheck{\isacharunderscore}{\kern0pt}cfuncs{\isacharcomma}{\kern0pt}\ blast{\isacharparenright}{\kern0pt}%
\endisatagproof
{\isafoldproof}%
%
\isadelimproof
\isanewline
%
\endisadelimproof
\isanewline
\isacommand{lemma}\isamarkupfalse%
\ right{\isacharunderscore}{\kern0pt}cart{\isacharunderscore}{\kern0pt}proj{\isacharunderscore}{\kern0pt}one{\isacharunderscore}{\kern0pt}left{\isacharunderscore}{\kern0pt}inverse{\isacharcolon}{\kern0pt}\isanewline
\ \ {\isachardoublequoteopen}{\isasymlangle}{\isasymbeta}\isactrlbsub X\isactrlesub {\isacharcomma}{\kern0pt}id\ X{\isasymrangle}\ {\isasymcirc}\isactrlsub c\ right{\isacharunderscore}{\kern0pt}cart{\isacharunderscore}{\kern0pt}proj\ {\isasymone}\ X\ {\isacharequal}{\kern0pt}\ id\ {\isacharparenleft}{\kern0pt}{\isasymone}\ {\isasymtimes}\isactrlsub c\ X{\isacharparenright}{\kern0pt}{\isachardoublequoteclose}\isanewline
%
\isadelimproof
\ \ %
\endisadelimproof
%
\isatagproof
\isacommand{by}\isamarkupfalse%
\ {\isacharparenleft}{\kern0pt}typecheck{\isacharunderscore}{\kern0pt}cfuncs{\isacharcomma}{\kern0pt}\ smt\ {\isacharparenleft}{\kern0pt}z{\isadigit{3}}{\isacharparenright}{\kern0pt}\ cart{\isacharunderscore}{\kern0pt}prod{\isacharunderscore}{\kern0pt}decomp\ cfunc{\isacharunderscore}{\kern0pt}prod{\isacharunderscore}{\kern0pt}comp\ id{\isacharunderscore}{\kern0pt}left{\isacharunderscore}{\kern0pt}unit{\isadigit{2}}\ id{\isacharunderscore}{\kern0pt}right{\isacharunderscore}{\kern0pt}unit{\isadigit{2}}\ right{\isacharunderscore}{\kern0pt}cart{\isacharunderscore}{\kern0pt}proj{\isacharunderscore}{\kern0pt}cfunc{\isacharunderscore}{\kern0pt}prod\ terminal{\isacharunderscore}{\kern0pt}func{\isacharunderscore}{\kern0pt}comp\ terminal{\isacharunderscore}{\kern0pt}func{\isacharunderscore}{\kern0pt}unique{\isacharparenright}{\kern0pt}%
\endisatagproof
{\isafoldproof}%
%
\isadelimproof
\isanewline
%
\endisadelimproof
\isanewline
\isacommand{lemma}\isamarkupfalse%
\ right{\isacharunderscore}{\kern0pt}cart{\isacharunderscore}{\kern0pt}proj{\isacharunderscore}{\kern0pt}one{\isacharunderscore}{\kern0pt}right{\isacharunderscore}{\kern0pt}inverse{\isacharcolon}{\kern0pt}\isanewline
\ \ {\isachardoublequoteopen}right{\isacharunderscore}{\kern0pt}cart{\isacharunderscore}{\kern0pt}proj\ {\isasymone}\ X\ {\isasymcirc}\isactrlsub c\ {\isasymlangle}{\isasymbeta}\isactrlbsub X\isactrlesub {\isacharcomma}{\kern0pt}id\ X{\isasymrangle}\ {\isacharequal}{\kern0pt}\ id\ X{\isachardoublequoteclose}\isanewline
%
\isadelimproof
\ \ %
\endisadelimproof
%
\isatagproof
\isacommand{using}\isamarkupfalse%
\ right{\isacharunderscore}{\kern0pt}cart{\isacharunderscore}{\kern0pt}proj{\isacharunderscore}{\kern0pt}cfunc{\isacharunderscore}{\kern0pt}prod\ \isacommand{by}\isamarkupfalse%
\ {\isacharparenleft}{\kern0pt}typecheck{\isacharunderscore}{\kern0pt}cfuncs{\isacharcomma}{\kern0pt}\ blast{\isacharparenright}{\kern0pt}%
\endisatagproof
{\isafoldproof}%
%
\isadelimproof
\isanewline
%
\endisadelimproof
\isanewline
\isacommand{lemma}\isamarkupfalse%
\ cfunc{\isacharunderscore}{\kern0pt}cross{\isacharunderscore}{\kern0pt}prod{\isacharunderscore}{\kern0pt}right{\isacharunderscore}{\kern0pt}terminal{\isacharunderscore}{\kern0pt}decomp{\isacharcolon}{\kern0pt}\isanewline
\ \ \isakeyword{assumes}\ {\isachardoublequoteopen}f\ {\isacharcolon}{\kern0pt}\ X\ {\isasymrightarrow}\ Y{\isachardoublequoteclose}\ {\isachardoublequoteopen}x\ {\isacharcolon}{\kern0pt}\ {\isasymone}\ {\isasymrightarrow}\ Z{\isachardoublequoteclose}\isanewline
\ \ \isakeyword{shows}\ {\isachardoublequoteopen}f\ {\isasymtimes}\isactrlsub f\ x\ {\isacharequal}{\kern0pt}\ {\isasymlangle}f{\isacharcomma}{\kern0pt}\ x\ {\isasymcirc}\isactrlsub c\ {\isasymbeta}\isactrlbsub X\isactrlesub {\isasymrangle}\ {\isasymcirc}\isactrlsub c\ left{\isacharunderscore}{\kern0pt}cart{\isacharunderscore}{\kern0pt}proj\ X\ {\isasymone}{\isachardoublequoteclose}\isanewline
%
\isadelimproof
\ \ %
\endisadelimproof
%
\isatagproof
\isacommand{using}\isamarkupfalse%
\ assms\ \isacommand{by}\isamarkupfalse%
\ {\isacharparenleft}{\kern0pt}typecheck{\isacharunderscore}{\kern0pt}cfuncs{\isacharcomma}{\kern0pt}\ smt\ {\isacharparenleft}{\kern0pt}z{\isadigit{3}}{\isacharparenright}{\kern0pt}\ cfunc{\isacharunderscore}{\kern0pt}cross{\isacharunderscore}{\kern0pt}prod{\isacharunderscore}{\kern0pt}def\ cfunc{\isacharunderscore}{\kern0pt}prod{\isacharunderscore}{\kern0pt}comp\ cfunc{\isacharunderscore}{\kern0pt}type{\isacharunderscore}{\kern0pt}def\isanewline
\ \ \ \ \ \ comp{\isacharunderscore}{\kern0pt}associative{\isadigit{2}}\ right{\isacharunderscore}{\kern0pt}cart{\isacharunderscore}{\kern0pt}proj{\isacharunderscore}{\kern0pt}type\ terminal{\isacharunderscore}{\kern0pt}func{\isacharunderscore}{\kern0pt}comp\ terminal{\isacharunderscore}{\kern0pt}func{\isacharunderscore}{\kern0pt}unique{\isacharparenright}{\kern0pt}%
\endisatagproof
{\isafoldproof}%
%
\isadelimproof
%
\endisadelimproof
%
\begin{isamarkuptext}%
The lemma below corresponds to Proposition 2.1.21 in Halvorson.%
\end{isamarkuptext}\isamarkuptrue%
\isacommand{lemma}\isamarkupfalse%
\ cart{\isacharunderscore}{\kern0pt}prod{\isacharunderscore}{\kern0pt}elem{\isacharunderscore}{\kern0pt}eq{\isacharcolon}{\kern0pt}\isanewline
\ \ \isakeyword{assumes}\ {\isachardoublequoteopen}a\ {\isasymin}\isactrlsub c\ X\ {\isasymtimes}\isactrlsub c\ Y{\isachardoublequoteclose}\ {\isachardoublequoteopen}b\ {\isasymin}\isactrlsub c\ X\ {\isasymtimes}\isactrlsub c\ Y{\isachardoublequoteclose}\isanewline
\ \ \isakeyword{shows}\ {\isachardoublequoteopen}a\ {\isacharequal}{\kern0pt}\ b\ {\isasymlongleftrightarrow}\ \isanewline
\ \ \ \ {\isacharparenleft}{\kern0pt}left{\isacharunderscore}{\kern0pt}cart{\isacharunderscore}{\kern0pt}proj\ X\ Y\ {\isasymcirc}\isactrlsub c\ a\ {\isacharequal}{\kern0pt}\ left{\isacharunderscore}{\kern0pt}cart{\isacharunderscore}{\kern0pt}proj\ X\ Y\ {\isasymcirc}\isactrlsub c\ b\ \isanewline
\ \ \ \ \ \ {\isasymand}\ right{\isacharunderscore}{\kern0pt}cart{\isacharunderscore}{\kern0pt}proj\ X\ Y\ {\isasymcirc}\isactrlsub c\ a\ {\isacharequal}{\kern0pt}\ right{\isacharunderscore}{\kern0pt}cart{\isacharunderscore}{\kern0pt}proj\ X\ Y\ {\isasymcirc}\isactrlsub c\ b{\isacharparenright}{\kern0pt}{\isachardoublequoteclose}\isanewline
%
\isadelimproof
\ \ %
\endisadelimproof
%
\isatagproof
\isacommand{by}\isamarkupfalse%
\ {\isacharparenleft}{\kern0pt}metis\ {\isacharparenleft}{\kern0pt}full{\isacharunderscore}{\kern0pt}types{\isacharparenright}{\kern0pt}\ assms\ cfunc{\isacharunderscore}{\kern0pt}prod{\isacharunderscore}{\kern0pt}unique\ comp{\isacharunderscore}{\kern0pt}type\ left{\isacharunderscore}{\kern0pt}cart{\isacharunderscore}{\kern0pt}proj{\isacharunderscore}{\kern0pt}type\ right{\isacharunderscore}{\kern0pt}cart{\isacharunderscore}{\kern0pt}proj{\isacharunderscore}{\kern0pt}type{\isacharparenright}{\kern0pt}%
\endisatagproof
{\isafoldproof}%
%
\isadelimproof
%
\endisadelimproof
%
\begin{isamarkuptext}%
The lemma below corresponds to Note 2.1.22 in Halvorson.%
\end{isamarkuptext}\isamarkuptrue%
\isacommand{lemma}\isamarkupfalse%
\ \ element{\isacharunderscore}{\kern0pt}pair{\isacharunderscore}{\kern0pt}eq{\isacharcolon}{\kern0pt}\isanewline
\ \ \isakeyword{assumes}\ {\isachardoublequoteopen}x\ {\isasymin}\isactrlsub c\ X{\isachardoublequoteclose}\ {\isachardoublequoteopen}x{\isacharprime}{\kern0pt}\ {\isasymin}\isactrlsub c\ X{\isachardoublequoteclose}\ {\isachardoublequoteopen}y\ {\isasymin}\isactrlsub c\ Y{\isachardoublequoteclose}\ {\isachardoublequoteopen}y{\isacharprime}{\kern0pt}\ {\isasymin}\isactrlsub c\ Y{\isachardoublequoteclose}\isanewline
\ \ \isakeyword{shows}\ {\isachardoublequoteopen}{\isasymlangle}x{\isacharcomma}{\kern0pt}\ y{\isasymrangle}\ {\isacharequal}{\kern0pt}\ {\isasymlangle}x{\isacharprime}{\kern0pt}{\isacharcomma}{\kern0pt}\ y{\isacharprime}{\kern0pt}{\isasymrangle}\ {\isasymlongleftrightarrow}\ x\ {\isacharequal}{\kern0pt}\ x{\isacharprime}{\kern0pt}\ {\isasymand}\ y\ {\isacharequal}{\kern0pt}\ y{\isacharprime}{\kern0pt}{\isachardoublequoteclose}\isanewline
%
\isadelimproof
\ \ %
\endisadelimproof
%
\isatagproof
\isacommand{by}\isamarkupfalse%
\ {\isacharparenleft}{\kern0pt}metis\ assms\ left{\isacharunderscore}{\kern0pt}cart{\isacharunderscore}{\kern0pt}proj{\isacharunderscore}{\kern0pt}cfunc{\isacharunderscore}{\kern0pt}prod\ right{\isacharunderscore}{\kern0pt}cart{\isacharunderscore}{\kern0pt}proj{\isacharunderscore}{\kern0pt}cfunc{\isacharunderscore}{\kern0pt}prod{\isacharparenright}{\kern0pt}%
\endisatagproof
{\isafoldproof}%
%
\isadelimproof
%
\endisadelimproof
%
\begin{isamarkuptext}%
The lemma below corresponds to Proposition 2.1.23 in Halvorson.%
\end{isamarkuptext}\isamarkuptrue%
\isacommand{lemma}\isamarkupfalse%
\ nonempty{\isacharunderscore}{\kern0pt}right{\isacharunderscore}{\kern0pt}imp{\isacharunderscore}{\kern0pt}left{\isacharunderscore}{\kern0pt}proj{\isacharunderscore}{\kern0pt}epimorphism{\isacharcolon}{\kern0pt}\isanewline
\ \ {\isachardoublequoteopen}nonempty\ Y\ {\isasymLongrightarrow}\ epimorphism\ {\isacharparenleft}{\kern0pt}left{\isacharunderscore}{\kern0pt}cart{\isacharunderscore}{\kern0pt}proj\ X\ Y{\isacharparenright}{\kern0pt}{\isachardoublequoteclose}\isanewline
%
\isadelimproof
%
\endisadelimproof
%
\isatagproof
\isacommand{proof}\isamarkupfalse%
\ {\isacharminus}{\kern0pt}\isanewline
\ \ \isacommand{assume}\isamarkupfalse%
\ {\isachardoublequoteopen}nonempty\ Y{\isachardoublequoteclose}\isanewline
\ \ \isacommand{then}\isamarkupfalse%
\ \isacommand{obtain}\isamarkupfalse%
\ y\ \isakeyword{where}\ y{\isacharunderscore}{\kern0pt}in{\isacharunderscore}{\kern0pt}Y{\isacharcolon}{\kern0pt}\ {\isachardoublequoteopen}y\ {\isacharcolon}{\kern0pt}\ {\isasymone}\ {\isasymrightarrow}\ Y{\isachardoublequoteclose}\isanewline
\ \ \ \ \isacommand{using}\isamarkupfalse%
\ nonempty{\isacharunderscore}{\kern0pt}def\ \isacommand{by}\isamarkupfalse%
\ blast\isanewline
\ \ \isacommand{then}\isamarkupfalse%
\ \isacommand{have}\isamarkupfalse%
\ id{\isacharunderscore}{\kern0pt}eq{\isacharcolon}{\kern0pt}\ {\isachardoublequoteopen}{\isacharparenleft}{\kern0pt}left{\isacharunderscore}{\kern0pt}cart{\isacharunderscore}{\kern0pt}proj\ X\ Y{\isacharparenright}{\kern0pt}\ {\isasymcirc}\isactrlsub c\ {\isasymlangle}id\ X{\isacharcomma}{\kern0pt}\ y\ {\isasymcirc}\isactrlsub c\ {\isasymbeta}\isactrlbsub X\isactrlesub {\isasymrangle}\ {\isacharequal}{\kern0pt}\ id\ X{\isachardoublequoteclose}\isanewline
\ \ \ \ \isacommand{using}\isamarkupfalse%
\ comp{\isacharunderscore}{\kern0pt}type\ id{\isacharunderscore}{\kern0pt}type\ left{\isacharunderscore}{\kern0pt}cart{\isacharunderscore}{\kern0pt}proj{\isacharunderscore}{\kern0pt}cfunc{\isacharunderscore}{\kern0pt}prod\ terminal{\isacharunderscore}{\kern0pt}func{\isacharunderscore}{\kern0pt}type\ \isacommand{by}\isamarkupfalse%
\ blast\isanewline
\ \ \isacommand{then}\isamarkupfalse%
\ \isacommand{show}\isamarkupfalse%
\ {\isachardoublequoteopen}epimorphism\ {\isacharparenleft}{\kern0pt}left{\isacharunderscore}{\kern0pt}cart{\isacharunderscore}{\kern0pt}proj\ X\ Y{\isacharparenright}{\kern0pt}{\isachardoublequoteclose}\isanewline
\ \ \ \ \isacommand{unfolding}\isamarkupfalse%
\ epimorphism{\isacharunderscore}{\kern0pt}def\isanewline
\ \ \isacommand{proof}\isamarkupfalse%
\ clarify\isanewline
\ \ \ \ \isacommand{fix}\isamarkupfalse%
\ g\ h\isanewline
\ \ \ \ \isacommand{assume}\isamarkupfalse%
\ domain{\isacharunderscore}{\kern0pt}g{\isacharcolon}{\kern0pt}\ {\isachardoublequoteopen}domain\ g\ {\isacharequal}{\kern0pt}\ codomain\ {\isacharparenleft}{\kern0pt}left{\isacharunderscore}{\kern0pt}cart{\isacharunderscore}{\kern0pt}proj\ X\ Y{\isacharparenright}{\kern0pt}{\isachardoublequoteclose}\isanewline
\ \ \ \ \isacommand{assume}\isamarkupfalse%
\ domain{\isacharunderscore}{\kern0pt}h{\isacharcolon}{\kern0pt}\ {\isachardoublequoteopen}domain\ h\ {\isacharequal}{\kern0pt}\ codomain\ {\isacharparenleft}{\kern0pt}left{\isacharunderscore}{\kern0pt}cart{\isacharunderscore}{\kern0pt}proj\ X\ Y{\isacharparenright}{\kern0pt}{\isachardoublequoteclose}\isanewline
\ \ \ \ \isacommand{assume}\isamarkupfalse%
\ {\isachardoublequoteopen}g\ {\isasymcirc}\isactrlsub c\ left{\isacharunderscore}{\kern0pt}cart{\isacharunderscore}{\kern0pt}proj\ X\ Y\ {\isacharequal}{\kern0pt}\ h\ {\isasymcirc}\isactrlsub c\ left{\isacharunderscore}{\kern0pt}cart{\isacharunderscore}{\kern0pt}proj\ X\ Y{\isachardoublequoteclose}\isanewline
\ \ \ \ \isacommand{then}\isamarkupfalse%
\ \isacommand{have}\isamarkupfalse%
\ {\isachardoublequoteopen}g\ {\isasymcirc}\isactrlsub c\ left{\isacharunderscore}{\kern0pt}cart{\isacharunderscore}{\kern0pt}proj\ X\ Y\ {\isasymcirc}\isactrlsub c\ {\isasymlangle}id\ X{\isacharcomma}{\kern0pt}\ y\ {\isasymcirc}\isactrlsub c\ {\isasymbeta}\isactrlbsub X\isactrlesub {\isasymrangle}\ {\isacharequal}{\kern0pt}\ h\ {\isasymcirc}\isactrlsub c\ left{\isacharunderscore}{\kern0pt}cart{\isacharunderscore}{\kern0pt}proj\ X\ Y\ {\isasymcirc}\isactrlsub c\ {\isasymlangle}id\ X{\isacharcomma}{\kern0pt}\ y\ {\isasymcirc}\isactrlsub c\ {\isasymbeta}\isactrlbsub X\isactrlesub {\isasymrangle}{\isachardoublequoteclose}\isanewline
\ \ \ \ \ \ \isacommand{using}\isamarkupfalse%
\ y{\isacharunderscore}{\kern0pt}in{\isacharunderscore}{\kern0pt}Y\ \isacommand{by}\isamarkupfalse%
\ {\isacharparenleft}{\kern0pt}typecheck{\isacharunderscore}{\kern0pt}cfuncs{\isacharcomma}{\kern0pt}\ simp\ add{\isacharcolon}{\kern0pt}\ cfunc{\isacharunderscore}{\kern0pt}type{\isacharunderscore}{\kern0pt}def\ comp{\isacharunderscore}{\kern0pt}associative\ domain{\isacharunderscore}{\kern0pt}g\ domain{\isacharunderscore}{\kern0pt}h{\isacharparenright}{\kern0pt}\isanewline
\ \ \ \ \isacommand{then}\isamarkupfalse%
\ \isacommand{show}\isamarkupfalse%
\ {\isachardoublequoteopen}g\ {\isacharequal}{\kern0pt}\ h{\isachardoublequoteclose}\isanewline
\ \ \ \ \ \ \isacommand{by}\isamarkupfalse%
\ {\isacharparenleft}{\kern0pt}metis\ cfunc{\isacharunderscore}{\kern0pt}type{\isacharunderscore}{\kern0pt}def\ domain{\isacharunderscore}{\kern0pt}g\ domain{\isacharunderscore}{\kern0pt}h\ id{\isacharunderscore}{\kern0pt}eq\ id{\isacharunderscore}{\kern0pt}right{\isacharunderscore}{\kern0pt}unit\ left{\isacharunderscore}{\kern0pt}cart{\isacharunderscore}{\kern0pt}proj{\isacharunderscore}{\kern0pt}type{\isacharparenright}{\kern0pt}\isanewline
\ \ \isacommand{qed}\isamarkupfalse%
\isanewline
\isacommand{qed}\isamarkupfalse%
%
\endisatagproof
{\isafoldproof}%
%
\isadelimproof
%
\endisadelimproof
%
\begin{isamarkuptext}%
The lemma below is the dual of Proposition 2.1.23 in Halvorson.%
\end{isamarkuptext}\isamarkuptrue%
\isacommand{lemma}\isamarkupfalse%
\ nonempty{\isacharunderscore}{\kern0pt}left{\isacharunderscore}{\kern0pt}imp{\isacharunderscore}{\kern0pt}right{\isacharunderscore}{\kern0pt}proj{\isacharunderscore}{\kern0pt}epimorphism{\isacharcolon}{\kern0pt}\isanewline
\ \ {\isachardoublequoteopen}nonempty\ X\ {\isasymLongrightarrow}\ epimorphism\ {\isacharparenleft}{\kern0pt}right{\isacharunderscore}{\kern0pt}cart{\isacharunderscore}{\kern0pt}proj\ X\ Y{\isacharparenright}{\kern0pt}{\isachardoublequoteclose}\isanewline
%
\isadelimproof
%
\endisadelimproof
%
\isatagproof
\isacommand{proof}\isamarkupfalse%
\ {\isacharminus}{\kern0pt}\ \isanewline
\ \ \isacommand{assume}\isamarkupfalse%
\ {\isachardoublequoteopen}nonempty\ X{\isachardoublequoteclose}\isanewline
\ \ \isacommand{then}\isamarkupfalse%
\ \isacommand{obtain}\isamarkupfalse%
\ y\ \isakeyword{where}\ y{\isacharunderscore}{\kern0pt}in{\isacharunderscore}{\kern0pt}Y{\isacharcolon}{\kern0pt}\ {\isachardoublequoteopen}y{\isacharcolon}{\kern0pt}\ {\isasymone}\ {\isasymrightarrow}\ X{\isachardoublequoteclose}\isanewline
\ \ \ \ \isacommand{using}\isamarkupfalse%
\ nonempty{\isacharunderscore}{\kern0pt}def\ \isacommand{by}\isamarkupfalse%
\ blast\isanewline
\ \ \isacommand{then}\isamarkupfalse%
\ \isacommand{have}\isamarkupfalse%
\ id{\isacharunderscore}{\kern0pt}eq{\isacharcolon}{\kern0pt}\ {\isachardoublequoteopen}{\isacharparenleft}{\kern0pt}right{\isacharunderscore}{\kern0pt}cart{\isacharunderscore}{\kern0pt}proj\ X\ Y{\isacharparenright}{\kern0pt}\ {\isasymcirc}\isactrlsub c\ {\isasymlangle}y\ {\isasymcirc}\isactrlsub c\ {\isasymbeta}\isactrlbsub Y\isactrlesub {\isacharcomma}{\kern0pt}\ id\ Y{\isasymrangle}\ {\isacharequal}{\kern0pt}\ id\ Y{\isachardoublequoteclose}\isanewline
\ \ \ \ \ \isacommand{using}\isamarkupfalse%
\ comp{\isacharunderscore}{\kern0pt}type\ id{\isacharunderscore}{\kern0pt}type\ right{\isacharunderscore}{\kern0pt}cart{\isacharunderscore}{\kern0pt}proj{\isacharunderscore}{\kern0pt}cfunc{\isacharunderscore}{\kern0pt}prod\ terminal{\isacharunderscore}{\kern0pt}func{\isacharunderscore}{\kern0pt}type\ \isacommand{by}\isamarkupfalse%
\ blast\isanewline
\ \ \isacommand{then}\isamarkupfalse%
\ \isacommand{show}\isamarkupfalse%
\ {\isachardoublequoteopen}epimorphism\ {\isacharparenleft}{\kern0pt}right{\isacharunderscore}{\kern0pt}cart{\isacharunderscore}{\kern0pt}proj\ X\ Y{\isacharparenright}{\kern0pt}{\isachardoublequoteclose}\isanewline
\ \ \ \ \isacommand{unfolding}\isamarkupfalse%
\ epimorphism{\isacharunderscore}{\kern0pt}def\isanewline
\ \ \isacommand{proof}\isamarkupfalse%
\ clarify\isanewline
\ \ \ \ \isacommand{fix}\isamarkupfalse%
\ g\ h\isanewline
\ \ \ \ \isacommand{assume}\isamarkupfalse%
\ domain{\isacharunderscore}{\kern0pt}g{\isacharcolon}{\kern0pt}\ {\isachardoublequoteopen}domain\ g\ {\isacharequal}{\kern0pt}\ codomain\ {\isacharparenleft}{\kern0pt}right{\isacharunderscore}{\kern0pt}cart{\isacharunderscore}{\kern0pt}proj\ X\ Y{\isacharparenright}{\kern0pt}{\isachardoublequoteclose}\isanewline
\ \ \ \ \isacommand{assume}\isamarkupfalse%
\ domain{\isacharunderscore}{\kern0pt}h{\isacharcolon}{\kern0pt}\ {\isachardoublequoteopen}domain\ h\ {\isacharequal}{\kern0pt}\ codomain\ {\isacharparenleft}{\kern0pt}right{\isacharunderscore}{\kern0pt}cart{\isacharunderscore}{\kern0pt}proj\ X\ Y{\isacharparenright}{\kern0pt}{\isachardoublequoteclose}\isanewline
\ \ \ \ \isacommand{assume}\isamarkupfalse%
\ {\isachardoublequoteopen}g\ {\isasymcirc}\isactrlsub c\ right{\isacharunderscore}{\kern0pt}cart{\isacharunderscore}{\kern0pt}proj\ X\ Y\ {\isacharequal}{\kern0pt}\ h\ {\isasymcirc}\isactrlsub c\ right{\isacharunderscore}{\kern0pt}cart{\isacharunderscore}{\kern0pt}proj\ X\ Y{\isachardoublequoteclose}\isanewline
\ \ \ \ \isacommand{then}\isamarkupfalse%
\ \isacommand{have}\isamarkupfalse%
\ {\isachardoublequoteopen}g\ {\isasymcirc}\isactrlsub c\ right{\isacharunderscore}{\kern0pt}cart{\isacharunderscore}{\kern0pt}proj\ X\ Y\ {\isasymcirc}\isactrlsub c\ {\isasymlangle}y\ {\isasymcirc}\isactrlsub c\ {\isasymbeta}\isactrlbsub Y\isactrlesub {\isacharcomma}{\kern0pt}\ id\ Y{\isasymrangle}\ {\isacharequal}{\kern0pt}\ h\ {\isasymcirc}\isactrlsub c\ right{\isacharunderscore}{\kern0pt}cart{\isacharunderscore}{\kern0pt}proj\ X\ Y\ {\isasymcirc}\isactrlsub c\ {\isasymlangle}y\ {\isasymcirc}\isactrlsub c\ {\isasymbeta}\isactrlbsub Y\isactrlesub {\isacharcomma}{\kern0pt}\ id\ Y{\isasymrangle}{\isachardoublequoteclose}\isanewline
\ \ \ \ \ \ \isacommand{using}\isamarkupfalse%
\ y{\isacharunderscore}{\kern0pt}in{\isacharunderscore}{\kern0pt}Y\ \isacommand{by}\isamarkupfalse%
\ {\isacharparenleft}{\kern0pt}typecheck{\isacharunderscore}{\kern0pt}cfuncs{\isacharcomma}{\kern0pt}\ simp\ add{\isacharcolon}{\kern0pt}\ cfunc{\isacharunderscore}{\kern0pt}type{\isacharunderscore}{\kern0pt}def\ comp{\isacharunderscore}{\kern0pt}associative\ domain{\isacharunderscore}{\kern0pt}g\ domain{\isacharunderscore}{\kern0pt}h{\isacharparenright}{\kern0pt}\isanewline
\ \ \ \ \isacommand{then}\isamarkupfalse%
\ \isacommand{show}\isamarkupfalse%
\ {\isachardoublequoteopen}g\ {\isacharequal}{\kern0pt}\ h{\isachardoublequoteclose}\isanewline
\ \ \ \ \ \ \isacommand{by}\isamarkupfalse%
\ {\isacharparenleft}{\kern0pt}metis\ cfunc{\isacharunderscore}{\kern0pt}type{\isacharunderscore}{\kern0pt}def\ domain{\isacharunderscore}{\kern0pt}g\ domain{\isacharunderscore}{\kern0pt}h\ id{\isacharunderscore}{\kern0pt}eq\ id{\isacharunderscore}{\kern0pt}right{\isacharunderscore}{\kern0pt}unit\ right{\isacharunderscore}{\kern0pt}cart{\isacharunderscore}{\kern0pt}proj{\isacharunderscore}{\kern0pt}type{\isacharparenright}{\kern0pt}\isanewline
\ \ \isacommand{qed}\isamarkupfalse%
\isanewline
\isacommand{qed}\isamarkupfalse%
%
\endisatagproof
{\isafoldproof}%
%
\isadelimproof
\isanewline
%
\endisadelimproof
\isanewline
\isacommand{lemma}\isamarkupfalse%
\ cart{\isacharunderscore}{\kern0pt}prod{\isacharunderscore}{\kern0pt}extract{\isacharunderscore}{\kern0pt}left{\isacharcolon}{\kern0pt}\isanewline
\ \ \isakeyword{assumes}\ {\isachardoublequoteopen}f\ {\isacharcolon}{\kern0pt}\ {\isasymone}\ {\isasymrightarrow}\ X{\isachardoublequoteclose}\ {\isachardoublequoteopen}g\ {\isacharcolon}{\kern0pt}\ {\isasymone}\ {\isasymrightarrow}\ Y{\isachardoublequoteclose}\isanewline
\ \ \isakeyword{shows}\ {\isachardoublequoteopen}{\isasymlangle}f{\isacharcomma}{\kern0pt}\ g{\isasymrangle}\ {\isacharequal}{\kern0pt}\ {\isasymlangle}id\ X{\isacharcomma}{\kern0pt}\ g\ {\isasymcirc}\isactrlsub c\ {\isasymbeta}\isactrlbsub X\isactrlesub {\isasymrangle}\ {\isasymcirc}\isactrlsub c\ f{\isachardoublequoteclose}\isanewline
%
\isadelimproof
%
\endisadelimproof
%
\isatagproof
\isacommand{proof}\isamarkupfalse%
\ {\isacharminus}{\kern0pt}\isanewline
\ \ \isacommand{have}\isamarkupfalse%
\ {\isachardoublequoteopen}{\isasymlangle}f{\isacharcomma}{\kern0pt}\ g{\isasymrangle}\ {\isacharequal}{\kern0pt}\ {\isasymlangle}id\ X\ {\isasymcirc}\isactrlsub c\ f{\isacharcomma}{\kern0pt}\ g\ {\isasymcirc}\isactrlsub c\ {\isasymbeta}\isactrlbsub X\isactrlesub \ {\isasymcirc}\isactrlsub c\ f{\isasymrangle}{\isachardoublequoteclose}\isanewline
\ \ \ \ \isacommand{using}\isamarkupfalse%
\ assms\ \isacommand{by}\isamarkupfalse%
\ {\isacharparenleft}{\kern0pt}typecheck{\isacharunderscore}{\kern0pt}cfuncs{\isacharcomma}{\kern0pt}\ metis\ id{\isacharunderscore}{\kern0pt}left{\isacharunderscore}{\kern0pt}unit{\isadigit{2}}\ id{\isacharunderscore}{\kern0pt}right{\isacharunderscore}{\kern0pt}unit{\isadigit{2}}\ id{\isacharunderscore}{\kern0pt}type\ one{\isacharunderscore}{\kern0pt}unique{\isacharunderscore}{\kern0pt}element{\isacharparenright}{\kern0pt}\isanewline
\ \ \isacommand{also}\isamarkupfalse%
\ \isacommand{have}\isamarkupfalse%
\ {\isachardoublequoteopen}{\isachardot}{\kern0pt}{\isachardot}{\kern0pt}{\isachardot}{\kern0pt}\ {\isacharequal}{\kern0pt}\ {\isasymlangle}id\ X{\isacharcomma}{\kern0pt}\ g\ {\isasymcirc}\isactrlsub c\ {\isasymbeta}\isactrlbsub X\isactrlesub {\isasymrangle}\ {\isasymcirc}\isactrlsub c\ f{\isachardoublequoteclose}\isanewline
\ \ \ \ \isacommand{using}\isamarkupfalse%
\ assms\ \isacommand{by}\isamarkupfalse%
\ {\isacharparenleft}{\kern0pt}typecheck{\isacharunderscore}{\kern0pt}cfuncs{\isacharcomma}{\kern0pt}\ simp\ add{\isacharcolon}{\kern0pt}\ cfunc{\isacharunderscore}{\kern0pt}prod{\isacharunderscore}{\kern0pt}comp\ comp{\isacharunderscore}{\kern0pt}associative{\isadigit{2}}{\isacharparenright}{\kern0pt}\isanewline
\ \ \isacommand{then}\isamarkupfalse%
\ \isacommand{show}\isamarkupfalse%
\ {\isacharquery}{\kern0pt}thesis\isanewline
\ \ \ \ \isacommand{using}\isamarkupfalse%
\ calculation\ \isacommand{by}\isamarkupfalse%
\ auto\isanewline
\isacommand{qed}\isamarkupfalse%
%
\endisatagproof
{\isafoldproof}%
%
\isadelimproof
\isanewline
%
\endisadelimproof
\isanewline
\isacommand{lemma}\isamarkupfalse%
\ cart{\isacharunderscore}{\kern0pt}prod{\isacharunderscore}{\kern0pt}extract{\isacharunderscore}{\kern0pt}right{\isacharcolon}{\kern0pt}\isanewline
\ \ \isakeyword{assumes}\ {\isachardoublequoteopen}f\ {\isacharcolon}{\kern0pt}\ {\isasymone}\ {\isasymrightarrow}\ X{\isachardoublequoteclose}\ {\isachardoublequoteopen}g\ {\isacharcolon}{\kern0pt}\ {\isasymone}\ {\isasymrightarrow}\ Y{\isachardoublequoteclose}\isanewline
\ \ \isakeyword{shows}\ {\isachardoublequoteopen}{\isasymlangle}f{\isacharcomma}{\kern0pt}\ g{\isasymrangle}\ {\isacharequal}{\kern0pt}\ {\isasymlangle}f\ {\isasymcirc}\isactrlsub c\ {\isasymbeta}\isactrlbsub Y\isactrlesub {\isacharcomma}{\kern0pt}\ id\ Y{\isasymrangle}\ {\isasymcirc}\isactrlsub c\ g{\isachardoublequoteclose}\isanewline
%
\isadelimproof
%
\endisadelimproof
%
\isatagproof
\isacommand{proof}\isamarkupfalse%
\ {\isacharminus}{\kern0pt}\isanewline
\ \ \isacommand{have}\isamarkupfalse%
\ {\isachardoublequoteopen}{\isasymlangle}f{\isacharcomma}{\kern0pt}\ g{\isasymrangle}\ {\isacharequal}{\kern0pt}\ {\isasymlangle}f\ {\isasymcirc}\isactrlsub c\ {\isasymbeta}\isactrlbsub Y\isactrlesub \ {\isasymcirc}\isactrlsub c\ g{\isacharcomma}{\kern0pt}\ id\ Y\ {\isasymcirc}\isactrlsub c\ g{\isasymrangle}{\isachardoublequoteclose}\isanewline
\ \ \ \ \isacommand{using}\isamarkupfalse%
\ assms\ \isacommand{by}\isamarkupfalse%
\ {\isacharparenleft}{\kern0pt}typecheck{\isacharunderscore}{\kern0pt}cfuncs{\isacharcomma}{\kern0pt}\ metis\ id{\isacharunderscore}{\kern0pt}left{\isacharunderscore}{\kern0pt}unit{\isadigit{2}}\ id{\isacharunderscore}{\kern0pt}right{\isacharunderscore}{\kern0pt}unit{\isadigit{2}}\ id{\isacharunderscore}{\kern0pt}type\ one{\isacharunderscore}{\kern0pt}unique{\isacharunderscore}{\kern0pt}element{\isacharparenright}{\kern0pt}\isanewline
\ \ \isacommand{also}\isamarkupfalse%
\ \isacommand{have}\isamarkupfalse%
\ {\isachardoublequoteopen}{\isachardot}{\kern0pt}{\isachardot}{\kern0pt}{\isachardot}{\kern0pt}\ {\isacharequal}{\kern0pt}\ {\isasymlangle}f\ {\isasymcirc}\isactrlsub c\ {\isasymbeta}\isactrlbsub Y\isactrlesub {\isacharcomma}{\kern0pt}\ id\ Y{\isasymrangle}\ {\isasymcirc}\isactrlsub c\ g{\isachardoublequoteclose}\isanewline
\ \ \ \ \isacommand{using}\isamarkupfalse%
\ assms\ \isacommand{by}\isamarkupfalse%
\ {\isacharparenleft}{\kern0pt}typecheck{\isacharunderscore}{\kern0pt}cfuncs{\isacharcomma}{\kern0pt}\ simp\ add{\isacharcolon}{\kern0pt}\ cfunc{\isacharunderscore}{\kern0pt}prod{\isacharunderscore}{\kern0pt}comp\ comp{\isacharunderscore}{\kern0pt}associative{\isadigit{2}}{\isacharparenright}{\kern0pt}\isanewline
\ \ \isacommand{then}\isamarkupfalse%
\ \isacommand{show}\isamarkupfalse%
\ {\isacharquery}{\kern0pt}thesis\isanewline
\ \ \ \ \isacommand{using}\isamarkupfalse%
\ calculation\ \isacommand{by}\isamarkupfalse%
\ auto\isanewline
\isacommand{qed}\isamarkupfalse%
%
\endisatagproof
{\isafoldproof}%
%
\isadelimproof
%
\endisadelimproof
%
\isadelimdocument
%
\endisadelimdocument
%
\isatagdocument
%
\isamarkupsubsubsection{Cartesian Products as Pullbacks%
}
\isamarkuptrue%
%
\endisatagdocument
{\isafolddocument}%
%
\isadelimdocument
%
\endisadelimdocument
%
\begin{isamarkuptext}%
The definition below corresponds to a definition stated between Definition 2.1.42 and Definition 2.1.43 in Halvorson.%
\end{isamarkuptext}\isamarkuptrue%
\isacommand{definition}\isamarkupfalse%
\ is{\isacharunderscore}{\kern0pt}pullback\ {\isacharcolon}{\kern0pt}{\isacharcolon}{\kern0pt}\ {\isachardoublequoteopen}cset\ {\isasymRightarrow}\ cset\ {\isasymRightarrow}\ cset\ {\isasymRightarrow}\ cset\ {\isasymRightarrow}\ cfunc\ {\isasymRightarrow}\ cfunc\ {\isasymRightarrow}\ cfunc\ {\isasymRightarrow}\ cfunc\ {\isasymRightarrow}\ bool{\isachardoublequoteclose}\ \isakeyword{where}\isanewline
\ \ {\isachardoublequoteopen}is{\isacharunderscore}{\kern0pt}pullback\ A\ B\ C\ D\ ab\ bd\ ac\ cd\ {\isasymlongleftrightarrow}\ \isanewline
\ \ \ \ {\isacharparenleft}{\kern0pt}ab\ {\isacharcolon}{\kern0pt}\ A\ {\isasymrightarrow}\ B\ {\isasymand}\ bd\ {\isacharcolon}{\kern0pt}\ B\ {\isasymrightarrow}\ D\ {\isasymand}\ ac\ {\isacharcolon}{\kern0pt}\ A\ {\isasymrightarrow}\ C\ {\isasymand}\ cd\ {\isacharcolon}{\kern0pt}\ C\ {\isasymrightarrow}\ D\ {\isasymand}\ bd\ {\isasymcirc}\isactrlsub c\ ab\ {\isacharequal}{\kern0pt}\ cd\ {\isasymcirc}\isactrlsub c\ ac\ {\isasymand}\ \isanewline
\ \ \ \ {\isacharparenleft}{\kern0pt}{\isasymforall}\ Z\ k\ h{\isachardot}{\kern0pt}\ {\isacharparenleft}{\kern0pt}k\ {\isacharcolon}{\kern0pt}\ Z\ {\isasymrightarrow}\ B\ {\isasymand}\ h\ {\isacharcolon}{\kern0pt}\ Z\ {\isasymrightarrow}\ C\ {\isasymand}\ bd\ {\isasymcirc}\isactrlsub c\ k\ {\isacharequal}{\kern0pt}\ cd\ {\isasymcirc}\isactrlsub c\ h{\isacharparenright}{\kern0pt}\ \ {\isasymlongrightarrow}\isanewline
\ \ \ \ \ \ {\isacharparenleft}{\kern0pt}{\isasymexists}{\isacharbang}{\kern0pt}\ j{\isachardot}{\kern0pt}\ j\ {\isacharcolon}{\kern0pt}\ Z\ {\isasymrightarrow}\ A\ {\isasymand}\ ab\ {\isasymcirc}\isactrlsub c\ j\ {\isacharequal}{\kern0pt}\ k\ {\isasymand}\ ac\ {\isasymcirc}\isactrlsub c\ j\ {\isacharequal}{\kern0pt}\ h{\isacharparenright}{\kern0pt}{\isacharparenright}{\kern0pt}{\isacharparenright}{\kern0pt}{\isachardoublequoteclose}\isanewline
\isanewline
\isacommand{lemma}\isamarkupfalse%
\ pullback{\isacharunderscore}{\kern0pt}unique{\isacharcolon}{\kern0pt}\isanewline
\ \ \isakeyword{assumes}\ {\isachardoublequoteopen}ab\ {\isacharcolon}{\kern0pt}\ A\ {\isasymrightarrow}\ B{\isachardoublequoteclose}\ {\isachardoublequoteopen}bd\ {\isacharcolon}{\kern0pt}\ B\ {\isasymrightarrow}\ D{\isachardoublequoteclose}\ {\isachardoublequoteopen}ac\ {\isacharcolon}{\kern0pt}\ A\ {\isasymrightarrow}\ C{\isachardoublequoteclose}\ {\isachardoublequoteopen}cd\ {\isacharcolon}{\kern0pt}\ C\ {\isasymrightarrow}\ D{\isachardoublequoteclose}\isanewline
\ \ \isakeyword{assumes}\ {\isachardoublequoteopen}k\ {\isacharcolon}{\kern0pt}\ Z\ {\isasymrightarrow}\ B{\isachardoublequoteclose}\ {\isachardoublequoteopen}h\ {\isacharcolon}{\kern0pt}\ Z\ {\isasymrightarrow}\ C{\isachardoublequoteclose}\isanewline
\ \ \isakeyword{assumes}\ {\isachardoublequoteopen}is{\isacharunderscore}{\kern0pt}pullback\ A\ B\ C\ D\ ab\ bd\ ac\ cd{\isachardoublequoteclose}\isanewline
\ \ \isakeyword{shows}\ {\isachardoublequoteopen}bd\ {\isasymcirc}\isactrlsub c\ k\ {\isacharequal}{\kern0pt}\ cd\ {\isasymcirc}\isactrlsub c\ h\ {\isasymLongrightarrow}\ {\isacharparenleft}{\kern0pt}{\isasymexists}{\isacharbang}{\kern0pt}\ j{\isachardot}{\kern0pt}\ j\ {\isacharcolon}{\kern0pt}\ Z\ {\isasymrightarrow}\ A\ {\isasymand}\ ab\ {\isasymcirc}\isactrlsub c\ j\ {\isacharequal}{\kern0pt}\ k\ {\isasymand}\ ac\ {\isasymcirc}\isactrlsub c\ j\ {\isacharequal}{\kern0pt}\ h{\isacharparenright}{\kern0pt}{\isachardoublequoteclose}\isanewline
%
\isadelimproof
\ \ %
\endisadelimproof
%
\isatagproof
\isacommand{using}\isamarkupfalse%
\ assms\ \isacommand{unfolding}\isamarkupfalse%
\ is{\isacharunderscore}{\kern0pt}pullback{\isacharunderscore}{\kern0pt}def\ \isacommand{by}\isamarkupfalse%
\ simp%
\endisatagproof
{\isafoldproof}%
%
\isadelimproof
\isanewline
%
\endisadelimproof
\isanewline
\isacommand{lemma}\isamarkupfalse%
\ pullback{\isacharunderscore}{\kern0pt}iff{\isacharunderscore}{\kern0pt}product{\isacharcolon}{\kern0pt}\isanewline
\ \ \isakeyword{assumes}\ {\isachardoublequoteopen}terminal{\isacharunderscore}{\kern0pt}object{\isacharparenleft}{\kern0pt}T{\isacharparenright}{\kern0pt}{\isachardoublequoteclose}\isanewline
\ \ \isakeyword{assumes}\ f{\isacharunderscore}{\kern0pt}type{\isacharbrackleft}{\kern0pt}type{\isacharunderscore}{\kern0pt}rule{\isacharbrackright}{\kern0pt}{\isacharcolon}{\kern0pt}\ {\isachardoublequoteopen}f\ {\isacharcolon}{\kern0pt}\ Y\ {\isasymrightarrow}\ T{\isachardoublequoteclose}\ \isanewline
\ \ \isakeyword{assumes}\ g{\isacharunderscore}{\kern0pt}type{\isacharbrackleft}{\kern0pt}type{\isacharunderscore}{\kern0pt}rule{\isacharbrackright}{\kern0pt}{\isacharcolon}{\kern0pt}\ {\isachardoublequoteopen}g\ {\isacharcolon}{\kern0pt}\ X\ {\isasymrightarrow}\ T{\isachardoublequoteclose}\isanewline
\ \ \isakeyword{shows}\ {\isachardoublequoteopen}{\isacharparenleft}{\kern0pt}is{\isacharunderscore}{\kern0pt}pullback\ P\ Y\ X\ T\ {\isacharparenleft}{\kern0pt}pY{\isacharparenright}{\kern0pt}\ f\ {\isacharparenleft}{\kern0pt}pX{\isacharparenright}{\kern0pt}\ g{\isacharparenright}{\kern0pt}\ {\isacharequal}{\kern0pt}\ {\isacharparenleft}{\kern0pt}is{\isacharunderscore}{\kern0pt}cart{\isacharunderscore}{\kern0pt}prod\ P\ pX\ pY\ X\ Y{\isacharparenright}{\kern0pt}{\isachardoublequoteclose}\isanewline
%
\isadelimproof
%
\endisadelimproof
%
\isatagproof
\isacommand{proof}\isamarkupfalse%
{\isacharparenleft}{\kern0pt}safe{\isacharparenright}{\kern0pt}\isanewline
\ \ \isacommand{assume}\isamarkupfalse%
\ pullback{\isacharcolon}{\kern0pt}\ {\isachardoublequoteopen}is{\isacharunderscore}{\kern0pt}pullback\ P\ Y\ X\ T\ pY\ f\ pX\ g{\isachardoublequoteclose}\isanewline
\ \ \isacommand{have}\isamarkupfalse%
\ f{\isacharunderscore}{\kern0pt}type{\isacharbrackleft}{\kern0pt}type{\isacharunderscore}{\kern0pt}rule{\isacharbrackright}{\kern0pt}{\isacharcolon}{\kern0pt}\ {\isachardoublequoteopen}f\ {\isacharcolon}{\kern0pt}\ Y\ {\isasymrightarrow}\ T{\isachardoublequoteclose}\isanewline
\ \ \ \ \isacommand{using}\isamarkupfalse%
\ is{\isacharunderscore}{\kern0pt}pullback{\isacharunderscore}{\kern0pt}def\ pullback\ \isacommand{by}\isamarkupfalse%
\ force\isanewline
\ \ \isacommand{have}\isamarkupfalse%
\ g{\isacharunderscore}{\kern0pt}type{\isacharbrackleft}{\kern0pt}type{\isacharunderscore}{\kern0pt}rule{\isacharbrackright}{\kern0pt}{\isacharcolon}{\kern0pt}\ {\isachardoublequoteopen}g\ {\isacharcolon}{\kern0pt}\ X\ {\isasymrightarrow}\ T{\isachardoublequoteclose}\isanewline
\ \ \ \ \isacommand{using}\isamarkupfalse%
\ is{\isacharunderscore}{\kern0pt}pullback{\isacharunderscore}{\kern0pt}def\ pullback\ \isacommand{by}\isamarkupfalse%
\ force\isanewline
\ \ \isacommand{show}\isamarkupfalse%
\ {\isachardoublequoteopen}is{\isacharunderscore}{\kern0pt}cart{\isacharunderscore}{\kern0pt}prod\ P\ pX\ pY\ X\ Y{\isachardoublequoteclose}\isanewline
\ \ \isacommand{proof}\isamarkupfalse%
{\isacharparenleft}{\kern0pt}unfold\ is{\isacharunderscore}{\kern0pt}cart{\isacharunderscore}{\kern0pt}prod{\isacharunderscore}{\kern0pt}def{\isacharcomma}{\kern0pt}\ safe{\isacharparenright}{\kern0pt}\isanewline
\ \ \ \ \isacommand{show}\isamarkupfalse%
\ pX{\isacharunderscore}{\kern0pt}type{\isacharbrackleft}{\kern0pt}type{\isacharunderscore}{\kern0pt}rule{\isacharbrackright}{\kern0pt}{\isacharcolon}{\kern0pt}\ {\isachardoublequoteopen}pX\ {\isacharcolon}{\kern0pt}\ P\ {\isasymrightarrow}\ X{\isachardoublequoteclose}\isanewline
\ \ \ \ \ \ \isacommand{using}\isamarkupfalse%
\ pullback\ is{\isacharunderscore}{\kern0pt}pullback{\isacharunderscore}{\kern0pt}def\ \isacommand{by}\isamarkupfalse%
\ force\isanewline
\ \ \ \ \isacommand{show}\isamarkupfalse%
\ pY{\isacharunderscore}{\kern0pt}type{\isacharbrackleft}{\kern0pt}type{\isacharunderscore}{\kern0pt}rule{\isacharbrackright}{\kern0pt}{\isacharcolon}{\kern0pt}\ {\isachardoublequoteopen}pY\ {\isacharcolon}{\kern0pt}\ P\ {\isasymrightarrow}\ Y{\isachardoublequoteclose}\isanewline
\ \ \ \ \ \ \isacommand{using}\isamarkupfalse%
\ pullback\ is{\isacharunderscore}{\kern0pt}pullback{\isacharunderscore}{\kern0pt}def\ \isacommand{by}\isamarkupfalse%
\ force\isanewline
\ \ \ \ \isacommand{show}\isamarkupfalse%
\ {\isachardoublequoteopen}{\isasymAnd}x\ y\ Z{\isachardot}{\kern0pt}\isanewline
\ \ \ \ \ \ \ x\ {\isacharcolon}{\kern0pt}\ Z\ {\isasymrightarrow}\ X\ {\isasymLongrightarrow}\isanewline
\ \ \ \ \ \ \ y\ {\isacharcolon}{\kern0pt}\ Z\ {\isasymrightarrow}\ Y\ {\isasymLongrightarrow}\isanewline
\ \ \ \ \ \ \ {\isasymexists}h{\isachardot}{\kern0pt}\ h\ {\isacharcolon}{\kern0pt}\ Z\ {\isasymrightarrow}\ P\ {\isasymand}\isanewline
\ \ \ \ \ \ \ \ \ \ \ pX\ {\isasymcirc}\isactrlsub c\ h\ {\isacharequal}{\kern0pt}\ x\ {\isasymand}\ pY\ {\isasymcirc}\isactrlsub c\ h\ {\isacharequal}{\kern0pt}\ y\ {\isasymand}\ {\isacharparenleft}{\kern0pt}{\isasymforall}h{\isadigit{2}}{\isachardot}{\kern0pt}\ h{\isadigit{2}}\ {\isacharcolon}{\kern0pt}\ Z\ {\isasymrightarrow}\ P\ {\isasymand}\ pX\ {\isasymcirc}\isactrlsub c\ h{\isadigit{2}}\ {\isacharequal}{\kern0pt}\ x\ {\isasymand}\ pY\ {\isasymcirc}\isactrlsub c\ h{\isadigit{2}}\ {\isacharequal}{\kern0pt}\ y\ {\isasymlongrightarrow}\ h{\isadigit{2}}\ {\isacharequal}{\kern0pt}\ h{\isacharparenright}{\kern0pt}{\isachardoublequoteclose}\isanewline
\ \ \ \ \isacommand{proof}\isamarkupfalse%
\ {\isacharminus}{\kern0pt}\ \isanewline
\ \ \ \ \ \ \isacommand{fix}\isamarkupfalse%
\ x\ y\ Z\isanewline
\ \ \ \ \ \ \isacommand{assume}\isamarkupfalse%
\ x{\isacharunderscore}{\kern0pt}type{\isacharbrackleft}{\kern0pt}type{\isacharunderscore}{\kern0pt}rule{\isacharbrackright}{\kern0pt}{\isacharcolon}{\kern0pt}\ {\isachardoublequoteopen}x\ {\isacharcolon}{\kern0pt}\ Z\ {\isasymrightarrow}\ X{\isachardoublequoteclose}\isanewline
\ \ \ \ \ \ \isacommand{assume}\isamarkupfalse%
\ y{\isacharunderscore}{\kern0pt}type{\isacharbrackleft}{\kern0pt}type{\isacharunderscore}{\kern0pt}rule{\isacharbrackright}{\kern0pt}{\isacharcolon}{\kern0pt}\ {\isachardoublequoteopen}y\ {\isacharcolon}{\kern0pt}\ Z\ {\isasymrightarrow}\ Y{\isachardoublequoteclose}\isanewline
\ \ \ \ \ \ \isacommand{have}\isamarkupfalse%
\ \ {\isachardoublequoteopen}{\isasymAnd}Z\ k\ h{\isachardot}{\kern0pt}\ k\ {\isacharcolon}{\kern0pt}\ Z\ {\isasymrightarrow}\ Y\ {\isasymLongrightarrow}\ h\ {\isacharcolon}{\kern0pt}\ Z\ {\isasymrightarrow}\ X\ {\isasymLongrightarrow}\ f\ {\isasymcirc}\isactrlsub c\ k\ {\isacharequal}{\kern0pt}\ g\ {\isasymcirc}\isactrlsub c\ h\ {\isasymLongrightarrow}\ {\isasymexists}j{\isachardot}{\kern0pt}\ j\ {\isacharcolon}{\kern0pt}\ Z\ {\isasymrightarrow}\ P\ {\isasymand}\ pY\ {\isasymcirc}\isactrlsub c\ j\ {\isacharequal}{\kern0pt}\ k\ {\isasymand}\ pX\ {\isasymcirc}\isactrlsub c\ j\ {\isacharequal}{\kern0pt}\ h{\isachardoublequoteclose}\isanewline
\ \ \ \ \ \ \ \ \isacommand{using}\isamarkupfalse%
\ is{\isacharunderscore}{\kern0pt}pullback{\isacharunderscore}{\kern0pt}def\ pullback\ \isacommand{by}\isamarkupfalse%
\ blast\isanewline
\ \ \ \ \ \ \isacommand{then}\isamarkupfalse%
\ \isacommand{have}\isamarkupfalse%
\ {\isachardoublequoteopen}{\isasymexists}h{\isachardot}{\kern0pt}\ h\ {\isacharcolon}{\kern0pt}\ Z\ {\isasymrightarrow}\ P\ {\isasymand}\isanewline
\ \ \ \ \ \ \ \ \ \ \ pX\ {\isasymcirc}\isactrlsub c\ h\ {\isacharequal}{\kern0pt}\ x\ {\isasymand}\ pY\ {\isasymcirc}\isactrlsub c\ h\ {\isacharequal}{\kern0pt}\ y{\isachardoublequoteclose}\isanewline
\ \ \ \ \ \ \ \ \isacommand{by}\isamarkupfalse%
\ {\isacharparenleft}{\kern0pt}smt\ {\isacharparenleft}{\kern0pt}verit{\isacharcomma}{\kern0pt}\ ccfv{\isacharunderscore}{\kern0pt}threshold{\isacharparenright}{\kern0pt}\ assms\ cfunc{\isacharunderscore}{\kern0pt}type{\isacharunderscore}{\kern0pt}def\ codomain{\isacharunderscore}{\kern0pt}comp\ domain{\isacharunderscore}{\kern0pt}comp\ f{\isacharunderscore}{\kern0pt}type\ g{\isacharunderscore}{\kern0pt}type\ terminal{\isacharunderscore}{\kern0pt}object{\isacharunderscore}{\kern0pt}def\ x{\isacharunderscore}{\kern0pt}type\ y{\isacharunderscore}{\kern0pt}type{\isacharparenright}{\kern0pt}\isanewline
\ \ \ \ \ \ \isacommand{then}\isamarkupfalse%
\ \isacommand{show}\isamarkupfalse%
\ {\isachardoublequoteopen}{\isasymexists}h{\isachardot}{\kern0pt}\ h\ {\isacharcolon}{\kern0pt}\ Z\ {\isasymrightarrow}\ P\ {\isasymand}\isanewline
\ \ \ \ \ \ \ \ \ \ \ pX\ {\isasymcirc}\isactrlsub c\ h\ {\isacharequal}{\kern0pt}\ x\ {\isasymand}\ pY\ {\isasymcirc}\isactrlsub c\ h\ {\isacharequal}{\kern0pt}\ y\ {\isasymand}\ {\isacharparenleft}{\kern0pt}{\isasymforall}h{\isadigit{2}}{\isachardot}{\kern0pt}\ h{\isadigit{2}}\ {\isacharcolon}{\kern0pt}\ Z\ {\isasymrightarrow}\ P\ {\isasymand}\ pX\ {\isasymcirc}\isactrlsub c\ h{\isadigit{2}}\ {\isacharequal}{\kern0pt}\ x\ {\isasymand}\ pY\ {\isasymcirc}\isactrlsub c\ h{\isadigit{2}}\ {\isacharequal}{\kern0pt}\ y\ {\isasymlongrightarrow}\ h{\isadigit{2}}\ {\isacharequal}{\kern0pt}\ h{\isacharparenright}{\kern0pt}{\isachardoublequoteclose}\isanewline
\ \ \ \ \ \ \ \ \isacommand{by}\isamarkupfalse%
\ {\isacharparenleft}{\kern0pt}typecheck{\isacharunderscore}{\kern0pt}cfuncs{\isacharcomma}{\kern0pt}\ smt\ {\isacharparenleft}{\kern0pt}verit{\isacharcomma}{\kern0pt}\ ccfv{\isacharunderscore}{\kern0pt}threshold{\isacharparenright}{\kern0pt}\ comp{\isacharunderscore}{\kern0pt}associative{\isadigit{2}}\ is{\isacharunderscore}{\kern0pt}pullback{\isacharunderscore}{\kern0pt}def\ pullback{\isacharparenright}{\kern0pt}\isanewline
\ \ \ \ \isacommand{qed}\isamarkupfalse%
\isanewline
\ \ \isacommand{qed}\isamarkupfalse%
\isanewline
\isacommand{next}\isamarkupfalse%
\isanewline
\ \ \isacommand{assume}\isamarkupfalse%
\ prod{\isacharcolon}{\kern0pt}\ {\isachardoublequoteopen}is{\isacharunderscore}{\kern0pt}cart{\isacharunderscore}{\kern0pt}prod\ P\ pX\ pY\ X\ Y{\isachardoublequoteclose}\isanewline
\ \ \isacommand{then}\isamarkupfalse%
\ \isacommand{show}\isamarkupfalse%
\ {\isachardoublequoteopen}is{\isacharunderscore}{\kern0pt}pullback\ P\ Y\ X\ T\ pY\ f\ pX\ g{\isachardoublequoteclose}\isanewline
\ \ \isacommand{proof}\isamarkupfalse%
{\isacharparenleft}{\kern0pt}unfold\ is{\isacharunderscore}{\kern0pt}cart{\isacharunderscore}{\kern0pt}prod{\isacharunderscore}{\kern0pt}def\ is{\isacharunderscore}{\kern0pt}pullback{\isacharunderscore}{\kern0pt}def{\isacharcomma}{\kern0pt}\ typecheck{\isacharunderscore}{\kern0pt}cfuncs{\isacharcomma}{\kern0pt}\ safe{\isacharparenright}{\kern0pt}\isanewline
\ \ \ \ \isacommand{assume}\isamarkupfalse%
\ pX{\isacharunderscore}{\kern0pt}type{\isacharbrackleft}{\kern0pt}type{\isacharunderscore}{\kern0pt}rule{\isacharbrackright}{\kern0pt}{\isacharcolon}{\kern0pt}\ {\isachardoublequoteopen}pX\ {\isacharcolon}{\kern0pt}\ P\ {\isasymrightarrow}\ X{\isachardoublequoteclose}\isanewline
\ \ \ \ \isacommand{assume}\isamarkupfalse%
\ pY{\isacharunderscore}{\kern0pt}type{\isacharbrackleft}{\kern0pt}type{\isacharunderscore}{\kern0pt}rule{\isacharbrackright}{\kern0pt}{\isacharcolon}{\kern0pt}\ {\isachardoublequoteopen}pY\ {\isacharcolon}{\kern0pt}\ P\ {\isasymrightarrow}\ Y{\isachardoublequoteclose}\isanewline
\ \ \ \ \isacommand{show}\isamarkupfalse%
\ {\isachardoublequoteopen}f\ {\isasymcirc}\isactrlsub c\ pY\ {\isacharequal}{\kern0pt}\ g\ {\isasymcirc}\isactrlsub c\ pX{\isachardoublequoteclose}\isanewline
\ \ \ \ \ \ \isacommand{using}\isamarkupfalse%
\ assms{\isacharparenleft}{\kern0pt}{\isadigit{1}}{\isacharparenright}{\kern0pt}\ terminal{\isacharunderscore}{\kern0pt}object{\isacharunderscore}{\kern0pt}def\ \isacommand{by}\isamarkupfalse%
\ {\isacharparenleft}{\kern0pt}typecheck{\isacharunderscore}{\kern0pt}cfuncs{\isacharcomma}{\kern0pt}\ auto{\isacharparenright}{\kern0pt}\ \ \isanewline
\ \ \ \ \isacommand{show}\isamarkupfalse%
\ {\isachardoublequoteopen}{\isasymAnd}Z\ k\ h{\isachardot}{\kern0pt}\ k\ {\isacharcolon}{\kern0pt}\ Z\ {\isasymrightarrow}\ Y\ {\isasymLongrightarrow}\ h\ {\isacharcolon}{\kern0pt}\ Z\ {\isasymrightarrow}\ X\ {\isasymLongrightarrow}\ f\ {\isasymcirc}\isactrlsub c\ k\ {\isacharequal}{\kern0pt}\ g\ {\isasymcirc}\isactrlsub c\ h\ {\isasymLongrightarrow}\ {\isasymexists}j{\isachardot}{\kern0pt}\ j\ {\isacharcolon}{\kern0pt}\ Z\ {\isasymrightarrow}\ P\ {\isasymand}\ pY\ {\isasymcirc}\isactrlsub c\ j\ {\isacharequal}{\kern0pt}\ k\ {\isasymand}\ pX\ {\isasymcirc}\isactrlsub c\ j\ {\isacharequal}{\kern0pt}\ h{\isachardoublequoteclose}\isanewline
\ \ \ \ \ \ \isacommand{using}\isamarkupfalse%
\ is{\isacharunderscore}{\kern0pt}cart{\isacharunderscore}{\kern0pt}prod{\isacharunderscore}{\kern0pt}def\ prod\ \isacommand{by}\isamarkupfalse%
\ blast\isanewline
\ \ \ \ \isacommand{show}\isamarkupfalse%
\ {\isachardoublequoteopen}{\isasymAnd}Z\ j\ y{\isachardot}{\kern0pt}\isanewline
\ \ \ \ \ \ \ pY\ {\isasymcirc}\isactrlsub c\ j\ {\isacharcolon}{\kern0pt}\ Z\ {\isasymrightarrow}\ Y\ {\isasymLongrightarrow}\isanewline
\ \ \ \ \ \ \ pX\ {\isasymcirc}\isactrlsub c\ j\ {\isacharcolon}{\kern0pt}\ Z\ {\isasymrightarrow}\ X\ {\isasymLongrightarrow}\isanewline
\ \ \ \ \ \ \ f\ {\isasymcirc}\isactrlsub c\ pY\ {\isasymcirc}\isactrlsub c\ j\ {\isacharequal}{\kern0pt}\ g\ {\isasymcirc}\isactrlsub c\ pX\ {\isasymcirc}\isactrlsub c\ j\ {\isasymLongrightarrow}\ j\ {\isacharcolon}{\kern0pt}\ Z\ {\isasymrightarrow}\ P\ {\isasymLongrightarrow}\ y\ {\isacharcolon}{\kern0pt}\ Z\ {\isasymrightarrow}\ P\ {\isasymLongrightarrow}\ pY\ {\isasymcirc}\isactrlsub c\ y\ {\isacharequal}{\kern0pt}\ pY\ {\isasymcirc}\isactrlsub c\ j\ {\isasymLongrightarrow}\ pX\ {\isasymcirc}\isactrlsub c\ y\ {\isacharequal}{\kern0pt}\ pX\ {\isasymcirc}\isactrlsub c\ j\ {\isasymLongrightarrow}\ j\ {\isacharequal}{\kern0pt}\ y{\isachardoublequoteclose}\isanewline
\ \ \ \ \ \ \isacommand{using}\isamarkupfalse%
\ is{\isacharunderscore}{\kern0pt}cart{\isacharunderscore}{\kern0pt}prod{\isacharunderscore}{\kern0pt}def\ prod\ \isacommand{by}\isamarkupfalse%
\ blast\isanewline
\ \ \isacommand{qed}\isamarkupfalse%
\isanewline
\isacommand{qed}\isamarkupfalse%
%
\endisatagproof
{\isafoldproof}%
%
\isadelimproof
\isanewline
%
\endisadelimproof
%
\isadelimtheory
\isanewline
%
\endisadelimtheory
%
\isatagtheory
\isacommand{end}\isamarkupfalse%
%
\endisatagtheory
{\isafoldtheory}%
%
\isadelimtheory
%
\endisadelimtheory
%
\end{isabellebody}%
\endinput
%:%file=~/ETCS/HOL-ETCS/Terminal.thy%:%
%:%11=1%:%
%:%27=3%:%
%:%28=3%:%
%:%29=4%:%
%:%30=5%:%
%:%39=7%:%
%:%41=8%:%
%:%42=8%:%
%:%43=9%:%
%:%44=10%:%
%:%45=11%:%
%:%46=12%:%
%:%47=13%:%
%:%48=14%:%
%:%49=15%:%
%:%50=16%:%
%:%51=16%:%
%:%52=17%:%
%:%53=18%:%
%:%56=19%:%
%:%60=19%:%
%:%61=19%:%
%:%62=19%:%
%:%67=19%:%
%:%70=20%:%
%:%71=21%:%
%:%72=21%:%
%:%73=22%:%
%:%76=23%:%
%:%80=23%:%
%:%81=23%:%
%:%86=23%:%
%:%89=24%:%
%:%90=25%:%
%:%91=25%:%
%:%92=26%:%
%:%95=27%:%
%:%99=27%:%
%:%100=27%:%
%:%114=29%:%
%:%126=31%:%
%:%128=32%:%
%:%129=32%:%
%:%130=33%:%
%:%131=34%:%
%:%132=35%:%
%:%133=35%:%
%:%134=36%:%
%:%135=37%:%
%:%136=38%:%
%:%137=38%:%
%:%138=39%:%
%:%140=41%:%
%:%142=42%:%
%:%143=42%:%
%:%144=43%:%
%:%147=44%:%
%:%151=44%:%
%:%152=44%:%
%:%153=45%:%
%:%154=45%:%
%:%159=45%:%
%:%162=46%:%
%:%163=47%:%
%:%164=47%:%
%:%165=48%:%
%:%168=49%:%
%:%172=49%:%
%:%173=49%:%
%:%174=49%:%
%:%179=49%:%
%:%182=50%:%
%:%183=51%:%
%:%184=51%:%
%:%185=52%:%
%:%186=53%:%
%:%189=54%:%
%:%193=54%:%
%:%194=54%:%
%:%199=54%:%
%:%202=55%:%
%:%203=56%:%
%:%204=56%:%
%:%205=57%:%
%:%206=58%:%
%:%209=59%:%
%:%213=59%:%
%:%214=59%:%
%:%215=59%:%
%:%229=61%:%
%:%239=63%:%
%:%240=63%:%
%:%241=64%:%
%:%242=65%:%
%:%243=66%:%
%:%244=66%:%
%:%247=67%:%
%:%251=67%:%
%:%252=67%:%
%:%253=67%:%
%:%254=67%:%
%:%263=69%:%
%:%265=70%:%
%:%266=70%:%
%:%267=71%:%
%:%268=72%:%
%:%269=73%:%
%:%272=74%:%
%:%276=74%:%
%:%277=74%:%
%:%278=75%:%
%:%279=75%:%
%:%288=77%:%
%:%290=78%:%
%:%291=78%:%
%:%292=79%:%
%:%293=80%:%
%:%296=81%:%
%:%300=81%:%
%:%301=81%:%
%:%302=82%:%
%:%303=82%:%
%:%304=83%:%
%:%305=83%:%
%:%306=84%:%
%:%307=84%:%
%:%308=84%:%
%:%309=85%:%
%:%310=86%:%
%:%311=86%:%
%:%312=87%:%
%:%313=87%:%
%:%314=87%:%
%:%315=88%:%
%:%316=89%:%
%:%317=89%:%
%:%318=90%:%
%:%319=90%:%
%:%320=90%:%
%:%321=91%:%
%:%322=92%:%
%:%323=92%:%
%:%324=93%:%
%:%325=93%:%
%:%326=93%:%
%:%327=94%:%
%:%328=95%:%
%:%329=95%:%
%:%330=96%:%
%:%331=96%:%
%:%332=97%:%
%:%333=97%:%
%:%334=98%:%
%:%335=98%:%
%:%336=99%:%
%:%337=100%:%
%:%338=100%:%
%:%339=101%:%
%:%340=101%:%
%:%341=101%:%
%:%342=102%:%
%:%352=104%:%
%:%354=105%:%
%:%355=105%:%
%:%356=106%:%
%:%357=107%:%
%:%360=108%:%
%:%364=108%:%
%:%365=108%:%
%:%366=109%:%
%:%367=109%:%
%:%368=110%:%
%:%369=110%:%
%:%370=111%:%
%:%371=111%:%
%:%372=112%:%
%:%373=112%:%
%:%374=113%:%
%:%375=113%:%
%:%376=114%:%
%:%377=114%:%
%:%378=115%:%
%:%379=115%:%
%:%380=116%:%
%:%381=116%:%
%:%382=117%:%
%:%383=117%:%
%:%384=118%:%
%:%385=118%:%
%:%386=119%:%
%:%387=119%:%
%:%388=120%:%
%:%389=120%:%
%:%390=121%:%
%:%391=121%:%
%:%392=122%:%
%:%393=122%:%
%:%394=123%:%
%:%395=123%:%
%:%396=123%:%
%:%397=124%:%
%:%398=124%:%
%:%399=124%:%
%:%400=125%:%
%:%401=125%:%
%:%402=125%:%
%:%403=126%:%
%:%404=126%:%
%:%405=127%:%
%:%406=127%:%
%:%407=128%:%
%:%408=128%:%
%:%409=129%:%
%:%415=129%:%
%:%418=130%:%
%:%419=131%:%
%:%420=131%:%
%:%421=132%:%
%:%422=133%:%
%:%423=134%:%
%:%426=135%:%
%:%430=135%:%
%:%431=135%:%
%:%440=137%:%
%:%442=138%:%
%:%443=138%:%
%:%444=139%:%
%:%451=140%:%
%:%452=140%:%
%:%453=141%:%
%:%454=141%:%
%:%455=142%:%
%:%456=142%:%
%:%457=142%:%
%:%458=143%:%
%:%459=143%:%
%:%460=144%:%
%:%461=144%:%
%:%462=144%:%
%:%463=145%:%
%:%464=145%:%
%:%465=145%:%
%:%466=146%:%
%:%467=146%:%
%:%468=147%:%
%:%469=147%:%
%:%470=148%:%
%:%471=148%:%
%:%472=149%:%
%:%473=149%:%
%:%474=150%:%
%:%475=150%:%
%:%476=151%:%
%:%477=151%:%
%:%478=152%:%
%:%479=152%:%
%:%480=153%:%
%:%481=153%:%
%:%482=154%:%
%:%483=154%:%
%:%484=155%:%
%:%485=155%:%
%:%486=155%:%
%:%487=156%:%
%:%488=156%:%
%:%489=157%:%
%:%490=157%:%
%:%491=157%:%
%:%492=158%:%
%:%493=158%:%
%:%494=158%:%
%:%495=159%:%
%:%496=159%:%
%:%497=160%:%
%:%498=160%:%
%:%499=160%:%
%:%500=161%:%
%:%501=161%:%
%:%502=161%:%
%:%503=162%:%
%:%504=162%:%
%:%505=163%:%
%:%506=163%:%
%:%507=164%:%
%:%508=164%:%
%:%509=165%:%
%:%510=165%:%
%:%511=165%:%
%:%512=166%:%
%:%513=166%:%
%:%514=167%:%
%:%515=167%:%
%:%516=168%:%
%:%517=168%:%
%:%518=169%:%
%:%519=169%:%
%:%520=170%:%
%:%521=170%:%
%:%522=171%:%
%:%523=171%:%
%:%524=172%:%
%:%525=172%:%
%:%526=173%:%
%:%527=173%:%
%:%528=174%:%
%:%529=174%:%
%:%530=174%:%
%:%531=175%:%
%:%532=175%:%
%:%533=176%:%
%:%534=176%:%
%:%535=177%:%
%:%536=177%:%
%:%537=178%:%
%:%538=178%:%
%:%539=179%:%
%:%540=179%:%
%:%541=180%:%
%:%542=180%:%
%:%543=181%:%
%:%544=181%:%
%:%545=181%:%
%:%546=182%:%
%:%547=182%:%
%:%548=183%:%
%:%549=183%:%
%:%550=183%:%
%:%551=184%:%
%:%552=184%:%
%:%553=185%:%
%:%554=185%:%
%:%555=185%:%
%:%556=186%:%
%:%557=186%:%
%:%558=187%:%
%:%559=187%:%
%:%560=187%:%
%:%561=188%:%
%:%562=188%:%
%:%563=189%:%
%:%564=189%:%
%:%565=189%:%
%:%566=190%:%
%:%567=190%:%
%:%568=191%:%
%:%569=191%:%
%:%570=192%:%
%:%571=192%:%
%:%572=193%:%
%:%573=193%:%
%:%574=193%:%
%:%575=194%:%
%:%576=194%:%
%:%577=195%:%
%:%592=197%:%
%:%604=199%:%
%:%606=200%:%
%:%607=200%:%
%:%608=201%:%
%:%609=202%:%
%:%610=203%:%
%:%611=203%:%
%:%612=204%:%
%:%613=205%:%
%:%616=206%:%
%:%620=206%:%
%:%621=206%:%
%:%622=206%:%
%:%631=208%:%
%:%633=209%:%
%:%634=209%:%
%:%635=210%:%
%:%638=211%:%
%:%642=211%:%
%:%643=211%:%
%:%652=213%:%
%:%654=214%:%
%:%655=214%:%
%:%656=215%:%
%:%659=216%:%
%:%663=216%:%
%:%664=216%:%
%:%665=217%:%
%:%666=217%:%
%:%667=218%:%
%:%668=218%:%
%:%669=219%:%
%:%670=219%:%
%:%671=220%:%
%:%672=220%:%
%:%673=221%:%
%:%674=221%:%
%:%675=222%:%
%:%676=222%:%
%:%677=223%:%
%:%678=224%:%
%:%679=224%:%
%:%680=225%:%
%:%681=225%:%
%:%682=225%:%
%:%683=226%:%
%:%684=226%:%
%:%685=227%:%
%:%686=227%:%
%:%687=228%:%
%:%688=229%:%
%:%689=229%:%
%:%690=230%:%
%:%691=230%:%
%:%692=231%:%
%:%693=231%:%
%:%694=232%:%
%:%695=232%:%
%:%696=233%:%
%:%697=234%:%
%:%698=234%:%
%:%699=235%:%
%:%700=235%:%
%:%701=235%:%
%:%702=236%:%
%:%703=236%:%
%:%704=236%:%
%:%705=237%:%
%:%706=237%:%
%:%707=237%:%
%:%708=238%:%
%:%709=238%:%
%:%710=239%:%
%:%711=239%:%
%:%712=239%:%
%:%713=240%:%
%:%714=240%:%
%:%715=240%:%
%:%716=241%:%
%:%722=241%:%
%:%725=242%:%
%:%726=243%:%
%:%727=243%:%
%:%728=244%:%
%:%729=245%:%
%:%730=246%:%
%:%733=247%:%
%:%737=247%:%
%:%738=247%:%
%:%743=247%:%
%:%746=248%:%
%:%747=249%:%
%:%748=249%:%
%:%749=250%:%
%:%750=251%:%
%:%751=252%:%
%:%752=253%:%
%:%755=254%:%
%:%759=254%:%
%:%760=254%:%
%:%761=255%:%
%:%762=255%:%
%:%763=256%:%
%:%764=256%:%
%:%765=257%:%
%:%766=257%:%
%:%767=258%:%
%:%768=258%:%
%:%769=259%:%
%:%770=259%:%
%:%771=260%:%
%:%772=260%:%
%:%773=261%:%
%:%774=261%:%
%:%775=261%:%
%:%776=262%:%
%:%777=262%:%
%:%778=263%:%
%:%779=263%:%
%:%780=263%:%
%:%781=264%:%
%:%782=264%:%
%:%783=265%:%
%:%784=265%:%
%:%785=265%:%
%:%786=266%:%
%:%787=266%:%
%:%788=267%:%
%:%789=267%:%
%:%790=268%:%
%:%791=268%:%
%:%792=269%:%
%:%793=269%:%
%:%794=269%:%
%:%795=270%:%
%:%796=271%:%
%:%797=271%:%
%:%798=272%:%
%:%799=272%:%
%:%800=273%:%
%:%801=273%:%
%:%802=274%:%
%:%803=274%:%
%:%804=275%:%
%:%805=275%:%
%:%806=276%:%
%:%807=276%:%
%:%808=276%:%
%:%809=277%:%
%:%810=277%:%
%:%811=277%:%
%:%812=278%:%
%:%813=278%:%
%:%814=279%:%
%:%815=279%:%
%:%816=279%:%
%:%817=280%:%
%:%818=280%:%
%:%819=280%:%
%:%820=281%:%
%:%821=281%:%
%:%822=281%:%
%:%823=282%:%
%:%824=282%:%
%:%825=283%:%
%:%826=283%:%
%:%827=283%:%
%:%828=284%:%
%:%829=284%:%
%:%830=284%:%
%:%831=285%:%
%:%832=285%:%
%:%833=286%:%
%:%834=286%:%
%:%835=287%:%
%:%836=287%:%
%:%837=288%:%
%:%838=288%:%
%:%839=289%:%
%:%840=289%:%
%:%841=290%:%
%:%842=290%:%
%:%843=291%:%
%:%844=291%:%
%:%845=292%:%
%:%846=292%:%
%:%847=292%:%
%:%848=293%:%
%:%849=293%:%
%:%850=294%:%
%:%851=294%:%
%:%852=294%:%
%:%853=295%:%
%:%854=295%:%
%:%855=296%:%
%:%856=296%:%
%:%857=296%:%
%:%858=297%:%
%:%859=297%:%
%:%860=298%:%
%:%861=298%:%
%:%862=299%:%
%:%863=299%:%
%:%864=300%:%
%:%865=300%:%
%:%866=300%:%
%:%867=301%:%
%:%868=301%:%
%:%869=302%:%
%:%870=302%:%
%:%871=303%:%
%:%872=303%:%
%:%873=304%:%
%:%874=304%:%
%:%875=305%:%
%:%876=305%:%
%:%877=306%:%
%:%878=306%:%
%:%879=306%:%
%:%880=307%:%
%:%881=307%:%
%:%882=307%:%
%:%883=308%:%
%:%884=308%:%
%:%885=309%:%
%:%886=309%:%
%:%887=309%:%
%:%888=310%:%
%:%889=310%:%
%:%890=310%:%
%:%891=311%:%
%:%892=311%:%
%:%893=311%:%
%:%894=312%:%
%:%895=312%:%
%:%896=313%:%
%:%897=313%:%
%:%898=313%:%
%:%899=314%:%
%:%900=314%:%
%:%901=314%:%
%:%902=315%:%
%:%903=315%:%
%:%904=316%:%
%:%914=318%:%
%:%915=319%:%
%:%916=320%:%
%:%918=321%:%
%:%919=321%:%
%:%920=322%:%
%:%921=323%:%
%:%922=324%:%
%:%925=325%:%
%:%929=325%:%
%:%930=325%:%
%:%931=326%:%
%:%932=326%:%
%:%933=327%:%
%:%934=327%:%
%:%935=328%:%
%:%936=328%:%
%:%937=329%:%
%:%938=329%:%
%:%939=330%:%
%:%940=330%:%
%:%941=331%:%
%:%942=331%:%
%:%943=331%:%
%:%944=332%:%
%:%945=332%:%
%:%946=332%:%
%:%947=333%:%
%:%948=333%:%
%:%949=334%:%
%:%950=334%:%
%:%951=335%:%
%:%952=335%:%
%:%953=335%:%
%:%954=336%:%
%:%955=336%:%
%:%956=336%:%
%:%957=337%:%
%:%958=337%:%
%:%959=337%:%
%:%960=338%:%
%:%961=338%:%
%:%962=338%:%
%:%963=339%:%
%:%964=339%:%
%:%965=339%:%
%:%966=340%:%
%:%967=340%:%
%:%968=340%:%
%:%969=341%:%
%:%970=341%:%
%:%971=341%:%
%:%972=342%:%
%:%973=342%:%
%:%974=343%:%
%:%975=343%:%
%:%976=344%:%
%:%977=344%:%
%:%978=345%:%
%:%979=345%:%
%:%980=346%:%
%:%981=346%:%
%:%982=347%:%
%:%983=347%:%
%:%984=348%:%
%:%985=348%:%
%:%986=349%:%
%:%987=349%:%
%:%988=350%:%
%:%989=350%:%
%:%990=350%:%
%:%991=351%:%
%:%992=351%:%
%:%993=351%:%
%:%994=352%:%
%:%995=352%:%
%:%996=352%:%
%:%997=353%:%
%:%998=353%:%
%:%999=353%:%
%:%1000=354%:%
%:%1001=354%:%
%:%1002=354%:%
%:%1003=355%:%
%:%1004=355%:%
%:%1005=355%:%
%:%1006=356%:%
%:%1007=356%:%
%:%1008=356%:%
%:%1009=357%:%
%:%1010=357%:%
%:%1011=358%:%
%:%1026=360%:%
%:%1038=362%:%
%:%1040=363%:%
%:%1041=363%:%
%:%1042=364%:%
%:%1043=365%:%
%:%1044=366%:%
%:%1045=366%:%
%:%1046=367%:%
%:%1047=368%:%
%:%1050=369%:%
%:%1054=369%:%
%:%1055=369%:%
%:%1056=369%:%
%:%1057=369%:%
%:%1066=371%:%
%:%1068=372%:%
%:%1069=372%:%
%:%1070=373%:%
%:%1073=374%:%
%:%1077=374%:%
%:%1078=374%:%
%:%1079=375%:%
%:%1080=375%:%
%:%1081=376%:%
%:%1082=376%:%
%:%1083=377%:%
%:%1084=377%:%
%:%1085=378%:%
%:%1086=378%:%
%:%1087=379%:%
%:%1088=379%:%
%:%1089=380%:%
%:%1090=380%:%
%:%1091=381%:%
%:%1092=381%:%
%:%1093=382%:%
%:%1094=383%:%
%:%1095=383%:%
%:%1096=384%:%
%:%1097=384%:%
%:%1098=384%:%
%:%1099=385%:%
%:%1100=385%:%
%:%1101=386%:%
%:%1102=386%:%
%:%1103=387%:%
%:%1104=387%:%
%:%1105=388%:%
%:%1106=388%:%
%:%1107=389%:%
%:%1108=389%:%
%:%1109=390%:%
%:%1110=390%:%
%:%1111=390%:%
%:%1112=391%:%
%:%1113=391%:%
%:%1114=391%:%
%:%1115=392%:%
%:%1116=392%:%
%:%1117=392%:%
%:%1118=393%:%
%:%1119=393%:%
%:%1120=393%:%
%:%1121=394%:%
%:%1122=394%:%
%:%1123=394%:%
%:%1124=395%:%
%:%1125=395%:%
%:%1126=395%:%
%:%1127=396%:%
%:%1128=396%:%
%:%1129=396%:%
%:%1130=397%:%
%:%1131=397%:%
%:%1132=397%:%
%:%1133=398%:%
%:%1134=398%:%
%:%1135=399%:%
%:%1136=399%:%
%:%1137=400%:%
%:%1138=400%:%
%:%1139=401%:%
%:%1140=401%:%
%:%1141=402%:%
%:%1142=402%:%
%:%1143=403%:%
%:%1144=403%:%
%:%1145=403%:%
%:%1146=404%:%
%:%1147=404%:%
%:%1148=405%:%
%:%1158=407%:%
%:%1160=408%:%
%:%1161=408%:%
%:%1162=409%:%
%:%1163=410%:%
%:%1164=411%:%
%:%1167=412%:%
%:%1171=412%:%
%:%1172=412%:%
%:%1173=413%:%
%:%1174=413%:%
%:%1175=414%:%
%:%1176=414%:%
%:%1177=415%:%
%:%1178=415%:%
%:%1179=416%:%
%:%1180=416%:%
%:%1181=417%:%
%:%1182=417%:%
%:%1183=417%:%
%:%1184=418%:%
%:%1185=418%:%
%:%1186=418%:%
%:%1187=419%:%
%:%1188=419%:%
%:%1189=419%:%
%:%1190=420%:%
%:%1191=420%:%
%:%1192=420%:%
%:%1193=421%:%
%:%1194=421%:%
%:%1195=421%:%
%:%1196=422%:%
%:%1197=422%:%
%:%1198=422%:%
%:%1199=423%:%
%:%1200=423%:%
%:%1201=424%:%
%:%1202=424%:%
%:%1203=424%:%
%:%1204=425%:%
%:%1205=425%:%
%:%1206=426%:%
%:%1207=426%:%
%:%1208=426%:%
%:%1209=427%:%
%:%1210=427%:%
%:%1211=428%:%
%:%1212=428%:%
%:%1213=428%:%
%:%1214=429%:%
%:%1215=429%:%
%:%1216=430%:%
%:%1217=430%:%
%:%1218=431%:%
%:%1219=431%:%
%:%1220=431%:%
%:%1221=432%:%
%:%1222=432%:%
%:%1223=433%:%
%:%1224=433%:%
%:%1225=433%:%
%:%1226=434%:%
%:%1227=434%:%
%:%1228=434%:%
%:%1229=435%:%
%:%1230=435%:%
%:%1231=435%:%
%:%1232=436%:%
%:%1238=436%:%
%:%1241=437%:%
%:%1242=438%:%
%:%1243=438%:%
%:%1244=439%:%
%:%1245=440%:%
%:%1246=441%:%
%:%1247=442%:%
%:%1250=443%:%
%:%1254=443%:%
%:%1255=443%:%
%:%1256=444%:%
%:%1257=444%:%
%:%1258=445%:%
%:%1259=445%:%
%:%1260=446%:%
%:%1261=446%:%
%:%1262=447%:%
%:%1263=447%:%
%:%1264=447%:%
%:%1265=448%:%
%:%1266=448%:%
%:%1267=448%:%
%:%1268=449%:%
%:%1269=449%:%
%:%1270=450%:%
%:%1271=450%:%
%:%1272=450%:%
%:%1273=451%:%
%:%1274=451%:%
%:%1275=451%:%
%:%1276=452%:%
%:%1277=452%:%
%:%1278=452%:%
%:%1279=453%:%
%:%1280=453%:%
%:%1281=453%:%
%:%1282=454%:%
%:%1283=454%:%
%:%1284=454%:%
%:%1285=455%:%
%:%1286=455%:%
%:%1287=456%:%
%:%1288=456%:%
%:%1289=457%:%
%:%1290=457%:%
%:%1291=458%:%
%:%1292=458%:%
%:%1293=458%:%
%:%1294=459%:%
%:%1295=459%:%
%:%1296=460%:%
%:%1297=460%:%
%:%1298=461%:%
%:%1299=461%:%
%:%1300=462%:%
%:%1301=462%:%
%:%1302=463%:%
%:%1303=463%:%
%:%1304=463%:%
%:%1305=464%:%
%:%1306=464%:%
%:%1307=464%:%
%:%1308=465%:%
%:%1309=465%:%
%:%1310=466%:%
%:%1311=466%:%
%:%1312=466%:%
%:%1313=467%:%
%:%1314=467%:%
%:%1315=467%:%
%:%1316=468%:%
%:%1317=468%:%
%:%1318=468%:%
%:%1319=469%:%
%:%1320=469%:%
%:%1321=469%:%
%:%1322=470%:%
%:%1323=470%:%
%:%1324=470%:%
%:%1325=471%:%
%:%1326=471%:%
%:%1327=471%:%
%:%1328=472%:%
%:%1329=472%:%
%:%1330=472%:%
%:%1331=473%:%
%:%1332=473%:%
%:%1333=474%:%
%:%1334=474%:%
%:%1335=474%:%
%:%1336=475%:%
%:%1337=475%:%
%:%1338=475%:%
%:%1339=476%:%
%:%1340=476%:%
%:%1341=477%:%
%:%1356=479%:%
%:%1366=481%:%
%:%1367=481%:%
%:%1368=482%:%
%:%1369=483%:%
%:%1372=484%:%
%:%1376=484%:%
%:%1377=484%:%
%:%1378=484%:%
%:%1383=484%:%
%:%1386=485%:%
%:%1387=486%:%
%:%1388=486%:%
%:%1389=487%:%
%:%1396=488%:%
%:%1397=488%:%
%:%1398=489%:%
%:%1399=489%:%
%:%1400=490%:%
%:%1401=490%:%
%:%1402=491%:%
%:%1403=491%:%
%:%1404=492%:%
%:%1405=492%:%
%:%1406=493%:%
%:%1407=493%:%
%:%1408=494%:%
%:%1409=495%:%
%:%1410=495%:%
%:%1411=496%:%
%:%1412=496%:%
%:%1413=496%:%
%:%1414=497%:%
%:%1415=497%:%
%:%1416=498%:%
%:%1417=498%:%
%:%1418=498%:%
%:%1419=499%:%
%:%1420=500%:%
%:%1421=500%:%
%:%1422=500%:%
%:%1423=501%:%
%:%1424=501%:%
%:%1425=501%:%
%:%1426=501%:%
%:%1427=502%:%
%:%1437=504%:%
%:%1439=505%:%
%:%1440=505%:%
%:%1441=506%:%
%:%1444=507%:%
%:%1448=507%:%
%:%1449=507%:%
%:%1450=508%:%
%:%1451=508%:%
%:%1452=509%:%
%:%1453=509%:%
%:%1454=510%:%
%:%1455=510%:%
%:%1456=511%:%
%:%1457=511%:%
%:%1458=512%:%
%:%1459=512%:%
%:%1460=513%:%
%:%1461=513%:%
%:%1462=514%:%
%:%1463=514%:%
%:%1464=515%:%
%:%1465=515%:%
%:%1466=516%:%
%:%1467=516%:%
%:%1468=517%:%
%:%1469=517%:%
%:%1470=517%:%
%:%1471=518%:%
%:%1472=519%:%
%:%1473=519%:%
%:%1474=520%:%
%:%1475=520%:%
%:%1476=521%:%
%:%1477=521%:%
%:%1478=521%:%
%:%1479=522%:%
%:%1480=522%:%
%:%1481=523%:%
%:%1482=523%:%
%:%1483=524%:%
%:%1484=524%:%
%:%1485=525%:%
%:%1486=525%:%
%:%1487=525%:%
%:%1488=526%:%
%:%1489=526%:%
%:%1490=527%:%
%:%1496=527%:%
%:%1499=528%:%
%:%1500=529%:%
%:%1501=529%:%
%:%1502=530%:%
%:%1505=531%:%
%:%1509=531%:%
%:%1510=531%:%
%:%1511=532%:%
%:%1512=532%:%
%:%1513=533%:%
%:%1514=533%:%
%:%1515=534%:%
%:%1516=534%:%
%:%1517=535%:%
%:%1518=535%:%
%:%1519=536%:%
%:%1520=536%:%
%:%1521=537%:%
%:%1522=537%:%
%:%1523=538%:%
%:%1524=538%:%
%:%1525=539%:%
%:%1526=539%:%
%:%1527=540%:%
%:%1528=540%:%
%:%1529=541%:%
%:%1530=541%:%
%:%1531=541%:%
%:%1532=542%:%
%:%1533=543%:%
%:%1534=543%:%
%:%1535=544%:%
%:%1536=544%:%
%:%1537=545%:%
%:%1538=545%:%
%:%1539=545%:%
%:%1540=546%:%
%:%1541=546%:%
%:%1542=547%:%
%:%1543=547%:%
%:%1544=548%:%
%:%1545=548%:%
%:%1546=549%:%
%:%1547=549%:%
%:%1548=549%:%
%:%1549=550%:%
%:%1550=550%:%
%:%1551=551%:%
%:%1557=551%:%
%:%1560=552%:%
%:%1561=553%:%
%:%1562=553%:%
%:%1563=554%:%
%:%1566=555%:%
%:%1570=555%:%
%:%1571=555%:%
%:%1576=555%:%
%:%1579=556%:%
%:%1580=557%:%
%:%1581=557%:%
%:%1582=558%:%
%:%1585=559%:%
%:%1589=559%:%
%:%1590=559%:%
%:%1599=561%:%
%:%1601=562%:%
%:%1602=562%:%
%:%1603=563%:%
%:%1606=564%:%
%:%1610=564%:%
%:%1611=564%:%
%:%1616=564%:%
%:%1619=565%:%
%:%1620=566%:%
%:%1621=566%:%
%:%1622=567%:%
%:%1625=568%:%
%:%1629=568%:%
%:%1630=568%:%
%:%1631=568%:%
%:%1636=568%:%
%:%1639=569%:%
%:%1640=570%:%
%:%1641=570%:%
%:%1642=571%:%
%:%1645=572%:%
%:%1649=572%:%
%:%1650=572%:%
%:%1655=572%:%
%:%1658=573%:%
%:%1659=574%:%
%:%1660=574%:%
%:%1661=575%:%
%:%1664=576%:%
%:%1668=576%:%
%:%1669=576%:%
%:%1670=576%:%
%:%1675=576%:%
%:%1678=577%:%
%:%1679=578%:%
%:%1680=578%:%
%:%1681=579%:%
%:%1682=580%:%
%:%1685=581%:%
%:%1689=581%:%
%:%1690=581%:%
%:%1691=581%:%
%:%1692=582%:%
%:%1701=584%:%
%:%1703=585%:%
%:%1704=585%:%
%:%1705=586%:%
%:%1706=587%:%
%:%1708=589%:%
%:%1711=590%:%
%:%1715=590%:%
%:%1716=590%:%
%:%1725=592%:%
%:%1727=593%:%
%:%1728=593%:%
%:%1729=594%:%
%:%1730=595%:%
%:%1733=596%:%
%:%1737=596%:%
%:%1738=596%:%
%:%1747=598%:%
%:%1749=599%:%
%:%1750=599%:%
%:%1751=600%:%
%:%1758=601%:%
%:%1759=601%:%
%:%1760=602%:%
%:%1761=602%:%
%:%1762=603%:%
%:%1763=603%:%
%:%1764=603%:%
%:%1765=604%:%
%:%1766=604%:%
%:%1767=604%:%
%:%1768=605%:%
%:%1769=605%:%
%:%1770=605%:%
%:%1771=606%:%
%:%1772=606%:%
%:%1773=606%:%
%:%1774=607%:%
%:%1775=607%:%
%:%1776=607%:%
%:%1777=608%:%
%:%1778=608%:%
%:%1779=609%:%
%:%1780=609%:%
%:%1781=610%:%
%:%1782=610%:%
%:%1783=611%:%
%:%1784=611%:%
%:%1785=612%:%
%:%1786=612%:%
%:%1787=613%:%
%:%1788=613%:%
%:%1789=614%:%
%:%1790=614%:%
%:%1791=614%:%
%:%1792=615%:%
%:%1793=615%:%
%:%1794=615%:%
%:%1795=616%:%
%:%1796=616%:%
%:%1797=616%:%
%:%1798=617%:%
%:%1799=617%:%
%:%1800=618%:%
%:%1801=618%:%
%:%1802=619%:%
%:%1812=621%:%
%:%1814=622%:%
%:%1815=622%:%
%:%1816=623%:%
%:%1823=624%:%
%:%1824=624%:%
%:%1825=625%:%
%:%1826=625%:%
%:%1827=626%:%
%:%1828=626%:%
%:%1829=626%:%
%:%1830=627%:%
%:%1831=627%:%
%:%1832=627%:%
%:%1833=628%:%
%:%1834=628%:%
%:%1835=628%:%
%:%1836=629%:%
%:%1837=629%:%
%:%1838=629%:%
%:%1839=630%:%
%:%1840=630%:%
%:%1841=630%:%
%:%1842=631%:%
%:%1843=631%:%
%:%1844=632%:%
%:%1845=632%:%
%:%1846=633%:%
%:%1847=633%:%
%:%1848=634%:%
%:%1849=634%:%
%:%1850=635%:%
%:%1851=635%:%
%:%1852=636%:%
%:%1853=636%:%
%:%1854=637%:%
%:%1855=637%:%
%:%1856=637%:%
%:%1857=638%:%
%:%1858=638%:%
%:%1859=638%:%
%:%1860=639%:%
%:%1861=639%:%
%:%1862=639%:%
%:%1863=640%:%
%:%1864=640%:%
%:%1865=641%:%
%:%1866=641%:%
%:%1867=642%:%
%:%1873=642%:%
%:%1876=643%:%
%:%1877=644%:%
%:%1878=644%:%
%:%1879=645%:%
%:%1880=646%:%
%:%1887=647%:%
%:%1888=647%:%
%:%1889=648%:%
%:%1890=648%:%
%:%1891=649%:%
%:%1892=649%:%
%:%1893=649%:%
%:%1894=650%:%
%:%1895=650%:%
%:%1896=650%:%
%:%1897=651%:%
%:%1898=651%:%
%:%1899=651%:%
%:%1900=652%:%
%:%1901=652%:%
%:%1902=652%:%
%:%1903=653%:%
%:%1904=653%:%
%:%1905=653%:%
%:%1906=654%:%
%:%1912=654%:%
%:%1915=655%:%
%:%1916=656%:%
%:%1917=656%:%
%:%1918=657%:%
%:%1919=658%:%
%:%1926=659%:%
%:%1927=659%:%
%:%1928=660%:%
%:%1929=660%:%
%:%1930=661%:%
%:%1931=661%:%
%:%1932=661%:%
%:%1933=662%:%
%:%1934=662%:%
%:%1935=662%:%
%:%1936=663%:%
%:%1937=663%:%
%:%1938=663%:%
%:%1939=664%:%
%:%1940=664%:%
%:%1941=664%:%
%:%1942=665%:%
%:%1943=665%:%
%:%1944=665%:%
%:%1945=666%:%
%:%1960=668%:%
%:%1972=670%:%
%:%1974=671%:%
%:%1975=671%:%
%:%1976=672%:%
%:%1979=675%:%
%:%1980=676%:%
%:%1981=677%:%
%:%1982=677%:%
%:%1983=678%:%
%:%1984=679%:%
%:%1985=680%:%
%:%1986=681%:%
%:%1989=682%:%
%:%1993=682%:%
%:%1994=682%:%
%:%1995=682%:%
%:%1996=682%:%
%:%2001=682%:%
%:%2004=683%:%
%:%2005=684%:%
%:%2006=684%:%
%:%2007=685%:%
%:%2008=686%:%
%:%2009=687%:%
%:%2010=688%:%
%:%2017=689%:%
%:%2018=689%:%
%:%2019=690%:%
%:%2020=690%:%
%:%2021=691%:%
%:%2022=691%:%
%:%2023=692%:%
%:%2024=692%:%
%:%2025=692%:%
%:%2026=693%:%
%:%2027=693%:%
%:%2028=694%:%
%:%2029=694%:%
%:%2030=694%:%
%:%2031=695%:%
%:%2032=695%:%
%:%2033=696%:%
%:%2034=696%:%
%:%2035=697%:%
%:%2036=697%:%
%:%2037=698%:%
%:%2038=698%:%
%:%2039=698%:%
%:%2040=699%:%
%:%2041=699%:%
%:%2042=700%:%
%:%2043=700%:%
%:%2044=700%:%
%:%2045=701%:%
%:%2046=701%:%
%:%2050=705%:%
%:%2051=706%:%
%:%2052=706%:%
%:%2053=707%:%
%:%2054=707%:%
%:%2055=708%:%
%:%2056=708%:%
%:%2057=709%:%
%:%2058=709%:%
%:%2059=710%:%
%:%2060=710%:%
%:%2061=711%:%
%:%2062=711%:%
%:%2063=711%:%
%:%2064=712%:%
%:%2065=712%:%
%:%2066=712%:%
%:%2067=713%:%
%:%2068=714%:%
%:%2069=714%:%
%:%2070=715%:%
%:%2071=715%:%
%:%2072=715%:%
%:%2073=716%:%
%:%2074=717%:%
%:%2075=717%:%
%:%2076=718%:%
%:%2077=718%:%
%:%2078=719%:%
%:%2079=719%:%
%:%2080=720%:%
%:%2081=720%:%
%:%2082=721%:%
%:%2083=721%:%
%:%2084=722%:%
%:%2085=722%:%
%:%2086=722%:%
%:%2087=723%:%
%:%2088=723%:%
%:%2089=724%:%
%:%2090=724%:%
%:%2091=725%:%
%:%2092=725%:%
%:%2093=726%:%
%:%2094=726%:%
%:%2095=727%:%
%:%2096=727%:%
%:%2097=727%:%
%:%2098=728%:%
%:%2099=728%:%
%:%2100=729%:%
%:%2101=729%:%
%:%2102=729%:%
%:%2103=730%:%
%:%2104=730%:%
%:%2107=733%:%
%:%2108=734%:%
%:%2109=734%:%
%:%2110=734%:%
%:%2111=735%:%
%:%2112=735%:%
%:%2113=736%:%
%:%2119=736%:%
%:%2124=737%:%
%:%2129=738%:%

%
\begin{isabellebody}%
\setisabellecontext{Equalizer}%
%
\isadelimdocument
%
\endisadelimdocument
%
\isatagdocument
%
\isamarkupsection{Equalizers and Subobjects%
}
\isamarkuptrue%
%
\endisatagdocument
{\isafolddocument}%
%
\isadelimdocument
%
\endisadelimdocument
%
\isadelimtheory
%
\endisadelimtheory
%
\isatagtheory
\isacommand{theory}\isamarkupfalse%
\ Equalizer\isanewline
\ \ \isakeyword{imports}\ Terminal\isanewline
\isakeyword{begin}%
\endisatagtheory
{\isafoldtheory}%
%
\isadelimtheory
%
\endisadelimtheory
%
\isadelimdocument
%
\endisadelimdocument
%
\isatagdocument
%
\isamarkupsubsection{Equalizers%
}
\isamarkuptrue%
%
\endisatagdocument
{\isafolddocument}%
%
\isadelimdocument
%
\endisadelimdocument
\isacommand{definition}\isamarkupfalse%
\ equalizer\ {\isacharcolon}{\kern0pt}{\isacharcolon}{\kern0pt}\ {\isachardoublequoteopen}cset\ {\isasymRightarrow}\ cfunc\ {\isasymRightarrow}\ cfunc\ {\isasymRightarrow}\ cfunc\ {\isasymRightarrow}\ bool{\isachardoublequoteclose}\ \isakeyword{where}\isanewline
\ \ {\isachardoublequoteopen}equalizer\ E\ m\ f\ g\ {\isasymlongleftrightarrow}\ {\isacharparenleft}{\kern0pt}{\isasymexists}\ X\ Y{\isachardot}{\kern0pt}\ {\isacharparenleft}{\kern0pt}f\ {\isacharcolon}{\kern0pt}\ X\ {\isasymrightarrow}\ Y{\isacharparenright}{\kern0pt}\ {\isasymand}\ {\isacharparenleft}{\kern0pt}g\ {\isacharcolon}{\kern0pt}\ X\ {\isasymrightarrow}\ Y{\isacharparenright}{\kern0pt}\ {\isasymand}\ {\isacharparenleft}{\kern0pt}m\ {\isacharcolon}{\kern0pt}\ E\ {\isasymrightarrow}\ X{\isacharparenright}{\kern0pt}\isanewline
\ \ \ \ {\isasymand}\ {\isacharparenleft}{\kern0pt}f\ {\isasymcirc}\isactrlsub c\ m\ {\isacharequal}{\kern0pt}\ g\ {\isasymcirc}\isactrlsub c\ m{\isacharparenright}{\kern0pt}\isanewline
\ \ \ \ {\isasymand}\ {\isacharparenleft}{\kern0pt}{\isasymforall}\ h\ F{\isachardot}{\kern0pt}\ {\isacharparenleft}{\kern0pt}{\isacharparenleft}{\kern0pt}h\ {\isacharcolon}{\kern0pt}\ F\ {\isasymrightarrow}\ X{\isacharparenright}{\kern0pt}\ {\isasymand}\ {\isacharparenleft}{\kern0pt}f\ {\isasymcirc}\isactrlsub c\ h\ {\isacharequal}{\kern0pt}\ g\ {\isasymcirc}\isactrlsub c\ h{\isacharparenright}{\kern0pt}{\isacharparenright}{\kern0pt}\ {\isasymlongrightarrow}\ {\isacharparenleft}{\kern0pt}{\isasymexists}{\isacharbang}{\kern0pt}\ k{\isachardot}{\kern0pt}\ {\isacharparenleft}{\kern0pt}k\ {\isacharcolon}{\kern0pt}\ F\ {\isasymrightarrow}\ E{\isacharparenright}{\kern0pt}\ {\isasymand}\ m\ {\isasymcirc}\isactrlsub c\ k\ {\isacharequal}{\kern0pt}\ h{\isacharparenright}{\kern0pt}{\isacharparenright}{\kern0pt}{\isacharparenright}{\kern0pt}{\isachardoublequoteclose}\isanewline
\isanewline
\isacommand{lemma}\isamarkupfalse%
\ equalizer{\isacharunderscore}{\kern0pt}def{\isadigit{2}}{\isacharcolon}{\kern0pt}\isanewline
\ \ \isakeyword{assumes}\ {\isachardoublequoteopen}f\ {\isacharcolon}{\kern0pt}\ X\ {\isasymrightarrow}\ Y{\isachardoublequoteclose}\ {\isachardoublequoteopen}g\ {\isacharcolon}{\kern0pt}\ X\ {\isasymrightarrow}\ Y{\isachardoublequoteclose}\ {\isachardoublequoteopen}m\ {\isacharcolon}{\kern0pt}\ E\ {\isasymrightarrow}\ X{\isachardoublequoteclose}\isanewline
\ \ \isakeyword{shows}\ {\isachardoublequoteopen}equalizer\ E\ m\ f\ g\ {\isasymlongleftrightarrow}\ {\isacharparenleft}{\kern0pt}{\isacharparenleft}{\kern0pt}f\ {\isasymcirc}\isactrlsub c\ m\ {\isacharequal}{\kern0pt}\ g\ {\isasymcirc}\isactrlsub c\ m{\isacharparenright}{\kern0pt}\isanewline
\ \ \ \ {\isasymand}\ {\isacharparenleft}{\kern0pt}{\isasymforall}\ h\ F{\isachardot}{\kern0pt}\ {\isacharparenleft}{\kern0pt}{\isacharparenleft}{\kern0pt}h\ {\isacharcolon}{\kern0pt}\ F\ {\isasymrightarrow}\ X{\isacharparenright}{\kern0pt}\ {\isasymand}\ {\isacharparenleft}{\kern0pt}f\ {\isasymcirc}\isactrlsub c\ h\ {\isacharequal}{\kern0pt}\ g\ {\isasymcirc}\isactrlsub c\ h{\isacharparenright}{\kern0pt}{\isacharparenright}{\kern0pt}\ {\isasymlongrightarrow}\ {\isacharparenleft}{\kern0pt}{\isasymexists}{\isacharbang}{\kern0pt}\ k{\isachardot}{\kern0pt}\ {\isacharparenleft}{\kern0pt}k\ {\isacharcolon}{\kern0pt}\ F\ {\isasymrightarrow}\ E{\isacharparenright}{\kern0pt}\ {\isasymand}\ m\ {\isasymcirc}\isactrlsub c\ k\ {\isacharequal}{\kern0pt}\ h{\isacharparenright}{\kern0pt}{\isacharparenright}{\kern0pt}{\isacharparenright}{\kern0pt}{\isachardoublequoteclose}\isanewline
%
\isadelimproof
\ \ %
\endisadelimproof
%
\isatagproof
\isacommand{using}\isamarkupfalse%
\ assms\ \isacommand{unfolding}\isamarkupfalse%
\ equalizer{\isacharunderscore}{\kern0pt}def\ \isacommand{by}\isamarkupfalse%
\ {\isacharparenleft}{\kern0pt}auto\ simp\ add{\isacharcolon}{\kern0pt}\ cfunc{\isacharunderscore}{\kern0pt}type{\isacharunderscore}{\kern0pt}def{\isacharparenright}{\kern0pt}%
\endisatagproof
{\isafoldproof}%
%
\isadelimproof
\isanewline
%
\endisadelimproof
\isanewline
\isacommand{lemma}\isamarkupfalse%
\ equalizer{\isacharunderscore}{\kern0pt}eq{\isacharcolon}{\kern0pt}\isanewline
\ \ \isakeyword{assumes}\ {\isachardoublequoteopen}f\ {\isacharcolon}{\kern0pt}\ X\ {\isasymrightarrow}\ Y{\isachardoublequoteclose}\ {\isachardoublequoteopen}g\ {\isacharcolon}{\kern0pt}\ X\ {\isasymrightarrow}\ Y{\isachardoublequoteclose}\ {\isachardoublequoteopen}m\ {\isacharcolon}{\kern0pt}\ E\ {\isasymrightarrow}\ X{\isachardoublequoteclose}\isanewline
\ \ \isakeyword{assumes}\ {\isachardoublequoteopen}equalizer\ E\ m\ f\ g{\isachardoublequoteclose}\isanewline
\ \ \isakeyword{shows}\ {\isachardoublequoteopen}f\ {\isasymcirc}\isactrlsub c\ m\ {\isacharequal}{\kern0pt}\ g\ {\isasymcirc}\isactrlsub c\ m{\isachardoublequoteclose}\isanewline
%
\isadelimproof
\ \ %
\endisadelimproof
%
\isatagproof
\isacommand{using}\isamarkupfalse%
\ assms\ equalizer{\isacharunderscore}{\kern0pt}def{\isadigit{2}}\ \isacommand{by}\isamarkupfalse%
\ auto%
\endisatagproof
{\isafoldproof}%
%
\isadelimproof
\isanewline
%
\endisadelimproof
\isanewline
\isacommand{lemma}\isamarkupfalse%
\ similar{\isacharunderscore}{\kern0pt}equalizers{\isacharcolon}{\kern0pt}\isanewline
\ \ \isakeyword{assumes}\ {\isachardoublequoteopen}f\ {\isacharcolon}{\kern0pt}\ X\ {\isasymrightarrow}\ Y{\isachardoublequoteclose}\ {\isachardoublequoteopen}g\ {\isacharcolon}{\kern0pt}\ X\ {\isasymrightarrow}\ Y{\isachardoublequoteclose}\ {\isachardoublequoteopen}m\ {\isacharcolon}{\kern0pt}\ E\ {\isasymrightarrow}\ X{\isachardoublequoteclose}\isanewline
\ \ \isakeyword{assumes}\ {\isachardoublequoteopen}equalizer\ E\ m\ f\ g{\isachardoublequoteclose}\isanewline
\ \ \isakeyword{assumes}\ {\isachardoublequoteopen}h\ {\isacharcolon}{\kern0pt}\ F\ {\isasymrightarrow}\ X{\isachardoublequoteclose}\ {\isachardoublequoteopen}f\ {\isasymcirc}\isactrlsub c\ h\ {\isacharequal}{\kern0pt}\ g\ {\isasymcirc}\isactrlsub c\ h{\isachardoublequoteclose}\isanewline
\ \ \isakeyword{shows}\ {\isachardoublequoteopen}{\isasymexists}{\isacharbang}{\kern0pt}\ k{\isachardot}{\kern0pt}\ k\ {\isacharcolon}{\kern0pt}\ F\ {\isasymrightarrow}\ E\ {\isasymand}\ m\ {\isasymcirc}\isactrlsub c\ k\ {\isacharequal}{\kern0pt}\ h{\isachardoublequoteclose}\isanewline
%
\isadelimproof
\ \ %
\endisadelimproof
%
\isatagproof
\isacommand{using}\isamarkupfalse%
\ assms\ equalizer{\isacharunderscore}{\kern0pt}def{\isadigit{2}}\ \isacommand{by}\isamarkupfalse%
\ auto%
\endisatagproof
{\isafoldproof}%
%
\isadelimproof
%
\endisadelimproof
%
\begin{isamarkuptext}%
The definition above and the axiomatization below correspond to Axiom 4 (Equalizers) in Halvorson.%
\end{isamarkuptext}\isamarkuptrue%
\isacommand{axiomatization}\isamarkupfalse%
\ \isakeyword{where}\isanewline
\ \ equalizer{\isacharunderscore}{\kern0pt}exists{\isacharcolon}{\kern0pt}\ {\isachardoublequoteopen}f\ {\isacharcolon}{\kern0pt}\ X\ {\isasymrightarrow}\ Y\ {\isasymLongrightarrow}\ g\ {\isacharcolon}{\kern0pt}\ X\ {\isasymrightarrow}\ Y\ {\isasymLongrightarrow}\ {\isasymexists}\ E\ m{\isachardot}{\kern0pt}\ equalizer\ E\ m\ f\ g{\isachardoublequoteclose}\isanewline
\isanewline
\isacommand{lemma}\isamarkupfalse%
\ equalizer{\isacharunderscore}{\kern0pt}exists{\isadigit{2}}{\isacharcolon}{\kern0pt}\isanewline
\ \ \isakeyword{assumes}\ {\isachardoublequoteopen}f\ {\isacharcolon}{\kern0pt}\ X\ {\isasymrightarrow}\ Y{\isachardoublequoteclose}\ {\isachardoublequoteopen}g\ {\isacharcolon}{\kern0pt}\ X\ {\isasymrightarrow}\ Y{\isachardoublequoteclose}\isanewline
\ \ \isakeyword{shows}\ {\isachardoublequoteopen}{\isasymexists}\ E\ m{\isachardot}{\kern0pt}\ m\ {\isacharcolon}{\kern0pt}\ E\ {\isasymrightarrow}\ X\ {\isasymand}\ f\ {\isasymcirc}\isactrlsub c\ m\ {\isacharequal}{\kern0pt}\ g\ {\isasymcirc}\isactrlsub c\ m\ {\isasymand}\ {\isacharparenleft}{\kern0pt}{\isasymforall}\ h\ F{\isachardot}{\kern0pt}\ {\isacharparenleft}{\kern0pt}{\isacharparenleft}{\kern0pt}h\ {\isacharcolon}{\kern0pt}\ F\ {\isasymrightarrow}\ X{\isacharparenright}{\kern0pt}\ {\isasymand}\ {\isacharparenleft}{\kern0pt}f\ {\isasymcirc}\isactrlsub c\ h\ {\isacharequal}{\kern0pt}\ g\ {\isasymcirc}\isactrlsub c\ h{\isacharparenright}{\kern0pt}{\isacharparenright}{\kern0pt}\ {\isasymlongrightarrow}\ {\isacharparenleft}{\kern0pt}{\isasymexists}{\isacharbang}{\kern0pt}\ k{\isachardot}{\kern0pt}\ {\isacharparenleft}{\kern0pt}k\ {\isacharcolon}{\kern0pt}\ F\ {\isasymrightarrow}\ E{\isacharparenright}{\kern0pt}\ {\isasymand}\ m\ {\isasymcirc}\isactrlsub c\ k\ {\isacharequal}{\kern0pt}\ h{\isacharparenright}{\kern0pt}{\isacharparenright}{\kern0pt}{\isachardoublequoteclose}\isanewline
%
\isadelimproof
%
\endisadelimproof
%
\isatagproof
\isacommand{proof}\isamarkupfalse%
\ {\isacharminus}{\kern0pt}\isanewline
\ \ \isacommand{obtain}\isamarkupfalse%
\ E\ m\ \isakeyword{where}\ {\isachardoublequoteopen}equalizer\ E\ m\ f\ g{\isachardoublequoteclose}\isanewline
\ \ \ \ \isacommand{using}\isamarkupfalse%
\ assms\ equalizer{\isacharunderscore}{\kern0pt}exists\ \isacommand{by}\isamarkupfalse%
\ blast\isanewline
\ \ \isacommand{then}\isamarkupfalse%
\ \isacommand{show}\isamarkupfalse%
\ {\isacharquery}{\kern0pt}thesis\isanewline
\ \ \ \ \isacommand{unfolding}\isamarkupfalse%
\ equalizer{\isacharunderscore}{\kern0pt}def\isanewline
\ \ \isacommand{proof}\isamarkupfalse%
\ {\isacharparenleft}{\kern0pt}rule{\isacharunderscore}{\kern0pt}tac\ x{\isacharequal}{\kern0pt}{\isachardoublequoteopen}E{\isachardoublequoteclose}\ \isakeyword{in}\ exI{\isacharcomma}{\kern0pt}\ rule{\isacharunderscore}{\kern0pt}tac\ x{\isacharequal}{\kern0pt}{\isachardoublequoteopen}m{\isachardoublequoteclose}\ \isakeyword{in}\ exI{\isacharcomma}{\kern0pt}\ safe{\isacharparenright}{\kern0pt}\isanewline
\ \ \ \ \isacommand{fix}\isamarkupfalse%
\ X{\isacharprime}{\kern0pt}\ Y{\isacharprime}{\kern0pt}\isanewline
\ \ \ \ \isacommand{assume}\isamarkupfalse%
\ f{\isacharunderscore}{\kern0pt}type{\isadigit{2}}{\isacharcolon}{\kern0pt}\ {\isachardoublequoteopen}f\ {\isacharcolon}{\kern0pt}\ X{\isacharprime}{\kern0pt}\ {\isasymrightarrow}\ Y{\isacharprime}{\kern0pt}{\isachardoublequoteclose}\isanewline
\ \ \ \ \isacommand{assume}\isamarkupfalse%
\ g{\isacharunderscore}{\kern0pt}type{\isadigit{2}}{\isacharcolon}{\kern0pt}\ {\isachardoublequoteopen}g\ {\isacharcolon}{\kern0pt}\ X{\isacharprime}{\kern0pt}\ {\isasymrightarrow}\ Y{\isacharprime}{\kern0pt}{\isachardoublequoteclose}\isanewline
\ \ \ \ \isacommand{assume}\isamarkupfalse%
\ m{\isacharunderscore}{\kern0pt}type{\isacharcolon}{\kern0pt}\ {\isachardoublequoteopen}m\ {\isacharcolon}{\kern0pt}\ E\ {\isasymrightarrow}\ X{\isacharprime}{\kern0pt}{\isachardoublequoteclose}\isanewline
\ \ \ \ \isacommand{assume}\isamarkupfalse%
\ fm{\isacharunderscore}{\kern0pt}eq{\isacharunderscore}{\kern0pt}gm{\isacharcolon}{\kern0pt}\ {\isachardoublequoteopen}f\ {\isasymcirc}\isactrlsub c\ m\ {\isacharequal}{\kern0pt}\ g\ {\isasymcirc}\isactrlsub c\ m{\isachardoublequoteclose}\isanewline
\ \ \ \ \isacommand{assume}\isamarkupfalse%
\ equalizer{\isacharunderscore}{\kern0pt}unique{\isacharcolon}{\kern0pt}\ {\isachardoublequoteopen}{\isasymforall}h\ F{\isachardot}{\kern0pt}\ h\ {\isacharcolon}{\kern0pt}\ F\ {\isasymrightarrow}\ X{\isacharprime}{\kern0pt}\ {\isasymand}\ f\ {\isasymcirc}\isactrlsub c\ h\ {\isacharequal}{\kern0pt}\ g\ {\isasymcirc}\isactrlsub c\ h\ {\isasymlongrightarrow}\ {\isacharparenleft}{\kern0pt}{\isasymexists}{\isacharbang}{\kern0pt}k{\isachardot}{\kern0pt}\ k\ {\isacharcolon}{\kern0pt}\ F\ {\isasymrightarrow}\ E\ {\isasymand}\ m\ {\isasymcirc}\isactrlsub c\ k\ {\isacharequal}{\kern0pt}\ h{\isacharparenright}{\kern0pt}{\isachardoublequoteclose}\isanewline
\isanewline
\ \ \ \ \isacommand{show}\isamarkupfalse%
\ m{\isacharunderscore}{\kern0pt}type{\isadigit{2}}{\isacharcolon}{\kern0pt}\ {\isachardoublequoteopen}m\ {\isacharcolon}{\kern0pt}\ E\ {\isasymrightarrow}\ X{\isachardoublequoteclose}\isanewline
\ \ \ \ \ \ \isacommand{using}\isamarkupfalse%
\ assms{\isacharparenleft}{\kern0pt}{\isadigit{2}}{\isacharparenright}{\kern0pt}\ cfunc{\isacharunderscore}{\kern0pt}type{\isacharunderscore}{\kern0pt}def\ g{\isacharunderscore}{\kern0pt}type{\isadigit{2}}\ m{\isacharunderscore}{\kern0pt}type\ \isacommand{by}\isamarkupfalse%
\ auto\isanewline
\isanewline
\ \ \ \ \isacommand{show}\isamarkupfalse%
\ {\isachardoublequoteopen}{\isasymAnd}\ h\ F{\isachardot}{\kern0pt}\ h\ {\isacharcolon}{\kern0pt}\ F\ {\isasymrightarrow}\ X\ {\isasymLongrightarrow}\ f\ {\isasymcirc}\isactrlsub c\ h\ {\isacharequal}{\kern0pt}\ g\ {\isasymcirc}\isactrlsub c\ h\ {\isasymLongrightarrow}\ {\isasymexists}k{\isachardot}{\kern0pt}\ k\ {\isacharcolon}{\kern0pt}\ F\ {\isasymrightarrow}\ E\ {\isasymand}\ m\ {\isasymcirc}\isactrlsub c\ k\ {\isacharequal}{\kern0pt}\ h{\isachardoublequoteclose}\isanewline
\ \ \ \ \ \ \isacommand{by}\isamarkupfalse%
\ {\isacharparenleft}{\kern0pt}metis\ m{\isacharunderscore}{\kern0pt}type{\isadigit{2}}\ cfunc{\isacharunderscore}{\kern0pt}type{\isacharunderscore}{\kern0pt}def\ equalizer{\isacharunderscore}{\kern0pt}unique\ m{\isacharunderscore}{\kern0pt}type{\isacharparenright}{\kern0pt}\isanewline
\isanewline
\ \ \ \ \isacommand{show}\isamarkupfalse%
\ {\isachardoublequoteopen}{\isasymAnd}\ F\ k\ y{\isachardot}{\kern0pt}\ m\ {\isasymcirc}\isactrlsub c\ k\ {\isacharcolon}{\kern0pt}\ F\ {\isasymrightarrow}\ X\ {\isasymLongrightarrow}\ f\ {\isasymcirc}\isactrlsub c\ m\ {\isasymcirc}\isactrlsub c\ k\ {\isacharequal}{\kern0pt}\ g\ {\isasymcirc}\isactrlsub c\ m\ {\isasymcirc}\isactrlsub c\ k\ {\isasymLongrightarrow}\ k\ {\isacharcolon}{\kern0pt}\ F\ {\isasymrightarrow}\ E\ {\isasymLongrightarrow}\ y\ {\isacharcolon}{\kern0pt}\ F\ {\isasymrightarrow}\ E\isanewline
\ \ \ \ \ \ \ \ {\isasymLongrightarrow}\ m\ {\isasymcirc}\isactrlsub c\ y\ {\isacharequal}{\kern0pt}\ m\ {\isasymcirc}\isactrlsub c\ k\ {\isasymLongrightarrow}\ k\ {\isacharequal}{\kern0pt}\ y{\isachardoublequoteclose}\isanewline
\ \ \ \ \ \ \isacommand{using}\isamarkupfalse%
\ comp{\isacharunderscore}{\kern0pt}type\ equalizer{\isacharunderscore}{\kern0pt}unique\ m{\isacharunderscore}{\kern0pt}type\ \isacommand{by}\isamarkupfalse%
\ blast\isanewline
\ \ \isacommand{qed}\isamarkupfalse%
\isanewline
\isacommand{qed}\isamarkupfalse%
%
\endisatagproof
{\isafoldproof}%
%
\isadelimproof
%
\endisadelimproof
%
\begin{isamarkuptext}%
The lemma below corresponds to Exercise 2.1.31 in Halvorson.%
\end{isamarkuptext}\isamarkuptrue%
\isacommand{lemma}\isamarkupfalse%
\ equalizers{\isacharunderscore}{\kern0pt}isomorphic{\isacharcolon}{\kern0pt}\isanewline
\ \ \isakeyword{assumes}\ {\isachardoublequoteopen}equalizer\ E\ m\ f\ g{\isachardoublequoteclose}\ {\isachardoublequoteopen}equalizer\ E{\isacharprime}{\kern0pt}\ m{\isacharprime}{\kern0pt}\ f\ g{\isachardoublequoteclose}\isanewline
\ \ \isakeyword{shows}\ {\isachardoublequoteopen}{\isasymexists}\ k{\isachardot}{\kern0pt}\ k\ {\isacharcolon}{\kern0pt}\ E\ {\isasymrightarrow}\ E{\isacharprime}{\kern0pt}\ {\isasymand}\ isomorphism\ k\ {\isasymand}\ m\ {\isacharequal}{\kern0pt}\ m{\isacharprime}{\kern0pt}\ {\isasymcirc}\isactrlsub c\ k{\isachardoublequoteclose}\isanewline
%
\isadelimproof
%
\endisadelimproof
%
\isatagproof
\isacommand{proof}\isamarkupfalse%
\ {\isacharminus}{\kern0pt}\isanewline
\ \ \isacommand{have}\isamarkupfalse%
\ fm{\isacharunderscore}{\kern0pt}eq{\isacharunderscore}{\kern0pt}gm{\isacharcolon}{\kern0pt}\ {\isachardoublequoteopen}f\ {\isasymcirc}\isactrlsub c\ m\ {\isacharequal}{\kern0pt}\ g\ {\isasymcirc}\isactrlsub c\ m{\isachardoublequoteclose}\isanewline
\ \ \ \ \isacommand{using}\isamarkupfalse%
\ assms{\isacharparenleft}{\kern0pt}{\isadigit{1}}{\isacharparenright}{\kern0pt}\ equalizer{\isacharunderscore}{\kern0pt}def\ \isacommand{by}\isamarkupfalse%
\ blast\isanewline
\ \ \isacommand{have}\isamarkupfalse%
\ fm{\isacharprime}{\kern0pt}{\isacharunderscore}{\kern0pt}eq{\isacharunderscore}{\kern0pt}gm{\isacharprime}{\kern0pt}{\isacharcolon}{\kern0pt}\ {\isachardoublequoteopen}f\ {\isasymcirc}\isactrlsub c\ m{\isacharprime}{\kern0pt}\ {\isacharequal}{\kern0pt}\ g\ {\isasymcirc}\isactrlsub c\ m{\isacharprime}{\kern0pt}{\isachardoublequoteclose}\isanewline
\ \ \ \ \isacommand{using}\isamarkupfalse%
\ assms{\isacharparenleft}{\kern0pt}{\isadigit{2}}{\isacharparenright}{\kern0pt}\ equalizer{\isacharunderscore}{\kern0pt}def\ \isacommand{by}\isamarkupfalse%
\ blast\isanewline
\isanewline
\ \ \isacommand{obtain}\isamarkupfalse%
\ X\ Y\ \isakeyword{where}\ f{\isacharunderscore}{\kern0pt}type{\isacharcolon}{\kern0pt}\ {\isachardoublequoteopen}f\ {\isacharcolon}{\kern0pt}\ X\ {\isasymrightarrow}\ Y{\isachardoublequoteclose}\ \isakeyword{and}\ g{\isacharunderscore}{\kern0pt}type{\isacharcolon}{\kern0pt}\ {\isachardoublequoteopen}g\ {\isacharcolon}{\kern0pt}\ X\ {\isasymrightarrow}\ Y{\isachardoublequoteclose}\ \isakeyword{and}\ m{\isacharunderscore}{\kern0pt}type{\isacharcolon}{\kern0pt}\ {\isachardoublequoteopen}m\ {\isacharcolon}{\kern0pt}\ E\ {\isasymrightarrow}\ X{\isachardoublequoteclose}\isanewline
\ \ \ \ \isacommand{using}\isamarkupfalse%
\ assms{\isacharparenleft}{\kern0pt}{\isadigit{1}}{\isacharparenright}{\kern0pt}\ \isacommand{unfolding}\isamarkupfalse%
\ equalizer{\isacharunderscore}{\kern0pt}def\ \isacommand{by}\isamarkupfalse%
\ auto\isanewline
\isanewline
\ \ \isacommand{obtain}\isamarkupfalse%
\ k\ \isakeyword{where}\ k{\isacharunderscore}{\kern0pt}type{\isacharcolon}{\kern0pt}\ {\isachardoublequoteopen}k\ {\isacharcolon}{\kern0pt}\ E{\isacharprime}{\kern0pt}\ {\isasymrightarrow}\ E{\isachardoublequoteclose}\ \isakeyword{and}\ mk{\isacharunderscore}{\kern0pt}eq{\isacharunderscore}{\kern0pt}m{\isacharprime}{\kern0pt}{\isacharcolon}{\kern0pt}\ {\isachardoublequoteopen}m\ {\isasymcirc}\isactrlsub c\ k\ {\isacharequal}{\kern0pt}\ m{\isacharprime}{\kern0pt}{\isachardoublequoteclose}\isanewline
\ \ \ \ \isacommand{by}\isamarkupfalse%
\ {\isacharparenleft}{\kern0pt}metis\ assms\ cfunc{\isacharunderscore}{\kern0pt}type{\isacharunderscore}{\kern0pt}def\ equalizer{\isacharunderscore}{\kern0pt}def{\isacharparenright}{\kern0pt}\isanewline
\ \ \isacommand{obtain}\isamarkupfalse%
\ k{\isacharprime}{\kern0pt}\ \isakeyword{where}\ k{\isacharprime}{\kern0pt}{\isacharunderscore}{\kern0pt}type{\isacharcolon}{\kern0pt}\ {\isachardoublequoteopen}k{\isacharprime}{\kern0pt}\ {\isacharcolon}{\kern0pt}\ E\ {\isasymrightarrow}\ E{\isacharprime}{\kern0pt}{\isachardoublequoteclose}\ \isakeyword{and}\ m{\isacharprime}{\kern0pt}k{\isacharunderscore}{\kern0pt}eq{\isacharunderscore}{\kern0pt}m{\isacharcolon}{\kern0pt}\ {\isachardoublequoteopen}m{\isacharprime}{\kern0pt}\ {\isasymcirc}\isactrlsub c\ k{\isacharprime}{\kern0pt}\ {\isacharequal}{\kern0pt}\ m{\isachardoublequoteclose}\isanewline
\ \ \ \ \isacommand{by}\isamarkupfalse%
\ {\isacharparenleft}{\kern0pt}metis\ assms\ cfunc{\isacharunderscore}{\kern0pt}type{\isacharunderscore}{\kern0pt}def\ equalizer{\isacharunderscore}{\kern0pt}def{\isacharparenright}{\kern0pt}\isanewline
\isanewline
\ \ \isacommand{have}\isamarkupfalse%
\ {\isachardoublequoteopen}f\ {\isasymcirc}\isactrlsub c\ m\ {\isasymcirc}\isactrlsub c\ k\ {\isasymcirc}\isactrlsub c\ k{\isacharprime}{\kern0pt}\ {\isacharequal}{\kern0pt}\ g\ {\isasymcirc}\isactrlsub c\ m\ {\isasymcirc}\isactrlsub c\ k\ {\isasymcirc}\isactrlsub c\ k{\isacharprime}{\kern0pt}{\isachardoublequoteclose}\isanewline
\ \ \ \ \isacommand{using}\isamarkupfalse%
\ comp{\isacharunderscore}{\kern0pt}associative{\isadigit{2}}\ m{\isacharunderscore}{\kern0pt}type\ fm{\isacharunderscore}{\kern0pt}eq{\isacharunderscore}{\kern0pt}gm\ k{\isacharprime}{\kern0pt}{\isacharunderscore}{\kern0pt}type\ k{\isacharunderscore}{\kern0pt}type\ m{\isacharprime}{\kern0pt}k{\isacharunderscore}{\kern0pt}eq{\isacharunderscore}{\kern0pt}m\ mk{\isacharunderscore}{\kern0pt}eq{\isacharunderscore}{\kern0pt}m{\isacharprime}{\kern0pt}\ \isacommand{by}\isamarkupfalse%
\ auto\isanewline
\isanewline
\ \ \isacommand{have}\isamarkupfalse%
\ {\isachardoublequoteopen}k\ {\isasymcirc}\isactrlsub c\ k{\isacharprime}{\kern0pt}\ {\isacharcolon}{\kern0pt}\ E\ {\isasymrightarrow}\ E\ {\isasymand}\ m\ {\isasymcirc}\isactrlsub c\ k\ {\isasymcirc}\isactrlsub c\ k{\isacharprime}{\kern0pt}\ {\isacharequal}{\kern0pt}\ m{\isachardoublequoteclose}\isanewline
\ \ \ \ \isacommand{using}\isamarkupfalse%
\ comp{\isacharunderscore}{\kern0pt}associative{\isadigit{2}}\ comp{\isacharunderscore}{\kern0pt}type\ k{\isacharprime}{\kern0pt}{\isacharunderscore}{\kern0pt}type\ k{\isacharunderscore}{\kern0pt}type\ m{\isacharunderscore}{\kern0pt}type\ m{\isacharprime}{\kern0pt}k{\isacharunderscore}{\kern0pt}eq{\isacharunderscore}{\kern0pt}m\ mk{\isacharunderscore}{\kern0pt}eq{\isacharunderscore}{\kern0pt}m{\isacharprime}{\kern0pt}\ \isacommand{by}\isamarkupfalse%
\ auto\isanewline
\ \ \isacommand{then}\isamarkupfalse%
\ \isacommand{have}\isamarkupfalse%
\ kk{\isacharprime}{\kern0pt}{\isacharunderscore}{\kern0pt}eq{\isacharunderscore}{\kern0pt}id{\isacharcolon}{\kern0pt}\ {\isachardoublequoteopen}k\ {\isasymcirc}\isactrlsub c\ k{\isacharprime}{\kern0pt}\ {\isacharequal}{\kern0pt}\ id\ E{\isachardoublequoteclose}\isanewline
\ \ \ \ \isacommand{using}\isamarkupfalse%
\ assms{\isacharparenleft}{\kern0pt}{\isadigit{1}}{\isacharparenright}{\kern0pt}\ equalizer{\isacharunderscore}{\kern0pt}def\ id{\isacharunderscore}{\kern0pt}right{\isacharunderscore}{\kern0pt}unit{\isadigit{2}}\ id{\isacharunderscore}{\kern0pt}type\ \isacommand{by}\isamarkupfalse%
\ blast\isanewline
\isanewline
\ \ \isacommand{have}\isamarkupfalse%
\ {\isachardoublequoteopen}k{\isacharprime}{\kern0pt}\ {\isasymcirc}\isactrlsub c\ k\ {\isacharcolon}{\kern0pt}\ E{\isacharprime}{\kern0pt}\ {\isasymrightarrow}\ E{\isacharprime}{\kern0pt}\ {\isasymand}\ m{\isacharprime}{\kern0pt}\ {\isasymcirc}\isactrlsub c\ k{\isacharprime}{\kern0pt}\ {\isasymcirc}\isactrlsub c\ k\ {\isacharequal}{\kern0pt}\ m{\isacharprime}{\kern0pt}{\isachardoublequoteclose}\isanewline
\ \ \ \ \isacommand{by}\isamarkupfalse%
\ {\isacharparenleft}{\kern0pt}smt\ comp{\isacharunderscore}{\kern0pt}associative{\isadigit{2}}\ comp{\isacharunderscore}{\kern0pt}type\ k{\isacharprime}{\kern0pt}{\isacharunderscore}{\kern0pt}type\ k{\isacharunderscore}{\kern0pt}type\ m{\isacharprime}{\kern0pt}k{\isacharunderscore}{\kern0pt}eq{\isacharunderscore}{\kern0pt}m\ m{\isacharunderscore}{\kern0pt}type\ mk{\isacharunderscore}{\kern0pt}eq{\isacharunderscore}{\kern0pt}m{\isacharprime}{\kern0pt}{\isacharparenright}{\kern0pt}\isanewline
\ \ \isacommand{then}\isamarkupfalse%
\ \isacommand{have}\isamarkupfalse%
\ k{\isacharprime}{\kern0pt}k{\isacharunderscore}{\kern0pt}eq{\isacharunderscore}{\kern0pt}id{\isacharcolon}{\kern0pt}\ {\isachardoublequoteopen}k{\isacharprime}{\kern0pt}\ {\isasymcirc}\isactrlsub c\ k\ {\isacharequal}{\kern0pt}\ id\ E{\isacharprime}{\kern0pt}{\isachardoublequoteclose}\isanewline
\ \ \ \ \isacommand{using}\isamarkupfalse%
\ assms{\isacharparenleft}{\kern0pt}{\isadigit{2}}{\isacharparenright}{\kern0pt}\ equalizer{\isacharunderscore}{\kern0pt}def\ id{\isacharunderscore}{\kern0pt}right{\isacharunderscore}{\kern0pt}unit{\isadigit{2}}\ id{\isacharunderscore}{\kern0pt}type\ \isacommand{by}\isamarkupfalse%
\ blast\isanewline
\isanewline
\ \ \isacommand{show}\isamarkupfalse%
\ {\isachardoublequoteopen}{\isasymexists}k{\isachardot}{\kern0pt}\ k\ {\isacharcolon}{\kern0pt}\ E\ {\isasymrightarrow}\ E{\isacharprime}{\kern0pt}\ {\isasymand}\ isomorphism\ k\ {\isasymand}\ m\ {\isacharequal}{\kern0pt}\ m{\isacharprime}{\kern0pt}\ {\isasymcirc}\isactrlsub c\ k{\isachardoublequoteclose}\isanewline
\ \ \ \ \isacommand{using}\isamarkupfalse%
\ cfunc{\isacharunderscore}{\kern0pt}type{\isacharunderscore}{\kern0pt}def\ isomorphism{\isacharunderscore}{\kern0pt}def\ k{\isacharprime}{\kern0pt}{\isacharunderscore}{\kern0pt}type\ k{\isacharprime}{\kern0pt}k{\isacharunderscore}{\kern0pt}eq{\isacharunderscore}{\kern0pt}id\ k{\isacharunderscore}{\kern0pt}type\ kk{\isacharprime}{\kern0pt}{\isacharunderscore}{\kern0pt}eq{\isacharunderscore}{\kern0pt}id\ m{\isacharprime}{\kern0pt}k{\isacharunderscore}{\kern0pt}eq{\isacharunderscore}{\kern0pt}m\ \isacommand{by}\isamarkupfalse%
\ {\isacharparenleft}{\kern0pt}rule{\isacharunderscore}{\kern0pt}tac\ x{\isacharequal}{\kern0pt}{\isachardoublequoteopen}k{\isacharprime}{\kern0pt}{\isachardoublequoteclose}\ \isakeyword{in}\ exI{\isacharcomma}{\kern0pt}\ auto{\isacharparenright}{\kern0pt}\isanewline
\isacommand{qed}\isamarkupfalse%
%
\endisatagproof
{\isafoldproof}%
%
\isadelimproof
\isanewline
%
\endisadelimproof
\isanewline
\isacommand{lemma}\isamarkupfalse%
\ isomorphic{\isacharunderscore}{\kern0pt}to{\isacharunderscore}{\kern0pt}equalizer{\isacharunderscore}{\kern0pt}is{\isacharunderscore}{\kern0pt}equalizer{\isacharcolon}{\kern0pt}\isanewline
\ \ \isakeyword{assumes}\ {\isachardoublequoteopen}{\isasymphi}{\isacharcolon}{\kern0pt}\ E{\isacharprime}{\kern0pt}\ {\isasymrightarrow}\ E{\isachardoublequoteclose}\isanewline
\ \ \isakeyword{assumes}\ {\isachardoublequoteopen}isomorphism\ {\isasymphi}{\isachardoublequoteclose}\isanewline
\ \ \isakeyword{assumes}\ {\isachardoublequoteopen}equalizer\ E\ m\ f\ g{\isachardoublequoteclose}\ \isanewline
\ \ \isakeyword{assumes}\ {\isachardoublequoteopen}f\ {\isacharcolon}{\kern0pt}\ X\ {\isasymrightarrow}\ Y{\isachardoublequoteclose}\isanewline
\ \ \isakeyword{assumes}\ {\isachardoublequoteopen}g\ {\isacharcolon}{\kern0pt}\ X\ {\isasymrightarrow}\ Y{\isachardoublequoteclose}\isanewline
\ \ \isakeyword{assumes}\ {\isachardoublequoteopen}m\ {\isacharcolon}{\kern0pt}\ E\ {\isasymrightarrow}\ X{\isachardoublequoteclose}\isanewline
\ \ \isakeyword{shows}\ \ \ {\isachardoublequoteopen}equalizer\ E{\isacharprime}{\kern0pt}\ {\isacharparenleft}{\kern0pt}m\ {\isasymcirc}\isactrlsub c\ {\isasymphi}{\isacharparenright}{\kern0pt}\ f\ g{\isachardoublequoteclose}\isanewline
%
\isadelimproof
%
\endisadelimproof
%
\isatagproof
\isacommand{proof}\isamarkupfalse%
\ {\isacharminus}{\kern0pt}\ \isanewline
\ \ \isacommand{obtain}\isamarkupfalse%
\ {\isasymphi}{\isacharunderscore}{\kern0pt}inv\ \isakeyword{where}\ {\isasymphi}{\isacharunderscore}{\kern0pt}inv{\isacharunderscore}{\kern0pt}type{\isacharbrackleft}{\kern0pt}type{\isacharunderscore}{\kern0pt}rule{\isacharbrackright}{\kern0pt}{\isacharcolon}{\kern0pt}\ {\isachardoublequoteopen}{\isasymphi}{\isacharunderscore}{\kern0pt}inv\ {\isacharcolon}{\kern0pt}\ E\ {\isasymrightarrow}\ E{\isacharprime}{\kern0pt}{\isachardoublequoteclose}\ \isakeyword{and}\ {\isasymphi}{\isacharunderscore}{\kern0pt}inv{\isacharunderscore}{\kern0pt}{\isasymphi}{\isacharcolon}{\kern0pt}\ {\isachardoublequoteopen}{\isasymphi}{\isacharunderscore}{\kern0pt}inv\ {\isasymcirc}\isactrlsub c\ {\isasymphi}\ {\isacharequal}{\kern0pt}\ id{\isacharparenleft}{\kern0pt}E{\isacharprime}{\kern0pt}{\isacharparenright}{\kern0pt}{\isachardoublequoteclose}\ \isakeyword{and}\ {\isasymphi}{\isasymphi}{\isacharunderscore}{\kern0pt}inv{\isacharcolon}{\kern0pt}\ {\isachardoublequoteopen}{\isasymphi}\ {\isasymcirc}\isactrlsub c\ {\isasymphi}{\isacharunderscore}{\kern0pt}inv\ {\isacharequal}{\kern0pt}\ id{\isacharparenleft}{\kern0pt}E{\isacharparenright}{\kern0pt}{\isachardoublequoteclose}\isanewline
\ \ \ \ \isacommand{using}\isamarkupfalse%
\ assms{\isacharparenleft}{\kern0pt}{\isadigit{1}}{\isacharcomma}{\kern0pt}{\isadigit{2}}{\isacharparenright}{\kern0pt}\ cfunc{\isacharunderscore}{\kern0pt}type{\isacharunderscore}{\kern0pt}def\ isomorphism{\isacharunderscore}{\kern0pt}def\ \isacommand{by}\isamarkupfalse%
\ auto\isanewline
\isanewline
\ \ \isacommand{have}\isamarkupfalse%
\ equalizes{\isacharcolon}{\kern0pt}\ {\isachardoublequoteopen}f\ {\isasymcirc}\isactrlsub c\ m\ {\isasymcirc}\isactrlsub c\ {\isasymphi}\ {\isacharequal}{\kern0pt}\ g\ {\isasymcirc}\isactrlsub c\ m\ {\isasymcirc}\isactrlsub c\ {\isasymphi}{\isachardoublequoteclose}\isanewline
\ \ \ \ \isacommand{using}\isamarkupfalse%
\ assms\ comp{\isacharunderscore}{\kern0pt}associative{\isadigit{2}}\ equalizer{\isacharunderscore}{\kern0pt}def\ \isacommand{by}\isamarkupfalse%
\ force\isanewline
\ \ \isacommand{have}\isamarkupfalse%
\ {\isachardoublequoteopen}{\isasymforall}h\ F{\isachardot}{\kern0pt}\ h\ {\isacharcolon}{\kern0pt}\ F\ {\isasymrightarrow}\ X\ {\isasymand}\ f\ {\isasymcirc}\isactrlsub c\ h\ {\isacharequal}{\kern0pt}\ g\ {\isasymcirc}\isactrlsub c\ h\ {\isasymlongrightarrow}\ {\isacharparenleft}{\kern0pt}{\isasymexists}{\isacharbang}{\kern0pt}k{\isachardot}{\kern0pt}\ k\ {\isacharcolon}{\kern0pt}\ F\ {\isasymrightarrow}\ E{\isacharprime}{\kern0pt}\ {\isasymand}\ {\isacharparenleft}{\kern0pt}m\ {\isasymcirc}\isactrlsub c\ {\isasymphi}{\isacharparenright}{\kern0pt}\ {\isasymcirc}\isactrlsub c\ k\ {\isacharequal}{\kern0pt}\ h{\isacharparenright}{\kern0pt}{\isachardoublequoteclose}\isanewline
\ \ \isacommand{proof}\isamarkupfalse%
{\isacharparenleft}{\kern0pt}safe{\isacharparenright}{\kern0pt}\isanewline
\ \ \ \ \isacommand{fix}\isamarkupfalse%
\ h\ F\isanewline
\ \ \ \ \isacommand{assume}\isamarkupfalse%
\ h{\isacharunderscore}{\kern0pt}type{\isacharbrackleft}{\kern0pt}type{\isacharunderscore}{\kern0pt}rule{\isacharbrackright}{\kern0pt}{\isacharcolon}{\kern0pt}\ {\isachardoublequoteopen}h\ {\isacharcolon}{\kern0pt}\ F\ {\isasymrightarrow}\ X{\isachardoublequoteclose}\isanewline
\ \ \ \ \isacommand{assume}\isamarkupfalse%
\ h{\isacharunderscore}{\kern0pt}equalizes{\isacharcolon}{\kern0pt}\ {\isachardoublequoteopen}f\ {\isasymcirc}\isactrlsub c\ h\ {\isacharequal}{\kern0pt}\ g\ {\isasymcirc}\isactrlsub c\ h{\isachardoublequoteclose}\isanewline
\ \ \ \ \isacommand{have}\isamarkupfalse%
\ k{\isacharunderscore}{\kern0pt}exists{\isacharunderscore}{\kern0pt}uniquely{\isacharcolon}{\kern0pt}\ {\isachardoublequoteopen}{\isasymexists}{\isacharbang}{\kern0pt}\ k{\isachardot}{\kern0pt}\ k{\isacharcolon}{\kern0pt}\ F\ \ {\isasymrightarrow}\ E\ {\isasymand}\ m\ {\isasymcirc}\isactrlsub c\ k\ {\isacharequal}{\kern0pt}\ h{\isachardoublequoteclose}\isanewline
\ \ \ \ \ \ \isacommand{using}\isamarkupfalse%
\ assms\ equalizer{\isacharunderscore}{\kern0pt}def{\isadigit{2}}\ h{\isacharunderscore}{\kern0pt}equalizes\ \isacommand{by}\isamarkupfalse%
\ {\isacharparenleft}{\kern0pt}typecheck{\isacharunderscore}{\kern0pt}cfuncs{\isacharcomma}{\kern0pt}\ auto{\isacharparenright}{\kern0pt}\isanewline
\ \ \ \ \isacommand{then}\isamarkupfalse%
\ \isacommand{obtain}\isamarkupfalse%
\ k\ \isakeyword{where}\ k{\isacharunderscore}{\kern0pt}type{\isacharbrackleft}{\kern0pt}type{\isacharunderscore}{\kern0pt}rule{\isacharbrackright}{\kern0pt}{\isacharcolon}{\kern0pt}\ {\isachardoublequoteopen}k{\isacharcolon}{\kern0pt}\ F\ \ {\isasymrightarrow}\ E{\isachardoublequoteclose}\ \isakeyword{and}\ k{\isacharunderscore}{\kern0pt}def{\isacharcolon}{\kern0pt}\ {\isachardoublequoteopen}m\ {\isasymcirc}\isactrlsub c\ k\ {\isacharequal}{\kern0pt}\ h{\isachardoublequoteclose}\isanewline
\ \ \ \ \ \ \isacommand{by}\isamarkupfalse%
\ blast\isanewline
\ \ \ \ \isacommand{then}\isamarkupfalse%
\ \isacommand{show}\isamarkupfalse%
\ {\isachardoublequoteopen}{\isasymexists}k{\isachardot}{\kern0pt}\ k\ {\isacharcolon}{\kern0pt}\ F\ {\isasymrightarrow}\ E{\isacharprime}{\kern0pt}\ {\isasymand}\ {\isacharparenleft}{\kern0pt}m\ {\isasymcirc}\isactrlsub c\ {\isasymphi}{\isacharparenright}{\kern0pt}\ {\isasymcirc}\isactrlsub c\ k\ {\isacharequal}{\kern0pt}\ h{\isachardoublequoteclose}\isanewline
\ \ \ \ \ \ \isacommand{using}\isamarkupfalse%
\ assms\ \isacommand{by}\isamarkupfalse%
\ {\isacharparenleft}{\kern0pt}typecheck{\isacharunderscore}{\kern0pt}cfuncs{\isacharcomma}{\kern0pt}\ smt\ {\isacharparenleft}{\kern0pt}z{\isadigit{3}}{\isacharparenright}{\kern0pt}\ {\isasymphi}{\isasymphi}{\isacharunderscore}{\kern0pt}inv\ {\isasymphi}{\isacharunderscore}{\kern0pt}inv{\isacharunderscore}{\kern0pt}type\ comp{\isacharunderscore}{\kern0pt}associative{\isadigit{2}}\ comp{\isacharunderscore}{\kern0pt}type\ id{\isacharunderscore}{\kern0pt}right{\isacharunderscore}{\kern0pt}unit{\isadigit{2}}\ k{\isacharunderscore}{\kern0pt}exists{\isacharunderscore}{\kern0pt}uniquely{\isacharparenright}{\kern0pt}\isanewline
\ \ \isacommand{next}\isamarkupfalse%
\isanewline
\ \ \ \ \isacommand{fix}\isamarkupfalse%
\ F\ k\ y\isanewline
\ \ \ \ \isacommand{assume}\isamarkupfalse%
\ {\isachardoublequoteopen}{\isacharparenleft}{\kern0pt}m\ {\isasymcirc}\isactrlsub c\ {\isasymphi}{\isacharparenright}{\kern0pt}\ {\isasymcirc}\isactrlsub c\ k\ {\isacharcolon}{\kern0pt}\ F\ {\isasymrightarrow}\ X{\isachardoublequoteclose}\isanewline
\ \ \ \ \isacommand{assume}\isamarkupfalse%
\ {\isachardoublequoteopen}f\ {\isasymcirc}\isactrlsub c\ {\isacharparenleft}{\kern0pt}m\ {\isasymcirc}\isactrlsub c\ {\isasymphi}{\isacharparenright}{\kern0pt}\ {\isasymcirc}\isactrlsub c\ k\ {\isacharequal}{\kern0pt}\ g\ {\isasymcirc}\isactrlsub c\ {\isacharparenleft}{\kern0pt}m\ {\isasymcirc}\isactrlsub c\ {\isasymphi}{\isacharparenright}{\kern0pt}\ {\isasymcirc}\isactrlsub c\ k{\isachardoublequoteclose}\isanewline
\ \ \ \ \isacommand{assume}\isamarkupfalse%
\ k{\isacharunderscore}{\kern0pt}type{\isacharbrackleft}{\kern0pt}type{\isacharunderscore}{\kern0pt}rule{\isacharbrackright}{\kern0pt}{\isacharcolon}{\kern0pt}\ {\isachardoublequoteopen}k\ {\isacharcolon}{\kern0pt}\ F\ {\isasymrightarrow}\ E{\isacharprime}{\kern0pt}{\isachardoublequoteclose}\isanewline
\ \ \ \ \isacommand{assume}\isamarkupfalse%
\ y{\isacharunderscore}{\kern0pt}type{\isacharbrackleft}{\kern0pt}type{\isacharunderscore}{\kern0pt}rule{\isacharbrackright}{\kern0pt}{\isacharcolon}{\kern0pt}\ {\isachardoublequoteopen}y\ {\isacharcolon}{\kern0pt}\ F\ {\isasymrightarrow}\ E{\isacharprime}{\kern0pt}{\isachardoublequoteclose}\isanewline
\ \ \ \ \isacommand{assume}\isamarkupfalse%
\ {\isachardoublequoteopen}{\isacharparenleft}{\kern0pt}m\ {\isasymcirc}\isactrlsub c\ {\isasymphi}{\isacharparenright}{\kern0pt}\ {\isasymcirc}\isactrlsub c\ y\ {\isacharequal}{\kern0pt}\ {\isacharparenleft}{\kern0pt}m\ {\isasymcirc}\isactrlsub c\ {\isasymphi}{\isacharparenright}{\kern0pt}\ {\isasymcirc}\isactrlsub c\ k{\isachardoublequoteclose}\isanewline
\ \ \ \ \isacommand{then}\isamarkupfalse%
\ \isacommand{show}\isamarkupfalse%
\ {\isachardoublequoteopen}k\ {\isacharequal}{\kern0pt}\ y{\isachardoublequoteclose}\isanewline
\ \ \ \ \ \ \isacommand{by}\isamarkupfalse%
\ {\isacharparenleft}{\kern0pt}typecheck{\isacharunderscore}{\kern0pt}cfuncs{\isacharcomma}{\kern0pt}\ smt\ {\isacharparenleft}{\kern0pt}verit{\isacharcomma}{\kern0pt}\ ccfv{\isacharunderscore}{\kern0pt}threshold{\isacharparenright}{\kern0pt}\ assms{\isacharparenleft}{\kern0pt}{\isadigit{1}}{\isacharcomma}{\kern0pt}{\isadigit{2}}{\isacharcomma}{\kern0pt}{\isadigit{3}}{\isacharparenright}{\kern0pt}\ cfunc{\isacharunderscore}{\kern0pt}type{\isacharunderscore}{\kern0pt}def\ comp{\isacharunderscore}{\kern0pt}associative\ comp{\isacharunderscore}{\kern0pt}type\ equalizer{\isacharunderscore}{\kern0pt}def\ id{\isacharunderscore}{\kern0pt}left{\isacharunderscore}{\kern0pt}unit{\isadigit{2}}\ isomorphism{\isacharunderscore}{\kern0pt}def{\isacharparenright}{\kern0pt}\isanewline
\ \ \isacommand{qed}\isamarkupfalse%
\isanewline
\ \ \isacommand{then}\isamarkupfalse%
\ \isacommand{show}\isamarkupfalse%
\ {\isacharquery}{\kern0pt}thesis\isanewline
\ \ \ \ \isacommand{by}\isamarkupfalse%
\ {\isacharparenleft}{\kern0pt}smt\ {\isacharparenleft}{\kern0pt}verit{\isacharcomma}{\kern0pt}\ best{\isacharparenright}{\kern0pt}\ assms{\isacharparenleft}{\kern0pt}{\isadigit{1}}{\isacharcomma}{\kern0pt}{\isadigit{4}}{\isacharcomma}{\kern0pt}{\isadigit{5}}{\isacharcomma}{\kern0pt}{\isadigit{6}}{\isacharparenright}{\kern0pt}\ comp{\isacharunderscore}{\kern0pt}type\ equalizer{\isacharunderscore}{\kern0pt}def\ equalizes{\isacharparenright}{\kern0pt}\isanewline
\isacommand{qed}\isamarkupfalse%
%
\endisatagproof
{\isafoldproof}%
%
\isadelimproof
%
\endisadelimproof
%
\begin{isamarkuptext}%
The lemma below corresponds to Exercise 2.1.34 in Halvorson.%
\end{isamarkuptext}\isamarkuptrue%
\isacommand{lemma}\isamarkupfalse%
\ equalizer{\isacharunderscore}{\kern0pt}is{\isacharunderscore}{\kern0pt}monomorphism{\isacharcolon}{\kern0pt}\isanewline
\ \ {\isachardoublequoteopen}equalizer\ E\ m\ f\ g\ {\isasymLongrightarrow}\ \ monomorphism{\isacharparenleft}{\kern0pt}m{\isacharparenright}{\kern0pt}{\isachardoublequoteclose}\isanewline
%
\isadelimproof
\ \ %
\endisadelimproof
%
\isatagproof
\isacommand{unfolding}\isamarkupfalse%
\ equalizer{\isacharunderscore}{\kern0pt}def\ monomorphism{\isacharunderscore}{\kern0pt}def\isanewline
\isacommand{proof}\isamarkupfalse%
\ clarify\isanewline
\ \ \isacommand{fix}\isamarkupfalse%
\ h{\isadigit{1}}\ h{\isadigit{2}}\ X\ Y\isanewline
\ \ \isacommand{assume}\isamarkupfalse%
\ f{\isacharunderscore}{\kern0pt}type{\isacharcolon}{\kern0pt}\ {\isachardoublequoteopen}f\ {\isacharcolon}{\kern0pt}\ X\ {\isasymrightarrow}\ Y{\isachardoublequoteclose}\isanewline
\ \ \isacommand{assume}\isamarkupfalse%
\ g{\isacharunderscore}{\kern0pt}type{\isacharcolon}{\kern0pt}\ {\isachardoublequoteopen}g\ {\isacharcolon}{\kern0pt}\ X\ {\isasymrightarrow}\ Y{\isachardoublequoteclose}\isanewline
\ \ \isacommand{assume}\isamarkupfalse%
\ m{\isacharunderscore}{\kern0pt}type{\isacharcolon}{\kern0pt}\ {\isachardoublequoteopen}m\ {\isacharcolon}{\kern0pt}\ E\ {\isasymrightarrow}\ X{\isachardoublequoteclose}\isanewline
\ \ \isacommand{assume}\isamarkupfalse%
\ fm{\isacharunderscore}{\kern0pt}gm{\isacharcolon}{\kern0pt}\ {\isachardoublequoteopen}f\ {\isasymcirc}\isactrlsub c\ m\ {\isacharequal}{\kern0pt}\ g\ {\isasymcirc}\isactrlsub c\ m{\isachardoublequoteclose}\isanewline
\ \ \isacommand{assume}\isamarkupfalse%
\ uniqueness{\isacharcolon}{\kern0pt}\ {\isachardoublequoteopen}{\isasymforall}h\ F{\isachardot}{\kern0pt}\ h\ {\isacharcolon}{\kern0pt}\ F\ {\isasymrightarrow}\ X\ {\isasymand}\ f\ {\isasymcirc}\isactrlsub c\ h\ {\isacharequal}{\kern0pt}\ g\ {\isasymcirc}\isactrlsub c\ h\ {\isasymlongrightarrow}\ {\isacharparenleft}{\kern0pt}{\isasymexists}{\isacharbang}{\kern0pt}k{\isachardot}{\kern0pt}\ k\ {\isacharcolon}{\kern0pt}\ F\ {\isasymrightarrow}\ E\ {\isasymand}\ m\ {\isasymcirc}\isactrlsub c\ k\ {\isacharequal}{\kern0pt}\ h{\isacharparenright}{\kern0pt}{\isachardoublequoteclose}\isanewline
\ \ \isacommand{assume}\isamarkupfalse%
\ relation{\isacharunderscore}{\kern0pt}ga{\isacharcolon}{\kern0pt}\ {\isachardoublequoteopen}codomain\ h{\isadigit{1}}\ {\isacharequal}{\kern0pt}\ domain\ m{\isachardoublequoteclose}\isanewline
\ \ \isacommand{assume}\isamarkupfalse%
\ relation{\isacharunderscore}{\kern0pt}h{\isacharcolon}{\kern0pt}\ {\isachardoublequoteopen}codomain\ h{\isadigit{2}}\ {\isacharequal}{\kern0pt}\ domain\ m{\isachardoublequoteclose}\ \isanewline
\ \ \isacommand{assume}\isamarkupfalse%
\ m{\isacharunderscore}{\kern0pt}ga{\isacharunderscore}{\kern0pt}mh{\isacharcolon}{\kern0pt}\ {\isachardoublequoteopen}m\ {\isasymcirc}\isactrlsub c\ h{\isadigit{1}}\ {\isacharequal}{\kern0pt}\ m\ {\isasymcirc}\isactrlsub c\ h{\isadigit{2}}{\isachardoublequoteclose}\ \ \ \isanewline
\ \ \isacommand{have}\isamarkupfalse%
\ \ {\isachardoublequoteopen}f\ {\isasymcirc}\isactrlsub c\ m\ {\isasymcirc}\isactrlsub c\ h{\isadigit{1}}\ {\isacharequal}{\kern0pt}\ \ g\ {\isasymcirc}\isactrlsub c\ m\ {\isasymcirc}\isactrlsub c\ h{\isadigit{2}}{\isachardoublequoteclose}\isanewline
\ \ \ \ \isacommand{using}\isamarkupfalse%
\ cfunc{\isacharunderscore}{\kern0pt}type{\isacharunderscore}{\kern0pt}def\ comp{\isacharunderscore}{\kern0pt}associative\ f{\isacharunderscore}{\kern0pt}type\ fm{\isacharunderscore}{\kern0pt}gm\ g{\isacharunderscore}{\kern0pt}type\ m{\isacharunderscore}{\kern0pt}ga{\isacharunderscore}{\kern0pt}mh\ m{\isacharunderscore}{\kern0pt}type\ relation{\isacharunderscore}{\kern0pt}h\ \isacommand{by}\isamarkupfalse%
\ auto\isanewline
\ \ \isacommand{then}\isamarkupfalse%
\ \isacommand{obtain}\isamarkupfalse%
\ z\ \isakeyword{where}\ {\isachardoublequoteopen}z{\isacharcolon}{\kern0pt}\ domain{\isacharparenleft}{\kern0pt}h{\isadigit{1}}{\isacharparenright}{\kern0pt}\ {\isasymrightarrow}\ E\ {\isasymand}\ m\ {\isasymcirc}\isactrlsub c\ z\ {\isacharequal}{\kern0pt}\ m\ {\isasymcirc}\isactrlsub c\ h{\isadigit{1}}\ {\isasymand}\ \isanewline
\ \ \ \ {\isacharparenleft}{\kern0pt}{\isasymforall}\ j{\isachardot}{\kern0pt}\ j{\isacharcolon}{\kern0pt}domain{\isacharparenleft}{\kern0pt}h{\isadigit{1}}{\isacharparenright}{\kern0pt}\ {\isasymrightarrow}\ E\ {\isasymand}\ \ m\ {\isasymcirc}\isactrlsub c\ j\ {\isacharequal}{\kern0pt}\ m\ {\isasymcirc}\isactrlsub c\ h{\isadigit{1}}\ {\isasymlongrightarrow}\ j\ {\isacharequal}{\kern0pt}\ z{\isacharparenright}{\kern0pt}{\isachardoublequoteclose}\isanewline
\ \ \ \ \isacommand{using}\isamarkupfalse%
\ uniqueness\ \isacommand{by}\isamarkupfalse%
\ {\isacharparenleft}{\kern0pt}erule{\isacharunderscore}{\kern0pt}tac\ x{\isacharequal}{\kern0pt}{\isachardoublequoteopen}m\ {\isasymcirc}\isactrlsub c\ h{\isadigit{1}}{\isachardoublequoteclose}\ \isakeyword{in}\ allE{\isacharcomma}{\kern0pt}\ erule{\isacharunderscore}{\kern0pt}tac\ x{\isacharequal}{\kern0pt}{\isachardoublequoteopen}domain{\isacharparenleft}{\kern0pt}h{\isadigit{1}}{\isacharparenright}{\kern0pt}{\isachardoublequoteclose}\ \isakeyword{in}\ allE{\isacharcomma}{\kern0pt}\isanewline
\ \ \ \ \ \ \ \ \ \ \ \ \ \ \ \ \ \ \ \ \ \ \ \ \ smt\ cfunc{\isacharunderscore}{\kern0pt}type{\isacharunderscore}{\kern0pt}def\ codomain{\isacharunderscore}{\kern0pt}comp\ domain{\isacharunderscore}{\kern0pt}comp\ m{\isacharunderscore}{\kern0pt}ga{\isacharunderscore}{\kern0pt}mh\ m{\isacharunderscore}{\kern0pt}type\ relation{\isacharunderscore}{\kern0pt}ga{\isacharparenright}{\kern0pt}\isanewline
\ \ \isacommand{then}\isamarkupfalse%
\ \isacommand{show}\isamarkupfalse%
\ {\isachardoublequoteopen}h{\isadigit{1}}\ {\isacharequal}{\kern0pt}\ h{\isadigit{2}}{\isachardoublequoteclose}\isanewline
\ \ \ \ \isacommand{by}\isamarkupfalse%
\ {\isacharparenleft}{\kern0pt}metis\ cfunc{\isacharunderscore}{\kern0pt}type{\isacharunderscore}{\kern0pt}def\ domain{\isacharunderscore}{\kern0pt}comp\ m{\isacharunderscore}{\kern0pt}ga{\isacharunderscore}{\kern0pt}mh\ m{\isacharunderscore}{\kern0pt}type\ relation{\isacharunderscore}{\kern0pt}ga\ relation{\isacharunderscore}{\kern0pt}h{\isacharparenright}{\kern0pt}\isanewline
\isacommand{qed}\isamarkupfalse%
%
\endisatagproof
{\isafoldproof}%
%
\isadelimproof
%
\endisadelimproof
%
\begin{isamarkuptext}%
The definition below corresponds to Definition 2.1.35 in Halvorson.%
\end{isamarkuptext}\isamarkuptrue%
\isacommand{definition}\isamarkupfalse%
\ regular{\isacharunderscore}{\kern0pt}monomorphism\ {\isacharcolon}{\kern0pt}{\isacharcolon}{\kern0pt}\ {\isachardoublequoteopen}cfunc\ {\isasymRightarrow}\ bool{\isachardoublequoteclose}\isanewline
\ \ \isakeyword{where}\ {\isachardoublequoteopen}regular{\isacharunderscore}{\kern0pt}monomorphism\ f\ \ {\isasymlongleftrightarrow}\ \ \isanewline
\ \ \ \ \ \ \ \ \ \ {\isacharparenleft}{\kern0pt}{\isasymexists}\ g\ h{\isachardot}{\kern0pt}\ domain\ g\ {\isacharequal}{\kern0pt}\ codomain\ f\ {\isasymand}\ domain\ h\ {\isacharequal}{\kern0pt}\ codomain\ f\ {\isasymand}\ equalizer\ {\isacharparenleft}{\kern0pt}domain\ f{\isacharparenright}{\kern0pt}\ f\ g\ h{\isacharparenright}{\kern0pt}{\isachardoublequoteclose}%
\begin{isamarkuptext}%
The lemma below corresponds to Exercise 2.1.36 in Halvorson.%
\end{isamarkuptext}\isamarkuptrue%
\isacommand{lemma}\isamarkupfalse%
\ epi{\isacharunderscore}{\kern0pt}regmon{\isacharunderscore}{\kern0pt}is{\isacharunderscore}{\kern0pt}iso{\isacharcolon}{\kern0pt}\isanewline
\ \ \isakeyword{assumes}\ {\isachardoublequoteopen}epimorphism\ f{\isachardoublequoteclose}\ {\isachardoublequoteopen}regular{\isacharunderscore}{\kern0pt}monomorphism\ f{\isachardoublequoteclose}\isanewline
\ \ \isakeyword{shows}\ {\isachardoublequoteopen}isomorphism\ f{\isachardoublequoteclose}\isanewline
%
\isadelimproof
%
\endisadelimproof
%
\isatagproof
\isacommand{proof}\isamarkupfalse%
\ {\isacharminus}{\kern0pt}\isanewline
\ \ \isacommand{obtain}\isamarkupfalse%
\ g\ h\ \isakeyword{where}\ g{\isacharunderscore}{\kern0pt}type{\isacharcolon}{\kern0pt}\ {\isachardoublequoteopen}domain\ g\ {\isacharequal}{\kern0pt}\ codomain\ f{\isachardoublequoteclose}\ \isakeyword{and}\isanewline
\ \ \ \ \ \ \ \ \ \ \ \ \ \ \ \ \ \ \ h{\isacharunderscore}{\kern0pt}type{\isacharcolon}{\kern0pt}\ {\isachardoublequoteopen}domain\ h\ {\isacharequal}{\kern0pt}\ codomain\ f{\isachardoublequoteclose}\ \isakeyword{and}\isanewline
\ \ \ \ \ \ \ \ \ \ \ \ \ \ \ \ \ \ \ f{\isacharunderscore}{\kern0pt}equalizer{\isacharcolon}{\kern0pt}\ {\isachardoublequoteopen}equalizer\ {\isacharparenleft}{\kern0pt}domain\ f{\isacharparenright}{\kern0pt}\ f\ g\ h{\isachardoublequoteclose}\isanewline
\ \ \ \ \isacommand{using}\isamarkupfalse%
\ assms{\isacharparenleft}{\kern0pt}{\isadigit{2}}{\isacharparenright}{\kern0pt}\ regular{\isacharunderscore}{\kern0pt}monomorphism{\isacharunderscore}{\kern0pt}def\ \isacommand{by}\isamarkupfalse%
\ auto\isanewline
\ \ \isacommand{then}\isamarkupfalse%
\ \isacommand{have}\isamarkupfalse%
\ {\isachardoublequoteopen}g\ {\isasymcirc}\isactrlsub c\ f\ {\isacharequal}{\kern0pt}\ h\ {\isasymcirc}\isactrlsub c\ f{\isachardoublequoteclose}\isanewline
\ \ \ \ \isacommand{using}\isamarkupfalse%
\ equalizer{\isacharunderscore}{\kern0pt}def\ \isacommand{by}\isamarkupfalse%
\ blast\isanewline
\ \ \isacommand{then}\isamarkupfalse%
\ \isacommand{have}\isamarkupfalse%
\ {\isachardoublequoteopen}g\ {\isacharequal}{\kern0pt}\ h{\isachardoublequoteclose}\isanewline
\ \ \ \ \isacommand{using}\isamarkupfalse%
\ assms{\isacharparenleft}{\kern0pt}{\isadigit{1}}{\isacharparenright}{\kern0pt}\ cfunc{\isacharunderscore}{\kern0pt}type{\isacharunderscore}{\kern0pt}def\ epimorphism{\isacharunderscore}{\kern0pt}def\ equalizer{\isacharunderscore}{\kern0pt}def\ f{\isacharunderscore}{\kern0pt}equalizer\ \isacommand{by}\isamarkupfalse%
\ auto\isanewline
\ \ \isacommand{then}\isamarkupfalse%
\ \isacommand{have}\isamarkupfalse%
\ {\isachardoublequoteopen}g\ {\isasymcirc}\isactrlsub c\ id{\isacharparenleft}{\kern0pt}codomain\ f{\isacharparenright}{\kern0pt}\ {\isacharequal}{\kern0pt}\ h\ {\isasymcirc}\isactrlsub c\ id{\isacharparenleft}{\kern0pt}codomain\ f{\isacharparenright}{\kern0pt}{\isachardoublequoteclose}\isanewline
\ \ \ \ \isacommand{by}\isamarkupfalse%
\ simp\isanewline
\ \ \isacommand{then}\isamarkupfalse%
\ \isacommand{obtain}\isamarkupfalse%
\ k\ \isakeyword{where}\ k{\isacharunderscore}{\kern0pt}type{\isacharcolon}{\kern0pt}\ {\isachardoublequoteopen}f\ {\isasymcirc}\isactrlsub c\ k\ {\isacharequal}{\kern0pt}\ id{\isacharparenleft}{\kern0pt}codomain{\isacharparenleft}{\kern0pt}f{\isacharparenright}{\kern0pt}{\isacharparenright}{\kern0pt}\ {\isasymand}\ codomain\ k\ {\isacharequal}{\kern0pt}\ domain\ f{\isachardoublequoteclose}\isanewline
\ \ \ \ \isacommand{by}\isamarkupfalse%
\ {\isacharparenleft}{\kern0pt}metis\ cfunc{\isacharunderscore}{\kern0pt}type{\isacharunderscore}{\kern0pt}def\ equalizer{\isacharunderscore}{\kern0pt}def\ f{\isacharunderscore}{\kern0pt}equalizer\ id{\isacharunderscore}{\kern0pt}type{\isacharparenright}{\kern0pt}\isanewline
\ \ \isacommand{then}\isamarkupfalse%
\ \isacommand{have}\isamarkupfalse%
\ {\isachardoublequoteopen}f\ {\isasymcirc}\isactrlsub c\ id{\isacharparenleft}{\kern0pt}domain{\isacharparenleft}{\kern0pt}f{\isacharparenright}{\kern0pt}{\isacharparenright}{\kern0pt}\ {\isacharequal}{\kern0pt}\ f\ {\isasymcirc}\isactrlsub c\ {\isacharparenleft}{\kern0pt}k\ {\isasymcirc}\isactrlsub c\ f{\isacharparenright}{\kern0pt}{\isachardoublequoteclose}\isanewline
\ \ \ \ \isacommand{by}\isamarkupfalse%
\ {\isacharparenleft}{\kern0pt}metis\ comp{\isacharunderscore}{\kern0pt}associative\ domain{\isacharunderscore}{\kern0pt}comp\ id{\isacharunderscore}{\kern0pt}domain\ id{\isacharunderscore}{\kern0pt}left{\isacharunderscore}{\kern0pt}unit\ id{\isacharunderscore}{\kern0pt}right{\isacharunderscore}{\kern0pt}unit{\isacharparenright}{\kern0pt}\isanewline
\ \ \isacommand{then}\isamarkupfalse%
\ \isacommand{have}\isamarkupfalse%
\ {\isachardoublequoteopen}monomorphism\ f\ {\isasymLongrightarrow}\ k\ {\isasymcirc}\isactrlsub c\ f\ {\isacharequal}{\kern0pt}\ id{\isacharparenleft}{\kern0pt}domain\ f{\isacharparenright}{\kern0pt}{\isachardoublequoteclose}\isanewline
\ \ \ \ \isacommand{by}\isamarkupfalse%
\ {\isacharparenleft}{\kern0pt}metis\ {\isacharparenleft}{\kern0pt}mono{\isacharunderscore}{\kern0pt}tags{\isacharparenright}{\kern0pt}\ codomain{\isacharunderscore}{\kern0pt}comp\ domain{\isacharunderscore}{\kern0pt}comp\ id{\isacharunderscore}{\kern0pt}codomain\ id{\isacharunderscore}{\kern0pt}domain\ k{\isacharunderscore}{\kern0pt}type\ monomorphism{\isacharunderscore}{\kern0pt}def{\isacharparenright}{\kern0pt}\isanewline
\ \ \isacommand{then}\isamarkupfalse%
\ \isacommand{have}\isamarkupfalse%
\ {\isachardoublequoteopen}k\ {\isasymcirc}\isactrlsub c\ f\ {\isacharequal}{\kern0pt}\ id{\isacharparenleft}{\kern0pt}domain\ f{\isacharparenright}{\kern0pt}{\isachardoublequoteclose}\isanewline
\ \ \ \ \isacommand{using}\isamarkupfalse%
\ equalizer{\isacharunderscore}{\kern0pt}is{\isacharunderscore}{\kern0pt}monomorphism\ f{\isacharunderscore}{\kern0pt}equalizer\ \isacommand{by}\isamarkupfalse%
\ blast\isanewline
\ \ \isacommand{then}\isamarkupfalse%
\ \isacommand{show}\isamarkupfalse%
\ {\isachardoublequoteopen}isomorphism\ f{\isachardoublequoteclose}\isanewline
\ \ \ \ \isacommand{by}\isamarkupfalse%
\ {\isacharparenleft}{\kern0pt}metis\ domain{\isacharunderscore}{\kern0pt}comp\ id{\isacharunderscore}{\kern0pt}domain\ isomorphism{\isacharunderscore}{\kern0pt}def\ k{\isacharunderscore}{\kern0pt}type{\isacharparenright}{\kern0pt}\ \ \isanewline
\isacommand{qed}\isamarkupfalse%
%
\endisatagproof
{\isafoldproof}%
%
\isadelimproof
%
\endisadelimproof
%
\isadelimdocument
%
\endisadelimdocument
%
\isatagdocument
%
\isamarkupsubsection{Subobjects%
}
\isamarkuptrue%
%
\endisatagdocument
{\isafolddocument}%
%
\isadelimdocument
%
\endisadelimdocument
%
\begin{isamarkuptext}%
The definition below corresponds to Definition 2.1.32 in Halvorson.%
\end{isamarkuptext}\isamarkuptrue%
\isacommand{definition}\isamarkupfalse%
\ factors{\isacharunderscore}{\kern0pt}through\ {\isacharcolon}{\kern0pt}{\isacharcolon}{\kern0pt}\ {\isachardoublequoteopen}cfunc\ \ {\isasymRightarrow}\ cfunc\ {\isasymRightarrow}\ bool{\isachardoublequoteclose}\ {\isacharparenleft}{\kern0pt}\isakeyword{infix}\ {\isachardoublequoteopen}factorsthru{\isachardoublequoteclose}\ {\isadigit{9}}{\isadigit{0}}{\isacharparenright}{\kern0pt}\isanewline
\ \ \isakeyword{where}\ {\isachardoublequoteopen}g\ factorsthru\ f\ {\isasymlongleftrightarrow}\ {\isacharparenleft}{\kern0pt}{\isasymexists}\ h{\isachardot}{\kern0pt}\ {\isacharparenleft}{\kern0pt}h{\isacharcolon}{\kern0pt}\ domain{\isacharparenleft}{\kern0pt}g{\isacharparenright}{\kern0pt}{\isasymrightarrow}\ domain{\isacharparenleft}{\kern0pt}f{\isacharparenright}{\kern0pt}{\isacharparenright}{\kern0pt}\ {\isasymand}\ f\ {\isasymcirc}\isactrlsub c\ h\ {\isacharequal}{\kern0pt}\ g{\isacharparenright}{\kern0pt}{\isachardoublequoteclose}\isanewline
\isanewline
\isacommand{lemma}\isamarkupfalse%
\ factors{\isacharunderscore}{\kern0pt}through{\isacharunderscore}{\kern0pt}def{\isadigit{2}}{\isacharcolon}{\kern0pt}\isanewline
\ \ \isakeyword{assumes}\ {\isachardoublequoteopen}g\ {\isacharcolon}{\kern0pt}\ X\ {\isasymrightarrow}\ Z{\isachardoublequoteclose}\ {\isachardoublequoteopen}f\ {\isacharcolon}{\kern0pt}\ Y\ {\isasymrightarrow}\ Z{\isachardoublequoteclose}\isanewline
\ \ \isakeyword{shows}\ {\isachardoublequoteopen}g\ factorsthru\ f\ {\isasymlongleftrightarrow}\ {\isacharparenleft}{\kern0pt}{\isasymexists}\ h{\isachardot}{\kern0pt}\ h{\isacharcolon}{\kern0pt}\ X\ {\isasymrightarrow}\ Y\ {\isasymand}\ f\ {\isasymcirc}\isactrlsub c\ h\ {\isacharequal}{\kern0pt}\ g{\isacharparenright}{\kern0pt}{\isachardoublequoteclose}\isanewline
%
\isadelimproof
\ \ %
\endisadelimproof
%
\isatagproof
\isacommand{unfolding}\isamarkupfalse%
\ factors{\isacharunderscore}{\kern0pt}through{\isacharunderscore}{\kern0pt}def\ \isacommand{using}\isamarkupfalse%
\ assms\ \isacommand{by}\isamarkupfalse%
\ {\isacharparenleft}{\kern0pt}simp\ add{\isacharcolon}{\kern0pt}\ cfunc{\isacharunderscore}{\kern0pt}type{\isacharunderscore}{\kern0pt}def{\isacharparenright}{\kern0pt}%
\endisatagproof
{\isafoldproof}%
%
\isadelimproof
%
\endisadelimproof
%
\begin{isamarkuptext}%
The lemma below corresponds to Exercise 2.1.33 in Halvorson.%
\end{isamarkuptext}\isamarkuptrue%
\isacommand{lemma}\isamarkupfalse%
\ xfactorthru{\isacharunderscore}{\kern0pt}equalizer{\isacharunderscore}{\kern0pt}iff{\isacharunderscore}{\kern0pt}fx{\isacharunderscore}{\kern0pt}eq{\isacharunderscore}{\kern0pt}gx{\isacharcolon}{\kern0pt}\isanewline
\ \ \isakeyword{assumes}\ {\isachardoublequoteopen}f{\isacharcolon}{\kern0pt}\ X{\isasymrightarrow}\ Y{\isachardoublequoteclose}\ {\isachardoublequoteopen}g{\isacharcolon}{\kern0pt}X\ {\isasymrightarrow}\ Y{\isachardoublequoteclose}\ {\isachardoublequoteopen}equalizer\ E\ m\ f\ g{\isachardoublequoteclose}\ {\isachardoublequoteopen}x{\isasymin}\isactrlsub c\ X{\isachardoublequoteclose}\isanewline
\ \ \isakeyword{shows}\ {\isachardoublequoteopen}x\ factorsthru\ m\ {\isasymlongleftrightarrow}\ f\ {\isasymcirc}\isactrlsub c\ x\ {\isacharequal}{\kern0pt}\ g\ \ {\isasymcirc}\isactrlsub c\ x{\isachardoublequoteclose}\isanewline
%
\isadelimproof
%
\endisadelimproof
%
\isatagproof
\isacommand{proof}\isamarkupfalse%
\ safe\isanewline
\ \ \isacommand{assume}\isamarkupfalse%
\ LHS{\isacharcolon}{\kern0pt}\ {\isachardoublequoteopen}x\ factorsthru\ m{\isachardoublequoteclose}\isanewline
\ \ \isacommand{then}\isamarkupfalse%
\ \isacommand{show}\isamarkupfalse%
\ {\isachardoublequoteopen}f\ {\isasymcirc}\isactrlsub c\ x\ {\isacharequal}{\kern0pt}\ g\ \ {\isasymcirc}\isactrlsub c\ x{\isachardoublequoteclose}\isanewline
\ \ \ \ \isacommand{using}\isamarkupfalse%
\ assms{\isacharparenleft}{\kern0pt}{\isadigit{3}}{\isacharparenright}{\kern0pt}\ cfunc{\isacharunderscore}{\kern0pt}type{\isacharunderscore}{\kern0pt}def\ comp{\isacharunderscore}{\kern0pt}associative\ equalizer{\isacharunderscore}{\kern0pt}def\ factors{\isacharunderscore}{\kern0pt}through{\isacharunderscore}{\kern0pt}def\ \isacommand{by}\isamarkupfalse%
\ auto\isanewline
\isacommand{next}\isamarkupfalse%
\isanewline
\ \ \isacommand{assume}\isamarkupfalse%
\ RHS{\isacharcolon}{\kern0pt}\ {\isachardoublequoteopen}f\ {\isasymcirc}\isactrlsub c\ x\ {\isacharequal}{\kern0pt}\ g\ \ {\isasymcirc}\isactrlsub c\ x{\isachardoublequoteclose}\isanewline
\ \ \isacommand{then}\isamarkupfalse%
\ \isacommand{show}\isamarkupfalse%
\ {\isachardoublequoteopen}x\ factorsthru\ m{\isachardoublequoteclose}\isanewline
\ \ \ \ \isacommand{unfolding}\isamarkupfalse%
\ cfunc{\isacharunderscore}{\kern0pt}type{\isacharunderscore}{\kern0pt}def\ factors{\isacharunderscore}{\kern0pt}through{\isacharunderscore}{\kern0pt}def\isanewline
\ \ \ \ \isacommand{by}\isamarkupfalse%
\ {\isacharparenleft}{\kern0pt}metis\ RHS\ assms{\isacharparenleft}{\kern0pt}{\isadigit{1}}{\isacharcomma}{\kern0pt}{\isadigit{3}}{\isacharcomma}{\kern0pt}{\isadigit{4}}{\isacharparenright}{\kern0pt}\ cfunc{\isacharunderscore}{\kern0pt}type{\isacharunderscore}{\kern0pt}def\ equalizer{\isacharunderscore}{\kern0pt}def{\isacharparenright}{\kern0pt}\isanewline
\isacommand{qed}\isamarkupfalse%
%
\endisatagproof
{\isafoldproof}%
%
\isadelimproof
%
\endisadelimproof
%
\begin{isamarkuptext}%
The definition below corresponds to Definition 2.1.37 in Halvorson.%
\end{isamarkuptext}\isamarkuptrue%
\isacommand{definition}\isamarkupfalse%
\ subobject{\isacharunderscore}{\kern0pt}of\ {\isacharcolon}{\kern0pt}{\isacharcolon}{\kern0pt}\ {\isachardoublequoteopen}cset\ {\isasymtimes}\ cfunc\ {\isasymRightarrow}\ cset\ {\isasymRightarrow}\ bool{\isachardoublequoteclose}\ {\isacharparenleft}{\kern0pt}\isakeyword{infix}\ {\isachardoublequoteopen}{\isasymsubseteq}\isactrlsub c{\isachardoublequoteclose}\ {\isadigit{5}}{\isadigit{0}}{\isacharparenright}{\kern0pt}\isanewline
\ \ \isakeyword{where}\ {\isachardoublequoteopen}B\ {\isasymsubseteq}\isactrlsub c\ X\ {\isasymlongleftrightarrow}\ {\isacharparenleft}{\kern0pt}snd\ B\ {\isacharcolon}{\kern0pt}\ fst\ B\ {\isasymrightarrow}\ X\ {\isasymand}\ monomorphism\ {\isacharparenleft}{\kern0pt}snd\ B{\isacharparenright}{\kern0pt}{\isacharparenright}{\kern0pt}{\isachardoublequoteclose}\isanewline
\isanewline
\isacommand{lemma}\isamarkupfalse%
\ subobject{\isacharunderscore}{\kern0pt}of{\isacharunderscore}{\kern0pt}def{\isadigit{2}}{\isacharcolon}{\kern0pt}\isanewline
\ \ {\isachardoublequoteopen}{\isacharparenleft}{\kern0pt}B{\isacharcomma}{\kern0pt}m{\isacharparenright}{\kern0pt}\ {\isasymsubseteq}\isactrlsub c\ X\ {\isacharequal}{\kern0pt}\ {\isacharparenleft}{\kern0pt}m\ {\isacharcolon}{\kern0pt}\ B\ {\isasymrightarrow}\ X\ {\isasymand}\ monomorphism\ m{\isacharparenright}{\kern0pt}{\isachardoublequoteclose}\isanewline
%
\isadelimproof
\ \ %
\endisadelimproof
%
\isatagproof
\isacommand{by}\isamarkupfalse%
\ {\isacharparenleft}{\kern0pt}simp\ add{\isacharcolon}{\kern0pt}\ subobject{\isacharunderscore}{\kern0pt}of{\isacharunderscore}{\kern0pt}def{\isacharparenright}{\kern0pt}%
\endisatagproof
{\isafoldproof}%
%
\isadelimproof
\isanewline
%
\endisadelimproof
\isanewline
\isacommand{definition}\isamarkupfalse%
\ relative{\isacharunderscore}{\kern0pt}subset\ {\isacharcolon}{\kern0pt}{\isacharcolon}{\kern0pt}\ {\isachardoublequoteopen}cset\ {\isasymtimes}\ cfunc\ {\isasymRightarrow}\ cset\ {\isasymRightarrow}\ cset\ {\isasymtimes}\ cfunc\ {\isasymRightarrow}\ bool{\isachardoublequoteclose}\ {\isacharparenleft}{\kern0pt}{\isachardoublequoteopen}{\isacharunderscore}{\kern0pt}{\isasymsubseteq}\isactrlbsub {\isacharunderscore}{\kern0pt}\isactrlesub {\isacharunderscore}{\kern0pt}{\isachardoublequoteclose}\ {\isacharbrackleft}{\kern0pt}{\isadigit{5}}{\isadigit{1}}{\isacharcomma}{\kern0pt}{\isadigit{5}}{\isadigit{0}}{\isacharcomma}{\kern0pt}{\isadigit{5}}{\isadigit{1}}{\isacharbrackright}{\kern0pt}{\isadigit{5}}{\isadigit{0}}{\isacharparenright}{\kern0pt}\isanewline
\ \ \isakeyword{where}\ {\isachardoublequoteopen}B\ {\isasymsubseteq}\isactrlbsub X\isactrlesub \ A\ \ {\isasymlongleftrightarrow}\ \isanewline
\ \ \ \ {\isacharparenleft}{\kern0pt}snd\ B\ {\isacharcolon}{\kern0pt}\ fst\ B\ {\isasymrightarrow}\ X\ {\isasymand}\ monomorphism\ {\isacharparenleft}{\kern0pt}snd\ B{\isacharparenright}{\kern0pt}\ {\isasymand}\ snd\ A\ {\isacharcolon}{\kern0pt}\ fst\ A\ {\isasymrightarrow}\ X\ {\isasymand}\ monomorphism\ {\isacharparenleft}{\kern0pt}snd\ A{\isacharparenright}{\kern0pt}\isanewline
\ \ \ \ \ \ \ \ \ \ {\isasymand}\ {\isacharparenleft}{\kern0pt}{\isasymexists}\ k{\isachardot}{\kern0pt}\ k{\isacharcolon}{\kern0pt}\ fst\ B\ {\isasymrightarrow}\ fst\ A\ {\isasymand}\ snd\ A\ {\isasymcirc}\isactrlsub c\ k\ {\isacharequal}{\kern0pt}\ snd\ B{\isacharparenright}{\kern0pt}{\isacharparenright}{\kern0pt}{\isachardoublequoteclose}\isanewline
\isanewline
\isacommand{lemma}\isamarkupfalse%
\ relative{\isacharunderscore}{\kern0pt}subset{\isacharunderscore}{\kern0pt}def{\isadigit{2}}{\isacharcolon}{\kern0pt}\ \isanewline
\ \ {\isachardoublequoteopen}{\isacharparenleft}{\kern0pt}B{\isacharcomma}{\kern0pt}m{\isacharparenright}{\kern0pt}\ {\isasymsubseteq}\isactrlbsub X\isactrlesub \ {\isacharparenleft}{\kern0pt}A{\isacharcomma}{\kern0pt}n{\isacharparenright}{\kern0pt}\ {\isacharequal}{\kern0pt}\ {\isacharparenleft}{\kern0pt}m\ {\isacharcolon}{\kern0pt}\ B\ {\isasymrightarrow}\ X\ {\isasymand}\ monomorphism\ m\ {\isasymand}\ n\ {\isacharcolon}{\kern0pt}\ A\ {\isasymrightarrow}\ X\ {\isasymand}\ monomorphism\ n\isanewline
\ \ \ \ \ \ \ \ \ \ {\isasymand}\ {\isacharparenleft}{\kern0pt}{\isasymexists}\ k{\isachardot}{\kern0pt}\ k{\isacharcolon}{\kern0pt}\ B\ {\isasymrightarrow}\ A\ {\isasymand}\ n\ {\isasymcirc}\isactrlsub c\ k\ {\isacharequal}{\kern0pt}\ m{\isacharparenright}{\kern0pt}{\isacharparenright}{\kern0pt}{\isachardoublequoteclose}\isanewline
%
\isadelimproof
\ \ %
\endisadelimproof
%
\isatagproof
\isacommand{unfolding}\isamarkupfalse%
\ relative{\isacharunderscore}{\kern0pt}subset{\isacharunderscore}{\kern0pt}def\ \isacommand{by}\isamarkupfalse%
\ auto%
\endisatagproof
{\isafoldproof}%
%
\isadelimproof
\isanewline
%
\endisadelimproof
\isanewline
\isacommand{lemma}\isamarkupfalse%
\ subobject{\isacharunderscore}{\kern0pt}is{\isacharunderscore}{\kern0pt}relative{\isacharunderscore}{\kern0pt}subset{\isacharcolon}{\kern0pt}\ {\isachardoublequoteopen}{\isacharparenleft}{\kern0pt}B{\isacharcomma}{\kern0pt}m{\isacharparenright}{\kern0pt}\ {\isasymsubseteq}\isactrlsub c\ A\ {\isasymlongleftrightarrow}\ {\isacharparenleft}{\kern0pt}B{\isacharcomma}{\kern0pt}m{\isacharparenright}{\kern0pt}\ {\isasymsubseteq}\isactrlbsub A\isactrlesub \ {\isacharparenleft}{\kern0pt}A{\isacharcomma}{\kern0pt}\ id{\isacharparenleft}{\kern0pt}A{\isacharparenright}{\kern0pt}{\isacharparenright}{\kern0pt}{\isachardoublequoteclose}\isanewline
%
\isadelimproof
\ \ %
\endisadelimproof
%
\isatagproof
\isacommand{unfolding}\isamarkupfalse%
\ relative{\isacharunderscore}{\kern0pt}subset{\isacharunderscore}{\kern0pt}def{\isadigit{2}}\ subobject{\isacharunderscore}{\kern0pt}of{\isacharunderscore}{\kern0pt}def{\isadigit{2}}\isanewline
\ \ \isacommand{using}\isamarkupfalse%
\ cfunc{\isacharunderscore}{\kern0pt}type{\isacharunderscore}{\kern0pt}def\ id{\isacharunderscore}{\kern0pt}isomorphism\ id{\isacharunderscore}{\kern0pt}left{\isacharunderscore}{\kern0pt}unit\ id{\isacharunderscore}{\kern0pt}type\ iso{\isacharunderscore}{\kern0pt}imp{\isacharunderscore}{\kern0pt}epi{\isacharunderscore}{\kern0pt}and{\isacharunderscore}{\kern0pt}monic\ \isacommand{by}\isamarkupfalse%
\ auto%
\endisatagproof
{\isafoldproof}%
%
\isadelimproof
%
\endisadelimproof
%
\begin{isamarkuptext}%
The definition below corresponds to Definition 2.1.39 in Halvorson.%
\end{isamarkuptext}\isamarkuptrue%
\isacommand{definition}\isamarkupfalse%
\ relative{\isacharunderscore}{\kern0pt}member\ {\isacharcolon}{\kern0pt}{\isacharcolon}{\kern0pt}\ {\isachardoublequoteopen}cfunc\ {\isasymRightarrow}\ cset\ {\isasymRightarrow}\ cset\ {\isasymtimes}\ cfunc\ {\isasymRightarrow}\ bool{\isachardoublequoteclose}\ {\isacharparenleft}{\kern0pt}{\isachardoublequoteopen}{\isacharunderscore}{\kern0pt}\ {\isasymin}\isactrlbsub {\isacharunderscore}{\kern0pt}\isactrlesub \ {\isacharunderscore}{\kern0pt}{\isachardoublequoteclose}\ {\isacharbrackleft}{\kern0pt}{\isadigit{5}}{\isadigit{1}}{\isacharcomma}{\kern0pt}{\isadigit{5}}{\isadigit{0}}{\isacharcomma}{\kern0pt}{\isadigit{5}}{\isadigit{1}}{\isacharbrackright}{\kern0pt}{\isadigit{5}}{\isadigit{0}}{\isacharparenright}{\kern0pt}\ \isakeyword{where}\isanewline
\ \ {\isachardoublequoteopen}x\ {\isasymin}\isactrlbsub X\isactrlesub \ B\ {\isasymlongleftrightarrow}\ {\isacharparenleft}{\kern0pt}x\ {\isasymin}\isactrlsub c\ X\ {\isasymand}\ monomorphism\ {\isacharparenleft}{\kern0pt}snd\ B{\isacharparenright}{\kern0pt}\ {\isasymand}\ snd\ B\ {\isacharcolon}{\kern0pt}\ fst\ B\ {\isasymrightarrow}\ X\ {\isasymand}\ x\ factorsthru\ {\isacharparenleft}{\kern0pt}snd\ B{\isacharparenright}{\kern0pt}{\isacharparenright}{\kern0pt}{\isachardoublequoteclose}\isanewline
\isanewline
\isacommand{lemma}\isamarkupfalse%
\ relative{\isacharunderscore}{\kern0pt}member{\isacharunderscore}{\kern0pt}def{\isadigit{2}}{\isacharcolon}{\kern0pt}\isanewline
\ \ {\isachardoublequoteopen}x\ {\isasymin}\isactrlbsub X\isactrlesub \ {\isacharparenleft}{\kern0pt}B{\isacharcomma}{\kern0pt}\ m{\isacharparenright}{\kern0pt}\ {\isacharequal}{\kern0pt}\ {\isacharparenleft}{\kern0pt}x\ {\isasymin}\isactrlsub c\ X\ {\isasymand}\ monomorphism\ m\ {\isasymand}\ m\ {\isacharcolon}{\kern0pt}\ B\ {\isasymrightarrow}\ X\ {\isasymand}\ x\ factorsthru\ m{\isacharparenright}{\kern0pt}{\isachardoublequoteclose}\isanewline
%
\isadelimproof
\ \ %
\endisadelimproof
%
\isatagproof
\isacommand{unfolding}\isamarkupfalse%
\ relative{\isacharunderscore}{\kern0pt}member{\isacharunderscore}{\kern0pt}def\ \isacommand{by}\isamarkupfalse%
\ auto%
\endisatagproof
{\isafoldproof}%
%
\isadelimproof
%
\endisadelimproof
%
\begin{isamarkuptext}%
The lemma below corresponds to Proposition 2.1.40 in Halvorson.%
\end{isamarkuptext}\isamarkuptrue%
\isacommand{lemma}\isamarkupfalse%
\ relative{\isacharunderscore}{\kern0pt}subobject{\isacharunderscore}{\kern0pt}member{\isacharcolon}{\kern0pt}\isanewline
\ \ \isakeyword{assumes}\ {\isachardoublequoteopen}{\isacharparenleft}{\kern0pt}A{\isacharcomma}{\kern0pt}n{\isacharparenright}{\kern0pt}\ {\isasymsubseteq}\isactrlbsub X\isactrlesub \ {\isacharparenleft}{\kern0pt}B{\isacharcomma}{\kern0pt}m{\isacharparenright}{\kern0pt}{\isachardoublequoteclose}\ {\isachardoublequoteopen}x\ {\isasymin}\isactrlsub c\ X{\isachardoublequoteclose}\isanewline
\ \ \isakeyword{shows}\ {\isachardoublequoteopen}x\ {\isasymin}\isactrlbsub X\isactrlesub \ {\isacharparenleft}{\kern0pt}A{\isacharcomma}{\kern0pt}n{\isacharparenright}{\kern0pt}\ {\isasymLongrightarrow}\ x\ {\isasymin}\isactrlbsub X\isactrlesub \ {\isacharparenleft}{\kern0pt}B{\isacharcomma}{\kern0pt}m{\isacharparenright}{\kern0pt}{\isachardoublequoteclose}\isanewline
%
\isadelimproof
\ \ %
\endisadelimproof
%
\isatagproof
\isacommand{using}\isamarkupfalse%
\ assms\ \isacommand{unfolding}\isamarkupfalse%
\ relative{\isacharunderscore}{\kern0pt}member{\isacharunderscore}{\kern0pt}def{\isadigit{2}}\ relative{\isacharunderscore}{\kern0pt}subset{\isacharunderscore}{\kern0pt}def{\isadigit{2}}\isanewline
\isacommand{proof}\isamarkupfalse%
\ clarify\isanewline
\ \ \isacommand{fix}\isamarkupfalse%
\ k\isanewline
\ \ \isacommand{assume}\isamarkupfalse%
\ m{\isacharunderscore}{\kern0pt}type{\isacharcolon}{\kern0pt}\ {\isachardoublequoteopen}m\ {\isacharcolon}{\kern0pt}\ B\ {\isasymrightarrow}\ X{\isachardoublequoteclose}\isanewline
\ \ \isacommand{assume}\isamarkupfalse%
\ k{\isacharunderscore}{\kern0pt}type{\isacharcolon}{\kern0pt}\ {\isachardoublequoteopen}k\ {\isacharcolon}{\kern0pt}\ A\ {\isasymrightarrow}\ B{\isachardoublequoteclose}\isanewline
\ \ \isacommand{assume}\isamarkupfalse%
\ m{\isacharunderscore}{\kern0pt}monomorphism{\isacharcolon}{\kern0pt}\ {\isachardoublequoteopen}monomorphism\ m{\isachardoublequoteclose}\isanewline
\ \ \isacommand{assume}\isamarkupfalse%
\ mk{\isacharunderscore}{\kern0pt}monomorphism{\isacharcolon}{\kern0pt}\ {\isachardoublequoteopen}monomorphism\ {\isacharparenleft}{\kern0pt}m\ {\isasymcirc}\isactrlsub c\ k{\isacharparenright}{\kern0pt}{\isachardoublequoteclose}\isanewline
\ \ \isacommand{assume}\isamarkupfalse%
\ n{\isacharunderscore}{\kern0pt}eq{\isacharunderscore}{\kern0pt}mk{\isacharcolon}{\kern0pt}\ {\isachardoublequoteopen}n\ {\isacharequal}{\kern0pt}\ m\ {\isasymcirc}\isactrlsub c\ k{\isachardoublequoteclose}\isanewline
\ \ \isacommand{assume}\isamarkupfalse%
\ factorsthru{\isacharunderscore}{\kern0pt}mk{\isacharcolon}{\kern0pt}\ {\isachardoublequoteopen}x\ factorsthru\ {\isacharparenleft}{\kern0pt}m\ {\isasymcirc}\isactrlsub c\ k{\isacharparenright}{\kern0pt}{\isachardoublequoteclose}\isanewline
\ \ \isanewline
\ \ \isacommand{obtain}\isamarkupfalse%
\ a\ \isakeyword{where}\ a{\isacharunderscore}{\kern0pt}assms{\isacharcolon}{\kern0pt}\ {\isachardoublequoteopen}a\ {\isasymin}\isactrlsub c\ A\ {\isasymand}\ {\isacharparenleft}{\kern0pt}m\ {\isasymcirc}\isactrlsub c\ k{\isacharparenright}{\kern0pt}\ {\isasymcirc}\isactrlsub c\ a\ {\isacharequal}{\kern0pt}\ x{\isachardoublequoteclose}\isanewline
\ \ \ \ \isacommand{using}\isamarkupfalse%
\ assms{\isacharparenleft}{\kern0pt}{\isadigit{2}}{\isacharparenright}{\kern0pt}\ cfunc{\isacharunderscore}{\kern0pt}type{\isacharunderscore}{\kern0pt}def\ domain{\isacharunderscore}{\kern0pt}comp\ factors{\isacharunderscore}{\kern0pt}through{\isacharunderscore}{\kern0pt}def\ factorsthru{\isacharunderscore}{\kern0pt}mk\ k{\isacharunderscore}{\kern0pt}type\ m{\isacharunderscore}{\kern0pt}type\ \isacommand{by}\isamarkupfalse%
\ auto\isanewline
\ \ \isacommand{then}\isamarkupfalse%
\ \isacommand{show}\isamarkupfalse%
\ {\isachardoublequoteopen}x\ factorsthru\ m\ {\isachardoublequoteclose}\isanewline
\ \ \ \ \isacommand{unfolding}\isamarkupfalse%
\ factors{\isacharunderscore}{\kern0pt}through{\isacharunderscore}{\kern0pt}def\ \isanewline
\ \ \ \ \isacommand{using}\isamarkupfalse%
\ cfunc{\isacharunderscore}{\kern0pt}type{\isacharunderscore}{\kern0pt}def\ comp{\isacharunderscore}{\kern0pt}type\ k{\isacharunderscore}{\kern0pt}type\ m{\isacharunderscore}{\kern0pt}type\ comp{\isacharunderscore}{\kern0pt}associative\isanewline
\ \ \ \ \isacommand{by}\isamarkupfalse%
\ {\isacharparenleft}{\kern0pt}rule{\isacharunderscore}{\kern0pt}tac\ x{\isacharequal}{\kern0pt}{\isachardoublequoteopen}k\ {\isasymcirc}\isactrlsub c\ a{\isachardoublequoteclose}\ \isakeyword{in}\ exI{\isacharcomma}{\kern0pt}\ auto{\isacharparenright}{\kern0pt}\isanewline
\isacommand{qed}\isamarkupfalse%
%
\endisatagproof
{\isafoldproof}%
%
\isadelimproof
%
\endisadelimproof
%
\isadelimdocument
%
\endisadelimdocument
%
\isatagdocument
%
\isamarkupsubsection{Inverse Image%
}
\isamarkuptrue%
%
\endisatagdocument
{\isafolddocument}%
%
\isadelimdocument
%
\endisadelimdocument
%
\begin{isamarkuptext}%
The definition below corresponds to a definition given by a diagram between Definition 2.1.37 and Proposition 2.1.38 in Halvorson.%
\end{isamarkuptext}\isamarkuptrue%
\isacommand{definition}\isamarkupfalse%
\ inverse{\isacharunderscore}{\kern0pt}image\ {\isacharcolon}{\kern0pt}{\isacharcolon}{\kern0pt}\ {\isachardoublequoteopen}cfunc\ {\isasymRightarrow}\ cset\ {\isasymRightarrow}\ cfunc\ {\isasymRightarrow}\ cset{\isachardoublequoteclose}\ {\isacharparenleft}{\kern0pt}{\isachardoublequoteopen}{\isacharunderscore}{\kern0pt}\isactrlsup {\isacharminus}{\kern0pt}\isactrlsup {\isadigit{1}}{\isasymlparr}{\isacharunderscore}{\kern0pt}{\isasymrparr}\isactrlbsub {\isacharunderscore}{\kern0pt}\isactrlesub {\isachardoublequoteclose}\ {\isacharbrackleft}{\kern0pt}{\isadigit{1}}{\isadigit{0}}{\isadigit{1}}{\isacharcomma}{\kern0pt}{\isadigit{0}}{\isacharcomma}{\kern0pt}{\isadigit{0}}{\isacharbrackright}{\kern0pt}{\isadigit{1}}{\isadigit{0}}{\isadigit{0}}{\isacharparenright}{\kern0pt}\ \isakeyword{where}\isanewline
\ \ {\isachardoublequoteopen}inverse{\isacharunderscore}{\kern0pt}image\ f\ B\ m\ {\isacharequal}{\kern0pt}\ {\isacharparenleft}{\kern0pt}SOME\ A{\isachardot}{\kern0pt}\ {\isasymexists}\ X\ Y\ k{\isachardot}{\kern0pt}\ f\ {\isacharcolon}{\kern0pt}\ X\ {\isasymrightarrow}\ Y\ {\isasymand}\ m\ {\isacharcolon}{\kern0pt}\ B\ {\isasymrightarrow}\ Y\ {\isasymand}\ monomorphism\ m\ {\isasymand}\isanewline
\ \ \ \ equalizer\ A\ k\ {\isacharparenleft}{\kern0pt}f\ {\isasymcirc}\isactrlsub c\ left{\isacharunderscore}{\kern0pt}cart{\isacharunderscore}{\kern0pt}proj\ X\ B{\isacharparenright}{\kern0pt}\ {\isacharparenleft}{\kern0pt}m\ {\isasymcirc}\isactrlsub c\ right{\isacharunderscore}{\kern0pt}cart{\isacharunderscore}{\kern0pt}proj\ X\ B{\isacharparenright}{\kern0pt}{\isacharparenright}{\kern0pt}{\isachardoublequoteclose}\isanewline
\isanewline
\isacommand{lemma}\isamarkupfalse%
\ inverse{\isacharunderscore}{\kern0pt}image{\isacharunderscore}{\kern0pt}is{\isacharunderscore}{\kern0pt}equalizer{\isacharcolon}{\kern0pt}\isanewline
\ \ \isakeyword{assumes}\ {\isachardoublequoteopen}m\ {\isacharcolon}{\kern0pt}\ B\ {\isasymrightarrow}\ Y{\isachardoublequoteclose}\ {\isachardoublequoteopen}f\ {\isacharcolon}{\kern0pt}\ X\ {\isasymrightarrow}\ Y{\isachardoublequoteclose}\ {\isachardoublequoteopen}monomorphism\ m{\isachardoublequoteclose}\isanewline
\ \ \isakeyword{shows}\ {\isachardoublequoteopen}{\isasymexists}k{\isachardot}{\kern0pt}\ equalizer\ {\isacharparenleft}{\kern0pt}f\isactrlsup {\isacharminus}{\kern0pt}\isactrlsup {\isadigit{1}}{\isasymlparr}B{\isasymrparr}\isactrlbsub m\isactrlesub {\isacharparenright}{\kern0pt}\ k\ {\isacharparenleft}{\kern0pt}f\ {\isasymcirc}\isactrlsub c\ left{\isacharunderscore}{\kern0pt}cart{\isacharunderscore}{\kern0pt}proj\ X\ B{\isacharparenright}{\kern0pt}\ {\isacharparenleft}{\kern0pt}m\ {\isasymcirc}\isactrlsub c\ right{\isacharunderscore}{\kern0pt}cart{\isacharunderscore}{\kern0pt}proj\ X\ B{\isacharparenright}{\kern0pt}{\isachardoublequoteclose}\isanewline
%
\isadelimproof
%
\endisadelimproof
%
\isatagproof
\isacommand{proof}\isamarkupfalse%
\ {\isacharminus}{\kern0pt}\isanewline
\ \ \isacommand{obtain}\isamarkupfalse%
\ A\ k\ \isakeyword{where}\ {\isachardoublequoteopen}equalizer\ A\ k\ {\isacharparenleft}{\kern0pt}f\ {\isasymcirc}\isactrlsub c\ left{\isacharunderscore}{\kern0pt}cart{\isacharunderscore}{\kern0pt}proj\ X\ B{\isacharparenright}{\kern0pt}\ {\isacharparenleft}{\kern0pt}m\ {\isasymcirc}\isactrlsub c\ right{\isacharunderscore}{\kern0pt}cart{\isacharunderscore}{\kern0pt}proj\ X\ B{\isacharparenright}{\kern0pt}{\isachardoublequoteclose}\isanewline
\ \ \ \ \isacommand{by}\isamarkupfalse%
\ {\isacharparenleft}{\kern0pt}meson\ assms{\isacharparenleft}{\kern0pt}{\isadigit{1}}{\isacharcomma}{\kern0pt}{\isadigit{2}}{\isacharparenright}{\kern0pt}\ comp{\isacharunderscore}{\kern0pt}type\ equalizer{\isacharunderscore}{\kern0pt}exists\ left{\isacharunderscore}{\kern0pt}cart{\isacharunderscore}{\kern0pt}proj{\isacharunderscore}{\kern0pt}type\ right{\isacharunderscore}{\kern0pt}cart{\isacharunderscore}{\kern0pt}proj{\isacharunderscore}{\kern0pt}type{\isacharparenright}{\kern0pt}\isanewline
\ \ \isacommand{then}\isamarkupfalse%
\ \isacommand{show}\isamarkupfalse%
\ {\isachardoublequoteopen}{\isasymexists}k{\isachardot}{\kern0pt}\ equalizer\ {\isacharparenleft}{\kern0pt}inverse{\isacharunderscore}{\kern0pt}image\ f\ B\ m{\isacharparenright}{\kern0pt}\ k\ {\isacharparenleft}{\kern0pt}f\ {\isasymcirc}\isactrlsub c\ left{\isacharunderscore}{\kern0pt}cart{\isacharunderscore}{\kern0pt}proj\ X\ B{\isacharparenright}{\kern0pt}\ {\isacharparenleft}{\kern0pt}m\ {\isasymcirc}\isactrlsub c\ right{\isacharunderscore}{\kern0pt}cart{\isacharunderscore}{\kern0pt}proj\ X\ B{\isacharparenright}{\kern0pt}{\isachardoublequoteclose}\isanewline
\ \ \ \ \isacommand{unfolding}\isamarkupfalse%
\ inverse{\isacharunderscore}{\kern0pt}image{\isacharunderscore}{\kern0pt}def\ \isacommand{using}\isamarkupfalse%
\ assms\ cfunc{\isacharunderscore}{\kern0pt}type{\isacharunderscore}{\kern0pt}def\ \isacommand{by}\isamarkupfalse%
\ {\isacharparenleft}{\kern0pt}rule{\isacharunderscore}{\kern0pt}tac\ someI{\isadigit{2}}{\isacharunderscore}{\kern0pt}ex{\isacharcomma}{\kern0pt}\ auto{\isacharparenright}{\kern0pt}\isanewline
\isacommand{qed}\isamarkupfalse%
%
\endisatagproof
{\isafoldproof}%
%
\isadelimproof
\isanewline
%
\endisadelimproof
\isanewline
\isacommand{definition}\isamarkupfalse%
\ inverse{\isacharunderscore}{\kern0pt}image{\isacharunderscore}{\kern0pt}mapping\ {\isacharcolon}{\kern0pt}{\isacharcolon}{\kern0pt}\ {\isachardoublequoteopen}cfunc\ {\isasymRightarrow}\ cset\ {\isasymRightarrow}\ cfunc\ {\isasymRightarrow}\ cfunc{\isachardoublequoteclose}\ \ \isakeyword{where}\isanewline
\ \ {\isachardoublequoteopen}inverse{\isacharunderscore}{\kern0pt}image{\isacharunderscore}{\kern0pt}mapping\ f\ B\ m\ {\isacharequal}{\kern0pt}\ {\isacharparenleft}{\kern0pt}SOME\ k{\isachardot}{\kern0pt}\ {\isasymexists}\ X\ Y{\isachardot}{\kern0pt}\ f\ {\isacharcolon}{\kern0pt}\ X\ {\isasymrightarrow}\ Y\ {\isasymand}\ m\ {\isacharcolon}{\kern0pt}\ B\ {\isasymrightarrow}\ Y\ {\isasymand}\ monomorphism\ m\ {\isasymand}\isanewline
\ \ \ \ equalizer\ {\isacharparenleft}{\kern0pt}inverse{\isacharunderscore}{\kern0pt}image\ f\ B\ m{\isacharparenright}{\kern0pt}\ k\ {\isacharparenleft}{\kern0pt}f\ {\isasymcirc}\isactrlsub c\ left{\isacharunderscore}{\kern0pt}cart{\isacharunderscore}{\kern0pt}proj\ X\ B{\isacharparenright}{\kern0pt}\ {\isacharparenleft}{\kern0pt}m\ {\isasymcirc}\isactrlsub c\ right{\isacharunderscore}{\kern0pt}cart{\isacharunderscore}{\kern0pt}proj\ X\ B{\isacharparenright}{\kern0pt}{\isacharparenright}{\kern0pt}{\isachardoublequoteclose}\isanewline
\isanewline
\isacommand{lemma}\isamarkupfalse%
\ inverse{\isacharunderscore}{\kern0pt}image{\isacharunderscore}{\kern0pt}is{\isacharunderscore}{\kern0pt}equalizer{\isadigit{2}}{\isacharcolon}{\kern0pt}\isanewline
\ \ \isakeyword{assumes}\ {\isachardoublequoteopen}m\ {\isacharcolon}{\kern0pt}\ B\ {\isasymrightarrow}\ Y{\isachardoublequoteclose}\ {\isachardoublequoteopen}f\ {\isacharcolon}{\kern0pt}\ X\ {\isasymrightarrow}\ Y{\isachardoublequoteclose}\ {\isachardoublequoteopen}monomorphism\ m{\isachardoublequoteclose}\isanewline
\ \ \isakeyword{shows}\ {\isachardoublequoteopen}equalizer\ {\isacharparenleft}{\kern0pt}inverse{\isacharunderscore}{\kern0pt}image\ f\ B\ m{\isacharparenright}{\kern0pt}\ {\isacharparenleft}{\kern0pt}inverse{\isacharunderscore}{\kern0pt}image{\isacharunderscore}{\kern0pt}mapping\ f\ B\ m{\isacharparenright}{\kern0pt}\ {\isacharparenleft}{\kern0pt}f\ {\isasymcirc}\isactrlsub c\ left{\isacharunderscore}{\kern0pt}cart{\isacharunderscore}{\kern0pt}proj\ X\ B{\isacharparenright}{\kern0pt}\ {\isacharparenleft}{\kern0pt}m\ {\isasymcirc}\isactrlsub c\ right{\isacharunderscore}{\kern0pt}cart{\isacharunderscore}{\kern0pt}proj\ X\ B{\isacharparenright}{\kern0pt}{\isachardoublequoteclose}\isanewline
%
\isadelimproof
%
\endisadelimproof
%
\isatagproof
\isacommand{proof}\isamarkupfalse%
\ {\isacharminus}{\kern0pt}\isanewline
\ \ \isacommand{obtain}\isamarkupfalse%
\ k\ \isakeyword{where}\ {\isachardoublequoteopen}equalizer\ {\isacharparenleft}{\kern0pt}inverse{\isacharunderscore}{\kern0pt}image\ f\ B\ m{\isacharparenright}{\kern0pt}\ k\ {\isacharparenleft}{\kern0pt}f\ {\isasymcirc}\isactrlsub c\ left{\isacharunderscore}{\kern0pt}cart{\isacharunderscore}{\kern0pt}proj\ X\ B{\isacharparenright}{\kern0pt}\ {\isacharparenleft}{\kern0pt}m\ {\isasymcirc}\isactrlsub c\ right{\isacharunderscore}{\kern0pt}cart{\isacharunderscore}{\kern0pt}proj\ X\ B{\isacharparenright}{\kern0pt}{\isachardoublequoteclose}\isanewline
\ \ \ \ \isacommand{using}\isamarkupfalse%
\ assms\ inverse{\isacharunderscore}{\kern0pt}image{\isacharunderscore}{\kern0pt}is{\isacharunderscore}{\kern0pt}equalizer\ \isacommand{by}\isamarkupfalse%
\ blast\isanewline
\ \ \isacommand{then}\isamarkupfalse%
\ \isacommand{have}\isamarkupfalse%
\ {\isachardoublequoteopen}{\isasymexists}\ X\ Y{\isachardot}{\kern0pt}\ f\ {\isacharcolon}{\kern0pt}\ X\ {\isasymrightarrow}\ Y\ {\isasymand}\ m\ {\isacharcolon}{\kern0pt}\ B\ {\isasymrightarrow}\ Y\ {\isasymand}\ monomorphism\ m\ {\isasymand}\isanewline
\ \ \ \ equalizer\ {\isacharparenleft}{\kern0pt}inverse{\isacharunderscore}{\kern0pt}image\ f\ B\ m{\isacharparenright}{\kern0pt}\ {\isacharparenleft}{\kern0pt}inverse{\isacharunderscore}{\kern0pt}image{\isacharunderscore}{\kern0pt}mapping\ f\ B\ m{\isacharparenright}{\kern0pt}\ {\isacharparenleft}{\kern0pt}f\ {\isasymcirc}\isactrlsub c\ left{\isacharunderscore}{\kern0pt}cart{\isacharunderscore}{\kern0pt}proj\ X\ B{\isacharparenright}{\kern0pt}\ {\isacharparenleft}{\kern0pt}m\ {\isasymcirc}\isactrlsub c\ right{\isacharunderscore}{\kern0pt}cart{\isacharunderscore}{\kern0pt}proj\ X\ B{\isacharparenright}{\kern0pt}{\isachardoublequoteclose}\isanewline
\ \ \ \ \isacommand{unfolding}\isamarkupfalse%
\ inverse{\isacharunderscore}{\kern0pt}image{\isacharunderscore}{\kern0pt}mapping{\isacharunderscore}{\kern0pt}def\ \isacommand{using}\isamarkupfalse%
\ assms\ \isacommand{by}\isamarkupfalse%
\ {\isacharparenleft}{\kern0pt}rule{\isacharunderscore}{\kern0pt}tac\ someI{\isacharunderscore}{\kern0pt}ex{\isacharcomma}{\kern0pt}\ auto{\isacharparenright}{\kern0pt}\isanewline
\ \ \isacommand{then}\isamarkupfalse%
\ \isacommand{show}\isamarkupfalse%
\ {\isachardoublequoteopen}equalizer\ {\isacharparenleft}{\kern0pt}inverse{\isacharunderscore}{\kern0pt}image\ f\ B\ m{\isacharparenright}{\kern0pt}\ {\isacharparenleft}{\kern0pt}inverse{\isacharunderscore}{\kern0pt}image{\isacharunderscore}{\kern0pt}mapping\ f\ B\ m{\isacharparenright}{\kern0pt}\ {\isacharparenleft}{\kern0pt}f\ {\isasymcirc}\isactrlsub c\ left{\isacharunderscore}{\kern0pt}cart{\isacharunderscore}{\kern0pt}proj\ X\ B{\isacharparenright}{\kern0pt}\ {\isacharparenleft}{\kern0pt}m\ {\isasymcirc}\isactrlsub c\ right{\isacharunderscore}{\kern0pt}cart{\isacharunderscore}{\kern0pt}proj\ X\ B{\isacharparenright}{\kern0pt}{\isachardoublequoteclose}\isanewline
\ \ \ \ \isacommand{using}\isamarkupfalse%
\ assms{\isacharparenleft}{\kern0pt}{\isadigit{2}}{\isacharparenright}{\kern0pt}\ cfunc{\isacharunderscore}{\kern0pt}type{\isacharunderscore}{\kern0pt}def\ \isacommand{by}\isamarkupfalse%
\ auto\isanewline
\isacommand{qed}\isamarkupfalse%
%
\endisatagproof
{\isafoldproof}%
%
\isadelimproof
\isanewline
%
\endisadelimproof
\isanewline
\isacommand{lemma}\isamarkupfalse%
\ inverse{\isacharunderscore}{\kern0pt}image{\isacharunderscore}{\kern0pt}mapping{\isacharunderscore}{\kern0pt}type{\isacharbrackleft}{\kern0pt}type{\isacharunderscore}{\kern0pt}rule{\isacharbrackright}{\kern0pt}{\isacharcolon}{\kern0pt}\isanewline
\ \ \isakeyword{assumes}\ {\isachardoublequoteopen}m\ {\isacharcolon}{\kern0pt}\ B\ {\isasymrightarrow}\ Y{\isachardoublequoteclose}\ {\isachardoublequoteopen}f\ {\isacharcolon}{\kern0pt}\ X\ {\isasymrightarrow}\ Y{\isachardoublequoteclose}\ {\isachardoublequoteopen}monomorphism\ m{\isachardoublequoteclose}\isanewline
\ \ \isakeyword{shows}\ {\isachardoublequoteopen}inverse{\isacharunderscore}{\kern0pt}image{\isacharunderscore}{\kern0pt}mapping\ f\ B\ m\ {\isacharcolon}{\kern0pt}\ {\isacharparenleft}{\kern0pt}inverse{\isacharunderscore}{\kern0pt}image\ f\ B\ m{\isacharparenright}{\kern0pt}\ {\isasymrightarrow}\ X\ {\isasymtimes}\isactrlsub c\ B{\isachardoublequoteclose}\isanewline
%
\isadelimproof
\ \ %
\endisadelimproof
%
\isatagproof
\isacommand{using}\isamarkupfalse%
\ assms\ cfunc{\isacharunderscore}{\kern0pt}type{\isacharunderscore}{\kern0pt}def\ domain{\isacharunderscore}{\kern0pt}comp\ equalizer{\isacharunderscore}{\kern0pt}def\ inverse{\isacharunderscore}{\kern0pt}image{\isacharunderscore}{\kern0pt}is{\isacharunderscore}{\kern0pt}equalizer{\isadigit{2}}\ left{\isacharunderscore}{\kern0pt}cart{\isacharunderscore}{\kern0pt}proj{\isacharunderscore}{\kern0pt}type\ \isacommand{by}\isamarkupfalse%
\ auto%
\endisatagproof
{\isafoldproof}%
%
\isadelimproof
\isanewline
%
\endisadelimproof
\isanewline
\isacommand{lemma}\isamarkupfalse%
\ inverse{\isacharunderscore}{\kern0pt}image{\isacharunderscore}{\kern0pt}mapping{\isacharunderscore}{\kern0pt}eq{\isacharcolon}{\kern0pt}\isanewline
\ \ \isakeyword{assumes}\ {\isachardoublequoteopen}m\ {\isacharcolon}{\kern0pt}\ B\ {\isasymrightarrow}\ Y{\isachardoublequoteclose}\ {\isachardoublequoteopen}f\ {\isacharcolon}{\kern0pt}\ X\ {\isasymrightarrow}\ Y{\isachardoublequoteclose}\ {\isachardoublequoteopen}monomorphism\ m{\isachardoublequoteclose}\isanewline
\ \ \isakeyword{shows}\ {\isachardoublequoteopen}f\ {\isasymcirc}\isactrlsub c\ left{\isacharunderscore}{\kern0pt}cart{\isacharunderscore}{\kern0pt}proj\ X\ B\ {\isasymcirc}\isactrlsub c\ inverse{\isacharunderscore}{\kern0pt}image{\isacharunderscore}{\kern0pt}mapping\ f\ B\ m\isanewline
\ \ \ \ \ \ {\isacharequal}{\kern0pt}\ m\ {\isasymcirc}\isactrlsub c\ right{\isacharunderscore}{\kern0pt}cart{\isacharunderscore}{\kern0pt}proj\ X\ B\ {\isasymcirc}\isactrlsub c\ inverse{\isacharunderscore}{\kern0pt}image{\isacharunderscore}{\kern0pt}mapping\ f\ B\ m{\isachardoublequoteclose}\isanewline
%
\isadelimproof
\ \ %
\endisadelimproof
%
\isatagproof
\isacommand{using}\isamarkupfalse%
\ assms\ cfunc{\isacharunderscore}{\kern0pt}type{\isacharunderscore}{\kern0pt}def\ comp{\isacharunderscore}{\kern0pt}associative\ equalizer{\isacharunderscore}{\kern0pt}def\ inverse{\isacharunderscore}{\kern0pt}image{\isacharunderscore}{\kern0pt}is{\isacharunderscore}{\kern0pt}equalizer{\isadigit{2}}\isanewline
\ \ \isacommand{by}\isamarkupfalse%
\ {\isacharparenleft}{\kern0pt}typecheck{\isacharunderscore}{\kern0pt}cfuncs{\isacharcomma}{\kern0pt}\ smt\ {\isacharparenleft}{\kern0pt}verit{\isacharparenright}{\kern0pt}{\isacharparenright}{\kern0pt}%
\endisatagproof
{\isafoldproof}%
%
\isadelimproof
\isanewline
%
\endisadelimproof
\isanewline
\isacommand{lemma}\isamarkupfalse%
\ inverse{\isacharunderscore}{\kern0pt}image{\isacharunderscore}{\kern0pt}mapping{\isacharunderscore}{\kern0pt}monomorphism{\isacharcolon}{\kern0pt}\isanewline
\ \ \isakeyword{assumes}\ {\isachardoublequoteopen}m\ {\isacharcolon}{\kern0pt}\ B\ {\isasymrightarrow}\ Y{\isachardoublequoteclose}\ {\isachardoublequoteopen}f\ {\isacharcolon}{\kern0pt}\ X\ {\isasymrightarrow}\ Y{\isachardoublequoteclose}\ {\isachardoublequoteopen}monomorphism\ m{\isachardoublequoteclose}\isanewline
\ \ \isakeyword{shows}\ {\isachardoublequoteopen}monomorphism\ {\isacharparenleft}{\kern0pt}inverse{\isacharunderscore}{\kern0pt}image{\isacharunderscore}{\kern0pt}mapping\ f\ B\ m{\isacharparenright}{\kern0pt}{\isachardoublequoteclose}\isanewline
%
\isadelimproof
\ \ %
\endisadelimproof
%
\isatagproof
\isacommand{using}\isamarkupfalse%
\ assms\ equalizer{\isacharunderscore}{\kern0pt}is{\isacharunderscore}{\kern0pt}monomorphism\ inverse{\isacharunderscore}{\kern0pt}image{\isacharunderscore}{\kern0pt}is{\isacharunderscore}{\kern0pt}equalizer{\isadigit{2}}\ \isacommand{by}\isamarkupfalse%
\ blast%
\endisatagproof
{\isafoldproof}%
%
\isadelimproof
%
\endisadelimproof
%
\begin{isamarkuptext}%
The lemma below is the dual of Proposition 2.1.38 in Halvorson.%
\end{isamarkuptext}\isamarkuptrue%
\isacommand{lemma}\isamarkupfalse%
\ inverse{\isacharunderscore}{\kern0pt}image{\isacharunderscore}{\kern0pt}monomorphism{\isacharcolon}{\kern0pt}\isanewline
\ \ \isakeyword{assumes}\ {\isachardoublequoteopen}m\ {\isacharcolon}{\kern0pt}\ B\ {\isasymrightarrow}\ Y{\isachardoublequoteclose}\ {\isachardoublequoteopen}f\ {\isacharcolon}{\kern0pt}\ X\ {\isasymrightarrow}\ Y{\isachardoublequoteclose}\ {\isachardoublequoteopen}monomorphism\ m{\isachardoublequoteclose}\isanewline
\ \ \isakeyword{shows}\ {\isachardoublequoteopen}monomorphism\ {\isacharparenleft}{\kern0pt}left{\isacharunderscore}{\kern0pt}cart{\isacharunderscore}{\kern0pt}proj\ X\ B\ {\isasymcirc}\isactrlsub c\ inverse{\isacharunderscore}{\kern0pt}image{\isacharunderscore}{\kern0pt}mapping\ f\ B\ m{\isacharparenright}{\kern0pt}{\isachardoublequoteclose}\isanewline
%
\isadelimproof
\ \ %
\endisadelimproof
%
\isatagproof
\isacommand{using}\isamarkupfalse%
\ assms\isanewline
\isacommand{proof}\isamarkupfalse%
\ {\isacharparenleft}{\kern0pt}typecheck{\isacharunderscore}{\kern0pt}cfuncs{\isacharcomma}{\kern0pt}\ unfold\ monomorphism{\isacharunderscore}{\kern0pt}def{\isadigit{3}}{\isacharcomma}{\kern0pt}\ clarify{\isacharparenright}{\kern0pt}\isanewline
\ \ \isacommand{fix}\isamarkupfalse%
\ g\ h\ A\isanewline
\ \ \isacommand{assume}\isamarkupfalse%
\ g{\isacharunderscore}{\kern0pt}type{\isacharcolon}{\kern0pt}\ {\isachardoublequoteopen}g\ {\isacharcolon}{\kern0pt}\ A\ {\isasymrightarrow}\ {\isacharparenleft}{\kern0pt}f\isactrlsup {\isacharminus}{\kern0pt}\isactrlsup {\isadigit{1}}{\isasymlparr}B{\isasymrparr}\isactrlbsub m\isactrlesub {\isacharparenright}{\kern0pt}{\isachardoublequoteclose}\isanewline
\ \ \isacommand{assume}\isamarkupfalse%
\ h{\isacharunderscore}{\kern0pt}type{\isacharcolon}{\kern0pt}\ {\isachardoublequoteopen}h\ {\isacharcolon}{\kern0pt}\ A\ {\isasymrightarrow}\ {\isacharparenleft}{\kern0pt}f\isactrlsup {\isacharminus}{\kern0pt}\isactrlsup {\isadigit{1}}{\isasymlparr}B{\isasymrparr}\isactrlbsub m\isactrlesub {\isacharparenright}{\kern0pt}{\isachardoublequoteclose}\isanewline
\ \ \isacommand{assume}\isamarkupfalse%
\ left{\isacharunderscore}{\kern0pt}eq{\isacharcolon}{\kern0pt}\ {\isachardoublequoteopen}{\isacharparenleft}{\kern0pt}left{\isacharunderscore}{\kern0pt}cart{\isacharunderscore}{\kern0pt}proj\ X\ B\ {\isasymcirc}\isactrlsub c\ inverse{\isacharunderscore}{\kern0pt}image{\isacharunderscore}{\kern0pt}mapping\ f\ B\ m{\isacharparenright}{\kern0pt}\ {\isasymcirc}\isactrlsub c\ g\isanewline
\ \ \ \ {\isacharequal}{\kern0pt}\ {\isacharparenleft}{\kern0pt}left{\isacharunderscore}{\kern0pt}cart{\isacharunderscore}{\kern0pt}proj\ X\ B\ {\isasymcirc}\isactrlsub c\ inverse{\isacharunderscore}{\kern0pt}image{\isacharunderscore}{\kern0pt}mapping\ f\ B\ m{\isacharparenright}{\kern0pt}\ {\isasymcirc}\isactrlsub c\ h{\isachardoublequoteclose}\isanewline
\ \ \isacommand{then}\isamarkupfalse%
\ \isacommand{have}\isamarkupfalse%
\ {\isachardoublequoteopen}f\ {\isasymcirc}\isactrlsub c\ {\isacharparenleft}{\kern0pt}left{\isacharunderscore}{\kern0pt}cart{\isacharunderscore}{\kern0pt}proj\ X\ B\ {\isasymcirc}\isactrlsub c\ inverse{\isacharunderscore}{\kern0pt}image{\isacharunderscore}{\kern0pt}mapping\ f\ B\ m{\isacharparenright}{\kern0pt}\ {\isasymcirc}\isactrlsub c\ g\isanewline
\ \ \ \ {\isacharequal}{\kern0pt}\ f\ {\isasymcirc}\isactrlsub c\ {\isacharparenleft}{\kern0pt}left{\isacharunderscore}{\kern0pt}cart{\isacharunderscore}{\kern0pt}proj\ X\ B\ {\isasymcirc}\isactrlsub c\ inverse{\isacharunderscore}{\kern0pt}image{\isacharunderscore}{\kern0pt}mapping\ f\ B\ m{\isacharparenright}{\kern0pt}\ {\isasymcirc}\isactrlsub c\ h{\isachardoublequoteclose}\isanewline
\ \ \ \ \isacommand{by}\isamarkupfalse%
\ auto\isanewline
\ \ \isacommand{then}\isamarkupfalse%
\ \isacommand{have}\isamarkupfalse%
\ {\isachardoublequoteopen}m\ {\isasymcirc}\isactrlsub c\ {\isacharparenleft}{\kern0pt}right{\isacharunderscore}{\kern0pt}cart{\isacharunderscore}{\kern0pt}proj\ X\ B\ {\isasymcirc}\isactrlsub c\ inverse{\isacharunderscore}{\kern0pt}image{\isacharunderscore}{\kern0pt}mapping\ f\ B\ m{\isacharparenright}{\kern0pt}\ {\isasymcirc}\isactrlsub c\ g\isanewline
\ \ \ \ {\isacharequal}{\kern0pt}\ m\ {\isasymcirc}\isactrlsub c\ {\isacharparenleft}{\kern0pt}right{\isacharunderscore}{\kern0pt}cart{\isacharunderscore}{\kern0pt}proj\ X\ B\ {\isasymcirc}\isactrlsub c\ inverse{\isacharunderscore}{\kern0pt}image{\isacharunderscore}{\kern0pt}mapping\ f\ B\ m{\isacharparenright}{\kern0pt}\ {\isasymcirc}\isactrlsub c\ h{\isachardoublequoteclose}\isanewline
\ \ \ \ \isacommand{using}\isamarkupfalse%
\ assms\ g{\isacharunderscore}{\kern0pt}type\ h{\isacharunderscore}{\kern0pt}type\isanewline
\ \ \ \ \isacommand{by}\isamarkupfalse%
\ {\isacharparenleft}{\kern0pt}typecheck{\isacharunderscore}{\kern0pt}cfuncs{\isacharcomma}{\kern0pt}\ smt\ cfunc{\isacharunderscore}{\kern0pt}type{\isacharunderscore}{\kern0pt}def\ codomain{\isacharunderscore}{\kern0pt}comp\ comp{\isacharunderscore}{\kern0pt}associative\ domain{\isacharunderscore}{\kern0pt}comp\ inverse{\isacharunderscore}{\kern0pt}image{\isacharunderscore}{\kern0pt}mapping{\isacharunderscore}{\kern0pt}eq\ left{\isacharunderscore}{\kern0pt}cart{\isacharunderscore}{\kern0pt}proj{\isacharunderscore}{\kern0pt}type{\isacharparenright}{\kern0pt}\ \isanewline
\ \ \isacommand{then}\isamarkupfalse%
\ \isacommand{have}\isamarkupfalse%
\ right{\isacharunderscore}{\kern0pt}eq{\isacharcolon}{\kern0pt}\ {\isachardoublequoteopen}{\isacharparenleft}{\kern0pt}right{\isacharunderscore}{\kern0pt}cart{\isacharunderscore}{\kern0pt}proj\ X\ B\ {\isasymcirc}\isactrlsub c\ inverse{\isacharunderscore}{\kern0pt}image{\isacharunderscore}{\kern0pt}mapping\ f\ B\ m{\isacharparenright}{\kern0pt}\ {\isasymcirc}\isactrlsub c\ g\isanewline
\ \ \ \ {\isacharequal}{\kern0pt}\ {\isacharparenleft}{\kern0pt}right{\isacharunderscore}{\kern0pt}cart{\isacharunderscore}{\kern0pt}proj\ X\ B\ {\isasymcirc}\isactrlsub c\ inverse{\isacharunderscore}{\kern0pt}image{\isacharunderscore}{\kern0pt}mapping\ f\ B\ m{\isacharparenright}{\kern0pt}\ {\isasymcirc}\isactrlsub c\ h{\isachardoublequoteclose}\isanewline
\ \ \ \ \isacommand{using}\isamarkupfalse%
\ assms\ g{\isacharunderscore}{\kern0pt}type\ h{\isacharunderscore}{\kern0pt}type\ monomorphism{\isacharunderscore}{\kern0pt}def{\isadigit{3}}\ \isacommand{by}\isamarkupfalse%
\ {\isacharparenleft}{\kern0pt}typecheck{\isacharunderscore}{\kern0pt}cfuncs{\isacharcomma}{\kern0pt}\ auto{\isacharparenright}{\kern0pt}\isanewline
\ \ \isacommand{then}\isamarkupfalse%
\ \isacommand{have}\isamarkupfalse%
\ {\isachardoublequoteopen}inverse{\isacharunderscore}{\kern0pt}image{\isacharunderscore}{\kern0pt}mapping\ f\ B\ m\ {\isasymcirc}\isactrlsub c\ g\ {\isacharequal}{\kern0pt}\ inverse{\isacharunderscore}{\kern0pt}image{\isacharunderscore}{\kern0pt}mapping\ f\ B\ m\ {\isasymcirc}\isactrlsub c\ h{\isachardoublequoteclose}\isanewline
\ \ \ \ \isacommand{using}\isamarkupfalse%
\ assms\ g{\isacharunderscore}{\kern0pt}type\ h{\isacharunderscore}{\kern0pt}type\ cfunc{\isacharunderscore}{\kern0pt}type{\isacharunderscore}{\kern0pt}def\ comp{\isacharunderscore}{\kern0pt}associative\ left{\isacharunderscore}{\kern0pt}eq\ left{\isacharunderscore}{\kern0pt}cart{\isacharunderscore}{\kern0pt}proj{\isacharunderscore}{\kern0pt}type\ right{\isacharunderscore}{\kern0pt}cart{\isacharunderscore}{\kern0pt}proj{\isacharunderscore}{\kern0pt}type\isanewline
\ \ \ \ \isacommand{by}\isamarkupfalse%
\ {\isacharparenleft}{\kern0pt}typecheck{\isacharunderscore}{\kern0pt}cfuncs{\isacharcomma}{\kern0pt}\ subst\ cart{\isacharunderscore}{\kern0pt}prod{\isacharunderscore}{\kern0pt}eq{\isacharcomma}{\kern0pt}\ auto{\isacharparenright}{\kern0pt}\isanewline
\ \ \isacommand{then}\isamarkupfalse%
\ \isacommand{show}\isamarkupfalse%
\ {\isachardoublequoteopen}g\ {\isacharequal}{\kern0pt}\ h{\isachardoublequoteclose}\isanewline
\ \ \ \ \isacommand{using}\isamarkupfalse%
\ assms\ g{\isacharunderscore}{\kern0pt}type\ h{\isacharunderscore}{\kern0pt}type\ inverse{\isacharunderscore}{\kern0pt}image{\isacharunderscore}{\kern0pt}mapping{\isacharunderscore}{\kern0pt}monomorphism\ inverse{\isacharunderscore}{\kern0pt}image{\isacharunderscore}{\kern0pt}mapping{\isacharunderscore}{\kern0pt}type\ monomorphism{\isacharunderscore}{\kern0pt}def{\isadigit{3}}\isanewline
\ \ \ \ \isacommand{by}\isamarkupfalse%
\ blast\isanewline
\isacommand{qed}\isamarkupfalse%
%
\endisatagproof
{\isafoldproof}%
%
\isadelimproof
\isanewline
%
\endisadelimproof
\isanewline
\isacommand{definition}\isamarkupfalse%
\ inverse{\isacharunderscore}{\kern0pt}image{\isacharunderscore}{\kern0pt}subobject{\isacharunderscore}{\kern0pt}mapping\ {\isacharcolon}{\kern0pt}{\isacharcolon}{\kern0pt}\ {\isachardoublequoteopen}cfunc\ {\isasymRightarrow}\ cset\ {\isasymRightarrow}\ cfunc\ {\isasymRightarrow}\ cfunc{\isachardoublequoteclose}\ {\isacharparenleft}{\kern0pt}{\isachardoublequoteopen}{\isacharbrackleft}{\kern0pt}{\isacharunderscore}{\kern0pt}\isactrlsup {\isacharminus}{\kern0pt}\isactrlsup {\isadigit{1}}{\isasymlparr}{\isacharunderscore}{\kern0pt}{\isasymrparr}\isactrlbsub {\isacharunderscore}{\kern0pt}\isactrlesub {\isacharbrackright}{\kern0pt}map{\isachardoublequoteclose}\ {\isacharbrackleft}{\kern0pt}{\isadigit{1}}{\isadigit{0}}{\isadigit{1}}{\isacharcomma}{\kern0pt}{\isadigit{0}}{\isacharcomma}{\kern0pt}{\isadigit{0}}{\isacharbrackright}{\kern0pt}{\isadigit{1}}{\isadigit{0}}{\isadigit{0}}{\isacharparenright}{\kern0pt}\ \isakeyword{where}\isanewline
\ \ {\isachardoublequoteopen}{\isacharbrackleft}{\kern0pt}f\isactrlsup {\isacharminus}{\kern0pt}\isactrlsup {\isadigit{1}}{\isasymlparr}B{\isasymrparr}\isactrlbsub m\isactrlesub {\isacharbrackright}{\kern0pt}map\ {\isacharequal}{\kern0pt}\ left{\isacharunderscore}{\kern0pt}cart{\isacharunderscore}{\kern0pt}proj\ {\isacharparenleft}{\kern0pt}domain\ f{\isacharparenright}{\kern0pt}\ B\ {\isasymcirc}\isactrlsub c\ inverse{\isacharunderscore}{\kern0pt}image{\isacharunderscore}{\kern0pt}mapping\ f\ B\ m{\isachardoublequoteclose}\isanewline
\isanewline
\isacommand{lemma}\isamarkupfalse%
\ inverse{\isacharunderscore}{\kern0pt}image{\isacharunderscore}{\kern0pt}subobject{\isacharunderscore}{\kern0pt}mapping{\isacharunderscore}{\kern0pt}def{\isadigit{2}}{\isacharcolon}{\kern0pt}\isanewline
\ \ \isakeyword{assumes}\ {\isachardoublequoteopen}f\ {\isacharcolon}{\kern0pt}\ X\ {\isasymrightarrow}\ Y{\isachardoublequoteclose}\isanewline
\ \ \isakeyword{shows}\ {\isachardoublequoteopen}{\isacharbrackleft}{\kern0pt}f\isactrlsup {\isacharminus}{\kern0pt}\isactrlsup {\isadigit{1}}{\isasymlparr}B{\isasymrparr}\isactrlbsub m\isactrlesub {\isacharbrackright}{\kern0pt}map\ {\isacharequal}{\kern0pt}\ left{\isacharunderscore}{\kern0pt}cart{\isacharunderscore}{\kern0pt}proj\ X\ B\ {\isasymcirc}\isactrlsub c\ inverse{\isacharunderscore}{\kern0pt}image{\isacharunderscore}{\kern0pt}mapping\ f\ B\ m{\isachardoublequoteclose}\isanewline
%
\isadelimproof
\ \ %
\endisadelimproof
%
\isatagproof
\isacommand{using}\isamarkupfalse%
\ assms\ \isacommand{unfolding}\isamarkupfalse%
\ inverse{\isacharunderscore}{\kern0pt}image{\isacharunderscore}{\kern0pt}subobject{\isacharunderscore}{\kern0pt}mapping{\isacharunderscore}{\kern0pt}def\ cfunc{\isacharunderscore}{\kern0pt}type{\isacharunderscore}{\kern0pt}def\ \isacommand{by}\isamarkupfalse%
\ auto%
\endisatagproof
{\isafoldproof}%
%
\isadelimproof
\isanewline
%
\endisadelimproof
\isanewline
\isacommand{lemma}\isamarkupfalse%
\ inverse{\isacharunderscore}{\kern0pt}image{\isacharunderscore}{\kern0pt}subobject{\isacharunderscore}{\kern0pt}mapping{\isacharunderscore}{\kern0pt}type{\isacharbrackleft}{\kern0pt}type{\isacharunderscore}{\kern0pt}rule{\isacharbrackright}{\kern0pt}{\isacharcolon}{\kern0pt}\isanewline
\ \ \isakeyword{assumes}\ {\isachardoublequoteopen}f\ {\isacharcolon}{\kern0pt}\ X\ {\isasymrightarrow}\ Y{\isachardoublequoteclose}\ {\isachardoublequoteopen}m\ {\isacharcolon}{\kern0pt}\ B\ {\isasymrightarrow}\ Y{\isachardoublequoteclose}\ {\isachardoublequoteopen}monomorphism\ m{\isachardoublequoteclose}\isanewline
\ \ \isakeyword{shows}\ {\isachardoublequoteopen}{\isacharbrackleft}{\kern0pt}f\isactrlsup {\isacharminus}{\kern0pt}\isactrlsup {\isadigit{1}}{\isasymlparr}B{\isasymrparr}\isactrlbsub m\isactrlesub {\isacharbrackright}{\kern0pt}map\ {\isacharcolon}{\kern0pt}\ f\isactrlsup {\isacharminus}{\kern0pt}\isactrlsup {\isadigit{1}}{\isasymlparr}B{\isasymrparr}\isactrlbsub m\isactrlesub \ {\isasymrightarrow}\ X{\isachardoublequoteclose}\isanewline
%
\isadelimproof
\ \ %
\endisadelimproof
%
\isatagproof
\isacommand{using}\isamarkupfalse%
\ assms\ \isacommand{by}\isamarkupfalse%
\ {\isacharparenleft}{\kern0pt}unfold\ inverse{\isacharunderscore}{\kern0pt}image{\isacharunderscore}{\kern0pt}subobject{\isacharunderscore}{\kern0pt}mapping{\isacharunderscore}{\kern0pt}def{\isadigit{2}}{\isacharcomma}{\kern0pt}\ typecheck{\isacharunderscore}{\kern0pt}cfuncs{\isacharparenright}{\kern0pt}%
\endisatagproof
{\isafoldproof}%
%
\isadelimproof
\isanewline
%
\endisadelimproof
\isanewline
\isacommand{lemma}\isamarkupfalse%
\ inverse{\isacharunderscore}{\kern0pt}image{\isacharunderscore}{\kern0pt}subobject{\isacharunderscore}{\kern0pt}mapping{\isacharunderscore}{\kern0pt}mono{\isacharcolon}{\kern0pt}\isanewline
\ \ \isakeyword{assumes}\ {\isachardoublequoteopen}f\ {\isacharcolon}{\kern0pt}\ X\ {\isasymrightarrow}\ Y{\isachardoublequoteclose}\ {\isachardoublequoteopen}m\ {\isacharcolon}{\kern0pt}\ B\ {\isasymrightarrow}\ Y{\isachardoublequoteclose}\ {\isachardoublequoteopen}monomorphism\ m{\isachardoublequoteclose}\isanewline
\ \ \isakeyword{shows}\ {\isachardoublequoteopen}monomorphism\ {\isacharparenleft}{\kern0pt}{\isacharbrackleft}{\kern0pt}f\isactrlsup {\isacharminus}{\kern0pt}\isactrlsup {\isadigit{1}}{\isasymlparr}B{\isasymrparr}\isactrlbsub m\isactrlesub {\isacharbrackright}{\kern0pt}map{\isacharparenright}{\kern0pt}{\isachardoublequoteclose}\isanewline
%
\isadelimproof
\ \ %
\endisadelimproof
%
\isatagproof
\isacommand{using}\isamarkupfalse%
\ assms\ cfunc{\isacharunderscore}{\kern0pt}type{\isacharunderscore}{\kern0pt}def\ inverse{\isacharunderscore}{\kern0pt}image{\isacharunderscore}{\kern0pt}monomorphism\ inverse{\isacharunderscore}{\kern0pt}image{\isacharunderscore}{\kern0pt}subobject{\isacharunderscore}{\kern0pt}mapping{\isacharunderscore}{\kern0pt}def\ \isacommand{by}\isamarkupfalse%
\ fastforce%
\endisatagproof
{\isafoldproof}%
%
\isadelimproof
\isanewline
%
\endisadelimproof
\isanewline
\isacommand{lemma}\isamarkupfalse%
\ inverse{\isacharunderscore}{\kern0pt}image{\isacharunderscore}{\kern0pt}subobject{\isacharcolon}{\kern0pt}\isanewline
\ \ \isakeyword{assumes}\ {\isachardoublequoteopen}m\ {\isacharcolon}{\kern0pt}\ B\ {\isasymrightarrow}\ Y{\isachardoublequoteclose}\ {\isachardoublequoteopen}f\ {\isacharcolon}{\kern0pt}\ X\ {\isasymrightarrow}\ Y{\isachardoublequoteclose}\ {\isachardoublequoteopen}monomorphism\ m{\isachardoublequoteclose}\isanewline
\ \ \isakeyword{shows}\ {\isachardoublequoteopen}{\isacharparenleft}{\kern0pt}f\isactrlsup {\isacharminus}{\kern0pt}\isactrlsup {\isadigit{1}}{\isasymlparr}B{\isasymrparr}\isactrlbsub m\isactrlesub {\isacharcomma}{\kern0pt}\ {\isacharbrackleft}{\kern0pt}f\isactrlsup {\isacharminus}{\kern0pt}\isactrlsup {\isadigit{1}}{\isasymlparr}B{\isasymrparr}\isactrlbsub m\isactrlesub {\isacharbrackright}{\kern0pt}map{\isacharparenright}{\kern0pt}\ {\isasymsubseteq}\isactrlsub c\ X{\isachardoublequoteclose}\isanewline
%
\isadelimproof
\ \ %
\endisadelimproof
%
\isatagproof
\isacommand{unfolding}\isamarkupfalse%
\ subobject{\isacharunderscore}{\kern0pt}of{\isacharunderscore}{\kern0pt}def{\isadigit{2}}\isanewline
\ \ \isacommand{using}\isamarkupfalse%
\ assms\ inverse{\isacharunderscore}{\kern0pt}image{\isacharunderscore}{\kern0pt}subobject{\isacharunderscore}{\kern0pt}mapping{\isacharunderscore}{\kern0pt}mono\ inverse{\isacharunderscore}{\kern0pt}image{\isacharunderscore}{\kern0pt}subobject{\isacharunderscore}{\kern0pt}mapping{\isacharunderscore}{\kern0pt}type\isanewline
\ \ \isacommand{by}\isamarkupfalse%
\ force%
\endisatagproof
{\isafoldproof}%
%
\isadelimproof
\isanewline
%
\endisadelimproof
\isanewline
\isacommand{lemma}\isamarkupfalse%
\ inverse{\isacharunderscore}{\kern0pt}image{\isacharunderscore}{\kern0pt}pullback{\isacharcolon}{\kern0pt}\isanewline
\ \ \isakeyword{assumes}\ {\isachardoublequoteopen}m\ {\isacharcolon}{\kern0pt}\ B\ {\isasymrightarrow}\ Y{\isachardoublequoteclose}\ {\isachardoublequoteopen}f\ {\isacharcolon}{\kern0pt}\ X\ {\isasymrightarrow}\ Y{\isachardoublequoteclose}\ {\isachardoublequoteopen}monomorphism\ m{\isachardoublequoteclose}\isanewline
\ \ \isakeyword{shows}\ {\isachardoublequoteopen}is{\isacharunderscore}{\kern0pt}pullback\ {\isacharparenleft}{\kern0pt}f\isactrlsup {\isacharminus}{\kern0pt}\isactrlsup {\isadigit{1}}{\isasymlparr}B{\isasymrparr}\isactrlbsub m\isactrlesub {\isacharparenright}{\kern0pt}\ B\ X\ Y\ \isanewline
\ \ \ \ {\isacharparenleft}{\kern0pt}right{\isacharunderscore}{\kern0pt}cart{\isacharunderscore}{\kern0pt}proj\ X\ B\ {\isasymcirc}\isactrlsub c\ inverse{\isacharunderscore}{\kern0pt}image{\isacharunderscore}{\kern0pt}mapping\ f\ B\ m{\isacharparenright}{\kern0pt}\ m\isanewline
\ \ \ \ {\isacharparenleft}{\kern0pt}left{\isacharunderscore}{\kern0pt}cart{\isacharunderscore}{\kern0pt}proj\ X\ B\ {\isasymcirc}\isactrlsub c\ inverse{\isacharunderscore}{\kern0pt}image{\isacharunderscore}{\kern0pt}mapping\ f\ B\ m{\isacharparenright}{\kern0pt}\ f{\isachardoublequoteclose}\isanewline
%
\isadelimproof
\ \ %
\endisadelimproof
%
\isatagproof
\isacommand{unfolding}\isamarkupfalse%
\ is{\isacharunderscore}{\kern0pt}pullback{\isacharunderscore}{\kern0pt}def\ \isacommand{using}\isamarkupfalse%
\ assms\isanewline
\isacommand{proof}\isamarkupfalse%
\ safe\isanewline
\ \ \isacommand{show}\isamarkupfalse%
\ right{\isacharunderscore}{\kern0pt}type{\isacharcolon}{\kern0pt}\ {\isachardoublequoteopen}right{\isacharunderscore}{\kern0pt}cart{\isacharunderscore}{\kern0pt}proj\ X\ B\ {\isasymcirc}\isactrlsub c\ inverse{\isacharunderscore}{\kern0pt}image{\isacharunderscore}{\kern0pt}mapping\ f\ B\ m\ {\isacharcolon}{\kern0pt}\ f\isactrlsup {\isacharminus}{\kern0pt}\isactrlsup {\isadigit{1}}{\isasymlparr}B{\isasymrparr}\isactrlbsub m\isactrlesub \ {\isasymrightarrow}\ B{\isachardoublequoteclose}\isanewline
\ \ \ \ \isacommand{using}\isamarkupfalse%
\ assms\ cfunc{\isacharunderscore}{\kern0pt}type{\isacharunderscore}{\kern0pt}def\ codomain{\isacharunderscore}{\kern0pt}comp\ domain{\isacharunderscore}{\kern0pt}comp\ inverse{\isacharunderscore}{\kern0pt}image{\isacharunderscore}{\kern0pt}mapping{\isacharunderscore}{\kern0pt}type\isanewline
\ \ \ \ \ \ right{\isacharunderscore}{\kern0pt}cart{\isacharunderscore}{\kern0pt}proj{\isacharunderscore}{\kern0pt}type\ \isacommand{by}\isamarkupfalse%
\ auto\isanewline
\ \ \isacommand{show}\isamarkupfalse%
\ left{\isacharunderscore}{\kern0pt}type{\isacharcolon}{\kern0pt}\ {\isachardoublequoteopen}left{\isacharunderscore}{\kern0pt}cart{\isacharunderscore}{\kern0pt}proj\ X\ B\ {\isasymcirc}\isactrlsub c\ inverse{\isacharunderscore}{\kern0pt}image{\isacharunderscore}{\kern0pt}mapping\ f\ B\ m\ {\isacharcolon}{\kern0pt}\ f\isactrlsup {\isacharminus}{\kern0pt}\isactrlsup {\isadigit{1}}{\isasymlparr}B{\isasymrparr}\isactrlbsub m\isactrlesub \ {\isasymrightarrow}\ X{\isachardoublequoteclose}\isanewline
\ \ \ \ \isacommand{using}\isamarkupfalse%
\ assms\ fst{\isacharunderscore}{\kern0pt}conv\ inverse{\isacharunderscore}{\kern0pt}image{\isacharunderscore}{\kern0pt}subobject\ subobject{\isacharunderscore}{\kern0pt}of{\isacharunderscore}{\kern0pt}def\ \isacommand{by}\isamarkupfalse%
\ {\isacharparenleft}{\kern0pt}typecheck{\isacharunderscore}{\kern0pt}cfuncs{\isacharparenright}{\kern0pt}\isanewline
\isanewline
\ \ \isacommand{show}\isamarkupfalse%
\ {\isachardoublequoteopen}m\ {\isasymcirc}\isactrlsub c\ right{\isacharunderscore}{\kern0pt}cart{\isacharunderscore}{\kern0pt}proj\ X\ B\ {\isasymcirc}\isactrlsub c\ inverse{\isacharunderscore}{\kern0pt}image{\isacharunderscore}{\kern0pt}mapping\ f\ B\ m\ {\isacharequal}{\kern0pt}\isanewline
\ \ \ \ \ \ f\ {\isasymcirc}\isactrlsub c\ left{\isacharunderscore}{\kern0pt}cart{\isacharunderscore}{\kern0pt}proj\ X\ B\ {\isasymcirc}\isactrlsub c\ inverse{\isacharunderscore}{\kern0pt}image{\isacharunderscore}{\kern0pt}mapping\ f\ B\ m{\isachardoublequoteclose}\isanewline
\ \ \ \ \isacommand{using}\isamarkupfalse%
\ assms\ inverse{\isacharunderscore}{\kern0pt}image{\isacharunderscore}{\kern0pt}mapping{\isacharunderscore}{\kern0pt}eq\ \isacommand{by}\isamarkupfalse%
\ auto\isanewline
\isacommand{next}\isamarkupfalse%
\isanewline
\ \ \isacommand{fix}\isamarkupfalse%
\ Z\ k\ h\isanewline
\ \ \isacommand{assume}\isamarkupfalse%
\ k{\isacharunderscore}{\kern0pt}type{\isacharcolon}{\kern0pt}\ {\isachardoublequoteopen}k\ {\isacharcolon}{\kern0pt}\ Z\ {\isasymrightarrow}\ B{\isachardoublequoteclose}\ \isakeyword{and}\ h{\isacharunderscore}{\kern0pt}type{\isacharcolon}{\kern0pt}\ {\isachardoublequoteopen}h\ {\isacharcolon}{\kern0pt}\ Z\ {\isasymrightarrow}\ X{\isachardoublequoteclose}\isanewline
\ \ \isacommand{assume}\isamarkupfalse%
\ mk{\isacharunderscore}{\kern0pt}eq{\isacharunderscore}{\kern0pt}fh{\isacharcolon}{\kern0pt}\ {\isachardoublequoteopen}m\ {\isasymcirc}\isactrlsub c\ k\ {\isacharequal}{\kern0pt}\ f\ {\isasymcirc}\isactrlsub c\ h{\isachardoublequoteclose}\isanewline
\isanewline
\ \ \isacommand{have}\isamarkupfalse%
\ {\isachardoublequoteopen}equalizer\ {\isacharparenleft}{\kern0pt}f\isactrlsup {\isacharminus}{\kern0pt}\isactrlsup {\isadigit{1}}{\isasymlparr}B{\isasymrparr}\isactrlbsub m\isactrlesub {\isacharparenright}{\kern0pt}\ {\isacharparenleft}{\kern0pt}inverse{\isacharunderscore}{\kern0pt}image{\isacharunderscore}{\kern0pt}mapping\ f\ B\ m{\isacharparenright}{\kern0pt}\ {\isacharparenleft}{\kern0pt}f\ {\isasymcirc}\isactrlsub c\ left{\isacharunderscore}{\kern0pt}cart{\isacharunderscore}{\kern0pt}proj\ X\ B{\isacharparenright}{\kern0pt}\ {\isacharparenleft}{\kern0pt}m\ {\isasymcirc}\isactrlsub c\ right{\isacharunderscore}{\kern0pt}cart{\isacharunderscore}{\kern0pt}proj\ X\ B\ {\isacharparenright}{\kern0pt}{\isachardoublequoteclose}\isanewline
\ \ \ \ \isacommand{using}\isamarkupfalse%
\ assms\ inverse{\isacharunderscore}{\kern0pt}image{\isacharunderscore}{\kern0pt}is{\isacharunderscore}{\kern0pt}equalizer{\isadigit{2}}\ \isacommand{by}\isamarkupfalse%
\ blast\isanewline
\ \ \isacommand{then}\isamarkupfalse%
\ \isacommand{have}\isamarkupfalse%
\ {\isachardoublequoteopen}{\isasymforall}h\ F{\isachardot}{\kern0pt}\ h\ {\isacharcolon}{\kern0pt}\ F\ {\isasymrightarrow}\ {\isacharparenleft}{\kern0pt}X\ {\isasymtimes}\isactrlsub c\ B{\isacharparenright}{\kern0pt}\ \isanewline
\ \ \ \ \ \ \ \ \ \ \ \ {\isasymand}\ {\isacharparenleft}{\kern0pt}f\ {\isasymcirc}\isactrlsub c\ left{\isacharunderscore}{\kern0pt}cart{\isacharunderscore}{\kern0pt}proj\ X\ B{\isacharparenright}{\kern0pt}\ {\isasymcirc}\isactrlsub c\ h\ {\isacharequal}{\kern0pt}\ {\isacharparenleft}{\kern0pt}m\ {\isasymcirc}\isactrlsub c\ right{\isacharunderscore}{\kern0pt}cart{\isacharunderscore}{\kern0pt}proj\ X\ B{\isacharparenright}{\kern0pt}\ {\isasymcirc}\isactrlsub c\ h\ {\isasymlongrightarrow}\isanewline
\ \ \ \ \ \ \ \ \ \ {\isacharparenleft}{\kern0pt}{\isasymexists}{\isacharbang}{\kern0pt}u{\isachardot}{\kern0pt}\ u\ {\isacharcolon}{\kern0pt}\ F\ {\isasymrightarrow}\ {\isacharparenleft}{\kern0pt}f\isactrlsup {\isacharminus}{\kern0pt}\isactrlsup {\isadigit{1}}{\isasymlparr}B{\isasymrparr}\isactrlbsub m\isactrlesub {\isacharparenright}{\kern0pt}\ {\isasymand}\ inverse{\isacharunderscore}{\kern0pt}image{\isacharunderscore}{\kern0pt}mapping\ f\ B\ m\ {\isasymcirc}\isactrlsub c\ u\ {\isacharequal}{\kern0pt}\ h{\isacharparenright}{\kern0pt}{\isachardoublequoteclose}\isanewline
\ \ \ \ \isacommand{unfolding}\isamarkupfalse%
\ equalizer{\isacharunderscore}{\kern0pt}def\ \isacommand{using}\isamarkupfalse%
\ assms{\isacharparenleft}{\kern0pt}{\isadigit{2}}{\isacharparenright}{\kern0pt}\ cfunc{\isacharunderscore}{\kern0pt}type{\isacharunderscore}{\kern0pt}def\ domain{\isacharunderscore}{\kern0pt}comp\ left{\isacharunderscore}{\kern0pt}cart{\isacharunderscore}{\kern0pt}proj{\isacharunderscore}{\kern0pt}type\ \isacommand{by}\isamarkupfalse%
\ auto\isanewline
\ \ \isacommand{then}\isamarkupfalse%
\ \isacommand{have}\isamarkupfalse%
\ {\isachardoublequoteopen}{\isasymlangle}h{\isacharcomma}{\kern0pt}k{\isasymrangle}\ {\isacharcolon}{\kern0pt}\ Z\ {\isasymrightarrow}\ X\ {\isasymtimes}\isactrlsub c\ B\ \ {\isasymLongrightarrow}\isanewline
\ \ \ \ \ \ {\isacharparenleft}{\kern0pt}f\ {\isasymcirc}\isactrlsub c\ left{\isacharunderscore}{\kern0pt}cart{\isacharunderscore}{\kern0pt}proj\ X\ B{\isacharparenright}{\kern0pt}\ {\isasymcirc}\isactrlsub c\ {\isasymlangle}h{\isacharcomma}{\kern0pt}k{\isasymrangle}\ {\isacharequal}{\kern0pt}\ {\isacharparenleft}{\kern0pt}m\ {\isasymcirc}\isactrlsub c\ right{\isacharunderscore}{\kern0pt}cart{\isacharunderscore}{\kern0pt}proj\ X\ B{\isacharparenright}{\kern0pt}\ {\isasymcirc}\isactrlsub c\ {\isasymlangle}h{\isacharcomma}{\kern0pt}k{\isasymrangle}\ {\isasymLongrightarrow}\isanewline
\ \ \ \ \ \ {\isacharparenleft}{\kern0pt}{\isasymexists}{\isacharbang}{\kern0pt}u{\isachardot}{\kern0pt}\ u\ {\isacharcolon}{\kern0pt}\ Z\ {\isasymrightarrow}\ {\isacharparenleft}{\kern0pt}f\isactrlsup {\isacharminus}{\kern0pt}\isactrlsup {\isadigit{1}}{\isasymlparr}B{\isasymrparr}\isactrlbsub m\isactrlesub {\isacharparenright}{\kern0pt}\ {\isasymand}\ inverse{\isacharunderscore}{\kern0pt}image{\isacharunderscore}{\kern0pt}mapping\ f\ B\ m\ {\isasymcirc}\isactrlsub c\ u\ {\isacharequal}{\kern0pt}\ {\isasymlangle}h{\isacharcomma}{\kern0pt}k{\isasymrangle}{\isacharparenright}{\kern0pt}{\isachardoublequoteclose}\isanewline
\ \ \ \ \isacommand{by}\isamarkupfalse%
\ {\isacharparenleft}{\kern0pt}erule{\isacharunderscore}{\kern0pt}tac\ x{\isacharequal}{\kern0pt}{\isachardoublequoteopen}{\isasymlangle}h{\isacharcomma}{\kern0pt}k{\isasymrangle}{\isachardoublequoteclose}\ \isakeyword{in}\ allE{\isacharcomma}{\kern0pt}\ erule{\isacharunderscore}{\kern0pt}tac\ x{\isacharequal}{\kern0pt}Z\ \isakeyword{in}\ allE{\isacharcomma}{\kern0pt}\ auto{\isacharparenright}{\kern0pt}\isanewline
\ \ \isacommand{then}\isamarkupfalse%
\ \isacommand{have}\isamarkupfalse%
\ {\isachardoublequoteopen}{\isasymexists}{\isacharbang}{\kern0pt}u{\isachardot}{\kern0pt}\ u\ {\isacharcolon}{\kern0pt}\ Z\ {\isasymrightarrow}\ {\isacharparenleft}{\kern0pt}f\isactrlsup {\isacharminus}{\kern0pt}\isactrlsup {\isadigit{1}}{\isasymlparr}B{\isasymrparr}\isactrlbsub m\isactrlesub {\isacharparenright}{\kern0pt}\ {\isasymand}\ inverse{\isacharunderscore}{\kern0pt}image{\isacharunderscore}{\kern0pt}mapping\ f\ B\ m\ {\isasymcirc}\isactrlsub c\ u\ {\isacharequal}{\kern0pt}\ {\isasymlangle}h{\isacharcomma}{\kern0pt}k{\isasymrangle}{\isachardoublequoteclose}\isanewline
\ \ \ \ \isacommand{using}\isamarkupfalse%
\ k{\isacharunderscore}{\kern0pt}type\ h{\isacharunderscore}{\kern0pt}type\ assms\isanewline
\ \ \ \ \isacommand{by}\isamarkupfalse%
\ {\isacharparenleft}{\kern0pt}typecheck{\isacharunderscore}{\kern0pt}cfuncs{\isacharcomma}{\kern0pt}\ smt\ comp{\isacharunderscore}{\kern0pt}associative{\isadigit{2}}\ left{\isacharunderscore}{\kern0pt}cart{\isacharunderscore}{\kern0pt}proj{\isacharunderscore}{\kern0pt}cfunc{\isacharunderscore}{\kern0pt}prod\ left{\isacharunderscore}{\kern0pt}cart{\isacharunderscore}{\kern0pt}proj{\isacharunderscore}{\kern0pt}type\isanewline
\ \ \ \ \ \ \ \ mk{\isacharunderscore}{\kern0pt}eq{\isacharunderscore}{\kern0pt}fh\ right{\isacharunderscore}{\kern0pt}cart{\isacharunderscore}{\kern0pt}proj{\isacharunderscore}{\kern0pt}cfunc{\isacharunderscore}{\kern0pt}prod\ right{\isacharunderscore}{\kern0pt}cart{\isacharunderscore}{\kern0pt}proj{\isacharunderscore}{\kern0pt}type{\isacharparenright}{\kern0pt}\isanewline
\ \ \isacommand{then}\isamarkupfalse%
\ \isacommand{show}\isamarkupfalse%
\ {\isachardoublequoteopen}{\isasymexists}j{\isachardot}{\kern0pt}\ j\ {\isacharcolon}{\kern0pt}\ Z\ {\isasymrightarrow}\ {\isacharparenleft}{\kern0pt}f\isactrlsup {\isacharminus}{\kern0pt}\isactrlsup {\isadigit{1}}{\isasymlparr}B{\isasymrparr}\isactrlbsub m\isactrlesub {\isacharparenright}{\kern0pt}\ {\isasymand}\isanewline
\ \ \ \ \ \ \ \ \ {\isacharparenleft}{\kern0pt}right{\isacharunderscore}{\kern0pt}cart{\isacharunderscore}{\kern0pt}proj\ X\ B\ {\isasymcirc}\isactrlsub c\ inverse{\isacharunderscore}{\kern0pt}image{\isacharunderscore}{\kern0pt}mapping\ f\ B\ m{\isacharparenright}{\kern0pt}\ {\isasymcirc}\isactrlsub c\ j\ {\isacharequal}{\kern0pt}\ k\ {\isasymand}\isanewline
\ \ \ \ \ \ \ \ \ {\isacharparenleft}{\kern0pt}left{\isacharunderscore}{\kern0pt}cart{\isacharunderscore}{\kern0pt}proj\ X\ B\ {\isasymcirc}\isactrlsub c\ inverse{\isacharunderscore}{\kern0pt}image{\isacharunderscore}{\kern0pt}mapping\ f\ B\ m{\isacharparenright}{\kern0pt}\ {\isasymcirc}\isactrlsub c\ j\ {\isacharequal}{\kern0pt}\ h{\isachardoublequoteclose}\isanewline
\ \ \isacommand{proof}\isamarkupfalse%
\ {\isacharparenleft}{\kern0pt}insert\ k{\isacharunderscore}{\kern0pt}type\ h{\isacharunderscore}{\kern0pt}type\ assms{\isacharcomma}{\kern0pt}\ safe{\isacharparenright}{\kern0pt}\isanewline
\ \ \ \ \isacommand{fix}\isamarkupfalse%
\ u\isanewline
\ \ \ \ \isacommand{assume}\isamarkupfalse%
\ u{\isacharunderscore}{\kern0pt}type{\isacharbrackleft}{\kern0pt}type{\isacharunderscore}{\kern0pt}rule{\isacharbrackright}{\kern0pt}{\isacharcolon}{\kern0pt}\ {\isachardoublequoteopen}u\ {\isacharcolon}{\kern0pt}\ Z\ {\isasymrightarrow}\ {\isacharparenleft}{\kern0pt}f\isactrlsup {\isacharminus}{\kern0pt}\isactrlsup {\isadigit{1}}{\isasymlparr}B{\isasymrparr}\isactrlbsub m\isactrlesub {\isacharparenright}{\kern0pt}{\isachardoublequoteclose}\isanewline
\ \ \ \ \isacommand{assume}\isamarkupfalse%
\ u{\isacharunderscore}{\kern0pt}eq{\isacharcolon}{\kern0pt}\ {\isachardoublequoteopen}inverse{\isacharunderscore}{\kern0pt}image{\isacharunderscore}{\kern0pt}mapping\ f\ B\ m\ {\isasymcirc}\isactrlsub c\ u\ {\isacharequal}{\kern0pt}\ {\isasymlangle}h{\isacharcomma}{\kern0pt}k{\isasymrangle}{\isachardoublequoteclose}\isanewline
\isanewline
\ \ \ \ \isacommand{show}\isamarkupfalse%
\ {\isachardoublequoteopen}{\isasymexists}j{\isachardot}{\kern0pt}\ j\ {\isacharcolon}{\kern0pt}\ Z\ {\isasymrightarrow}\ f\isactrlsup {\isacharminus}{\kern0pt}\isactrlsup {\isadigit{1}}{\isasymlparr}B{\isasymrparr}\isactrlbsub m\isactrlesub \ {\isasymand}\isanewline
\ \ \ \ \ \ \ \ \ \ \ \ \ {\isacharparenleft}{\kern0pt}right{\isacharunderscore}{\kern0pt}cart{\isacharunderscore}{\kern0pt}proj\ X\ B\ {\isasymcirc}\isactrlsub c\ inverse{\isacharunderscore}{\kern0pt}image{\isacharunderscore}{\kern0pt}mapping\ f\ B\ m{\isacharparenright}{\kern0pt}\ {\isasymcirc}\isactrlsub c\ j\ {\isacharequal}{\kern0pt}\ k\ {\isasymand}\isanewline
\ \ \ \ \ \ \ \ \ \ \ \ \ {\isacharparenleft}{\kern0pt}left{\isacharunderscore}{\kern0pt}cart{\isacharunderscore}{\kern0pt}proj\ X\ B\ {\isasymcirc}\isactrlsub c\ inverse{\isacharunderscore}{\kern0pt}image{\isacharunderscore}{\kern0pt}mapping\ f\ B\ m{\isacharparenright}{\kern0pt}\ {\isasymcirc}\isactrlsub c\ j\ {\isacharequal}{\kern0pt}\ h{\isachardoublequoteclose}\isanewline
\ \ \ \ \isacommand{proof}\isamarkupfalse%
\ {\isacharparenleft}{\kern0pt}rule\ exI{\isacharbrackleft}{\kern0pt}\isakeyword{where}\ x{\isacharequal}{\kern0pt}u{\isacharbrackright}{\kern0pt}{\isacharcomma}{\kern0pt}\ typecheck{\isacharunderscore}{\kern0pt}cfuncs{\isacharcomma}{\kern0pt}\ safe{\isacharparenright}{\kern0pt}\isanewline
\isanewline
\ \ \ \ \ \ \isacommand{show}\isamarkupfalse%
\ {\isachardoublequoteopen}{\isacharparenleft}{\kern0pt}right{\isacharunderscore}{\kern0pt}cart{\isacharunderscore}{\kern0pt}proj\ X\ B\ {\isasymcirc}\isactrlsub c\ inverse{\isacharunderscore}{\kern0pt}image{\isacharunderscore}{\kern0pt}mapping\ f\ B\ m{\isacharparenright}{\kern0pt}\ {\isasymcirc}\isactrlsub c\ u\ {\isacharequal}{\kern0pt}\ k{\isachardoublequoteclose}\isanewline
\ \ \ \ \ \ \ \ \isacommand{using}\isamarkupfalse%
\ assms\ u{\isacharunderscore}{\kern0pt}type\ h{\isacharunderscore}{\kern0pt}type\ k{\isacharunderscore}{\kern0pt}type\ u{\isacharunderscore}{\kern0pt}eq\isanewline
\ \ \ \ \ \ \ \ \isacommand{by}\isamarkupfalse%
\ {\isacharparenleft}{\kern0pt}typecheck{\isacharunderscore}{\kern0pt}cfuncs{\isacharcomma}{\kern0pt}\ metis\ {\isacharparenleft}{\kern0pt}full{\isacharunderscore}{\kern0pt}types{\isacharparenright}{\kern0pt}\ comp{\isacharunderscore}{\kern0pt}associative{\isadigit{2}}\ right{\isacharunderscore}{\kern0pt}cart{\isacharunderscore}{\kern0pt}proj{\isacharunderscore}{\kern0pt}cfunc{\isacharunderscore}{\kern0pt}prod{\isacharparenright}{\kern0pt}\isanewline
\ \ \isanewline
\ \ \ \ \ \ \isacommand{show}\isamarkupfalse%
\ {\isachardoublequoteopen}{\isacharparenleft}{\kern0pt}left{\isacharunderscore}{\kern0pt}cart{\isacharunderscore}{\kern0pt}proj\ X\ B\ {\isasymcirc}\isactrlsub c\ inverse{\isacharunderscore}{\kern0pt}image{\isacharunderscore}{\kern0pt}mapping\ f\ B\ m{\isacharparenright}{\kern0pt}\ {\isasymcirc}\isactrlsub c\ u\ {\isacharequal}{\kern0pt}\ h{\isachardoublequoteclose}\isanewline
\ \ \ \ \ \ \ \ \isacommand{using}\isamarkupfalse%
\ assms\ u{\isacharunderscore}{\kern0pt}type\ h{\isacharunderscore}{\kern0pt}type\ k{\isacharunderscore}{\kern0pt}type\ u{\isacharunderscore}{\kern0pt}eq\isanewline
\ \ \ \ \ \ \ \ \isacommand{by}\isamarkupfalse%
\ {\isacharparenleft}{\kern0pt}typecheck{\isacharunderscore}{\kern0pt}cfuncs{\isacharcomma}{\kern0pt}\ metis\ {\isacharparenleft}{\kern0pt}full{\isacharunderscore}{\kern0pt}types{\isacharparenright}{\kern0pt}\ comp{\isacharunderscore}{\kern0pt}associative{\isadigit{2}}\ left{\isacharunderscore}{\kern0pt}cart{\isacharunderscore}{\kern0pt}proj{\isacharunderscore}{\kern0pt}cfunc{\isacharunderscore}{\kern0pt}prod{\isacharparenright}{\kern0pt}\isanewline
\ \ \ \ \isacommand{qed}\isamarkupfalse%
\isanewline
\ \ \isacommand{qed}\isamarkupfalse%
\isanewline
\isacommand{next}\isamarkupfalse%
\isanewline
\ \ \isacommand{fix}\isamarkupfalse%
\ Z\ j\ y\isanewline
\ \ \isacommand{assume}\isamarkupfalse%
\ j{\isacharunderscore}{\kern0pt}type{\isacharcolon}{\kern0pt}\ {\isachardoublequoteopen}j\ {\isacharcolon}{\kern0pt}\ Z\ {\isasymrightarrow}\ {\isacharparenleft}{\kern0pt}f\isactrlsup {\isacharminus}{\kern0pt}\isactrlsup {\isadigit{1}}{\isasymlparr}B{\isasymrparr}\isactrlbsub m\isactrlesub {\isacharparenright}{\kern0pt}{\isachardoublequoteclose}\isanewline
\ \ \isacommand{assume}\isamarkupfalse%
\ y{\isacharunderscore}{\kern0pt}type{\isacharcolon}{\kern0pt}\ {\isachardoublequoteopen}y\ {\isacharcolon}{\kern0pt}\ Z\ {\isasymrightarrow}\ {\isacharparenleft}{\kern0pt}f\isactrlsup {\isacharminus}{\kern0pt}\isactrlsup {\isadigit{1}}{\isasymlparr}B{\isasymrparr}\isactrlbsub m\isactrlesub {\isacharparenright}{\kern0pt}{\isachardoublequoteclose}\isanewline
\ \ \isacommand{assume}\isamarkupfalse%
\ {\isachardoublequoteopen}{\isacharparenleft}{\kern0pt}left{\isacharunderscore}{\kern0pt}cart{\isacharunderscore}{\kern0pt}proj\ X\ B\ {\isasymcirc}\isactrlsub c\ inverse{\isacharunderscore}{\kern0pt}image{\isacharunderscore}{\kern0pt}mapping\ f\ B\ m{\isacharparenright}{\kern0pt}\ {\isasymcirc}\isactrlsub c\ y\ {\isacharequal}{\kern0pt}\isanewline
\ \ \ \ \ \ \ {\isacharparenleft}{\kern0pt}left{\isacharunderscore}{\kern0pt}cart{\isacharunderscore}{\kern0pt}proj\ X\ B\ {\isasymcirc}\isactrlsub c\ inverse{\isacharunderscore}{\kern0pt}image{\isacharunderscore}{\kern0pt}mapping\ f\ B\ m{\isacharparenright}{\kern0pt}\ {\isasymcirc}\isactrlsub c\ j{\isachardoublequoteclose}\isanewline
\ \ \isacommand{then}\isamarkupfalse%
\ \isacommand{show}\isamarkupfalse%
\ {\isachardoublequoteopen}j\ {\isacharequal}{\kern0pt}\ y{\isachardoublequoteclose}\isanewline
\ \ \ \ \isacommand{using}\isamarkupfalse%
\ assms\ j{\isacharunderscore}{\kern0pt}type\ y{\isacharunderscore}{\kern0pt}type\ inverse{\isacharunderscore}{\kern0pt}image{\isacharunderscore}{\kern0pt}mapping{\isacharunderscore}{\kern0pt}type\ comp{\isacharunderscore}{\kern0pt}type\isanewline
\ \ \ \ \isacommand{by}\isamarkupfalse%
\ {\isacharparenleft}{\kern0pt}smt\ {\isacharparenleft}{\kern0pt}verit{\isacharcomma}{\kern0pt}\ ccfv{\isacharunderscore}{\kern0pt}threshold{\isacharparenright}{\kern0pt}\ inverse{\isacharunderscore}{\kern0pt}image{\isacharunderscore}{\kern0pt}monomorphism\ left{\isacharunderscore}{\kern0pt}cart{\isacharunderscore}{\kern0pt}proj{\isacharunderscore}{\kern0pt}type\ monomorphism{\isacharunderscore}{\kern0pt}def{\isadigit{3}}{\isacharparenright}{\kern0pt}\isanewline
\isacommand{qed}\isamarkupfalse%
%
\endisatagproof
{\isafoldproof}%
%
\isadelimproof
%
\endisadelimproof
%
\begin{isamarkuptext}%
The lemma below corresponds to Proposition 2.1.41 in Halvorson.%
\end{isamarkuptext}\isamarkuptrue%
\isacommand{lemma}\isamarkupfalse%
\ in{\isacharunderscore}{\kern0pt}inverse{\isacharunderscore}{\kern0pt}image{\isacharcolon}{\kern0pt}\isanewline
\ \ \isakeyword{assumes}\ {\isachardoublequoteopen}f\ {\isacharcolon}{\kern0pt}\ X\ {\isasymrightarrow}\ Y{\isachardoublequoteclose}\ {\isachardoublequoteopen}{\isacharparenleft}{\kern0pt}B{\isacharcomma}{\kern0pt}m{\isacharparenright}{\kern0pt}\ {\isasymsubseteq}\isactrlsub c\ Y{\isachardoublequoteclose}\ {\isachardoublequoteopen}x\ {\isasymin}\isactrlsub c\ X{\isachardoublequoteclose}\isanewline
\ \ \isakeyword{shows}\ {\isachardoublequoteopen}{\isacharparenleft}{\kern0pt}x\ {\isasymin}\isactrlbsub X\isactrlesub \ {\isacharparenleft}{\kern0pt}f\isactrlsup {\isacharminus}{\kern0pt}\isactrlsup {\isadigit{1}}{\isasymlparr}B{\isasymrparr}\isactrlbsub m\isactrlesub {\isacharcomma}{\kern0pt}\ left{\isacharunderscore}{\kern0pt}cart{\isacharunderscore}{\kern0pt}proj\ X\ B\ {\isasymcirc}\isactrlsub c\ inverse{\isacharunderscore}{\kern0pt}image{\isacharunderscore}{\kern0pt}mapping\ f\ B\ m{\isacharparenright}{\kern0pt}{\isacharparenright}{\kern0pt}\ {\isacharequal}{\kern0pt}\ {\isacharparenleft}{\kern0pt}f\ {\isasymcirc}\isactrlsub c\ x\ {\isasymin}\isactrlbsub Y\isactrlesub \ {\isacharparenleft}{\kern0pt}B{\isacharcomma}{\kern0pt}m{\isacharparenright}{\kern0pt}{\isacharparenright}{\kern0pt}{\isachardoublequoteclose}\isanewline
%
\isadelimproof
%
\endisadelimproof
%
\isatagproof
\isacommand{proof}\isamarkupfalse%
\isanewline
\ \ \isacommand{have}\isamarkupfalse%
\ m{\isacharunderscore}{\kern0pt}type{\isacharcolon}{\kern0pt}\ {\isachardoublequoteopen}m\ {\isacharcolon}{\kern0pt}\ B\ {\isasymrightarrow}\ Y{\isachardoublequoteclose}\ {\isachardoublequoteopen}monomorphism\ m{\isachardoublequoteclose}\isanewline
\ \ \ \ \isacommand{using}\isamarkupfalse%
\ assms{\isacharparenleft}{\kern0pt}{\isadigit{2}}{\isacharparenright}{\kern0pt}\ \isacommand{unfolding}\isamarkupfalse%
\ subobject{\isacharunderscore}{\kern0pt}of{\isacharunderscore}{\kern0pt}def{\isadigit{2}}\ \isacommand{by}\isamarkupfalse%
\ auto\isanewline
\isanewline
\ \ \isacommand{assume}\isamarkupfalse%
\ {\isachardoublequoteopen}x\ {\isasymin}\isactrlbsub X\isactrlesub \ {\isacharparenleft}{\kern0pt}f\isactrlsup {\isacharminus}{\kern0pt}\isactrlsup {\isadigit{1}}{\isasymlparr}B{\isasymrparr}\isactrlbsub m\isactrlesub {\isacharcomma}{\kern0pt}\ left{\isacharunderscore}{\kern0pt}cart{\isacharunderscore}{\kern0pt}proj\ X\ B\ {\isasymcirc}\isactrlsub c\ inverse{\isacharunderscore}{\kern0pt}image{\isacharunderscore}{\kern0pt}mapping\ f\ B\ m{\isacharparenright}{\kern0pt}{\isachardoublequoteclose}\isanewline
\ \ \isacommand{then}\isamarkupfalse%
\ \isacommand{obtain}\isamarkupfalse%
\ h\ \isakeyword{where}\ h{\isacharunderscore}{\kern0pt}type{\isacharcolon}{\kern0pt}\ {\isachardoublequoteopen}h\ {\isasymin}\isactrlsub c\ {\isacharparenleft}{\kern0pt}f\isactrlsup {\isacharminus}{\kern0pt}\isactrlsup {\isadigit{1}}{\isasymlparr}B{\isasymrparr}\isactrlbsub m\isactrlesub {\isacharparenright}{\kern0pt}{\isachardoublequoteclose}\isanewline
\ \ \ \ \ \ \isakeyword{and}\ h{\isacharunderscore}{\kern0pt}def{\isacharcolon}{\kern0pt}\ {\isachardoublequoteopen}{\isacharparenleft}{\kern0pt}left{\isacharunderscore}{\kern0pt}cart{\isacharunderscore}{\kern0pt}proj\ X\ B\ {\isasymcirc}\isactrlsub c\ inverse{\isacharunderscore}{\kern0pt}image{\isacharunderscore}{\kern0pt}mapping\ f\ B\ m{\isacharparenright}{\kern0pt}\ {\isasymcirc}\isactrlsub c\ h\ {\isacharequal}{\kern0pt}\ x{\isachardoublequoteclose}\isanewline
\ \ \ \ \isacommand{unfolding}\isamarkupfalse%
\ relative{\isacharunderscore}{\kern0pt}member{\isacharunderscore}{\kern0pt}def{\isadigit{2}}\ factors{\isacharunderscore}{\kern0pt}through{\isacharunderscore}{\kern0pt}def\ \isacommand{by}\isamarkupfalse%
\ {\isacharparenleft}{\kern0pt}auto\ simp\ add{\isacharcolon}{\kern0pt}\ cfunc{\isacharunderscore}{\kern0pt}type{\isacharunderscore}{\kern0pt}def{\isacharparenright}{\kern0pt}\isanewline
\ \ \isacommand{then}\isamarkupfalse%
\ \isacommand{have}\isamarkupfalse%
\ {\isachardoublequoteopen}f\ {\isasymcirc}\isactrlsub c\ x\ {\isacharequal}{\kern0pt}\ f\ {\isasymcirc}\isactrlsub c\ left{\isacharunderscore}{\kern0pt}cart{\isacharunderscore}{\kern0pt}proj\ X\ B\ {\isasymcirc}\isactrlsub c\ inverse{\isacharunderscore}{\kern0pt}image{\isacharunderscore}{\kern0pt}mapping\ f\ B\ m\ {\isasymcirc}\isactrlsub c\ h{\isachardoublequoteclose}\isanewline
\ \ \ \ \isacommand{using}\isamarkupfalse%
\ assms\ m{\isacharunderscore}{\kern0pt}type\ \isacommand{by}\isamarkupfalse%
\ {\isacharparenleft}{\kern0pt}typecheck{\isacharunderscore}{\kern0pt}cfuncs{\isacharcomma}{\kern0pt}\ simp\ add{\isacharcolon}{\kern0pt}\ comp{\isacharunderscore}{\kern0pt}associative{\isadigit{2}}\ h{\isacharunderscore}{\kern0pt}def{\isacharparenright}{\kern0pt}\isanewline
\ \ \isacommand{then}\isamarkupfalse%
\ \isacommand{have}\isamarkupfalse%
\ {\isachardoublequoteopen}f\ {\isasymcirc}\isactrlsub c\ x\ {\isacharequal}{\kern0pt}\ {\isacharparenleft}{\kern0pt}f\ {\isasymcirc}\isactrlsub c\ left{\isacharunderscore}{\kern0pt}cart{\isacharunderscore}{\kern0pt}proj\ X\ B\ {\isasymcirc}\isactrlsub c\ inverse{\isacharunderscore}{\kern0pt}image{\isacharunderscore}{\kern0pt}mapping\ f\ B\ m{\isacharparenright}{\kern0pt}\ {\isasymcirc}\isactrlsub c\ h{\isachardoublequoteclose}\isanewline
\ \ \ \ \isacommand{using}\isamarkupfalse%
\ assms\ m{\isacharunderscore}{\kern0pt}type\ h{\isacharunderscore}{\kern0pt}type\ h{\isacharunderscore}{\kern0pt}def\ comp{\isacharunderscore}{\kern0pt}associative{\isadigit{2}}\ \isacommand{by}\isamarkupfalse%
\ {\isacharparenleft}{\kern0pt}typecheck{\isacharunderscore}{\kern0pt}cfuncs{\isacharcomma}{\kern0pt}\ blast{\isacharparenright}{\kern0pt}\isanewline
\ \ \isacommand{then}\isamarkupfalse%
\ \isacommand{have}\isamarkupfalse%
\ {\isachardoublequoteopen}f\ {\isasymcirc}\isactrlsub c\ x\ {\isacharequal}{\kern0pt}\ {\isacharparenleft}{\kern0pt}m\ {\isasymcirc}\isactrlsub c\ right{\isacharunderscore}{\kern0pt}cart{\isacharunderscore}{\kern0pt}proj\ X\ B\ {\isasymcirc}\isactrlsub c\ inverse{\isacharunderscore}{\kern0pt}image{\isacharunderscore}{\kern0pt}mapping\ f\ B\ m{\isacharparenright}{\kern0pt}\ {\isasymcirc}\isactrlsub c\ h{\isachardoublequoteclose}\isanewline
\ \ \ \ \isacommand{using}\isamarkupfalse%
\ assms\ h{\isacharunderscore}{\kern0pt}type\ m{\isacharunderscore}{\kern0pt}type\ \isacommand{by}\isamarkupfalse%
\ {\isacharparenleft}{\kern0pt}typecheck{\isacharunderscore}{\kern0pt}cfuncs{\isacharcomma}{\kern0pt}\ simp\ add{\isacharcolon}{\kern0pt}\ inverse{\isacharunderscore}{\kern0pt}image{\isacharunderscore}{\kern0pt}mapping{\isacharunderscore}{\kern0pt}eq\ m{\isacharunderscore}{\kern0pt}type{\isacharparenright}{\kern0pt}\isanewline
\ \ \isacommand{then}\isamarkupfalse%
\ \isacommand{have}\isamarkupfalse%
\ {\isachardoublequoteopen}f\ {\isasymcirc}\isactrlsub c\ x\ {\isacharequal}{\kern0pt}\ m\ {\isasymcirc}\isactrlsub c\ right{\isacharunderscore}{\kern0pt}cart{\isacharunderscore}{\kern0pt}proj\ X\ B\ {\isasymcirc}\isactrlsub c\ inverse{\isacharunderscore}{\kern0pt}image{\isacharunderscore}{\kern0pt}mapping\ f\ B\ m\ {\isasymcirc}\isactrlsub c\ h{\isachardoublequoteclose}\isanewline
\ \ \ \ \isacommand{using}\isamarkupfalse%
\ assms\ m{\isacharunderscore}{\kern0pt}type\ h{\isacharunderscore}{\kern0pt}type\ \isacommand{by}\isamarkupfalse%
\ {\isacharparenleft}{\kern0pt}typecheck{\isacharunderscore}{\kern0pt}cfuncs{\isacharcomma}{\kern0pt}\ smt\ cfunc{\isacharunderscore}{\kern0pt}type{\isacharunderscore}{\kern0pt}def\ comp{\isacharunderscore}{\kern0pt}associative\ domain{\isacharunderscore}{\kern0pt}comp{\isacharparenright}{\kern0pt}\isanewline
\ \ \isacommand{then}\isamarkupfalse%
\ \isacommand{have}\isamarkupfalse%
\ {\isachardoublequoteopen}{\isacharparenleft}{\kern0pt}f\ {\isasymcirc}\isactrlsub c\ x{\isacharparenright}{\kern0pt}\ factorsthru\ m{\isachardoublequoteclose}\isanewline
\ \ \ \ \isacommand{unfolding}\isamarkupfalse%
\ factors{\isacharunderscore}{\kern0pt}through{\isacharunderscore}{\kern0pt}def\ \isacommand{using}\isamarkupfalse%
\ assms\ h{\isacharunderscore}{\kern0pt}type\ m{\isacharunderscore}{\kern0pt}type\isanewline
\ \ \ \ \isacommand{by}\isamarkupfalse%
\ {\isacharparenleft}{\kern0pt}rule{\isacharunderscore}{\kern0pt}tac\ x{\isacharequal}{\kern0pt}{\isachardoublequoteopen}right{\isacharunderscore}{\kern0pt}cart{\isacharunderscore}{\kern0pt}proj\ X\ B\ {\isasymcirc}\isactrlsub c\ inverse{\isacharunderscore}{\kern0pt}image{\isacharunderscore}{\kern0pt}mapping\ f\ B\ m\ {\isasymcirc}\isactrlsub c\ h{\isachardoublequoteclose}\ \isakeyword{in}\ exI{\isacharcomma}{\kern0pt}\isanewline
\ \ \ \ \ \ \ \ typecheck{\isacharunderscore}{\kern0pt}cfuncs{\isacharcomma}{\kern0pt}\ auto\ simp\ add{\isacharcolon}{\kern0pt}\ cfunc{\isacharunderscore}{\kern0pt}type{\isacharunderscore}{\kern0pt}def{\isacharparenright}{\kern0pt}\isanewline
\ \ \isacommand{then}\isamarkupfalse%
\ \isacommand{show}\isamarkupfalse%
\ {\isachardoublequoteopen}f\ {\isasymcirc}\isactrlsub c\ x\ {\isasymin}\isactrlbsub Y\isactrlesub \ {\isacharparenleft}{\kern0pt}B{\isacharcomma}{\kern0pt}\ m{\isacharparenright}{\kern0pt}{\isachardoublequoteclose}\isanewline
\ \ \ \ \isacommand{unfolding}\isamarkupfalse%
\ relative{\isacharunderscore}{\kern0pt}member{\isacharunderscore}{\kern0pt}def{\isadigit{2}}\ \isacommand{using}\isamarkupfalse%
\ assms\ m{\isacharunderscore}{\kern0pt}type\ \isacommand{by}\isamarkupfalse%
\ {\isacharparenleft}{\kern0pt}typecheck{\isacharunderscore}{\kern0pt}cfuncs{\isacharcomma}{\kern0pt}\ auto{\isacharparenright}{\kern0pt}\isanewline
\isacommand{next}\isamarkupfalse%
\isanewline
\ \ \isacommand{have}\isamarkupfalse%
\ m{\isacharunderscore}{\kern0pt}type{\isacharcolon}{\kern0pt}\ {\isachardoublequoteopen}m\ {\isacharcolon}{\kern0pt}\ B\ {\isasymrightarrow}\ Y{\isachardoublequoteclose}\ {\isachardoublequoteopen}monomorphism\ m{\isachardoublequoteclose}\isanewline
\ \ \ \ \isacommand{using}\isamarkupfalse%
\ assms{\isacharparenleft}{\kern0pt}{\isadigit{2}}{\isacharparenright}{\kern0pt}\ \isacommand{unfolding}\isamarkupfalse%
\ subobject{\isacharunderscore}{\kern0pt}of{\isacharunderscore}{\kern0pt}def{\isadigit{2}}\ \isacommand{by}\isamarkupfalse%
\ auto\isanewline
\isanewline
\ \ \isacommand{assume}\isamarkupfalse%
\ {\isachardoublequoteopen}f\ {\isasymcirc}\isactrlsub c\ x\ {\isasymin}\isactrlbsub Y\isactrlesub \ {\isacharparenleft}{\kern0pt}B{\isacharcomma}{\kern0pt}\ m{\isacharparenright}{\kern0pt}{\isachardoublequoteclose}\isanewline
\ \ \isacommand{then}\isamarkupfalse%
\ \isacommand{have}\isamarkupfalse%
\ {\isachardoublequoteopen}{\isasymexists}h{\isachardot}{\kern0pt}\ h\ {\isacharcolon}{\kern0pt}\ domain\ {\isacharparenleft}{\kern0pt}f\ {\isasymcirc}\isactrlsub c\ x{\isacharparenright}{\kern0pt}\ {\isasymrightarrow}\ domain\ m\ {\isasymand}\ m\ {\isasymcirc}\isactrlsub c\ h\ {\isacharequal}{\kern0pt}\ f\ {\isasymcirc}\isactrlsub c\ x{\isachardoublequoteclose}\isanewline
\ \ \ \ \isacommand{unfolding}\isamarkupfalse%
\ relative{\isacharunderscore}{\kern0pt}member{\isacharunderscore}{\kern0pt}def{\isadigit{2}}\ factors{\isacharunderscore}{\kern0pt}through{\isacharunderscore}{\kern0pt}def\ \isacommand{by}\isamarkupfalse%
\ auto\isanewline
\ \ \isacommand{then}\isamarkupfalse%
\ \isacommand{obtain}\isamarkupfalse%
\ h\ \isakeyword{where}\ h{\isacharunderscore}{\kern0pt}type{\isacharcolon}{\kern0pt}\ {\isachardoublequoteopen}h\ {\isasymin}\isactrlsub c\ B{\isachardoublequoteclose}\ \isakeyword{and}\ h{\isacharunderscore}{\kern0pt}def{\isacharcolon}{\kern0pt}\ {\isachardoublequoteopen}m\ {\isasymcirc}\isactrlsub c\ h\ {\isacharequal}{\kern0pt}\ f\ {\isasymcirc}\isactrlsub c\ x{\isachardoublequoteclose}\isanewline
\ \ \ \ \isacommand{unfolding}\isamarkupfalse%
\ relative{\isacharunderscore}{\kern0pt}member{\isacharunderscore}{\kern0pt}def{\isadigit{2}}\ factors{\isacharunderscore}{\kern0pt}through{\isacharunderscore}{\kern0pt}def\ \isanewline
\ \ \ \ \isacommand{using}\isamarkupfalse%
\ assms\ cfunc{\isacharunderscore}{\kern0pt}type{\isacharunderscore}{\kern0pt}def\ domain{\isacharunderscore}{\kern0pt}comp\ m{\isacharunderscore}{\kern0pt}type\ \isacommand{by}\isamarkupfalse%
\ auto\isanewline
\ \ \isacommand{then}\isamarkupfalse%
\ \isacommand{have}\isamarkupfalse%
\ {\isachardoublequoteopen}{\isasymexists}j{\isachardot}{\kern0pt}\ j\ {\isasymin}\isactrlsub c\ {\isacharparenleft}{\kern0pt}f\isactrlsup {\isacharminus}{\kern0pt}\isactrlsup {\isadigit{1}}{\isasymlparr}B{\isasymrparr}\isactrlbsub m\isactrlesub {\isacharparenright}{\kern0pt}\ {\isasymand}\isanewline
\ \ \ \ \ \ \ \ \ {\isacharparenleft}{\kern0pt}right{\isacharunderscore}{\kern0pt}cart{\isacharunderscore}{\kern0pt}proj\ X\ B\ {\isasymcirc}\isactrlsub c\ inverse{\isacharunderscore}{\kern0pt}image{\isacharunderscore}{\kern0pt}mapping\ f\ B\ m{\isacharparenright}{\kern0pt}\ {\isasymcirc}\isactrlsub c\ j\ {\isacharequal}{\kern0pt}\ h\ {\isasymand}\isanewline
\ \ \ \ \ \ \ \ \ {\isacharparenleft}{\kern0pt}left{\isacharunderscore}{\kern0pt}cart{\isacharunderscore}{\kern0pt}proj\ X\ B\ {\isasymcirc}\isactrlsub c\ inverse{\isacharunderscore}{\kern0pt}image{\isacharunderscore}{\kern0pt}mapping\ f\ B\ m{\isacharparenright}{\kern0pt}\ {\isasymcirc}\isactrlsub c\ j\ {\isacharequal}{\kern0pt}\ x{\isachardoublequoteclose}\isanewline
\ \ \ \ \isacommand{using}\isamarkupfalse%
\ inverse{\isacharunderscore}{\kern0pt}image{\isacharunderscore}{\kern0pt}pullback\ assms\ m{\isacharunderscore}{\kern0pt}type\ \isacommand{unfolding}\isamarkupfalse%
\ is{\isacharunderscore}{\kern0pt}pullback{\isacharunderscore}{\kern0pt}def\ \isacommand{by}\isamarkupfalse%
\ blast\isanewline
\ \ \isacommand{then}\isamarkupfalse%
\ \isacommand{have}\isamarkupfalse%
\ {\isachardoublequoteopen}x\ factorsthru\ {\isacharparenleft}{\kern0pt}left{\isacharunderscore}{\kern0pt}cart{\isacharunderscore}{\kern0pt}proj\ X\ B\ {\isasymcirc}\isactrlsub c\ inverse{\isacharunderscore}{\kern0pt}image{\isacharunderscore}{\kern0pt}mapping\ f\ B\ m{\isacharparenright}{\kern0pt}{\isachardoublequoteclose}\isanewline
\ \ \ \ \isacommand{using}\isamarkupfalse%
\ m{\isacharunderscore}{\kern0pt}type\ assms\ cfunc{\isacharunderscore}{\kern0pt}type{\isacharunderscore}{\kern0pt}def\ \isacommand{by}\isamarkupfalse%
\ {\isacharparenleft}{\kern0pt}typecheck{\isacharunderscore}{\kern0pt}cfuncs{\isacharcomma}{\kern0pt}\ unfold\ factors{\isacharunderscore}{\kern0pt}through{\isacharunderscore}{\kern0pt}def{\isacharcomma}{\kern0pt}\ auto{\isacharparenright}{\kern0pt}\isanewline
\ \ \isacommand{then}\isamarkupfalse%
\ \isacommand{show}\isamarkupfalse%
\ {\isachardoublequoteopen}x\ {\isasymin}\isactrlbsub X\isactrlesub \ {\isacharparenleft}{\kern0pt}f\isactrlsup {\isacharminus}{\kern0pt}\isactrlsup {\isadigit{1}}{\isasymlparr}B{\isasymrparr}\isactrlbsub m\isactrlesub {\isacharcomma}{\kern0pt}\ left{\isacharunderscore}{\kern0pt}cart{\isacharunderscore}{\kern0pt}proj\ X\ B\ {\isasymcirc}\isactrlsub c\ inverse{\isacharunderscore}{\kern0pt}image{\isacharunderscore}{\kern0pt}mapping\ f\ B\ m{\isacharparenright}{\kern0pt}{\isachardoublequoteclose}\isanewline
\ \ \ \ \isacommand{unfolding}\isamarkupfalse%
\ relative{\isacharunderscore}{\kern0pt}member{\isacharunderscore}{\kern0pt}def{\isadigit{2}}\ \isacommand{using}\isamarkupfalse%
\ m{\isacharunderscore}{\kern0pt}type\ assms\isanewline
\ \ \ \ \isacommand{by}\isamarkupfalse%
\ {\isacharparenleft}{\kern0pt}typecheck{\isacharunderscore}{\kern0pt}cfuncs{\isacharcomma}{\kern0pt}\ simp\ add{\isacharcolon}{\kern0pt}\ inverse{\isacharunderscore}{\kern0pt}image{\isacharunderscore}{\kern0pt}monomorphism{\isacharparenright}{\kern0pt}\isanewline
\isacommand{qed}\isamarkupfalse%
%
\endisatagproof
{\isafoldproof}%
%
\isadelimproof
%
\endisadelimproof
%
\isadelimdocument
%
\endisadelimdocument
%
\isatagdocument
%
\isamarkupsubsection{Fibered Products%
}
\isamarkuptrue%
%
\endisatagdocument
{\isafolddocument}%
%
\isadelimdocument
%
\endisadelimdocument
%
\begin{isamarkuptext}%
The definition below corresponds to Definition 2.1.42 in Halvorson.%
\end{isamarkuptext}\isamarkuptrue%
\isacommand{definition}\isamarkupfalse%
\ fibered{\isacharunderscore}{\kern0pt}product\ {\isacharcolon}{\kern0pt}{\isacharcolon}{\kern0pt}\ {\isachardoublequoteopen}cset\ {\isasymRightarrow}\ cfunc\ {\isasymRightarrow}\ cfunc\ {\isasymRightarrow}\ cset\ {\isasymRightarrow}\ cset{\isachardoublequoteclose}\ {\isacharparenleft}{\kern0pt}{\isachardoublequoteopen}{\isacharunderscore}{\kern0pt}\ \isactrlbsub {\isacharunderscore}{\kern0pt}\isactrlesub {\isasymtimes}\isactrlsub c\isactrlbsub {\isacharunderscore}{\kern0pt}\isactrlesub \ {\isacharunderscore}{\kern0pt}{\isachardoublequoteclose}\ {\isacharbrackleft}{\kern0pt}{\isadigit{6}}{\isadigit{6}}{\isacharcomma}{\kern0pt}{\isadigit{5}}{\isadigit{0}}{\isacharcomma}{\kern0pt}{\isadigit{5}}{\isadigit{0}}{\isacharcomma}{\kern0pt}{\isadigit{6}}{\isadigit{5}}{\isacharbrackright}{\kern0pt}{\isadigit{6}}{\isadigit{5}}{\isacharparenright}{\kern0pt}\ \isakeyword{where}\isanewline
\ \ {\isachardoublequoteopen}X\ \isactrlbsub f\isactrlesub {\isasymtimes}\isactrlsub c\isactrlbsub g\isactrlesub \ Y\ {\isacharequal}{\kern0pt}\ {\isacharparenleft}{\kern0pt}SOME\ E{\isachardot}{\kern0pt}\ {\isasymexists}\ Z\ m{\isachardot}{\kern0pt}\ f\ {\isacharcolon}{\kern0pt}\ X\ {\isasymrightarrow}\ Z\ {\isasymand}\ g\ {\isacharcolon}{\kern0pt}\ Y\ {\isasymrightarrow}\ Z\ {\isasymand}\isanewline
\ \ \ \ equalizer\ E\ m\ {\isacharparenleft}{\kern0pt}f\ {\isasymcirc}\isactrlsub c\ left{\isacharunderscore}{\kern0pt}cart{\isacharunderscore}{\kern0pt}proj\ X\ Y{\isacharparenright}{\kern0pt}\ {\isacharparenleft}{\kern0pt}g\ {\isasymcirc}\isactrlsub c\ right{\isacharunderscore}{\kern0pt}cart{\isacharunderscore}{\kern0pt}proj\ X\ Y{\isacharparenright}{\kern0pt}{\isacharparenright}{\kern0pt}{\isachardoublequoteclose}\isanewline
\isanewline
\isacommand{lemma}\isamarkupfalse%
\ fibered{\isacharunderscore}{\kern0pt}product{\isacharunderscore}{\kern0pt}equalizer{\isacharcolon}{\kern0pt}\isanewline
\ \ \isakeyword{assumes}\ {\isachardoublequoteopen}f\ {\isacharcolon}{\kern0pt}\ X\ {\isasymrightarrow}\ Z{\isachardoublequoteclose}\ {\isachardoublequoteopen}g\ {\isacharcolon}{\kern0pt}\ Y\ {\isasymrightarrow}\ Z{\isachardoublequoteclose}\isanewline
\ \ \isakeyword{shows}\ {\isachardoublequoteopen}{\isasymexists}\ m{\isachardot}{\kern0pt}\ equalizer\ {\isacharparenleft}{\kern0pt}X\ \isactrlbsub f\isactrlesub {\isasymtimes}\isactrlsub c\isactrlbsub g\isactrlesub \ Y{\isacharparenright}{\kern0pt}\ m\ {\isacharparenleft}{\kern0pt}f\ {\isasymcirc}\isactrlsub c\ left{\isacharunderscore}{\kern0pt}cart{\isacharunderscore}{\kern0pt}proj\ X\ Y{\isacharparenright}{\kern0pt}\ {\isacharparenleft}{\kern0pt}g\ {\isasymcirc}\isactrlsub c\ right{\isacharunderscore}{\kern0pt}cart{\isacharunderscore}{\kern0pt}proj\ X\ Y{\isacharparenright}{\kern0pt}{\isachardoublequoteclose}\isanewline
%
\isadelimproof
%
\endisadelimproof
%
\isatagproof
\isacommand{proof}\isamarkupfalse%
\ {\isacharminus}{\kern0pt}\isanewline
\ \ \isacommand{obtain}\isamarkupfalse%
\ E\ m\ \isakeyword{where}\ {\isachardoublequoteopen}equalizer\ E\ m\ {\isacharparenleft}{\kern0pt}f\ {\isasymcirc}\isactrlsub c\ left{\isacharunderscore}{\kern0pt}cart{\isacharunderscore}{\kern0pt}proj\ X\ Y{\isacharparenright}{\kern0pt}\ {\isacharparenleft}{\kern0pt}g\ {\isasymcirc}\isactrlsub c\ right{\isacharunderscore}{\kern0pt}cart{\isacharunderscore}{\kern0pt}proj\ X\ Y{\isacharparenright}{\kern0pt}{\isachardoublequoteclose}\isanewline
\ \ \ \ \isacommand{using}\isamarkupfalse%
\ assms\ equalizer{\isacharunderscore}{\kern0pt}exists\ \isacommand{by}\isamarkupfalse%
\ {\isacharparenleft}{\kern0pt}typecheck{\isacharunderscore}{\kern0pt}cfuncs{\isacharcomma}{\kern0pt}\ blast{\isacharparenright}{\kern0pt}\isanewline
\ \ \isacommand{then}\isamarkupfalse%
\ \isacommand{have}\isamarkupfalse%
\ {\isachardoublequoteopen}{\isasymexists}x\ Z\ m{\isachardot}{\kern0pt}\ f\ {\isacharcolon}{\kern0pt}\ X\ {\isasymrightarrow}\ Z\ {\isasymand}\ g\ {\isacharcolon}{\kern0pt}\ Y\ {\isasymrightarrow}\ Z\ {\isasymand}\isanewline
\ \ \ \ \ \ equalizer\ x\ m\ {\isacharparenleft}{\kern0pt}f\ {\isasymcirc}\isactrlsub c\ left{\isacharunderscore}{\kern0pt}cart{\isacharunderscore}{\kern0pt}proj\ X\ Y{\isacharparenright}{\kern0pt}\ {\isacharparenleft}{\kern0pt}g\ {\isasymcirc}\isactrlsub c\ right{\isacharunderscore}{\kern0pt}cart{\isacharunderscore}{\kern0pt}proj\ X\ Y{\isacharparenright}{\kern0pt}{\isachardoublequoteclose}\isanewline
\ \ \ \ \isacommand{using}\isamarkupfalse%
\ assms\ \isacommand{by}\isamarkupfalse%
\ blast\isanewline
\ \ \isacommand{then}\isamarkupfalse%
\ \isacommand{have}\isamarkupfalse%
\ {\isachardoublequoteopen}{\isasymexists}\ Z\ m{\isachardot}{\kern0pt}\ f\ {\isacharcolon}{\kern0pt}\ X\ {\isasymrightarrow}\ Z\ {\isasymand}\ g\ {\isacharcolon}{\kern0pt}\ Y\ {\isasymrightarrow}\ Z\ {\isasymand}\ \isanewline
\ \ \ \ \ \ equalizer\ {\isacharparenleft}{\kern0pt}X\ \isactrlbsub f\isactrlesub {\isasymtimes}\isactrlsub c\isactrlbsub g\isactrlesub \ Y{\isacharparenright}{\kern0pt}\ m\ {\isacharparenleft}{\kern0pt}f\ {\isasymcirc}\isactrlsub c\ left{\isacharunderscore}{\kern0pt}cart{\isacharunderscore}{\kern0pt}proj\ X\ Y{\isacharparenright}{\kern0pt}\ {\isacharparenleft}{\kern0pt}g\ {\isasymcirc}\isactrlsub c\ right{\isacharunderscore}{\kern0pt}cart{\isacharunderscore}{\kern0pt}proj\ X\ Y{\isacharparenright}{\kern0pt}{\isachardoublequoteclose}\isanewline
\ \ \ \ \isacommand{unfolding}\isamarkupfalse%
\ fibered{\isacharunderscore}{\kern0pt}product{\isacharunderscore}{\kern0pt}def\ \isacommand{by}\isamarkupfalse%
\ {\isacharparenleft}{\kern0pt}rule\ someI{\isacharunderscore}{\kern0pt}ex{\isacharparenright}{\kern0pt}\isanewline
\ \ \isacommand{then}\isamarkupfalse%
\ \isacommand{show}\isamarkupfalse%
\ {\isachardoublequoteopen}{\isasymexists}m{\isachardot}{\kern0pt}\ equalizer\ {\isacharparenleft}{\kern0pt}X\ \isactrlbsub f\isactrlesub {\isasymtimes}\isactrlsub c\isactrlbsub g\isactrlesub \ Y{\isacharparenright}{\kern0pt}\ m\ {\isacharparenleft}{\kern0pt}f\ {\isasymcirc}\isactrlsub c\ left{\isacharunderscore}{\kern0pt}cart{\isacharunderscore}{\kern0pt}proj\ X\ Y{\isacharparenright}{\kern0pt}\ {\isacharparenleft}{\kern0pt}g\ {\isasymcirc}\isactrlsub c\ right{\isacharunderscore}{\kern0pt}cart{\isacharunderscore}{\kern0pt}proj\ X\ Y{\isacharparenright}{\kern0pt}{\isachardoublequoteclose}\isanewline
\ \ \ \ \isacommand{by}\isamarkupfalse%
\ auto\isanewline
\isacommand{qed}\isamarkupfalse%
%
\endisatagproof
{\isafoldproof}%
%
\isadelimproof
\isanewline
%
\endisadelimproof
\isanewline
\isacommand{definition}\isamarkupfalse%
\ fibered{\isacharunderscore}{\kern0pt}product{\isacharunderscore}{\kern0pt}morphism\ {\isacharcolon}{\kern0pt}{\isacharcolon}{\kern0pt}\ {\isachardoublequoteopen}cset\ {\isasymRightarrow}\ cfunc\ {\isasymRightarrow}\ cfunc\ {\isasymRightarrow}\ cset\ {\isasymRightarrow}\ cfunc{\isachardoublequoteclose}\ \isakeyword{where}\isanewline
\ \ {\isachardoublequoteopen}fibered{\isacharunderscore}{\kern0pt}product{\isacharunderscore}{\kern0pt}morphism\ X\ f\ g\ Y\ {\isacharequal}{\kern0pt}\ {\isacharparenleft}{\kern0pt}SOME\ m{\isachardot}{\kern0pt}\ {\isasymexists}\ Z{\isachardot}{\kern0pt}\ f\ {\isacharcolon}{\kern0pt}\ X\ {\isasymrightarrow}\ Z\ {\isasymand}\ g\ {\isacharcolon}{\kern0pt}\ Y\ {\isasymrightarrow}\ Z\ {\isasymand}\isanewline
\ \ \ \ equalizer\ {\isacharparenleft}{\kern0pt}X\ \isactrlbsub f\isactrlesub {\isasymtimes}\isactrlsub c\isactrlbsub g\isactrlesub \ Y{\isacharparenright}{\kern0pt}\ m\ {\isacharparenleft}{\kern0pt}f\ {\isasymcirc}\isactrlsub c\ left{\isacharunderscore}{\kern0pt}cart{\isacharunderscore}{\kern0pt}proj\ X\ Y{\isacharparenright}{\kern0pt}\ {\isacharparenleft}{\kern0pt}g\ {\isasymcirc}\isactrlsub c\ right{\isacharunderscore}{\kern0pt}cart{\isacharunderscore}{\kern0pt}proj\ X\ Y{\isacharparenright}{\kern0pt}{\isacharparenright}{\kern0pt}{\isachardoublequoteclose}\isanewline
\isanewline
\isacommand{lemma}\isamarkupfalse%
\ fibered{\isacharunderscore}{\kern0pt}product{\isacharunderscore}{\kern0pt}morphism{\isacharunderscore}{\kern0pt}equalizer{\isacharcolon}{\kern0pt}\isanewline
\ \ \isakeyword{assumes}\ {\isachardoublequoteopen}f\ {\isacharcolon}{\kern0pt}\ X\ {\isasymrightarrow}\ Z{\isachardoublequoteclose}\ {\isachardoublequoteopen}g\ {\isacharcolon}{\kern0pt}\ Y\ {\isasymrightarrow}\ Z{\isachardoublequoteclose}\isanewline
\ \ \isakeyword{shows}\ {\isachardoublequoteopen}equalizer\ {\isacharparenleft}{\kern0pt}X\ \isactrlbsub f\isactrlesub {\isasymtimes}\isactrlsub c\isactrlbsub g\isactrlesub \ Y{\isacharparenright}{\kern0pt}\ {\isacharparenleft}{\kern0pt}fibered{\isacharunderscore}{\kern0pt}product{\isacharunderscore}{\kern0pt}morphism\ X\ f\ g\ Y{\isacharparenright}{\kern0pt}\ {\isacharparenleft}{\kern0pt}f\ {\isasymcirc}\isactrlsub c\ left{\isacharunderscore}{\kern0pt}cart{\isacharunderscore}{\kern0pt}proj\ X\ Y{\isacharparenright}{\kern0pt}\ {\isacharparenleft}{\kern0pt}g\ {\isasymcirc}\isactrlsub c\ right{\isacharunderscore}{\kern0pt}cart{\isacharunderscore}{\kern0pt}proj\ X\ Y{\isacharparenright}{\kern0pt}{\isachardoublequoteclose}\isanewline
%
\isadelimproof
%
\endisadelimproof
%
\isatagproof
\isacommand{proof}\isamarkupfalse%
\ {\isacharminus}{\kern0pt}\isanewline
\ \ \isacommand{have}\isamarkupfalse%
\ {\isachardoublequoteopen}{\isasymexists}x\ Z{\isachardot}{\kern0pt}\ f\ {\isacharcolon}{\kern0pt}\ X\ {\isasymrightarrow}\ Z\ {\isasymand}\isanewline
\ \ \ \ \ \ g\ {\isacharcolon}{\kern0pt}\ Y\ {\isasymrightarrow}\ Z\ {\isasymand}\ equalizer\ {\isacharparenleft}{\kern0pt}X\ \isactrlbsub f\isactrlesub {\isasymtimes}\isactrlsub c\isactrlbsub g\isactrlesub \ Y{\isacharparenright}{\kern0pt}\ x\ {\isacharparenleft}{\kern0pt}f\ {\isasymcirc}\isactrlsub c\ left{\isacharunderscore}{\kern0pt}cart{\isacharunderscore}{\kern0pt}proj\ X\ Y{\isacharparenright}{\kern0pt}\ {\isacharparenleft}{\kern0pt}g\ {\isasymcirc}\isactrlsub c\ right{\isacharunderscore}{\kern0pt}cart{\isacharunderscore}{\kern0pt}proj\ X\ Y{\isacharparenright}{\kern0pt}{\isachardoublequoteclose}\isanewline
\ \ \ \ \isacommand{using}\isamarkupfalse%
\ assms\ fibered{\isacharunderscore}{\kern0pt}product{\isacharunderscore}{\kern0pt}equalizer\ \isacommand{by}\isamarkupfalse%
\ blast\isanewline
\ \ \isacommand{then}\isamarkupfalse%
\ \isacommand{have}\isamarkupfalse%
\ {\isachardoublequoteopen}{\isasymexists}Z{\isachardot}{\kern0pt}\ f\ {\isacharcolon}{\kern0pt}\ X\ {\isasymrightarrow}\ Z\ {\isasymand}\ g\ {\isacharcolon}{\kern0pt}\ Y\ {\isasymrightarrow}\ Z\ {\isasymand}\isanewline
\ \ \ \ equalizer\ {\isacharparenleft}{\kern0pt}X\ \isactrlbsub f\isactrlesub {\isasymtimes}\isactrlsub c\isactrlbsub g\isactrlesub \ Y{\isacharparenright}{\kern0pt}\ {\isacharparenleft}{\kern0pt}fibered{\isacharunderscore}{\kern0pt}product{\isacharunderscore}{\kern0pt}morphism\ X\ f\ g\ Y{\isacharparenright}{\kern0pt}\ {\isacharparenleft}{\kern0pt}f\ {\isasymcirc}\isactrlsub c\ left{\isacharunderscore}{\kern0pt}cart{\isacharunderscore}{\kern0pt}proj\ X\ Y{\isacharparenright}{\kern0pt}\ {\isacharparenleft}{\kern0pt}g\ {\isasymcirc}\isactrlsub c\ right{\isacharunderscore}{\kern0pt}cart{\isacharunderscore}{\kern0pt}proj\ X\ Y{\isacharparenright}{\kern0pt}{\isachardoublequoteclose}\isanewline
\ \ \ \ \isacommand{unfolding}\isamarkupfalse%
\ fibered{\isacharunderscore}{\kern0pt}product{\isacharunderscore}{\kern0pt}morphism{\isacharunderscore}{\kern0pt}def\ \isacommand{by}\isamarkupfalse%
\ {\isacharparenleft}{\kern0pt}rule\ someI{\isacharunderscore}{\kern0pt}ex{\isacharparenright}{\kern0pt}\isanewline
\ \ \isacommand{then}\isamarkupfalse%
\ \isacommand{show}\isamarkupfalse%
\ {\isachardoublequoteopen}equalizer\ {\isacharparenleft}{\kern0pt}X\ \isactrlbsub f\isactrlesub {\isasymtimes}\isactrlsub c\isactrlbsub g\isactrlesub \ Y{\isacharparenright}{\kern0pt}\ {\isacharparenleft}{\kern0pt}fibered{\isacharunderscore}{\kern0pt}product{\isacharunderscore}{\kern0pt}morphism\ X\ f\ g\ Y{\isacharparenright}{\kern0pt}\ {\isacharparenleft}{\kern0pt}f\ {\isasymcirc}\isactrlsub c\ left{\isacharunderscore}{\kern0pt}cart{\isacharunderscore}{\kern0pt}proj\ X\ Y{\isacharparenright}{\kern0pt}\ {\isacharparenleft}{\kern0pt}g\ {\isasymcirc}\isactrlsub c\ right{\isacharunderscore}{\kern0pt}cart{\isacharunderscore}{\kern0pt}proj\ X\ Y{\isacharparenright}{\kern0pt}{\isachardoublequoteclose}\isanewline
\ \ \ \ \isacommand{by}\isamarkupfalse%
\ auto\isanewline
\isacommand{qed}\isamarkupfalse%
%
\endisatagproof
{\isafoldproof}%
%
\isadelimproof
\isanewline
%
\endisadelimproof
\isanewline
\isacommand{lemma}\isamarkupfalse%
\ fibered{\isacharunderscore}{\kern0pt}product{\isacharunderscore}{\kern0pt}morphism{\isacharunderscore}{\kern0pt}type{\isacharbrackleft}{\kern0pt}type{\isacharunderscore}{\kern0pt}rule{\isacharbrackright}{\kern0pt}{\isacharcolon}{\kern0pt}\isanewline
\ \ \isakeyword{assumes}\ {\isachardoublequoteopen}f\ {\isacharcolon}{\kern0pt}\ X\ {\isasymrightarrow}\ Z{\isachardoublequoteclose}\ {\isachardoublequoteopen}g\ {\isacharcolon}{\kern0pt}\ Y\ {\isasymrightarrow}\ Z{\isachardoublequoteclose}\isanewline
\ \ \isakeyword{shows}\ {\isachardoublequoteopen}fibered{\isacharunderscore}{\kern0pt}product{\isacharunderscore}{\kern0pt}morphism\ X\ f\ g\ Y\ {\isacharcolon}{\kern0pt}\ X\ \isactrlbsub f\isactrlesub {\isasymtimes}\isactrlsub c\isactrlbsub g\isactrlesub \ Y\ {\isasymrightarrow}\ X\ {\isasymtimes}\isactrlsub c\ Y{\isachardoublequoteclose}\isanewline
%
\isadelimproof
\ \ %
\endisadelimproof
%
\isatagproof
\isacommand{using}\isamarkupfalse%
\ assms\ cfunc{\isacharunderscore}{\kern0pt}type{\isacharunderscore}{\kern0pt}def\ domain{\isacharunderscore}{\kern0pt}comp\ equalizer{\isacharunderscore}{\kern0pt}def\ fibered{\isacharunderscore}{\kern0pt}product{\isacharunderscore}{\kern0pt}morphism{\isacharunderscore}{\kern0pt}equalizer\ left{\isacharunderscore}{\kern0pt}cart{\isacharunderscore}{\kern0pt}proj{\isacharunderscore}{\kern0pt}type\ \isacommand{by}\isamarkupfalse%
\ auto%
\endisatagproof
{\isafoldproof}%
%
\isadelimproof
\isanewline
%
\endisadelimproof
\isanewline
\isacommand{lemma}\isamarkupfalse%
\ fibered{\isacharunderscore}{\kern0pt}product{\isacharunderscore}{\kern0pt}morphism{\isacharunderscore}{\kern0pt}monomorphism{\isacharcolon}{\kern0pt}\isanewline
\ \ \isakeyword{assumes}\ {\isachardoublequoteopen}f\ {\isacharcolon}{\kern0pt}\ X\ {\isasymrightarrow}\ Z{\isachardoublequoteclose}\ {\isachardoublequoteopen}g\ {\isacharcolon}{\kern0pt}\ Y\ {\isasymrightarrow}\ Z{\isachardoublequoteclose}\isanewline
\ \ \isakeyword{shows}\ {\isachardoublequoteopen}monomorphism\ {\isacharparenleft}{\kern0pt}fibered{\isacharunderscore}{\kern0pt}product{\isacharunderscore}{\kern0pt}morphism\ X\ f\ g\ Y{\isacharparenright}{\kern0pt}{\isachardoublequoteclose}\isanewline
%
\isadelimproof
\ \ %
\endisadelimproof
%
\isatagproof
\isacommand{using}\isamarkupfalse%
\ assms\ equalizer{\isacharunderscore}{\kern0pt}is{\isacharunderscore}{\kern0pt}monomorphism\ fibered{\isacharunderscore}{\kern0pt}product{\isacharunderscore}{\kern0pt}morphism{\isacharunderscore}{\kern0pt}equalizer\ \isacommand{by}\isamarkupfalse%
\ blast%
\endisatagproof
{\isafoldproof}%
%
\isadelimproof
\isanewline
%
\endisadelimproof
\isanewline
\isacommand{definition}\isamarkupfalse%
\ fibered{\isacharunderscore}{\kern0pt}product{\isacharunderscore}{\kern0pt}left{\isacharunderscore}{\kern0pt}proj\ {\isacharcolon}{\kern0pt}{\isacharcolon}{\kern0pt}\ {\isachardoublequoteopen}cset\ {\isasymRightarrow}\ cfunc\ {\isasymRightarrow}\ cfunc\ {\isasymRightarrow}\ cset\ {\isasymRightarrow}\ cfunc{\isachardoublequoteclose}\ \isakeyword{where}\isanewline
\ \ {\isachardoublequoteopen}fibered{\isacharunderscore}{\kern0pt}product{\isacharunderscore}{\kern0pt}left{\isacharunderscore}{\kern0pt}proj\ X\ f\ g\ Y\ {\isacharequal}{\kern0pt}\ {\isacharparenleft}{\kern0pt}left{\isacharunderscore}{\kern0pt}cart{\isacharunderscore}{\kern0pt}proj\ X\ Y{\isacharparenright}{\kern0pt}\ {\isasymcirc}\isactrlsub c\ {\isacharparenleft}{\kern0pt}fibered{\isacharunderscore}{\kern0pt}product{\isacharunderscore}{\kern0pt}morphism\ X\ f\ g\ Y{\isacharparenright}{\kern0pt}{\isachardoublequoteclose}\isanewline
\isanewline
\isacommand{lemma}\isamarkupfalse%
\ fibered{\isacharunderscore}{\kern0pt}product{\isacharunderscore}{\kern0pt}left{\isacharunderscore}{\kern0pt}proj{\isacharunderscore}{\kern0pt}type{\isacharbrackleft}{\kern0pt}type{\isacharunderscore}{\kern0pt}rule{\isacharbrackright}{\kern0pt}{\isacharcolon}{\kern0pt}\isanewline
\ \ \isakeyword{assumes}\ {\isachardoublequoteopen}f\ {\isacharcolon}{\kern0pt}\ X\ {\isasymrightarrow}\ Z{\isachardoublequoteclose}\ {\isachardoublequoteopen}g\ {\isacharcolon}{\kern0pt}\ Y\ {\isasymrightarrow}\ Z{\isachardoublequoteclose}\isanewline
\ \ \isakeyword{shows}\ {\isachardoublequoteopen}fibered{\isacharunderscore}{\kern0pt}product{\isacharunderscore}{\kern0pt}left{\isacharunderscore}{\kern0pt}proj\ X\ f\ g\ Y\ {\isacharcolon}{\kern0pt}\ X\ \isactrlbsub f\isactrlesub {\isasymtimes}\isactrlsub c\isactrlbsub g\isactrlesub \ Y\ {\isasymrightarrow}\ X{\isachardoublequoteclose}\isanewline
%
\isadelimproof
\ \ %
\endisadelimproof
%
\isatagproof
\isacommand{by}\isamarkupfalse%
\ {\isacharparenleft}{\kern0pt}metis\ assms\ comp{\isacharunderscore}{\kern0pt}type\ fibered{\isacharunderscore}{\kern0pt}product{\isacharunderscore}{\kern0pt}left{\isacharunderscore}{\kern0pt}proj{\isacharunderscore}{\kern0pt}def\ fibered{\isacharunderscore}{\kern0pt}product{\isacharunderscore}{\kern0pt}morphism{\isacharunderscore}{\kern0pt}type\ left{\isacharunderscore}{\kern0pt}cart{\isacharunderscore}{\kern0pt}proj{\isacharunderscore}{\kern0pt}type{\isacharparenright}{\kern0pt}%
\endisatagproof
{\isafoldproof}%
%
\isadelimproof
\isanewline
%
\endisadelimproof
\isanewline
\isacommand{definition}\isamarkupfalse%
\ fibered{\isacharunderscore}{\kern0pt}product{\isacharunderscore}{\kern0pt}right{\isacharunderscore}{\kern0pt}proj\ {\isacharcolon}{\kern0pt}{\isacharcolon}{\kern0pt}\ {\isachardoublequoteopen}cset\ {\isasymRightarrow}\ cfunc\ {\isasymRightarrow}\ cfunc\ {\isasymRightarrow}\ cset\ {\isasymRightarrow}\ cfunc{\isachardoublequoteclose}\ \isakeyword{where}\isanewline
\ \ {\isachardoublequoteopen}fibered{\isacharunderscore}{\kern0pt}product{\isacharunderscore}{\kern0pt}right{\isacharunderscore}{\kern0pt}proj\ X\ f\ g\ Y\ {\isacharequal}{\kern0pt}\ {\isacharparenleft}{\kern0pt}right{\isacharunderscore}{\kern0pt}cart{\isacharunderscore}{\kern0pt}proj\ X\ Y{\isacharparenright}{\kern0pt}\ {\isasymcirc}\isactrlsub c\ {\isacharparenleft}{\kern0pt}fibered{\isacharunderscore}{\kern0pt}product{\isacharunderscore}{\kern0pt}morphism\ X\ f\ g\ Y{\isacharparenright}{\kern0pt}{\isachardoublequoteclose}\isanewline
\isanewline
\isacommand{lemma}\isamarkupfalse%
\ fibered{\isacharunderscore}{\kern0pt}product{\isacharunderscore}{\kern0pt}right{\isacharunderscore}{\kern0pt}proj{\isacharunderscore}{\kern0pt}type{\isacharbrackleft}{\kern0pt}type{\isacharunderscore}{\kern0pt}rule{\isacharbrackright}{\kern0pt}{\isacharcolon}{\kern0pt}\isanewline
\ \ \isakeyword{assumes}\ {\isachardoublequoteopen}f\ {\isacharcolon}{\kern0pt}\ X\ {\isasymrightarrow}\ Z{\isachardoublequoteclose}\ {\isachardoublequoteopen}g\ {\isacharcolon}{\kern0pt}\ Y\ {\isasymrightarrow}\ Z{\isachardoublequoteclose}\isanewline
\ \ \isakeyword{shows}\ {\isachardoublequoteopen}fibered{\isacharunderscore}{\kern0pt}product{\isacharunderscore}{\kern0pt}right{\isacharunderscore}{\kern0pt}proj\ X\ f\ g\ Y\ {\isacharcolon}{\kern0pt}\ X\ \isactrlbsub f\isactrlesub {\isasymtimes}\isactrlsub c\isactrlbsub g\isactrlesub \ Y\ {\isasymrightarrow}\ Y{\isachardoublequoteclose}\isanewline
%
\isadelimproof
\ \ %
\endisadelimproof
%
\isatagproof
\isacommand{by}\isamarkupfalse%
\ {\isacharparenleft}{\kern0pt}metis\ assms\ comp{\isacharunderscore}{\kern0pt}type\ fibered{\isacharunderscore}{\kern0pt}product{\isacharunderscore}{\kern0pt}right{\isacharunderscore}{\kern0pt}proj{\isacharunderscore}{\kern0pt}def\ fibered{\isacharunderscore}{\kern0pt}product{\isacharunderscore}{\kern0pt}morphism{\isacharunderscore}{\kern0pt}type\ right{\isacharunderscore}{\kern0pt}cart{\isacharunderscore}{\kern0pt}proj{\isacharunderscore}{\kern0pt}type{\isacharparenright}{\kern0pt}%
\endisatagproof
{\isafoldproof}%
%
\isadelimproof
\isanewline
%
\endisadelimproof
\isanewline
\isacommand{lemma}\isamarkupfalse%
\ pair{\isacharunderscore}{\kern0pt}factorsthru{\isacharunderscore}{\kern0pt}fibered{\isacharunderscore}{\kern0pt}product{\isacharunderscore}{\kern0pt}morphism{\isacharcolon}{\kern0pt}\isanewline
\ \ \isakeyword{assumes}\ {\isachardoublequoteopen}f\ {\isacharcolon}{\kern0pt}\ X\ {\isasymrightarrow}\ Z{\isachardoublequoteclose}\ {\isachardoublequoteopen}g\ {\isacharcolon}{\kern0pt}\ Y\ {\isasymrightarrow}\ Z{\isachardoublequoteclose}\ {\isachardoublequoteopen}x\ {\isacharcolon}{\kern0pt}\ A\ {\isasymrightarrow}\ X{\isachardoublequoteclose}\ {\isachardoublequoteopen}y\ {\isacharcolon}{\kern0pt}\ A\ {\isasymrightarrow}\ Y{\isachardoublequoteclose}\isanewline
\ \ \isakeyword{shows}\ {\isachardoublequoteopen}f\ {\isasymcirc}\isactrlsub c\ x\ {\isacharequal}{\kern0pt}\ g\ {\isasymcirc}\isactrlsub c\ y\ {\isasymLongrightarrow}\ {\isasymlangle}x{\isacharcomma}{\kern0pt}y{\isasymrangle}\ factorsthru\ fibered{\isacharunderscore}{\kern0pt}product{\isacharunderscore}{\kern0pt}morphism\ X\ f\ g\ Y{\isachardoublequoteclose}\isanewline
%
\isadelimproof
\ \ %
\endisadelimproof
%
\isatagproof
\isacommand{unfolding}\isamarkupfalse%
\ factors{\isacharunderscore}{\kern0pt}through{\isacharunderscore}{\kern0pt}def\isanewline
\isacommand{proof}\isamarkupfalse%
\ {\isacharminus}{\kern0pt}\isanewline
\ \ \isacommand{have}\isamarkupfalse%
\ equalizer{\isacharcolon}{\kern0pt}\ {\isachardoublequoteopen}equalizer\ {\isacharparenleft}{\kern0pt}X\ \isactrlbsub f\isactrlesub {\isasymtimes}\isactrlsub c\isactrlbsub g\isactrlesub \ Y{\isacharparenright}{\kern0pt}\ {\isacharparenleft}{\kern0pt}fibered{\isacharunderscore}{\kern0pt}product{\isacharunderscore}{\kern0pt}morphism\ X\ f\ g\ Y{\isacharparenright}{\kern0pt}\ {\isacharparenleft}{\kern0pt}f\ {\isasymcirc}\isactrlsub c\ left{\isacharunderscore}{\kern0pt}cart{\isacharunderscore}{\kern0pt}proj\ X\ Y{\isacharparenright}{\kern0pt}\ {\isacharparenleft}{\kern0pt}g\ {\isasymcirc}\isactrlsub c\ right{\isacharunderscore}{\kern0pt}cart{\isacharunderscore}{\kern0pt}proj\ X\ Y{\isacharparenright}{\kern0pt}{\isachardoublequoteclose}\isanewline
\ \ \ \ \isacommand{using}\isamarkupfalse%
\ fibered{\isacharunderscore}{\kern0pt}product{\isacharunderscore}{\kern0pt}morphism{\isacharunderscore}{\kern0pt}equalizer\ assms\ \isacommand{by}\isamarkupfalse%
\ {\isacharparenleft}{\kern0pt}typecheck{\isacharunderscore}{\kern0pt}cfuncs{\isacharcomma}{\kern0pt}\ auto{\isacharparenright}{\kern0pt}\isanewline
\isanewline
\ \ \isacommand{assume}\isamarkupfalse%
\ {\isachardoublequoteopen}f\ {\isasymcirc}\isactrlsub c\ x\ {\isacharequal}{\kern0pt}\ g\ {\isasymcirc}\isactrlsub c\ y{\isachardoublequoteclose}\isanewline
\ \ \isacommand{then}\isamarkupfalse%
\ \isacommand{have}\isamarkupfalse%
\ {\isachardoublequoteopen}{\isacharparenleft}{\kern0pt}f\ {\isasymcirc}\isactrlsub c\ left{\isacharunderscore}{\kern0pt}cart{\isacharunderscore}{\kern0pt}proj\ X\ Y{\isacharparenright}{\kern0pt}\ {\isasymcirc}\isactrlsub c\ {\isasymlangle}x{\isacharcomma}{\kern0pt}y{\isasymrangle}\ {\isacharequal}{\kern0pt}\ {\isacharparenleft}{\kern0pt}g\ {\isasymcirc}\isactrlsub c\ right{\isacharunderscore}{\kern0pt}cart{\isacharunderscore}{\kern0pt}proj\ X\ Y{\isacharparenright}{\kern0pt}\ {\isasymcirc}\isactrlsub c\ {\isasymlangle}x{\isacharcomma}{\kern0pt}y{\isasymrangle}{\isachardoublequoteclose}\isanewline
\ \ \ \ \isacommand{using}\isamarkupfalse%
\ assms\ \isacommand{by}\isamarkupfalse%
\ {\isacharparenleft}{\kern0pt}typecheck{\isacharunderscore}{\kern0pt}cfuncs{\isacharcomma}{\kern0pt}\ smt\ comp{\isacharunderscore}{\kern0pt}associative{\isadigit{2}}\ left{\isacharunderscore}{\kern0pt}cart{\isacharunderscore}{\kern0pt}proj{\isacharunderscore}{\kern0pt}cfunc{\isacharunderscore}{\kern0pt}prod\ right{\isacharunderscore}{\kern0pt}cart{\isacharunderscore}{\kern0pt}proj{\isacharunderscore}{\kern0pt}cfunc{\isacharunderscore}{\kern0pt}prod{\isacharparenright}{\kern0pt}\isanewline
\ \ \isacommand{then}\isamarkupfalse%
\ \isacommand{have}\isamarkupfalse%
\ {\isachardoublequoteopen}{\isasymexists}{\isacharbang}{\kern0pt}\ h{\isachardot}{\kern0pt}\ h\ {\isacharcolon}{\kern0pt}\ A\ {\isasymrightarrow}\ X\ \isactrlbsub f\isactrlesub {\isasymtimes}\isactrlsub c\isactrlbsub g\isactrlesub \ Y\ {\isasymand}\ fibered{\isacharunderscore}{\kern0pt}product{\isacharunderscore}{\kern0pt}morphism\ X\ f\ g\ Y\ {\isasymcirc}\isactrlsub c\ h\ {\isacharequal}{\kern0pt}\ {\isasymlangle}x{\isacharcomma}{\kern0pt}y{\isasymrangle}{\isachardoublequoteclose}\isanewline
\ \ \ \ \isacommand{using}\isamarkupfalse%
\ assms\ similar{\isacharunderscore}{\kern0pt}equalizers\ \isacommand{by}\isamarkupfalse%
\ {\isacharparenleft}{\kern0pt}typecheck{\isacharunderscore}{\kern0pt}cfuncs{\isacharcomma}{\kern0pt}\ smt\ {\isacharparenleft}{\kern0pt}verit{\isacharcomma}{\kern0pt}\ del{\isacharunderscore}{\kern0pt}insts{\isacharparenright}{\kern0pt}\ \ cfunc{\isacharunderscore}{\kern0pt}type{\isacharunderscore}{\kern0pt}def\ equalizer\ equalizer{\isacharunderscore}{\kern0pt}def{\isacharparenright}{\kern0pt}\isanewline
\ \ \isacommand{then}\isamarkupfalse%
\ \isacommand{show}\isamarkupfalse%
\ {\isachardoublequoteopen}{\isasymexists}h{\isachardot}{\kern0pt}\ h\ {\isacharcolon}{\kern0pt}\ domain\ {\isasymlangle}x{\isacharcomma}{\kern0pt}y{\isasymrangle}\ {\isasymrightarrow}\ domain\ {\isacharparenleft}{\kern0pt}fibered{\isacharunderscore}{\kern0pt}product{\isacharunderscore}{\kern0pt}morphism\ X\ f\ g\ Y{\isacharparenright}{\kern0pt}\ {\isasymand}\isanewline
\ \ \ \ \ \ \ \ fibered{\isacharunderscore}{\kern0pt}product{\isacharunderscore}{\kern0pt}morphism\ X\ f\ g\ Y\ {\isasymcirc}\isactrlsub c\ h\ {\isacharequal}{\kern0pt}\ {\isasymlangle}x{\isacharcomma}{\kern0pt}y{\isasymrangle}{\isachardoublequoteclose}\isanewline
\ \ \ \ \isacommand{by}\isamarkupfalse%
\ {\isacharparenleft}{\kern0pt}metis\ assms{\isacharparenleft}{\kern0pt}{\isadigit{1}}{\isacharcomma}{\kern0pt}{\isadigit{2}}{\isacharparenright}{\kern0pt}\ cfunc{\isacharunderscore}{\kern0pt}type{\isacharunderscore}{\kern0pt}def\ domain{\isacharunderscore}{\kern0pt}comp\ fibered{\isacharunderscore}{\kern0pt}product{\isacharunderscore}{\kern0pt}morphism{\isacharunderscore}{\kern0pt}type{\isacharparenright}{\kern0pt}\isanewline
\isacommand{qed}\isamarkupfalse%
%
\endisatagproof
{\isafoldproof}%
%
\isadelimproof
\isanewline
%
\endisadelimproof
\isanewline
\isacommand{lemma}\isamarkupfalse%
\ fibered{\isacharunderscore}{\kern0pt}product{\isacharunderscore}{\kern0pt}is{\isacharunderscore}{\kern0pt}pullback{\isacharcolon}{\kern0pt}\isanewline
\ \ \isakeyword{assumes}\ {\isachardoublequoteopen}f\ {\isacharcolon}{\kern0pt}\ X\ {\isasymrightarrow}\ Z{\isachardoublequoteclose}\ {\isachardoublequoteopen}g\ {\isacharcolon}{\kern0pt}\ Y\ {\isasymrightarrow}\ Z{\isachardoublequoteclose}\isanewline
\ \ \isakeyword{shows}\ {\isachardoublequoteopen}is{\isacharunderscore}{\kern0pt}pullback\ {\isacharparenleft}{\kern0pt}X\ \isactrlbsub f\isactrlesub {\isasymtimes}\isactrlsub c\isactrlbsub g\isactrlesub \ Y{\isacharparenright}{\kern0pt}\ Y\ X\ Z\ \ {\isacharparenleft}{\kern0pt}fibered{\isacharunderscore}{\kern0pt}product{\isacharunderscore}{\kern0pt}right{\isacharunderscore}{\kern0pt}proj\ X\ f\ g\ Y{\isacharparenright}{\kern0pt}\ g\ {\isacharparenleft}{\kern0pt}fibered{\isacharunderscore}{\kern0pt}product{\isacharunderscore}{\kern0pt}left{\isacharunderscore}{\kern0pt}proj\ X\ f\ g\ Y{\isacharparenright}{\kern0pt}\ f{\isachardoublequoteclose}\isanewline
%
\isadelimproof
\ \ %
\endisadelimproof
%
\isatagproof
\isacommand{unfolding}\isamarkupfalse%
\ is{\isacharunderscore}{\kern0pt}pullback{\isacharunderscore}{\kern0pt}def\isanewline
\ \ \isacommand{using}\isamarkupfalse%
\ assms\ fibered{\isacharunderscore}{\kern0pt}product{\isacharunderscore}{\kern0pt}left{\isacharunderscore}{\kern0pt}proj{\isacharunderscore}{\kern0pt}type\ fibered{\isacharunderscore}{\kern0pt}product{\isacharunderscore}{\kern0pt}right{\isacharunderscore}{\kern0pt}proj{\isacharunderscore}{\kern0pt}type\isanewline
\isacommand{proof}\isamarkupfalse%
\ safe\isanewline
\ \ \isacommand{show}\isamarkupfalse%
\ {\isachardoublequoteopen}g\ {\isasymcirc}\isactrlsub c\ fibered{\isacharunderscore}{\kern0pt}product{\isacharunderscore}{\kern0pt}right{\isacharunderscore}{\kern0pt}proj\ X\ f\ g\ Y\ {\isacharequal}{\kern0pt}\ f\ {\isasymcirc}\isactrlsub c\ fibered{\isacharunderscore}{\kern0pt}product{\isacharunderscore}{\kern0pt}left{\isacharunderscore}{\kern0pt}proj\ X\ f\ g\ Y{\isachardoublequoteclose}\isanewline
\ \ \ \ \isacommand{unfolding}\isamarkupfalse%
\ fibered{\isacharunderscore}{\kern0pt}product{\isacharunderscore}{\kern0pt}right{\isacharunderscore}{\kern0pt}proj{\isacharunderscore}{\kern0pt}def\ fibered{\isacharunderscore}{\kern0pt}product{\isacharunderscore}{\kern0pt}left{\isacharunderscore}{\kern0pt}proj{\isacharunderscore}{\kern0pt}def\isanewline
\ \ \ \ \isacommand{using}\isamarkupfalse%
\ assms\ cfunc{\isacharunderscore}{\kern0pt}type{\isacharunderscore}{\kern0pt}def\ comp{\isacharunderscore}{\kern0pt}associative{\isadigit{2}}\ equalizer{\isacharunderscore}{\kern0pt}def\ fibered{\isacharunderscore}{\kern0pt}product{\isacharunderscore}{\kern0pt}morphism{\isacharunderscore}{\kern0pt}equalizer\isanewline
\ \ \ \ \isacommand{by}\isamarkupfalse%
\ {\isacharparenleft}{\kern0pt}typecheck{\isacharunderscore}{\kern0pt}cfuncs{\isacharcomma}{\kern0pt}\ auto{\isacharparenright}{\kern0pt}\isanewline
\isacommand{next}\isamarkupfalse%
\isanewline
\ \ \isacommand{fix}\isamarkupfalse%
\ A\ k\ h\isanewline
\ \ \isacommand{assume}\isamarkupfalse%
\ k{\isacharunderscore}{\kern0pt}type{\isacharcolon}{\kern0pt}\ {\isachardoublequoteopen}k\ {\isacharcolon}{\kern0pt}\ A\ {\isasymrightarrow}\ Y{\isachardoublequoteclose}\ \isakeyword{and}\ h{\isacharunderscore}{\kern0pt}type{\isacharcolon}{\kern0pt}\ {\isachardoublequoteopen}h\ {\isacharcolon}{\kern0pt}\ A\ {\isasymrightarrow}\ X{\isachardoublequoteclose}\isanewline
\ \ \isacommand{assume}\isamarkupfalse%
\ k{\isacharunderscore}{\kern0pt}h{\isacharunderscore}{\kern0pt}commutes{\isacharcolon}{\kern0pt}\ {\isachardoublequoteopen}g\ {\isasymcirc}\isactrlsub c\ k\ {\isacharequal}{\kern0pt}\ f\ {\isasymcirc}\isactrlsub c\ h{\isachardoublequoteclose}\isanewline
\isanewline
\ \ \isacommand{have}\isamarkupfalse%
\ {\isachardoublequoteopen}{\isasymlangle}h{\isacharcomma}{\kern0pt}k{\isasymrangle}\ factorsthru\ fibered{\isacharunderscore}{\kern0pt}product{\isacharunderscore}{\kern0pt}morphism\ X\ f\ g\ Y{\isachardoublequoteclose}\isanewline
\ \ \ \ \isacommand{using}\isamarkupfalse%
\ assms\ h{\isacharunderscore}{\kern0pt}type\ k{\isacharunderscore}{\kern0pt}h{\isacharunderscore}{\kern0pt}commutes\ k{\isacharunderscore}{\kern0pt}type\ pair{\isacharunderscore}{\kern0pt}factorsthru{\isacharunderscore}{\kern0pt}fibered{\isacharunderscore}{\kern0pt}product{\isacharunderscore}{\kern0pt}morphism\ \isacommand{by}\isamarkupfalse%
\ auto\isanewline
\ \ \isacommand{then}\isamarkupfalse%
\ \isacommand{have}\isamarkupfalse%
\ {\isachardoublequoteopen}{\isasymexists}j{\isachardot}{\kern0pt}\ j\ {\isacharcolon}{\kern0pt}\ A\ {\isasymrightarrow}\ X\ \isactrlbsub f\isactrlesub {\isasymtimes}\isactrlsub c\isactrlbsub g\isactrlesub \ Y\ {\isasymand}\ fibered{\isacharunderscore}{\kern0pt}product{\isacharunderscore}{\kern0pt}morphism\ X\ f\ g\ Y\ {\isasymcirc}\isactrlsub c\ j\ {\isacharequal}{\kern0pt}\ {\isasymlangle}h{\isacharcomma}{\kern0pt}k{\isasymrangle}{\isachardoublequoteclose}\isanewline
\ \ \ \ \isacommand{by}\isamarkupfalse%
\ {\isacharparenleft}{\kern0pt}meson\ assms\ cfunc{\isacharunderscore}{\kern0pt}prod{\isacharunderscore}{\kern0pt}type\ factors{\isacharunderscore}{\kern0pt}through{\isacharunderscore}{\kern0pt}def{\isadigit{2}}\ fibered{\isacharunderscore}{\kern0pt}product{\isacharunderscore}{\kern0pt}morphism{\isacharunderscore}{\kern0pt}type\ h{\isacharunderscore}{\kern0pt}type\ k{\isacharunderscore}{\kern0pt}type{\isacharparenright}{\kern0pt}\isanewline
\ \ \isacommand{then}\isamarkupfalse%
\ \isacommand{show}\isamarkupfalse%
\ {\isachardoublequoteopen}{\isasymexists}j{\isachardot}{\kern0pt}\ j\ {\isacharcolon}{\kern0pt}\ A\ {\isasymrightarrow}\ X\ \isactrlbsub f\isactrlesub {\isasymtimes}\isactrlsub c\isactrlbsub g\isactrlesub \ Y\ {\isasymand}\isanewline
\ \ \ \ \ \ \ \ \ \ \ fibered{\isacharunderscore}{\kern0pt}product{\isacharunderscore}{\kern0pt}right{\isacharunderscore}{\kern0pt}proj\ X\ f\ g\ Y\ {\isasymcirc}\isactrlsub c\ j\ {\isacharequal}{\kern0pt}\ k\ {\isasymand}\ fibered{\isacharunderscore}{\kern0pt}product{\isacharunderscore}{\kern0pt}left{\isacharunderscore}{\kern0pt}proj\ X\ f\ g\ Y\ {\isasymcirc}\isactrlsub c\ j\ {\isacharequal}{\kern0pt}\ h{\isachardoublequoteclose}\isanewline
\ \ \ \ \isacommand{unfolding}\isamarkupfalse%
\ fibered{\isacharunderscore}{\kern0pt}product{\isacharunderscore}{\kern0pt}right{\isacharunderscore}{\kern0pt}proj{\isacharunderscore}{\kern0pt}def\ fibered{\isacharunderscore}{\kern0pt}product{\isacharunderscore}{\kern0pt}left{\isacharunderscore}{\kern0pt}proj{\isacharunderscore}{\kern0pt}def\ \isanewline
\ \ \isacommand{proof}\isamarkupfalse%
\ {\isacharparenleft}{\kern0pt}clarify{\isacharcomma}{\kern0pt}\ rule{\isacharunderscore}{\kern0pt}tac\ x{\isacharequal}{\kern0pt}j\ \isakeyword{in}\ exI{\isacharcomma}{\kern0pt}\ safe{\isacharparenright}{\kern0pt}\isanewline
\ \ \ \ \isacommand{fix}\isamarkupfalse%
\ j\isanewline
\ \ \ \ \isacommand{assume}\isamarkupfalse%
\ j{\isacharunderscore}{\kern0pt}type{\isacharcolon}{\kern0pt}\ {\isachardoublequoteopen}j\ {\isacharcolon}{\kern0pt}\ A\ {\isasymrightarrow}\ X\ \isactrlbsub f\isactrlesub {\isasymtimes}\isactrlsub c\isactrlbsub g\isactrlesub \ Y{\isachardoublequoteclose}\isanewline
\isanewline
\ \ \ \ \isacommand{show}\isamarkupfalse%
\ {\isachardoublequoteopen}fibered{\isacharunderscore}{\kern0pt}product{\isacharunderscore}{\kern0pt}morphism\ X\ f\ g\ Y\ {\isasymcirc}\isactrlsub c\ j\ {\isacharequal}{\kern0pt}\ {\isasymlangle}h{\isacharcomma}{\kern0pt}k{\isasymrangle}\ {\isasymLongrightarrow}\isanewline
\ \ \ \ \ \ \ \ {\isacharparenleft}{\kern0pt}right{\isacharunderscore}{\kern0pt}cart{\isacharunderscore}{\kern0pt}proj\ X\ Y\ {\isasymcirc}\isactrlsub c\ fibered{\isacharunderscore}{\kern0pt}product{\isacharunderscore}{\kern0pt}morphism\ X\ f\ g\ Y{\isacharparenright}{\kern0pt}\ {\isasymcirc}\isactrlsub c\ j\ {\isacharequal}{\kern0pt}\ k{\isachardoublequoteclose}\isanewline
\ \ \ \ \ \ \isacommand{using}\isamarkupfalse%
\ assms\ h{\isacharunderscore}{\kern0pt}type\ k{\isacharunderscore}{\kern0pt}type\ j{\isacharunderscore}{\kern0pt}type\isanewline
\ \ \ \ \ \ \isacommand{by}\isamarkupfalse%
\ {\isacharparenleft}{\kern0pt}typecheck{\isacharunderscore}{\kern0pt}cfuncs{\isacharcomma}{\kern0pt}\ metis\ cfunc{\isacharunderscore}{\kern0pt}type{\isacharunderscore}{\kern0pt}def\ comp{\isacharunderscore}{\kern0pt}associative\ right{\isacharunderscore}{\kern0pt}cart{\isacharunderscore}{\kern0pt}proj{\isacharunderscore}{\kern0pt}cfunc{\isacharunderscore}{\kern0pt}prod{\isacharparenright}{\kern0pt}\isanewline
\isanewline
\ \ \ \ \isacommand{show}\isamarkupfalse%
\ {\isachardoublequoteopen}fibered{\isacharunderscore}{\kern0pt}product{\isacharunderscore}{\kern0pt}morphism\ X\ f\ g\ Y\ {\isasymcirc}\isactrlsub c\ j\ {\isacharequal}{\kern0pt}\ {\isasymlangle}h{\isacharcomma}{\kern0pt}k{\isasymrangle}\ {\isasymLongrightarrow}\isanewline
\ \ \ \ \ \ \ \ {\isacharparenleft}{\kern0pt}left{\isacharunderscore}{\kern0pt}cart{\isacharunderscore}{\kern0pt}proj\ X\ Y\ {\isasymcirc}\isactrlsub c\ fibered{\isacharunderscore}{\kern0pt}product{\isacharunderscore}{\kern0pt}morphism\ X\ f\ g\ Y{\isacharparenright}{\kern0pt}\ {\isasymcirc}\isactrlsub c\ j\ {\isacharequal}{\kern0pt}\ h{\isachardoublequoteclose}\isanewline
\ \ \ \ \ \ \isacommand{using}\isamarkupfalse%
\ assms\ h{\isacharunderscore}{\kern0pt}type\ k{\isacharunderscore}{\kern0pt}type\ j{\isacharunderscore}{\kern0pt}type\isanewline
\ \ \ \ \ \ \isacommand{by}\isamarkupfalse%
\ {\isacharparenleft}{\kern0pt}typecheck{\isacharunderscore}{\kern0pt}cfuncs{\isacharcomma}{\kern0pt}\ metis\ cfunc{\isacharunderscore}{\kern0pt}type{\isacharunderscore}{\kern0pt}def\ comp{\isacharunderscore}{\kern0pt}associative\ left{\isacharunderscore}{\kern0pt}cart{\isacharunderscore}{\kern0pt}proj{\isacharunderscore}{\kern0pt}cfunc{\isacharunderscore}{\kern0pt}prod{\isacharparenright}{\kern0pt}\isanewline
\ \ \isacommand{qed}\isamarkupfalse%
\isanewline
\isacommand{next}\isamarkupfalse%
\isanewline
\ \ \isacommand{fix}\isamarkupfalse%
\ A\ j\ y\isanewline
\ \ \isacommand{assume}\isamarkupfalse%
\ j{\isacharunderscore}{\kern0pt}type{\isacharcolon}{\kern0pt}\ {\isachardoublequoteopen}j\ {\isacharcolon}{\kern0pt}\ A\ {\isasymrightarrow}\ X\ \isactrlbsub f\isactrlesub {\isasymtimes}\isactrlsub c\isactrlbsub g\isactrlesub \ Y{\isachardoublequoteclose}\ \isakeyword{and}\ y{\isacharunderscore}{\kern0pt}type{\isacharcolon}{\kern0pt}\ {\isachardoublequoteopen}y\ {\isacharcolon}{\kern0pt}\ A\ {\isasymrightarrow}\ X\ \isactrlbsub f\isactrlesub {\isasymtimes}\isactrlsub c\isactrlbsub g\isactrlesub \ Y{\isachardoublequoteclose}\isanewline
\ \ \isacommand{assume}\isamarkupfalse%
\ {\isachardoublequoteopen}fibered{\isacharunderscore}{\kern0pt}product{\isacharunderscore}{\kern0pt}right{\isacharunderscore}{\kern0pt}proj\ X\ f\ g\ Y\ {\isasymcirc}\isactrlsub c\ y\ {\isacharequal}{\kern0pt}\ fibered{\isacharunderscore}{\kern0pt}product{\isacharunderscore}{\kern0pt}right{\isacharunderscore}{\kern0pt}proj\ X\ f\ g\ Y\ {\isasymcirc}\isactrlsub c\ j{\isachardoublequoteclose}\isanewline
\ \ \isacommand{then}\isamarkupfalse%
\ \isacommand{have}\isamarkupfalse%
\ right{\isacharunderscore}{\kern0pt}eq{\isacharcolon}{\kern0pt}\ {\isachardoublequoteopen}right{\isacharunderscore}{\kern0pt}cart{\isacharunderscore}{\kern0pt}proj\ X\ Y\ {\isasymcirc}\isactrlsub c\ {\isacharparenleft}{\kern0pt}fibered{\isacharunderscore}{\kern0pt}product{\isacharunderscore}{\kern0pt}morphism\ X\ f\ g\ Y\ {\isasymcirc}\isactrlsub c\ y{\isacharparenright}{\kern0pt}\ {\isacharequal}{\kern0pt}\isanewline
\ \ \ \ \ \ right{\isacharunderscore}{\kern0pt}cart{\isacharunderscore}{\kern0pt}proj\ X\ Y\ {\isasymcirc}\isactrlsub c\ {\isacharparenleft}{\kern0pt}fibered{\isacharunderscore}{\kern0pt}product{\isacharunderscore}{\kern0pt}morphism\ X\ f\ g\ Y\ {\isasymcirc}\isactrlsub c\ j{\isacharparenright}{\kern0pt}{\isachardoublequoteclose}\isanewline
\ \ \ \ \isacommand{unfolding}\isamarkupfalse%
\ fibered{\isacharunderscore}{\kern0pt}product{\isacharunderscore}{\kern0pt}right{\isacharunderscore}{\kern0pt}proj{\isacharunderscore}{\kern0pt}def\ \isacommand{using}\isamarkupfalse%
\ assms\ j{\isacharunderscore}{\kern0pt}type\ y{\isacharunderscore}{\kern0pt}type\isanewline
\ \ \ \ \isacommand{by}\isamarkupfalse%
\ {\isacharparenleft}{\kern0pt}typecheck{\isacharunderscore}{\kern0pt}cfuncs{\isacharcomma}{\kern0pt}\ simp\ add{\isacharcolon}{\kern0pt}\ comp{\isacharunderscore}{\kern0pt}associative{\isadigit{2}}{\isacharparenright}{\kern0pt}\isanewline
\ \ \isacommand{assume}\isamarkupfalse%
\ {\isachardoublequoteopen}fibered{\isacharunderscore}{\kern0pt}product{\isacharunderscore}{\kern0pt}left{\isacharunderscore}{\kern0pt}proj\ X\ f\ g\ Y\ {\isasymcirc}\isactrlsub c\ y\ {\isacharequal}{\kern0pt}\ fibered{\isacharunderscore}{\kern0pt}product{\isacharunderscore}{\kern0pt}left{\isacharunderscore}{\kern0pt}proj\ X\ f\ g\ Y\ {\isasymcirc}\isactrlsub c\ j{\isachardoublequoteclose}\isanewline
\ \ \isacommand{then}\isamarkupfalse%
\ \isacommand{have}\isamarkupfalse%
\ left{\isacharunderscore}{\kern0pt}eq{\isacharcolon}{\kern0pt}\ {\isachardoublequoteopen}left{\isacharunderscore}{\kern0pt}cart{\isacharunderscore}{\kern0pt}proj\ X\ Y\ {\isasymcirc}\isactrlsub c\ {\isacharparenleft}{\kern0pt}fibered{\isacharunderscore}{\kern0pt}product{\isacharunderscore}{\kern0pt}morphism\ X\ f\ g\ Y\ {\isasymcirc}\isactrlsub c\ y{\isacharparenright}{\kern0pt}\ {\isacharequal}{\kern0pt}\isanewline
\ \ \ \ \ \ left{\isacharunderscore}{\kern0pt}cart{\isacharunderscore}{\kern0pt}proj\ X\ Y\ {\isasymcirc}\isactrlsub c\ {\isacharparenleft}{\kern0pt}fibered{\isacharunderscore}{\kern0pt}product{\isacharunderscore}{\kern0pt}morphism\ X\ f\ g\ Y\ {\isasymcirc}\isactrlsub c\ j{\isacharparenright}{\kern0pt}{\isachardoublequoteclose}\isanewline
\ \ \ \ \isacommand{unfolding}\isamarkupfalse%
\ fibered{\isacharunderscore}{\kern0pt}product{\isacharunderscore}{\kern0pt}left{\isacharunderscore}{\kern0pt}proj{\isacharunderscore}{\kern0pt}def\ \ \isacommand{using}\isamarkupfalse%
\ assms\ j{\isacharunderscore}{\kern0pt}type\ y{\isacharunderscore}{\kern0pt}type\isanewline
\ \ \ \ \isacommand{by}\isamarkupfalse%
\ {\isacharparenleft}{\kern0pt}typecheck{\isacharunderscore}{\kern0pt}cfuncs{\isacharcomma}{\kern0pt}\ simp\ add{\isacharcolon}{\kern0pt}\ comp{\isacharunderscore}{\kern0pt}associative{\isadigit{2}}{\isacharparenright}{\kern0pt}\isanewline
\isanewline
\ \ \isacommand{have}\isamarkupfalse%
\ mono{\isacharcolon}{\kern0pt}\ {\isachardoublequoteopen}monomorphism\ {\isacharparenleft}{\kern0pt}fibered{\isacharunderscore}{\kern0pt}product{\isacharunderscore}{\kern0pt}morphism\ X\ f\ g\ Y{\isacharparenright}{\kern0pt}{\isachardoublequoteclose}\isanewline
\ \ \ \ \isacommand{using}\isamarkupfalse%
\ assms\ fibered{\isacharunderscore}{\kern0pt}product{\isacharunderscore}{\kern0pt}morphism{\isacharunderscore}{\kern0pt}monomorphism\ \isacommand{by}\isamarkupfalse%
\ auto\isanewline
\isanewline
\ \ \isacommand{have}\isamarkupfalse%
\ {\isachardoublequoteopen}fibered{\isacharunderscore}{\kern0pt}product{\isacharunderscore}{\kern0pt}morphism\ X\ f\ g\ Y\ {\isasymcirc}\isactrlsub c\ y\ {\isacharequal}{\kern0pt}\ fibered{\isacharunderscore}{\kern0pt}product{\isacharunderscore}{\kern0pt}morphism\ X\ f\ g\ Y\ {\isasymcirc}\isactrlsub c\ j{\isachardoublequoteclose}\isanewline
\ \ \ \ \isacommand{using}\isamarkupfalse%
\ right{\isacharunderscore}{\kern0pt}eq\ left{\isacharunderscore}{\kern0pt}eq\ cart{\isacharunderscore}{\kern0pt}prod{\isacharunderscore}{\kern0pt}eq\ fibered{\isacharunderscore}{\kern0pt}product{\isacharunderscore}{\kern0pt}morphism{\isacharunderscore}{\kern0pt}type\ y{\isacharunderscore}{\kern0pt}type\ j{\isacharunderscore}{\kern0pt}type\ assms\ comp{\isacharunderscore}{\kern0pt}type\isanewline
\ \ \ \ \isacommand{by}\isamarkupfalse%
\ {\isacharparenleft}{\kern0pt}subst\ cart{\isacharunderscore}{\kern0pt}prod{\isacharunderscore}{\kern0pt}eq{\isacharbrackleft}{\kern0pt}\isakeyword{where}\ Z{\isacharequal}{\kern0pt}A{\isacharcomma}{\kern0pt}\ \isakeyword{where}\ X{\isacharequal}{\kern0pt}X{\isacharcomma}{\kern0pt}\ \isakeyword{where}\ Y{\isacharequal}{\kern0pt}Y{\isacharbrackright}{\kern0pt}{\isacharcomma}{\kern0pt}\ auto{\isacharparenright}{\kern0pt}\isanewline
\ \ \isacommand{then}\isamarkupfalse%
\ \isacommand{show}\isamarkupfalse%
\ {\isachardoublequoteopen}j\ {\isacharequal}{\kern0pt}\ y{\isachardoublequoteclose}\isanewline
\ \ \ \ \isacommand{using}\isamarkupfalse%
\ mono\ assms\ cfunc{\isacharunderscore}{\kern0pt}type{\isacharunderscore}{\kern0pt}def\ fibered{\isacharunderscore}{\kern0pt}product{\isacharunderscore}{\kern0pt}morphism{\isacharunderscore}{\kern0pt}type\ j{\isacharunderscore}{\kern0pt}type\ y{\isacharunderscore}{\kern0pt}type\isanewline
\ \ \ \ \isacommand{unfolding}\isamarkupfalse%
\ monomorphism{\isacharunderscore}{\kern0pt}def\isanewline
\ \ \ \ \isacommand{by}\isamarkupfalse%
\ auto\ \isanewline
\isacommand{qed}\isamarkupfalse%
%
\endisatagproof
{\isafoldproof}%
%
\isadelimproof
\isanewline
%
\endisadelimproof
\isanewline
\isacommand{lemma}\isamarkupfalse%
\ fibered{\isacharunderscore}{\kern0pt}product{\isacharunderscore}{\kern0pt}proj{\isacharunderscore}{\kern0pt}eq{\isacharcolon}{\kern0pt}\isanewline
\ \ \isakeyword{assumes}\ {\isachardoublequoteopen}f\ {\isacharcolon}{\kern0pt}\ X\ {\isasymrightarrow}\ Z{\isachardoublequoteclose}\ {\isachardoublequoteopen}g\ {\isacharcolon}{\kern0pt}\ Y\ {\isasymrightarrow}\ Z{\isachardoublequoteclose}\isanewline
\ \ \isakeyword{shows}\ {\isachardoublequoteopen}f\ {\isasymcirc}\isactrlsub c\ fibered{\isacharunderscore}{\kern0pt}product{\isacharunderscore}{\kern0pt}left{\isacharunderscore}{\kern0pt}proj\ X\ f\ g\ Y\ {\isacharequal}{\kern0pt}\ g\ {\isasymcirc}\isactrlsub c\ fibered{\isacharunderscore}{\kern0pt}product{\isacharunderscore}{\kern0pt}right{\isacharunderscore}{\kern0pt}proj\ X\ f\ g\ Y{\isachardoublequoteclose}\isanewline
%
\isadelimproof
\ \ \ \ %
\endisadelimproof
%
\isatagproof
\isacommand{using}\isamarkupfalse%
\ fibered{\isacharunderscore}{\kern0pt}product{\isacharunderscore}{\kern0pt}is{\isacharunderscore}{\kern0pt}pullback\ assms\isanewline
\ \ \ \ \isacommand{unfolding}\isamarkupfalse%
\ is{\isacharunderscore}{\kern0pt}pullback{\isacharunderscore}{\kern0pt}def\ \isacommand{by}\isamarkupfalse%
\ auto%
\endisatagproof
{\isafoldproof}%
%
\isadelimproof
\isanewline
%
\endisadelimproof
\isanewline
\isacommand{lemma}\isamarkupfalse%
\ fibered{\isacharunderscore}{\kern0pt}product{\isacharunderscore}{\kern0pt}pair{\isacharunderscore}{\kern0pt}member{\isacharcolon}{\kern0pt}\isanewline
\ \ \isakeyword{assumes}\ {\isachardoublequoteopen}f\ {\isacharcolon}{\kern0pt}\ X\ {\isasymrightarrow}\ Z{\isachardoublequoteclose}\ {\isachardoublequoteopen}g\ {\isacharcolon}{\kern0pt}\ Y\ {\isasymrightarrow}\ Z{\isachardoublequoteclose}\ {\isachardoublequoteopen}x\ {\isasymin}\isactrlsub c\ X{\isachardoublequoteclose}\ {\isachardoublequoteopen}y\ {\isasymin}\isactrlsub c\ Y{\isachardoublequoteclose}\isanewline
\ \ \isakeyword{shows}\ {\isachardoublequoteopen}{\isacharparenleft}{\kern0pt}{\isasymlangle}x{\isacharcomma}{\kern0pt}\ y{\isasymrangle}\ {\isasymin}\isactrlbsub X\ {\isasymtimes}\isactrlsub c\ Y\isactrlesub \ {\isacharparenleft}{\kern0pt}X\isactrlbsub f\isactrlesub {\isasymtimes}\isactrlsub c\isactrlbsub g\isactrlesub Y{\isacharcomma}{\kern0pt}\ \ fibered{\isacharunderscore}{\kern0pt}product{\isacharunderscore}{\kern0pt}morphism\ X\ f\ g\ Y{\isacharparenright}{\kern0pt}{\isacharparenright}{\kern0pt}\ {\isacharequal}{\kern0pt}\ {\isacharparenleft}{\kern0pt}f\ {\isasymcirc}\isactrlsub c\ x\ {\isacharequal}{\kern0pt}\ g\ {\isasymcirc}\isactrlsub c\ y{\isacharparenright}{\kern0pt}{\isachardoublequoteclose}\isanewline
%
\isadelimproof
%
\endisadelimproof
%
\isatagproof
\isacommand{proof}\isamarkupfalse%
\isanewline
\ \ \isacommand{assume}\isamarkupfalse%
\ {\isachardoublequoteopen}{\isasymlangle}x{\isacharcomma}{\kern0pt}y{\isasymrangle}\ {\isasymin}\isactrlbsub X\ {\isasymtimes}\isactrlsub c\ Y\isactrlesub \ {\isacharparenleft}{\kern0pt}X\ \isactrlbsub f\isactrlesub {\isasymtimes}\isactrlsub c\isactrlbsub g\isactrlesub \ Y{\isacharcomma}{\kern0pt}\ fibered{\isacharunderscore}{\kern0pt}product{\isacharunderscore}{\kern0pt}morphism\ X\ f\ g\ Y{\isacharparenright}{\kern0pt}{\isachardoublequoteclose}\isanewline
\ \ \isacommand{then}\isamarkupfalse%
\ \isacommand{obtain}\isamarkupfalse%
\ h\ \isakeyword{where}\isanewline
\ \ \ \ h{\isacharunderscore}{\kern0pt}type{\isacharcolon}{\kern0pt}\ {\isachardoublequoteopen}h\ {\isasymin}\isactrlsub c\ X\isactrlbsub f\isactrlesub {\isasymtimes}\isactrlsub c\isactrlbsub g\isactrlesub Y{\isachardoublequoteclose}\ \isakeyword{and}\ h{\isacharunderscore}{\kern0pt}eq{\isacharcolon}{\kern0pt}\ {\isachardoublequoteopen}fibered{\isacharunderscore}{\kern0pt}product{\isacharunderscore}{\kern0pt}morphism\ X\ f\ g\ Y\ {\isasymcirc}\isactrlsub c\ h\ {\isacharequal}{\kern0pt}\ {\isasymlangle}x{\isacharcomma}{\kern0pt}y{\isasymrangle}{\isachardoublequoteclose}\isanewline
\ \ \ \ \isacommand{unfolding}\isamarkupfalse%
\ relative{\isacharunderscore}{\kern0pt}member{\isacharunderscore}{\kern0pt}def{\isadigit{2}}\ factors{\isacharunderscore}{\kern0pt}through{\isacharunderscore}{\kern0pt}def\isanewline
\ \ \ \ \isacommand{using}\isamarkupfalse%
\ assms{\isacharparenleft}{\kern0pt}{\isadigit{3}}{\isacharcomma}{\kern0pt}{\isadigit{4}}{\isacharparenright}{\kern0pt}\ cfunc{\isacharunderscore}{\kern0pt}prod{\isacharunderscore}{\kern0pt}type\ cfunc{\isacharunderscore}{\kern0pt}type{\isacharunderscore}{\kern0pt}def\ \isacommand{by}\isamarkupfalse%
\ auto\isanewline
\ \ \isanewline
\ \ \isacommand{have}\isamarkupfalse%
\ left{\isacharunderscore}{\kern0pt}eq{\isacharcolon}{\kern0pt}\ {\isachardoublequoteopen}fibered{\isacharunderscore}{\kern0pt}product{\isacharunderscore}{\kern0pt}left{\isacharunderscore}{\kern0pt}proj\ X\ f\ g\ Y\ {\isasymcirc}\isactrlsub c\ h\ {\isacharequal}{\kern0pt}\ x{\isachardoublequoteclose}\isanewline
\ \ \ \ \isacommand{unfolding}\isamarkupfalse%
\ fibered{\isacharunderscore}{\kern0pt}product{\isacharunderscore}{\kern0pt}left{\isacharunderscore}{\kern0pt}proj{\isacharunderscore}{\kern0pt}def\isanewline
\ \ \ \ \isacommand{using}\isamarkupfalse%
\ assms\ h{\isacharunderscore}{\kern0pt}type\isanewline
\ \ \ \ \isacommand{by}\isamarkupfalse%
\ {\isacharparenleft}{\kern0pt}typecheck{\isacharunderscore}{\kern0pt}cfuncs{\isacharcomma}{\kern0pt}\ smt\ comp{\isacharunderscore}{\kern0pt}associative{\isadigit{2}}\ h{\isacharunderscore}{\kern0pt}eq\ left{\isacharunderscore}{\kern0pt}cart{\isacharunderscore}{\kern0pt}proj{\isacharunderscore}{\kern0pt}cfunc{\isacharunderscore}{\kern0pt}prod{\isacharparenright}{\kern0pt}\isanewline
\isanewline
\ \ \isacommand{have}\isamarkupfalse%
\ right{\isacharunderscore}{\kern0pt}eq{\isacharcolon}{\kern0pt}\ {\isachardoublequoteopen}fibered{\isacharunderscore}{\kern0pt}product{\isacharunderscore}{\kern0pt}right{\isacharunderscore}{\kern0pt}proj\ X\ f\ g\ Y\ {\isasymcirc}\isactrlsub c\ h\ {\isacharequal}{\kern0pt}\ y{\isachardoublequoteclose}\isanewline
\ \ \ \ \isacommand{unfolding}\isamarkupfalse%
\ fibered{\isacharunderscore}{\kern0pt}product{\isacharunderscore}{\kern0pt}right{\isacharunderscore}{\kern0pt}proj{\isacharunderscore}{\kern0pt}def\isanewline
\ \ \ \ \isacommand{using}\isamarkupfalse%
\ assms\ h{\isacharunderscore}{\kern0pt}type\isanewline
\ \ \ \ \isacommand{by}\isamarkupfalse%
\ {\isacharparenleft}{\kern0pt}typecheck{\isacharunderscore}{\kern0pt}cfuncs{\isacharcomma}{\kern0pt}\ smt\ comp{\isacharunderscore}{\kern0pt}associative{\isadigit{2}}\ h{\isacharunderscore}{\kern0pt}eq\ right{\isacharunderscore}{\kern0pt}cart{\isacharunderscore}{\kern0pt}proj{\isacharunderscore}{\kern0pt}cfunc{\isacharunderscore}{\kern0pt}prod{\isacharparenright}{\kern0pt}\isanewline
\isanewline
\ \ \isacommand{have}\isamarkupfalse%
\ {\isachardoublequoteopen}f\ {\isasymcirc}\isactrlsub c\ fibered{\isacharunderscore}{\kern0pt}product{\isacharunderscore}{\kern0pt}left{\isacharunderscore}{\kern0pt}proj\ X\ f\ g\ Y\ {\isasymcirc}\isactrlsub c\ h\ {\isacharequal}{\kern0pt}\ g\ {\isasymcirc}\isactrlsub c\ fibered{\isacharunderscore}{\kern0pt}product{\isacharunderscore}{\kern0pt}right{\isacharunderscore}{\kern0pt}proj\ X\ f\ g\ Y\ {\isasymcirc}\isactrlsub c\ h{\isachardoublequoteclose}\isanewline
\ \ \ \ \isacommand{using}\isamarkupfalse%
\ assms\ h{\isacharunderscore}{\kern0pt}type\ \isacommand{by}\isamarkupfalse%
\ {\isacharparenleft}{\kern0pt}typecheck{\isacharunderscore}{\kern0pt}cfuncs{\isacharcomma}{\kern0pt}\ simp\ add{\isacharcolon}{\kern0pt}\ comp{\isacharunderscore}{\kern0pt}associative{\isadigit{2}}\ fibered{\isacharunderscore}{\kern0pt}product{\isacharunderscore}{\kern0pt}proj{\isacharunderscore}{\kern0pt}eq{\isacharparenright}{\kern0pt}\isanewline
\ \ \isacommand{then}\isamarkupfalse%
\ \isacommand{show}\isamarkupfalse%
\ {\isachardoublequoteopen}f\ {\isasymcirc}\isactrlsub c\ x\ {\isacharequal}{\kern0pt}\ g\ {\isasymcirc}\isactrlsub c\ y{\isachardoublequoteclose}\isanewline
\ \ \ \ \isacommand{using}\isamarkupfalse%
\ left{\isacharunderscore}{\kern0pt}eq\ right{\isacharunderscore}{\kern0pt}eq\ \isacommand{by}\isamarkupfalse%
\ auto\isanewline
\isacommand{next}\isamarkupfalse%
\isanewline
\ \ \isacommand{assume}\isamarkupfalse%
\ f{\isacharunderscore}{\kern0pt}g{\isacharunderscore}{\kern0pt}eq{\isacharcolon}{\kern0pt}\ {\isachardoublequoteopen}f\ {\isasymcirc}\isactrlsub c\ x\ {\isacharequal}{\kern0pt}\ g\ {\isasymcirc}\isactrlsub c\ y{\isachardoublequoteclose}\isanewline
\ \ \isacommand{show}\isamarkupfalse%
\ {\isachardoublequoteopen}{\isasymlangle}x{\isacharcomma}{\kern0pt}y{\isasymrangle}\ {\isasymin}\isactrlbsub X\ {\isasymtimes}\isactrlsub c\ Y\isactrlesub \ {\isacharparenleft}{\kern0pt}X\ \isactrlbsub f\isactrlesub {\isasymtimes}\isactrlsub c\isactrlbsub g\isactrlesub \ Y{\isacharcomma}{\kern0pt}\ fibered{\isacharunderscore}{\kern0pt}product{\isacharunderscore}{\kern0pt}morphism\ X\ f\ g\ Y{\isacharparenright}{\kern0pt}{\isachardoublequoteclose}\isanewline
\ \ \ \ \isacommand{unfolding}\isamarkupfalse%
\ relative{\isacharunderscore}{\kern0pt}member{\isacharunderscore}{\kern0pt}def\ factors{\isacharunderscore}{\kern0pt}through{\isacharunderscore}{\kern0pt}def\isanewline
\ \ \isacommand{proof}\isamarkupfalse%
\ {\isacharparenleft}{\kern0pt}safe{\isacharparenright}{\kern0pt}\isanewline
\ \ \ \ \isacommand{show}\isamarkupfalse%
\ {\isachardoublequoteopen}{\isasymlangle}x{\isacharcomma}{\kern0pt}y{\isasymrangle}\ {\isasymin}\isactrlsub c\ X\ {\isasymtimes}\isactrlsub c\ Y{\isachardoublequoteclose}\isanewline
\ \ \ \ \ \ \isacommand{using}\isamarkupfalse%
\ assms\ \isacommand{by}\isamarkupfalse%
\ typecheck{\isacharunderscore}{\kern0pt}cfuncs\isanewline
\ \ \ \ \isacommand{show}\isamarkupfalse%
\ {\isachardoublequoteopen}monomorphism\ {\isacharparenleft}{\kern0pt}snd\ {\isacharparenleft}{\kern0pt}X\ \isactrlbsub f\isactrlesub {\isasymtimes}\isactrlsub c\isactrlbsub g\isactrlesub \ Y{\isacharcomma}{\kern0pt}\ fibered{\isacharunderscore}{\kern0pt}product{\isacharunderscore}{\kern0pt}morphism\ X\ f\ g\ Y{\isacharparenright}{\kern0pt}{\isacharparenright}{\kern0pt}{\isachardoublequoteclose}\isanewline
\ \ \ \ \ \ \isacommand{using}\isamarkupfalse%
\ assms{\isacharparenleft}{\kern0pt}{\isadigit{1}}{\isacharcomma}{\kern0pt}{\isadigit{2}}{\isacharparenright}{\kern0pt}\ fibered{\isacharunderscore}{\kern0pt}product{\isacharunderscore}{\kern0pt}morphism{\isacharunderscore}{\kern0pt}monomorphism\ \isacommand{by}\isamarkupfalse%
\ auto\isanewline
\ \ \ \ \isacommand{show}\isamarkupfalse%
\ {\isachardoublequoteopen}snd\ {\isacharparenleft}{\kern0pt}X\ \isactrlbsub f\isactrlesub {\isasymtimes}\isactrlsub c\isactrlbsub g\isactrlesub \ Y{\isacharcomma}{\kern0pt}\ fibered{\isacharunderscore}{\kern0pt}product{\isacharunderscore}{\kern0pt}morphism\ X\ f\ g\ Y{\isacharparenright}{\kern0pt}\ {\isacharcolon}{\kern0pt}\ fst\ {\isacharparenleft}{\kern0pt}X\ \isactrlbsub f\isactrlesub {\isasymtimes}\isactrlsub c\isactrlbsub g\isactrlesub \ Y{\isacharcomma}{\kern0pt}\ fibered{\isacharunderscore}{\kern0pt}product{\isacharunderscore}{\kern0pt}morphism\ X\ f\ g\ Y{\isacharparenright}{\kern0pt}\ {\isasymrightarrow}\ X\ {\isasymtimes}\isactrlsub c\ Y{\isachardoublequoteclose}\isanewline
\ \ \ \ \ \ \isacommand{using}\isamarkupfalse%
\ assms{\isacharparenleft}{\kern0pt}{\isadigit{1}}{\isacharcomma}{\kern0pt}{\isadigit{2}}{\isacharparenright}{\kern0pt}\ fibered{\isacharunderscore}{\kern0pt}product{\isacharunderscore}{\kern0pt}morphism{\isacharunderscore}{\kern0pt}type\ \isacommand{by}\isamarkupfalse%
\ force\isanewline
\ \ \ \ \isacommand{have}\isamarkupfalse%
\ j{\isacharunderscore}{\kern0pt}exists{\isacharcolon}{\kern0pt}\ {\isachardoublequoteopen}{\isasymAnd}\ Z\ k\ h{\isachardot}{\kern0pt}\ k\ {\isacharcolon}{\kern0pt}\ Z\ {\isasymrightarrow}\ Y\ {\isasymLongrightarrow}\ h\ {\isacharcolon}{\kern0pt}\ Z\ {\isasymrightarrow}\ X\ {\isasymLongrightarrow}\ g\ {\isasymcirc}\isactrlsub c\ k\ {\isacharequal}{\kern0pt}\ f\ {\isasymcirc}\isactrlsub c\ h\ {\isasymLongrightarrow}\isanewline
\ \ \ \ \ \ {\isacharparenleft}{\kern0pt}{\isasymexists}{\isacharbang}{\kern0pt}j{\isachardot}{\kern0pt}\ j\ {\isacharcolon}{\kern0pt}\ Z\ {\isasymrightarrow}\ X\ \isactrlbsub f\isactrlesub {\isasymtimes}\isactrlsub c\isactrlbsub g\isactrlesub \ Y\ {\isasymand}\isanewline
\ \ \ \ \ \ \ \ \ \ \ \ fibered{\isacharunderscore}{\kern0pt}product{\isacharunderscore}{\kern0pt}right{\isacharunderscore}{\kern0pt}proj\ X\ f\ g\ Y\ {\isasymcirc}\isactrlsub c\ j\ {\isacharequal}{\kern0pt}\ k\ {\isasymand}\isanewline
\ \ \ \ \ \ \ \ \ \ \ \ fibered{\isacharunderscore}{\kern0pt}product{\isacharunderscore}{\kern0pt}left{\isacharunderscore}{\kern0pt}proj\ X\ f\ g\ Y\ {\isasymcirc}\isactrlsub c\ j\ {\isacharequal}{\kern0pt}\ h{\isacharparenright}{\kern0pt}{\isachardoublequoteclose}\isanewline
\ \ \ \ \ \ \isacommand{using}\isamarkupfalse%
\ fibered{\isacharunderscore}{\kern0pt}product{\isacharunderscore}{\kern0pt}is{\isacharunderscore}{\kern0pt}pullback\ assms\ \isacommand{unfolding}\isamarkupfalse%
\ is{\isacharunderscore}{\kern0pt}pullback{\isacharunderscore}{\kern0pt}def\ \isacommand{by}\isamarkupfalse%
\ auto\isanewline
\isanewline
\ \ \ \ \isacommand{obtain}\isamarkupfalse%
\ j\ \isakeyword{where}\ j{\isacharunderscore}{\kern0pt}type{\isacharcolon}{\kern0pt}\ {\isachardoublequoteopen}j\ {\isasymin}\isactrlsub c\ X\ \isactrlbsub f\isactrlesub {\isasymtimes}\isactrlsub c\isactrlbsub g\isactrlesub \ Y{\isachardoublequoteclose}\ \isakeyword{and}\ \isanewline
\ \ \ \ \ \ j{\isacharunderscore}{\kern0pt}projs{\isacharcolon}{\kern0pt}\ {\isachardoublequoteopen}fibered{\isacharunderscore}{\kern0pt}product{\isacharunderscore}{\kern0pt}right{\isacharunderscore}{\kern0pt}proj\ X\ f\ g\ Y\ {\isasymcirc}\isactrlsub c\ j\ {\isacharequal}{\kern0pt}\ y{\isachardoublequoteclose}\ {\isachardoublequoteopen}fibered{\isacharunderscore}{\kern0pt}product{\isacharunderscore}{\kern0pt}left{\isacharunderscore}{\kern0pt}proj\ X\ f\ g\ Y\ {\isasymcirc}\isactrlsub c\ j\ {\isacharequal}{\kern0pt}\ x{\isachardoublequoteclose}\isanewline
\ \ \ \ \ \ \isacommand{using}\isamarkupfalse%
\ j{\isacharunderscore}{\kern0pt}exists{\isacharbrackleft}{\kern0pt}\isakeyword{where}\ Z{\isacharequal}{\kern0pt}{\isasymone}{\isacharcomma}{\kern0pt}\ \isakeyword{where}\ k{\isacharequal}{\kern0pt}y{\isacharcomma}{\kern0pt}\ \isakeyword{where}\ h{\isacharequal}{\kern0pt}x{\isacharbrackright}{\kern0pt}\ assms\ f{\isacharunderscore}{\kern0pt}g{\isacharunderscore}{\kern0pt}eq\ \isacommand{by}\isamarkupfalse%
\ auto\isanewline
\ \ \ \ \isacommand{show}\isamarkupfalse%
\ {\isachardoublequoteopen}{\isasymexists}h{\isachardot}{\kern0pt}\ h\ {\isacharcolon}{\kern0pt}\ domain\ {\isasymlangle}x{\isacharcomma}{\kern0pt}y{\isasymrangle}\ {\isasymrightarrow}\ domain\ {\isacharparenleft}{\kern0pt}snd\ {\isacharparenleft}{\kern0pt}X\ \isactrlbsub f\isactrlesub {\isasymtimes}\isactrlsub c\isactrlbsub g\isactrlesub \ Y{\isacharcomma}{\kern0pt}\ fibered{\isacharunderscore}{\kern0pt}product{\isacharunderscore}{\kern0pt}morphism\ X\ f\ g\ Y{\isacharparenright}{\kern0pt}{\isacharparenright}{\kern0pt}\ {\isasymand}\isanewline
\ \ \ \ \ \ \ \ \ \ \ snd\ {\isacharparenleft}{\kern0pt}X\ \isactrlbsub f\isactrlesub {\isasymtimes}\isactrlsub c\isactrlbsub g\isactrlesub \ Y{\isacharcomma}{\kern0pt}\ fibered{\isacharunderscore}{\kern0pt}product{\isacharunderscore}{\kern0pt}morphism\ X\ f\ g\ Y{\isacharparenright}{\kern0pt}\ {\isasymcirc}\isactrlsub c\ h\ {\isacharequal}{\kern0pt}\ {\isasymlangle}x{\isacharcomma}{\kern0pt}y{\isasymrangle}{\isachardoublequoteclose}\isanewline
\ \ \ \ \isacommand{proof}\isamarkupfalse%
\ {\isacharparenleft}{\kern0pt}rule{\isacharunderscore}{\kern0pt}tac\ x{\isacharequal}{\kern0pt}j\ \isakeyword{in}\ exI{\isacharcomma}{\kern0pt}\ safe{\isacharparenright}{\kern0pt}\isanewline
\ \ \ \ \ \ \isacommand{show}\isamarkupfalse%
\ {\isachardoublequoteopen}j\ {\isacharcolon}{\kern0pt}\ domain\ {\isasymlangle}x{\isacharcomma}{\kern0pt}y{\isasymrangle}\ {\isasymrightarrow}\ domain\ {\isacharparenleft}{\kern0pt}snd\ {\isacharparenleft}{\kern0pt}X\ \isactrlbsub f\isactrlesub {\isasymtimes}\isactrlsub c\isactrlbsub g\isactrlesub \ Y{\isacharcomma}{\kern0pt}\ fibered{\isacharunderscore}{\kern0pt}product{\isacharunderscore}{\kern0pt}morphism\ X\ f\ g\ Y{\isacharparenright}{\kern0pt}{\isacharparenright}{\kern0pt}{\isachardoublequoteclose}\isanewline
\ \ \ \ \ \ \ \ \isacommand{using}\isamarkupfalse%
\ assms\ j{\isacharunderscore}{\kern0pt}type\ cfunc{\isacharunderscore}{\kern0pt}type{\isacharunderscore}{\kern0pt}def\ \isacommand{by}\isamarkupfalse%
\ {\isacharparenleft}{\kern0pt}typecheck{\isacharunderscore}{\kern0pt}cfuncs{\isacharcomma}{\kern0pt}\ auto{\isacharparenright}{\kern0pt}\isanewline
\isanewline
\ \ \ \ \ \ \isacommand{have}\isamarkupfalse%
\ left{\isacharunderscore}{\kern0pt}eq{\isacharcolon}{\kern0pt}\ {\isachardoublequoteopen}left{\isacharunderscore}{\kern0pt}cart{\isacharunderscore}{\kern0pt}proj\ X\ Y\ {\isasymcirc}\isactrlsub c\ fibered{\isacharunderscore}{\kern0pt}product{\isacharunderscore}{\kern0pt}morphism\ X\ f\ g\ Y\ {\isasymcirc}\isactrlsub c\ j\ {\isacharequal}{\kern0pt}\ x{\isachardoublequoteclose}\isanewline
\ \ \ \ \ \ \ \ \isacommand{using}\isamarkupfalse%
\ j{\isacharunderscore}{\kern0pt}projs\ assms\ j{\isacharunderscore}{\kern0pt}type\ comp{\isacharunderscore}{\kern0pt}associative{\isadigit{2}}\isanewline
\ \ \ \ \ \ \ \ \isacommand{unfolding}\isamarkupfalse%
\ fibered{\isacharunderscore}{\kern0pt}product{\isacharunderscore}{\kern0pt}left{\isacharunderscore}{\kern0pt}proj{\isacharunderscore}{\kern0pt}def\ \isacommand{by}\isamarkupfalse%
\ {\isacharparenleft}{\kern0pt}typecheck{\isacharunderscore}{\kern0pt}cfuncs{\isacharcomma}{\kern0pt}\ auto{\isacharparenright}{\kern0pt}\isanewline
\isanewline
\ \ \ \ \ \ \isacommand{have}\isamarkupfalse%
\ right{\isacharunderscore}{\kern0pt}eq{\isacharcolon}{\kern0pt}\ {\isachardoublequoteopen}right{\isacharunderscore}{\kern0pt}cart{\isacharunderscore}{\kern0pt}proj\ X\ Y\ {\isasymcirc}\isactrlsub c\ fibered{\isacharunderscore}{\kern0pt}product{\isacharunderscore}{\kern0pt}morphism\ X\ f\ g\ Y\ {\isasymcirc}\isactrlsub c\ j\ {\isacharequal}{\kern0pt}\ y{\isachardoublequoteclose}\isanewline
\ \ \ \ \ \ \ \ \isacommand{using}\isamarkupfalse%
\ j{\isacharunderscore}{\kern0pt}projs\ assms\ j{\isacharunderscore}{\kern0pt}type\ comp{\isacharunderscore}{\kern0pt}associative{\isadigit{2}}\isanewline
\ \ \ \ \ \ \ \ \isacommand{unfolding}\isamarkupfalse%
\ fibered{\isacharunderscore}{\kern0pt}product{\isacharunderscore}{\kern0pt}right{\isacharunderscore}{\kern0pt}proj{\isacharunderscore}{\kern0pt}def\ \isacommand{by}\isamarkupfalse%
\ {\isacharparenleft}{\kern0pt}typecheck{\isacharunderscore}{\kern0pt}cfuncs{\isacharcomma}{\kern0pt}\ auto{\isacharparenright}{\kern0pt}\isanewline
\isanewline
\ \ \ \ \ \ \isacommand{show}\isamarkupfalse%
\ {\isachardoublequoteopen}snd\ {\isacharparenleft}{\kern0pt}X\ \isactrlbsub f\isactrlesub {\isasymtimes}\isactrlsub c\isactrlbsub g\isactrlesub \ Y{\isacharcomma}{\kern0pt}\ fibered{\isacharunderscore}{\kern0pt}product{\isacharunderscore}{\kern0pt}morphism\ X\ f\ g\ Y{\isacharparenright}{\kern0pt}\ {\isasymcirc}\isactrlsub c\ j\ {\isacharequal}{\kern0pt}\ {\isasymlangle}x{\isacharcomma}{\kern0pt}y{\isasymrangle}{\isachardoublequoteclose}\isanewline
\ \ \ \ \ \ \ \ \isacommand{using}\isamarkupfalse%
\ left{\isacharunderscore}{\kern0pt}eq\ right{\isacharunderscore}{\kern0pt}eq\ assms\ j{\isacharunderscore}{\kern0pt}type\ \isacommand{by}\isamarkupfalse%
\ {\isacharparenleft}{\kern0pt}typecheck{\isacharunderscore}{\kern0pt}cfuncs{\isacharcomma}{\kern0pt}\ simp\ add{\isacharcolon}{\kern0pt}\ cfunc{\isacharunderscore}{\kern0pt}prod{\isacharunderscore}{\kern0pt}unique{\isacharparenright}{\kern0pt}\isanewline
\ \ \ \ \isacommand{qed}\isamarkupfalse%
\isanewline
\ \ \isacommand{qed}\isamarkupfalse%
\isanewline
\isacommand{qed}\isamarkupfalse%
%
\endisatagproof
{\isafoldproof}%
%
\isadelimproof
\isanewline
%
\endisadelimproof
\isanewline
\isacommand{lemma}\isamarkupfalse%
\ fibered{\isacharunderscore}{\kern0pt}product{\isacharunderscore}{\kern0pt}pair{\isacharunderscore}{\kern0pt}member{\isadigit{2}}{\isacharcolon}{\kern0pt}\isanewline
\ \ \isakeyword{assumes}\ {\isachardoublequoteopen}f\ {\isacharcolon}{\kern0pt}\ X\ {\isasymrightarrow}\ Y{\isachardoublequoteclose}\ {\isachardoublequoteopen}g\ {\isacharcolon}{\kern0pt}\ X\ {\isasymrightarrow}\ E{\isachardoublequoteclose}\ {\isachardoublequoteopen}x\ {\isasymin}\isactrlsub c\ X{\isachardoublequoteclose}\ {\isachardoublequoteopen}y\ {\isasymin}\isactrlsub c\ X{\isachardoublequoteclose}\isanewline
\ \ \isakeyword{assumes}\ {\isachardoublequoteopen}g\ {\isasymcirc}\isactrlsub c\ fibered{\isacharunderscore}{\kern0pt}product{\isacharunderscore}{\kern0pt}left{\isacharunderscore}{\kern0pt}proj\ X\ f\ f\ X\ {\isacharequal}{\kern0pt}\ g\ {\isasymcirc}\isactrlsub c\ fibered{\isacharunderscore}{\kern0pt}product{\isacharunderscore}{\kern0pt}right{\isacharunderscore}{\kern0pt}proj\ X\ f\ f\ X{\isachardoublequoteclose}\isanewline
\ \ \isakeyword{shows}\ {\isachardoublequoteopen}{\isasymforall}x\ y{\isachardot}{\kern0pt}\ x\ {\isasymin}\isactrlsub c\ X\ {\isasymlongrightarrow}\ y\ {\isasymin}\isactrlsub c\ X\ {\isasymlongrightarrow}\ {\isasymlangle}x{\isacharcomma}{\kern0pt}y{\isasymrangle}\ {\isasymin}\isactrlbsub X\ {\isasymtimes}\isactrlsub c\ X\isactrlesub \ {\isacharparenleft}{\kern0pt}X\ \isactrlbsub f\isactrlesub {\isasymtimes}\isactrlsub c\isactrlbsub f\isactrlesub \ X{\isacharcomma}{\kern0pt}\ fibered{\isacharunderscore}{\kern0pt}product{\isacharunderscore}{\kern0pt}morphism\ X\ f\ f\ X{\isacharparenright}{\kern0pt}\ {\isasymlongrightarrow}\ g\ {\isasymcirc}\isactrlsub c\ x\ {\isacharequal}{\kern0pt}\ g\ {\isasymcirc}\isactrlsub c\ y{\isachardoublequoteclose}\isanewline
%
\isadelimproof
%
\endisadelimproof
%
\isatagproof
\isacommand{proof}\isamarkupfalse%
{\isacharparenleft}{\kern0pt}clarify{\isacharparenright}{\kern0pt}\isanewline
\ \ \isacommand{fix}\isamarkupfalse%
\ x\ y\ \ \isanewline
\ \ \isacommand{assume}\isamarkupfalse%
\ x{\isacharunderscore}{\kern0pt}type{\isacharbrackleft}{\kern0pt}type{\isacharunderscore}{\kern0pt}rule{\isacharbrackright}{\kern0pt}{\isacharcolon}{\kern0pt}\ {\isachardoublequoteopen}x\ {\isasymin}\isactrlsub c\ X{\isachardoublequoteclose}\isanewline
\ \ \isacommand{assume}\isamarkupfalse%
\ y{\isacharunderscore}{\kern0pt}type{\isacharbrackleft}{\kern0pt}type{\isacharunderscore}{\kern0pt}rule{\isacharbrackright}{\kern0pt}{\isacharcolon}{\kern0pt}\ {\isachardoublequoteopen}y\ {\isasymin}\isactrlsub c\ X{\isachardoublequoteclose}\isanewline
\ \ \isacommand{assume}\isamarkupfalse%
\ a{\isadigit{3}}{\isacharcolon}{\kern0pt}\ {\isachardoublequoteopen}{\isasymlangle}x{\isacharcomma}{\kern0pt}y{\isasymrangle}\ {\isasymin}\isactrlbsub X\ {\isasymtimes}\isactrlsub c\ X\isactrlesub \ {\isacharparenleft}{\kern0pt}X\ \isactrlbsub f\isactrlesub {\isasymtimes}\isactrlsub c\isactrlbsub f\isactrlesub \ X{\isacharcomma}{\kern0pt}\ fibered{\isacharunderscore}{\kern0pt}product{\isacharunderscore}{\kern0pt}morphism\ X\ f\ f\ X{\isacharparenright}{\kern0pt}{\isachardoublequoteclose}\isanewline
\ \ \isacommand{then}\isamarkupfalse%
\ \isacommand{obtain}\isamarkupfalse%
\ h\ \isakeyword{where}\isanewline
\ \ \ \ h{\isacharunderscore}{\kern0pt}type{\isacharcolon}{\kern0pt}\ {\isachardoublequoteopen}h\ {\isasymin}\isactrlsub c\ X\isactrlbsub f\isactrlesub {\isasymtimes}\isactrlsub c\isactrlbsub f\isactrlesub X{\isachardoublequoteclose}\ \isakeyword{and}\ h{\isacharunderscore}{\kern0pt}eq{\isacharcolon}{\kern0pt}\ {\isachardoublequoteopen}fibered{\isacharunderscore}{\kern0pt}product{\isacharunderscore}{\kern0pt}morphism\ X\ f\ f\ X\ {\isasymcirc}\isactrlsub c\ h\ {\isacharequal}{\kern0pt}\ {\isasymlangle}x{\isacharcomma}{\kern0pt}y{\isasymrangle}{\isachardoublequoteclose}\isanewline
\ \ \ \ \isacommand{by}\isamarkupfalse%
\ {\isacharparenleft}{\kern0pt}meson\ factors{\isacharunderscore}{\kern0pt}through{\isacharunderscore}{\kern0pt}def{\isadigit{2}}\ relative{\isacharunderscore}{\kern0pt}member{\isacharunderscore}{\kern0pt}def{\isadigit{2}}{\isacharparenright}{\kern0pt}\isanewline
\isanewline
\ \ \isacommand{have}\isamarkupfalse%
\ left{\isacharunderscore}{\kern0pt}eq{\isacharcolon}{\kern0pt}\ {\isachardoublequoteopen}fibered{\isacharunderscore}{\kern0pt}product{\isacharunderscore}{\kern0pt}left{\isacharunderscore}{\kern0pt}proj\ X\ f\ f\ X\ {\isasymcirc}\isactrlsub c\ h\ {\isacharequal}{\kern0pt}\ x{\isachardoublequoteclose}\isanewline
\ \ \ \ \ \ \isacommand{unfolding}\isamarkupfalse%
\ fibered{\isacharunderscore}{\kern0pt}product{\isacharunderscore}{\kern0pt}left{\isacharunderscore}{\kern0pt}proj{\isacharunderscore}{\kern0pt}def\isanewline
\ \ \ \ \ \ \isacommand{by}\isamarkupfalse%
\ {\isacharparenleft}{\kern0pt}typecheck{\isacharunderscore}{\kern0pt}cfuncs{\isacharcomma}{\kern0pt}\ smt\ {\isacharparenleft}{\kern0pt}z{\isadigit{3}}{\isacharparenright}{\kern0pt}\ assms{\isacharparenleft}{\kern0pt}{\isadigit{1}}{\isacharparenright}{\kern0pt}\ comp{\isacharunderscore}{\kern0pt}associative{\isadigit{2}}\ h{\isacharunderscore}{\kern0pt}eq\ h{\isacharunderscore}{\kern0pt}type\ left{\isacharunderscore}{\kern0pt}cart{\isacharunderscore}{\kern0pt}proj{\isacharunderscore}{\kern0pt}cfunc{\isacharunderscore}{\kern0pt}prod\ y{\isacharunderscore}{\kern0pt}type{\isacharparenright}{\kern0pt}\isanewline
\ \ \ \ \isanewline
\ \ \isacommand{have}\isamarkupfalse%
\ right{\isacharunderscore}{\kern0pt}eq{\isacharcolon}{\kern0pt}\ {\isachardoublequoteopen}fibered{\isacharunderscore}{\kern0pt}product{\isacharunderscore}{\kern0pt}right{\isacharunderscore}{\kern0pt}proj\ X\ f\ f\ X\ {\isasymcirc}\isactrlsub c\ h\ {\isacharequal}{\kern0pt}\ y{\isachardoublequoteclose}\isanewline
\ \ \ \ \isacommand{unfolding}\isamarkupfalse%
\ fibered{\isacharunderscore}{\kern0pt}product{\isacharunderscore}{\kern0pt}right{\isacharunderscore}{\kern0pt}proj{\isacharunderscore}{\kern0pt}def\isanewline
\ \ \ \ \isacommand{by}\isamarkupfalse%
\ {\isacharparenleft}{\kern0pt}typecheck{\isacharunderscore}{\kern0pt}cfuncs{\isacharcomma}{\kern0pt}\ metis\ {\isacharparenleft}{\kern0pt}full{\isacharunderscore}{\kern0pt}types{\isacharparenright}{\kern0pt}\ a{\isadigit{3}}\ comp{\isacharunderscore}{\kern0pt}associative{\isadigit{2}}\ h{\isacharunderscore}{\kern0pt}eq\ h{\isacharunderscore}{\kern0pt}type\ relative{\isacharunderscore}{\kern0pt}member{\isacharunderscore}{\kern0pt}def{\isadigit{2}}\ right{\isacharunderscore}{\kern0pt}cart{\isacharunderscore}{\kern0pt}proj{\isacharunderscore}{\kern0pt}cfunc{\isacharunderscore}{\kern0pt}prod\ x{\isacharunderscore}{\kern0pt}type{\isacharparenright}{\kern0pt}\isanewline
\isanewline
\ \ \isacommand{then}\isamarkupfalse%
\ \isacommand{show}\isamarkupfalse%
\ {\isachardoublequoteopen}g\ {\isasymcirc}\isactrlsub c\ x\ {\isacharequal}{\kern0pt}\ g\ {\isasymcirc}\isactrlsub c\ y{\isachardoublequoteclose}\isanewline
\ \ \ \ \isacommand{using}\isamarkupfalse%
\ assms{\isacharparenleft}{\kern0pt}{\isadigit{1}}{\isacharcomma}{\kern0pt}{\isadigit{2}}{\isacharcomma}{\kern0pt}{\isadigit{5}}{\isacharparenright}{\kern0pt}\ cfunc{\isacharunderscore}{\kern0pt}type{\isacharunderscore}{\kern0pt}def\ comp{\isacharunderscore}{\kern0pt}associative\ fibered{\isacharunderscore}{\kern0pt}product{\isacharunderscore}{\kern0pt}left{\isacharunderscore}{\kern0pt}proj{\isacharunderscore}{\kern0pt}type\ fibered{\isacharunderscore}{\kern0pt}product{\isacharunderscore}{\kern0pt}right{\isacharunderscore}{\kern0pt}proj{\isacharunderscore}{\kern0pt}type\ h{\isacharunderscore}{\kern0pt}type\ left{\isacharunderscore}{\kern0pt}eq\ right{\isacharunderscore}{\kern0pt}eq\ \isacommand{by}\isamarkupfalse%
\ fastforce\isanewline
\isacommand{qed}\isamarkupfalse%
%
\endisatagproof
{\isafoldproof}%
%
\isadelimproof
\isanewline
%
\endisadelimproof
\isanewline
\isacommand{lemma}\isamarkupfalse%
\ kernel{\isacharunderscore}{\kern0pt}pair{\isacharunderscore}{\kern0pt}subset{\isacharcolon}{\kern0pt}\isanewline
\ \ \isakeyword{assumes}\ {\isachardoublequoteopen}f{\isacharcolon}{\kern0pt}\ X\ {\isasymrightarrow}\ Y{\isachardoublequoteclose}\isanewline
\ \ \isakeyword{shows}\ {\isachardoublequoteopen}{\isacharparenleft}{\kern0pt}X\ \isactrlbsub f\isactrlesub {\isasymtimes}\isactrlsub c\isactrlbsub f\isactrlesub \ X{\isacharcomma}{\kern0pt}\ fibered{\isacharunderscore}{\kern0pt}product{\isacharunderscore}{\kern0pt}morphism\ X\ f\ f\ X{\isacharparenright}{\kern0pt}\ {\isasymsubseteq}\isactrlsub c\ X\ {\isasymtimes}\isactrlsub c\ X{\isachardoublequoteclose}\isanewline
%
\isadelimproof
\ \ %
\endisadelimproof
%
\isatagproof
\isacommand{using}\isamarkupfalse%
\ assms\ fibered{\isacharunderscore}{\kern0pt}product{\isacharunderscore}{\kern0pt}morphism{\isacharunderscore}{\kern0pt}monomorphism\ fibered{\isacharunderscore}{\kern0pt}product{\isacharunderscore}{\kern0pt}morphism{\isacharunderscore}{\kern0pt}type\ subobject{\isacharunderscore}{\kern0pt}of{\isacharunderscore}{\kern0pt}def{\isadigit{2}}\ \isacommand{by}\isamarkupfalse%
\ auto%
\endisatagproof
{\isafoldproof}%
%
\isadelimproof
%
\endisadelimproof
%
\begin{isamarkuptext}%
The three lemmas below correspond to Exercise 2.1.44 in Halvorson.%
\end{isamarkuptext}\isamarkuptrue%
\isacommand{lemma}\isamarkupfalse%
\ kern{\isacharunderscore}{\kern0pt}pair{\isacharunderscore}{\kern0pt}proj{\isacharunderscore}{\kern0pt}iso{\isacharunderscore}{\kern0pt}TFAE{\isadigit{1}}{\isacharcolon}{\kern0pt}\isanewline
\ \ \isakeyword{assumes}\ {\isachardoublequoteopen}f{\isacharcolon}{\kern0pt}\ X\ {\isasymrightarrow}\ Y{\isachardoublequoteclose}\ {\isachardoublequoteopen}monomorphism\ f{\isachardoublequoteclose}\isanewline
\ \ \isakeyword{shows}\ {\isachardoublequoteopen}{\isacharparenleft}{\kern0pt}fibered{\isacharunderscore}{\kern0pt}product{\isacharunderscore}{\kern0pt}left{\isacharunderscore}{\kern0pt}proj\ X\ f\ f\ X{\isacharparenright}{\kern0pt}\ {\isacharequal}{\kern0pt}\ {\isacharparenleft}{\kern0pt}fibered{\isacharunderscore}{\kern0pt}product{\isacharunderscore}{\kern0pt}right{\isacharunderscore}{\kern0pt}proj\ X\ f\ f\ X{\isacharparenright}{\kern0pt}{\isachardoublequoteclose}\isanewline
%
\isadelimproof
%
\endisadelimproof
%
\isatagproof
\isacommand{proof}\isamarkupfalse%
\ {\isacharparenleft}{\kern0pt}cases\ {\isachardoublequoteopen}{\isasymexists}x{\isachardot}{\kern0pt}\ x{\isasymin}\isactrlsub c\ X\isactrlbsub f\isactrlesub {\isasymtimes}\isactrlsub c\isactrlbsub f\isactrlesub X{\isachardoublequoteclose}{\isacharcomma}{\kern0pt}\ clarify{\isacharparenright}{\kern0pt}\isanewline
\ \ \isacommand{fix}\isamarkupfalse%
\ x\isanewline
\ \ \isacommand{assume}\isamarkupfalse%
\ x{\isacharunderscore}{\kern0pt}type{\isacharcolon}{\kern0pt}\ {\isachardoublequoteopen}x{\isasymin}\isactrlsub c\ X\isactrlbsub f\isactrlesub {\isasymtimes}\isactrlsub c\isactrlbsub f\isactrlesub X{\isachardoublequoteclose}\isanewline
\ \ \isacommand{then}\isamarkupfalse%
\ \isacommand{have}\isamarkupfalse%
\ {\isachardoublequoteopen}{\isacharparenleft}{\kern0pt}f\ {\isasymcirc}\isactrlsub c\ {\isacharparenleft}{\kern0pt}fibered{\isacharunderscore}{\kern0pt}product{\isacharunderscore}{\kern0pt}left{\isacharunderscore}{\kern0pt}proj\ X\ f\ f\ X{\isacharparenright}{\kern0pt}{\isacharparenright}{\kern0pt}{\isasymcirc}\isactrlsub c\ x\ {\isacharequal}{\kern0pt}\ {\isacharparenleft}{\kern0pt}f{\isasymcirc}\isactrlsub c\ {\isacharparenleft}{\kern0pt}fibered{\isacharunderscore}{\kern0pt}product{\isacharunderscore}{\kern0pt}right{\isacharunderscore}{\kern0pt}proj\ X\ f\ f\ X{\isacharparenright}{\kern0pt}{\isacharparenright}{\kern0pt}{\isasymcirc}\isactrlsub c\ x{\isachardoublequoteclose}\isanewline
\ \ \ \ \isacommand{using}\isamarkupfalse%
\ assms\ cfunc{\isacharunderscore}{\kern0pt}type{\isacharunderscore}{\kern0pt}def\ comp{\isacharunderscore}{\kern0pt}associative\ equalizer{\isacharunderscore}{\kern0pt}def\ fibered{\isacharunderscore}{\kern0pt}product{\isacharunderscore}{\kern0pt}morphism{\isacharunderscore}{\kern0pt}equalizer\isanewline
\ \ \ \ \isacommand{unfolding}\isamarkupfalse%
\ fibered{\isacharunderscore}{\kern0pt}product{\isacharunderscore}{\kern0pt}right{\isacharunderscore}{\kern0pt}proj{\isacharunderscore}{\kern0pt}def\ fibered{\isacharunderscore}{\kern0pt}product{\isacharunderscore}{\kern0pt}left{\isacharunderscore}{\kern0pt}proj{\isacharunderscore}{\kern0pt}def\isanewline
\ \ \ \ \isacommand{by}\isamarkupfalse%
\ {\isacharparenleft}{\kern0pt}typecheck{\isacharunderscore}{\kern0pt}cfuncs{\isacharcomma}{\kern0pt}\ smt\ {\isacharparenleft}{\kern0pt}verit{\isacharparenright}{\kern0pt}{\isacharparenright}{\kern0pt}\isanewline
\ \ \isacommand{then}\isamarkupfalse%
\ \isacommand{have}\isamarkupfalse%
\ {\isachardoublequoteopen}f\ {\isasymcirc}\isactrlsub c\ {\isacharparenleft}{\kern0pt}fibered{\isacharunderscore}{\kern0pt}product{\isacharunderscore}{\kern0pt}left{\isacharunderscore}{\kern0pt}proj\ X\ f\ f\ X{\isacharparenright}{\kern0pt}\ {\isacharequal}{\kern0pt}\ f{\isasymcirc}\isactrlsub c\ {\isacharparenleft}{\kern0pt}fibered{\isacharunderscore}{\kern0pt}product{\isacharunderscore}{\kern0pt}right{\isacharunderscore}{\kern0pt}proj\ X\ f\ f\ X{\isacharparenright}{\kern0pt}{\isachardoublequoteclose}\isanewline
\ \ \ \ \isacommand{using}\isamarkupfalse%
\ assms\ fibered{\isacharunderscore}{\kern0pt}product{\isacharunderscore}{\kern0pt}is{\isacharunderscore}{\kern0pt}pullback\ is{\isacharunderscore}{\kern0pt}pullback{\isacharunderscore}{\kern0pt}def\ \isacommand{by}\isamarkupfalse%
\ auto\isanewline
\ \ \isacommand{then}\isamarkupfalse%
\ \isacommand{show}\isamarkupfalse%
\ {\isachardoublequoteopen}{\isacharparenleft}{\kern0pt}fibered{\isacharunderscore}{\kern0pt}product{\isacharunderscore}{\kern0pt}left{\isacharunderscore}{\kern0pt}proj\ X\ f\ f\ X{\isacharparenright}{\kern0pt}\ {\isacharequal}{\kern0pt}\ {\isacharparenleft}{\kern0pt}fibered{\isacharunderscore}{\kern0pt}product{\isacharunderscore}{\kern0pt}right{\isacharunderscore}{\kern0pt}proj\ X\ f\ f\ X{\isacharparenright}{\kern0pt}{\isachardoublequoteclose}\isanewline
\ \ \ \ \isacommand{using}\isamarkupfalse%
\ assms\ cfunc{\isacharunderscore}{\kern0pt}type{\isacharunderscore}{\kern0pt}def\ fibered{\isacharunderscore}{\kern0pt}product{\isacharunderscore}{\kern0pt}left{\isacharunderscore}{\kern0pt}proj{\isacharunderscore}{\kern0pt}type\ fibered{\isacharunderscore}{\kern0pt}product{\isacharunderscore}{\kern0pt}right{\isacharunderscore}{\kern0pt}proj{\isacharunderscore}{\kern0pt}type\ monomorphism{\isacharunderscore}{\kern0pt}def\ \isacommand{by}\isamarkupfalse%
\ auto\isanewline
\isacommand{next}\isamarkupfalse%
\isanewline
\ \ \isacommand{assume}\isamarkupfalse%
\ {\isachardoublequoteopen}{\isasymnexists}x{\isachardot}{\kern0pt}\ x\ {\isasymin}\isactrlsub c\ X\ \isactrlbsub f\isactrlesub {\isasymtimes}\isactrlsub c\isactrlbsub f\isactrlesub \ X{\isachardoublequoteclose}\isanewline
\ \ \isacommand{then}\isamarkupfalse%
\ \isacommand{show}\isamarkupfalse%
\ {\isachardoublequoteopen}fibered{\isacharunderscore}{\kern0pt}product{\isacharunderscore}{\kern0pt}left{\isacharunderscore}{\kern0pt}proj\ X\ f\ f\ X\ {\isacharequal}{\kern0pt}\ fibered{\isacharunderscore}{\kern0pt}product{\isacharunderscore}{\kern0pt}right{\isacharunderscore}{\kern0pt}proj\ X\ f\ f\ X{\isachardoublequoteclose}\isanewline
\ \ \ \ \isacommand{using}\isamarkupfalse%
\ assms\ fibered{\isacharunderscore}{\kern0pt}product{\isacharunderscore}{\kern0pt}left{\isacharunderscore}{\kern0pt}proj{\isacharunderscore}{\kern0pt}type\ fibered{\isacharunderscore}{\kern0pt}product{\isacharunderscore}{\kern0pt}right{\isacharunderscore}{\kern0pt}proj{\isacharunderscore}{\kern0pt}type\ one{\isacharunderscore}{\kern0pt}separator\ \isacommand{by}\isamarkupfalse%
\ blast\isanewline
\isacommand{qed}\isamarkupfalse%
%
\endisatagproof
{\isafoldproof}%
%
\isadelimproof
\isanewline
%
\endisadelimproof
\isanewline
\isacommand{lemma}\isamarkupfalse%
\ kern{\isacharunderscore}{\kern0pt}pair{\isacharunderscore}{\kern0pt}proj{\isacharunderscore}{\kern0pt}iso{\isacharunderscore}{\kern0pt}TFAE{\isadigit{2}}{\isacharcolon}{\kern0pt}\isanewline
\ \ \isakeyword{assumes}\ {\isachardoublequoteopen}f{\isacharcolon}{\kern0pt}\ X\ {\isasymrightarrow}\ Y{\isachardoublequoteclose}\ {\isachardoublequoteopen}fibered{\isacharunderscore}{\kern0pt}product{\isacharunderscore}{\kern0pt}left{\isacharunderscore}{\kern0pt}proj\ X\ f\ f\ X\ {\isacharequal}{\kern0pt}\ fibered{\isacharunderscore}{\kern0pt}product{\isacharunderscore}{\kern0pt}right{\isacharunderscore}{\kern0pt}proj\ X\ f\ f\ X{\isachardoublequoteclose}\isanewline
\ \ \isakeyword{shows}\ {\isachardoublequoteopen}monomorphism\ f\ {\isasymand}\ isomorphism\ {\isacharparenleft}{\kern0pt}fibered{\isacharunderscore}{\kern0pt}product{\isacharunderscore}{\kern0pt}left{\isacharunderscore}{\kern0pt}proj\ X\ f\ f\ X{\isacharparenright}{\kern0pt}\ {\isasymand}\ isomorphism\ {\isacharparenleft}{\kern0pt}fibered{\isacharunderscore}{\kern0pt}product{\isacharunderscore}{\kern0pt}right{\isacharunderscore}{\kern0pt}proj\ X\ f\ f\ X{\isacharparenright}{\kern0pt}{\isachardoublequoteclose}\isanewline
%
\isadelimproof
\ \ %
\endisadelimproof
%
\isatagproof
\isacommand{using}\isamarkupfalse%
\ assms\isanewline
\isacommand{proof}\isamarkupfalse%
\ safe\isanewline
\ \ \isacommand{have}\isamarkupfalse%
\ {\isachardoublequoteopen}injective\ f{\isachardoublequoteclose}\isanewline
\ \ \ \ \isacommand{unfolding}\isamarkupfalse%
\ injective{\isacharunderscore}{\kern0pt}def\isanewline
\ \ \isacommand{proof}\isamarkupfalse%
\ clarify\isanewline
\ \ \ \ \isacommand{fix}\isamarkupfalse%
\ x\ y\isanewline
\ \ \ \ \isacommand{assume}\isamarkupfalse%
\ x{\isacharunderscore}{\kern0pt}type{\isacharcolon}{\kern0pt}\ {\isachardoublequoteopen}x\ {\isasymin}\isactrlsub c\ domain\ f{\isachardoublequoteclose}\ \isakeyword{and}\ y{\isacharunderscore}{\kern0pt}type{\isacharcolon}{\kern0pt}\ {\isachardoublequoteopen}y\ {\isasymin}\isactrlsub c\ domain\ f{\isachardoublequoteclose}\isanewline
\ \ \ \ \isacommand{then}\isamarkupfalse%
\ \isacommand{have}\isamarkupfalse%
\ x{\isacharunderscore}{\kern0pt}type{\isadigit{2}}{\isacharcolon}{\kern0pt}\ {\isachardoublequoteopen}x\ {\isasymin}\isactrlsub c\ X{\isachardoublequoteclose}\ \isakeyword{and}\ y{\isacharunderscore}{\kern0pt}type{\isadigit{2}}{\isacharcolon}{\kern0pt}\ {\isachardoublequoteopen}y\ {\isasymin}\isactrlsub c\ X{\isachardoublequoteclose}\isanewline
\ \ \ \ \ \ \isacommand{using}\isamarkupfalse%
\ assms{\isacharparenleft}{\kern0pt}{\isadigit{1}}{\isacharparenright}{\kern0pt}\ cfunc{\isacharunderscore}{\kern0pt}type{\isacharunderscore}{\kern0pt}def\ \isacommand{by}\isamarkupfalse%
\ auto\isanewline
\isanewline
\ \ \ \ \isacommand{have}\isamarkupfalse%
\ x{\isacharunderscore}{\kern0pt}y{\isacharunderscore}{\kern0pt}type{\isacharcolon}{\kern0pt}\ {\isachardoublequoteopen}{\isasymlangle}x{\isacharcomma}{\kern0pt}y{\isasymrangle}\ {\isacharcolon}{\kern0pt}\ {\isasymone}\ {\isasymrightarrow}\ X\ {\isasymtimes}\isactrlsub c\ X{\isachardoublequoteclose}\isanewline
\ \ \ \ \ \ \isacommand{using}\isamarkupfalse%
\ x{\isacharunderscore}{\kern0pt}type{\isadigit{2}}\ y{\isacharunderscore}{\kern0pt}type{\isadigit{2}}\ \isacommand{by}\isamarkupfalse%
\ {\isacharparenleft}{\kern0pt}typecheck{\isacharunderscore}{\kern0pt}cfuncs{\isacharparenright}{\kern0pt}\isanewline
\ \ \ \ \isacommand{have}\isamarkupfalse%
\ fibered{\isacharunderscore}{\kern0pt}product{\isacharunderscore}{\kern0pt}type{\isacharcolon}{\kern0pt}\ {\isachardoublequoteopen}fibered{\isacharunderscore}{\kern0pt}product{\isacharunderscore}{\kern0pt}morphism\ X\ f\ f\ X\ {\isacharcolon}{\kern0pt}\ X\ \isactrlbsub f\isactrlesub {\isasymtimes}\isactrlsub c\isactrlbsub f\isactrlesub \ X\ {\isasymrightarrow}\ X\ {\isasymtimes}\isactrlsub c\ X{\isachardoublequoteclose}\isanewline
\ \ \ \ \ \ \isacommand{using}\isamarkupfalse%
\ assms\ \isacommand{by}\isamarkupfalse%
\ typecheck{\isacharunderscore}{\kern0pt}cfuncs\isanewline
\isanewline
\ \ \ \ \isacommand{assume}\isamarkupfalse%
\ {\isachardoublequoteopen}f\ {\isasymcirc}\isactrlsub c\ x\ {\isacharequal}{\kern0pt}\ f\ {\isasymcirc}\isactrlsub c\ y{\isachardoublequoteclose}\isanewline
\ \ \ \ \isacommand{then}\isamarkupfalse%
\ \isacommand{have}\isamarkupfalse%
\ factorsthru{\isacharcolon}{\kern0pt}\ {\isachardoublequoteopen}{\isasymlangle}x{\isacharcomma}{\kern0pt}y{\isasymrangle}\ factorsthru\ fibered{\isacharunderscore}{\kern0pt}product{\isacharunderscore}{\kern0pt}morphism\ X\ f\ f\ X{\isachardoublequoteclose}\isanewline
\ \ \ \ \ \ \isacommand{using}\isamarkupfalse%
\ assms{\isacharparenleft}{\kern0pt}{\isadigit{1}}{\isacharparenright}{\kern0pt}\ pair{\isacharunderscore}{\kern0pt}factorsthru{\isacharunderscore}{\kern0pt}fibered{\isacharunderscore}{\kern0pt}product{\isacharunderscore}{\kern0pt}morphism\ x{\isacharunderscore}{\kern0pt}type{\isadigit{2}}\ y{\isacharunderscore}{\kern0pt}type{\isadigit{2}}\ \isacommand{by}\isamarkupfalse%
\ auto\isanewline
\ \ \ \ \isacommand{then}\isamarkupfalse%
\ \isacommand{obtain}\isamarkupfalse%
\ xy\ \isakeyword{where}\ xy{\isacharunderscore}{\kern0pt}assms{\isacharcolon}{\kern0pt}\ {\isachardoublequoteopen}xy\ {\isacharcolon}{\kern0pt}\ {\isasymone}\ {\isasymrightarrow}\ X\ \isactrlbsub f\isactrlesub {\isasymtimes}\isactrlsub c\isactrlbsub f\isactrlesub \ X{\isachardoublequoteclose}\ {\isachardoublequoteopen}fibered{\isacharunderscore}{\kern0pt}product{\isacharunderscore}{\kern0pt}morphism\ X\ f\ f\ X\ {\isasymcirc}\isactrlsub c\ xy\ {\isacharequal}{\kern0pt}\ {\isasymlangle}x{\isacharcomma}{\kern0pt}y{\isasymrangle}{\isachardoublequoteclose}\isanewline
\ \ \ \ \ \ \isacommand{using}\isamarkupfalse%
\ factors{\isacharunderscore}{\kern0pt}through{\isacharunderscore}{\kern0pt}def{\isadigit{2}}\ fibered{\isacharunderscore}{\kern0pt}product{\isacharunderscore}{\kern0pt}type\ x{\isacharunderscore}{\kern0pt}y{\isacharunderscore}{\kern0pt}type\ \isacommand{by}\isamarkupfalse%
\ blast\isanewline
\isanewline
\ \ \ \ \isacommand{have}\isamarkupfalse%
\ left{\isacharunderscore}{\kern0pt}proj{\isacharcolon}{\kern0pt}\ {\isachardoublequoteopen}fibered{\isacharunderscore}{\kern0pt}product{\isacharunderscore}{\kern0pt}left{\isacharunderscore}{\kern0pt}proj\ X\ f\ f\ X\ {\isasymcirc}\isactrlsub c\ xy\ {\isacharequal}{\kern0pt}\ x{\isachardoublequoteclose}\isanewline
\ \ \ \ \ \ \isacommand{unfolding}\isamarkupfalse%
\ fibered{\isacharunderscore}{\kern0pt}product{\isacharunderscore}{\kern0pt}left{\isacharunderscore}{\kern0pt}proj{\isacharunderscore}{\kern0pt}def\ \isacommand{using}\isamarkupfalse%
\ assms\ xy{\isacharunderscore}{\kern0pt}assms\isanewline
\ \ \ \ \ \ \isacommand{by}\isamarkupfalse%
\ {\isacharparenleft}{\kern0pt}typecheck{\isacharunderscore}{\kern0pt}cfuncs{\isacharcomma}{\kern0pt}\ metis\ cfunc{\isacharunderscore}{\kern0pt}type{\isacharunderscore}{\kern0pt}def\ comp{\isacharunderscore}{\kern0pt}associative\ left{\isacharunderscore}{\kern0pt}cart{\isacharunderscore}{\kern0pt}proj{\isacharunderscore}{\kern0pt}cfunc{\isacharunderscore}{\kern0pt}prod\ x{\isacharunderscore}{\kern0pt}type{\isadigit{2}}\ xy{\isacharunderscore}{\kern0pt}assms{\isacharparenleft}{\kern0pt}{\isadigit{2}}{\isacharparenright}{\kern0pt}\ y{\isacharunderscore}{\kern0pt}type{\isadigit{2}}{\isacharparenright}{\kern0pt}\isanewline
\ \ \ \ \isacommand{have}\isamarkupfalse%
\ right{\isacharunderscore}{\kern0pt}proj{\isacharcolon}{\kern0pt}\ {\isachardoublequoteopen}fibered{\isacharunderscore}{\kern0pt}product{\isacharunderscore}{\kern0pt}right{\isacharunderscore}{\kern0pt}proj\ X\ f\ f\ X\ {\isasymcirc}\isactrlsub c\ xy\ {\isacharequal}{\kern0pt}\ y{\isachardoublequoteclose}\isanewline
\ \ \ \ \ \ \isacommand{unfolding}\isamarkupfalse%
\ fibered{\isacharunderscore}{\kern0pt}product{\isacharunderscore}{\kern0pt}right{\isacharunderscore}{\kern0pt}proj{\isacharunderscore}{\kern0pt}def\ \isacommand{using}\isamarkupfalse%
\ assms\ xy{\isacharunderscore}{\kern0pt}assms\isanewline
\ \ \ \ \ \ \isacommand{by}\isamarkupfalse%
\ {\isacharparenleft}{\kern0pt}typecheck{\isacharunderscore}{\kern0pt}cfuncs{\isacharcomma}{\kern0pt}\ metis\ cfunc{\isacharunderscore}{\kern0pt}type{\isacharunderscore}{\kern0pt}def\ comp{\isacharunderscore}{\kern0pt}associative\ right{\isacharunderscore}{\kern0pt}cart{\isacharunderscore}{\kern0pt}proj{\isacharunderscore}{\kern0pt}cfunc{\isacharunderscore}{\kern0pt}prod\ x{\isacharunderscore}{\kern0pt}type{\isadigit{2}}\ xy{\isacharunderscore}{\kern0pt}assms{\isacharparenleft}{\kern0pt}{\isadigit{2}}{\isacharparenright}{\kern0pt}\ y{\isacharunderscore}{\kern0pt}type{\isadigit{2}}{\isacharparenright}{\kern0pt}\isanewline
\ \ \ \ \ \ \isanewline
\ \ \ \ \isacommand{show}\isamarkupfalse%
\ {\isachardoublequoteopen}x\ {\isacharequal}{\kern0pt}\ y{\isachardoublequoteclose}\isanewline
\ \ \ \ \ \ \isacommand{using}\isamarkupfalse%
\ assms{\isacharparenleft}{\kern0pt}{\isadigit{2}}{\isacharparenright}{\kern0pt}\ left{\isacharunderscore}{\kern0pt}proj\ right{\isacharunderscore}{\kern0pt}proj\ \isacommand{by}\isamarkupfalse%
\ auto\isanewline
\ \ \isacommand{qed}\isamarkupfalse%
\isanewline
\ \ \isacommand{then}\isamarkupfalse%
\ \isacommand{show}\isamarkupfalse%
\ {\isachardoublequoteopen}monomorphism\ f{\isachardoublequoteclose}\isanewline
\ \ \ \ \isacommand{using}\isamarkupfalse%
\ injective{\isacharunderscore}{\kern0pt}imp{\isacharunderscore}{\kern0pt}monomorphism\ \isacommand{by}\isamarkupfalse%
\ blast\isanewline
\isacommand{next}\isamarkupfalse%
\isanewline
\ \ \isacommand{have}\isamarkupfalse%
\ {\isachardoublequoteopen}diagonal\ X\ factorsthru\ fibered{\isacharunderscore}{\kern0pt}product{\isacharunderscore}{\kern0pt}morphism\ X\ f\ f\ X{\isachardoublequoteclose}\isanewline
\ \ \ \ \isacommand{using}\isamarkupfalse%
\ assms{\isacharparenleft}{\kern0pt}{\isadigit{1}}{\isacharparenright}{\kern0pt}\ diagonal{\isacharunderscore}{\kern0pt}def\ id{\isacharunderscore}{\kern0pt}type\ pair{\isacharunderscore}{\kern0pt}factorsthru{\isacharunderscore}{\kern0pt}fibered{\isacharunderscore}{\kern0pt}product{\isacharunderscore}{\kern0pt}morphism\ \isacommand{by}\isamarkupfalse%
\ fastforce\isanewline
\ \ \isacommand{then}\isamarkupfalse%
\ \isacommand{obtain}\isamarkupfalse%
\ xx\ \isakeyword{where}\ xx{\isacharunderscore}{\kern0pt}assms{\isacharcolon}{\kern0pt}\ {\isachardoublequoteopen}xx\ {\isacharcolon}{\kern0pt}\ X\ {\isasymrightarrow}\ X\ \isactrlbsub f\isactrlesub {\isasymtimes}\isactrlsub c\isactrlbsub f\isactrlesub \ X{\isachardoublequoteclose}\ {\isachardoublequoteopen}diagonal\ X\ {\isacharequal}{\kern0pt}\ fibered{\isacharunderscore}{\kern0pt}product{\isacharunderscore}{\kern0pt}morphism\ X\ f\ f\ X\ {\isasymcirc}\isactrlsub c\ xx{\isachardoublequoteclose}\isanewline
\ \ \ \ \isacommand{using}\isamarkupfalse%
\ assms{\isacharparenleft}{\kern0pt}{\isadigit{1}}{\isacharparenright}{\kern0pt}\ cfunc{\isacharunderscore}{\kern0pt}type{\isacharunderscore}{\kern0pt}def\ diagonal{\isacharunderscore}{\kern0pt}type\ factors{\isacharunderscore}{\kern0pt}through{\isacharunderscore}{\kern0pt}def\ fibered{\isacharunderscore}{\kern0pt}product{\isacharunderscore}{\kern0pt}morphism{\isacharunderscore}{\kern0pt}type\ \isacommand{by}\isamarkupfalse%
\ fastforce\isanewline
\ \ \isacommand{have}\isamarkupfalse%
\ eq{\isadigit{1}}{\isacharcolon}{\kern0pt}\ {\isachardoublequoteopen}fibered{\isacharunderscore}{\kern0pt}product{\isacharunderscore}{\kern0pt}right{\isacharunderscore}{\kern0pt}proj\ X\ f\ f\ X\ {\isasymcirc}\isactrlsub c\ xx\ {\isacharequal}{\kern0pt}\ id\ X{\isachardoublequoteclose}\isanewline
\ \ \ \ \isacommand{by}\isamarkupfalse%
\ {\isacharparenleft}{\kern0pt}smt\ assms{\isacharparenleft}{\kern0pt}{\isadigit{1}}{\isacharparenright}{\kern0pt}\ comp{\isacharunderscore}{\kern0pt}associative{\isadigit{2}}\ diagonal{\isacharunderscore}{\kern0pt}def\ fibered{\isacharunderscore}{\kern0pt}product{\isacharunderscore}{\kern0pt}morphism{\isacharunderscore}{\kern0pt}type\ fibered{\isacharunderscore}{\kern0pt}product{\isacharunderscore}{\kern0pt}right{\isacharunderscore}{\kern0pt}proj{\isacharunderscore}{\kern0pt}def\ id{\isacharunderscore}{\kern0pt}type\ right{\isacharunderscore}{\kern0pt}cart{\isacharunderscore}{\kern0pt}proj{\isacharunderscore}{\kern0pt}cfunc{\isacharunderscore}{\kern0pt}prod\ right{\isacharunderscore}{\kern0pt}cart{\isacharunderscore}{\kern0pt}proj{\isacharunderscore}{\kern0pt}type\ xx{\isacharunderscore}{\kern0pt}assms{\isacharparenright}{\kern0pt}\isanewline
\isanewline
\ \ \isacommand{have}\isamarkupfalse%
\ eq{\isadigit{2}}{\isacharcolon}{\kern0pt}\ {\isachardoublequoteopen}xx\ {\isasymcirc}\isactrlsub c\ fibered{\isacharunderscore}{\kern0pt}product{\isacharunderscore}{\kern0pt}right{\isacharunderscore}{\kern0pt}proj\ X\ f\ f\ X\ {\isacharequal}{\kern0pt}\ id\ {\isacharparenleft}{\kern0pt}X\ \isactrlbsub f\isactrlesub {\isasymtimes}\isactrlsub c\isactrlbsub f\isactrlesub \ X{\isacharparenright}{\kern0pt}{\isachardoublequoteclose}\isanewline
\ \ \isacommand{proof}\isamarkupfalse%
\ {\isacharparenleft}{\kern0pt}rule\ one{\isacharunderscore}{\kern0pt}separator{\isacharbrackleft}{\kern0pt}\isakeyword{where}\ X{\isacharequal}{\kern0pt}{\isachardoublequoteopen}X\ \isactrlbsub f\isactrlesub {\isasymtimes}\isactrlsub c\isactrlbsub f\isactrlesub \ X{\isachardoublequoteclose}{\isacharcomma}{\kern0pt}\ \isakeyword{where}\ Y{\isacharequal}{\kern0pt}{\isachardoublequoteopen}X\ \isactrlbsub f\isactrlesub {\isasymtimes}\isactrlsub c\isactrlbsub f\isactrlesub \ X{\isachardoublequoteclose}{\isacharbrackright}{\kern0pt}{\isacharparenright}{\kern0pt}\isanewline
\ \ \ \ \isacommand{show}\isamarkupfalse%
\ {\isachardoublequoteopen}xx\ {\isasymcirc}\isactrlsub c\ fibered{\isacharunderscore}{\kern0pt}product{\isacharunderscore}{\kern0pt}right{\isacharunderscore}{\kern0pt}proj\ X\ f\ f\ X\ {\isacharcolon}{\kern0pt}\ X\ \isactrlbsub f\isactrlesub {\isasymtimes}\isactrlsub c\isactrlbsub f\isactrlesub \ X\ {\isasymrightarrow}\ X\ \isactrlbsub f\isactrlesub {\isasymtimes}\isactrlsub c\isactrlbsub f\isactrlesub \ X{\isachardoublequoteclose}\isanewline
\ \ \ \ \ \ \isacommand{using}\isamarkupfalse%
\ assms{\isacharparenleft}{\kern0pt}{\isadigit{1}}{\isacharparenright}{\kern0pt}\ comp{\isacharunderscore}{\kern0pt}type\ fibered{\isacharunderscore}{\kern0pt}product{\isacharunderscore}{\kern0pt}right{\isacharunderscore}{\kern0pt}proj{\isacharunderscore}{\kern0pt}type\ xx{\isacharunderscore}{\kern0pt}assms\ \isacommand{by}\isamarkupfalse%
\ blast\isanewline
\ \ \ \ \isacommand{show}\isamarkupfalse%
\ {\isachardoublequoteopen}id\isactrlsub c\ {\isacharparenleft}{\kern0pt}X\ \isactrlbsub f\isactrlesub {\isasymtimes}\isactrlsub c\isactrlbsub f\isactrlesub \ X{\isacharparenright}{\kern0pt}\ {\isacharcolon}{\kern0pt}\ X\ \isactrlbsub f\isactrlesub {\isasymtimes}\isactrlsub c\isactrlbsub f\isactrlesub \ X\ {\isasymrightarrow}\ X\ \isactrlbsub f\isactrlesub {\isasymtimes}\isactrlsub c\isactrlbsub f\isactrlesub \ X{\isachardoublequoteclose}\isanewline
\ \ \ \ \ \ \isacommand{by}\isamarkupfalse%
\ {\isacharparenleft}{\kern0pt}simp\ add{\isacharcolon}{\kern0pt}\ id{\isacharunderscore}{\kern0pt}type{\isacharparenright}{\kern0pt}\isanewline
\ \ \isacommand{next}\isamarkupfalse%
\isanewline
\ \ \ \ \isacommand{fix}\isamarkupfalse%
\ x\isanewline
\ \ \ \ \isacommand{assume}\isamarkupfalse%
\ x{\isacharunderscore}{\kern0pt}type{\isacharcolon}{\kern0pt}\ {\isachardoublequoteopen}x\ {\isasymin}\isactrlsub c\ X\ \isactrlbsub f\isactrlesub {\isasymtimes}\isactrlsub c\isactrlbsub f\isactrlesub \ X{\isachardoublequoteclose}\isanewline
\ \ \ \ \isacommand{then}\isamarkupfalse%
\ \isacommand{obtain}\isamarkupfalse%
\ a\ \isakeyword{where}\ a{\isacharunderscore}{\kern0pt}assms{\isacharcolon}{\kern0pt}\ {\isachardoublequoteopen}{\isasymlangle}a{\isacharcomma}{\kern0pt}a{\isasymrangle}\ {\isacharequal}{\kern0pt}\ fibered{\isacharunderscore}{\kern0pt}product{\isacharunderscore}{\kern0pt}morphism\ X\ f\ f\ X\ {\isasymcirc}\isactrlsub c\ x{\isachardoublequoteclose}\ {\isachardoublequoteopen}a\ {\isasymin}\isactrlsub c\ X{\isachardoublequoteclose}\isanewline
\ \ \ \ \ \ \isacommand{by}\isamarkupfalse%
\ {\isacharparenleft}{\kern0pt}smt\ assms\ cfunc{\isacharunderscore}{\kern0pt}prod{\isacharunderscore}{\kern0pt}comp\ cfunc{\isacharunderscore}{\kern0pt}prod{\isacharunderscore}{\kern0pt}unique\ comp{\isacharunderscore}{\kern0pt}type\ fibered{\isacharunderscore}{\kern0pt}product{\isacharunderscore}{\kern0pt}left{\isacharunderscore}{\kern0pt}proj{\isacharunderscore}{\kern0pt}def\isanewline
\ \ \ \ \ \ \ \ \ \ fibered{\isacharunderscore}{\kern0pt}product{\isacharunderscore}{\kern0pt}morphism{\isacharunderscore}{\kern0pt}type\ fibered{\isacharunderscore}{\kern0pt}product{\isacharunderscore}{\kern0pt}right{\isacharunderscore}{\kern0pt}proj{\isacharunderscore}{\kern0pt}def\ fibered{\isacharunderscore}{\kern0pt}product{\isacharunderscore}{\kern0pt}right{\isacharunderscore}{\kern0pt}proj{\isacharunderscore}{\kern0pt}type{\isacharparenright}{\kern0pt}\isanewline
\isanewline
\ \ \ \ \isacommand{have}\isamarkupfalse%
\ {\isachardoublequoteopen}{\isacharparenleft}{\kern0pt}xx\ {\isasymcirc}\isactrlsub c\ fibered{\isacharunderscore}{\kern0pt}product{\isacharunderscore}{\kern0pt}right{\isacharunderscore}{\kern0pt}proj\ X\ f\ f\ X{\isacharparenright}{\kern0pt}\ {\isasymcirc}\isactrlsub c\ x\ {\isacharequal}{\kern0pt}\ xx\ {\isasymcirc}\isactrlsub c\ right{\isacharunderscore}{\kern0pt}cart{\isacharunderscore}{\kern0pt}proj\ X\ X\ {\isasymcirc}\isactrlsub c\ {\isasymlangle}a{\isacharcomma}{\kern0pt}a{\isasymrangle}{\isachardoublequoteclose}\isanewline
\ \ \ \ \ \ \isacommand{using}\isamarkupfalse%
\ xx{\isacharunderscore}{\kern0pt}assms\ x{\isacharunderscore}{\kern0pt}type\ a{\isacharunderscore}{\kern0pt}assms\ assms\ comp{\isacharunderscore}{\kern0pt}associative{\isadigit{2}}\isanewline
\ \ \ \ \ \ \isacommand{unfolding}\isamarkupfalse%
\ fibered{\isacharunderscore}{\kern0pt}product{\isacharunderscore}{\kern0pt}right{\isacharunderscore}{\kern0pt}proj{\isacharunderscore}{\kern0pt}def\isanewline
\ \ \ \ \ \ \isacommand{by}\isamarkupfalse%
\ {\isacharparenleft}{\kern0pt}typecheck{\isacharunderscore}{\kern0pt}cfuncs{\isacharcomma}{\kern0pt}\ auto{\isacharparenright}{\kern0pt}\isanewline
\ \ \ \ \isacommand{also}\isamarkupfalse%
\ \isacommand{have}\isamarkupfalse%
\ {\isachardoublequoteopen}{\isachardot}{\kern0pt}{\isachardot}{\kern0pt}{\isachardot}{\kern0pt}\ {\isacharequal}{\kern0pt}\ xx\ {\isasymcirc}\isactrlsub c\ a{\isachardoublequoteclose}\isanewline
\ \ \ \ \ \ \isacommand{using}\isamarkupfalse%
\ a{\isacharunderscore}{\kern0pt}assms{\isacharparenleft}{\kern0pt}{\isadigit{2}}{\isacharparenright}{\kern0pt}\ right{\isacharunderscore}{\kern0pt}cart{\isacharunderscore}{\kern0pt}proj{\isacharunderscore}{\kern0pt}cfunc{\isacharunderscore}{\kern0pt}prod\ \isacommand{by}\isamarkupfalse%
\ auto\isanewline
\ \ \ \ \isacommand{also}\isamarkupfalse%
\ \isacommand{have}\isamarkupfalse%
\ {\isachardoublequoteopen}{\isachardot}{\kern0pt}{\isachardot}{\kern0pt}{\isachardot}{\kern0pt}\ {\isacharequal}{\kern0pt}\ x{\isachardoublequoteclose}\isanewline
\ \ \ \ \isacommand{proof}\isamarkupfalse%
\ {\isacharminus}{\kern0pt}\isanewline
\ \ \ \ \ \ \isacommand{have}\isamarkupfalse%
\ f{\isadigit{2}}{\isacharcolon}{\kern0pt}\ {\isachardoublequoteopen}{\isasymforall}c{\isachardot}{\kern0pt}\ c\ {\isacharcolon}{\kern0pt}\ {\isasymone}\ {\isasymrightarrow}\ X\ {\isasymlongrightarrow}\ fibered{\isacharunderscore}{\kern0pt}product{\isacharunderscore}{\kern0pt}morphism\ X\ f\ f\ X\ {\isasymcirc}\isactrlsub c\ xx\ {\isasymcirc}\isactrlsub c\ c\ {\isacharequal}{\kern0pt}\ diagonal\ X\ {\isasymcirc}\isactrlsub c\ c{\isachardoublequoteclose}\isanewline
\ \ \ \ \ \ \isacommand{proof}\isamarkupfalse%
\ safe\isanewline
\ \ \ \ \ \ \ \ \isacommand{fix}\isamarkupfalse%
\ c\isanewline
\ \ \ \ \ \ \ \ \isacommand{assume}\isamarkupfalse%
\ {\isachardoublequoteopen}c\ {\isasymin}\isactrlsub c\ X{\isachardoublequoteclose}\isanewline
\ \ \ \ \ \ \ \ \isacommand{then}\isamarkupfalse%
\ \isacommand{show}\isamarkupfalse%
\ {\isachardoublequoteopen}fibered{\isacharunderscore}{\kern0pt}product{\isacharunderscore}{\kern0pt}morphism\ X\ f\ f\ X\ {\isasymcirc}\isactrlsub c\ xx\ {\isasymcirc}\isactrlsub c\ c\ {\isacharequal}{\kern0pt}\ diagonal\ X\ {\isasymcirc}\isactrlsub c\ c{\isachardoublequoteclose}\isanewline
\ \ \ \ \ \ \ \ \ \ \isacommand{using}\isamarkupfalse%
\ assms\ xx{\isacharunderscore}{\kern0pt}assms\ \isacommand{by}\isamarkupfalse%
\ {\isacharparenleft}{\kern0pt}typecheck{\isacharunderscore}{\kern0pt}cfuncs{\isacharcomma}{\kern0pt}\ simp\ add{\isacharcolon}{\kern0pt}\ comp{\isacharunderscore}{\kern0pt}associative{\isadigit{2}}\ xx{\isacharunderscore}{\kern0pt}assms{\isacharparenleft}{\kern0pt}{\isadigit{2}}{\isacharparenright}{\kern0pt}{\isacharparenright}{\kern0pt}\isanewline
\ \ \ \ \ \ \isacommand{qed}\isamarkupfalse%
\isanewline
\ \ \ \ \ \ \isacommand{have}\isamarkupfalse%
\ f{\isadigit{4}}{\isacharcolon}{\kern0pt}\ {\isachardoublequoteopen}xx\ {\isacharcolon}{\kern0pt}\ X\ {\isasymrightarrow}\ codomain\ xx{\isachardoublequoteclose}\isanewline
\ \ \ \ \ \ \ \ \isacommand{using}\isamarkupfalse%
\ cfunc{\isacharunderscore}{\kern0pt}type{\isacharunderscore}{\kern0pt}def\ xx{\isacharunderscore}{\kern0pt}assms\ \isacommand{by}\isamarkupfalse%
\ presburger\isanewline
\ \ \ \ \ \ \isacommand{have}\isamarkupfalse%
\ f{\isadigit{5}}{\isacharcolon}{\kern0pt}\ {\isachardoublequoteopen}diagonal\ X\ {\isasymcirc}\isactrlsub c\ a\ {\isacharequal}{\kern0pt}\ {\isasymlangle}a{\isacharcomma}{\kern0pt}a{\isasymrangle}{\isachardoublequoteclose}\isanewline
\ \ \ \ \ \ \ \ \isacommand{using}\isamarkupfalse%
\ a{\isacharunderscore}{\kern0pt}assms\ diag{\isacharunderscore}{\kern0pt}on{\isacharunderscore}{\kern0pt}elements\ \isacommand{by}\isamarkupfalse%
\ blast\isanewline
\ \ \ \ \ \ \isacommand{have}\isamarkupfalse%
\ f{\isadigit{6}}{\isacharcolon}{\kern0pt}\ {\isachardoublequoteopen}codomain\ {\isacharparenleft}{\kern0pt}xx\ {\isasymcirc}\isactrlsub c\ a{\isacharparenright}{\kern0pt}\ {\isacharequal}{\kern0pt}\ codomain\ xx{\isachardoublequoteclose}\isanewline
\ \ \ \ \ \ \ \ \isacommand{using}\isamarkupfalse%
\ f{\isadigit{4}}\ \isacommand{by}\isamarkupfalse%
\ {\isacharparenleft}{\kern0pt}meson\ a{\isacharunderscore}{\kern0pt}assms\ cfunc{\isacharunderscore}{\kern0pt}type{\isacharunderscore}{\kern0pt}def\ comp{\isacharunderscore}{\kern0pt}type{\isacharparenright}{\kern0pt}\isanewline
\ \ \ \ \ \ \isacommand{then}\isamarkupfalse%
\ \isacommand{have}\isamarkupfalse%
\ f{\isadigit{9}}{\isacharcolon}{\kern0pt}\ {\isachardoublequoteopen}x\ {\isacharcolon}{\kern0pt}\ domain\ x\ {\isasymrightarrow}\ codomain\ xx{\isachardoublequoteclose}\isanewline
\ \ \ \ \ \ \ \ \isacommand{using}\isamarkupfalse%
\ cfunc{\isacharunderscore}{\kern0pt}type{\isacharunderscore}{\kern0pt}def\ x{\isacharunderscore}{\kern0pt}type\ xx{\isacharunderscore}{\kern0pt}assms\ \isacommand{by}\isamarkupfalse%
\ auto\isanewline
\ \ \ \ \ \ \isacommand{have}\isamarkupfalse%
\ f{\isadigit{1}}{\isadigit{0}}{\isacharcolon}{\kern0pt}\ {\isachardoublequoteopen}{\isasymforall}c\ ca{\isachardot}{\kern0pt}\ domain\ {\isacharparenleft}{\kern0pt}ca\ {\isasymcirc}\isactrlsub c\ a{\isacharparenright}{\kern0pt}\ {\isacharequal}{\kern0pt}\ {\isasymone}\ {\isasymor}\ {\isasymnot}\ ca\ {\isacharcolon}{\kern0pt}\ X\ {\isasymrightarrow}\ c{\isachardoublequoteclose}\isanewline
\ \ \ \ \ \ \ \ \isacommand{by}\isamarkupfalse%
\ {\isacharparenleft}{\kern0pt}meson\ a{\isacharunderscore}{\kern0pt}assms\ cfunc{\isacharunderscore}{\kern0pt}type{\isacharunderscore}{\kern0pt}def\ comp{\isacharunderscore}{\kern0pt}type{\isacharparenright}{\kern0pt}\isanewline
\ \ \ \ \ \ \isacommand{then}\isamarkupfalse%
\ \isacommand{have}\isamarkupfalse%
\ {\isachardoublequoteopen}domain\ {\isasymlangle}a{\isacharcomma}{\kern0pt}a{\isasymrangle}\ {\isacharequal}{\kern0pt}\ {\isasymone}{\isachardoublequoteclose}\isanewline
\ \ \ \ \ \ \ \ \isacommand{using}\isamarkupfalse%
\ diagonal{\isacharunderscore}{\kern0pt}type\ f{\isadigit{5}}\ \isacommand{by}\isamarkupfalse%
\ force\isanewline
\ \ \ \ \ \ \isacommand{then}\isamarkupfalse%
\ \isacommand{have}\isamarkupfalse%
\ f{\isadigit{1}}{\isadigit{1}}{\isacharcolon}{\kern0pt}\ {\isachardoublequoteopen}domain\ x\ {\isacharequal}{\kern0pt}\ {\isasymone}{\isachardoublequoteclose}\isanewline
\ \ \ \ \ \ \ \ \isacommand{using}\isamarkupfalse%
\ cfunc{\isacharunderscore}{\kern0pt}type{\isacharunderscore}{\kern0pt}def\ x{\isacharunderscore}{\kern0pt}type\ \isacommand{by}\isamarkupfalse%
\ blast\isanewline
\ \ \ \ \ \ \isacommand{have}\isamarkupfalse%
\ {\isachardoublequoteopen}xx\ {\isasymcirc}\isactrlsub c\ a\ {\isasymin}\isactrlsub c\ codomain\ xx{\isachardoublequoteclose}\isanewline
\ \ \ \ \ \ \ \ \isacommand{using}\isamarkupfalse%
\ a{\isacharunderscore}{\kern0pt}assms\ comp{\isacharunderscore}{\kern0pt}type\ f{\isadigit{4}}\ \isacommand{by}\isamarkupfalse%
\ auto\isanewline
\ \ \ \ \ \ \isacommand{then}\isamarkupfalse%
\ \isacommand{show}\isamarkupfalse%
\ {\isacharquery}{\kern0pt}thesis\isanewline
\ \ \ \ \ \ \ \ \isacommand{using}\isamarkupfalse%
\ f{\isadigit{1}}{\isadigit{1}}\ f{\isadigit{9}}\ f{\isadigit{5}}\ f{\isadigit{2}}\ a{\isacharunderscore}{\kern0pt}assms\ assms{\isacharparenleft}{\kern0pt}{\isadigit{1}}{\isacharparenright}{\kern0pt}\ cfunc{\isacharunderscore}{\kern0pt}type{\isacharunderscore}{\kern0pt}def\ fibered{\isacharunderscore}{\kern0pt}product{\isacharunderscore}{\kern0pt}morphism{\isacharunderscore}{\kern0pt}monomorphism\ \isanewline
\ \ \ \ \ \ \ \ \ \ \ \ \ \ fibered{\isacharunderscore}{\kern0pt}product{\isacharunderscore}{\kern0pt}morphism{\isacharunderscore}{\kern0pt}type\ monomorphism{\isacharunderscore}{\kern0pt}def\ x{\isacharunderscore}{\kern0pt}type\isanewline
\ \ \ \ \ \ \ \ \isacommand{by}\isamarkupfalse%
\ auto\isanewline
\ \ \ \ \isacommand{qed}\isamarkupfalse%
\isanewline
\ \ \ \ \isacommand{also}\isamarkupfalse%
\ \isacommand{have}\isamarkupfalse%
\ {\isachardoublequoteopen}{\isachardot}{\kern0pt}{\isachardot}{\kern0pt}{\isachardot}{\kern0pt}\ {\isacharequal}{\kern0pt}\ id\isactrlsub c\ {\isacharparenleft}{\kern0pt}X\ \isactrlbsub f\isactrlesub {\isasymtimes}\isactrlsub c\isactrlbsub f\isactrlesub \ X{\isacharparenright}{\kern0pt}\ {\isasymcirc}\isactrlsub c\ x{\isachardoublequoteclose}\isanewline
\ \ \ \ \ \ \isacommand{by}\isamarkupfalse%
\ {\isacharparenleft}{\kern0pt}metis\ cfunc{\isacharunderscore}{\kern0pt}type{\isacharunderscore}{\kern0pt}def\ id{\isacharunderscore}{\kern0pt}left{\isacharunderscore}{\kern0pt}unit\ x{\isacharunderscore}{\kern0pt}type{\isacharparenright}{\kern0pt}\isanewline
\ \ \ \ \isacommand{then}\isamarkupfalse%
\ \isacommand{show}\isamarkupfalse%
\ {\isachardoublequoteopen}{\isacharparenleft}{\kern0pt}xx\ {\isasymcirc}\isactrlsub c\ fibered{\isacharunderscore}{\kern0pt}product{\isacharunderscore}{\kern0pt}right{\isacharunderscore}{\kern0pt}proj\ X\ f\ f\ X{\isacharparenright}{\kern0pt}\ {\isasymcirc}\isactrlsub c\ x\ {\isacharequal}{\kern0pt}\ id\isactrlsub c\ {\isacharparenleft}{\kern0pt}X\ \isactrlbsub f\isactrlesub {\isasymtimes}\isactrlsub c\isactrlbsub f\isactrlesub \ X{\isacharparenright}{\kern0pt}\ {\isasymcirc}\isactrlsub c\ x{\isachardoublequoteclose}\isanewline
\ \ \ \ \ \ \isacommand{using}\isamarkupfalse%
\ calculation\ \isacommand{by}\isamarkupfalse%
\ auto\isanewline
\ \ \isacommand{qed}\isamarkupfalse%
\isanewline
\isanewline
\ \ \isacommand{show}\isamarkupfalse%
\ {\isachardoublequoteopen}isomorphism\ {\isacharparenleft}{\kern0pt}fibered{\isacharunderscore}{\kern0pt}product{\isacharunderscore}{\kern0pt}left{\isacharunderscore}{\kern0pt}proj\ X\ f\ f\ X{\isacharparenright}{\kern0pt}{\isachardoublequoteclose}\isanewline
\ \ \ \ \isacommand{unfolding}\isamarkupfalse%
\ isomorphism{\isacharunderscore}{\kern0pt}def\isanewline
\ \ \ \ \isacommand{by}\isamarkupfalse%
\ {\isacharparenleft}{\kern0pt}metis\ assms\ cfunc{\isacharunderscore}{\kern0pt}type{\isacharunderscore}{\kern0pt}def\ eq{\isadigit{1}}\ eq{\isadigit{2}}\ fibered{\isacharunderscore}{\kern0pt}product{\isacharunderscore}{\kern0pt}right{\isacharunderscore}{\kern0pt}proj{\isacharunderscore}{\kern0pt}type\ xx{\isacharunderscore}{\kern0pt}assms{\isacharparenleft}{\kern0pt}{\isadigit{1}}{\isacharparenright}{\kern0pt}{\isacharparenright}{\kern0pt}\isanewline
\isanewline
\ \ \isacommand{then}\isamarkupfalse%
\ \isacommand{show}\isamarkupfalse%
\ {\isachardoublequoteopen}isomorphism\ {\isacharparenleft}{\kern0pt}fibered{\isacharunderscore}{\kern0pt}product{\isacharunderscore}{\kern0pt}right{\isacharunderscore}{\kern0pt}proj\ X\ f\ f\ X{\isacharparenright}{\kern0pt}{\isachardoublequoteclose}\isanewline
\ \ \ \ \isacommand{unfolding}\isamarkupfalse%
\ isomorphism{\isacharunderscore}{\kern0pt}def\isanewline
\ \ \ \ \isacommand{using}\isamarkupfalse%
\ assms{\isacharparenleft}{\kern0pt}{\isadigit{2}}{\isacharparenright}{\kern0pt}\ isomorphism{\isacharunderscore}{\kern0pt}def\ \isacommand{by}\isamarkupfalse%
\ auto\isanewline
\isacommand{qed}\isamarkupfalse%
%
\endisatagproof
{\isafoldproof}%
%
\isadelimproof
\isanewline
%
\endisadelimproof
\isanewline
\isacommand{lemma}\isamarkupfalse%
\ kern{\isacharunderscore}{\kern0pt}pair{\isacharunderscore}{\kern0pt}proj{\isacharunderscore}{\kern0pt}iso{\isacharunderscore}{\kern0pt}TFAE{\isadigit{3}}{\isacharcolon}{\kern0pt}\isanewline
\ \ \isakeyword{assumes}\ {\isachardoublequoteopen}f{\isacharcolon}{\kern0pt}\ X\ {\isasymrightarrow}\ Y{\isachardoublequoteclose}\isanewline
\ \ \isakeyword{assumes}\ {\isachardoublequoteopen}isomorphism\ {\isacharparenleft}{\kern0pt}fibered{\isacharunderscore}{\kern0pt}product{\isacharunderscore}{\kern0pt}left{\isacharunderscore}{\kern0pt}proj\ X\ f\ f\ X{\isacharparenright}{\kern0pt}{\isachardoublequoteclose}\ {\isachardoublequoteopen}isomorphism\ {\isacharparenleft}{\kern0pt}fibered{\isacharunderscore}{\kern0pt}product{\isacharunderscore}{\kern0pt}right{\isacharunderscore}{\kern0pt}proj\ X\ f\ f\ X{\isacharparenright}{\kern0pt}{\isachardoublequoteclose}\isanewline
\ \ \isakeyword{shows}\ {\isachardoublequoteopen}fibered{\isacharunderscore}{\kern0pt}product{\isacharunderscore}{\kern0pt}left{\isacharunderscore}{\kern0pt}proj\ X\ f\ f\ X\ {\isacharequal}{\kern0pt}\ fibered{\isacharunderscore}{\kern0pt}product{\isacharunderscore}{\kern0pt}right{\isacharunderscore}{\kern0pt}proj\ X\ f\ f\ X{\isachardoublequoteclose}\isanewline
%
\isadelimproof
%
\endisadelimproof
%
\isatagproof
\isacommand{proof}\isamarkupfalse%
\ {\isacharminus}{\kern0pt}\isanewline
\ \ \isacommand{obtain}\isamarkupfalse%
\ q{\isadigit{0}}\ \isakeyword{where}\ \isanewline
\ \ \ \ q{\isadigit{0}}{\isacharunderscore}{\kern0pt}assms{\isacharcolon}{\kern0pt}\ {\isachardoublequoteopen}q{\isadigit{0}}\ {\isacharcolon}{\kern0pt}\ X\ {\isasymrightarrow}\ X\ \isactrlbsub f\isactrlesub {\isasymtimes}\isactrlsub c\isactrlbsub f\isactrlesub \ X{\isachardoublequoteclose}\isanewline
\ \ \ \ \ \ {\isachardoublequoteopen}fibered{\isacharunderscore}{\kern0pt}product{\isacharunderscore}{\kern0pt}left{\isacharunderscore}{\kern0pt}proj\ X\ f\ f\ X\ {\isasymcirc}\isactrlsub c\ q{\isadigit{0}}\ {\isacharequal}{\kern0pt}\ id\ X{\isachardoublequoteclose}\isanewline
\ \ \ \ \ \ {\isachardoublequoteopen}q{\isadigit{0}}\ {\isasymcirc}\isactrlsub c\ fibered{\isacharunderscore}{\kern0pt}product{\isacharunderscore}{\kern0pt}left{\isacharunderscore}{\kern0pt}proj\ X\ f\ f\ X\ {\isacharequal}{\kern0pt}\ id\ {\isacharparenleft}{\kern0pt}X\ \isactrlbsub f\isactrlesub {\isasymtimes}\isactrlsub c\isactrlbsub f\isactrlesub \ X{\isacharparenright}{\kern0pt}{\isachardoublequoteclose}\isanewline
\ \ \ \ \isacommand{using}\isamarkupfalse%
\ assms{\isacharparenleft}{\kern0pt}{\isadigit{1}}{\isacharcomma}{\kern0pt}{\isadigit{2}}{\isacharparenright}{\kern0pt}\ cfunc{\isacharunderscore}{\kern0pt}type{\isacharunderscore}{\kern0pt}def\ isomorphism{\isacharunderscore}{\kern0pt}def\ \isacommand{by}\isamarkupfalse%
\ {\isacharparenleft}{\kern0pt}typecheck{\isacharunderscore}{\kern0pt}cfuncs{\isacharcomma}{\kern0pt}\ force{\isacharparenright}{\kern0pt}\isanewline
\isanewline
\ \ \isacommand{obtain}\isamarkupfalse%
\ q{\isadigit{1}}\ \isakeyword{where}\ \isanewline
\ \ \ \ q{\isadigit{1}}{\isacharunderscore}{\kern0pt}assms{\isacharcolon}{\kern0pt}\ {\isachardoublequoteopen}q{\isadigit{1}}\ {\isacharcolon}{\kern0pt}\ X\ {\isasymrightarrow}\ X\ \isactrlbsub f\isactrlesub {\isasymtimes}\isactrlsub c\isactrlbsub f\isactrlesub \ X{\isachardoublequoteclose}\isanewline
\ \ \ \ \ \ {\isachardoublequoteopen}fibered{\isacharunderscore}{\kern0pt}product{\isacharunderscore}{\kern0pt}right{\isacharunderscore}{\kern0pt}proj\ X\ f\ f\ X\ {\isasymcirc}\isactrlsub c\ q{\isadigit{1}}\ {\isacharequal}{\kern0pt}\ id\ X{\isachardoublequoteclose}\isanewline
\ \ \ \ \ \ {\isachardoublequoteopen}q{\isadigit{1}}\ {\isasymcirc}\isactrlsub c\ fibered{\isacharunderscore}{\kern0pt}product{\isacharunderscore}{\kern0pt}right{\isacharunderscore}{\kern0pt}proj\ X\ f\ f\ X\ {\isacharequal}{\kern0pt}\ id\ {\isacharparenleft}{\kern0pt}X\ \isactrlbsub f\isactrlesub {\isasymtimes}\isactrlsub c\isactrlbsub f\isactrlesub \ X{\isacharparenright}{\kern0pt}{\isachardoublequoteclose}\isanewline
\ \ \ \ \isacommand{using}\isamarkupfalse%
\ assms{\isacharparenleft}{\kern0pt}{\isadigit{1}}{\isacharcomma}{\kern0pt}{\isadigit{3}}{\isacharparenright}{\kern0pt}\ cfunc{\isacharunderscore}{\kern0pt}type{\isacharunderscore}{\kern0pt}def\ isomorphism{\isacharunderscore}{\kern0pt}def\ \isacommand{by}\isamarkupfalse%
\ {\isacharparenleft}{\kern0pt}typecheck{\isacharunderscore}{\kern0pt}cfuncs{\isacharcomma}{\kern0pt}\ force{\isacharparenright}{\kern0pt}\isanewline
\isanewline
\ \ \isacommand{have}\isamarkupfalse%
\ {\isachardoublequoteopen}{\isasymAnd}x{\isachardot}{\kern0pt}\ x\ {\isasymin}\isactrlsub c\ domain\ f\ {\isasymLongrightarrow}\ q{\isadigit{0}}\ {\isasymcirc}\isactrlsub c\ x\ {\isacharequal}{\kern0pt}\ q{\isadigit{1}}\ {\isasymcirc}\isactrlsub c\ x{\isachardoublequoteclose}\isanewline
\ \ \isacommand{proof}\isamarkupfalse%
\ {\isacharminus}{\kern0pt}\isanewline
\ \ \ \ \isacommand{fix}\isamarkupfalse%
\ x\ \isanewline
\ \ \ \ \isacommand{have}\isamarkupfalse%
\ fxfx{\isacharcolon}{\kern0pt}\ \ {\isachardoublequoteopen}f{\isasymcirc}\isactrlsub c\ x\ {\isacharequal}{\kern0pt}\ f{\isasymcirc}\isactrlsub c\ x{\isachardoublequoteclose}\isanewline
\ \ \ \ \ \ \ \isacommand{by}\isamarkupfalse%
\ simp\isanewline
\ \ \ \ \isacommand{assume}\isamarkupfalse%
\ x{\isacharunderscore}{\kern0pt}type{\isacharcolon}{\kern0pt}\ {\isachardoublequoteopen}x\ {\isasymin}\isactrlsub c\ domain\ f{\isachardoublequoteclose}\isanewline
\ \ \ \ \isacommand{have}\isamarkupfalse%
\ factorsthru{\isacharcolon}{\kern0pt}\ {\isachardoublequoteopen}{\isasymlangle}x{\isacharcomma}{\kern0pt}x{\isasymrangle}\ factorsthru\ fibered{\isacharunderscore}{\kern0pt}product{\isacharunderscore}{\kern0pt}morphism\ X\ f\ f\ X{\isachardoublequoteclose}\isanewline
\ \ \ \ \ \ \isacommand{using}\isamarkupfalse%
\ assms{\isacharparenleft}{\kern0pt}{\isadigit{1}}{\isacharparenright}{\kern0pt}\ cfunc{\isacharunderscore}{\kern0pt}type{\isacharunderscore}{\kern0pt}def\ fxfx\ pair{\isacharunderscore}{\kern0pt}factorsthru{\isacharunderscore}{\kern0pt}fibered{\isacharunderscore}{\kern0pt}product{\isacharunderscore}{\kern0pt}morphism\ x{\isacharunderscore}{\kern0pt}type\ \ \isacommand{by}\isamarkupfalse%
\ auto\isanewline
\ \ \ \ \isacommand{then}\isamarkupfalse%
\ \isacommand{obtain}\isamarkupfalse%
\ xx\ \isakeyword{where}\ xx{\isacharunderscore}{\kern0pt}assms{\isacharcolon}{\kern0pt}\ {\isachardoublequoteopen}xx\ {\isacharcolon}{\kern0pt}\ {\isasymone}\ {\isasymrightarrow}\ X\ \isactrlbsub f\isactrlesub {\isasymtimes}\isactrlsub c\isactrlbsub f\isactrlesub \ X{\isachardoublequoteclose}\ {\isachardoublequoteopen}{\isasymlangle}x{\isacharcomma}{\kern0pt}x{\isasymrangle}\ {\isacharequal}{\kern0pt}\ fibered{\isacharunderscore}{\kern0pt}product{\isacharunderscore}{\kern0pt}morphism\ X\ f\ f\ X\ {\isasymcirc}\isactrlsub c\ xx{\isachardoublequoteclose}\isanewline
\ \ \ \ \ \ \isacommand{by}\isamarkupfalse%
\ {\isacharparenleft}{\kern0pt}smt\ assms{\isacharparenleft}{\kern0pt}{\isadigit{1}}{\isacharparenright}{\kern0pt}\ cfunc{\isacharunderscore}{\kern0pt}type{\isacharunderscore}{\kern0pt}def\ diag{\isacharunderscore}{\kern0pt}on{\isacharunderscore}{\kern0pt}elements\ diagonal{\isacharunderscore}{\kern0pt}type\ domain{\isacharunderscore}{\kern0pt}comp\ factors{\isacharunderscore}{\kern0pt}through{\isacharunderscore}{\kern0pt}def\ factorsthru\ fibered{\isacharunderscore}{\kern0pt}product{\isacharunderscore}{\kern0pt}morphism{\isacharunderscore}{\kern0pt}type\ x{\isacharunderscore}{\kern0pt}type{\isacharparenright}{\kern0pt}\isanewline
\ \ \ \ \ \ \isanewline
\ \ \ \ \isacommand{have}\isamarkupfalse%
\ projection{\isacharunderscore}{\kern0pt}prop{\isacharcolon}{\kern0pt}\ {\isachardoublequoteopen}q{\isadigit{0}}\ {\isasymcirc}\isactrlsub c\ {\isacharparenleft}{\kern0pt}{\isacharparenleft}{\kern0pt}fibered{\isacharunderscore}{\kern0pt}product{\isacharunderscore}{\kern0pt}left{\isacharunderscore}{\kern0pt}proj\ X\ f\ f\ X{\isacharparenright}{\kern0pt}{\isasymcirc}\isactrlsub c\ xx{\isacharparenright}{\kern0pt}\ {\isacharequal}{\kern0pt}\ \isanewline
\ \ \ \ \ \ \ \ \ \ \ \ \ \ \ \ \ \ \ \ \ \ \ \ \ \ \ \ \ \ \ q{\isadigit{1}}\ {\isasymcirc}\isactrlsub c\ {\isacharparenleft}{\kern0pt}{\isacharparenleft}{\kern0pt}fibered{\isacharunderscore}{\kern0pt}product{\isacharunderscore}{\kern0pt}right{\isacharunderscore}{\kern0pt}proj\ X\ f\ f\ X{\isacharparenright}{\kern0pt}{\isasymcirc}\isactrlsub c\ xx{\isacharparenright}{\kern0pt}{\isachardoublequoteclose}\isanewline
\ \ \ \ \ \ \isacommand{using}\isamarkupfalse%
\ q{\isadigit{0}}{\isacharunderscore}{\kern0pt}assms\ q{\isadigit{1}}{\isacharunderscore}{\kern0pt}assms\ xx{\isacharunderscore}{\kern0pt}assms\ assms\ \isacommand{by}\isamarkupfalse%
\ {\isacharparenleft}{\kern0pt}typecheck{\isacharunderscore}{\kern0pt}cfuncs{\isacharcomma}{\kern0pt}\ simp\ add{\isacharcolon}{\kern0pt}\ comp{\isacharunderscore}{\kern0pt}associative{\isadigit{2}}{\isacharparenright}{\kern0pt}\isanewline
\ \ \ \ \isacommand{then}\isamarkupfalse%
\ \isacommand{have}\isamarkupfalse%
\ fun{\isacharunderscore}{\kern0pt}fact{\isacharcolon}{\kern0pt}\ {\isachardoublequoteopen}x\ {\isacharequal}{\kern0pt}\ {\isacharparenleft}{\kern0pt}{\isacharparenleft}{\kern0pt}fibered{\isacharunderscore}{\kern0pt}product{\isacharunderscore}{\kern0pt}left{\isacharunderscore}{\kern0pt}proj\ X\ f\ f\ X{\isacharparenright}{\kern0pt}\ {\isasymcirc}\isactrlsub c\ q{\isadigit{1}}{\isacharparenright}{\kern0pt}{\isasymcirc}\isactrlsub c\ {\isacharparenleft}{\kern0pt}{\isacharparenleft}{\kern0pt}{\isacharparenleft}{\kern0pt}fibered{\isacharunderscore}{\kern0pt}product{\isacharunderscore}{\kern0pt}left{\isacharunderscore}{\kern0pt}proj\ X\ f\ f\ X{\isacharparenright}{\kern0pt}{\isasymcirc}\isactrlsub c\ xx{\isacharparenright}{\kern0pt}{\isacharparenright}{\kern0pt}{\isachardoublequoteclose}\isanewline
\ \ \ \ \ \ \isacommand{by}\isamarkupfalse%
\ {\isacharparenleft}{\kern0pt}smt\ assms{\isacharparenleft}{\kern0pt}{\isadigit{1}}{\isacharparenright}{\kern0pt}\ cfunc{\isacharunderscore}{\kern0pt}type{\isacharunderscore}{\kern0pt}def\ comp{\isacharunderscore}{\kern0pt}associative{\isadigit{2}}\ fibered{\isacharunderscore}{\kern0pt}product{\isacharunderscore}{\kern0pt}left{\isacharunderscore}{\kern0pt}proj{\isacharunderscore}{\kern0pt}def\isanewline
\ \ \ \ \ \ \ \ \ \ fibered{\isacharunderscore}{\kern0pt}product{\isacharunderscore}{\kern0pt}left{\isacharunderscore}{\kern0pt}proj{\isacharunderscore}{\kern0pt}type\ fibered{\isacharunderscore}{\kern0pt}product{\isacharunderscore}{\kern0pt}morphism{\isacharunderscore}{\kern0pt}type\ fibered{\isacharunderscore}{\kern0pt}product{\isacharunderscore}{\kern0pt}right{\isacharunderscore}{\kern0pt}proj{\isacharunderscore}{\kern0pt}def\isanewline
\ \ \ \ \ \ \ \ \ \ fibered{\isacharunderscore}{\kern0pt}product{\isacharunderscore}{\kern0pt}right{\isacharunderscore}{\kern0pt}proj{\isacharunderscore}{\kern0pt}type\ id{\isacharunderscore}{\kern0pt}left{\isacharunderscore}{\kern0pt}unit{\isadigit{2}}\ left{\isacharunderscore}{\kern0pt}cart{\isacharunderscore}{\kern0pt}proj{\isacharunderscore}{\kern0pt}cfunc{\isacharunderscore}{\kern0pt}prod\ left{\isacharunderscore}{\kern0pt}cart{\isacharunderscore}{\kern0pt}proj{\isacharunderscore}{\kern0pt}type\isanewline
\ \ \ \ \ \ \ \ \ \ q{\isadigit{1}}{\isacharunderscore}{\kern0pt}assms\ right{\isacharunderscore}{\kern0pt}cart{\isacharunderscore}{\kern0pt}proj{\isacharunderscore}{\kern0pt}cfunc{\isacharunderscore}{\kern0pt}prod\ right{\isacharunderscore}{\kern0pt}cart{\isacharunderscore}{\kern0pt}proj{\isacharunderscore}{\kern0pt}type\ x{\isacharunderscore}{\kern0pt}type\ xx{\isacharunderscore}{\kern0pt}assms{\isacharparenright}{\kern0pt}\isanewline
\ \ \ \ \isacommand{then}\isamarkupfalse%
\ \isacommand{have}\isamarkupfalse%
\ {\isachardoublequoteopen}q{\isadigit{1}}\ \ {\isasymcirc}\isactrlsub c\ {\isacharparenleft}{\kern0pt}{\isacharparenleft}{\kern0pt}fibered{\isacharunderscore}{\kern0pt}product{\isacharunderscore}{\kern0pt}left{\isacharunderscore}{\kern0pt}proj\ X\ f\ f\ X{\isacharparenright}{\kern0pt}{\isasymcirc}\isactrlsub c\ xx{\isacharparenright}{\kern0pt}\ {\isacharequal}{\kern0pt}\ \isanewline
\ \ \ \ \ \ \ \ \ \ \ \ \ q{\isadigit{0}}\ \ {\isasymcirc}\isactrlsub c\ {\isacharparenleft}{\kern0pt}{\isacharparenleft}{\kern0pt}fibered{\isacharunderscore}{\kern0pt}product{\isacharunderscore}{\kern0pt}left{\isacharunderscore}{\kern0pt}proj\ X\ f\ f\ X{\isacharparenright}{\kern0pt}{\isasymcirc}\isactrlsub c\ xx{\isacharparenright}{\kern0pt}{\isachardoublequoteclose}\isanewline
\ \ \ \ \ \ \isacommand{using}\isamarkupfalse%
\ q{\isadigit{0}}{\isacharunderscore}{\kern0pt}assms\ q{\isadigit{1}}{\isacharunderscore}{\kern0pt}assms\ xx{\isacharunderscore}{\kern0pt}assms\ assms\ \isanewline
\ \ \ \ \ \ \isacommand{by}\isamarkupfalse%
\ {\isacharparenleft}{\kern0pt}typecheck{\isacharunderscore}{\kern0pt}cfuncs{\isacharcomma}{\kern0pt}\ smt\ cfunc{\isacharunderscore}{\kern0pt}type{\isacharunderscore}{\kern0pt}def\ comp{\isacharunderscore}{\kern0pt}associative{\isadigit{2}}\ fibered{\isacharunderscore}{\kern0pt}product{\isacharunderscore}{\kern0pt}left{\isacharunderscore}{\kern0pt}proj{\isacharunderscore}{\kern0pt}def\isanewline
\ \ \ \ \ \ \ \ \ \ fibered{\isacharunderscore}{\kern0pt}product{\isacharunderscore}{\kern0pt}morphism{\isacharunderscore}{\kern0pt}type\ fibered{\isacharunderscore}{\kern0pt}product{\isacharunderscore}{\kern0pt}right{\isacharunderscore}{\kern0pt}proj{\isacharunderscore}{\kern0pt}def\ left{\isacharunderscore}{\kern0pt}cart{\isacharunderscore}{\kern0pt}proj{\isacharunderscore}{\kern0pt}cfunc{\isacharunderscore}{\kern0pt}prod\isanewline
\ \ \ \ \ \ \ \ \ \ left{\isacharunderscore}{\kern0pt}cart{\isacharunderscore}{\kern0pt}proj{\isacharunderscore}{\kern0pt}type\ projection{\isacharunderscore}{\kern0pt}prop\ right{\isacharunderscore}{\kern0pt}cart{\isacharunderscore}{\kern0pt}proj{\isacharunderscore}{\kern0pt}cfunc{\isacharunderscore}{\kern0pt}prod\ right{\isacharunderscore}{\kern0pt}cart{\isacharunderscore}{\kern0pt}proj{\isacharunderscore}{\kern0pt}type\ x{\isacharunderscore}{\kern0pt}type\ xx{\isacharunderscore}{\kern0pt}assms{\isacharparenleft}{\kern0pt}{\isadigit{2}}{\isacharparenright}{\kern0pt}{\isacharparenright}{\kern0pt}\isanewline
\ \ \ \ \isacommand{then}\isamarkupfalse%
\ \isacommand{show}\isamarkupfalse%
\ {\isachardoublequoteopen}q{\isadigit{0}}\ {\isasymcirc}\isactrlsub c\ x\ {\isacharequal}{\kern0pt}\ q{\isadigit{1}}\ {\isasymcirc}\isactrlsub c\ x{\isachardoublequoteclose}\ \ \ \ \ \ \isanewline
\ \ \ \ \ \ \isacommand{by}\isamarkupfalse%
\ {\isacharparenleft}{\kern0pt}smt\ assms{\isacharparenleft}{\kern0pt}{\isadigit{1}}{\isacharparenright}{\kern0pt}\ cfunc{\isacharunderscore}{\kern0pt}type{\isacharunderscore}{\kern0pt}def\ codomain{\isacharunderscore}{\kern0pt}comp\ comp{\isacharunderscore}{\kern0pt}associative\ fibered{\isacharunderscore}{\kern0pt}product{\isacharunderscore}{\kern0pt}left{\isacharunderscore}{\kern0pt}proj{\isacharunderscore}{\kern0pt}type\isanewline
\ \ \ \ \ \ \ \ \ \ fun{\isacharunderscore}{\kern0pt}fact\ id{\isacharunderscore}{\kern0pt}left{\isacharunderscore}{\kern0pt}unit{\isadigit{2}}\ q{\isadigit{0}}{\isacharunderscore}{\kern0pt}assms\ q{\isadigit{1}}{\isacharunderscore}{\kern0pt}assms\ xx{\isacharunderscore}{\kern0pt}assms{\isacharparenright}{\kern0pt}\isanewline
\ \ \isacommand{qed}\isamarkupfalse%
\isanewline
\ \ \isacommand{then}\isamarkupfalse%
\ \isacommand{have}\isamarkupfalse%
\ {\isachardoublequoteopen}q{\isadigit{0}}\ {\isacharequal}{\kern0pt}\ q{\isadigit{1}}{\isachardoublequoteclose}\isanewline
\ \ \ \ \isacommand{by}\isamarkupfalse%
\ {\isacharparenleft}{\kern0pt}metis\ assms{\isacharparenleft}{\kern0pt}{\isadigit{1}}{\isacharparenright}{\kern0pt}\ cfunc{\isacharunderscore}{\kern0pt}type{\isacharunderscore}{\kern0pt}def\ one{\isacharunderscore}{\kern0pt}separator{\isacharunderscore}{\kern0pt}contrapos\ q{\isadigit{0}}{\isacharunderscore}{\kern0pt}assms{\isacharparenleft}{\kern0pt}{\isadigit{1}}{\isacharparenright}{\kern0pt}\ q{\isadigit{1}}{\isacharunderscore}{\kern0pt}assms{\isacharparenleft}{\kern0pt}{\isadigit{1}}{\isacharparenright}{\kern0pt}{\isacharparenright}{\kern0pt}\isanewline
\ \ \isacommand{then}\isamarkupfalse%
\ \isacommand{show}\isamarkupfalse%
\ {\isachardoublequoteopen}fibered{\isacharunderscore}{\kern0pt}product{\isacharunderscore}{\kern0pt}left{\isacharunderscore}{\kern0pt}proj\ X\ f\ f\ X\ {\isacharequal}{\kern0pt}\ fibered{\isacharunderscore}{\kern0pt}product{\isacharunderscore}{\kern0pt}right{\isacharunderscore}{\kern0pt}proj\ X\ f\ f\ X{\isachardoublequoteclose}\isanewline
\ \ \ \ \isacommand{by}\isamarkupfalse%
\ {\isacharparenleft}{\kern0pt}smt\ assms{\isacharparenleft}{\kern0pt}{\isadigit{1}}{\isacharparenright}{\kern0pt}\ comp{\isacharunderscore}{\kern0pt}associative{\isadigit{2}}\ fibered{\isacharunderscore}{\kern0pt}product{\isacharunderscore}{\kern0pt}left{\isacharunderscore}{\kern0pt}proj{\isacharunderscore}{\kern0pt}type\ fibered{\isacharunderscore}{\kern0pt}product{\isacharunderscore}{\kern0pt}right{\isacharunderscore}{\kern0pt}proj{\isacharunderscore}{\kern0pt}type\isanewline
\ \ \ \ \ \ \ \ id{\isacharunderscore}{\kern0pt}left{\isacharunderscore}{\kern0pt}unit{\isadigit{2}}\ id{\isacharunderscore}{\kern0pt}right{\isacharunderscore}{\kern0pt}unit{\isadigit{2}}\ q{\isadigit{0}}{\isacharunderscore}{\kern0pt}assms\ q{\isadigit{1}}{\isacharunderscore}{\kern0pt}assms{\isacharparenright}{\kern0pt}\isanewline
\isacommand{qed}\isamarkupfalse%
%
\endisatagproof
{\isafoldproof}%
%
\isadelimproof
\isanewline
%
\endisadelimproof
\isanewline
\isacommand{lemma}\isamarkupfalse%
\ terminal{\isacharunderscore}{\kern0pt}fib{\isacharunderscore}{\kern0pt}prod{\isacharunderscore}{\kern0pt}iso{\isacharcolon}{\kern0pt}\isanewline
\ \ \isakeyword{assumes}\ {\isachardoublequoteopen}terminal{\isacharunderscore}{\kern0pt}object{\isacharparenleft}{\kern0pt}T{\isacharparenright}{\kern0pt}{\isachardoublequoteclose}\isanewline
\ \ \isakeyword{assumes}\ f{\isacharunderscore}{\kern0pt}type{\isacharcolon}{\kern0pt}\ {\isachardoublequoteopen}f\ {\isacharcolon}{\kern0pt}\ Y\ {\isasymrightarrow}\ T{\isachardoublequoteclose}\ \isanewline
\ \ \isakeyword{assumes}\ g{\isacharunderscore}{\kern0pt}type{\isacharcolon}{\kern0pt}\ {\isachardoublequoteopen}g\ {\isacharcolon}{\kern0pt}\ X\ {\isasymrightarrow}\ T{\isachardoublequoteclose}\isanewline
\ \ \isakeyword{shows}\ {\isachardoublequoteopen}{\isacharparenleft}{\kern0pt}X\ \isactrlbsub g\isactrlesub {\isasymtimes}\isactrlsub c\isactrlbsub f\isactrlesub \ Y{\isacharparenright}{\kern0pt}\ {\isasymcong}\ X\ {\isasymtimes}\isactrlsub c\ Y{\isachardoublequoteclose}\isanewline
%
\isadelimproof
%
\endisadelimproof
%
\isatagproof
\isacommand{proof}\isamarkupfalse%
\ {\isacharminus}{\kern0pt}\ \isanewline
\ \ \isacommand{have}\isamarkupfalse%
\ {\isachardoublequoteopen}{\isacharparenleft}{\kern0pt}is{\isacharunderscore}{\kern0pt}pullback\ {\isacharparenleft}{\kern0pt}X\ \isactrlbsub g\isactrlesub {\isasymtimes}\isactrlsub c\isactrlbsub f\isactrlesub \ Y{\isacharparenright}{\kern0pt}\ Y\ X\ T\ {\isacharparenleft}{\kern0pt}fibered{\isacharunderscore}{\kern0pt}product{\isacharunderscore}{\kern0pt}right{\isacharunderscore}{\kern0pt}proj\ X\ g\ f\ Y{\isacharparenright}{\kern0pt}\ f\ {\isacharparenleft}{\kern0pt}fibered{\isacharunderscore}{\kern0pt}product{\isacharunderscore}{\kern0pt}left{\isacharunderscore}{\kern0pt}proj\ X\ g\ f\ Y{\isacharparenright}{\kern0pt}\ g{\isacharparenright}{\kern0pt}{\isachardoublequoteclose}\isanewline
\ \ \ \ \isacommand{using}\isamarkupfalse%
\ assms\ pullback{\isacharunderscore}{\kern0pt}iff{\isacharunderscore}{\kern0pt}product\ fibered{\isacharunderscore}{\kern0pt}product{\isacharunderscore}{\kern0pt}is{\isacharunderscore}{\kern0pt}pullback\ \isacommand{by}\isamarkupfalse%
\ {\isacharparenleft}{\kern0pt}typecheck{\isacharunderscore}{\kern0pt}cfuncs{\isacharcomma}{\kern0pt}\ blast{\isacharparenright}{\kern0pt}\isanewline
\ \ \isacommand{then}\isamarkupfalse%
\ \isacommand{have}\isamarkupfalse%
\ {\isachardoublequoteopen}{\isacharparenleft}{\kern0pt}is{\isacharunderscore}{\kern0pt}cart{\isacharunderscore}{\kern0pt}prod\ {\isacharparenleft}{\kern0pt}X\ \isactrlbsub g\isactrlesub {\isasymtimes}\isactrlsub c\isactrlbsub f\isactrlesub \ Y{\isacharparenright}{\kern0pt}\ {\isacharparenleft}{\kern0pt}fibered{\isacharunderscore}{\kern0pt}product{\isacharunderscore}{\kern0pt}left{\isacharunderscore}{\kern0pt}proj\ X\ g\ f\ Y{\isacharparenright}{\kern0pt}\ {\isacharparenleft}{\kern0pt}fibered{\isacharunderscore}{\kern0pt}product{\isacharunderscore}{\kern0pt}right{\isacharunderscore}{\kern0pt}proj\ X\ g\ f\ Y{\isacharparenright}{\kern0pt}\ \ X\ Y{\isacharparenright}{\kern0pt}{\isachardoublequoteclose}\isanewline
\ \ \ \ \isacommand{using}\isamarkupfalse%
\ assms\ \isacommand{by}\isamarkupfalse%
\ {\isacharparenleft}{\kern0pt}meson\ one{\isacharunderscore}{\kern0pt}terminal{\isacharunderscore}{\kern0pt}object\ pullback{\isacharunderscore}{\kern0pt}iff{\isacharunderscore}{\kern0pt}product\ terminal{\isacharunderscore}{\kern0pt}func{\isacharunderscore}{\kern0pt}type{\isacharparenright}{\kern0pt}\isanewline
\ \ \isacommand{then}\isamarkupfalse%
\ \isacommand{show}\isamarkupfalse%
\ {\isacharquery}{\kern0pt}thesis\isanewline
\ \ \ \ \isacommand{using}\isamarkupfalse%
\ assms\ \isacommand{by}\isamarkupfalse%
\ {\isacharparenleft}{\kern0pt}metis\ canonical{\isacharunderscore}{\kern0pt}cart{\isacharunderscore}{\kern0pt}prod{\isacharunderscore}{\kern0pt}is{\isacharunderscore}{\kern0pt}cart{\isacharunderscore}{\kern0pt}prod\ cart{\isacharunderscore}{\kern0pt}prods{\isacharunderscore}{\kern0pt}isomorphic\ fst{\isacharunderscore}{\kern0pt}conv\ is{\isacharunderscore}{\kern0pt}isomorphic{\isacharunderscore}{\kern0pt}def\ snd{\isacharunderscore}{\kern0pt}conv{\isacharparenright}{\kern0pt}\isanewline
\isacommand{qed}\isamarkupfalse%
%
\endisatagproof
{\isafoldproof}%
%
\isadelimproof
\isanewline
%
\endisadelimproof
%
\isadelimtheory
\isanewline
%
\endisadelimtheory
%
\isatagtheory
\isacommand{end}\isamarkupfalse%
%
\endisatagtheory
{\isafoldtheory}%
%
\isadelimtheory
%
\endisadelimtheory
%
\end{isabellebody}%
\endinput
%:%file=~/ETCS/HOL-ETCS/Equalizer.thy%:%
%:%11=1%:%
%:%27=3%:%
%:%28=3%:%
%:%29=4%:%
%:%30=5%:%
%:%44=7%:%
%:%54=9%:%
%:%55=9%:%
%:%56=10%:%
%:%58=12%:%
%:%59=13%:%
%:%60=14%:%
%:%61=14%:%
%:%62=15%:%
%:%63=16%:%
%:%64=17%:%
%:%67=18%:%
%:%71=18%:%
%:%72=18%:%
%:%73=18%:%
%:%74=18%:%
%:%79=18%:%
%:%82=19%:%
%:%83=20%:%
%:%84=20%:%
%:%85=21%:%
%:%86=22%:%
%:%87=23%:%
%:%90=24%:%
%:%94=24%:%
%:%95=24%:%
%:%96=24%:%
%:%101=24%:%
%:%104=25%:%
%:%105=26%:%
%:%106=26%:%
%:%107=27%:%
%:%108=28%:%
%:%109=29%:%
%:%110=30%:%
%:%113=31%:%
%:%117=31%:%
%:%118=31%:%
%:%119=31%:%
%:%128=33%:%
%:%130=34%:%
%:%131=34%:%
%:%132=35%:%
%:%133=36%:%
%:%134=37%:%
%:%135=37%:%
%:%136=38%:%
%:%137=39%:%
%:%144=40%:%
%:%145=40%:%
%:%146=41%:%
%:%147=41%:%
%:%148=42%:%
%:%149=42%:%
%:%150=42%:%
%:%151=43%:%
%:%152=43%:%
%:%153=43%:%
%:%154=44%:%
%:%155=44%:%
%:%156=45%:%
%:%157=45%:%
%:%158=46%:%
%:%159=46%:%
%:%160=47%:%
%:%161=47%:%
%:%162=48%:%
%:%163=48%:%
%:%164=49%:%
%:%165=49%:%
%:%166=50%:%
%:%167=50%:%
%:%168=51%:%
%:%169=51%:%
%:%170=52%:%
%:%171=53%:%
%:%172=53%:%
%:%173=54%:%
%:%174=54%:%
%:%175=54%:%
%:%176=55%:%
%:%177=56%:%
%:%178=56%:%
%:%179=57%:%
%:%180=57%:%
%:%181=58%:%
%:%182=59%:%
%:%183=59%:%
%:%184=60%:%
%:%185=61%:%
%:%186=61%:%
%:%187=61%:%
%:%188=62%:%
%:%189=62%:%
%:%190=63%:%
%:%200=65%:%
%:%202=66%:%
%:%203=66%:%
%:%204=67%:%
%:%205=68%:%
%:%212=69%:%
%:%213=69%:%
%:%214=70%:%
%:%215=70%:%
%:%216=71%:%
%:%217=71%:%
%:%218=71%:%
%:%219=72%:%
%:%220=72%:%
%:%221=73%:%
%:%222=73%:%
%:%223=73%:%
%:%224=74%:%
%:%225=75%:%
%:%226=75%:%
%:%227=76%:%
%:%228=76%:%
%:%229=76%:%
%:%230=76%:%
%:%231=77%:%
%:%232=78%:%
%:%233=78%:%
%:%234=79%:%
%:%235=79%:%
%:%236=80%:%
%:%237=80%:%
%:%238=81%:%
%:%239=81%:%
%:%240=82%:%
%:%241=83%:%
%:%242=83%:%
%:%243=84%:%
%:%244=84%:%
%:%245=84%:%
%:%246=85%:%
%:%247=86%:%
%:%248=86%:%
%:%249=87%:%
%:%250=87%:%
%:%251=87%:%
%:%252=88%:%
%:%253=88%:%
%:%254=88%:%
%:%255=89%:%
%:%256=89%:%
%:%257=89%:%
%:%258=90%:%
%:%259=91%:%
%:%260=91%:%
%:%261=92%:%
%:%262=92%:%
%:%263=93%:%
%:%264=93%:%
%:%265=93%:%
%:%266=94%:%
%:%267=94%:%
%:%268=94%:%
%:%269=95%:%
%:%270=96%:%
%:%271=96%:%
%:%272=97%:%
%:%273=97%:%
%:%274=97%:%
%:%275=98%:%
%:%281=98%:%
%:%284=99%:%
%:%285=100%:%
%:%286=100%:%
%:%287=101%:%
%:%288=102%:%
%:%289=103%:%
%:%290=104%:%
%:%291=105%:%
%:%292=106%:%
%:%293=107%:%
%:%300=108%:%
%:%301=108%:%
%:%302=109%:%
%:%303=109%:%
%:%304=110%:%
%:%305=110%:%
%:%306=110%:%
%:%307=111%:%
%:%308=112%:%
%:%309=112%:%
%:%310=113%:%
%:%311=113%:%
%:%312=113%:%
%:%313=114%:%
%:%314=114%:%
%:%315=115%:%
%:%316=115%:%
%:%317=116%:%
%:%318=116%:%
%:%319=117%:%
%:%320=117%:%
%:%321=118%:%
%:%322=118%:%
%:%323=119%:%
%:%324=119%:%
%:%325=120%:%
%:%326=120%:%
%:%327=120%:%
%:%328=121%:%
%:%329=121%:%
%:%330=121%:%
%:%331=122%:%
%:%332=122%:%
%:%333=123%:%
%:%334=123%:%
%:%335=123%:%
%:%336=124%:%
%:%337=124%:%
%:%338=124%:%
%:%339=125%:%
%:%340=125%:%
%:%341=126%:%
%:%342=126%:%
%:%343=127%:%
%:%344=127%:%
%:%345=128%:%
%:%346=128%:%
%:%347=129%:%
%:%348=129%:%
%:%349=130%:%
%:%350=130%:%
%:%351=131%:%
%:%352=131%:%
%:%353=132%:%
%:%354=132%:%
%:%355=132%:%
%:%356=133%:%
%:%357=133%:%
%:%358=134%:%
%:%359=134%:%
%:%360=135%:%
%:%361=135%:%
%:%362=135%:%
%:%363=136%:%
%:%364=136%:%
%:%365=137%:%
%:%375=139%:%
%:%377=140%:%
%:%378=140%:%
%:%379=141%:%
%:%382=142%:%
%:%386=142%:%
%:%387=142%:%
%:%388=143%:%
%:%389=143%:%
%:%390=144%:%
%:%391=144%:%
%:%392=145%:%
%:%393=145%:%
%:%394=146%:%
%:%395=146%:%
%:%396=147%:%
%:%397=147%:%
%:%398=148%:%
%:%399=148%:%
%:%400=149%:%
%:%401=149%:%
%:%402=150%:%
%:%403=150%:%
%:%404=151%:%
%:%405=151%:%
%:%406=152%:%
%:%407=152%:%
%:%408=153%:%
%:%409=153%:%
%:%410=154%:%
%:%411=154%:%
%:%412=154%:%
%:%413=155%:%
%:%414=155%:%
%:%415=155%:%
%:%416=156%:%
%:%417=157%:%
%:%418=157%:%
%:%419=157%:%
%:%420=158%:%
%:%421=159%:%
%:%422=159%:%
%:%423=159%:%
%:%424=160%:%
%:%425=160%:%
%:%426=161%:%
%:%436=163%:%
%:%438=164%:%
%:%439=164%:%
%:%440=165%:%
%:%443=168%:%
%:%445=169%:%
%:%446=169%:%
%:%447=170%:%
%:%448=171%:%
%:%455=172%:%
%:%456=172%:%
%:%457=173%:%
%:%458=173%:%
%:%459=174%:%
%:%460=175%:%
%:%461=176%:%
%:%462=176%:%
%:%463=176%:%
%:%464=177%:%
%:%465=177%:%
%:%466=177%:%
%:%467=178%:%
%:%468=178%:%
%:%469=178%:%
%:%470=179%:%
%:%471=179%:%
%:%472=179%:%
%:%473=180%:%
%:%474=180%:%
%:%475=180%:%
%:%476=181%:%
%:%477=181%:%
%:%478=181%:%
%:%479=182%:%
%:%480=182%:%
%:%481=183%:%
%:%482=183%:%
%:%483=183%:%
%:%484=184%:%
%:%485=184%:%
%:%486=185%:%
%:%487=185%:%
%:%488=185%:%
%:%489=186%:%
%:%490=186%:%
%:%491=187%:%
%:%492=187%:%
%:%493=187%:%
%:%494=188%:%
%:%495=188%:%
%:%496=189%:%
%:%497=189%:%
%:%498=189%:%
%:%499=190%:%
%:%500=190%:%
%:%501=190%:%
%:%502=191%:%
%:%503=191%:%
%:%504=191%:%
%:%505=192%:%
%:%506=192%:%
%:%507=193%:%
%:%522=195%:%
%:%534=197%:%
%:%536=198%:%
%:%537=198%:%
%:%538=199%:%
%:%539=200%:%
%:%540=201%:%
%:%541=201%:%
%:%542=202%:%
%:%543=203%:%
%:%546=204%:%
%:%550=204%:%
%:%551=204%:%
%:%552=204%:%
%:%553=204%:%
%:%562=206%:%
%:%564=207%:%
%:%565=207%:%
%:%566=208%:%
%:%567=209%:%
%:%574=210%:%
%:%575=210%:%
%:%576=211%:%
%:%577=211%:%
%:%578=212%:%
%:%579=212%:%
%:%580=212%:%
%:%581=213%:%
%:%582=213%:%
%:%583=213%:%
%:%584=214%:%
%:%585=214%:%
%:%586=215%:%
%:%587=215%:%
%:%588=216%:%
%:%589=216%:%
%:%590=216%:%
%:%591=217%:%
%:%592=217%:%
%:%593=218%:%
%:%594=218%:%
%:%595=219%:%
%:%605=221%:%
%:%607=222%:%
%:%608=222%:%
%:%609=223%:%
%:%610=224%:%
%:%611=225%:%
%:%612=225%:%
%:%613=226%:%
%:%616=227%:%
%:%620=227%:%
%:%621=227%:%
%:%626=227%:%
%:%629=228%:%
%:%630=229%:%
%:%631=229%:%
%:%632=230%:%
%:%634=232%:%
%:%635=233%:%
%:%636=234%:%
%:%637=234%:%
%:%638=235%:%
%:%639=236%:%
%:%642=237%:%
%:%646=237%:%
%:%647=237%:%
%:%648=237%:%
%:%653=237%:%
%:%656=238%:%
%:%657=239%:%
%:%658=239%:%
%:%661=240%:%
%:%665=240%:%
%:%666=240%:%
%:%667=241%:%
%:%668=241%:%
%:%669=241%:%
%:%678=243%:%
%:%680=244%:%
%:%681=244%:%
%:%682=245%:%
%:%683=246%:%
%:%684=247%:%
%:%685=247%:%
%:%686=248%:%
%:%689=249%:%
%:%693=249%:%
%:%694=249%:%
%:%695=249%:%
%:%704=251%:%
%:%706=252%:%
%:%707=252%:%
%:%708=253%:%
%:%709=254%:%
%:%712=255%:%
%:%716=255%:%
%:%717=255%:%
%:%718=255%:%
%:%719=256%:%
%:%720=256%:%
%:%721=257%:%
%:%722=257%:%
%:%723=258%:%
%:%724=258%:%
%:%725=259%:%
%:%726=259%:%
%:%727=260%:%
%:%728=260%:%
%:%729=261%:%
%:%730=261%:%
%:%731=262%:%
%:%732=262%:%
%:%733=263%:%
%:%734=263%:%
%:%735=264%:%
%:%736=265%:%
%:%737=265%:%
%:%738=266%:%
%:%739=266%:%
%:%740=266%:%
%:%741=267%:%
%:%742=267%:%
%:%743=267%:%
%:%744=268%:%
%:%745=268%:%
%:%746=269%:%
%:%747=269%:%
%:%748=270%:%
%:%749=270%:%
%:%750=271%:%
%:%765=273%:%
%:%777=275%:%
%:%779=276%:%
%:%780=276%:%
%:%781=277%:%
%:%782=278%:%
%:%783=279%:%
%:%784=280%:%
%:%785=280%:%
%:%786=281%:%
%:%787=282%:%
%:%794=283%:%
%:%795=283%:%
%:%796=284%:%
%:%797=284%:%
%:%798=285%:%
%:%799=285%:%
%:%800=286%:%
%:%801=286%:%
%:%802=286%:%
%:%803=287%:%
%:%804=287%:%
%:%805=287%:%
%:%806=287%:%
%:%807=288%:%
%:%813=288%:%
%:%816=289%:%
%:%817=290%:%
%:%818=290%:%
%:%819=291%:%
%:%820=292%:%
%:%821=293%:%
%:%822=294%:%
%:%823=294%:%
%:%824=295%:%
%:%825=296%:%
%:%832=297%:%
%:%833=297%:%
%:%834=298%:%
%:%835=298%:%
%:%836=299%:%
%:%837=299%:%
%:%838=299%:%
%:%839=300%:%
%:%840=300%:%
%:%841=300%:%
%:%842=301%:%
%:%843=302%:%
%:%844=302%:%
%:%845=302%:%
%:%846=302%:%
%:%847=303%:%
%:%848=303%:%
%:%849=303%:%
%:%850=304%:%
%:%851=304%:%
%:%852=304%:%
%:%853=305%:%
%:%859=305%:%
%:%862=306%:%
%:%863=307%:%
%:%864=307%:%
%:%865=308%:%
%:%866=309%:%
%:%869=310%:%
%:%873=310%:%
%:%874=310%:%
%:%875=310%:%
%:%880=310%:%
%:%883=311%:%
%:%884=312%:%
%:%885=312%:%
%:%886=313%:%
%:%887=314%:%
%:%888=315%:%
%:%891=316%:%
%:%895=316%:%
%:%896=316%:%
%:%897=317%:%
%:%898=317%:%
%:%903=317%:%
%:%906=318%:%
%:%907=319%:%
%:%908=319%:%
%:%909=320%:%
%:%910=321%:%
%:%913=322%:%
%:%917=322%:%
%:%918=322%:%
%:%919=322%:%
%:%928=324%:%
%:%930=325%:%
%:%931=325%:%
%:%932=326%:%
%:%933=327%:%
%:%936=328%:%
%:%940=328%:%
%:%941=328%:%
%:%942=329%:%
%:%943=329%:%
%:%944=330%:%
%:%945=330%:%
%:%946=331%:%
%:%947=331%:%
%:%948=332%:%
%:%949=332%:%
%:%950=333%:%
%:%951=333%:%
%:%952=334%:%
%:%953=335%:%
%:%954=335%:%
%:%955=335%:%
%:%956=336%:%
%:%957=337%:%
%:%958=337%:%
%:%959=338%:%
%:%960=338%:%
%:%961=338%:%
%:%962=339%:%
%:%963=340%:%
%:%964=340%:%
%:%965=341%:%
%:%966=341%:%
%:%967=342%:%
%:%968=342%:%
%:%969=342%:%
%:%970=343%:%
%:%971=344%:%
%:%972=344%:%
%:%973=344%:%
%:%974=345%:%
%:%975=345%:%
%:%976=345%:%
%:%977=346%:%
%:%978=346%:%
%:%979=347%:%
%:%980=347%:%
%:%981=348%:%
%:%982=348%:%
%:%983=348%:%
%:%984=349%:%
%:%985=349%:%
%:%986=350%:%
%:%987=350%:%
%:%988=351%:%
%:%994=351%:%
%:%997=352%:%
%:%998=353%:%
%:%999=353%:%
%:%1000=354%:%
%:%1001=355%:%
%:%1002=356%:%
%:%1003=356%:%
%:%1004=357%:%
%:%1005=358%:%
%:%1008=359%:%
%:%1012=359%:%
%:%1013=359%:%
%:%1014=359%:%
%:%1015=359%:%
%:%1020=359%:%
%:%1023=360%:%
%:%1024=361%:%
%:%1025=361%:%
%:%1026=362%:%
%:%1027=363%:%
%:%1030=364%:%
%:%1034=364%:%
%:%1035=364%:%
%:%1036=364%:%
%:%1041=364%:%
%:%1044=365%:%
%:%1045=366%:%
%:%1046=366%:%
%:%1047=367%:%
%:%1048=368%:%
%:%1051=369%:%
%:%1055=369%:%
%:%1056=369%:%
%:%1057=369%:%
%:%1062=369%:%
%:%1065=370%:%
%:%1066=371%:%
%:%1067=371%:%
%:%1068=372%:%
%:%1069=373%:%
%:%1072=374%:%
%:%1076=374%:%
%:%1077=374%:%
%:%1078=375%:%
%:%1079=375%:%
%:%1080=376%:%
%:%1081=376%:%
%:%1086=376%:%
%:%1089=377%:%
%:%1090=378%:%
%:%1091=378%:%
%:%1092=379%:%
%:%1093=380%:%
%:%1095=382%:%
%:%1098=383%:%
%:%1102=383%:%
%:%1103=383%:%
%:%1104=383%:%
%:%1105=384%:%
%:%1106=384%:%
%:%1107=385%:%
%:%1108=385%:%
%:%1109=386%:%
%:%1110=386%:%
%:%1111=387%:%
%:%1112=387%:%
%:%1113=388%:%
%:%1114=388%:%
%:%1115=389%:%
%:%1116=389%:%
%:%1117=389%:%
%:%1118=390%:%
%:%1119=391%:%
%:%1120=391%:%
%:%1121=392%:%
%:%1122=393%:%
%:%1123=393%:%
%:%1124=393%:%
%:%1125=394%:%
%:%1126=394%:%
%:%1127=395%:%
%:%1128=395%:%
%:%1129=396%:%
%:%1130=396%:%
%:%1131=397%:%
%:%1132=397%:%
%:%1133=398%:%
%:%1134=399%:%
%:%1135=399%:%
%:%1136=400%:%
%:%1137=400%:%
%:%1138=400%:%
%:%1139=401%:%
%:%1140=401%:%
%:%1141=401%:%
%:%1143=403%:%
%:%1144=404%:%
%:%1145=404%:%
%:%1146=404%:%
%:%1147=404%:%
%:%1148=405%:%
%:%1149=405%:%
%:%1150=405%:%
%:%1152=407%:%
%:%1153=408%:%
%:%1154=408%:%
%:%1155=409%:%
%:%1156=409%:%
%:%1157=409%:%
%:%1158=410%:%
%:%1159=410%:%
%:%1160=411%:%
%:%1161=411%:%
%:%1162=412%:%
%:%1163=413%:%
%:%1164=413%:%
%:%1165=413%:%
%:%1167=415%:%
%:%1168=416%:%
%:%1169=416%:%
%:%1170=417%:%
%:%1171=417%:%
%:%1172=418%:%
%:%1173=418%:%
%:%1174=419%:%
%:%1175=419%:%
%:%1176=420%:%
%:%1177=421%:%
%:%1178=421%:%
%:%1180=423%:%
%:%1181=424%:%
%:%1182=424%:%
%:%1183=425%:%
%:%1184=426%:%
%:%1185=426%:%
%:%1186=427%:%
%:%1187=427%:%
%:%1188=428%:%
%:%1189=428%:%
%:%1190=429%:%
%:%1191=430%:%
%:%1192=430%:%
%:%1193=431%:%
%:%1194=431%:%
%:%1195=432%:%
%:%1196=432%:%
%:%1197=433%:%
%:%1198=433%:%
%:%1199=434%:%
%:%1200=434%:%
%:%1201=435%:%
%:%1202=435%:%
%:%1203=436%:%
%:%1204=436%:%
%:%1205=437%:%
%:%1206=437%:%
%:%1207=438%:%
%:%1208=438%:%
%:%1209=439%:%
%:%1210=439%:%
%:%1211=440%:%
%:%1212=441%:%
%:%1213=441%:%
%:%1214=441%:%
%:%1215=442%:%
%:%1216=442%:%
%:%1217=443%:%
%:%1218=443%:%
%:%1219=444%:%
%:%1229=446%:%
%:%1231=447%:%
%:%1232=447%:%
%:%1233=448%:%
%:%1234=449%:%
%:%1241=450%:%
%:%1242=450%:%
%:%1243=451%:%
%:%1244=451%:%
%:%1245=452%:%
%:%1246=452%:%
%:%1247=452%:%
%:%1248=452%:%
%:%1249=453%:%
%:%1250=454%:%
%:%1251=454%:%
%:%1252=455%:%
%:%1253=455%:%
%:%1254=455%:%
%:%1255=456%:%
%:%1256=457%:%
%:%1257=457%:%
%:%1258=457%:%
%:%1259=458%:%
%:%1260=458%:%
%:%1261=458%:%
%:%1262=459%:%
%:%1263=459%:%
%:%1264=459%:%
%:%1265=460%:%
%:%1266=460%:%
%:%1267=460%:%
%:%1268=461%:%
%:%1269=461%:%
%:%1270=461%:%
%:%1271=462%:%
%:%1272=462%:%
%:%1273=462%:%
%:%1274=463%:%
%:%1275=463%:%
%:%1276=463%:%
%:%1277=464%:%
%:%1278=464%:%
%:%1279=464%:%
%:%1280=465%:%
%:%1281=465%:%
%:%1282=465%:%
%:%1283=466%:%
%:%1284=466%:%
%:%1285=466%:%
%:%1286=467%:%
%:%1287=467%:%
%:%1288=467%:%
%:%1289=468%:%
%:%1290=468%:%
%:%1291=469%:%
%:%1292=470%:%
%:%1293=470%:%
%:%1294=470%:%
%:%1295=471%:%
%:%1296=471%:%
%:%1297=471%:%
%:%1298=471%:%
%:%1299=472%:%
%:%1300=472%:%
%:%1301=473%:%
%:%1302=473%:%
%:%1303=474%:%
%:%1304=474%:%
%:%1305=474%:%
%:%1306=474%:%
%:%1307=475%:%
%:%1308=476%:%
%:%1309=476%:%
%:%1310=477%:%
%:%1311=477%:%
%:%1312=477%:%
%:%1313=478%:%
%:%1314=478%:%
%:%1315=478%:%
%:%1316=479%:%
%:%1317=479%:%
%:%1318=479%:%
%:%1319=480%:%
%:%1320=480%:%
%:%1321=481%:%
%:%1322=481%:%
%:%1323=481%:%
%:%1324=482%:%
%:%1325=482%:%
%:%1326=482%:%
%:%1328=484%:%
%:%1329=485%:%
%:%1330=485%:%
%:%1331=485%:%
%:%1332=485%:%
%:%1333=486%:%
%:%1334=486%:%
%:%1335=486%:%
%:%1336=487%:%
%:%1337=487%:%
%:%1338=487%:%
%:%1339=488%:%
%:%1340=488%:%
%:%1341=488%:%
%:%1342=489%:%
%:%1343=489%:%
%:%1344=489%:%
%:%1345=490%:%
%:%1346=490%:%
%:%1347=491%:%
%:%1362=493%:%
%:%1374=495%:%
%:%1376=496%:%
%:%1377=496%:%
%:%1378=497%:%
%:%1379=498%:%
%:%1380=499%:%
%:%1381=500%:%
%:%1382=500%:%
%:%1383=501%:%
%:%1384=502%:%
%:%1391=503%:%
%:%1392=503%:%
%:%1393=504%:%
%:%1394=504%:%
%:%1395=505%:%
%:%1396=505%:%
%:%1397=505%:%
%:%1398=506%:%
%:%1399=506%:%
%:%1400=506%:%
%:%1401=507%:%
%:%1402=508%:%
%:%1403=508%:%
%:%1404=508%:%
%:%1405=509%:%
%:%1406=509%:%
%:%1407=509%:%
%:%1408=510%:%
%:%1409=511%:%
%:%1410=511%:%
%:%1411=511%:%
%:%1412=512%:%
%:%1413=512%:%
%:%1414=512%:%
%:%1415=513%:%
%:%1416=513%:%
%:%1417=514%:%
%:%1423=514%:%
%:%1426=515%:%
%:%1427=516%:%
%:%1428=516%:%
%:%1429=517%:%
%:%1430=518%:%
%:%1431=519%:%
%:%1432=520%:%
%:%1433=520%:%
%:%1434=521%:%
%:%1435=522%:%
%:%1442=523%:%
%:%1443=523%:%
%:%1444=524%:%
%:%1445=524%:%
%:%1446=525%:%
%:%1447=526%:%
%:%1448=526%:%
%:%1449=526%:%
%:%1450=527%:%
%:%1451=527%:%
%:%1452=527%:%
%:%1453=528%:%
%:%1454=529%:%
%:%1455=529%:%
%:%1456=529%:%
%:%1457=530%:%
%:%1458=530%:%
%:%1459=530%:%
%:%1460=531%:%
%:%1461=531%:%
%:%1462=532%:%
%:%1468=532%:%
%:%1471=533%:%
%:%1472=534%:%
%:%1473=534%:%
%:%1474=535%:%
%:%1475=536%:%
%:%1478=537%:%
%:%1482=537%:%
%:%1483=537%:%
%:%1484=537%:%
%:%1489=537%:%
%:%1492=538%:%
%:%1493=539%:%
%:%1494=539%:%
%:%1495=540%:%
%:%1496=541%:%
%:%1499=542%:%
%:%1503=542%:%
%:%1504=542%:%
%:%1505=542%:%
%:%1510=542%:%
%:%1513=543%:%
%:%1514=544%:%
%:%1515=544%:%
%:%1516=545%:%
%:%1517=546%:%
%:%1518=547%:%
%:%1519=547%:%
%:%1520=548%:%
%:%1521=549%:%
%:%1524=550%:%
%:%1528=550%:%
%:%1529=550%:%
%:%1534=550%:%
%:%1537=551%:%
%:%1538=552%:%
%:%1539=552%:%
%:%1540=553%:%
%:%1541=554%:%
%:%1542=555%:%
%:%1543=555%:%
%:%1544=556%:%
%:%1545=557%:%
%:%1548=558%:%
%:%1552=558%:%
%:%1553=558%:%
%:%1558=558%:%
%:%1561=559%:%
%:%1562=560%:%
%:%1563=560%:%
%:%1564=561%:%
%:%1565=562%:%
%:%1568=563%:%
%:%1572=563%:%
%:%1573=563%:%
%:%1574=564%:%
%:%1575=564%:%
%:%1576=565%:%
%:%1577=565%:%
%:%1578=566%:%
%:%1579=566%:%
%:%1580=566%:%
%:%1581=567%:%
%:%1582=568%:%
%:%1583=568%:%
%:%1584=569%:%
%:%1585=569%:%
%:%1586=569%:%
%:%1587=570%:%
%:%1588=570%:%
%:%1589=570%:%
%:%1590=571%:%
%:%1591=571%:%
%:%1592=571%:%
%:%1593=572%:%
%:%1594=572%:%
%:%1595=572%:%
%:%1596=573%:%
%:%1597=573%:%
%:%1598=573%:%
%:%1599=574%:%
%:%1600=575%:%
%:%1601=575%:%
%:%1602=576%:%
%:%1608=576%:%
%:%1611=577%:%
%:%1612=578%:%
%:%1613=578%:%
%:%1614=579%:%
%:%1615=580%:%
%:%1618=581%:%
%:%1622=581%:%
%:%1623=581%:%
%:%1624=582%:%
%:%1625=582%:%
%:%1626=583%:%
%:%1627=583%:%
%:%1628=584%:%
%:%1629=584%:%
%:%1630=585%:%
%:%1631=585%:%
%:%1632=586%:%
%:%1633=586%:%
%:%1634=587%:%
%:%1635=587%:%
%:%1636=588%:%
%:%1637=588%:%
%:%1638=589%:%
%:%1639=589%:%
%:%1640=590%:%
%:%1641=590%:%
%:%1642=591%:%
%:%1643=591%:%
%:%1644=592%:%
%:%1645=593%:%
%:%1646=593%:%
%:%1647=594%:%
%:%1648=594%:%
%:%1649=594%:%
%:%1650=595%:%
%:%1651=595%:%
%:%1652=595%:%
%:%1653=596%:%
%:%1654=596%:%
%:%1655=597%:%
%:%1656=597%:%
%:%1657=597%:%
%:%1658=598%:%
%:%1659=599%:%
%:%1660=599%:%
%:%1661=600%:%
%:%1662=600%:%
%:%1663=601%:%
%:%1664=601%:%
%:%1665=602%:%
%:%1666=602%:%
%:%1667=603%:%
%:%1668=604%:%
%:%1669=604%:%
%:%1670=605%:%
%:%1671=606%:%
%:%1672=606%:%
%:%1673=607%:%
%:%1674=607%:%
%:%1675=608%:%
%:%1676=609%:%
%:%1677=609%:%
%:%1678=610%:%
%:%1679=611%:%
%:%1680=611%:%
%:%1681=612%:%
%:%1682=612%:%
%:%1683=613%:%
%:%1684=613%:%
%:%1685=614%:%
%:%1686=614%:%
%:%1687=615%:%
%:%1688=615%:%
%:%1689=616%:%
%:%1690=616%:%
%:%1691=617%:%
%:%1692=617%:%
%:%1693=618%:%
%:%1694=618%:%
%:%1695=618%:%
%:%1696=619%:%
%:%1697=620%:%
%:%1698=620%:%
%:%1699=620%:%
%:%1700=621%:%
%:%1701=621%:%
%:%1702=622%:%
%:%1703=622%:%
%:%1704=623%:%
%:%1705=623%:%
%:%1706=623%:%
%:%1707=624%:%
%:%1708=625%:%
%:%1709=625%:%
%:%1710=625%:%
%:%1711=626%:%
%:%1712=626%:%
%:%1713=627%:%
%:%1714=628%:%
%:%1715=628%:%
%:%1716=629%:%
%:%1717=629%:%
%:%1718=629%:%
%:%1719=630%:%
%:%1720=631%:%
%:%1721=631%:%
%:%1722=632%:%
%:%1723=632%:%
%:%1724=633%:%
%:%1725=633%:%
%:%1726=634%:%
%:%1727=634%:%
%:%1728=634%:%
%:%1729=635%:%
%:%1730=635%:%
%:%1731=636%:%
%:%1732=636%:%
%:%1733=637%:%
%:%1734=637%:%
%:%1735=638%:%
%:%1741=638%:%
%:%1744=639%:%
%:%1745=640%:%
%:%1746=640%:%
%:%1747=641%:%
%:%1748=642%:%
%:%1751=643%:%
%:%1755=643%:%
%:%1756=643%:%
%:%1757=644%:%
%:%1758=644%:%
%:%1759=644%:%
%:%1764=644%:%
%:%1767=645%:%
%:%1768=646%:%
%:%1769=646%:%
%:%1770=647%:%
%:%1771=648%:%
%:%1778=649%:%
%:%1779=649%:%
%:%1780=650%:%
%:%1781=650%:%
%:%1782=651%:%
%:%1783=651%:%
%:%1784=651%:%
%:%1785=652%:%
%:%1786=653%:%
%:%1787=653%:%
%:%1788=654%:%
%:%1789=654%:%
%:%1790=654%:%
%:%1791=655%:%
%:%1792=656%:%
%:%1793=656%:%
%:%1794=657%:%
%:%1795=657%:%
%:%1796=658%:%
%:%1797=658%:%
%:%1798=659%:%
%:%1799=659%:%
%:%1800=660%:%
%:%1801=661%:%
%:%1802=661%:%
%:%1803=662%:%
%:%1804=662%:%
%:%1805=663%:%
%:%1806=663%:%
%:%1807=664%:%
%:%1808=664%:%
%:%1809=665%:%
%:%1810=666%:%
%:%1811=666%:%
%:%1812=667%:%
%:%1813=667%:%
%:%1814=667%:%
%:%1815=668%:%
%:%1816=668%:%
%:%1817=668%:%
%:%1818=669%:%
%:%1819=669%:%
%:%1820=669%:%
%:%1821=670%:%
%:%1822=670%:%
%:%1823=671%:%
%:%1824=671%:%
%:%1825=672%:%
%:%1826=672%:%
%:%1827=673%:%
%:%1828=673%:%
%:%1829=674%:%
%:%1830=674%:%
%:%1831=675%:%
%:%1832=675%:%
%:%1833=676%:%
%:%1834=676%:%
%:%1835=676%:%
%:%1836=677%:%
%:%1837=677%:%
%:%1838=678%:%
%:%1839=678%:%
%:%1840=678%:%
%:%1841=679%:%
%:%1842=679%:%
%:%1843=680%:%
%:%1844=680%:%
%:%1845=680%:%
%:%1846=681%:%
%:%1847=681%:%
%:%1850=684%:%
%:%1851=685%:%
%:%1852=685%:%
%:%1853=685%:%
%:%1854=685%:%
%:%1855=686%:%
%:%1856=687%:%
%:%1857=687%:%
%:%1858=688%:%
%:%1859=689%:%
%:%1860=689%:%
%:%1861=689%:%
%:%1862=690%:%
%:%1863=690%:%
%:%1864=691%:%
%:%1865=692%:%
%:%1866=692%:%
%:%1867=693%:%
%:%1868=693%:%
%:%1869=694%:%
%:%1870=694%:%
%:%1871=694%:%
%:%1872=695%:%
%:%1873=696%:%
%:%1874=696%:%
%:%1875=697%:%
%:%1876=697%:%
%:%1877=698%:%
%:%1878=698%:%
%:%1879=698%:%
%:%1880=699%:%
%:%1881=700%:%
%:%1882=700%:%
%:%1883=701%:%
%:%1884=701%:%
%:%1885=702%:%
%:%1886=702%:%
%:%1887=702%:%
%:%1888=703%:%
%:%1889=704%:%
%:%1890=704%:%
%:%1891=705%:%
%:%1892=705%:%
%:%1893=705%:%
%:%1894=706%:%
%:%1895=706%:%
%:%1896=707%:%
%:%1897=707%:%
%:%1898=708%:%
%:%1904=708%:%
%:%1907=709%:%
%:%1908=710%:%
%:%1909=710%:%
%:%1910=711%:%
%:%1911=712%:%
%:%1912=713%:%
%:%1919=714%:%
%:%1920=714%:%
%:%1921=715%:%
%:%1922=715%:%
%:%1923=716%:%
%:%1924=716%:%
%:%1925=717%:%
%:%1926=717%:%
%:%1927=718%:%
%:%1928=718%:%
%:%1929=719%:%
%:%1930=719%:%
%:%1931=719%:%
%:%1932=720%:%
%:%1933=721%:%
%:%1934=721%:%
%:%1935=722%:%
%:%1936=723%:%
%:%1937=723%:%
%:%1938=724%:%
%:%1939=724%:%
%:%1940=725%:%
%:%1941=725%:%
%:%1942=726%:%
%:%1943=727%:%
%:%1944=727%:%
%:%1945=728%:%
%:%1946=728%:%
%:%1947=729%:%
%:%1948=729%:%
%:%1949=730%:%
%:%1950=731%:%
%:%1951=731%:%
%:%1952=731%:%
%:%1953=732%:%
%:%1954=732%:%
%:%1955=732%:%
%:%1956=733%:%
%:%1962=733%:%
%:%1965=734%:%
%:%1966=735%:%
%:%1967=735%:%
%:%1968=736%:%
%:%1969=737%:%
%:%1972=738%:%
%:%1976=738%:%
%:%1977=738%:%
%:%1978=738%:%
%:%1987=740%:%
%:%1989=741%:%
%:%1990=741%:%
%:%1991=742%:%
%:%1992=743%:%
%:%1999=744%:%
%:%2000=744%:%
%:%2001=745%:%
%:%2002=745%:%
%:%2003=746%:%
%:%2004=746%:%
%:%2005=747%:%
%:%2006=747%:%
%:%2007=747%:%
%:%2008=748%:%
%:%2009=748%:%
%:%2010=749%:%
%:%2011=749%:%
%:%2012=750%:%
%:%2013=750%:%
%:%2014=751%:%
%:%2015=751%:%
%:%2016=751%:%
%:%2017=752%:%
%:%2018=752%:%
%:%2019=752%:%
%:%2020=753%:%
%:%2021=753%:%
%:%2022=753%:%
%:%2023=754%:%
%:%2024=754%:%
%:%2025=754%:%
%:%2026=755%:%
%:%2027=755%:%
%:%2028=756%:%
%:%2029=756%:%
%:%2030=757%:%
%:%2031=757%:%
%:%2032=757%:%
%:%2033=758%:%
%:%2034=758%:%
%:%2035=758%:%
%:%2036=759%:%
%:%2042=759%:%
%:%2045=760%:%
%:%2046=761%:%
%:%2047=761%:%
%:%2048=762%:%
%:%2049=763%:%
%:%2052=764%:%
%:%2056=764%:%
%:%2057=764%:%
%:%2058=765%:%
%:%2059=765%:%
%:%2060=766%:%
%:%2061=766%:%
%:%2062=767%:%
%:%2063=767%:%
%:%2064=768%:%
%:%2065=768%:%
%:%2066=769%:%
%:%2067=769%:%
%:%2068=770%:%
%:%2069=770%:%
%:%2070=771%:%
%:%2071=771%:%
%:%2072=771%:%
%:%2073=772%:%
%:%2074=772%:%
%:%2075=772%:%
%:%2076=773%:%
%:%2077=774%:%
%:%2078=774%:%
%:%2079=775%:%
%:%2080=775%:%
%:%2081=775%:%
%:%2082=776%:%
%:%2083=776%:%
%:%2084=777%:%
%:%2085=777%:%
%:%2086=777%:%
%:%2087=778%:%
%:%2088=779%:%
%:%2089=779%:%
%:%2090=780%:%
%:%2091=780%:%
%:%2092=780%:%
%:%2093=781%:%
%:%2094=781%:%
%:%2095=781%:%
%:%2096=782%:%
%:%2097=782%:%
%:%2098=782%:%
%:%2099=783%:%
%:%2100=783%:%
%:%2101=783%:%
%:%2102=784%:%
%:%2103=785%:%
%:%2104=785%:%
%:%2105=786%:%
%:%2106=786%:%
%:%2107=786%:%
%:%2108=787%:%
%:%2109=787%:%
%:%2110=788%:%
%:%2111=788%:%
%:%2112=789%:%
%:%2113=789%:%
%:%2114=789%:%
%:%2115=790%:%
%:%2116=790%:%
%:%2117=791%:%
%:%2118=792%:%
%:%2119=792%:%
%:%2120=793%:%
%:%2121=793%:%
%:%2122=793%:%
%:%2123=794%:%
%:%2124=794%:%
%:%2125=795%:%
%:%2126=795%:%
%:%2127=795%:%
%:%2128=796%:%
%:%2129=796%:%
%:%2130=796%:%
%:%2131=797%:%
%:%2132=797%:%
%:%2133=798%:%
%:%2134=798%:%
%:%2135=799%:%
%:%2136=799%:%
%:%2137=799%:%
%:%2138=800%:%
%:%2139=800%:%
%:%2140=800%:%
%:%2141=801%:%
%:%2142=801%:%
%:%2143=801%:%
%:%2144=802%:%
%:%2145=802%:%
%:%2146=803%:%
%:%2147=803%:%
%:%2148=804%:%
%:%2149=805%:%
%:%2150=805%:%
%:%2151=806%:%
%:%2152=806%:%
%:%2153=807%:%
%:%2154=807%:%
%:%2155=808%:%
%:%2156=808%:%
%:%2157=808%:%
%:%2158=809%:%
%:%2159=809%:%
%:%2160=810%:%
%:%2161=810%:%
%:%2162=811%:%
%:%2163=811%:%
%:%2164=812%:%
%:%2165=812%:%
%:%2166=813%:%
%:%2167=813%:%
%:%2168=814%:%
%:%2169=814%:%
%:%2170=814%:%
%:%2171=815%:%
%:%2172=815%:%
%:%2173=816%:%
%:%2174=817%:%
%:%2175=818%:%
%:%2176=818%:%
%:%2177=819%:%
%:%2178=819%:%
%:%2179=820%:%
%:%2180=820%:%
%:%2181=821%:%
%:%2182=821%:%
%:%2183=822%:%
%:%2184=822%:%
%:%2185=822%:%
%:%2186=823%:%
%:%2187=823%:%
%:%2188=823%:%
%:%2189=824%:%
%:%2190=824%:%
%:%2191=824%:%
%:%2192=825%:%
%:%2193=825%:%
%:%2194=826%:%
%:%2195=826%:%
%:%2196=827%:%
%:%2197=827%:%
%:%2198=828%:%
%:%2199=828%:%
%:%2200=829%:%
%:%2201=829%:%
%:%2202=830%:%
%:%2203=830%:%
%:%2204=830%:%
%:%2205=831%:%
%:%2206=831%:%
%:%2207=831%:%
%:%2208=832%:%
%:%2209=832%:%
%:%2210=833%:%
%:%2211=833%:%
%:%2212=834%:%
%:%2213=834%:%
%:%2214=834%:%
%:%2215=835%:%
%:%2216=835%:%
%:%2217=836%:%
%:%2218=836%:%
%:%2219=836%:%
%:%2220=837%:%
%:%2221=837%:%
%:%2222=838%:%
%:%2223=838%:%
%:%2224=838%:%
%:%2225=839%:%
%:%2226=839%:%
%:%2227=839%:%
%:%2228=840%:%
%:%2229=840%:%
%:%2230=840%:%
%:%2231=841%:%
%:%2232=841%:%
%:%2233=842%:%
%:%2234=842%:%
%:%2235=843%:%
%:%2236=843%:%
%:%2237=843%:%
%:%2238=844%:%
%:%2239=844%:%
%:%2240=844%:%
%:%2241=845%:%
%:%2242=845%:%
%:%2243=845%:%
%:%2244=846%:%
%:%2245=846%:%
%:%2246=846%:%
%:%2247=847%:%
%:%2248=847%:%
%:%2249=848%:%
%:%2250=848%:%
%:%2251=848%:%
%:%2252=849%:%
%:%2253=849%:%
%:%2254=849%:%
%:%2255=850%:%
%:%2256=850%:%
%:%2257=851%:%
%:%2258=852%:%
%:%2259=852%:%
%:%2260=853%:%
%:%2261=853%:%
%:%2262=854%:%
%:%2263=854%:%
%:%2264=854%:%
%:%2265=855%:%
%:%2266=855%:%
%:%2267=856%:%
%:%2268=856%:%
%:%2269=856%:%
%:%2270=857%:%
%:%2271=857%:%
%:%2272=857%:%
%:%2273=858%:%
%:%2274=858%:%
%:%2275=859%:%
%:%2276=860%:%
%:%2277=860%:%
%:%2278=861%:%
%:%2279=861%:%
%:%2280=862%:%
%:%2281=862%:%
%:%2282=863%:%
%:%2283=864%:%
%:%2284=864%:%
%:%2285=864%:%
%:%2286=865%:%
%:%2287=865%:%
%:%2288=866%:%
%:%2289=866%:%
%:%2290=866%:%
%:%2291=867%:%
%:%2297=867%:%
%:%2300=868%:%
%:%2301=869%:%
%:%2302=869%:%
%:%2303=870%:%
%:%2304=871%:%
%:%2305=872%:%
%:%2312=873%:%
%:%2313=873%:%
%:%2314=874%:%
%:%2315=874%:%
%:%2316=875%:%
%:%2317=876%:%
%:%2318=877%:%
%:%2319=878%:%
%:%2320=878%:%
%:%2321=878%:%
%:%2322=879%:%
%:%2323=880%:%
%:%2324=880%:%
%:%2325=881%:%
%:%2326=882%:%
%:%2327=883%:%
%:%2328=884%:%
%:%2329=884%:%
%:%2330=884%:%
%:%2331=885%:%
%:%2332=886%:%
%:%2333=886%:%
%:%2334=887%:%
%:%2335=887%:%
%:%2336=888%:%
%:%2337=888%:%
%:%2338=889%:%
%:%2339=889%:%
%:%2340=890%:%
%:%2341=890%:%
%:%2342=891%:%
%:%2343=891%:%
%:%2344=892%:%
%:%2345=892%:%
%:%2346=893%:%
%:%2347=893%:%
%:%2348=893%:%
%:%2349=894%:%
%:%2350=894%:%
%:%2351=894%:%
%:%2352=895%:%
%:%2353=895%:%
%:%2354=896%:%
%:%2355=897%:%
%:%2356=897%:%
%:%2357=898%:%
%:%2358=899%:%
%:%2359=899%:%
%:%2360=899%:%
%:%2361=900%:%
%:%2362=900%:%
%:%2363=900%:%
%:%2364=901%:%
%:%2365=901%:%
%:%2366=902%:%
%:%2367=903%:%
%:%2368=904%:%
%:%2369=905%:%
%:%2370=905%:%
%:%2371=905%:%
%:%2372=906%:%
%:%2373=907%:%
%:%2374=907%:%
%:%2375=908%:%
%:%2376=908%:%
%:%2377=909%:%
%:%2378=910%:%
%:%2379=911%:%
%:%2380=911%:%
%:%2381=911%:%
%:%2382=912%:%
%:%2383=912%:%
%:%2384=913%:%
%:%2385=914%:%
%:%2386=914%:%
%:%2387=915%:%
%:%2388=915%:%
%:%2389=915%:%
%:%2390=916%:%
%:%2391=916%:%
%:%2392=917%:%
%:%2393=917%:%
%:%2394=917%:%
%:%2395=918%:%
%:%2396=918%:%
%:%2397=919%:%
%:%2398=920%:%
%:%2404=920%:%
%:%2407=921%:%
%:%2408=922%:%
%:%2409=922%:%
%:%2410=923%:%
%:%2411=924%:%
%:%2412=925%:%
%:%2413=926%:%
%:%2420=927%:%
%:%2421=927%:%
%:%2422=928%:%
%:%2423=928%:%
%:%2424=929%:%
%:%2425=929%:%
%:%2426=929%:%
%:%2427=930%:%
%:%2428=930%:%
%:%2429=930%:%
%:%2430=931%:%
%:%2431=931%:%
%:%2432=931%:%
%:%2433=932%:%
%:%2434=932%:%
%:%2435=932%:%
%:%2436=933%:%
%:%2437=933%:%
%:%2438=933%:%
%:%2439=934%:%
%:%2445=934%:%
%:%2450=935%:%
%:%2455=936%:%

%
\begin{isabellebody}%
\setisabellecontext{Truth}%
%
\isadelimdocument
%
\endisadelimdocument
%
\isatagdocument
%
\isamarkupsection{Truth Values and Characteristic Functions%
}
\isamarkuptrue%
%
\endisatagdocument
{\isafolddocument}%
%
\isadelimdocument
%
\endisadelimdocument
%
\isadelimtheory
%
\endisadelimtheory
%
\isatagtheory
\isacommand{theory}\isamarkupfalse%
\ Truth\isanewline
\ \ \isakeyword{imports}\ Equalizer\isanewline
\isakeyword{begin}%
\endisatagtheory
{\isafoldtheory}%
%
\isadelimtheory
%
\endisadelimtheory
%
\begin{isamarkuptext}%
The axiomatization below corresponds to Axiom 5 (Truth-Value Object) in Halvorson.%
\end{isamarkuptext}\isamarkuptrue%
\isacommand{axiomatization}\isamarkupfalse%
\isanewline
\ \ true{\isacharunderscore}{\kern0pt}func\ {\isacharcolon}{\kern0pt}{\isacharcolon}{\kern0pt}\ {\isachardoublequoteopen}cfunc{\isachardoublequoteclose}\ {\isacharparenleft}{\kern0pt}{\isachardoublequoteopen}{\isasymt}{\isachardoublequoteclose}{\isacharparenright}{\kern0pt}\ \isakeyword{and}\isanewline
\ \ false{\isacharunderscore}{\kern0pt}func\ \ {\isacharcolon}{\kern0pt}{\isacharcolon}{\kern0pt}\ {\isachardoublequoteopen}cfunc{\isachardoublequoteclose}\ {\isacharparenleft}{\kern0pt}{\isachardoublequoteopen}{\isasymf}{\isachardoublequoteclose}{\isacharparenright}{\kern0pt}\ \isakeyword{and}\isanewline
\ \ truth{\isacharunderscore}{\kern0pt}value{\isacharunderscore}{\kern0pt}set\ {\isacharcolon}{\kern0pt}{\isacharcolon}{\kern0pt}\ {\isachardoublequoteopen}cset{\isachardoublequoteclose}\ {\isacharparenleft}{\kern0pt}{\isachardoublequoteopen}{\isasymOmega}{\isachardoublequoteclose}{\isacharparenright}{\kern0pt}\isanewline
\isakeyword{where}\isanewline
\ \ true{\isacharunderscore}{\kern0pt}func{\isacharunderscore}{\kern0pt}type{\isacharbrackleft}{\kern0pt}type{\isacharunderscore}{\kern0pt}rule{\isacharbrackright}{\kern0pt}{\isacharcolon}{\kern0pt}\ {\isachardoublequoteopen}{\isasymt}\ {\isasymin}\isactrlsub c\ {\isasymOmega}{\isachardoublequoteclose}\ \isakeyword{and}\isanewline
\ \ false{\isacharunderscore}{\kern0pt}func{\isacharunderscore}{\kern0pt}type{\isacharbrackleft}{\kern0pt}type{\isacharunderscore}{\kern0pt}rule{\isacharbrackright}{\kern0pt}{\isacharcolon}{\kern0pt}\ {\isachardoublequoteopen}{\isasymf}\ {\isasymin}\isactrlsub c\ {\isasymOmega}{\isachardoublequoteclose}\ \isakeyword{and}\isanewline
\ \ true{\isacharunderscore}{\kern0pt}false{\isacharunderscore}{\kern0pt}distinct{\isacharcolon}{\kern0pt}\ {\isachardoublequoteopen}{\isasymt}\ {\isasymnoteq}\ {\isasymf}{\isachardoublequoteclose}\ \isakeyword{and}\isanewline
\ \ true{\isacharunderscore}{\kern0pt}false{\isacharunderscore}{\kern0pt}only{\isacharunderscore}{\kern0pt}truth{\isacharunderscore}{\kern0pt}values{\isacharcolon}{\kern0pt}\ {\isachardoublequoteopen}x\ {\isasymin}\isactrlsub c\ {\isasymOmega}\ {\isasymLongrightarrow}\ x\ {\isacharequal}{\kern0pt}\ {\isasymf}\ {\isasymor}\ x\ {\isacharequal}{\kern0pt}\ {\isasymt}{\isachardoublequoteclose}\ \isakeyword{and}\isanewline
\ \ characteristic{\isacharunderscore}{\kern0pt}function{\isacharunderscore}{\kern0pt}exists{\isacharcolon}{\kern0pt}\isanewline
\ \ \ \ {\isachardoublequoteopen}m\ {\isacharcolon}{\kern0pt}\ B\ {\isasymrightarrow}\ X\ {\isasymLongrightarrow}\ monomorphism\ m\ {\isasymLongrightarrow}\ {\isasymexists}{\isacharbang}{\kern0pt}\ {\isasymchi}{\isachardot}{\kern0pt}\ is{\isacharunderscore}{\kern0pt}pullback\ B\ {\isasymone}\ X\ {\isasymOmega}\ {\isacharparenleft}{\kern0pt}{\isasymbeta}\isactrlbsub B\isactrlesub {\isacharparenright}{\kern0pt}\ {\isasymt}\ m\ {\isasymchi}{\isachardoublequoteclose}\isanewline
\isanewline
\isacommand{definition}\isamarkupfalse%
\ characteristic{\isacharunderscore}{\kern0pt}func\ {\isacharcolon}{\kern0pt}{\isacharcolon}{\kern0pt}\ {\isachardoublequoteopen}cfunc\ {\isasymRightarrow}\ cfunc{\isachardoublequoteclose}\ \isakeyword{where}\isanewline
\ \ {\isachardoublequoteopen}characteristic{\isacharunderscore}{\kern0pt}func\ m\ {\isacharequal}{\kern0pt}\isanewline
\ \ \ \ {\isacharparenleft}{\kern0pt}THE\ {\isasymchi}{\isachardot}{\kern0pt}\ monomorphism\ m\ {\isasymlongrightarrow}\ is{\isacharunderscore}{\kern0pt}pullback\ {\isacharparenleft}{\kern0pt}domain\ m{\isacharparenright}{\kern0pt}\ {\isasymone}\ {\isacharparenleft}{\kern0pt}codomain\ m{\isacharparenright}{\kern0pt}\ {\isasymOmega}\ {\isacharparenleft}{\kern0pt}{\isasymbeta}\isactrlbsub domain\ m\isactrlesub {\isacharparenright}{\kern0pt}\ {\isasymt}\ m\ {\isasymchi}{\isacharparenright}{\kern0pt}{\isachardoublequoteclose}\isanewline
\isanewline
\isacommand{lemma}\isamarkupfalse%
\ characteristic{\isacharunderscore}{\kern0pt}func{\isacharunderscore}{\kern0pt}is{\isacharunderscore}{\kern0pt}pullback{\isacharcolon}{\kern0pt}\isanewline
\ \ \isakeyword{assumes}\ {\isachardoublequoteopen}m\ {\isacharcolon}{\kern0pt}\ B\ {\isasymrightarrow}\ X{\isachardoublequoteclose}\ {\isachardoublequoteopen}monomorphism\ m{\isachardoublequoteclose}\isanewline
\ \ \isakeyword{shows}\ {\isachardoublequoteopen}is{\isacharunderscore}{\kern0pt}pullback\ B\ {\isasymone}\ X\ {\isasymOmega}\ {\isacharparenleft}{\kern0pt}{\isasymbeta}\isactrlbsub B\isactrlesub {\isacharparenright}{\kern0pt}\ {\isasymt}\ m\ {\isacharparenleft}{\kern0pt}characteristic{\isacharunderscore}{\kern0pt}func\ m{\isacharparenright}{\kern0pt}{\isachardoublequoteclose}\isanewline
%
\isadelimproof
%
\endisadelimproof
%
\isatagproof
\isacommand{proof}\isamarkupfalse%
\ {\isacharminus}{\kern0pt}\isanewline
\ \ \isacommand{obtain}\isamarkupfalse%
\ {\isasymchi}\ \isakeyword{where}\ chi{\isacharunderscore}{\kern0pt}is{\isacharunderscore}{\kern0pt}pullback{\isacharcolon}{\kern0pt}\ {\isachardoublequoteopen}is{\isacharunderscore}{\kern0pt}pullback\ B\ {\isasymone}\ X\ {\isasymOmega}\ {\isacharparenleft}{\kern0pt}{\isasymbeta}\isactrlbsub B\isactrlesub {\isacharparenright}{\kern0pt}\ {\isasymt}\ m\ {\isasymchi}{\isachardoublequoteclose}\isanewline
\ \ \ \ \isacommand{using}\isamarkupfalse%
\ assms\ characteristic{\isacharunderscore}{\kern0pt}function{\isacharunderscore}{\kern0pt}exists\ \isacommand{by}\isamarkupfalse%
\ blast\isanewline
\isanewline
\ \ \isacommand{have}\isamarkupfalse%
\ {\isachardoublequoteopen}monomorphism\ m\ {\isasymlongrightarrow}\ is{\isacharunderscore}{\kern0pt}pullback\ {\isacharparenleft}{\kern0pt}domain\ m{\isacharparenright}{\kern0pt}\ {\isasymone}\ {\isacharparenleft}{\kern0pt}codomain\ m{\isacharparenright}{\kern0pt}\ {\isasymOmega}\ {\isacharparenleft}{\kern0pt}{\isasymbeta}\isactrlbsub domain\ m\isactrlesub {\isacharparenright}{\kern0pt}\ {\isasymt}\ m\ {\isacharparenleft}{\kern0pt}characteristic{\isacharunderscore}{\kern0pt}func\ m{\isacharparenright}{\kern0pt}{\isachardoublequoteclose}\isanewline
\ \ \isacommand{proof}\isamarkupfalse%
\ {\isacharparenleft}{\kern0pt}unfold\ characteristic{\isacharunderscore}{\kern0pt}func{\isacharunderscore}{\kern0pt}def{\isacharcomma}{\kern0pt}\ rule\ theI{\isacharprime}{\kern0pt}{\isacharcomma}{\kern0pt}\ rule{\isacharunderscore}{\kern0pt}tac\ a{\isacharequal}{\kern0pt}{\isasymchi}\ \isakeyword{in}\ ex{\isadigit{1}}I{\isacharcomma}{\kern0pt}\ clarify{\isacharparenright}{\kern0pt}\isanewline
\ \ \ \ \isacommand{show}\isamarkupfalse%
\ {\isachardoublequoteopen}is{\isacharunderscore}{\kern0pt}pullback\ {\isacharparenleft}{\kern0pt}domain\ m{\isacharparenright}{\kern0pt}\ {\isasymone}\ {\isacharparenleft}{\kern0pt}codomain\ m{\isacharparenright}{\kern0pt}\ {\isasymOmega}\ {\isacharparenleft}{\kern0pt}{\isasymbeta}\isactrlbsub domain\ m\isactrlesub {\isacharparenright}{\kern0pt}\ {\isasymt}\ m\ {\isasymchi}{\isachardoublequoteclose}\isanewline
\ \ \ \ \ \ \isacommand{using}\isamarkupfalse%
\ assms{\isacharparenleft}{\kern0pt}{\isadigit{1}}{\isacharparenright}{\kern0pt}\ cfunc{\isacharunderscore}{\kern0pt}type{\isacharunderscore}{\kern0pt}def\ chi{\isacharunderscore}{\kern0pt}is{\isacharunderscore}{\kern0pt}pullback\ \isacommand{by}\isamarkupfalse%
\ auto\isanewline
\ \ \ \ \isacommand{show}\isamarkupfalse%
\ {\isachardoublequoteopen}{\isasymAnd}x{\isachardot}{\kern0pt}\ monomorphism\ m\ {\isasymlongrightarrow}\ is{\isacharunderscore}{\kern0pt}pullback\ {\isacharparenleft}{\kern0pt}domain\ m{\isacharparenright}{\kern0pt}\ {\isasymone}\ {\isacharparenleft}{\kern0pt}codomain\ m{\isacharparenright}{\kern0pt}\ {\isasymOmega}\ {\isacharparenleft}{\kern0pt}{\isasymbeta}\isactrlbsub domain\ m\isactrlesub {\isacharparenright}{\kern0pt}\ {\isasymt}\ m\ x\ {\isasymLongrightarrow}\ x\ {\isacharequal}{\kern0pt}\ {\isasymchi}{\isachardoublequoteclose}\isanewline
\ \ \ \ \ \ \isacommand{using}\isamarkupfalse%
\ assms\ cfunc{\isacharunderscore}{\kern0pt}type{\isacharunderscore}{\kern0pt}def\ characteristic{\isacharunderscore}{\kern0pt}function{\isacharunderscore}{\kern0pt}exists\ chi{\isacharunderscore}{\kern0pt}is{\isacharunderscore}{\kern0pt}pullback\ \isacommand{by}\isamarkupfalse%
\ fastforce\isanewline
\ \ \isacommand{qed}\isamarkupfalse%
\isanewline
\ \ \isacommand{then}\isamarkupfalse%
\ \isacommand{show}\isamarkupfalse%
\ {\isachardoublequoteopen}is{\isacharunderscore}{\kern0pt}pullback\ B\ {\isasymone}\ X\ {\isasymOmega}\ {\isacharparenleft}{\kern0pt}{\isasymbeta}\isactrlbsub B\isactrlesub {\isacharparenright}{\kern0pt}\ {\isasymt}\ m\ {\isacharparenleft}{\kern0pt}characteristic{\isacharunderscore}{\kern0pt}func\ m{\isacharparenright}{\kern0pt}{\isachardoublequoteclose}\isanewline
\ \ \ \ \isacommand{using}\isamarkupfalse%
\ assms\ cfunc{\isacharunderscore}{\kern0pt}type{\isacharunderscore}{\kern0pt}def\ \isacommand{by}\isamarkupfalse%
\ auto\isanewline
\isacommand{qed}\isamarkupfalse%
%
\endisatagproof
{\isafoldproof}%
%
\isadelimproof
\isanewline
%
\endisadelimproof
\isanewline
\isacommand{lemma}\isamarkupfalse%
\ characteristic{\isacharunderscore}{\kern0pt}func{\isacharunderscore}{\kern0pt}type{\isacharbrackleft}{\kern0pt}type{\isacharunderscore}{\kern0pt}rule{\isacharbrackright}{\kern0pt}{\isacharcolon}{\kern0pt}\isanewline
\ \ \isakeyword{assumes}\ {\isachardoublequoteopen}m\ {\isacharcolon}{\kern0pt}\ B\ {\isasymrightarrow}\ X{\isachardoublequoteclose}\ {\isachardoublequoteopen}monomorphism\ m{\isachardoublequoteclose}\isanewline
\ \ \isakeyword{shows}\ {\isachardoublequoteopen}characteristic{\isacharunderscore}{\kern0pt}func\ m\ {\isacharcolon}{\kern0pt}\ X\ {\isasymrightarrow}\ {\isasymOmega}{\isachardoublequoteclose}\isanewline
%
\isadelimproof
%
\endisadelimproof
%
\isatagproof
\isacommand{proof}\isamarkupfalse%
\ {\isacharminus}{\kern0pt}\isanewline
\ \ \isacommand{have}\isamarkupfalse%
\ {\isachardoublequoteopen}is{\isacharunderscore}{\kern0pt}pullback\ B\ {\isasymone}\ X\ {\isasymOmega}\ {\isacharparenleft}{\kern0pt}{\isasymbeta}\isactrlbsub B\isactrlesub {\isacharparenright}{\kern0pt}\ {\isasymt}\ m\ {\isacharparenleft}{\kern0pt}characteristic{\isacharunderscore}{\kern0pt}func\ m{\isacharparenright}{\kern0pt}{\isachardoublequoteclose}\isanewline
\ \ \ \ \isacommand{using}\isamarkupfalse%
\ assms\ \isacommand{by}\isamarkupfalse%
\ {\isacharparenleft}{\kern0pt}rule\ characteristic{\isacharunderscore}{\kern0pt}func{\isacharunderscore}{\kern0pt}is{\isacharunderscore}{\kern0pt}pullback{\isacharparenright}{\kern0pt}\isanewline
\ \ \isacommand{then}\isamarkupfalse%
\ \isacommand{show}\isamarkupfalse%
\ {\isachardoublequoteopen}characteristic{\isacharunderscore}{\kern0pt}func\ m\ {\isacharcolon}{\kern0pt}\ X\ {\isasymrightarrow}\ {\isasymOmega}{\isachardoublequoteclose}\isanewline
\ \ \ \ \isacommand{unfolding}\isamarkupfalse%
\ is{\isacharunderscore}{\kern0pt}pullback{\isacharunderscore}{\kern0pt}def\ \isacommand{by}\isamarkupfalse%
\ auto\isanewline
\isacommand{qed}\isamarkupfalse%
%
\endisatagproof
{\isafoldproof}%
%
\isadelimproof
\isanewline
%
\endisadelimproof
\isanewline
\isacommand{lemma}\isamarkupfalse%
\ characteristic{\isacharunderscore}{\kern0pt}func{\isacharunderscore}{\kern0pt}eq{\isacharcolon}{\kern0pt}\isanewline
\ \ \isakeyword{assumes}\ {\isachardoublequoteopen}m\ {\isacharcolon}{\kern0pt}\ B\ {\isasymrightarrow}\ X{\isachardoublequoteclose}\ {\isachardoublequoteopen}monomorphism\ m{\isachardoublequoteclose}\isanewline
\ \ \isakeyword{shows}\ {\isachardoublequoteopen}characteristic{\isacharunderscore}{\kern0pt}func\ m\ {\isasymcirc}\isactrlsub c\ m\ {\isacharequal}{\kern0pt}\ {\isasymt}\ {\isasymcirc}\isactrlsub c\ {\isasymbeta}\isactrlbsub B\isactrlesub {\isachardoublequoteclose}\isanewline
%
\isadelimproof
\ \ %
\endisadelimproof
%
\isatagproof
\isacommand{using}\isamarkupfalse%
\ assms\ characteristic{\isacharunderscore}{\kern0pt}func{\isacharunderscore}{\kern0pt}is{\isacharunderscore}{\kern0pt}pullback\ \isacommand{unfolding}\isamarkupfalse%
\ is{\isacharunderscore}{\kern0pt}pullback{\isacharunderscore}{\kern0pt}def\ \isacommand{by}\isamarkupfalse%
\ auto%
\endisatagproof
{\isafoldproof}%
%
\isadelimproof
\isanewline
%
\endisadelimproof
\isanewline
\isacommand{lemma}\isamarkupfalse%
\ monomorphism{\isacharunderscore}{\kern0pt}equalizes{\isacharunderscore}{\kern0pt}char{\isacharunderscore}{\kern0pt}func{\isacharcolon}{\kern0pt}\isanewline
\ \ \isakeyword{assumes}\ m{\isacharunderscore}{\kern0pt}type{\isacharbrackleft}{\kern0pt}type{\isacharunderscore}{\kern0pt}rule{\isacharbrackright}{\kern0pt}{\isacharcolon}{\kern0pt}\ {\isachardoublequoteopen}m\ {\isacharcolon}{\kern0pt}\ B\ {\isasymrightarrow}\ X{\isachardoublequoteclose}\ \isakeyword{and}\ m{\isacharunderscore}{\kern0pt}mono{\isacharbrackleft}{\kern0pt}type{\isacharunderscore}{\kern0pt}rule{\isacharbrackright}{\kern0pt}{\isacharcolon}{\kern0pt}\ {\isachardoublequoteopen}monomorphism\ m{\isachardoublequoteclose}\isanewline
\ \ \isakeyword{shows}\ {\isachardoublequoteopen}equalizer\ B\ m\ {\isacharparenleft}{\kern0pt}characteristic{\isacharunderscore}{\kern0pt}func\ m{\isacharparenright}{\kern0pt}\ {\isacharparenleft}{\kern0pt}{\isasymt}\ {\isasymcirc}\isactrlsub c\ {\isasymbeta}\isactrlbsub X\isactrlesub {\isacharparenright}{\kern0pt}{\isachardoublequoteclose}\isanewline
%
\isadelimproof
\ \ %
\endisadelimproof
%
\isatagproof
\isacommand{unfolding}\isamarkupfalse%
\ equalizer{\isacharunderscore}{\kern0pt}def\isanewline
\isacommand{proof}\isamarkupfalse%
\ {\isacharparenleft}{\kern0pt}typecheck{\isacharunderscore}{\kern0pt}cfuncs{\isacharcomma}{\kern0pt}\ rule{\isacharunderscore}{\kern0pt}tac\ x{\isacharequal}{\kern0pt}{\isachardoublequoteopen}X{\isachardoublequoteclose}\ \isakeyword{in}\ exI{\isacharcomma}{\kern0pt}\ rule{\isacharunderscore}{\kern0pt}tac\ x{\isacharequal}{\kern0pt}{\isachardoublequoteopen}{\isasymOmega}{\isachardoublequoteclose}\ \isakeyword{in}\ exI{\isacharcomma}{\kern0pt}\ safe{\isacharparenright}{\kern0pt}\isanewline
\ \ \isacommand{have}\isamarkupfalse%
\ comm{\isacharcolon}{\kern0pt}\ {\isachardoublequoteopen}{\isasymt}\ {\isasymcirc}\isactrlsub c\ {\isasymbeta}\isactrlbsub B\isactrlesub \ {\isacharequal}{\kern0pt}\ characteristic{\isacharunderscore}{\kern0pt}func\ m\ {\isasymcirc}\isactrlsub c\ m{\isachardoublequoteclose}\isanewline
\ \ \ \ \isacommand{using}\isamarkupfalse%
\ characteristic{\isacharunderscore}{\kern0pt}func{\isacharunderscore}{\kern0pt}eq\ m{\isacharunderscore}{\kern0pt}mono\ m{\isacharunderscore}{\kern0pt}type\ \isacommand{by}\isamarkupfalse%
\ auto\isanewline
\ \ \isacommand{then}\isamarkupfalse%
\ \isacommand{have}\isamarkupfalse%
\ {\isachardoublequoteopen}{\isasymbeta}\isactrlbsub B\isactrlesub \ {\isacharequal}{\kern0pt}\ {\isasymbeta}\isactrlbsub X\isactrlesub \ {\isasymcirc}\isactrlsub c\ m{\isachardoublequoteclose}\isanewline
\ \ \ \ \isacommand{using}\isamarkupfalse%
\ m{\isacharunderscore}{\kern0pt}type\ terminal{\isacharunderscore}{\kern0pt}func{\isacharunderscore}{\kern0pt}comp\ \isacommand{by}\isamarkupfalse%
\ auto\isanewline
\ \ \isacommand{then}\isamarkupfalse%
\ \isacommand{show}\isamarkupfalse%
\ {\isachardoublequoteopen}characteristic{\isacharunderscore}{\kern0pt}func\ m\ {\isasymcirc}\isactrlsub c\ m\ {\isacharequal}{\kern0pt}\ {\isacharparenleft}{\kern0pt}{\isasymt}\ {\isasymcirc}\isactrlsub c\ {\isasymbeta}\isactrlbsub X\isactrlesub {\isacharparenright}{\kern0pt}\ {\isasymcirc}\isactrlsub c\ m{\isachardoublequoteclose}\isanewline
\ \ \ \ \isacommand{using}\isamarkupfalse%
\ comm\ comp{\isacharunderscore}{\kern0pt}associative{\isadigit{2}}\ \isacommand{by}\isamarkupfalse%
\ {\isacharparenleft}{\kern0pt}typecheck{\isacharunderscore}{\kern0pt}cfuncs{\isacharcomma}{\kern0pt}\ auto{\isacharparenright}{\kern0pt}\isanewline
\isacommand{next}\isamarkupfalse%
\isanewline
\ \ \isacommand{show}\isamarkupfalse%
\ {\isachardoublequoteopen}{\isasymAnd}h\ F{\isachardot}{\kern0pt}\ h\ {\isacharcolon}{\kern0pt}\ F\ {\isasymrightarrow}\ X\ {\isasymLongrightarrow}\ characteristic{\isacharunderscore}{\kern0pt}func\ m\ {\isasymcirc}\isactrlsub c\ h\ {\isacharequal}{\kern0pt}\ {\isacharparenleft}{\kern0pt}{\isasymt}\ {\isasymcirc}\isactrlsub c\ {\isasymbeta}\isactrlbsub X\isactrlesub {\isacharparenright}{\kern0pt}\ {\isasymcirc}\isactrlsub c\ h\ {\isasymLongrightarrow}\ {\isasymexists}k{\isachardot}{\kern0pt}\ k\ {\isacharcolon}{\kern0pt}\ F\ {\isasymrightarrow}\ B\ {\isasymand}\ m\ {\isasymcirc}\isactrlsub c\ k\ {\isacharequal}{\kern0pt}\ h{\isachardoublequoteclose}\isanewline
\ \ \ \ \isacommand{by}\isamarkupfalse%
\ {\isacharparenleft}{\kern0pt}typecheck{\isacharunderscore}{\kern0pt}cfuncs{\isacharcomma}{\kern0pt}\ smt\ {\isacharparenleft}{\kern0pt}verit{\isacharcomma}{\kern0pt}\ ccfv{\isacharunderscore}{\kern0pt}threshold{\isacharparenright}{\kern0pt}\ cfunc{\isacharunderscore}{\kern0pt}type{\isacharunderscore}{\kern0pt}def\ characteristic{\isacharunderscore}{\kern0pt}func{\isacharunderscore}{\kern0pt}is{\isacharunderscore}{\kern0pt}pullback\ comp{\isacharunderscore}{\kern0pt}associative\ comp{\isacharunderscore}{\kern0pt}type\ is{\isacharunderscore}{\kern0pt}pullback{\isacharunderscore}{\kern0pt}def\ m{\isacharunderscore}{\kern0pt}mono{\isacharparenright}{\kern0pt}\isanewline
\isacommand{next}\isamarkupfalse%
\isanewline
\ \ \isacommand{show}\isamarkupfalse%
\ {\isachardoublequoteopen}{\isasymAnd}F\ k\ y{\isachardot}{\kern0pt}\ \ characteristic{\isacharunderscore}{\kern0pt}func\ m\ {\isasymcirc}\isactrlsub c\ m\ {\isasymcirc}\isactrlsub c\ k\ {\isacharequal}{\kern0pt}\ {\isacharparenleft}{\kern0pt}{\isasymt}\ {\isasymcirc}\isactrlsub c\ {\isasymbeta}\isactrlbsub X\isactrlesub {\isacharparenright}{\kern0pt}\ {\isasymcirc}\isactrlsub c\ m\ {\isasymcirc}\isactrlsub c\ k\ {\isasymLongrightarrow}\ k\ {\isacharcolon}{\kern0pt}\ F\ {\isasymrightarrow}\ B\ {\isasymLongrightarrow}\ y\ {\isacharcolon}{\kern0pt}\ F\ {\isasymrightarrow}\ B\ {\isasymLongrightarrow}\ m\ {\isasymcirc}\isactrlsub c\ y\ {\isacharequal}{\kern0pt}\ m\ {\isasymcirc}\isactrlsub c\ k\ {\isasymLongrightarrow}\ k\ {\isacharequal}{\kern0pt}\ y{\isachardoublequoteclose}\isanewline
\ \ \ \ \ \ \isacommand{by}\isamarkupfalse%
\ {\isacharparenleft}{\kern0pt}typecheck{\isacharunderscore}{\kern0pt}cfuncs{\isacharcomma}{\kern0pt}\ smt\ m{\isacharunderscore}{\kern0pt}mono\ monomorphism{\isacharunderscore}{\kern0pt}def{\isadigit{2}}{\isacharparenright}{\kern0pt}\isanewline
\isacommand{qed}\isamarkupfalse%
%
\endisatagproof
{\isafoldproof}%
%
\isadelimproof
\isanewline
%
\endisadelimproof
\isanewline
\isacommand{lemma}\isamarkupfalse%
\ characteristic{\isacharunderscore}{\kern0pt}func{\isacharunderscore}{\kern0pt}true{\isacharunderscore}{\kern0pt}relative{\isacharunderscore}{\kern0pt}member{\isacharcolon}{\kern0pt}\isanewline
\ \ \isakeyword{assumes}\ {\isachardoublequoteopen}m\ {\isacharcolon}{\kern0pt}\ B\ {\isasymrightarrow}\ X{\isachardoublequoteclose}\ {\isachardoublequoteopen}monomorphism\ m{\isachardoublequoteclose}\ {\isachardoublequoteopen}x\ {\isasymin}\isactrlsub c\ X{\isachardoublequoteclose}\isanewline
\ \ \isakeyword{assumes}\ characteristic{\isacharunderscore}{\kern0pt}func{\isacharunderscore}{\kern0pt}true{\isacharcolon}{\kern0pt}\ {\isachardoublequoteopen}characteristic{\isacharunderscore}{\kern0pt}func\ m\ {\isasymcirc}\isactrlsub c\ x\ {\isacharequal}{\kern0pt}\ {\isasymt}{\isachardoublequoteclose}\isanewline
\ \ \isakeyword{shows}\ {\isachardoublequoteopen}x\ {\isasymin}\isactrlbsub X\isactrlesub \ {\isacharparenleft}{\kern0pt}B{\isacharcomma}{\kern0pt}m{\isacharparenright}{\kern0pt}{\isachardoublequoteclose}\isanewline
%
\isadelimproof
%
\endisadelimproof
%
\isatagproof
\isacommand{proof}\isamarkupfalse%
\ {\isacharparenleft}{\kern0pt}insert\ assms{\isacharcomma}{\kern0pt}\ unfold\ relative{\isacharunderscore}{\kern0pt}member{\isacharunderscore}{\kern0pt}def{\isadigit{2}}\ factors{\isacharunderscore}{\kern0pt}through{\isacharunderscore}{\kern0pt}def{\isacharcomma}{\kern0pt}\ clarify{\isacharparenright}{\kern0pt}\isanewline
\ \ \isacommand{have}\isamarkupfalse%
\ {\isachardoublequoteopen}is{\isacharunderscore}{\kern0pt}pullback\ B\ {\isasymone}\ X\ {\isasymOmega}\ {\isacharparenleft}{\kern0pt}{\isasymbeta}\isactrlbsub B\isactrlesub {\isacharparenright}{\kern0pt}\ {\isasymt}\ m\ {\isacharparenleft}{\kern0pt}characteristic{\isacharunderscore}{\kern0pt}func\ m{\isacharparenright}{\kern0pt}{\isachardoublequoteclose}\isanewline
\ \ \ \ \isacommand{by}\isamarkupfalse%
\ {\isacharparenleft}{\kern0pt}simp\ add{\isacharcolon}{\kern0pt}\ assms\ characteristic{\isacharunderscore}{\kern0pt}func{\isacharunderscore}{\kern0pt}is{\isacharunderscore}{\kern0pt}pullback{\isacharparenright}{\kern0pt}\isanewline
\ \ \isacommand{then}\isamarkupfalse%
\ \isacommand{have}\isamarkupfalse%
\ {\isachardoublequoteopen}{\isasymexists}j{\isachardot}{\kern0pt}\ j\ {\isacharcolon}{\kern0pt}\ {\isasymone}\ {\isasymrightarrow}\ B\ {\isasymand}\ {\isasymbeta}\isactrlbsub B\isactrlesub \ {\isasymcirc}\isactrlsub c\ j\ {\isacharequal}{\kern0pt}\ id\ {\isasymone}\ {\isasymand}\ m\ {\isasymcirc}\isactrlsub c\ j\ {\isacharequal}{\kern0pt}\ x{\isachardoublequoteclose}\isanewline
\ \ \ \ \isacommand{unfolding}\isamarkupfalse%
\ is{\isacharunderscore}{\kern0pt}pullback{\isacharunderscore}{\kern0pt}def\ \isacommand{using}\isamarkupfalse%
\ assms\ \isacommand{by}\isamarkupfalse%
\ {\isacharparenleft}{\kern0pt}metis\ id{\isacharunderscore}{\kern0pt}right{\isacharunderscore}{\kern0pt}unit{\isadigit{2}}\ id{\isacharunderscore}{\kern0pt}type\ true{\isacharunderscore}{\kern0pt}func{\isacharunderscore}{\kern0pt}type{\isacharparenright}{\kern0pt}\isanewline
\ \ \isacommand{then}\isamarkupfalse%
\ \isacommand{show}\isamarkupfalse%
\ {\isachardoublequoteopen}{\isasymexists}j{\isachardot}{\kern0pt}\ j\ {\isacharcolon}{\kern0pt}\ domain\ x\ {\isasymrightarrow}\ domain\ m\ {\isasymand}\ m\ {\isasymcirc}\isactrlsub c\ j\ {\isacharequal}{\kern0pt}\ x{\isachardoublequoteclose}\isanewline
\ \ \ \ \isacommand{using}\isamarkupfalse%
\ assms{\isacharparenleft}{\kern0pt}{\isadigit{1}}{\isacharcomma}{\kern0pt}{\isadigit{3}}{\isacharparenright}{\kern0pt}\ cfunc{\isacharunderscore}{\kern0pt}type{\isacharunderscore}{\kern0pt}def\ \isacommand{by}\isamarkupfalse%
\ auto\isanewline
\isacommand{qed}\isamarkupfalse%
%
\endisatagproof
{\isafoldproof}%
%
\isadelimproof
\isanewline
%
\endisadelimproof
\isanewline
\isacommand{lemma}\isamarkupfalse%
\ characteristic{\isacharunderscore}{\kern0pt}func{\isacharunderscore}{\kern0pt}false{\isacharunderscore}{\kern0pt}not{\isacharunderscore}{\kern0pt}relative{\isacharunderscore}{\kern0pt}member{\isacharcolon}{\kern0pt}\isanewline
\ \ \isakeyword{assumes}\ {\isachardoublequoteopen}m\ {\isacharcolon}{\kern0pt}\ B\ {\isasymrightarrow}\ X{\isachardoublequoteclose}\ {\isachardoublequoteopen}monomorphism\ m{\isachardoublequoteclose}\ {\isachardoublequoteopen}x\ {\isasymin}\isactrlsub c\ X{\isachardoublequoteclose}\isanewline
\ \ \isakeyword{assumes}\ characteristic{\isacharunderscore}{\kern0pt}func{\isacharunderscore}{\kern0pt}true{\isacharcolon}{\kern0pt}\ {\isachardoublequoteopen}characteristic{\isacharunderscore}{\kern0pt}func\ m\ {\isasymcirc}\isactrlsub c\ x\ {\isacharequal}{\kern0pt}\ {\isasymf}{\isachardoublequoteclose}\isanewline
\ \ \isakeyword{shows}\ {\isachardoublequoteopen}{\isasymnot}\ {\isacharparenleft}{\kern0pt}x\ {\isasymin}\isactrlbsub X\isactrlesub \ {\isacharparenleft}{\kern0pt}B{\isacharcomma}{\kern0pt}m{\isacharparenright}{\kern0pt}{\isacharparenright}{\kern0pt}{\isachardoublequoteclose}\isanewline
%
\isadelimproof
%
\endisadelimproof
%
\isatagproof
\isacommand{proof}\isamarkupfalse%
\ {\isacharparenleft}{\kern0pt}insert\ assms{\isacharcomma}{\kern0pt}\ unfold\ relative{\isacharunderscore}{\kern0pt}member{\isacharunderscore}{\kern0pt}def{\isadigit{2}}\ factors{\isacharunderscore}{\kern0pt}through{\isacharunderscore}{\kern0pt}def{\isacharcomma}{\kern0pt}\ clarify{\isacharparenright}{\kern0pt}\isanewline
\ \ \isacommand{fix}\isamarkupfalse%
\ h\isanewline
\ \ \isacommand{assume}\isamarkupfalse%
\ x{\isacharunderscore}{\kern0pt}def{\isacharcolon}{\kern0pt}\ {\isachardoublequoteopen}x\ {\isacharequal}{\kern0pt}\ m\ {\isasymcirc}\isactrlsub c\ h{\isachardoublequoteclose}\isanewline
\ \ \isacommand{assume}\isamarkupfalse%
\ {\isachardoublequoteopen}h\ {\isacharcolon}{\kern0pt}\ domain\ {\isacharparenleft}{\kern0pt}m\ {\isasymcirc}\isactrlsub c\ h{\isacharparenright}{\kern0pt}\ {\isasymrightarrow}\ domain\ m{\isachardoublequoteclose}\isanewline
\ \ \isacommand{then}\isamarkupfalse%
\ \isacommand{have}\isamarkupfalse%
\ h{\isacharunderscore}{\kern0pt}type{\isacharcolon}{\kern0pt}\ {\isachardoublequoteopen}h\ {\isasymin}\isactrlsub c\ B{\isachardoublequoteclose}\isanewline
\ \ \ \ \isacommand{using}\isamarkupfalse%
\ assms{\isacharparenleft}{\kern0pt}{\isadigit{1}}{\isacharcomma}{\kern0pt}{\isadigit{3}}{\isacharparenright}{\kern0pt}\ cfunc{\isacharunderscore}{\kern0pt}type{\isacharunderscore}{\kern0pt}def\ x{\isacharunderscore}{\kern0pt}def\ \isacommand{by}\isamarkupfalse%
\ auto\isanewline
\isanewline
\ \ \isacommand{have}\isamarkupfalse%
\ {\isachardoublequoteopen}is{\isacharunderscore}{\kern0pt}pullback\ B\ {\isasymone}\ X\ {\isasymOmega}\ {\isacharparenleft}{\kern0pt}{\isasymbeta}\isactrlbsub B\isactrlesub {\isacharparenright}{\kern0pt}\ {\isasymt}\ m\ {\isacharparenleft}{\kern0pt}characteristic{\isacharunderscore}{\kern0pt}func\ m{\isacharparenright}{\kern0pt}{\isachardoublequoteclose}\isanewline
\ \ \ \ \isacommand{by}\isamarkupfalse%
\ {\isacharparenleft}{\kern0pt}simp\ add{\isacharcolon}{\kern0pt}\ assms\ characteristic{\isacharunderscore}{\kern0pt}func{\isacharunderscore}{\kern0pt}is{\isacharunderscore}{\kern0pt}pullback{\isacharparenright}{\kern0pt}\isanewline
\ \ \isacommand{then}\isamarkupfalse%
\ \isacommand{have}\isamarkupfalse%
\ char{\isacharunderscore}{\kern0pt}m{\isacharunderscore}{\kern0pt}true{\isacharcolon}{\kern0pt}\ {\isachardoublequoteopen}characteristic{\isacharunderscore}{\kern0pt}func\ m\ {\isasymcirc}\isactrlsub c\ m\ {\isacharequal}{\kern0pt}\ {\isasymt}\ {\isasymcirc}\isactrlsub c\ {\isasymbeta}\isactrlbsub B\isactrlesub {\isachardoublequoteclose}\isanewline
\ \ \ \ \isacommand{unfolding}\isamarkupfalse%
\ is{\isacharunderscore}{\kern0pt}pullback{\isacharunderscore}{\kern0pt}def\ \isacommand{by}\isamarkupfalse%
\ auto\isanewline
\isanewline
\ \ \isacommand{then}\isamarkupfalse%
\ \isacommand{have}\isamarkupfalse%
\ {\isachardoublequoteopen}characteristic{\isacharunderscore}{\kern0pt}func\ m\ {\isasymcirc}\isactrlsub c\ m\ {\isasymcirc}\isactrlsub c\ h\ {\isacharequal}{\kern0pt}\ {\isasymf}{\isachardoublequoteclose}\isanewline
\ \ \ \ \isacommand{using}\isamarkupfalse%
\ x{\isacharunderscore}{\kern0pt}def\ characteristic{\isacharunderscore}{\kern0pt}func{\isacharunderscore}{\kern0pt}true\ \isacommand{by}\isamarkupfalse%
\ auto\isanewline
\ \ \isacommand{then}\isamarkupfalse%
\ \isacommand{have}\isamarkupfalse%
\ {\isachardoublequoteopen}{\isacharparenleft}{\kern0pt}characteristic{\isacharunderscore}{\kern0pt}func\ m\ {\isasymcirc}\isactrlsub c\ m{\isacharparenright}{\kern0pt}\ {\isasymcirc}\isactrlsub c\ h\ {\isacharequal}{\kern0pt}\ {\isasymf}{\isachardoublequoteclose}\isanewline
\ \ \ \ \isacommand{using}\isamarkupfalse%
\ assms\ h{\isacharunderscore}{\kern0pt}type\ \isacommand{by}\isamarkupfalse%
\ {\isacharparenleft}{\kern0pt}typecheck{\isacharunderscore}{\kern0pt}cfuncs{\isacharcomma}{\kern0pt}\ simp\ add{\isacharcolon}{\kern0pt}\ comp{\isacharunderscore}{\kern0pt}associative{\isadigit{2}}{\isacharparenright}{\kern0pt}\isanewline
\ \ \isacommand{then}\isamarkupfalse%
\ \isacommand{have}\isamarkupfalse%
\ {\isachardoublequoteopen}{\isacharparenleft}{\kern0pt}{\isasymt}\ {\isasymcirc}\isactrlsub c\ {\isasymbeta}\isactrlbsub B\isactrlesub {\isacharparenright}{\kern0pt}\ {\isasymcirc}\isactrlsub c\ h\ {\isacharequal}{\kern0pt}\ {\isasymf}{\isachardoublequoteclose}\ \ \ \ \isanewline
\ \ \ \ \isacommand{using}\isamarkupfalse%
\ char{\isacharunderscore}{\kern0pt}m{\isacharunderscore}{\kern0pt}true\ \isacommand{by}\isamarkupfalse%
\ auto\isanewline
\ \ \isacommand{then}\isamarkupfalse%
\ \isacommand{have}\isamarkupfalse%
\ {\isachardoublequoteopen}{\isasymt}\ {\isacharequal}{\kern0pt}\ {\isasymf}{\isachardoublequoteclose}\isanewline
\ \ \ \ \isacommand{by}\isamarkupfalse%
\ {\isacharparenleft}{\kern0pt}metis\ cfunc{\isacharunderscore}{\kern0pt}type{\isacharunderscore}{\kern0pt}def\ comp{\isacharunderscore}{\kern0pt}associative\ h{\isacharunderscore}{\kern0pt}type\ id{\isacharunderscore}{\kern0pt}right{\isacharunderscore}{\kern0pt}unit{\isadigit{2}}\ id{\isacharunderscore}{\kern0pt}type\ one{\isacharunderscore}{\kern0pt}unique{\isacharunderscore}{\kern0pt}element\isanewline
\ \ \ \ \ \ \ \ terminal{\isacharunderscore}{\kern0pt}func{\isacharunderscore}{\kern0pt}comp\ terminal{\isacharunderscore}{\kern0pt}func{\isacharunderscore}{\kern0pt}type\ true{\isacharunderscore}{\kern0pt}func{\isacharunderscore}{\kern0pt}type{\isacharparenright}{\kern0pt}\isanewline
\ \ \isacommand{then}\isamarkupfalse%
\ \isacommand{show}\isamarkupfalse%
\ False\isanewline
\ \ \ \ \isacommand{using}\isamarkupfalse%
\ true{\isacharunderscore}{\kern0pt}false{\isacharunderscore}{\kern0pt}distinct\ \isacommand{by}\isamarkupfalse%
\ auto\isanewline
\isacommand{qed}\isamarkupfalse%
%
\endisatagproof
{\isafoldproof}%
%
\isadelimproof
\isanewline
%
\endisadelimproof
\isanewline
\isacommand{lemma}\isamarkupfalse%
\ rel{\isacharunderscore}{\kern0pt}mem{\isacharunderscore}{\kern0pt}char{\isacharunderscore}{\kern0pt}func{\isacharunderscore}{\kern0pt}true{\isacharcolon}{\kern0pt}\isanewline
\ \ \isakeyword{assumes}\ {\isachardoublequoteopen}m\ {\isacharcolon}{\kern0pt}\ B\ {\isasymrightarrow}\ X{\isachardoublequoteclose}\ {\isachardoublequoteopen}monomorphism\ m{\isachardoublequoteclose}\ {\isachardoublequoteopen}x\ {\isasymin}\isactrlsub c\ X{\isachardoublequoteclose}\isanewline
\ \ \isakeyword{assumes}\ {\isachardoublequoteopen}x\ {\isasymin}\isactrlbsub X\isactrlesub \ {\isacharparenleft}{\kern0pt}B{\isacharcomma}{\kern0pt}m{\isacharparenright}{\kern0pt}{\isachardoublequoteclose}\isanewline
\ \ \isakeyword{shows}\ {\isachardoublequoteopen}characteristic{\isacharunderscore}{\kern0pt}func\ m\ {\isasymcirc}\isactrlsub c\ x\ {\isacharequal}{\kern0pt}\ {\isasymt}{\isachardoublequoteclose}\isanewline
%
\isadelimproof
\ \ %
\endisadelimproof
%
\isatagproof
\isacommand{by}\isamarkupfalse%
\ {\isacharparenleft}{\kern0pt}meson\ assms{\isacharparenleft}{\kern0pt}{\isadigit{4}}{\isacharparenright}{\kern0pt}\ characteristic{\isacharunderscore}{\kern0pt}func{\isacharunderscore}{\kern0pt}false{\isacharunderscore}{\kern0pt}not{\isacharunderscore}{\kern0pt}relative{\isacharunderscore}{\kern0pt}member\ characteristic{\isacharunderscore}{\kern0pt}func{\isacharunderscore}{\kern0pt}type\ comp{\isacharunderscore}{\kern0pt}type\ relative{\isacharunderscore}{\kern0pt}member{\isacharunderscore}{\kern0pt}def{\isadigit{2}}\ true{\isacharunderscore}{\kern0pt}false{\isacharunderscore}{\kern0pt}only{\isacharunderscore}{\kern0pt}truth{\isacharunderscore}{\kern0pt}values{\isacharparenright}{\kern0pt}%
\endisatagproof
{\isafoldproof}%
%
\isadelimproof
\isanewline
%
\endisadelimproof
\isanewline
\isacommand{lemma}\isamarkupfalse%
\ not{\isacharunderscore}{\kern0pt}rel{\isacharunderscore}{\kern0pt}mem{\isacharunderscore}{\kern0pt}char{\isacharunderscore}{\kern0pt}func{\isacharunderscore}{\kern0pt}false{\isacharcolon}{\kern0pt}\isanewline
\ \ \isakeyword{assumes}\ {\isachardoublequoteopen}m\ {\isacharcolon}{\kern0pt}\ B\ {\isasymrightarrow}\ X{\isachardoublequoteclose}\ {\isachardoublequoteopen}monomorphism\ m{\isachardoublequoteclose}\ {\isachardoublequoteopen}x\ {\isasymin}\isactrlsub c\ X{\isachardoublequoteclose}\isanewline
\ \ \isakeyword{assumes}\ {\isachardoublequoteopen}{\isasymnot}\ {\isacharparenleft}{\kern0pt}x\ {\isasymin}\isactrlbsub X\isactrlesub \ {\isacharparenleft}{\kern0pt}B{\isacharcomma}{\kern0pt}m{\isacharparenright}{\kern0pt}{\isacharparenright}{\kern0pt}{\isachardoublequoteclose}\isanewline
\ \ \isakeyword{shows}\ {\isachardoublequoteopen}characteristic{\isacharunderscore}{\kern0pt}func\ m\ {\isasymcirc}\isactrlsub c\ x\ {\isacharequal}{\kern0pt}\ {\isasymf}{\isachardoublequoteclose}\isanewline
%
\isadelimproof
\ \ %
\endisadelimproof
%
\isatagproof
\isacommand{by}\isamarkupfalse%
\ {\isacharparenleft}{\kern0pt}meson\ assms\ characteristic{\isacharunderscore}{\kern0pt}func{\isacharunderscore}{\kern0pt}true{\isacharunderscore}{\kern0pt}relative{\isacharunderscore}{\kern0pt}member\ characteristic{\isacharunderscore}{\kern0pt}func{\isacharunderscore}{\kern0pt}type\ comp{\isacharunderscore}{\kern0pt}type\ true{\isacharunderscore}{\kern0pt}false{\isacharunderscore}{\kern0pt}only{\isacharunderscore}{\kern0pt}truth{\isacharunderscore}{\kern0pt}values{\isacharparenright}{\kern0pt}%
\endisatagproof
{\isafoldproof}%
%
\isadelimproof
%
\endisadelimproof
%
\begin{isamarkuptext}%
The lemma below corresponds to Proposition 2.2.2 in Halvorson.%
\end{isamarkuptext}\isamarkuptrue%
\isacommand{lemma}\isamarkupfalse%
\ {\isachardoublequoteopen}card\ {\isacharbraceleft}{\kern0pt}x{\isachardot}{\kern0pt}\ x\ {\isasymin}\isactrlsub c\ {\isasymOmega}\ {\isasymtimes}\isactrlsub c\ {\isasymOmega}{\isacharbraceright}{\kern0pt}\ {\isacharequal}{\kern0pt}\ {\isadigit{4}}{\isachardoublequoteclose}\isanewline
%
\isadelimproof
%
\endisadelimproof
%
\isatagproof
\isacommand{proof}\isamarkupfalse%
\ {\isacharminus}{\kern0pt}\isanewline
\ \ \isacommand{have}\isamarkupfalse%
\ {\isachardoublequoteopen}{\isacharbraceleft}{\kern0pt}x{\isachardot}{\kern0pt}\ x\ {\isasymin}\isactrlsub c\ {\isasymOmega}\ {\isasymtimes}\isactrlsub c\ {\isasymOmega}{\isacharbraceright}{\kern0pt}\ {\isacharequal}{\kern0pt}\ {\isacharbraceleft}{\kern0pt}{\isasymlangle}{\isasymt}{\isacharcomma}{\kern0pt}{\isasymt}{\isasymrangle}{\isacharcomma}{\kern0pt}\ {\isasymlangle}{\isasymt}{\isacharcomma}{\kern0pt}{\isasymf}{\isasymrangle}{\isacharcomma}{\kern0pt}\ {\isasymlangle}{\isasymf}{\isacharcomma}{\kern0pt}{\isasymt}{\isasymrangle}{\isacharcomma}{\kern0pt}\ {\isasymlangle}{\isasymf}{\isacharcomma}{\kern0pt}{\isasymf}{\isasymrangle}{\isacharbraceright}{\kern0pt}{\isachardoublequoteclose}\isanewline
\ \ \ \ \isacommand{by}\isamarkupfalse%
\ {\isacharparenleft}{\kern0pt}auto\ simp\ add{\isacharcolon}{\kern0pt}\ cfunc{\isacharunderscore}{\kern0pt}prod{\isacharunderscore}{\kern0pt}type\ true{\isacharunderscore}{\kern0pt}func{\isacharunderscore}{\kern0pt}type\ false{\isacharunderscore}{\kern0pt}func{\isacharunderscore}{\kern0pt}type{\isacharcomma}{\kern0pt}\isanewline
\ \ \ \ \ \ \ \ smt\ cfunc{\isacharunderscore}{\kern0pt}prod{\isacharunderscore}{\kern0pt}unique\ comp{\isacharunderscore}{\kern0pt}type\ left{\isacharunderscore}{\kern0pt}cart{\isacharunderscore}{\kern0pt}proj{\isacharunderscore}{\kern0pt}type\ right{\isacharunderscore}{\kern0pt}cart{\isacharunderscore}{\kern0pt}proj{\isacharunderscore}{\kern0pt}type\ true{\isacharunderscore}{\kern0pt}false{\isacharunderscore}{\kern0pt}only{\isacharunderscore}{\kern0pt}truth{\isacharunderscore}{\kern0pt}values{\isacharparenright}{\kern0pt}\isanewline
\ \ \isacommand{then}\isamarkupfalse%
\ \isacommand{show}\isamarkupfalse%
\ {\isachardoublequoteopen}card\ {\isacharbraceleft}{\kern0pt}x{\isachardot}{\kern0pt}\ x\ {\isasymin}\isactrlsub c\ {\isasymOmega}\ {\isasymtimes}\isactrlsub c\ {\isasymOmega}{\isacharbraceright}{\kern0pt}\ {\isacharequal}{\kern0pt}\ {\isadigit{4}}{\isachardoublequoteclose}\isanewline
\ \ \ \ \isacommand{using}\isamarkupfalse%
\ element{\isacharunderscore}{\kern0pt}pair{\isacharunderscore}{\kern0pt}eq\ false{\isacharunderscore}{\kern0pt}func{\isacharunderscore}{\kern0pt}type\ true{\isacharunderscore}{\kern0pt}false{\isacharunderscore}{\kern0pt}distinct\ true{\isacharunderscore}{\kern0pt}func{\isacharunderscore}{\kern0pt}type\ \isacommand{by}\isamarkupfalse%
\ auto\isanewline
\isacommand{qed}\isamarkupfalse%
%
\endisatagproof
{\isafoldproof}%
%
\isadelimproof
%
\endisadelimproof
%
\isadelimdocument
%
\endisadelimdocument
%
\isatagdocument
%
\isamarkupsubsection{Equality Predicate%
}
\isamarkuptrue%
%
\endisatagdocument
{\isafolddocument}%
%
\isadelimdocument
%
\endisadelimdocument
\isacommand{definition}\isamarkupfalse%
\ eq{\isacharunderscore}{\kern0pt}pred\ {\isacharcolon}{\kern0pt}{\isacharcolon}{\kern0pt}\ {\isachardoublequoteopen}cset\ {\isasymRightarrow}\ cfunc{\isachardoublequoteclose}\ \isakeyword{where}\isanewline
\ \ {\isachardoublequoteopen}eq{\isacharunderscore}{\kern0pt}pred\ X\ {\isacharequal}{\kern0pt}\ {\isacharparenleft}{\kern0pt}THE\ {\isasymchi}{\isachardot}{\kern0pt}\ is{\isacharunderscore}{\kern0pt}pullback\ X\ {\isasymone}\ {\isacharparenleft}{\kern0pt}X\ {\isasymtimes}\isactrlsub c\ X{\isacharparenright}{\kern0pt}\ {\isasymOmega}\ {\isacharparenleft}{\kern0pt}{\isasymbeta}\isactrlbsub X\isactrlesub {\isacharparenright}{\kern0pt}\ {\isasymt}\ {\isacharparenleft}{\kern0pt}diagonal\ X{\isacharparenright}{\kern0pt}\ {\isasymchi}{\isacharparenright}{\kern0pt}{\isachardoublequoteclose}\isanewline
\isanewline
\isacommand{lemma}\isamarkupfalse%
\ eq{\isacharunderscore}{\kern0pt}pred{\isacharunderscore}{\kern0pt}pullback{\isacharcolon}{\kern0pt}\ {\isachardoublequoteopen}is{\isacharunderscore}{\kern0pt}pullback\ X\ {\isasymone}\ {\isacharparenleft}{\kern0pt}X\ {\isasymtimes}\isactrlsub c\ X{\isacharparenright}{\kern0pt}\ {\isasymOmega}\ {\isacharparenleft}{\kern0pt}{\isasymbeta}\isactrlbsub X\isactrlesub {\isacharparenright}{\kern0pt}\ {\isasymt}\ {\isacharparenleft}{\kern0pt}diagonal\ X{\isacharparenright}{\kern0pt}\ {\isacharparenleft}{\kern0pt}eq{\isacharunderscore}{\kern0pt}pred\ X{\isacharparenright}{\kern0pt}{\isachardoublequoteclose}\isanewline
%
\isadelimproof
\ \ %
\endisadelimproof
%
\isatagproof
\isacommand{unfolding}\isamarkupfalse%
\ eq{\isacharunderscore}{\kern0pt}pred{\isacharunderscore}{\kern0pt}def\isanewline
\ \ \isacommand{by}\isamarkupfalse%
\ {\isacharparenleft}{\kern0pt}rule\ the{\isadigit{1}}I{\isadigit{2}}{\isacharcomma}{\kern0pt}\ simp{\isacharunderscore}{\kern0pt}all\ add{\isacharcolon}{\kern0pt}\ characteristic{\isacharunderscore}{\kern0pt}function{\isacharunderscore}{\kern0pt}exists\ diag{\isacharunderscore}{\kern0pt}mono\ diagonal{\isacharunderscore}{\kern0pt}type{\isacharparenright}{\kern0pt}%
\endisatagproof
{\isafoldproof}%
%
\isadelimproof
\isanewline
%
\endisadelimproof
\isanewline
\isacommand{lemma}\isamarkupfalse%
\ eq{\isacharunderscore}{\kern0pt}pred{\isacharunderscore}{\kern0pt}type{\isacharbrackleft}{\kern0pt}type{\isacharunderscore}{\kern0pt}rule{\isacharbrackright}{\kern0pt}{\isacharcolon}{\kern0pt}\isanewline
\ \ {\isachardoublequoteopen}eq{\isacharunderscore}{\kern0pt}pred\ X\ {\isacharcolon}{\kern0pt}\ X\ {\isasymtimes}\isactrlsub c\ X\ {\isasymrightarrow}\ {\isasymOmega}{\isachardoublequoteclose}\isanewline
%
\isadelimproof
\ \ %
\endisadelimproof
%
\isatagproof
\isacommand{using}\isamarkupfalse%
\ eq{\isacharunderscore}{\kern0pt}pred{\isacharunderscore}{\kern0pt}pullback\ \isacommand{unfolding}\isamarkupfalse%
\ is{\isacharunderscore}{\kern0pt}pullback{\isacharunderscore}{\kern0pt}def\ \ \isacommand{by}\isamarkupfalse%
\ auto%
\endisatagproof
{\isafoldproof}%
%
\isadelimproof
\isanewline
%
\endisadelimproof
\isanewline
\isacommand{lemma}\isamarkupfalse%
\ eq{\isacharunderscore}{\kern0pt}pred{\isacharunderscore}{\kern0pt}square{\isacharcolon}{\kern0pt}\ {\isachardoublequoteopen}eq{\isacharunderscore}{\kern0pt}pred\ X\ {\isasymcirc}\isactrlsub c\ diagonal\ X\ {\isacharequal}{\kern0pt}\ {\isasymt}\ {\isasymcirc}\isactrlsub c\ {\isasymbeta}\isactrlbsub X\isactrlesub {\isachardoublequoteclose}\isanewline
%
\isadelimproof
\ \ %
\endisadelimproof
%
\isatagproof
\isacommand{using}\isamarkupfalse%
\ eq{\isacharunderscore}{\kern0pt}pred{\isacharunderscore}{\kern0pt}pullback\ \isacommand{unfolding}\isamarkupfalse%
\ is{\isacharunderscore}{\kern0pt}pullback{\isacharunderscore}{\kern0pt}def\ \ \isacommand{by}\isamarkupfalse%
\ auto%
\endisatagproof
{\isafoldproof}%
%
\isadelimproof
\isanewline
%
\endisadelimproof
\isanewline
\isacommand{lemma}\isamarkupfalse%
\ eq{\isacharunderscore}{\kern0pt}pred{\isacharunderscore}{\kern0pt}iff{\isacharunderscore}{\kern0pt}eq{\isacharcolon}{\kern0pt}\isanewline
\ \ \isakeyword{assumes}\ {\isachardoublequoteopen}x\ {\isacharcolon}{\kern0pt}\ {\isasymone}\ {\isasymrightarrow}\ X{\isachardoublequoteclose}\ {\isachardoublequoteopen}y\ {\isacharcolon}{\kern0pt}\ {\isasymone}\ {\isasymrightarrow}\ X{\isachardoublequoteclose}\isanewline
\ \ \isakeyword{shows}\ {\isachardoublequoteopen}{\isacharparenleft}{\kern0pt}x\ {\isacharequal}{\kern0pt}\ y{\isacharparenright}{\kern0pt}\ {\isacharequal}{\kern0pt}\ {\isacharparenleft}{\kern0pt}eq{\isacharunderscore}{\kern0pt}pred\ X\ {\isasymcirc}\isactrlsub c\ {\isasymlangle}x{\isacharcomma}{\kern0pt}\ y{\isasymrangle}\ {\isacharequal}{\kern0pt}\ {\isasymt}{\isacharparenright}{\kern0pt}{\isachardoublequoteclose}\isanewline
%
\isadelimproof
%
\endisadelimproof
%
\isatagproof
\isacommand{proof}\isamarkupfalse%
\ safe\isanewline
\ \ \isacommand{assume}\isamarkupfalse%
\ x{\isacharunderscore}{\kern0pt}eq{\isacharunderscore}{\kern0pt}y{\isacharcolon}{\kern0pt}\ {\isachardoublequoteopen}x\ {\isacharequal}{\kern0pt}\ y{\isachardoublequoteclose}\isanewline
\isanewline
\ \ \isacommand{have}\isamarkupfalse%
\ {\isachardoublequoteopen}{\isacharparenleft}{\kern0pt}eq{\isacharunderscore}{\kern0pt}pred\ X\ {\isasymcirc}\isactrlsub c\ {\isasymlangle}id\isactrlsub c\ X{\isacharcomma}{\kern0pt}id\isactrlsub c\ X{\isasymrangle}{\isacharparenright}{\kern0pt}\ {\isasymcirc}\isactrlsub c\ y\ {\isacharequal}{\kern0pt}\ {\isacharparenleft}{\kern0pt}{\isasymt}\ {\isasymcirc}\isactrlsub c\ {\isasymbeta}\isactrlbsub X\isactrlesub {\isacharparenright}{\kern0pt}\ {\isasymcirc}\isactrlsub c\ y{\isachardoublequoteclose}\isanewline
\ \ \ \ \isacommand{using}\isamarkupfalse%
\ eq{\isacharunderscore}{\kern0pt}pred{\isacharunderscore}{\kern0pt}square\ \isacommand{unfolding}\isamarkupfalse%
\ diagonal{\isacharunderscore}{\kern0pt}def\ \isacommand{by}\isamarkupfalse%
\ auto\isanewline
\ \ \isacommand{then}\isamarkupfalse%
\ \isacommand{have}\isamarkupfalse%
\ {\isachardoublequoteopen}eq{\isacharunderscore}{\kern0pt}pred\ X\ {\isasymcirc}\isactrlsub c\ {\isasymlangle}y{\isacharcomma}{\kern0pt}\ y{\isasymrangle}\ {\isacharequal}{\kern0pt}\ {\isacharparenleft}{\kern0pt}{\isasymt}\ {\isasymcirc}\isactrlsub c\ {\isasymbeta}\isactrlbsub X\isactrlesub {\isacharparenright}{\kern0pt}\ {\isasymcirc}\isactrlsub c\ y{\isachardoublequoteclose}\isanewline
\ \ \ \ \isacommand{using}\isamarkupfalse%
\ assms\ diagonal{\isacharunderscore}{\kern0pt}type\ id{\isacharunderscore}{\kern0pt}type\isanewline
\ \ \ \ \isacommand{by}\isamarkupfalse%
\ {\isacharparenleft}{\kern0pt}typecheck{\isacharunderscore}{\kern0pt}cfuncs{\isacharcomma}{\kern0pt}\ smt\ cfunc{\isacharunderscore}{\kern0pt}prod{\isacharunderscore}{\kern0pt}comp\ comp{\isacharunderscore}{\kern0pt}associative{\isadigit{2}}\ diagonal{\isacharunderscore}{\kern0pt}def\ id{\isacharunderscore}{\kern0pt}left{\isacharunderscore}{\kern0pt}unit{\isadigit{2}}{\isacharparenright}{\kern0pt}\isanewline
\ \ \isacommand{then}\isamarkupfalse%
\ \isacommand{show}\isamarkupfalse%
\ {\isachardoublequoteopen}eq{\isacharunderscore}{\kern0pt}pred\ X\ {\isasymcirc}\isactrlsub c\ {\isasymlangle}y{\isacharcomma}{\kern0pt}\ y{\isasymrangle}\ {\isacharequal}{\kern0pt}\ {\isasymt}{\isachardoublequoteclose}\isanewline
\ \ \ \ \isacommand{using}\isamarkupfalse%
\ assms\ id{\isacharunderscore}{\kern0pt}type\isanewline
\ \ \ \ \isacommand{by}\isamarkupfalse%
\ {\isacharparenleft}{\kern0pt}typecheck{\isacharunderscore}{\kern0pt}cfuncs{\isacharcomma}{\kern0pt}\ smt\ comp{\isacharunderscore}{\kern0pt}associative{\isadigit{2}}\ terminal{\isacharunderscore}{\kern0pt}func{\isacharunderscore}{\kern0pt}comp\ terminal{\isacharunderscore}{\kern0pt}func{\isacharunderscore}{\kern0pt}type\ terminal{\isacharunderscore}{\kern0pt}func{\isacharunderscore}{\kern0pt}unique\ id{\isacharunderscore}{\kern0pt}right{\isacharunderscore}{\kern0pt}unit{\isadigit{2}}{\isacharparenright}{\kern0pt}\isanewline
\isacommand{next}\isamarkupfalse%
\isanewline
\ \ \isacommand{assume}\isamarkupfalse%
\ {\isachardoublequoteopen}eq{\isacharunderscore}{\kern0pt}pred\ X\ {\isasymcirc}\isactrlsub c\ {\isasymlangle}x{\isacharcomma}{\kern0pt}y{\isasymrangle}\ {\isacharequal}{\kern0pt}\ {\isasymt}{\isachardoublequoteclose}\isanewline
\ \ \isacommand{then}\isamarkupfalse%
\ \isacommand{have}\isamarkupfalse%
\ {\isachardoublequoteopen}eq{\isacharunderscore}{\kern0pt}pred\ X\ {\isasymcirc}\isactrlsub c\ {\isasymlangle}x{\isacharcomma}{\kern0pt}y{\isasymrangle}\ {\isacharequal}{\kern0pt}\ {\isasymt}\ {\isasymcirc}\isactrlsub c\ id\ {\isasymone}{\isachardoublequoteclose}\isanewline
\ \ \ \ \isacommand{using}\isamarkupfalse%
\ id{\isacharunderscore}{\kern0pt}right{\isacharunderscore}{\kern0pt}unit{\isadigit{2}}\ true{\isacharunderscore}{\kern0pt}func{\isacharunderscore}{\kern0pt}type\ \isacommand{by}\isamarkupfalse%
\ auto\isanewline
\ \ \isacommand{then}\isamarkupfalse%
\ \isacommand{obtain}\isamarkupfalse%
\ j\ \ \isakeyword{where}\ j{\isacharunderscore}{\kern0pt}type{\isacharcolon}{\kern0pt}\ {\isachardoublequoteopen}j\ {\isacharcolon}{\kern0pt}\ {\isasymone}\ {\isasymrightarrow}\ X{\isachardoublequoteclose}\ \isakeyword{and}\ {\isachardoublequoteopen}diagonal\ X\ {\isasymcirc}\isactrlsub c\ j\ {\isacharequal}{\kern0pt}\ {\isasymlangle}x{\isacharcomma}{\kern0pt}y{\isasymrangle}{\isachardoublequoteclose}\isanewline
\ \ \ \ \isacommand{using}\isamarkupfalse%
\ eq{\isacharunderscore}{\kern0pt}pred{\isacharunderscore}{\kern0pt}pullback\ assms\ \isacommand{unfolding}\isamarkupfalse%
\ is{\isacharunderscore}{\kern0pt}pullback{\isacharunderscore}{\kern0pt}def\ \isacommand{by}\isamarkupfalse%
\ {\isacharparenleft}{\kern0pt}metis\ cfunc{\isacharunderscore}{\kern0pt}prod{\isacharunderscore}{\kern0pt}type\ id{\isacharunderscore}{\kern0pt}type{\isacharparenright}{\kern0pt}\isanewline
\ \ \isacommand{then}\isamarkupfalse%
\ \isacommand{have}\isamarkupfalse%
\ {\isachardoublequoteopen}{\isasymlangle}j{\isacharcomma}{\kern0pt}j{\isasymrangle}\ {\isacharequal}{\kern0pt}\ {\isasymlangle}x{\isacharcomma}{\kern0pt}y{\isasymrangle}{\isachardoublequoteclose}\isanewline
\ \ \ \ \isacommand{using}\isamarkupfalse%
\ diag{\isacharunderscore}{\kern0pt}on{\isacharunderscore}{\kern0pt}elements\ \isacommand{by}\isamarkupfalse%
\ auto\isanewline
\ \ \isacommand{then}\isamarkupfalse%
\ \isacommand{show}\isamarkupfalse%
\ {\isachardoublequoteopen}x\ {\isacharequal}{\kern0pt}\ y{\isachardoublequoteclose}\isanewline
\ \ \ \ \isacommand{using}\isamarkupfalse%
\ assms\ element{\isacharunderscore}{\kern0pt}pair{\isacharunderscore}{\kern0pt}eq\ j{\isacharunderscore}{\kern0pt}type\ \isacommand{by}\isamarkupfalse%
\ auto\isanewline
\isacommand{qed}\isamarkupfalse%
%
\endisatagproof
{\isafoldproof}%
%
\isadelimproof
\isanewline
%
\endisadelimproof
\isanewline
\isacommand{lemma}\isamarkupfalse%
\ eq{\isacharunderscore}{\kern0pt}pred{\isacharunderscore}{\kern0pt}iff{\isacharunderscore}{\kern0pt}eq{\isacharunderscore}{\kern0pt}conv{\isacharcolon}{\kern0pt}\isanewline
\ \ \isakeyword{assumes}\ {\isachardoublequoteopen}x\ {\isacharcolon}{\kern0pt}\ {\isasymone}\ {\isasymrightarrow}\ X{\isachardoublequoteclose}\ {\isachardoublequoteopen}y\ {\isacharcolon}{\kern0pt}\ {\isasymone}\ {\isasymrightarrow}\ X{\isachardoublequoteclose}\isanewline
\ \ \isakeyword{shows}\ {\isachardoublequoteopen}{\isacharparenleft}{\kern0pt}x\ {\isasymnoteq}\ y{\isacharparenright}{\kern0pt}\ {\isacharequal}{\kern0pt}\ {\isacharparenleft}{\kern0pt}eq{\isacharunderscore}{\kern0pt}pred\ X\ {\isasymcirc}\isactrlsub c\ {\isasymlangle}x{\isacharcomma}{\kern0pt}\ y{\isasymrangle}\ {\isacharequal}{\kern0pt}\ {\isasymf}{\isacharparenright}{\kern0pt}{\isachardoublequoteclose}\isanewline
%
\isadelimproof
%
\endisadelimproof
%
\isatagproof
\isacommand{proof}\isamarkupfalse%
{\isacharparenleft}{\kern0pt}safe{\isacharparenright}{\kern0pt}\isanewline
\ \ \isacommand{assume}\isamarkupfalse%
\ {\isachardoublequoteopen}x\ {\isasymnoteq}\ y{\isachardoublequoteclose}\isanewline
\ \ \isacommand{then}\isamarkupfalse%
\ \isacommand{show}\isamarkupfalse%
\ {\isachardoublequoteopen}eq{\isacharunderscore}{\kern0pt}pred\ X\ {\isasymcirc}\isactrlsub c\ {\isasymlangle}x{\isacharcomma}{\kern0pt}y{\isasymrangle}\ {\isacharequal}{\kern0pt}\ {\isasymf}{\isachardoublequoteclose}\isanewline
\ \ \ \ \isacommand{using}\isamarkupfalse%
\ assms\ eq{\isacharunderscore}{\kern0pt}pred{\isacharunderscore}{\kern0pt}iff{\isacharunderscore}{\kern0pt}eq\ true{\isacharunderscore}{\kern0pt}false{\isacharunderscore}{\kern0pt}only{\isacharunderscore}{\kern0pt}truth{\isacharunderscore}{\kern0pt}values\ \isacommand{by}\isamarkupfalse%
\ {\isacharparenleft}{\kern0pt}typecheck{\isacharunderscore}{\kern0pt}cfuncs{\isacharcomma}{\kern0pt}\ blast{\isacharparenright}{\kern0pt}\isanewline
\isacommand{next}\isamarkupfalse%
\isanewline
\ \ \isacommand{show}\isamarkupfalse%
\ {\isachardoublequoteopen}eq{\isacharunderscore}{\kern0pt}pred\ X\ {\isasymcirc}\isactrlsub c\ {\isasymlangle}y{\isacharcomma}{\kern0pt}y{\isasymrangle}\ {\isacharequal}{\kern0pt}\ {\isasymf}\ {\isasymLongrightarrow}\ x\ {\isacharequal}{\kern0pt}\ y\ {\isasymLongrightarrow}\ False{\isachardoublequoteclose}\isanewline
\ \ \ \ \isacommand{by}\isamarkupfalse%
\ {\isacharparenleft}{\kern0pt}metis\ assms{\isacharparenleft}{\kern0pt}{\isadigit{1}}{\isacharparenright}{\kern0pt}\ eq{\isacharunderscore}{\kern0pt}pred{\isacharunderscore}{\kern0pt}iff{\isacharunderscore}{\kern0pt}eq\ true{\isacharunderscore}{\kern0pt}false{\isacharunderscore}{\kern0pt}distinct{\isacharparenright}{\kern0pt}\isanewline
\isacommand{qed}\isamarkupfalse%
%
\endisatagproof
{\isafoldproof}%
%
\isadelimproof
\isanewline
%
\endisadelimproof
\isanewline
\isacommand{lemma}\isamarkupfalse%
\ eq{\isacharunderscore}{\kern0pt}pred{\isacharunderscore}{\kern0pt}iff{\isacharunderscore}{\kern0pt}eq{\isacharunderscore}{\kern0pt}conv{\isadigit{2}}{\isacharcolon}{\kern0pt}\isanewline
\ \ \isakeyword{assumes}\ {\isachardoublequoteopen}x\ {\isacharcolon}{\kern0pt}\ {\isasymone}\ {\isasymrightarrow}\ X{\isachardoublequoteclose}\ {\isachardoublequoteopen}y\ {\isacharcolon}{\kern0pt}\ {\isasymone}\ {\isasymrightarrow}\ X{\isachardoublequoteclose}\isanewline
\ \ \isakeyword{shows}\ {\isachardoublequoteopen}{\isacharparenleft}{\kern0pt}x\ {\isasymnoteq}\ y{\isacharparenright}{\kern0pt}\ {\isacharequal}{\kern0pt}\ {\isacharparenleft}{\kern0pt}eq{\isacharunderscore}{\kern0pt}pred\ X\ {\isasymcirc}\isactrlsub c\ {\isasymlangle}x{\isacharcomma}{\kern0pt}\ y{\isasymrangle}\ {\isasymnoteq}\ {\isasymt}{\isacharparenright}{\kern0pt}{\isachardoublequoteclose}\isanewline
%
\isadelimproof
\ \ %
\endisadelimproof
%
\isatagproof
\isacommand{using}\isamarkupfalse%
\ assms\ eq{\isacharunderscore}{\kern0pt}pred{\isacharunderscore}{\kern0pt}iff{\isacharunderscore}{\kern0pt}eq\ \isacommand{by}\isamarkupfalse%
\ presburger%
\endisatagproof
{\isafoldproof}%
%
\isadelimproof
\isanewline
%
\endisadelimproof
\isanewline
\isacommand{lemma}\isamarkupfalse%
\ eq{\isacharunderscore}{\kern0pt}pred{\isacharunderscore}{\kern0pt}of{\isacharunderscore}{\kern0pt}monomorphism{\isacharcolon}{\kern0pt}\isanewline
\ \ \isakeyword{assumes}\ m{\isacharunderscore}{\kern0pt}type{\isacharbrackleft}{\kern0pt}type{\isacharunderscore}{\kern0pt}rule{\isacharbrackright}{\kern0pt}{\isacharcolon}{\kern0pt}\ {\isachardoublequoteopen}m\ {\isacharcolon}{\kern0pt}\ X\ {\isasymrightarrow}\ Y{\isachardoublequoteclose}\ \isakeyword{and}\ m{\isacharunderscore}{\kern0pt}mono{\isacharcolon}{\kern0pt}\ {\isachardoublequoteopen}monomorphism\ m{\isachardoublequoteclose}\isanewline
\ \ \isakeyword{shows}\ {\isachardoublequoteopen}eq{\isacharunderscore}{\kern0pt}pred\ Y\ {\isasymcirc}\isactrlsub c\ {\isacharparenleft}{\kern0pt}m\ {\isasymtimes}\isactrlsub f\ m{\isacharparenright}{\kern0pt}\ {\isacharequal}{\kern0pt}\ eq{\isacharunderscore}{\kern0pt}pred\ X{\isachardoublequoteclose}\isanewline
%
\isadelimproof
%
\endisadelimproof
%
\isatagproof
\isacommand{proof}\isamarkupfalse%
\ {\isacharparenleft}{\kern0pt}rule\ one{\isacharunderscore}{\kern0pt}separator{\isacharbrackleft}{\kern0pt}\isakeyword{where}\ X{\isacharequal}{\kern0pt}{\isachardoublequoteopen}X\ {\isasymtimes}\isactrlsub c\ X{\isachardoublequoteclose}{\isacharcomma}{\kern0pt}\ \isakeyword{where}\ Y{\isacharequal}{\kern0pt}{\isasymOmega}{\isacharbrackright}{\kern0pt}{\isacharparenright}{\kern0pt}\isanewline
\ \ \isacommand{show}\isamarkupfalse%
\ {\isachardoublequoteopen}eq{\isacharunderscore}{\kern0pt}pred\ Y\ {\isasymcirc}\isactrlsub c\ m\ {\isasymtimes}\isactrlsub f\ m\ {\isacharcolon}{\kern0pt}\ X\ {\isasymtimes}\isactrlsub c\ X\ {\isasymrightarrow}\ {\isasymOmega}{\isachardoublequoteclose}\isanewline
\ \ \ \ \isacommand{by}\isamarkupfalse%
\ typecheck{\isacharunderscore}{\kern0pt}cfuncs\isanewline
\ \ \isacommand{show}\isamarkupfalse%
\ {\isachardoublequoteopen}eq{\isacharunderscore}{\kern0pt}pred\ X\ {\isacharcolon}{\kern0pt}\ X\ {\isasymtimes}\isactrlsub c\ X\ {\isasymrightarrow}\ {\isasymOmega}{\isachardoublequoteclose}\isanewline
\ \ \ \ \isacommand{by}\isamarkupfalse%
\ typecheck{\isacharunderscore}{\kern0pt}cfuncs\isanewline
\isacommand{next}\isamarkupfalse%
\isanewline
\ \ \isacommand{fix}\isamarkupfalse%
\ x\isanewline
\ \ \isacommand{assume}\isamarkupfalse%
\ {\isachardoublequoteopen}x\ {\isasymin}\isactrlsub c\ X\ {\isasymtimes}\isactrlsub c\ X{\isachardoublequoteclose}\isanewline
\ \ \isacommand{then}\isamarkupfalse%
\ \isacommand{obtain}\isamarkupfalse%
\ x{\isadigit{1}}\ x{\isadigit{2}}\ \isakeyword{where}\ x{\isacharunderscore}{\kern0pt}def{\isacharcolon}{\kern0pt}\ {\isachardoublequoteopen}x\ {\isacharequal}{\kern0pt}\ {\isasymlangle}x{\isadigit{1}}{\isacharcomma}{\kern0pt}\ x{\isadigit{2}}{\isasymrangle}{\isachardoublequoteclose}\ \isakeyword{and}\ x{\isadigit{1}}{\isacharunderscore}{\kern0pt}type{\isacharbrackleft}{\kern0pt}type{\isacharunderscore}{\kern0pt}rule{\isacharbrackright}{\kern0pt}{\isacharcolon}{\kern0pt}\ {\isachardoublequoteopen}x{\isadigit{1}}\ {\isasymin}\isactrlsub c\ X{\isachardoublequoteclose}\ \isakeyword{and}\ x{\isadigit{2}}{\isacharunderscore}{\kern0pt}type{\isacharbrackleft}{\kern0pt}type{\isacharunderscore}{\kern0pt}rule{\isacharbrackright}{\kern0pt}{\isacharcolon}{\kern0pt}\ {\isachardoublequoteopen}x{\isadigit{2}}\ {\isasymin}\isactrlsub c\ X{\isachardoublequoteclose}\isanewline
\ \ \ \ \isacommand{using}\isamarkupfalse%
\ cart{\isacharunderscore}{\kern0pt}prod{\isacharunderscore}{\kern0pt}decomp\ \isacommand{by}\isamarkupfalse%
\ blast\isanewline
\ \ \isacommand{show}\isamarkupfalse%
\ {\isachardoublequoteopen}{\isacharparenleft}{\kern0pt}eq{\isacharunderscore}{\kern0pt}pred\ Y\ {\isasymcirc}\isactrlsub c\ m\ {\isasymtimes}\isactrlsub f\ m{\isacharparenright}{\kern0pt}\ {\isasymcirc}\isactrlsub c\ x\ {\isacharequal}{\kern0pt}\ eq{\isacharunderscore}{\kern0pt}pred\ X\ {\isasymcirc}\isactrlsub c\ x{\isachardoublequoteclose}\isanewline
\ \ \isacommand{proof}\isamarkupfalse%
\ {\isacharparenleft}{\kern0pt}unfold\ x{\isacharunderscore}{\kern0pt}def{\isacharcomma}{\kern0pt}\ cases\ {\isachardoublequoteopen}{\isacharparenleft}{\kern0pt}eq{\isacharunderscore}{\kern0pt}pred\ Y\ {\isasymcirc}\isactrlsub c\ m\ {\isasymtimes}\isactrlsub f\ m{\isacharparenright}{\kern0pt}\ {\isasymcirc}\isactrlsub c\ {\isasymlangle}x{\isadigit{1}}{\isacharcomma}{\kern0pt}x{\isadigit{2}}{\isasymrangle}\ {\isacharequal}{\kern0pt}\ {\isasymt}{\isachardoublequoteclose}{\isacharparenright}{\kern0pt}\isanewline
\ \ \ \ \isacommand{assume}\isamarkupfalse%
\ LHS{\isacharcolon}{\kern0pt}\ {\isachardoublequoteopen}{\isacharparenleft}{\kern0pt}eq{\isacharunderscore}{\kern0pt}pred\ Y\ {\isasymcirc}\isactrlsub c\ m\ {\isasymtimes}\isactrlsub f\ m{\isacharparenright}{\kern0pt}\ {\isasymcirc}\isactrlsub c\ {\isasymlangle}x{\isadigit{1}}{\isacharcomma}{\kern0pt}x{\isadigit{2}}{\isasymrangle}\ {\isacharequal}{\kern0pt}\ {\isasymt}{\isachardoublequoteclose}\isanewline
\ \ \ \ \isacommand{then}\isamarkupfalse%
\ \isacommand{have}\isamarkupfalse%
\ {\isachardoublequoteopen}eq{\isacharunderscore}{\kern0pt}pred\ Y\ {\isasymcirc}\isactrlsub c\ {\isacharparenleft}{\kern0pt}m\ {\isasymtimes}\isactrlsub f\ m{\isacharparenright}{\kern0pt}\ {\isasymcirc}\isactrlsub c\ {\isasymlangle}x{\isadigit{1}}{\isacharcomma}{\kern0pt}x{\isadigit{2}}{\isasymrangle}\ {\isacharequal}{\kern0pt}\ {\isasymt}{\isachardoublequoteclose}\isanewline
\ \ \ \ \ \ \isacommand{by}\isamarkupfalse%
\ {\isacharparenleft}{\kern0pt}typecheck{\isacharunderscore}{\kern0pt}cfuncs{\isacharcomma}{\kern0pt}\ simp\ add{\isacharcolon}{\kern0pt}\ comp{\isacharunderscore}{\kern0pt}associative{\isadigit{2}}{\isacharparenright}{\kern0pt}\isanewline
\ \ \ \ \isacommand{then}\isamarkupfalse%
\ \isacommand{have}\isamarkupfalse%
\ {\isachardoublequoteopen}eq{\isacharunderscore}{\kern0pt}pred\ Y\ {\isasymcirc}\isactrlsub c\ {\isasymlangle}m\ {\isasymcirc}\isactrlsub c\ x{\isadigit{1}}{\isacharcomma}{\kern0pt}\ m\ {\isasymcirc}\isactrlsub c\ x{\isadigit{2}}{\isasymrangle}\ {\isacharequal}{\kern0pt}\ {\isasymt}{\isachardoublequoteclose}\isanewline
\ \ \ \ \ \ \isacommand{by}\isamarkupfalse%
\ {\isacharparenleft}{\kern0pt}typecheck{\isacharunderscore}{\kern0pt}cfuncs{\isacharcomma}{\kern0pt}\ auto\ simp\ add{\isacharcolon}{\kern0pt}\ cfunc{\isacharunderscore}{\kern0pt}cross{\isacharunderscore}{\kern0pt}prod{\isacharunderscore}{\kern0pt}comp{\isacharunderscore}{\kern0pt}cfunc{\isacharunderscore}{\kern0pt}prod{\isacharparenright}{\kern0pt}\isanewline
\ \ \ \ \isacommand{then}\isamarkupfalse%
\ \isacommand{have}\isamarkupfalse%
\ {\isachardoublequoteopen}m\ {\isasymcirc}\isactrlsub c\ x{\isadigit{1}}\ {\isacharequal}{\kern0pt}\ m\ {\isasymcirc}\isactrlsub c\ x{\isadigit{2}}{\isachardoublequoteclose}\isanewline
\ \ \ \ \ \ \isacommand{by}\isamarkupfalse%
\ {\isacharparenleft}{\kern0pt}typecheck{\isacharunderscore}{\kern0pt}cfuncs{\isacharunderscore}{\kern0pt}prems{\isacharcomma}{\kern0pt}\ simp\ add{\isacharcolon}{\kern0pt}\ eq{\isacharunderscore}{\kern0pt}pred{\isacharunderscore}{\kern0pt}iff{\isacharunderscore}{\kern0pt}eq{\isacharparenright}{\kern0pt}\isanewline
\ \ \ \ \isacommand{then}\isamarkupfalse%
\ \isacommand{have}\isamarkupfalse%
\ {\isachardoublequoteopen}x{\isadigit{1}}\ {\isacharequal}{\kern0pt}\ x{\isadigit{2}}{\isachardoublequoteclose}\isanewline
\ \ \ \ \ \ \isacommand{using}\isamarkupfalse%
\ m{\isacharunderscore}{\kern0pt}mono\ m{\isacharunderscore}{\kern0pt}type\ monomorphism{\isacharunderscore}{\kern0pt}def{\isadigit{3}}\ x{\isadigit{1}}{\isacharunderscore}{\kern0pt}type\ x{\isadigit{2}}{\isacharunderscore}{\kern0pt}type\ \isacommand{by}\isamarkupfalse%
\ blast\isanewline
\ \ \ \ \isacommand{then}\isamarkupfalse%
\ \isacommand{have}\isamarkupfalse%
\ RHS{\isacharcolon}{\kern0pt}\ {\isachardoublequoteopen}eq{\isacharunderscore}{\kern0pt}pred\ X\ {\isasymcirc}\isactrlsub c\ {\isasymlangle}x{\isadigit{1}}{\isacharcomma}{\kern0pt}x{\isadigit{2}}{\isasymrangle}\ {\isacharequal}{\kern0pt}\ {\isasymt}{\isachardoublequoteclose}\isanewline
\ \ \ \ \ \ \isacommand{by}\isamarkupfalse%
\ {\isacharparenleft}{\kern0pt}typecheck{\isacharunderscore}{\kern0pt}cfuncs{\isacharcomma}{\kern0pt}\ insert\ eq{\isacharunderscore}{\kern0pt}pred{\isacharunderscore}{\kern0pt}iff{\isacharunderscore}{\kern0pt}eq{\isacharcomma}{\kern0pt}\ blast{\isacharparenright}{\kern0pt}\isanewline
\ \ \ \ \isacommand{show}\isamarkupfalse%
\ {\isachardoublequoteopen}{\isacharparenleft}{\kern0pt}eq{\isacharunderscore}{\kern0pt}pred\ Y\ {\isasymcirc}\isactrlsub c\ m\ {\isasymtimes}\isactrlsub f\ m{\isacharparenright}{\kern0pt}\ {\isasymcirc}\isactrlsub c\ {\isasymlangle}x{\isadigit{1}}{\isacharcomma}{\kern0pt}x{\isadigit{2}}{\isasymrangle}\ {\isacharequal}{\kern0pt}\ eq{\isacharunderscore}{\kern0pt}pred\ X\ {\isasymcirc}\isactrlsub c\ {\isasymlangle}x{\isadigit{1}}{\isacharcomma}{\kern0pt}x{\isadigit{2}}{\isasymrangle}{\isachardoublequoteclose}\isanewline
\ \ \ \ \ \ \isacommand{using}\isamarkupfalse%
\ LHS\ RHS\ \isacommand{by}\isamarkupfalse%
\ auto\isanewline
\ \ \isacommand{next}\isamarkupfalse%
\isanewline
\ \ \ \ \isacommand{assume}\isamarkupfalse%
\ {\isachardoublequoteopen}{\isacharparenleft}{\kern0pt}eq{\isacharunderscore}{\kern0pt}pred\ Y\ {\isasymcirc}\isactrlsub c\ m\ {\isasymtimes}\isactrlsub f\ m{\isacharparenright}{\kern0pt}\ {\isasymcirc}\isactrlsub c\ {\isasymlangle}x{\isadigit{1}}{\isacharcomma}{\kern0pt}x{\isadigit{2}}{\isasymrangle}\ {\isasymnoteq}\ {\isasymt}{\isachardoublequoteclose}\isanewline
\ \ \ \ \isacommand{then}\isamarkupfalse%
\ \isacommand{have}\isamarkupfalse%
\ LHS{\isacharcolon}{\kern0pt}\ {\isachardoublequoteopen}{\isacharparenleft}{\kern0pt}eq{\isacharunderscore}{\kern0pt}pred\ Y\ {\isasymcirc}\isactrlsub c\ m\ {\isasymtimes}\isactrlsub f\ m{\isacharparenright}{\kern0pt}\ {\isasymcirc}\isactrlsub c\ {\isasymlangle}x{\isadigit{1}}{\isacharcomma}{\kern0pt}x{\isadigit{2}}{\isasymrangle}\ {\isacharequal}{\kern0pt}\ {\isasymf}{\isachardoublequoteclose}\isanewline
\ \ \ \ \ \ \isacommand{by}\isamarkupfalse%
\ {\isacharparenleft}{\kern0pt}typecheck{\isacharunderscore}{\kern0pt}cfuncs{\isacharcomma}{\kern0pt}\ meson\ true{\isacharunderscore}{\kern0pt}false{\isacharunderscore}{\kern0pt}only{\isacharunderscore}{\kern0pt}truth{\isacharunderscore}{\kern0pt}values{\isacharparenright}{\kern0pt}\isanewline
\ \ \ \ \isacommand{then}\isamarkupfalse%
\ \isacommand{have}\isamarkupfalse%
\ {\isachardoublequoteopen}eq{\isacharunderscore}{\kern0pt}pred\ Y\ {\isasymcirc}\isactrlsub c\ {\isacharparenleft}{\kern0pt}m\ {\isasymtimes}\isactrlsub f\ m{\isacharparenright}{\kern0pt}\ {\isasymcirc}\isactrlsub c\ {\isasymlangle}x{\isadigit{1}}{\isacharcomma}{\kern0pt}x{\isadigit{2}}{\isasymrangle}\ {\isacharequal}{\kern0pt}\ {\isasymf}{\isachardoublequoteclose}\isanewline
\ \ \ \ \ \ \isacommand{by}\isamarkupfalse%
\ {\isacharparenleft}{\kern0pt}typecheck{\isacharunderscore}{\kern0pt}cfuncs{\isacharcomma}{\kern0pt}\ simp\ add{\isacharcolon}{\kern0pt}\ comp{\isacharunderscore}{\kern0pt}associative{\isadigit{2}}{\isacharparenright}{\kern0pt}\isanewline
\ \ \ \ \isacommand{then}\isamarkupfalse%
\ \isacommand{have}\isamarkupfalse%
\ {\isachardoublequoteopen}eq{\isacharunderscore}{\kern0pt}pred\ Y\ {\isasymcirc}\isactrlsub c\ {\isasymlangle}m\ {\isasymcirc}\isactrlsub c\ x{\isadigit{1}}{\isacharcomma}{\kern0pt}\ m\ {\isasymcirc}\isactrlsub c\ x{\isadigit{2}}{\isasymrangle}\ {\isacharequal}{\kern0pt}\ {\isasymf}{\isachardoublequoteclose}\isanewline
\ \ \ \ \ \ \isacommand{by}\isamarkupfalse%
\ {\isacharparenleft}{\kern0pt}typecheck{\isacharunderscore}{\kern0pt}cfuncs{\isacharcomma}{\kern0pt}\ auto\ simp\ add{\isacharcolon}{\kern0pt}\ cfunc{\isacharunderscore}{\kern0pt}cross{\isacharunderscore}{\kern0pt}prod{\isacharunderscore}{\kern0pt}comp{\isacharunderscore}{\kern0pt}cfunc{\isacharunderscore}{\kern0pt}prod{\isacharparenright}{\kern0pt}\isanewline
\ \ \ \ \isacommand{then}\isamarkupfalse%
\ \isacommand{have}\isamarkupfalse%
\ {\isachardoublequoteopen}m\ {\isasymcirc}\isactrlsub c\ x{\isadigit{1}}\ {\isasymnoteq}\ m\ {\isasymcirc}\isactrlsub c\ x{\isadigit{2}}{\isachardoublequoteclose}\isanewline
\ \ \ \ \ \ \isacommand{using}\isamarkupfalse%
\ eq{\isacharunderscore}{\kern0pt}pred{\isacharunderscore}{\kern0pt}iff{\isacharunderscore}{\kern0pt}eq{\isacharunderscore}{\kern0pt}conv\ \isacommand{by}\isamarkupfalse%
\ {\isacharparenleft}{\kern0pt}typecheck{\isacharunderscore}{\kern0pt}cfuncs{\isacharunderscore}{\kern0pt}prems{\isacharcomma}{\kern0pt}\ blast{\isacharparenright}{\kern0pt}\isanewline
\ \ \ \ \isacommand{then}\isamarkupfalse%
\ \isacommand{have}\isamarkupfalse%
\ {\isachardoublequoteopen}x{\isadigit{1}}\ {\isasymnoteq}\ x{\isadigit{2}}{\isachardoublequoteclose}\isanewline
\ \ \ \ \ \ \isacommand{by}\isamarkupfalse%
\ auto\isanewline
\ \ \ \ \isacommand{then}\isamarkupfalse%
\ \isacommand{have}\isamarkupfalse%
\ RHS{\isacharcolon}{\kern0pt}\ {\isachardoublequoteopen}eq{\isacharunderscore}{\kern0pt}pred\ X\ {\isasymcirc}\isactrlsub c\ {\isasymlangle}x{\isadigit{1}}{\isacharcomma}{\kern0pt}x{\isadigit{2}}{\isasymrangle}\ {\isacharequal}{\kern0pt}\ {\isasymf}{\isachardoublequoteclose}\isanewline
\ \ \ \ \ \ \isacommand{using}\isamarkupfalse%
\ eq{\isacharunderscore}{\kern0pt}pred{\isacharunderscore}{\kern0pt}iff{\isacharunderscore}{\kern0pt}eq{\isacharunderscore}{\kern0pt}conv\ \isacommand{by}\isamarkupfalse%
\ {\isacharparenleft}{\kern0pt}typecheck{\isacharunderscore}{\kern0pt}cfuncs{\isacharcomma}{\kern0pt}\ blast{\isacharparenright}{\kern0pt}\isanewline
\ \ \ \ \isacommand{show}\isamarkupfalse%
\ {\isachardoublequoteopen}{\isacharparenleft}{\kern0pt}eq{\isacharunderscore}{\kern0pt}pred\ Y\ {\isasymcirc}\isactrlsub c\ m\ {\isasymtimes}\isactrlsub f\ m{\isacharparenright}{\kern0pt}\ {\isasymcirc}\isactrlsub c\ {\isasymlangle}x{\isadigit{1}}{\isacharcomma}{\kern0pt}x{\isadigit{2}}{\isasymrangle}\ {\isacharequal}{\kern0pt}\ eq{\isacharunderscore}{\kern0pt}pred\ X\ {\isasymcirc}\isactrlsub c\ {\isasymlangle}x{\isadigit{1}}{\isacharcomma}{\kern0pt}x{\isadigit{2}}{\isasymrangle}{\isachardoublequoteclose}\isanewline
\ \ \ \ \ \ \isacommand{using}\isamarkupfalse%
\ LHS\ RHS\ \isacommand{by}\isamarkupfalse%
\ auto\isanewline
\ \ \isacommand{qed}\isamarkupfalse%
\isanewline
\isacommand{qed}\isamarkupfalse%
%
\endisatagproof
{\isafoldproof}%
%
\isadelimproof
\isanewline
%
\endisadelimproof
\isanewline
\isacommand{lemma}\isamarkupfalse%
\ eq{\isacharunderscore}{\kern0pt}pred{\isacharunderscore}{\kern0pt}true{\isacharunderscore}{\kern0pt}extract{\isacharunderscore}{\kern0pt}right{\isacharcolon}{\kern0pt}\ \isanewline
\ \ \ \ \isakeyword{assumes}\ {\isachardoublequoteopen}x\ {\isasymin}\isactrlsub c\ X{\isachardoublequoteclose}\ \isanewline
\ \ \ \ \isakeyword{shows}\ \ {\isachardoublequoteopen}eq{\isacharunderscore}{\kern0pt}pred\ X\ {\isasymcirc}\isactrlsub c\ {\isasymlangle}x\ {\isasymcirc}\isactrlsub c\ {\isasymbeta}\isactrlbsub X\isactrlesub {\isacharcomma}{\kern0pt}\ id\ X{\isasymrangle}\ {\isasymcirc}\isactrlsub c\ x\ {\isacharequal}{\kern0pt}\ {\isasymt}{\isachardoublequoteclose}\isanewline
%
\isadelimproof
\ \ \ \ %
\endisadelimproof
%
\isatagproof
\isacommand{using}\isamarkupfalse%
\ assms\ cart{\isacharunderscore}{\kern0pt}prod{\isacharunderscore}{\kern0pt}extract{\isacharunderscore}{\kern0pt}right\ eq{\isacharunderscore}{\kern0pt}pred{\isacharunderscore}{\kern0pt}iff{\isacharunderscore}{\kern0pt}eq\ \isacommand{by}\isamarkupfalse%
\ fastforce%
\endisatagproof
{\isafoldproof}%
%
\isadelimproof
\isanewline
%
\endisadelimproof
\isanewline
\isacommand{lemma}\isamarkupfalse%
\ eq{\isacharunderscore}{\kern0pt}pred{\isacharunderscore}{\kern0pt}false{\isacharunderscore}{\kern0pt}extract{\isacharunderscore}{\kern0pt}right{\isacharcolon}{\kern0pt}\ \isanewline
\ \ \ \ \isakeyword{assumes}\ {\isachardoublequoteopen}x\ {\isasymin}\isactrlsub c\ X{\isachardoublequoteclose}\ \ {\isachardoublequoteopen}y\ {\isasymin}\isactrlsub c\ X{\isachardoublequoteclose}\ {\isachardoublequoteopen}x\ {\isasymnoteq}\ y{\isachardoublequoteclose}\isanewline
\ \ \ \ \isakeyword{shows}\ \ {\isachardoublequoteopen}eq{\isacharunderscore}{\kern0pt}pred\ X\ {\isasymcirc}\isactrlsub c\ {\isasymlangle}x\ {\isasymcirc}\isactrlsub c\ {\isasymbeta}\isactrlbsub X\isactrlesub {\isacharcomma}{\kern0pt}\ id\ X{\isasymrangle}\ {\isasymcirc}\isactrlsub c\ y\ {\isacharequal}{\kern0pt}\ {\isasymf}{\isachardoublequoteclose}\isanewline
%
\isadelimproof
\ \ \ \ %
\endisadelimproof
%
\isatagproof
\isacommand{using}\isamarkupfalse%
\ assms\ cart{\isacharunderscore}{\kern0pt}prod{\isacharunderscore}{\kern0pt}extract{\isacharunderscore}{\kern0pt}right\ eq{\isacharunderscore}{\kern0pt}pred{\isacharunderscore}{\kern0pt}iff{\isacharunderscore}{\kern0pt}eq\ true{\isacharunderscore}{\kern0pt}false{\isacharunderscore}{\kern0pt}only{\isacharunderscore}{\kern0pt}truth{\isacharunderscore}{\kern0pt}values\ \ \isacommand{by}\isamarkupfalse%
\ {\isacharparenleft}{\kern0pt}typecheck{\isacharunderscore}{\kern0pt}cfuncs{\isacharcomma}{\kern0pt}\ fastforce{\isacharparenright}{\kern0pt}%
\endisatagproof
{\isafoldproof}%
%
\isadelimproof
%
\endisadelimproof
%
\isadelimdocument
%
\endisadelimdocument
%
\isatagdocument
%
\isamarkupsubsection{Properties of Monomorphisms and Epimorphisms%
}
\isamarkuptrue%
%
\endisatagdocument
{\isafolddocument}%
%
\isadelimdocument
%
\endisadelimdocument
%
\begin{isamarkuptext}%
The lemma below corresponds to Exercise 2.2.3 in Halvorson.%
\end{isamarkuptext}\isamarkuptrue%
\isacommand{lemma}\isamarkupfalse%
\ regmono{\isacharunderscore}{\kern0pt}is{\isacharunderscore}{\kern0pt}mono{\isacharcolon}{\kern0pt}\ {\isachardoublequoteopen}regular{\isacharunderscore}{\kern0pt}monomorphism\ m\ {\isasymLongrightarrow}\ monomorphism\ m{\isachardoublequoteclose}\isanewline
%
\isadelimproof
\ \ %
\endisadelimproof
%
\isatagproof
\isacommand{using}\isamarkupfalse%
\ equalizer{\isacharunderscore}{\kern0pt}is{\isacharunderscore}{\kern0pt}monomorphism\ regular{\isacharunderscore}{\kern0pt}monomorphism{\isacharunderscore}{\kern0pt}def\ \isacommand{by}\isamarkupfalse%
\ blast%
\endisatagproof
{\isafoldproof}%
%
\isadelimproof
%
\endisadelimproof
%
\begin{isamarkuptext}%
The lemma below corresponds to Proposition 2.2.4 in Halvorson.%
\end{isamarkuptext}\isamarkuptrue%
\isacommand{lemma}\isamarkupfalse%
\ mono{\isacharunderscore}{\kern0pt}is{\isacharunderscore}{\kern0pt}regmono{\isacharcolon}{\kern0pt}\isanewline
\ \ \isakeyword{shows}\ {\isachardoublequoteopen}monomorphism\ m\ {\isasymLongrightarrow}\ regular{\isacharunderscore}{\kern0pt}monomorphism\ m{\isachardoublequoteclose}\isanewline
%
\isadelimproof
\ \ %
\endisadelimproof
%
\isatagproof
\isacommand{unfolding}\isamarkupfalse%
\ monomorphism{\isacharunderscore}{\kern0pt}def\ regular{\isacharunderscore}{\kern0pt}monomorphism{\isacharunderscore}{\kern0pt}def\isanewline
\ \ \isacommand{using}\isamarkupfalse%
\ cfunc{\isacharunderscore}{\kern0pt}type{\isacharunderscore}{\kern0pt}def\ characteristic{\isacharunderscore}{\kern0pt}func{\isacharunderscore}{\kern0pt}type\ monomorphism{\isacharunderscore}{\kern0pt}def\ domain{\isacharunderscore}{\kern0pt}comp\ terminal{\isacharunderscore}{\kern0pt}func{\isacharunderscore}{\kern0pt}type\ true{\isacharunderscore}{\kern0pt}func{\isacharunderscore}{\kern0pt}type\ monomorphism{\isacharunderscore}{\kern0pt}equalizes{\isacharunderscore}{\kern0pt}char{\isacharunderscore}{\kern0pt}func\isanewline
\ \ \isacommand{by}\isamarkupfalse%
\ {\isacharparenleft}{\kern0pt}rule{\isacharunderscore}{\kern0pt}tac\ x{\isacharequal}{\kern0pt}{\isachardoublequoteopen}characteristic{\isacharunderscore}{\kern0pt}func\ m{\isachardoublequoteclose}\ \isakeyword{in}\ exI{\isacharcomma}{\kern0pt}\ rule{\isacharunderscore}{\kern0pt}tac\ x{\isacharequal}{\kern0pt}{\isachardoublequoteopen}{\isasymt}\ {\isasymcirc}\isactrlsub c\ {\isasymbeta}\isactrlbsub codomain{\isacharparenleft}{\kern0pt}m{\isacharparenright}{\kern0pt}\isactrlesub {\isachardoublequoteclose}\ \isakeyword{in}\ exI{\isacharcomma}{\kern0pt}\ auto{\isacharparenright}{\kern0pt}%
\endisatagproof
{\isafoldproof}%
%
\isadelimproof
%
\endisadelimproof
%
\begin{isamarkuptext}%
The lemma below corresponds to Proposition 2.2.5 in Halvorson.%
\end{isamarkuptext}\isamarkuptrue%
\isacommand{lemma}\isamarkupfalse%
\ epi{\isacharunderscore}{\kern0pt}mon{\isacharunderscore}{\kern0pt}is{\isacharunderscore}{\kern0pt}iso{\isacharcolon}{\kern0pt}\isanewline
\ \ \isakeyword{assumes}\ {\isachardoublequoteopen}epimorphism\ f{\isachardoublequoteclose}\ {\isachardoublequoteopen}monomorphism\ f{\isachardoublequoteclose}\isanewline
\ \ \isakeyword{shows}\ {\isachardoublequoteopen}isomorphism\ f{\isachardoublequoteclose}\isanewline
%
\isadelimproof
\ \ %
\endisadelimproof
%
\isatagproof
\isacommand{using}\isamarkupfalse%
\ assms\ epi{\isacharunderscore}{\kern0pt}regmon{\isacharunderscore}{\kern0pt}is{\isacharunderscore}{\kern0pt}iso\ mono{\isacharunderscore}{\kern0pt}is{\isacharunderscore}{\kern0pt}regmono\ \isacommand{by}\isamarkupfalse%
\ auto%
\endisatagproof
{\isafoldproof}%
%
\isadelimproof
%
\endisadelimproof
%
\begin{isamarkuptext}%
The lemma below corresponds to Proposition 2.2.8 in Halvorson.%
\end{isamarkuptext}\isamarkuptrue%
\isacommand{lemma}\isamarkupfalse%
\ epi{\isacharunderscore}{\kern0pt}is{\isacharunderscore}{\kern0pt}surj{\isacharcolon}{\kern0pt}\isanewline
\ \ \isakeyword{assumes}\ {\isachardoublequoteopen}p{\isacharcolon}{\kern0pt}\ X\ {\isasymrightarrow}\ Y{\isachardoublequoteclose}\ {\isachardoublequoteopen}epimorphism\ p{\isachardoublequoteclose}\isanewline
\ \ \isakeyword{shows}\ {\isachardoublequoteopen}surjective\ p{\isachardoublequoteclose}\isanewline
%
\isadelimproof
\ \ %
\endisadelimproof
%
\isatagproof
\isacommand{unfolding}\isamarkupfalse%
\ surjective{\isacharunderscore}{\kern0pt}def\isanewline
\isacommand{proof}\isamarkupfalse%
{\isacharparenleft}{\kern0pt}rule\ ccontr{\isacharparenright}{\kern0pt}\isanewline
\ \ \isacommand{assume}\isamarkupfalse%
\ a{\isadigit{1}}{\isacharcolon}{\kern0pt}\ {\isachardoublequoteopen}{\isasymnot}\ {\isacharparenleft}{\kern0pt}{\isasymforall}y{\isachardot}{\kern0pt}\ y\ {\isasymin}\isactrlsub c\ codomain\ p\ {\isasymlongrightarrow}\ {\isacharparenleft}{\kern0pt}{\isasymexists}x{\isachardot}{\kern0pt}\ x\ {\isasymin}\isactrlsub c\ domain\ p\ {\isasymand}\ p\ {\isasymcirc}\isactrlsub c\ x\ {\isacharequal}{\kern0pt}\ y{\isacharparenright}{\kern0pt}{\isacharparenright}{\kern0pt}{\isachardoublequoteclose}\isanewline
\ \ \isacommand{have}\isamarkupfalse%
\ {\isachardoublequoteopen}{\isasymexists}y{\isachardot}{\kern0pt}\ y\ {\isasymin}\isactrlsub c\ Y\ {\isasymand}\ {\isasymnot}{\isacharparenleft}{\kern0pt}{\isasymexists}x{\isachardot}{\kern0pt}\ x\ {\isasymin}\isactrlsub c\ X\ {\isasymand}\ p\ {\isasymcirc}\isactrlsub c\ x\ {\isacharequal}{\kern0pt}\ y{\isacharparenright}{\kern0pt}{\isachardoublequoteclose}\isanewline
\ \ \ \ \isacommand{using}\isamarkupfalse%
\ a{\isadigit{1}}\ assms{\isacharparenleft}{\kern0pt}{\isadigit{1}}{\isacharparenright}{\kern0pt}\ cfunc{\isacharunderscore}{\kern0pt}type{\isacharunderscore}{\kern0pt}def\ \isacommand{by}\isamarkupfalse%
\ auto\isanewline
\ \ \isacommand{then}\isamarkupfalse%
\ \isacommand{obtain}\isamarkupfalse%
\ y{\isadigit{0}}\ \isakeyword{where}\ y{\isacharunderscore}{\kern0pt}def{\isacharcolon}{\kern0pt}\ {\isachardoublequoteopen}y{\isadigit{0}}\ {\isasymin}\isactrlsub c\ Y\ {\isasymand}\ {\isacharparenleft}{\kern0pt}{\isasymforall}x{\isachardot}{\kern0pt}\ x\ {\isasymin}\isactrlsub c\ X\ {\isasymlongrightarrow}\ p\ {\isasymcirc}\isactrlsub c\ x\ {\isasymnoteq}\ y{\isadigit{0}}{\isacharparenright}{\kern0pt}{\isachardoublequoteclose}\isanewline
\ \ \ \ \isacommand{by}\isamarkupfalse%
\ auto\isanewline
\ \ \isacommand{have}\isamarkupfalse%
\ mono{\isacharcolon}{\kern0pt}\ {\isachardoublequoteopen}monomorphism\ y{\isadigit{0}}{\isachardoublequoteclose}\isanewline
\ \ \ \ \isacommand{using}\isamarkupfalse%
\ element{\isacharunderscore}{\kern0pt}monomorphism\ y{\isacharunderscore}{\kern0pt}def\ \isacommand{by}\isamarkupfalse%
\ blast\isanewline
\ \ \isacommand{obtain}\isamarkupfalse%
\ g\ \isakeyword{where}\ g{\isacharunderscore}{\kern0pt}def{\isacharcolon}{\kern0pt}\ {\isachardoublequoteopen}g\ {\isacharequal}{\kern0pt}\ eq{\isacharunderscore}{\kern0pt}pred\ Y\ {\isasymcirc}\isactrlsub c\ {\isasymlangle}y{\isadigit{0}}\ {\isasymcirc}\isactrlsub c\ {\isasymbeta}\isactrlbsub Y\isactrlesub {\isacharcomma}{\kern0pt}\ id\ Y{\isasymrangle}{\isachardoublequoteclose}\isanewline
\ \ \ \ \isacommand{by}\isamarkupfalse%
\ simp\isanewline
\ \ \isacommand{have}\isamarkupfalse%
\ g{\isacharunderscore}{\kern0pt}right{\isacharunderscore}{\kern0pt}arg{\isacharunderscore}{\kern0pt}type{\isacharcolon}{\kern0pt}\ {\isachardoublequoteopen}{\isasymlangle}y{\isadigit{0}}\ {\isasymcirc}\isactrlsub c\ {\isasymbeta}\isactrlbsub Y\isactrlesub {\isacharcomma}{\kern0pt}\ id\ Y{\isasymrangle}\ {\isacharcolon}{\kern0pt}\ Y\ {\isasymrightarrow}\ Y\ {\isasymtimes}\isactrlsub c\ Y{\isachardoublequoteclose}\isanewline
\ \ \ \ \isacommand{by}\isamarkupfalse%
\ {\isacharparenleft}{\kern0pt}meson\ cfunc{\isacharunderscore}{\kern0pt}prod{\isacharunderscore}{\kern0pt}type\ comp{\isacharunderscore}{\kern0pt}type\ id{\isacharunderscore}{\kern0pt}type\ terminal{\isacharunderscore}{\kern0pt}func{\isacharunderscore}{\kern0pt}type\ y{\isacharunderscore}{\kern0pt}def{\isacharparenright}{\kern0pt}\isanewline
\ \ \isacommand{then}\isamarkupfalse%
\ \isacommand{have}\isamarkupfalse%
\ g{\isacharunderscore}{\kern0pt}type{\isacharbrackleft}{\kern0pt}type{\isacharunderscore}{\kern0pt}rule{\isacharbrackright}{\kern0pt}{\isacharcolon}{\kern0pt}\ {\isachardoublequoteopen}g{\isacharcolon}{\kern0pt}\ Y\ {\isasymrightarrow}\ {\isasymOmega}{\isachardoublequoteclose}\isanewline
\ \ \ \ \isacommand{using}\isamarkupfalse%
\ comp{\isacharunderscore}{\kern0pt}type\ eq{\isacharunderscore}{\kern0pt}pred{\isacharunderscore}{\kern0pt}type\ g{\isacharunderscore}{\kern0pt}def\ \isacommand{by}\isamarkupfalse%
\ blast\isanewline
\isanewline
\ \ \isacommand{have}\isamarkupfalse%
\ gpx{\isacharunderscore}{\kern0pt}Eqs{\isacharunderscore}{\kern0pt}f{\isacharcolon}{\kern0pt}\ {\isachardoublequoteopen}{\isasymforall}x{\isachardot}{\kern0pt}\ x\ {\isasymin}\isactrlsub c\ X\ {\isasymlongrightarrow}\ g\ {\isasymcirc}\isactrlsub c\ p\ {\isasymcirc}\isactrlsub c\ x\ {\isacharequal}{\kern0pt}\ {\isasymf}{\isachardoublequoteclose}\isanewline
\ \ \isacommand{proof}\isamarkupfalse%
{\isacharparenleft}{\kern0pt}rule\ ccontr{\isacharparenright}{\kern0pt}\isanewline
\ \ \ \ \isacommand{assume}\isamarkupfalse%
\ {\isachardoublequoteopen}{\isasymnot}\ {\isacharparenleft}{\kern0pt}{\isasymforall}x{\isachardot}{\kern0pt}\ x\ {\isasymin}\isactrlsub c\ X\ {\isasymlongrightarrow}\ g\ {\isasymcirc}\isactrlsub c\ p\ {\isasymcirc}\isactrlsub c\ x\ {\isacharequal}{\kern0pt}\ {\isasymf}{\isacharparenright}{\kern0pt}{\isachardoublequoteclose}\isanewline
\ \ \ \ \isacommand{then}\isamarkupfalse%
\ \isacommand{obtain}\isamarkupfalse%
\ x\ \isakeyword{where}\ x{\isacharunderscore}{\kern0pt}type{\isacharcolon}{\kern0pt}\ {\isachardoublequoteopen}x\ {\isasymin}\isactrlsub c\ X{\isachardoublequoteclose}\ \isakeyword{and}\ bwoc{\isacharcolon}{\kern0pt}\ {\isachardoublequoteopen}g\ {\isasymcirc}\isactrlsub c\ p\ {\isasymcirc}\isactrlsub c\ x\ {\isasymnoteq}\ {\isasymf}{\isachardoublequoteclose}\isanewline
\ \ \ \ \ \ \isacommand{by}\isamarkupfalse%
\ blast\isanewline
\ \ \ \ \isanewline
\ \ \ \ \isacommand{show}\isamarkupfalse%
\ False\isanewline
\ \ \ \ \ \ \isacommand{by}\isamarkupfalse%
\ {\isacharparenleft}{\kern0pt}smt\ {\isacharparenleft}{\kern0pt}verit{\isacharparenright}{\kern0pt}\ assms{\isacharparenleft}{\kern0pt}{\isadigit{1}}{\isacharparenright}{\kern0pt}\ bwoc\ cfunc{\isacharunderscore}{\kern0pt}type{\isacharunderscore}{\kern0pt}def\ comp{\isacharunderscore}{\kern0pt}associative\ comp{\isacharunderscore}{\kern0pt}type\ eq{\isacharunderscore}{\kern0pt}pred{\isacharunderscore}{\kern0pt}false{\isacharunderscore}{\kern0pt}extract{\isacharunderscore}{\kern0pt}right\ eq{\isacharunderscore}{\kern0pt}pred{\isacharunderscore}{\kern0pt}type\ g{\isacharunderscore}{\kern0pt}def\ g{\isacharunderscore}{\kern0pt}right{\isacharunderscore}{\kern0pt}arg{\isacharunderscore}{\kern0pt}type\ x{\isacharunderscore}{\kern0pt}type\ y{\isacharunderscore}{\kern0pt}def{\isacharparenright}{\kern0pt}\isanewline
\ \ \isacommand{qed}\isamarkupfalse%
\isanewline
\ \ \isacommand{obtain}\isamarkupfalse%
\ h\ \isakeyword{where}\ h{\isacharunderscore}{\kern0pt}def{\isacharcolon}{\kern0pt}\ {\isachardoublequoteopen}h\ {\isacharequal}{\kern0pt}\ {\isasymf}\ {\isasymcirc}\isactrlsub c\ {\isasymbeta}\isactrlbsub Y\isactrlesub {\isachardoublequoteclose}\ \isakeyword{and}\ h{\isacharunderscore}{\kern0pt}type{\isacharbrackleft}{\kern0pt}type{\isacharunderscore}{\kern0pt}rule{\isacharbrackright}{\kern0pt}{\isacharcolon}{\kern0pt}{\isachardoublequoteopen}h{\isacharcolon}{\kern0pt}\ Y\ {\isasymrightarrow}\ {\isasymOmega}{\isachardoublequoteclose}\isanewline
\ \ \ \ \isacommand{by}\isamarkupfalse%
\ {\isacharparenleft}{\kern0pt}typecheck{\isacharunderscore}{\kern0pt}cfuncs{\isacharcomma}{\kern0pt}\ simp{\isacharparenright}{\kern0pt}\isanewline
\ \ \isacommand{have}\isamarkupfalse%
\ hpx{\isacharunderscore}{\kern0pt}eqs{\isacharunderscore}{\kern0pt}f{\isacharcolon}{\kern0pt}\ {\isachardoublequoteopen}{\isasymforall}x{\isachardot}{\kern0pt}\ x\ {\isasymin}\isactrlsub c\ X\ {\isasymlongrightarrow}\ h\ {\isasymcirc}\isactrlsub c\ p\ {\isasymcirc}\isactrlsub c\ x\ {\isacharequal}{\kern0pt}\ {\isasymf}{\isachardoublequoteclose}\isanewline
\ \ \ \ \isacommand{by}\isamarkupfalse%
\ {\isacharparenleft}{\kern0pt}smt\ assms{\isacharparenleft}{\kern0pt}{\isadigit{1}}{\isacharparenright}{\kern0pt}\ cfunc{\isacharunderscore}{\kern0pt}type{\isacharunderscore}{\kern0pt}def\ codomain{\isacharunderscore}{\kern0pt}comp\ comp{\isacharunderscore}{\kern0pt}associative\ false{\isacharunderscore}{\kern0pt}func{\isacharunderscore}{\kern0pt}type\ h{\isacharunderscore}{\kern0pt}def\ id{\isacharunderscore}{\kern0pt}right{\isacharunderscore}{\kern0pt}unit{\isadigit{2}}\ id{\isacharunderscore}{\kern0pt}type\ terminal{\isacharunderscore}{\kern0pt}func{\isacharunderscore}{\kern0pt}comp\ terminal{\isacharunderscore}{\kern0pt}func{\isacharunderscore}{\kern0pt}type\ terminal{\isacharunderscore}{\kern0pt}func{\isacharunderscore}{\kern0pt}unique{\isacharparenright}{\kern0pt}\isanewline
\ \ \isacommand{have}\isamarkupfalse%
\ gp{\isacharunderscore}{\kern0pt}eqs{\isacharunderscore}{\kern0pt}hp{\isacharcolon}{\kern0pt}\ {\isachardoublequoteopen}g\ {\isasymcirc}\isactrlsub c\ p\ {\isacharequal}{\kern0pt}\ h\ {\isasymcirc}\isactrlsub c\ p{\isachardoublequoteclose}\isanewline
\ \ \isacommand{proof}\isamarkupfalse%
{\isacharparenleft}{\kern0pt}rule\ one{\isacharunderscore}{\kern0pt}separator{\isacharbrackleft}{\kern0pt}\isakeyword{where}\ X{\isacharequal}{\kern0pt}X{\isacharcomma}{\kern0pt}\isakeyword{where}\ Y{\isacharequal}{\kern0pt}{\isasymOmega}{\isacharbrackright}{\kern0pt}{\isacharparenright}{\kern0pt}\isanewline
\ \ \ \ \isacommand{show}\isamarkupfalse%
\ {\isachardoublequoteopen}g\ {\isasymcirc}\isactrlsub c\ p\ {\isacharcolon}{\kern0pt}\ X\ {\isasymrightarrow}\ {\isasymOmega}{\isachardoublequoteclose}\isanewline
\ \ \ \ \ \ \isacommand{using}\isamarkupfalse%
\ assms\ \isacommand{by}\isamarkupfalse%
\ typecheck{\isacharunderscore}{\kern0pt}cfuncs\isanewline
\ \ \ \ \isacommand{show}\isamarkupfalse%
\ {\isachardoublequoteopen}h\ {\isasymcirc}\isactrlsub c\ p\ {\isacharcolon}{\kern0pt}\ X\ {\isasymrightarrow}\ {\isasymOmega}{\isachardoublequoteclose}\isanewline
\ \ \ \ \ \ \isacommand{using}\isamarkupfalse%
\ assms\ \isacommand{by}\isamarkupfalse%
\ typecheck{\isacharunderscore}{\kern0pt}cfuncs\isanewline
\ \ \ \ \isacommand{show}\isamarkupfalse%
\ {\isachardoublequoteopen}{\isasymAnd}x{\isachardot}{\kern0pt}\ x\ {\isasymin}\isactrlsub c\ X\ {\isasymLongrightarrow}\ {\isacharparenleft}{\kern0pt}g\ {\isasymcirc}\isactrlsub c\ p{\isacharparenright}{\kern0pt}\ {\isasymcirc}\isactrlsub c\ x\ {\isacharequal}{\kern0pt}\ {\isacharparenleft}{\kern0pt}h\ {\isasymcirc}\isactrlsub c\ p{\isacharparenright}{\kern0pt}\ {\isasymcirc}\isactrlsub c\ x{\isachardoublequoteclose}\isanewline
\ \ \ \ \ \ \isacommand{using}\isamarkupfalse%
\ assms{\isacharparenleft}{\kern0pt}{\isadigit{1}}{\isacharparenright}{\kern0pt}\ comp{\isacharunderscore}{\kern0pt}associative{\isadigit{2}}\ g{\isacharunderscore}{\kern0pt}type\ gpx{\isacharunderscore}{\kern0pt}Eqs{\isacharunderscore}{\kern0pt}f\ h{\isacharunderscore}{\kern0pt}type\ hpx{\isacharunderscore}{\kern0pt}eqs{\isacharunderscore}{\kern0pt}f\ \isacommand{by}\isamarkupfalse%
\ auto\isanewline
\ \ \isacommand{qed}\isamarkupfalse%
\isanewline
\ \ \isacommand{have}\isamarkupfalse%
\ g{\isacharunderscore}{\kern0pt}not{\isacharunderscore}{\kern0pt}h{\isacharcolon}{\kern0pt}\ {\isachardoublequoteopen}g\ {\isasymnoteq}\ h{\isachardoublequoteclose}\isanewline
\ \ \isacommand{proof}\isamarkupfalse%
\ {\isacharminus}{\kern0pt}\isanewline
\ \ \ \isacommand{have}\isamarkupfalse%
\ f{\isadigit{1}}{\isacharcolon}{\kern0pt}\ {\isachardoublequoteopen}{\isasymforall}c{\isachardot}{\kern0pt}\ {\isasymbeta}\isactrlbsub codomain\ c\isactrlesub \ {\isasymcirc}\isactrlsub c\ c\ {\isacharequal}{\kern0pt}\ {\isasymbeta}\isactrlbsub domain\ c\isactrlesub {\isachardoublequoteclose}\isanewline
\ \ \ \ \isacommand{by}\isamarkupfalse%
\ {\isacharparenleft}{\kern0pt}simp\ add{\isacharcolon}{\kern0pt}\ cfunc{\isacharunderscore}{\kern0pt}type{\isacharunderscore}{\kern0pt}def\ terminal{\isacharunderscore}{\kern0pt}func{\isacharunderscore}{\kern0pt}comp{\isacharparenright}{\kern0pt}\isanewline
\ \ \ \isacommand{have}\isamarkupfalse%
\ f{\isadigit{2}}{\isacharcolon}{\kern0pt}\ {\isachardoublequoteopen}domain\ {\isasymlangle}y{\isadigit{0}}\ {\isasymcirc}\isactrlsub c\ {\isasymbeta}\isactrlbsub Y\isactrlesub {\isacharcomma}{\kern0pt}id\isactrlsub c\ Y{\isasymrangle}\ {\isacharequal}{\kern0pt}\ Y{\isachardoublequoteclose}\isanewline
\ \ \ \ \isacommand{using}\isamarkupfalse%
\ cfunc{\isacharunderscore}{\kern0pt}type{\isacharunderscore}{\kern0pt}def\ g{\isacharunderscore}{\kern0pt}right{\isacharunderscore}{\kern0pt}arg{\isacharunderscore}{\kern0pt}type\ \isacommand{by}\isamarkupfalse%
\ blast\isanewline
\ \ \isacommand{have}\isamarkupfalse%
\ f{\isadigit{3}}{\isacharcolon}{\kern0pt}\ {\isachardoublequoteopen}codomain\ {\isasymlangle}y{\isadigit{0}}\ {\isasymcirc}\isactrlsub c\ {\isasymbeta}\isactrlbsub Y\isactrlesub {\isacharcomma}{\kern0pt}id\isactrlsub c\ Y{\isasymrangle}\ {\isacharequal}{\kern0pt}\ Y\ {\isasymtimes}\isactrlsub c\ Y{\isachardoublequoteclose}\isanewline
\ \ \ \ \isacommand{using}\isamarkupfalse%
\ cfunc{\isacharunderscore}{\kern0pt}type{\isacharunderscore}{\kern0pt}def\ g{\isacharunderscore}{\kern0pt}right{\isacharunderscore}{\kern0pt}arg{\isacharunderscore}{\kern0pt}type\ \isacommand{by}\isamarkupfalse%
\ blast\isanewline
\ \ \isacommand{have}\isamarkupfalse%
\ f{\isadigit{4}}{\isacharcolon}{\kern0pt}\ {\isachardoublequoteopen}codomain\ y{\isadigit{0}}\ {\isacharequal}{\kern0pt}\ Y{\isachardoublequoteclose}\isanewline
\ \ \ \ \isacommand{using}\isamarkupfalse%
\ cfunc{\isacharunderscore}{\kern0pt}type{\isacharunderscore}{\kern0pt}def\ y{\isacharunderscore}{\kern0pt}def\ \isacommand{by}\isamarkupfalse%
\ presburger\isanewline
\ \ \isacommand{have}\isamarkupfalse%
\ {\isachardoublequoteopen}{\isasymforall}c{\isachardot}{\kern0pt}\ domain\ {\isacharparenleft}{\kern0pt}eq{\isacharunderscore}{\kern0pt}pred\ c{\isacharparenright}{\kern0pt}\ {\isacharequal}{\kern0pt}\ c\ {\isasymtimes}\isactrlsub c\ c{\isachardoublequoteclose}\isanewline
\ \ \ \ \isacommand{using}\isamarkupfalse%
\ cfunc{\isacharunderscore}{\kern0pt}type{\isacharunderscore}{\kern0pt}def\ eq{\isacharunderscore}{\kern0pt}pred{\isacharunderscore}{\kern0pt}type\ \isacommand{by}\isamarkupfalse%
\ auto\isanewline
\ \ \isacommand{then}\isamarkupfalse%
\ \isacommand{have}\isamarkupfalse%
\ {\isachardoublequoteopen}g\ {\isasymcirc}\isactrlsub c\ y{\isadigit{0}}\ {\isasymnoteq}\ {\isasymf}{\isachardoublequoteclose}\isanewline
\ \ \ \ \isacommand{using}\isamarkupfalse%
\ f{\isadigit{4}}\ f{\isadigit{3}}\ f{\isadigit{2}}\ \isacommand{by}\isamarkupfalse%
\ {\isacharparenleft}{\kern0pt}metis\ {\isacharparenleft}{\kern0pt}no{\isacharunderscore}{\kern0pt}types{\isacharparenright}{\kern0pt}\ eq{\isacharunderscore}{\kern0pt}pred{\isacharunderscore}{\kern0pt}true{\isacharunderscore}{\kern0pt}extract{\isacharunderscore}{\kern0pt}right\ comp{\isacharunderscore}{\kern0pt}associative\ g{\isacharunderscore}{\kern0pt}def\ true{\isacharunderscore}{\kern0pt}false{\isacharunderscore}{\kern0pt}distinct\ y{\isacharunderscore}{\kern0pt}def{\isacharparenright}{\kern0pt}\isanewline
\ \ \isacommand{then}\isamarkupfalse%
\ \isacommand{show}\isamarkupfalse%
\ {\isacharquery}{\kern0pt}thesis\isanewline
\ \ \ \ \isacommand{using}\isamarkupfalse%
\ f{\isadigit{1}}\ \isacommand{by}\isamarkupfalse%
\ {\isacharparenleft}{\kern0pt}metis\ {\isacharparenleft}{\kern0pt}no{\isacharunderscore}{\kern0pt}types{\isacharparenright}{\kern0pt}\ cfunc{\isacharunderscore}{\kern0pt}type{\isacharunderscore}{\kern0pt}def\ comp{\isacharunderscore}{\kern0pt}associative\ false{\isacharunderscore}{\kern0pt}func{\isacharunderscore}{\kern0pt}type\ h{\isacharunderscore}{\kern0pt}def\ id{\isacharunderscore}{\kern0pt}right{\isacharunderscore}{\kern0pt}unit{\isadigit{2}}\ id{\isacharunderscore}{\kern0pt}type\ one{\isacharunderscore}{\kern0pt}unique{\isacharunderscore}{\kern0pt}element\ terminal{\isacharunderscore}{\kern0pt}func{\isacharunderscore}{\kern0pt}type\ y{\isacharunderscore}{\kern0pt}def{\isacharparenright}{\kern0pt}\isanewline
\isacommand{qed}\isamarkupfalse%
\isanewline
\ \ \isacommand{then}\isamarkupfalse%
\ \isacommand{show}\isamarkupfalse%
\ False\isanewline
\ \ \ \ \isacommand{using}\isamarkupfalse%
\ gp{\isacharunderscore}{\kern0pt}eqs{\isacharunderscore}{\kern0pt}hp\ assms\ cfunc{\isacharunderscore}{\kern0pt}type{\isacharunderscore}{\kern0pt}def\ epimorphism{\isacharunderscore}{\kern0pt}def\ g{\isacharunderscore}{\kern0pt}type\ h{\isacharunderscore}{\kern0pt}type\ \isacommand{by}\isamarkupfalse%
\ auto\isanewline
\isacommand{qed}\isamarkupfalse%
%
\endisatagproof
{\isafoldproof}%
%
\isadelimproof
%
\endisadelimproof
%
\begin{isamarkuptext}%
The lemma below corresponds to Proposition 2.2.9 in Halvorson.%
\end{isamarkuptext}\isamarkuptrue%
\isacommand{lemma}\isamarkupfalse%
\ pullback{\isacharunderscore}{\kern0pt}of{\isacharunderscore}{\kern0pt}epi{\isacharunderscore}{\kern0pt}is{\isacharunderscore}{\kern0pt}epi{\isadigit{1}}{\isacharcolon}{\kern0pt}\isanewline
\isakeyword{assumes}\ {\isachardoublequoteopen}f{\isacharcolon}{\kern0pt}\ Y\ {\isasymrightarrow}\ Z{\isachardoublequoteclose}\ {\isachardoublequoteopen}epimorphism\ f{\isachardoublequoteclose}\ {\isachardoublequoteopen}is{\isacharunderscore}{\kern0pt}pullback\ A\ Y\ X\ Z\ q{\isadigit{1}}\ f\ q{\isadigit{0}}\ g{\isachardoublequoteclose}\isanewline
\isakeyword{shows}\ {\isachardoublequoteopen}epimorphism\ q{\isadigit{0}}{\isachardoublequoteclose}\ \isanewline
%
\isadelimproof
%
\endisadelimproof
%
\isatagproof
\isacommand{proof}\isamarkupfalse%
\ {\isacharminus}{\kern0pt}\ \isanewline
\ \ \isacommand{have}\isamarkupfalse%
\ surj{\isacharunderscore}{\kern0pt}f{\isacharcolon}{\kern0pt}\ {\isachardoublequoteopen}surjective\ f{\isachardoublequoteclose}\isanewline
\ \ \ \ \isacommand{using}\isamarkupfalse%
\ assms{\isacharparenleft}{\kern0pt}{\isadigit{1}}{\isacharcomma}{\kern0pt}{\isadigit{2}}{\isacharparenright}{\kern0pt}\ epi{\isacharunderscore}{\kern0pt}is{\isacharunderscore}{\kern0pt}surj\ \isacommand{by}\isamarkupfalse%
\ auto\isanewline
\ \ \isacommand{have}\isamarkupfalse%
\ {\isachardoublequoteopen}surjective\ {\isacharparenleft}{\kern0pt}q{\isadigit{0}}{\isacharparenright}{\kern0pt}{\isachardoublequoteclose}\isanewline
\ \ \ \ \isacommand{unfolding}\isamarkupfalse%
\ surjective{\isacharunderscore}{\kern0pt}def\isanewline
\ \ \isacommand{proof}\isamarkupfalse%
{\isacharparenleft}{\kern0pt}clarify{\isacharparenright}{\kern0pt}\isanewline
\ \ \ \ \isacommand{fix}\isamarkupfalse%
\ y\isanewline
\ \ \ \ \isacommand{assume}\isamarkupfalse%
\ y{\isacharunderscore}{\kern0pt}type{\isacharcolon}{\kern0pt}\ {\isachardoublequoteopen}y\ {\isasymin}\isactrlsub c\ codomain\ q{\isadigit{0}}{\isachardoublequoteclose}\isanewline
\ \ \ \ \isacommand{then}\isamarkupfalse%
\ \isacommand{have}\isamarkupfalse%
\ codomain{\isacharunderscore}{\kern0pt}gy{\isacharcolon}{\kern0pt}\ {\isachardoublequoteopen}g\ {\isasymcirc}\isactrlsub c\ y\ {\isasymin}\isactrlsub c\ Z{\isachardoublequoteclose}\isanewline
\ \ \ \ \ \ \isacommand{using}\isamarkupfalse%
\ assms{\isacharparenleft}{\kern0pt}{\isadigit{3}}{\isacharparenright}{\kern0pt}\ cfunc{\isacharunderscore}{\kern0pt}type{\isacharunderscore}{\kern0pt}def\ is{\isacharunderscore}{\kern0pt}pullback{\isacharunderscore}{\kern0pt}def\ \ \isacommand{by}\isamarkupfalse%
\ {\isacharparenleft}{\kern0pt}typecheck{\isacharunderscore}{\kern0pt}cfuncs{\isacharcomma}{\kern0pt}\ auto{\isacharparenright}{\kern0pt}\isanewline
\ \ \ \ \isacommand{then}\isamarkupfalse%
\ \isacommand{have}\isamarkupfalse%
\ z{\isacharunderscore}{\kern0pt}exists{\isacharcolon}{\kern0pt}\ {\isachardoublequoteopen}{\isasymexists}\ z{\isachardot}{\kern0pt}\ z\ {\isasymin}\isactrlsub c\ Y\ {\isasymand}\ f\ {\isasymcirc}\isactrlsub c\ z\ {\isacharequal}{\kern0pt}\ g\ {\isasymcirc}\isactrlsub c\ y{\isachardoublequoteclose}\isanewline
\ \ \ \ \ \ \isacommand{using}\isamarkupfalse%
\ assms{\isacharparenleft}{\kern0pt}{\isadigit{1}}{\isacharparenright}{\kern0pt}\ cfunc{\isacharunderscore}{\kern0pt}type{\isacharunderscore}{\kern0pt}def\ surj{\isacharunderscore}{\kern0pt}f\ surjective{\isacharunderscore}{\kern0pt}def\ \isacommand{by}\isamarkupfalse%
\ auto\isanewline
\ \ \ \ \isacommand{then}\isamarkupfalse%
\ \isacommand{obtain}\isamarkupfalse%
\ z\ \isakeyword{where}\ z{\isacharunderscore}{\kern0pt}def{\isacharcolon}{\kern0pt}\ {\isachardoublequoteopen}z\ {\isasymin}\isactrlsub c\ Y\ {\isasymand}\ f\ {\isasymcirc}\isactrlsub c\ z\ {\isacharequal}{\kern0pt}\ g\ {\isasymcirc}\isactrlsub c\ y{\isachardoublequoteclose}\isanewline
\ \ \ \ \ \ \isacommand{by}\isamarkupfalse%
\ blast\isanewline
\ \ \ \ \isacommand{then}\isamarkupfalse%
\ \isacommand{have}\isamarkupfalse%
\ {\isachardoublequoteopen}{\isasymexists}{\isacharbang}{\kern0pt}\ k{\isachardot}{\kern0pt}\ k{\isacharcolon}{\kern0pt}\ {\isasymone}\ {\isasymrightarrow}\ A\ {\isasymand}\ q{\isadigit{0}}\ {\isasymcirc}\isactrlsub c\ k\ {\isacharequal}{\kern0pt}\ y\ {\isasymand}\ q{\isadigit{1}}\ {\isasymcirc}\isactrlsub c\ k\ {\isacharequal}{\kern0pt}z{\isachardoublequoteclose}\isanewline
\ \ \ \ \ \ \isacommand{by}\isamarkupfalse%
\ {\isacharparenleft}{\kern0pt}smt\ {\isacharparenleft}{\kern0pt}verit{\isacharcomma}{\kern0pt}\ ccfv{\isacharunderscore}{\kern0pt}threshold{\isacharparenright}{\kern0pt}\ assms{\isacharparenleft}{\kern0pt}{\isadigit{3}}{\isacharparenright}{\kern0pt}\ cfunc{\isacharunderscore}{\kern0pt}type{\isacharunderscore}{\kern0pt}def\ is{\isacharunderscore}{\kern0pt}pullback{\isacharunderscore}{\kern0pt}def\ y{\isacharunderscore}{\kern0pt}type{\isacharparenright}{\kern0pt}\isanewline
\ \ \ \ \isacommand{then}\isamarkupfalse%
\ \isacommand{show}\isamarkupfalse%
\ {\isachardoublequoteopen}{\isasymexists}x{\isachardot}{\kern0pt}\ x\ {\isasymin}\isactrlsub c\ domain\ q{\isadigit{0}}\ {\isasymand}\ q{\isadigit{0}}\ {\isasymcirc}\isactrlsub c\ x\ {\isacharequal}{\kern0pt}\ y{\isachardoublequoteclose}\isanewline
\ \ \ \ \ \ \isacommand{using}\isamarkupfalse%
\ assms{\isacharparenleft}{\kern0pt}{\isadigit{3}}{\isacharparenright}{\kern0pt}\ cfunc{\isacharunderscore}{\kern0pt}type{\isacharunderscore}{\kern0pt}def\ is{\isacharunderscore}{\kern0pt}pullback{\isacharunderscore}{\kern0pt}def\ \ \isacommand{by}\isamarkupfalse%
\ auto\isanewline
\ \ \isacommand{qed}\isamarkupfalse%
\ \isanewline
\ \ \isacommand{then}\isamarkupfalse%
\ \isacommand{show}\isamarkupfalse%
\ {\isacharquery}{\kern0pt}thesis\isanewline
\ \ \ \ \isacommand{using}\isamarkupfalse%
\ surjective{\isacharunderscore}{\kern0pt}is{\isacharunderscore}{\kern0pt}epimorphism\ \isacommand{by}\isamarkupfalse%
\ blast\isanewline
\isacommand{qed}\isamarkupfalse%
%
\endisatagproof
{\isafoldproof}%
%
\isadelimproof
%
\endisadelimproof
%
\begin{isamarkuptext}%
The lemma below corresponds to Proposition 2.2.9b in Halvorson.%
\end{isamarkuptext}\isamarkuptrue%
\isacommand{lemma}\isamarkupfalse%
\ pullback{\isacharunderscore}{\kern0pt}of{\isacharunderscore}{\kern0pt}epi{\isacharunderscore}{\kern0pt}is{\isacharunderscore}{\kern0pt}epi{\isadigit{2}}{\isacharcolon}{\kern0pt}\isanewline
\isakeyword{assumes}\ {\isachardoublequoteopen}g{\isacharcolon}{\kern0pt}\ X\ {\isasymrightarrow}\ Z{\isachardoublequoteclose}\ {\isachardoublequoteopen}epimorphism\ g{\isachardoublequoteclose}\ {\isachardoublequoteopen}is{\isacharunderscore}{\kern0pt}pullback\ A\ Y\ X\ Z\ q{\isadigit{1}}\ f\ q{\isadigit{0}}\ g{\isachardoublequoteclose}\isanewline
\isakeyword{shows}\ {\isachardoublequoteopen}epimorphism\ q{\isadigit{1}}{\isachardoublequoteclose}\ \isanewline
%
\isadelimproof
%
\endisadelimproof
%
\isatagproof
\isacommand{proof}\isamarkupfalse%
\ {\isacharminus}{\kern0pt}\ \isanewline
\ \ \isacommand{have}\isamarkupfalse%
\ surj{\isacharunderscore}{\kern0pt}g{\isacharcolon}{\kern0pt}\ {\isachardoublequoteopen}surjective\ g{\isachardoublequoteclose}\isanewline
\ \ \ \ \isacommand{using}\isamarkupfalse%
\ assms{\isacharparenleft}{\kern0pt}{\isadigit{1}}{\isacharparenright}{\kern0pt}\ assms{\isacharparenleft}{\kern0pt}{\isadigit{2}}{\isacharparenright}{\kern0pt}\ epi{\isacharunderscore}{\kern0pt}is{\isacharunderscore}{\kern0pt}surj\ \isacommand{by}\isamarkupfalse%
\ auto\isanewline
\ \ \isacommand{have}\isamarkupfalse%
\ {\isachardoublequoteopen}surjective\ q{\isadigit{1}}{\isachardoublequoteclose}\isanewline
\ \ \ \ \isacommand{unfolding}\isamarkupfalse%
\ surjective{\isacharunderscore}{\kern0pt}def\isanewline
\ \ \isacommand{proof}\isamarkupfalse%
{\isacharparenleft}{\kern0pt}clarify{\isacharparenright}{\kern0pt}\isanewline
\ \ \ \ \isacommand{fix}\isamarkupfalse%
\ y\isanewline
\ \ \ \ \isacommand{assume}\isamarkupfalse%
\ y{\isacharunderscore}{\kern0pt}type{\isacharcolon}{\kern0pt}\ {\isachardoublequoteopen}y\ {\isasymin}\isactrlsub c\ codomain\ q{\isadigit{1}}{\isachardoublequoteclose}\isanewline
\ \ \ \ \isacommand{then}\isamarkupfalse%
\ \isacommand{have}\isamarkupfalse%
\ codomain{\isacharunderscore}{\kern0pt}gy{\isacharcolon}{\kern0pt}\ {\isachardoublequoteopen}f\ {\isasymcirc}\isactrlsub c\ y\ {\isasymin}\isactrlsub c\ Z{\isachardoublequoteclose}\isanewline
\ \ \ \ \ \ \isacommand{using}\isamarkupfalse%
\ assms{\isacharparenleft}{\kern0pt}{\isadigit{3}}{\isacharparenright}{\kern0pt}\ cfunc{\isacharunderscore}{\kern0pt}type{\isacharunderscore}{\kern0pt}def\ comp{\isacharunderscore}{\kern0pt}type\ is{\isacharunderscore}{\kern0pt}pullback{\isacharunderscore}{\kern0pt}def\ \ \isacommand{by}\isamarkupfalse%
\ auto\isanewline
\ \ \ \ \isacommand{then}\isamarkupfalse%
\ \isacommand{have}\isamarkupfalse%
\ z{\isacharunderscore}{\kern0pt}exists{\isacharcolon}{\kern0pt}\ {\isachardoublequoteopen}{\isasymexists}\ z{\isachardot}{\kern0pt}\ z\ {\isasymin}\isactrlsub c\ X\ {\isasymand}\ g\ {\isasymcirc}\isactrlsub c\ z\ {\isacharequal}{\kern0pt}\ f\ {\isasymcirc}\isactrlsub c\ y{\isachardoublequoteclose}\isanewline
\ \ \ \ \ \ \isacommand{using}\isamarkupfalse%
\ assms{\isacharparenleft}{\kern0pt}{\isadigit{1}}{\isacharparenright}{\kern0pt}\ cfunc{\isacharunderscore}{\kern0pt}type{\isacharunderscore}{\kern0pt}def\ surj{\isacharunderscore}{\kern0pt}g\ surjective{\isacharunderscore}{\kern0pt}def\ \isacommand{by}\isamarkupfalse%
\ auto\isanewline
\ \ \ \ \isacommand{then}\isamarkupfalse%
\ \isacommand{obtain}\isamarkupfalse%
\ z\ \isakeyword{where}\ z{\isacharunderscore}{\kern0pt}def{\isacharcolon}{\kern0pt}\ {\isachardoublequoteopen}z\ {\isasymin}\isactrlsub c\ X\ {\isasymand}\ g\ {\isasymcirc}\isactrlsub c\ z\ {\isacharequal}{\kern0pt}\ f\ {\isasymcirc}\isactrlsub c\ y{\isachardoublequoteclose}\isanewline
\ \ \ \ \ \ \isacommand{by}\isamarkupfalse%
\ blast\isanewline
\ \ \ \ \isacommand{then}\isamarkupfalse%
\ \isacommand{have}\isamarkupfalse%
\ {\isachardoublequoteopen}{\isasymexists}{\isacharbang}{\kern0pt}\ k{\isachardot}{\kern0pt}\ k{\isacharcolon}{\kern0pt}\ {\isasymone}\ {\isasymrightarrow}\ A\ {\isasymand}\ q{\isadigit{0}}\ {\isasymcirc}\isactrlsub c\ k\ {\isacharequal}{\kern0pt}\ z\ {\isasymand}\ q{\isadigit{1}}\ {\isasymcirc}\isactrlsub c\ k\ {\isacharequal}{\kern0pt}y{\isachardoublequoteclose}\isanewline
\ \ \ \ \ \ \isacommand{by}\isamarkupfalse%
\ {\isacharparenleft}{\kern0pt}smt\ {\isacharparenleft}{\kern0pt}verit{\isacharcomma}{\kern0pt}\ ccfv{\isacharunderscore}{\kern0pt}threshold{\isacharparenright}{\kern0pt}\ assms{\isacharparenleft}{\kern0pt}{\isadigit{3}}{\isacharparenright}{\kern0pt}\ cfunc{\isacharunderscore}{\kern0pt}type{\isacharunderscore}{\kern0pt}def\ is{\isacharunderscore}{\kern0pt}pullback{\isacharunderscore}{\kern0pt}def\ \ y{\isacharunderscore}{\kern0pt}type{\isacharparenright}{\kern0pt}\ \ \ \ \ \ \isanewline
\ \ \ \ \isacommand{then}\isamarkupfalse%
\ \isacommand{show}\isamarkupfalse%
\ {\isachardoublequoteopen}{\isasymexists}x{\isachardot}{\kern0pt}\ x\ {\isasymin}\isactrlsub c\ domain\ q{\isadigit{1}}\ {\isasymand}\ q{\isadigit{1}}\ {\isasymcirc}\isactrlsub c\ x\ {\isacharequal}{\kern0pt}\ y{\isachardoublequoteclose}\isanewline
\ \ \ \ \ \ \isacommand{using}\isamarkupfalse%
\ assms{\isacharparenleft}{\kern0pt}{\isadigit{3}}{\isacharparenright}{\kern0pt}\ cfunc{\isacharunderscore}{\kern0pt}type{\isacharunderscore}{\kern0pt}def\ is{\isacharunderscore}{\kern0pt}pullback{\isacharunderscore}{\kern0pt}def\ \ \isacommand{by}\isamarkupfalse%
\ auto\isanewline
\ \ \isacommand{qed}\isamarkupfalse%
\isanewline
\ \ \isacommand{then}\isamarkupfalse%
\ \isacommand{show}\isamarkupfalse%
\ {\isacharquery}{\kern0pt}thesis\isanewline
\ \ \ \ \isacommand{using}\isamarkupfalse%
\ surjective{\isacharunderscore}{\kern0pt}is{\isacharunderscore}{\kern0pt}epimorphism\ \isacommand{by}\isamarkupfalse%
\ blast\isanewline
\isacommand{qed}\isamarkupfalse%
%
\endisatagproof
{\isafoldproof}%
%
\isadelimproof
%
\endisadelimproof
%
\begin{isamarkuptext}%
The lemma below corresponds to Proposition 2.2.9c in Halvorson.%
\end{isamarkuptext}\isamarkuptrue%
\isacommand{lemma}\isamarkupfalse%
\ pullback{\isacharunderscore}{\kern0pt}of{\isacharunderscore}{\kern0pt}mono{\isacharunderscore}{\kern0pt}is{\isacharunderscore}{\kern0pt}mono{\isadigit{1}}{\isacharcolon}{\kern0pt}\isanewline
\isakeyword{assumes}\ {\isachardoublequoteopen}g{\isacharcolon}{\kern0pt}\ X\ {\isasymrightarrow}\ Z{\isachardoublequoteclose}\ {\isachardoublequoteopen}monomorphism\ f{\isachardoublequoteclose}\ {\isachardoublequoteopen}is{\isacharunderscore}{\kern0pt}pullback\ A\ Y\ X\ Z\ q{\isadigit{1}}\ f\ q{\isadigit{0}}\ g{\isachardoublequoteclose}\isanewline
\isakeyword{shows}\ {\isachardoublequoteopen}monomorphism\ q{\isadigit{0}}{\isachardoublequoteclose}\ \isanewline
%
\isadelimproof
%
\endisadelimproof
%
\isatagproof
\isacommand{proof}\isamarkupfalse%
{\isacharparenleft}{\kern0pt}unfold\ monomorphism{\isacharunderscore}{\kern0pt}def{\isadigit{2}}{\isacharcomma}{\kern0pt}\ clarify{\isacharparenright}{\kern0pt}\isanewline
\ \ \isacommand{fix}\isamarkupfalse%
\ u\ v\ Q\ a\ x\isanewline
\ \ \isacommand{assume}\isamarkupfalse%
\ u{\isacharunderscore}{\kern0pt}type{\isacharcolon}{\kern0pt}\ {\isachardoublequoteopen}u\ {\isacharcolon}{\kern0pt}\ Q\ {\isasymrightarrow}\ a{\isachardoublequoteclose}\ \ \isanewline
\ \ \isacommand{assume}\isamarkupfalse%
\ v{\isacharunderscore}{\kern0pt}type{\isacharcolon}{\kern0pt}\ {\isachardoublequoteopen}v\ {\isacharcolon}{\kern0pt}\ Q\ {\isasymrightarrow}\ a{\isachardoublequoteclose}\isanewline
\ \ \isacommand{assume}\isamarkupfalse%
\ q{\isadigit{0}}{\isacharunderscore}{\kern0pt}type{\isacharcolon}{\kern0pt}\ {\isachardoublequoteopen}q{\isadigit{0}}\ {\isacharcolon}{\kern0pt}\ \ a\ {\isasymrightarrow}\ x{\isachardoublequoteclose}\ \ \ \ \isanewline
\ \ \isacommand{assume}\isamarkupfalse%
\ equals{\isacharcolon}{\kern0pt}\ {\isachardoublequoteopen}q{\isadigit{0}}\ {\isasymcirc}\isactrlsub c\ u\ {\isacharequal}{\kern0pt}\ q{\isadigit{0}}\ {\isasymcirc}\isactrlsub c\ v{\isachardoublequoteclose}\ \isanewline
\isanewline
\ \ \isacommand{have}\isamarkupfalse%
\ a{\isacharunderscore}{\kern0pt}is{\isacharunderscore}{\kern0pt}A{\isacharcolon}{\kern0pt}\ {\isachardoublequoteopen}a\ {\isacharequal}{\kern0pt}\ A{\isachardoublequoteclose}\isanewline
\ \ \ \ \isacommand{using}\isamarkupfalse%
\ assms{\isacharparenleft}{\kern0pt}{\isadigit{3}}{\isacharparenright}{\kern0pt}\ cfunc{\isacharunderscore}{\kern0pt}type{\isacharunderscore}{\kern0pt}def\ is{\isacharunderscore}{\kern0pt}pullback{\isacharunderscore}{\kern0pt}def\ q{\isadigit{0}}{\isacharunderscore}{\kern0pt}type\ \ \isacommand{by}\isamarkupfalse%
\ force\isanewline
\ \ \isacommand{have}\isamarkupfalse%
\ x{\isacharunderscore}{\kern0pt}is{\isacharunderscore}{\kern0pt}X{\isacharcolon}{\kern0pt}\ {\isachardoublequoteopen}x\ {\isacharequal}{\kern0pt}\ X{\isachardoublequoteclose}\isanewline
\ \ \ \ \isacommand{using}\isamarkupfalse%
\ assms{\isacharparenleft}{\kern0pt}{\isadigit{3}}{\isacharparenright}{\kern0pt}\ cfunc{\isacharunderscore}{\kern0pt}type{\isacharunderscore}{\kern0pt}def\ is{\isacharunderscore}{\kern0pt}pullback{\isacharunderscore}{\kern0pt}def\ q{\isadigit{0}}{\isacharunderscore}{\kern0pt}type\ \ \isacommand{by}\isamarkupfalse%
\ fastforce\isanewline
\ \ \isacommand{have}\isamarkupfalse%
\ u{\isacharunderscore}{\kern0pt}type{\isadigit{2}}{\isacharbrackleft}{\kern0pt}type{\isacharunderscore}{\kern0pt}rule{\isacharbrackright}{\kern0pt}{\isacharcolon}{\kern0pt}\ {\isachardoublequoteopen}u\ {\isacharcolon}{\kern0pt}\ Q\ {\isasymrightarrow}\ A{\isachardoublequoteclose}\isanewline
\ \ \ \ \isacommand{using}\isamarkupfalse%
\ a{\isacharunderscore}{\kern0pt}is{\isacharunderscore}{\kern0pt}A\ u{\isacharunderscore}{\kern0pt}type\ \isacommand{by}\isamarkupfalse%
\ blast\isanewline
\ \ \isacommand{have}\isamarkupfalse%
\ v{\isacharunderscore}{\kern0pt}type{\isadigit{2}}{\isacharbrackleft}{\kern0pt}type{\isacharunderscore}{\kern0pt}rule{\isacharbrackright}{\kern0pt}{\isacharcolon}{\kern0pt}\ {\isachardoublequoteopen}v\ {\isacharcolon}{\kern0pt}\ Q\ {\isasymrightarrow}\ A{\isachardoublequoteclose}\isanewline
\ \ \ \ \isacommand{using}\isamarkupfalse%
\ a{\isacharunderscore}{\kern0pt}is{\isacharunderscore}{\kern0pt}A\ v{\isacharunderscore}{\kern0pt}type\ \isacommand{by}\isamarkupfalse%
\ blast\isanewline
\ \ \isacommand{have}\isamarkupfalse%
\ q{\isadigit{1}}{\isacharunderscore}{\kern0pt}type{\isadigit{2}}{\isacharbrackleft}{\kern0pt}type{\isacharunderscore}{\kern0pt}rule{\isacharbrackright}{\kern0pt}{\isacharcolon}{\kern0pt}\ {\isachardoublequoteopen}q{\isadigit{0}}\ {\isacharcolon}{\kern0pt}\ A\ {\isasymrightarrow}\ X{\isachardoublequoteclose}\isanewline
\ \ \ \ \isacommand{using}\isamarkupfalse%
\ a{\isacharunderscore}{\kern0pt}is{\isacharunderscore}{\kern0pt}A\ q{\isadigit{0}}{\isacharunderscore}{\kern0pt}type\ x{\isacharunderscore}{\kern0pt}is{\isacharunderscore}{\kern0pt}X\ \isacommand{by}\isamarkupfalse%
\ blast\isanewline
\isanewline
\ \ \isacommand{have}\isamarkupfalse%
\ eqn{\isadigit{1}}{\isacharcolon}{\kern0pt}\ {\isachardoublequoteopen}g\ {\isasymcirc}\isactrlsub c\ {\isacharparenleft}{\kern0pt}q{\isadigit{0}}\ {\isasymcirc}\isactrlsub c\ u{\isacharparenright}{\kern0pt}\ {\isacharequal}{\kern0pt}\ f\ {\isasymcirc}\isactrlsub c\ {\isacharparenleft}{\kern0pt}q{\isadigit{1}}\ {\isasymcirc}\isactrlsub c\ v{\isacharparenright}{\kern0pt}{\isachardoublequoteclose}\isanewline
\ \ \isacommand{proof}\isamarkupfalse%
\ {\isacharminus}{\kern0pt}\ \isanewline
\ \ \ \ \isacommand{have}\isamarkupfalse%
\ {\isachardoublequoteopen}g\ {\isasymcirc}\isactrlsub c\ {\isacharparenleft}{\kern0pt}q{\isadigit{0}}\ {\isasymcirc}\isactrlsub c\ u{\isacharparenright}{\kern0pt}\ {\isacharequal}{\kern0pt}\ g\ {\isasymcirc}\isactrlsub c\ q{\isadigit{0}}\ {\isasymcirc}\isactrlsub c\ v{\isachardoublequoteclose}\isanewline
\ \ \ \ \ \ \isacommand{by}\isamarkupfalse%
\ {\isacharparenleft}{\kern0pt}simp\ add{\isacharcolon}{\kern0pt}\ equals{\isacharparenright}{\kern0pt}\isanewline
\ \ \ \ \isacommand{also}\isamarkupfalse%
\ \isacommand{have}\isamarkupfalse%
\ {\isachardoublequoteopen}{\isachardot}{\kern0pt}{\isachardot}{\kern0pt}{\isachardot}{\kern0pt}\ {\isacharequal}{\kern0pt}\ f\ {\isasymcirc}\isactrlsub c\ {\isacharparenleft}{\kern0pt}q{\isadigit{1}}\ {\isasymcirc}\isactrlsub c\ v{\isacharparenright}{\kern0pt}{\isachardoublequoteclose}\isanewline
\ \ \ \ \ \ \isacommand{using}\isamarkupfalse%
\ assms{\isacharparenleft}{\kern0pt}{\isadigit{3}}{\isacharparenright}{\kern0pt}\ cfunc{\isacharunderscore}{\kern0pt}type{\isacharunderscore}{\kern0pt}def\ comp{\isacharunderscore}{\kern0pt}associative\ is{\isacharunderscore}{\kern0pt}pullback{\isacharunderscore}{\kern0pt}def\ \isacommand{by}\isamarkupfalse%
\ {\isacharparenleft}{\kern0pt}typecheck{\isacharunderscore}{\kern0pt}cfuncs{\isacharcomma}{\kern0pt}\ force{\isacharparenright}{\kern0pt}\isanewline
\ \ \ \ \isacommand{then}\isamarkupfalse%
\ \isacommand{show}\isamarkupfalse%
\ {\isacharquery}{\kern0pt}thesis\isanewline
\ \ \ \ \ \ \isacommand{by}\isamarkupfalse%
\ {\isacharparenleft}{\kern0pt}simp\ add{\isacharcolon}{\kern0pt}\ calculation{\isacharparenright}{\kern0pt}\isanewline
\ \ \isacommand{qed}\isamarkupfalse%
\ \isanewline
\isanewline
\ \ \isacommand{have}\isamarkupfalse%
\ eqn{\isadigit{2}}{\isacharcolon}{\kern0pt}\ {\isachardoublequoteopen}q{\isadigit{1}}\ {\isasymcirc}\isactrlsub c\ u\ {\isacharequal}{\kern0pt}\ \ q{\isadigit{1}}\ \ {\isasymcirc}\isactrlsub c\ v{\isachardoublequoteclose}\isanewline
\ \ \isacommand{proof}\isamarkupfalse%
\ {\isacharminus}{\kern0pt}\ \isanewline
\ \ \ \ \isacommand{have}\isamarkupfalse%
\ f{\isadigit{1}}{\isacharcolon}{\kern0pt}\ {\isachardoublequoteopen}f\ {\isasymcirc}\isactrlsub c\ q{\isadigit{1}}\ {\isasymcirc}\isactrlsub c\ u\ {\isacharequal}{\kern0pt}\ g\ {\isasymcirc}\isactrlsub c\ q{\isadigit{0}}\ {\isasymcirc}\isactrlsub c\ u{\isachardoublequoteclose}\isanewline
\ \ \ \ \ \ \isacommand{using}\isamarkupfalse%
\ assms{\isacharparenleft}{\kern0pt}{\isadigit{3}}{\isacharparenright}{\kern0pt}\ comp{\isacharunderscore}{\kern0pt}associative{\isadigit{2}}\ is{\isacharunderscore}{\kern0pt}pullback{\isacharunderscore}{\kern0pt}def\ \ \isacommand{by}\isamarkupfalse%
\ {\isacharparenleft}{\kern0pt}typecheck{\isacharunderscore}{\kern0pt}cfuncs{\isacharcomma}{\kern0pt}\ auto{\isacharparenright}{\kern0pt}\isanewline
\ \ \ \ \isacommand{also}\isamarkupfalse%
\ \isacommand{have}\isamarkupfalse%
\ {\isachardoublequoteopen}{\isachardot}{\kern0pt}{\isachardot}{\kern0pt}{\isachardot}{\kern0pt}\ {\isacharequal}{\kern0pt}\ g\ {\isasymcirc}\isactrlsub c\ q{\isadigit{0}}\ {\isasymcirc}\isactrlsub c\ v{\isachardoublequoteclose}\isanewline
\ \ \ \ \ \ \isacommand{by}\isamarkupfalse%
\ {\isacharparenleft}{\kern0pt}simp\ add{\isacharcolon}{\kern0pt}\ equals{\isacharparenright}{\kern0pt}\isanewline
\ \ \ \ \isacommand{also}\isamarkupfalse%
\ \isacommand{have}\isamarkupfalse%
\ {\isachardoublequoteopen}{\isachardot}{\kern0pt}{\isachardot}{\kern0pt}{\isachardot}{\kern0pt}\ {\isacharequal}{\kern0pt}\ f\ {\isasymcirc}\isactrlsub c\ q{\isadigit{1}}\ {\isasymcirc}\isactrlsub c\ v{\isachardoublequoteclose}\isanewline
\ \ \ \ \ \ \isacommand{using}\isamarkupfalse%
\ eqn{\isadigit{1}}\ equals\ \isacommand{by}\isamarkupfalse%
\ fastforce\isanewline
\ \ \ \ \isacommand{then}\isamarkupfalse%
\ \isacommand{show}\isamarkupfalse%
\ {\isacharquery}{\kern0pt}thesis\isanewline
\ \ \ \ \ \ \isacommand{by}\isamarkupfalse%
\ {\isacharparenleft}{\kern0pt}typecheck{\isacharunderscore}{\kern0pt}cfuncs{\isacharcomma}{\kern0pt}\ smt\ {\isacharparenleft}{\kern0pt}verit{\isacharcomma}{\kern0pt}\ ccfv{\isacharunderscore}{\kern0pt}threshold{\isacharparenright}{\kern0pt}\ f{\isadigit{1}}\ assms{\isacharparenleft}{\kern0pt}{\isadigit{2}}{\isacharcomma}{\kern0pt}{\isadigit{3}}{\isacharparenright}{\kern0pt}\ eqn{\isadigit{1}}\ is{\isacharunderscore}{\kern0pt}pullback{\isacharunderscore}{\kern0pt}def\ monomorphism{\isacharunderscore}{\kern0pt}def{\isadigit{3}}{\isacharparenright}{\kern0pt}\isanewline
\ \ \isacommand{qed}\isamarkupfalse%
\isanewline
\isanewline
\ \ \isacommand{have}\isamarkupfalse%
\ uniqueness{\isacharcolon}{\kern0pt}\ {\isachardoublequoteopen}{\isasymexists}{\isacharbang}{\kern0pt}\ j{\isachardot}{\kern0pt}\ {\isacharparenleft}{\kern0pt}j\ {\isacharcolon}{\kern0pt}\ Q\ {\isasymrightarrow}\ A\ {\isasymand}\ q{\isadigit{1}}\ {\isasymcirc}\isactrlsub c\ j\ {\isacharequal}{\kern0pt}\ q{\isadigit{1}}\ {\isasymcirc}\isactrlsub c\ v\ {\isasymand}\ q{\isadigit{0}}\ {\isasymcirc}\isactrlsub c\ j\ {\isacharequal}{\kern0pt}\ q{\isadigit{0}}\ {\isasymcirc}\isactrlsub c\ u{\isacharparenright}{\kern0pt}{\isachardoublequoteclose}\isanewline
\ \ \ \isacommand{by}\isamarkupfalse%
\ {\isacharparenleft}{\kern0pt}typecheck{\isacharunderscore}{\kern0pt}cfuncs{\isacharcomma}{\kern0pt}\ smt\ {\isacharparenleft}{\kern0pt}verit{\isacharcomma}{\kern0pt}\ ccfv{\isacharunderscore}{\kern0pt}threshold{\isacharparenright}{\kern0pt}\ assms{\isacharparenleft}{\kern0pt}{\isadigit{3}}{\isacharparenright}{\kern0pt}\ eqn{\isadigit{1}}\ is{\isacharunderscore}{\kern0pt}pullback{\isacharunderscore}{\kern0pt}def{\isacharparenright}{\kern0pt}\isanewline
\ \ \isacommand{then}\isamarkupfalse%
\ \isacommand{show}\isamarkupfalse%
\ {\isachardoublequoteopen}u\ {\isacharequal}{\kern0pt}\ v{\isachardoublequoteclose}\isanewline
\ \ \ \ \isacommand{using}\isamarkupfalse%
\ eqn{\isadigit{2}}\ equals\ uniqueness\ \isacommand{by}\isamarkupfalse%
\ {\isacharparenleft}{\kern0pt}typecheck{\isacharunderscore}{\kern0pt}cfuncs{\isacharcomma}{\kern0pt}\ auto{\isacharparenright}{\kern0pt}\isanewline
\isacommand{qed}\isamarkupfalse%
%
\endisatagproof
{\isafoldproof}%
%
\isadelimproof
%
\endisadelimproof
%
\begin{isamarkuptext}%
The lemma below corresponds to Proposition 2.2.9d in Halvorson.%
\end{isamarkuptext}\isamarkuptrue%
\isacommand{lemma}\isamarkupfalse%
\ pullback{\isacharunderscore}{\kern0pt}of{\isacharunderscore}{\kern0pt}mono{\isacharunderscore}{\kern0pt}is{\isacharunderscore}{\kern0pt}mono{\isadigit{2}}{\isacharcolon}{\kern0pt}\isanewline
\isakeyword{assumes}\ {\isachardoublequoteopen}g{\isacharcolon}{\kern0pt}\ X\ {\isasymrightarrow}\ Z{\isachardoublequoteclose}\ {\isachardoublequoteopen}monomorphism\ g{\isachardoublequoteclose}\ {\isachardoublequoteopen}is{\isacharunderscore}{\kern0pt}pullback\ A\ Y\ X\ Z\ q{\isadigit{1}}\ f\ q{\isadigit{0}}\ g{\isachardoublequoteclose}\isanewline
\isakeyword{shows}\ {\isachardoublequoteopen}monomorphism\ q{\isadigit{1}}{\isachardoublequoteclose}\ \isanewline
%
\isadelimproof
%
\endisadelimproof
%
\isatagproof
\isacommand{proof}\isamarkupfalse%
{\isacharparenleft}{\kern0pt}unfold\ monomorphism{\isacharunderscore}{\kern0pt}def{\isadigit{2}}{\isacharcomma}{\kern0pt}\ clarify{\isacharparenright}{\kern0pt}\isanewline
\ \ \isacommand{fix}\isamarkupfalse%
\ u\ v\ Q\ a\ y\isanewline
\ \ \isacommand{assume}\isamarkupfalse%
\ u{\isacharunderscore}{\kern0pt}type{\isacharcolon}{\kern0pt}\ {\isachardoublequoteopen}u\ {\isacharcolon}{\kern0pt}\ Q\ {\isasymrightarrow}\ a{\isachardoublequoteclose}\ \ \isanewline
\ \ \isacommand{assume}\isamarkupfalse%
\ v{\isacharunderscore}{\kern0pt}type{\isacharcolon}{\kern0pt}\ {\isachardoublequoteopen}v\ {\isacharcolon}{\kern0pt}\ Q\ {\isasymrightarrow}\ a{\isachardoublequoteclose}\isanewline
\ \ \isacommand{assume}\isamarkupfalse%
\ q{\isadigit{1}}{\isacharunderscore}{\kern0pt}type{\isacharcolon}{\kern0pt}\ {\isachardoublequoteopen}q{\isadigit{1}}\ {\isacharcolon}{\kern0pt}\ \ a\ {\isasymrightarrow}\ y{\isachardoublequoteclose}\ \isanewline
\ \ \isacommand{assume}\isamarkupfalse%
\ equals{\isacharcolon}{\kern0pt}\ {\isachardoublequoteopen}q{\isadigit{1}}\ {\isasymcirc}\isactrlsub c\ u\ {\isacharequal}{\kern0pt}\ q{\isadigit{1}}\ {\isasymcirc}\isactrlsub c\ v{\isachardoublequoteclose}\ \isanewline
\isanewline
\ \ \isacommand{have}\isamarkupfalse%
\ a{\isacharunderscore}{\kern0pt}is{\isacharunderscore}{\kern0pt}A{\isacharcolon}{\kern0pt}\ {\isachardoublequoteopen}a\ {\isacharequal}{\kern0pt}\ A{\isachardoublequoteclose}\isanewline
\ \ \ \ \isacommand{using}\isamarkupfalse%
\ assms{\isacharparenleft}{\kern0pt}{\isadigit{3}}{\isacharparenright}{\kern0pt}\ cfunc{\isacharunderscore}{\kern0pt}type{\isacharunderscore}{\kern0pt}def\ is{\isacharunderscore}{\kern0pt}pullback{\isacharunderscore}{\kern0pt}def\ q{\isadigit{1}}{\isacharunderscore}{\kern0pt}type\ \ \isacommand{by}\isamarkupfalse%
\ force\isanewline
\ \ \isacommand{have}\isamarkupfalse%
\ y{\isacharunderscore}{\kern0pt}is{\isacharunderscore}{\kern0pt}Y{\isacharcolon}{\kern0pt}\ {\isachardoublequoteopen}y\ {\isacharequal}{\kern0pt}\ Y{\isachardoublequoteclose}\isanewline
\ \ \ \ \isacommand{using}\isamarkupfalse%
\ assms{\isacharparenleft}{\kern0pt}{\isadigit{3}}{\isacharparenright}{\kern0pt}\ cfunc{\isacharunderscore}{\kern0pt}type{\isacharunderscore}{\kern0pt}def\ is{\isacharunderscore}{\kern0pt}pullback{\isacharunderscore}{\kern0pt}def\ q{\isadigit{1}}{\isacharunderscore}{\kern0pt}type\ \ \isacommand{by}\isamarkupfalse%
\ fastforce\isanewline
\ \ \isacommand{have}\isamarkupfalse%
\ u{\isacharunderscore}{\kern0pt}type{\isadigit{2}}{\isacharbrackleft}{\kern0pt}type{\isacharunderscore}{\kern0pt}rule{\isacharbrackright}{\kern0pt}{\isacharcolon}{\kern0pt}\ {\isachardoublequoteopen}u\ {\isacharcolon}{\kern0pt}\ Q\ {\isasymrightarrow}\ A{\isachardoublequoteclose}\isanewline
\ \ \ \ \isacommand{using}\isamarkupfalse%
\ a{\isacharunderscore}{\kern0pt}is{\isacharunderscore}{\kern0pt}A\ u{\isacharunderscore}{\kern0pt}type\ \isacommand{by}\isamarkupfalse%
\ blast\isanewline
\ \ \isacommand{have}\isamarkupfalse%
\ v{\isacharunderscore}{\kern0pt}type{\isadigit{2}}{\isacharbrackleft}{\kern0pt}type{\isacharunderscore}{\kern0pt}rule{\isacharbrackright}{\kern0pt}{\isacharcolon}{\kern0pt}\ {\isachardoublequoteopen}v\ {\isacharcolon}{\kern0pt}\ Q\ {\isasymrightarrow}\ A{\isachardoublequoteclose}\isanewline
\ \ \ \ \isacommand{using}\isamarkupfalse%
\ a{\isacharunderscore}{\kern0pt}is{\isacharunderscore}{\kern0pt}A\ v{\isacharunderscore}{\kern0pt}type\ \isacommand{by}\isamarkupfalse%
\ blast\isanewline
\ \ \isacommand{have}\isamarkupfalse%
\ q{\isadigit{1}}{\isacharunderscore}{\kern0pt}type{\isadigit{2}}{\isacharbrackleft}{\kern0pt}type{\isacharunderscore}{\kern0pt}rule{\isacharbrackright}{\kern0pt}{\isacharcolon}{\kern0pt}\ {\isachardoublequoteopen}q{\isadigit{1}}\ {\isacharcolon}{\kern0pt}\ A\ {\isasymrightarrow}\ Y{\isachardoublequoteclose}\isanewline
\ \ \ \ \isacommand{using}\isamarkupfalse%
\ a{\isacharunderscore}{\kern0pt}is{\isacharunderscore}{\kern0pt}A\ q{\isadigit{1}}{\isacharunderscore}{\kern0pt}type\ y{\isacharunderscore}{\kern0pt}is{\isacharunderscore}{\kern0pt}Y\ \isacommand{by}\isamarkupfalse%
\ blast\isanewline
\isanewline
\ \ \isacommand{have}\isamarkupfalse%
\ eqn{\isadigit{1}}{\isacharcolon}{\kern0pt}\ {\isachardoublequoteopen}f\ {\isasymcirc}\isactrlsub c\ {\isacharparenleft}{\kern0pt}q{\isadigit{1}}\ {\isasymcirc}\isactrlsub c\ u{\isacharparenright}{\kern0pt}\ {\isacharequal}{\kern0pt}\ g\ {\isasymcirc}\isactrlsub c\ {\isacharparenleft}{\kern0pt}q{\isadigit{0}}\ {\isasymcirc}\isactrlsub c\ v{\isacharparenright}{\kern0pt}{\isachardoublequoteclose}\isanewline
\ \ \isacommand{proof}\isamarkupfalse%
\ {\isacharminus}{\kern0pt}\ \isanewline
\ \ \ \ \isacommand{have}\isamarkupfalse%
\ {\isachardoublequoteopen}f\ {\isasymcirc}\isactrlsub c\ {\isacharparenleft}{\kern0pt}q{\isadigit{1}}\ {\isasymcirc}\isactrlsub c\ u{\isacharparenright}{\kern0pt}\ {\isacharequal}{\kern0pt}\ f\ {\isasymcirc}\isactrlsub c\ q{\isadigit{1}}\ {\isasymcirc}\isactrlsub c\ v{\isachardoublequoteclose}\isanewline
\ \ \ \ \ \ \isacommand{by}\isamarkupfalse%
\ {\isacharparenleft}{\kern0pt}simp\ add{\isacharcolon}{\kern0pt}\ equals{\isacharparenright}{\kern0pt}\isanewline
\ \ \ \ \isacommand{also}\isamarkupfalse%
\ \isacommand{have}\isamarkupfalse%
\ {\isachardoublequoteopen}{\isachardot}{\kern0pt}{\isachardot}{\kern0pt}{\isachardot}{\kern0pt}\ {\isacharequal}{\kern0pt}\ g\ {\isasymcirc}\isactrlsub c\ {\isacharparenleft}{\kern0pt}q{\isadigit{0}}\ {\isasymcirc}\isactrlsub c\ v{\isacharparenright}{\kern0pt}{\isachardoublequoteclose}\isanewline
\ \ \ \ \ \ \isacommand{using}\isamarkupfalse%
\ assms{\isacharparenleft}{\kern0pt}{\isadigit{3}}{\isacharparenright}{\kern0pt}\ cfunc{\isacharunderscore}{\kern0pt}type{\isacharunderscore}{\kern0pt}def\ comp{\isacharunderscore}{\kern0pt}associative\ is{\isacharunderscore}{\kern0pt}pullback{\isacharunderscore}{\kern0pt}def\ \ \isacommand{by}\isamarkupfalse%
\ {\isacharparenleft}{\kern0pt}typecheck{\isacharunderscore}{\kern0pt}cfuncs{\isacharcomma}{\kern0pt}\ force{\isacharparenright}{\kern0pt}\isanewline
\ \ \ \ \isacommand{then}\isamarkupfalse%
\ \isacommand{show}\isamarkupfalse%
\ {\isacharquery}{\kern0pt}thesis\isanewline
\ \ \ \ \ \ \isacommand{by}\isamarkupfalse%
\ {\isacharparenleft}{\kern0pt}simp\ add{\isacharcolon}{\kern0pt}\ calculation{\isacharparenright}{\kern0pt}\isanewline
\ \ \isacommand{qed}\isamarkupfalse%
\ \isanewline
\isanewline
\ \ \isacommand{have}\isamarkupfalse%
\ eqn{\isadigit{2}}{\isacharcolon}{\kern0pt}\ {\isachardoublequoteopen}q{\isadigit{0}}\ {\isasymcirc}\isactrlsub c\ u\ {\isacharequal}{\kern0pt}\ \ q{\isadigit{0}}\ \ {\isasymcirc}\isactrlsub c\ v{\isachardoublequoteclose}\isanewline
\ \ \isacommand{proof}\isamarkupfalse%
\ {\isacharminus}{\kern0pt}\ \isanewline
\ \ \ \ \isacommand{have}\isamarkupfalse%
\ f{\isadigit{1}}{\isacharcolon}{\kern0pt}\ {\isachardoublequoteopen}g\ {\isasymcirc}\isactrlsub c\ q{\isadigit{0}}\ {\isasymcirc}\isactrlsub c\ u\ {\isacharequal}{\kern0pt}\ f\ {\isasymcirc}\isactrlsub c\ q{\isadigit{1}}\ {\isasymcirc}\isactrlsub c\ u{\isachardoublequoteclose}\isanewline
\ \ \ \ \ \ \isacommand{using}\isamarkupfalse%
\ assms{\isacharparenleft}{\kern0pt}{\isadigit{3}}{\isacharparenright}{\kern0pt}\ comp{\isacharunderscore}{\kern0pt}associative{\isadigit{2}}\ is{\isacharunderscore}{\kern0pt}pullback{\isacharunderscore}{\kern0pt}def\ \ \isacommand{by}\isamarkupfalse%
\ {\isacharparenleft}{\kern0pt}typecheck{\isacharunderscore}{\kern0pt}cfuncs{\isacharcomma}{\kern0pt}\ auto{\isacharparenright}{\kern0pt}\isanewline
\ \ \ \ \isacommand{also}\isamarkupfalse%
\ \isacommand{have}\isamarkupfalse%
\ {\isachardoublequoteopen}{\isachardot}{\kern0pt}{\isachardot}{\kern0pt}{\isachardot}{\kern0pt}\ {\isacharequal}{\kern0pt}\ f\ {\isasymcirc}\isactrlsub c\ q{\isadigit{1}}\ {\isasymcirc}\isactrlsub c\ v{\isachardoublequoteclose}\isanewline
\ \ \ \ \ \ \isacommand{by}\isamarkupfalse%
\ {\isacharparenleft}{\kern0pt}simp\ add{\isacharcolon}{\kern0pt}\ equals{\isacharparenright}{\kern0pt}\isanewline
\ \ \ \ \isacommand{also}\isamarkupfalse%
\ \isacommand{have}\isamarkupfalse%
\ {\isachardoublequoteopen}{\isachardot}{\kern0pt}{\isachardot}{\kern0pt}{\isachardot}{\kern0pt}\ {\isacharequal}{\kern0pt}\ g\ {\isasymcirc}\isactrlsub c\ q{\isadigit{0}}\ {\isasymcirc}\isactrlsub c\ v{\isachardoublequoteclose}\isanewline
\ \ \ \ \ \ \isacommand{using}\isamarkupfalse%
\ eqn{\isadigit{1}}\ equals\ \isacommand{by}\isamarkupfalse%
\ fastforce\isanewline
\ \ \ \ \isacommand{then}\isamarkupfalse%
\ \isacommand{show}\isamarkupfalse%
\ {\isacharquery}{\kern0pt}thesis\isanewline
\ \ \ \ \ \ \isacommand{by}\isamarkupfalse%
\ {\isacharparenleft}{\kern0pt}typecheck{\isacharunderscore}{\kern0pt}cfuncs{\isacharcomma}{\kern0pt}\ smt\ {\isacharparenleft}{\kern0pt}verit{\isacharcomma}{\kern0pt}\ ccfv{\isacharunderscore}{\kern0pt}threshold{\isacharparenright}{\kern0pt}\ f{\isadigit{1}}\ assms{\isacharparenleft}{\kern0pt}{\isadigit{2}}{\isacharcomma}{\kern0pt}{\isadigit{3}}{\isacharparenright}{\kern0pt}\ eqn{\isadigit{1}}\ is{\isacharunderscore}{\kern0pt}pullback{\isacharunderscore}{\kern0pt}def\ monomorphism{\isacharunderscore}{\kern0pt}def{\isadigit{3}}{\isacharparenright}{\kern0pt}\isanewline
\ \ \isacommand{qed}\isamarkupfalse%
\isanewline
\ \ \isacommand{have}\isamarkupfalse%
\ uniqueness{\isacharcolon}{\kern0pt}\ {\isachardoublequoteopen}{\isasymexists}{\isacharbang}{\kern0pt}\ j{\isachardot}{\kern0pt}\ {\isacharparenleft}{\kern0pt}j\ {\isacharcolon}{\kern0pt}\ Q\ {\isasymrightarrow}\ A\ {\isasymand}\ q{\isadigit{0}}\ {\isasymcirc}\isactrlsub c\ j\ {\isacharequal}{\kern0pt}\ q{\isadigit{0}}\ {\isasymcirc}\isactrlsub c\ v\ {\isasymand}\ q{\isadigit{1}}\ {\isasymcirc}\isactrlsub c\ j\ {\isacharequal}{\kern0pt}\ q{\isadigit{1}}\ {\isasymcirc}\isactrlsub c\ u{\isacharparenright}{\kern0pt}{\isachardoublequoteclose}\isanewline
\ \ \ \isacommand{by}\isamarkupfalse%
\ {\isacharparenleft}{\kern0pt}typecheck{\isacharunderscore}{\kern0pt}cfuncs{\isacharcomma}{\kern0pt}\ smt\ {\isacharparenleft}{\kern0pt}verit{\isacharcomma}{\kern0pt}\ ccfv{\isacharunderscore}{\kern0pt}threshold{\isacharparenright}{\kern0pt}\ assms{\isacharparenleft}{\kern0pt}{\isadigit{3}}{\isacharparenright}{\kern0pt}\ eqn{\isadigit{1}}\ is{\isacharunderscore}{\kern0pt}pullback{\isacharunderscore}{\kern0pt}def{\isacharparenright}{\kern0pt}\isanewline
\ \ \isacommand{then}\isamarkupfalse%
\ \isacommand{show}\isamarkupfalse%
\ {\isachardoublequoteopen}u\ {\isacharequal}{\kern0pt}\ v{\isachardoublequoteclose}\isanewline
\ \ \ \ \isacommand{using}\isamarkupfalse%
\ eqn{\isadigit{2}}\ equals\ uniqueness\ \isacommand{by}\isamarkupfalse%
\ {\isacharparenleft}{\kern0pt}typecheck{\isacharunderscore}{\kern0pt}cfuncs{\isacharcomma}{\kern0pt}\ auto{\isacharparenright}{\kern0pt}\isanewline
\isacommand{qed}\isamarkupfalse%
%
\endisatagproof
{\isafoldproof}%
%
\isadelimproof
%
\endisadelimproof
%
\isadelimdocument
%
\endisadelimdocument
%
\isatagdocument
%
\isamarkupsubsection{Fiber Over an Element and its Connection to the Fibered Product%
}
\isamarkuptrue%
%
\endisatagdocument
{\isafolddocument}%
%
\isadelimdocument
%
\endisadelimdocument
%
\begin{isamarkuptext}%
The definition below corresponds to Definition 2.2.6 in Halvorson.%
\end{isamarkuptext}\isamarkuptrue%
\isacommand{definition}\isamarkupfalse%
\ fiber\ {\isacharcolon}{\kern0pt}{\isacharcolon}{\kern0pt}\ {\isachardoublequoteopen}cfunc\ {\isasymRightarrow}\ cfunc\ {\isasymRightarrow}\ cset{\isachardoublequoteclose}\ {\isacharparenleft}{\kern0pt}{\isachardoublequoteopen}{\isacharunderscore}{\kern0pt}\isactrlsup {\isacharminus}{\kern0pt}\isactrlsup {\isadigit{1}}{\isacharbraceleft}{\kern0pt}{\isacharunderscore}{\kern0pt}{\isacharbraceright}{\kern0pt}{\isachardoublequoteclose}\ {\isacharbrackleft}{\kern0pt}{\isadigit{1}}{\isadigit{0}}{\isadigit{0}}{\isacharcomma}{\kern0pt}{\isadigit{1}}{\isadigit{0}}{\isadigit{0}}{\isacharbrackright}{\kern0pt}{\isadigit{1}}{\isadigit{0}}{\isadigit{0}}{\isacharparenright}{\kern0pt}\ \isakeyword{where}\isanewline
\ \ {\isachardoublequoteopen}f\isactrlsup {\isacharminus}{\kern0pt}\isactrlsup {\isadigit{1}}{\isacharbraceleft}{\kern0pt}y{\isacharbraceright}{\kern0pt}\ {\isacharequal}{\kern0pt}\ {\isacharparenleft}{\kern0pt}f\isactrlsup {\isacharminus}{\kern0pt}\isactrlsup {\isadigit{1}}{\isasymlparr}{\isasymone}{\isasymrparr}\isactrlbsub y\isactrlesub {\isacharparenright}{\kern0pt}{\isachardoublequoteclose}\isanewline
\isanewline
\isacommand{definition}\isamarkupfalse%
\ fiber{\isacharunderscore}{\kern0pt}morphism\ {\isacharcolon}{\kern0pt}{\isacharcolon}{\kern0pt}\ {\isachardoublequoteopen}cfunc\ {\isasymRightarrow}\ cfunc\ {\isasymRightarrow}\ cfunc{\isachardoublequoteclose}\ \isakeyword{where}\isanewline
\ \ {\isachardoublequoteopen}fiber{\isacharunderscore}{\kern0pt}morphism\ f\ y\ {\isacharequal}{\kern0pt}\ left{\isacharunderscore}{\kern0pt}cart{\isacharunderscore}{\kern0pt}proj\ {\isacharparenleft}{\kern0pt}domain\ f{\isacharparenright}{\kern0pt}\ {\isasymone}\ {\isasymcirc}\isactrlsub c\ inverse{\isacharunderscore}{\kern0pt}image{\isacharunderscore}{\kern0pt}mapping\ f\ {\isasymone}\ y{\isachardoublequoteclose}\isanewline
\isanewline
\isacommand{lemma}\isamarkupfalse%
\ fiber{\isacharunderscore}{\kern0pt}morphism{\isacharunderscore}{\kern0pt}type{\isacharbrackleft}{\kern0pt}type{\isacharunderscore}{\kern0pt}rule{\isacharbrackright}{\kern0pt}{\isacharcolon}{\kern0pt}\isanewline
\ \ \isakeyword{assumes}\ {\isachardoublequoteopen}f\ {\isacharcolon}{\kern0pt}\ X\ {\isasymrightarrow}\ Y{\isachardoublequoteclose}\ {\isachardoublequoteopen}y\ {\isasymin}\isactrlsub c\ Y{\isachardoublequoteclose}\isanewline
\ \ \isakeyword{shows}\ {\isachardoublequoteopen}fiber{\isacharunderscore}{\kern0pt}morphism\ f\ y\ {\isacharcolon}{\kern0pt}\ f\isactrlsup {\isacharminus}{\kern0pt}\isactrlsup {\isadigit{1}}{\isacharbraceleft}{\kern0pt}y{\isacharbraceright}{\kern0pt}\ {\isasymrightarrow}\ X{\isachardoublequoteclose}\isanewline
%
\isadelimproof
\ \ %
\endisadelimproof
%
\isatagproof
\isacommand{unfolding}\isamarkupfalse%
\ fiber{\isacharunderscore}{\kern0pt}def\ fiber{\isacharunderscore}{\kern0pt}morphism{\isacharunderscore}{\kern0pt}def\isanewline
\ \ \isacommand{using}\isamarkupfalse%
\ assms\ cfunc{\isacharunderscore}{\kern0pt}type{\isacharunderscore}{\kern0pt}def\ element{\isacharunderscore}{\kern0pt}monomorphism\ inverse{\isacharunderscore}{\kern0pt}image{\isacharunderscore}{\kern0pt}subobject\ subobject{\isacharunderscore}{\kern0pt}of{\isacharunderscore}{\kern0pt}def{\isadigit{2}}\isanewline
\ \ \isacommand{by}\isamarkupfalse%
\ {\isacharparenleft}{\kern0pt}typecheck{\isacharunderscore}{\kern0pt}cfuncs{\isacharcomma}{\kern0pt}\ auto{\isacharparenright}{\kern0pt}%
\endisatagproof
{\isafoldproof}%
%
\isadelimproof
\isanewline
%
\endisadelimproof
\isanewline
\isacommand{lemma}\isamarkupfalse%
\ fiber{\isacharunderscore}{\kern0pt}subset{\isacharcolon}{\kern0pt}\ \isanewline
\ \ \isakeyword{assumes}\ {\isachardoublequoteopen}f\ {\isacharcolon}{\kern0pt}\ X\ {\isasymrightarrow}\ Y{\isachardoublequoteclose}\ {\isachardoublequoteopen}y\ {\isasymin}\isactrlsub c\ Y{\isachardoublequoteclose}\isanewline
\ \ \isakeyword{shows}\ {\isachardoublequoteopen}{\isacharparenleft}{\kern0pt}f\isactrlsup {\isacharminus}{\kern0pt}\isactrlsup {\isadigit{1}}{\isacharbraceleft}{\kern0pt}y{\isacharbraceright}{\kern0pt}{\isacharcomma}{\kern0pt}\ fiber{\isacharunderscore}{\kern0pt}morphism\ f\ y{\isacharparenright}{\kern0pt}\ {\isasymsubseteq}\isactrlsub c\ X{\isachardoublequoteclose}\isanewline
%
\isadelimproof
\ \ %
\endisadelimproof
%
\isatagproof
\isacommand{unfolding}\isamarkupfalse%
\ fiber{\isacharunderscore}{\kern0pt}def\ fiber{\isacharunderscore}{\kern0pt}morphism{\isacharunderscore}{\kern0pt}def\isanewline
\ \ \isacommand{using}\isamarkupfalse%
\ assms\ cfunc{\isacharunderscore}{\kern0pt}type{\isacharunderscore}{\kern0pt}def\ element{\isacharunderscore}{\kern0pt}monomorphism\ inverse{\isacharunderscore}{\kern0pt}image{\isacharunderscore}{\kern0pt}subobject\ inverse{\isacharunderscore}{\kern0pt}image{\isacharunderscore}{\kern0pt}subobject{\isacharunderscore}{\kern0pt}mapping{\isacharunderscore}{\kern0pt}def\isanewline
\ \ \isacommand{by}\isamarkupfalse%
\ {\isacharparenleft}{\kern0pt}typecheck{\isacharunderscore}{\kern0pt}cfuncs{\isacharcomma}{\kern0pt}\ auto{\isacharparenright}{\kern0pt}%
\endisatagproof
{\isafoldproof}%
%
\isadelimproof
\isanewline
%
\endisadelimproof
\isanewline
\isacommand{lemma}\isamarkupfalse%
\ fiber{\isacharunderscore}{\kern0pt}morphism{\isacharunderscore}{\kern0pt}monomorphism{\isacharcolon}{\kern0pt}\isanewline
\ \ \isakeyword{assumes}\ {\isachardoublequoteopen}f\ {\isacharcolon}{\kern0pt}\ X\ {\isasymrightarrow}\ Y{\isachardoublequoteclose}\ {\isachardoublequoteopen}y\ {\isasymin}\isactrlsub c\ Y{\isachardoublequoteclose}\isanewline
\ \ \isakeyword{shows}\ {\isachardoublequoteopen}monomorphism\ {\isacharparenleft}{\kern0pt}fiber{\isacharunderscore}{\kern0pt}morphism\ f\ y{\isacharparenright}{\kern0pt}{\isachardoublequoteclose}\isanewline
%
\isadelimproof
\ \ %
\endisadelimproof
%
\isatagproof
\isacommand{using}\isamarkupfalse%
\ assms\ cfunc{\isacharunderscore}{\kern0pt}type{\isacharunderscore}{\kern0pt}def\ element{\isacharunderscore}{\kern0pt}monomorphism\ fiber{\isacharunderscore}{\kern0pt}morphism{\isacharunderscore}{\kern0pt}def\ inverse{\isacharunderscore}{\kern0pt}image{\isacharunderscore}{\kern0pt}monomorphism\ \isacommand{by}\isamarkupfalse%
\ auto%
\endisatagproof
{\isafoldproof}%
%
\isadelimproof
\isanewline
%
\endisadelimproof
\isanewline
\isacommand{lemma}\isamarkupfalse%
\ fiber{\isacharunderscore}{\kern0pt}morphism{\isacharunderscore}{\kern0pt}eq{\isacharcolon}{\kern0pt}\isanewline
\ \ \isakeyword{assumes}\ {\isachardoublequoteopen}f\ {\isacharcolon}{\kern0pt}\ X\ {\isasymrightarrow}\ Y{\isachardoublequoteclose}\ {\isachardoublequoteopen}y\ {\isasymin}\isactrlsub c\ Y{\isachardoublequoteclose}\isanewline
\ \ \isakeyword{shows}\ {\isachardoublequoteopen}f\ {\isasymcirc}\isactrlsub c\ fiber{\isacharunderscore}{\kern0pt}morphism\ f\ y\ \ {\isacharequal}{\kern0pt}\ y\ {\isasymcirc}\isactrlsub c\ {\isasymbeta}\isactrlbsub f\isactrlsup {\isacharminus}{\kern0pt}\isactrlsup {\isadigit{1}}{\isacharbraceleft}{\kern0pt}y{\isacharbraceright}{\kern0pt}\isactrlesub {\isachardoublequoteclose}\isanewline
%
\isadelimproof
%
\endisadelimproof
%
\isatagproof
\isacommand{proof}\isamarkupfalse%
\ {\isacharminus}{\kern0pt}\isanewline
\ \ \isacommand{have}\isamarkupfalse%
\ {\isachardoublequoteopen}f\ {\isasymcirc}\isactrlsub c\ fiber{\isacharunderscore}{\kern0pt}morphism\ f\ y\ {\isacharequal}{\kern0pt}\ f\ {\isasymcirc}\isactrlsub c\ left{\isacharunderscore}{\kern0pt}cart{\isacharunderscore}{\kern0pt}proj\ {\isacharparenleft}{\kern0pt}domain\ f{\isacharparenright}{\kern0pt}\ {\isasymone}\ {\isasymcirc}\isactrlsub c\ inverse{\isacharunderscore}{\kern0pt}image{\isacharunderscore}{\kern0pt}mapping\ f\ {\isasymone}\ y{\isachardoublequoteclose}\isanewline
\ \ \ \ \isacommand{unfolding}\isamarkupfalse%
\ fiber{\isacharunderscore}{\kern0pt}morphism{\isacharunderscore}{\kern0pt}def\ \isacommand{by}\isamarkupfalse%
\ auto\isanewline
\ \ \isacommand{also}\isamarkupfalse%
\ \isacommand{have}\isamarkupfalse%
\ {\isachardoublequoteopen}{\isachardot}{\kern0pt}{\isachardot}{\kern0pt}{\isachardot}{\kern0pt}\ {\isacharequal}{\kern0pt}\ y\ {\isasymcirc}\isactrlsub c\ right{\isacharunderscore}{\kern0pt}cart{\isacharunderscore}{\kern0pt}proj\ X\ {\isasymone}\ {\isasymcirc}\isactrlsub c\ inverse{\isacharunderscore}{\kern0pt}image{\isacharunderscore}{\kern0pt}mapping\ f\ {\isasymone}\ y{\isachardoublequoteclose}\isanewline
\ \ \ \ \isacommand{using}\isamarkupfalse%
\ assms\ cfunc{\isacharunderscore}{\kern0pt}type{\isacharunderscore}{\kern0pt}def\ element{\isacharunderscore}{\kern0pt}monomorphism\ inverse{\isacharunderscore}{\kern0pt}image{\isacharunderscore}{\kern0pt}mapping{\isacharunderscore}{\kern0pt}eq\ \isacommand{by}\isamarkupfalse%
\ auto\isanewline
\ \ \isacommand{also}\isamarkupfalse%
\ \isacommand{have}\isamarkupfalse%
\ {\isachardoublequoteopen}{\isachardot}{\kern0pt}{\isachardot}{\kern0pt}{\isachardot}{\kern0pt}\ {\isacharequal}{\kern0pt}\ y\ {\isasymcirc}\isactrlsub c\ {\isasymbeta}\isactrlbsub f\isactrlsup {\isacharminus}{\kern0pt}\isactrlsup {\isadigit{1}}{\isasymlparr}{\isasymone}{\isasymrparr}\isactrlbsub y\isactrlesub \isactrlesub {\isachardoublequoteclose}\isanewline
\ \ \ \ \isacommand{using}\isamarkupfalse%
\ assms\ \isacommand{by}\isamarkupfalse%
\ {\isacharparenleft}{\kern0pt}typecheck{\isacharunderscore}{\kern0pt}cfuncs{\isacharcomma}{\kern0pt}\ metis\ element{\isacharunderscore}{\kern0pt}monomorphism\ terminal{\isacharunderscore}{\kern0pt}func{\isacharunderscore}{\kern0pt}unique{\isacharparenright}{\kern0pt}\isanewline
\ \ \isacommand{also}\isamarkupfalse%
\ \isacommand{have}\isamarkupfalse%
\ {\isachardoublequoteopen}{\isachardot}{\kern0pt}{\isachardot}{\kern0pt}{\isachardot}{\kern0pt}\ {\isacharequal}{\kern0pt}\ y\ {\isasymcirc}\isactrlsub c\ {\isasymbeta}\isactrlbsub f\isactrlsup {\isacharminus}{\kern0pt}\isactrlsup {\isadigit{1}}{\isacharbraceleft}{\kern0pt}y{\isacharbraceright}{\kern0pt}\isactrlesub {\isachardoublequoteclose}\isanewline
\ \ \ \ \isacommand{unfolding}\isamarkupfalse%
\ fiber{\isacharunderscore}{\kern0pt}def\ \isacommand{by}\isamarkupfalse%
\ auto\isanewline
\ \ \isacommand{then}\isamarkupfalse%
\ \isacommand{show}\isamarkupfalse%
\ {\isacharquery}{\kern0pt}thesis\isanewline
\ \ \ \ \isacommand{using}\isamarkupfalse%
\ calculation\ \isacommand{by}\isamarkupfalse%
\ auto\isanewline
\isacommand{qed}\isamarkupfalse%
%
\endisatagproof
{\isafoldproof}%
%
\isadelimproof
%
\endisadelimproof
%
\begin{isamarkuptext}%
The lemma below corresponds to Proposition 2.2.7 in Halvorson.%
\end{isamarkuptext}\isamarkuptrue%
\isacommand{lemma}\isamarkupfalse%
\ not{\isacharunderscore}{\kern0pt}surjective{\isacharunderscore}{\kern0pt}has{\isacharunderscore}{\kern0pt}some{\isacharunderscore}{\kern0pt}empty{\isacharunderscore}{\kern0pt}preimage{\isacharcolon}{\kern0pt}\isanewline
\ \ \isakeyword{assumes}\ p{\isacharunderscore}{\kern0pt}type{\isacharbrackleft}{\kern0pt}type{\isacharunderscore}{\kern0pt}rule{\isacharbrackright}{\kern0pt}{\isacharcolon}{\kern0pt}\ {\isachardoublequoteopen}p{\isacharcolon}{\kern0pt}\ X\ {\isasymrightarrow}\ Y{\isachardoublequoteclose}\ \isakeyword{and}\ p{\isacharunderscore}{\kern0pt}not{\isacharunderscore}{\kern0pt}surj{\isacharcolon}{\kern0pt}\ {\isachardoublequoteopen}{\isasymnot}\ surjective\ p{\isachardoublequoteclose}\isanewline
\ \ \isakeyword{shows}\ {\isachardoublequoteopen}{\isasymexists}\ y{\isachardot}{\kern0pt}\ y\ {\isasymin}\isactrlsub c\ Y\ {\isasymand}\ is{\isacharunderscore}{\kern0pt}empty{\isacharparenleft}{\kern0pt}p\isactrlsup {\isacharminus}{\kern0pt}\isactrlsup {\isadigit{1}}{\isacharbraceleft}{\kern0pt}y{\isacharbraceright}{\kern0pt}{\isacharparenright}{\kern0pt}{\isachardoublequoteclose}\isanewline
%
\isadelimproof
%
\endisadelimproof
%
\isatagproof
\isacommand{proof}\isamarkupfalse%
\ {\isacharminus}{\kern0pt}\isanewline
\ \ \isacommand{have}\isamarkupfalse%
\ nonempty{\isacharcolon}{\kern0pt}\ {\isachardoublequoteopen}nonempty{\isacharparenleft}{\kern0pt}Y{\isacharparenright}{\kern0pt}{\isachardoublequoteclose}\isanewline
\ \ \ \ \isacommand{using}\isamarkupfalse%
\ assms\ cfunc{\isacharunderscore}{\kern0pt}type{\isacharunderscore}{\kern0pt}def\ nonempty{\isacharunderscore}{\kern0pt}def\ surjective{\isacharunderscore}{\kern0pt}def\ \isacommand{by}\isamarkupfalse%
\ auto\isanewline
\ \ \isacommand{obtain}\isamarkupfalse%
\ y{\isadigit{0}}\ \isakeyword{where}\ y{\isadigit{0}}{\isacharunderscore}{\kern0pt}type{\isacharbrackleft}{\kern0pt}type{\isacharunderscore}{\kern0pt}rule{\isacharbrackright}{\kern0pt}{\isacharcolon}{\kern0pt}\ {\isachardoublequoteopen}y{\isadigit{0}}\ {\isasymin}\isactrlsub c\ Y{\isachardoublequoteclose}\ {\isachardoublequoteopen}{\isasymforall}\ x{\isachardot}{\kern0pt}\ x\ {\isasymin}\isactrlsub c\ X\ {\isasymlongrightarrow}\ p{\isasymcirc}\isactrlsub c\ x\ {\isasymnoteq}\ y{\isadigit{0}}{\isachardoublequoteclose}\isanewline
\ \ \ \ \isacommand{using}\isamarkupfalse%
\ assms\ cfunc{\isacharunderscore}{\kern0pt}type{\isacharunderscore}{\kern0pt}def\ surjective{\isacharunderscore}{\kern0pt}def\ \isacommand{by}\isamarkupfalse%
\ auto\isanewline
\isanewline
\ \ \isacommand{have}\isamarkupfalse%
\ {\isachardoublequoteopen}{\isasymnot}nonempty{\isacharparenleft}{\kern0pt}p\isactrlsup {\isacharminus}{\kern0pt}\isactrlsup {\isadigit{1}}{\isacharbraceleft}{\kern0pt}y{\isadigit{0}}{\isacharbraceright}{\kern0pt}{\isacharparenright}{\kern0pt}{\isachardoublequoteclose}\isanewline
\ \ \isacommand{proof}\isamarkupfalse%
\ {\isacharparenleft}{\kern0pt}rule\ ccontr{\isacharcomma}{\kern0pt}\ clarify{\isacharparenright}{\kern0pt}\isanewline
\ \ \ \ \isacommand{assume}\isamarkupfalse%
\ a{\isadigit{1}}{\isacharcolon}{\kern0pt}\ {\isachardoublequoteopen}nonempty{\isacharparenleft}{\kern0pt}p\isactrlsup {\isacharminus}{\kern0pt}\isactrlsup {\isadigit{1}}{\isacharbraceleft}{\kern0pt}y{\isadigit{0}}{\isacharbraceright}{\kern0pt}{\isacharparenright}{\kern0pt}{\isachardoublequoteclose}\isanewline
\ \ \ \ \isacommand{obtain}\isamarkupfalse%
\ z\ \isakeyword{where}\ z{\isacharunderscore}{\kern0pt}type{\isacharbrackleft}{\kern0pt}type{\isacharunderscore}{\kern0pt}rule{\isacharbrackright}{\kern0pt}{\isacharcolon}{\kern0pt}\ {\isachardoublequoteopen}z\ {\isasymin}\isactrlsub c\ p\isactrlsup {\isacharminus}{\kern0pt}\isactrlsup {\isadigit{1}}{\isacharbraceleft}{\kern0pt}y{\isadigit{0}}{\isacharbraceright}{\kern0pt}{\isachardoublequoteclose}\isanewline
\ \ \ \ \ \ \isacommand{using}\isamarkupfalse%
\ a{\isadigit{1}}\ nonempty{\isacharunderscore}{\kern0pt}def\ \isacommand{by}\isamarkupfalse%
\ blast\isanewline
\ \ \ \ \isacommand{have}\isamarkupfalse%
\ fiber{\isacharunderscore}{\kern0pt}z{\isacharunderscore}{\kern0pt}type{\isacharcolon}{\kern0pt}\ {\isachardoublequoteopen}fiber{\isacharunderscore}{\kern0pt}morphism\ p\ y{\isadigit{0}}\ {\isasymcirc}\isactrlsub c\ z\ {\isasymin}\isactrlsub c\ X{\isachardoublequoteclose}\isanewline
\ \ \ \ \ \ \isacommand{using}\isamarkupfalse%
\ assms{\isacharparenleft}{\kern0pt}{\isadigit{1}}{\isacharparenright}{\kern0pt}\ comp{\isacharunderscore}{\kern0pt}type\ fiber{\isacharunderscore}{\kern0pt}morphism{\isacharunderscore}{\kern0pt}type\ y{\isadigit{0}}{\isacharunderscore}{\kern0pt}type\ z{\isacharunderscore}{\kern0pt}type\ \isacommand{by}\isamarkupfalse%
\ auto\isanewline
\ \ \ \ \isacommand{have}\isamarkupfalse%
\ contradiction{\isacharcolon}{\kern0pt}\ {\isachardoublequoteopen}p\ {\isasymcirc}\isactrlsub c\ fiber{\isacharunderscore}{\kern0pt}morphism\ p\ y{\isadigit{0}}\ {\isasymcirc}\isactrlsub c\ z\ {\isacharequal}{\kern0pt}\ y{\isadigit{0}}{\isachardoublequoteclose}\isanewline
\ \ \ \ \ \ \isacommand{by}\isamarkupfalse%
\ {\isacharparenleft}{\kern0pt}typecheck{\isacharunderscore}{\kern0pt}cfuncs{\isacharcomma}{\kern0pt}\ smt\ {\isacharparenleft}{\kern0pt}z{\isadigit{3}}{\isacharparenright}{\kern0pt}\ comp{\isacharunderscore}{\kern0pt}associative{\isadigit{2}}\ fiber{\isacharunderscore}{\kern0pt}morphism{\isacharunderscore}{\kern0pt}eq\ id{\isacharunderscore}{\kern0pt}right{\isacharunderscore}{\kern0pt}unit{\isadigit{2}}\ id{\isacharunderscore}{\kern0pt}type\ one{\isacharunderscore}{\kern0pt}unique{\isacharunderscore}{\kern0pt}element\ terminal{\isacharunderscore}{\kern0pt}func{\isacharunderscore}{\kern0pt}comp\ terminal{\isacharunderscore}{\kern0pt}func{\isacharunderscore}{\kern0pt}type{\isacharparenright}{\kern0pt}\isanewline
\ \ \ \ \isacommand{have}\isamarkupfalse%
\ {\isachardoublequoteopen}p\ {\isasymcirc}\isactrlsub c\ {\isacharparenleft}{\kern0pt}fiber{\isacharunderscore}{\kern0pt}morphism\ p\ y{\isadigit{0}}\ {\isasymcirc}\isactrlsub c\ z{\isacharparenright}{\kern0pt}\ {\isasymnoteq}\ y{\isadigit{0}}{\isachardoublequoteclose}\isanewline
\ \ \ \ \ \ \isacommand{by}\isamarkupfalse%
\ {\isacharparenleft}{\kern0pt}simp\ add{\isacharcolon}{\kern0pt}\ fiber{\isacharunderscore}{\kern0pt}z{\isacharunderscore}{\kern0pt}type\ y{\isadigit{0}}{\isacharunderscore}{\kern0pt}type{\isacharparenright}{\kern0pt}\isanewline
\ \ \ \ \isacommand{then}\isamarkupfalse%
\ \isacommand{show}\isamarkupfalse%
\ False\isanewline
\ \ \ \ \ \ \isacommand{using}\isamarkupfalse%
\ contradiction\ \isacommand{by}\isamarkupfalse%
\ blast\isanewline
\ \ \isacommand{qed}\isamarkupfalse%
\isanewline
\ \ \isacommand{then}\isamarkupfalse%
\ \isacommand{show}\isamarkupfalse%
\ {\isacharquery}{\kern0pt}thesis\isanewline
\ \ \ \ \isacommand{using}\isamarkupfalse%
\ is{\isacharunderscore}{\kern0pt}empty{\isacharunderscore}{\kern0pt}def\ nonempty{\isacharunderscore}{\kern0pt}def\ y{\isadigit{0}}{\isacharunderscore}{\kern0pt}type\ \isacommand{by}\isamarkupfalse%
\ blast\isanewline
\isacommand{qed}\isamarkupfalse%
%
\endisatagproof
{\isafoldproof}%
%
\isadelimproof
\isanewline
%
\endisadelimproof
\isanewline
\isacommand{lemma}\isamarkupfalse%
\ fiber{\isacharunderscore}{\kern0pt}iso{\isacharunderscore}{\kern0pt}fibered{\isacharunderscore}{\kern0pt}prod{\isacharcolon}{\kern0pt}\isanewline
\ \ \isakeyword{assumes}\ f{\isacharunderscore}{\kern0pt}type{\isacharbrackleft}{\kern0pt}type{\isacharunderscore}{\kern0pt}rule{\isacharbrackright}{\kern0pt}{\isacharcolon}{\kern0pt}\ {\isachardoublequoteopen}f\ {\isacharcolon}{\kern0pt}\ X\ {\isasymrightarrow}\ Y{\isachardoublequoteclose}\isanewline
\ \ \isakeyword{assumes}\ y{\isacharunderscore}{\kern0pt}type{\isacharbrackleft}{\kern0pt}type{\isacharunderscore}{\kern0pt}rule{\isacharbrackright}{\kern0pt}{\isacharcolon}{\kern0pt}\ {\isachardoublequoteopen}y\ {\isacharcolon}{\kern0pt}\ {\isasymone}\ {\isasymrightarrow}\ Y{\isachardoublequoteclose}\isanewline
\ \ \isakeyword{shows}\ {\isachardoublequoteopen}f\isactrlsup {\isacharminus}{\kern0pt}\isactrlsup {\isadigit{1}}{\isacharbraceleft}{\kern0pt}y{\isacharbraceright}{\kern0pt}\ {\isasymcong}\ X\ \isactrlbsub f\isactrlesub {\isasymtimes}\isactrlsub c\isactrlbsub y\isactrlesub {\isasymone}{\isachardoublequoteclose}\isanewline
%
\isadelimproof
\ \ %
\endisadelimproof
%
\isatagproof
\isacommand{using}\isamarkupfalse%
\ element{\isacharunderscore}{\kern0pt}monomorphism\ equalizers{\isacharunderscore}{\kern0pt}isomorphic\ f{\isacharunderscore}{\kern0pt}type\ fiber{\isacharunderscore}{\kern0pt}def\ fibered{\isacharunderscore}{\kern0pt}product{\isacharunderscore}{\kern0pt}equalizer\ inverse{\isacharunderscore}{\kern0pt}image{\isacharunderscore}{\kern0pt}is{\isacharunderscore}{\kern0pt}equalizer\ is{\isacharunderscore}{\kern0pt}isomorphic{\isacharunderscore}{\kern0pt}def\ y{\isacharunderscore}{\kern0pt}type\ \isacommand{by}\isamarkupfalse%
\ moura%
\endisatagproof
{\isafoldproof}%
%
\isadelimproof
\isanewline
%
\endisadelimproof
\isanewline
\isacommand{lemma}\isamarkupfalse%
\ fib{\isacharunderscore}{\kern0pt}prod{\isacharunderscore}{\kern0pt}left{\isacharunderscore}{\kern0pt}id{\isacharunderscore}{\kern0pt}iso{\isacharcolon}{\kern0pt}\isanewline
\ \ \isakeyword{assumes}\ {\isachardoublequoteopen}g\ {\isacharcolon}{\kern0pt}\ Y\ {\isasymrightarrow}\ X{\isachardoublequoteclose}\isanewline
\ \ \isakeyword{shows}\ {\isachardoublequoteopen}{\isacharparenleft}{\kern0pt}X\ \isactrlbsub id{\isacharparenleft}{\kern0pt}X{\isacharparenright}{\kern0pt}\isactrlesub {\isasymtimes}\isactrlsub c\isactrlbsub g\isactrlesub \ Y{\isacharparenright}{\kern0pt}\ {\isasymcong}\ Y{\isachardoublequoteclose}\isanewline
%
\isadelimproof
%
\endisadelimproof
%
\isatagproof
\isacommand{proof}\isamarkupfalse%
\ {\isacharminus}{\kern0pt}\ \isanewline
\ \ \isacommand{have}\isamarkupfalse%
\ is{\isacharunderscore}{\kern0pt}pullback{\isacharcolon}{\kern0pt}\ {\isachardoublequoteopen}is{\isacharunderscore}{\kern0pt}pullback\ {\isacharparenleft}{\kern0pt}X\ \isactrlbsub id{\isacharparenleft}{\kern0pt}X{\isacharparenright}{\kern0pt}\isactrlesub {\isasymtimes}\isactrlsub c\isactrlbsub g\isactrlesub \ Y{\isacharparenright}{\kern0pt}\ Y\ X\ X\ {\isacharparenleft}{\kern0pt}fibered{\isacharunderscore}{\kern0pt}product{\isacharunderscore}{\kern0pt}right{\isacharunderscore}{\kern0pt}proj\ X\ {\isacharparenleft}{\kern0pt}id{\isacharparenleft}{\kern0pt}X{\isacharparenright}{\kern0pt}{\isacharparenright}{\kern0pt}\ g\ Y{\isacharparenright}{\kern0pt}\ g\ {\isacharparenleft}{\kern0pt}fibered{\isacharunderscore}{\kern0pt}product{\isacharunderscore}{\kern0pt}left{\isacharunderscore}{\kern0pt}proj\ X\ {\isacharparenleft}{\kern0pt}id{\isacharparenleft}{\kern0pt}X{\isacharparenright}{\kern0pt}{\isacharparenright}{\kern0pt}\ g\ Y{\isacharparenright}{\kern0pt}\ {\isacharparenleft}{\kern0pt}id{\isacharparenleft}{\kern0pt}X{\isacharparenright}{\kern0pt}{\isacharparenright}{\kern0pt}{\isachardoublequoteclose}\isanewline
\ \ \ \ \isacommand{using}\isamarkupfalse%
\ assms\ fibered{\isacharunderscore}{\kern0pt}product{\isacharunderscore}{\kern0pt}is{\isacharunderscore}{\kern0pt}pullback\ \isacommand{by}\isamarkupfalse%
\ {\isacharparenleft}{\kern0pt}typecheck{\isacharunderscore}{\kern0pt}cfuncs{\isacharcomma}{\kern0pt}\ blast{\isacharparenright}{\kern0pt}\isanewline
\ \ \isacommand{then}\isamarkupfalse%
\ \isacommand{have}\isamarkupfalse%
\ mono{\isacharcolon}{\kern0pt}\ {\isachardoublequoteopen}monomorphism{\isacharparenleft}{\kern0pt}fibered{\isacharunderscore}{\kern0pt}product{\isacharunderscore}{\kern0pt}right{\isacharunderscore}{\kern0pt}proj\ X\ {\isacharparenleft}{\kern0pt}id{\isacharparenleft}{\kern0pt}X{\isacharparenright}{\kern0pt}{\isacharparenright}{\kern0pt}\ g\ Y{\isacharparenright}{\kern0pt}{\isachardoublequoteclose}\isanewline
\ \ \ \ \isacommand{using}\isamarkupfalse%
\ assms\ \isacommand{by}\isamarkupfalse%
\ {\isacharparenleft}{\kern0pt}typecheck{\isacharunderscore}{\kern0pt}cfuncs{\isacharcomma}{\kern0pt}\ meson\ id{\isacharunderscore}{\kern0pt}isomorphism\ iso{\isacharunderscore}{\kern0pt}imp{\isacharunderscore}{\kern0pt}epi{\isacharunderscore}{\kern0pt}and{\isacharunderscore}{\kern0pt}monic\ pullback{\isacharunderscore}{\kern0pt}of{\isacharunderscore}{\kern0pt}mono{\isacharunderscore}{\kern0pt}is{\isacharunderscore}{\kern0pt}mono{\isadigit{2}}{\isacharparenright}{\kern0pt}\isanewline
\ \ \isacommand{have}\isamarkupfalse%
\ {\isachardoublequoteopen}epimorphism{\isacharparenleft}{\kern0pt}fibered{\isacharunderscore}{\kern0pt}product{\isacharunderscore}{\kern0pt}right{\isacharunderscore}{\kern0pt}proj\ X\ {\isacharparenleft}{\kern0pt}id{\isacharparenleft}{\kern0pt}X{\isacharparenright}{\kern0pt}{\isacharparenright}{\kern0pt}\ g\ Y{\isacharparenright}{\kern0pt}{\isachardoublequoteclose}\isanewline
\ \ \ \ \isacommand{by}\isamarkupfalse%
\ {\isacharparenleft}{\kern0pt}meson\ id{\isacharunderscore}{\kern0pt}isomorphism\ id{\isacharunderscore}{\kern0pt}type\ is{\isacharunderscore}{\kern0pt}pullback\ iso{\isacharunderscore}{\kern0pt}imp{\isacharunderscore}{\kern0pt}epi{\isacharunderscore}{\kern0pt}and{\isacharunderscore}{\kern0pt}monic\ pullback{\isacharunderscore}{\kern0pt}of{\isacharunderscore}{\kern0pt}epi{\isacharunderscore}{\kern0pt}is{\isacharunderscore}{\kern0pt}epi{\isadigit{2}}{\isacharparenright}{\kern0pt}\isanewline
\ \ \isacommand{then}\isamarkupfalse%
\ \isacommand{have}\isamarkupfalse%
\ {\isachardoublequoteopen}isomorphism{\isacharparenleft}{\kern0pt}fibered{\isacharunderscore}{\kern0pt}product{\isacharunderscore}{\kern0pt}right{\isacharunderscore}{\kern0pt}proj\ X\ {\isacharparenleft}{\kern0pt}id{\isacharparenleft}{\kern0pt}X{\isacharparenright}{\kern0pt}{\isacharparenright}{\kern0pt}\ g\ Y{\isacharparenright}{\kern0pt}{\isachardoublequoteclose}\isanewline
\ \ \ \ \isacommand{by}\isamarkupfalse%
\ {\isacharparenleft}{\kern0pt}simp\ add{\isacharcolon}{\kern0pt}\ epi{\isacharunderscore}{\kern0pt}mon{\isacharunderscore}{\kern0pt}is{\isacharunderscore}{\kern0pt}iso\ mono{\isacharparenright}{\kern0pt}\isanewline
\ \ \isacommand{then}\isamarkupfalse%
\ \isacommand{show}\isamarkupfalse%
\ {\isacharquery}{\kern0pt}thesis\isanewline
\ \ \ \ \isacommand{using}\isamarkupfalse%
\ assms\ fibered{\isacharunderscore}{\kern0pt}product{\isacharunderscore}{\kern0pt}right{\isacharunderscore}{\kern0pt}proj{\isacharunderscore}{\kern0pt}type\ id{\isacharunderscore}{\kern0pt}type\ is{\isacharunderscore}{\kern0pt}isomorphic{\isacharunderscore}{\kern0pt}def\ \isacommand{by}\isamarkupfalse%
\ blast\isanewline
\isacommand{qed}\isamarkupfalse%
%
\endisatagproof
{\isafoldproof}%
%
\isadelimproof
\isanewline
%
\endisadelimproof
\isanewline
\isacommand{lemma}\isamarkupfalse%
\ fib{\isacharunderscore}{\kern0pt}prod{\isacharunderscore}{\kern0pt}right{\isacharunderscore}{\kern0pt}id{\isacharunderscore}{\kern0pt}iso{\isacharcolon}{\kern0pt}\isanewline
\ \ \isakeyword{assumes}\ {\isachardoublequoteopen}f\ {\isacharcolon}{\kern0pt}\ X\ {\isasymrightarrow}\ Y{\isachardoublequoteclose}\isanewline
\ \ \isakeyword{shows}\ {\isachardoublequoteopen}{\isacharparenleft}{\kern0pt}X\ \isactrlbsub f\isactrlesub {\isasymtimes}\isactrlsub c\isactrlbsub id{\isacharparenleft}{\kern0pt}Y{\isacharparenright}{\kern0pt}\isactrlesub \ Y{\isacharparenright}{\kern0pt}\ {\isasymcong}\ X{\isachardoublequoteclose}\isanewline
%
\isadelimproof
%
\endisadelimproof
%
\isatagproof
\isacommand{proof}\isamarkupfalse%
\ {\isacharminus}{\kern0pt}\ \isanewline
\ \ \isacommand{have}\isamarkupfalse%
\ is{\isacharunderscore}{\kern0pt}pullback{\isacharcolon}{\kern0pt}\ {\isachardoublequoteopen}is{\isacharunderscore}{\kern0pt}pullback\ {\isacharparenleft}{\kern0pt}X\ \isactrlbsub f\isactrlesub {\isasymtimes}\isactrlsub c\isactrlbsub id{\isacharparenleft}{\kern0pt}Y{\isacharparenright}{\kern0pt}\isactrlesub \ Y{\isacharparenright}{\kern0pt}\ Y\ X\ Y\ {\isacharparenleft}{\kern0pt}fibered{\isacharunderscore}{\kern0pt}product{\isacharunderscore}{\kern0pt}right{\isacharunderscore}{\kern0pt}proj\ X\ f\ {\isacharparenleft}{\kern0pt}id{\isacharparenleft}{\kern0pt}Y{\isacharparenright}{\kern0pt}{\isacharparenright}{\kern0pt}\ Y{\isacharparenright}{\kern0pt}\ {\isacharparenleft}{\kern0pt}id{\isacharparenleft}{\kern0pt}Y{\isacharparenright}{\kern0pt}{\isacharparenright}{\kern0pt}\ {\isacharparenleft}{\kern0pt}fibered{\isacharunderscore}{\kern0pt}product{\isacharunderscore}{\kern0pt}left{\isacharunderscore}{\kern0pt}proj\ X\ f\ {\isacharparenleft}{\kern0pt}id{\isacharparenleft}{\kern0pt}Y{\isacharparenright}{\kern0pt}{\isacharparenright}{\kern0pt}\ Y{\isacharparenright}{\kern0pt}\ f\ {\isachardoublequoteclose}\isanewline
\ \ \ \ \isacommand{using}\isamarkupfalse%
\ assms\ fibered{\isacharunderscore}{\kern0pt}product{\isacharunderscore}{\kern0pt}is{\isacharunderscore}{\kern0pt}pullback\ \isacommand{by}\isamarkupfalse%
\ {\isacharparenleft}{\kern0pt}typecheck{\isacharunderscore}{\kern0pt}cfuncs{\isacharcomma}{\kern0pt}\ blast{\isacharparenright}{\kern0pt}\isanewline
\ \ \ \ \isanewline
\ \ \isacommand{then}\isamarkupfalse%
\ \isacommand{have}\isamarkupfalse%
\ mono{\isacharcolon}{\kern0pt}\ {\isachardoublequoteopen}monomorphism{\isacharparenleft}{\kern0pt}fibered{\isacharunderscore}{\kern0pt}product{\isacharunderscore}{\kern0pt}left{\isacharunderscore}{\kern0pt}proj\ X\ f\ {\isacharparenleft}{\kern0pt}id{\isacharparenleft}{\kern0pt}Y{\isacharparenright}{\kern0pt}{\isacharparenright}{\kern0pt}\ Y{\isacharparenright}{\kern0pt}{\isachardoublequoteclose}\isanewline
\ \ \ \ \isacommand{using}\isamarkupfalse%
\ assms\ \isacommand{by}\isamarkupfalse%
\ {\isacharparenleft}{\kern0pt}typecheck{\isacharunderscore}{\kern0pt}cfuncs{\isacharcomma}{\kern0pt}\ meson\ id{\isacharunderscore}{\kern0pt}isomorphism\ is{\isacharunderscore}{\kern0pt}pullback\ iso{\isacharunderscore}{\kern0pt}imp{\isacharunderscore}{\kern0pt}epi{\isacharunderscore}{\kern0pt}and{\isacharunderscore}{\kern0pt}monic\ pullback{\isacharunderscore}{\kern0pt}of{\isacharunderscore}{\kern0pt}mono{\isacharunderscore}{\kern0pt}is{\isacharunderscore}{\kern0pt}mono{\isadigit{1}}{\isacharparenright}{\kern0pt}\isanewline
\ \ \isacommand{have}\isamarkupfalse%
\ {\isachardoublequoteopen}epimorphism{\isacharparenleft}{\kern0pt}fibered{\isacharunderscore}{\kern0pt}product{\isacharunderscore}{\kern0pt}left{\isacharunderscore}{\kern0pt}proj\ X\ f\ {\isacharparenleft}{\kern0pt}id{\isacharparenleft}{\kern0pt}Y{\isacharparenright}{\kern0pt}{\isacharparenright}{\kern0pt}\ Y{\isacharparenright}{\kern0pt}{\isachardoublequoteclose}\isanewline
\ \ \ \ \isacommand{by}\isamarkupfalse%
\ {\isacharparenleft}{\kern0pt}meson\ id{\isacharunderscore}{\kern0pt}isomorphism\ id{\isacharunderscore}{\kern0pt}type\ is{\isacharunderscore}{\kern0pt}pullback\ iso{\isacharunderscore}{\kern0pt}imp{\isacharunderscore}{\kern0pt}epi{\isacharunderscore}{\kern0pt}and{\isacharunderscore}{\kern0pt}monic\ pullback{\isacharunderscore}{\kern0pt}of{\isacharunderscore}{\kern0pt}epi{\isacharunderscore}{\kern0pt}is{\isacharunderscore}{\kern0pt}epi{\isadigit{1}}{\isacharparenright}{\kern0pt}\isanewline
\ \ \isacommand{then}\isamarkupfalse%
\ \isacommand{have}\isamarkupfalse%
\ {\isachardoublequoteopen}isomorphism{\isacharparenleft}{\kern0pt}fibered{\isacharunderscore}{\kern0pt}product{\isacharunderscore}{\kern0pt}left{\isacharunderscore}{\kern0pt}proj\ X\ f\ {\isacharparenleft}{\kern0pt}id{\isacharparenleft}{\kern0pt}Y{\isacharparenright}{\kern0pt}{\isacharparenright}{\kern0pt}\ Y{\isacharparenright}{\kern0pt}{\isachardoublequoteclose}\isanewline
\ \ \ \ \isacommand{by}\isamarkupfalse%
\ {\isacharparenleft}{\kern0pt}simp\ add{\isacharcolon}{\kern0pt}\ epi{\isacharunderscore}{\kern0pt}mon{\isacharunderscore}{\kern0pt}is{\isacharunderscore}{\kern0pt}iso\ mono{\isacharparenright}{\kern0pt}\isanewline
\ \ \isacommand{then}\isamarkupfalse%
\ \isacommand{show}\isamarkupfalse%
\ {\isacharquery}{\kern0pt}thesis\isanewline
\ \ \ \ \isacommand{using}\isamarkupfalse%
\ assms\ fibered{\isacharunderscore}{\kern0pt}product{\isacharunderscore}{\kern0pt}left{\isacharunderscore}{\kern0pt}proj{\isacharunderscore}{\kern0pt}type\ id{\isacharunderscore}{\kern0pt}type\ is{\isacharunderscore}{\kern0pt}isomorphic{\isacharunderscore}{\kern0pt}def\ \isacommand{by}\isamarkupfalse%
\ blast\isanewline
\isacommand{qed}\isamarkupfalse%
%
\endisatagproof
{\isafoldproof}%
%
\isadelimproof
%
\endisadelimproof
%
\begin{isamarkuptext}%
The lemma below corresponds to the discussion at the top of page 42 in Halvorson.%
\end{isamarkuptext}\isamarkuptrue%
\isacommand{lemma}\isamarkupfalse%
\ kernel{\isacharunderscore}{\kern0pt}pair{\isacharunderscore}{\kern0pt}connection{\isacharcolon}{\kern0pt}\isanewline
\ \ \isakeyword{assumes}\ f{\isacharunderscore}{\kern0pt}type{\isacharbrackleft}{\kern0pt}type{\isacharunderscore}{\kern0pt}rule{\isacharbrackright}{\kern0pt}{\isacharcolon}{\kern0pt}\ {\isachardoublequoteopen}f\ {\isacharcolon}{\kern0pt}\ X\ {\isasymrightarrow}\ Y{\isachardoublequoteclose}\ \isakeyword{and}\ g{\isacharunderscore}{\kern0pt}type{\isacharbrackleft}{\kern0pt}type{\isacharunderscore}{\kern0pt}rule{\isacharbrackright}{\kern0pt}{\isacharcolon}{\kern0pt}\ {\isachardoublequoteopen}g\ {\isacharcolon}{\kern0pt}\ X\ {\isasymrightarrow}\ E{\isachardoublequoteclose}\isanewline
\ \ \isakeyword{assumes}\ g{\isacharunderscore}{\kern0pt}epi{\isacharcolon}{\kern0pt}\ {\isachardoublequoteopen}epimorphism\ g{\isachardoublequoteclose}\isanewline
\ \ \isakeyword{assumes}\ h{\isacharunderscore}{\kern0pt}g{\isacharunderscore}{\kern0pt}eq{\isacharunderscore}{\kern0pt}f{\isacharcolon}{\kern0pt}\ {\isachardoublequoteopen}h\ {\isasymcirc}\isactrlsub c\ g\ {\isacharequal}{\kern0pt}\ f{\isachardoublequoteclose}\isanewline
\ \ \isakeyword{assumes}\ g{\isacharunderscore}{\kern0pt}eq{\isacharcolon}{\kern0pt}\ {\isachardoublequoteopen}g\ {\isasymcirc}\isactrlsub c\ fibered{\isacharunderscore}{\kern0pt}product{\isacharunderscore}{\kern0pt}left{\isacharunderscore}{\kern0pt}proj\ X\ f\ f\ X\ {\isacharequal}{\kern0pt}\ g\ {\isasymcirc}\isactrlsub c\ fibered{\isacharunderscore}{\kern0pt}product{\isacharunderscore}{\kern0pt}right{\isacharunderscore}{\kern0pt}proj\ X\ f\ f\ X\ {\isachardoublequoteclose}\isanewline
\ \ \isakeyword{assumes}\ h{\isacharunderscore}{\kern0pt}type{\isacharbrackleft}{\kern0pt}type{\isacharunderscore}{\kern0pt}rule{\isacharbrackright}{\kern0pt}{\isacharcolon}{\kern0pt}\ {\isachardoublequoteopen}h\ {\isacharcolon}{\kern0pt}\ E\ {\isasymrightarrow}\ Y{\isachardoublequoteclose}\isanewline
\ \ \isakeyword{shows}\ {\isachardoublequoteopen}{\isasymexists}{\isacharbang}{\kern0pt}\ b{\isachardot}{\kern0pt}\ b\ {\isacharcolon}{\kern0pt}\ X\ \isactrlbsub f\isactrlesub {\isasymtimes}\isactrlsub c\isactrlbsub f\isactrlesub \ X\ {\isasymrightarrow}\ E\ \isactrlbsub h\isactrlesub {\isasymtimes}\isactrlsub c\isactrlbsub h\isactrlesub \ E\ {\isasymand}\isanewline
\ \ \ \ fibered{\isacharunderscore}{\kern0pt}product{\isacharunderscore}{\kern0pt}left{\isacharunderscore}{\kern0pt}proj\ E\ h\ h\ E\ {\isasymcirc}\isactrlsub c\ b\ {\isacharequal}{\kern0pt}\ g\ {\isasymcirc}\isactrlsub c\ fibered{\isacharunderscore}{\kern0pt}product{\isacharunderscore}{\kern0pt}left{\isacharunderscore}{\kern0pt}proj\ X\ f\ f\ X\ {\isasymand}\isanewline
\ \ \ \ fibered{\isacharunderscore}{\kern0pt}product{\isacharunderscore}{\kern0pt}right{\isacharunderscore}{\kern0pt}proj\ E\ h\ h\ E\ {\isasymcirc}\isactrlsub c\ b\ {\isacharequal}{\kern0pt}\ g\ {\isasymcirc}\isactrlsub c\ fibered{\isacharunderscore}{\kern0pt}product{\isacharunderscore}{\kern0pt}right{\isacharunderscore}{\kern0pt}proj\ X\ f\ f\ X\ {\isasymand}\isanewline
\ \ \ \ epimorphism\ b{\isachardoublequoteclose}\isanewline
%
\isadelimproof
%
\endisadelimproof
%
\isatagproof
\isacommand{proof}\isamarkupfalse%
\ {\isacharminus}{\kern0pt}\isanewline
\ \ \isacommand{have}\isamarkupfalse%
\ gxg{\isacharunderscore}{\kern0pt}fpmorph{\isacharunderscore}{\kern0pt}eq{\isacharcolon}{\kern0pt}\ {\isachardoublequoteopen}{\isacharparenleft}{\kern0pt}h\ {\isasymcirc}\isactrlsub c\ left{\isacharunderscore}{\kern0pt}cart{\isacharunderscore}{\kern0pt}proj\ E\ E{\isacharparenright}{\kern0pt}\ {\isasymcirc}\isactrlsub c\ {\isacharparenleft}{\kern0pt}g\ {\isasymtimes}\isactrlsub f\ g{\isacharparenright}{\kern0pt}\ {\isasymcirc}\isactrlsub c\ fibered{\isacharunderscore}{\kern0pt}product{\isacharunderscore}{\kern0pt}morphism\ X\ f\ f\ X\isanewline
\ \ \ \ \ \ \ \ {\isacharequal}{\kern0pt}\ {\isacharparenleft}{\kern0pt}h\ {\isasymcirc}\isactrlsub c\ right{\isacharunderscore}{\kern0pt}cart{\isacharunderscore}{\kern0pt}proj\ E\ E{\isacharparenright}{\kern0pt}\ {\isasymcirc}\isactrlsub c\ {\isacharparenleft}{\kern0pt}g\ {\isasymtimes}\isactrlsub f\ g{\isacharparenright}{\kern0pt}\ {\isasymcirc}\isactrlsub c\ fibered{\isacharunderscore}{\kern0pt}product{\isacharunderscore}{\kern0pt}morphism\ X\ f\ f\ X{\isachardoublequoteclose}\isanewline
\ \ \isacommand{proof}\isamarkupfalse%
\ {\isacharminus}{\kern0pt}\isanewline
\ \ \ \ \isacommand{have}\isamarkupfalse%
\ {\isachardoublequoteopen}{\isacharparenleft}{\kern0pt}h\ {\isasymcirc}\isactrlsub c\ left{\isacharunderscore}{\kern0pt}cart{\isacharunderscore}{\kern0pt}proj\ E\ E{\isacharparenright}{\kern0pt}\ {\isasymcirc}\isactrlsub c\ {\isacharparenleft}{\kern0pt}g\ {\isasymtimes}\isactrlsub f\ g{\isacharparenright}{\kern0pt}\ {\isasymcirc}\isactrlsub c\ fibered{\isacharunderscore}{\kern0pt}product{\isacharunderscore}{\kern0pt}morphism\ X\ f\ f\ X\isanewline
\ \ \ \ \ \ \ \ {\isacharequal}{\kern0pt}\ h\ {\isasymcirc}\isactrlsub c\ {\isacharparenleft}{\kern0pt}left{\isacharunderscore}{\kern0pt}cart{\isacharunderscore}{\kern0pt}proj\ E\ E\ {\isasymcirc}\isactrlsub c\ {\isacharparenleft}{\kern0pt}g\ {\isasymtimes}\isactrlsub f\ g{\isacharparenright}{\kern0pt}{\isacharparenright}{\kern0pt}\ {\isasymcirc}\isactrlsub c\ fibered{\isacharunderscore}{\kern0pt}product{\isacharunderscore}{\kern0pt}morphism\ X\ f\ f\ X{\isachardoublequoteclose}\isanewline
\ \ \ \ \ \ \isacommand{by}\isamarkupfalse%
\ {\isacharparenleft}{\kern0pt}typecheck{\isacharunderscore}{\kern0pt}cfuncs{\isacharcomma}{\kern0pt}\ simp\ add{\isacharcolon}{\kern0pt}\ comp{\isacharunderscore}{\kern0pt}associative{\isadigit{2}}{\isacharparenright}{\kern0pt}\isanewline
\ \ \ \ \isacommand{also}\isamarkupfalse%
\ \isacommand{have}\isamarkupfalse%
\ {\isachardoublequoteopen}{\isachardot}{\kern0pt}{\isachardot}{\kern0pt}{\isachardot}{\kern0pt}\ {\isacharequal}{\kern0pt}\ h\ {\isasymcirc}\isactrlsub c\ {\isacharparenleft}{\kern0pt}g\ {\isasymcirc}\isactrlsub c\ left{\isacharunderscore}{\kern0pt}cart{\isacharunderscore}{\kern0pt}proj\ X\ X{\isacharparenright}{\kern0pt}\ {\isasymcirc}\isactrlsub c\ fibered{\isacharunderscore}{\kern0pt}product{\isacharunderscore}{\kern0pt}morphism\ X\ f\ f\ X{\isachardoublequoteclose}\isanewline
\ \ \ \ \ \ \isacommand{by}\isamarkupfalse%
\ {\isacharparenleft}{\kern0pt}typecheck{\isacharunderscore}{\kern0pt}cfuncs{\isacharcomma}{\kern0pt}\ simp\ add{\isacharcolon}{\kern0pt}\ comp{\isacharunderscore}{\kern0pt}associative{\isadigit{2}}\ left{\isacharunderscore}{\kern0pt}cart{\isacharunderscore}{\kern0pt}proj{\isacharunderscore}{\kern0pt}cfunc{\isacharunderscore}{\kern0pt}cross{\isacharunderscore}{\kern0pt}prod{\isacharparenright}{\kern0pt}\isanewline
\ \ \ \ \isacommand{also}\isamarkupfalse%
\ \isacommand{have}\isamarkupfalse%
\ {\isachardoublequoteopen}{\isachardot}{\kern0pt}{\isachardot}{\kern0pt}{\isachardot}{\kern0pt}\ {\isacharequal}{\kern0pt}\ {\isacharparenleft}{\kern0pt}h\ {\isasymcirc}\isactrlsub c\ g{\isacharparenright}{\kern0pt}\ {\isasymcirc}\isactrlsub c\ left{\isacharunderscore}{\kern0pt}cart{\isacharunderscore}{\kern0pt}proj\ X\ X\ {\isasymcirc}\isactrlsub c\ fibered{\isacharunderscore}{\kern0pt}product{\isacharunderscore}{\kern0pt}morphism\ X\ f\ f\ X{\isachardoublequoteclose}\isanewline
\ \ \ \ \ \ \isacommand{by}\isamarkupfalse%
\ {\isacharparenleft}{\kern0pt}typecheck{\isacharunderscore}{\kern0pt}cfuncs{\isacharcomma}{\kern0pt}\ smt\ comp{\isacharunderscore}{\kern0pt}associative{\isadigit{2}}{\isacharparenright}{\kern0pt}\isanewline
\ \ \ \ \isacommand{also}\isamarkupfalse%
\ \isacommand{have}\isamarkupfalse%
\ {\isachardoublequoteopen}{\isachardot}{\kern0pt}{\isachardot}{\kern0pt}{\isachardot}{\kern0pt}\ {\isacharequal}{\kern0pt}\ f\ {\isasymcirc}\isactrlsub c\ left{\isacharunderscore}{\kern0pt}cart{\isacharunderscore}{\kern0pt}proj\ X\ X\ {\isasymcirc}\isactrlsub c\ fibered{\isacharunderscore}{\kern0pt}product{\isacharunderscore}{\kern0pt}morphism\ X\ f\ f\ X{\isachardoublequoteclose}\isanewline
\ \ \ \ \ \ \isacommand{by}\isamarkupfalse%
\ {\isacharparenleft}{\kern0pt}simp\ add{\isacharcolon}{\kern0pt}\ h{\isacharunderscore}{\kern0pt}g{\isacharunderscore}{\kern0pt}eq{\isacharunderscore}{\kern0pt}f{\isacharparenright}{\kern0pt}\isanewline
\ \ \ \ \isacommand{also}\isamarkupfalse%
\ \isacommand{have}\isamarkupfalse%
\ {\isachardoublequoteopen}{\isachardot}{\kern0pt}{\isachardot}{\kern0pt}{\isachardot}{\kern0pt}\ {\isacharequal}{\kern0pt}\ f\ {\isasymcirc}\isactrlsub c\ right{\isacharunderscore}{\kern0pt}cart{\isacharunderscore}{\kern0pt}proj\ X\ X\ {\isasymcirc}\isactrlsub c\ fibered{\isacharunderscore}{\kern0pt}product{\isacharunderscore}{\kern0pt}morphism\ X\ f\ f\ X{\isachardoublequoteclose}\isanewline
\ \ \ \ \ \ \isacommand{using}\isamarkupfalse%
\ f{\isacharunderscore}{\kern0pt}type\ fibered{\isacharunderscore}{\kern0pt}product{\isacharunderscore}{\kern0pt}left{\isacharunderscore}{\kern0pt}proj{\isacharunderscore}{\kern0pt}def\ fibered{\isacharunderscore}{\kern0pt}product{\isacharunderscore}{\kern0pt}proj{\isacharunderscore}{\kern0pt}eq\ fibered{\isacharunderscore}{\kern0pt}product{\isacharunderscore}{\kern0pt}right{\isacharunderscore}{\kern0pt}proj{\isacharunderscore}{\kern0pt}def\ \isacommand{by}\isamarkupfalse%
\ auto\isanewline
\ \ \ \ \isacommand{also}\isamarkupfalse%
\ \isacommand{have}\isamarkupfalse%
\ {\isachardoublequoteopen}{\isachardot}{\kern0pt}{\isachardot}{\kern0pt}{\isachardot}{\kern0pt}\ {\isacharequal}{\kern0pt}\ {\isacharparenleft}{\kern0pt}h\ {\isasymcirc}\isactrlsub c\ g{\isacharparenright}{\kern0pt}\ {\isasymcirc}\isactrlsub c\ right{\isacharunderscore}{\kern0pt}cart{\isacharunderscore}{\kern0pt}proj\ X\ X\ {\isasymcirc}\isactrlsub c\ fibered{\isacharunderscore}{\kern0pt}product{\isacharunderscore}{\kern0pt}morphism\ X\ f\ f\ X{\isachardoublequoteclose}\isanewline
\ \ \ \ \ \ \isacommand{by}\isamarkupfalse%
\ {\isacharparenleft}{\kern0pt}simp\ add{\isacharcolon}{\kern0pt}\ h{\isacharunderscore}{\kern0pt}g{\isacharunderscore}{\kern0pt}eq{\isacharunderscore}{\kern0pt}f{\isacharparenright}{\kern0pt}\isanewline
\ \ \ \ \isacommand{also}\isamarkupfalse%
\ \isacommand{have}\isamarkupfalse%
\ {\isachardoublequoteopen}{\isachardot}{\kern0pt}{\isachardot}{\kern0pt}{\isachardot}{\kern0pt}\ {\isacharequal}{\kern0pt}\ h\ {\isasymcirc}\isactrlsub c\ {\isacharparenleft}{\kern0pt}g\ {\isasymcirc}\isactrlsub c\ right{\isacharunderscore}{\kern0pt}cart{\isacharunderscore}{\kern0pt}proj\ X\ X{\isacharparenright}{\kern0pt}\ {\isasymcirc}\isactrlsub c\ fibered{\isacharunderscore}{\kern0pt}product{\isacharunderscore}{\kern0pt}morphism\ X\ f\ f\ X{\isachardoublequoteclose}\isanewline
\ \ \ \ \ \ \isacommand{by}\isamarkupfalse%
\ {\isacharparenleft}{\kern0pt}typecheck{\isacharunderscore}{\kern0pt}cfuncs{\isacharcomma}{\kern0pt}\ smt\ comp{\isacharunderscore}{\kern0pt}associative{\isadigit{2}}{\isacharparenright}{\kern0pt}\isanewline
\ \ \ \ \isacommand{also}\isamarkupfalse%
\ \isacommand{have}\isamarkupfalse%
\ {\isachardoublequoteopen}{\isachardot}{\kern0pt}{\isachardot}{\kern0pt}{\isachardot}{\kern0pt}\ {\isacharequal}{\kern0pt}\ h\ {\isasymcirc}\isactrlsub c\ right{\isacharunderscore}{\kern0pt}cart{\isacharunderscore}{\kern0pt}proj\ E\ E\ {\isasymcirc}\isactrlsub c\ {\isacharparenleft}{\kern0pt}g\ {\isasymtimes}\isactrlsub f\ g{\isacharparenright}{\kern0pt}\ {\isasymcirc}\isactrlsub c\ fibered{\isacharunderscore}{\kern0pt}product{\isacharunderscore}{\kern0pt}morphism\ X\ f\ f\ X{\isachardoublequoteclose}\isanewline
\ \ \ \ \ \ \isacommand{by}\isamarkupfalse%
\ {\isacharparenleft}{\kern0pt}typecheck{\isacharunderscore}{\kern0pt}cfuncs{\isacharcomma}{\kern0pt}\ simp\ add{\isacharcolon}{\kern0pt}\ comp{\isacharunderscore}{\kern0pt}associative{\isadigit{2}}\ right{\isacharunderscore}{\kern0pt}cart{\isacharunderscore}{\kern0pt}proj{\isacharunderscore}{\kern0pt}cfunc{\isacharunderscore}{\kern0pt}cross{\isacharunderscore}{\kern0pt}prod{\isacharparenright}{\kern0pt}\isanewline
\ \ \ \ \isacommand{also}\isamarkupfalse%
\ \isacommand{have}\isamarkupfalse%
\ {\isachardoublequoteopen}{\isachardot}{\kern0pt}{\isachardot}{\kern0pt}{\isachardot}{\kern0pt}\ {\isacharequal}{\kern0pt}\ {\isacharparenleft}{\kern0pt}h\ {\isasymcirc}\isactrlsub c\ right{\isacharunderscore}{\kern0pt}cart{\isacharunderscore}{\kern0pt}proj\ E\ E{\isacharparenright}{\kern0pt}\ {\isasymcirc}\isactrlsub c\ {\isacharparenleft}{\kern0pt}g\ {\isasymtimes}\isactrlsub f\ g{\isacharparenright}{\kern0pt}\ {\isasymcirc}\isactrlsub c\ fibered{\isacharunderscore}{\kern0pt}product{\isacharunderscore}{\kern0pt}morphism\ X\ f\ f\ X{\isachardoublequoteclose}\isanewline
\ \ \ \ \ \ \isacommand{by}\isamarkupfalse%
\ {\isacharparenleft}{\kern0pt}typecheck{\isacharunderscore}{\kern0pt}cfuncs{\isacharcomma}{\kern0pt}\ smt\ comp{\isacharunderscore}{\kern0pt}associative{\isadigit{2}}{\isacharparenright}{\kern0pt}\isanewline
\ \ \ \ \isacommand{then}\isamarkupfalse%
\ \isacommand{show}\isamarkupfalse%
\ {\isacharquery}{\kern0pt}thesis\isanewline
\ \ \ \ \ \ \isacommand{using}\isamarkupfalse%
\ calculation\ \isacommand{by}\isamarkupfalse%
\ auto\isanewline
\ \ \isacommand{qed}\isamarkupfalse%
\isanewline
\ \ \isacommand{have}\isamarkupfalse%
\ h{\isacharunderscore}{\kern0pt}equalizer{\isacharcolon}{\kern0pt}\ {\isachardoublequoteopen}equalizer\ {\isacharparenleft}{\kern0pt}E\ \isactrlbsub h\isactrlesub {\isasymtimes}\isactrlsub c\isactrlbsub h\isactrlesub \ E{\isacharparenright}{\kern0pt}\ {\isacharparenleft}{\kern0pt}fibered{\isacharunderscore}{\kern0pt}product{\isacharunderscore}{\kern0pt}morphism\ E\ h\ h\ E{\isacharparenright}{\kern0pt}\ {\isacharparenleft}{\kern0pt}h\ {\isasymcirc}\isactrlsub c\ left{\isacharunderscore}{\kern0pt}cart{\isacharunderscore}{\kern0pt}proj\ E\ E{\isacharparenright}{\kern0pt}\ {\isacharparenleft}{\kern0pt}h\ {\isasymcirc}\isactrlsub c\ right{\isacharunderscore}{\kern0pt}cart{\isacharunderscore}{\kern0pt}proj\ E\ E{\isacharparenright}{\kern0pt}{\isachardoublequoteclose}\isanewline
\ \ \ \ \isacommand{using}\isamarkupfalse%
\ fibered{\isacharunderscore}{\kern0pt}product{\isacharunderscore}{\kern0pt}morphism{\isacharunderscore}{\kern0pt}equalizer\ h{\isacharunderscore}{\kern0pt}type\ \isacommand{by}\isamarkupfalse%
\ auto\isanewline
\ \ \isacommand{then}\isamarkupfalse%
\ \isacommand{have}\isamarkupfalse%
\ {\isachardoublequoteopen}{\isasymforall}j\ F{\isachardot}{\kern0pt}\ j\ {\isacharcolon}{\kern0pt}\ F\ {\isasymrightarrow}\ E\ {\isasymtimes}\isactrlsub c\ E\ {\isasymand}\ {\isacharparenleft}{\kern0pt}h\ {\isasymcirc}\isactrlsub c\ left{\isacharunderscore}{\kern0pt}cart{\isacharunderscore}{\kern0pt}proj\ E\ E{\isacharparenright}{\kern0pt}\ {\isasymcirc}\isactrlsub c\ j\ {\isacharequal}{\kern0pt}\ {\isacharparenleft}{\kern0pt}h\ {\isasymcirc}\isactrlsub c\ right{\isacharunderscore}{\kern0pt}cart{\isacharunderscore}{\kern0pt}proj\ E\ E{\isacharparenright}{\kern0pt}\ {\isasymcirc}\isactrlsub c\ j\ {\isasymlongrightarrow}\isanewline
\ \ \ \ \ \ \ \ \ \ \ \ \ \ \ {\isacharparenleft}{\kern0pt}{\isasymexists}{\isacharbang}{\kern0pt}k{\isachardot}{\kern0pt}\ k\ {\isacharcolon}{\kern0pt}\ F\ {\isasymrightarrow}\ E\ \isactrlbsub h\isactrlesub {\isasymtimes}\isactrlsub c\isactrlbsub h\isactrlesub \ E\ {\isasymand}\ fibered{\isacharunderscore}{\kern0pt}product{\isacharunderscore}{\kern0pt}morphism\ E\ h\ h\ E\ {\isasymcirc}\isactrlsub c\ k\ {\isacharequal}{\kern0pt}\ j{\isacharparenright}{\kern0pt}{\isachardoublequoteclose}\isanewline
\ \ \ \ \isacommand{unfolding}\isamarkupfalse%
\ equalizer{\isacharunderscore}{\kern0pt}def\ \isacommand{using}\isamarkupfalse%
\ cfunc{\isacharunderscore}{\kern0pt}type{\isacharunderscore}{\kern0pt}def\ fibered{\isacharunderscore}{\kern0pt}product{\isacharunderscore}{\kern0pt}morphism{\isacharunderscore}{\kern0pt}type\ h{\isacharunderscore}{\kern0pt}type\ \isacommand{by}\isamarkupfalse%
\ {\isacharparenleft}{\kern0pt}smt\ {\isacharparenleft}{\kern0pt}verit{\isacharparenright}{\kern0pt}{\isacharparenright}{\kern0pt}\isanewline
\ \ \isacommand{then}\isamarkupfalse%
\ \isacommand{have}\isamarkupfalse%
\ {\isachardoublequoteopen}{\isacharparenleft}{\kern0pt}g\ {\isasymtimes}\isactrlsub f\ g{\isacharparenright}{\kern0pt}\ {\isasymcirc}\isactrlsub c\ fibered{\isacharunderscore}{\kern0pt}product{\isacharunderscore}{\kern0pt}morphism\ X\ f\ f\ X\ \ {\isacharcolon}{\kern0pt}\ X\ \isactrlbsub f\isactrlesub {\isasymtimes}\isactrlsub c\isactrlbsub f\isactrlesub \ X\ {\isasymrightarrow}\ E\ {\isasymtimes}\isactrlsub c\ E\ {\isasymand}\ {\isacharparenleft}{\kern0pt}h\ {\isasymcirc}\isactrlsub c\ left{\isacharunderscore}{\kern0pt}cart{\isacharunderscore}{\kern0pt}proj\ E\ E{\isacharparenright}{\kern0pt}\ {\isasymcirc}\isactrlsub c\ {\isacharparenleft}{\kern0pt}g\ {\isasymtimes}\isactrlsub f\ g{\isacharparenright}{\kern0pt}\ {\isasymcirc}\isactrlsub c\ fibered{\isacharunderscore}{\kern0pt}product{\isacharunderscore}{\kern0pt}morphism\ X\ f\ f\ X\ {\isacharequal}{\kern0pt}\ {\isacharparenleft}{\kern0pt}h\ {\isasymcirc}\isactrlsub c\ right{\isacharunderscore}{\kern0pt}cart{\isacharunderscore}{\kern0pt}proj\ E\ E{\isacharparenright}{\kern0pt}\ {\isasymcirc}\isactrlsub c\ {\isacharparenleft}{\kern0pt}g\ {\isasymtimes}\isactrlsub f\ g{\isacharparenright}{\kern0pt}\ {\isasymcirc}\isactrlsub c\ fibered{\isacharunderscore}{\kern0pt}product{\isacharunderscore}{\kern0pt}morphism\ X\ f\ f\ X\ {\isasymlongrightarrow}\isanewline
\ \ \ \ \ \ \ \ \ \ \ \ \ \ \ {\isacharparenleft}{\kern0pt}{\isasymexists}{\isacharbang}{\kern0pt}k{\isachardot}{\kern0pt}\ k\ {\isacharcolon}{\kern0pt}\ X\ \isactrlbsub f\isactrlesub {\isasymtimes}\isactrlsub c\isactrlbsub f\isactrlesub \ X\ {\isasymrightarrow}\ E\ \isactrlbsub h\isactrlesub {\isasymtimes}\isactrlsub c\isactrlbsub h\isactrlesub \ E\ {\isasymand}\ fibered{\isacharunderscore}{\kern0pt}product{\isacharunderscore}{\kern0pt}morphism\ E\ h\ h\ E\ {\isasymcirc}\isactrlsub c\ k\ {\isacharequal}{\kern0pt}\ {\isacharparenleft}{\kern0pt}g\ {\isasymtimes}\isactrlsub f\ g{\isacharparenright}{\kern0pt}\ {\isasymcirc}\isactrlsub c\ fibered{\isacharunderscore}{\kern0pt}product{\isacharunderscore}{\kern0pt}morphism\ X\ f\ f\ X{\isacharparenright}{\kern0pt}{\isachardoublequoteclose}\isanewline
\ \ \ \ \isacommand{by}\isamarkupfalse%
\ auto\isanewline
\ \ \isacommand{then}\isamarkupfalse%
\ \isacommand{obtain}\isamarkupfalse%
\ b\ \isakeyword{where}\ b{\isacharunderscore}{\kern0pt}type{\isacharbrackleft}{\kern0pt}type{\isacharunderscore}{\kern0pt}rule{\isacharbrackright}{\kern0pt}{\isacharcolon}{\kern0pt}\ {\isachardoublequoteopen}b\ {\isacharcolon}{\kern0pt}\ X\ \isactrlbsub f\isactrlesub {\isasymtimes}\isactrlsub c\isactrlbsub f\isactrlesub \ X\ {\isasymrightarrow}\ E\ \isactrlbsub h\isactrlesub {\isasymtimes}\isactrlsub c\isactrlbsub h\isactrlesub \ E{\isachardoublequoteclose}\isanewline
\ \ \ \ \ \ \ \ \ \ \isakeyword{and}\ b{\isacharunderscore}{\kern0pt}eq{\isacharcolon}{\kern0pt}\ {\isachardoublequoteopen}fibered{\isacharunderscore}{\kern0pt}product{\isacharunderscore}{\kern0pt}morphism\ E\ h\ h\ E\ {\isasymcirc}\isactrlsub c\ b\ {\isacharequal}{\kern0pt}\ {\isacharparenleft}{\kern0pt}g\ {\isasymtimes}\isactrlsub f\ g{\isacharparenright}{\kern0pt}\ {\isasymcirc}\isactrlsub c\ fibered{\isacharunderscore}{\kern0pt}product{\isacharunderscore}{\kern0pt}morphism\ X\ f\ f\ X{\isachardoublequoteclose}\isanewline
\ \ \ \ \isacommand{by}\isamarkupfalse%
\ {\isacharparenleft}{\kern0pt}meson\ cfunc{\isacharunderscore}{\kern0pt}cross{\isacharunderscore}{\kern0pt}prod{\isacharunderscore}{\kern0pt}type\ comp{\isacharunderscore}{\kern0pt}type\ f{\isacharunderscore}{\kern0pt}type\ fibered{\isacharunderscore}{\kern0pt}product{\isacharunderscore}{\kern0pt}morphism{\isacharunderscore}{\kern0pt}type\ g{\isacharunderscore}{\kern0pt}type\ gxg{\isacharunderscore}{\kern0pt}fpmorph{\isacharunderscore}{\kern0pt}eq{\isacharparenright}{\kern0pt}\isanewline
\isanewline
\ \ \isacommand{have}\isamarkupfalse%
\ {\isachardoublequoteopen}is{\isacharunderscore}{\kern0pt}pullback\ {\isacharparenleft}{\kern0pt}X\ \isactrlbsub f\isactrlesub {\isasymtimes}\isactrlsub c\isactrlbsub f\isactrlesub \ X{\isacharparenright}{\kern0pt}\ {\isacharparenleft}{\kern0pt}X\ {\isasymtimes}\isactrlsub c\ X{\isacharparenright}{\kern0pt}\ {\isacharparenleft}{\kern0pt}E\ \isactrlbsub h\isactrlesub {\isasymtimes}\isactrlsub c\isactrlbsub h\isactrlesub \ E{\isacharparenright}{\kern0pt}\ {\isacharparenleft}{\kern0pt}E\ {\isasymtimes}\isactrlsub c\ E{\isacharparenright}{\kern0pt}\isanewline
\ \ \ \ \ \ {\isacharparenleft}{\kern0pt}fibered{\isacharunderscore}{\kern0pt}product{\isacharunderscore}{\kern0pt}morphism\ X\ f\ f\ X{\isacharparenright}{\kern0pt}\ {\isacharparenleft}{\kern0pt}g\ {\isasymtimes}\isactrlsub f\ g{\isacharparenright}{\kern0pt}\ b\ {\isacharparenleft}{\kern0pt}fibered{\isacharunderscore}{\kern0pt}product{\isacharunderscore}{\kern0pt}morphism\ E\ h\ h\ E{\isacharparenright}{\kern0pt}{\isachardoublequoteclose}\isanewline
\ \ \isacommand{proof}\isamarkupfalse%
\ {\isacharparenleft}{\kern0pt}unfold\ is{\isacharunderscore}{\kern0pt}pullback{\isacharunderscore}{\kern0pt}def{\isacharcomma}{\kern0pt}\ typecheck{\isacharunderscore}{\kern0pt}cfuncs{\isacharcomma}{\kern0pt}\ safe{\isacharcomma}{\kern0pt}\ metis\ b{\isacharunderscore}{\kern0pt}eq{\isacharparenright}{\kern0pt}\isanewline
\ \ \ \ \isacommand{fix}\isamarkupfalse%
\ Z\ k\ j\isanewline
\ \ \ \ \isacommand{assume}\isamarkupfalse%
\ k{\isacharunderscore}{\kern0pt}type{\isacharbrackleft}{\kern0pt}type{\isacharunderscore}{\kern0pt}rule{\isacharbrackright}{\kern0pt}{\isacharcolon}{\kern0pt}\ {\isachardoublequoteopen}k\ {\isacharcolon}{\kern0pt}\ Z\ {\isasymrightarrow}\ X\ {\isasymtimes}\isactrlsub c\ X{\isachardoublequoteclose}\ \isakeyword{and}\ h{\isacharunderscore}{\kern0pt}type{\isacharbrackleft}{\kern0pt}type{\isacharunderscore}{\kern0pt}rule{\isacharbrackright}{\kern0pt}{\isacharcolon}{\kern0pt}\ {\isachardoublequoteopen}j\ {\isacharcolon}{\kern0pt}\ Z\ {\isasymrightarrow}\ E\ \isactrlbsub h\isactrlesub {\isasymtimes}\isactrlsub c\isactrlbsub h\isactrlesub \ E{\isachardoublequoteclose}\isanewline
\ \ \ \ \isacommand{assume}\isamarkupfalse%
\ k{\isacharunderscore}{\kern0pt}h{\isacharunderscore}{\kern0pt}eq{\isacharcolon}{\kern0pt}\ {\isachardoublequoteopen}{\isacharparenleft}{\kern0pt}g\ {\isasymtimes}\isactrlsub f\ g{\isacharparenright}{\kern0pt}\ {\isasymcirc}\isactrlsub c\ k\ {\isacharequal}{\kern0pt}\ fibered{\isacharunderscore}{\kern0pt}product{\isacharunderscore}{\kern0pt}morphism\ E\ h\ h\ E\ {\isasymcirc}\isactrlsub c\ j{\isachardoublequoteclose}\isanewline
\isanewline
\ \ \ \ \isacommand{have}\isamarkupfalse%
\ left{\isacharunderscore}{\kern0pt}k{\isacharunderscore}{\kern0pt}right{\isacharunderscore}{\kern0pt}k{\isacharunderscore}{\kern0pt}eq{\isacharcolon}{\kern0pt}\ {\isachardoublequoteopen}f\ {\isasymcirc}\isactrlsub c\ left{\isacharunderscore}{\kern0pt}cart{\isacharunderscore}{\kern0pt}proj\ X\ X\ {\isasymcirc}\isactrlsub c\ k\ {\isacharequal}{\kern0pt}\ f\ {\isasymcirc}\isactrlsub c\ right{\isacharunderscore}{\kern0pt}cart{\isacharunderscore}{\kern0pt}proj\ X\ X\ {\isasymcirc}\isactrlsub c\ k{\isachardoublequoteclose}\isanewline
\ \ \ \ \isacommand{proof}\isamarkupfalse%
\ {\isacharminus}{\kern0pt}\isanewline
\ \ \ \ \ \ \isacommand{have}\isamarkupfalse%
\ {\isachardoublequoteopen}f\ {\isasymcirc}\isactrlsub c\ left{\isacharunderscore}{\kern0pt}cart{\isacharunderscore}{\kern0pt}proj\ X\ X\ {\isasymcirc}\isactrlsub c\ k\ {\isacharequal}{\kern0pt}\ h\ {\isasymcirc}\isactrlsub c\ g\ {\isasymcirc}\isactrlsub c\ left{\isacharunderscore}{\kern0pt}cart{\isacharunderscore}{\kern0pt}proj\ X\ X\ {\isasymcirc}\isactrlsub c\ k{\isachardoublequoteclose}\isanewline
\ \ \ \ \ \ \ \ \isacommand{by}\isamarkupfalse%
\ {\isacharparenleft}{\kern0pt}smt\ {\isacharparenleft}{\kern0pt}z{\isadigit{3}}{\isacharparenright}{\kern0pt}\ assms{\isacharparenleft}{\kern0pt}{\isadigit{6}}{\isacharparenright}{\kern0pt}\ comp{\isacharunderscore}{\kern0pt}associative{\isadigit{2}}\ comp{\isacharunderscore}{\kern0pt}type\ g{\isacharunderscore}{\kern0pt}type\ h{\isacharunderscore}{\kern0pt}g{\isacharunderscore}{\kern0pt}eq{\isacharunderscore}{\kern0pt}f\ k{\isacharunderscore}{\kern0pt}type\ left{\isacharunderscore}{\kern0pt}cart{\isacharunderscore}{\kern0pt}proj{\isacharunderscore}{\kern0pt}type{\isacharparenright}{\kern0pt}\isanewline
\ \ \ \ \ \ \isacommand{also}\isamarkupfalse%
\ \isacommand{have}\isamarkupfalse%
\ {\isachardoublequoteopen}{\isachardot}{\kern0pt}{\isachardot}{\kern0pt}{\isachardot}{\kern0pt}\ {\isacharequal}{\kern0pt}\ h\ {\isasymcirc}\isactrlsub c\ left{\isacharunderscore}{\kern0pt}cart{\isacharunderscore}{\kern0pt}proj\ E\ E\ {\isasymcirc}\isactrlsub c\ {\isacharparenleft}{\kern0pt}g\ {\isasymtimes}\isactrlsub f\ g{\isacharparenright}{\kern0pt}\ {\isasymcirc}\isactrlsub c\ k{\isachardoublequoteclose}\isanewline
\ \ \ \ \ \ \ \ \isacommand{by}\isamarkupfalse%
\ {\isacharparenleft}{\kern0pt}typecheck{\isacharunderscore}{\kern0pt}cfuncs{\isacharcomma}{\kern0pt}\ simp\ add{\isacharcolon}{\kern0pt}\ comp{\isacharunderscore}{\kern0pt}associative{\isadigit{2}}\ left{\isacharunderscore}{\kern0pt}cart{\isacharunderscore}{\kern0pt}proj{\isacharunderscore}{\kern0pt}cfunc{\isacharunderscore}{\kern0pt}cross{\isacharunderscore}{\kern0pt}prod{\isacharparenright}{\kern0pt}\isanewline
\ \ \ \ \ \ \isacommand{also}\isamarkupfalse%
\ \isacommand{have}\isamarkupfalse%
\ {\isachardoublequoteopen}{\isachardot}{\kern0pt}{\isachardot}{\kern0pt}{\isachardot}{\kern0pt}\ {\isacharequal}{\kern0pt}\ h\ {\isasymcirc}\isactrlsub c\ left{\isacharunderscore}{\kern0pt}cart{\isacharunderscore}{\kern0pt}proj\ E\ E\ {\isasymcirc}\isactrlsub c\ fibered{\isacharunderscore}{\kern0pt}product{\isacharunderscore}{\kern0pt}morphism\ E\ h\ h\ E\ {\isasymcirc}\isactrlsub c\ j{\isachardoublequoteclose}\isanewline
\ \ \ \ \ \ \ \ \isacommand{by}\isamarkupfalse%
\ {\isacharparenleft}{\kern0pt}simp\ add{\isacharcolon}{\kern0pt}\ k{\isacharunderscore}{\kern0pt}h{\isacharunderscore}{\kern0pt}eq{\isacharparenright}{\kern0pt}\isanewline
\ \ \ \ \ \ \isacommand{also}\isamarkupfalse%
\ \isacommand{have}\isamarkupfalse%
\ {\isachardoublequoteopen}{\isachardot}{\kern0pt}{\isachardot}{\kern0pt}{\isachardot}{\kern0pt}\ {\isacharequal}{\kern0pt}\ {\isacharparenleft}{\kern0pt}{\isacharparenleft}{\kern0pt}h\ {\isasymcirc}\isactrlsub c\ left{\isacharunderscore}{\kern0pt}cart{\isacharunderscore}{\kern0pt}proj\ E\ E{\isacharparenright}{\kern0pt}\ {\isasymcirc}\isactrlsub c\ fibered{\isacharunderscore}{\kern0pt}product{\isacharunderscore}{\kern0pt}morphism\ E\ h\ h\ E{\isacharparenright}{\kern0pt}\ {\isasymcirc}\isactrlsub c\ j{\isachardoublequoteclose}\isanewline
\ \ \ \ \ \ \ \ \isacommand{by}\isamarkupfalse%
\ {\isacharparenleft}{\kern0pt}typecheck{\isacharunderscore}{\kern0pt}cfuncs{\isacharcomma}{\kern0pt}\ smt\ comp{\isacharunderscore}{\kern0pt}associative{\isadigit{2}}{\isacharparenright}{\kern0pt}\isanewline
\ \ \ \ \ \ \isacommand{also}\isamarkupfalse%
\ \isacommand{have}\isamarkupfalse%
\ {\isachardoublequoteopen}{\isachardot}{\kern0pt}{\isachardot}{\kern0pt}{\isachardot}{\kern0pt}\ {\isacharequal}{\kern0pt}\ {\isacharparenleft}{\kern0pt}{\isacharparenleft}{\kern0pt}h\ {\isasymcirc}\isactrlsub c\ right{\isacharunderscore}{\kern0pt}cart{\isacharunderscore}{\kern0pt}proj\ E\ E{\isacharparenright}{\kern0pt}\ {\isasymcirc}\isactrlsub c\ fibered{\isacharunderscore}{\kern0pt}product{\isacharunderscore}{\kern0pt}morphism\ E\ h\ h\ E{\isacharparenright}{\kern0pt}\ {\isasymcirc}\isactrlsub c\ j{\isachardoublequoteclose}\isanewline
\ \ \ \ \ \ \ \ \isacommand{using}\isamarkupfalse%
\ equalizer{\isacharunderscore}{\kern0pt}def\ h{\isacharunderscore}{\kern0pt}equalizer\ \isacommand{by}\isamarkupfalse%
\ auto\isanewline
\ \ \ \ \ \ \isacommand{also}\isamarkupfalse%
\ \isacommand{have}\isamarkupfalse%
\ {\isachardoublequoteopen}{\isachardot}{\kern0pt}{\isachardot}{\kern0pt}{\isachardot}{\kern0pt}\ {\isacharequal}{\kern0pt}\ h\ {\isasymcirc}\isactrlsub c\ right{\isacharunderscore}{\kern0pt}cart{\isacharunderscore}{\kern0pt}proj\ E\ E\ {\isasymcirc}\isactrlsub c\ fibered{\isacharunderscore}{\kern0pt}product{\isacharunderscore}{\kern0pt}morphism\ E\ h\ h\ E\ {\isasymcirc}\isactrlsub c\ j{\isachardoublequoteclose}\isanewline
\ \ \ \ \ \ \ \ \isacommand{by}\isamarkupfalse%
\ {\isacharparenleft}{\kern0pt}typecheck{\isacharunderscore}{\kern0pt}cfuncs{\isacharcomma}{\kern0pt}\ smt\ comp{\isacharunderscore}{\kern0pt}associative{\isadigit{2}}{\isacharparenright}{\kern0pt}\isanewline
\ \ \ \ \ \ \isacommand{also}\isamarkupfalse%
\ \isacommand{have}\isamarkupfalse%
\ {\isachardoublequoteopen}{\isachardot}{\kern0pt}{\isachardot}{\kern0pt}{\isachardot}{\kern0pt}\ {\isacharequal}{\kern0pt}\ h\ {\isasymcirc}\isactrlsub c\ right{\isacharunderscore}{\kern0pt}cart{\isacharunderscore}{\kern0pt}proj\ E\ E\ {\isasymcirc}\isactrlsub c\ {\isacharparenleft}{\kern0pt}g\ {\isasymtimes}\isactrlsub f\ g{\isacharparenright}{\kern0pt}\ {\isasymcirc}\isactrlsub c\ k{\isachardoublequoteclose}\isanewline
\ \ \ \ \ \ \ \ \isacommand{by}\isamarkupfalse%
\ {\isacharparenleft}{\kern0pt}simp\ add{\isacharcolon}{\kern0pt}\ k{\isacharunderscore}{\kern0pt}h{\isacharunderscore}{\kern0pt}eq{\isacharparenright}{\kern0pt}\isanewline
\ \ \ \ \ \ \isacommand{also}\isamarkupfalse%
\ \isacommand{have}\isamarkupfalse%
\ {\isachardoublequoteopen}{\isachardot}{\kern0pt}{\isachardot}{\kern0pt}{\isachardot}{\kern0pt}\ {\isacharequal}{\kern0pt}\ h\ {\isasymcirc}\isactrlsub c\ g\ {\isasymcirc}\isactrlsub c\ right{\isacharunderscore}{\kern0pt}cart{\isacharunderscore}{\kern0pt}proj\ X\ X\ {\isasymcirc}\isactrlsub c\ k{\isachardoublequoteclose}\isanewline
\ \ \ \ \ \ \ \ \isacommand{by}\isamarkupfalse%
\ {\isacharparenleft}{\kern0pt}typecheck{\isacharunderscore}{\kern0pt}cfuncs{\isacharcomma}{\kern0pt}\ simp\ add{\isacharcolon}{\kern0pt}\ comp{\isacharunderscore}{\kern0pt}associative{\isadigit{2}}\ right{\isacharunderscore}{\kern0pt}cart{\isacharunderscore}{\kern0pt}proj{\isacharunderscore}{\kern0pt}cfunc{\isacharunderscore}{\kern0pt}cross{\isacharunderscore}{\kern0pt}prod{\isacharparenright}{\kern0pt}\isanewline
\ \ \ \ \ \ \isacommand{also}\isamarkupfalse%
\ \isacommand{have}\isamarkupfalse%
\ {\isachardoublequoteopen}{\isachardot}{\kern0pt}{\isachardot}{\kern0pt}{\isachardot}{\kern0pt}\ {\isacharequal}{\kern0pt}\ f\ {\isasymcirc}\isactrlsub c\ right{\isacharunderscore}{\kern0pt}cart{\isacharunderscore}{\kern0pt}proj\ X\ X\ {\isasymcirc}\isactrlsub c\ k{\isachardoublequoteclose}\isanewline
\ \ \ \ \ \ \ \ \isacommand{using}\isamarkupfalse%
\ assms{\isacharparenleft}{\kern0pt}{\isadigit{6}}{\isacharparenright}{\kern0pt}\ comp{\isacharunderscore}{\kern0pt}associative{\isadigit{2}}\ comp{\isacharunderscore}{\kern0pt}type\ g{\isacharunderscore}{\kern0pt}type\ h{\isacharunderscore}{\kern0pt}g{\isacharunderscore}{\kern0pt}eq{\isacharunderscore}{\kern0pt}f\ k{\isacharunderscore}{\kern0pt}type\ right{\isacharunderscore}{\kern0pt}cart{\isacharunderscore}{\kern0pt}proj{\isacharunderscore}{\kern0pt}type\ \isacommand{by}\isamarkupfalse%
\ blast\isanewline
\ \ \ \ \ \ \isacommand{then}\isamarkupfalse%
\ \isacommand{show}\isamarkupfalse%
\ {\isacharquery}{\kern0pt}thesis\isanewline
\ \ \ \ \ \ \ \ \isacommand{using}\isamarkupfalse%
\ calculation\ \isacommand{by}\isamarkupfalse%
\ auto\isanewline
\ \ \ \ \isacommand{qed}\isamarkupfalse%
\isanewline
\isanewline
\ \ \ \ \isacommand{have}\isamarkupfalse%
\ {\isachardoublequoteopen}is{\isacharunderscore}{\kern0pt}pullback\ {\isacharparenleft}{\kern0pt}X\ \isactrlbsub f\isactrlesub {\isasymtimes}\isactrlsub c\isactrlbsub f\isactrlesub \ X{\isacharparenright}{\kern0pt}\ X\ X\ Y\isanewline
\ \ \ \ \ \ \ \ {\isacharparenleft}{\kern0pt}fibered{\isacharunderscore}{\kern0pt}product{\isacharunderscore}{\kern0pt}right{\isacharunderscore}{\kern0pt}proj\ X\ f\ f\ X{\isacharparenright}{\kern0pt}\ f\ {\isacharparenleft}{\kern0pt}fibered{\isacharunderscore}{\kern0pt}product{\isacharunderscore}{\kern0pt}left{\isacharunderscore}{\kern0pt}proj\ X\ f\ f\ X{\isacharparenright}{\kern0pt}\ f{\isachardoublequoteclose}\isanewline
\ \ \ \ \ \ \isacommand{by}\isamarkupfalse%
\ {\isacharparenleft}{\kern0pt}simp\ add{\isacharcolon}{\kern0pt}\ f{\isacharunderscore}{\kern0pt}type\ fibered{\isacharunderscore}{\kern0pt}product{\isacharunderscore}{\kern0pt}is{\isacharunderscore}{\kern0pt}pullback{\isacharparenright}{\kern0pt}\isanewline
\ \ \ \ \isacommand{then}\isamarkupfalse%
\ \isacommand{have}\isamarkupfalse%
\ {\isachardoublequoteopen}right{\isacharunderscore}{\kern0pt}cart{\isacharunderscore}{\kern0pt}proj\ X\ X\ {\isasymcirc}\isactrlsub c\ k\ {\isacharcolon}{\kern0pt}\ Z\ {\isasymrightarrow}\ X\ {\isasymLongrightarrow}\ left{\isacharunderscore}{\kern0pt}cart{\isacharunderscore}{\kern0pt}proj\ X\ X\ {\isasymcirc}\isactrlsub c\ k\ {\isacharcolon}{\kern0pt}\ Z\ {\isasymrightarrow}\ X\ {\isasymLongrightarrow}\ f\ {\isasymcirc}\isactrlsub c\ right{\isacharunderscore}{\kern0pt}cart{\isacharunderscore}{\kern0pt}proj\ X\ X\ {\isasymcirc}\isactrlsub c\ k\ {\isacharequal}{\kern0pt}\ f\ {\isasymcirc}\isactrlsub c\ left{\isacharunderscore}{\kern0pt}cart{\isacharunderscore}{\kern0pt}proj\ X\ X\ {\isasymcirc}\isactrlsub c\ k\ {\isasymLongrightarrow}\isanewline
\ \ \ \ \ \ {\isacharparenleft}{\kern0pt}{\isasymexists}{\isacharbang}{\kern0pt}j{\isachardot}{\kern0pt}\ j\ {\isacharcolon}{\kern0pt}\ Z\ {\isasymrightarrow}\ X\ \isactrlbsub f\isactrlesub {\isasymtimes}\isactrlsub c\isactrlbsub f\isactrlesub \ X\ {\isasymand}\isanewline
\ \ \ \ \ \ \ \ fibered{\isacharunderscore}{\kern0pt}product{\isacharunderscore}{\kern0pt}right{\isacharunderscore}{\kern0pt}proj\ X\ f\ f\ X\ {\isasymcirc}\isactrlsub c\ j\ {\isacharequal}{\kern0pt}\ right{\isacharunderscore}{\kern0pt}cart{\isacharunderscore}{\kern0pt}proj\ X\ X\ {\isasymcirc}\isactrlsub c\ k\isanewline
\ \ \ \ \ \ \ \ {\isasymand}\ fibered{\isacharunderscore}{\kern0pt}product{\isacharunderscore}{\kern0pt}left{\isacharunderscore}{\kern0pt}proj\ X\ f\ f\ X\ {\isasymcirc}\isactrlsub c\ j\ {\isacharequal}{\kern0pt}\ left{\isacharunderscore}{\kern0pt}cart{\isacharunderscore}{\kern0pt}proj\ X\ X\ {\isasymcirc}\isactrlsub c\ k{\isacharparenright}{\kern0pt}{\isachardoublequoteclose}\isanewline
\ \ \ \ \ \ \isacommand{unfolding}\isamarkupfalse%
\ is{\isacharunderscore}{\kern0pt}pullback{\isacharunderscore}{\kern0pt}def\ \isacommand{by}\isamarkupfalse%
\ auto\isanewline
\ \ \ \ \isacommand{then}\isamarkupfalse%
\ \isacommand{obtain}\isamarkupfalse%
\ z\ \isakeyword{where}\ z{\isacharunderscore}{\kern0pt}type{\isacharbrackleft}{\kern0pt}type{\isacharunderscore}{\kern0pt}rule{\isacharbrackright}{\kern0pt}{\isacharcolon}{\kern0pt}\ {\isachardoublequoteopen}z\ {\isacharcolon}{\kern0pt}\ Z\ {\isasymrightarrow}\ X\ \isactrlbsub f\isactrlesub {\isasymtimes}\isactrlsub c\isactrlbsub f\isactrlesub \ X{\isachardoublequoteclose}\isanewline
\ \ \ \ \ \ \ \ \isakeyword{and}\ k{\isacharunderscore}{\kern0pt}right{\isacharunderscore}{\kern0pt}eq{\isacharcolon}{\kern0pt}\ {\isachardoublequoteopen}fibered{\isacharunderscore}{\kern0pt}product{\isacharunderscore}{\kern0pt}right{\isacharunderscore}{\kern0pt}proj\ X\ f\ f\ X\ {\isasymcirc}\isactrlsub c\ z\ {\isacharequal}{\kern0pt}\ right{\isacharunderscore}{\kern0pt}cart{\isacharunderscore}{\kern0pt}proj\ X\ X\ {\isasymcirc}\isactrlsub c\ k{\isachardoublequoteclose}\ \isanewline
\ \ \ \ \ \ \ \ \isakeyword{and}\ k{\isacharunderscore}{\kern0pt}left{\isacharunderscore}{\kern0pt}eq{\isacharcolon}{\kern0pt}\ {\isachardoublequoteopen}fibered{\isacharunderscore}{\kern0pt}product{\isacharunderscore}{\kern0pt}left{\isacharunderscore}{\kern0pt}proj\ X\ f\ f\ X\ {\isasymcirc}\isactrlsub c\ z\ {\isacharequal}{\kern0pt}\ left{\isacharunderscore}{\kern0pt}cart{\isacharunderscore}{\kern0pt}proj\ X\ X\ {\isasymcirc}\isactrlsub c\ k{\isachardoublequoteclose}\isanewline
\ \ \ \ \ \ \ \ \isakeyword{and}\ z{\isacharunderscore}{\kern0pt}unique{\isacharcolon}{\kern0pt}\ {\isachardoublequoteopen}{\isasymAnd}j{\isachardot}{\kern0pt}\ j\ {\isacharcolon}{\kern0pt}\ Z\ {\isasymrightarrow}\ X\ \isactrlbsub f\isactrlesub {\isasymtimes}\isactrlsub c\isactrlbsub f\isactrlesub \ X\ \isanewline
\ \ \ \ \ \ \ \ \ \ {\isasymand}\ fibered{\isacharunderscore}{\kern0pt}product{\isacharunderscore}{\kern0pt}right{\isacharunderscore}{\kern0pt}proj\ X\ f\ f\ X\ {\isasymcirc}\isactrlsub c\ j\ {\isacharequal}{\kern0pt}\ right{\isacharunderscore}{\kern0pt}cart{\isacharunderscore}{\kern0pt}proj\ X\ X\ {\isasymcirc}\isactrlsub c\ k\isanewline
\ \ \ \ \ \ \ \ \ \ {\isasymand}\ fibered{\isacharunderscore}{\kern0pt}product{\isacharunderscore}{\kern0pt}left{\isacharunderscore}{\kern0pt}proj\ X\ f\ f\ X\ {\isasymcirc}\isactrlsub c\ j\ {\isacharequal}{\kern0pt}\ left{\isacharunderscore}{\kern0pt}cart{\isacharunderscore}{\kern0pt}proj\ X\ X\ {\isasymcirc}\isactrlsub c\ k\ {\isasymLongrightarrow}\ z\ {\isacharequal}{\kern0pt}\ j{\isachardoublequoteclose}\isanewline
\ \ \ \ \ \ \isacommand{using}\isamarkupfalse%
\ left{\isacharunderscore}{\kern0pt}k{\isacharunderscore}{\kern0pt}right{\isacharunderscore}{\kern0pt}k{\isacharunderscore}{\kern0pt}eq\ \isacommand{by}\isamarkupfalse%
\ {\isacharparenleft}{\kern0pt}typecheck{\isacharunderscore}{\kern0pt}cfuncs{\isacharcomma}{\kern0pt}\ auto{\isacharparenright}{\kern0pt}\isanewline
\isanewline
\ \ \ \ \isacommand{have}\isamarkupfalse%
\ k{\isacharunderscore}{\kern0pt}eq{\isacharcolon}{\kern0pt}\ {\isachardoublequoteopen}fibered{\isacharunderscore}{\kern0pt}product{\isacharunderscore}{\kern0pt}morphism\ X\ f\ f\ X\ {\isasymcirc}\isactrlsub c\ z\ {\isacharequal}{\kern0pt}\ k{\isachardoublequoteclose}\isanewline
\ \ \ \ \ \ \isacommand{using}\isamarkupfalse%
\ k{\isacharunderscore}{\kern0pt}right{\isacharunderscore}{\kern0pt}eq\ k{\isacharunderscore}{\kern0pt}left{\isacharunderscore}{\kern0pt}eq\isanewline
\ \ \ \ \ \ \isacommand{unfolding}\isamarkupfalse%
\ fibered{\isacharunderscore}{\kern0pt}product{\isacharunderscore}{\kern0pt}right{\isacharunderscore}{\kern0pt}proj{\isacharunderscore}{\kern0pt}def\ fibered{\isacharunderscore}{\kern0pt}product{\isacharunderscore}{\kern0pt}left{\isacharunderscore}{\kern0pt}proj{\isacharunderscore}{\kern0pt}def\isanewline
\ \ \ \ \ \ \isacommand{by}\isamarkupfalse%
\ {\isacharparenleft}{\kern0pt}typecheck{\isacharunderscore}{\kern0pt}cfuncs{\isacharunderscore}{\kern0pt}prems{\isacharcomma}{\kern0pt}\ smt\ cfunc{\isacharunderscore}{\kern0pt}prod{\isacharunderscore}{\kern0pt}comp\ cfunc{\isacharunderscore}{\kern0pt}prod{\isacharunderscore}{\kern0pt}unique{\isacharparenright}{\kern0pt}\isanewline
\isanewline
\ \ \ \ \isacommand{show}\isamarkupfalse%
\ {\isachardoublequoteopen}{\isasymexists}l{\isachardot}{\kern0pt}\ l\ {\isacharcolon}{\kern0pt}\ Z\ {\isasymrightarrow}\ X\ \isactrlbsub f\isactrlesub {\isasymtimes}\isactrlsub c\isactrlbsub f\isactrlesub \ X\ {\isasymand}\ fibered{\isacharunderscore}{\kern0pt}product{\isacharunderscore}{\kern0pt}morphism\ X\ f\ f\ X\ {\isasymcirc}\isactrlsub c\ l\ {\isacharequal}{\kern0pt}\ k\ {\isasymand}\ b\ {\isasymcirc}\isactrlsub c\ l\ {\isacharequal}{\kern0pt}\ j{\isachardoublequoteclose}\isanewline
\ \ \ \ \isacommand{proof}\isamarkupfalse%
\ {\isacharparenleft}{\kern0pt}rule{\isacharunderscore}{\kern0pt}tac\ x{\isacharequal}{\kern0pt}z\ \isakeyword{in}\ exI{\isacharcomma}{\kern0pt}\ typecheck{\isacharunderscore}{\kern0pt}cfuncs{\isacharcomma}{\kern0pt}\ insert\ k{\isacharunderscore}{\kern0pt}eq{\isacharcomma}{\kern0pt}\ safe{\isacharparenright}{\kern0pt}\isanewline
\ \ \ \ \ \ \isacommand{have}\isamarkupfalse%
\ {\isachardoublequoteopen}fibered{\isacharunderscore}{\kern0pt}product{\isacharunderscore}{\kern0pt}morphism\ E\ h\ h\ E\ {\isasymcirc}\isactrlsub c\ j\ {\isacharequal}{\kern0pt}\ {\isacharparenleft}{\kern0pt}g\ {\isasymtimes}\isactrlsub f\ g{\isacharparenright}{\kern0pt}\ {\isasymcirc}\isactrlsub c\ k{\isachardoublequoteclose}\isanewline
\ \ \ \ \ \ \ \ \isacommand{by}\isamarkupfalse%
\ {\isacharparenleft}{\kern0pt}simp\ add{\isacharcolon}{\kern0pt}\ k{\isacharunderscore}{\kern0pt}h{\isacharunderscore}{\kern0pt}eq{\isacharparenright}{\kern0pt}\isanewline
\ \ \ \ \ \ \isacommand{also}\isamarkupfalse%
\ \isacommand{have}\isamarkupfalse%
\ {\isachardoublequoteopen}{\isachardot}{\kern0pt}{\isachardot}{\kern0pt}{\isachardot}{\kern0pt}\ {\isacharequal}{\kern0pt}\ {\isacharparenleft}{\kern0pt}g\ {\isasymtimes}\isactrlsub f\ g{\isacharparenright}{\kern0pt}\ {\isasymcirc}\isactrlsub c\ fibered{\isacharunderscore}{\kern0pt}product{\isacharunderscore}{\kern0pt}morphism\ X\ f\ f\ X\ {\isasymcirc}\isactrlsub c\ z{\isachardoublequoteclose}\isanewline
\ \ \ \ \ \ \ \ \isacommand{by}\isamarkupfalse%
\ {\isacharparenleft}{\kern0pt}simp\ add{\isacharcolon}{\kern0pt}\ k{\isacharunderscore}{\kern0pt}eq{\isacharparenright}{\kern0pt}\isanewline
\ \ \ \ \ \ \isacommand{also}\isamarkupfalse%
\ \isacommand{have}\isamarkupfalse%
\ {\isachardoublequoteopen}{\isachardot}{\kern0pt}{\isachardot}{\kern0pt}{\isachardot}{\kern0pt}\ {\isacharequal}{\kern0pt}\ fibered{\isacharunderscore}{\kern0pt}product{\isacharunderscore}{\kern0pt}morphism\ E\ h\ h\ E\ {\isasymcirc}\isactrlsub c\ b\ {\isasymcirc}\isactrlsub c\ z{\isachardoublequoteclose}\isanewline
\ \ \ \ \ \ \ \ \isacommand{by}\isamarkupfalse%
\ {\isacharparenleft}{\kern0pt}typecheck{\isacharunderscore}{\kern0pt}cfuncs{\isacharcomma}{\kern0pt}\ simp\ add{\isacharcolon}{\kern0pt}\ b{\isacharunderscore}{\kern0pt}eq\ comp{\isacharunderscore}{\kern0pt}associative{\isadigit{2}}{\isacharparenright}{\kern0pt}\isanewline
\ \ \ \ \ \ \isacommand{then}\isamarkupfalse%
\ \isacommand{show}\isamarkupfalse%
\ {\isachardoublequoteopen}b\ {\isasymcirc}\isactrlsub c\ z\ {\isacharequal}{\kern0pt}\ j{\isachardoublequoteclose}\isanewline
\ \ \ \ \ \ \ \ \isacommand{using}\isamarkupfalse%
\ calculation\ fibered{\isacharunderscore}{\kern0pt}product{\isacharunderscore}{\kern0pt}morphism{\isacharunderscore}{\kern0pt}monomorphism\ monomorphism{\isacharunderscore}{\kern0pt}def{\isadigit{2}}\ \isacommand{by}\isamarkupfalse%
\ {\isacharparenleft}{\kern0pt}typecheck{\isacharunderscore}{\kern0pt}cfuncs{\isacharunderscore}{\kern0pt}prems{\isacharcomma}{\kern0pt}\ metis{\isacharparenright}{\kern0pt}\isanewline
\ \ \ \ \isacommand{qed}\isamarkupfalse%
\isanewline
\isanewline
\ \ \ \ \isacommand{show}\isamarkupfalse%
\ {\isachardoublequoteopen}{\isasymAnd}\ j\ y{\isachardot}{\kern0pt}\ j\ {\isacharcolon}{\kern0pt}\ Z\ {\isasymrightarrow}\ X\ \isactrlbsub f\isactrlesub {\isasymtimes}\isactrlsub c\isactrlbsub f\isactrlesub \ X\ {\isasymLongrightarrow}\ y\ {\isacharcolon}{\kern0pt}\ Z\ {\isasymrightarrow}\ X\ \isactrlbsub f\isactrlesub {\isasymtimes}\isactrlsub c\isactrlbsub f\isactrlesub \ X\ {\isasymLongrightarrow}\isanewline
\ \ \ \ \ \ \ \ fibered{\isacharunderscore}{\kern0pt}product{\isacharunderscore}{\kern0pt}morphism\ X\ f\ f\ X\ {\isasymcirc}\isactrlsub c\ y\ {\isacharequal}{\kern0pt}\ fibered{\isacharunderscore}{\kern0pt}product{\isacharunderscore}{\kern0pt}morphism\ X\ f\ f\ X\ {\isasymcirc}\isactrlsub c\ j\ {\isasymLongrightarrow}\isanewline
\ \ \ \ \ \ \ \ j\ {\isacharequal}{\kern0pt}\ y{\isachardoublequoteclose}\isanewline
\ \ \ \ \ \ \isacommand{using}\isamarkupfalse%
\ fibered{\isacharunderscore}{\kern0pt}product{\isacharunderscore}{\kern0pt}morphism{\isacharunderscore}{\kern0pt}monomorphism\ monomorphism{\isacharunderscore}{\kern0pt}def{\isadigit{2}}\ \isacommand{by}\isamarkupfalse%
\ {\isacharparenleft}{\kern0pt}typecheck{\isacharunderscore}{\kern0pt}cfuncs{\isacharunderscore}{\kern0pt}prems{\isacharcomma}{\kern0pt}\ metis{\isacharparenright}{\kern0pt}\isanewline
\ \ \isacommand{qed}\isamarkupfalse%
\isanewline
\ \ \isacommand{then}\isamarkupfalse%
\ \isacommand{have}\isamarkupfalse%
\ b{\isacharunderscore}{\kern0pt}epi{\isacharcolon}{\kern0pt}\ {\isachardoublequoteopen}epimorphism\ b{\isachardoublequoteclose}\isanewline
\ \ \ \ \isacommand{using}\isamarkupfalse%
\ g{\isacharunderscore}{\kern0pt}epi\ g{\isacharunderscore}{\kern0pt}type\ cfunc{\isacharunderscore}{\kern0pt}cross{\isacharunderscore}{\kern0pt}prod{\isacharunderscore}{\kern0pt}type\ cfunc{\isacharunderscore}{\kern0pt}cross{\isacharunderscore}{\kern0pt}prod{\isacharunderscore}{\kern0pt}surj\ \ pullback{\isacharunderscore}{\kern0pt}of{\isacharunderscore}{\kern0pt}epi{\isacharunderscore}{\kern0pt}is{\isacharunderscore}{\kern0pt}epi{\isadigit{1}}\ h{\isacharunderscore}{\kern0pt}type\isanewline
\ \ \ \ \isacommand{by}\isamarkupfalse%
\ {\isacharparenleft}{\kern0pt}meson\ epi{\isacharunderscore}{\kern0pt}is{\isacharunderscore}{\kern0pt}surj\ surjective{\isacharunderscore}{\kern0pt}is{\isacharunderscore}{\kern0pt}epimorphism{\isacharparenright}{\kern0pt}\isanewline
\isanewline
\ \ \isacommand{have}\isamarkupfalse%
\ existence{\isacharcolon}{\kern0pt}\ {\isachardoublequoteopen}{\isasymexists}b{\isachardot}{\kern0pt}\ b\ {\isacharcolon}{\kern0pt}\ X\ \isactrlbsub f\isactrlesub {\isasymtimes}\isactrlsub c\isactrlbsub f\isactrlesub \ X\ {\isasymrightarrow}\ E\ \isactrlbsub h\isactrlesub {\isasymtimes}\isactrlsub c\isactrlbsub h\isactrlesub \ E\ {\isasymand}\isanewline
\ \ \ \ \ \ \ \ fibered{\isacharunderscore}{\kern0pt}product{\isacharunderscore}{\kern0pt}left{\isacharunderscore}{\kern0pt}proj\ E\ h\ h\ E\ {\isasymcirc}\isactrlsub c\ b\ {\isacharequal}{\kern0pt}\ g\ {\isasymcirc}\isactrlsub c\ fibered{\isacharunderscore}{\kern0pt}product{\isacharunderscore}{\kern0pt}left{\isacharunderscore}{\kern0pt}proj\ X\ f\ f\ X\ {\isasymand}\isanewline
\ \ \ \ \ \ \ \ fibered{\isacharunderscore}{\kern0pt}product{\isacharunderscore}{\kern0pt}right{\isacharunderscore}{\kern0pt}proj\ E\ h\ h\ E\ {\isasymcirc}\isactrlsub c\ b\ {\isacharequal}{\kern0pt}\ g\ {\isasymcirc}\isactrlsub c\ fibered{\isacharunderscore}{\kern0pt}product{\isacharunderscore}{\kern0pt}right{\isacharunderscore}{\kern0pt}proj\ X\ f\ f\ X\ {\isasymand}\isanewline
\ \ \ \ \ \ \ \ epimorphism\ b{\isachardoublequoteclose}\isanewline
\ \ \isacommand{proof}\isamarkupfalse%
\ {\isacharparenleft}{\kern0pt}rule{\isacharunderscore}{\kern0pt}tac\ x{\isacharequal}{\kern0pt}b\ \isakeyword{in}\ exI{\isacharcomma}{\kern0pt}\ safe{\isacharparenright}{\kern0pt}\isanewline
\ \ \ \ \isacommand{show}\isamarkupfalse%
\ {\isachardoublequoteopen}b\ {\isacharcolon}{\kern0pt}\ X\ \isactrlbsub f\isactrlesub {\isasymtimes}\isactrlsub c\isactrlbsub f\isactrlesub \ X\ {\isasymrightarrow}\ E\ \isactrlbsub h\isactrlesub {\isasymtimes}\isactrlsub c\isactrlbsub h\isactrlesub \ E{\isachardoublequoteclose}\isanewline
\ \ \ \ \ \ \isacommand{by}\isamarkupfalse%
\ typecheck{\isacharunderscore}{\kern0pt}cfuncs\isanewline
\ \ \ \ \isacommand{show}\isamarkupfalse%
\ {\isachardoublequoteopen}fibered{\isacharunderscore}{\kern0pt}product{\isacharunderscore}{\kern0pt}left{\isacharunderscore}{\kern0pt}proj\ E\ h\ h\ E\ {\isasymcirc}\isactrlsub c\ b\ {\isacharequal}{\kern0pt}\ g\ {\isasymcirc}\isactrlsub c\ fibered{\isacharunderscore}{\kern0pt}product{\isacharunderscore}{\kern0pt}left{\isacharunderscore}{\kern0pt}proj\ X\ f\ f\ X{\isachardoublequoteclose}\isanewline
\ \ \ \ \isacommand{proof}\isamarkupfalse%
\ {\isacharminus}{\kern0pt}\isanewline
\ \ \ \ \ \ \isacommand{have}\isamarkupfalse%
\ {\isachardoublequoteopen}fibered{\isacharunderscore}{\kern0pt}product{\isacharunderscore}{\kern0pt}left{\isacharunderscore}{\kern0pt}proj\ E\ h\ h\ E\ {\isasymcirc}\isactrlsub c\ b\isanewline
\ \ \ \ \ \ \ \ \ \ {\isacharequal}{\kern0pt}\ left{\isacharunderscore}{\kern0pt}cart{\isacharunderscore}{\kern0pt}proj\ E\ E\ {\isasymcirc}\isactrlsub c\ fibered{\isacharunderscore}{\kern0pt}product{\isacharunderscore}{\kern0pt}morphism\ E\ h\ h\ E\ {\isasymcirc}\isactrlsub c\ b{\isachardoublequoteclose}\isanewline
\ \ \ \ \ \ \ \ \isacommand{unfolding}\isamarkupfalse%
\ fibered{\isacharunderscore}{\kern0pt}product{\isacharunderscore}{\kern0pt}left{\isacharunderscore}{\kern0pt}proj{\isacharunderscore}{\kern0pt}def\ \isacommand{by}\isamarkupfalse%
\ {\isacharparenleft}{\kern0pt}typecheck{\isacharunderscore}{\kern0pt}cfuncs{\isacharcomma}{\kern0pt}\ simp\ add{\isacharcolon}{\kern0pt}\ comp{\isacharunderscore}{\kern0pt}associative{\isadigit{2}}{\isacharparenright}{\kern0pt}\isanewline
\ \ \ \ \ \ \isacommand{also}\isamarkupfalse%
\ \isacommand{have}\isamarkupfalse%
\ {\isachardoublequoteopen}{\isachardot}{\kern0pt}{\isachardot}{\kern0pt}{\isachardot}{\kern0pt}\ {\isacharequal}{\kern0pt}\ left{\isacharunderscore}{\kern0pt}cart{\isacharunderscore}{\kern0pt}proj\ E\ E\ {\isasymcirc}\isactrlsub c\ {\isacharparenleft}{\kern0pt}g\ {\isasymtimes}\isactrlsub f\ g{\isacharparenright}{\kern0pt}\ {\isasymcirc}\isactrlsub c\ fibered{\isacharunderscore}{\kern0pt}product{\isacharunderscore}{\kern0pt}morphism\ X\ f\ f\ X{\isachardoublequoteclose}\isanewline
\ \ \ \ \ \ \ \ \isacommand{by}\isamarkupfalse%
\ {\isacharparenleft}{\kern0pt}simp\ add{\isacharcolon}{\kern0pt}\ b{\isacharunderscore}{\kern0pt}eq{\isacharparenright}{\kern0pt}\isanewline
\ \ \ \ \ \ \isacommand{also}\isamarkupfalse%
\ \isacommand{have}\isamarkupfalse%
\ {\isachardoublequoteopen}{\isachardot}{\kern0pt}{\isachardot}{\kern0pt}{\isachardot}{\kern0pt}\ {\isacharequal}{\kern0pt}\ g\ {\isasymcirc}\isactrlsub c\ left{\isacharunderscore}{\kern0pt}cart{\isacharunderscore}{\kern0pt}proj\ X\ X\ {\isasymcirc}\isactrlsub c\ fibered{\isacharunderscore}{\kern0pt}product{\isacharunderscore}{\kern0pt}morphism\ X\ f\ f\ X{\isachardoublequoteclose}\isanewline
\ \ \ \ \ \ \ \ \isacommand{by}\isamarkupfalse%
\ {\isacharparenleft}{\kern0pt}typecheck{\isacharunderscore}{\kern0pt}cfuncs{\isacharcomma}{\kern0pt}\ simp\ add{\isacharcolon}{\kern0pt}\ comp{\isacharunderscore}{\kern0pt}associative{\isadigit{2}}\ left{\isacharunderscore}{\kern0pt}cart{\isacharunderscore}{\kern0pt}proj{\isacharunderscore}{\kern0pt}cfunc{\isacharunderscore}{\kern0pt}cross{\isacharunderscore}{\kern0pt}prod{\isacharparenright}{\kern0pt}\isanewline
\ \ \ \ \ \ \isacommand{also}\isamarkupfalse%
\ \isacommand{have}\isamarkupfalse%
\ {\isachardoublequoteopen}{\isachardot}{\kern0pt}{\isachardot}{\kern0pt}{\isachardot}{\kern0pt}\ {\isacharequal}{\kern0pt}\ g\ {\isasymcirc}\isactrlsub c\ fibered{\isacharunderscore}{\kern0pt}product{\isacharunderscore}{\kern0pt}left{\isacharunderscore}{\kern0pt}proj\ X\ f\ f\ X{\isachardoublequoteclose}\isanewline
\ \ \ \ \ \ \ \ \isacommand{unfolding}\isamarkupfalse%
\ fibered{\isacharunderscore}{\kern0pt}product{\isacharunderscore}{\kern0pt}left{\isacharunderscore}{\kern0pt}proj{\isacharunderscore}{\kern0pt}def\ \isacommand{by}\isamarkupfalse%
\ {\isacharparenleft}{\kern0pt}typecheck{\isacharunderscore}{\kern0pt}cfuncs{\isacharparenright}{\kern0pt}\isanewline
\ \ \ \ \ \ \isacommand{then}\isamarkupfalse%
\ \isacommand{show}\isamarkupfalse%
\ {\isacharquery}{\kern0pt}thesis\isanewline
\ \ \ \ \ \ \ \ \isacommand{using}\isamarkupfalse%
\ calculation\ \isacommand{by}\isamarkupfalse%
\ auto\isanewline
\ \ \ \ \isacommand{qed}\isamarkupfalse%
\isanewline
\ \ \ \ \isacommand{show}\isamarkupfalse%
\ {\isachardoublequoteopen}fibered{\isacharunderscore}{\kern0pt}product{\isacharunderscore}{\kern0pt}right{\isacharunderscore}{\kern0pt}proj\ E\ h\ h\ E\ {\isasymcirc}\isactrlsub c\ b\ {\isacharequal}{\kern0pt}\ g\ {\isasymcirc}\isactrlsub c\ fibered{\isacharunderscore}{\kern0pt}product{\isacharunderscore}{\kern0pt}right{\isacharunderscore}{\kern0pt}proj\ X\ f\ f\ X{\isachardoublequoteclose}\isanewline
\ \ \ \ \isacommand{proof}\isamarkupfalse%
\ {\isacharminus}{\kern0pt}\isanewline
\ \ \ \ \ \ \isacommand{have}\isamarkupfalse%
\ {\isachardoublequoteopen}fibered{\isacharunderscore}{\kern0pt}product{\isacharunderscore}{\kern0pt}right{\isacharunderscore}{\kern0pt}proj\ E\ h\ h\ E\ {\isasymcirc}\isactrlsub c\ b\isanewline
\ \ \ \ \ \ \ \ \ \ {\isacharequal}{\kern0pt}\ right{\isacharunderscore}{\kern0pt}cart{\isacharunderscore}{\kern0pt}proj\ E\ E\ {\isasymcirc}\isactrlsub c\ fibered{\isacharunderscore}{\kern0pt}product{\isacharunderscore}{\kern0pt}morphism\ E\ h\ h\ E\ {\isasymcirc}\isactrlsub c\ b{\isachardoublequoteclose}\isanewline
\ \ \ \ \ \ \ \ \isacommand{unfolding}\isamarkupfalse%
\ fibered{\isacharunderscore}{\kern0pt}product{\isacharunderscore}{\kern0pt}right{\isacharunderscore}{\kern0pt}proj{\isacharunderscore}{\kern0pt}def\ \isacommand{by}\isamarkupfalse%
\ {\isacharparenleft}{\kern0pt}typecheck{\isacharunderscore}{\kern0pt}cfuncs{\isacharcomma}{\kern0pt}\ simp\ add{\isacharcolon}{\kern0pt}\ comp{\isacharunderscore}{\kern0pt}associative{\isadigit{2}}{\isacharparenright}{\kern0pt}\isanewline
\ \ \ \ \ \ \isacommand{also}\isamarkupfalse%
\ \isacommand{have}\isamarkupfalse%
\ {\isachardoublequoteopen}{\isachardot}{\kern0pt}{\isachardot}{\kern0pt}{\isachardot}{\kern0pt}\ {\isacharequal}{\kern0pt}\ right{\isacharunderscore}{\kern0pt}cart{\isacharunderscore}{\kern0pt}proj\ E\ E\ {\isasymcirc}\isactrlsub c\ {\isacharparenleft}{\kern0pt}g\ {\isasymtimes}\isactrlsub f\ g{\isacharparenright}{\kern0pt}\ {\isasymcirc}\isactrlsub c\ fibered{\isacharunderscore}{\kern0pt}product{\isacharunderscore}{\kern0pt}morphism\ X\ f\ f\ X{\isachardoublequoteclose}\isanewline
\ \ \ \ \ \ \ \ \isacommand{by}\isamarkupfalse%
\ {\isacharparenleft}{\kern0pt}simp\ add{\isacharcolon}{\kern0pt}\ b{\isacharunderscore}{\kern0pt}eq{\isacharparenright}{\kern0pt}\isanewline
\ \ \ \ \ \ \isacommand{also}\isamarkupfalse%
\ \isacommand{have}\isamarkupfalse%
\ {\isachardoublequoteopen}{\isachardot}{\kern0pt}{\isachardot}{\kern0pt}{\isachardot}{\kern0pt}\ {\isacharequal}{\kern0pt}\ g\ {\isasymcirc}\isactrlsub c\ right{\isacharunderscore}{\kern0pt}cart{\isacharunderscore}{\kern0pt}proj\ X\ X\ {\isasymcirc}\isactrlsub c\ fibered{\isacharunderscore}{\kern0pt}product{\isacharunderscore}{\kern0pt}morphism\ X\ f\ f\ X{\isachardoublequoteclose}\isanewline
\ \ \ \ \ \ \ \ \isacommand{by}\isamarkupfalse%
\ {\isacharparenleft}{\kern0pt}typecheck{\isacharunderscore}{\kern0pt}cfuncs{\isacharcomma}{\kern0pt}\ simp\ add{\isacharcolon}{\kern0pt}\ comp{\isacharunderscore}{\kern0pt}associative{\isadigit{2}}\ right{\isacharunderscore}{\kern0pt}cart{\isacharunderscore}{\kern0pt}proj{\isacharunderscore}{\kern0pt}cfunc{\isacharunderscore}{\kern0pt}cross{\isacharunderscore}{\kern0pt}prod{\isacharparenright}{\kern0pt}\isanewline
\ \ \ \ \ \ \isacommand{also}\isamarkupfalse%
\ \isacommand{have}\isamarkupfalse%
\ {\isachardoublequoteopen}{\isachardot}{\kern0pt}{\isachardot}{\kern0pt}{\isachardot}{\kern0pt}\ {\isacharequal}{\kern0pt}\ g\ {\isasymcirc}\isactrlsub c\ fibered{\isacharunderscore}{\kern0pt}product{\isacharunderscore}{\kern0pt}right{\isacharunderscore}{\kern0pt}proj\ X\ f\ f\ X{\isachardoublequoteclose}\isanewline
\ \ \ \ \ \ \ \ \isacommand{unfolding}\isamarkupfalse%
\ fibered{\isacharunderscore}{\kern0pt}product{\isacharunderscore}{\kern0pt}right{\isacharunderscore}{\kern0pt}proj{\isacharunderscore}{\kern0pt}def\ \isacommand{by}\isamarkupfalse%
\ {\isacharparenleft}{\kern0pt}typecheck{\isacharunderscore}{\kern0pt}cfuncs{\isacharparenright}{\kern0pt}\isanewline
\ \ \ \ \ \ \isacommand{then}\isamarkupfalse%
\ \isacommand{show}\isamarkupfalse%
\ {\isacharquery}{\kern0pt}thesis\isanewline
\ \ \ \ \ \ \ \ \isacommand{using}\isamarkupfalse%
\ calculation\ \isacommand{by}\isamarkupfalse%
\ auto\isanewline
\ \ \ \ \isacommand{qed}\isamarkupfalse%
\isanewline
\ \ \ \ \isacommand{show}\isamarkupfalse%
\ {\isachardoublequoteopen}epimorphism\ b{\isachardoublequoteclose}\isanewline
\ \ \ \ \ \ \isacommand{by}\isamarkupfalse%
\ {\isacharparenleft}{\kern0pt}simp\ add{\isacharcolon}{\kern0pt}\ b{\isacharunderscore}{\kern0pt}epi{\isacharparenright}{\kern0pt}\isanewline
\ \ \isacommand{qed}\isamarkupfalse%
\ \ \isanewline
\ \ \isacommand{show}\isamarkupfalse%
\ {\isachardoublequoteopen}{\isasymexists}{\isacharbang}{\kern0pt}b{\isachardot}{\kern0pt}\ b\ {\isacharcolon}{\kern0pt}\ X\ \isactrlbsub f\isactrlesub {\isasymtimes}\isactrlsub c\isactrlbsub f\isactrlesub \ X\ {\isasymrightarrow}\ E\ \isactrlbsub h\isactrlesub {\isasymtimes}\isactrlsub c\isactrlbsub h\isactrlesub \ E\ {\isasymand}\isanewline
\ \ \ \ \ \ \ \ \ fibered{\isacharunderscore}{\kern0pt}product{\isacharunderscore}{\kern0pt}left{\isacharunderscore}{\kern0pt}proj\ E\ h\ h\ E\ {\isasymcirc}\isactrlsub c\ b\ {\isacharequal}{\kern0pt}\ g\ {\isasymcirc}\isactrlsub c\ fibered{\isacharunderscore}{\kern0pt}product{\isacharunderscore}{\kern0pt}left{\isacharunderscore}{\kern0pt}proj\ X\ f\ f\ X\ {\isasymand}\isanewline
\ \ \ \ \ \ \ \ \ fibered{\isacharunderscore}{\kern0pt}product{\isacharunderscore}{\kern0pt}right{\isacharunderscore}{\kern0pt}proj\ E\ h\ h\ E\ {\isasymcirc}\isactrlsub c\ b\ {\isacharequal}{\kern0pt}\ g\ {\isasymcirc}\isactrlsub c\ fibered{\isacharunderscore}{\kern0pt}product{\isacharunderscore}{\kern0pt}right{\isacharunderscore}{\kern0pt}proj\ X\ f\ f\ X\ {\isasymand}\isanewline
\ \ \ \ \ \ \ \ \ epimorphism\ b{\isachardoublequoteclose}\isanewline
\ \ \ \ \isacommand{by}\isamarkupfalse%
\ {\isacharparenleft}{\kern0pt}typecheck{\isacharunderscore}{\kern0pt}cfuncs{\isacharcomma}{\kern0pt}\ metis\ epimorphism{\isacharunderscore}{\kern0pt}def{\isadigit{2}}\ existence\ g{\isacharunderscore}{\kern0pt}eq\ iso{\isacharunderscore}{\kern0pt}imp{\isacharunderscore}{\kern0pt}epi{\isacharunderscore}{\kern0pt}and{\isacharunderscore}{\kern0pt}monic\ kern{\isacharunderscore}{\kern0pt}pair{\isacharunderscore}{\kern0pt}proj{\isacharunderscore}{\kern0pt}iso{\isacharunderscore}{\kern0pt}TFAE{\isadigit{2}}\ monomorphism{\isacharunderscore}{\kern0pt}def{\isadigit{3}}{\isacharparenright}{\kern0pt}\isanewline
\isacommand{qed}\isamarkupfalse%
%
\endisatagproof
{\isafoldproof}%
%
\isadelimproof
%
\endisadelimproof
%
\isadelimdocument
%
\endisadelimdocument
%
\isatagdocument
%
\isamarkupsection{Set Subtraction%
}
\isamarkuptrue%
%
\endisatagdocument
{\isafolddocument}%
%
\isadelimdocument
%
\endisadelimdocument
\isacommand{definition}\isamarkupfalse%
\ set{\isacharunderscore}{\kern0pt}subtraction\ {\isacharcolon}{\kern0pt}{\isacharcolon}{\kern0pt}\ {\isachardoublequoteopen}cset\ {\isasymRightarrow}\ cset\ {\isasymtimes}\ cfunc\ {\isasymRightarrow}\ cset{\isachardoublequoteclose}\ {\isacharparenleft}{\kern0pt}\isakeyword{infix}\ {\isachardoublequoteopen}{\isasymsetminus}{\isachardoublequoteclose}\ {\isadigit{6}}{\isadigit{0}}{\isacharparenright}{\kern0pt}\ \isakeyword{where}\isanewline
\ \ {\isachardoublequoteopen}Y\ {\isasymsetminus}\ X\ {\isacharequal}{\kern0pt}\ {\isacharparenleft}{\kern0pt}SOME\ E{\isachardot}{\kern0pt}\ {\isasymexists}\ m{\isacharprime}{\kern0pt}{\isachardot}{\kern0pt}\ \ equalizer\ E\ m{\isacharprime}{\kern0pt}\ {\isacharparenleft}{\kern0pt}characteristic{\isacharunderscore}{\kern0pt}func\ {\isacharparenleft}{\kern0pt}snd\ X{\isacharparenright}{\kern0pt}{\isacharparenright}{\kern0pt}\ {\isacharparenleft}{\kern0pt}{\isasymf}\ {\isasymcirc}\isactrlsub c\ {\isasymbeta}\isactrlbsub Y\isactrlesub {\isacharparenright}{\kern0pt}{\isacharparenright}{\kern0pt}{\isachardoublequoteclose}\isanewline
\isanewline
\isacommand{lemma}\isamarkupfalse%
\ set{\isacharunderscore}{\kern0pt}subtraction{\isacharunderscore}{\kern0pt}equalizer{\isacharcolon}{\kern0pt}\isanewline
\ \ \isakeyword{assumes}\ {\isachardoublequoteopen}m\ {\isacharcolon}{\kern0pt}\ X\ {\isasymrightarrow}\ Y{\isachardoublequoteclose}\ {\isachardoublequoteopen}monomorphism\ m{\isachardoublequoteclose}\isanewline
\ \ \isakeyword{shows}\ {\isachardoublequoteopen}{\isasymexists}\ m{\isacharprime}{\kern0pt}{\isachardot}{\kern0pt}\ \ equalizer\ {\isacharparenleft}{\kern0pt}Y\ {\isasymsetminus}\ {\isacharparenleft}{\kern0pt}X{\isacharcomma}{\kern0pt}m{\isacharparenright}{\kern0pt}{\isacharparenright}{\kern0pt}\ m{\isacharprime}{\kern0pt}\ {\isacharparenleft}{\kern0pt}characteristic{\isacharunderscore}{\kern0pt}func\ m{\isacharparenright}{\kern0pt}\ {\isacharparenleft}{\kern0pt}{\isasymf}\ {\isasymcirc}\isactrlsub c\ {\isasymbeta}\isactrlbsub Y\isactrlesub {\isacharparenright}{\kern0pt}{\isachardoublequoteclose}\isanewline
%
\isadelimproof
%
\endisadelimproof
%
\isatagproof
\isacommand{proof}\isamarkupfalse%
\ {\isacharminus}{\kern0pt}\isanewline
\ \ \isacommand{have}\isamarkupfalse%
\ {\isachardoublequoteopen}{\isasymexists}\ E\ m{\isacharprime}{\kern0pt}{\isachardot}{\kern0pt}\ equalizer\ E\ m{\isacharprime}{\kern0pt}\ {\isacharparenleft}{\kern0pt}characteristic{\isacharunderscore}{\kern0pt}func\ m{\isacharparenright}{\kern0pt}\ {\isacharparenleft}{\kern0pt}{\isasymf}\ {\isasymcirc}\isactrlsub c\ {\isasymbeta}\isactrlbsub Y\isactrlesub {\isacharparenright}{\kern0pt}{\isachardoublequoteclose}\isanewline
\ \ \ \ \isacommand{using}\isamarkupfalse%
\ assms\ equalizer{\isacharunderscore}{\kern0pt}exists\ \isacommand{by}\isamarkupfalse%
\ {\isacharparenleft}{\kern0pt}typecheck{\isacharunderscore}{\kern0pt}cfuncs{\isacharcomma}{\kern0pt}\ auto{\isacharparenright}{\kern0pt}\isanewline
\ \ \isacommand{then}\isamarkupfalse%
\ \isacommand{have}\isamarkupfalse%
\ {\isachardoublequoteopen}{\isasymexists}\ m{\isacharprime}{\kern0pt}{\isachardot}{\kern0pt}\ \ equalizer\ {\isacharparenleft}{\kern0pt}Y\ {\isasymsetminus}\ {\isacharparenleft}{\kern0pt}X{\isacharcomma}{\kern0pt}m{\isacharparenright}{\kern0pt}{\isacharparenright}{\kern0pt}\ m{\isacharprime}{\kern0pt}\ {\isacharparenleft}{\kern0pt}characteristic{\isacharunderscore}{\kern0pt}func\ {\isacharparenleft}{\kern0pt}snd\ {\isacharparenleft}{\kern0pt}X{\isacharcomma}{\kern0pt}m{\isacharparenright}{\kern0pt}{\isacharparenright}{\kern0pt}{\isacharparenright}{\kern0pt}\ {\isacharparenleft}{\kern0pt}{\isasymf}\ {\isasymcirc}\isactrlsub c\ {\isasymbeta}\isactrlbsub Y\isactrlesub {\isacharparenright}{\kern0pt}{\isachardoublequoteclose}\isanewline
\ \ \ \ \isacommand{by}\isamarkupfalse%
\ {\isacharparenleft}{\kern0pt}unfold\ set{\isacharunderscore}{\kern0pt}subtraction{\isacharunderscore}{\kern0pt}def{\isacharcomma}{\kern0pt}\ rule{\isacharunderscore}{\kern0pt}tac\ someI{\isacharunderscore}{\kern0pt}ex{\isacharcomma}{\kern0pt}\ auto{\isacharparenright}{\kern0pt}\isanewline
\ \ \isacommand{then}\isamarkupfalse%
\ \isacommand{show}\isamarkupfalse%
\ {\isachardoublequoteopen}{\isasymexists}\ m{\isacharprime}{\kern0pt}{\isachardot}{\kern0pt}\ \ equalizer\ {\isacharparenleft}{\kern0pt}Y\ {\isasymsetminus}\ {\isacharparenleft}{\kern0pt}X{\isacharcomma}{\kern0pt}m{\isacharparenright}{\kern0pt}{\isacharparenright}{\kern0pt}\ m{\isacharprime}{\kern0pt}\ {\isacharparenleft}{\kern0pt}characteristic{\isacharunderscore}{\kern0pt}func\ m{\isacharparenright}{\kern0pt}\ {\isacharparenleft}{\kern0pt}{\isasymf}\ {\isasymcirc}\isactrlsub c\ {\isasymbeta}\isactrlbsub Y\isactrlesub {\isacharparenright}{\kern0pt}{\isachardoublequoteclose}\isanewline
\ \ \ \ \isacommand{by}\isamarkupfalse%
\ auto\isanewline
\isacommand{qed}\isamarkupfalse%
%
\endisatagproof
{\isafoldproof}%
%
\isadelimproof
\isanewline
%
\endisadelimproof
\isanewline
\isacommand{definition}\isamarkupfalse%
\ complement{\isacharunderscore}{\kern0pt}morphism\ {\isacharcolon}{\kern0pt}{\isacharcolon}{\kern0pt}\ {\isachardoublequoteopen}cfunc\ {\isasymRightarrow}\ cfunc{\isachardoublequoteclose}\ {\isacharparenleft}{\kern0pt}{\isachardoublequoteopen}{\isacharunderscore}{\kern0pt}\isactrlsup c{\isachardoublequoteclose}\ {\isacharbrackleft}{\kern0pt}{\isadigit{1}}{\isadigit{0}}{\isadigit{0}}{\isadigit{0}}{\isacharbrackright}{\kern0pt}{\isacharparenright}{\kern0pt}\ \isakeyword{where}\isanewline
\ \ {\isachardoublequoteopen}m\isactrlsup c\ {\isacharequal}{\kern0pt}\ {\isacharparenleft}{\kern0pt}SOME\ m{\isacharprime}{\kern0pt}{\isachardot}{\kern0pt}\ \ equalizer\ {\isacharparenleft}{\kern0pt}codomain\ m\ {\isasymsetminus}\ {\isacharparenleft}{\kern0pt}domain\ m{\isacharcomma}{\kern0pt}\ m{\isacharparenright}{\kern0pt}{\isacharparenright}{\kern0pt}\ m{\isacharprime}{\kern0pt}\ {\isacharparenleft}{\kern0pt}characteristic{\isacharunderscore}{\kern0pt}func\ m{\isacharparenright}{\kern0pt}\ {\isacharparenleft}{\kern0pt}{\isasymf}\ {\isasymcirc}\isactrlsub c\ {\isasymbeta}\isactrlbsub codomain\ m\isactrlesub {\isacharparenright}{\kern0pt}{\isacharparenright}{\kern0pt}{\isachardoublequoteclose}\isanewline
\isanewline
\isacommand{lemma}\isamarkupfalse%
\ complement{\isacharunderscore}{\kern0pt}morphism{\isacharunderscore}{\kern0pt}equalizer{\isacharcolon}{\kern0pt}\isanewline
\ \ \isakeyword{assumes}\ {\isachardoublequoteopen}m\ {\isacharcolon}{\kern0pt}\ X\ {\isasymrightarrow}\ Y{\isachardoublequoteclose}\ {\isachardoublequoteopen}monomorphism\ m{\isachardoublequoteclose}\isanewline
\ \ \isakeyword{shows}\ {\isachardoublequoteopen}equalizer\ {\isacharparenleft}{\kern0pt}Y\ {\isasymsetminus}\ {\isacharparenleft}{\kern0pt}X{\isacharcomma}{\kern0pt}m{\isacharparenright}{\kern0pt}{\isacharparenright}{\kern0pt}\ m\isactrlsup c\ {\isacharparenleft}{\kern0pt}characteristic{\isacharunderscore}{\kern0pt}func\ m{\isacharparenright}{\kern0pt}\ {\isacharparenleft}{\kern0pt}{\isasymf}\ {\isasymcirc}\isactrlsub c\ {\isasymbeta}\isactrlbsub Y\isactrlesub {\isacharparenright}{\kern0pt}{\isachardoublequoteclose}\isanewline
%
\isadelimproof
%
\endisadelimproof
%
\isatagproof
\isacommand{proof}\isamarkupfalse%
\ {\isacharminus}{\kern0pt}\isanewline
\ \ \isacommand{have}\isamarkupfalse%
\ {\isachardoublequoteopen}{\isasymexists}\ m{\isacharprime}{\kern0pt}{\isachardot}{\kern0pt}\ equalizer\ {\isacharparenleft}{\kern0pt}codomain\ m\ {\isasymsetminus}\ {\isacharparenleft}{\kern0pt}domain\ m{\isacharcomma}{\kern0pt}\ m{\isacharparenright}{\kern0pt}{\isacharparenright}{\kern0pt}\ m{\isacharprime}{\kern0pt}\ {\isacharparenleft}{\kern0pt}characteristic{\isacharunderscore}{\kern0pt}func\ m{\isacharparenright}{\kern0pt}\ {\isacharparenleft}{\kern0pt}{\isasymf}\ {\isasymcirc}\isactrlsub c\ {\isasymbeta}\isactrlbsub codomain\ m\isactrlesub {\isacharparenright}{\kern0pt}{\isachardoublequoteclose}\isanewline
\ \ \ \ \isacommand{by}\isamarkupfalse%
\ {\isacharparenleft}{\kern0pt}simp\ add{\isacharcolon}{\kern0pt}\ assms\ cfunc{\isacharunderscore}{\kern0pt}type{\isacharunderscore}{\kern0pt}def\ set{\isacharunderscore}{\kern0pt}subtraction{\isacharunderscore}{\kern0pt}equalizer{\isacharparenright}{\kern0pt}\isanewline
\ \ \isacommand{then}\isamarkupfalse%
\ \isacommand{have}\isamarkupfalse%
\ {\isachardoublequoteopen}equalizer\ {\isacharparenleft}{\kern0pt}codomain\ m\ {\isasymsetminus}\ {\isacharparenleft}{\kern0pt}domain\ m{\isacharcomma}{\kern0pt}\ m{\isacharparenright}{\kern0pt}{\isacharparenright}{\kern0pt}\ m\isactrlsup c\ {\isacharparenleft}{\kern0pt}characteristic{\isacharunderscore}{\kern0pt}func\ m{\isacharparenright}{\kern0pt}\ {\isacharparenleft}{\kern0pt}{\isasymf}\ {\isasymcirc}\isactrlsub c\ {\isasymbeta}\isactrlbsub codomain\ m\isactrlesub {\isacharparenright}{\kern0pt}{\isachardoublequoteclose}\isanewline
\ \ \ \ \isacommand{by}\isamarkupfalse%
\ {\isacharparenleft}{\kern0pt}unfold\ complement{\isacharunderscore}{\kern0pt}morphism{\isacharunderscore}{\kern0pt}def{\isacharcomma}{\kern0pt}\ rule{\isacharunderscore}{\kern0pt}tac\ someI{\isacharunderscore}{\kern0pt}ex{\isacharcomma}{\kern0pt}\ auto{\isacharparenright}{\kern0pt}\isanewline
\ \ \isacommand{then}\isamarkupfalse%
\ \isacommand{show}\isamarkupfalse%
\ {\isachardoublequoteopen}equalizer\ {\isacharparenleft}{\kern0pt}Y\ {\isasymsetminus}\ {\isacharparenleft}{\kern0pt}X{\isacharcomma}{\kern0pt}\ m{\isacharparenright}{\kern0pt}{\isacharparenright}{\kern0pt}\ m\isactrlsup c\ {\isacharparenleft}{\kern0pt}characteristic{\isacharunderscore}{\kern0pt}func\ m{\isacharparenright}{\kern0pt}\ {\isacharparenleft}{\kern0pt}{\isasymf}\ {\isasymcirc}\isactrlsub c\ {\isasymbeta}\isactrlbsub Y\isactrlesub {\isacharparenright}{\kern0pt}{\isachardoublequoteclose}\isanewline
\ \ \ \ \isacommand{using}\isamarkupfalse%
\ assms\ \isacommand{unfolding}\isamarkupfalse%
\ cfunc{\isacharunderscore}{\kern0pt}type{\isacharunderscore}{\kern0pt}def\ \isacommand{by}\isamarkupfalse%
\ auto\isanewline
\isacommand{qed}\isamarkupfalse%
%
\endisatagproof
{\isafoldproof}%
%
\isadelimproof
\isanewline
%
\endisadelimproof
\isanewline
\isacommand{lemma}\isamarkupfalse%
\ complement{\isacharunderscore}{\kern0pt}morphism{\isacharunderscore}{\kern0pt}type{\isacharbrackleft}{\kern0pt}type{\isacharunderscore}{\kern0pt}rule{\isacharbrackright}{\kern0pt}{\isacharcolon}{\kern0pt}\isanewline
\ \ \isakeyword{assumes}\ {\isachardoublequoteopen}m\ {\isacharcolon}{\kern0pt}\ X\ {\isasymrightarrow}\ Y{\isachardoublequoteclose}\ {\isachardoublequoteopen}monomorphism\ m{\isachardoublequoteclose}\isanewline
\ \ \isakeyword{shows}\ {\isachardoublequoteopen}m\isactrlsup c\ {\isacharcolon}{\kern0pt}\ Y\ {\isasymsetminus}\ {\isacharparenleft}{\kern0pt}X{\isacharcomma}{\kern0pt}m{\isacharparenright}{\kern0pt}\ {\isasymrightarrow}\ Y{\isachardoublequoteclose}\isanewline
%
\isadelimproof
\ \ %
\endisadelimproof
%
\isatagproof
\isacommand{using}\isamarkupfalse%
\ assms\ cfunc{\isacharunderscore}{\kern0pt}type{\isacharunderscore}{\kern0pt}def\ characteristic{\isacharunderscore}{\kern0pt}func{\isacharunderscore}{\kern0pt}type\ complement{\isacharunderscore}{\kern0pt}morphism{\isacharunderscore}{\kern0pt}equalizer\ equalizer{\isacharunderscore}{\kern0pt}def\ \isacommand{by}\isamarkupfalse%
\ auto%
\endisatagproof
{\isafoldproof}%
%
\isadelimproof
\isanewline
%
\endisadelimproof
\isanewline
\isacommand{lemma}\isamarkupfalse%
\ complement{\isacharunderscore}{\kern0pt}morphism{\isacharunderscore}{\kern0pt}mono{\isacharcolon}{\kern0pt}\isanewline
\ \ \isakeyword{assumes}\ {\isachardoublequoteopen}m\ {\isacharcolon}{\kern0pt}\ X\ {\isasymrightarrow}\ Y{\isachardoublequoteclose}\ {\isachardoublequoteopen}monomorphism\ m{\isachardoublequoteclose}\isanewline
\ \ \isakeyword{shows}\ {\isachardoublequoteopen}monomorphism\ m\isactrlsup c{\isachardoublequoteclose}\isanewline
%
\isadelimproof
\ \ %
\endisadelimproof
%
\isatagproof
\isacommand{using}\isamarkupfalse%
\ assms\ complement{\isacharunderscore}{\kern0pt}morphism{\isacharunderscore}{\kern0pt}equalizer\ equalizer{\isacharunderscore}{\kern0pt}is{\isacharunderscore}{\kern0pt}monomorphism\ \isacommand{by}\isamarkupfalse%
\ blast%
\endisatagproof
{\isafoldproof}%
%
\isadelimproof
\isanewline
%
\endisadelimproof
\isanewline
\isacommand{lemma}\isamarkupfalse%
\ complement{\isacharunderscore}{\kern0pt}morphism{\isacharunderscore}{\kern0pt}eq{\isacharcolon}{\kern0pt}\isanewline
\ \ \isakeyword{assumes}\ {\isachardoublequoteopen}m\ {\isacharcolon}{\kern0pt}\ X\ {\isasymrightarrow}\ Y{\isachardoublequoteclose}\ {\isachardoublequoteopen}monomorphism\ m{\isachardoublequoteclose}\isanewline
\ \ \isakeyword{shows}\ {\isachardoublequoteopen}characteristic{\isacharunderscore}{\kern0pt}func\ m\ {\isasymcirc}\isactrlsub c\ m\isactrlsup c\ \ {\isacharequal}{\kern0pt}\ {\isacharparenleft}{\kern0pt}{\isasymf}\ {\isasymcirc}\isactrlsub c\ {\isasymbeta}\isactrlbsub Y\isactrlesub {\isacharparenright}{\kern0pt}\ {\isasymcirc}\isactrlsub c\ m\isactrlsup c{\isachardoublequoteclose}\isanewline
%
\isadelimproof
\ \ %
\endisadelimproof
%
\isatagproof
\isacommand{using}\isamarkupfalse%
\ assms\ complement{\isacharunderscore}{\kern0pt}morphism{\isacharunderscore}{\kern0pt}equalizer\ \isacommand{unfolding}\isamarkupfalse%
\ equalizer{\isacharunderscore}{\kern0pt}def\ \isacommand{by}\isamarkupfalse%
\ auto%
\endisatagproof
{\isafoldproof}%
%
\isadelimproof
\isanewline
%
\endisadelimproof
\isanewline
\isacommand{lemma}\isamarkupfalse%
\ characteristic{\isacharunderscore}{\kern0pt}func{\isacharunderscore}{\kern0pt}true{\isacharunderscore}{\kern0pt}not{\isacharunderscore}{\kern0pt}complement{\isacharunderscore}{\kern0pt}member{\isacharcolon}{\kern0pt}\isanewline
\ \ \isakeyword{assumes}\ {\isachardoublequoteopen}m\ {\isacharcolon}{\kern0pt}\ B\ {\isasymrightarrow}\ X{\isachardoublequoteclose}\ {\isachardoublequoteopen}monomorphism\ m{\isachardoublequoteclose}\ {\isachardoublequoteopen}x\ {\isasymin}\isactrlsub c\ X{\isachardoublequoteclose}\isanewline
\ \ \isakeyword{assumes}\ characteristic{\isacharunderscore}{\kern0pt}func{\isacharunderscore}{\kern0pt}true{\isacharcolon}{\kern0pt}\ {\isachardoublequoteopen}characteristic{\isacharunderscore}{\kern0pt}func\ m\ {\isasymcirc}\isactrlsub c\ x\ {\isacharequal}{\kern0pt}\ {\isasymt}{\isachardoublequoteclose}\isanewline
\ \ \isakeyword{shows}\ {\isachardoublequoteopen}{\isasymnot}\ x\ {\isasymin}\isactrlbsub X\isactrlesub \ {\isacharparenleft}{\kern0pt}X\ {\isasymsetminus}\ {\isacharparenleft}{\kern0pt}B{\isacharcomma}{\kern0pt}\ m{\isacharparenright}{\kern0pt}{\isacharcomma}{\kern0pt}m\isactrlsup c{\isacharparenright}{\kern0pt}{\isachardoublequoteclose}\isanewline
%
\isadelimproof
%
\endisadelimproof
%
\isatagproof
\isacommand{proof}\isamarkupfalse%
\isanewline
\ \ \isacommand{assume}\isamarkupfalse%
\ in{\isacharunderscore}{\kern0pt}complement{\isacharcolon}{\kern0pt}\ {\isachardoublequoteopen}x\ {\isasymin}\isactrlbsub X\isactrlesub \ {\isacharparenleft}{\kern0pt}X\ {\isasymsetminus}\ {\isacharparenleft}{\kern0pt}B{\isacharcomma}{\kern0pt}\ m{\isacharparenright}{\kern0pt}{\isacharcomma}{\kern0pt}\ m\isactrlsup c{\isacharparenright}{\kern0pt}{\isachardoublequoteclose}\isanewline
\ \ \isacommand{then}\isamarkupfalse%
\ \isacommand{obtain}\isamarkupfalse%
\ x{\isacharprime}{\kern0pt}\ \isakeyword{where}\ x{\isacharprime}{\kern0pt}{\isacharunderscore}{\kern0pt}type{\isacharcolon}{\kern0pt}\ {\isachardoublequoteopen}x{\isacharprime}{\kern0pt}\ {\isasymin}\isactrlsub c\ X\ {\isasymsetminus}\ {\isacharparenleft}{\kern0pt}B{\isacharcomma}{\kern0pt}m{\isacharparenright}{\kern0pt}{\isachardoublequoteclose}\ \isakeyword{and}\ x{\isacharprime}{\kern0pt}{\isacharunderscore}{\kern0pt}def{\isacharcolon}{\kern0pt}\ {\isachardoublequoteopen}m\isactrlsup c\ {\isasymcirc}\isactrlsub c\ x{\isacharprime}{\kern0pt}\ {\isacharequal}{\kern0pt}\ x{\isachardoublequoteclose}\isanewline
\ \ \ \ \isacommand{using}\isamarkupfalse%
\ assms\ cfunc{\isacharunderscore}{\kern0pt}type{\isacharunderscore}{\kern0pt}def\ complement{\isacharunderscore}{\kern0pt}morphism{\isacharunderscore}{\kern0pt}type\ factors{\isacharunderscore}{\kern0pt}through{\isacharunderscore}{\kern0pt}def\ relative{\isacharunderscore}{\kern0pt}member{\isacharunderscore}{\kern0pt}def{\isadigit{2}}\isanewline
\ \ \ \ \isacommand{by}\isamarkupfalse%
\ auto\isanewline
\ \ \isacommand{then}\isamarkupfalse%
\ \isacommand{have}\isamarkupfalse%
\ {\isachardoublequoteopen}characteristic{\isacharunderscore}{\kern0pt}func\ m\ {\isasymcirc}\isactrlsub c\ m\isactrlsup c\ {\isacharequal}{\kern0pt}\ {\isacharparenleft}{\kern0pt}{\isasymf}\ {\isasymcirc}\isactrlsub c\ {\isasymbeta}\isactrlbsub X\isactrlesub {\isacharparenright}{\kern0pt}\ {\isasymcirc}\isactrlsub c\ m\isactrlsup c{\isachardoublequoteclose}\isanewline
\ \ \ \ \isacommand{using}\isamarkupfalse%
\ assms\ complement{\isacharunderscore}{\kern0pt}morphism{\isacharunderscore}{\kern0pt}equalizer\ equalizer{\isacharunderscore}{\kern0pt}def\ \isacommand{by}\isamarkupfalse%
\ blast\isanewline
\ \ \isacommand{then}\isamarkupfalse%
\ \isacommand{have}\isamarkupfalse%
\ {\isachardoublequoteopen}characteristic{\isacharunderscore}{\kern0pt}func\ m\ {\isasymcirc}\isactrlsub c\ x\ {\isacharequal}{\kern0pt}\ {\isasymf}\ {\isasymcirc}\isactrlsub c\ {\isasymbeta}\isactrlbsub X\isactrlesub \ {\isasymcirc}\isactrlsub c\ x{\isachardoublequoteclose}\isanewline
\ \ \ \ \isacommand{using}\isamarkupfalse%
\ assms\ x{\isacharprime}{\kern0pt}{\isacharunderscore}{\kern0pt}type\ complement{\isacharunderscore}{\kern0pt}morphism{\isacharunderscore}{\kern0pt}type\isanewline
\ \ \ \ \isacommand{by}\isamarkupfalse%
\ {\isacharparenleft}{\kern0pt}typecheck{\isacharunderscore}{\kern0pt}cfuncs{\isacharcomma}{\kern0pt}\ smt\ x{\isacharprime}{\kern0pt}{\isacharunderscore}{\kern0pt}def\ assms\ cfunc{\isacharunderscore}{\kern0pt}type{\isacharunderscore}{\kern0pt}def\ comp{\isacharunderscore}{\kern0pt}associative\ domain{\isacharunderscore}{\kern0pt}comp{\isacharparenright}{\kern0pt}\isanewline
\ \ \isacommand{then}\isamarkupfalse%
\ \isacommand{have}\isamarkupfalse%
\ {\isachardoublequoteopen}characteristic{\isacharunderscore}{\kern0pt}func\ m\ {\isasymcirc}\isactrlsub c\ x\ {\isacharequal}{\kern0pt}\ {\isasymf}{\isachardoublequoteclose}\isanewline
\ \ \ \ \isacommand{using}\isamarkupfalse%
\ assms\ \isacommand{by}\isamarkupfalse%
\ {\isacharparenleft}{\kern0pt}typecheck{\isacharunderscore}{\kern0pt}cfuncs{\isacharcomma}{\kern0pt}\ metis\ id{\isacharunderscore}{\kern0pt}right{\isacharunderscore}{\kern0pt}unit{\isadigit{2}}\ id{\isacharunderscore}{\kern0pt}type\ one{\isacharunderscore}{\kern0pt}unique{\isacharunderscore}{\kern0pt}element\ terminal{\isacharunderscore}{\kern0pt}func{\isacharunderscore}{\kern0pt}comp\ terminal{\isacharunderscore}{\kern0pt}func{\isacharunderscore}{\kern0pt}type{\isacharparenright}{\kern0pt}\isanewline
\ \ \isacommand{then}\isamarkupfalse%
\ \isacommand{show}\isamarkupfalse%
\ False\isanewline
\ \ \ \ \isacommand{using}\isamarkupfalse%
\ characteristic{\isacharunderscore}{\kern0pt}func{\isacharunderscore}{\kern0pt}true\ true{\isacharunderscore}{\kern0pt}false{\isacharunderscore}{\kern0pt}distinct\ \isacommand{by}\isamarkupfalse%
\ auto\isanewline
\isacommand{qed}\isamarkupfalse%
%
\endisatagproof
{\isafoldproof}%
%
\isadelimproof
\isanewline
%
\endisadelimproof
\isanewline
\isacommand{lemma}\isamarkupfalse%
\ characteristic{\isacharunderscore}{\kern0pt}func{\isacharunderscore}{\kern0pt}false{\isacharunderscore}{\kern0pt}complement{\isacharunderscore}{\kern0pt}member{\isacharcolon}{\kern0pt}\isanewline
\ \ \isakeyword{assumes}\ {\isachardoublequoteopen}m\ {\isacharcolon}{\kern0pt}\ B\ {\isasymrightarrow}\ X{\isachardoublequoteclose}\ {\isachardoublequoteopen}monomorphism\ m{\isachardoublequoteclose}\ {\isachardoublequoteopen}x\ {\isasymin}\isactrlsub c\ X{\isachardoublequoteclose}\isanewline
\ \ \isakeyword{assumes}\ characteristic{\isacharunderscore}{\kern0pt}func{\isacharunderscore}{\kern0pt}false{\isacharcolon}{\kern0pt}\ {\isachardoublequoteopen}characteristic{\isacharunderscore}{\kern0pt}func\ m\ {\isasymcirc}\isactrlsub c\ x\ {\isacharequal}{\kern0pt}\ {\isasymf}{\isachardoublequoteclose}\isanewline
\ \ \isakeyword{shows}\ {\isachardoublequoteopen}x\ {\isasymin}\isactrlbsub X\isactrlesub \ {\isacharparenleft}{\kern0pt}X\ {\isasymsetminus}\ {\isacharparenleft}{\kern0pt}B{\isacharcomma}{\kern0pt}\ m{\isacharparenright}{\kern0pt}{\isacharcomma}{\kern0pt}m\isactrlsup c{\isacharparenright}{\kern0pt}{\isachardoublequoteclose}\isanewline
%
\isadelimproof
%
\endisadelimproof
%
\isatagproof
\isacommand{proof}\isamarkupfalse%
\ {\isacharminus}{\kern0pt}\isanewline
\ \ \isacommand{have}\isamarkupfalse%
\ x{\isacharunderscore}{\kern0pt}equalizes{\isacharcolon}{\kern0pt}\ {\isachardoublequoteopen}characteristic{\isacharunderscore}{\kern0pt}func\ m\ {\isasymcirc}\isactrlsub c\ x\ {\isacharequal}{\kern0pt}\ {\isasymf}\ {\isasymcirc}\isactrlsub c\ {\isasymbeta}\isactrlbsub X\isactrlesub \ {\isasymcirc}\isactrlsub c\ x{\isachardoublequoteclose}\isanewline
\ \ \ \ \isacommand{by}\isamarkupfalse%
\ {\isacharparenleft}{\kern0pt}metis\ assms{\isacharparenleft}{\kern0pt}{\isadigit{3}}{\isacharparenright}{\kern0pt}\ characteristic{\isacharunderscore}{\kern0pt}func{\isacharunderscore}{\kern0pt}false\ false{\isacharunderscore}{\kern0pt}func{\isacharunderscore}{\kern0pt}type\ id{\isacharunderscore}{\kern0pt}right{\isacharunderscore}{\kern0pt}unit{\isadigit{2}}\ id{\isacharunderscore}{\kern0pt}type\ one{\isacharunderscore}{\kern0pt}unique{\isacharunderscore}{\kern0pt}element\ terminal{\isacharunderscore}{\kern0pt}func{\isacharunderscore}{\kern0pt}comp\ terminal{\isacharunderscore}{\kern0pt}func{\isacharunderscore}{\kern0pt}type{\isacharparenright}{\kern0pt}\isanewline
\ \ \isacommand{have}\isamarkupfalse%
\ {\isachardoublequoteopen}{\isasymAnd}h\ F{\isachardot}{\kern0pt}\ h\ {\isacharcolon}{\kern0pt}\ F\ {\isasymrightarrow}\ X\ {\isasymand}\ characteristic{\isacharunderscore}{\kern0pt}func\ m\ {\isasymcirc}\isactrlsub c\ h\ {\isacharequal}{\kern0pt}\ {\isacharparenleft}{\kern0pt}{\isasymf}\ {\isasymcirc}\isactrlsub c\ {\isasymbeta}\isactrlbsub X\isactrlesub {\isacharparenright}{\kern0pt}\ {\isasymcirc}\isactrlsub c\ h\ {\isasymlongrightarrow}\isanewline
\ \ \ \ \ \ \ \ \ \ \ \ \ \ \ \ \ \ {\isacharparenleft}{\kern0pt}{\isasymexists}{\isacharbang}{\kern0pt}k{\isachardot}{\kern0pt}\ k\ {\isacharcolon}{\kern0pt}\ F\ {\isasymrightarrow}\ X\ {\isasymsetminus}\ {\isacharparenleft}{\kern0pt}B{\isacharcomma}{\kern0pt}\ m{\isacharparenright}{\kern0pt}\ {\isasymand}\ m\isactrlsup c\ {\isasymcirc}\isactrlsub c\ k\ {\isacharequal}{\kern0pt}\ h{\isacharparenright}{\kern0pt}{\isachardoublequoteclose}\isanewline
\ \ \ \ \isacommand{using}\isamarkupfalse%
\ assms\ complement{\isacharunderscore}{\kern0pt}morphism{\isacharunderscore}{\kern0pt}equalizer\ \isacommand{unfolding}\isamarkupfalse%
\ equalizer{\isacharunderscore}{\kern0pt}def\isanewline
\ \ \ \ \isacommand{by}\isamarkupfalse%
\ {\isacharparenleft}{\kern0pt}smt\ cfunc{\isacharunderscore}{\kern0pt}type{\isacharunderscore}{\kern0pt}def\ characteristic{\isacharunderscore}{\kern0pt}func{\isacharunderscore}{\kern0pt}type{\isacharparenright}{\kern0pt}\ \isanewline
\ \ \isacommand{then}\isamarkupfalse%
\ \isacommand{obtain}\isamarkupfalse%
\ x{\isacharprime}{\kern0pt}\ \isakeyword{where}\ x{\isacharprime}{\kern0pt}{\isacharunderscore}{\kern0pt}type{\isacharcolon}{\kern0pt}\ {\isachardoublequoteopen}x{\isacharprime}{\kern0pt}\ {\isasymin}\isactrlsub c\ X\ {\isasymsetminus}\ {\isacharparenleft}{\kern0pt}B{\isacharcomma}{\kern0pt}\ m{\isacharparenright}{\kern0pt}{\isachardoublequoteclose}\ \isakeyword{and}\ x{\isacharprime}{\kern0pt}{\isacharunderscore}{\kern0pt}def{\isacharcolon}{\kern0pt}\ {\isachardoublequoteopen}m\isactrlsup c\ {\isasymcirc}\isactrlsub c\ x{\isacharprime}{\kern0pt}\ {\isacharequal}{\kern0pt}\ x{\isachardoublequoteclose}\isanewline
\ \ \ \ \isacommand{by}\isamarkupfalse%
\ {\isacharparenleft}{\kern0pt}metis\ assms{\isacharparenleft}{\kern0pt}{\isadigit{3}}{\isacharparenright}{\kern0pt}\ cfunc{\isacharunderscore}{\kern0pt}type{\isacharunderscore}{\kern0pt}def\ comp{\isacharunderscore}{\kern0pt}associative\ false{\isacharunderscore}{\kern0pt}func{\isacharunderscore}{\kern0pt}type\ terminal{\isacharunderscore}{\kern0pt}func{\isacharunderscore}{\kern0pt}type\ x{\isacharunderscore}{\kern0pt}equalizes{\isacharparenright}{\kern0pt}\isanewline
\ \ \isacommand{then}\isamarkupfalse%
\ \isacommand{show}\isamarkupfalse%
\ {\isachardoublequoteopen}x\ {\isasymin}\isactrlbsub X\isactrlesub \ {\isacharparenleft}{\kern0pt}X\ {\isasymsetminus}\ {\isacharparenleft}{\kern0pt}B{\isacharcomma}{\kern0pt}\ m{\isacharparenright}{\kern0pt}{\isacharcomma}{\kern0pt}m\isactrlsup c{\isacharparenright}{\kern0pt}{\isachardoublequoteclose}\isanewline
\ \ \ \ \isacommand{unfolding}\isamarkupfalse%
\ relative{\isacharunderscore}{\kern0pt}member{\isacharunderscore}{\kern0pt}def\ factors{\isacharunderscore}{\kern0pt}through{\isacharunderscore}{\kern0pt}def\isanewline
\ \ \ \ \isacommand{using}\isamarkupfalse%
\ assms\ complement{\isacharunderscore}{\kern0pt}morphism{\isacharunderscore}{\kern0pt}mono\ complement{\isacharunderscore}{\kern0pt}morphism{\isacharunderscore}{\kern0pt}type\ cfunc{\isacharunderscore}{\kern0pt}type{\isacharunderscore}{\kern0pt}def\ \isacommand{by}\isamarkupfalse%
\ auto\isanewline
\isacommand{qed}\isamarkupfalse%
%
\endisatagproof
{\isafoldproof}%
%
\isadelimproof
\isanewline
%
\endisadelimproof
\isanewline
\isacommand{lemma}\isamarkupfalse%
\ in{\isacharunderscore}{\kern0pt}complement{\isacharunderscore}{\kern0pt}not{\isacharunderscore}{\kern0pt}in{\isacharunderscore}{\kern0pt}subset{\isacharcolon}{\kern0pt}\isanewline
\ \ \isakeyword{assumes}\ {\isachardoublequoteopen}m\ {\isacharcolon}{\kern0pt}\ X\ {\isasymrightarrow}\ Y{\isachardoublequoteclose}\ {\isachardoublequoteopen}monomorphism\ m{\isachardoublequoteclose}\ {\isachardoublequoteopen}x\ {\isasymin}\isactrlsub c\ Y{\isachardoublequoteclose}\isanewline
\ \ \isakeyword{assumes}\ {\isachardoublequoteopen}x\ {\isasymin}\isactrlbsub Y\isactrlesub \ {\isacharparenleft}{\kern0pt}Y\ {\isasymsetminus}\ {\isacharparenleft}{\kern0pt}X{\isacharcomma}{\kern0pt}m{\isacharparenright}{\kern0pt}{\isacharcomma}{\kern0pt}\ m\isactrlsup c{\isacharparenright}{\kern0pt}{\isachardoublequoteclose}\isanewline
\ \ \isakeyword{shows}\ {\isachardoublequoteopen}{\isasymnot}\ x\ {\isasymin}\isactrlbsub Y\isactrlesub \ {\isacharparenleft}{\kern0pt}X{\isacharcomma}{\kern0pt}\ m{\isacharparenright}{\kern0pt}{\isachardoublequoteclose}\isanewline
%
\isadelimproof
\ \ %
\endisadelimproof
%
\isatagproof
\isacommand{using}\isamarkupfalse%
\ assms\ characteristic{\isacharunderscore}{\kern0pt}func{\isacharunderscore}{\kern0pt}false{\isacharunderscore}{\kern0pt}not{\isacharunderscore}{\kern0pt}relative{\isacharunderscore}{\kern0pt}member\isanewline
\ \ \ \ characteristic{\isacharunderscore}{\kern0pt}func{\isacharunderscore}{\kern0pt}true{\isacharunderscore}{\kern0pt}not{\isacharunderscore}{\kern0pt}complement{\isacharunderscore}{\kern0pt}member\ characteristic{\isacharunderscore}{\kern0pt}func{\isacharunderscore}{\kern0pt}type\ comp{\isacharunderscore}{\kern0pt}type\isanewline
\ \ \ \ true{\isacharunderscore}{\kern0pt}false{\isacharunderscore}{\kern0pt}only{\isacharunderscore}{\kern0pt}truth{\isacharunderscore}{\kern0pt}values\ \isacommand{by}\isamarkupfalse%
\ blast%
\endisatagproof
{\isafoldproof}%
%
\isadelimproof
\isanewline
%
\endisadelimproof
\isanewline
\isacommand{lemma}\isamarkupfalse%
\ not{\isacharunderscore}{\kern0pt}in{\isacharunderscore}{\kern0pt}subset{\isacharunderscore}{\kern0pt}in{\isacharunderscore}{\kern0pt}complement{\isacharcolon}{\kern0pt}\isanewline
\ \ \isakeyword{assumes}\ {\isachardoublequoteopen}m\ {\isacharcolon}{\kern0pt}\ X\ {\isasymrightarrow}\ Y{\isachardoublequoteclose}\ {\isachardoublequoteopen}monomorphism\ m{\isachardoublequoteclose}\ {\isachardoublequoteopen}x\ {\isasymin}\isactrlsub c\ Y{\isachardoublequoteclose}\isanewline
\ \ \isakeyword{assumes}\ {\isachardoublequoteopen}{\isasymnot}\ x\ {\isasymin}\isactrlbsub Y\isactrlesub \ {\isacharparenleft}{\kern0pt}X{\isacharcomma}{\kern0pt}\ m{\isacharparenright}{\kern0pt}{\isachardoublequoteclose}\isanewline
\ \ \isakeyword{shows}\ {\isachardoublequoteopen}x\ {\isasymin}\isactrlbsub Y\isactrlesub \ {\isacharparenleft}{\kern0pt}Y\ {\isasymsetminus}\ {\isacharparenleft}{\kern0pt}X{\isacharcomma}{\kern0pt}m{\isacharparenright}{\kern0pt}{\isacharcomma}{\kern0pt}\ m\isactrlsup c{\isacharparenright}{\kern0pt}{\isachardoublequoteclose}\isanewline
%
\isadelimproof
\ \ %
\endisadelimproof
%
\isatagproof
\isacommand{using}\isamarkupfalse%
\ assms\ characteristic{\isacharunderscore}{\kern0pt}func{\isacharunderscore}{\kern0pt}false{\isacharunderscore}{\kern0pt}complement{\isacharunderscore}{\kern0pt}member\ characteristic{\isacharunderscore}{\kern0pt}func{\isacharunderscore}{\kern0pt}true{\isacharunderscore}{\kern0pt}relative{\isacharunderscore}{\kern0pt}member\isanewline
\ \ \ \ characteristic{\isacharunderscore}{\kern0pt}func{\isacharunderscore}{\kern0pt}type\ comp{\isacharunderscore}{\kern0pt}type\ true{\isacharunderscore}{\kern0pt}false{\isacharunderscore}{\kern0pt}only{\isacharunderscore}{\kern0pt}truth{\isacharunderscore}{\kern0pt}values\ \isacommand{by}\isamarkupfalse%
\ blast%
\endisatagproof
{\isafoldproof}%
%
\isadelimproof
\isanewline
%
\endisadelimproof
\isanewline
\isacommand{lemma}\isamarkupfalse%
\ complement{\isacharunderscore}{\kern0pt}disjoint{\isacharcolon}{\kern0pt}\isanewline
\ \ \isakeyword{assumes}\ {\isachardoublequoteopen}m\ {\isacharcolon}{\kern0pt}\ X\ {\isasymrightarrow}\ Y{\isachardoublequoteclose}\ {\isachardoublequoteopen}monomorphism\ m{\isachardoublequoteclose}\isanewline
\ \ \isakeyword{assumes}\ {\isachardoublequoteopen}x\ {\isasymin}\isactrlsub c\ X{\isachardoublequoteclose}\ {\isachardoublequoteopen}x{\isacharprime}{\kern0pt}\ {\isasymin}\isactrlsub c\ Y\ {\isasymsetminus}\ {\isacharparenleft}{\kern0pt}X{\isacharcomma}{\kern0pt}m{\isacharparenright}{\kern0pt}{\isachardoublequoteclose}\isanewline
\ \ \isakeyword{shows}\ {\isachardoublequoteopen}m\ {\isasymcirc}\isactrlsub c\ x\ {\isasymnoteq}\ m\isactrlsup c\ {\isasymcirc}\isactrlsub c\ x{\isacharprime}{\kern0pt}{\isachardoublequoteclose}\isanewline
%
\isadelimproof
%
\endisadelimproof
%
\isatagproof
\isacommand{proof}\isamarkupfalse%
\ \isanewline
\ \ \isacommand{assume}\isamarkupfalse%
\ {\isachardoublequoteopen}m\ {\isasymcirc}\isactrlsub c\ x\ {\isacharequal}{\kern0pt}\ m\isactrlsup c\ {\isasymcirc}\isactrlsub c\ x{\isacharprime}{\kern0pt}{\isachardoublequoteclose}\isanewline
\ \ \isacommand{then}\isamarkupfalse%
\ \isacommand{have}\isamarkupfalse%
\ {\isachardoublequoteopen}characteristic{\isacharunderscore}{\kern0pt}func\ m\ {\isasymcirc}\isactrlsub c\ m\ {\isasymcirc}\isactrlsub c\ x\ {\isacharequal}{\kern0pt}\ characteristic{\isacharunderscore}{\kern0pt}func\ m\ {\isasymcirc}\isactrlsub c\ m\isactrlsup c\ {\isasymcirc}\isactrlsub c\ x{\isacharprime}{\kern0pt}{\isachardoublequoteclose}\isanewline
\ \ \ \ \isacommand{by}\isamarkupfalse%
\ auto\isanewline
\ \ \isacommand{then}\isamarkupfalse%
\ \isacommand{have}\isamarkupfalse%
\ {\isachardoublequoteopen}{\isacharparenleft}{\kern0pt}characteristic{\isacharunderscore}{\kern0pt}func\ m\ {\isasymcirc}\isactrlsub c\ m{\isacharparenright}{\kern0pt}\ {\isasymcirc}\isactrlsub c\ x\ {\isacharequal}{\kern0pt}\ {\isacharparenleft}{\kern0pt}characteristic{\isacharunderscore}{\kern0pt}func\ m\ {\isasymcirc}\isactrlsub c\ m\isactrlsup c{\isacharparenright}{\kern0pt}\ {\isasymcirc}\isactrlsub c\ x{\isacharprime}{\kern0pt}{\isachardoublequoteclose}\isanewline
\ \ \ \ \isacommand{using}\isamarkupfalse%
\ assms\ comp{\isacharunderscore}{\kern0pt}associative{\isadigit{2}}\ \isacommand{by}\isamarkupfalse%
\ {\isacharparenleft}{\kern0pt}typecheck{\isacharunderscore}{\kern0pt}cfuncs{\isacharcomma}{\kern0pt}\ auto{\isacharparenright}{\kern0pt}\isanewline
\ \ \isacommand{then}\isamarkupfalse%
\ \isacommand{have}\isamarkupfalse%
\ {\isachardoublequoteopen}{\isacharparenleft}{\kern0pt}{\isasymt}\ {\isasymcirc}\isactrlsub c\ {\isasymbeta}\isactrlbsub X\isactrlesub {\isacharparenright}{\kern0pt}\ {\isasymcirc}\isactrlsub c\ x\ {\isacharequal}{\kern0pt}\ {\isacharparenleft}{\kern0pt}{\isacharparenleft}{\kern0pt}{\isasymf}\ {\isasymcirc}\isactrlsub c\ {\isasymbeta}\isactrlbsub Y\isactrlesub {\isacharparenright}{\kern0pt}\ {\isasymcirc}\isactrlsub c\ m\isactrlsup c{\isacharparenright}{\kern0pt}\ {\isasymcirc}\isactrlsub c\ x{\isacharprime}{\kern0pt}{\isachardoublequoteclose}\isanewline
\ \ \ \ \isacommand{using}\isamarkupfalse%
\ assms\ characteristic{\isacharunderscore}{\kern0pt}func{\isacharunderscore}{\kern0pt}eq\ complement{\isacharunderscore}{\kern0pt}morphism{\isacharunderscore}{\kern0pt}eq\ \isacommand{by}\isamarkupfalse%
\ auto\isanewline
\ \ \isacommand{then}\isamarkupfalse%
\ \isacommand{have}\isamarkupfalse%
\ {\isachardoublequoteopen}{\isasymt}\ {\isasymcirc}\isactrlsub c\ {\isasymbeta}\isactrlbsub X\isactrlesub \ {\isasymcirc}\isactrlsub c\ x\ {\isacharequal}{\kern0pt}\ {\isasymf}\ {\isasymcirc}\isactrlsub c\ {\isasymbeta}\isactrlbsub Y\isactrlesub \ {\isasymcirc}\isactrlsub c\ m\isactrlsup c\ {\isasymcirc}\isactrlsub c\ x{\isacharprime}{\kern0pt}{\isachardoublequoteclose}\isanewline
\ \ \ \ \isacommand{using}\isamarkupfalse%
\ assms\ comp{\isacharunderscore}{\kern0pt}associative{\isadigit{2}}\ \isacommand{by}\isamarkupfalse%
\ {\isacharparenleft}{\kern0pt}typecheck{\isacharunderscore}{\kern0pt}cfuncs{\isacharcomma}{\kern0pt}\ smt\ terminal{\isacharunderscore}{\kern0pt}func{\isacharunderscore}{\kern0pt}comp\ terminal{\isacharunderscore}{\kern0pt}func{\isacharunderscore}{\kern0pt}type{\isacharparenright}{\kern0pt}\isanewline
\ \ \isacommand{then}\isamarkupfalse%
\ \isacommand{have}\isamarkupfalse%
\ {\isachardoublequoteopen}{\isasymt}\ {\isasymcirc}\isactrlsub c\ id\ {\isasymone}\ {\isacharequal}{\kern0pt}\ {\isasymf}\ {\isasymcirc}\isactrlsub c\ id\ {\isasymone}{\isachardoublequoteclose}\isanewline
\ \ \ \ \isacommand{using}\isamarkupfalse%
\ assms\ \isacommand{by}\isamarkupfalse%
\ {\isacharparenleft}{\kern0pt}smt\ cfunc{\isacharunderscore}{\kern0pt}type{\isacharunderscore}{\kern0pt}def\ comp{\isacharunderscore}{\kern0pt}associative\ complement{\isacharunderscore}{\kern0pt}morphism{\isacharunderscore}{\kern0pt}type\ id{\isacharunderscore}{\kern0pt}type\ one{\isacharunderscore}{\kern0pt}unique{\isacharunderscore}{\kern0pt}element\ terminal{\isacharunderscore}{\kern0pt}func{\isacharunderscore}{\kern0pt}comp\ terminal{\isacharunderscore}{\kern0pt}func{\isacharunderscore}{\kern0pt}type{\isacharparenright}{\kern0pt}\isanewline
\ \ \isacommand{then}\isamarkupfalse%
\ \isacommand{have}\isamarkupfalse%
\ {\isachardoublequoteopen}{\isasymt}\ {\isacharequal}{\kern0pt}\ {\isasymf}{\isachardoublequoteclose}\isanewline
\ \ \ \ \isacommand{using}\isamarkupfalse%
\ false{\isacharunderscore}{\kern0pt}func{\isacharunderscore}{\kern0pt}type\ id{\isacharunderscore}{\kern0pt}right{\isacharunderscore}{\kern0pt}unit{\isadigit{2}}\ true{\isacharunderscore}{\kern0pt}func{\isacharunderscore}{\kern0pt}type\ \isacommand{by}\isamarkupfalse%
\ auto\isanewline
\ \ \isacommand{then}\isamarkupfalse%
\ \isacommand{show}\isamarkupfalse%
\ False\isanewline
\ \ \ \ \isacommand{using}\isamarkupfalse%
\ true{\isacharunderscore}{\kern0pt}false{\isacharunderscore}{\kern0pt}distinct\ \isacommand{by}\isamarkupfalse%
\ auto\isanewline
\isacommand{qed}\isamarkupfalse%
%
\endisatagproof
{\isafoldproof}%
%
\isadelimproof
\isanewline
%
\endisadelimproof
\isanewline
\isacommand{lemma}\isamarkupfalse%
\ set{\isacharunderscore}{\kern0pt}subtraction{\isacharunderscore}{\kern0pt}right{\isacharunderscore}{\kern0pt}iso{\isacharcolon}{\kern0pt}\isanewline
\ \ \isakeyword{assumes}\ m{\isacharunderscore}{\kern0pt}type{\isacharbrackleft}{\kern0pt}type{\isacharunderscore}{\kern0pt}rule{\isacharbrackright}{\kern0pt}{\isacharcolon}{\kern0pt}\ {\isachardoublequoteopen}m\ {\isacharcolon}{\kern0pt}\ A\ {\isasymrightarrow}\ C{\isachardoublequoteclose}\ \isakeyword{and}\ m{\isacharunderscore}{\kern0pt}mono{\isacharbrackleft}{\kern0pt}type{\isacharunderscore}{\kern0pt}rule{\isacharbrackright}{\kern0pt}{\isacharcolon}{\kern0pt}\ {\isachardoublequoteopen}monomorphism\ m{\isachardoublequoteclose}\isanewline
\ \ \isakeyword{assumes}\ i{\isacharunderscore}{\kern0pt}type{\isacharbrackleft}{\kern0pt}type{\isacharunderscore}{\kern0pt}rule{\isacharbrackright}{\kern0pt}{\isacharcolon}{\kern0pt}\ {\isachardoublequoteopen}i\ {\isacharcolon}{\kern0pt}\ B\ {\isasymrightarrow}\ A{\isachardoublequoteclose}\ \isakeyword{and}\ i{\isacharunderscore}{\kern0pt}iso{\isacharcolon}{\kern0pt}\ {\isachardoublequoteopen}isomorphism\ i{\isachardoublequoteclose}\isanewline
\ \ \isakeyword{shows}\ {\isachardoublequoteopen}C\ {\isasymsetminus}\ {\isacharparenleft}{\kern0pt}A{\isacharcomma}{\kern0pt}m{\isacharparenright}{\kern0pt}\ {\isacharequal}{\kern0pt}\ C\ {\isasymsetminus}\ {\isacharparenleft}{\kern0pt}B{\isacharcomma}{\kern0pt}\ m\ {\isasymcirc}\isactrlsub c\ i{\isacharparenright}{\kern0pt}{\isachardoublequoteclose}\isanewline
%
\isadelimproof
%
\endisadelimproof
%
\isatagproof
\isacommand{proof}\isamarkupfalse%
\ {\isacharminus}{\kern0pt}\isanewline
\ \ \isacommand{have}\isamarkupfalse%
\ mi{\isacharunderscore}{\kern0pt}mono{\isacharbrackleft}{\kern0pt}type{\isacharunderscore}{\kern0pt}rule{\isacharbrackright}{\kern0pt}{\isacharcolon}{\kern0pt}\ {\isachardoublequoteopen}monomorphism\ {\isacharparenleft}{\kern0pt}m\ {\isasymcirc}\isactrlsub c\ i{\isacharparenright}{\kern0pt}{\isachardoublequoteclose}\isanewline
\ \ \ \ \isacommand{using}\isamarkupfalse%
\ cfunc{\isacharunderscore}{\kern0pt}type{\isacharunderscore}{\kern0pt}def\ composition{\isacharunderscore}{\kern0pt}of{\isacharunderscore}{\kern0pt}monic{\isacharunderscore}{\kern0pt}pair{\isacharunderscore}{\kern0pt}is{\isacharunderscore}{\kern0pt}monic\ i{\isacharunderscore}{\kern0pt}iso\ i{\isacharunderscore}{\kern0pt}type\ iso{\isacharunderscore}{\kern0pt}imp{\isacharunderscore}{\kern0pt}epi{\isacharunderscore}{\kern0pt}and{\isacharunderscore}{\kern0pt}monic\ m{\isacharunderscore}{\kern0pt}mono\ m{\isacharunderscore}{\kern0pt}type\ \isacommand{by}\isamarkupfalse%
\ presburger\isanewline
\ \ \isacommand{obtain}\isamarkupfalse%
\ {\isasymchi}m\ \isakeyword{where}\ {\isasymchi}m{\isacharunderscore}{\kern0pt}type{\isacharbrackleft}{\kern0pt}type{\isacharunderscore}{\kern0pt}rule{\isacharbrackright}{\kern0pt}{\isacharcolon}{\kern0pt}\ {\isachardoublequoteopen}{\isasymchi}m\ {\isacharcolon}{\kern0pt}\ C\ {\isasymrightarrow}\ {\isasymOmega}{\isachardoublequoteclose}\ \isakeyword{and}\ {\isasymchi}m{\isacharunderscore}{\kern0pt}def{\isacharcolon}{\kern0pt}\ {\isachardoublequoteopen}{\isasymchi}m\ {\isacharequal}{\kern0pt}\ characteristic{\isacharunderscore}{\kern0pt}func\ m{\isachardoublequoteclose}\isanewline
\ \ \ \ \isacommand{using}\isamarkupfalse%
\ characteristic{\isacharunderscore}{\kern0pt}func{\isacharunderscore}{\kern0pt}type\ m{\isacharunderscore}{\kern0pt}mono\ m{\isacharunderscore}{\kern0pt}type\ \isacommand{by}\isamarkupfalse%
\ blast\isanewline
\ \ \isacommand{obtain}\isamarkupfalse%
\ {\isasymchi}mi\ \isakeyword{where}\ {\isasymchi}mi{\isacharunderscore}{\kern0pt}type{\isacharbrackleft}{\kern0pt}type{\isacharunderscore}{\kern0pt}rule{\isacharbrackright}{\kern0pt}{\isacharcolon}{\kern0pt}\ {\isachardoublequoteopen}{\isasymchi}mi\ {\isacharcolon}{\kern0pt}\ C\ {\isasymrightarrow}\ {\isasymOmega}{\isachardoublequoteclose}\ \isakeyword{and}\ {\isasymchi}mi{\isacharunderscore}{\kern0pt}def{\isacharcolon}{\kern0pt}\ {\isachardoublequoteopen}{\isasymchi}mi\ {\isacharequal}{\kern0pt}\ characteristic{\isacharunderscore}{\kern0pt}func\ {\isacharparenleft}{\kern0pt}m\ {\isasymcirc}\isactrlsub c\ i{\isacharparenright}{\kern0pt}{\isachardoublequoteclose}\isanewline
\ \ \ \ \isacommand{by}\isamarkupfalse%
\ {\isacharparenleft}{\kern0pt}typecheck{\isacharunderscore}{\kern0pt}cfuncs{\isacharcomma}{\kern0pt}\ simp{\isacharparenright}{\kern0pt}\isanewline
\ \ \isacommand{have}\isamarkupfalse%
\ {\isachardoublequoteopen}{\isasymAnd}\ c{\isachardot}{\kern0pt}\ c\ {\isasymin}\isactrlsub c\ C\ {\isasymLongrightarrow}\ {\isacharparenleft}{\kern0pt}{\isasymchi}m\ {\isasymcirc}\isactrlsub c\ c\ {\isacharequal}{\kern0pt}\ {\isasymt}{\isacharparenright}{\kern0pt}\ {\isacharequal}{\kern0pt}\ {\isacharparenleft}{\kern0pt}{\isasymchi}mi\ {\isasymcirc}\isactrlsub c\ c\ {\isacharequal}{\kern0pt}\ {\isasymt}{\isacharparenright}{\kern0pt}{\isachardoublequoteclose}\isanewline
\ \ \isacommand{proof}\isamarkupfalse%
\ {\isacharminus}{\kern0pt}\isanewline
\ \ \ \ \isacommand{fix}\isamarkupfalse%
\ c\isanewline
\ \ \ \ \isacommand{assume}\isamarkupfalse%
\ c{\isacharunderscore}{\kern0pt}type{\isacharbrackleft}{\kern0pt}type{\isacharunderscore}{\kern0pt}rule{\isacharbrackright}{\kern0pt}{\isacharcolon}{\kern0pt}\ {\isachardoublequoteopen}c\ {\isasymin}\isactrlsub c\ C{\isachardoublequoteclose}\isanewline
\ \ \ \ \isacommand{have}\isamarkupfalse%
\ {\isachardoublequoteopen}{\isacharparenleft}{\kern0pt}{\isasymchi}m\ {\isasymcirc}\isactrlsub c\ c\ {\isacharequal}{\kern0pt}\ {\isasymt}{\isacharparenright}{\kern0pt}\ {\isacharequal}{\kern0pt}\ {\isacharparenleft}{\kern0pt}c\ {\isasymin}\isactrlbsub C\isactrlesub \ {\isacharparenleft}{\kern0pt}A{\isacharcomma}{\kern0pt}m{\isacharparenright}{\kern0pt}{\isacharparenright}{\kern0pt}{\isachardoublequoteclose}\isanewline
\ \ \ \ \ \ \isacommand{by}\isamarkupfalse%
\ {\isacharparenleft}{\kern0pt}typecheck{\isacharunderscore}{\kern0pt}cfuncs{\isacharcomma}{\kern0pt}\ metis\ {\isasymchi}m{\isacharunderscore}{\kern0pt}def\ m{\isacharunderscore}{\kern0pt}mono\ not{\isacharunderscore}{\kern0pt}rel{\isacharunderscore}{\kern0pt}mem{\isacharunderscore}{\kern0pt}char{\isacharunderscore}{\kern0pt}func{\isacharunderscore}{\kern0pt}false\ rel{\isacharunderscore}{\kern0pt}mem{\isacharunderscore}{\kern0pt}char{\isacharunderscore}{\kern0pt}func{\isacharunderscore}{\kern0pt}true\ true{\isacharunderscore}{\kern0pt}false{\isacharunderscore}{\kern0pt}distinct{\isacharparenright}{\kern0pt}\isanewline
\ \ \ \ \isacommand{also}\isamarkupfalse%
\ \isacommand{have}\isamarkupfalse%
\ {\isachardoublequoteopen}{\isachardot}{\kern0pt}{\isachardot}{\kern0pt}{\isachardot}{\kern0pt}\ {\isacharequal}{\kern0pt}\ {\isacharparenleft}{\kern0pt}{\isasymexists}\ a{\isachardot}{\kern0pt}\ a\ {\isasymin}\isactrlsub c\ A\ {\isasymand}\ c\ {\isacharequal}{\kern0pt}\ m\ {\isasymcirc}\isactrlsub c\ a{\isacharparenright}{\kern0pt}{\isachardoublequoteclose}\isanewline
\ \ \ \ \ \ \isacommand{using}\isamarkupfalse%
\ cfunc{\isacharunderscore}{\kern0pt}type{\isacharunderscore}{\kern0pt}def\ factors{\isacharunderscore}{\kern0pt}through{\isacharunderscore}{\kern0pt}def\ m{\isacharunderscore}{\kern0pt}mono\ relative{\isacharunderscore}{\kern0pt}member{\isacharunderscore}{\kern0pt}def{\isadigit{2}}\ \isacommand{by}\isamarkupfalse%
\ {\isacharparenleft}{\kern0pt}typecheck{\isacharunderscore}{\kern0pt}cfuncs{\isacharcomma}{\kern0pt}\ auto{\isacharparenright}{\kern0pt}\isanewline
\ \ \ \ \isacommand{also}\isamarkupfalse%
\ \isacommand{have}\isamarkupfalse%
\ {\isachardoublequoteopen}{\isachardot}{\kern0pt}{\isachardot}{\kern0pt}{\isachardot}{\kern0pt}\ {\isacharequal}{\kern0pt}\ {\isacharparenleft}{\kern0pt}{\isasymexists}\ b{\isachardot}{\kern0pt}\ b\ {\isasymin}\isactrlsub c\ B\ {\isasymand}\ c\ {\isacharequal}{\kern0pt}\ m\ {\isasymcirc}\isactrlsub c\ i\ {\isasymcirc}\isactrlsub c\ b{\isacharparenright}{\kern0pt}{\isachardoublequoteclose}\isanewline
\ \ \ \ \ \ \isacommand{by}\isamarkupfalse%
\ {\isacharparenleft}{\kern0pt}typecheck{\isacharunderscore}{\kern0pt}cfuncs{\isacharcomma}{\kern0pt}\ smt\ {\isacharparenleft}{\kern0pt}z{\isadigit{3}}{\isacharparenright}{\kern0pt}\ cfunc{\isacharunderscore}{\kern0pt}type{\isacharunderscore}{\kern0pt}def\ comp{\isacharunderscore}{\kern0pt}type\ epi{\isacharunderscore}{\kern0pt}is{\isacharunderscore}{\kern0pt}surj\ i{\isacharunderscore}{\kern0pt}iso\ iso{\isacharunderscore}{\kern0pt}imp{\isacharunderscore}{\kern0pt}epi{\isacharunderscore}{\kern0pt}and{\isacharunderscore}{\kern0pt}monic\ surjective{\isacharunderscore}{\kern0pt}def{\isacharparenright}{\kern0pt}\isanewline
\ \ \ \ \isacommand{also}\isamarkupfalse%
\ \isacommand{have}\isamarkupfalse%
\ {\isachardoublequoteopen}{\isachardot}{\kern0pt}{\isachardot}{\kern0pt}{\isachardot}{\kern0pt}\ {\isacharequal}{\kern0pt}\ {\isacharparenleft}{\kern0pt}c\ {\isasymin}\isactrlbsub C\isactrlesub \ {\isacharparenleft}{\kern0pt}B{\isacharcomma}{\kern0pt}m\ {\isasymcirc}\isactrlsub c\ i{\isacharparenright}{\kern0pt}{\isacharparenright}{\kern0pt}{\isachardoublequoteclose}\isanewline
\ \ \ \ \ \ \isacommand{using}\isamarkupfalse%
\ cfunc{\isacharunderscore}{\kern0pt}type{\isacharunderscore}{\kern0pt}def\ comp{\isacharunderscore}{\kern0pt}associative{\isadigit{2}}\ composition{\isacharunderscore}{\kern0pt}of{\isacharunderscore}{\kern0pt}monic{\isacharunderscore}{\kern0pt}pair{\isacharunderscore}{\kern0pt}is{\isacharunderscore}{\kern0pt}monic\ factors{\isacharunderscore}{\kern0pt}through{\isacharunderscore}{\kern0pt}def{\isadigit{2}}\ i{\isacharunderscore}{\kern0pt}iso\ iso{\isacharunderscore}{\kern0pt}imp{\isacharunderscore}{\kern0pt}epi{\isacharunderscore}{\kern0pt}and{\isacharunderscore}{\kern0pt}monic\ m{\isacharunderscore}{\kern0pt}mono\ relative{\isacharunderscore}{\kern0pt}member{\isacharunderscore}{\kern0pt}def{\isadigit{2}}\isanewline
\ \ \ \ \ \ \isacommand{by}\isamarkupfalse%
\ {\isacharparenleft}{\kern0pt}typecheck{\isacharunderscore}{\kern0pt}cfuncs{\isacharcomma}{\kern0pt}\ auto{\isacharparenright}{\kern0pt}\isanewline
\ \ \ \ \isacommand{also}\isamarkupfalse%
\ \isacommand{have}\isamarkupfalse%
\ {\isachardoublequoteopen}{\isachardot}{\kern0pt}{\isachardot}{\kern0pt}{\isachardot}{\kern0pt}\ {\isacharequal}{\kern0pt}\ {\isacharparenleft}{\kern0pt}{\isasymchi}mi\ {\isasymcirc}\isactrlsub c\ c\ {\isacharequal}{\kern0pt}\ {\isasymt}{\isacharparenright}{\kern0pt}{\isachardoublequoteclose}\isanewline
\ \ \ \ \ \ \isacommand{by}\isamarkupfalse%
\ {\isacharparenleft}{\kern0pt}typecheck{\isacharunderscore}{\kern0pt}cfuncs{\isacharcomma}{\kern0pt}\ metis\ {\isasymchi}mi{\isacharunderscore}{\kern0pt}def\ mi{\isacharunderscore}{\kern0pt}mono\ not{\isacharunderscore}{\kern0pt}rel{\isacharunderscore}{\kern0pt}mem{\isacharunderscore}{\kern0pt}char{\isacharunderscore}{\kern0pt}func{\isacharunderscore}{\kern0pt}false\ rel{\isacharunderscore}{\kern0pt}mem{\isacharunderscore}{\kern0pt}char{\isacharunderscore}{\kern0pt}func{\isacharunderscore}{\kern0pt}true\ true{\isacharunderscore}{\kern0pt}false{\isacharunderscore}{\kern0pt}distinct{\isacharparenright}{\kern0pt}\isanewline
\ \ \ \ \isacommand{then}\isamarkupfalse%
\ \isacommand{show}\isamarkupfalse%
\ {\isachardoublequoteopen}{\isacharparenleft}{\kern0pt}{\isasymchi}m\ {\isasymcirc}\isactrlsub c\ c\ {\isacharequal}{\kern0pt}\ {\isasymt}{\isacharparenright}{\kern0pt}\ {\isacharequal}{\kern0pt}\ {\isacharparenleft}{\kern0pt}{\isasymchi}mi\ {\isasymcirc}\isactrlsub c\ c\ {\isacharequal}{\kern0pt}\ {\isasymt}{\isacharparenright}{\kern0pt}{\isachardoublequoteclose}\isanewline
\ \ \ \ \ \ \isacommand{using}\isamarkupfalse%
\ calculation\ \isacommand{by}\isamarkupfalse%
\ auto\isanewline
\ \ \isacommand{qed}\isamarkupfalse%
\isanewline
\ \ \isacommand{then}\isamarkupfalse%
\ \isacommand{have}\isamarkupfalse%
\ {\isachardoublequoteopen}{\isasymchi}m\ {\isacharequal}{\kern0pt}\ {\isasymchi}mi{\isachardoublequoteclose}\isanewline
\ \ \ \ \isacommand{by}\isamarkupfalse%
\ {\isacharparenleft}{\kern0pt}typecheck{\isacharunderscore}{\kern0pt}cfuncs{\isacharcomma}{\kern0pt}\ smt\ {\isacharparenleft}{\kern0pt}verit{\isacharcomma}{\kern0pt}\ best{\isacharparenright}{\kern0pt}\ comp{\isacharunderscore}{\kern0pt}type\ one{\isacharunderscore}{\kern0pt}separator\ true{\isacharunderscore}{\kern0pt}false{\isacharunderscore}{\kern0pt}only{\isacharunderscore}{\kern0pt}truth{\isacharunderscore}{\kern0pt}values{\isacharparenright}{\kern0pt}\ \isanewline
\ \ \isacommand{then}\isamarkupfalse%
\ \isacommand{show}\isamarkupfalse%
\ {\isachardoublequoteopen}C\ {\isasymsetminus}\ {\isacharparenleft}{\kern0pt}A{\isacharcomma}{\kern0pt}m{\isacharparenright}{\kern0pt}\ {\isacharequal}{\kern0pt}\ C\ {\isasymsetminus}\ {\isacharparenleft}{\kern0pt}B{\isacharcomma}{\kern0pt}\ m\ {\isasymcirc}\isactrlsub c\ i{\isacharparenright}{\kern0pt}{\isachardoublequoteclose}\isanewline
\ \ \ \ \isacommand{using}\isamarkupfalse%
\ {\isasymchi}m{\isacharunderscore}{\kern0pt}def\ {\isasymchi}mi{\isacharunderscore}{\kern0pt}def\ isomorphic{\isacharunderscore}{\kern0pt}is{\isacharunderscore}{\kern0pt}reflexive\ set{\isacharunderscore}{\kern0pt}subtraction{\isacharunderscore}{\kern0pt}def\ \isacommand{by}\isamarkupfalse%
\ auto\isanewline
\isacommand{qed}\isamarkupfalse%
%
\endisatagproof
{\isafoldproof}%
%
\isadelimproof
\isanewline
%
\endisadelimproof
\isanewline
\isacommand{lemma}\isamarkupfalse%
\ set{\isacharunderscore}{\kern0pt}subtraction{\isacharunderscore}{\kern0pt}left{\isacharunderscore}{\kern0pt}iso{\isacharcolon}{\kern0pt}\isanewline
\ \ \isakeyword{assumes}\ m{\isacharunderscore}{\kern0pt}type{\isacharbrackleft}{\kern0pt}type{\isacharunderscore}{\kern0pt}rule{\isacharbrackright}{\kern0pt}{\isacharcolon}{\kern0pt}\ {\isachardoublequoteopen}m\ {\isacharcolon}{\kern0pt}\ C\ {\isasymrightarrow}\ A{\isachardoublequoteclose}\ \isakeyword{and}\ m{\isacharunderscore}{\kern0pt}mono{\isacharbrackleft}{\kern0pt}type{\isacharunderscore}{\kern0pt}rule{\isacharbrackright}{\kern0pt}{\isacharcolon}{\kern0pt}\ {\isachardoublequoteopen}monomorphism\ m{\isachardoublequoteclose}\isanewline
\ \ \isakeyword{assumes}\ i{\isacharunderscore}{\kern0pt}type{\isacharbrackleft}{\kern0pt}type{\isacharunderscore}{\kern0pt}rule{\isacharbrackright}{\kern0pt}{\isacharcolon}{\kern0pt}\ {\isachardoublequoteopen}i\ {\isacharcolon}{\kern0pt}\ A\ {\isasymrightarrow}\ B{\isachardoublequoteclose}\ \isakeyword{and}\ i{\isacharunderscore}{\kern0pt}iso{\isacharcolon}{\kern0pt}\ {\isachardoublequoteopen}isomorphism\ i{\isachardoublequoteclose}\isanewline
\ \ \isakeyword{shows}\ {\isachardoublequoteopen}A\ {\isasymsetminus}\ {\isacharparenleft}{\kern0pt}C{\isacharcomma}{\kern0pt}m{\isacharparenright}{\kern0pt}\ {\isasymcong}\ B\ {\isasymsetminus}\ {\isacharparenleft}{\kern0pt}C{\isacharcomma}{\kern0pt}\ i\ {\isasymcirc}\isactrlsub c\ m{\isacharparenright}{\kern0pt}{\isachardoublequoteclose}\isanewline
%
\isadelimproof
%
\endisadelimproof
%
\isatagproof
\isacommand{proof}\isamarkupfalse%
\ {\isacharminus}{\kern0pt}\isanewline
\ \ \isacommand{have}\isamarkupfalse%
\ im{\isacharunderscore}{\kern0pt}mono{\isacharbrackleft}{\kern0pt}type{\isacharunderscore}{\kern0pt}rule{\isacharbrackright}{\kern0pt}{\isacharcolon}{\kern0pt}\ {\isachardoublequoteopen}monomorphism\ {\isacharparenleft}{\kern0pt}i\ {\isasymcirc}\isactrlsub c\ m{\isacharparenright}{\kern0pt}{\isachardoublequoteclose}\isanewline
\ \ \ \ \isacommand{using}\isamarkupfalse%
\ cfunc{\isacharunderscore}{\kern0pt}type{\isacharunderscore}{\kern0pt}def\ composition{\isacharunderscore}{\kern0pt}of{\isacharunderscore}{\kern0pt}monic{\isacharunderscore}{\kern0pt}pair{\isacharunderscore}{\kern0pt}is{\isacharunderscore}{\kern0pt}monic\ i{\isacharunderscore}{\kern0pt}iso\ i{\isacharunderscore}{\kern0pt}type\ iso{\isacharunderscore}{\kern0pt}imp{\isacharunderscore}{\kern0pt}epi{\isacharunderscore}{\kern0pt}and{\isacharunderscore}{\kern0pt}monic\ m{\isacharunderscore}{\kern0pt}mono\ m{\isacharunderscore}{\kern0pt}type\ \isacommand{by}\isamarkupfalse%
\ presburger\isanewline
\ \ \isacommand{obtain}\isamarkupfalse%
\ {\isasymchi}m\ \isakeyword{where}\ {\isasymchi}m{\isacharunderscore}{\kern0pt}type{\isacharbrackleft}{\kern0pt}type{\isacharunderscore}{\kern0pt}rule{\isacharbrackright}{\kern0pt}{\isacharcolon}{\kern0pt}\ {\isachardoublequoteopen}{\isasymchi}m\ {\isacharcolon}{\kern0pt}\ A\ {\isasymrightarrow}\ {\isasymOmega}{\isachardoublequoteclose}\ \isakeyword{and}\ {\isasymchi}m{\isacharunderscore}{\kern0pt}def{\isacharcolon}{\kern0pt}\ {\isachardoublequoteopen}{\isasymchi}m\ {\isacharequal}{\kern0pt}\ characteristic{\isacharunderscore}{\kern0pt}func\ m{\isachardoublequoteclose}\isanewline
\ \ \ \ \isacommand{using}\isamarkupfalse%
\ characteristic{\isacharunderscore}{\kern0pt}func{\isacharunderscore}{\kern0pt}type\ m{\isacharunderscore}{\kern0pt}mono\ m{\isacharunderscore}{\kern0pt}type\ \isacommand{by}\isamarkupfalse%
\ blast\isanewline
\ \ \isacommand{obtain}\isamarkupfalse%
\ {\isasymchi}im\ \isakeyword{where}\ {\isasymchi}im{\isacharunderscore}{\kern0pt}type{\isacharbrackleft}{\kern0pt}type{\isacharunderscore}{\kern0pt}rule{\isacharbrackright}{\kern0pt}{\isacharcolon}{\kern0pt}\ {\isachardoublequoteopen}{\isasymchi}im\ {\isacharcolon}{\kern0pt}\ B\ {\isasymrightarrow}\ {\isasymOmega}{\isachardoublequoteclose}\ \isakeyword{and}\ {\isasymchi}im{\isacharunderscore}{\kern0pt}def{\isacharcolon}{\kern0pt}\ {\isachardoublequoteopen}{\isasymchi}im\ {\isacharequal}{\kern0pt}\ characteristic{\isacharunderscore}{\kern0pt}func\ {\isacharparenleft}{\kern0pt}i\ {\isasymcirc}\isactrlsub c\ m{\isacharparenright}{\kern0pt}{\isachardoublequoteclose}\isanewline
\ \ \ \ \isacommand{by}\isamarkupfalse%
\ {\isacharparenleft}{\kern0pt}typecheck{\isacharunderscore}{\kern0pt}cfuncs{\isacharcomma}{\kern0pt}\ simp{\isacharparenright}{\kern0pt}\isanewline
\ \ \isacommand{have}\isamarkupfalse%
\ {\isasymchi}im{\isacharunderscore}{\kern0pt}pullback{\isacharcolon}{\kern0pt}\ {\isachardoublequoteopen}is{\isacharunderscore}{\kern0pt}pullback\ C\ {\isasymone}\ B\ {\isasymOmega}\ {\isacharparenleft}{\kern0pt}{\isasymbeta}\isactrlbsub C\isactrlesub {\isacharparenright}{\kern0pt}\ {\isasymt}\ {\isacharparenleft}{\kern0pt}i\ {\isasymcirc}\isactrlsub c\ m{\isacharparenright}{\kern0pt}\ {\isasymchi}im{\isachardoublequoteclose}\isanewline
\ \ \ \ \isacommand{using}\isamarkupfalse%
\ {\isasymchi}im{\isacharunderscore}{\kern0pt}def\ characteristic{\isacharunderscore}{\kern0pt}func{\isacharunderscore}{\kern0pt}is{\isacharunderscore}{\kern0pt}pullback\ comp{\isacharunderscore}{\kern0pt}type\ i{\isacharunderscore}{\kern0pt}type\ im{\isacharunderscore}{\kern0pt}mono\ m{\isacharunderscore}{\kern0pt}type\ \isacommand{by}\isamarkupfalse%
\ blast\isanewline
\ \ \isacommand{have}\isamarkupfalse%
\ {\isachardoublequoteopen}is{\isacharunderscore}{\kern0pt}pullback\ C\ {\isasymone}\ A\ {\isasymOmega}\ {\isacharparenleft}{\kern0pt}{\isasymbeta}\isactrlbsub C\isactrlesub {\isacharparenright}{\kern0pt}\ {\isasymt}\ m\ {\isacharparenleft}{\kern0pt}{\isasymchi}im\ {\isasymcirc}\isactrlsub c\ i{\isacharparenright}{\kern0pt}{\isachardoublequoteclose}\isanewline
\ \ \isacommand{proof}\isamarkupfalse%
\ {\isacharparenleft}{\kern0pt}unfold\ is{\isacharunderscore}{\kern0pt}pullback{\isacharunderscore}{\kern0pt}def{\isacharcomma}{\kern0pt}\ typecheck{\isacharunderscore}{\kern0pt}cfuncs{\isacharcomma}{\kern0pt}\ safe{\isacharparenright}{\kern0pt}\isanewline
\ \ \ \ \isacommand{show}\isamarkupfalse%
\ {\isachardoublequoteopen}{\isasymt}\ {\isasymcirc}\isactrlsub c\ {\isasymbeta}\isactrlbsub C\isactrlesub \ {\isacharequal}{\kern0pt}\ {\isacharparenleft}{\kern0pt}{\isasymchi}im\ {\isasymcirc}\isactrlsub c\ i{\isacharparenright}{\kern0pt}\ {\isasymcirc}\isactrlsub c\ m{\isachardoublequoteclose}\isanewline
\ \ \ \ \ \ \isacommand{by}\isamarkupfalse%
\ {\isacharparenleft}{\kern0pt}typecheck{\isacharunderscore}{\kern0pt}cfuncs{\isacharcomma}{\kern0pt}\ etcs{\isacharunderscore}{\kern0pt}assocr{\isacharcomma}{\kern0pt}\ metis\ {\isasymchi}im{\isacharunderscore}{\kern0pt}def\ characteristic{\isacharunderscore}{\kern0pt}func{\isacharunderscore}{\kern0pt}eq\ comp{\isacharunderscore}{\kern0pt}type\ im{\isacharunderscore}{\kern0pt}mono{\isacharparenright}{\kern0pt}\isanewline
\ \ \isacommand{next}\isamarkupfalse%
\isanewline
\ \ \ \ \isacommand{fix}\isamarkupfalse%
\ Z\ k\ h\isanewline
\ \ \ \ \isacommand{assume}\isamarkupfalse%
\ k{\isacharunderscore}{\kern0pt}type{\isacharbrackleft}{\kern0pt}type{\isacharunderscore}{\kern0pt}rule{\isacharbrackright}{\kern0pt}{\isacharcolon}{\kern0pt}\ {\isachardoublequoteopen}k\ {\isacharcolon}{\kern0pt}\ Z\ {\isasymrightarrow}\ {\isasymone}{\isachardoublequoteclose}\ \isakeyword{and}\ h{\isacharunderscore}{\kern0pt}type{\isacharbrackleft}{\kern0pt}type{\isacharunderscore}{\kern0pt}rule{\isacharbrackright}{\kern0pt}{\isacharcolon}{\kern0pt}\ {\isachardoublequoteopen}h\ {\isacharcolon}{\kern0pt}\ Z\ {\isasymrightarrow}\ A{\isachardoublequoteclose}\isanewline
\ \ \ \ \isacommand{assume}\isamarkupfalse%
\ eq{\isacharcolon}{\kern0pt}\ {\isachardoublequoteopen}{\isasymt}\ {\isasymcirc}\isactrlsub c\ k\ {\isacharequal}{\kern0pt}\ {\isacharparenleft}{\kern0pt}{\isasymchi}im\ {\isasymcirc}\isactrlsub c\ i{\isacharparenright}{\kern0pt}\ {\isasymcirc}\isactrlsub c\ h{\isachardoublequoteclose}\isanewline
\ \ \ \ \isacommand{then}\isamarkupfalse%
\ \isacommand{obtain}\isamarkupfalse%
\ j\ \isakeyword{where}\ j{\isacharunderscore}{\kern0pt}type{\isacharbrackleft}{\kern0pt}type{\isacharunderscore}{\kern0pt}rule{\isacharbrackright}{\kern0pt}{\isacharcolon}{\kern0pt}\ {\isachardoublequoteopen}j\ {\isacharcolon}{\kern0pt}\ Z\ {\isasymrightarrow}\ C{\isachardoublequoteclose}\ \isakeyword{and}\ j{\isacharunderscore}{\kern0pt}def{\isacharcolon}{\kern0pt}\ {\isachardoublequoteopen}i\ {\isasymcirc}\isactrlsub c\ h\ {\isacharequal}{\kern0pt}\ {\isacharparenleft}{\kern0pt}i\ {\isasymcirc}\isactrlsub c\ m{\isacharparenright}{\kern0pt}\ {\isasymcirc}\isactrlsub c\ j{\isachardoublequoteclose}\isanewline
\ \ \ \ \ \ \isacommand{using}\isamarkupfalse%
\ {\isasymchi}im{\isacharunderscore}{\kern0pt}pullback\ \isacommand{unfolding}\isamarkupfalse%
\ is{\isacharunderscore}{\kern0pt}pullback{\isacharunderscore}{\kern0pt}def\ \isacommand{by}\isamarkupfalse%
\ {\isacharparenleft}{\kern0pt}typecheck{\isacharunderscore}{\kern0pt}cfuncs{\isacharcomma}{\kern0pt}\ smt\ {\isacharparenleft}{\kern0pt}verit{\isacharcomma}{\kern0pt}\ ccfv{\isacharunderscore}{\kern0pt}threshold{\isacharparenright}{\kern0pt}\ comp{\isacharunderscore}{\kern0pt}associative{\isadigit{2}}\ k{\isacharunderscore}{\kern0pt}type{\isacharparenright}{\kern0pt}\isanewline
\ \ \ \ \isacommand{then}\isamarkupfalse%
\ \isacommand{show}\isamarkupfalse%
\ {\isachardoublequoteopen}{\isasymexists}j{\isachardot}{\kern0pt}\ j\ {\isacharcolon}{\kern0pt}\ Z\ {\isasymrightarrow}\ C\ {\isasymand}\ {\isasymbeta}\isactrlbsub C\isactrlesub \ {\isasymcirc}\isactrlsub c\ j\ {\isacharequal}{\kern0pt}\ k\ {\isasymand}\ m\ {\isasymcirc}\isactrlsub c\ j\ {\isacharequal}{\kern0pt}\ h{\isachardoublequoteclose}\isanewline
\ \ \ \ \ \ \isacommand{by}\isamarkupfalse%
\ {\isacharparenleft}{\kern0pt}rule{\isacharunderscore}{\kern0pt}tac\ x{\isacharequal}{\kern0pt}{\isachardoublequoteopen}j{\isachardoublequoteclose}\ \isakeyword{in}\ exI{\isacharcomma}{\kern0pt}\ typecheck{\isacharunderscore}{\kern0pt}cfuncs{\isacharcomma}{\kern0pt}\ smt\ comp{\isacharunderscore}{\kern0pt}associative{\isadigit{2}}\ i{\isacharunderscore}{\kern0pt}iso\ iso{\isacharunderscore}{\kern0pt}imp{\isacharunderscore}{\kern0pt}epi{\isacharunderscore}{\kern0pt}and{\isacharunderscore}{\kern0pt}monic\ monomorphism{\isacharunderscore}{\kern0pt}def{\isadigit{2}}\ terminal{\isacharunderscore}{\kern0pt}func{\isacharunderscore}{\kern0pt}unique{\isacharparenright}{\kern0pt}\isanewline
\ \ \isacommand{next}\isamarkupfalse%
\isanewline
\ \ \ \ \isacommand{fix}\isamarkupfalse%
\ Z\ j\ y\isanewline
\ \ \ \ \isacommand{assume}\isamarkupfalse%
\ j{\isacharunderscore}{\kern0pt}type{\isacharbrackleft}{\kern0pt}type{\isacharunderscore}{\kern0pt}rule{\isacharbrackright}{\kern0pt}{\isacharcolon}{\kern0pt}\ {\isachardoublequoteopen}j\ {\isacharcolon}{\kern0pt}\ Z\ {\isasymrightarrow}\ C{\isachardoublequoteclose}\ \isakeyword{and}\ y{\isacharunderscore}{\kern0pt}type{\isacharbrackleft}{\kern0pt}type{\isacharunderscore}{\kern0pt}rule{\isacharbrackright}{\kern0pt}{\isacharcolon}{\kern0pt}\ {\isachardoublequoteopen}y\ {\isacharcolon}{\kern0pt}\ Z\ {\isasymrightarrow}\ C{\isachardoublequoteclose}\isanewline
\ \ \ \ \isacommand{assume}\isamarkupfalse%
\ {\isachardoublequoteopen}{\isasymt}\ {\isasymcirc}\isactrlsub c\ {\isasymbeta}\isactrlbsub C\isactrlesub \ {\isasymcirc}\isactrlsub c\ j\ {\isacharequal}{\kern0pt}\ {\isacharparenleft}{\kern0pt}{\isasymchi}im\ {\isasymcirc}\isactrlsub c\ i{\isacharparenright}{\kern0pt}\ {\isasymcirc}\isactrlsub c\ m\ {\isasymcirc}\isactrlsub c\ j{\isachardoublequoteclose}\ {\isachardoublequoteopen}{\isasymbeta}\isactrlbsub C\isactrlesub \ {\isasymcirc}\isactrlsub c\ y\ {\isacharequal}{\kern0pt}\ {\isasymbeta}\isactrlbsub C\isactrlesub \ {\isasymcirc}\isactrlsub c\ j{\isachardoublequoteclose}\ {\isachardoublequoteopen}m\ {\isasymcirc}\isactrlsub c\ y\ {\isacharequal}{\kern0pt}\ m\ {\isasymcirc}\isactrlsub c\ j{\isachardoublequoteclose}\isanewline
\ \ \ \ \isacommand{then}\isamarkupfalse%
\ \isacommand{show}\isamarkupfalse%
\ {\isachardoublequoteopen}j\ {\isacharequal}{\kern0pt}\ y{\isachardoublequoteclose}\isanewline
\ \ \ \ \ \ \isacommand{using}\isamarkupfalse%
\ m{\isacharunderscore}{\kern0pt}mono\ monomorphism{\isacharunderscore}{\kern0pt}def{\isadigit{2}}\ \isacommand{by}\isamarkupfalse%
\ {\isacharparenleft}{\kern0pt}typecheck{\isacharunderscore}{\kern0pt}cfuncs{\isacharunderscore}{\kern0pt}prems{\isacharcomma}{\kern0pt}\ blast{\isacharparenright}{\kern0pt}\isanewline
\ \ \isacommand{qed}\isamarkupfalse%
\isanewline
\ \ \isacommand{then}\isamarkupfalse%
\ \isacommand{have}\isamarkupfalse%
\ {\isasymchi}im{\isacharunderscore}{\kern0pt}i{\isacharunderscore}{\kern0pt}eq{\isacharunderscore}{\kern0pt}{\isasymchi}m{\isacharcolon}{\kern0pt}\ {\isachardoublequoteopen}{\isasymchi}im\ {\isasymcirc}\isactrlsub c\ i\ {\isacharequal}{\kern0pt}\ {\isasymchi}m{\isachardoublequoteclose}\isanewline
\ \ \ \ \isacommand{using}\isamarkupfalse%
\ {\isasymchi}m{\isacharunderscore}{\kern0pt}def\ characteristic{\isacharunderscore}{\kern0pt}func{\isacharunderscore}{\kern0pt}is{\isacharunderscore}{\kern0pt}pullback\ characteristic{\isacharunderscore}{\kern0pt}function{\isacharunderscore}{\kern0pt}exists\ m{\isacharunderscore}{\kern0pt}mono\ m{\isacharunderscore}{\kern0pt}type\ \isacommand{by}\isamarkupfalse%
\ blast\isanewline
\ \ \isacommand{then}\isamarkupfalse%
\ \isacommand{have}\isamarkupfalse%
\ {\isachardoublequoteopen}{\isasymchi}im\ {\isasymcirc}\isactrlsub c\ {\isacharparenleft}{\kern0pt}i\ {\isasymcirc}\isactrlsub c\ m\isactrlsup c{\isacharparenright}{\kern0pt}\ {\isacharequal}{\kern0pt}\ {\isasymf}\ {\isasymcirc}\isactrlsub c\ {\isasymbeta}\isactrlbsub B\isactrlesub \ {\isasymcirc}\isactrlsub c\ {\isacharparenleft}{\kern0pt}i\ {\isasymcirc}\isactrlsub c\ m\isactrlsup c{\isacharparenright}{\kern0pt}{\isachardoublequoteclose}\isanewline
\ \ \ \ \isacommand{by}\isamarkupfalse%
\ {\isacharparenleft}{\kern0pt}etcs{\isacharunderscore}{\kern0pt}assocl{\isacharcomma}{\kern0pt}\ typecheck{\isacharunderscore}{\kern0pt}cfuncs{\isacharcomma}{\kern0pt}\ smt\ {\isacharparenleft}{\kern0pt}verit{\isacharcomma}{\kern0pt}\ best{\isacharparenright}{\kern0pt}\ {\isasymchi}m{\isacharunderscore}{\kern0pt}def\ comp{\isacharunderscore}{\kern0pt}associative{\isadigit{2}}\ complement{\isacharunderscore}{\kern0pt}morphism{\isacharunderscore}{\kern0pt}eq\ m{\isacharunderscore}{\kern0pt}mono\ terminal{\isacharunderscore}{\kern0pt}func{\isacharunderscore}{\kern0pt}comp{\isacharparenright}{\kern0pt}\isanewline
\ \ \isacommand{then}\isamarkupfalse%
\ \isacommand{obtain}\isamarkupfalse%
\ i{\isacharprime}{\kern0pt}\ \isakeyword{where}\ i{\isacharprime}{\kern0pt}{\isacharunderscore}{\kern0pt}type{\isacharbrackleft}{\kern0pt}type{\isacharunderscore}{\kern0pt}rule{\isacharbrackright}{\kern0pt}{\isacharcolon}{\kern0pt}\ {\isachardoublequoteopen}i{\isacharprime}{\kern0pt}\ {\isacharcolon}{\kern0pt}\ A\ {\isasymsetminus}\ {\isacharparenleft}{\kern0pt}C{\isacharcomma}{\kern0pt}\ m{\isacharparenright}{\kern0pt}\ {\isasymrightarrow}\ B\ {\isasymsetminus}\ {\isacharparenleft}{\kern0pt}C{\isacharcomma}{\kern0pt}\ i\ {\isasymcirc}\isactrlsub c\ m{\isacharparenright}{\kern0pt}{\isachardoublequoteclose}\ \isakeyword{and}\ i{\isacharprime}{\kern0pt}{\isacharunderscore}{\kern0pt}def{\isacharcolon}{\kern0pt}\ {\isachardoublequoteopen}i\ {\isasymcirc}\isactrlsub c\ m\isactrlsup c\ {\isacharequal}{\kern0pt}\ {\isacharparenleft}{\kern0pt}i\ {\isasymcirc}\isactrlsub c\ m{\isacharparenright}{\kern0pt}\isactrlsup c\ {\isasymcirc}\isactrlsub c\ i{\isacharprime}{\kern0pt}{\isachardoublequoteclose}\isanewline
\ \ \ \ \isacommand{using}\isamarkupfalse%
\ complement{\isacharunderscore}{\kern0pt}morphism{\isacharunderscore}{\kern0pt}equalizer{\isacharbrackleft}{\kern0pt}\isakeyword{where}\ m{\isacharequal}{\kern0pt}{\isachardoublequoteopen}i\ {\isasymcirc}\isactrlsub c\ m{\isachardoublequoteclose}{\isacharcomma}{\kern0pt}\ \isakeyword{where}\ X{\isacharequal}{\kern0pt}C{\isacharcomma}{\kern0pt}\ \isakeyword{where}\ Y{\isacharequal}{\kern0pt}B{\isacharbrackright}{\kern0pt}\ \isacommand{unfolding}\isamarkupfalse%
\ equalizer{\isacharunderscore}{\kern0pt}def\isanewline
\ \ \ \ \isacommand{by}\isamarkupfalse%
\ {\isacharparenleft}{\kern0pt}{\isacharminus}{\kern0pt}{\isacharcomma}{\kern0pt}\ typecheck{\isacharunderscore}{\kern0pt}cfuncs{\isacharcomma}{\kern0pt}\ smt\ {\isasymchi}im{\isacharunderscore}{\kern0pt}def\ cfunc{\isacharunderscore}{\kern0pt}type{\isacharunderscore}{\kern0pt}def\ comp{\isacharunderscore}{\kern0pt}associative{\isadigit{2}}\ im{\isacharunderscore}{\kern0pt}mono{\isacharparenright}{\kern0pt}\isanewline
\isanewline
\ \ \isacommand{have}\isamarkupfalse%
\ {\isachardoublequoteopen}{\isasymchi}m\ {\isasymcirc}\isactrlsub c\ {\isacharparenleft}{\kern0pt}i\isactrlbold {\isasyminverse}\ {\isasymcirc}\isactrlsub c\ {\isacharparenleft}{\kern0pt}i\ {\isasymcirc}\isactrlsub c\ m{\isacharparenright}{\kern0pt}\isactrlsup c{\isacharparenright}{\kern0pt}\ {\isacharequal}{\kern0pt}\ {\isasymf}\ {\isasymcirc}\isactrlsub c\ {\isasymbeta}\isactrlbsub A\isactrlesub \ {\isasymcirc}\isactrlsub c\ {\isacharparenleft}{\kern0pt}i\isactrlbold {\isasyminverse}\ {\isasymcirc}\isactrlsub c\ {\isacharparenleft}{\kern0pt}i\ {\isasymcirc}\isactrlsub c\ m{\isacharparenright}{\kern0pt}\isactrlsup c{\isacharparenright}{\kern0pt}{\isachardoublequoteclose}\isanewline
\ \ \isacommand{proof}\isamarkupfalse%
\ {\isacharminus}{\kern0pt}\isanewline
\ \ \ \ \isacommand{have}\isamarkupfalse%
\ {\isachardoublequoteopen}{\isasymchi}m\ {\isasymcirc}\isactrlsub c\ {\isacharparenleft}{\kern0pt}i\isactrlbold {\isasyminverse}\ {\isasymcirc}\isactrlsub c\ {\isacharparenleft}{\kern0pt}i\ {\isasymcirc}\isactrlsub c\ m{\isacharparenright}{\kern0pt}\isactrlsup c{\isacharparenright}{\kern0pt}\ {\isacharequal}{\kern0pt}\ {\isasymchi}im\ {\isasymcirc}\isactrlsub c\ {\isacharparenleft}{\kern0pt}i\ {\isasymcirc}\isactrlsub c\ i\isactrlbold {\isasyminverse}{\isacharparenright}{\kern0pt}\ {\isasymcirc}\isactrlsub c\ {\isacharparenleft}{\kern0pt}i\ {\isasymcirc}\isactrlsub c\ m{\isacharparenright}{\kern0pt}\isactrlsup c{\isachardoublequoteclose}\isanewline
\ \ \ \ \ \ \isacommand{by}\isamarkupfalse%
\ {\isacharparenleft}{\kern0pt}typecheck{\isacharunderscore}{\kern0pt}cfuncs{\isacharcomma}{\kern0pt}\ simp\ add{\isacharcolon}{\kern0pt}\ {\isasymchi}im{\isacharunderscore}{\kern0pt}i{\isacharunderscore}{\kern0pt}eq{\isacharunderscore}{\kern0pt}{\isasymchi}m\ cfunc{\isacharunderscore}{\kern0pt}type{\isacharunderscore}{\kern0pt}def\ comp{\isacharunderscore}{\kern0pt}associative\ i{\isacharunderscore}{\kern0pt}iso{\isacharparenright}{\kern0pt}\isanewline
\ \ \ \ \isacommand{also}\isamarkupfalse%
\ \isacommand{have}\isamarkupfalse%
\ {\isachardoublequoteopen}{\isachardot}{\kern0pt}{\isachardot}{\kern0pt}{\isachardot}{\kern0pt}\ {\isacharequal}{\kern0pt}\ {\isasymchi}im\ {\isasymcirc}\isactrlsub c\ {\isacharparenleft}{\kern0pt}i\ {\isasymcirc}\isactrlsub c\ m{\isacharparenright}{\kern0pt}\isactrlsup c{\isachardoublequoteclose}\isanewline
\ \ \ \ \ \ \isacommand{using}\isamarkupfalse%
\ i{\isacharunderscore}{\kern0pt}iso\ id{\isacharunderscore}{\kern0pt}left{\isacharunderscore}{\kern0pt}unit{\isadigit{2}}\ inv{\isacharunderscore}{\kern0pt}right\ \isacommand{by}\isamarkupfalse%
\ {\isacharparenleft}{\kern0pt}typecheck{\isacharunderscore}{\kern0pt}cfuncs{\isacharcomma}{\kern0pt}\ auto{\isacharparenright}{\kern0pt}\isanewline
\ \ \ \ \isacommand{also}\isamarkupfalse%
\ \isacommand{have}\isamarkupfalse%
\ {\isachardoublequoteopen}{\isachardot}{\kern0pt}{\isachardot}{\kern0pt}{\isachardot}{\kern0pt}\ {\isacharequal}{\kern0pt}\ {\isasymf}\ {\isasymcirc}\isactrlsub c\ {\isasymbeta}\isactrlbsub B\isactrlesub \ {\isasymcirc}\isactrlsub c\ {\isacharparenleft}{\kern0pt}i\ {\isasymcirc}\isactrlsub c\ m{\isacharparenright}{\kern0pt}\isactrlsup c{\isachardoublequoteclose}\isanewline
\ \ \ \ \ \ \isacommand{by}\isamarkupfalse%
\ {\isacharparenleft}{\kern0pt}typecheck{\isacharunderscore}{\kern0pt}cfuncs{\isacharcomma}{\kern0pt}\ simp\ add{\isacharcolon}{\kern0pt}\ {\isasymchi}im{\isacharunderscore}{\kern0pt}def\ comp{\isacharunderscore}{\kern0pt}associative{\isadigit{2}}\ complement{\isacharunderscore}{\kern0pt}morphism{\isacharunderscore}{\kern0pt}eq\ im{\isacharunderscore}{\kern0pt}mono{\isacharparenright}{\kern0pt}\isanewline
\ \ \ \ \isacommand{also}\isamarkupfalse%
\ \isacommand{have}\isamarkupfalse%
\ {\isachardoublequoteopen}{\isachardot}{\kern0pt}{\isachardot}{\kern0pt}{\isachardot}{\kern0pt}\ {\isacharequal}{\kern0pt}\ {\isasymf}\ {\isasymcirc}\isactrlsub c\ {\isasymbeta}\isactrlbsub A\isactrlesub \ {\isasymcirc}\isactrlsub c\ {\isacharparenleft}{\kern0pt}i\isactrlbold {\isasyminverse}\ {\isasymcirc}\isactrlsub c\ {\isacharparenleft}{\kern0pt}i\ {\isasymcirc}\isactrlsub c\ m{\isacharparenright}{\kern0pt}\isactrlsup c{\isacharparenright}{\kern0pt}{\isachardoublequoteclose}\isanewline
\ \ \ \ \ \ \isacommand{by}\isamarkupfalse%
\ {\isacharparenleft}{\kern0pt}typecheck{\isacharunderscore}{\kern0pt}cfuncs{\isacharcomma}{\kern0pt}\ metis\ i{\isacharunderscore}{\kern0pt}iso\ terminal{\isacharunderscore}{\kern0pt}func{\isacharunderscore}{\kern0pt}unique{\isacharparenright}{\kern0pt}\isanewline
\ \ \ \ \isacommand{then}\isamarkupfalse%
\ \isacommand{show}\isamarkupfalse%
\ {\isacharquery}{\kern0pt}thesis\ \isacommand{using}\isamarkupfalse%
\ calculation\ \isacommand{by}\isamarkupfalse%
\ auto\isanewline
\ \ \isacommand{qed}\isamarkupfalse%
\isanewline
\ \ \isacommand{then}\isamarkupfalse%
\ \isacommand{obtain}\isamarkupfalse%
\ i{\isacharprime}{\kern0pt}{\isacharunderscore}{\kern0pt}inv\ \isakeyword{where}\ i{\isacharprime}{\kern0pt}{\isacharunderscore}{\kern0pt}inv{\isacharunderscore}{\kern0pt}type{\isacharbrackleft}{\kern0pt}type{\isacharunderscore}{\kern0pt}rule{\isacharbrackright}{\kern0pt}{\isacharcolon}{\kern0pt}\ {\isachardoublequoteopen}i{\isacharprime}{\kern0pt}{\isacharunderscore}{\kern0pt}inv\ {\isacharcolon}{\kern0pt}\ B\ {\isasymsetminus}\ {\isacharparenleft}{\kern0pt}C{\isacharcomma}{\kern0pt}\ i\ {\isasymcirc}\isactrlsub c\ m{\isacharparenright}{\kern0pt}\ {\isasymrightarrow}\ A\ {\isasymsetminus}\ {\isacharparenleft}{\kern0pt}C{\isacharcomma}{\kern0pt}\ m{\isacharparenright}{\kern0pt}{\isachardoublequoteclose}\isanewline
\ \ \ \ \isakeyword{and}\ i{\isacharprime}{\kern0pt}{\isacharunderscore}{\kern0pt}inv{\isacharunderscore}{\kern0pt}def{\isacharcolon}{\kern0pt}\ {\isachardoublequoteopen}{\isacharparenleft}{\kern0pt}i\ {\isasymcirc}\isactrlsub c\ m{\isacharparenright}{\kern0pt}\isactrlsup c\ {\isacharequal}{\kern0pt}\ {\isacharparenleft}{\kern0pt}i\ {\isasymcirc}\isactrlsub c\ m\isactrlsup c{\isacharparenright}{\kern0pt}\ {\isasymcirc}\isactrlsub c\ i{\isacharprime}{\kern0pt}{\isacharunderscore}{\kern0pt}inv{\isachardoublequoteclose}\isanewline
\ \ \ \ \isacommand{using}\isamarkupfalse%
\ complement{\isacharunderscore}{\kern0pt}morphism{\isacharunderscore}{\kern0pt}equalizer{\isacharbrackleft}{\kern0pt}\isakeyword{where}\ m{\isacharequal}{\kern0pt}{\isachardoublequoteopen}m{\isachardoublequoteclose}{\isacharcomma}{\kern0pt}\ \isakeyword{where}\ X{\isacharequal}{\kern0pt}C{\isacharcomma}{\kern0pt}\ \isakeyword{where}\ Y{\isacharequal}{\kern0pt}A{\isacharbrackright}{\kern0pt}\ \isacommand{unfolding}\isamarkupfalse%
\ equalizer{\isacharunderscore}{\kern0pt}def\isanewline
\ \ \ \ \isacommand{by}\isamarkupfalse%
\ {\isacharparenleft}{\kern0pt}{\isacharminus}{\kern0pt}{\isacharcomma}{\kern0pt}\ typecheck{\isacharunderscore}{\kern0pt}cfuncs{\isacharcomma}{\kern0pt}\ smt\ {\isacharparenleft}{\kern0pt}z{\isadigit{3}}{\isacharparenright}{\kern0pt}\ {\isasymchi}m{\isacharunderscore}{\kern0pt}def\ cfunc{\isacharunderscore}{\kern0pt}type{\isacharunderscore}{\kern0pt}def\ comp{\isacharunderscore}{\kern0pt}associative{\isadigit{2}}\ i{\isacharunderscore}{\kern0pt}iso\ id{\isacharunderscore}{\kern0pt}left{\isacharunderscore}{\kern0pt}unit{\isadigit{2}}\ inv{\isacharunderscore}{\kern0pt}right\ m{\isacharunderscore}{\kern0pt}mono{\isacharparenright}{\kern0pt}\isanewline
\isanewline
\ \ \isacommand{have}\isamarkupfalse%
\ {\isachardoublequoteopen}isomorphism\ i{\isacharprime}{\kern0pt}{\isachardoublequoteclose}\isanewline
\ \ \isacommand{proof}\isamarkupfalse%
\ {\isacharparenleft}{\kern0pt}etcs{\isacharunderscore}{\kern0pt}subst\ isomorphism{\isacharunderscore}{\kern0pt}def{\isadigit{3}}{\isacharcomma}{\kern0pt}\ rule{\isacharunderscore}{\kern0pt}tac\ x{\isacharequal}{\kern0pt}{\isachardoublequoteopen}i{\isacharprime}{\kern0pt}{\isacharunderscore}{\kern0pt}inv{\isachardoublequoteclose}\ \isakeyword{in}\ exI{\isacharcomma}{\kern0pt}\ typecheck{\isacharunderscore}{\kern0pt}cfuncs{\isacharcomma}{\kern0pt}\ safe{\isacharparenright}{\kern0pt}\isanewline
\ \ \ \ \isacommand{have}\isamarkupfalse%
\ {\isachardoublequoteopen}i\ {\isasymcirc}\isactrlsub c\ m\isactrlsup c\ {\isacharequal}{\kern0pt}\ {\isacharparenleft}{\kern0pt}i\ {\isasymcirc}\isactrlsub c\ m\isactrlsup c{\isacharparenright}{\kern0pt}\ {\isasymcirc}\isactrlsub c\ i{\isacharprime}{\kern0pt}{\isacharunderscore}{\kern0pt}inv\ {\isasymcirc}\isactrlsub c\ i{\isacharprime}{\kern0pt}{\isachardoublequoteclose}\isanewline
\ \ \ \ \ \ \isacommand{using}\isamarkupfalse%
\ i{\isacharprime}{\kern0pt}{\isacharunderscore}{\kern0pt}inv{\isacharunderscore}{\kern0pt}def\ \isacommand{by}\isamarkupfalse%
\ {\isacharparenleft}{\kern0pt}etcs{\isacharunderscore}{\kern0pt}subst\ i{\isacharprime}{\kern0pt}{\isacharunderscore}{\kern0pt}def{\isacharcomma}{\kern0pt}\ etcs{\isacharunderscore}{\kern0pt}assocl{\isacharcomma}{\kern0pt}\ auto{\isacharparenright}{\kern0pt}\isanewline
\ \ \ \ \isacommand{then}\isamarkupfalse%
\ \isacommand{show}\isamarkupfalse%
\ {\isachardoublequoteopen}i{\isacharprime}{\kern0pt}{\isacharunderscore}{\kern0pt}inv\ {\isasymcirc}\isactrlsub c\ i{\isacharprime}{\kern0pt}\ {\isacharequal}{\kern0pt}\ id\isactrlsub c\ {\isacharparenleft}{\kern0pt}A\ {\isasymsetminus}\ {\isacharparenleft}{\kern0pt}C{\isacharcomma}{\kern0pt}\ m{\isacharparenright}{\kern0pt}{\isacharparenright}{\kern0pt}{\isachardoublequoteclose}\isanewline
\ \ \ \ \ \ \isacommand{by}\isamarkupfalse%
\ {\isacharparenleft}{\kern0pt}typecheck{\isacharunderscore}{\kern0pt}cfuncs{\isacharunderscore}{\kern0pt}prems{\isacharcomma}{\kern0pt}\ smt\ {\isacharparenleft}{\kern0pt}verit{\isacharcomma}{\kern0pt}\ best{\isacharparenright}{\kern0pt}\ cfunc{\isacharunderscore}{\kern0pt}type{\isacharunderscore}{\kern0pt}def\ complement{\isacharunderscore}{\kern0pt}morphism{\isacharunderscore}{\kern0pt}mono\ composition{\isacharunderscore}{\kern0pt}of{\isacharunderscore}{\kern0pt}monic{\isacharunderscore}{\kern0pt}pair{\isacharunderscore}{\kern0pt}is{\isacharunderscore}{\kern0pt}monic\ i{\isacharunderscore}{\kern0pt}iso\ id{\isacharunderscore}{\kern0pt}right{\isacharunderscore}{\kern0pt}unit{\isadigit{2}}\ id{\isacharunderscore}{\kern0pt}type\ iso{\isacharunderscore}{\kern0pt}imp{\isacharunderscore}{\kern0pt}epi{\isacharunderscore}{\kern0pt}and{\isacharunderscore}{\kern0pt}monic\ m{\isacharunderscore}{\kern0pt}mono\ monomorphism{\isacharunderscore}{\kern0pt}def{\isadigit{3}}{\isacharparenright}{\kern0pt}\isanewline
\ \ \isacommand{next}\isamarkupfalse%
\isanewline
\ \ \ \ \isacommand{have}\isamarkupfalse%
\ {\isachardoublequoteopen}{\isacharparenleft}{\kern0pt}i\ {\isasymcirc}\isactrlsub c\ m{\isacharparenright}{\kern0pt}\isactrlsup c\ {\isacharequal}{\kern0pt}\ {\isacharparenleft}{\kern0pt}i\ {\isasymcirc}\isactrlsub c\ m{\isacharparenright}{\kern0pt}\isactrlsup c\ {\isasymcirc}\isactrlsub c\ i{\isacharprime}{\kern0pt}\ {\isasymcirc}\isactrlsub c\ i{\isacharprime}{\kern0pt}{\isacharunderscore}{\kern0pt}inv{\isachardoublequoteclose}\isanewline
\ \ \ \ \ \ \isacommand{using}\isamarkupfalse%
\ i{\isacharprime}{\kern0pt}{\isacharunderscore}{\kern0pt}def\ \isacommand{by}\isamarkupfalse%
\ {\isacharparenleft}{\kern0pt}etcs{\isacharunderscore}{\kern0pt}subst\ i{\isacharprime}{\kern0pt}{\isacharunderscore}{\kern0pt}inv{\isacharunderscore}{\kern0pt}def{\isacharcomma}{\kern0pt}\ etcs{\isacharunderscore}{\kern0pt}assocl{\isacharcomma}{\kern0pt}\ auto{\isacharparenright}{\kern0pt}\isanewline
\ \ \ \ \isacommand{then}\isamarkupfalse%
\ \isacommand{show}\isamarkupfalse%
\ {\isachardoublequoteopen}i{\isacharprime}{\kern0pt}\ {\isasymcirc}\isactrlsub c\ i{\isacharprime}{\kern0pt}{\isacharunderscore}{\kern0pt}inv\ {\isacharequal}{\kern0pt}\ id\isactrlsub c\ {\isacharparenleft}{\kern0pt}B\ {\isasymsetminus}\ {\isacharparenleft}{\kern0pt}C{\isacharcomma}{\kern0pt}\ i\ {\isasymcirc}\isactrlsub c\ m{\isacharparenright}{\kern0pt}{\isacharparenright}{\kern0pt}{\isachardoublequoteclose}\isanewline
\ \ \ \ \ \ \isacommand{by}\isamarkupfalse%
\ {\isacharparenleft}{\kern0pt}typecheck{\isacharunderscore}{\kern0pt}cfuncs{\isacharunderscore}{\kern0pt}prems{\isacharcomma}{\kern0pt}\ metis\ complement{\isacharunderscore}{\kern0pt}morphism{\isacharunderscore}{\kern0pt}mono\ id{\isacharunderscore}{\kern0pt}right{\isacharunderscore}{\kern0pt}unit{\isadigit{2}}\ id{\isacharunderscore}{\kern0pt}type\ im{\isacharunderscore}{\kern0pt}mono\ monomorphism{\isacharunderscore}{\kern0pt}def{\isadigit{3}}{\isacharparenright}{\kern0pt}\isanewline
\ \ \isacommand{qed}\isamarkupfalse%
\isanewline
\ \ \isacommand{then}\isamarkupfalse%
\ \isacommand{show}\isamarkupfalse%
\ {\isachardoublequoteopen}A\ {\isasymsetminus}\ {\isacharparenleft}{\kern0pt}C{\isacharcomma}{\kern0pt}\ m{\isacharparenright}{\kern0pt}\ {\isasymcong}\ B\ {\isasymsetminus}\ {\isacharparenleft}{\kern0pt}C{\isacharcomma}{\kern0pt}\ i\ {\isasymcirc}\isactrlsub c\ m{\isacharparenright}{\kern0pt}{\isachardoublequoteclose}\isanewline
\ \ \ \ \isacommand{using}\isamarkupfalse%
\ i{\isacharprime}{\kern0pt}{\isacharunderscore}{\kern0pt}type\ is{\isacharunderscore}{\kern0pt}isomorphic{\isacharunderscore}{\kern0pt}def\ \isacommand{by}\isamarkupfalse%
\ blast\isanewline
\isacommand{qed}\isamarkupfalse%
%
\endisatagproof
{\isafoldproof}%
%
\isadelimproof
%
\endisadelimproof
%
\isadelimdocument
%
\endisadelimdocument
%
\isatagdocument
%
\isamarkupsection{Graphs%
}
\isamarkuptrue%
%
\endisatagdocument
{\isafolddocument}%
%
\isadelimdocument
%
\endisadelimdocument
\isacommand{definition}\isamarkupfalse%
\ functional{\isacharunderscore}{\kern0pt}on\ {\isacharcolon}{\kern0pt}{\isacharcolon}{\kern0pt}\ {\isachardoublequoteopen}cset\ {\isasymRightarrow}\ cset\ {\isasymRightarrow}\ cset\ {\isasymtimes}\ cfunc\ {\isasymRightarrow}\ bool{\isachardoublequoteclose}\ \isakeyword{where}\isanewline
\ \ {\isachardoublequoteopen}functional{\isacharunderscore}{\kern0pt}on\ X\ Y\ R\ {\isacharequal}{\kern0pt}\ {\isacharparenleft}{\kern0pt}R\ \ {\isasymsubseteq}\isactrlsub c\ X\ {\isasymtimes}\isactrlsub c\ Y\ {\isasymand}\isanewline
\ \ \ \ {\isacharparenleft}{\kern0pt}{\isasymforall}x{\isachardot}{\kern0pt}\ x\ {\isasymin}\isactrlsub c\ X\ {\isasymlongrightarrow}\ {\isacharparenleft}{\kern0pt}{\isasymexists}{\isacharbang}{\kern0pt}\ y{\isachardot}{\kern0pt}\ \ y\ {\isasymin}\isactrlsub c\ Y\ {\isasymand}\ \ \isanewline
\ \ \ \ \ \ {\isasymlangle}x{\isacharcomma}{\kern0pt}y{\isasymrangle}\ {\isasymin}\isactrlbsub X{\isasymtimes}\isactrlsub cY\isactrlesub \ R{\isacharparenright}{\kern0pt}{\isacharparenright}{\kern0pt}{\isacharparenright}{\kern0pt}{\isachardoublequoteclose}%
\begin{isamarkuptext}%
The definition below corresponds to Definition 2.3.12 in Halvorson.%
\end{isamarkuptext}\isamarkuptrue%
\isacommand{definition}\isamarkupfalse%
\ graph\ {\isacharcolon}{\kern0pt}{\isacharcolon}{\kern0pt}\ {\isachardoublequoteopen}cfunc\ {\isasymRightarrow}\ cset{\isachardoublequoteclose}\ \isakeyword{where}\isanewline
\ {\isachardoublequoteopen}graph\ f\ {\isacharequal}{\kern0pt}\ {\isacharparenleft}{\kern0pt}SOME\ E{\isachardot}{\kern0pt}\ {\isasymexists}\ m{\isachardot}{\kern0pt}\ equalizer\ E\ m\ {\isacharparenleft}{\kern0pt}f\ {\isasymcirc}\isactrlsub c\ left{\isacharunderscore}{\kern0pt}cart{\isacharunderscore}{\kern0pt}proj\ {\isacharparenleft}{\kern0pt}domain\ f{\isacharparenright}{\kern0pt}\ {\isacharparenleft}{\kern0pt}codomain\ f{\isacharparenright}{\kern0pt}{\isacharparenright}{\kern0pt}\ {\isacharparenleft}{\kern0pt}right{\isacharunderscore}{\kern0pt}cart{\isacharunderscore}{\kern0pt}proj\ {\isacharparenleft}{\kern0pt}domain\ f{\isacharparenright}{\kern0pt}\ {\isacharparenleft}{\kern0pt}codomain\ f{\isacharparenright}{\kern0pt}{\isacharparenright}{\kern0pt}{\isacharparenright}{\kern0pt}{\isachardoublequoteclose}\isanewline
\isanewline
\isacommand{lemma}\isamarkupfalse%
\ graph{\isacharunderscore}{\kern0pt}equalizer{\isacharcolon}{\kern0pt}\isanewline
\ \ {\isachardoublequoteopen}{\isasymexists}\ m{\isachardot}{\kern0pt}\ equalizer\ {\isacharparenleft}{\kern0pt}graph\ f{\isacharparenright}{\kern0pt}\ m\ {\isacharparenleft}{\kern0pt}f\ {\isasymcirc}\isactrlsub c\ left{\isacharunderscore}{\kern0pt}cart{\isacharunderscore}{\kern0pt}proj\ {\isacharparenleft}{\kern0pt}domain\ f{\isacharparenright}{\kern0pt}\ {\isacharparenleft}{\kern0pt}codomain\ f{\isacharparenright}{\kern0pt}{\isacharparenright}{\kern0pt}\ {\isacharparenleft}{\kern0pt}right{\isacharunderscore}{\kern0pt}cart{\isacharunderscore}{\kern0pt}proj\ {\isacharparenleft}{\kern0pt}domain\ f{\isacharparenright}{\kern0pt}\ {\isacharparenleft}{\kern0pt}codomain\ f{\isacharparenright}{\kern0pt}{\isacharparenright}{\kern0pt}{\isachardoublequoteclose}\isanewline
%
\isadelimproof
\ \ %
\endisadelimproof
%
\isatagproof
\isacommand{by}\isamarkupfalse%
\ {\isacharparenleft}{\kern0pt}unfold\ graph{\isacharunderscore}{\kern0pt}def{\isacharcomma}{\kern0pt}\ typecheck{\isacharunderscore}{\kern0pt}cfuncs{\isacharcomma}{\kern0pt}\ rule{\isacharunderscore}{\kern0pt}tac\ someI{\isacharunderscore}{\kern0pt}ex{\isacharcomma}{\kern0pt}\ simp\ add{\isacharcolon}{\kern0pt}\ cfunc{\isacharunderscore}{\kern0pt}type{\isacharunderscore}{\kern0pt}def\ equalizer{\isacharunderscore}{\kern0pt}exists{\isacharparenright}{\kern0pt}%
\endisatagproof
{\isafoldproof}%
%
\isadelimproof
\isanewline
%
\endisadelimproof
\ \ \isanewline
\isacommand{lemma}\isamarkupfalse%
\ graph{\isacharunderscore}{\kern0pt}equalizer{\isadigit{2}}{\isacharcolon}{\kern0pt}\isanewline
\ \ \isakeyword{assumes}\ {\isachardoublequoteopen}f\ {\isacharcolon}{\kern0pt}\ X\ {\isasymrightarrow}\ Y{\isachardoublequoteclose}\isanewline
\ \ \isakeyword{shows}\ {\isachardoublequoteopen}{\isasymexists}\ m{\isachardot}{\kern0pt}\ equalizer\ {\isacharparenleft}{\kern0pt}graph\ f{\isacharparenright}{\kern0pt}\ m\ {\isacharparenleft}{\kern0pt}f\ {\isasymcirc}\isactrlsub c\ left{\isacharunderscore}{\kern0pt}cart{\isacharunderscore}{\kern0pt}proj\ X\ Y{\isacharparenright}{\kern0pt}\ {\isacharparenleft}{\kern0pt}right{\isacharunderscore}{\kern0pt}cart{\isacharunderscore}{\kern0pt}proj\ X\ Y{\isacharparenright}{\kern0pt}{\isachardoublequoteclose}\isanewline
%
\isadelimproof
\ \ %
\endisadelimproof
%
\isatagproof
\isacommand{using}\isamarkupfalse%
\ assms\ \isacommand{by}\isamarkupfalse%
\ {\isacharparenleft}{\kern0pt}typecheck{\isacharunderscore}{\kern0pt}cfuncs{\isacharcomma}{\kern0pt}\ metis\ cfunc{\isacharunderscore}{\kern0pt}type{\isacharunderscore}{\kern0pt}def\ graph{\isacharunderscore}{\kern0pt}equalizer{\isacharparenright}{\kern0pt}%
\endisatagproof
{\isafoldproof}%
%
\isadelimproof
\isanewline
%
\endisadelimproof
\isanewline
\isacommand{definition}\isamarkupfalse%
\ graph{\isacharunderscore}{\kern0pt}morph\ {\isacharcolon}{\kern0pt}{\isacharcolon}{\kern0pt}\ {\isachardoublequoteopen}cfunc\ {\isasymRightarrow}\ cfunc{\isachardoublequoteclose}\ \isakeyword{where}\isanewline
\ {\isachardoublequoteopen}graph{\isacharunderscore}{\kern0pt}morph\ f\ {\isacharequal}{\kern0pt}\ {\isacharparenleft}{\kern0pt}SOME\ m{\isachardot}{\kern0pt}\ equalizer\ {\isacharparenleft}{\kern0pt}graph\ f{\isacharparenright}{\kern0pt}\ m\ {\isacharparenleft}{\kern0pt}f\ {\isasymcirc}\isactrlsub c\ left{\isacharunderscore}{\kern0pt}cart{\isacharunderscore}{\kern0pt}proj\ {\isacharparenleft}{\kern0pt}domain\ f{\isacharparenright}{\kern0pt}\ {\isacharparenleft}{\kern0pt}codomain\ f{\isacharparenright}{\kern0pt}{\isacharparenright}{\kern0pt}\ {\isacharparenleft}{\kern0pt}right{\isacharunderscore}{\kern0pt}cart{\isacharunderscore}{\kern0pt}proj\ {\isacharparenleft}{\kern0pt}domain\ f{\isacharparenright}{\kern0pt}\ {\isacharparenleft}{\kern0pt}codomain\ f{\isacharparenright}{\kern0pt}{\isacharparenright}{\kern0pt}{\isacharparenright}{\kern0pt}{\isachardoublequoteclose}\isanewline
\isanewline
\isacommand{lemma}\isamarkupfalse%
\ graph{\isacharunderscore}{\kern0pt}equalizer{\isadigit{3}}{\isacharcolon}{\kern0pt}\isanewline
\ \ {\isachardoublequoteopen}equalizer\ {\isacharparenleft}{\kern0pt}graph\ f{\isacharparenright}{\kern0pt}\ {\isacharparenleft}{\kern0pt}graph{\isacharunderscore}{\kern0pt}morph\ f{\isacharparenright}{\kern0pt}\ {\isacharparenleft}{\kern0pt}f\ {\isasymcirc}\isactrlsub c\ left{\isacharunderscore}{\kern0pt}cart{\isacharunderscore}{\kern0pt}proj\ {\isacharparenleft}{\kern0pt}domain\ f{\isacharparenright}{\kern0pt}\ {\isacharparenleft}{\kern0pt}codomain\ f{\isacharparenright}{\kern0pt}{\isacharparenright}{\kern0pt}\ {\isacharparenleft}{\kern0pt}right{\isacharunderscore}{\kern0pt}cart{\isacharunderscore}{\kern0pt}proj\ {\isacharparenleft}{\kern0pt}domain\ f{\isacharparenright}{\kern0pt}\ {\isacharparenleft}{\kern0pt}codomain\ f{\isacharparenright}{\kern0pt}{\isacharparenright}{\kern0pt}{\isachardoublequoteclose}\isanewline
%
\isadelimproof
\ \ \ %
\endisadelimproof
%
\isatagproof
\isacommand{using}\isamarkupfalse%
\ graph{\isacharunderscore}{\kern0pt}equalizer\ \isacommand{by}\isamarkupfalse%
\ {\isacharparenleft}{\kern0pt}unfold\ graph{\isacharunderscore}{\kern0pt}morph{\isacharunderscore}{\kern0pt}def{\isacharcomma}{\kern0pt}\ typecheck{\isacharunderscore}{\kern0pt}cfuncs{\isacharcomma}{\kern0pt}\ rule{\isacharunderscore}{\kern0pt}tac\ someI{\isacharunderscore}{\kern0pt}ex{\isacharcomma}{\kern0pt}\ blast{\isacharparenright}{\kern0pt}%
\endisatagproof
{\isafoldproof}%
%
\isadelimproof
\isanewline
%
\endisadelimproof
\isanewline
\isacommand{lemma}\isamarkupfalse%
\ graph{\isacharunderscore}{\kern0pt}equalizer{\isadigit{4}}{\isacharcolon}{\kern0pt}\isanewline
\ \ \isakeyword{assumes}\ {\isachardoublequoteopen}f\ {\isacharcolon}{\kern0pt}\ X\ {\isasymrightarrow}\ Y{\isachardoublequoteclose}\isanewline
\ \ \isakeyword{shows}\ {\isachardoublequoteopen}equalizer\ {\isacharparenleft}{\kern0pt}graph\ f{\isacharparenright}{\kern0pt}\ {\isacharparenleft}{\kern0pt}graph{\isacharunderscore}{\kern0pt}morph\ f{\isacharparenright}{\kern0pt}\ {\isacharparenleft}{\kern0pt}f\ {\isasymcirc}\isactrlsub c\ left{\isacharunderscore}{\kern0pt}cart{\isacharunderscore}{\kern0pt}proj\ X\ Y{\isacharparenright}{\kern0pt}\ {\isacharparenleft}{\kern0pt}right{\isacharunderscore}{\kern0pt}cart{\isacharunderscore}{\kern0pt}proj\ X\ Y{\isacharparenright}{\kern0pt}{\isachardoublequoteclose}\isanewline
%
\isadelimproof
\ \ %
\endisadelimproof
%
\isatagproof
\isacommand{using}\isamarkupfalse%
\ assms\ cfunc{\isacharunderscore}{\kern0pt}type{\isacharunderscore}{\kern0pt}def\ graph{\isacharunderscore}{\kern0pt}equalizer{\isadigit{3}}\ \isacommand{by}\isamarkupfalse%
\ auto%
\endisatagproof
{\isafoldproof}%
%
\isadelimproof
\isanewline
%
\endisadelimproof
\isanewline
\isacommand{lemma}\isamarkupfalse%
\ graph{\isacharunderscore}{\kern0pt}subobject{\isacharcolon}{\kern0pt}\isanewline
\ \ \isakeyword{assumes}\ {\isachardoublequoteopen}f\ {\isacharcolon}{\kern0pt}\ X\ {\isasymrightarrow}\ Y{\isachardoublequoteclose}\isanewline
\ \ \isakeyword{shows}\ {\isachardoublequoteopen}{\isacharparenleft}{\kern0pt}graph\ f{\isacharcomma}{\kern0pt}\ graph{\isacharunderscore}{\kern0pt}morph\ f{\isacharparenright}{\kern0pt}\ {\isasymsubseteq}\isactrlsub c\ {\isacharparenleft}{\kern0pt}X\ {\isasymtimes}\isactrlsub c\ Y{\isacharparenright}{\kern0pt}{\isachardoublequoteclose}\isanewline
%
\isadelimproof
\ \ %
\endisadelimproof
%
\isatagproof
\isacommand{by}\isamarkupfalse%
\ {\isacharparenleft}{\kern0pt}metis\ assms\ cfunc{\isacharunderscore}{\kern0pt}type{\isacharunderscore}{\kern0pt}def\ equalizer{\isacharunderscore}{\kern0pt}def\ equalizer{\isacharunderscore}{\kern0pt}is{\isacharunderscore}{\kern0pt}monomorphism\ graph{\isacharunderscore}{\kern0pt}equalizer{\isadigit{3}}\ right{\isacharunderscore}{\kern0pt}cart{\isacharunderscore}{\kern0pt}proj{\isacharunderscore}{\kern0pt}type\ subobject{\isacharunderscore}{\kern0pt}of{\isacharunderscore}{\kern0pt}def{\isadigit{2}}{\isacharparenright}{\kern0pt}%
\endisatagproof
{\isafoldproof}%
%
\isadelimproof
\isanewline
%
\endisadelimproof
\isanewline
\isacommand{lemma}\isamarkupfalse%
\ graph{\isacharunderscore}{\kern0pt}morph{\isacharunderscore}{\kern0pt}type{\isacharbrackleft}{\kern0pt}type{\isacharunderscore}{\kern0pt}rule{\isacharbrackright}{\kern0pt}{\isacharcolon}{\kern0pt}\isanewline
\ \ \isakeyword{assumes}\ {\isachardoublequoteopen}f\ {\isacharcolon}{\kern0pt}\ X\ {\isasymrightarrow}\ Y{\isachardoublequoteclose}\isanewline
\ \ \isakeyword{shows}\ {\isachardoublequoteopen}graph{\isacharunderscore}{\kern0pt}morph{\isacharparenleft}{\kern0pt}f{\isacharparenright}{\kern0pt}\ {\isacharcolon}{\kern0pt}\ graph\ f\ {\isasymrightarrow}\ X\ {\isasymtimes}\isactrlsub c\ Y{\isachardoublequoteclose}\isanewline
%
\isadelimproof
\ \ %
\endisadelimproof
%
\isatagproof
\isacommand{using}\isamarkupfalse%
\ graph{\isacharunderscore}{\kern0pt}subobject\ subobject{\isacharunderscore}{\kern0pt}of{\isacharunderscore}{\kern0pt}def{\isadigit{2}}\ assms\ \isacommand{by}\isamarkupfalse%
\ auto%
\endisatagproof
{\isafoldproof}%
%
\isadelimproof
%
\endisadelimproof
%
\begin{isamarkuptext}%
The lemma below corresponds to Exercise 2.3.13 in Halvorson.%
\end{isamarkuptext}\isamarkuptrue%
\isacommand{lemma}\isamarkupfalse%
\ graphs{\isacharunderscore}{\kern0pt}are{\isacharunderscore}{\kern0pt}functional{\isacharcolon}{\kern0pt}\isanewline
\ \ \isakeyword{assumes}\ {\isachardoublequoteopen}f\ {\isacharcolon}{\kern0pt}\ X\ {\isasymrightarrow}\ Y{\isachardoublequoteclose}\isanewline
\ \ \isakeyword{shows}\ {\isachardoublequoteopen}functional{\isacharunderscore}{\kern0pt}on\ X\ Y\ {\isacharparenleft}{\kern0pt}graph\ f{\isacharcomma}{\kern0pt}\ graph{\isacharunderscore}{\kern0pt}morph\ f{\isacharparenright}{\kern0pt}{\isachardoublequoteclose}\isanewline
%
\isadelimproof
%
\endisadelimproof
%
\isatagproof
\isacommand{proof}\isamarkupfalse%
{\isacharparenleft}{\kern0pt}unfold\ functional{\isacharunderscore}{\kern0pt}on{\isacharunderscore}{\kern0pt}def{\isacharcomma}{\kern0pt}\ safe{\isacharparenright}{\kern0pt}\isanewline
\ \ \isacommand{show}\isamarkupfalse%
\ graph{\isacharunderscore}{\kern0pt}subobj{\isacharcolon}{\kern0pt}\ {\isachardoublequoteopen}{\isacharparenleft}{\kern0pt}graph\ f{\isacharcomma}{\kern0pt}\ graph{\isacharunderscore}{\kern0pt}morph\ f{\isacharparenright}{\kern0pt}\ \ {\isasymsubseteq}\isactrlsub c\ {\isacharparenleft}{\kern0pt}X{\isasymtimes}\isactrlsub c\ Y{\isacharparenright}{\kern0pt}{\isachardoublequoteclose}\isanewline
\ \ \ \ \isacommand{by}\isamarkupfalse%
\ {\isacharparenleft}{\kern0pt}simp\ add{\isacharcolon}{\kern0pt}\ assms\ graph{\isacharunderscore}{\kern0pt}subobject{\isacharparenright}{\kern0pt}\isanewline
\ \ \isacommand{show}\isamarkupfalse%
\ {\isachardoublequoteopen}{\isasymAnd}x{\isachardot}{\kern0pt}\ x\ {\isasymin}\isactrlsub c\ X\ {\isasymLongrightarrow}\ {\isasymexists}y{\isachardot}{\kern0pt}\ y\ {\isasymin}\isactrlsub c\ Y\ {\isasymand}\ {\isasymlangle}x{\isacharcomma}{\kern0pt}y{\isasymrangle}\ {\isasymin}\isactrlbsub X\ {\isasymtimes}\isactrlsub c\ Y\isactrlesub \ {\isacharparenleft}{\kern0pt}graph\ f{\isacharcomma}{\kern0pt}\ graph{\isacharunderscore}{\kern0pt}morph\ f{\isacharparenright}{\kern0pt}{\isachardoublequoteclose}\isanewline
\ \ \isacommand{proof}\isamarkupfalse%
\ {\isacharminus}{\kern0pt}\ \isanewline
\ \ \ \ \isacommand{fix}\isamarkupfalse%
\ x\ \isanewline
\ \ \ \ \isacommand{assume}\isamarkupfalse%
\ x{\isacharunderscore}{\kern0pt}type{\isacharbrackleft}{\kern0pt}type{\isacharunderscore}{\kern0pt}rule{\isacharbrackright}{\kern0pt}{\isacharcolon}{\kern0pt}\ {\isachardoublequoteopen}x\ {\isasymin}\isactrlsub c\ X{\isachardoublequoteclose}\isanewline
\ \ \ \ \isacommand{obtain}\isamarkupfalse%
\ y\ \isakeyword{where}\ y{\isacharunderscore}{\kern0pt}def{\isacharcolon}{\kern0pt}\ {\isachardoublequoteopen}y\ {\isacharequal}{\kern0pt}\ f\ {\isasymcirc}\isactrlsub c\ x{\isachardoublequoteclose}\isanewline
\ \ \ \ \ \ \isacommand{by}\isamarkupfalse%
\ simp\isanewline
\ \ \ \ \isacommand{then}\isamarkupfalse%
\ \isacommand{have}\isamarkupfalse%
\ y{\isacharunderscore}{\kern0pt}type{\isacharbrackleft}{\kern0pt}type{\isacharunderscore}{\kern0pt}rule{\isacharbrackright}{\kern0pt}{\isacharcolon}{\kern0pt}\ {\isachardoublequoteopen}y\ {\isasymin}\isactrlsub c\ Y{\isachardoublequoteclose}\isanewline
\ \ \ \ \ \ \isacommand{using}\isamarkupfalse%
\ assms\ comp{\isacharunderscore}{\kern0pt}type\ x{\isacharunderscore}{\kern0pt}type\ y{\isacharunderscore}{\kern0pt}def\ \isacommand{by}\isamarkupfalse%
\ blast\isanewline
\isanewline
\ \ \ \ \isacommand{have}\isamarkupfalse%
\ {\isachardoublequoteopen}{\isasymlangle}x{\isacharcomma}{\kern0pt}y{\isasymrangle}\ {\isasymin}\isactrlbsub X\ {\isasymtimes}\isactrlsub c\ Y\isactrlesub \ {\isacharparenleft}{\kern0pt}graph\ f{\isacharcomma}{\kern0pt}\ graph{\isacharunderscore}{\kern0pt}morph\ f{\isacharparenright}{\kern0pt}{\isachardoublequoteclose}\isanewline
\ \ \ \ \isacommand{proof}\isamarkupfalse%
{\isacharparenleft}{\kern0pt}unfold\ relative{\isacharunderscore}{\kern0pt}member{\isacharunderscore}{\kern0pt}def{\isacharcomma}{\kern0pt}\ safe{\isacharparenright}{\kern0pt}\isanewline
\ \ \ \ \ \ \isacommand{show}\isamarkupfalse%
\ {\isachardoublequoteopen}{\isasymlangle}x{\isacharcomma}{\kern0pt}y{\isasymrangle}\ {\isasymin}\isactrlsub c\ X\ {\isasymtimes}\isactrlsub c\ Y{\isachardoublequoteclose}\isanewline
\ \ \ \ \ \ \ \ \isacommand{by}\isamarkupfalse%
\ typecheck{\isacharunderscore}{\kern0pt}cfuncs\ \isanewline
\ \ \ \ \ \ \isacommand{show}\isamarkupfalse%
\ {\isachardoublequoteopen}monomorphism\ {\isacharparenleft}{\kern0pt}snd\ {\isacharparenleft}{\kern0pt}graph\ f{\isacharcomma}{\kern0pt}\ graph{\isacharunderscore}{\kern0pt}morph\ f{\isacharparenright}{\kern0pt}{\isacharparenright}{\kern0pt}{\isachardoublequoteclose}\isanewline
\ \ \ \ \ \ \ \ \isacommand{using}\isamarkupfalse%
\ graph{\isacharunderscore}{\kern0pt}subobj\ subobject{\isacharunderscore}{\kern0pt}of{\isacharunderscore}{\kern0pt}def\ \isacommand{by}\isamarkupfalse%
\ auto\isanewline
\ \ \ \ \ \ \isacommand{show}\isamarkupfalse%
\ {\isachardoublequoteopen}\ snd\ {\isacharparenleft}{\kern0pt}graph\ f{\isacharcomma}{\kern0pt}\ graph{\isacharunderscore}{\kern0pt}morph\ f{\isacharparenright}{\kern0pt}\ {\isacharcolon}{\kern0pt}\ fst\ {\isacharparenleft}{\kern0pt}graph\ f{\isacharcomma}{\kern0pt}\ graph{\isacharunderscore}{\kern0pt}morph\ f{\isacharparenright}{\kern0pt}\ {\isasymrightarrow}\ X\ {\isasymtimes}\isactrlsub c\ Y{\isachardoublequoteclose}\isanewline
\ \ \ \ \ \ \ \ \isacommand{by}\isamarkupfalse%
\ {\isacharparenleft}{\kern0pt}simp\ add{\isacharcolon}{\kern0pt}\ assms\ graph{\isacharunderscore}{\kern0pt}morph{\isacharunderscore}{\kern0pt}type{\isacharparenright}{\kern0pt}\isanewline
\ \ \ \ \ \ \isacommand{have}\isamarkupfalse%
\ {\isachardoublequoteopen}{\isasymlangle}x{\isacharcomma}{\kern0pt}y{\isasymrangle}\ factorsthru\ graph{\isacharunderscore}{\kern0pt}morph\ f{\isachardoublequoteclose}\isanewline
\ \ \ \ \ \ \isacommand{proof}\isamarkupfalse%
{\isacharparenleft}{\kern0pt}subst\ xfactorthru{\isacharunderscore}{\kern0pt}equalizer{\isacharunderscore}{\kern0pt}iff{\isacharunderscore}{\kern0pt}fx{\isacharunderscore}{\kern0pt}eq{\isacharunderscore}{\kern0pt}gx{\isacharbrackleft}{\kern0pt}\isakeyword{where}\ E\ {\isacharequal}{\kern0pt}\ {\isachardoublequoteopen}graph\ f{\isachardoublequoteclose}{\isacharcomma}{\kern0pt}\ \isakeyword{where}\ m\ {\isacharequal}{\kern0pt}\ {\isachardoublequoteopen}graph{\isacharunderscore}{\kern0pt}morph\ f{\isachardoublequoteclose}{\isacharcomma}{\kern0pt}\ \ \isanewline
\ \ \ \ \ \ \ \ \ \ \ \ \ \ \ \ \ \ \ \ \ \ \ \ \ \ \ \ \ \ \ \ \ \ \ \ \ \ \ \ \ \ \ \ \ \ \ \ \ \ \ \ \ \isakeyword{where}\ f\ {\isacharequal}{\kern0pt}\ {\isachardoublequoteopen}{\isacharparenleft}{\kern0pt}f\ {\isasymcirc}\isactrlsub c\ left{\isacharunderscore}{\kern0pt}cart{\isacharunderscore}{\kern0pt}proj\ X\ Y{\isacharparenright}{\kern0pt}{\isachardoublequoteclose}{\isacharcomma}{\kern0pt}\ \isakeyword{where}\ g\ {\isacharequal}{\kern0pt}\ {\isachardoublequoteopen}right{\isacharunderscore}{\kern0pt}cart{\isacharunderscore}{\kern0pt}proj\ X\ Y{\isachardoublequoteclose}{\isacharcomma}{\kern0pt}\ \isakeyword{where}\ X\ {\isacharequal}{\kern0pt}\ {\isachardoublequoteopen}X\ {\isasymtimes}\isactrlsub c\ Y{\isachardoublequoteclose}{\isacharcomma}{\kern0pt}\ \isakeyword{where}\ Y\ {\isacharequal}{\kern0pt}\ Y{\isacharcomma}{\kern0pt}\isanewline
\ \ \ \ \ \ \ \ \ \ \ \ \ \ \ \ \ \ \ \ \ \ \ \ \ \ \ \ \ \ \ \ \ \ \ \ \ \ \ \ \ \ \ \ \ \ \ \ \ \ \ \ \ \isakeyword{where}\ x\ {\isacharequal}{\kern0pt}{\isachardoublequoteopen}{\isasymlangle}x{\isacharcomma}{\kern0pt}y{\isasymrangle}{\isachardoublequoteclose}{\isacharbrackright}{\kern0pt}{\isacharparenright}{\kern0pt}\isanewline
\ \ \ \ \ \ \ \ \isacommand{show}\isamarkupfalse%
\ {\isachardoublequoteopen}f\ {\isasymcirc}\isactrlsub c\ left{\isacharunderscore}{\kern0pt}cart{\isacharunderscore}{\kern0pt}proj\ X\ Y\ {\isacharcolon}{\kern0pt}\ X\ {\isasymtimes}\isactrlsub c\ Y\ {\isasymrightarrow}\ Y{\isachardoublequoteclose}\isanewline
\ \ \ \ \ \ \ \ \ \ \isacommand{using}\isamarkupfalse%
\ assms\ \isacommand{by}\isamarkupfalse%
\ typecheck{\isacharunderscore}{\kern0pt}cfuncs\isanewline
\ \ \ \ \ \ \ \ \isacommand{show}\isamarkupfalse%
\ {\isachardoublequoteopen}right{\isacharunderscore}{\kern0pt}cart{\isacharunderscore}{\kern0pt}proj\ X\ Y\ {\isacharcolon}{\kern0pt}\ X\ {\isasymtimes}\isactrlsub c\ Y\ {\isasymrightarrow}\ Y{\isachardoublequoteclose}\isanewline
\ \ \ \ \ \ \ \ \ \ \isacommand{by}\isamarkupfalse%
\ \ typecheck{\isacharunderscore}{\kern0pt}cfuncs\isanewline
\ \ \ \ \ \ \ \ \isacommand{show}\isamarkupfalse%
\ {\isachardoublequoteopen}equalizer\ {\isacharparenleft}{\kern0pt}graph\ f{\isacharparenright}{\kern0pt}\ {\isacharparenleft}{\kern0pt}graph{\isacharunderscore}{\kern0pt}morph\ f{\isacharparenright}{\kern0pt}\ {\isacharparenleft}{\kern0pt}f\ {\isasymcirc}\isactrlsub c\ left{\isacharunderscore}{\kern0pt}cart{\isacharunderscore}{\kern0pt}proj\ X\ Y{\isacharparenright}{\kern0pt}\ {\isacharparenleft}{\kern0pt}right{\isacharunderscore}{\kern0pt}cart{\isacharunderscore}{\kern0pt}proj\ X\ Y{\isacharparenright}{\kern0pt}{\isachardoublequoteclose}\isanewline
\ \ \ \ \ \ \ \ \ \ \isacommand{by}\isamarkupfalse%
\ {\isacharparenleft}{\kern0pt}simp\ add{\isacharcolon}{\kern0pt}\ assms\ graph{\isacharunderscore}{\kern0pt}equalizer{\isadigit{4}}{\isacharparenright}{\kern0pt}\isanewline
\ \ \ \ \ \ \ \ \isacommand{show}\isamarkupfalse%
\ {\isachardoublequoteopen}{\isasymlangle}x{\isacharcomma}{\kern0pt}y{\isasymrangle}\ {\isasymin}\isactrlsub c\ X\ {\isasymtimes}\isactrlsub c\ Y{\isachardoublequoteclose}\isanewline
\ \ \ \ \ \ \ \ \ \ \isacommand{by}\isamarkupfalse%
\ typecheck{\isacharunderscore}{\kern0pt}cfuncs\isanewline
\ \ \ \ \ \ \ \ \isacommand{show}\isamarkupfalse%
\ {\isachardoublequoteopen}{\isacharparenleft}{\kern0pt}f\ {\isasymcirc}\isactrlsub c\ left{\isacharunderscore}{\kern0pt}cart{\isacharunderscore}{\kern0pt}proj\ X\ Y{\isacharparenright}{\kern0pt}\ {\isasymcirc}\isactrlsub c\ {\isasymlangle}x{\isacharcomma}{\kern0pt}y{\isasymrangle}\ {\isacharequal}{\kern0pt}\ right{\isacharunderscore}{\kern0pt}cart{\isacharunderscore}{\kern0pt}proj\ X\ Y\ {\isasymcirc}\isactrlsub c\ {\isasymlangle}x{\isacharcomma}{\kern0pt}y{\isasymrangle}{\isachardoublequoteclose}\isanewline
\ \ \ \ \ \ \ \ \ \ \isacommand{using}\isamarkupfalse%
\ assms\ \ \isanewline
\ \ \ \ \ \ \ \ \ \ \isacommand{by}\isamarkupfalse%
\ {\isacharparenleft}{\kern0pt}typecheck{\isacharunderscore}{\kern0pt}cfuncs{\isacharcomma}{\kern0pt}\ smt\ {\isacharparenleft}{\kern0pt}z{\isadigit{3}}{\isacharparenright}{\kern0pt}\ comp{\isacharunderscore}{\kern0pt}associative{\isadigit{2}}\ left{\isacharunderscore}{\kern0pt}cart{\isacharunderscore}{\kern0pt}proj{\isacharunderscore}{\kern0pt}cfunc{\isacharunderscore}{\kern0pt}prod\ right{\isacharunderscore}{\kern0pt}cart{\isacharunderscore}{\kern0pt}proj{\isacharunderscore}{\kern0pt}cfunc{\isacharunderscore}{\kern0pt}prod\ y{\isacharunderscore}{\kern0pt}def{\isacharparenright}{\kern0pt}\isanewline
\ \ \ \ \ \ \isacommand{qed}\isamarkupfalse%
\isanewline
\ \ \ \ \ \ \isacommand{then}\isamarkupfalse%
\ \isacommand{show}\isamarkupfalse%
\ {\isachardoublequoteopen}{\isasymlangle}x{\isacharcomma}{\kern0pt}y{\isasymrangle}\ factorsthru\ snd\ {\isacharparenleft}{\kern0pt}graph\ f{\isacharcomma}{\kern0pt}\ graph{\isacharunderscore}{\kern0pt}morph\ f{\isacharparenright}{\kern0pt}{\isachardoublequoteclose}\isanewline
\ \ \ \ \ \ \ \ \isacommand{by}\isamarkupfalse%
\ simp\isanewline
\ \ \ \ \isacommand{qed}\isamarkupfalse%
\isanewline
\ \ \ \ \isacommand{then}\isamarkupfalse%
\ \isacommand{show}\isamarkupfalse%
\ {\isachardoublequoteopen}{\isasymexists}y{\isachardot}{\kern0pt}\ y\ {\isasymin}\isactrlsub c\ Y\ {\isasymand}\ {\isasymlangle}x{\isacharcomma}{\kern0pt}y{\isasymrangle}\ {\isasymin}\isactrlbsub X\ {\isasymtimes}\isactrlsub c\ Y\isactrlesub \ {\isacharparenleft}{\kern0pt}graph\ f{\isacharcomma}{\kern0pt}\ graph{\isacharunderscore}{\kern0pt}morph\ f{\isacharparenright}{\kern0pt}{\isachardoublequoteclose}\isanewline
\ \ \ \ \ \ \isacommand{using}\isamarkupfalse%
\ y{\isacharunderscore}{\kern0pt}type\ \isacommand{by}\isamarkupfalse%
\ blast\isanewline
\ \ \isacommand{qed}\isamarkupfalse%
\isanewline
\ \ \isacommand{show}\isamarkupfalse%
\ {\isachardoublequoteopen}{\isasymAnd}x\ y\ ya{\isachardot}{\kern0pt}\isanewline
\ \ \ \ \ \ \ x\ {\isasymin}\isactrlsub c\ X\ {\isasymLongrightarrow}\isanewline
\ \ \ \ \ \ \ y\ {\isasymin}\isactrlsub c\ Y\ {\isasymLongrightarrow}\isanewline
\ \ \ \ \ \ \ {\isasymlangle}x{\isacharcomma}{\kern0pt}y{\isasymrangle}\ {\isasymin}\isactrlbsub X\ {\isasymtimes}\isactrlsub c\ Y\isactrlesub \ {\isacharparenleft}{\kern0pt}graph\ f{\isacharcomma}{\kern0pt}\ graph{\isacharunderscore}{\kern0pt}morph\ f{\isacharparenright}{\kern0pt}\ {\isasymLongrightarrow}\ \isanewline
\ \ \ \ \ \ \ \ ya\ {\isasymin}\isactrlsub c\ Y\ {\isasymLongrightarrow}\ \isanewline
\ \ \ \ \ \ \ \ {\isasymlangle}x{\isacharcomma}{\kern0pt}ya{\isasymrangle}\ {\isasymin}\isactrlbsub X\ {\isasymtimes}\isactrlsub c\ Y\isactrlesub \ {\isacharparenleft}{\kern0pt}graph\ f{\isacharcomma}{\kern0pt}\ graph{\isacharunderscore}{\kern0pt}morph\ f{\isacharparenright}{\kern0pt}\isanewline
\ \ \ \ \ \ \ \ \ {\isasymLongrightarrow}\ y\ {\isacharequal}{\kern0pt}\ ya{\isachardoublequoteclose}\isanewline
\ \ \ \ \isacommand{using}\isamarkupfalse%
\ assms\ \ \isanewline
\ \ \ \ \isacommand{by}\isamarkupfalse%
\ {\isacharparenleft}{\kern0pt}smt\ {\isacharparenleft}{\kern0pt}z{\isadigit{3}}{\isacharparenright}{\kern0pt}\ comp{\isacharunderscore}{\kern0pt}associative{\isadigit{2}}\ equalizer{\isacharunderscore}{\kern0pt}def\ factors{\isacharunderscore}{\kern0pt}through{\isacharunderscore}{\kern0pt}def{\isadigit{2}}\ graph{\isacharunderscore}{\kern0pt}equalizer{\isadigit{4}}\ left{\isacharunderscore}{\kern0pt}cart{\isacharunderscore}{\kern0pt}proj{\isacharunderscore}{\kern0pt}cfunc{\isacharunderscore}{\kern0pt}prod\ left{\isacharunderscore}{\kern0pt}cart{\isacharunderscore}{\kern0pt}proj{\isacharunderscore}{\kern0pt}type\ relative{\isacharunderscore}{\kern0pt}member{\isacharunderscore}{\kern0pt}def{\isadigit{2}}\ right{\isacharunderscore}{\kern0pt}cart{\isacharunderscore}{\kern0pt}proj{\isacharunderscore}{\kern0pt}cfunc{\isacharunderscore}{\kern0pt}prod{\isacharparenright}{\kern0pt}\isanewline
\isacommand{qed}\isamarkupfalse%
%
\endisatagproof
{\isafoldproof}%
%
\isadelimproof
\isanewline
%
\endisadelimproof
\isanewline
\isacommand{lemma}\isamarkupfalse%
\ functional{\isacharunderscore}{\kern0pt}on{\isacharunderscore}{\kern0pt}isomorphism{\isacharcolon}{\kern0pt}\isanewline
\ \ \isakeyword{assumes}\ {\isachardoublequoteopen}functional{\isacharunderscore}{\kern0pt}on\ X\ Y\ {\isacharparenleft}{\kern0pt}R{\isacharcomma}{\kern0pt}m{\isacharparenright}{\kern0pt}{\isachardoublequoteclose}\isanewline
\ \ \isakeyword{shows}\ {\isachardoublequoteopen}isomorphism{\isacharparenleft}{\kern0pt}left{\isacharunderscore}{\kern0pt}cart{\isacharunderscore}{\kern0pt}proj\ X\ Y\ {\isasymcirc}\isactrlsub c\ m{\isacharparenright}{\kern0pt}{\isachardoublequoteclose}\isanewline
%
\isadelimproof
%
\endisadelimproof
%
\isatagproof
\isacommand{proof}\isamarkupfalse%
{\isacharminus}{\kern0pt}\isanewline
\ \ \isacommand{have}\isamarkupfalse%
\ m{\isacharunderscore}{\kern0pt}mono{\isacharcolon}{\kern0pt}\ {\isachardoublequoteopen}monomorphism{\isacharparenleft}{\kern0pt}m{\isacharparenright}{\kern0pt}{\isachardoublequoteclose}\isanewline
\ \ \ \ \isacommand{using}\isamarkupfalse%
\ assms\ functional{\isacharunderscore}{\kern0pt}on{\isacharunderscore}{\kern0pt}def\ subobject{\isacharunderscore}{\kern0pt}of{\isacharunderscore}{\kern0pt}def{\isadigit{2}}\ \isacommand{by}\isamarkupfalse%
\ blast\isanewline
\ \ \isacommand{have}\isamarkupfalse%
\ pi{\isadigit{0}}{\isacharunderscore}{\kern0pt}m{\isacharunderscore}{\kern0pt}type{\isacharbrackleft}{\kern0pt}type{\isacharunderscore}{\kern0pt}rule{\isacharbrackright}{\kern0pt}{\isacharcolon}{\kern0pt}\ {\isachardoublequoteopen}left{\isacharunderscore}{\kern0pt}cart{\isacharunderscore}{\kern0pt}proj\ X\ Y\ {\isasymcirc}\isactrlsub c\ m\ {\isacharcolon}{\kern0pt}\ R\ {\isasymrightarrow}\ X{\isachardoublequoteclose}\isanewline
\ \ \ \ \isacommand{using}\isamarkupfalse%
\ assms\ functional{\isacharunderscore}{\kern0pt}on{\isacharunderscore}{\kern0pt}def\ subobject{\isacharunderscore}{\kern0pt}of{\isacharunderscore}{\kern0pt}def{\isadigit{2}}\ \isacommand{by}\isamarkupfalse%
\ {\isacharparenleft}{\kern0pt}typecheck{\isacharunderscore}{\kern0pt}cfuncs{\isacharcomma}{\kern0pt}\ blast{\isacharparenright}{\kern0pt}\isanewline
\ \ \isacommand{have}\isamarkupfalse%
\ surj{\isacharcolon}{\kern0pt}\ {\isachardoublequoteopen}surjective{\isacharparenleft}{\kern0pt}left{\isacharunderscore}{\kern0pt}cart{\isacharunderscore}{\kern0pt}proj\ X\ Y\ {\isasymcirc}\isactrlsub c\ m{\isacharparenright}{\kern0pt}{\isachardoublequoteclose}\isanewline
\ \ \isacommand{proof}\isamarkupfalse%
{\isacharparenleft}{\kern0pt}unfold\ surjective{\isacharunderscore}{\kern0pt}def{\isacharcomma}{\kern0pt}\ clarify{\isacharparenright}{\kern0pt}\isanewline
\ \ \ \ \isacommand{fix}\isamarkupfalse%
\ x\ \isanewline
\ \ \ \ \isacommand{assume}\isamarkupfalse%
\ {\isachardoublequoteopen}x\ {\isasymin}\isactrlsub c\ codomain\ {\isacharparenleft}{\kern0pt}left{\isacharunderscore}{\kern0pt}cart{\isacharunderscore}{\kern0pt}proj\ X\ Y\ {\isasymcirc}\isactrlsub c\ m{\isacharparenright}{\kern0pt}{\isachardoublequoteclose}\isanewline
\ \ \ \ \isacommand{then}\isamarkupfalse%
\ \isacommand{have}\isamarkupfalse%
\ {\isacharbrackleft}{\kern0pt}type{\isacharunderscore}{\kern0pt}rule{\isacharbrackright}{\kern0pt}{\isacharcolon}{\kern0pt}\ {\isachardoublequoteopen}x\ {\isasymin}\isactrlsub c\ X{\isachardoublequoteclose}\isanewline
\ \ \ \ \ \ \isacommand{using}\isamarkupfalse%
\ cfunc{\isacharunderscore}{\kern0pt}type{\isacharunderscore}{\kern0pt}def\ pi{\isadigit{0}}{\isacharunderscore}{\kern0pt}m{\isacharunderscore}{\kern0pt}type\ \isacommand{by}\isamarkupfalse%
\ force\isanewline
\ \ \ \ \isacommand{then}\isamarkupfalse%
\ \isacommand{have}\isamarkupfalse%
\ {\isachardoublequoteopen}{\isasymexists}{\isacharbang}{\kern0pt}\ y{\isachardot}{\kern0pt}\ {\isacharparenleft}{\kern0pt}y\ {\isasymin}\isactrlsub c\ Y\ {\isasymand}\ \ {\isasymlangle}x{\isacharcomma}{\kern0pt}y{\isasymrangle}\ {\isasymin}\isactrlbsub X{\isasymtimes}\isactrlsub cY\isactrlesub \ {\isacharparenleft}{\kern0pt}R{\isacharcomma}{\kern0pt}m{\isacharparenright}{\kern0pt}{\isacharparenright}{\kern0pt}{\isachardoublequoteclose}\isanewline
\ \ \ \ \ \ \isacommand{using}\isamarkupfalse%
\ assms\ functional{\isacharunderscore}{\kern0pt}on{\isacharunderscore}{\kern0pt}def\ \ \isacommand{by}\isamarkupfalse%
\ force\isanewline
\ \ \ \ \isacommand{then}\isamarkupfalse%
\ \isacommand{show}\isamarkupfalse%
\ {\isachardoublequoteopen}{\isasymexists}z{\isachardot}{\kern0pt}\ z\ {\isasymin}\isactrlsub c\ domain\ {\isacharparenleft}{\kern0pt}left{\isacharunderscore}{\kern0pt}cart{\isacharunderscore}{\kern0pt}proj\ X\ Y\ {\isasymcirc}\isactrlsub c\ m{\isacharparenright}{\kern0pt}\ {\isasymand}\ {\isacharparenleft}{\kern0pt}left{\isacharunderscore}{\kern0pt}cart{\isacharunderscore}{\kern0pt}proj\ X\ Y\ {\isasymcirc}\isactrlsub c\ m{\isacharparenright}{\kern0pt}\ {\isasymcirc}\isactrlsub c\ z\ {\isacharequal}{\kern0pt}\ x{\isachardoublequoteclose}\isanewline
\ \ \ \ \ \ \isacommand{by}\isamarkupfalse%
\ {\isacharparenleft}{\kern0pt}typecheck{\isacharunderscore}{\kern0pt}cfuncs{\isacharcomma}{\kern0pt}\ smt\ {\isacharparenleft}{\kern0pt}verit{\isacharcomma}{\kern0pt}\ best{\isacharparenright}{\kern0pt}\ cfunc{\isacharunderscore}{\kern0pt}type{\isacharunderscore}{\kern0pt}def\ comp{\isacharunderscore}{\kern0pt}associative\ factors{\isacharunderscore}{\kern0pt}through{\isacharunderscore}{\kern0pt}def{\isadigit{2}}\ left{\isacharunderscore}{\kern0pt}cart{\isacharunderscore}{\kern0pt}proj{\isacharunderscore}{\kern0pt}cfunc{\isacharunderscore}{\kern0pt}prod\ relative{\isacharunderscore}{\kern0pt}member{\isacharunderscore}{\kern0pt}def{\isadigit{2}}{\isacharparenright}{\kern0pt}\isanewline
\ \ \isacommand{qed}\isamarkupfalse%
\isanewline
\ \ \isacommand{have}\isamarkupfalse%
\ inj{\isacharcolon}{\kern0pt}\ {\isachardoublequoteopen}injective{\isacharparenleft}{\kern0pt}left{\isacharunderscore}{\kern0pt}cart{\isacharunderscore}{\kern0pt}proj\ X\ Y\ {\isasymcirc}\isactrlsub c\ m{\isacharparenright}{\kern0pt}{\isachardoublequoteclose}\isanewline
\ \ \isacommand{proof}\isamarkupfalse%
{\isacharparenleft}{\kern0pt}unfold\ injective{\isacharunderscore}{\kern0pt}def{\isacharcomma}{\kern0pt}\ clarify{\isacharparenright}{\kern0pt}\isanewline
\ \ \ \ \isacommand{fix}\isamarkupfalse%
\ r{\isadigit{1}}\ r{\isadigit{2}}\ \isanewline
\ \ \ \ \isacommand{assume}\isamarkupfalse%
\ {\isachardoublequoteopen}r{\isadigit{1}}\ {\isasymin}\isactrlsub c\ domain\ {\isacharparenleft}{\kern0pt}left{\isacharunderscore}{\kern0pt}cart{\isacharunderscore}{\kern0pt}proj\ X\ Y\ {\isasymcirc}\isactrlsub c\ m{\isacharparenright}{\kern0pt}{\isachardoublequoteclose}\ \isacommand{then}\isamarkupfalse%
\ \isacommand{have}\isamarkupfalse%
\ r{\isadigit{1}}{\isacharunderscore}{\kern0pt}type{\isacharbrackleft}{\kern0pt}type{\isacharunderscore}{\kern0pt}rule{\isacharbrackright}{\kern0pt}{\isacharcolon}{\kern0pt}\ {\isachardoublequoteopen}r{\isadigit{1}}\ {\isasymin}\isactrlsub c\ R{\isachardoublequoteclose}\isanewline
\ \ \ \ \ \ \isacommand{by}\isamarkupfalse%
\ {\isacharparenleft}{\kern0pt}metis\ cfunc{\isacharunderscore}{\kern0pt}type{\isacharunderscore}{\kern0pt}def\ pi{\isadigit{0}}{\isacharunderscore}{\kern0pt}m{\isacharunderscore}{\kern0pt}type{\isacharparenright}{\kern0pt}\isanewline
\ \ \ \ \isacommand{assume}\isamarkupfalse%
\ {\isachardoublequoteopen}r{\isadigit{2}}\ {\isasymin}\isactrlsub c\ domain\ {\isacharparenleft}{\kern0pt}left{\isacharunderscore}{\kern0pt}cart{\isacharunderscore}{\kern0pt}proj\ X\ Y\ {\isasymcirc}\isactrlsub c\ m{\isacharparenright}{\kern0pt}{\isachardoublequoteclose}\ \isacommand{then}\isamarkupfalse%
\ \isacommand{have}\isamarkupfalse%
\ r{\isadigit{2}}{\isacharunderscore}{\kern0pt}type{\isacharbrackleft}{\kern0pt}type{\isacharunderscore}{\kern0pt}rule{\isacharbrackright}{\kern0pt}{\isacharcolon}{\kern0pt}\ {\isachardoublequoteopen}r{\isadigit{2}}\ {\isasymin}\isactrlsub c\ R{\isachardoublequoteclose}\isanewline
\ \ \ \ \ \ \isacommand{by}\isamarkupfalse%
\ {\isacharparenleft}{\kern0pt}metis\ cfunc{\isacharunderscore}{\kern0pt}type{\isacharunderscore}{\kern0pt}def\ pi{\isadigit{0}}{\isacharunderscore}{\kern0pt}m{\isacharunderscore}{\kern0pt}type{\isacharparenright}{\kern0pt}\isanewline
\ \ \ \ \isacommand{assume}\isamarkupfalse%
\ {\isachardoublequoteopen}{\isacharparenleft}{\kern0pt}left{\isacharunderscore}{\kern0pt}cart{\isacharunderscore}{\kern0pt}proj\ X\ Y\ {\isasymcirc}\isactrlsub c\ m{\isacharparenright}{\kern0pt}\ {\isasymcirc}\isactrlsub c\ r{\isadigit{1}}\ {\isacharequal}{\kern0pt}\ {\isacharparenleft}{\kern0pt}left{\isacharunderscore}{\kern0pt}cart{\isacharunderscore}{\kern0pt}proj\ X\ Y\ {\isasymcirc}\isactrlsub c\ m{\isacharparenright}{\kern0pt}\ {\isasymcirc}\isactrlsub c\ r{\isadigit{2}}{\isachardoublequoteclose}\isanewline
\ \ \ \ \isacommand{then}\isamarkupfalse%
\ \isacommand{have}\isamarkupfalse%
\ eq{\isacharcolon}{\kern0pt}\ {\isachardoublequoteopen}left{\isacharunderscore}{\kern0pt}cart{\isacharunderscore}{\kern0pt}proj\ X\ Y\ {\isasymcirc}\isactrlsub c\ m\ {\isasymcirc}\isactrlsub c\ r{\isadigit{1}}\ {\isacharequal}{\kern0pt}\ left{\isacharunderscore}{\kern0pt}cart{\isacharunderscore}{\kern0pt}proj\ X\ Y\ {\isasymcirc}\isactrlsub c\ m\ {\isasymcirc}\isactrlsub c\ r{\isadigit{2}}{\isachardoublequoteclose}\isanewline
\ \ \ \ \ \ \isacommand{using}\isamarkupfalse%
\ assms\ cfunc{\isacharunderscore}{\kern0pt}type{\isacharunderscore}{\kern0pt}def\ comp{\isacharunderscore}{\kern0pt}associative\ functional{\isacharunderscore}{\kern0pt}on{\isacharunderscore}{\kern0pt}def\ subobject{\isacharunderscore}{\kern0pt}of{\isacharunderscore}{\kern0pt}def{\isadigit{2}}\ \isacommand{by}\isamarkupfalse%
\ {\isacharparenleft}{\kern0pt}typecheck{\isacharunderscore}{\kern0pt}cfuncs{\isacharcomma}{\kern0pt}\ auto{\isacharparenright}{\kern0pt}\isanewline
\ \ \ \ \isacommand{have}\isamarkupfalse%
\ mx{\isacharunderscore}{\kern0pt}type{\isacharbrackleft}{\kern0pt}type{\isacharunderscore}{\kern0pt}rule{\isacharbrackright}{\kern0pt}{\isacharcolon}{\kern0pt}\ {\isachardoublequoteopen}m\ {\isasymcirc}\isactrlsub c\ r{\isadigit{1}}\ {\isasymin}\isactrlsub c\ X{\isasymtimes}\isactrlsub cY{\isachardoublequoteclose}\isanewline
\ \ \ \ \ \ \isacommand{using}\isamarkupfalse%
\ assms\ functional{\isacharunderscore}{\kern0pt}on{\isacharunderscore}{\kern0pt}def\ subobject{\isacharunderscore}{\kern0pt}of{\isacharunderscore}{\kern0pt}def{\isadigit{2}}\ \isacommand{by}\isamarkupfalse%
\ {\isacharparenleft}{\kern0pt}typecheck{\isacharunderscore}{\kern0pt}cfuncs{\isacharcomma}{\kern0pt}\ blast{\isacharparenright}{\kern0pt}\isanewline
\ \ \ \ \isacommand{then}\isamarkupfalse%
\ \isacommand{obtain}\isamarkupfalse%
\ x{\isadigit{1}}\ \isakeyword{and}\ y{\isadigit{1}}\ \isakeyword{where}\ m{\isadigit{1}}r{\isadigit{1}}{\isacharunderscore}{\kern0pt}eqs{\isacharcolon}{\kern0pt}\ {\isachardoublequoteopen}m\ {\isasymcirc}\isactrlsub c\ r{\isadigit{1}}\ {\isacharequal}{\kern0pt}\ {\isasymlangle}x{\isadigit{1}}{\isacharcomma}{\kern0pt}\ y{\isadigit{1}}{\isasymrangle}\ {\isasymand}\ x{\isadigit{1}}\ {\isasymin}\isactrlsub c\ X\ {\isasymand}\ y{\isadigit{1}}\ {\isasymin}\isactrlsub c\ Y{\isachardoublequoteclose}\isanewline
\ \ \ \ \ \ \isacommand{using}\isamarkupfalse%
\ cart{\isacharunderscore}{\kern0pt}prod{\isacharunderscore}{\kern0pt}decomp\ \isacommand{by}\isamarkupfalse%
\ presburger\isanewline
\ \ \ \ \isacommand{have}\isamarkupfalse%
\ my{\isacharunderscore}{\kern0pt}type{\isacharbrackleft}{\kern0pt}type{\isacharunderscore}{\kern0pt}rule{\isacharbrackright}{\kern0pt}{\isacharcolon}{\kern0pt}\ {\isachardoublequoteopen}m\ {\isasymcirc}\isactrlsub c\ r{\isadigit{2}}\ {\isasymin}\isactrlsub c\ X{\isasymtimes}\isactrlsub cY{\isachardoublequoteclose}\isanewline
\ \ \ \ \ \ \isacommand{using}\isamarkupfalse%
\ assms\ functional{\isacharunderscore}{\kern0pt}on{\isacharunderscore}{\kern0pt}def\ subobject{\isacharunderscore}{\kern0pt}of{\isacharunderscore}{\kern0pt}def{\isadigit{2}}\ \isacommand{by}\isamarkupfalse%
\ {\isacharparenleft}{\kern0pt}typecheck{\isacharunderscore}{\kern0pt}cfuncs{\isacharcomma}{\kern0pt}\ blast{\isacharparenright}{\kern0pt}\isanewline
\ \ \ \ \isacommand{then}\isamarkupfalse%
\ \isacommand{obtain}\isamarkupfalse%
\ x{\isadigit{2}}\ \isakeyword{and}\ y{\isadigit{2}}\ \isakeyword{where}\ m{\isadigit{2}}r{\isadigit{2}}{\isacharunderscore}{\kern0pt}eqs{\isacharcolon}{\kern0pt}{\isachardoublequoteopen}m\ {\isasymcirc}\isactrlsub c\ r{\isadigit{2}}\ {\isacharequal}{\kern0pt}\ {\isasymlangle}x{\isadigit{2}}{\isacharcomma}{\kern0pt}\ y{\isadigit{2}}{\isasymrangle}\ {\isasymand}\ x{\isadigit{2}}\ {\isasymin}\isactrlsub c\ X\ {\isasymand}\ y{\isadigit{2}}\ {\isasymin}\isactrlsub c\ Y{\isachardoublequoteclose}\isanewline
\ \ \ \ \ \ \isacommand{using}\isamarkupfalse%
\ cart{\isacharunderscore}{\kern0pt}prod{\isacharunderscore}{\kern0pt}decomp\ \isacommand{by}\isamarkupfalse%
\ presburger\isanewline
\ \ \ \ \isacommand{have}\isamarkupfalse%
\ x{\isacharunderscore}{\kern0pt}equal{\isacharcolon}{\kern0pt}\ {\isachardoublequoteopen}x{\isadigit{1}}\ {\isacharequal}{\kern0pt}\ x{\isadigit{2}}{\isachardoublequoteclose}\isanewline
\ \ \ \ \ \ \isacommand{using}\isamarkupfalse%
\ eq\ left{\isacharunderscore}{\kern0pt}cart{\isacharunderscore}{\kern0pt}proj{\isacharunderscore}{\kern0pt}cfunc{\isacharunderscore}{\kern0pt}prod\ m{\isadigit{1}}r{\isadigit{1}}{\isacharunderscore}{\kern0pt}eqs\ m{\isadigit{2}}r{\isadigit{2}}{\isacharunderscore}{\kern0pt}eqs\ \isacommand{by}\isamarkupfalse%
\ force\isanewline
\ \ \ \ \isacommand{have}\isamarkupfalse%
\ functional{\isacharcolon}{\kern0pt}\ {\isachardoublequoteopen}{\isasymexists}{\isacharbang}{\kern0pt}\ y{\isachardot}{\kern0pt}\ {\isacharparenleft}{\kern0pt}y\ {\isasymin}\isactrlsub c\ Y\ {\isasymand}\ \ {\isasymlangle}x{\isadigit{1}}{\isacharcomma}{\kern0pt}y{\isasymrangle}\ {\isasymin}\isactrlbsub X{\isasymtimes}\isactrlsub cY\isactrlesub \ {\isacharparenleft}{\kern0pt}R{\isacharcomma}{\kern0pt}m{\isacharparenright}{\kern0pt}{\isacharparenright}{\kern0pt}{\isachardoublequoteclose}\isanewline
\ \ \ \ \ \ \isacommand{using}\isamarkupfalse%
\ assms\ functional{\isacharunderscore}{\kern0pt}on{\isacharunderscore}{\kern0pt}def\ m{\isadigit{1}}r{\isadigit{1}}{\isacharunderscore}{\kern0pt}eqs\ \isacommand{by}\isamarkupfalse%
\ force\isanewline
\ \ \ \ \isacommand{then}\isamarkupfalse%
\ \isacommand{have}\isamarkupfalse%
\ y{\isacharunderscore}{\kern0pt}equal{\isacharcolon}{\kern0pt}\ {\isachardoublequoteopen}y{\isadigit{1}}\ {\isacharequal}{\kern0pt}\ y{\isadigit{2}}{\isachardoublequoteclose}\isanewline
\ \ \ \ \ \ \isacommand{by}\isamarkupfalse%
\ {\isacharparenleft}{\kern0pt}metis\ prod{\isachardot}{\kern0pt}sel\ factors{\isacharunderscore}{\kern0pt}through{\isacharunderscore}{\kern0pt}def{\isadigit{2}}\ m{\isadigit{1}}r{\isadigit{1}}{\isacharunderscore}{\kern0pt}eqs\ m{\isadigit{2}}r{\isadigit{2}}{\isacharunderscore}{\kern0pt}eqs\ mx{\isacharunderscore}{\kern0pt}type\ my{\isacharunderscore}{\kern0pt}type\ r{\isadigit{1}}{\isacharunderscore}{\kern0pt}type\ r{\isadigit{2}}{\isacharunderscore}{\kern0pt}type\ relative{\isacharunderscore}{\kern0pt}member{\isacharunderscore}{\kern0pt}def\ x{\isacharunderscore}{\kern0pt}equal{\isacharparenright}{\kern0pt}\isanewline
\ \ \ \ \isacommand{then}\isamarkupfalse%
\ \isacommand{show}\isamarkupfalse%
\ {\isachardoublequoteopen}r{\isadigit{1}}\ {\isacharequal}{\kern0pt}\ r{\isadigit{2}}{\isachardoublequoteclose}\isanewline
\ \ \ \ \ \ \isacommand{by}\isamarkupfalse%
\ {\isacharparenleft}{\kern0pt}metis\ functional\ cfunc{\isacharunderscore}{\kern0pt}type{\isacharunderscore}{\kern0pt}def\ m{\isadigit{1}}r{\isadigit{1}}{\isacharunderscore}{\kern0pt}eqs\ m{\isadigit{2}}r{\isadigit{2}}{\isacharunderscore}{\kern0pt}eqs\ monomorphism{\isacharunderscore}{\kern0pt}def\ r{\isadigit{1}}{\isacharunderscore}{\kern0pt}type\ r{\isadigit{2}}{\isacharunderscore}{\kern0pt}type\ relative{\isacharunderscore}{\kern0pt}member{\isacharunderscore}{\kern0pt}def{\isadigit{2}}\ x{\isacharunderscore}{\kern0pt}equal{\isacharparenright}{\kern0pt}\isanewline
\ \ \isacommand{qed}\isamarkupfalse%
\isanewline
\ \ \isacommand{show}\isamarkupfalse%
\ {\isachardoublequoteopen}isomorphism{\isacharparenleft}{\kern0pt}left{\isacharunderscore}{\kern0pt}cart{\isacharunderscore}{\kern0pt}proj\ X\ Y\ {\isasymcirc}\isactrlsub c\ m{\isacharparenright}{\kern0pt}{\isachardoublequoteclose}\isanewline
\ \ \ \ \isacommand{by}\isamarkupfalse%
\ {\isacharparenleft}{\kern0pt}metis\ epi{\isacharunderscore}{\kern0pt}mon{\isacharunderscore}{\kern0pt}is{\isacharunderscore}{\kern0pt}iso\ inj\ injective{\isacharunderscore}{\kern0pt}imp{\isacharunderscore}{\kern0pt}monomorphism\ surj\ surjective{\isacharunderscore}{\kern0pt}is{\isacharunderscore}{\kern0pt}epimorphism{\isacharparenright}{\kern0pt}\isanewline
\isacommand{qed}\isamarkupfalse%
%
\endisatagproof
{\isafoldproof}%
%
\isadelimproof
%
\endisadelimproof
%
\begin{isamarkuptext}%
The lemma below corresponds to Proposition 2.3.14 in Halvorson.%
\end{isamarkuptext}\isamarkuptrue%
\isacommand{lemma}\isamarkupfalse%
\ functional{\isacharunderscore}{\kern0pt}relations{\isacharunderscore}{\kern0pt}are{\isacharunderscore}{\kern0pt}graphs{\isacharcolon}{\kern0pt}\isanewline
\ \ \isakeyword{assumes}\ {\isachardoublequoteopen}functional{\isacharunderscore}{\kern0pt}on\ X\ Y\ {\isacharparenleft}{\kern0pt}R{\isacharcomma}{\kern0pt}m{\isacharparenright}{\kern0pt}{\isachardoublequoteclose}\isanewline
\ \ \isakeyword{shows}\ {\isachardoublequoteopen}{\isasymexists}{\isacharbang}{\kern0pt}\ f{\isachardot}{\kern0pt}\ f\ {\isacharcolon}{\kern0pt}\ X\ {\isasymrightarrow}\ Y\ {\isasymand}\ \isanewline
\ \ \ \ {\isacharparenleft}{\kern0pt}{\isasymexists}\ i{\isachardot}{\kern0pt}\ i\ {\isacharcolon}{\kern0pt}\ R\ {\isasymrightarrow}\ graph{\isacharparenleft}{\kern0pt}f{\isacharparenright}{\kern0pt}\ {\isasymand}\ isomorphism{\isacharparenleft}{\kern0pt}i{\isacharparenright}{\kern0pt}\ {\isasymand}\ m\ {\isacharequal}{\kern0pt}\ graph{\isacharunderscore}{\kern0pt}morph{\isacharparenleft}{\kern0pt}f{\isacharparenright}{\kern0pt}\ {\isasymcirc}\isactrlsub c\ i{\isacharparenright}{\kern0pt}{\isachardoublequoteclose}\isanewline
%
\isadelimproof
%
\endisadelimproof
%
\isatagproof
\isacommand{proof}\isamarkupfalse%
\ safe\isanewline
\ \ \isacommand{have}\isamarkupfalse%
\ m{\isacharunderscore}{\kern0pt}type{\isacharbrackleft}{\kern0pt}type{\isacharunderscore}{\kern0pt}rule{\isacharbrackright}{\kern0pt}{\isacharcolon}{\kern0pt}\ {\isachardoublequoteopen}m\ {\isacharcolon}{\kern0pt}\ R\ {\isasymrightarrow}\ X\ {\isasymtimes}\isactrlsub c\ Y{\isachardoublequoteclose}\isanewline
\ \ \ \ \isacommand{using}\isamarkupfalse%
\ assms\ \isacommand{unfolding}\isamarkupfalse%
\ functional{\isacharunderscore}{\kern0pt}on{\isacharunderscore}{\kern0pt}def\ subobject{\isacharunderscore}{\kern0pt}of{\isacharunderscore}{\kern0pt}def{\isadigit{2}}\ \isacommand{by}\isamarkupfalse%
\ auto\isanewline
\ \ \isacommand{have}\isamarkupfalse%
\ m{\isacharunderscore}{\kern0pt}mono{\isacharbrackleft}{\kern0pt}type{\isacharunderscore}{\kern0pt}rule{\isacharbrackright}{\kern0pt}{\isacharcolon}{\kern0pt}\ {\isachardoublequoteopen}monomorphism{\isacharparenleft}{\kern0pt}m{\isacharparenright}{\kern0pt}{\isachardoublequoteclose}\isanewline
\ \ \ \ \isacommand{using}\isamarkupfalse%
\ assms\ functional{\isacharunderscore}{\kern0pt}on{\isacharunderscore}{\kern0pt}def\ subobject{\isacharunderscore}{\kern0pt}of{\isacharunderscore}{\kern0pt}def{\isadigit{2}}\ \isacommand{by}\isamarkupfalse%
\ blast\isanewline
\ \ \isacommand{have}\isamarkupfalse%
\ isomorphism{\isacharbrackleft}{\kern0pt}type{\isacharunderscore}{\kern0pt}rule{\isacharbrackright}{\kern0pt}{\isacharcolon}{\kern0pt}\ {\isachardoublequoteopen}isomorphism{\isacharparenleft}{\kern0pt}left{\isacharunderscore}{\kern0pt}cart{\isacharunderscore}{\kern0pt}proj\ X\ Y\ {\isasymcirc}\isactrlsub c\ m{\isacharparenright}{\kern0pt}{\isachardoublequoteclose}\isanewline
\ \ \ \ \isacommand{using}\isamarkupfalse%
\ assms\ functional{\isacharunderscore}{\kern0pt}on{\isacharunderscore}{\kern0pt}isomorphism\ \isacommand{by}\isamarkupfalse%
\ force\isanewline
\ \ \isanewline
\ \ \isacommand{obtain}\isamarkupfalse%
\ h\ \isakeyword{where}\ h{\isacharunderscore}{\kern0pt}type{\isacharbrackleft}{\kern0pt}type{\isacharunderscore}{\kern0pt}rule{\isacharbrackright}{\kern0pt}{\isacharcolon}{\kern0pt}\ {\isachardoublequoteopen}h{\isacharcolon}{\kern0pt}\ X\ {\isasymrightarrow}\ R{\isachardoublequoteclose}\ \isakeyword{and}\ h{\isacharunderscore}{\kern0pt}def{\isacharcolon}{\kern0pt}\ {\isachardoublequoteopen}h\ {\isacharequal}{\kern0pt}\ {\isacharparenleft}{\kern0pt}left{\isacharunderscore}{\kern0pt}cart{\isacharunderscore}{\kern0pt}proj\ X\ Y\ {\isasymcirc}\isactrlsub c\ m{\isacharparenright}{\kern0pt}\isactrlbold {\isasyminverse}{\isachardoublequoteclose}\isanewline
\ \ \ \ \isacommand{by}\isamarkupfalse%
\ {\isacharparenleft}{\kern0pt}typecheck{\isacharunderscore}{\kern0pt}cfuncs{\isacharcomma}{\kern0pt}\ simp{\isacharparenright}{\kern0pt}\isanewline
\ \ \isacommand{obtain}\isamarkupfalse%
\ f\ \isakeyword{where}\ f{\isacharunderscore}{\kern0pt}def{\isacharcolon}{\kern0pt}\ {\isachardoublequoteopen}f\ {\isacharequal}{\kern0pt}\ {\isacharparenleft}{\kern0pt}right{\isacharunderscore}{\kern0pt}cart{\isacharunderscore}{\kern0pt}proj\ X\ Y{\isacharparenright}{\kern0pt}\ {\isasymcirc}\isactrlsub c\ m\ {\isasymcirc}\isactrlsub c\ h{\isachardoublequoteclose}\isanewline
\ \ \ \ \isacommand{by}\isamarkupfalse%
\ auto\isanewline
\ \ \isacommand{then}\isamarkupfalse%
\ \isacommand{have}\isamarkupfalse%
\ f{\isacharunderscore}{\kern0pt}type{\isacharbrackleft}{\kern0pt}type{\isacharunderscore}{\kern0pt}rule{\isacharbrackright}{\kern0pt}{\isacharcolon}{\kern0pt}\ {\isachardoublequoteopen}f\ {\isacharcolon}{\kern0pt}\ X\ {\isasymrightarrow}\ Y{\isachardoublequoteclose}\isanewline
\ \ \ \ \isacommand{by}\isamarkupfalse%
\ {\isacharparenleft}{\kern0pt}metis\ assms\ comp{\isacharunderscore}{\kern0pt}type\ f{\isacharunderscore}{\kern0pt}def\ functional{\isacharunderscore}{\kern0pt}on{\isacharunderscore}{\kern0pt}def\ h{\isacharunderscore}{\kern0pt}type\ right{\isacharunderscore}{\kern0pt}cart{\isacharunderscore}{\kern0pt}proj{\isacharunderscore}{\kern0pt}type\ subobject{\isacharunderscore}{\kern0pt}of{\isacharunderscore}{\kern0pt}def{\isadigit{2}}{\isacharparenright}{\kern0pt}\isanewline
\isanewline
\ \ \isacommand{have}\isamarkupfalse%
\ eq{\isacharcolon}{\kern0pt}\ {\isachardoublequoteopen}f\ {\isasymcirc}\isactrlsub c\ left{\isacharunderscore}{\kern0pt}cart{\isacharunderscore}{\kern0pt}proj\ X\ Y\ {\isasymcirc}\isactrlsub c\ m\ {\isacharequal}{\kern0pt}\ right{\isacharunderscore}{\kern0pt}cart{\isacharunderscore}{\kern0pt}proj\ X\ Y\ {\isasymcirc}\isactrlsub c\ m{\isachardoublequoteclose}\isanewline
\ \ \ \ \isacommand{unfolding}\isamarkupfalse%
\ f{\isacharunderscore}{\kern0pt}def\ h{\isacharunderscore}{\kern0pt}def\ \isacommand{by}\isamarkupfalse%
\ {\isacharparenleft}{\kern0pt}typecheck{\isacharunderscore}{\kern0pt}cfuncs{\isacharcomma}{\kern0pt}\ smt\ comp{\isacharunderscore}{\kern0pt}associative{\isadigit{2}}\ id{\isacharunderscore}{\kern0pt}right{\isacharunderscore}{\kern0pt}unit{\isadigit{2}}\ inv{\isacharunderscore}{\kern0pt}left\ isomorphism{\isacharparenright}{\kern0pt}\isanewline
\isanewline
\ \ \isacommand{show}\isamarkupfalse%
\ {\isachardoublequoteopen}{\isasymexists}f{\isachardot}{\kern0pt}\ f\ {\isacharcolon}{\kern0pt}\ X\ {\isasymrightarrow}\ Y\ {\isasymand}\ {\isacharparenleft}{\kern0pt}{\isasymexists}i{\isachardot}{\kern0pt}\ i\ {\isacharcolon}{\kern0pt}\ R\ {\isasymrightarrow}\ graph\ f\ {\isasymand}\ isomorphism\ i\ {\isasymand}\ m\ {\isacharequal}{\kern0pt}\ graph{\isacharunderscore}{\kern0pt}morph\ f\ {\isasymcirc}\isactrlsub c\ i{\isacharparenright}{\kern0pt}{\isachardoublequoteclose}\isanewline
\ \ \isacommand{proof}\isamarkupfalse%
\ {\isacharparenleft}{\kern0pt}rule{\isacharunderscore}{\kern0pt}tac\ x{\isacharequal}{\kern0pt}f\ \isakeyword{in}\ exI{\isacharcomma}{\kern0pt}\ safe{\isacharcomma}{\kern0pt}\ typecheck{\isacharunderscore}{\kern0pt}cfuncs{\isacharparenright}{\kern0pt}\isanewline
\ \ \ \ \isacommand{have}\isamarkupfalse%
\ graph{\isacharunderscore}{\kern0pt}equalizer{\isacharcolon}{\kern0pt}\ {\isachardoublequoteopen}equalizer\ {\isacharparenleft}{\kern0pt}graph\ f{\isacharparenright}{\kern0pt}\ {\isacharparenleft}{\kern0pt}graph{\isacharunderscore}{\kern0pt}morph\ f{\isacharparenright}{\kern0pt}\ {\isacharparenleft}{\kern0pt}f\ {\isasymcirc}\isactrlsub c\ left{\isacharunderscore}{\kern0pt}cart{\isacharunderscore}{\kern0pt}proj\ X\ Y{\isacharparenright}{\kern0pt}\ {\isacharparenleft}{\kern0pt}right{\isacharunderscore}{\kern0pt}cart{\isacharunderscore}{\kern0pt}proj\ X\ Y{\isacharparenright}{\kern0pt}{\isachardoublequoteclose}\isanewline
\ \ \ \ \ \ \isacommand{by}\isamarkupfalse%
\ {\isacharparenleft}{\kern0pt}simp\ add{\isacharcolon}{\kern0pt}\ f{\isacharunderscore}{\kern0pt}type\ graph{\isacharunderscore}{\kern0pt}equalizer{\isadigit{4}}{\isacharparenright}{\kern0pt}\isanewline
\ \ \ \ \isacommand{then}\isamarkupfalse%
\ \isacommand{have}\isamarkupfalse%
\ {\isachardoublequoteopen}{\isasymforall}h\ F{\isachardot}{\kern0pt}\ h\ {\isacharcolon}{\kern0pt}\ F\ {\isasymrightarrow}\ X\ {\isasymtimes}\isactrlsub c\ Y\ {\isasymand}\ {\isacharparenleft}{\kern0pt}f\ {\isasymcirc}\isactrlsub c\ left{\isacharunderscore}{\kern0pt}cart{\isacharunderscore}{\kern0pt}proj\ X\ Y{\isacharparenright}{\kern0pt}\ {\isasymcirc}\isactrlsub c\ h\ {\isacharequal}{\kern0pt}\ right{\isacharunderscore}{\kern0pt}cart{\isacharunderscore}{\kern0pt}proj\ X\ Y\ {\isasymcirc}\isactrlsub c\ h\ {\isasymlongrightarrow}\isanewline
\ \ \ \ \ \ \ \ \ \ {\isacharparenleft}{\kern0pt}{\isasymexists}{\isacharbang}{\kern0pt}k{\isachardot}{\kern0pt}\ k\ {\isacharcolon}{\kern0pt}\ F\ {\isasymrightarrow}\ graph\ f\ {\isasymand}\ graph{\isacharunderscore}{\kern0pt}morph\ f\ {\isasymcirc}\isactrlsub c\ k\ {\isacharequal}{\kern0pt}\ h{\isacharparenright}{\kern0pt}{\isachardoublequoteclose}\isanewline
\ \ \ \ \ \ \isacommand{unfolding}\isamarkupfalse%
\ equalizer{\isacharunderscore}{\kern0pt}def\ \isacommand{using}\isamarkupfalse%
\ cfunc{\isacharunderscore}{\kern0pt}type{\isacharunderscore}{\kern0pt}def\ \isacommand{by}\isamarkupfalse%
\ {\isacharparenleft}{\kern0pt}typecheck{\isacharunderscore}{\kern0pt}cfuncs{\isacharcomma}{\kern0pt}\ auto{\isacharparenright}{\kern0pt}\isanewline
\ \ \ \ \isacommand{then}\isamarkupfalse%
\ \isacommand{obtain}\isamarkupfalse%
\ i\ \isakeyword{where}\ i{\isacharunderscore}{\kern0pt}type{\isacharbrackleft}{\kern0pt}type{\isacharunderscore}{\kern0pt}rule{\isacharbrackright}{\kern0pt}{\isacharcolon}{\kern0pt}\ {\isachardoublequoteopen}i\ {\isacharcolon}{\kern0pt}\ R\ {\isasymrightarrow}\ graph\ f{\isachardoublequoteclose}\ \isakeyword{and}\ i{\isacharunderscore}{\kern0pt}eq{\isacharcolon}{\kern0pt}\ {\isachardoublequoteopen}graph{\isacharunderscore}{\kern0pt}morph\ f\ {\isasymcirc}\isactrlsub c\ i\ {\isacharequal}{\kern0pt}\ m{\isachardoublequoteclose}\isanewline
\ \ \ \ \ \ \isacommand{by}\isamarkupfalse%
\ {\isacharparenleft}{\kern0pt}typecheck{\isacharunderscore}{\kern0pt}cfuncs{\isacharcomma}{\kern0pt}\ smt\ comp{\isacharunderscore}{\kern0pt}associative{\isadigit{2}}\ eq\ left{\isacharunderscore}{\kern0pt}cart{\isacharunderscore}{\kern0pt}proj{\isacharunderscore}{\kern0pt}type{\isacharparenright}{\kern0pt}\isanewline
\ \ \ \ \isacommand{have}\isamarkupfalse%
\ {\isachardoublequoteopen}surjective\ i{\isachardoublequoteclose}\isanewline
\ \ \ \ \isacommand{proof}\isamarkupfalse%
\ {\isacharparenleft}{\kern0pt}etcs{\isacharunderscore}{\kern0pt}subst\ surjective{\isacharunderscore}{\kern0pt}def{\isadigit{2}}{\isacharcomma}{\kern0pt}\ clarify{\isacharparenright}{\kern0pt}\isanewline
\ \ \ \ \ \ \isacommand{fix}\isamarkupfalse%
\ y{\isacharprime}{\kern0pt}\isanewline
\ \ \ \ \ \ \isacommand{assume}\isamarkupfalse%
\ y{\isacharprime}{\kern0pt}{\isacharunderscore}{\kern0pt}type{\isacharbrackleft}{\kern0pt}type{\isacharunderscore}{\kern0pt}rule{\isacharbrackright}{\kern0pt}{\isacharcolon}{\kern0pt}\ {\isachardoublequoteopen}y{\isacharprime}{\kern0pt}\ {\isasymin}\isactrlsub c\ graph\ f{\isachardoublequoteclose}\isanewline
\isanewline
\ \ \ \ \ \ \isacommand{define}\isamarkupfalse%
\ x\ \isakeyword{where}\ {\isachardoublequoteopen}x\ {\isacharequal}{\kern0pt}\ left{\isacharunderscore}{\kern0pt}cart{\isacharunderscore}{\kern0pt}proj\ X\ Y\ {\isasymcirc}\isactrlsub c\ graph{\isacharunderscore}{\kern0pt}morph{\isacharparenleft}{\kern0pt}f{\isacharparenright}{\kern0pt}\ {\isasymcirc}\isactrlsub c\ y{\isacharprime}{\kern0pt}{\isachardoublequoteclose}\isanewline
\ \ \ \ \ \ \isacommand{then}\isamarkupfalse%
\ \isacommand{have}\isamarkupfalse%
\ x{\isacharunderscore}{\kern0pt}type{\isacharbrackleft}{\kern0pt}type{\isacharunderscore}{\kern0pt}rule{\isacharbrackright}{\kern0pt}{\isacharcolon}{\kern0pt}\ {\isachardoublequoteopen}x\ {\isasymin}\isactrlsub c\ X{\isachardoublequoteclose}\isanewline
\ \ \ \ \ \ \ \ \isacommand{unfolding}\isamarkupfalse%
\ x{\isacharunderscore}{\kern0pt}def\ \isacommand{by}\isamarkupfalse%
\ typecheck{\isacharunderscore}{\kern0pt}cfuncs\isanewline
\isanewline
\ \ \ \ \ \ \isacommand{obtain}\isamarkupfalse%
\ y\ \isakeyword{where}\ y{\isacharunderscore}{\kern0pt}type{\isacharbrackleft}{\kern0pt}type{\isacharunderscore}{\kern0pt}rule{\isacharbrackright}{\kern0pt}{\isacharcolon}{\kern0pt}\ {\isachardoublequoteopen}y\ {\isasymin}\isactrlsub c\ Y{\isachardoublequoteclose}\ \isakeyword{and}\ x{\isacharunderscore}{\kern0pt}y{\isacharunderscore}{\kern0pt}in{\isacharunderscore}{\kern0pt}R{\isacharcolon}{\kern0pt}\ {\isachardoublequoteopen}{\isasymlangle}x{\isacharcomma}{\kern0pt}y{\isasymrangle}\ {\isasymin}\isactrlbsub X\ {\isasymtimes}\isactrlsub c\ Y\isactrlesub \ {\isacharparenleft}{\kern0pt}R{\isacharcomma}{\kern0pt}\ m{\isacharparenright}{\kern0pt}{\isachardoublequoteclose}\isanewline
\ \ \ \ \ \ \ \ \isakeyword{and}\ y{\isacharunderscore}{\kern0pt}unique{\isacharcolon}{\kern0pt}\ {\isachardoublequoteopen}{\isasymforall}\ z{\isachardot}{\kern0pt}\ {\isacharparenleft}{\kern0pt}z\ {\isasymin}\isactrlsub c\ Y\ {\isasymand}\ {\isasymlangle}x{\isacharcomma}{\kern0pt}z{\isasymrangle}\ {\isasymin}\isactrlbsub X\ {\isasymtimes}\isactrlsub c\ Y\isactrlesub \ {\isacharparenleft}{\kern0pt}R{\isacharcomma}{\kern0pt}\ m{\isacharparenright}{\kern0pt}{\isacharparenright}{\kern0pt}\ {\isasymlongrightarrow}\ z\ {\isacharequal}{\kern0pt}\ y{\isachardoublequoteclose}\isanewline
\ \ \ \ \ \ \ \ \isacommand{by}\isamarkupfalse%
\ {\isacharparenleft}{\kern0pt}metis\ assms\ functional{\isacharunderscore}{\kern0pt}on{\isacharunderscore}{\kern0pt}def\ x{\isacharunderscore}{\kern0pt}type{\isacharparenright}{\kern0pt}\isanewline
\isanewline
\ \ \ \ \ \ \isacommand{obtain}\isamarkupfalse%
\ x{\isacharprime}{\kern0pt}\ \isakeyword{where}\ x{\isacharprime}{\kern0pt}{\isacharunderscore}{\kern0pt}type{\isacharbrackleft}{\kern0pt}type{\isacharunderscore}{\kern0pt}rule{\isacharbrackright}{\kern0pt}{\isacharcolon}{\kern0pt}\ {\isachardoublequoteopen}x{\isacharprime}{\kern0pt}\ {\isasymin}\isactrlsub c\ R{\isachardoublequoteclose}\ \isakeyword{and}\ x{\isacharprime}{\kern0pt}{\isacharunderscore}{\kern0pt}eq{\isacharcolon}{\kern0pt}\ {\isachardoublequoteopen}m\ {\isasymcirc}\isactrlsub c\ x{\isacharprime}{\kern0pt}\ {\isacharequal}{\kern0pt}\ {\isasymlangle}x{\isacharcomma}{\kern0pt}\ y{\isasymrangle}{\isachardoublequoteclose}\isanewline
\ \ \ \ \ \ \ \ \isacommand{using}\isamarkupfalse%
\ x{\isacharunderscore}{\kern0pt}y{\isacharunderscore}{\kern0pt}in{\isacharunderscore}{\kern0pt}R\ \isacommand{unfolding}\isamarkupfalse%
\ relative{\isacharunderscore}{\kern0pt}member{\isacharunderscore}{\kern0pt}def{\isadigit{2}}\ \isacommand{by}\isamarkupfalse%
\ {\isacharparenleft}{\kern0pt}{\isacharminus}{\kern0pt}{\isacharcomma}{\kern0pt}\ etcs{\isacharunderscore}{\kern0pt}subst{\isacharunderscore}{\kern0pt}asm\ factors{\isacharunderscore}{\kern0pt}through{\isacharunderscore}{\kern0pt}def{\isadigit{2}}{\isacharcomma}{\kern0pt}\ auto{\isacharparenright}{\kern0pt}\isanewline
\isanewline
\ \ \ \ \ \ \isacommand{have}\isamarkupfalse%
\ {\isachardoublequoteopen}graph{\isacharunderscore}{\kern0pt}morph{\isacharparenleft}{\kern0pt}f{\isacharparenright}{\kern0pt}\ {\isasymcirc}\isactrlsub c\ i\ {\isasymcirc}\isactrlsub c\ x{\isacharprime}{\kern0pt}\ {\isacharequal}{\kern0pt}\ graph{\isacharunderscore}{\kern0pt}morph{\isacharparenleft}{\kern0pt}f{\isacharparenright}{\kern0pt}\ {\isasymcirc}\isactrlsub c\ y{\isacharprime}{\kern0pt}{\isachardoublequoteclose}\isanewline
\ \ \ \ \ \ \isacommand{proof}\isamarkupfalse%
\ {\isacharparenleft}{\kern0pt}typecheck{\isacharunderscore}{\kern0pt}cfuncs{\isacharcomma}{\kern0pt}\ rule\ cart{\isacharunderscore}{\kern0pt}prod{\isacharunderscore}{\kern0pt}eqI{\isacharcomma}{\kern0pt}\ safe{\isacharparenright}{\kern0pt}\isanewline
\ \ \ \ \ \ \ \ \isacommand{show}\isamarkupfalse%
\ left{\isacharcolon}{\kern0pt}\ {\isachardoublequoteopen}left{\isacharunderscore}{\kern0pt}cart{\isacharunderscore}{\kern0pt}proj\ X\ Y\ {\isasymcirc}\isactrlsub c\ graph{\isacharunderscore}{\kern0pt}morph\ f\ {\isasymcirc}\isactrlsub c\ i\ {\isasymcirc}\isactrlsub c\ x{\isacharprime}{\kern0pt}\ {\isacharequal}{\kern0pt}\ left{\isacharunderscore}{\kern0pt}cart{\isacharunderscore}{\kern0pt}proj\ X\ Y\ {\isasymcirc}\isactrlsub c\ graph{\isacharunderscore}{\kern0pt}morph\ f\ {\isasymcirc}\isactrlsub c\ y{\isacharprime}{\kern0pt}{\isachardoublequoteclose}\isanewline
\ \ \ \ \ \ \ \ \isacommand{proof}\isamarkupfalse%
\ {\isacharminus}{\kern0pt}\isanewline
\ \ \ \ \ \ \ \ \ \ \isacommand{have}\isamarkupfalse%
\ {\isachardoublequoteopen}left{\isacharunderscore}{\kern0pt}cart{\isacharunderscore}{\kern0pt}proj\ X\ Y\ {\isasymcirc}\isactrlsub c\ graph{\isacharunderscore}{\kern0pt}morph{\isacharparenleft}{\kern0pt}f{\isacharparenright}{\kern0pt}\ {\isasymcirc}\isactrlsub c\ i\ {\isasymcirc}\isactrlsub c\ x{\isacharprime}{\kern0pt}\ {\isacharequal}{\kern0pt}\ left{\isacharunderscore}{\kern0pt}cart{\isacharunderscore}{\kern0pt}proj\ X\ Y\ {\isasymcirc}\isactrlsub c\ m\ {\isasymcirc}\isactrlsub c\ x{\isacharprime}{\kern0pt}{\isachardoublequoteclose}\isanewline
\ \ \ \ \ \ \ \ \ \ \ \ \isacommand{by}\isamarkupfalse%
\ {\isacharparenleft}{\kern0pt}typecheck{\isacharunderscore}{\kern0pt}cfuncs{\isacharcomma}{\kern0pt}\ smt\ comp{\isacharunderscore}{\kern0pt}associative{\isadigit{2}}\ i{\isacharunderscore}{\kern0pt}eq{\isacharparenright}{\kern0pt}\isanewline
\ \ \ \ \ \ \ \ \ \ \isacommand{also}\isamarkupfalse%
\ \isacommand{have}\isamarkupfalse%
\ {\isachardoublequoteopen}{\isachardot}{\kern0pt}{\isachardot}{\kern0pt}{\isachardot}{\kern0pt}\ {\isacharequal}{\kern0pt}\ x{\isachardoublequoteclose}\isanewline
\ \ \ \ \ \ \ \ \ \ \ \ \isacommand{unfolding}\isamarkupfalse%
\ x{\isacharprime}{\kern0pt}{\isacharunderscore}{\kern0pt}eq\ \isacommand{using}\isamarkupfalse%
\ left{\isacharunderscore}{\kern0pt}cart{\isacharunderscore}{\kern0pt}proj{\isacharunderscore}{\kern0pt}cfunc{\isacharunderscore}{\kern0pt}prod\ \isacommand{by}\isamarkupfalse%
\ {\isacharparenleft}{\kern0pt}typecheck{\isacharunderscore}{\kern0pt}cfuncs{\isacharcomma}{\kern0pt}\ blast{\isacharparenright}{\kern0pt}\isanewline
\ \ \ \ \ \ \ \ \ \ \isacommand{also}\isamarkupfalse%
\ \isacommand{have}\isamarkupfalse%
\ {\isachardoublequoteopen}{\isachardot}{\kern0pt}{\isachardot}{\kern0pt}{\isachardot}{\kern0pt}\ {\isacharequal}{\kern0pt}\ left{\isacharunderscore}{\kern0pt}cart{\isacharunderscore}{\kern0pt}proj\ X\ Y\ {\isasymcirc}\isactrlsub c\ graph{\isacharunderscore}{\kern0pt}morph\ f\ {\isasymcirc}\isactrlsub c\ y{\isacharprime}{\kern0pt}{\isachardoublequoteclose}\isanewline
\ \ \ \ \ \ \ \ \ \ \ \ \isacommand{unfolding}\isamarkupfalse%
\ x{\isacharunderscore}{\kern0pt}def\ \isacommand{by}\isamarkupfalse%
\ auto\isanewline
\ \ \ \ \ \ \ \ \ \ \isacommand{then}\isamarkupfalse%
\ \isacommand{show}\isamarkupfalse%
\ {\isacharquery}{\kern0pt}thesis\ \isacommand{using}\isamarkupfalse%
\ calculation\ \isacommand{by}\isamarkupfalse%
\ auto\isanewline
\ \ \ \ \ \ \ \ \isacommand{qed}\isamarkupfalse%
\isanewline
\isanewline
\ \ \ \ \ \ \ \ \isacommand{show}\isamarkupfalse%
\ {\isachardoublequoteopen}right{\isacharunderscore}{\kern0pt}cart{\isacharunderscore}{\kern0pt}proj\ X\ Y\ {\isasymcirc}\isactrlsub c\ graph{\isacharunderscore}{\kern0pt}morph\ f\ {\isasymcirc}\isactrlsub c\ i\ {\isasymcirc}\isactrlsub c\ x{\isacharprime}{\kern0pt}\ {\isacharequal}{\kern0pt}\ right{\isacharunderscore}{\kern0pt}cart{\isacharunderscore}{\kern0pt}proj\ X\ Y\ {\isasymcirc}\isactrlsub c\ graph{\isacharunderscore}{\kern0pt}morph\ f\ {\isasymcirc}\isactrlsub c\ y{\isacharprime}{\kern0pt}{\isachardoublequoteclose}\isanewline
\ \ \ \ \ \ \ \ \isacommand{proof}\isamarkupfalse%
\ {\isacharminus}{\kern0pt}\isanewline
\ \ \ \ \ \ \ \ \ \ \isacommand{have}\isamarkupfalse%
\ {\isachardoublequoteopen}right{\isacharunderscore}{\kern0pt}cart{\isacharunderscore}{\kern0pt}proj\ X\ Y\ {\isasymcirc}\isactrlsub c\ graph{\isacharunderscore}{\kern0pt}morph\ f\ {\isasymcirc}\isactrlsub c\ i\ {\isasymcirc}\isactrlsub c\ x{\isacharprime}{\kern0pt}\ {\isacharequal}{\kern0pt}\ f\ {\isasymcirc}\isactrlsub c\ left{\isacharunderscore}{\kern0pt}cart{\isacharunderscore}{\kern0pt}proj\ X\ Y\ {\isasymcirc}\isactrlsub c\ graph{\isacharunderscore}{\kern0pt}morph\ f\ {\isasymcirc}\isactrlsub c\ i\ {\isasymcirc}\isactrlsub c\ x{\isacharprime}{\kern0pt}{\isachardoublequoteclose}\isanewline
\ \ \ \ \ \ \ \ \ \ \ \ \isacommand{by}\isamarkupfalse%
\ {\isacharparenleft}{\kern0pt}etcs{\isacharunderscore}{\kern0pt}assocl{\isacharcomma}{\kern0pt}\ typecheck{\isacharunderscore}{\kern0pt}cfuncs{\isacharcomma}{\kern0pt}\ metis\ graph{\isacharunderscore}{\kern0pt}equalizer\ equalizer{\isacharunderscore}{\kern0pt}eq{\isacharparenright}{\kern0pt}\isanewline
\ \ \ \ \ \ \ \ \ \ \isacommand{also}\isamarkupfalse%
\ \isacommand{have}\isamarkupfalse%
\ {\isachardoublequoteopen}{\isachardot}{\kern0pt}{\isachardot}{\kern0pt}{\isachardot}{\kern0pt}\ {\isacharequal}{\kern0pt}\ f\ {\isasymcirc}\isactrlsub c\ left{\isacharunderscore}{\kern0pt}cart{\isacharunderscore}{\kern0pt}proj\ X\ Y\ {\isasymcirc}\isactrlsub c\ graph{\isacharunderscore}{\kern0pt}morph\ f\ {\isasymcirc}\isactrlsub c\ y{\isacharprime}{\kern0pt}{\isachardoublequoteclose}\isanewline
\ \ \ \ \ \ \ \ \ \ \ \ \isacommand{by}\isamarkupfalse%
\ {\isacharparenleft}{\kern0pt}subst\ left{\isacharcomma}{\kern0pt}\ simp{\isacharparenright}{\kern0pt}\isanewline
\ \ \ \ \ \ \ \ \ \ \isacommand{also}\isamarkupfalse%
\ \isacommand{have}\isamarkupfalse%
\ {\isachardoublequoteopen}{\isachardot}{\kern0pt}{\isachardot}{\kern0pt}{\isachardot}{\kern0pt}\ {\isacharequal}{\kern0pt}\ right{\isacharunderscore}{\kern0pt}cart{\isacharunderscore}{\kern0pt}proj\ X\ Y\ {\isasymcirc}\isactrlsub c\ graph{\isacharunderscore}{\kern0pt}morph\ f\ {\isasymcirc}\isactrlsub c\ y{\isacharprime}{\kern0pt}{\isachardoublequoteclose}\isanewline
\ \ \ \ \ \ \ \ \ \ \ \ \isacommand{by}\isamarkupfalse%
\ {\isacharparenleft}{\kern0pt}etcs{\isacharunderscore}{\kern0pt}assocl{\isacharcomma}{\kern0pt}\ typecheck{\isacharunderscore}{\kern0pt}cfuncs{\isacharcomma}{\kern0pt}\ metis\ graph{\isacharunderscore}{\kern0pt}equalizer\ equalizer{\isacharunderscore}{\kern0pt}eq{\isacharparenright}{\kern0pt}\isanewline
\ \ \ \ \ \ \ \ \ \ \isacommand{then}\isamarkupfalse%
\ \isacommand{show}\isamarkupfalse%
\ {\isacharquery}{\kern0pt}thesis\ \isacommand{using}\isamarkupfalse%
\ calculation\ \isacommand{by}\isamarkupfalse%
\ auto\isanewline
\ \ \ \ \ \ \ \ \isacommand{qed}\isamarkupfalse%
\isanewline
\ \ \ \ \ \ \isacommand{qed}\isamarkupfalse%
\isanewline
\ \ \ \ \ \ \isacommand{then}\isamarkupfalse%
\ \isacommand{have}\isamarkupfalse%
\ {\isachardoublequoteopen}i\ {\isasymcirc}\isactrlsub c\ x{\isacharprime}{\kern0pt}\ {\isacharequal}{\kern0pt}\ y{\isacharprime}{\kern0pt}{\isachardoublequoteclose}\isanewline
\ \ \ \ \ \ \ \ \isacommand{using}\isamarkupfalse%
\ equalizer{\isacharunderscore}{\kern0pt}is{\isacharunderscore}{\kern0pt}monomorphism\ graph{\isacharunderscore}{\kern0pt}equalizer\ monomorphism{\isacharunderscore}{\kern0pt}def{\isadigit{2}}\ \isacommand{by}\isamarkupfalse%
\ {\isacharparenleft}{\kern0pt}typecheck{\isacharunderscore}{\kern0pt}cfuncs{\isacharunderscore}{\kern0pt}prems{\isacharcomma}{\kern0pt}\ blast{\isacharparenright}{\kern0pt}\isanewline
\ \ \ \ \ \ \isacommand{then}\isamarkupfalse%
\ \isacommand{show}\isamarkupfalse%
\ {\isachardoublequoteopen}{\isasymexists}x{\isacharprime}{\kern0pt}{\isachardot}{\kern0pt}\ x{\isacharprime}{\kern0pt}\ {\isasymin}\isactrlsub c\ R\ {\isasymand}\ i\ {\isasymcirc}\isactrlsub c\ x{\isacharprime}{\kern0pt}\ {\isacharequal}{\kern0pt}\ y{\isacharprime}{\kern0pt}{\isachardoublequoteclose}\isanewline
\ \ \ \ \ \ \ \ \isacommand{by}\isamarkupfalse%
\ {\isacharparenleft}{\kern0pt}rule{\isacharunderscore}{\kern0pt}tac\ x{\isacharequal}{\kern0pt}x{\isacharprime}{\kern0pt}\ \isakeyword{in}\ exI{\isacharcomma}{\kern0pt}\ simp\ add{\isacharcolon}{\kern0pt}\ x{\isacharprime}{\kern0pt}{\isacharunderscore}{\kern0pt}type{\isacharparenright}{\kern0pt}\isanewline
\ \ \ \ \isacommand{qed}\isamarkupfalse%
\isanewline
\ \ \ \ \isacommand{then}\isamarkupfalse%
\ \isacommand{have}\isamarkupfalse%
\ {\isachardoublequoteopen}isomorphism\ i{\isachardoublequoteclose}\isanewline
\ \ \ \ \ \ \isacommand{by}\isamarkupfalse%
\ {\isacharparenleft}{\kern0pt}metis\ comp{\isacharunderscore}{\kern0pt}monic{\isacharunderscore}{\kern0pt}imp{\isacharunderscore}{\kern0pt}monic{\isacharprime}{\kern0pt}\ epi{\isacharunderscore}{\kern0pt}mon{\isacharunderscore}{\kern0pt}is{\isacharunderscore}{\kern0pt}iso\ f{\isacharunderscore}{\kern0pt}type\ graph{\isacharunderscore}{\kern0pt}morph{\isacharunderscore}{\kern0pt}type\ i{\isacharunderscore}{\kern0pt}eq\ i{\isacharunderscore}{\kern0pt}type\ m{\isacharunderscore}{\kern0pt}mono\ surjective{\isacharunderscore}{\kern0pt}is{\isacharunderscore}{\kern0pt}epimorphism{\isacharparenright}{\kern0pt}\isanewline
\ \ \ \ \isacommand{then}\isamarkupfalse%
\ \isacommand{show}\isamarkupfalse%
\ {\isachardoublequoteopen}{\isasymexists}i{\isachardot}{\kern0pt}\ i\ {\isacharcolon}{\kern0pt}\ R\ {\isasymrightarrow}\ graph\ f\ {\isasymand}\ isomorphism\ i\ {\isasymand}\ m\ {\isacharequal}{\kern0pt}\ graph{\isacharunderscore}{\kern0pt}morph\ f\ {\isasymcirc}\isactrlsub c\ i{\isachardoublequoteclose}\isanewline
\ \ \ \ \ \ \isacommand{by}\isamarkupfalse%
\ {\isacharparenleft}{\kern0pt}rule{\isacharunderscore}{\kern0pt}tac\ x{\isacharequal}{\kern0pt}i\ \isakeyword{in}\ exI{\isacharcomma}{\kern0pt}\ simp\ add{\isacharcolon}{\kern0pt}\ i{\isacharunderscore}{\kern0pt}type\ i{\isacharunderscore}{\kern0pt}eq{\isacharparenright}{\kern0pt}\isanewline
\ \ \isacommand{qed}\isamarkupfalse%
\isanewline
\isacommand{next}\isamarkupfalse%
\isanewline
\ \ \isacommand{fix}\isamarkupfalse%
\ f{\isadigit{1}}\ f{\isadigit{2}}\ i{\isadigit{1}}\ i{\isadigit{2}}\isanewline
\ \ \isacommand{assume}\isamarkupfalse%
\ f{\isadigit{1}}{\isacharunderscore}{\kern0pt}type{\isacharbrackleft}{\kern0pt}type{\isacharunderscore}{\kern0pt}rule{\isacharbrackright}{\kern0pt}{\isacharcolon}{\kern0pt}\ {\isachardoublequoteopen}f{\isadigit{1}}\ {\isacharcolon}{\kern0pt}\ X\ {\isasymrightarrow}\ Y{\isachardoublequoteclose}\isanewline
\ \ \isacommand{assume}\isamarkupfalse%
\ f{\isadigit{2}}{\isacharunderscore}{\kern0pt}type{\isacharbrackleft}{\kern0pt}type{\isacharunderscore}{\kern0pt}rule{\isacharbrackright}{\kern0pt}{\isacharcolon}{\kern0pt}\ {\isachardoublequoteopen}f{\isadigit{2}}\ {\isacharcolon}{\kern0pt}\ X\ {\isasymrightarrow}\ Y{\isachardoublequoteclose}\isanewline
\ \ \isacommand{assume}\isamarkupfalse%
\ i{\isadigit{1}}{\isacharunderscore}{\kern0pt}type{\isacharbrackleft}{\kern0pt}type{\isacharunderscore}{\kern0pt}rule{\isacharbrackright}{\kern0pt}{\isacharcolon}{\kern0pt}\ {\isachardoublequoteopen}i{\isadigit{1}}\ {\isacharcolon}{\kern0pt}\ R\ {\isasymrightarrow}\ graph\ f{\isadigit{1}}{\isachardoublequoteclose}\isanewline
\ \ \isacommand{assume}\isamarkupfalse%
\ i{\isadigit{2}}{\isacharunderscore}{\kern0pt}type{\isacharbrackleft}{\kern0pt}type{\isacharunderscore}{\kern0pt}rule{\isacharbrackright}{\kern0pt}{\isacharcolon}{\kern0pt}\ {\isachardoublequoteopen}i{\isadigit{2}}\ {\isacharcolon}{\kern0pt}\ R\ {\isasymrightarrow}\ graph\ f{\isadigit{2}}{\isachardoublequoteclose}\isanewline
\ \ \isacommand{assume}\isamarkupfalse%
\ i{\isadigit{1}}{\isacharunderscore}{\kern0pt}iso{\isacharcolon}{\kern0pt}\ {\isachardoublequoteopen}isomorphism\ i{\isadigit{1}}{\isachardoublequoteclose}\isanewline
\ \ \isacommand{assume}\isamarkupfalse%
\ i{\isadigit{2}}{\isacharunderscore}{\kern0pt}iso{\isacharcolon}{\kern0pt}\ {\isachardoublequoteopen}isomorphism\ i{\isadigit{2}}{\isachardoublequoteclose}\isanewline
\ \ \isacommand{assume}\isamarkupfalse%
\ eq{\isadigit{1}}{\isacharcolon}{\kern0pt}\ {\isachardoublequoteopen}m\ {\isacharequal}{\kern0pt}\ graph{\isacharunderscore}{\kern0pt}morph\ f{\isadigit{1}}\ {\isasymcirc}\isactrlsub c\ i{\isadigit{1}}{\isachardoublequoteclose}\isanewline
\ \ \isacommand{assume}\isamarkupfalse%
\ eq{\isadigit{2}}{\isacharcolon}{\kern0pt}\ {\isachardoublequoteopen}graph{\isacharunderscore}{\kern0pt}morph\ f{\isadigit{1}}\ {\isasymcirc}\isactrlsub c\ i{\isadigit{1}}\ {\isacharequal}{\kern0pt}\ graph{\isacharunderscore}{\kern0pt}morph\ f{\isadigit{2}}\ {\isasymcirc}\isactrlsub c\ i{\isadigit{2}}{\isachardoublequoteclose}\ \isanewline
\isanewline
\ \ \isacommand{have}\isamarkupfalse%
\ m{\isacharunderscore}{\kern0pt}type{\isacharbrackleft}{\kern0pt}type{\isacharunderscore}{\kern0pt}rule{\isacharbrackright}{\kern0pt}{\isacharcolon}{\kern0pt}\ {\isachardoublequoteopen}m\ {\isacharcolon}{\kern0pt}\ R\ {\isasymrightarrow}\ X\ {\isasymtimes}\isactrlsub c\ Y{\isachardoublequoteclose}\isanewline
\ \ \ \ \isacommand{using}\isamarkupfalse%
\ assms\ \isacommand{unfolding}\isamarkupfalse%
\ functional{\isacharunderscore}{\kern0pt}on{\isacharunderscore}{\kern0pt}def\ subobject{\isacharunderscore}{\kern0pt}of{\isacharunderscore}{\kern0pt}def{\isadigit{2}}\ \isacommand{by}\isamarkupfalse%
\ auto\isanewline
\ \ \isacommand{have}\isamarkupfalse%
\ isomorphism{\isacharbrackleft}{\kern0pt}type{\isacharunderscore}{\kern0pt}rule{\isacharbrackright}{\kern0pt}{\isacharcolon}{\kern0pt}\ {\isachardoublequoteopen}isomorphism{\isacharparenleft}{\kern0pt}left{\isacharunderscore}{\kern0pt}cart{\isacharunderscore}{\kern0pt}proj\ X\ Y\ {\isasymcirc}\isactrlsub c\ m{\isacharparenright}{\kern0pt}{\isachardoublequoteclose}\isanewline
\ \ \ \ \isacommand{using}\isamarkupfalse%
\ assms\ functional{\isacharunderscore}{\kern0pt}on{\isacharunderscore}{\kern0pt}isomorphism\ \isacommand{by}\isamarkupfalse%
\ force\ \ \isanewline
\ \ \isacommand{obtain}\isamarkupfalse%
\ h\ \isakeyword{where}\ h{\isacharunderscore}{\kern0pt}type{\isacharbrackleft}{\kern0pt}type{\isacharunderscore}{\kern0pt}rule{\isacharbrackright}{\kern0pt}{\isacharcolon}{\kern0pt}\ {\isachardoublequoteopen}h{\isacharcolon}{\kern0pt}\ X\ {\isasymrightarrow}\ R{\isachardoublequoteclose}\ \isakeyword{and}\ h{\isacharunderscore}{\kern0pt}def{\isacharcolon}{\kern0pt}\ {\isachardoublequoteopen}h\ {\isacharequal}{\kern0pt}\ {\isacharparenleft}{\kern0pt}left{\isacharunderscore}{\kern0pt}cart{\isacharunderscore}{\kern0pt}proj\ X\ Y\ {\isasymcirc}\isactrlsub c\ m{\isacharparenright}{\kern0pt}\isactrlbold {\isasyminverse}{\isachardoublequoteclose}\isanewline
\ \ \ \ \isacommand{by}\isamarkupfalse%
\ {\isacharparenleft}{\kern0pt}typecheck{\isacharunderscore}{\kern0pt}cfuncs{\isacharcomma}{\kern0pt}\ simp{\isacharparenright}{\kern0pt}\isanewline
\ \ \isacommand{have}\isamarkupfalse%
\ {\isachardoublequoteopen}f{\isadigit{1}}\ {\isasymcirc}\isactrlsub c\ left{\isacharunderscore}{\kern0pt}cart{\isacharunderscore}{\kern0pt}proj\ X\ Y\ {\isasymcirc}\isactrlsub c\ m\ {\isacharequal}{\kern0pt}\ f{\isadigit{2}}\ {\isasymcirc}\isactrlsub c\ left{\isacharunderscore}{\kern0pt}cart{\isacharunderscore}{\kern0pt}proj\ X\ Y\ {\isasymcirc}\isactrlsub c\ m{\isachardoublequoteclose}\isanewline
\ \ \isacommand{proof}\isamarkupfalse%
\ {\isacharminus}{\kern0pt}\ \isanewline
\ \ \ \ \isacommand{have}\isamarkupfalse%
\ {\isachardoublequoteopen}f{\isadigit{1}}\ {\isasymcirc}\isactrlsub c\ left{\isacharunderscore}{\kern0pt}cart{\isacharunderscore}{\kern0pt}proj\ X\ Y\ {\isasymcirc}\isactrlsub c\ m\ {\isacharequal}{\kern0pt}\ {\isacharparenleft}{\kern0pt}f{\isadigit{1}}\ {\isasymcirc}\isactrlsub c\ left{\isacharunderscore}{\kern0pt}cart{\isacharunderscore}{\kern0pt}proj\ X\ Y{\isacharparenright}{\kern0pt}\ {\isasymcirc}\isactrlsub c\ graph{\isacharunderscore}{\kern0pt}morph\ f{\isadigit{1}}\ {\isasymcirc}\isactrlsub c\ i{\isadigit{1}}{\isachardoublequoteclose}\isanewline
\ \ \ \ \ \ \isacommand{using}\isamarkupfalse%
\ comp{\isacharunderscore}{\kern0pt}associative{\isadigit{2}}\ eq{\isadigit{1}}\ eq{\isadigit{2}}\ \isacommand{by}\isamarkupfalse%
\ {\isacharparenleft}{\kern0pt}typecheck{\isacharunderscore}{\kern0pt}cfuncs{\isacharcomma}{\kern0pt}\ force{\isacharparenright}{\kern0pt}\isanewline
\ \ \ \ \isacommand{also}\isamarkupfalse%
\ \isacommand{have}\isamarkupfalse%
\ {\isachardoublequoteopen}{\isachardot}{\kern0pt}{\isachardot}{\kern0pt}{\isachardot}{\kern0pt}\ {\isacharequal}{\kern0pt}\ {\isacharparenleft}{\kern0pt}right{\isacharunderscore}{\kern0pt}cart{\isacharunderscore}{\kern0pt}proj\ X\ Y{\isacharparenright}{\kern0pt}\ {\isasymcirc}\isactrlsub c\ graph{\isacharunderscore}{\kern0pt}morph\ f{\isadigit{1}}\ {\isasymcirc}\isactrlsub c\ i{\isadigit{1}}{\isachardoublequoteclose}\isanewline
\ \ \ \ \ \ \isacommand{by}\isamarkupfalse%
\ {\isacharparenleft}{\kern0pt}typecheck{\isacharunderscore}{\kern0pt}cfuncs{\isacharcomma}{\kern0pt}\ smt\ comp{\isacharunderscore}{\kern0pt}associative{\isadigit{2}}\ equalizer{\isacharunderscore}{\kern0pt}def\ graph{\isacharunderscore}{\kern0pt}equalizer{\isadigit{4}}{\isacharparenright}{\kern0pt}\isanewline
\ \ \ \ \isacommand{also}\isamarkupfalse%
\ \isacommand{have}\isamarkupfalse%
\ {\isachardoublequoteopen}{\isachardot}{\kern0pt}{\isachardot}{\kern0pt}{\isachardot}{\kern0pt}\ {\isacharequal}{\kern0pt}\ {\isacharparenleft}{\kern0pt}right{\isacharunderscore}{\kern0pt}cart{\isacharunderscore}{\kern0pt}proj\ X\ Y{\isacharparenright}{\kern0pt}\ {\isasymcirc}\isactrlsub c\ graph{\isacharunderscore}{\kern0pt}morph\ f{\isadigit{2}}\ {\isasymcirc}\isactrlsub c\ i{\isadigit{2}}{\isachardoublequoteclose}\isanewline
\ \ \ \ \ \ \isacommand{by}\isamarkupfalse%
\ {\isacharparenleft}{\kern0pt}simp\ add{\isacharcolon}{\kern0pt}\ eq{\isadigit{2}}{\isacharparenright}{\kern0pt}\isanewline
\ \ \ \ \isacommand{also}\isamarkupfalse%
\ \isacommand{have}\isamarkupfalse%
\ {\isachardoublequoteopen}{\isachardot}{\kern0pt}{\isachardot}{\kern0pt}{\isachardot}{\kern0pt}\ {\isacharequal}{\kern0pt}\ {\isacharparenleft}{\kern0pt}f{\isadigit{2}}\ {\isasymcirc}\isactrlsub c\ left{\isacharunderscore}{\kern0pt}cart{\isacharunderscore}{\kern0pt}proj\ X\ Y{\isacharparenright}{\kern0pt}\ {\isasymcirc}\isactrlsub c\ graph{\isacharunderscore}{\kern0pt}morph\ f{\isadigit{2}}\ {\isasymcirc}\isactrlsub c\ i{\isadigit{2}}{\isachardoublequoteclose}\isanewline
\ \ \ \ \ \ \isacommand{by}\isamarkupfalse%
\ {\isacharparenleft}{\kern0pt}typecheck{\isacharunderscore}{\kern0pt}cfuncs{\isacharcomma}{\kern0pt}\ smt\ comp{\isacharunderscore}{\kern0pt}associative{\isadigit{2}}\ equalizer{\isacharunderscore}{\kern0pt}eq\ graph{\isacharunderscore}{\kern0pt}equalizer{\isadigit{4}}{\isacharparenright}{\kern0pt}\isanewline
\ \ \ \ \isacommand{also}\isamarkupfalse%
\ \isacommand{have}\isamarkupfalse%
\ {\isachardoublequoteopen}{\isachardot}{\kern0pt}{\isachardot}{\kern0pt}{\isachardot}{\kern0pt}\ {\isacharequal}{\kern0pt}\ f{\isadigit{2}}\ {\isasymcirc}\isactrlsub c\ left{\isacharunderscore}{\kern0pt}cart{\isacharunderscore}{\kern0pt}proj\ X\ Y\ {\isasymcirc}\isactrlsub c\ m{\isachardoublequoteclose}\isanewline
\ \ \ \ \ \ \isacommand{by}\isamarkupfalse%
\ {\isacharparenleft}{\kern0pt}typecheck{\isacharunderscore}{\kern0pt}cfuncs{\isacharcomma}{\kern0pt}\ metis\ comp{\isacharunderscore}{\kern0pt}associative{\isadigit{2}}\ eq{\isadigit{1}}\ eq{\isadigit{2}}{\isacharparenright}{\kern0pt}\isanewline
\ \ \ \ \isacommand{then}\isamarkupfalse%
\ \isacommand{show}\isamarkupfalse%
\ {\isacharquery}{\kern0pt}thesis\ \isacommand{using}\isamarkupfalse%
\ calculation\ \isacommand{by}\isamarkupfalse%
\ auto\isanewline
\ \ \isacommand{qed}\isamarkupfalse%
\isanewline
\ \ \isacommand{then}\isamarkupfalse%
\ \isacommand{show}\isamarkupfalse%
\ {\isachardoublequoteopen}f{\isadigit{1}}\ {\isacharequal}{\kern0pt}\ f{\isadigit{2}}{\isachardoublequoteclose}\isanewline
\ \ \ \ \isacommand{by}\isamarkupfalse%
\ {\isacharparenleft}{\kern0pt}typecheck{\isacharunderscore}{\kern0pt}cfuncs{\isacharcomma}{\kern0pt}\ metis\ cfunc{\isacharunderscore}{\kern0pt}type{\isacharunderscore}{\kern0pt}def\ comp{\isacharunderscore}{\kern0pt}associative\ h{\isacharunderscore}{\kern0pt}def\ h{\isacharunderscore}{\kern0pt}type\ id{\isacharunderscore}{\kern0pt}right{\isacharunderscore}{\kern0pt}unit{\isadigit{2}}\ inverse{\isacharunderscore}{\kern0pt}def{\isadigit{2}}\ isomorphism{\isacharparenright}{\kern0pt}\isanewline
\isacommand{qed}\isamarkupfalse%
%
\endisatagproof
{\isafoldproof}%
%
\isadelimproof
\isanewline
%
\endisadelimproof
%
\isadelimtheory
\isanewline
%
\endisadelimtheory
%
\isatagtheory
\isacommand{end}\isamarkupfalse%
%
\endisatagtheory
{\isafoldtheory}%
%
\isadelimtheory
%
\endisadelimtheory
%
\end{isabellebody}%
\endinput
%:%file=~/ETCS/HOL-ETCS/Truth.thy%:%
%:%11=1%:%
%:%27=3%:%
%:%28=3%:%
%:%29=4%:%
%:%30=5%:%
%:%39=7%:%
%:%41=8%:%
%:%42=8%:%
%:%43=9%:%
%:%44=10%:%
%:%45=11%:%
%:%46=12%:%
%:%47=13%:%
%:%48=14%:%
%:%49=15%:%
%:%50=16%:%
%:%51=17%:%
%:%52=18%:%
%:%53=19%:%
%:%54=20%:%
%:%55=20%:%
%:%56=21%:%
%:%57=22%:%
%:%58=23%:%
%:%59=24%:%
%:%60=24%:%
%:%61=25%:%
%:%62=26%:%
%:%69=27%:%
%:%70=27%:%
%:%71=28%:%
%:%72=28%:%
%:%73=29%:%
%:%74=29%:%
%:%75=29%:%
%:%76=30%:%
%:%77=31%:%
%:%78=31%:%
%:%79=32%:%
%:%80=32%:%
%:%81=33%:%
%:%82=33%:%
%:%83=34%:%
%:%84=34%:%
%:%85=34%:%
%:%86=35%:%
%:%87=35%:%
%:%88=36%:%
%:%89=36%:%
%:%90=36%:%
%:%91=37%:%
%:%92=37%:%
%:%93=38%:%
%:%94=38%:%
%:%95=38%:%
%:%96=39%:%
%:%97=39%:%
%:%98=39%:%
%:%99=40%:%
%:%105=40%:%
%:%108=41%:%
%:%109=42%:%
%:%110=42%:%
%:%111=43%:%
%:%112=44%:%
%:%119=45%:%
%:%120=45%:%
%:%121=46%:%
%:%122=46%:%
%:%123=47%:%
%:%124=47%:%
%:%125=47%:%
%:%126=48%:%
%:%127=48%:%
%:%128=48%:%
%:%129=49%:%
%:%130=49%:%
%:%131=49%:%
%:%132=50%:%
%:%138=50%:%
%:%141=51%:%
%:%142=52%:%
%:%143=52%:%
%:%144=53%:%
%:%145=54%:%
%:%148=55%:%
%:%152=55%:%
%:%153=55%:%
%:%154=55%:%
%:%155=55%:%
%:%160=55%:%
%:%163=56%:%
%:%164=57%:%
%:%165=57%:%
%:%166=58%:%
%:%167=59%:%
%:%170=60%:%
%:%174=60%:%
%:%175=60%:%
%:%176=61%:%
%:%177=61%:%
%:%178=62%:%
%:%179=62%:%
%:%180=63%:%
%:%181=63%:%
%:%182=63%:%
%:%183=64%:%
%:%184=64%:%
%:%185=64%:%
%:%186=65%:%
%:%187=65%:%
%:%188=65%:%
%:%189=66%:%
%:%190=66%:%
%:%191=66%:%
%:%192=67%:%
%:%193=67%:%
%:%194=67%:%
%:%195=68%:%
%:%196=68%:%
%:%197=69%:%
%:%198=69%:%
%:%199=70%:%
%:%200=70%:%
%:%201=71%:%
%:%202=71%:%
%:%203=72%:%
%:%204=72%:%
%:%205=73%:%
%:%206=73%:%
%:%207=74%:%
%:%213=74%:%
%:%216=75%:%
%:%217=76%:%
%:%218=76%:%
%:%219=77%:%
%:%220=78%:%
%:%221=79%:%
%:%228=80%:%
%:%229=80%:%
%:%230=81%:%
%:%231=81%:%
%:%232=82%:%
%:%233=82%:%
%:%234=83%:%
%:%235=83%:%
%:%236=83%:%
%:%237=84%:%
%:%238=84%:%
%:%239=84%:%
%:%240=84%:%
%:%241=85%:%
%:%242=85%:%
%:%243=85%:%
%:%244=86%:%
%:%245=86%:%
%:%246=86%:%
%:%247=87%:%
%:%253=87%:%
%:%256=88%:%
%:%257=89%:%
%:%258=89%:%
%:%259=90%:%
%:%260=91%:%
%:%261=92%:%
%:%268=93%:%
%:%269=93%:%
%:%270=94%:%
%:%271=94%:%
%:%272=95%:%
%:%273=95%:%
%:%274=96%:%
%:%275=96%:%
%:%276=97%:%
%:%277=97%:%
%:%278=97%:%
%:%279=98%:%
%:%280=98%:%
%:%281=98%:%
%:%282=99%:%
%:%283=100%:%
%:%284=100%:%
%:%285=101%:%
%:%286=101%:%
%:%287=102%:%
%:%288=102%:%
%:%289=102%:%
%:%290=103%:%
%:%291=103%:%
%:%292=103%:%
%:%293=104%:%
%:%294=105%:%
%:%295=105%:%
%:%296=105%:%
%:%297=106%:%
%:%298=106%:%
%:%299=106%:%
%:%300=107%:%
%:%301=107%:%
%:%302=107%:%
%:%303=108%:%
%:%304=108%:%
%:%305=108%:%
%:%306=109%:%
%:%307=109%:%
%:%308=109%:%
%:%309=110%:%
%:%310=110%:%
%:%311=110%:%
%:%312=111%:%
%:%313=111%:%
%:%314=111%:%
%:%315=112%:%
%:%316=112%:%
%:%317=113%:%
%:%318=114%:%
%:%319=114%:%
%:%320=114%:%
%:%321=115%:%
%:%322=115%:%
%:%323=115%:%
%:%324=116%:%
%:%330=116%:%
%:%333=117%:%
%:%334=118%:%
%:%335=118%:%
%:%336=119%:%
%:%337=120%:%
%:%338=121%:%
%:%341=122%:%
%:%345=122%:%
%:%346=122%:%
%:%351=122%:%
%:%354=123%:%
%:%355=124%:%
%:%356=124%:%
%:%357=125%:%
%:%358=126%:%
%:%359=127%:%
%:%362=128%:%
%:%366=128%:%
%:%367=128%:%
%:%376=130%:%
%:%378=131%:%
%:%379=131%:%
%:%386=132%:%
%:%387=132%:%
%:%388=133%:%
%:%389=133%:%
%:%390=134%:%
%:%391=134%:%
%:%392=135%:%
%:%393=136%:%
%:%394=136%:%
%:%395=136%:%
%:%396=137%:%
%:%397=137%:%
%:%398=137%:%
%:%399=138%:%
%:%414=140%:%
%:%424=142%:%
%:%425=142%:%
%:%426=143%:%
%:%427=144%:%
%:%428=145%:%
%:%429=145%:%
%:%432=146%:%
%:%436=146%:%
%:%437=146%:%
%:%438=147%:%
%:%439=147%:%
%:%444=147%:%
%:%447=148%:%
%:%448=149%:%
%:%449=149%:%
%:%450=150%:%
%:%453=151%:%
%:%457=151%:%
%:%458=151%:%
%:%459=151%:%
%:%460=151%:%
%:%465=151%:%
%:%468=152%:%
%:%469=153%:%
%:%470=153%:%
%:%473=154%:%
%:%477=154%:%
%:%478=154%:%
%:%479=154%:%
%:%480=154%:%
%:%485=154%:%
%:%488=155%:%
%:%489=156%:%
%:%490=156%:%
%:%491=157%:%
%:%492=158%:%
%:%499=159%:%
%:%500=159%:%
%:%501=160%:%
%:%502=160%:%
%:%503=161%:%
%:%504=162%:%
%:%505=162%:%
%:%506=163%:%
%:%507=163%:%
%:%508=163%:%
%:%509=163%:%
%:%510=164%:%
%:%511=164%:%
%:%512=164%:%
%:%513=165%:%
%:%514=165%:%
%:%515=166%:%
%:%516=166%:%
%:%517=167%:%
%:%518=167%:%
%:%519=167%:%
%:%520=168%:%
%:%521=168%:%
%:%522=169%:%
%:%523=169%:%
%:%524=170%:%
%:%525=170%:%
%:%526=171%:%
%:%527=171%:%
%:%528=172%:%
%:%529=172%:%
%:%530=172%:%
%:%531=173%:%
%:%532=173%:%
%:%533=173%:%
%:%534=174%:%
%:%535=174%:%
%:%536=174%:%
%:%537=175%:%
%:%538=175%:%
%:%539=175%:%
%:%540=175%:%
%:%541=176%:%
%:%542=176%:%
%:%543=176%:%
%:%544=177%:%
%:%545=177%:%
%:%546=177%:%
%:%547=178%:%
%:%548=178%:%
%:%549=178%:%
%:%550=179%:%
%:%551=179%:%
%:%552=179%:%
%:%553=180%:%
%:%559=180%:%
%:%562=181%:%
%:%563=182%:%
%:%564=182%:%
%:%565=183%:%
%:%566=184%:%
%:%573=185%:%
%:%574=185%:%
%:%575=186%:%
%:%576=186%:%
%:%577=187%:%
%:%578=187%:%
%:%579=187%:%
%:%580=188%:%
%:%581=188%:%
%:%582=188%:%
%:%583=189%:%
%:%584=189%:%
%:%585=190%:%
%:%586=190%:%
%:%587=191%:%
%:%588=191%:%
%:%589=192%:%
%:%595=192%:%
%:%598=193%:%
%:%599=194%:%
%:%600=194%:%
%:%601=195%:%
%:%602=196%:%
%:%605=197%:%
%:%609=197%:%
%:%610=197%:%
%:%611=197%:%
%:%616=197%:%
%:%619=198%:%
%:%620=199%:%
%:%621=199%:%
%:%622=200%:%
%:%623=201%:%
%:%630=202%:%
%:%631=202%:%
%:%632=203%:%
%:%633=203%:%
%:%634=204%:%
%:%635=204%:%
%:%636=205%:%
%:%637=205%:%
%:%638=206%:%
%:%639=206%:%
%:%640=207%:%
%:%641=207%:%
%:%642=208%:%
%:%643=208%:%
%:%644=209%:%
%:%645=209%:%
%:%646=210%:%
%:%647=210%:%
%:%648=210%:%
%:%649=211%:%
%:%650=211%:%
%:%651=211%:%
%:%652=212%:%
%:%653=212%:%
%:%654=213%:%
%:%655=213%:%
%:%656=214%:%
%:%657=214%:%
%:%658=215%:%
%:%659=215%:%
%:%660=215%:%
%:%661=216%:%
%:%662=216%:%
%:%663=217%:%
%:%664=217%:%
%:%665=217%:%
%:%666=218%:%
%:%667=218%:%
%:%668=219%:%
%:%669=219%:%
%:%670=219%:%
%:%671=220%:%
%:%672=220%:%
%:%673=221%:%
%:%674=221%:%
%:%675=221%:%
%:%676=222%:%
%:%677=222%:%
%:%678=222%:%
%:%679=223%:%
%:%680=223%:%
%:%681=223%:%
%:%682=224%:%
%:%683=224%:%
%:%684=225%:%
%:%685=225%:%
%:%686=226%:%
%:%687=226%:%
%:%688=226%:%
%:%689=227%:%
%:%690=227%:%
%:%691=228%:%
%:%692=228%:%
%:%693=229%:%
%:%694=229%:%
%:%695=229%:%
%:%696=230%:%
%:%697=230%:%
%:%698=231%:%
%:%699=231%:%
%:%700=231%:%
%:%701=232%:%
%:%702=232%:%
%:%703=233%:%
%:%704=233%:%
%:%705=233%:%
%:%706=234%:%
%:%707=234%:%
%:%708=235%:%
%:%709=235%:%
%:%710=235%:%
%:%711=236%:%
%:%712=236%:%
%:%713=236%:%
%:%714=237%:%
%:%715=237%:%
%:%716=237%:%
%:%717=238%:%
%:%718=238%:%
%:%719=239%:%
%:%720=239%:%
%:%721=239%:%
%:%722=240%:%
%:%723=240%:%
%:%724=240%:%
%:%725=241%:%
%:%726=241%:%
%:%727=242%:%
%:%728=242%:%
%:%729=242%:%
%:%730=243%:%
%:%731=243%:%
%:%732=244%:%
%:%738=244%:%
%:%741=245%:%
%:%742=246%:%
%:%743=246%:%
%:%744=247%:%
%:%745=248%:%
%:%748=249%:%
%:%752=249%:%
%:%753=249%:%
%:%754=249%:%
%:%759=249%:%
%:%762=250%:%
%:%763=251%:%
%:%764=251%:%
%:%765=252%:%
%:%766=253%:%
%:%769=254%:%
%:%773=254%:%
%:%774=254%:%
%:%775=254%:%
%:%789=256%:%
%:%801=258%:%
%:%803=259%:%
%:%804=259%:%
%:%807=260%:%
%:%811=260%:%
%:%812=260%:%
%:%813=260%:%
%:%822=262%:%
%:%824=263%:%
%:%825=263%:%
%:%826=264%:%
%:%829=265%:%
%:%833=265%:%
%:%834=265%:%
%:%835=266%:%
%:%836=266%:%
%:%837=267%:%
%:%838=267%:%
%:%847=269%:%
%:%849=270%:%
%:%850=270%:%
%:%851=271%:%
%:%852=272%:%
%:%855=273%:%
%:%859=273%:%
%:%860=273%:%
%:%861=273%:%
%:%870=275%:%
%:%872=276%:%
%:%873=276%:%
%:%874=277%:%
%:%875=278%:%
%:%878=279%:%
%:%882=279%:%
%:%883=279%:%
%:%884=280%:%
%:%885=280%:%
%:%886=281%:%
%:%887=281%:%
%:%888=282%:%
%:%889=282%:%
%:%890=283%:%
%:%891=283%:%
%:%892=283%:%
%:%893=284%:%
%:%894=284%:%
%:%895=284%:%
%:%896=285%:%
%:%897=285%:%
%:%898=286%:%
%:%899=286%:%
%:%900=287%:%
%:%901=287%:%
%:%902=287%:%
%:%903=288%:%
%:%904=288%:%
%:%905=289%:%
%:%906=289%:%
%:%907=290%:%
%:%908=290%:%
%:%909=291%:%
%:%910=291%:%
%:%911=292%:%
%:%912=292%:%
%:%913=292%:%
%:%914=293%:%
%:%915=293%:%
%:%916=293%:%
%:%917=294%:%
%:%918=295%:%
%:%919=295%:%
%:%920=296%:%
%:%921=296%:%
%:%922=297%:%
%:%923=297%:%
%:%924=298%:%
%:%925=298%:%
%:%926=298%:%
%:%927=299%:%
%:%928=299%:%
%:%929=300%:%
%:%930=301%:%
%:%931=301%:%
%:%932=302%:%
%:%933=302%:%
%:%934=303%:%
%:%935=303%:%
%:%936=304%:%
%:%937=304%:%
%:%938=305%:%
%:%939=305%:%
%:%940=306%:%
%:%941=306%:%
%:%942=307%:%
%:%943=307%:%
%:%944=308%:%
%:%945=308%:%
%:%946=309%:%
%:%947=309%:%
%:%948=310%:%
%:%949=310%:%
%:%950=311%:%
%:%951=311%:%
%:%952=311%:%
%:%953=312%:%
%:%954=312%:%
%:%955=313%:%
%:%956=313%:%
%:%957=313%:%
%:%958=314%:%
%:%959=314%:%
%:%960=315%:%
%:%961=315%:%
%:%962=315%:%
%:%963=316%:%
%:%964=316%:%
%:%965=317%:%
%:%966=317%:%
%:%967=318%:%
%:%968=318%:%
%:%969=319%:%
%:%970=319%:%
%:%971=320%:%
%:%972=320%:%
%:%973=321%:%
%:%974=321%:%
%:%975=322%:%
%:%976=322%:%
%:%977=322%:%
%:%978=323%:%
%:%979=323%:%
%:%980=324%:%
%:%981=324%:%
%:%982=324%:%
%:%983=325%:%
%:%984=325%:%
%:%985=326%:%
%:%986=326%:%
%:%987=326%:%
%:%988=327%:%
%:%989=327%:%
%:%990=328%:%
%:%991=328%:%
%:%992=328%:%
%:%993=329%:%
%:%994=329%:%
%:%995=329%:%
%:%996=330%:%
%:%997=330%:%
%:%998=330%:%
%:%999=331%:%
%:%1000=331%:%
%:%1001=331%:%
%:%1002=332%:%
%:%1003=332%:%
%:%1004=332%:%
%:%1005=333%:%
%:%1006=333%:%
%:%1007=334%:%
%:%1008=334%:%
%:%1009=334%:%
%:%1010=335%:%
%:%1011=335%:%
%:%1012=335%:%
%:%1013=336%:%
%:%1023=338%:%
%:%1025=339%:%
%:%1026=339%:%
%:%1027=340%:%
%:%1028=341%:%
%:%1035=342%:%
%:%1036=342%:%
%:%1037=343%:%
%:%1038=343%:%
%:%1039=344%:%
%:%1040=344%:%
%:%1041=344%:%
%:%1042=345%:%
%:%1043=345%:%
%:%1044=346%:%
%:%1045=346%:%
%:%1046=347%:%
%:%1047=347%:%
%:%1048=348%:%
%:%1049=348%:%
%:%1050=349%:%
%:%1051=349%:%
%:%1052=350%:%
%:%1053=350%:%
%:%1054=350%:%
%:%1055=351%:%
%:%1056=351%:%
%:%1057=351%:%
%:%1058=352%:%
%:%1059=352%:%
%:%1060=352%:%
%:%1061=353%:%
%:%1062=353%:%
%:%1063=353%:%
%:%1064=354%:%
%:%1065=354%:%
%:%1066=354%:%
%:%1067=355%:%
%:%1068=355%:%
%:%1069=356%:%
%:%1070=356%:%
%:%1071=356%:%
%:%1072=357%:%
%:%1073=357%:%
%:%1074=358%:%
%:%1075=358%:%
%:%1076=358%:%
%:%1077=359%:%
%:%1078=359%:%
%:%1079=359%:%
%:%1080=360%:%
%:%1081=360%:%
%:%1082=361%:%
%:%1083=361%:%
%:%1084=361%:%
%:%1085=362%:%
%:%1086=362%:%
%:%1087=362%:%
%:%1088=363%:%
%:%1098=365%:%
%:%1100=366%:%
%:%1101=366%:%
%:%1102=367%:%
%:%1103=368%:%
%:%1110=369%:%
%:%1111=369%:%
%:%1112=370%:%
%:%1113=370%:%
%:%1114=371%:%
%:%1115=371%:%
%:%1116=371%:%
%:%1117=372%:%
%:%1118=372%:%
%:%1119=373%:%
%:%1120=373%:%
%:%1121=374%:%
%:%1122=374%:%
%:%1123=375%:%
%:%1124=375%:%
%:%1125=376%:%
%:%1126=376%:%
%:%1127=377%:%
%:%1128=377%:%
%:%1129=377%:%
%:%1130=378%:%
%:%1131=378%:%
%:%1132=378%:%
%:%1133=379%:%
%:%1134=379%:%
%:%1135=379%:%
%:%1136=380%:%
%:%1137=380%:%
%:%1138=380%:%
%:%1139=381%:%
%:%1140=381%:%
%:%1141=381%:%
%:%1142=382%:%
%:%1143=382%:%
%:%1144=383%:%
%:%1145=383%:%
%:%1146=383%:%
%:%1147=384%:%
%:%1148=384%:%
%:%1149=385%:%
%:%1150=385%:%
%:%1151=385%:%
%:%1152=386%:%
%:%1153=386%:%
%:%1154=386%:%
%:%1155=387%:%
%:%1156=387%:%
%:%1157=388%:%
%:%1158=388%:%
%:%1159=388%:%
%:%1160=389%:%
%:%1161=389%:%
%:%1162=389%:%
%:%1163=390%:%
%:%1173=392%:%
%:%1175=393%:%
%:%1176=393%:%
%:%1177=394%:%
%:%1178=395%:%
%:%1185=396%:%
%:%1186=396%:%
%:%1187=397%:%
%:%1188=397%:%
%:%1189=398%:%
%:%1190=398%:%
%:%1191=399%:%
%:%1192=399%:%
%:%1193=400%:%
%:%1194=400%:%
%:%1195=401%:%
%:%1196=401%:%
%:%1197=402%:%
%:%1198=403%:%
%:%1199=403%:%
%:%1200=404%:%
%:%1201=404%:%
%:%1202=404%:%
%:%1203=405%:%
%:%1204=405%:%
%:%1205=406%:%
%:%1206=406%:%
%:%1207=406%:%
%:%1208=407%:%
%:%1209=407%:%
%:%1210=408%:%
%:%1211=408%:%
%:%1212=408%:%
%:%1213=409%:%
%:%1214=409%:%
%:%1215=410%:%
%:%1216=410%:%
%:%1217=410%:%
%:%1218=411%:%
%:%1219=411%:%
%:%1220=412%:%
%:%1221=412%:%
%:%1222=412%:%
%:%1223=413%:%
%:%1224=414%:%
%:%1225=414%:%
%:%1226=415%:%
%:%1227=415%:%
%:%1228=416%:%
%:%1229=416%:%
%:%1230=417%:%
%:%1231=417%:%
%:%1232=418%:%
%:%1233=418%:%
%:%1234=418%:%
%:%1235=419%:%
%:%1236=419%:%
%:%1237=419%:%
%:%1238=420%:%
%:%1239=420%:%
%:%1240=420%:%
%:%1241=421%:%
%:%1242=421%:%
%:%1243=422%:%
%:%1244=422%:%
%:%1245=423%:%
%:%1246=424%:%
%:%1247=424%:%
%:%1248=425%:%
%:%1249=425%:%
%:%1250=426%:%
%:%1251=426%:%
%:%1252=427%:%
%:%1253=427%:%
%:%1254=427%:%
%:%1255=428%:%
%:%1256=428%:%
%:%1257=428%:%
%:%1258=429%:%
%:%1259=429%:%
%:%1260=430%:%
%:%1261=430%:%
%:%1262=430%:%
%:%1263=431%:%
%:%1264=431%:%
%:%1265=431%:%
%:%1266=432%:%
%:%1267=432%:%
%:%1268=432%:%
%:%1269=433%:%
%:%1270=433%:%
%:%1271=434%:%
%:%1272=434%:%
%:%1273=435%:%
%:%1274=436%:%
%:%1275=436%:%
%:%1276=437%:%
%:%1277=437%:%
%:%1278=438%:%
%:%1279=438%:%
%:%1280=438%:%
%:%1281=439%:%
%:%1282=439%:%
%:%1283=439%:%
%:%1284=440%:%
%:%1294=442%:%
%:%1296=443%:%
%:%1297=443%:%
%:%1298=444%:%
%:%1299=445%:%
%:%1306=446%:%
%:%1307=446%:%
%:%1308=447%:%
%:%1309=447%:%
%:%1310=448%:%
%:%1311=448%:%
%:%1312=449%:%
%:%1313=449%:%
%:%1314=450%:%
%:%1315=450%:%
%:%1316=451%:%
%:%1317=451%:%
%:%1318=452%:%
%:%1319=453%:%
%:%1320=453%:%
%:%1321=454%:%
%:%1322=454%:%
%:%1323=454%:%
%:%1324=455%:%
%:%1325=455%:%
%:%1326=456%:%
%:%1327=456%:%
%:%1328=456%:%
%:%1329=457%:%
%:%1330=457%:%
%:%1331=458%:%
%:%1332=458%:%
%:%1333=458%:%
%:%1334=459%:%
%:%1335=459%:%
%:%1336=460%:%
%:%1337=460%:%
%:%1338=460%:%
%:%1339=461%:%
%:%1340=461%:%
%:%1341=462%:%
%:%1342=462%:%
%:%1343=462%:%
%:%1344=463%:%
%:%1345=464%:%
%:%1346=464%:%
%:%1347=465%:%
%:%1348=465%:%
%:%1349=466%:%
%:%1350=466%:%
%:%1351=467%:%
%:%1352=467%:%
%:%1353=468%:%
%:%1354=468%:%
%:%1355=468%:%
%:%1356=469%:%
%:%1357=469%:%
%:%1358=469%:%
%:%1359=470%:%
%:%1360=470%:%
%:%1361=470%:%
%:%1362=471%:%
%:%1363=471%:%
%:%1364=472%:%
%:%1365=472%:%
%:%1366=473%:%
%:%1367=474%:%
%:%1368=474%:%
%:%1369=475%:%
%:%1370=475%:%
%:%1371=476%:%
%:%1372=476%:%
%:%1373=477%:%
%:%1374=477%:%
%:%1375=477%:%
%:%1376=478%:%
%:%1377=478%:%
%:%1378=478%:%
%:%1379=479%:%
%:%1380=479%:%
%:%1381=480%:%
%:%1382=480%:%
%:%1383=480%:%
%:%1384=481%:%
%:%1385=481%:%
%:%1386=481%:%
%:%1387=482%:%
%:%1388=482%:%
%:%1389=482%:%
%:%1390=483%:%
%:%1391=483%:%
%:%1392=484%:%
%:%1393=484%:%
%:%1394=485%:%
%:%1395=485%:%
%:%1396=486%:%
%:%1397=486%:%
%:%1398=487%:%
%:%1399=487%:%
%:%1400=487%:%
%:%1401=488%:%
%:%1402=488%:%
%:%1403=488%:%
%:%1404=489%:%
%:%1419=491%:%
%:%1431=493%:%
%:%1433=494%:%
%:%1434=494%:%
%:%1435=495%:%
%:%1436=496%:%
%:%1437=497%:%
%:%1438=497%:%
%:%1439=498%:%
%:%1440=499%:%
%:%1441=500%:%
%:%1442=500%:%
%:%1443=501%:%
%:%1444=502%:%
%:%1447=503%:%
%:%1451=503%:%
%:%1452=503%:%
%:%1453=504%:%
%:%1454=504%:%
%:%1455=505%:%
%:%1456=505%:%
%:%1461=505%:%
%:%1464=506%:%
%:%1465=507%:%
%:%1466=507%:%
%:%1467=508%:%
%:%1468=509%:%
%:%1471=510%:%
%:%1475=510%:%
%:%1476=510%:%
%:%1477=511%:%
%:%1478=511%:%
%:%1479=512%:%
%:%1480=512%:%
%:%1485=512%:%
%:%1488=513%:%
%:%1489=514%:%
%:%1490=514%:%
%:%1491=515%:%
%:%1492=516%:%
%:%1495=517%:%
%:%1499=517%:%
%:%1500=517%:%
%:%1501=517%:%
%:%1506=517%:%
%:%1509=518%:%
%:%1510=519%:%
%:%1511=519%:%
%:%1512=520%:%
%:%1513=521%:%
%:%1520=522%:%
%:%1521=522%:%
%:%1522=523%:%
%:%1523=523%:%
%:%1524=524%:%
%:%1525=524%:%
%:%1526=524%:%
%:%1527=525%:%
%:%1528=525%:%
%:%1529=525%:%
%:%1530=526%:%
%:%1531=526%:%
%:%1532=526%:%
%:%1533=527%:%
%:%1534=527%:%
%:%1535=527%:%
%:%1536=528%:%
%:%1537=528%:%
%:%1538=528%:%
%:%1539=529%:%
%:%1540=529%:%
%:%1541=529%:%
%:%1542=530%:%
%:%1543=530%:%
%:%1544=530%:%
%:%1545=531%:%
%:%1546=531%:%
%:%1547=531%:%
%:%1548=532%:%
%:%1549=532%:%
%:%1550=532%:%
%:%1551=533%:%
%:%1561=535%:%
%:%1563=536%:%
%:%1564=536%:%
%:%1565=537%:%
%:%1566=538%:%
%:%1573=539%:%
%:%1574=539%:%
%:%1575=540%:%
%:%1576=540%:%
%:%1577=541%:%
%:%1578=541%:%
%:%1579=541%:%
%:%1580=542%:%
%:%1581=542%:%
%:%1582=543%:%
%:%1583=543%:%
%:%1584=543%:%
%:%1585=544%:%
%:%1586=545%:%
%:%1587=545%:%
%:%1588=546%:%
%:%1589=546%:%
%:%1590=547%:%
%:%1591=547%:%
%:%1592=548%:%
%:%1593=548%:%
%:%1594=549%:%
%:%1595=549%:%
%:%1596=549%:%
%:%1597=550%:%
%:%1598=550%:%
%:%1599=551%:%
%:%1600=551%:%
%:%1601=551%:%
%:%1602=552%:%
%:%1603=552%:%
%:%1604=553%:%
%:%1605=553%:%
%:%1606=554%:%
%:%1607=554%:%
%:%1608=555%:%
%:%1609=555%:%
%:%1610=556%:%
%:%1611=556%:%
%:%1612=556%:%
%:%1613=557%:%
%:%1614=557%:%
%:%1615=557%:%
%:%1616=558%:%
%:%1617=558%:%
%:%1618=559%:%
%:%1619=559%:%
%:%1620=559%:%
%:%1621=560%:%
%:%1622=560%:%
%:%1623=560%:%
%:%1624=561%:%
%:%1630=561%:%
%:%1633=562%:%
%:%1634=563%:%
%:%1635=563%:%
%:%1636=564%:%
%:%1637=565%:%
%:%1638=566%:%
%:%1641=567%:%
%:%1645=567%:%
%:%1646=567%:%
%:%1647=567%:%
%:%1652=567%:%
%:%1655=568%:%
%:%1656=569%:%
%:%1657=569%:%
%:%1658=570%:%
%:%1659=571%:%
%:%1666=572%:%
%:%1667=572%:%
%:%1668=573%:%
%:%1669=573%:%
%:%1670=574%:%
%:%1671=574%:%
%:%1672=574%:%
%:%1673=575%:%
%:%1674=575%:%
%:%1675=575%:%
%:%1676=576%:%
%:%1677=576%:%
%:%1678=576%:%
%:%1679=577%:%
%:%1680=577%:%
%:%1681=578%:%
%:%1682=578%:%
%:%1683=579%:%
%:%1684=579%:%
%:%1685=579%:%
%:%1686=580%:%
%:%1687=580%:%
%:%1688=581%:%
%:%1689=581%:%
%:%1690=581%:%
%:%1691=582%:%
%:%1692=582%:%
%:%1693=582%:%
%:%1694=583%:%
%:%1700=583%:%
%:%1703=584%:%
%:%1704=585%:%
%:%1705=585%:%
%:%1706=586%:%
%:%1707=587%:%
%:%1714=588%:%
%:%1715=588%:%
%:%1716=589%:%
%:%1717=589%:%
%:%1718=590%:%
%:%1719=590%:%
%:%1720=590%:%
%:%1721=591%:%
%:%1722=592%:%
%:%1723=592%:%
%:%1724=592%:%
%:%1725=593%:%
%:%1726=593%:%
%:%1727=593%:%
%:%1728=594%:%
%:%1729=594%:%
%:%1730=595%:%
%:%1731=595%:%
%:%1732=596%:%
%:%1733=596%:%
%:%1734=596%:%
%:%1735=597%:%
%:%1736=597%:%
%:%1737=598%:%
%:%1738=598%:%
%:%1739=598%:%
%:%1740=599%:%
%:%1741=599%:%
%:%1742=599%:%
%:%1743=600%:%
%:%1753=602%:%
%:%1755=603%:%
%:%1756=603%:%
%:%1757=604%:%
%:%1758=605%:%
%:%1759=606%:%
%:%1760=607%:%
%:%1761=608%:%
%:%1762=609%:%
%:%1765=612%:%
%:%1772=613%:%
%:%1773=613%:%
%:%1774=614%:%
%:%1775=614%:%
%:%1776=615%:%
%:%1777=616%:%
%:%1778=616%:%
%:%1779=617%:%
%:%1780=617%:%
%:%1781=618%:%
%:%1782=619%:%
%:%1783=619%:%
%:%1784=620%:%
%:%1785=620%:%
%:%1786=620%:%
%:%1787=621%:%
%:%1788=621%:%
%:%1789=622%:%
%:%1790=622%:%
%:%1791=622%:%
%:%1792=623%:%
%:%1793=623%:%
%:%1794=624%:%
%:%1795=624%:%
%:%1796=624%:%
%:%1797=625%:%
%:%1798=625%:%
%:%1799=626%:%
%:%1800=626%:%
%:%1801=626%:%
%:%1802=627%:%
%:%1803=627%:%
%:%1804=627%:%
%:%1805=628%:%
%:%1806=628%:%
%:%1807=628%:%
%:%1808=629%:%
%:%1809=629%:%
%:%1810=630%:%
%:%1811=630%:%
%:%1812=630%:%
%:%1813=631%:%
%:%1814=631%:%
%:%1815=632%:%
%:%1816=632%:%
%:%1817=632%:%
%:%1818=633%:%
%:%1819=633%:%
%:%1820=634%:%
%:%1821=634%:%
%:%1822=634%:%
%:%1823=635%:%
%:%1824=635%:%
%:%1825=636%:%
%:%1826=636%:%
%:%1827=636%:%
%:%1828=637%:%
%:%1829=637%:%
%:%1830=637%:%
%:%1831=638%:%
%:%1832=638%:%
%:%1833=639%:%
%:%1834=639%:%
%:%1835=640%:%
%:%1836=640%:%
%:%1837=640%:%
%:%1838=641%:%
%:%1839=641%:%
%:%1840=641%:%
%:%1841=642%:%
%:%1842=643%:%
%:%1843=643%:%
%:%1844=643%:%
%:%1845=643%:%
%:%1846=644%:%
%:%1847=644%:%
%:%1848=644%:%
%:%1849=645%:%
%:%1850=646%:%
%:%1851=646%:%
%:%1852=647%:%
%:%1853=647%:%
%:%1854=647%:%
%:%1855=648%:%
%:%1856=649%:%
%:%1857=649%:%
%:%1858=650%:%
%:%1859=651%:%
%:%1860=651%:%
%:%1861=652%:%
%:%1862=653%:%
%:%1863=653%:%
%:%1864=654%:%
%:%1865=654%:%
%:%1866=655%:%
%:%1867=655%:%
%:%1868=656%:%
%:%1869=656%:%
%:%1870=657%:%
%:%1871=658%:%
%:%1872=658%:%
%:%1873=659%:%
%:%1874=659%:%
%:%1875=660%:%
%:%1876=660%:%
%:%1877=661%:%
%:%1878=661%:%
%:%1879=662%:%
%:%1880=662%:%
%:%1881=662%:%
%:%1882=663%:%
%:%1883=663%:%
%:%1884=664%:%
%:%1885=664%:%
%:%1886=664%:%
%:%1887=665%:%
%:%1888=665%:%
%:%1889=666%:%
%:%1890=666%:%
%:%1891=666%:%
%:%1892=667%:%
%:%1893=667%:%
%:%1894=668%:%
%:%1895=668%:%
%:%1896=668%:%
%:%1897=669%:%
%:%1898=669%:%
%:%1899=669%:%
%:%1900=670%:%
%:%1901=670%:%
%:%1902=670%:%
%:%1903=671%:%
%:%1904=671%:%
%:%1905=672%:%
%:%1906=672%:%
%:%1907=672%:%
%:%1908=673%:%
%:%1909=673%:%
%:%1910=674%:%
%:%1911=674%:%
%:%1912=674%:%
%:%1913=675%:%
%:%1914=675%:%
%:%1915=676%:%
%:%1916=676%:%
%:%1917=676%:%
%:%1918=677%:%
%:%1919=677%:%
%:%1920=677%:%
%:%1921=678%:%
%:%1922=678%:%
%:%1923=678%:%
%:%1924=679%:%
%:%1925=679%:%
%:%1926=679%:%
%:%1927=680%:%
%:%1928=680%:%
%:%1929=681%:%
%:%1930=682%:%
%:%1931=682%:%
%:%1932=683%:%
%:%1933=684%:%
%:%1934=684%:%
%:%1935=685%:%
%:%1936=685%:%
%:%1937=685%:%
%:%1940=688%:%
%:%1941=689%:%
%:%1942=689%:%
%:%1943=689%:%
%:%1944=690%:%
%:%1945=690%:%
%:%1946=690%:%
%:%1947=691%:%
%:%1948=692%:%
%:%1949=693%:%
%:%1951=695%:%
%:%1952=696%:%
%:%1953=696%:%
%:%1954=696%:%
%:%1955=697%:%
%:%1956=698%:%
%:%1957=698%:%
%:%1958=699%:%
%:%1959=699%:%
%:%1960=700%:%
%:%1961=700%:%
%:%1962=701%:%
%:%1963=701%:%
%:%1964=702%:%
%:%1965=703%:%
%:%1966=703%:%
%:%1967=704%:%
%:%1968=704%:%
%:%1969=705%:%
%:%1970=705%:%
%:%1971=706%:%
%:%1972=706%:%
%:%1973=707%:%
%:%1974=707%:%
%:%1975=707%:%
%:%1976=708%:%
%:%1977=708%:%
%:%1978=709%:%
%:%1979=709%:%
%:%1980=709%:%
%:%1981=710%:%
%:%1982=710%:%
%:%1983=711%:%
%:%1984=711%:%
%:%1985=711%:%
%:%1986=712%:%
%:%1987=712%:%
%:%1988=712%:%
%:%1989=713%:%
%:%1990=713%:%
%:%1991=714%:%
%:%1992=715%:%
%:%1993=715%:%
%:%1995=717%:%
%:%1996=718%:%
%:%1997=718%:%
%:%1998=718%:%
%:%1999=719%:%
%:%2000=719%:%
%:%2001=720%:%
%:%2002=720%:%
%:%2003=720%:%
%:%2004=721%:%
%:%2005=721%:%
%:%2006=722%:%
%:%2007=722%:%
%:%2008=723%:%
%:%2009=724%:%
%:%2010=724%:%
%:%2013=727%:%
%:%2014=728%:%
%:%2015=728%:%
%:%2016=729%:%
%:%2017=729%:%
%:%2018=730%:%
%:%2019=730%:%
%:%2020=731%:%
%:%2021=731%:%
%:%2022=732%:%
%:%2023=732%:%
%:%2024=733%:%
%:%2025=733%:%
%:%2026=734%:%
%:%2027=735%:%
%:%2028=735%:%
%:%2029=735%:%
%:%2030=736%:%
%:%2031=736%:%
%:%2032=736%:%
%:%2033=737%:%
%:%2034=737%:%
%:%2035=738%:%
%:%2036=738%:%
%:%2037=738%:%
%:%2038=739%:%
%:%2039=739%:%
%:%2040=740%:%
%:%2041=740%:%
%:%2042=740%:%
%:%2043=741%:%
%:%2044=741%:%
%:%2045=741%:%
%:%2046=742%:%
%:%2047=742%:%
%:%2048=742%:%
%:%2049=743%:%
%:%2050=743%:%
%:%2051=743%:%
%:%2052=744%:%
%:%2053=744%:%
%:%2054=745%:%
%:%2055=745%:%
%:%2056=746%:%
%:%2057=746%:%
%:%2058=747%:%
%:%2059=747%:%
%:%2060=748%:%
%:%2061=749%:%
%:%2062=749%:%
%:%2063=749%:%
%:%2064=750%:%
%:%2065=750%:%
%:%2066=750%:%
%:%2067=751%:%
%:%2068=751%:%
%:%2069=752%:%
%:%2070=752%:%
%:%2071=752%:%
%:%2072=753%:%
%:%2073=753%:%
%:%2074=754%:%
%:%2075=754%:%
%:%2076=754%:%
%:%2077=755%:%
%:%2078=755%:%
%:%2079=755%:%
%:%2080=756%:%
%:%2081=756%:%
%:%2082=756%:%
%:%2083=757%:%
%:%2084=757%:%
%:%2085=757%:%
%:%2086=758%:%
%:%2087=758%:%
%:%2088=759%:%
%:%2089=759%:%
%:%2090=760%:%
%:%2091=760%:%
%:%2092=761%:%
%:%2093=761%:%
%:%2094=762%:%
%:%2095=762%:%
%:%2098=765%:%
%:%2099=766%:%
%:%2100=766%:%
%:%2101=767%:%
%:%2116=769%:%
%:%2126=771%:%
%:%2127=771%:%
%:%2128=772%:%
%:%2129=773%:%
%:%2130=774%:%
%:%2131=774%:%
%:%2132=775%:%
%:%2133=776%:%
%:%2140=777%:%
%:%2141=777%:%
%:%2142=778%:%
%:%2143=778%:%
%:%2144=779%:%
%:%2145=779%:%
%:%2146=779%:%
%:%2147=780%:%
%:%2148=780%:%
%:%2149=780%:%
%:%2150=781%:%
%:%2151=781%:%
%:%2152=782%:%
%:%2153=782%:%
%:%2154=782%:%
%:%2155=783%:%
%:%2156=783%:%
%:%2157=784%:%
%:%2163=784%:%
%:%2166=785%:%
%:%2167=786%:%
%:%2168=786%:%
%:%2169=787%:%
%:%2170=788%:%
%:%2171=789%:%
%:%2172=789%:%
%:%2173=790%:%
%:%2174=791%:%
%:%2181=792%:%
%:%2182=792%:%
%:%2183=793%:%
%:%2184=793%:%
%:%2185=794%:%
%:%2186=794%:%
%:%2187=795%:%
%:%2188=795%:%
%:%2189=795%:%
%:%2190=796%:%
%:%2191=796%:%
%:%2192=797%:%
%:%2193=797%:%
%:%2194=797%:%
%:%2195=798%:%
%:%2196=798%:%
%:%2197=798%:%
%:%2198=798%:%
%:%2199=799%:%
%:%2205=799%:%
%:%2208=800%:%
%:%2209=801%:%
%:%2210=801%:%
%:%2211=802%:%
%:%2212=803%:%
%:%2215=804%:%
%:%2219=804%:%
%:%2220=804%:%
%:%2221=804%:%
%:%2226=804%:%
%:%2229=805%:%
%:%2230=806%:%
%:%2231=806%:%
%:%2232=807%:%
%:%2233=808%:%
%:%2236=809%:%
%:%2240=809%:%
%:%2241=809%:%
%:%2242=809%:%
%:%2247=809%:%
%:%2250=810%:%
%:%2251=811%:%
%:%2252=811%:%
%:%2253=812%:%
%:%2254=813%:%
%:%2257=814%:%
%:%2261=814%:%
%:%2262=814%:%
%:%2263=814%:%
%:%2264=814%:%
%:%2269=814%:%
%:%2272=815%:%
%:%2273=816%:%
%:%2274=816%:%
%:%2275=817%:%
%:%2276=818%:%
%:%2277=819%:%
%:%2284=820%:%
%:%2285=820%:%
%:%2286=821%:%
%:%2287=821%:%
%:%2288=822%:%
%:%2289=822%:%
%:%2290=822%:%
%:%2291=823%:%
%:%2292=823%:%
%:%2293=824%:%
%:%2294=824%:%
%:%2295=825%:%
%:%2296=825%:%
%:%2297=825%:%
%:%2298=826%:%
%:%2299=826%:%
%:%2300=826%:%
%:%2301=827%:%
%:%2302=827%:%
%:%2303=827%:%
%:%2304=828%:%
%:%2305=828%:%
%:%2306=829%:%
%:%2307=829%:%
%:%2308=830%:%
%:%2309=830%:%
%:%2310=830%:%
%:%2311=831%:%
%:%2312=831%:%
%:%2313=831%:%
%:%2314=832%:%
%:%2315=832%:%
%:%2316=832%:%
%:%2317=833%:%
%:%2318=833%:%
%:%2319=833%:%
%:%2320=834%:%
%:%2326=834%:%
%:%2329=835%:%
%:%2330=836%:%
%:%2331=836%:%
%:%2332=837%:%
%:%2333=838%:%
%:%2334=839%:%
%:%2341=840%:%
%:%2342=840%:%
%:%2343=841%:%
%:%2344=841%:%
%:%2345=842%:%
%:%2346=842%:%
%:%2347=843%:%
%:%2348=843%:%
%:%2349=844%:%
%:%2350=845%:%
%:%2351=845%:%
%:%2352=845%:%
%:%2353=846%:%
%:%2354=846%:%
%:%2355=847%:%
%:%2356=847%:%
%:%2357=847%:%
%:%2358=848%:%
%:%2359=848%:%
%:%2360=849%:%
%:%2361=849%:%
%:%2362=849%:%
%:%2363=850%:%
%:%2364=850%:%
%:%2365=851%:%
%:%2366=851%:%
%:%2367=851%:%
%:%2368=852%:%
%:%2374=852%:%
%:%2377=853%:%
%:%2378=854%:%
%:%2379=854%:%
%:%2380=855%:%
%:%2381=856%:%
%:%2382=857%:%
%:%2385=858%:%
%:%2389=858%:%
%:%2390=858%:%
%:%2391=859%:%
%:%2392=860%:%
%:%2393=860%:%
%:%2398=860%:%
%:%2401=861%:%
%:%2402=862%:%
%:%2403=862%:%
%:%2404=863%:%
%:%2405=864%:%
%:%2406=865%:%
%:%2409=866%:%
%:%2413=866%:%
%:%2414=866%:%
%:%2415=867%:%
%:%2416=867%:%
%:%2421=867%:%
%:%2424=868%:%
%:%2425=869%:%
%:%2426=869%:%
%:%2427=870%:%
%:%2428=871%:%
%:%2429=872%:%
%:%2436=873%:%
%:%2437=873%:%
%:%2438=874%:%
%:%2439=874%:%
%:%2440=875%:%
%:%2441=875%:%
%:%2442=875%:%
%:%2443=876%:%
%:%2444=876%:%
%:%2445=877%:%
%:%2446=877%:%
%:%2447=877%:%
%:%2448=878%:%
%:%2449=878%:%
%:%2450=878%:%
%:%2451=879%:%
%:%2452=879%:%
%:%2453=879%:%
%:%2454=880%:%
%:%2455=880%:%
%:%2456=880%:%
%:%2457=881%:%
%:%2458=881%:%
%:%2459=881%:%
%:%2460=882%:%
%:%2461=882%:%
%:%2462=882%:%
%:%2463=883%:%
%:%2464=883%:%
%:%2465=883%:%
%:%2466=884%:%
%:%2467=884%:%
%:%2468=884%:%
%:%2469=885%:%
%:%2470=885%:%
%:%2471=885%:%
%:%2472=886%:%
%:%2473=886%:%
%:%2474=886%:%
%:%2475=887%:%
%:%2476=887%:%
%:%2477=887%:%
%:%2478=888%:%
%:%2479=888%:%
%:%2480=888%:%
%:%2481=889%:%
%:%2487=889%:%
%:%2490=890%:%
%:%2491=891%:%
%:%2492=891%:%
%:%2493=892%:%
%:%2494=893%:%
%:%2495=894%:%
%:%2502=895%:%
%:%2503=895%:%
%:%2504=896%:%
%:%2505=896%:%
%:%2506=897%:%
%:%2507=897%:%
%:%2508=897%:%
%:%2509=898%:%
%:%2510=898%:%
%:%2511=899%:%
%:%2512=899%:%
%:%2513=899%:%
%:%2514=900%:%
%:%2515=900%:%
%:%2516=901%:%
%:%2517=901%:%
%:%2518=902%:%
%:%2519=902%:%
%:%2520=903%:%
%:%2521=903%:%
%:%2522=904%:%
%:%2523=904%:%
%:%2524=905%:%
%:%2525=905%:%
%:%2526=906%:%
%:%2527=906%:%
%:%2528=907%:%
%:%2529=907%:%
%:%2530=908%:%
%:%2531=908%:%
%:%2532=908%:%
%:%2533=909%:%
%:%2534=909%:%
%:%2535=909%:%
%:%2536=910%:%
%:%2537=910%:%
%:%2538=910%:%
%:%2539=911%:%
%:%2540=911%:%
%:%2541=912%:%
%:%2542=912%:%
%:%2543=912%:%
%:%2544=913%:%
%:%2545=913%:%
%:%2546=914%:%
%:%2547=914%:%
%:%2548=915%:%
%:%2549=915%:%
%:%2550=915%:%
%:%2551=916%:%
%:%2552=916%:%
%:%2553=917%:%
%:%2554=917%:%
%:%2555=917%:%
%:%2556=918%:%
%:%2557=918%:%
%:%2558=918%:%
%:%2559=919%:%
%:%2560=919%:%
%:%2561=920%:%
%:%2562=920%:%
%:%2563=920%:%
%:%2564=921%:%
%:%2565=921%:%
%:%2566=922%:%
%:%2567=922%:%
%:%2568=922%:%
%:%2569=923%:%
%:%2570=923%:%
%:%2571=923%:%
%:%2572=924%:%
%:%2578=924%:%
%:%2581=925%:%
%:%2582=926%:%
%:%2583=926%:%
%:%2584=927%:%
%:%2585=928%:%
%:%2586=929%:%
%:%2593=930%:%
%:%2594=930%:%
%:%2595=931%:%
%:%2596=931%:%
%:%2597=932%:%
%:%2598=932%:%
%:%2599=932%:%
%:%2600=933%:%
%:%2601=933%:%
%:%2602=934%:%
%:%2603=934%:%
%:%2604=934%:%
%:%2605=935%:%
%:%2606=935%:%
%:%2607=936%:%
%:%2608=936%:%
%:%2609=937%:%
%:%2610=937%:%
%:%2611=938%:%
%:%2612=938%:%
%:%2613=938%:%
%:%2614=939%:%
%:%2615=939%:%
%:%2616=940%:%
%:%2617=940%:%
%:%2618=941%:%
%:%2619=941%:%
%:%2620=942%:%
%:%2621=942%:%
%:%2622=943%:%
%:%2623=943%:%
%:%2624=944%:%
%:%2625=944%:%
%:%2626=945%:%
%:%2627=945%:%
%:%2628=946%:%
%:%2629=946%:%
%:%2630=947%:%
%:%2631=947%:%
%:%2632=947%:%
%:%2633=948%:%
%:%2634=948%:%
%:%2635=948%:%
%:%2636=948%:%
%:%2637=949%:%
%:%2638=949%:%
%:%2639=949%:%
%:%2640=950%:%
%:%2641=950%:%
%:%2642=951%:%
%:%2643=951%:%
%:%2644=952%:%
%:%2645=952%:%
%:%2646=953%:%
%:%2647=953%:%
%:%2648=954%:%
%:%2649=954%:%
%:%2650=955%:%
%:%2651=955%:%
%:%2652=955%:%
%:%2653=956%:%
%:%2654=956%:%
%:%2655=956%:%
%:%2656=957%:%
%:%2657=957%:%
%:%2658=958%:%
%:%2659=958%:%
%:%2660=958%:%
%:%2661=959%:%
%:%2662=959%:%
%:%2663=959%:%
%:%2664=960%:%
%:%2665=960%:%
%:%2666=960%:%
%:%2667=961%:%
%:%2668=961%:%
%:%2669=962%:%
%:%2670=962%:%
%:%2671=962%:%
%:%2672=963%:%
%:%2673=963%:%
%:%2674=963%:%
%:%2675=964%:%
%:%2676=964%:%
%:%2677=965%:%
%:%2678=966%:%
%:%2679=966%:%
%:%2680=967%:%
%:%2681=967%:%
%:%2682=968%:%
%:%2683=968%:%
%:%2684=969%:%
%:%2685=969%:%
%:%2686=970%:%
%:%2687=970%:%
%:%2688=970%:%
%:%2689=971%:%
%:%2690=971%:%
%:%2691=971%:%
%:%2692=972%:%
%:%2693=972%:%
%:%2694=972%:%
%:%2695=973%:%
%:%2696=973%:%
%:%2697=974%:%
%:%2698=974%:%
%:%2699=974%:%
%:%2700=975%:%
%:%2701=975%:%
%:%2702=976%:%
%:%2703=976%:%
%:%2704=976%:%
%:%2705=976%:%
%:%2706=976%:%
%:%2707=977%:%
%:%2708=977%:%
%:%2709=978%:%
%:%2710=978%:%
%:%2711=978%:%
%:%2712=979%:%
%:%2713=980%:%
%:%2714=980%:%
%:%2715=980%:%
%:%2716=981%:%
%:%2717=981%:%
%:%2718=982%:%
%:%2719=983%:%
%:%2720=983%:%
%:%2721=984%:%
%:%2722=984%:%
%:%2723=985%:%
%:%2724=985%:%
%:%2725=986%:%
%:%2726=986%:%
%:%2727=986%:%
%:%2728=987%:%
%:%2729=987%:%
%:%2730=987%:%
%:%2731=988%:%
%:%2732=988%:%
%:%2733=989%:%
%:%2734=989%:%
%:%2735=990%:%
%:%2736=990%:%
%:%2737=991%:%
%:%2738=991%:%
%:%2739=991%:%
%:%2740=992%:%
%:%2741=992%:%
%:%2742=992%:%
%:%2743=993%:%
%:%2744=993%:%
%:%2745=994%:%
%:%2746=994%:%
%:%2747=995%:%
%:%2748=995%:%
%:%2749=995%:%
%:%2750=996%:%
%:%2751=996%:%
%:%2752=996%:%
%:%2753=997%:%
%:%2768=999%:%
%:%2778=1001%:%
%:%2779=1001%:%
%:%2780=1002%:%
%:%2784=1006%:%
%:%2786=1007%:%
%:%2787=1007%:%
%:%2788=1008%:%
%:%2789=1009%:%
%:%2790=1010%:%
%:%2791=1010%:%
%:%2792=1011%:%
%:%2795=1012%:%
%:%2799=1012%:%
%:%2800=1012%:%
%:%2805=1012%:%
%:%2808=1013%:%
%:%2809=1014%:%
%:%2810=1014%:%
%:%2811=1015%:%
%:%2812=1016%:%
%:%2815=1017%:%
%:%2819=1017%:%
%:%2820=1017%:%
%:%2821=1017%:%
%:%2826=1017%:%
%:%2829=1018%:%
%:%2830=1019%:%
%:%2831=1019%:%
%:%2832=1020%:%
%:%2833=1021%:%
%:%2834=1022%:%
%:%2835=1022%:%
%:%2836=1023%:%
%:%2839=1024%:%
%:%2843=1024%:%
%:%2844=1024%:%
%:%2845=1024%:%
%:%2850=1024%:%
%:%2853=1025%:%
%:%2854=1026%:%
%:%2855=1026%:%
%:%2856=1027%:%
%:%2857=1028%:%
%:%2860=1029%:%
%:%2864=1029%:%
%:%2865=1029%:%
%:%2866=1029%:%
%:%2871=1029%:%
%:%2874=1030%:%
%:%2875=1031%:%
%:%2876=1031%:%
%:%2877=1032%:%
%:%2878=1033%:%
%:%2881=1034%:%
%:%2885=1034%:%
%:%2886=1034%:%
%:%2891=1034%:%
%:%2894=1035%:%
%:%2895=1036%:%
%:%2896=1036%:%
%:%2897=1037%:%
%:%2898=1038%:%
%:%2901=1039%:%
%:%2905=1039%:%
%:%2906=1039%:%
%:%2907=1039%:%
%:%2916=1041%:%
%:%2918=1042%:%
%:%2919=1042%:%
%:%2920=1043%:%
%:%2921=1044%:%
%:%2928=1045%:%
%:%2929=1045%:%
%:%2930=1046%:%
%:%2931=1046%:%
%:%2932=1047%:%
%:%2933=1047%:%
%:%2934=1048%:%
%:%2935=1048%:%
%:%2936=1049%:%
%:%2937=1049%:%
%:%2938=1050%:%
%:%2939=1050%:%
%:%2940=1051%:%
%:%2941=1051%:%
%:%2942=1052%:%
%:%2943=1052%:%
%:%2944=1053%:%
%:%2945=1053%:%
%:%2946=1054%:%
%:%2947=1054%:%
%:%2948=1054%:%
%:%2949=1055%:%
%:%2950=1055%:%
%:%2951=1055%:%
%:%2952=1056%:%
%:%2953=1057%:%
%:%2954=1057%:%
%:%2955=1058%:%
%:%2956=1058%:%
%:%2957=1059%:%
%:%2958=1059%:%
%:%2959=1060%:%
%:%2960=1060%:%
%:%2961=1061%:%
%:%2962=1061%:%
%:%2963=1062%:%
%:%2964=1062%:%
%:%2965=1062%:%
%:%2966=1063%:%
%:%2967=1063%:%
%:%2968=1064%:%
%:%2969=1064%:%
%:%2970=1065%:%
%:%2971=1065%:%
%:%2972=1066%:%
%:%2973=1066%:%
%:%2974=1067%:%
%:%2975=1068%:%
%:%2976=1069%:%
%:%2977=1069%:%
%:%2978=1070%:%
%:%2979=1070%:%
%:%2980=1070%:%
%:%2981=1071%:%
%:%2982=1071%:%
%:%2983=1072%:%
%:%2984=1072%:%
%:%2985=1073%:%
%:%2986=1073%:%
%:%2987=1074%:%
%:%2988=1074%:%
%:%2989=1075%:%
%:%2990=1075%:%
%:%2991=1076%:%
%:%2992=1076%:%
%:%2993=1077%:%
%:%2994=1077%:%
%:%2995=1078%:%
%:%2996=1078%:%
%:%2997=1079%:%
%:%2998=1079%:%
%:%2999=1080%:%
%:%3000=1080%:%
%:%3001=1081%:%
%:%3002=1081%:%
%:%3003=1081%:%
%:%3004=1082%:%
%:%3005=1082%:%
%:%3006=1083%:%
%:%3007=1083%:%
%:%3008=1084%:%
%:%3009=1084%:%
%:%3010=1084%:%
%:%3011=1085%:%
%:%3012=1085%:%
%:%3013=1085%:%
%:%3014=1086%:%
%:%3015=1086%:%
%:%3016=1087%:%
%:%3017=1087%:%
%:%3023=1093%:%
%:%3024=1094%:%
%:%3025=1094%:%
%:%3026=1095%:%
%:%3027=1095%:%
%:%3028=1096%:%
%:%3034=1096%:%
%:%3037=1097%:%
%:%3038=1098%:%
%:%3039=1098%:%
%:%3040=1099%:%
%:%3041=1100%:%
%:%3048=1101%:%
%:%3049=1101%:%
%:%3050=1102%:%
%:%3051=1102%:%
%:%3052=1103%:%
%:%3053=1103%:%
%:%3054=1103%:%
%:%3055=1104%:%
%:%3056=1104%:%
%:%3057=1105%:%
%:%3058=1105%:%
%:%3059=1105%:%
%:%3060=1106%:%
%:%3061=1106%:%
%:%3062=1107%:%
%:%3063=1107%:%
%:%3064=1108%:%
%:%3065=1108%:%
%:%3066=1109%:%
%:%3067=1109%:%
%:%3068=1110%:%
%:%3069=1110%:%
%:%3070=1110%:%
%:%3071=1111%:%
%:%3072=1111%:%
%:%3073=1111%:%
%:%3074=1112%:%
%:%3075=1112%:%
%:%3076=1112%:%
%:%3077=1113%:%
%:%3078=1113%:%
%:%3079=1113%:%
%:%3080=1114%:%
%:%3081=1114%:%
%:%3082=1114%:%
%:%3083=1115%:%
%:%3084=1115%:%
%:%3085=1116%:%
%:%3086=1116%:%
%:%3087=1117%:%
%:%3088=1117%:%
%:%3089=1118%:%
%:%3090=1118%:%
%:%3091=1119%:%
%:%3092=1119%:%
%:%3093=1120%:%
%:%3094=1120%:%
%:%3095=1120%:%
%:%3096=1120%:%
%:%3097=1121%:%
%:%3098=1121%:%
%:%3099=1122%:%
%:%3100=1122%:%
%:%3101=1122%:%
%:%3102=1122%:%
%:%3103=1123%:%
%:%3104=1123%:%
%:%3105=1124%:%
%:%3106=1124%:%
%:%3107=1125%:%
%:%3108=1125%:%
%:%3109=1125%:%
%:%3110=1126%:%
%:%3111=1126%:%
%:%3112=1126%:%
%:%3113=1127%:%
%:%3114=1127%:%
%:%3115=1128%:%
%:%3116=1128%:%
%:%3117=1128%:%
%:%3118=1129%:%
%:%3119=1129%:%
%:%3120=1129%:%
%:%3121=1130%:%
%:%3122=1130%:%
%:%3123=1130%:%
%:%3124=1131%:%
%:%3125=1131%:%
%:%3126=1132%:%
%:%3127=1132%:%
%:%3128=1132%:%
%:%3129=1133%:%
%:%3130=1133%:%
%:%3131=1133%:%
%:%3132=1134%:%
%:%3133=1134%:%
%:%3134=1134%:%
%:%3135=1135%:%
%:%3136=1135%:%
%:%3137=1136%:%
%:%3138=1136%:%
%:%3139=1136%:%
%:%3140=1137%:%
%:%3141=1137%:%
%:%3142=1138%:%
%:%3143=1138%:%
%:%3144=1138%:%
%:%3145=1139%:%
%:%3146=1139%:%
%:%3147=1139%:%
%:%3148=1140%:%
%:%3149=1140%:%
%:%3150=1141%:%
%:%3151=1141%:%
%:%3152=1141%:%
%:%3153=1142%:%
%:%3154=1142%:%
%:%3155=1143%:%
%:%3156=1143%:%
%:%3157=1144%:%
%:%3158=1144%:%
%:%3159=1145%:%
%:%3160=1145%:%
%:%3161=1146%:%
%:%3171=1148%:%
%:%3173=1149%:%
%:%3174=1149%:%
%:%3175=1150%:%
%:%3176=1151%:%
%:%3177=1152%:%
%:%3184=1153%:%
%:%3185=1153%:%
%:%3186=1154%:%
%:%3187=1154%:%
%:%3188=1155%:%
%:%3189=1155%:%
%:%3190=1155%:%
%:%3191=1155%:%
%:%3192=1156%:%
%:%3193=1156%:%
%:%3194=1157%:%
%:%3195=1157%:%
%:%3196=1157%:%
%:%3197=1158%:%
%:%3198=1158%:%
%:%3199=1159%:%
%:%3200=1159%:%
%:%3201=1159%:%
%:%3202=1160%:%
%:%3203=1161%:%
%:%3204=1161%:%
%:%3205=1162%:%
%:%3206=1162%:%
%:%3207=1163%:%
%:%3208=1163%:%
%:%3209=1164%:%
%:%3210=1164%:%
%:%3211=1165%:%
%:%3212=1165%:%
%:%3213=1165%:%
%:%3214=1166%:%
%:%3215=1166%:%
%:%3216=1167%:%
%:%3217=1168%:%
%:%3218=1168%:%
%:%3219=1169%:%
%:%3220=1169%:%
%:%3221=1169%:%
%:%3222=1170%:%
%:%3223=1171%:%
%:%3224=1171%:%
%:%3225=1172%:%
%:%3226=1172%:%
%:%3227=1173%:%
%:%3228=1173%:%
%:%3229=1174%:%
%:%3230=1174%:%
%:%3231=1175%:%
%:%3232=1175%:%
%:%3233=1175%:%
%:%3234=1176%:%
%:%3235=1177%:%
%:%3236=1177%:%
%:%3237=1177%:%
%:%3238=1177%:%
%:%3239=1178%:%
%:%3240=1178%:%
%:%3241=1178%:%
%:%3242=1179%:%
%:%3243=1179%:%
%:%3244=1180%:%
%:%3245=1180%:%
%:%3246=1181%:%
%:%3247=1181%:%
%:%3248=1182%:%
%:%3249=1182%:%
%:%3250=1183%:%
%:%3251=1183%:%
%:%3252=1184%:%
%:%3253=1185%:%
%:%3254=1185%:%
%:%3255=1186%:%
%:%3256=1186%:%
%:%3257=1186%:%
%:%3258=1187%:%
%:%3259=1187%:%
%:%3260=1187%:%
%:%3261=1188%:%
%:%3262=1189%:%
%:%3263=1189%:%
%:%3264=1190%:%
%:%3265=1191%:%
%:%3266=1191%:%
%:%3267=1192%:%
%:%3268=1193%:%
%:%3269=1193%:%
%:%3270=1194%:%
%:%3271=1194%:%
%:%3272=1194%:%
%:%3273=1194%:%
%:%3274=1195%:%
%:%3275=1196%:%
%:%3276=1196%:%
%:%3277=1197%:%
%:%3278=1197%:%
%:%3279=1198%:%
%:%3280=1198%:%
%:%3281=1199%:%
%:%3282=1199%:%
%:%3283=1200%:%
%:%3284=1200%:%
%:%3285=1201%:%
%:%3286=1201%:%
%:%3287=1202%:%
%:%3288=1202%:%
%:%3289=1202%:%
%:%3290=1203%:%
%:%3291=1203%:%
%:%3292=1203%:%
%:%3293=1203%:%
%:%3294=1204%:%
%:%3295=1204%:%
%:%3296=1204%:%
%:%3297=1205%:%
%:%3298=1205%:%
%:%3299=1205%:%
%:%3300=1206%:%
%:%3301=1206%:%
%:%3302=1206%:%
%:%3303=1206%:%
%:%3304=1206%:%
%:%3305=1207%:%
%:%3306=1207%:%
%:%3307=1208%:%
%:%3308=1209%:%
%:%3309=1209%:%
%:%3310=1210%:%
%:%3311=1210%:%
%:%3312=1211%:%
%:%3313=1211%:%
%:%3314=1212%:%
%:%3315=1212%:%
%:%3316=1213%:%
%:%3317=1213%:%
%:%3318=1213%:%
%:%3319=1214%:%
%:%3320=1214%:%
%:%3321=1215%:%
%:%3322=1215%:%
%:%3323=1215%:%
%:%3324=1216%:%
%:%3325=1216%:%
%:%3326=1217%:%
%:%3327=1217%:%
%:%3328=1217%:%
%:%3329=1217%:%
%:%3330=1217%:%
%:%3331=1218%:%
%:%3332=1218%:%
%:%3333=1219%:%
%:%3334=1219%:%
%:%3335=1220%:%
%:%3336=1220%:%
%:%3337=1220%:%
%:%3338=1221%:%
%:%3339=1221%:%
%:%3340=1221%:%
%:%3341=1222%:%
%:%3342=1222%:%
%:%3343=1222%:%
%:%3344=1223%:%
%:%3345=1223%:%
%:%3346=1224%:%
%:%3347=1224%:%
%:%3348=1225%:%
%:%3349=1225%:%
%:%3350=1225%:%
%:%3351=1226%:%
%:%3352=1226%:%
%:%3353=1227%:%
%:%3354=1227%:%
%:%3355=1227%:%
%:%3356=1228%:%
%:%3357=1228%:%
%:%3358=1229%:%
%:%3359=1229%:%
%:%3360=1230%:%
%:%3361=1230%:%
%:%3362=1231%:%
%:%3363=1231%:%
%:%3364=1232%:%
%:%3365=1232%:%
%:%3366=1233%:%
%:%3367=1233%:%
%:%3368=1234%:%
%:%3369=1234%:%
%:%3370=1235%:%
%:%3371=1235%:%
%:%3372=1236%:%
%:%3373=1236%:%
%:%3374=1237%:%
%:%3375=1237%:%
%:%3376=1238%:%
%:%3377=1238%:%
%:%3378=1239%:%
%:%3379=1239%:%
%:%3380=1240%:%
%:%3381=1241%:%
%:%3382=1241%:%
%:%3383=1242%:%
%:%3384=1242%:%
%:%3385=1242%:%
%:%3386=1242%:%
%:%3387=1243%:%
%:%3388=1243%:%
%:%3389=1244%:%
%:%3390=1244%:%
%:%3391=1244%:%
%:%3392=1245%:%
%:%3393=1245%:%
%:%3394=1246%:%
%:%3395=1246%:%
%:%3396=1247%:%
%:%3397=1247%:%
%:%3398=1248%:%
%:%3399=1248%:%
%:%3400=1249%:%
%:%3401=1249%:%
%:%3402=1250%:%
%:%3403=1250%:%
%:%3404=1250%:%
%:%3405=1251%:%
%:%3406=1251%:%
%:%3407=1251%:%
%:%3408=1252%:%
%:%3409=1252%:%
%:%3410=1253%:%
%:%3411=1253%:%
%:%3412=1253%:%
%:%3413=1254%:%
%:%3414=1254%:%
%:%3415=1255%:%
%:%3416=1255%:%
%:%3417=1255%:%
%:%3418=1256%:%
%:%3419=1256%:%
%:%3420=1257%:%
%:%3421=1257%:%
%:%3422=1257%:%
%:%3423=1258%:%
%:%3424=1258%:%
%:%3425=1259%:%
%:%3426=1259%:%
%:%3427=1259%:%
%:%3428=1259%:%
%:%3429=1259%:%
%:%3430=1260%:%
%:%3431=1260%:%
%:%3432=1261%:%
%:%3433=1261%:%
%:%3434=1261%:%
%:%3435=1262%:%
%:%3436=1262%:%
%:%3437=1263%:%
%:%3443=1263%:%
%:%3448=1264%:%
%:%3453=1265%:%

%
\begin{isabellebody}%
\setisabellecontext{Equivalence}%
%
\isadelimtheory
%
\endisadelimtheory
%
\isatagtheory
\isacommand{theory}\isamarkupfalse%
\ Equivalence\isanewline
\ \ \isakeyword{imports}\ Truth\isanewline
\isakeyword{begin}%
\endisatagtheory
{\isafoldtheory}%
%
\isadelimtheory
%
\endisadelimtheory
%
\isadelimdocument
%
\endisadelimdocument
%
\isatagdocument
%
\isamarkupsection{Equivalence Classes%
}
\isamarkuptrue%
%
\endisatagdocument
{\isafolddocument}%
%
\isadelimdocument
%
\endisadelimdocument
\isacommand{definition}\isamarkupfalse%
\ reflexive{\isacharunderscore}{\kern0pt}on\ {\isacharcolon}{\kern0pt}{\isacharcolon}{\kern0pt}\ {\isachardoublequoteopen}cset\ {\isasymRightarrow}\ cset\ {\isasymtimes}\ cfunc\ {\isasymRightarrow}\ bool{\isachardoublequoteclose}\ \isakeyword{where}\isanewline
\ \ {\isachardoublequoteopen}reflexive{\isacharunderscore}{\kern0pt}on\ X\ R\ {\isacharequal}{\kern0pt}\ {\isacharparenleft}{\kern0pt}R\ {\isasymsubseteq}\isactrlsub c\ X{\isasymtimes}\isactrlsub cX\ {\isasymand}\ \isanewline
\ \ \ \ {\isacharparenleft}{\kern0pt}{\isasymforall}x{\isachardot}{\kern0pt}\ x\ {\isasymin}\isactrlsub c\ X\ {\isasymlongrightarrow}\ {\isacharparenleft}{\kern0pt}{\isasymlangle}x{\isacharcomma}{\kern0pt}x{\isasymrangle}\ {\isasymin}\isactrlbsub X{\isasymtimes}\isactrlsub cX\isactrlesub \ R{\isacharparenright}{\kern0pt}{\isacharparenright}{\kern0pt}{\isacharparenright}{\kern0pt}{\isachardoublequoteclose}\isanewline
\isanewline
\isacommand{definition}\isamarkupfalse%
\ symmetric{\isacharunderscore}{\kern0pt}on\ {\isacharcolon}{\kern0pt}{\isacharcolon}{\kern0pt}\ {\isachardoublequoteopen}cset\ {\isasymRightarrow}\ cset\ {\isasymtimes}\ cfunc\ {\isasymRightarrow}\ bool{\isachardoublequoteclose}\ \isakeyword{where}\isanewline
\ \ {\isachardoublequoteopen}symmetric{\isacharunderscore}{\kern0pt}on\ X\ R\ {\isacharequal}{\kern0pt}\ {\isacharparenleft}{\kern0pt}R\ \ {\isasymsubseteq}\isactrlsub c\ X{\isasymtimes}\isactrlsub cX\ {\isasymand}\isanewline
\ \ \ \ {\isacharparenleft}{\kern0pt}{\isasymforall}x\ y{\isachardot}{\kern0pt}\ x\ {\isasymin}\isactrlsub c\ X\ {\isasymand}\ \ y\ {\isasymin}\isactrlsub c\ X\ {\isasymlongrightarrow}\ \isanewline
\ \ \ \ \ \ {\isacharparenleft}{\kern0pt}{\isasymlangle}x{\isacharcomma}{\kern0pt}y{\isasymrangle}\ {\isasymin}\isactrlbsub X{\isasymtimes}\isactrlsub cX\isactrlesub \ R\ {\isasymlongrightarrow}\ {\isasymlangle}y{\isacharcomma}{\kern0pt}x{\isasymrangle}\ {\isasymin}\isactrlbsub X{\isasymtimes}\isactrlsub cX\isactrlesub \ R{\isacharparenright}{\kern0pt}{\isacharparenright}{\kern0pt}{\isacharparenright}{\kern0pt}{\isachardoublequoteclose}\ \isanewline
\isanewline
\isacommand{definition}\isamarkupfalse%
\ transitive{\isacharunderscore}{\kern0pt}on\ {\isacharcolon}{\kern0pt}{\isacharcolon}{\kern0pt}\ {\isachardoublequoteopen}cset\ {\isasymRightarrow}\ cset\ {\isasymtimes}\ cfunc\ {\isasymRightarrow}\ bool{\isachardoublequoteclose}\ \isakeyword{where}\isanewline
\ \ {\isachardoublequoteopen}transitive{\isacharunderscore}{\kern0pt}on\ X\ R\ {\isacharequal}{\kern0pt}\ {\isacharparenleft}{\kern0pt}R\ \ {\isasymsubseteq}\isactrlsub c\ X{\isasymtimes}\isactrlsub cX\ {\isasymand}\isanewline
\ \ \ \ {\isacharparenleft}{\kern0pt}{\isasymforall}x\ y\ z{\isachardot}{\kern0pt}\ x\ {\isasymin}\isactrlsub c\ X\ {\isasymand}\ \ y\ {\isasymin}\isactrlsub c\ X\ {\isasymand}\ z\ {\isasymin}\isactrlsub c\ X\ \ {\isasymlongrightarrow}\isanewline
\ \ \ \ \ \ {\isacharparenleft}{\kern0pt}{\isasymlangle}x{\isacharcomma}{\kern0pt}y{\isasymrangle}\ {\isasymin}\isactrlbsub X{\isasymtimes}\isactrlsub cX\isactrlesub \ R\ {\isasymand}\ {\isasymlangle}y{\isacharcomma}{\kern0pt}z{\isasymrangle}\ {\isasymin}\isactrlbsub X{\isasymtimes}\isactrlsub cX\isactrlesub \ R\ {\isasymlongrightarrow}\ {\isasymlangle}x{\isacharcomma}{\kern0pt}z{\isasymrangle}\ {\isasymin}\isactrlbsub X{\isasymtimes}\isactrlsub cX\isactrlesub \ R{\isacharparenright}{\kern0pt}{\isacharparenright}{\kern0pt}{\isacharparenright}{\kern0pt}{\isachardoublequoteclose}\isanewline
\isanewline
\isacommand{definition}\isamarkupfalse%
\ equiv{\isacharunderscore}{\kern0pt}rel{\isacharunderscore}{\kern0pt}on\ {\isacharcolon}{\kern0pt}{\isacharcolon}{\kern0pt}\ {\isachardoublequoteopen}cset\ {\isasymRightarrow}\ cset\ {\isasymtimes}\ cfunc\ {\isasymRightarrow}\ bool{\isachardoublequoteclose}\ \isakeyword{where}\isanewline
\ \ {\isachardoublequoteopen}equiv{\isacharunderscore}{\kern0pt}rel{\isacharunderscore}{\kern0pt}on\ X\ R\ \ {\isasymlongleftrightarrow}\ {\isacharparenleft}{\kern0pt}reflexive{\isacharunderscore}{\kern0pt}on\ X\ R\ {\isasymand}\ symmetric{\isacharunderscore}{\kern0pt}on\ X\ R\ {\isasymand}\ transitive{\isacharunderscore}{\kern0pt}on\ X\ R{\isacharparenright}{\kern0pt}{\isachardoublequoteclose}\isanewline
\isanewline
\isacommand{definition}\isamarkupfalse%
\ const{\isacharunderscore}{\kern0pt}on{\isacharunderscore}{\kern0pt}rel\ {\isacharcolon}{\kern0pt}{\isacharcolon}{\kern0pt}\ {\isachardoublequoteopen}cset\ {\isasymRightarrow}\ cset\ {\isasymtimes}\ cfunc\ {\isasymRightarrow}\ cfunc\ {\isasymRightarrow}\ bool{\isachardoublequoteclose}\ \isakeyword{where}\isanewline
\ \ {\isachardoublequoteopen}const{\isacharunderscore}{\kern0pt}on{\isacharunderscore}{\kern0pt}rel\ X\ R\ f\ {\isacharequal}{\kern0pt}\ {\isacharparenleft}{\kern0pt}{\isasymforall}x\ y{\isachardot}{\kern0pt}\ x\ {\isasymin}\isactrlsub c\ X\ {\isasymlongrightarrow}\ y\ {\isasymin}\isactrlsub c\ X\ {\isasymlongrightarrow}\ {\isasymlangle}x{\isacharcomma}{\kern0pt}\ y{\isasymrangle}\ {\isasymin}\isactrlbsub X{\isasymtimes}\isactrlsub cX\isactrlesub \ R\ {\isasymlongrightarrow}\ f\ {\isasymcirc}\isactrlsub c\ x\ {\isacharequal}{\kern0pt}\ f\ {\isasymcirc}\isactrlsub c\ y{\isacharparenright}{\kern0pt}{\isachardoublequoteclose}\isanewline
\isanewline
\isacommand{lemma}\isamarkupfalse%
\ reflexive{\isacharunderscore}{\kern0pt}def{\isadigit{2}}{\isacharcolon}{\kern0pt}\isanewline
\ \ \isakeyword{assumes}\ reflexive{\isacharunderscore}{\kern0pt}Y{\isacharcolon}{\kern0pt}\ {\isachardoublequoteopen}reflexive{\isacharunderscore}{\kern0pt}on\ X\ {\isacharparenleft}{\kern0pt}Y{\isacharcomma}{\kern0pt}\ m{\isacharparenright}{\kern0pt}{\isachardoublequoteclose}\isanewline
\ \ \isakeyword{assumes}\ x{\isacharunderscore}{\kern0pt}type{\isacharcolon}{\kern0pt}\ {\isachardoublequoteopen}x\ {\isasymin}\isactrlsub c\ X{\isachardoublequoteclose}\ \isanewline
\ \ \isakeyword{shows}\ {\isachardoublequoteopen}{\isasymexists}y{\isachardot}{\kern0pt}\ y\ {\isasymin}\isactrlsub c\ Y\ {\isasymand}\ \ m\ {\isasymcirc}\isactrlsub c\ y\ {\isacharequal}{\kern0pt}\ {\isasymlangle}x{\isacharcomma}{\kern0pt}x{\isasymrangle}{\isachardoublequoteclose}\isanewline
%
\isadelimproof
\ \ %
\endisadelimproof
%
\isatagproof
\isacommand{using}\isamarkupfalse%
\ assms\ \isacommand{unfolding}\isamarkupfalse%
\ reflexive{\isacharunderscore}{\kern0pt}on{\isacharunderscore}{\kern0pt}def\ relative{\isacharunderscore}{\kern0pt}member{\isacharunderscore}{\kern0pt}def\ factors{\isacharunderscore}{\kern0pt}through{\isacharunderscore}{\kern0pt}def{\isadigit{2}}\isanewline
\isacommand{proof}\isamarkupfalse%
\ {\isacharminus}{\kern0pt}\isanewline
\ \ \ \ \isacommand{assume}\isamarkupfalse%
\ a{\isadigit{1}}{\isacharcolon}{\kern0pt}\ {\isachardoublequoteopen}{\isacharparenleft}{\kern0pt}Y{\isacharcomma}{\kern0pt}\ m{\isacharparenright}{\kern0pt}\ {\isasymsubseteq}\isactrlsub c\ X\ {\isasymtimes}\isactrlsub c\ X\ {\isasymand}\ {\isacharparenleft}{\kern0pt}{\isasymforall}x{\isachardot}{\kern0pt}\ x\ {\isasymin}\isactrlsub c\ X\ {\isasymlongrightarrow}\ {\isasymlangle}x{\isacharcomma}{\kern0pt}x{\isasymrangle}\ {\isasymin}\isactrlsub c\ X\ {\isasymtimes}\isactrlsub c\ X\ {\isasymand}\ monomorphism\ {\isacharparenleft}{\kern0pt}snd\ {\isacharparenleft}{\kern0pt}Y{\isacharcomma}{\kern0pt}\ m{\isacharparenright}{\kern0pt}{\isacharparenright}{\kern0pt}\ {\isasymand}\ snd\ {\isacharparenleft}{\kern0pt}Y{\isacharcomma}{\kern0pt}\ m{\isacharparenright}{\kern0pt}\ {\isacharcolon}{\kern0pt}\ fst\ {\isacharparenleft}{\kern0pt}Y{\isacharcomma}{\kern0pt}\ m{\isacharparenright}{\kern0pt}\ {\isasymrightarrow}\ X\ {\isasymtimes}\isactrlsub c\ X\ {\isasymand}\ {\isasymlangle}x{\isacharcomma}{\kern0pt}x{\isasymrangle}\ factorsthru\ snd\ {\isacharparenleft}{\kern0pt}Y{\isacharcomma}{\kern0pt}\ m{\isacharparenright}{\kern0pt}{\isacharparenright}{\kern0pt}{\isachardoublequoteclose}\isanewline
\ \ \ \ \isacommand{have}\isamarkupfalse%
\ xx{\isacharunderscore}{\kern0pt}type{\isacharcolon}{\kern0pt}\ {\isachardoublequoteopen}{\isasymlangle}x{\isacharcomma}{\kern0pt}x{\isasymrangle}\ {\isasymin}\isactrlsub c\ X\ {\isasymtimes}\isactrlsub c\ X{\isachardoublequoteclose}\isanewline
\ \ \ \ \ \ \isacommand{by}\isamarkupfalse%
\ {\isacharparenleft}{\kern0pt}typecheck{\isacharunderscore}{\kern0pt}cfuncs{\isacharcomma}{\kern0pt}\ simp\ add{\isacharcolon}{\kern0pt}\ x{\isacharunderscore}{\kern0pt}type{\isacharparenright}{\kern0pt}\isanewline
\ \ \ \ \isacommand{have}\isamarkupfalse%
\ {\isachardoublequoteopen}{\isasymlangle}x{\isacharcomma}{\kern0pt}x{\isasymrangle}\ factorsthru\ m{\isachardoublequoteclose}\isanewline
\ \ \ \ \ \ \isacommand{using}\isamarkupfalse%
\ a{\isadigit{1}}\ x{\isacharunderscore}{\kern0pt}type\ \isacommand{by}\isamarkupfalse%
\ auto\isanewline
\ \ \ \ \isacommand{then}\isamarkupfalse%
\ \isacommand{show}\isamarkupfalse%
\ {\isacharquery}{\kern0pt}thesis\isanewline
\ \ \ \ \ \ \isacommand{using}\isamarkupfalse%
\ a{\isadigit{1}}\ xx{\isacharunderscore}{\kern0pt}type\ cfunc{\isacharunderscore}{\kern0pt}type{\isacharunderscore}{\kern0pt}def\ factors{\isacharunderscore}{\kern0pt}through{\isacharunderscore}{\kern0pt}def\ subobject{\isacharunderscore}{\kern0pt}of{\isacharunderscore}{\kern0pt}def{\isadigit{2}}\ \isacommand{by}\isamarkupfalse%
\ force\isanewline
\isacommand{qed}\isamarkupfalse%
%
\endisatagproof
{\isafoldproof}%
%
\isadelimproof
\isanewline
%
\endisadelimproof
\isanewline
\isacommand{lemma}\isamarkupfalse%
\ symmetric{\isacharunderscore}{\kern0pt}def{\isadigit{2}}{\isacharcolon}{\kern0pt}\isanewline
\ \ \isakeyword{assumes}\ symmetric{\isacharunderscore}{\kern0pt}Y{\isacharcolon}{\kern0pt}\ {\isachardoublequoteopen}symmetric{\isacharunderscore}{\kern0pt}on\ X\ {\isacharparenleft}{\kern0pt}Y{\isacharcomma}{\kern0pt}\ m{\isacharparenright}{\kern0pt}{\isachardoublequoteclose}\isanewline
\ \ \isakeyword{assumes}\ x{\isacharunderscore}{\kern0pt}type{\isacharcolon}{\kern0pt}\ {\isachardoublequoteopen}x\ {\isasymin}\isactrlsub c\ X{\isachardoublequoteclose}\isanewline
\ \ \isakeyword{assumes}\ y{\isacharunderscore}{\kern0pt}type{\isacharcolon}{\kern0pt}\ {\isachardoublequoteopen}y\ {\isasymin}\isactrlsub c\ X{\isachardoublequoteclose}\isanewline
\ \ \isakeyword{assumes}\ relation{\isacharcolon}{\kern0pt}\ {\isachardoublequoteopen}{\isasymexists}v{\isachardot}{\kern0pt}\ v\ {\isasymin}\isactrlsub c\ Y\ {\isasymand}\ \ m\ {\isasymcirc}\isactrlsub c\ v\ {\isacharequal}{\kern0pt}\ {\isasymlangle}x{\isacharcomma}{\kern0pt}y{\isasymrangle}{\isachardoublequoteclose}\isanewline
\ \ \isakeyword{shows}\ {\isachardoublequoteopen}{\isasymexists}w{\isachardot}{\kern0pt}\ w\ {\isasymin}\isactrlsub c\ Y\ {\isasymand}\ \ m\ {\isasymcirc}\isactrlsub c\ w\ {\isacharequal}{\kern0pt}\ {\isasymlangle}y{\isacharcomma}{\kern0pt}x{\isasymrangle}{\isachardoublequoteclose}\isanewline
%
\isadelimproof
\ \ %
\endisadelimproof
%
\isatagproof
\isacommand{using}\isamarkupfalse%
\ assms\ \isacommand{unfolding}\isamarkupfalse%
\ symmetric{\isacharunderscore}{\kern0pt}on{\isacharunderscore}{\kern0pt}def\ relative{\isacharunderscore}{\kern0pt}member{\isacharunderscore}{\kern0pt}def\ factors{\isacharunderscore}{\kern0pt}through{\isacharunderscore}{\kern0pt}def{\isadigit{2}}\isanewline
\ \ \isacommand{by}\isamarkupfalse%
\ {\isacharparenleft}{\kern0pt}metis\ cfunc{\isacharunderscore}{\kern0pt}prod{\isacharunderscore}{\kern0pt}type\ factors{\isacharunderscore}{\kern0pt}through{\isacharunderscore}{\kern0pt}def{\isadigit{2}}\ fst{\isacharunderscore}{\kern0pt}conv\ snd{\isacharunderscore}{\kern0pt}conv\ subobject{\isacharunderscore}{\kern0pt}of{\isacharunderscore}{\kern0pt}def{\isadigit{2}}{\isacharparenright}{\kern0pt}%
\endisatagproof
{\isafoldproof}%
%
\isadelimproof
\isanewline
%
\endisadelimproof
\isanewline
\isacommand{lemma}\isamarkupfalse%
\ transitive{\isacharunderscore}{\kern0pt}def{\isadigit{2}}{\isacharcolon}{\kern0pt}\isanewline
\ \ \isakeyword{assumes}\ transitive{\isacharunderscore}{\kern0pt}Y{\isacharcolon}{\kern0pt}\ {\isachardoublequoteopen}transitive{\isacharunderscore}{\kern0pt}on\ X\ {\isacharparenleft}{\kern0pt}Y{\isacharcomma}{\kern0pt}\ m{\isacharparenright}{\kern0pt}{\isachardoublequoteclose}\isanewline
\ \ \isakeyword{assumes}\ x{\isacharunderscore}{\kern0pt}type{\isacharcolon}{\kern0pt}\ {\isachardoublequoteopen}x\ {\isasymin}\isactrlsub c\ X{\isachardoublequoteclose}\isanewline
\ \ \isakeyword{assumes}\ y{\isacharunderscore}{\kern0pt}type{\isacharcolon}{\kern0pt}\ {\isachardoublequoteopen}y\ {\isasymin}\isactrlsub c\ X{\isachardoublequoteclose}\isanewline
\ \ \isakeyword{assumes}\ z{\isacharunderscore}{\kern0pt}type{\isacharcolon}{\kern0pt}\ {\isachardoublequoteopen}z\ {\isasymin}\isactrlsub c\ X{\isachardoublequoteclose}\isanewline
\ \ \isakeyword{assumes}\ relation{\isadigit{1}}{\isacharcolon}{\kern0pt}\ {\isachardoublequoteopen}{\isasymexists}v{\isachardot}{\kern0pt}\ v\ {\isasymin}\isactrlsub c\ Y\ {\isasymand}\ \ m\ {\isasymcirc}\isactrlsub c\ v\ {\isacharequal}{\kern0pt}\ {\isasymlangle}x{\isacharcomma}{\kern0pt}y{\isasymrangle}{\isachardoublequoteclose}\isanewline
\ \ \isakeyword{assumes}\ relation{\isadigit{2}}{\isacharcolon}{\kern0pt}\ {\isachardoublequoteopen}{\isasymexists}w{\isachardot}{\kern0pt}\ w\ {\isasymin}\isactrlsub c\ Y\ {\isasymand}\ \ m\ {\isasymcirc}\isactrlsub c\ w\ {\isacharequal}{\kern0pt}\ {\isasymlangle}y{\isacharcomma}{\kern0pt}z{\isasymrangle}{\isachardoublequoteclose}\isanewline
\ \ \isakeyword{shows}\ {\isachardoublequoteopen}{\isasymexists}u{\isachardot}{\kern0pt}\ u\ {\isasymin}\isactrlsub c\ Y\ {\isasymand}\ \ m\ {\isasymcirc}\isactrlsub c\ u\ {\isacharequal}{\kern0pt}\ {\isasymlangle}x{\isacharcomma}{\kern0pt}z{\isasymrangle}{\isachardoublequoteclose}\isanewline
%
\isadelimproof
\ \ %
\endisadelimproof
%
\isatagproof
\isacommand{using}\isamarkupfalse%
\ assms\ \isacommand{unfolding}\isamarkupfalse%
\ transitive{\isacharunderscore}{\kern0pt}on{\isacharunderscore}{\kern0pt}def\ relative{\isacharunderscore}{\kern0pt}member{\isacharunderscore}{\kern0pt}def\ factors{\isacharunderscore}{\kern0pt}through{\isacharunderscore}{\kern0pt}def{\isadigit{2}}\isanewline
\ \ \isacommand{by}\isamarkupfalse%
\ {\isacharparenleft}{\kern0pt}metis\ cfunc{\isacharunderscore}{\kern0pt}prod{\isacharunderscore}{\kern0pt}type\ factors{\isacharunderscore}{\kern0pt}through{\isacharunderscore}{\kern0pt}def{\isadigit{2}}\ fst{\isacharunderscore}{\kern0pt}conv\ snd{\isacharunderscore}{\kern0pt}conv\ subobject{\isacharunderscore}{\kern0pt}of{\isacharunderscore}{\kern0pt}def{\isadigit{2}}{\isacharparenright}{\kern0pt}%
\endisatagproof
{\isafoldproof}%
%
\isadelimproof
%
\endisadelimproof
%
\begin{isamarkuptext}%
The lemma below corresponds to Exercise 2.3.3 in Halvorson.%
\end{isamarkuptext}\isamarkuptrue%
\isacommand{lemma}\isamarkupfalse%
\ kernel{\isacharunderscore}{\kern0pt}pair{\isacharunderscore}{\kern0pt}equiv{\isacharunderscore}{\kern0pt}rel{\isacharcolon}{\kern0pt}\isanewline
\ \ \isakeyword{assumes}\ {\isachardoublequoteopen}f\ {\isacharcolon}{\kern0pt}\ X\ {\isasymrightarrow}\ Y{\isachardoublequoteclose}\isanewline
\ \ \isakeyword{shows}\ {\isachardoublequoteopen}equiv{\isacharunderscore}{\kern0pt}rel{\isacharunderscore}{\kern0pt}on\ X\ {\isacharparenleft}{\kern0pt}X\ \isactrlbsub f\isactrlesub {\isasymtimes}\isactrlsub c\isactrlbsub f\isactrlesub \ X{\isacharcomma}{\kern0pt}\ fibered{\isacharunderscore}{\kern0pt}product{\isacharunderscore}{\kern0pt}morphism\ X\ f\ f\ X{\isacharparenright}{\kern0pt}{\isachardoublequoteclose}\isanewline
%
\isadelimproof
%
\endisadelimproof
%
\isatagproof
\isacommand{proof}\isamarkupfalse%
\ {\isacharparenleft}{\kern0pt}unfold\ equiv{\isacharunderscore}{\kern0pt}rel{\isacharunderscore}{\kern0pt}on{\isacharunderscore}{\kern0pt}def{\isacharcomma}{\kern0pt}\ auto{\isacharparenright}{\kern0pt}\isanewline
\ \ \isacommand{show}\isamarkupfalse%
\ {\isachardoublequoteopen}reflexive{\isacharunderscore}{\kern0pt}on\ X\ {\isacharparenleft}{\kern0pt}X\ \isactrlbsub f\isactrlesub {\isasymtimes}\isactrlsub c\isactrlbsub f\isactrlesub \ X{\isacharcomma}{\kern0pt}\ fibered{\isacharunderscore}{\kern0pt}product{\isacharunderscore}{\kern0pt}morphism\ X\ f\ f\ X{\isacharparenright}{\kern0pt}{\isachardoublequoteclose}\isanewline
\ \ \isacommand{proof}\isamarkupfalse%
\ {\isacharparenleft}{\kern0pt}unfold\ reflexive{\isacharunderscore}{\kern0pt}on{\isacharunderscore}{\kern0pt}def{\isacharcomma}{\kern0pt}\ auto{\isacharparenright}{\kern0pt}\isanewline
\ \ \ \ \isacommand{show}\isamarkupfalse%
\ {\isachardoublequoteopen}{\isacharparenleft}{\kern0pt}X\ \isactrlbsub f\isactrlesub {\isasymtimes}\isactrlsub c\isactrlbsub f\isactrlesub \ X{\isacharcomma}{\kern0pt}\ fibered{\isacharunderscore}{\kern0pt}product{\isacharunderscore}{\kern0pt}morphism\ X\ f\ f\ X{\isacharparenright}{\kern0pt}\ {\isasymsubseteq}\isactrlsub c\ X\ {\isasymtimes}\isactrlsub c\ X{\isachardoublequoteclose}\isanewline
\ \ \ \ \ \ \isacommand{using}\isamarkupfalse%
\ assms\ kernel{\isacharunderscore}{\kern0pt}pair{\isacharunderscore}{\kern0pt}subset\ \isacommand{by}\isamarkupfalse%
\ auto\isanewline
\ \ \isacommand{next}\isamarkupfalse%
\isanewline
\ \ \ \ \isacommand{fix}\isamarkupfalse%
\ x\isanewline
\ \ \ \ \isacommand{assume}\isamarkupfalse%
\ x{\isacharunderscore}{\kern0pt}type{\isacharcolon}{\kern0pt}\ {\isachardoublequoteopen}x\ {\isasymin}\isactrlsub c\ X{\isachardoublequoteclose}\isanewline
\ \ \ \ \isacommand{then}\isamarkupfalse%
\ \isacommand{show}\isamarkupfalse%
\ {\isachardoublequoteopen}{\isasymlangle}x{\isacharcomma}{\kern0pt}x{\isasymrangle}\ {\isasymin}\isactrlbsub X\ {\isasymtimes}\isactrlsub c\ X\isactrlesub \ {\isacharparenleft}{\kern0pt}X\ \isactrlbsub f\isactrlesub {\isasymtimes}\isactrlsub c\isactrlbsub f\isactrlesub \ X{\isacharcomma}{\kern0pt}\ fibered{\isacharunderscore}{\kern0pt}product{\isacharunderscore}{\kern0pt}morphism\ X\ f\ f\ X{\isacharparenright}{\kern0pt}{\isachardoublequoteclose}\isanewline
\ \ \ \ \ \ \isacommand{by}\isamarkupfalse%
\ {\isacharparenleft}{\kern0pt}smt\ assms\ comp{\isacharunderscore}{\kern0pt}type\ diag{\isacharunderscore}{\kern0pt}on{\isacharunderscore}{\kern0pt}elements\ diagonal{\isacharunderscore}{\kern0pt}type\ fibered{\isacharunderscore}{\kern0pt}product{\isacharunderscore}{\kern0pt}morphism{\isacharunderscore}{\kern0pt}monomorphism\isanewline
\ \ \ \ \ \ \ \ \ \ fibered{\isacharunderscore}{\kern0pt}product{\isacharunderscore}{\kern0pt}morphism{\isacharunderscore}{\kern0pt}type\ pair{\isacharunderscore}{\kern0pt}factorsthru{\isacharunderscore}{\kern0pt}fibered{\isacharunderscore}{\kern0pt}product{\isacharunderscore}{\kern0pt}morphism\ relative{\isacharunderscore}{\kern0pt}member{\isacharunderscore}{\kern0pt}def{\isadigit{2}}{\isacharparenright}{\kern0pt}\isanewline
\ \ \isacommand{qed}\isamarkupfalse%
\isanewline
\isanewline
\ \ \isacommand{show}\isamarkupfalse%
\ {\isachardoublequoteopen}symmetric{\isacharunderscore}{\kern0pt}on\ X\ {\isacharparenleft}{\kern0pt}X\ \isactrlbsub f\isactrlesub {\isasymtimes}\isactrlsub c\isactrlbsub f\isactrlesub \ X{\isacharcomma}{\kern0pt}\ fibered{\isacharunderscore}{\kern0pt}product{\isacharunderscore}{\kern0pt}morphism\ X\ f\ f\ X{\isacharparenright}{\kern0pt}{\isachardoublequoteclose}\isanewline
\ \ \isacommand{proof}\isamarkupfalse%
\ {\isacharparenleft}{\kern0pt}unfold\ symmetric{\isacharunderscore}{\kern0pt}on{\isacharunderscore}{\kern0pt}def{\isacharcomma}{\kern0pt}\ auto{\isacharparenright}{\kern0pt}\isanewline
\ \ \ \ \isacommand{show}\isamarkupfalse%
\ {\isachardoublequoteopen}{\isacharparenleft}{\kern0pt}X\ \isactrlbsub f\isactrlesub {\isasymtimes}\isactrlsub c\isactrlbsub f\isactrlesub \ X{\isacharcomma}{\kern0pt}\ fibered{\isacharunderscore}{\kern0pt}product{\isacharunderscore}{\kern0pt}morphism\ X\ f\ f\ X{\isacharparenright}{\kern0pt}\ {\isasymsubseteq}\isactrlsub c\ X\ {\isasymtimes}\isactrlsub c\ X{\isachardoublequoteclose}\isanewline
\ \ \ \ \ \ \isacommand{using}\isamarkupfalse%
\ assms\ kernel{\isacharunderscore}{\kern0pt}pair{\isacharunderscore}{\kern0pt}subset\ \isacommand{by}\isamarkupfalse%
\ auto\isanewline
\ \ \isacommand{next}\isamarkupfalse%
\ \isanewline
\ \ \ \ \isacommand{fix}\isamarkupfalse%
\ x\ y\isanewline
\ \ \ \ \isacommand{assume}\isamarkupfalse%
\ x{\isacharunderscore}{\kern0pt}type{\isacharcolon}{\kern0pt}\ {\isachardoublequoteopen}x\ {\isasymin}\isactrlsub c\ X{\isachardoublequoteclose}\ \isakeyword{and}\ y{\isacharunderscore}{\kern0pt}type{\isacharcolon}{\kern0pt}\ {\isachardoublequoteopen}y\ {\isasymin}\isactrlsub c\ X{\isachardoublequoteclose}\isanewline
\ \ \ \ \isacommand{assume}\isamarkupfalse%
\ xy{\isacharunderscore}{\kern0pt}in{\isacharcolon}{\kern0pt}\ {\isachardoublequoteopen}{\isasymlangle}x{\isacharcomma}{\kern0pt}y{\isasymrangle}\ {\isasymin}\isactrlbsub X\ {\isasymtimes}\isactrlsub c\ X\isactrlesub \ {\isacharparenleft}{\kern0pt}X\ \isactrlbsub f\isactrlesub {\isasymtimes}\isactrlsub c\isactrlbsub f\isactrlesub \ X{\isacharcomma}{\kern0pt}\ fibered{\isacharunderscore}{\kern0pt}product{\isacharunderscore}{\kern0pt}morphism\ X\ f\ f\ X{\isacharparenright}{\kern0pt}{\isachardoublequoteclose}\isanewline
\ \ \ \ \isacommand{then}\isamarkupfalse%
\ \isacommand{have}\isamarkupfalse%
\ {\isachardoublequoteopen}f\ {\isasymcirc}\isactrlsub c\ x\ {\isacharequal}{\kern0pt}\ f\ {\isasymcirc}\isactrlsub c\ y{\isachardoublequoteclose}\isanewline
\ \ \ \ \ \ \isacommand{using}\isamarkupfalse%
\ assms\ fibered{\isacharunderscore}{\kern0pt}product{\isacharunderscore}{\kern0pt}pair{\isacharunderscore}{\kern0pt}member\ x{\isacharunderscore}{\kern0pt}type\ y{\isacharunderscore}{\kern0pt}type\ \isacommand{by}\isamarkupfalse%
\ blast\isanewline
\ \ \ \ \isanewline
\ \ \ \ \isacommand{then}\isamarkupfalse%
\ \isacommand{show}\isamarkupfalse%
\ {\isachardoublequoteopen}{\isasymlangle}y{\isacharcomma}{\kern0pt}x{\isasymrangle}\ {\isasymin}\isactrlbsub X\ {\isasymtimes}\isactrlsub c\ X\isactrlesub \ {\isacharparenleft}{\kern0pt}X\ \isactrlbsub f\isactrlesub {\isasymtimes}\isactrlsub c\isactrlbsub f\isactrlesub \ X{\isacharcomma}{\kern0pt}\ fibered{\isacharunderscore}{\kern0pt}product{\isacharunderscore}{\kern0pt}morphism\ X\ f\ f\ X{\isacharparenright}{\kern0pt}{\isachardoublequoteclose}\isanewline
\ \ \ \ \ \ \isacommand{using}\isamarkupfalse%
\ assms\ fibered{\isacharunderscore}{\kern0pt}product{\isacharunderscore}{\kern0pt}pair{\isacharunderscore}{\kern0pt}member\ x{\isacharunderscore}{\kern0pt}type\ y{\isacharunderscore}{\kern0pt}type\ \isacommand{by}\isamarkupfalse%
\ auto\isanewline
\ \ \isacommand{qed}\isamarkupfalse%
\isanewline
\isanewline
\ \ \isacommand{show}\isamarkupfalse%
\ {\isachardoublequoteopen}transitive{\isacharunderscore}{\kern0pt}on\ X\ {\isacharparenleft}{\kern0pt}X\ \isactrlbsub f\isactrlesub {\isasymtimes}\isactrlsub c\isactrlbsub f\isactrlesub \ X{\isacharcomma}{\kern0pt}\ fibered{\isacharunderscore}{\kern0pt}product{\isacharunderscore}{\kern0pt}morphism\ X\ f\ f\ X{\isacharparenright}{\kern0pt}{\isachardoublequoteclose}\isanewline
\ \ \isacommand{proof}\isamarkupfalse%
\ {\isacharparenleft}{\kern0pt}unfold\ transitive{\isacharunderscore}{\kern0pt}on{\isacharunderscore}{\kern0pt}def{\isacharcomma}{\kern0pt}\ auto{\isacharparenright}{\kern0pt}\isanewline
\ \ \ \ \isacommand{show}\isamarkupfalse%
\ {\isachardoublequoteopen}{\isacharparenleft}{\kern0pt}X\ \isactrlbsub f\isactrlesub {\isasymtimes}\isactrlsub c\isactrlbsub f\isactrlesub \ X{\isacharcomma}{\kern0pt}\ fibered{\isacharunderscore}{\kern0pt}product{\isacharunderscore}{\kern0pt}morphism\ X\ f\ f\ X{\isacharparenright}{\kern0pt}\ {\isasymsubseteq}\isactrlsub c\ X\ {\isasymtimes}\isactrlsub c\ X{\isachardoublequoteclose}\isanewline
\ \ \ \ \ \ \isacommand{using}\isamarkupfalse%
\ assms\ kernel{\isacharunderscore}{\kern0pt}pair{\isacharunderscore}{\kern0pt}subset\ \isacommand{by}\isamarkupfalse%
\ auto\isanewline
\ \ \isacommand{next}\isamarkupfalse%
\ \isanewline
\ \ \ \ \isacommand{fix}\isamarkupfalse%
\ x\ y\ z\ \isanewline
\ \ \ \ \isacommand{assume}\isamarkupfalse%
\ x{\isacharunderscore}{\kern0pt}type{\isacharcolon}{\kern0pt}\ {\isachardoublequoteopen}x\ {\isasymin}\isactrlsub c\ X{\isachardoublequoteclose}\ \isakeyword{and}\ y{\isacharunderscore}{\kern0pt}type{\isacharcolon}{\kern0pt}\ {\isachardoublequoteopen}y\ {\isasymin}\isactrlsub c\ X{\isachardoublequoteclose}\ \isakeyword{and}\ z{\isacharunderscore}{\kern0pt}type{\isacharcolon}{\kern0pt}\ {\isachardoublequoteopen}z\ {\isasymin}\isactrlsub c\ X{\isachardoublequoteclose}\isanewline
\ \ \ \ \isacommand{assume}\isamarkupfalse%
\ xy{\isacharunderscore}{\kern0pt}in{\isacharcolon}{\kern0pt}\ {\isachardoublequoteopen}{\isasymlangle}x{\isacharcomma}{\kern0pt}y{\isasymrangle}\ {\isasymin}\isactrlbsub X\ {\isasymtimes}\isactrlsub c\ X\isactrlesub \ {\isacharparenleft}{\kern0pt}X\ \isactrlbsub f\isactrlesub {\isasymtimes}\isactrlsub c\isactrlbsub f\isactrlesub \ X{\isacharcomma}{\kern0pt}\ fibered{\isacharunderscore}{\kern0pt}product{\isacharunderscore}{\kern0pt}morphism\ X\ f\ f\ X{\isacharparenright}{\kern0pt}{\isachardoublequoteclose}\isanewline
\ \ \ \ \isacommand{assume}\isamarkupfalse%
\ yz{\isacharunderscore}{\kern0pt}in{\isacharcolon}{\kern0pt}\ {\isachardoublequoteopen}{\isasymlangle}y{\isacharcomma}{\kern0pt}z{\isasymrangle}\ {\isasymin}\isactrlbsub X\ {\isasymtimes}\isactrlsub c\ X\isactrlesub \ {\isacharparenleft}{\kern0pt}X\ \isactrlbsub f\isactrlesub {\isasymtimes}\isactrlsub c\isactrlbsub f\isactrlesub \ X{\isacharcomma}{\kern0pt}\ fibered{\isacharunderscore}{\kern0pt}product{\isacharunderscore}{\kern0pt}morphism\ X\ f\ f\ X{\isacharparenright}{\kern0pt}{\isachardoublequoteclose}\isanewline
\isanewline
\ \ \ \ \isacommand{have}\isamarkupfalse%
\ eqn{\isadigit{1}}{\isacharcolon}{\kern0pt}\ {\isachardoublequoteopen}f\ {\isasymcirc}\isactrlsub c\ x\ {\isacharequal}{\kern0pt}\ f\ {\isasymcirc}\isactrlsub c\ y{\isachardoublequoteclose}\isanewline
\ \ \ \ \ \ \isacommand{using}\isamarkupfalse%
\ assms\ fibered{\isacharunderscore}{\kern0pt}product{\isacharunderscore}{\kern0pt}pair{\isacharunderscore}{\kern0pt}member\ x{\isacharunderscore}{\kern0pt}type\ xy{\isacharunderscore}{\kern0pt}in\ y{\isacharunderscore}{\kern0pt}type\ \isacommand{by}\isamarkupfalse%
\ blast\isanewline
\isanewline
\ \ \ \ \isacommand{have}\isamarkupfalse%
\ eqn{\isadigit{2}}{\isacharcolon}{\kern0pt}\ {\isachardoublequoteopen}f\ {\isasymcirc}\isactrlsub c\ y\ {\isacharequal}{\kern0pt}\ f\ {\isasymcirc}\isactrlsub c\ z{\isachardoublequoteclose}\isanewline
\ \ \ \ \ \ \isacommand{using}\isamarkupfalse%
\ assms\ fibered{\isacharunderscore}{\kern0pt}product{\isacharunderscore}{\kern0pt}pair{\isacharunderscore}{\kern0pt}member\ y{\isacharunderscore}{\kern0pt}type\ yz{\isacharunderscore}{\kern0pt}in\ z{\isacharunderscore}{\kern0pt}type\ \isacommand{by}\isamarkupfalse%
\ blast\isanewline
\isanewline
\ \ \ \ \isacommand{show}\isamarkupfalse%
\ {\isachardoublequoteopen}{\isasymlangle}x{\isacharcomma}{\kern0pt}z{\isasymrangle}\ {\isasymin}\isactrlbsub X\ {\isasymtimes}\isactrlsub c\ X\isactrlesub \ {\isacharparenleft}{\kern0pt}X\ \isactrlbsub f\isactrlesub {\isasymtimes}\isactrlsub c\isactrlbsub f\isactrlesub \ X{\isacharcomma}{\kern0pt}\ fibered{\isacharunderscore}{\kern0pt}product{\isacharunderscore}{\kern0pt}morphism\ X\ f\ f\ X{\isacharparenright}{\kern0pt}{\isachardoublequoteclose}\isanewline
\ \ \ \ \ \ \isacommand{using}\isamarkupfalse%
\ assms\ eqn{\isadigit{1}}\ eqn{\isadigit{2}}\ fibered{\isacharunderscore}{\kern0pt}product{\isacharunderscore}{\kern0pt}pair{\isacharunderscore}{\kern0pt}member\ x{\isacharunderscore}{\kern0pt}type\ z{\isacharunderscore}{\kern0pt}type\ \isacommand{by}\isamarkupfalse%
\ auto\isanewline
\ \ \isacommand{qed}\isamarkupfalse%
\isanewline
\isacommand{qed}\isamarkupfalse%
%
\endisatagproof
{\isafoldproof}%
%
\isadelimproof
%
\endisadelimproof
%
\begin{isamarkuptext}%
The axiomatization below corresponds to Axiom 6 (Equivalence Classes) in Halvorson.%
\end{isamarkuptext}\isamarkuptrue%
\isacommand{axiomatization}\isamarkupfalse%
\ \isanewline
\ \ quotient{\isacharunderscore}{\kern0pt}set\ {\isacharcolon}{\kern0pt}{\isacharcolon}{\kern0pt}\ {\isachardoublequoteopen}cset\ {\isasymRightarrow}\ {\isacharparenleft}{\kern0pt}cset\ {\isasymtimes}\ cfunc{\isacharparenright}{\kern0pt}\ {\isasymRightarrow}\ cset{\isachardoublequoteclose}\ {\isacharparenleft}{\kern0pt}\isakeyword{infix}\ {\isachardoublequoteopen}{\isasymsslash}{\isachardoublequoteclose}\ {\isadigit{5}}{\isadigit{0}}{\isacharparenright}{\kern0pt}\ \isakeyword{and}\isanewline
\ \ equiv{\isacharunderscore}{\kern0pt}class\ {\isacharcolon}{\kern0pt}{\isacharcolon}{\kern0pt}\ {\isachardoublequoteopen}cset\ {\isasymtimes}\ cfunc\ {\isasymRightarrow}\ cfunc{\isachardoublequoteclose}\ \isakeyword{and}\isanewline
\ \ quotient{\isacharunderscore}{\kern0pt}func\ {\isacharcolon}{\kern0pt}{\isacharcolon}{\kern0pt}\ {\isachardoublequoteopen}cfunc\ {\isasymRightarrow}\ cset\ {\isasymtimes}\ cfunc\ {\isasymRightarrow}\ cfunc{\isachardoublequoteclose}\isanewline
\isakeyword{where}\isanewline
\ \ equiv{\isacharunderscore}{\kern0pt}class{\isacharunderscore}{\kern0pt}type{\isacharbrackleft}{\kern0pt}type{\isacharunderscore}{\kern0pt}rule{\isacharbrackright}{\kern0pt}{\isacharcolon}{\kern0pt}\ {\isachardoublequoteopen}equiv{\isacharunderscore}{\kern0pt}rel{\isacharunderscore}{\kern0pt}on\ X\ R\ {\isasymLongrightarrow}\ equiv{\isacharunderscore}{\kern0pt}class\ R\ {\isacharcolon}{\kern0pt}\ X\ {\isasymrightarrow}\ quotient{\isacharunderscore}{\kern0pt}set\ X\ R{\isachardoublequoteclose}\ \isakeyword{and}\isanewline
\ \ equiv{\isacharunderscore}{\kern0pt}class{\isacharunderscore}{\kern0pt}eq{\isacharcolon}{\kern0pt}\ {\isachardoublequoteopen}equiv{\isacharunderscore}{\kern0pt}rel{\isacharunderscore}{\kern0pt}on\ X\ R\ {\isasymLongrightarrow}\ {\isasymlangle}x{\isacharcomma}{\kern0pt}\ y{\isasymrangle}\ {\isasymin}\isactrlsub c\ X{\isasymtimes}\isactrlsub cX\ {\isasymLongrightarrow}\isanewline
\ \ \ \ {\isasymlangle}x{\isacharcomma}{\kern0pt}\ y{\isasymrangle}\ {\isasymin}\isactrlbsub X{\isasymtimes}\isactrlsub cX\isactrlesub \ R\ {\isasymlongleftrightarrow}\ equiv{\isacharunderscore}{\kern0pt}class\ R\ {\isasymcirc}\isactrlsub c\ x\ {\isacharequal}{\kern0pt}\ equiv{\isacharunderscore}{\kern0pt}class\ R\ {\isasymcirc}\isactrlsub c\ y{\isachardoublequoteclose}\ \isakeyword{and}\isanewline
\ \ quotient{\isacharunderscore}{\kern0pt}func{\isacharunderscore}{\kern0pt}type{\isacharbrackleft}{\kern0pt}type{\isacharunderscore}{\kern0pt}rule{\isacharbrackright}{\kern0pt}{\isacharcolon}{\kern0pt}\ \isanewline
\ \ \ \ {\isachardoublequoteopen}equiv{\isacharunderscore}{\kern0pt}rel{\isacharunderscore}{\kern0pt}on\ X\ R\ {\isasymLongrightarrow}\ f\ {\isacharcolon}{\kern0pt}\ X\ {\isasymrightarrow}\ Y\ {\isasymLongrightarrow}\ {\isacharparenleft}{\kern0pt}const{\isacharunderscore}{\kern0pt}on{\isacharunderscore}{\kern0pt}rel\ X\ R\ f{\isacharparenright}{\kern0pt}\ {\isasymLongrightarrow}\isanewline
\ \ \ \ \ \ quotient{\isacharunderscore}{\kern0pt}func\ f\ R\ {\isacharcolon}{\kern0pt}\ quotient{\isacharunderscore}{\kern0pt}set\ X\ R\ {\isasymrightarrow}\ Y{\isachardoublequoteclose}\ \isakeyword{and}\ \isanewline
\ \ quotient{\isacharunderscore}{\kern0pt}func{\isacharunderscore}{\kern0pt}eq{\isacharcolon}{\kern0pt}\ {\isachardoublequoteopen}equiv{\isacharunderscore}{\kern0pt}rel{\isacharunderscore}{\kern0pt}on\ X\ R\ {\isasymLongrightarrow}\ f\ {\isacharcolon}{\kern0pt}\ X\ {\isasymrightarrow}\ Y\ {\isasymLongrightarrow}\ {\isacharparenleft}{\kern0pt}const{\isacharunderscore}{\kern0pt}on{\isacharunderscore}{\kern0pt}rel\ X\ R\ f{\isacharparenright}{\kern0pt}\ {\isasymLongrightarrow}\isanewline
\ \ \ \ \ quotient{\isacharunderscore}{\kern0pt}func\ f\ R\ {\isasymcirc}\isactrlsub c\ equiv{\isacharunderscore}{\kern0pt}class\ R\ {\isacharequal}{\kern0pt}\ f{\isachardoublequoteclose}\ \isakeyword{and}\ \ \isanewline
\ \ quotient{\isacharunderscore}{\kern0pt}func{\isacharunderscore}{\kern0pt}unique{\isacharcolon}{\kern0pt}\ {\isachardoublequoteopen}equiv{\isacharunderscore}{\kern0pt}rel{\isacharunderscore}{\kern0pt}on\ X\ R\ {\isasymLongrightarrow}\ f\ {\isacharcolon}{\kern0pt}\ X\ {\isasymrightarrow}\ Y\ {\isasymLongrightarrow}\ {\isacharparenleft}{\kern0pt}const{\isacharunderscore}{\kern0pt}on{\isacharunderscore}{\kern0pt}rel\ X\ R\ f{\isacharparenright}{\kern0pt}\ {\isasymLongrightarrow}\isanewline
\ \ \ \ h\ {\isacharcolon}{\kern0pt}\ quotient{\isacharunderscore}{\kern0pt}set\ X\ R\ {\isasymrightarrow}\ Y\ {\isasymLongrightarrow}\ h\ {\isasymcirc}\isactrlsub c\ equiv{\isacharunderscore}{\kern0pt}class\ R\ {\isacharequal}{\kern0pt}\ f\ {\isasymLongrightarrow}\ h\ {\isacharequal}{\kern0pt}\ quotient{\isacharunderscore}{\kern0pt}func\ f\ R{\isachardoublequoteclose}%
\begin{isamarkuptext}%
Note that \isa{{\isacharparenleft}{\kern0pt}{\isasymsslash}{\isacharparenright}{\kern0pt}} corresponds to $X/R$, \isa{equiv{\isacharunderscore}{\kern0pt}class} corresponds to the
  canonical quotient mapping $q$, and \isa{quotient{\isacharunderscore}{\kern0pt}func} corresponds to $\bar{f}$ in Halvorson's
  formulation of this axiom.%
\end{isamarkuptext}\isamarkuptrue%
\isacommand{abbreviation}\isamarkupfalse%
\ equiv{\isacharunderscore}{\kern0pt}class{\isacharprime}{\kern0pt}\ {\isacharcolon}{\kern0pt}{\isacharcolon}{\kern0pt}\ {\isachardoublequoteopen}cfunc\ {\isasymRightarrow}\ cset\ {\isasymtimes}\ cfunc\ {\isasymRightarrow}\ cfunc{\isachardoublequoteclose}\ {\isacharparenleft}{\kern0pt}{\isachardoublequoteopen}{\isacharbrackleft}{\kern0pt}{\isacharunderscore}{\kern0pt}{\isacharbrackright}{\kern0pt}\isactrlbsub {\isacharunderscore}{\kern0pt}\isactrlesub {\isachardoublequoteclose}{\isacharparenright}{\kern0pt}\ \isakeyword{where}\isanewline
\ \ {\isachardoublequoteopen}{\isacharbrackleft}{\kern0pt}x{\isacharbrackright}{\kern0pt}\isactrlbsub R\isactrlesub \ {\isasymequiv}\ equiv{\isacharunderscore}{\kern0pt}class\ R\ {\isasymcirc}\isactrlsub c\ x{\isachardoublequoteclose}%
\isadelimdocument
%
\endisadelimdocument
%
\isatagdocument
%
\isamarkupsection{Coequalizers and Epimorphisms%
}
\isamarkuptrue%
%
\isamarkupsubsection{Coequalizers%
}
\isamarkuptrue%
%
\endisatagdocument
{\isafolddocument}%
%
\isadelimdocument
%
\endisadelimdocument
%
\begin{isamarkuptext}%
The definition below corresponds to a comment after Axiom 6 (Equivalence Classes) in Halvorson.%
\end{isamarkuptext}\isamarkuptrue%
\isacommand{definition}\isamarkupfalse%
\ coequalizer\ {\isacharcolon}{\kern0pt}{\isacharcolon}{\kern0pt}\ {\isachardoublequoteopen}cset\ {\isasymRightarrow}\ cfunc\ {\isasymRightarrow}\ cfunc\ {\isasymRightarrow}\ cfunc\ {\isasymRightarrow}\ bool{\isachardoublequoteclose}\ \isakeyword{where}\isanewline
\ \ {\isachardoublequoteopen}coequalizer\ E\ m\ f\ g\ {\isasymlongleftrightarrow}\ {\isacharparenleft}{\kern0pt}{\isasymexists}\ X\ Y{\isachardot}{\kern0pt}\ {\isacharparenleft}{\kern0pt}f\ {\isacharcolon}{\kern0pt}\ Y\ {\isasymrightarrow}\ X{\isacharparenright}{\kern0pt}\ {\isasymand}\ {\isacharparenleft}{\kern0pt}g\ {\isacharcolon}{\kern0pt}\ Y\ {\isasymrightarrow}\ X{\isacharparenright}{\kern0pt}\ {\isasymand}\ {\isacharparenleft}{\kern0pt}m\ {\isacharcolon}{\kern0pt}\ X\ {\isasymrightarrow}\ E{\isacharparenright}{\kern0pt}\isanewline
\ \ \ \ {\isasymand}\ {\isacharparenleft}{\kern0pt}m\ {\isasymcirc}\isactrlsub c\ f\ {\isacharequal}{\kern0pt}\ m\ {\isasymcirc}\isactrlsub c\ g{\isacharparenright}{\kern0pt}\isanewline
\ \ \ \ {\isasymand}\ {\isacharparenleft}{\kern0pt}{\isasymforall}\ h\ F{\isachardot}{\kern0pt}\ {\isacharparenleft}{\kern0pt}{\isacharparenleft}{\kern0pt}h\ {\isacharcolon}{\kern0pt}\ X\ {\isasymrightarrow}\ F{\isacharparenright}{\kern0pt}\ {\isasymand}\ {\isacharparenleft}{\kern0pt}h\ {\isasymcirc}\isactrlsub c\ f\ {\isacharequal}{\kern0pt}\ h\ {\isasymcirc}\isactrlsub c\ g{\isacharparenright}{\kern0pt}{\isacharparenright}{\kern0pt}\ {\isasymlongrightarrow}\ {\isacharparenleft}{\kern0pt}{\isasymexists}{\isacharbang}{\kern0pt}\ k{\isachardot}{\kern0pt}\ {\isacharparenleft}{\kern0pt}k\ {\isacharcolon}{\kern0pt}\ E\ {\isasymrightarrow}\ F{\isacharparenright}{\kern0pt}\ {\isasymand}\ k\ {\isasymcirc}\isactrlsub c\ m\ {\isacharequal}{\kern0pt}\ h{\isacharparenright}{\kern0pt}{\isacharparenright}{\kern0pt}{\isacharparenright}{\kern0pt}{\isachardoublequoteclose}\isanewline
\isanewline
\isacommand{lemma}\isamarkupfalse%
\ coequalizer{\isacharunderscore}{\kern0pt}def{\isadigit{2}}{\isacharcolon}{\kern0pt}\isanewline
\ \ \isakeyword{assumes}\ {\isachardoublequoteopen}f\ {\isacharcolon}{\kern0pt}\ Y\ {\isasymrightarrow}\ X{\isachardoublequoteclose}\ {\isachardoublequoteopen}g\ {\isacharcolon}{\kern0pt}\ Y\ {\isasymrightarrow}\ X{\isachardoublequoteclose}\ {\isachardoublequoteopen}m\ {\isacharcolon}{\kern0pt}\ X\ {\isasymrightarrow}\ E{\isachardoublequoteclose}\isanewline
\ \ \isakeyword{shows}\ {\isachardoublequoteopen}coequalizer\ E\ m\ f\ g\ {\isasymlongleftrightarrow}\isanewline
\ \ \ \ {\isacharparenleft}{\kern0pt}m\ {\isasymcirc}\isactrlsub c\ f\ {\isacharequal}{\kern0pt}\ m\ {\isasymcirc}\isactrlsub c\ g{\isacharparenright}{\kern0pt}\isanewline
\ \ \ \ \ \ {\isasymand}\ {\isacharparenleft}{\kern0pt}{\isasymforall}\ h\ F{\isachardot}{\kern0pt}\ {\isacharparenleft}{\kern0pt}{\isacharparenleft}{\kern0pt}h\ {\isacharcolon}{\kern0pt}\ X\ {\isasymrightarrow}\ F{\isacharparenright}{\kern0pt}\ {\isasymand}\ {\isacharparenleft}{\kern0pt}h\ {\isasymcirc}\isactrlsub c\ f\ {\isacharequal}{\kern0pt}\ h\ {\isasymcirc}\isactrlsub c\ g{\isacharparenright}{\kern0pt}{\isacharparenright}{\kern0pt}\ {\isasymlongrightarrow}\ {\isacharparenleft}{\kern0pt}{\isasymexists}{\isacharbang}{\kern0pt}\ k{\isachardot}{\kern0pt}\ {\isacharparenleft}{\kern0pt}k\ {\isacharcolon}{\kern0pt}\ E\ {\isasymrightarrow}\ F{\isacharparenright}{\kern0pt}\ {\isasymand}\ k\ {\isasymcirc}\isactrlsub c\ m\ {\isacharequal}{\kern0pt}\ h{\isacharparenright}{\kern0pt}{\isacharparenright}{\kern0pt}{\isachardoublequoteclose}\isanewline
%
\isadelimproof
\ \ %
\endisadelimproof
%
\isatagproof
\isacommand{using}\isamarkupfalse%
\ assms\ \isacommand{unfolding}\isamarkupfalse%
\ coequalizer{\isacharunderscore}{\kern0pt}def\ cfunc{\isacharunderscore}{\kern0pt}type{\isacharunderscore}{\kern0pt}def\ \isacommand{by}\isamarkupfalse%
\ auto%
\endisatagproof
{\isafoldproof}%
%
\isadelimproof
%
\endisadelimproof
%
\begin{isamarkuptext}%
The lemma below corresponds to Exercise 2.3.1 in Halvorson.%
\end{isamarkuptext}\isamarkuptrue%
\isacommand{lemma}\isamarkupfalse%
\ coequalizer{\isacharunderscore}{\kern0pt}unique{\isacharcolon}{\kern0pt}\isanewline
\ \ \isakeyword{assumes}\ {\isachardoublequoteopen}coequalizer\ E\ m\ f\ g{\isachardoublequoteclose}\ {\isachardoublequoteopen}coequalizer\ F\ n\ f\ g{\isachardoublequoteclose}\isanewline
\ \ \isakeyword{shows}\ {\isachardoublequoteopen}E\ {\isasymcong}\ F{\isachardoublequoteclose}\isanewline
%
\isadelimproof
%
\endisadelimproof
%
\isatagproof
\isacommand{proof}\isamarkupfalse%
\ {\isacharminus}{\kern0pt}\ \isanewline
\ \ \isacommand{obtain}\isamarkupfalse%
\ k\ \isakeyword{where}\ k{\isacharunderscore}{\kern0pt}def{\isacharcolon}{\kern0pt}\ {\isachardoublequoteopen}k{\isacharcolon}{\kern0pt}\ E\ {\isasymrightarrow}\ F\ {\isasymand}\ k\ {\isasymcirc}\isactrlsub c\ m\ {\isacharequal}{\kern0pt}\ \ n{\isachardoublequoteclose}\isanewline
\ \ \ \ \ \isacommand{by}\isamarkupfalse%
\ {\isacharparenleft}{\kern0pt}typecheck{\isacharunderscore}{\kern0pt}cfuncs{\isacharcomma}{\kern0pt}\ metis\ assms\ cfunc{\isacharunderscore}{\kern0pt}type{\isacharunderscore}{\kern0pt}def\ coequalizer{\isacharunderscore}{\kern0pt}def{\isacharparenright}{\kern0pt}\isanewline
\ \ \isacommand{obtain}\isamarkupfalse%
\ k{\isacharprime}{\kern0pt}\ \isakeyword{where}\ k{\isacharprime}{\kern0pt}{\isacharunderscore}{\kern0pt}def{\isacharcolon}{\kern0pt}\ {\isachardoublequoteopen}k{\isacharprime}{\kern0pt}{\isacharcolon}{\kern0pt}\ F\ {\isasymrightarrow}\ E\ {\isasymand}\ k{\isacharprime}{\kern0pt}\ {\isasymcirc}\isactrlsub c\ n\ {\isacharequal}{\kern0pt}\ \ m{\isachardoublequoteclose}\isanewline
\ \ \ \ \ \isacommand{by}\isamarkupfalse%
\ {\isacharparenleft}{\kern0pt}typecheck{\isacharunderscore}{\kern0pt}cfuncs{\isacharcomma}{\kern0pt}\ metis\ assms\ cfunc{\isacharunderscore}{\kern0pt}type{\isacharunderscore}{\kern0pt}def\ coequalizer{\isacharunderscore}{\kern0pt}def{\isacharparenright}{\kern0pt}\isanewline
\ \ \isacommand{obtain}\isamarkupfalse%
\ k{\isacharprime}{\kern0pt}{\isacharprime}{\kern0pt}\ \isakeyword{where}\ k{\isacharprime}{\kern0pt}{\isacharprime}{\kern0pt}{\isacharunderscore}{\kern0pt}def{\isacharcolon}{\kern0pt}\ {\isachardoublequoteopen}k{\isacharprime}{\kern0pt}{\isacharprime}{\kern0pt}{\isacharcolon}{\kern0pt}\ F\ {\isasymrightarrow}\ F\ {\isasymand}\ k{\isacharprime}{\kern0pt}{\isacharprime}{\kern0pt}\ {\isasymcirc}\isactrlsub c\ n\ {\isacharequal}{\kern0pt}\ \ n{\isachardoublequoteclose}\isanewline
\ \ \ \ \isacommand{by}\isamarkupfalse%
\ {\isacharparenleft}{\kern0pt}typecheck{\isacharunderscore}{\kern0pt}cfuncs{\isacharcomma}{\kern0pt}\ smt\ {\isacharparenleft}{\kern0pt}verit{\isacharparenright}{\kern0pt}\ assms{\isacharparenleft}{\kern0pt}{\isadigit{2}}{\isacharparenright}{\kern0pt}\ \ cfunc{\isacharunderscore}{\kern0pt}type{\isacharunderscore}{\kern0pt}def\ coequalizer{\isacharunderscore}{\kern0pt}def{\isacharparenright}{\kern0pt}\isanewline
\isanewline
\ \ \isacommand{have}\isamarkupfalse%
\ k{\isacharprime}{\kern0pt}{\isacharprime}{\kern0pt}{\isacharunderscore}{\kern0pt}def{\isadigit{2}}{\isacharcolon}{\kern0pt}\ {\isachardoublequoteopen}k{\isacharprime}{\kern0pt}{\isacharprime}{\kern0pt}\ {\isacharequal}{\kern0pt}\ id\ F{\isachardoublequoteclose}\isanewline
\ \ \ \ \isacommand{using}\isamarkupfalse%
\ assms{\isacharparenleft}{\kern0pt}{\isadigit{2}}{\isacharparenright}{\kern0pt}\ coequalizer{\isacharunderscore}{\kern0pt}def\ id{\isacharunderscore}{\kern0pt}left{\isacharunderscore}{\kern0pt}unit{\isadigit{2}}\ k{\isacharprime}{\kern0pt}{\isacharprime}{\kern0pt}{\isacharunderscore}{\kern0pt}def\ \isacommand{by}\isamarkupfalse%
\ {\isacharparenleft}{\kern0pt}typecheck{\isacharunderscore}{\kern0pt}cfuncs{\isacharcomma}{\kern0pt}\ blast{\isacharparenright}{\kern0pt}\isanewline
\ \ \isacommand{have}\isamarkupfalse%
\ kk{\isacharprime}{\kern0pt}{\isacharunderscore}{\kern0pt}idF{\isacharcolon}{\kern0pt}\ {\isachardoublequoteopen}k\ {\isasymcirc}\isactrlsub c\ k{\isacharprime}{\kern0pt}\ {\isacharequal}{\kern0pt}\ id\ F{\isachardoublequoteclose}\isanewline
\ \ \ \ \isacommand{by}\isamarkupfalse%
\ {\isacharparenleft}{\kern0pt}typecheck{\isacharunderscore}{\kern0pt}cfuncs{\isacharcomma}{\kern0pt}\ smt\ {\isacharparenleft}{\kern0pt}verit{\isacharparenright}{\kern0pt}\ assms{\isacharparenleft}{\kern0pt}{\isadigit{2}}{\isacharparenright}{\kern0pt}\ cfunc{\isacharunderscore}{\kern0pt}type{\isacharunderscore}{\kern0pt}def\ coequalizer{\isacharunderscore}{\kern0pt}def\ comp{\isacharunderscore}{\kern0pt}associative\ k{\isacharprime}{\kern0pt}{\isacharprime}{\kern0pt}{\isacharunderscore}{\kern0pt}def\ k{\isacharprime}{\kern0pt}{\isacharprime}{\kern0pt}{\isacharunderscore}{\kern0pt}def{\isadigit{2}}\ k{\isacharprime}{\kern0pt}{\isacharunderscore}{\kern0pt}def\ k{\isacharunderscore}{\kern0pt}def{\isacharparenright}{\kern0pt}\isanewline
\ \ \isacommand{have}\isamarkupfalse%
\ k{\isacharprime}{\kern0pt}k{\isacharunderscore}{\kern0pt}idE{\isacharcolon}{\kern0pt}\ {\isachardoublequoteopen}k{\isacharprime}{\kern0pt}\ {\isasymcirc}\isactrlsub c\ k\ {\isacharequal}{\kern0pt}\ id\ E{\isachardoublequoteclose}\isanewline
\ \ \ \ \isacommand{by}\isamarkupfalse%
\ {\isacharparenleft}{\kern0pt}typecheck{\isacharunderscore}{\kern0pt}cfuncs{\isacharcomma}{\kern0pt}\ smt\ {\isacharparenleft}{\kern0pt}verit{\isacharparenright}{\kern0pt}\ assms{\isacharparenleft}{\kern0pt}{\isadigit{1}}{\isacharparenright}{\kern0pt}\ coequalizer{\isacharunderscore}{\kern0pt}def\ comp{\isacharunderscore}{\kern0pt}associative{\isadigit{2}}\ id{\isacharunderscore}{\kern0pt}left{\isacharunderscore}{\kern0pt}unit{\isadigit{2}}\ k{\isacharprime}{\kern0pt}{\isacharunderscore}{\kern0pt}def\ k{\isacharunderscore}{\kern0pt}def{\isacharparenright}{\kern0pt}\isanewline
\isanewline
\ \ \isacommand{show}\isamarkupfalse%
\ {\isachardoublequoteopen}E\ {\isasymcong}\ F{\isachardoublequoteclose}\isanewline
\ \ \ \ \isacommand{using}\isamarkupfalse%
\ cfunc{\isacharunderscore}{\kern0pt}type{\isacharunderscore}{\kern0pt}def\ is{\isacharunderscore}{\kern0pt}isomorphic{\isacharunderscore}{\kern0pt}def\ isomorphism{\isacharunderscore}{\kern0pt}def\ k{\isacharprime}{\kern0pt}{\isacharunderscore}{\kern0pt}def\ k{\isacharprime}{\kern0pt}k{\isacharunderscore}{\kern0pt}idE\ k{\isacharunderscore}{\kern0pt}def\ kk{\isacharprime}{\kern0pt}{\isacharunderscore}{\kern0pt}idF\ \isacommand{by}\isamarkupfalse%
\ fastforce\isanewline
\isacommand{qed}\isamarkupfalse%
%
\endisatagproof
{\isafoldproof}%
%
\isadelimproof
%
\endisadelimproof
%
\begin{isamarkuptext}%
The lemma below corresponds to Exercise 2.3.2 in Halvorson.%
\end{isamarkuptext}\isamarkuptrue%
\isacommand{lemma}\isamarkupfalse%
\ coequalizer{\isacharunderscore}{\kern0pt}is{\isacharunderscore}{\kern0pt}epimorphism{\isacharcolon}{\kern0pt}\isanewline
\ \ {\isachardoublequoteopen}coequalizer\ E\ m\ f\ g\ {\isasymLongrightarrow}\ \ epimorphism{\isacharparenleft}{\kern0pt}m{\isacharparenright}{\kern0pt}{\isachardoublequoteclose}\isanewline
%
\isadelimproof
\ \ %
\endisadelimproof
%
\isatagproof
\isacommand{unfolding}\isamarkupfalse%
\ coequalizer{\isacharunderscore}{\kern0pt}def\ epimorphism{\isacharunderscore}{\kern0pt}def\isanewline
\isacommand{proof}\isamarkupfalse%
\ auto\isanewline
\ \ \isacommand{fix}\isamarkupfalse%
\ k\ h\ X\ Y\isanewline
\ \ \isacommand{assume}\isamarkupfalse%
\ f{\isacharunderscore}{\kern0pt}type{\isacharcolon}{\kern0pt}\ {\isachardoublequoteopen}f\ {\isacharcolon}{\kern0pt}\ Y\ {\isasymrightarrow}\ X{\isachardoublequoteclose}\isanewline
\ \ \isacommand{assume}\isamarkupfalse%
\ g{\isacharunderscore}{\kern0pt}type{\isacharcolon}{\kern0pt}\ {\isachardoublequoteopen}g\ {\isacharcolon}{\kern0pt}\ Y\ {\isasymrightarrow}\ X{\isachardoublequoteclose}\isanewline
\ \ \isacommand{assume}\isamarkupfalse%
\ m{\isacharunderscore}{\kern0pt}type{\isacharcolon}{\kern0pt}\ {\isachardoublequoteopen}m\ {\isacharcolon}{\kern0pt}\ X\ {\isasymrightarrow}\ E{\isachardoublequoteclose}\isanewline
\ \ \isacommand{assume}\isamarkupfalse%
\ fm{\isacharunderscore}{\kern0pt}gm{\isacharcolon}{\kern0pt}\ {\isachardoublequoteopen}m\ {\isasymcirc}\isactrlsub c\ f\ {\isacharequal}{\kern0pt}\ m\ {\isasymcirc}\isactrlsub c\ g{\isachardoublequoteclose}\isanewline
\ \ \isacommand{assume}\isamarkupfalse%
\ uniqueness{\isacharcolon}{\kern0pt}\ {\isachardoublequoteopen}{\isasymforall}h\ F{\isachardot}{\kern0pt}\ h\ {\isacharcolon}{\kern0pt}\ X\ {\isasymrightarrow}\ F\ {\isasymand}\ h\ {\isasymcirc}\isactrlsub c\ f\ {\isacharequal}{\kern0pt}\ h\ {\isasymcirc}\isactrlsub c\ g\ {\isasymlongrightarrow}\ {\isacharparenleft}{\kern0pt}{\isasymexists}{\isacharbang}{\kern0pt}k{\isachardot}{\kern0pt}\ k\ {\isacharcolon}{\kern0pt}\ E\ {\isasymrightarrow}\ F\ {\isasymand}\ k\ {\isasymcirc}\isactrlsub c\ m\ {\isacharequal}{\kern0pt}\ h{\isacharparenright}{\kern0pt}{\isachardoublequoteclose}\isanewline
\ \ \isacommand{assume}\isamarkupfalse%
\ relation{\isacharunderscore}{\kern0pt}k{\isacharcolon}{\kern0pt}\ {\isachardoublequoteopen}domain\ k\ {\isacharequal}{\kern0pt}codomain\ m\ {\isachardoublequoteclose}\isanewline
\ \ \isacommand{assume}\isamarkupfalse%
\ relation{\isacharunderscore}{\kern0pt}h{\isacharcolon}{\kern0pt}\ {\isachardoublequoteopen}domain\ h\ {\isacharequal}{\kern0pt}\ codomain\ m{\isachardoublequoteclose}\ \isanewline
\ \ \isacommand{assume}\isamarkupfalse%
\ m{\isacharunderscore}{\kern0pt}k{\isacharunderscore}{\kern0pt}mh{\isacharcolon}{\kern0pt}\ {\isachardoublequoteopen}k\ {\isasymcirc}\isactrlsub c\ m\ {\isacharequal}{\kern0pt}\ h\ {\isasymcirc}\isactrlsub c\ m{\isachardoublequoteclose}\ \isanewline
\isanewline
\ \ \isacommand{have}\isamarkupfalse%
\ {\isachardoublequoteopen}k\ {\isasymcirc}\isactrlsub c\ m\ {\isasymcirc}\isactrlsub c\ f\ {\isacharequal}{\kern0pt}\ h\ {\isasymcirc}\isactrlsub c\ m\ {\isasymcirc}\isactrlsub c\ g{\isachardoublequoteclose}\isanewline
\ \ \ \ \isacommand{using}\isamarkupfalse%
\ cfunc{\isacharunderscore}{\kern0pt}type{\isacharunderscore}{\kern0pt}def\ comp{\isacharunderscore}{\kern0pt}associative\ fm{\isacharunderscore}{\kern0pt}gm\ g{\isacharunderscore}{\kern0pt}type\ m{\isacharunderscore}{\kern0pt}k{\isacharunderscore}{\kern0pt}mh\ m{\isacharunderscore}{\kern0pt}type\ relation{\isacharunderscore}{\kern0pt}k\ relation{\isacharunderscore}{\kern0pt}h\ \isacommand{by}\isamarkupfalse%
\ auto\isanewline
\isanewline
\ \ \isacommand{then}\isamarkupfalse%
\ \isacommand{obtain}\isamarkupfalse%
\ z\ \isakeyword{where}\ {\isachardoublequoteopen}z{\isacharcolon}{\kern0pt}\ E\ {\isasymrightarrow}\ codomain{\isacharparenleft}{\kern0pt}k{\isacharparenright}{\kern0pt}\ {\isasymand}\ z\ {\isasymcirc}\isactrlsub c\ m\ \ {\isacharequal}{\kern0pt}\ k\ {\isasymcirc}\isactrlsub c\ m\ {\isasymand}\isanewline
\ \ \ \ {\isacharparenleft}{\kern0pt}{\isasymforall}\ j{\isachardot}{\kern0pt}\ j{\isacharcolon}{\kern0pt}E\ {\isasymrightarrow}\ codomain{\isacharparenleft}{\kern0pt}k{\isacharparenright}{\kern0pt}\ {\isasymand}\ \ j\ {\isasymcirc}\isactrlsub c\ m\ {\isacharequal}{\kern0pt}\ k\ {\isasymcirc}\isactrlsub c\ m\ {\isasymlongrightarrow}\ j\ {\isacharequal}{\kern0pt}\ z{\isacharparenright}{\kern0pt}{\isachardoublequoteclose}\isanewline
\ \ \ \ \isacommand{using}\isamarkupfalse%
\ uniqueness\ \isacommand{by}\isamarkupfalse%
\ {\isacharparenleft}{\kern0pt}erule{\isacharunderscore}{\kern0pt}tac\ x{\isacharequal}{\kern0pt}{\isachardoublequoteopen}k\ {\isasymcirc}\isactrlsub c\ m{\isachardoublequoteclose}\ \isakeyword{in}\ allE{\isacharcomma}{\kern0pt}\ erule{\isacharunderscore}{\kern0pt}tac\ x{\isacharequal}{\kern0pt}{\isachardoublequoteopen}codomain{\isacharparenleft}{\kern0pt}k{\isacharparenright}{\kern0pt}{\isachardoublequoteclose}\ \isakeyword{in}\ allE{\isacharcomma}{\kern0pt}\isanewline
\ \ \ \ smt\ cfunc{\isacharunderscore}{\kern0pt}type{\isacharunderscore}{\kern0pt}def\ codomain{\isacharunderscore}{\kern0pt}comp\ comp{\isacharunderscore}{\kern0pt}associative\ domain{\isacharunderscore}{\kern0pt}comp\ f{\isacharunderscore}{\kern0pt}type\ g{\isacharunderscore}{\kern0pt}type\ m{\isacharunderscore}{\kern0pt}k{\isacharunderscore}{\kern0pt}mh\ m{\isacharunderscore}{\kern0pt}type\ relation{\isacharunderscore}{\kern0pt}k\ relation{\isacharunderscore}{\kern0pt}h{\isacharparenright}{\kern0pt}\isanewline
\isanewline
\ \ \isacommand{then}\isamarkupfalse%
\ \isacommand{show}\isamarkupfalse%
\ {\isachardoublequoteopen}k\ {\isacharequal}{\kern0pt}\ h{\isachardoublequoteclose}\isanewline
\ \ \ \ \isacommand{by}\isamarkupfalse%
\ {\isacharparenleft}{\kern0pt}metis\ cfunc{\isacharunderscore}{\kern0pt}type{\isacharunderscore}{\kern0pt}def\ codomain{\isacharunderscore}{\kern0pt}comp\ m{\isacharunderscore}{\kern0pt}k{\isacharunderscore}{\kern0pt}mh\ m{\isacharunderscore}{\kern0pt}type\ relation{\isacharunderscore}{\kern0pt}k\ relation{\isacharunderscore}{\kern0pt}h{\isacharparenright}{\kern0pt}\isanewline
\isacommand{qed}\isamarkupfalse%
%
\endisatagproof
{\isafoldproof}%
%
\isadelimproof
\isanewline
%
\endisadelimproof
\isanewline
\isacommand{lemma}\isamarkupfalse%
\ canonical{\isacharunderscore}{\kern0pt}quotient{\isacharunderscore}{\kern0pt}map{\isacharunderscore}{\kern0pt}is{\isacharunderscore}{\kern0pt}coequalizer{\isacharcolon}{\kern0pt}\isanewline
\ \ \isakeyword{assumes}\ {\isachardoublequoteopen}equiv{\isacharunderscore}{\kern0pt}rel{\isacharunderscore}{\kern0pt}on\ X\ {\isacharparenleft}{\kern0pt}R{\isacharcomma}{\kern0pt}m{\isacharparenright}{\kern0pt}{\isachardoublequoteclose}\isanewline
\ \ \isakeyword{shows}\ {\isachardoublequoteopen}coequalizer\ {\isacharparenleft}{\kern0pt}quotient{\isacharunderscore}{\kern0pt}set\ X\ {\isacharparenleft}{\kern0pt}R{\isacharcomma}{\kern0pt}m{\isacharparenright}{\kern0pt}{\isacharparenright}{\kern0pt}\ {\isacharparenleft}{\kern0pt}equiv{\isacharunderscore}{\kern0pt}class\ {\isacharparenleft}{\kern0pt}R{\isacharcomma}{\kern0pt}m{\isacharparenright}{\kern0pt}{\isacharparenright}{\kern0pt}\isanewline
\ \ \ \ \ \ \ \ \ \ \ \ \ \ \ \ \ \ \ \ \ {\isacharparenleft}{\kern0pt}left{\isacharunderscore}{\kern0pt}cart{\isacharunderscore}{\kern0pt}proj\ X\ X\ {\isasymcirc}\isactrlsub c\ m{\isacharparenright}{\kern0pt}\ {\isacharparenleft}{\kern0pt}right{\isacharunderscore}{\kern0pt}cart{\isacharunderscore}{\kern0pt}proj\ X\ X\ {\isasymcirc}\isactrlsub c\ m{\isacharparenright}{\kern0pt}{\isachardoublequoteclose}\isanewline
%
\isadelimproof
\ \ %
\endisadelimproof
%
\isatagproof
\isacommand{unfolding}\isamarkupfalse%
\ coequalizer{\isacharunderscore}{\kern0pt}def\ \isanewline
\isacommand{proof}\isamarkupfalse%
{\isacharparenleft}{\kern0pt}rule{\isacharunderscore}{\kern0pt}tac\ x{\isacharequal}{\kern0pt}X\ \isakeyword{in}\ exI{\isacharcomma}{\kern0pt}\ rule{\isacharunderscore}{\kern0pt}tac\ x{\isacharequal}{\kern0pt}\ {\isachardoublequoteopen}R{\isachardoublequoteclose}\ \isakeyword{in}\ exI{\isacharcomma}{\kern0pt}auto{\isacharparenright}{\kern0pt}\isanewline
\ \ \isacommand{have}\isamarkupfalse%
\ m{\isacharunderscore}{\kern0pt}type{\isacharcolon}{\kern0pt}\ {\isachardoublequoteopen}m{\isacharcolon}{\kern0pt}\ R\ {\isasymrightarrow}X\ {\isasymtimes}\isactrlsub c\ X{\isachardoublequoteclose}\isanewline
\ \ \ \ \isacommand{using}\isamarkupfalse%
\ assms\ equiv{\isacharunderscore}{\kern0pt}rel{\isacharunderscore}{\kern0pt}on{\isacharunderscore}{\kern0pt}def\ subobject{\isacharunderscore}{\kern0pt}of{\isacharunderscore}{\kern0pt}def{\isadigit{2}}\ transitive{\isacharunderscore}{\kern0pt}on{\isacharunderscore}{\kern0pt}def\ \isacommand{by}\isamarkupfalse%
\ blast\isanewline
\ \ \isacommand{show}\isamarkupfalse%
\ {\isachardoublequoteopen}left{\isacharunderscore}{\kern0pt}cart{\isacharunderscore}{\kern0pt}proj\ X\ X\ {\isasymcirc}\isactrlsub c\ m\ {\isacharcolon}{\kern0pt}\ R\ {\isasymrightarrow}\ X{\isachardoublequoteclose}\isanewline
\ \ \ \ \isacommand{using}\isamarkupfalse%
\ m{\isacharunderscore}{\kern0pt}type\ \isacommand{by}\isamarkupfalse%
\ typecheck{\isacharunderscore}{\kern0pt}cfuncs\isanewline
\ \ \isacommand{show}\isamarkupfalse%
\ {\isachardoublequoteopen}right{\isacharunderscore}{\kern0pt}cart{\isacharunderscore}{\kern0pt}proj\ X\ X\ {\isasymcirc}\isactrlsub c\ m\ {\isacharcolon}{\kern0pt}\ R\ {\isasymrightarrow}\ X{\isachardoublequoteclose}\isanewline
\ \ \ \ \isacommand{using}\isamarkupfalse%
\ m{\isacharunderscore}{\kern0pt}type\ \isacommand{by}\isamarkupfalse%
\ typecheck{\isacharunderscore}{\kern0pt}cfuncs\isanewline
\ \ \isacommand{show}\isamarkupfalse%
\ {\isachardoublequoteopen}equiv{\isacharunderscore}{\kern0pt}class\ {\isacharparenleft}{\kern0pt}R{\isacharcomma}{\kern0pt}\ m{\isacharparenright}{\kern0pt}\ {\isacharcolon}{\kern0pt}\ X\ {\isasymrightarrow}\ quotient{\isacharunderscore}{\kern0pt}set\ X\ {\isacharparenleft}{\kern0pt}R{\isacharcomma}{\kern0pt}m{\isacharparenright}{\kern0pt}{\isachardoublequoteclose}\isanewline
\ \ \ \ \isacommand{by}\isamarkupfalse%
\ {\isacharparenleft}{\kern0pt}simp\ add{\isacharcolon}{\kern0pt}\ assms\ equiv{\isacharunderscore}{\kern0pt}class{\isacharunderscore}{\kern0pt}type{\isacharparenright}{\kern0pt}\isanewline
\ \ \isacommand{show}\isamarkupfalse%
\ {\isachardoublequoteopen}equiv{\isacharunderscore}{\kern0pt}class\ {\isacharparenleft}{\kern0pt}R{\isacharcomma}{\kern0pt}\ m{\isacharparenright}{\kern0pt}\ {\isasymcirc}\isactrlsub c\ left{\isacharunderscore}{\kern0pt}cart{\isacharunderscore}{\kern0pt}proj\ X\ X\ {\isasymcirc}\isactrlsub c\ m\ {\isacharequal}{\kern0pt}\ equiv{\isacharunderscore}{\kern0pt}class\ {\isacharparenleft}{\kern0pt}R{\isacharcomma}{\kern0pt}\ m{\isacharparenright}{\kern0pt}\ {\isasymcirc}\isactrlsub c\ right{\isacharunderscore}{\kern0pt}cart{\isacharunderscore}{\kern0pt}proj\ X\ X\ {\isasymcirc}\isactrlsub c\ m{\isachardoublequoteclose}\isanewline
\ \ \isacommand{proof}\isamarkupfalse%
{\isacharparenleft}{\kern0pt}rule\ one{\isacharunderscore}{\kern0pt}separator{\isacharbrackleft}{\kern0pt}\isakeyword{where}\ X{\isacharequal}{\kern0pt}{\isachardoublequoteopen}R{\isachardoublequoteclose}{\isacharcomma}{\kern0pt}\ \isakeyword{where}\ Y\ {\isacharequal}{\kern0pt}\ {\isachardoublequoteopen}quotient{\isacharunderscore}{\kern0pt}set\ X\ {\isacharparenleft}{\kern0pt}R{\isacharcomma}{\kern0pt}m{\isacharparenright}{\kern0pt}{\isachardoublequoteclose}{\isacharbrackright}{\kern0pt}{\isacharparenright}{\kern0pt}\isanewline
\ \ \ \ \isacommand{show}\isamarkupfalse%
\ {\isachardoublequoteopen}equiv{\isacharunderscore}{\kern0pt}class\ {\isacharparenleft}{\kern0pt}R{\isacharcomma}{\kern0pt}\ m{\isacharparenright}{\kern0pt}\ {\isasymcirc}\isactrlsub c\ left{\isacharunderscore}{\kern0pt}cart{\isacharunderscore}{\kern0pt}proj\ X\ X\ {\isasymcirc}\isactrlsub c\ m\ {\isacharcolon}{\kern0pt}\ R\ {\isasymrightarrow}\ quotient{\isacharunderscore}{\kern0pt}set\ X\ {\isacharparenleft}{\kern0pt}R{\isacharcomma}{\kern0pt}\ m{\isacharparenright}{\kern0pt}{\isachardoublequoteclose}\isanewline
\ \ \ \ \ \ \isacommand{using}\isamarkupfalse%
\ m{\isacharunderscore}{\kern0pt}type\ assms\ \isacommand{by}\isamarkupfalse%
\ typecheck{\isacharunderscore}{\kern0pt}cfuncs\isanewline
\ \ \ \ \isacommand{show}\isamarkupfalse%
\ {\isachardoublequoteopen}equiv{\isacharunderscore}{\kern0pt}class\ {\isacharparenleft}{\kern0pt}R{\isacharcomma}{\kern0pt}\ m{\isacharparenright}{\kern0pt}\ {\isasymcirc}\isactrlsub c\ right{\isacharunderscore}{\kern0pt}cart{\isacharunderscore}{\kern0pt}proj\ X\ X\ {\isasymcirc}\isactrlsub c\ m\ {\isacharcolon}{\kern0pt}\ R\ {\isasymrightarrow}\ quotient{\isacharunderscore}{\kern0pt}set\ X\ {\isacharparenleft}{\kern0pt}R{\isacharcomma}{\kern0pt}\ m{\isacharparenright}{\kern0pt}{\isachardoublequoteclose}\isanewline
\ \ \ \ \ \ \isacommand{using}\isamarkupfalse%
\ m{\isacharunderscore}{\kern0pt}type\ assms\ \isacommand{by}\isamarkupfalse%
\ typecheck{\isacharunderscore}{\kern0pt}cfuncs\isanewline
\ \ \isacommand{next}\isamarkupfalse%
\isanewline
\ \ \ \ \isacommand{fix}\isamarkupfalse%
\ x\ \isanewline
\ \ \ \ \isacommand{assume}\isamarkupfalse%
\ x{\isacharunderscore}{\kern0pt}type{\isacharcolon}{\kern0pt}\ {\isachardoublequoteopen}x\ {\isasymin}\isactrlsub c\ R{\isachardoublequoteclose}\isanewline
\ \ \ \ \isacommand{then}\isamarkupfalse%
\ \isacommand{have}\isamarkupfalse%
\ m{\isacharunderscore}{\kern0pt}x{\isacharunderscore}{\kern0pt}type{\isacharcolon}{\kern0pt}\ {\isachardoublequoteopen}m\ {\isasymcirc}\isactrlsub c\ x\ {\isasymin}\isactrlsub c\ X\ {\isasymtimes}\isactrlsub c\ X{\isachardoublequoteclose}\isanewline
\ \ \ \ \ \ \isacommand{using}\isamarkupfalse%
\ m{\isacharunderscore}{\kern0pt}type\ \isacommand{by}\isamarkupfalse%
\ typecheck{\isacharunderscore}{\kern0pt}cfuncs\isanewline
\ \ \ \ \isacommand{then}\isamarkupfalse%
\ \isacommand{obtain}\isamarkupfalse%
\ a\ b\ \isakeyword{where}\ a{\isacharunderscore}{\kern0pt}type{\isacharcolon}{\kern0pt}\ {\isachardoublequoteopen}a\ {\isasymin}\isactrlsub c\ X{\isachardoublequoteclose}\ \isakeyword{and}\ b{\isacharunderscore}{\kern0pt}type{\isacharcolon}{\kern0pt}\ {\isachardoublequoteopen}b\ {\isasymin}\isactrlsub c\ X{\isachardoublequoteclose}\ \isakeyword{and}\ m{\isacharunderscore}{\kern0pt}x{\isacharunderscore}{\kern0pt}eq{\isacharcolon}{\kern0pt}\ {\isachardoublequoteopen}m\ {\isasymcirc}\isactrlsub c\ x\ {\isacharequal}{\kern0pt}\ {\isasymlangle}a{\isacharcomma}{\kern0pt}b{\isasymrangle}{\isachardoublequoteclose}\isanewline
\ \ \ \ \ \ \isacommand{using}\isamarkupfalse%
\ cart{\isacharunderscore}{\kern0pt}prod{\isacharunderscore}{\kern0pt}decomp\ \isacommand{by}\isamarkupfalse%
\ blast\isanewline
\ \ \ \ \isacommand{then}\isamarkupfalse%
\ \isacommand{have}\isamarkupfalse%
\ ab{\isacharunderscore}{\kern0pt}inR{\isacharunderscore}{\kern0pt}relXX{\isacharcolon}{\kern0pt}\ {\isachardoublequoteopen}{\isasymlangle}a{\isacharcomma}{\kern0pt}b{\isasymrangle}\ {\isasymin}\isactrlbsub X\ {\isasymtimes}\isactrlsub c\ X\isactrlesub {\isacharparenleft}{\kern0pt}R{\isacharcomma}{\kern0pt}m{\isacharparenright}{\kern0pt}{\isachardoublequoteclose}\isanewline
\ \ \ \ \ \ \isacommand{using}\isamarkupfalse%
\ assms\ cfunc{\isacharunderscore}{\kern0pt}type{\isacharunderscore}{\kern0pt}def\ equiv{\isacharunderscore}{\kern0pt}rel{\isacharunderscore}{\kern0pt}on{\isacharunderscore}{\kern0pt}def\ factors{\isacharunderscore}{\kern0pt}through{\isacharunderscore}{\kern0pt}def\ m{\isacharunderscore}{\kern0pt}x{\isacharunderscore}{\kern0pt}type\ reflexive{\isacharunderscore}{\kern0pt}on{\isacharunderscore}{\kern0pt}def\ relative{\isacharunderscore}{\kern0pt}member{\isacharunderscore}{\kern0pt}def{\isadigit{2}}\ x{\isacharunderscore}{\kern0pt}type\ \isacommand{by}\isamarkupfalse%
\ auto\isanewline
\ \ \ \ \isacommand{then}\isamarkupfalse%
\ \isacommand{have}\isamarkupfalse%
\ {\isachardoublequoteopen}equiv{\isacharunderscore}{\kern0pt}class\ {\isacharparenleft}{\kern0pt}R{\isacharcomma}{\kern0pt}\ m{\isacharparenright}{\kern0pt}\ {\isasymcirc}\isactrlsub c\ a\ {\isacharequal}{\kern0pt}\ equiv{\isacharunderscore}{\kern0pt}class\ {\isacharparenleft}{\kern0pt}R{\isacharcomma}{\kern0pt}\ m{\isacharparenright}{\kern0pt}\ {\isasymcirc}\isactrlsub c\ b{\isachardoublequoteclose}\isanewline
\ \ \ \ \ \ \isacommand{using}\isamarkupfalse%
\ equiv{\isacharunderscore}{\kern0pt}class{\isacharunderscore}{\kern0pt}eq\ assms\ relative{\isacharunderscore}{\kern0pt}member{\isacharunderscore}{\kern0pt}def\ \isacommand{by}\isamarkupfalse%
\ blast\isanewline
\ \ \ \ \isacommand{then}\isamarkupfalse%
\ \isacommand{have}\isamarkupfalse%
\ {\isachardoublequoteopen}equiv{\isacharunderscore}{\kern0pt}class\ {\isacharparenleft}{\kern0pt}R{\isacharcomma}{\kern0pt}\ m{\isacharparenright}{\kern0pt}\ {\isasymcirc}\isactrlsub c\ left{\isacharunderscore}{\kern0pt}cart{\isacharunderscore}{\kern0pt}proj\ X\ X\ {\isasymcirc}\isactrlsub c\ {\isasymlangle}a{\isacharcomma}{\kern0pt}b{\isasymrangle}\ {\isacharequal}{\kern0pt}\ equiv{\isacharunderscore}{\kern0pt}class\ {\isacharparenleft}{\kern0pt}R{\isacharcomma}{\kern0pt}\ m{\isacharparenright}{\kern0pt}\ {\isasymcirc}\isactrlsub c\ right{\isacharunderscore}{\kern0pt}cart{\isacharunderscore}{\kern0pt}proj\ X\ X\ {\isasymcirc}\isactrlsub c\ {\isasymlangle}a{\isacharcomma}{\kern0pt}b{\isasymrangle}{\isachardoublequoteclose}\isanewline
\ \ \ \ \ \ \isacommand{using}\isamarkupfalse%
\ a{\isacharunderscore}{\kern0pt}type\ b{\isacharunderscore}{\kern0pt}type\ left{\isacharunderscore}{\kern0pt}cart{\isacharunderscore}{\kern0pt}proj{\isacharunderscore}{\kern0pt}cfunc{\isacharunderscore}{\kern0pt}prod\ right{\isacharunderscore}{\kern0pt}cart{\isacharunderscore}{\kern0pt}proj{\isacharunderscore}{\kern0pt}cfunc{\isacharunderscore}{\kern0pt}prod\ \isacommand{by}\isamarkupfalse%
\ auto\isanewline
\ \ \ \ \isacommand{then}\isamarkupfalse%
\ \isacommand{have}\isamarkupfalse%
\ {\isachardoublequoteopen}equiv{\isacharunderscore}{\kern0pt}class\ {\isacharparenleft}{\kern0pt}R{\isacharcomma}{\kern0pt}\ m{\isacharparenright}{\kern0pt}\ {\isasymcirc}\isactrlsub c\ left{\isacharunderscore}{\kern0pt}cart{\isacharunderscore}{\kern0pt}proj\ X\ X\ {\isasymcirc}\isactrlsub c\ m\ {\isasymcirc}\isactrlsub c\ x\ {\isacharequal}{\kern0pt}\ equiv{\isacharunderscore}{\kern0pt}class\ {\isacharparenleft}{\kern0pt}R{\isacharcomma}{\kern0pt}\ m{\isacharparenright}{\kern0pt}\ {\isasymcirc}\isactrlsub c\ right{\isacharunderscore}{\kern0pt}cart{\isacharunderscore}{\kern0pt}proj\ X\ X\ {\isasymcirc}\isactrlsub c\ m\ {\isasymcirc}\isactrlsub c\ x{\isachardoublequoteclose}\isanewline
\ \ \ \ \ \ \isacommand{by}\isamarkupfalse%
\ {\isacharparenleft}{\kern0pt}simp\ add{\isacharcolon}{\kern0pt}\ m{\isacharunderscore}{\kern0pt}x{\isacharunderscore}{\kern0pt}eq{\isacharparenright}{\kern0pt}\isanewline
\ \ \ \ \isacommand{then}\isamarkupfalse%
\ \isacommand{show}\isamarkupfalse%
\ {\isachardoublequoteopen}{\isacharparenleft}{\kern0pt}equiv{\isacharunderscore}{\kern0pt}class\ {\isacharparenleft}{\kern0pt}R{\isacharcomma}{\kern0pt}\ m{\isacharparenright}{\kern0pt}\ {\isasymcirc}\isactrlsub c\ left{\isacharunderscore}{\kern0pt}cart{\isacharunderscore}{\kern0pt}proj\ X\ X\ {\isasymcirc}\isactrlsub c\ m{\isacharparenright}{\kern0pt}\ {\isasymcirc}\isactrlsub c\ x\ {\isacharequal}{\kern0pt}\ {\isacharparenleft}{\kern0pt}equiv{\isacharunderscore}{\kern0pt}class\ {\isacharparenleft}{\kern0pt}R{\isacharcomma}{\kern0pt}\ m{\isacharparenright}{\kern0pt}\ {\isasymcirc}\isactrlsub c\ right{\isacharunderscore}{\kern0pt}cart{\isacharunderscore}{\kern0pt}proj\ X\ X\ {\isasymcirc}\isactrlsub c\ m{\isacharparenright}{\kern0pt}\ {\isasymcirc}\isactrlsub c\ x{\isachardoublequoteclose}\isanewline
\ \ \ \ \ \ \isacommand{using}\isamarkupfalse%
\ x{\isacharunderscore}{\kern0pt}type\ m{\isacharunderscore}{\kern0pt}type\ assms\ \isacommand{by}\isamarkupfalse%
\ {\isacharparenleft}{\kern0pt}typecheck{\isacharunderscore}{\kern0pt}cfuncs{\isacharcomma}{\kern0pt}\ metis\ cfunc{\isacharunderscore}{\kern0pt}type{\isacharunderscore}{\kern0pt}def\ comp{\isacharunderscore}{\kern0pt}associative\ m{\isacharunderscore}{\kern0pt}x{\isacharunderscore}{\kern0pt}eq{\isacharparenright}{\kern0pt}\isanewline
\ \ \isacommand{qed}\isamarkupfalse%
\ \ \ \isanewline
\isacommand{next}\isamarkupfalse%
\isanewline
\ \ \isacommand{fix}\isamarkupfalse%
\ h\ F\ \isanewline
\ \ \isacommand{assume}\isamarkupfalse%
\ h{\isacharunderscore}{\kern0pt}type{\isacharcolon}{\kern0pt}\ {\isachardoublequoteopen}\ h\ {\isacharcolon}{\kern0pt}\ X\ {\isasymrightarrow}\ F{\isachardoublequoteclose}\isanewline
\ \ \isacommand{assume}\isamarkupfalse%
\ h{\isacharunderscore}{\kern0pt}proj{\isadigit{1}}{\isacharunderscore}{\kern0pt}eqs{\isacharunderscore}{\kern0pt}h{\isacharunderscore}{\kern0pt}proj{\isadigit{2}}{\isacharcolon}{\kern0pt}\ {\isachardoublequoteopen}h\ {\isasymcirc}\isactrlsub c\ left{\isacharunderscore}{\kern0pt}cart{\isacharunderscore}{\kern0pt}proj\ X\ X\ {\isasymcirc}\isactrlsub c\ m\ {\isacharequal}{\kern0pt}\ h\ {\isasymcirc}\isactrlsub c\ right{\isacharunderscore}{\kern0pt}cart{\isacharunderscore}{\kern0pt}proj\ X\ X\ {\isasymcirc}\isactrlsub c\ m{\isachardoublequoteclose}\isanewline
\isanewline
\ \ \isacommand{have}\isamarkupfalse%
\ m{\isacharunderscore}{\kern0pt}type{\isacharcolon}{\kern0pt}\ {\isachardoublequoteopen}m\ {\isacharcolon}{\kern0pt}\ R\ {\isasymrightarrow}\ X\ {\isasymtimes}\isactrlsub c\ X{\isachardoublequoteclose}\isanewline
\ \ \ \ \isacommand{using}\isamarkupfalse%
\ assms\ equiv{\isacharunderscore}{\kern0pt}rel{\isacharunderscore}{\kern0pt}on{\isacharunderscore}{\kern0pt}def\ reflexive{\isacharunderscore}{\kern0pt}on{\isacharunderscore}{\kern0pt}def\ subobject{\isacharunderscore}{\kern0pt}of{\isacharunderscore}{\kern0pt}def{\isadigit{2}}\ \isacommand{by}\isamarkupfalse%
\ blast\isanewline
\ \ \isacommand{have}\isamarkupfalse%
\ {\isachardoublequoteopen}const{\isacharunderscore}{\kern0pt}on{\isacharunderscore}{\kern0pt}rel\ X\ {\isacharparenleft}{\kern0pt}R{\isacharcomma}{\kern0pt}\ m{\isacharparenright}{\kern0pt}\ h{\isachardoublequoteclose}\isanewline
\ \ \isacommand{proof}\isamarkupfalse%
\ {\isacharparenleft}{\kern0pt}unfold\ const{\isacharunderscore}{\kern0pt}on{\isacharunderscore}{\kern0pt}rel{\isacharunderscore}{\kern0pt}def{\isacharcomma}{\kern0pt}\ auto{\isacharparenright}{\kern0pt}\isanewline
\ \ \ \ \isacommand{fix}\isamarkupfalse%
\ x\ y\isanewline
\ \ \ \ \isacommand{assume}\isamarkupfalse%
\ x{\isacharunderscore}{\kern0pt}type{\isacharcolon}{\kern0pt}\ {\isachardoublequoteopen}x\ {\isasymin}\isactrlsub c\ X{\isachardoublequoteclose}\ \isakeyword{and}\ y{\isacharunderscore}{\kern0pt}type{\isacharcolon}{\kern0pt}\ {\isachardoublequoteopen}y\ {\isasymin}\isactrlsub c\ X{\isachardoublequoteclose}\isanewline
\ \ \ \ \isacommand{assume}\isamarkupfalse%
\ {\isachardoublequoteopen}{\isasymlangle}x{\isacharcomma}{\kern0pt}y{\isasymrangle}\ {\isasymin}\isactrlbsub X\ {\isasymtimes}\isactrlsub c\ X\isactrlesub \ {\isacharparenleft}{\kern0pt}R{\isacharcomma}{\kern0pt}\ m{\isacharparenright}{\kern0pt}{\isachardoublequoteclose}\isanewline
\ \ \ \ \isacommand{then}\isamarkupfalse%
\ \isacommand{obtain}\isamarkupfalse%
\ xy\ \isakeyword{where}\ xy{\isacharunderscore}{\kern0pt}type{\isacharcolon}{\kern0pt}\ {\isachardoublequoteopen}xy\ {\isasymin}\isactrlsub c\ R{\isachardoublequoteclose}\ \isakeyword{and}\ m{\isacharunderscore}{\kern0pt}h{\isacharunderscore}{\kern0pt}eq{\isacharcolon}{\kern0pt}\ {\isachardoublequoteopen}m\ {\isasymcirc}\isactrlsub c\ xy\ {\isacharequal}{\kern0pt}\ {\isasymlangle}x{\isacharcomma}{\kern0pt}y{\isasymrangle}{\isachardoublequoteclose}\isanewline
\ \ \ \ \ \ \isacommand{unfolding}\isamarkupfalse%
\ relative{\isacharunderscore}{\kern0pt}member{\isacharunderscore}{\kern0pt}def{\isadigit{2}}\ factors{\isacharunderscore}{\kern0pt}through{\isacharunderscore}{\kern0pt}def\ \isacommand{using}\isamarkupfalse%
\ cfunc{\isacharunderscore}{\kern0pt}type{\isacharunderscore}{\kern0pt}def\ \isacommand{by}\isamarkupfalse%
\ auto\isanewline
\isanewline
\ \ \ \ \isacommand{have}\isamarkupfalse%
\ {\isachardoublequoteopen}h\ {\isasymcirc}\isactrlsub c\ left{\isacharunderscore}{\kern0pt}cart{\isacharunderscore}{\kern0pt}proj\ X\ X\ {\isasymcirc}\isactrlsub c\ m\ {\isasymcirc}\isactrlsub c\ xy\ {\isacharequal}{\kern0pt}\ h\ {\isasymcirc}\isactrlsub c\ right{\isacharunderscore}{\kern0pt}cart{\isacharunderscore}{\kern0pt}proj\ X\ X\ {\isasymcirc}\isactrlsub c\ m\ {\isasymcirc}\isactrlsub c\ xy{\isachardoublequoteclose}\isanewline
\ \ \ \ \ \ \isacommand{using}\isamarkupfalse%
\ h{\isacharunderscore}{\kern0pt}type\ m{\isacharunderscore}{\kern0pt}type\ xy{\isacharunderscore}{\kern0pt}type\ \isacommand{by}\isamarkupfalse%
\ {\isacharparenleft}{\kern0pt}typecheck{\isacharunderscore}{\kern0pt}cfuncs{\isacharcomma}{\kern0pt}\ smt\ comp{\isacharunderscore}{\kern0pt}associative{\isadigit{2}}\ comp{\isacharunderscore}{\kern0pt}type\ h{\isacharunderscore}{\kern0pt}proj{\isadigit{1}}{\isacharunderscore}{\kern0pt}eqs{\isacharunderscore}{\kern0pt}h{\isacharunderscore}{\kern0pt}proj{\isadigit{2}}{\isacharparenright}{\kern0pt}\isanewline
\ \ \ \ \isacommand{then}\isamarkupfalse%
\ \isacommand{have}\isamarkupfalse%
\ {\isachardoublequoteopen}h\ {\isasymcirc}\isactrlsub c\ left{\isacharunderscore}{\kern0pt}cart{\isacharunderscore}{\kern0pt}proj\ X\ X\ {\isasymcirc}\isactrlsub c\ {\isasymlangle}x{\isacharcomma}{\kern0pt}y{\isasymrangle}\ {\isacharequal}{\kern0pt}\ h\ {\isasymcirc}\isactrlsub c\ right{\isacharunderscore}{\kern0pt}cart{\isacharunderscore}{\kern0pt}proj\ X\ X\ {\isasymcirc}\isactrlsub c\ {\isasymlangle}x{\isacharcomma}{\kern0pt}y{\isasymrangle}{\isachardoublequoteclose}\isanewline
\ \ \ \ \ \ \isacommand{using}\isamarkupfalse%
\ m{\isacharunderscore}{\kern0pt}h{\isacharunderscore}{\kern0pt}eq\ \isacommand{by}\isamarkupfalse%
\ auto\isanewline
\ \ \ \ \isacommand{then}\isamarkupfalse%
\ \isacommand{show}\isamarkupfalse%
\ {\isachardoublequoteopen}h\ {\isasymcirc}\isactrlsub c\ x\ {\isacharequal}{\kern0pt}\ h\ {\isasymcirc}\isactrlsub c\ y{\isachardoublequoteclose}\isanewline
\ \ \ \ \ \ \isacommand{using}\isamarkupfalse%
\ left{\isacharunderscore}{\kern0pt}cart{\isacharunderscore}{\kern0pt}proj{\isacharunderscore}{\kern0pt}cfunc{\isacharunderscore}{\kern0pt}prod\ right{\isacharunderscore}{\kern0pt}cart{\isacharunderscore}{\kern0pt}proj{\isacharunderscore}{\kern0pt}cfunc{\isacharunderscore}{\kern0pt}prod\ x{\isacharunderscore}{\kern0pt}type\ y{\isacharunderscore}{\kern0pt}type\ \isacommand{by}\isamarkupfalse%
\ auto\isanewline
\ \ \isacommand{qed}\isamarkupfalse%
\isanewline
\ \ \isacommand{then}\isamarkupfalse%
\ \isacommand{show}\isamarkupfalse%
\ {\isachardoublequoteopen}{\isasymexists}k{\isachardot}{\kern0pt}\ k\ {\isacharcolon}{\kern0pt}\ quotient{\isacharunderscore}{\kern0pt}set\ X\ {\isacharparenleft}{\kern0pt}R{\isacharcomma}{\kern0pt}\ m{\isacharparenright}{\kern0pt}\ {\isasymrightarrow}\ F\ {\isasymand}\ k\ {\isasymcirc}\isactrlsub c\ equiv{\isacharunderscore}{\kern0pt}class\ {\isacharparenleft}{\kern0pt}R{\isacharcomma}{\kern0pt}\ m{\isacharparenright}{\kern0pt}\ {\isacharequal}{\kern0pt}\ h{\isachardoublequoteclose}\isanewline
\ \ \ \ \isacommand{using}\isamarkupfalse%
\ assms\ h{\isacharunderscore}{\kern0pt}type\ quotient{\isacharunderscore}{\kern0pt}func{\isacharunderscore}{\kern0pt}type\ quotient{\isacharunderscore}{\kern0pt}func{\isacharunderscore}{\kern0pt}eq\isanewline
\ \ \ \ \isacommand{by}\isamarkupfalse%
\ {\isacharparenleft}{\kern0pt}rule{\isacharunderscore}{\kern0pt}tac\ x{\isacharequal}{\kern0pt}{\isachardoublequoteopen}quotient{\isacharunderscore}{\kern0pt}func\ h\ {\isacharparenleft}{\kern0pt}R{\isacharcomma}{\kern0pt}\ m{\isacharparenright}{\kern0pt}{\isachardoublequoteclose}\ \isakeyword{in}\ exI{\isacharcomma}{\kern0pt}\ auto{\isacharparenright}{\kern0pt}\isanewline
\isacommand{next}\isamarkupfalse%
\isanewline
\ \ \isacommand{fix}\isamarkupfalse%
\ F\ k\ y\isanewline
\ \ \isacommand{assume}\isamarkupfalse%
\ k{\isacharunderscore}{\kern0pt}type{\isacharcolon}{\kern0pt}\ {\isachardoublequoteopen}k\ {\isacharcolon}{\kern0pt}\ quotient{\isacharunderscore}{\kern0pt}set\ X\ {\isacharparenleft}{\kern0pt}R{\isacharcomma}{\kern0pt}\ m{\isacharparenright}{\kern0pt}\ {\isasymrightarrow}\ F{\isachardoublequoteclose}\isanewline
\ \ \isacommand{assume}\isamarkupfalse%
\ y{\isacharunderscore}{\kern0pt}type{\isacharcolon}{\kern0pt}\ {\isachardoublequoteopen}y\ {\isacharcolon}{\kern0pt}\ quotient{\isacharunderscore}{\kern0pt}set\ X\ {\isacharparenleft}{\kern0pt}R{\isacharcomma}{\kern0pt}\ m{\isacharparenright}{\kern0pt}\ {\isasymrightarrow}\ F{\isachardoublequoteclose}\isanewline
\ \ \isacommand{assume}\isamarkupfalse%
\ k{\isacharunderscore}{\kern0pt}equiv{\isacharunderscore}{\kern0pt}class{\isacharunderscore}{\kern0pt}type{\isacharcolon}{\kern0pt}\ {\isachardoublequoteopen}k\ {\isasymcirc}\isactrlsub c\ equiv{\isacharunderscore}{\kern0pt}class\ {\isacharparenleft}{\kern0pt}R{\isacharcomma}{\kern0pt}\ m{\isacharparenright}{\kern0pt}\ {\isacharcolon}{\kern0pt}\ X\ {\isasymrightarrow}\ F{\isachardoublequoteclose}\isanewline
\ \ \isacommand{assume}\isamarkupfalse%
\ k{\isacharunderscore}{\kern0pt}equiv{\isacharunderscore}{\kern0pt}class{\isacharunderscore}{\kern0pt}eq{\isacharcolon}{\kern0pt}\ {\isachardoublequoteopen}{\isacharparenleft}{\kern0pt}k\ {\isasymcirc}\isactrlsub c\ equiv{\isacharunderscore}{\kern0pt}class\ {\isacharparenleft}{\kern0pt}R{\isacharcomma}{\kern0pt}\ m{\isacharparenright}{\kern0pt}{\isacharparenright}{\kern0pt}\ {\isasymcirc}\isactrlsub c\ left{\isacharunderscore}{\kern0pt}cart{\isacharunderscore}{\kern0pt}proj\ X\ X\ {\isasymcirc}\isactrlsub c\ m\ {\isacharequal}{\kern0pt}\isanewline
\ \ \ \ \ \ \ {\isacharparenleft}{\kern0pt}k\ {\isasymcirc}\isactrlsub c\ equiv{\isacharunderscore}{\kern0pt}class\ {\isacharparenleft}{\kern0pt}R{\isacharcomma}{\kern0pt}\ m{\isacharparenright}{\kern0pt}{\isacharparenright}{\kern0pt}\ {\isasymcirc}\isactrlsub c\ right{\isacharunderscore}{\kern0pt}cart{\isacharunderscore}{\kern0pt}proj\ X\ X\ {\isasymcirc}\isactrlsub c\ m{\isachardoublequoteclose}\isanewline
\ \ \isacommand{assume}\isamarkupfalse%
\ y{\isacharunderscore}{\kern0pt}k{\isacharunderscore}{\kern0pt}eq{\isacharcolon}{\kern0pt}\ {\isachardoublequoteopen}y\ {\isasymcirc}\isactrlsub c\ equiv{\isacharunderscore}{\kern0pt}class\ {\isacharparenleft}{\kern0pt}R{\isacharcomma}{\kern0pt}\ m{\isacharparenright}{\kern0pt}\ {\isacharequal}{\kern0pt}\ k\ {\isasymcirc}\isactrlsub c\ equiv{\isacharunderscore}{\kern0pt}class\ {\isacharparenleft}{\kern0pt}R{\isacharcomma}{\kern0pt}\ m{\isacharparenright}{\kern0pt}{\isachardoublequoteclose}\isanewline
\isanewline
\ \ \isacommand{have}\isamarkupfalse%
\ m{\isacharunderscore}{\kern0pt}type{\isacharcolon}{\kern0pt}\ {\isachardoublequoteopen}m\ {\isacharcolon}{\kern0pt}\ R\ {\isasymrightarrow}\ X\ {\isasymtimes}\isactrlsub c\ X{\isachardoublequoteclose}\isanewline
\ \ \ \ \isacommand{using}\isamarkupfalse%
\ assms\ equiv{\isacharunderscore}{\kern0pt}rel{\isacharunderscore}{\kern0pt}on{\isacharunderscore}{\kern0pt}def\ reflexive{\isacharunderscore}{\kern0pt}on{\isacharunderscore}{\kern0pt}def\ subobject{\isacharunderscore}{\kern0pt}of{\isacharunderscore}{\kern0pt}def{\isadigit{2}}\ \isacommand{by}\isamarkupfalse%
\ blast\isanewline
\isanewline
\ \ \isacommand{have}\isamarkupfalse%
\ y{\isacharunderscore}{\kern0pt}eq{\isacharcolon}{\kern0pt}\ {\isachardoublequoteopen}y\ {\isacharequal}{\kern0pt}\ quotient{\isacharunderscore}{\kern0pt}func\ {\isacharparenleft}{\kern0pt}y\ {\isasymcirc}\isactrlsub c\ equiv{\isacharunderscore}{\kern0pt}class\ {\isacharparenleft}{\kern0pt}R{\isacharcomma}{\kern0pt}\ m{\isacharparenright}{\kern0pt}{\isacharparenright}{\kern0pt}\ {\isacharparenleft}{\kern0pt}R{\isacharcomma}{\kern0pt}\ m{\isacharparenright}{\kern0pt}{\isachardoublequoteclose}\isanewline
\ \ \ \ \isacommand{using}\isamarkupfalse%
\ assms\ y{\isacharunderscore}{\kern0pt}type\ k{\isacharunderscore}{\kern0pt}equiv{\isacharunderscore}{\kern0pt}class{\isacharunderscore}{\kern0pt}type\ y{\isacharunderscore}{\kern0pt}k{\isacharunderscore}{\kern0pt}eq\isanewline
\ \ \isacommand{proof}\isamarkupfalse%
\ {\isacharparenleft}{\kern0pt}rule{\isacharunderscore}{\kern0pt}tac\ quotient{\isacharunderscore}{\kern0pt}func{\isacharunderscore}{\kern0pt}unique{\isacharbrackleft}{\kern0pt}\isakeyword{where}\ X{\isacharequal}{\kern0pt}X{\isacharcomma}{\kern0pt}\ \isakeyword{where}\ Y{\isacharequal}{\kern0pt}F{\isacharbrackright}{\kern0pt}{\isacharcomma}{\kern0pt}\ simp{\isacharunderscore}{\kern0pt}all{\isacharcomma}{\kern0pt}\ unfold\ const{\isacharunderscore}{\kern0pt}on{\isacharunderscore}{\kern0pt}rel{\isacharunderscore}{\kern0pt}def{\isacharcomma}{\kern0pt}\ auto{\isacharparenright}{\kern0pt}\isanewline
\ \ \ \ \isacommand{fix}\isamarkupfalse%
\ a\ b\isanewline
\ \ \ \ \isacommand{assume}\isamarkupfalse%
\ a{\isacharunderscore}{\kern0pt}type{\isacharcolon}{\kern0pt}\ {\isachardoublequoteopen}a\ {\isasymin}\isactrlsub c\ X{\isachardoublequoteclose}\ \isakeyword{and}\ b{\isacharunderscore}{\kern0pt}type{\isacharcolon}{\kern0pt}\ {\isachardoublequoteopen}b\ {\isasymin}\isactrlsub c\ X{\isachardoublequoteclose}\isanewline
\ \ \ \ \isacommand{assume}\isamarkupfalse%
\ ab{\isacharunderscore}{\kern0pt}in{\isacharunderscore}{\kern0pt}R{\isacharcolon}{\kern0pt}\ {\isachardoublequoteopen}{\isasymlangle}a{\isacharcomma}{\kern0pt}b{\isasymrangle}\ {\isasymin}\isactrlbsub X\ {\isasymtimes}\isactrlsub c\ X\isactrlesub \ {\isacharparenleft}{\kern0pt}R{\isacharcomma}{\kern0pt}\ m{\isacharparenright}{\kern0pt}{\isachardoublequoteclose}\isanewline
\ \ \ \ \isacommand{then}\isamarkupfalse%
\ \isacommand{obtain}\isamarkupfalse%
\ h\ \isakeyword{where}\ h{\isacharunderscore}{\kern0pt}type{\isacharcolon}{\kern0pt}\ {\isachardoublequoteopen}h\ {\isasymin}\isactrlsub c\ R{\isachardoublequoteclose}\ \isakeyword{and}\ m{\isacharunderscore}{\kern0pt}h{\isacharunderscore}{\kern0pt}eq{\isacharcolon}{\kern0pt}\ {\isachardoublequoteopen}m\ {\isasymcirc}\isactrlsub c\ h\ {\isacharequal}{\kern0pt}\ {\isasymlangle}a{\isacharcomma}{\kern0pt}\ b{\isasymrangle}{\isachardoublequoteclose}\isanewline
\ \ \ \ \ \ \isacommand{unfolding}\isamarkupfalse%
\ relative{\isacharunderscore}{\kern0pt}member{\isacharunderscore}{\kern0pt}def\ factors{\isacharunderscore}{\kern0pt}through{\isacharunderscore}{\kern0pt}def\ \isacommand{using}\isamarkupfalse%
\ cfunc{\isacharunderscore}{\kern0pt}type{\isacharunderscore}{\kern0pt}def\ \isacommand{by}\isamarkupfalse%
\ auto\ \isanewline
\isanewline
\ \ \ \ \isacommand{have}\isamarkupfalse%
\ {\isachardoublequoteopen}{\isacharparenleft}{\kern0pt}k\ {\isasymcirc}\isactrlsub c\ equiv{\isacharunderscore}{\kern0pt}class\ {\isacharparenleft}{\kern0pt}R{\isacharcomma}{\kern0pt}\ m{\isacharparenright}{\kern0pt}{\isacharparenright}{\kern0pt}\ {\isasymcirc}\isactrlsub c\ left{\isacharunderscore}{\kern0pt}cart{\isacharunderscore}{\kern0pt}proj\ X\ X\ {\isasymcirc}\isactrlsub c\ m\ {\isasymcirc}\isactrlsub c\ h\ {\isacharequal}{\kern0pt}\isanewline
\ \ \ \ \ \ \ {\isacharparenleft}{\kern0pt}k\ {\isasymcirc}\isactrlsub c\ equiv{\isacharunderscore}{\kern0pt}class\ {\isacharparenleft}{\kern0pt}R{\isacharcomma}{\kern0pt}\ m{\isacharparenright}{\kern0pt}{\isacharparenright}{\kern0pt}\ {\isasymcirc}\isactrlsub c\ right{\isacharunderscore}{\kern0pt}cart{\isacharunderscore}{\kern0pt}proj\ X\ X\ {\isasymcirc}\isactrlsub c\ m\ {\isasymcirc}\isactrlsub c\ h{\isachardoublequoteclose}\isanewline
\ \ \ \ \ \ \isacommand{using}\isamarkupfalse%
\ k{\isacharunderscore}{\kern0pt}type\ m{\isacharunderscore}{\kern0pt}type\ h{\isacharunderscore}{\kern0pt}type\ assms\ \isanewline
\ \ \ \ \ \ \isacommand{by}\isamarkupfalse%
\ {\isacharparenleft}{\kern0pt}typecheck{\isacharunderscore}{\kern0pt}cfuncs{\isacharcomma}{\kern0pt}\ smt\ comp{\isacharunderscore}{\kern0pt}associative{\isadigit{2}}\ comp{\isacharunderscore}{\kern0pt}type\ k{\isacharunderscore}{\kern0pt}equiv{\isacharunderscore}{\kern0pt}class{\isacharunderscore}{\kern0pt}eq{\isacharparenright}{\kern0pt}\isanewline
\ \ \ \ \isacommand{then}\isamarkupfalse%
\ \isacommand{have}\isamarkupfalse%
\ {\isachardoublequoteopen}{\isacharparenleft}{\kern0pt}k\ {\isasymcirc}\isactrlsub c\ equiv{\isacharunderscore}{\kern0pt}class\ {\isacharparenleft}{\kern0pt}R{\isacharcomma}{\kern0pt}\ m{\isacharparenright}{\kern0pt}{\isacharparenright}{\kern0pt}\ {\isasymcirc}\isactrlsub c\ left{\isacharunderscore}{\kern0pt}cart{\isacharunderscore}{\kern0pt}proj\ X\ X\ {\isasymcirc}\isactrlsub c\ {\isasymlangle}a{\isacharcomma}{\kern0pt}\ b{\isasymrangle}\ {\isacharequal}{\kern0pt}\isanewline
\ \ \ \ \ \ \ {\isacharparenleft}{\kern0pt}k\ {\isasymcirc}\isactrlsub c\ equiv{\isacharunderscore}{\kern0pt}class\ {\isacharparenleft}{\kern0pt}R{\isacharcomma}{\kern0pt}\ m{\isacharparenright}{\kern0pt}{\isacharparenright}{\kern0pt}\ {\isasymcirc}\isactrlsub c\ right{\isacharunderscore}{\kern0pt}cart{\isacharunderscore}{\kern0pt}proj\ X\ X\ {\isasymcirc}\isactrlsub c\ {\isasymlangle}a{\isacharcomma}{\kern0pt}\ b{\isasymrangle}{\isachardoublequoteclose}\isanewline
\ \ \ \ \ \ \isacommand{by}\isamarkupfalse%
\ {\isacharparenleft}{\kern0pt}simp\ add{\isacharcolon}{\kern0pt}\ m{\isacharunderscore}{\kern0pt}h{\isacharunderscore}{\kern0pt}eq{\isacharparenright}{\kern0pt}\isanewline
\ \ \ \ \isacommand{then}\isamarkupfalse%
\ \isacommand{show}\isamarkupfalse%
\ {\isachardoublequoteopen}{\isacharparenleft}{\kern0pt}k\ {\isasymcirc}\isactrlsub c\ equiv{\isacharunderscore}{\kern0pt}class\ {\isacharparenleft}{\kern0pt}R{\isacharcomma}{\kern0pt}\ m{\isacharparenright}{\kern0pt}{\isacharparenright}{\kern0pt}\ {\isasymcirc}\isactrlsub c\ a\ {\isacharequal}{\kern0pt}\ {\isacharparenleft}{\kern0pt}k\ {\isasymcirc}\isactrlsub c\ equiv{\isacharunderscore}{\kern0pt}class\ {\isacharparenleft}{\kern0pt}R{\isacharcomma}{\kern0pt}\ m{\isacharparenright}{\kern0pt}{\isacharparenright}{\kern0pt}\ {\isasymcirc}\isactrlsub c\ b{\isachardoublequoteclose}\isanewline
\ \ \ \ \ \ \isacommand{using}\isamarkupfalse%
\ a{\isacharunderscore}{\kern0pt}type\ b{\isacharunderscore}{\kern0pt}type\ left{\isacharunderscore}{\kern0pt}cart{\isacharunderscore}{\kern0pt}proj{\isacharunderscore}{\kern0pt}cfunc{\isacharunderscore}{\kern0pt}prod\ right{\isacharunderscore}{\kern0pt}cart{\isacharunderscore}{\kern0pt}proj{\isacharunderscore}{\kern0pt}cfunc{\isacharunderscore}{\kern0pt}prod\ \isacommand{by}\isamarkupfalse%
\ auto\isanewline
\ \ \isacommand{qed}\isamarkupfalse%
\isanewline
\isanewline
\ \ \isacommand{have}\isamarkupfalse%
\ k{\isacharunderscore}{\kern0pt}eq{\isacharcolon}{\kern0pt}\ {\isachardoublequoteopen}k\ {\isacharequal}{\kern0pt}\ quotient{\isacharunderscore}{\kern0pt}func\ {\isacharparenleft}{\kern0pt}y\ {\isasymcirc}\isactrlsub c\ equiv{\isacharunderscore}{\kern0pt}class\ {\isacharparenleft}{\kern0pt}R{\isacharcomma}{\kern0pt}\ m{\isacharparenright}{\kern0pt}{\isacharparenright}{\kern0pt}\ {\isacharparenleft}{\kern0pt}R{\isacharcomma}{\kern0pt}\ m{\isacharparenright}{\kern0pt}{\isachardoublequoteclose}\isanewline
\ \ \ \ \isacommand{using}\isamarkupfalse%
\ assms\ k{\isacharunderscore}{\kern0pt}type\ k{\isacharunderscore}{\kern0pt}equiv{\isacharunderscore}{\kern0pt}class{\isacharunderscore}{\kern0pt}type\ y{\isacharunderscore}{\kern0pt}k{\isacharunderscore}{\kern0pt}eq\isanewline
\ \ \isacommand{proof}\isamarkupfalse%
\ {\isacharparenleft}{\kern0pt}rule{\isacharunderscore}{\kern0pt}tac\ quotient{\isacharunderscore}{\kern0pt}func{\isacharunderscore}{\kern0pt}unique{\isacharbrackleft}{\kern0pt}\isakeyword{where}\ X{\isacharequal}{\kern0pt}X{\isacharcomma}{\kern0pt}\ \isakeyword{where}\ Y{\isacharequal}{\kern0pt}F{\isacharbrackright}{\kern0pt}{\isacharcomma}{\kern0pt}\ simp{\isacharunderscore}{\kern0pt}all{\isacharcomma}{\kern0pt}\ unfold\ const{\isacharunderscore}{\kern0pt}on{\isacharunderscore}{\kern0pt}rel{\isacharunderscore}{\kern0pt}def{\isacharcomma}{\kern0pt}\ auto{\isacharparenright}{\kern0pt}\isanewline
\ \ \ \ \isacommand{fix}\isamarkupfalse%
\ a\ b\isanewline
\ \ \ \ \isacommand{assume}\isamarkupfalse%
\ a{\isacharunderscore}{\kern0pt}type{\isacharcolon}{\kern0pt}\ {\isachardoublequoteopen}a\ {\isasymin}\isactrlsub c\ X{\isachardoublequoteclose}\ \isakeyword{and}\ b{\isacharunderscore}{\kern0pt}type{\isacharcolon}{\kern0pt}\ {\isachardoublequoteopen}b\ {\isasymin}\isactrlsub c\ X{\isachardoublequoteclose}\isanewline
\ \ \ \ \isacommand{assume}\isamarkupfalse%
\ ab{\isacharunderscore}{\kern0pt}in{\isacharunderscore}{\kern0pt}R{\isacharcolon}{\kern0pt}\ {\isachardoublequoteopen}{\isasymlangle}a{\isacharcomma}{\kern0pt}b{\isasymrangle}\ {\isasymin}\isactrlbsub X\ {\isasymtimes}\isactrlsub c\ X\isactrlesub \ {\isacharparenleft}{\kern0pt}R{\isacharcomma}{\kern0pt}\ m{\isacharparenright}{\kern0pt}{\isachardoublequoteclose}\isanewline
\ \ \ \ \isacommand{then}\isamarkupfalse%
\ \isacommand{obtain}\isamarkupfalse%
\ h\ \isakeyword{where}\ h{\isacharunderscore}{\kern0pt}type{\isacharcolon}{\kern0pt}\ {\isachardoublequoteopen}h\ {\isasymin}\isactrlsub c\ R{\isachardoublequoteclose}\ \isakeyword{and}\ m{\isacharunderscore}{\kern0pt}h{\isacharunderscore}{\kern0pt}eq{\isacharcolon}{\kern0pt}\ {\isachardoublequoteopen}m\ {\isasymcirc}\isactrlsub c\ h\ {\isacharequal}{\kern0pt}\ {\isasymlangle}a{\isacharcomma}{\kern0pt}\ b{\isasymrangle}{\isachardoublequoteclose}\isanewline
\ \ \ \ \ \ \isacommand{unfolding}\isamarkupfalse%
\ relative{\isacharunderscore}{\kern0pt}member{\isacharunderscore}{\kern0pt}def\ factors{\isacharunderscore}{\kern0pt}through{\isacharunderscore}{\kern0pt}def\ \isacommand{using}\isamarkupfalse%
\ cfunc{\isacharunderscore}{\kern0pt}type{\isacharunderscore}{\kern0pt}def\ \isacommand{by}\isamarkupfalse%
\ auto\ \isanewline
\ \ \ \ \isanewline
\ \ \ \ \isacommand{have}\isamarkupfalse%
\ {\isachardoublequoteopen}{\isacharparenleft}{\kern0pt}k\ {\isasymcirc}\isactrlsub c\ equiv{\isacharunderscore}{\kern0pt}class\ {\isacharparenleft}{\kern0pt}R{\isacharcomma}{\kern0pt}\ m{\isacharparenright}{\kern0pt}{\isacharparenright}{\kern0pt}\ {\isasymcirc}\isactrlsub c\ left{\isacharunderscore}{\kern0pt}cart{\isacharunderscore}{\kern0pt}proj\ X\ X\ {\isasymcirc}\isactrlsub c\ m\ {\isasymcirc}\isactrlsub c\ h\ {\isacharequal}{\kern0pt}\isanewline
\ \ \ \ \ \ \ {\isacharparenleft}{\kern0pt}k\ {\isasymcirc}\isactrlsub c\ equiv{\isacharunderscore}{\kern0pt}class\ {\isacharparenleft}{\kern0pt}R{\isacharcomma}{\kern0pt}\ m{\isacharparenright}{\kern0pt}{\isacharparenright}{\kern0pt}\ {\isasymcirc}\isactrlsub c\ right{\isacharunderscore}{\kern0pt}cart{\isacharunderscore}{\kern0pt}proj\ X\ X\ {\isasymcirc}\isactrlsub c\ m\ {\isasymcirc}\isactrlsub c\ h{\isachardoublequoteclose}\isanewline
\ \ \ \ \ \ \isacommand{using}\isamarkupfalse%
\ k{\isacharunderscore}{\kern0pt}type\ m{\isacharunderscore}{\kern0pt}type\ h{\isacharunderscore}{\kern0pt}type\ assms\ \isanewline
\ \ \ \ \ \ \isacommand{by}\isamarkupfalse%
\ {\isacharparenleft}{\kern0pt}typecheck{\isacharunderscore}{\kern0pt}cfuncs{\isacharcomma}{\kern0pt}\ smt\ comp{\isacharunderscore}{\kern0pt}associative{\isadigit{2}}\ comp{\isacharunderscore}{\kern0pt}type\ k{\isacharunderscore}{\kern0pt}equiv{\isacharunderscore}{\kern0pt}class{\isacharunderscore}{\kern0pt}eq{\isacharparenright}{\kern0pt}\isanewline
\ \ \ \ \isacommand{then}\isamarkupfalse%
\ \isacommand{have}\isamarkupfalse%
\ {\isachardoublequoteopen}{\isacharparenleft}{\kern0pt}k\ {\isasymcirc}\isactrlsub c\ equiv{\isacharunderscore}{\kern0pt}class\ {\isacharparenleft}{\kern0pt}R{\isacharcomma}{\kern0pt}\ m{\isacharparenright}{\kern0pt}{\isacharparenright}{\kern0pt}\ {\isasymcirc}\isactrlsub c\ left{\isacharunderscore}{\kern0pt}cart{\isacharunderscore}{\kern0pt}proj\ X\ X\ {\isasymcirc}\isactrlsub c\ {\isasymlangle}a{\isacharcomma}{\kern0pt}\ b{\isasymrangle}\ {\isacharequal}{\kern0pt}\isanewline
\ \ \ \ \ \ \ {\isacharparenleft}{\kern0pt}k\ {\isasymcirc}\isactrlsub c\ equiv{\isacharunderscore}{\kern0pt}class\ {\isacharparenleft}{\kern0pt}R{\isacharcomma}{\kern0pt}\ m{\isacharparenright}{\kern0pt}{\isacharparenright}{\kern0pt}\ {\isasymcirc}\isactrlsub c\ right{\isacharunderscore}{\kern0pt}cart{\isacharunderscore}{\kern0pt}proj\ X\ X\ {\isasymcirc}\isactrlsub c\ {\isasymlangle}a{\isacharcomma}{\kern0pt}\ b{\isasymrangle}{\isachardoublequoteclose}\isanewline
\ \ \ \ \ \ \isacommand{by}\isamarkupfalse%
\ {\isacharparenleft}{\kern0pt}simp\ add{\isacharcolon}{\kern0pt}\ m{\isacharunderscore}{\kern0pt}h{\isacharunderscore}{\kern0pt}eq{\isacharparenright}{\kern0pt}\isanewline
\ \ \ \ \isacommand{then}\isamarkupfalse%
\ \isacommand{show}\isamarkupfalse%
\ {\isachardoublequoteopen}{\isacharparenleft}{\kern0pt}k\ {\isasymcirc}\isactrlsub c\ equiv{\isacharunderscore}{\kern0pt}class\ {\isacharparenleft}{\kern0pt}R{\isacharcomma}{\kern0pt}\ m{\isacharparenright}{\kern0pt}{\isacharparenright}{\kern0pt}\ {\isasymcirc}\isactrlsub c\ a\ {\isacharequal}{\kern0pt}\ {\isacharparenleft}{\kern0pt}k\ {\isasymcirc}\isactrlsub c\ equiv{\isacharunderscore}{\kern0pt}class\ {\isacharparenleft}{\kern0pt}R{\isacharcomma}{\kern0pt}\ m{\isacharparenright}{\kern0pt}{\isacharparenright}{\kern0pt}\ {\isasymcirc}\isactrlsub c\ b{\isachardoublequoteclose}\isanewline
\ \ \ \ \ \ \isacommand{using}\isamarkupfalse%
\ a{\isacharunderscore}{\kern0pt}type\ b{\isacharunderscore}{\kern0pt}type\ left{\isacharunderscore}{\kern0pt}cart{\isacharunderscore}{\kern0pt}proj{\isacharunderscore}{\kern0pt}cfunc{\isacharunderscore}{\kern0pt}prod\ right{\isacharunderscore}{\kern0pt}cart{\isacharunderscore}{\kern0pt}proj{\isacharunderscore}{\kern0pt}cfunc{\isacharunderscore}{\kern0pt}prod\ \isacommand{by}\isamarkupfalse%
\ auto\isanewline
\ \ \isacommand{qed}\isamarkupfalse%
\isanewline
\ \ \isacommand{show}\isamarkupfalse%
\ {\isachardoublequoteopen}k\ {\isacharequal}{\kern0pt}\ y{\isachardoublequoteclose}\isanewline
\ \ \ \ \isacommand{using}\isamarkupfalse%
\ y{\isacharunderscore}{\kern0pt}eq\ k{\isacharunderscore}{\kern0pt}eq\ \isacommand{by}\isamarkupfalse%
\ auto\isanewline
\isacommand{qed}\isamarkupfalse%
%
\endisatagproof
{\isafoldproof}%
%
\isadelimproof
\isanewline
%
\endisadelimproof
\isanewline
\isacommand{lemma}\isamarkupfalse%
\ canonical{\isacharunderscore}{\kern0pt}quot{\isacharunderscore}{\kern0pt}map{\isacharunderscore}{\kern0pt}is{\isacharunderscore}{\kern0pt}epi{\isacharcolon}{\kern0pt}\isanewline
\ \ \isakeyword{assumes}\ {\isachardoublequoteopen}equiv{\isacharunderscore}{\kern0pt}rel{\isacharunderscore}{\kern0pt}on\ X\ {\isacharparenleft}{\kern0pt}R{\isacharcomma}{\kern0pt}m{\isacharparenright}{\kern0pt}{\isachardoublequoteclose}\isanewline
\ \ \isakeyword{shows}\ {\isachardoublequoteopen}epimorphism{\isacharparenleft}{\kern0pt}{\isacharparenleft}{\kern0pt}equiv{\isacharunderscore}{\kern0pt}class\ {\isacharparenleft}{\kern0pt}R{\isacharcomma}{\kern0pt}m{\isacharparenright}{\kern0pt}{\isacharparenright}{\kern0pt}{\isacharparenright}{\kern0pt}{\isachardoublequoteclose}\isanewline
%
\isadelimproof
\ \ %
\endisadelimproof
%
\isatagproof
\isacommand{by}\isamarkupfalse%
\ {\isacharparenleft}{\kern0pt}meson\ assms\ canonical{\isacharunderscore}{\kern0pt}quotient{\isacharunderscore}{\kern0pt}map{\isacharunderscore}{\kern0pt}is{\isacharunderscore}{\kern0pt}coequalizer\ coequalizer{\isacharunderscore}{\kern0pt}is{\isacharunderscore}{\kern0pt}epimorphism{\isacharparenright}{\kern0pt}%
\endisatagproof
{\isafoldproof}%
%
\isadelimproof
%
\endisadelimproof
%
\isadelimdocument
%
\endisadelimdocument
%
\isatagdocument
%
\isamarkupsubsection{Regular Epimorphisms%
}
\isamarkuptrue%
%
\endisatagdocument
{\isafolddocument}%
%
\isadelimdocument
%
\endisadelimdocument
%
\begin{isamarkuptext}%
The definition below corresponds to Definition 2.3.4 in Halvorson.%
\end{isamarkuptext}\isamarkuptrue%
\isacommand{definition}\isamarkupfalse%
\ regular{\isacharunderscore}{\kern0pt}epimorphism\ {\isacharcolon}{\kern0pt}{\isacharcolon}{\kern0pt}\ {\isachardoublequoteopen}cfunc\ {\isasymRightarrow}\ bool{\isachardoublequoteclose}\ \isakeyword{where}\isanewline
\ \ {\isachardoublequoteopen}regular{\isacharunderscore}{\kern0pt}epimorphism\ f\ {\isacharequal}{\kern0pt}\ {\isacharparenleft}{\kern0pt}{\isasymexists}\ g\ h{\isachardot}{\kern0pt}\ coequalizer\ {\isacharparenleft}{\kern0pt}codomain\ f{\isacharparenright}{\kern0pt}\ f\ g\ h{\isacharparenright}{\kern0pt}{\isachardoublequoteclose}%
\begin{isamarkuptext}%
The lemma below corresponds to Exercise 2.3.5 in Halvorson.%
\end{isamarkuptext}\isamarkuptrue%
\isacommand{lemma}\isamarkupfalse%
\ reg{\isacharunderscore}{\kern0pt}epi{\isacharunderscore}{\kern0pt}and{\isacharunderscore}{\kern0pt}mono{\isacharunderscore}{\kern0pt}is{\isacharunderscore}{\kern0pt}iso{\isacharcolon}{\kern0pt}\isanewline
\ \ \isakeyword{assumes}\ {\isachardoublequoteopen}f\ {\isacharcolon}{\kern0pt}\ X\ {\isasymrightarrow}\ Y{\isachardoublequoteclose}\ {\isachardoublequoteopen}regular{\isacharunderscore}{\kern0pt}epimorphism\ f{\isachardoublequoteclose}\ {\isachardoublequoteopen}monomorphism\ f{\isachardoublequoteclose}\isanewline
\ \ \isakeyword{shows}\ {\isachardoublequoteopen}isomorphism\ f{\isachardoublequoteclose}\isanewline
%
\isadelimproof
%
\endisadelimproof
%
\isatagproof
\isacommand{proof}\isamarkupfalse%
\ {\isacharminus}{\kern0pt}\ \ \ \isanewline
\ \ \isacommand{obtain}\isamarkupfalse%
\ g\ h\ \isakeyword{where}\ gh{\isacharunderscore}{\kern0pt}def{\isacharcolon}{\kern0pt}\ {\isachardoublequoteopen}coequalizer\ {\isacharparenleft}{\kern0pt}codomain\ f{\isacharparenright}{\kern0pt}\ f\ g\ h{\isachardoublequoteclose}\isanewline
\ \ \ \ \isacommand{using}\isamarkupfalse%
\ assms{\isacharparenleft}{\kern0pt}{\isadigit{2}}{\isacharparenright}{\kern0pt}\ regular{\isacharunderscore}{\kern0pt}epimorphism{\isacharunderscore}{\kern0pt}def\ \isacommand{by}\isamarkupfalse%
\ auto\isanewline
\ \ \isacommand{obtain}\isamarkupfalse%
\ W\ \isakeyword{where}\ W{\isacharunderscore}{\kern0pt}def{\isacharcolon}{\kern0pt}\ {\isachardoublequoteopen}{\isacharparenleft}{\kern0pt}g{\isacharcolon}{\kern0pt}\ W\ {\isasymrightarrow}\ X{\isacharparenright}{\kern0pt}\ {\isasymand}\ {\isacharparenleft}{\kern0pt}h{\isacharcolon}{\kern0pt}\ W\ {\isasymrightarrow}\ X{\isacharparenright}{\kern0pt}\ {\isasymand}\ {\isacharparenleft}{\kern0pt}coequalizer\ Y\ f\ g\ h{\isacharparenright}{\kern0pt}{\isachardoublequoteclose}\isanewline
\ \ \ \ \isacommand{using}\isamarkupfalse%
\ assms{\isacharparenleft}{\kern0pt}{\isadigit{1}}{\isacharparenright}{\kern0pt}\ cfunc{\isacharunderscore}{\kern0pt}type{\isacharunderscore}{\kern0pt}def\ coequalizer{\isacharunderscore}{\kern0pt}def\ gh{\isacharunderscore}{\kern0pt}def\ \isacommand{by}\isamarkupfalse%
\ fastforce\isanewline
\ \ \isacommand{have}\isamarkupfalse%
\ fg{\isacharunderscore}{\kern0pt}eqs{\isacharunderscore}{\kern0pt}fh{\isacharcolon}{\kern0pt}\ {\isachardoublequoteopen}f\ {\isasymcirc}\isactrlsub c\ g\ {\isacharequal}{\kern0pt}\ f\ {\isasymcirc}\isactrlsub c\ h{\isachardoublequoteclose}\isanewline
\ \ \ \ \isacommand{using}\isamarkupfalse%
\ coequalizer{\isacharunderscore}{\kern0pt}def\ gh{\isacharunderscore}{\kern0pt}def\ \isacommand{by}\isamarkupfalse%
\ blast\ \ \ \ \isanewline
\ \ \isacommand{then}\isamarkupfalse%
\ \isacommand{have}\isamarkupfalse%
\ {\isachardoublequoteopen}id{\isacharparenleft}{\kern0pt}X{\isacharparenright}{\kern0pt}{\isasymcirc}\isactrlsub c\ g\ {\isacharequal}{\kern0pt}\ id{\isacharparenleft}{\kern0pt}X{\isacharparenright}{\kern0pt}\ {\isasymcirc}\isactrlsub c\ \ h{\isachardoublequoteclose}\isanewline
\ \ \ \ \isacommand{using}\isamarkupfalse%
\ W{\isacharunderscore}{\kern0pt}def\ assms{\isacharparenleft}{\kern0pt}{\isadigit{1}}{\isacharcomma}{\kern0pt}{\isadigit{3}}{\isacharparenright}{\kern0pt}\ monomorphism{\isacharunderscore}{\kern0pt}def{\isadigit{2}}\ \isacommand{by}\isamarkupfalse%
\ blast\ \ \ \ \ \isanewline
\ \ \isacommand{then}\isamarkupfalse%
\ \isacommand{obtain}\isamarkupfalse%
\ j\ \isakeyword{where}\ j{\isacharunderscore}{\kern0pt}def{\isacharcolon}{\kern0pt}\ {\isachardoublequoteopen}j{\isacharcolon}{\kern0pt}\ Y\ {\isasymrightarrow}\ X\ {\isasymand}\ j\ {\isasymcirc}\isactrlsub c\ f\ {\isacharequal}{\kern0pt}\ \ id{\isacharparenleft}{\kern0pt}X{\isacharparenright}{\kern0pt}{\isachardoublequoteclose}\isanewline
\ \ \ \ \isacommand{using}\isamarkupfalse%
\ assms{\isacharparenleft}{\kern0pt}{\isadigit{1}}{\isacharparenright}{\kern0pt}\ \ W{\isacharunderscore}{\kern0pt}def\ \ coequalizer{\isacharunderscore}{\kern0pt}def{\isadigit{2}}\ \isacommand{by}\isamarkupfalse%
\ {\isacharparenleft}{\kern0pt}typecheck{\isacharunderscore}{\kern0pt}cfuncs{\isacharcomma}{\kern0pt}\ blast{\isacharparenright}{\kern0pt}\isanewline
\ \ \isacommand{have}\isamarkupfalse%
\ {\isachardoublequoteopen}id{\isacharparenleft}{\kern0pt}Y{\isacharparenright}{\kern0pt}\ {\isasymcirc}\isactrlsub c\ f\ {\isacharequal}{\kern0pt}\ f\ {\isasymcirc}\isactrlsub c\ id{\isacharparenleft}{\kern0pt}X{\isacharparenright}{\kern0pt}{\isachardoublequoteclose}\isanewline
\ \ \ \ \isacommand{using}\isamarkupfalse%
\ assms{\isacharparenleft}{\kern0pt}{\isadigit{1}}{\isacharparenright}{\kern0pt}\ id{\isacharunderscore}{\kern0pt}left{\isacharunderscore}{\kern0pt}unit{\isadigit{2}}\ id{\isacharunderscore}{\kern0pt}right{\isacharunderscore}{\kern0pt}unit{\isadigit{2}}\ \isacommand{by}\isamarkupfalse%
\ auto\isanewline
\ \ \isacommand{also}\isamarkupfalse%
\ \isacommand{have}\isamarkupfalse%
\ {\isachardoublequoteopen}{\isachardot}{\kern0pt}{\isachardot}{\kern0pt}{\isachardot}{\kern0pt}\ {\isacharequal}{\kern0pt}\ {\isacharparenleft}{\kern0pt}f\ {\isasymcirc}\isactrlsub c\ j{\isacharparenright}{\kern0pt}\ {\isasymcirc}\isactrlsub c\ f{\isachardoublequoteclose}\isanewline
\ \ \ \ \ \isacommand{using}\isamarkupfalse%
\ assms{\isacharparenleft}{\kern0pt}{\isadigit{1}}{\isacharparenright}{\kern0pt}\ comp{\isacharunderscore}{\kern0pt}associative{\isadigit{2}}\ j{\isacharunderscore}{\kern0pt}def\ \isacommand{by}\isamarkupfalse%
\ fastforce\isanewline
\ \ \isacommand{then}\isamarkupfalse%
\ \isacommand{have}\isamarkupfalse%
\ {\isachardoublequoteopen}id{\isacharparenleft}{\kern0pt}Y{\isacharparenright}{\kern0pt}\ {\isacharequal}{\kern0pt}\ f\ {\isasymcirc}\isactrlsub c\ j{\isachardoublequoteclose}\isanewline
\ \ \ \ \isacommand{by}\isamarkupfalse%
\ {\isacharparenleft}{\kern0pt}typecheck{\isacharunderscore}{\kern0pt}cfuncs{\isacharcomma}{\kern0pt}\ metis\ W{\isacharunderscore}{\kern0pt}def\ assms{\isacharparenleft}{\kern0pt}{\isadigit{1}}{\isacharparenright}{\kern0pt}\ calculation\ coequalizer{\isacharunderscore}{\kern0pt}is{\isacharunderscore}{\kern0pt}epimorphism\ epimorphism{\isacharunderscore}{\kern0pt}def{\isadigit{3}}\ j{\isacharunderscore}{\kern0pt}def{\isacharparenright}{\kern0pt}\isanewline
\ \ \isacommand{then}\isamarkupfalse%
\ \isacommand{show}\isamarkupfalse%
\ {\isachardoublequoteopen}isomorphism\ f{\isachardoublequoteclose}\isanewline
\ \ \ \ \isacommand{using}\isamarkupfalse%
\ \ assms{\isacharparenleft}{\kern0pt}{\isadigit{1}}{\isacharparenright}{\kern0pt}\ cfunc{\isacharunderscore}{\kern0pt}type{\isacharunderscore}{\kern0pt}def\ isomorphism{\isacharunderscore}{\kern0pt}def\ j{\isacharunderscore}{\kern0pt}def\ \isacommand{by}\isamarkupfalse%
\ fastforce\ \ \isanewline
\isacommand{qed}\isamarkupfalse%
%
\endisatagproof
{\isafoldproof}%
%
\isadelimproof
%
\endisadelimproof
%
\begin{isamarkuptext}%
The two lemmas below correspond to Proposition 2.3.6 in Halvorson.%
\end{isamarkuptext}\isamarkuptrue%
\isacommand{lemma}\isamarkupfalse%
\ epimorphism{\isacharunderscore}{\kern0pt}coequalizer{\isacharunderscore}{\kern0pt}kernel{\isacharunderscore}{\kern0pt}pair{\isacharcolon}{\kern0pt}\isanewline
\ \ \isakeyword{assumes}\ {\isachardoublequoteopen}f\ {\isacharcolon}{\kern0pt}\ X\ {\isasymrightarrow}\ Y{\isachardoublequoteclose}\ {\isachardoublequoteopen}epimorphism\ f{\isachardoublequoteclose}\isanewline
\ \ \isakeyword{shows}\ {\isachardoublequoteopen}coequalizer\ Y\ f\ {\isacharparenleft}{\kern0pt}fibered{\isacharunderscore}{\kern0pt}product{\isacharunderscore}{\kern0pt}left{\isacharunderscore}{\kern0pt}proj\ X\ f\ f\ X{\isacharparenright}{\kern0pt}\ {\isacharparenleft}{\kern0pt}fibered{\isacharunderscore}{\kern0pt}product{\isacharunderscore}{\kern0pt}right{\isacharunderscore}{\kern0pt}proj\ X\ f\ f\ X{\isacharparenright}{\kern0pt}{\isachardoublequoteclose}\isanewline
%
\isadelimproof
%
\endisadelimproof
%
\isatagproof
\isacommand{proof}\isamarkupfalse%
\ {\isacharparenleft}{\kern0pt}unfold\ coequalizer{\isacharunderscore}{\kern0pt}def{\isacharcomma}{\kern0pt}\ rule{\isacharunderscore}{\kern0pt}tac\ x{\isacharequal}{\kern0pt}X\ \isakeyword{in}\ exI{\isacharcomma}{\kern0pt}\ rule{\isacharunderscore}{\kern0pt}tac\ x{\isacharequal}{\kern0pt}{\isachardoublequoteopen}X\ \isactrlbsub f\isactrlesub {\isasymtimes}\isactrlsub c\isactrlbsub f\isactrlesub \ X{\isachardoublequoteclose}\ \isakeyword{in}\ exI{\isacharcomma}{\kern0pt}\ auto{\isacharparenright}{\kern0pt}\isanewline
\ \ \isacommand{show}\isamarkupfalse%
\ {\isachardoublequoteopen}fibered{\isacharunderscore}{\kern0pt}product{\isacharunderscore}{\kern0pt}left{\isacharunderscore}{\kern0pt}proj\ X\ f\ f\ X\ {\isacharcolon}{\kern0pt}\ X\ \isactrlbsub f\isactrlesub {\isasymtimes}\isactrlsub c\isactrlbsub f\isactrlesub \ X\ {\isasymrightarrow}\ X{\isachardoublequoteclose}\isanewline
\ \ \ \ \isacommand{using}\isamarkupfalse%
\ assms\ \isacommand{by}\isamarkupfalse%
\ typecheck{\isacharunderscore}{\kern0pt}cfuncs\isanewline
\ \ \isacommand{show}\isamarkupfalse%
\ {\isachardoublequoteopen}fibered{\isacharunderscore}{\kern0pt}product{\isacharunderscore}{\kern0pt}right{\isacharunderscore}{\kern0pt}proj\ X\ f\ f\ X\ {\isacharcolon}{\kern0pt}\ X\ \isactrlbsub f\isactrlesub {\isasymtimes}\isactrlsub c\isactrlbsub f\isactrlesub \ X\ {\isasymrightarrow}\ X{\isachardoublequoteclose}\isanewline
\ \ \ \ \isacommand{using}\isamarkupfalse%
\ assms\ \isacommand{by}\isamarkupfalse%
\ typecheck{\isacharunderscore}{\kern0pt}cfuncs\isanewline
\ \ \isacommand{show}\isamarkupfalse%
\ {\isachardoublequoteopen}f\ {\isacharcolon}{\kern0pt}\ X\ {\isasymrightarrow}Y{\isachardoublequoteclose}\isanewline
\ \ \ \ \isacommand{using}\isamarkupfalse%
\ assms\ \isacommand{by}\isamarkupfalse%
\ typecheck{\isacharunderscore}{\kern0pt}cfuncs\isanewline
\ \ \isacommand{show}\isamarkupfalse%
\ {\isachardoublequoteopen}f\ {\isasymcirc}\isactrlsub c\ fibered{\isacharunderscore}{\kern0pt}product{\isacharunderscore}{\kern0pt}left{\isacharunderscore}{\kern0pt}proj\ X\ f\ f\ X\ {\isacharequal}{\kern0pt}\ f\ {\isasymcirc}\isactrlsub c\ fibered{\isacharunderscore}{\kern0pt}product{\isacharunderscore}{\kern0pt}right{\isacharunderscore}{\kern0pt}proj\ X\ f\ f\ X{\isachardoublequoteclose}\isanewline
\ \ \ \ \isacommand{using}\isamarkupfalse%
\ fibered{\isacharunderscore}{\kern0pt}product{\isacharunderscore}{\kern0pt}is{\isacharunderscore}{\kern0pt}pullback\ assms\ \isacommand{unfolding}\isamarkupfalse%
\ is{\isacharunderscore}{\kern0pt}pullback{\isacharunderscore}{\kern0pt}def\ \isacommand{by}\isamarkupfalse%
\ auto\isanewline
\isacommand{next}\isamarkupfalse%
\isanewline
\ \ \isacommand{fix}\isamarkupfalse%
\ g\ E\isanewline
\ \ \isacommand{assume}\isamarkupfalse%
\ g{\isacharunderscore}{\kern0pt}type{\isacharcolon}{\kern0pt}\ {\isachardoublequoteopen}g\ {\isacharcolon}{\kern0pt}\ X\ {\isasymrightarrow}\ E{\isachardoublequoteclose}\isanewline
\ \ \isacommand{assume}\isamarkupfalse%
\ g{\isacharunderscore}{\kern0pt}eq{\isacharcolon}{\kern0pt}\ {\isachardoublequoteopen}g\ {\isasymcirc}\isactrlsub c\ fibered{\isacharunderscore}{\kern0pt}product{\isacharunderscore}{\kern0pt}left{\isacharunderscore}{\kern0pt}proj\ X\ f\ f\ X\ {\isacharequal}{\kern0pt}\ g\ {\isasymcirc}\isactrlsub c\ fibered{\isacharunderscore}{\kern0pt}product{\isacharunderscore}{\kern0pt}right{\isacharunderscore}{\kern0pt}proj\ X\ f\ f\ X{\isachardoublequoteclose}\isanewline
\isanewline
\ \ \isacommand{obtain}\isamarkupfalse%
\ F\ \isakeyword{where}\ F{\isacharunderscore}{\kern0pt}def{\isacharcolon}{\kern0pt}\ {\isachardoublequoteopen}F\ {\isacharequal}{\kern0pt}\ quotient{\isacharunderscore}{\kern0pt}set\ X\ {\isacharparenleft}{\kern0pt}X\ \isactrlbsub f\isactrlesub {\isasymtimes}\isactrlsub c\isactrlbsub f\isactrlesub \ X{\isacharcomma}{\kern0pt}\ fibered{\isacharunderscore}{\kern0pt}product{\isacharunderscore}{\kern0pt}morphism\ X\ f\ f\ X{\isacharparenright}{\kern0pt}{\isachardoublequoteclose}\isanewline
\ \ \ \ \isacommand{by}\isamarkupfalse%
\ auto\isanewline
\ \ \isacommand{obtain}\isamarkupfalse%
\ q\ \isakeyword{where}\ q{\isacharunderscore}{\kern0pt}def{\isacharcolon}{\kern0pt}\ {\isachardoublequoteopen}q\ {\isacharequal}{\kern0pt}\ equiv{\isacharunderscore}{\kern0pt}class\ {\isacharparenleft}{\kern0pt}X\ \isactrlbsub f\isactrlesub {\isasymtimes}\isactrlsub c\isactrlbsub f\isactrlesub \ X{\isacharcomma}{\kern0pt}\ fibered{\isacharunderscore}{\kern0pt}product{\isacharunderscore}{\kern0pt}morphism\ X\ f\ f\ X{\isacharparenright}{\kern0pt}{\isachardoublequoteclose}\isanewline
\ \ \ \ \isacommand{by}\isamarkupfalse%
\ auto\isanewline
\ \ \isacommand{have}\isamarkupfalse%
\ q{\isacharunderscore}{\kern0pt}type{\isacharbrackleft}{\kern0pt}type{\isacharunderscore}{\kern0pt}rule{\isacharbrackright}{\kern0pt}{\isacharcolon}{\kern0pt}\ {\isachardoublequoteopen}q\ {\isacharcolon}{\kern0pt}\ X\ {\isasymrightarrow}\ F{\isachardoublequoteclose}\isanewline
\ \ \ \ \isacommand{using}\isamarkupfalse%
\ F{\isacharunderscore}{\kern0pt}def\ assms{\isacharparenleft}{\kern0pt}{\isadigit{1}}{\isacharparenright}{\kern0pt}\ equiv{\isacharunderscore}{\kern0pt}class{\isacharunderscore}{\kern0pt}type\ kernel{\isacharunderscore}{\kern0pt}pair{\isacharunderscore}{\kern0pt}equiv{\isacharunderscore}{\kern0pt}rel\ q{\isacharunderscore}{\kern0pt}def\ \isacommand{by}\isamarkupfalse%
\ blast\isanewline
\ \ \ \ \isanewline
\ \ \isacommand{obtain}\isamarkupfalse%
\ f{\isacharunderscore}{\kern0pt}bar\ \isakeyword{where}\ f{\isacharunderscore}{\kern0pt}bar{\isacharunderscore}{\kern0pt}def{\isacharcolon}{\kern0pt}\ {\isachardoublequoteopen}f{\isacharunderscore}{\kern0pt}bar\ {\isacharequal}{\kern0pt}\ quotient{\isacharunderscore}{\kern0pt}func\ f\ {\isacharparenleft}{\kern0pt}X\ \isactrlbsub f\isactrlesub {\isasymtimes}\isactrlsub c\isactrlbsub f\isactrlesub \ X{\isacharcomma}{\kern0pt}\ fibered{\isacharunderscore}{\kern0pt}product{\isacharunderscore}{\kern0pt}morphism\ X\ f\ f\ X{\isacharparenright}{\kern0pt}{\isachardoublequoteclose}\isanewline
\ \ \ \ \isacommand{by}\isamarkupfalse%
\ auto\isanewline
\ \ \isacommand{have}\isamarkupfalse%
\ f{\isacharunderscore}{\kern0pt}bar{\isacharunderscore}{\kern0pt}type{\isacharbrackleft}{\kern0pt}type{\isacharunderscore}{\kern0pt}rule{\isacharbrackright}{\kern0pt}{\isacharcolon}{\kern0pt}\ {\isachardoublequoteopen}f{\isacharunderscore}{\kern0pt}bar\ {\isacharcolon}{\kern0pt}\ F\ {\isasymrightarrow}\ Y{\isachardoublequoteclose}\ \isanewline
\ \ \ \ \ \ \isacommand{using}\isamarkupfalse%
\ F{\isacharunderscore}{\kern0pt}def\ assms{\isacharparenleft}{\kern0pt}{\isadigit{1}}{\isacharparenright}{\kern0pt}\ const{\isacharunderscore}{\kern0pt}on{\isacharunderscore}{\kern0pt}rel{\isacharunderscore}{\kern0pt}def\ f{\isacharunderscore}{\kern0pt}bar{\isacharunderscore}{\kern0pt}def\ fibered{\isacharunderscore}{\kern0pt}product{\isacharunderscore}{\kern0pt}pair{\isacharunderscore}{\kern0pt}member\ kernel{\isacharunderscore}{\kern0pt}pair{\isacharunderscore}{\kern0pt}equiv{\isacharunderscore}{\kern0pt}rel\ quotient{\isacharunderscore}{\kern0pt}func{\isacharunderscore}{\kern0pt}type\ \isacommand{by}\isamarkupfalse%
\ auto\isanewline
\ \ \isacommand{have}\isamarkupfalse%
\ fibr{\isacharunderscore}{\kern0pt}proj{\isacharunderscore}{\kern0pt}left{\isacharunderscore}{\kern0pt}type{\isacharbrackleft}{\kern0pt}type{\isacharunderscore}{\kern0pt}rule{\isacharbrackright}{\kern0pt}{\isacharcolon}{\kern0pt}\ {\isachardoublequoteopen}fibered{\isacharunderscore}{\kern0pt}product{\isacharunderscore}{\kern0pt}left{\isacharunderscore}{\kern0pt}proj\ F\ {\isacharparenleft}{\kern0pt}f{\isacharunderscore}{\kern0pt}bar{\isacharparenright}{\kern0pt}\ {\isacharparenleft}{\kern0pt}f{\isacharunderscore}{\kern0pt}bar{\isacharparenright}{\kern0pt}\ F\ {\isacharcolon}{\kern0pt}\ F\ \isactrlbsub {\isacharparenleft}{\kern0pt}f{\isacharunderscore}{\kern0pt}bar{\isacharparenright}{\kern0pt}\isactrlesub {\isasymtimes}\isactrlsub c\isactrlbsub {\isacharparenleft}{\kern0pt}f{\isacharunderscore}{\kern0pt}bar{\isacharparenright}{\kern0pt}\isactrlesub \ F\ {\isasymrightarrow}\ F{\isachardoublequoteclose}\isanewline
\ \ \ \ \isacommand{by}\isamarkupfalse%
\ typecheck{\isacharunderscore}{\kern0pt}cfuncs\isanewline
\ \ \isacommand{have}\isamarkupfalse%
\ fibr{\isacharunderscore}{\kern0pt}proj{\isacharunderscore}{\kern0pt}right{\isacharunderscore}{\kern0pt}type{\isacharbrackleft}{\kern0pt}type{\isacharunderscore}{\kern0pt}rule{\isacharbrackright}{\kern0pt}{\isacharcolon}{\kern0pt}\ {\isachardoublequoteopen}fibered{\isacharunderscore}{\kern0pt}product{\isacharunderscore}{\kern0pt}right{\isacharunderscore}{\kern0pt}proj\ F\ {\isacharparenleft}{\kern0pt}f{\isacharunderscore}{\kern0pt}bar{\isacharparenright}{\kern0pt}\ {\isacharparenleft}{\kern0pt}f{\isacharunderscore}{\kern0pt}bar{\isacharparenright}{\kern0pt}\ F\ {\isacharcolon}{\kern0pt}\ F\ \isactrlbsub {\isacharparenleft}{\kern0pt}f{\isacharunderscore}{\kern0pt}bar{\isacharparenright}{\kern0pt}\isactrlesub {\isasymtimes}\isactrlsub c\isactrlbsub {\isacharparenleft}{\kern0pt}f{\isacharunderscore}{\kern0pt}bar{\isacharparenright}{\kern0pt}\isactrlesub \ F\ {\isasymrightarrow}\ F{\isachardoublequoteclose}\isanewline
\ \ \ \ \isacommand{by}\isamarkupfalse%
\ typecheck{\isacharunderscore}{\kern0pt}cfuncs\isanewline
\ \ \isanewline
\ \ \isanewline
\ \ \ \ \isanewline
\ \ \ \ \isanewline
\ \ \ \ \isanewline
\ \ \ \ \isanewline
\ \ \ \ \isanewline
\ \ \ \ \isanewline
\ \ \ \ \isanewline
\ \ \isanewline
\ \ \isacommand{have}\isamarkupfalse%
\ f{\isacharunderscore}{\kern0pt}eqs{\isacharcolon}{\kern0pt}\ {\isachardoublequoteopen}f{\isacharunderscore}{\kern0pt}bar\ {\isasymcirc}\isactrlsub c\ q\ {\isacharequal}{\kern0pt}\ f{\isachardoublequoteclose}\isanewline
\ \ \ \ \isacommand{proof}\isamarkupfalse%
\ {\isacharminus}{\kern0pt}\ \isanewline
\ \ \ \ \ \ \isacommand{have}\isamarkupfalse%
\ fact{\isadigit{1}}{\isacharcolon}{\kern0pt}\ {\isachardoublequoteopen}equiv{\isacharunderscore}{\kern0pt}rel{\isacharunderscore}{\kern0pt}on\ X\ {\isacharparenleft}{\kern0pt}X\ \isactrlbsub f\isactrlesub {\isasymtimes}\isactrlsub c\isactrlbsub f\isactrlesub \ X{\isacharcomma}{\kern0pt}\ fibered{\isacharunderscore}{\kern0pt}product{\isacharunderscore}{\kern0pt}morphism\ X\ f\ f\ X{\isacharparenright}{\kern0pt}{\isachardoublequoteclose}\isanewline
\ \ \ \ \ \ \ \ \isacommand{by}\isamarkupfalse%
\ {\isacharparenleft}{\kern0pt}meson\ assms{\isacharparenleft}{\kern0pt}{\isadigit{1}}{\isacharparenright}{\kern0pt}\ kernel{\isacharunderscore}{\kern0pt}pair{\isacharunderscore}{\kern0pt}equiv{\isacharunderscore}{\kern0pt}rel{\isacharparenright}{\kern0pt}\isanewline
\isanewline
\ \ \ \ \ \ \isacommand{have}\isamarkupfalse%
\ fact{\isadigit{2}}{\isacharcolon}{\kern0pt}\ {\isachardoublequoteopen}const{\isacharunderscore}{\kern0pt}on{\isacharunderscore}{\kern0pt}rel\ X\ {\isacharparenleft}{\kern0pt}X\ \isactrlbsub f\isactrlesub {\isasymtimes}\isactrlsub c\isactrlbsub f\isactrlesub \ X{\isacharcomma}{\kern0pt}\ fibered{\isacharunderscore}{\kern0pt}product{\isacharunderscore}{\kern0pt}morphism\ X\ f\ f\ X{\isacharparenright}{\kern0pt}\ f{\isachardoublequoteclose}\isanewline
\ \ \ \ \ \ \ \ \isacommand{using}\isamarkupfalse%
\ assms{\isacharparenleft}{\kern0pt}{\isadigit{1}}{\isacharparenright}{\kern0pt}\ const{\isacharunderscore}{\kern0pt}on{\isacharunderscore}{\kern0pt}rel{\isacharunderscore}{\kern0pt}def\ fibered{\isacharunderscore}{\kern0pt}product{\isacharunderscore}{\kern0pt}pair{\isacharunderscore}{\kern0pt}member\ \isacommand{by}\isamarkupfalse%
\ presburger\isanewline
\ \ \ \ \ \ \isacommand{show}\isamarkupfalse%
\ {\isacharquery}{\kern0pt}thesis\isanewline
\ \ \ \ \ \ \ \ \isacommand{using}\isamarkupfalse%
\ assms{\isacharparenleft}{\kern0pt}{\isadigit{1}}{\isacharparenright}{\kern0pt}\ f{\isacharunderscore}{\kern0pt}bar{\isacharunderscore}{\kern0pt}def\ fact{\isadigit{1}}\ fact{\isadigit{2}}\ q{\isacharunderscore}{\kern0pt}def\ quotient{\isacharunderscore}{\kern0pt}func{\isacharunderscore}{\kern0pt}eq\ \isacommand{by}\isamarkupfalse%
\ blast\isanewline
\ \ \isacommand{qed}\isamarkupfalse%
\isanewline
\isanewline
\ \ \isacommand{have}\isamarkupfalse%
\ {\isachardoublequoteopen}{\isasymexists}{\isacharbang}{\kern0pt}\ b{\isachardot}{\kern0pt}\ b\ {\isacharcolon}{\kern0pt}\ X\ \isactrlbsub f\isactrlesub {\isasymtimes}\isactrlsub c\isactrlbsub f\isactrlesub \ X\ {\isasymrightarrow}\ F\ \isactrlbsub {\isacharparenleft}{\kern0pt}f{\isacharunderscore}{\kern0pt}bar{\isacharparenright}{\kern0pt}\isactrlesub {\isasymtimes}\isactrlsub c\isactrlbsub {\isacharparenleft}{\kern0pt}f{\isacharunderscore}{\kern0pt}bar{\isacharparenright}{\kern0pt}\isactrlesub \ F\ {\isasymand}\isanewline
\ \ \ \ fibered{\isacharunderscore}{\kern0pt}product{\isacharunderscore}{\kern0pt}left{\isacharunderscore}{\kern0pt}proj\ F\ {\isacharparenleft}{\kern0pt}f{\isacharunderscore}{\kern0pt}bar{\isacharparenright}{\kern0pt}\ {\isacharparenleft}{\kern0pt}f{\isacharunderscore}{\kern0pt}bar{\isacharparenright}{\kern0pt}\ F\ {\isasymcirc}\isactrlsub c\ b\ {\isacharequal}{\kern0pt}\ q\ {\isasymcirc}\isactrlsub c\ fibered{\isacharunderscore}{\kern0pt}product{\isacharunderscore}{\kern0pt}left{\isacharunderscore}{\kern0pt}proj\ X\ f\ f\ X\ {\isasymand}\isanewline
\ \ \ \ fibered{\isacharunderscore}{\kern0pt}product{\isacharunderscore}{\kern0pt}right{\isacharunderscore}{\kern0pt}proj\ F\ {\isacharparenleft}{\kern0pt}f{\isacharunderscore}{\kern0pt}bar{\isacharparenright}{\kern0pt}\ {\isacharparenleft}{\kern0pt}f{\isacharunderscore}{\kern0pt}bar{\isacharparenright}{\kern0pt}\ F\ {\isasymcirc}\isactrlsub c\ b\ {\isacharequal}{\kern0pt}\ q\ {\isasymcirc}\isactrlsub c\ fibered{\isacharunderscore}{\kern0pt}product{\isacharunderscore}{\kern0pt}right{\isacharunderscore}{\kern0pt}proj\ X\ f\ f\ X\ {\isasymand}\isanewline
\ \ \ \ epimorphism\ b{\isachardoublequoteclose}\isanewline
\ \ \isacommand{proof}\isamarkupfalse%
{\isacharparenleft}{\kern0pt}rule\ kernel{\isacharunderscore}{\kern0pt}pair{\isacharunderscore}{\kern0pt}connection{\isacharbrackleft}{\kern0pt}\isakeyword{where}\ Y\ {\isacharequal}{\kern0pt}\ Y{\isacharbrackright}{\kern0pt}{\isacharparenright}{\kern0pt}\isanewline
\ \ \ \ \isacommand{show}\isamarkupfalse%
\ {\isachardoublequoteopen}f\ {\isacharcolon}{\kern0pt}\ X\ {\isasymrightarrow}\ Y{\isachardoublequoteclose}\isanewline
\ \ \ \ \ \ \isacommand{using}\isamarkupfalse%
\ assms\ \isacommand{by}\isamarkupfalse%
\ typecheck{\isacharunderscore}{\kern0pt}cfuncs\isanewline
\ \ \ \ \isacommand{show}\isamarkupfalse%
\ {\isachardoublequoteopen}q\ {\isacharcolon}{\kern0pt}\ X\ {\isasymrightarrow}\ F{\isachardoublequoteclose}\isanewline
\ \ \ \ \ \ \isacommand{by}\isamarkupfalse%
\ typecheck{\isacharunderscore}{\kern0pt}cfuncs\isanewline
\ \ \ \ \isacommand{show}\isamarkupfalse%
\ {\isachardoublequoteopen}epimorphism\ q{\isachardoublequoteclose}\isanewline
\ \ \ \ \ \ \isacommand{using}\isamarkupfalse%
\ assms{\isacharparenleft}{\kern0pt}{\isadigit{1}}{\isacharparenright}{\kern0pt}\ canonical{\isacharunderscore}{\kern0pt}quot{\isacharunderscore}{\kern0pt}map{\isacharunderscore}{\kern0pt}is{\isacharunderscore}{\kern0pt}epi\ kernel{\isacharunderscore}{\kern0pt}pair{\isacharunderscore}{\kern0pt}equiv{\isacharunderscore}{\kern0pt}rel\ q{\isacharunderscore}{\kern0pt}def\ \isacommand{by}\isamarkupfalse%
\ blast\isanewline
\ \ \ \ \isacommand{show}\isamarkupfalse%
\ {\isachardoublequoteopen}f{\isacharunderscore}{\kern0pt}bar\ {\isasymcirc}\isactrlsub c\ q\ {\isacharequal}{\kern0pt}\ f{\isachardoublequoteclose}\isanewline
\ \ \ \ \ \ \isacommand{by}\isamarkupfalse%
\ {\isacharparenleft}{\kern0pt}simp\ add{\isacharcolon}{\kern0pt}\ f{\isacharunderscore}{\kern0pt}eqs{\isacharparenright}{\kern0pt}\isanewline
\ \ \ \ \isacommand{show}\isamarkupfalse%
\ {\isachardoublequoteopen}q\ {\isasymcirc}\isactrlsub c\ fibered{\isacharunderscore}{\kern0pt}product{\isacharunderscore}{\kern0pt}left{\isacharunderscore}{\kern0pt}proj\ X\ f\ f\ X\ {\isacharequal}{\kern0pt}\ q\ {\isasymcirc}\isactrlsub c\ fibered{\isacharunderscore}{\kern0pt}product{\isacharunderscore}{\kern0pt}right{\isacharunderscore}{\kern0pt}proj\ X\ f\ f\ X{\isachardoublequoteclose}\isanewline
\ \ \ \ \ \ \isacommand{by}\isamarkupfalse%
\ {\isacharparenleft}{\kern0pt}metis\ assms{\isacharparenleft}{\kern0pt}{\isadigit{1}}{\isacharparenright}{\kern0pt}\ canonical{\isacharunderscore}{\kern0pt}quotient{\isacharunderscore}{\kern0pt}map{\isacharunderscore}{\kern0pt}is{\isacharunderscore}{\kern0pt}coequalizer\ coequalizer{\isacharunderscore}{\kern0pt}def\ fibered{\isacharunderscore}{\kern0pt}product{\isacharunderscore}{\kern0pt}left{\isacharunderscore}{\kern0pt}proj{\isacharunderscore}{\kern0pt}def\ fibered{\isacharunderscore}{\kern0pt}product{\isacharunderscore}{\kern0pt}right{\isacharunderscore}{\kern0pt}proj{\isacharunderscore}{\kern0pt}def\ kernel{\isacharunderscore}{\kern0pt}pair{\isacharunderscore}{\kern0pt}equiv{\isacharunderscore}{\kern0pt}rel\ q{\isacharunderscore}{\kern0pt}def{\isacharparenright}{\kern0pt}\isanewline
\ \ \ \ \isacommand{show}\isamarkupfalse%
\ {\isachardoublequoteopen}f{\isacharunderscore}{\kern0pt}bar\ {\isacharcolon}{\kern0pt}\ F\ {\isasymrightarrow}\ Y{\isachardoublequoteclose}\ \isanewline
\ \ \ \ \ \ \isacommand{by}\isamarkupfalse%
\ typecheck{\isacharunderscore}{\kern0pt}cfuncs\isanewline
\ \ \isacommand{qed}\isamarkupfalse%
\isanewline
\isanewline
\ \ \isanewline
\ \ \isanewline
\ \ \isacommand{then}\isamarkupfalse%
\ \isacommand{obtain}\isamarkupfalse%
\ b\ \isakeyword{where}\ b{\isacharunderscore}{\kern0pt}type{\isacharbrackleft}{\kern0pt}type{\isacharunderscore}{\kern0pt}rule{\isacharbrackright}{\kern0pt}{\isacharcolon}{\kern0pt}\ {\isachardoublequoteopen}b\ {\isacharcolon}{\kern0pt}\ X\ \isactrlbsub f\isactrlesub {\isasymtimes}\isactrlsub c\isactrlbsub f\isactrlesub \ X\ {\isasymrightarrow}\ F\ \isactrlbsub {\isacharparenleft}{\kern0pt}f{\isacharunderscore}{\kern0pt}bar{\isacharparenright}{\kern0pt}\isactrlesub {\isasymtimes}\isactrlsub c\isactrlbsub {\isacharparenleft}{\kern0pt}f{\isacharunderscore}{\kern0pt}bar{\isacharparenright}{\kern0pt}\isactrlesub \ F{\isachardoublequoteclose}\ \isakeyword{and}\isanewline
\ \ \ left{\isacharunderscore}{\kern0pt}b{\isacharunderscore}{\kern0pt}eqs{\isacharcolon}{\kern0pt}\ {\isachardoublequoteopen}fibered{\isacharunderscore}{\kern0pt}product{\isacharunderscore}{\kern0pt}left{\isacharunderscore}{\kern0pt}proj\ F\ {\isacharparenleft}{\kern0pt}f{\isacharunderscore}{\kern0pt}bar{\isacharparenright}{\kern0pt}\ {\isacharparenleft}{\kern0pt}f{\isacharunderscore}{\kern0pt}bar{\isacharparenright}{\kern0pt}\ F\ {\isasymcirc}\isactrlsub c\ b\ {\isacharequal}{\kern0pt}\ q\ {\isasymcirc}\isactrlsub c\ fibered{\isacharunderscore}{\kern0pt}product{\isacharunderscore}{\kern0pt}left{\isacharunderscore}{\kern0pt}proj\ X\ f\ f\ X{\isachardoublequoteclose}\ \isakeyword{and}\isanewline
\ \ \ right{\isacharunderscore}{\kern0pt}b{\isacharunderscore}{\kern0pt}eqs{\isacharcolon}{\kern0pt}\ \ {\isachardoublequoteopen}fibered{\isacharunderscore}{\kern0pt}product{\isacharunderscore}{\kern0pt}right{\isacharunderscore}{\kern0pt}proj\ F\ {\isacharparenleft}{\kern0pt}f{\isacharunderscore}{\kern0pt}bar{\isacharparenright}{\kern0pt}\ {\isacharparenleft}{\kern0pt}f{\isacharunderscore}{\kern0pt}bar{\isacharparenright}{\kern0pt}\ F\ {\isasymcirc}\isactrlsub c\ b\ {\isacharequal}{\kern0pt}\ q\ {\isasymcirc}\isactrlsub c\ fibered{\isacharunderscore}{\kern0pt}product{\isacharunderscore}{\kern0pt}right{\isacharunderscore}{\kern0pt}proj\ X\ f\ f\ X{\isachardoublequoteclose}\ \isakeyword{and}\isanewline
\ \ \ epi{\isacharunderscore}{\kern0pt}b{\isacharcolon}{\kern0pt}\ {\isachardoublequoteopen}epimorphism\ b{\isachardoublequoteclose}\isanewline
\ \ \ \ \isacommand{by}\isamarkupfalse%
\ auto\isanewline
\isanewline
\ \isanewline
\ \ \isacommand{have}\isamarkupfalse%
\ {\isachardoublequoteopen}fibered{\isacharunderscore}{\kern0pt}product{\isacharunderscore}{\kern0pt}left{\isacharunderscore}{\kern0pt}proj\ F\ {\isacharparenleft}{\kern0pt}f{\isacharunderscore}{\kern0pt}bar{\isacharparenright}{\kern0pt}\ {\isacharparenleft}{\kern0pt}f{\isacharunderscore}{\kern0pt}bar{\isacharparenright}{\kern0pt}\ F\ {\isacharequal}{\kern0pt}\ fibered{\isacharunderscore}{\kern0pt}product{\isacharunderscore}{\kern0pt}right{\isacharunderscore}{\kern0pt}proj\ F\ {\isacharparenleft}{\kern0pt}f{\isacharunderscore}{\kern0pt}bar{\isacharparenright}{\kern0pt}\ {\isacharparenleft}{\kern0pt}f{\isacharunderscore}{\kern0pt}bar{\isacharparenright}{\kern0pt}\ F{\isachardoublequoteclose}\isanewline
\ \ \isacommand{proof}\isamarkupfalse%
\ {\isacharminus}{\kern0pt}\ \isanewline
\ \ \ \ \isacommand{have}\isamarkupfalse%
\ {\isachardoublequoteopen}{\isacharparenleft}{\kern0pt}fibered{\isacharunderscore}{\kern0pt}product{\isacharunderscore}{\kern0pt}left{\isacharunderscore}{\kern0pt}proj\ F\ {\isacharparenleft}{\kern0pt}f{\isacharunderscore}{\kern0pt}bar{\isacharparenright}{\kern0pt}\ {\isacharparenleft}{\kern0pt}f{\isacharunderscore}{\kern0pt}bar{\isacharparenright}{\kern0pt}\ F{\isacharparenright}{\kern0pt}\ {\isasymcirc}\isactrlsub c\ b\ {\isacharequal}{\kern0pt}\ q\ {\isasymcirc}\isactrlsub c\ fibered{\isacharunderscore}{\kern0pt}product{\isacharunderscore}{\kern0pt}left{\isacharunderscore}{\kern0pt}proj\ X\ f\ f\ X{\isachardoublequoteclose}\isanewline
\ \ \ \ \ \ \isacommand{by}\isamarkupfalse%
\ {\isacharparenleft}{\kern0pt}simp\ add{\isacharcolon}{\kern0pt}\ left{\isacharunderscore}{\kern0pt}b{\isacharunderscore}{\kern0pt}eqs{\isacharparenright}{\kern0pt}\isanewline
\ \ \ \ \isacommand{also}\isamarkupfalse%
\ \isacommand{have}\isamarkupfalse%
\ {\isachardoublequoteopen}{\isachardot}{\kern0pt}{\isachardot}{\kern0pt}{\isachardot}{\kern0pt}\ {\isacharequal}{\kern0pt}\ q\ {\isasymcirc}\isactrlsub c\ fibered{\isacharunderscore}{\kern0pt}product{\isacharunderscore}{\kern0pt}right{\isacharunderscore}{\kern0pt}proj\ X\ f\ f\ X{\isachardoublequoteclose}\isanewline
\ \ \ \ \ \ \isacommand{using}\isamarkupfalse%
\ assms{\isacharparenleft}{\kern0pt}{\isadigit{1}}{\isacharparenright}{\kern0pt}\ canonical{\isacharunderscore}{\kern0pt}quotient{\isacharunderscore}{\kern0pt}map{\isacharunderscore}{\kern0pt}is{\isacharunderscore}{\kern0pt}coequalizer\ coequalizer{\isacharunderscore}{\kern0pt}def\ fibered{\isacharunderscore}{\kern0pt}product{\isacharunderscore}{\kern0pt}left{\isacharunderscore}{\kern0pt}proj{\isacharunderscore}{\kern0pt}def\ fibered{\isacharunderscore}{\kern0pt}product{\isacharunderscore}{\kern0pt}right{\isacharunderscore}{\kern0pt}proj{\isacharunderscore}{\kern0pt}def\ kernel{\isacharunderscore}{\kern0pt}pair{\isacharunderscore}{\kern0pt}equiv{\isacharunderscore}{\kern0pt}rel\ q{\isacharunderscore}{\kern0pt}def\ \isacommand{by}\isamarkupfalse%
\ fastforce\isanewline
\ \ \ \ \isacommand{also}\isamarkupfalse%
\ \isacommand{have}\isamarkupfalse%
\ {\isachardoublequoteopen}{\isachardot}{\kern0pt}{\isachardot}{\kern0pt}{\isachardot}{\kern0pt}\ {\isacharequal}{\kern0pt}\ fibered{\isacharunderscore}{\kern0pt}product{\isacharunderscore}{\kern0pt}right{\isacharunderscore}{\kern0pt}proj\ F\ {\isacharparenleft}{\kern0pt}f{\isacharunderscore}{\kern0pt}bar{\isacharparenright}{\kern0pt}\ {\isacharparenleft}{\kern0pt}f{\isacharunderscore}{\kern0pt}bar{\isacharparenright}{\kern0pt}\ F\ {\isasymcirc}\isactrlsub c\ b{\isachardoublequoteclose}\isanewline
\ \ \ \ \ \ \isacommand{by}\isamarkupfalse%
\ {\isacharparenleft}{\kern0pt}simp\ add{\isacharcolon}{\kern0pt}\ right{\isacharunderscore}{\kern0pt}b{\isacharunderscore}{\kern0pt}eqs{\isacharparenright}{\kern0pt}\isanewline
\ \ \ \ \isacommand{then}\isamarkupfalse%
\ \isacommand{have}\isamarkupfalse%
\ {\isachardoublequoteopen}fibered{\isacharunderscore}{\kern0pt}product{\isacharunderscore}{\kern0pt}left{\isacharunderscore}{\kern0pt}proj\ F\ {\isacharparenleft}{\kern0pt}f{\isacharunderscore}{\kern0pt}bar{\isacharparenright}{\kern0pt}\ {\isacharparenleft}{\kern0pt}f{\isacharunderscore}{\kern0pt}bar{\isacharparenright}{\kern0pt}\ F\ {\isasymcirc}\isactrlsub c\ b\ {\isacharequal}{\kern0pt}\ fibered{\isacharunderscore}{\kern0pt}product{\isacharunderscore}{\kern0pt}right{\isacharunderscore}{\kern0pt}proj\ F\ {\isacharparenleft}{\kern0pt}f{\isacharunderscore}{\kern0pt}bar{\isacharparenright}{\kern0pt}\ {\isacharparenleft}{\kern0pt}f{\isacharunderscore}{\kern0pt}bar{\isacharparenright}{\kern0pt}\ F\ {\isasymcirc}\isactrlsub c\ b{\isachardoublequoteclose}\isanewline
\ \ \ \ \ \ \isacommand{by}\isamarkupfalse%
\ {\isacharparenleft}{\kern0pt}simp\ add{\isacharcolon}{\kern0pt}\ calculation{\isacharparenright}{\kern0pt}\isanewline
\ \ \ \ \isacommand{then}\isamarkupfalse%
\ \isacommand{show}\isamarkupfalse%
\ {\isacharquery}{\kern0pt}thesis\isanewline
\ \ \ \ \ \ \isacommand{using}\isamarkupfalse%
\ b{\isacharunderscore}{\kern0pt}type\ epi{\isacharunderscore}{\kern0pt}b\ epimorphism{\isacharunderscore}{\kern0pt}def{\isadigit{2}}\ fibr{\isacharunderscore}{\kern0pt}proj{\isacharunderscore}{\kern0pt}left{\isacharunderscore}{\kern0pt}type\ fibr{\isacharunderscore}{\kern0pt}proj{\isacharunderscore}{\kern0pt}right{\isacharunderscore}{\kern0pt}type\ \isacommand{by}\isamarkupfalse%
\ blast\isanewline
\ \ \isacommand{qed}\isamarkupfalse%
\isanewline
\isanewline
\ \ \isanewline
\ \ \isanewline
\ \ \isacommand{then}\isamarkupfalse%
\ \isacommand{obtain}\isamarkupfalse%
\ b\ \isakeyword{where}\ b{\isacharunderscore}{\kern0pt}type{\isacharbrackleft}{\kern0pt}type{\isacharunderscore}{\kern0pt}rule{\isacharbrackright}{\kern0pt}{\isacharcolon}{\kern0pt}\ {\isachardoublequoteopen}b\ {\isacharcolon}{\kern0pt}\ X\ \isactrlbsub f\isactrlesub {\isasymtimes}\isactrlsub c\isactrlbsub f\isactrlesub \ X\ {\isasymrightarrow}\ F\ \isactrlbsub {\isacharparenleft}{\kern0pt}f{\isacharunderscore}{\kern0pt}bar{\isacharparenright}{\kern0pt}\isactrlesub {\isasymtimes}\isactrlsub c\isactrlbsub {\isacharparenleft}{\kern0pt}f{\isacharunderscore}{\kern0pt}bar{\isacharparenright}{\kern0pt}\isactrlesub \ F{\isachardoublequoteclose}\ \isakeyword{and}\isanewline
\ \ \ left{\isacharunderscore}{\kern0pt}b{\isacharunderscore}{\kern0pt}eqs{\isacharcolon}{\kern0pt}\ {\isachardoublequoteopen}fibered{\isacharunderscore}{\kern0pt}product{\isacharunderscore}{\kern0pt}left{\isacharunderscore}{\kern0pt}proj\ F\ {\isacharparenleft}{\kern0pt}f{\isacharunderscore}{\kern0pt}bar{\isacharparenright}{\kern0pt}\ {\isacharparenleft}{\kern0pt}f{\isacharunderscore}{\kern0pt}bar{\isacharparenright}{\kern0pt}\ F\ {\isasymcirc}\isactrlsub c\ b\ {\isacharequal}{\kern0pt}\ q\ {\isasymcirc}\isactrlsub c\ fibered{\isacharunderscore}{\kern0pt}product{\isacharunderscore}{\kern0pt}left{\isacharunderscore}{\kern0pt}proj\ X\ f\ f\ X{\isachardoublequoteclose}\ \isakeyword{and}\isanewline
\ \ \ right{\isacharunderscore}{\kern0pt}b{\isacharunderscore}{\kern0pt}eqs{\isacharcolon}{\kern0pt}\ \ {\isachardoublequoteopen}fibered{\isacharunderscore}{\kern0pt}product{\isacharunderscore}{\kern0pt}right{\isacharunderscore}{\kern0pt}proj\ F\ {\isacharparenleft}{\kern0pt}f{\isacharunderscore}{\kern0pt}bar{\isacharparenright}{\kern0pt}\ {\isacharparenleft}{\kern0pt}f{\isacharunderscore}{\kern0pt}bar{\isacharparenright}{\kern0pt}\ F\ {\isasymcirc}\isactrlsub c\ b\ {\isacharequal}{\kern0pt}\ q\ {\isasymcirc}\isactrlsub c\ fibered{\isacharunderscore}{\kern0pt}product{\isacharunderscore}{\kern0pt}right{\isacharunderscore}{\kern0pt}proj\ X\ f\ f\ X{\isachardoublequoteclose}\ \isakeyword{and}\isanewline
\ \ \ epi{\isacharunderscore}{\kern0pt}b{\isacharcolon}{\kern0pt}\ {\isachardoublequoteopen}epimorphism\ b{\isachardoublequoteclose}\isanewline
\ \ \ \ \isacommand{using}\isamarkupfalse%
\ b{\isacharunderscore}{\kern0pt}type\ epi{\isacharunderscore}{\kern0pt}b\ left{\isacharunderscore}{\kern0pt}b{\isacharunderscore}{\kern0pt}eqs\ right{\isacharunderscore}{\kern0pt}b{\isacharunderscore}{\kern0pt}eqs\ \isacommand{by}\isamarkupfalse%
\ blast\isanewline
\ \ \isanewline
\ \isanewline
\ \ \isacommand{have}\isamarkupfalse%
\ {\isachardoublequoteopen}fibered{\isacharunderscore}{\kern0pt}product{\isacharunderscore}{\kern0pt}left{\isacharunderscore}{\kern0pt}proj\ F\ {\isacharparenleft}{\kern0pt}f{\isacharunderscore}{\kern0pt}bar{\isacharparenright}{\kern0pt}\ {\isacharparenleft}{\kern0pt}f{\isacharunderscore}{\kern0pt}bar{\isacharparenright}{\kern0pt}\ F\ {\isacharequal}{\kern0pt}\ fibered{\isacharunderscore}{\kern0pt}product{\isacharunderscore}{\kern0pt}right{\isacharunderscore}{\kern0pt}proj\ F\ {\isacharparenleft}{\kern0pt}f{\isacharunderscore}{\kern0pt}bar{\isacharparenright}{\kern0pt}\ {\isacharparenleft}{\kern0pt}f{\isacharunderscore}{\kern0pt}bar{\isacharparenright}{\kern0pt}\ F{\isachardoublequoteclose}\isanewline
\ \ \isacommand{proof}\isamarkupfalse%
\ {\isacharminus}{\kern0pt}\ \isanewline
\ \ \ \ \isacommand{have}\isamarkupfalse%
\ {\isachardoublequoteopen}{\isacharparenleft}{\kern0pt}fibered{\isacharunderscore}{\kern0pt}product{\isacharunderscore}{\kern0pt}left{\isacharunderscore}{\kern0pt}proj\ F\ {\isacharparenleft}{\kern0pt}f{\isacharunderscore}{\kern0pt}bar{\isacharparenright}{\kern0pt}\ {\isacharparenleft}{\kern0pt}f{\isacharunderscore}{\kern0pt}bar{\isacharparenright}{\kern0pt}\ F{\isacharparenright}{\kern0pt}\ {\isasymcirc}\isactrlsub c\ b\ {\isacharequal}{\kern0pt}\ q\ {\isasymcirc}\isactrlsub c\ fibered{\isacharunderscore}{\kern0pt}product{\isacharunderscore}{\kern0pt}left{\isacharunderscore}{\kern0pt}proj\ X\ f\ f\ X{\isachardoublequoteclose}\isanewline
\ \ \ \ \ \ \isacommand{by}\isamarkupfalse%
\ {\isacharparenleft}{\kern0pt}simp\ add{\isacharcolon}{\kern0pt}\ left{\isacharunderscore}{\kern0pt}b{\isacharunderscore}{\kern0pt}eqs{\isacharparenright}{\kern0pt}\isanewline
\ \ \ \ \isacommand{also}\isamarkupfalse%
\ \isacommand{have}\isamarkupfalse%
\ {\isachardoublequoteopen}{\isachardot}{\kern0pt}{\isachardot}{\kern0pt}{\isachardot}{\kern0pt}\ {\isacharequal}{\kern0pt}\ q\ {\isasymcirc}\isactrlsub c\ fibered{\isacharunderscore}{\kern0pt}product{\isacharunderscore}{\kern0pt}right{\isacharunderscore}{\kern0pt}proj\ X\ f\ f\ X{\isachardoublequoteclose}\isanewline
\ \ \ \ \ \ \isacommand{using}\isamarkupfalse%
\ assms{\isacharparenleft}{\kern0pt}{\isadigit{1}}{\isacharparenright}{\kern0pt}\ canonical{\isacharunderscore}{\kern0pt}quotient{\isacharunderscore}{\kern0pt}map{\isacharunderscore}{\kern0pt}is{\isacharunderscore}{\kern0pt}coequalizer\ coequalizer{\isacharunderscore}{\kern0pt}def\ fibered{\isacharunderscore}{\kern0pt}product{\isacharunderscore}{\kern0pt}left{\isacharunderscore}{\kern0pt}proj{\isacharunderscore}{\kern0pt}def\ fibered{\isacharunderscore}{\kern0pt}product{\isacharunderscore}{\kern0pt}right{\isacharunderscore}{\kern0pt}proj{\isacharunderscore}{\kern0pt}def\ kernel{\isacharunderscore}{\kern0pt}pair{\isacharunderscore}{\kern0pt}equiv{\isacharunderscore}{\kern0pt}rel\ q{\isacharunderscore}{\kern0pt}def\ \isacommand{by}\isamarkupfalse%
\ fastforce\isanewline
\ \ \ \ \isacommand{also}\isamarkupfalse%
\ \isacommand{have}\isamarkupfalse%
\ {\isachardoublequoteopen}{\isachardot}{\kern0pt}{\isachardot}{\kern0pt}{\isachardot}{\kern0pt}\ {\isacharequal}{\kern0pt}\ fibered{\isacharunderscore}{\kern0pt}product{\isacharunderscore}{\kern0pt}right{\isacharunderscore}{\kern0pt}proj\ F\ {\isacharparenleft}{\kern0pt}f{\isacharunderscore}{\kern0pt}bar{\isacharparenright}{\kern0pt}\ {\isacharparenleft}{\kern0pt}f{\isacharunderscore}{\kern0pt}bar{\isacharparenright}{\kern0pt}\ F\ {\isasymcirc}\isactrlsub c\ b{\isachardoublequoteclose}\isanewline
\ \ \ \ \ \ \isacommand{by}\isamarkupfalse%
\ {\isacharparenleft}{\kern0pt}simp\ add{\isacharcolon}{\kern0pt}\ right{\isacharunderscore}{\kern0pt}b{\isacharunderscore}{\kern0pt}eqs{\isacharparenright}{\kern0pt}\isanewline
\ \ \ \ \isacommand{then}\isamarkupfalse%
\ \isacommand{have}\isamarkupfalse%
\ {\isachardoublequoteopen}fibered{\isacharunderscore}{\kern0pt}product{\isacharunderscore}{\kern0pt}left{\isacharunderscore}{\kern0pt}proj\ F\ {\isacharparenleft}{\kern0pt}f{\isacharunderscore}{\kern0pt}bar{\isacharparenright}{\kern0pt}\ {\isacharparenleft}{\kern0pt}f{\isacharunderscore}{\kern0pt}bar{\isacharparenright}{\kern0pt}\ F\ {\isasymcirc}\isactrlsub c\ b\ {\isacharequal}{\kern0pt}\ fibered{\isacharunderscore}{\kern0pt}product{\isacharunderscore}{\kern0pt}right{\isacharunderscore}{\kern0pt}proj\ F\ {\isacharparenleft}{\kern0pt}f{\isacharunderscore}{\kern0pt}bar{\isacharparenright}{\kern0pt}\ {\isacharparenleft}{\kern0pt}f{\isacharunderscore}{\kern0pt}bar{\isacharparenright}{\kern0pt}\ F\ {\isasymcirc}\isactrlsub c\ b{\isachardoublequoteclose}\isanewline
\ \ \ \ \ \ \isacommand{by}\isamarkupfalse%
\ {\isacharparenleft}{\kern0pt}simp\ add{\isacharcolon}{\kern0pt}\ calculation{\isacharparenright}{\kern0pt}\isanewline
\ \ \ \ \isacommand{then}\isamarkupfalse%
\ \isacommand{show}\isamarkupfalse%
\ {\isacharquery}{\kern0pt}thesis\isanewline
\ \ \ \ \ \ \isacommand{using}\isamarkupfalse%
\ b{\isacharunderscore}{\kern0pt}type\ epi{\isacharunderscore}{\kern0pt}b\ epimorphism{\isacharunderscore}{\kern0pt}def{\isadigit{2}}\ fibr{\isacharunderscore}{\kern0pt}proj{\isacharunderscore}{\kern0pt}left{\isacharunderscore}{\kern0pt}type\ fibr{\isacharunderscore}{\kern0pt}proj{\isacharunderscore}{\kern0pt}right{\isacharunderscore}{\kern0pt}type\ \isacommand{by}\isamarkupfalse%
\ blast\isanewline
\ \ \isacommand{qed}\isamarkupfalse%
\isanewline
\ \ \isanewline
\ \ \isacommand{then}\isamarkupfalse%
\ \isacommand{have}\isamarkupfalse%
\ mono{\isacharunderscore}{\kern0pt}fbar{\isacharcolon}{\kern0pt}\ {\isachardoublequoteopen}monomorphism{\isacharparenleft}{\kern0pt}f{\isacharunderscore}{\kern0pt}bar{\isacharparenright}{\kern0pt}{\isachardoublequoteclose}\isanewline
\ \ \ \ \isacommand{by}\isamarkupfalse%
\ {\isacharparenleft}{\kern0pt}typecheck{\isacharunderscore}{\kern0pt}cfuncs{\isacharcomma}{\kern0pt}\ simp\ add{\isacharcolon}{\kern0pt}\ \ kern{\isacharunderscore}{\kern0pt}pair{\isacharunderscore}{\kern0pt}proj{\isacharunderscore}{\kern0pt}iso{\isacharunderscore}{\kern0pt}TFAE{\isadigit{2}}{\isacharparenright}{\kern0pt}\isanewline
\ \ \isanewline
\ \ \isacommand{have}\isamarkupfalse%
\ {\isachardoublequoteopen}epimorphism{\isacharparenleft}{\kern0pt}f{\isacharunderscore}{\kern0pt}bar{\isacharparenright}{\kern0pt}{\isachardoublequoteclose}\isanewline
\ \ \ \ \isacommand{by}\isamarkupfalse%
\ {\isacharparenleft}{\kern0pt}typecheck{\isacharunderscore}{\kern0pt}cfuncs{\isacharcomma}{\kern0pt}\ metis\ assms{\isacharparenleft}{\kern0pt}{\isadigit{2}}{\isacharparenright}{\kern0pt}\ cfunc{\isacharunderscore}{\kern0pt}type{\isacharunderscore}{\kern0pt}def\ comp{\isacharunderscore}{\kern0pt}epi{\isacharunderscore}{\kern0pt}imp{\isacharunderscore}{\kern0pt}epi\ f{\isacharunderscore}{\kern0pt}eqs\ q{\isacharunderscore}{\kern0pt}type{\isacharparenright}{\kern0pt}\isanewline
\ \ \isanewline
\ \ \isacommand{then}\isamarkupfalse%
\ \isacommand{have}\isamarkupfalse%
\ {\isachardoublequoteopen}isomorphism{\isacharparenleft}{\kern0pt}f{\isacharunderscore}{\kern0pt}bar{\isacharparenright}{\kern0pt}{\isachardoublequoteclose}\isanewline
\ \ \ \ \isacommand{by}\isamarkupfalse%
\ {\isacharparenleft}{\kern0pt}simp\ add{\isacharcolon}{\kern0pt}\ epi{\isacharunderscore}{\kern0pt}mon{\isacharunderscore}{\kern0pt}is{\isacharunderscore}{\kern0pt}iso\ mono{\isacharunderscore}{\kern0pt}fbar{\isacharparenright}{\kern0pt}\isanewline
\isanewline
\ \ \isanewline
\ \ \isanewline
\isanewline
\ \ \isacommand{obtain}\isamarkupfalse%
\ f{\isacharunderscore}{\kern0pt}bar{\isacharunderscore}{\kern0pt}inv\ \isakeyword{where}\ f{\isacharunderscore}{\kern0pt}bar{\isacharunderscore}{\kern0pt}inv{\isacharunderscore}{\kern0pt}type{\isacharbrackleft}{\kern0pt}type{\isacharunderscore}{\kern0pt}rule{\isacharbrackright}{\kern0pt}{\isacharcolon}{\kern0pt}\ {\isachardoublequoteopen}f{\isacharunderscore}{\kern0pt}bar{\isacharunderscore}{\kern0pt}inv{\isacharcolon}{\kern0pt}\ Y\ {\isasymrightarrow}\ F{\isachardoublequoteclose}\ \isakeyword{and}\isanewline
\ \ \ \ \ \ \ \ \ \ \ \ \ \ \ \ \ \ \ \ \ \ \ \ \ \ \ \ f{\isacharunderscore}{\kern0pt}bar{\isacharunderscore}{\kern0pt}inv{\isacharunderscore}{\kern0pt}eq{\isadigit{1}}{\isacharcolon}{\kern0pt}\ {\isachardoublequoteopen}f{\isacharunderscore}{\kern0pt}bar{\isacharunderscore}{\kern0pt}inv\ {\isasymcirc}\isactrlsub c\ f{\isacharunderscore}{\kern0pt}bar\ {\isacharequal}{\kern0pt}\ id{\isacharparenleft}{\kern0pt}F{\isacharparenright}{\kern0pt}{\isachardoublequoteclose}\ \isakeyword{and}\ \ \isanewline
\ \ \ \ \ \ \ \ \ \ \ \ \ \ \ \ \ \ \ \ \ \ \ \ \ \ \ \ f{\isacharunderscore}{\kern0pt}bar{\isacharunderscore}{\kern0pt}inv{\isacharunderscore}{\kern0pt}eq{\isadigit{2}}{\isacharcolon}{\kern0pt}\ {\isachardoublequoteopen}f{\isacharunderscore}{\kern0pt}bar\ {\isasymcirc}\isactrlsub c\ f{\isacharunderscore}{\kern0pt}bar{\isacharunderscore}{\kern0pt}inv\ {\isacharequal}{\kern0pt}\ id{\isacharparenleft}{\kern0pt}Y{\isacharparenright}{\kern0pt}{\isachardoublequoteclose}\isanewline
\ \ \ \ \isacommand{using}\isamarkupfalse%
\ {\isacartoucheopen}isomorphism\ f{\isacharunderscore}{\kern0pt}bar{\isacartoucheclose}\ cfunc{\isacharunderscore}{\kern0pt}type{\isacharunderscore}{\kern0pt}def\ isomorphism{\isacharunderscore}{\kern0pt}def\ \isacommand{by}\isamarkupfalse%
\ {\isacharparenleft}{\kern0pt}typecheck{\isacharunderscore}{\kern0pt}cfuncs{\isacharcomma}{\kern0pt}\ force{\isacharparenright}{\kern0pt}\isanewline
\ \ \isanewline
\ \ \isacommand{obtain}\isamarkupfalse%
\ g{\isacharunderscore}{\kern0pt}bar\ \isakeyword{where}\ g{\isacharunderscore}{\kern0pt}bar{\isacharunderscore}{\kern0pt}def{\isacharcolon}{\kern0pt}\ {\isachardoublequoteopen}g{\isacharunderscore}{\kern0pt}bar\ {\isacharequal}{\kern0pt}\ quotient{\isacharunderscore}{\kern0pt}func\ g\ {\isacharparenleft}{\kern0pt}X\ \isactrlbsub f\isactrlesub {\isasymtimes}\isactrlsub c\isactrlbsub f\isactrlesub \ X{\isacharcomma}{\kern0pt}\ fibered{\isacharunderscore}{\kern0pt}product{\isacharunderscore}{\kern0pt}morphism\ X\ f\ f\ X{\isacharparenright}{\kern0pt}{\isachardoublequoteclose}\isanewline
\ \ \ \ \isacommand{by}\isamarkupfalse%
\ auto\isanewline
\ \ \isacommand{have}\isamarkupfalse%
\ {\isachardoublequoteopen}const{\isacharunderscore}{\kern0pt}on{\isacharunderscore}{\kern0pt}rel\ X\ {\isacharparenleft}{\kern0pt}X\ \isactrlbsub f\isactrlesub {\isasymtimes}\isactrlsub c\isactrlbsub f\isactrlesub \ X{\isacharcomma}{\kern0pt}\ fibered{\isacharunderscore}{\kern0pt}product{\isacharunderscore}{\kern0pt}morphism\ X\ f\ f\ X{\isacharparenright}{\kern0pt}\ g{\isachardoublequoteclose}\isanewline
\ \ \ \ \isacommand{unfolding}\isamarkupfalse%
\ const{\isacharunderscore}{\kern0pt}on{\isacharunderscore}{\kern0pt}rel{\isacharunderscore}{\kern0pt}def\ \isanewline
\ \ \ \ \isacommand{by}\isamarkupfalse%
\ {\isacharparenleft}{\kern0pt}meson\ assms{\isacharparenleft}{\kern0pt}{\isadigit{1}}{\isacharparenright}{\kern0pt}\ fibered{\isacharunderscore}{\kern0pt}product{\isacharunderscore}{\kern0pt}pair{\isacharunderscore}{\kern0pt}member{\isadigit{2}}\ g{\isacharunderscore}{\kern0pt}eq\ g{\isacharunderscore}{\kern0pt}type{\isacharparenright}{\kern0pt}\isanewline
\ \ \isacommand{then}\isamarkupfalse%
\ \isacommand{have}\isamarkupfalse%
\ g{\isacharunderscore}{\kern0pt}bar{\isacharunderscore}{\kern0pt}type{\isacharbrackleft}{\kern0pt}type{\isacharunderscore}{\kern0pt}rule{\isacharbrackright}{\kern0pt}{\isacharcolon}{\kern0pt}\ {\isachardoublequoteopen}g{\isacharunderscore}{\kern0pt}bar\ {\isacharcolon}{\kern0pt}\ F\ {\isasymrightarrow}\ E{\isachardoublequoteclose}\isanewline
\ \ \ \ \isacommand{using}\isamarkupfalse%
\ F{\isacharunderscore}{\kern0pt}def\ assms{\isacharparenleft}{\kern0pt}{\isadigit{1}}{\isacharparenright}{\kern0pt}\ g{\isacharunderscore}{\kern0pt}bar{\isacharunderscore}{\kern0pt}def\ g{\isacharunderscore}{\kern0pt}type\ kernel{\isacharunderscore}{\kern0pt}pair{\isacharunderscore}{\kern0pt}equiv{\isacharunderscore}{\kern0pt}rel\ quotient{\isacharunderscore}{\kern0pt}func{\isacharunderscore}{\kern0pt}type\ \isacommand{by}\isamarkupfalse%
\ blast\isanewline
\ \ \isacommand{obtain}\isamarkupfalse%
\ k\ \isakeyword{where}\ k{\isacharunderscore}{\kern0pt}def{\isacharcolon}{\kern0pt}\ {\isachardoublequoteopen}k\ {\isacharequal}{\kern0pt}\ g{\isacharunderscore}{\kern0pt}bar\ {\isasymcirc}\isactrlsub c\ f{\isacharunderscore}{\kern0pt}bar{\isacharunderscore}{\kern0pt}inv{\isachardoublequoteclose}\ \isakeyword{and}\ k{\isacharunderscore}{\kern0pt}type{\isacharbrackleft}{\kern0pt}type{\isacharunderscore}{\kern0pt}rule{\isacharbrackright}{\kern0pt}{\isacharcolon}{\kern0pt}\ {\isachardoublequoteopen}k\ {\isacharcolon}{\kern0pt}\ Y\ {\isasymrightarrow}\ E{\isachardoublequoteclose}\isanewline
\ \ \ \ \isacommand{by}\isamarkupfalse%
\ typecheck{\isacharunderscore}{\kern0pt}cfuncs\ \ \ \isanewline
\ \ \isacommand{then}\isamarkupfalse%
\ \isacommand{show}\isamarkupfalse%
\ {\isachardoublequoteopen}{\isasymexists}k{\isachardot}{\kern0pt}\ k\ {\isacharcolon}{\kern0pt}\ Y\ {\isasymrightarrow}\ E\ {\isasymand}\ k\ {\isasymcirc}\isactrlsub c\ f\ {\isacharequal}{\kern0pt}\ g{\isachardoublequoteclose}\isanewline
\ \ \ \ \isacommand{by}\isamarkupfalse%
\ {\isacharparenleft}{\kern0pt}smt\ {\isacharparenleft}{\kern0pt}z{\isadigit{3}}{\isacharparenright}{\kern0pt}\ {\isacartoucheopen}const{\isacharunderscore}{\kern0pt}on{\isacharunderscore}{\kern0pt}rel\ X\ {\isacharparenleft}{\kern0pt}X\ \isactrlbsub f\isactrlesub {\isasymtimes}\isactrlsub c\isactrlbsub f\isactrlesub \ X{\isacharcomma}{\kern0pt}\ fibered{\isacharunderscore}{\kern0pt}product{\isacharunderscore}{\kern0pt}morphism\ X\ f\ f\ X{\isacharparenright}{\kern0pt}\ g{\isacartoucheclose}\ assms{\isacharparenleft}{\kern0pt}{\isadigit{1}}{\isacharparenright}{\kern0pt}\ comp{\isacharunderscore}{\kern0pt}associative{\isadigit{2}}\ f{\isacharunderscore}{\kern0pt}bar{\isacharunderscore}{\kern0pt}inv{\isacharunderscore}{\kern0pt}eq{\isadigit{1}}\ f{\isacharunderscore}{\kern0pt}bar{\isacharunderscore}{\kern0pt}inv{\isacharunderscore}{\kern0pt}type\ f{\isacharunderscore}{\kern0pt}bar{\isacharunderscore}{\kern0pt}type\ f{\isacharunderscore}{\kern0pt}eqs\ g{\isacharunderscore}{\kern0pt}bar{\isacharunderscore}{\kern0pt}def\ g{\isacharunderscore}{\kern0pt}bar{\isacharunderscore}{\kern0pt}type\ g{\isacharunderscore}{\kern0pt}type\ id{\isacharunderscore}{\kern0pt}left{\isacharunderscore}{\kern0pt}unit{\isadigit{2}}\ kernel{\isacharunderscore}{\kern0pt}pair{\isacharunderscore}{\kern0pt}equiv{\isacharunderscore}{\kern0pt}rel\ q{\isacharunderscore}{\kern0pt}def\ q{\isacharunderscore}{\kern0pt}type\ quotient{\isacharunderscore}{\kern0pt}func{\isacharunderscore}{\kern0pt}eq{\isacharparenright}{\kern0pt}\isanewline
\isacommand{next}\isamarkupfalse%
\isanewline
\ \ \isacommand{show}\isamarkupfalse%
\ {\isachardoublequoteopen}{\isasymAnd}F\ k\ y{\isachardot}{\kern0pt}\isanewline
\ \ \ \ \ \ \ k\ {\isasymcirc}\isactrlsub c\ f\ {\isacharcolon}{\kern0pt}\ X\ {\isasymrightarrow}\ F\ {\isasymLongrightarrow}\isanewline
\ \ \ \ \ \ \ {\isacharparenleft}{\kern0pt}k\ {\isasymcirc}\isactrlsub c\ f{\isacharparenright}{\kern0pt}\ {\isasymcirc}\isactrlsub c\ fibered{\isacharunderscore}{\kern0pt}product{\isacharunderscore}{\kern0pt}left{\isacharunderscore}{\kern0pt}proj\ X\ f\ f\ X\ {\isacharequal}{\kern0pt}\ {\isacharparenleft}{\kern0pt}k\ {\isasymcirc}\isactrlsub c\ f{\isacharparenright}{\kern0pt}\ {\isasymcirc}\isactrlsub c\ fibered{\isacharunderscore}{\kern0pt}product{\isacharunderscore}{\kern0pt}right{\isacharunderscore}{\kern0pt}proj\ X\ f\ f\ X\ {\isasymLongrightarrow}\isanewline
\ \ \ \ \ \ \ k\ {\isacharcolon}{\kern0pt}\ Y\ {\isasymrightarrow}\ F\ {\isasymLongrightarrow}\ y\ {\isacharcolon}{\kern0pt}\ Y\ {\isasymrightarrow}\ F\ {\isasymLongrightarrow}\ y\ {\isasymcirc}\isactrlsub c\ f\ {\isacharequal}{\kern0pt}\ k\ {\isasymcirc}\isactrlsub c\ f\ {\isasymLongrightarrow}\ k\ {\isacharequal}{\kern0pt}\ y{\isachardoublequoteclose}\isanewline
\ \ \ \ \isacommand{using}\isamarkupfalse%
\ assms\ epimorphism{\isacharunderscore}{\kern0pt}def{\isadigit{2}}\ \isacommand{by}\isamarkupfalse%
\ blast\isanewline
\isacommand{qed}\isamarkupfalse%
%
\endisatagproof
{\isafoldproof}%
%
\isadelimproof
\isanewline
%
\endisadelimproof
\isanewline
\isacommand{lemma}\isamarkupfalse%
\ epimorphisms{\isacharunderscore}{\kern0pt}are{\isacharunderscore}{\kern0pt}regular{\isacharcolon}{\kern0pt}\isanewline
\ \ \isakeyword{assumes}\ {\isachardoublequoteopen}f\ {\isacharcolon}{\kern0pt}\ X\ {\isasymrightarrow}\ Y{\isachardoublequoteclose}\ {\isachardoublequoteopen}epimorphism\ f{\isachardoublequoteclose}\isanewline
\ \ \isakeyword{shows}\ {\isachardoublequoteopen}regular{\isacharunderscore}{\kern0pt}epimorphism\ f{\isachardoublequoteclose}\isanewline
%
\isadelimproof
\ \ %
\endisadelimproof
%
\isatagproof
\isacommand{by}\isamarkupfalse%
\ {\isacharparenleft}{\kern0pt}meson\ assms{\isacharparenleft}{\kern0pt}{\isadigit{2}}{\isacharparenright}{\kern0pt}\ cfunc{\isacharunderscore}{\kern0pt}type{\isacharunderscore}{\kern0pt}def\ epimorphism{\isacharunderscore}{\kern0pt}coequalizer{\isacharunderscore}{\kern0pt}kernel{\isacharunderscore}{\kern0pt}pair\ regular{\isacharunderscore}{\kern0pt}epimorphism{\isacharunderscore}{\kern0pt}def{\isacharparenright}{\kern0pt}%
\endisatagproof
{\isafoldproof}%
%
\isadelimproof
%
\endisadelimproof
%
\isadelimdocument
%
\endisadelimdocument
%
\isatagdocument
%
\isamarkupsubsection{Epi-monic Factorization%
}
\isamarkuptrue%
%
\endisatagdocument
{\isafolddocument}%
%
\isadelimdocument
%
\endisadelimdocument
\isacommand{lemma}\isamarkupfalse%
\ epi{\isacharunderscore}{\kern0pt}monic{\isacharunderscore}{\kern0pt}factorization{\isacharcolon}{\kern0pt}\isanewline
\ \ \isakeyword{assumes}\ f{\isacharunderscore}{\kern0pt}type{\isacharbrackleft}{\kern0pt}type{\isacharunderscore}{\kern0pt}rule{\isacharbrackright}{\kern0pt}{\isacharcolon}{\kern0pt}\ {\isachardoublequoteopen}f\ {\isacharcolon}{\kern0pt}\ X\ {\isasymrightarrow}\ Y{\isachardoublequoteclose}\isanewline
\ \ \isakeyword{shows}\ {\isachardoublequoteopen}{\isasymexists}\ g\ m\ E{\isachardot}{\kern0pt}\ g\ {\isacharcolon}{\kern0pt}\ X\ {\isasymrightarrow}\ E\ {\isasymand}\ m\ {\isacharcolon}{\kern0pt}\ E\ {\isasymrightarrow}\ Y\ \isanewline
\ \ \ \ {\isasymand}\ coequalizer\ E\ g\ {\isacharparenleft}{\kern0pt}fibered{\isacharunderscore}{\kern0pt}product{\isacharunderscore}{\kern0pt}left{\isacharunderscore}{\kern0pt}proj\ X\ f\ f\ X{\isacharparenright}{\kern0pt}\ {\isacharparenleft}{\kern0pt}fibered{\isacharunderscore}{\kern0pt}product{\isacharunderscore}{\kern0pt}right{\isacharunderscore}{\kern0pt}proj\ X\ f\ f\ X{\isacharparenright}{\kern0pt}\isanewline
\ \ \ \ {\isasymand}\ monomorphism\ m\ {\isasymand}\ f\ {\isacharequal}{\kern0pt}\ m\ {\isasymcirc}\isactrlsub c\ g\isanewline
\ \ \ \ {\isasymand}\ {\isacharparenleft}{\kern0pt}{\isasymforall}x{\isachardot}{\kern0pt}\ x\ {\isacharcolon}{\kern0pt}\ E\ {\isasymrightarrow}\ Y\ {\isasymlongrightarrow}\ f\ {\isacharequal}{\kern0pt}\ x\ {\isasymcirc}\isactrlsub c\ g\ {\isasymlongrightarrow}\ x\ {\isacharequal}{\kern0pt}\ m{\isacharparenright}{\kern0pt}{\isachardoublequoteclose}\isanewline
%
\isadelimproof
%
\endisadelimproof
%
\isatagproof
\isacommand{proof}\isamarkupfalse%
\ {\isacharminus}{\kern0pt}\isanewline
\ \ \isacommand{obtain}\isamarkupfalse%
\ q\ \isakeyword{where}\ q{\isacharunderscore}{\kern0pt}def{\isacharcolon}{\kern0pt}\ {\isachardoublequoteopen}q\ {\isacharequal}{\kern0pt}\ equiv{\isacharunderscore}{\kern0pt}class\ {\isacharparenleft}{\kern0pt}X\ \isactrlbsub f\isactrlesub {\isasymtimes}\isactrlsub c\isactrlbsub f\isactrlesub \ X{\isacharcomma}{\kern0pt}\ fibered{\isacharunderscore}{\kern0pt}product{\isacharunderscore}{\kern0pt}morphism\ X\ f\ f\ X{\isacharparenright}{\kern0pt}{\isachardoublequoteclose}\isanewline
\ \ \ \ \isacommand{by}\isamarkupfalse%
\ auto\isanewline
\ \ \isacommand{obtain}\isamarkupfalse%
\ E\ \isakeyword{where}\ E{\isacharunderscore}{\kern0pt}def{\isacharcolon}{\kern0pt}\ {\isachardoublequoteopen}E\ {\isacharequal}{\kern0pt}\ quotient{\isacharunderscore}{\kern0pt}set\ X\ {\isacharparenleft}{\kern0pt}X\ \isactrlbsub f\isactrlesub {\isasymtimes}\isactrlsub c\isactrlbsub f\isactrlesub \ X{\isacharcomma}{\kern0pt}\ fibered{\isacharunderscore}{\kern0pt}product{\isacharunderscore}{\kern0pt}morphism\ X\ f\ f\ X{\isacharparenright}{\kern0pt}{\isachardoublequoteclose}\isanewline
\ \ \ \ \isacommand{by}\isamarkupfalse%
\ auto\isanewline
\ \ \isacommand{obtain}\isamarkupfalse%
\ m\ \isakeyword{where}\ m{\isacharunderscore}{\kern0pt}def{\isacharcolon}{\kern0pt}\ {\isachardoublequoteopen}m\ {\isacharequal}{\kern0pt}\ quotient{\isacharunderscore}{\kern0pt}func\ f\ {\isacharparenleft}{\kern0pt}X\ \isactrlbsub f\isactrlesub {\isasymtimes}\isactrlsub c\isactrlbsub f\isactrlesub \ X{\isacharcomma}{\kern0pt}\ fibered{\isacharunderscore}{\kern0pt}product{\isacharunderscore}{\kern0pt}morphism\ X\ f\ f\ X{\isacharparenright}{\kern0pt}{\isachardoublequoteclose}\isanewline
\ \ \ \ \isacommand{by}\isamarkupfalse%
\ auto\isanewline
\ \ \isacommand{show}\isamarkupfalse%
\ {\isachardoublequoteopen}{\isasymexists}\ g\ m\ E{\isachardot}{\kern0pt}\ g\ {\isacharcolon}{\kern0pt}\ X\ {\isasymrightarrow}\ E\ {\isasymand}\ m\ {\isacharcolon}{\kern0pt}\ E\ {\isasymrightarrow}\ Y\ \isanewline
\ \ \ \ {\isasymand}\ coequalizer\ E\ g\ {\isacharparenleft}{\kern0pt}fibered{\isacharunderscore}{\kern0pt}product{\isacharunderscore}{\kern0pt}left{\isacharunderscore}{\kern0pt}proj\ X\ f\ f\ X{\isacharparenright}{\kern0pt}\ {\isacharparenleft}{\kern0pt}fibered{\isacharunderscore}{\kern0pt}product{\isacharunderscore}{\kern0pt}right{\isacharunderscore}{\kern0pt}proj\ X\ f\ f\ X{\isacharparenright}{\kern0pt}\isanewline
\ \ \ \ {\isasymand}\ monomorphism\ m\ {\isasymand}\ f\ {\isacharequal}{\kern0pt}\ m\ {\isasymcirc}\isactrlsub c\ g\isanewline
\ \ \ \ {\isasymand}\ {\isacharparenleft}{\kern0pt}{\isasymforall}x{\isachardot}{\kern0pt}\ x\ {\isacharcolon}{\kern0pt}\ E\ {\isasymrightarrow}\ Y\ {\isasymlongrightarrow}\ f\ {\isacharequal}{\kern0pt}\ x\ {\isasymcirc}\isactrlsub c\ g\ {\isasymlongrightarrow}\ x\ {\isacharequal}{\kern0pt}\ m{\isacharparenright}{\kern0pt}{\isachardoublequoteclose}\isanewline
\ \ \isacommand{proof}\isamarkupfalse%
\ {\isacharparenleft}{\kern0pt}rule{\isacharunderscore}{\kern0pt}tac\ x{\isacharequal}{\kern0pt}{\isachardoublequoteopen}q{\isachardoublequoteclose}\ \isakeyword{in}\ exI{\isacharcomma}{\kern0pt}\ rule{\isacharunderscore}{\kern0pt}tac\ x{\isacharequal}{\kern0pt}{\isachardoublequoteopen}m{\isachardoublequoteclose}\ \isakeyword{in}\ exI{\isacharcomma}{\kern0pt}\ rule{\isacharunderscore}{\kern0pt}tac\ x{\isacharequal}{\kern0pt}{\isachardoublequoteopen}E{\isachardoublequoteclose}\ \isakeyword{in}\ exI{\isacharcomma}{\kern0pt}\ auto{\isacharparenright}{\kern0pt}\isanewline
\ \ \ \ \isacommand{show}\isamarkupfalse%
\ q{\isacharunderscore}{\kern0pt}type{\isacharbrackleft}{\kern0pt}type{\isacharunderscore}{\kern0pt}rule{\isacharbrackright}{\kern0pt}{\isacharcolon}{\kern0pt}\ {\isachardoublequoteopen}q\ {\isacharcolon}{\kern0pt}\ X\ {\isasymrightarrow}\ E{\isachardoublequoteclose}\isanewline
\ \ \ \ \ \ \isacommand{unfolding}\isamarkupfalse%
\ q{\isacharunderscore}{\kern0pt}def\ E{\isacharunderscore}{\kern0pt}def\ \isacommand{using}\isamarkupfalse%
\ kernel{\isacharunderscore}{\kern0pt}pair{\isacharunderscore}{\kern0pt}equiv{\isacharunderscore}{\kern0pt}rel\ \isacommand{by}\isamarkupfalse%
\ {\isacharparenleft}{\kern0pt}typecheck{\isacharunderscore}{\kern0pt}cfuncs{\isacharcomma}{\kern0pt}\ blast{\isacharparenright}{\kern0pt}\isanewline
\ \ \ \ \isanewline
\ \ \ \ \isacommand{have}\isamarkupfalse%
\ f{\isacharunderscore}{\kern0pt}const{\isacharcolon}{\kern0pt}\ {\isachardoublequoteopen}const{\isacharunderscore}{\kern0pt}on{\isacharunderscore}{\kern0pt}rel\ X\ {\isacharparenleft}{\kern0pt}X\ \isactrlbsub f\isactrlesub {\isasymtimes}\isactrlsub c\isactrlbsub f\isactrlesub \ X{\isacharcomma}{\kern0pt}\ fibered{\isacharunderscore}{\kern0pt}product{\isacharunderscore}{\kern0pt}morphism\ X\ f\ f\ X{\isacharparenright}{\kern0pt}\ f{\isachardoublequoteclose}\isanewline
\ \ \ \ \ \ \isacommand{unfolding}\isamarkupfalse%
\ const{\isacharunderscore}{\kern0pt}on{\isacharunderscore}{\kern0pt}rel{\isacharunderscore}{\kern0pt}def\ \isacommand{using}\isamarkupfalse%
\ assms\ fibered{\isacharunderscore}{\kern0pt}product{\isacharunderscore}{\kern0pt}pair{\isacharunderscore}{\kern0pt}member\ \isacommand{by}\isamarkupfalse%
\ auto\isanewline
\ \ \ \ \isacommand{then}\isamarkupfalse%
\ \isacommand{show}\isamarkupfalse%
\ m{\isacharunderscore}{\kern0pt}type{\isacharbrackleft}{\kern0pt}type{\isacharunderscore}{\kern0pt}rule{\isacharbrackright}{\kern0pt}{\isacharcolon}{\kern0pt}\ {\isachardoublequoteopen}m\ {\isacharcolon}{\kern0pt}\ E\ {\isasymrightarrow}\ Y{\isachardoublequoteclose}\isanewline
\ \ \ \ \ \ \isacommand{unfolding}\isamarkupfalse%
\ m{\isacharunderscore}{\kern0pt}def\ E{\isacharunderscore}{\kern0pt}def\ \isacommand{using}\isamarkupfalse%
\ kernel{\isacharunderscore}{\kern0pt}pair{\isacharunderscore}{\kern0pt}equiv{\isacharunderscore}{\kern0pt}rel\ \isacommand{by}\isamarkupfalse%
\ {\isacharparenleft}{\kern0pt}typecheck{\isacharunderscore}{\kern0pt}cfuncs{\isacharcomma}{\kern0pt}\ blast{\isacharparenright}{\kern0pt}\isanewline
\ \ \ \ \isanewline
\ \ \ \ \isacommand{show}\isamarkupfalse%
\ q{\isacharunderscore}{\kern0pt}coequalizer{\isacharcolon}{\kern0pt}\ {\isachardoublequoteopen}coequalizer\ E\ q\ {\isacharparenleft}{\kern0pt}fibered{\isacharunderscore}{\kern0pt}product{\isacharunderscore}{\kern0pt}left{\isacharunderscore}{\kern0pt}proj\ X\ f\ f\ X{\isacharparenright}{\kern0pt}\ {\isacharparenleft}{\kern0pt}fibered{\isacharunderscore}{\kern0pt}product{\isacharunderscore}{\kern0pt}right{\isacharunderscore}{\kern0pt}proj\ X\ f\ f\ X{\isacharparenright}{\kern0pt}{\isachardoublequoteclose}\isanewline
\ \ \ \ \ \ \isacommand{unfolding}\isamarkupfalse%
\ q{\isacharunderscore}{\kern0pt}def\ fibered{\isacharunderscore}{\kern0pt}product{\isacharunderscore}{\kern0pt}left{\isacharunderscore}{\kern0pt}proj{\isacharunderscore}{\kern0pt}def\ fibered{\isacharunderscore}{\kern0pt}product{\isacharunderscore}{\kern0pt}right{\isacharunderscore}{\kern0pt}proj{\isacharunderscore}{\kern0pt}def\ E{\isacharunderscore}{\kern0pt}def\isanewline
\ \ \ \ \ \ \isacommand{using}\isamarkupfalse%
\ canonical{\isacharunderscore}{\kern0pt}quotient{\isacharunderscore}{\kern0pt}map{\isacharunderscore}{\kern0pt}is{\isacharunderscore}{\kern0pt}coequalizer\ f{\isacharunderscore}{\kern0pt}type\ kernel{\isacharunderscore}{\kern0pt}pair{\isacharunderscore}{\kern0pt}equiv{\isacharunderscore}{\kern0pt}rel\ \isacommand{by}\isamarkupfalse%
\ auto\ \isanewline
\ \ \ \ \isacommand{then}\isamarkupfalse%
\ \isacommand{have}\isamarkupfalse%
\ q{\isacharunderscore}{\kern0pt}epi{\isacharcolon}{\kern0pt}\ {\isachardoublequoteopen}epimorphism\ q{\isachardoublequoteclose}\isanewline
\ \ \ \ \ \ \isacommand{using}\isamarkupfalse%
\ coequalizer{\isacharunderscore}{\kern0pt}is{\isacharunderscore}{\kern0pt}epimorphism\ \isacommand{by}\isamarkupfalse%
\ auto\ \isanewline
\isanewline
\ \ \ \ \isacommand{show}\isamarkupfalse%
\ m{\isacharunderscore}{\kern0pt}mono{\isacharcolon}{\kern0pt}\ {\isachardoublequoteopen}monomorphism\ m{\isachardoublequoteclose}\isanewline
\ \ \ \ \isacommand{proof}\isamarkupfalse%
\ {\isacharminus}{\kern0pt}\isanewline
\ \ \ \ \ \ \isacommand{thm}\isamarkupfalse%
\ kernel{\isacharunderscore}{\kern0pt}pair{\isacharunderscore}{\kern0pt}connection{\isacharbrackleft}{\kern0pt}\isakeyword{where}\ E{\isacharequal}{\kern0pt}E{\isacharcomma}{\kern0pt}\isakeyword{where}\ X{\isacharequal}{\kern0pt}X{\isacharcomma}{\kern0pt}\ \isakeyword{where}\ h{\isacharequal}{\kern0pt}m{\isacharcomma}{\kern0pt}\ \isakeyword{where}\ f{\isacharequal}{\kern0pt}f{\isacharcomma}{\kern0pt}\ \isakeyword{where}\ g{\isacharequal}{\kern0pt}q{\isacharcomma}{\kern0pt}\ \isakeyword{where}\ Y{\isacharequal}{\kern0pt}Y{\isacharbrackright}{\kern0pt}\isanewline
\ \ \ \ \ \ \isacommand{have}\isamarkupfalse%
\ q{\isacharunderscore}{\kern0pt}eq{\isacharcolon}{\kern0pt}\ {\isachardoublequoteopen}q\ {\isasymcirc}\isactrlsub c\ fibered{\isacharunderscore}{\kern0pt}product{\isacharunderscore}{\kern0pt}left{\isacharunderscore}{\kern0pt}proj\ X\ f\ f\ X\ {\isacharequal}{\kern0pt}\ q\ {\isasymcirc}\isactrlsub c\ fibered{\isacharunderscore}{\kern0pt}product{\isacharunderscore}{\kern0pt}right{\isacharunderscore}{\kern0pt}proj\ X\ f\ f\ X{\isachardoublequoteclose}\isanewline
\ \ \ \ \ \ \ \ \isacommand{using}\isamarkupfalse%
\ canonical{\isacharunderscore}{\kern0pt}quotient{\isacharunderscore}{\kern0pt}map{\isacharunderscore}{\kern0pt}is{\isacharunderscore}{\kern0pt}coequalizer\ coequalizer{\isacharunderscore}{\kern0pt}def\ f{\isacharunderscore}{\kern0pt}type\ fibered{\isacharunderscore}{\kern0pt}product{\isacharunderscore}{\kern0pt}left{\isacharunderscore}{\kern0pt}proj{\isacharunderscore}{\kern0pt}def\ fibered{\isacharunderscore}{\kern0pt}product{\isacharunderscore}{\kern0pt}right{\isacharunderscore}{\kern0pt}proj{\isacharunderscore}{\kern0pt}def\ kernel{\isacharunderscore}{\kern0pt}pair{\isacharunderscore}{\kern0pt}equiv{\isacharunderscore}{\kern0pt}rel\ q{\isacharunderscore}{\kern0pt}def\ \isacommand{by}\isamarkupfalse%
\ fastforce\isanewline
\ \ \ \ \ \ \isacommand{then}\isamarkupfalse%
\ \isacommand{have}\isamarkupfalse%
\ {\isachardoublequoteopen}{\isasymexists}{\isacharbang}{\kern0pt}b{\isachardot}{\kern0pt}\ b\ {\isacharcolon}{\kern0pt}\ X\ \isactrlbsub f\isactrlesub {\isasymtimes}\isactrlsub c\isactrlbsub f\isactrlesub \ X\ {\isasymrightarrow}\ E\ \isactrlbsub m\isactrlesub {\isasymtimes}\isactrlsub c\isactrlbsub m\isactrlesub \ E\ {\isasymand}\isanewline
\ \ \ \ \ \ \ \ fibered{\isacharunderscore}{\kern0pt}product{\isacharunderscore}{\kern0pt}left{\isacharunderscore}{\kern0pt}proj\ E\ m\ m\ E\ {\isasymcirc}\isactrlsub c\ b\ {\isacharequal}{\kern0pt}\ q\ {\isasymcirc}\isactrlsub c\ fibered{\isacharunderscore}{\kern0pt}product{\isacharunderscore}{\kern0pt}left{\isacharunderscore}{\kern0pt}proj\ X\ f\ f\ X\ {\isasymand}\isanewline
\ \ \ \ \ \ \ \ fibered{\isacharunderscore}{\kern0pt}product{\isacharunderscore}{\kern0pt}right{\isacharunderscore}{\kern0pt}proj\ E\ m\ m\ E\ {\isasymcirc}\isactrlsub c\ b\ {\isacharequal}{\kern0pt}\ q\ {\isasymcirc}\isactrlsub c\ fibered{\isacharunderscore}{\kern0pt}product{\isacharunderscore}{\kern0pt}right{\isacharunderscore}{\kern0pt}proj\ X\ f\ f\ X\ {\isasymand}\isanewline
\ \ \ \ \ \ \ \ epimorphism\ b{\isachardoublequoteclose}\isanewline
\ \ \ \ \ \ \ \ \isacommand{by}\isamarkupfalse%
\ {\isacharparenleft}{\kern0pt}typecheck{\isacharunderscore}{\kern0pt}cfuncs{\isacharcomma}{\kern0pt}\ rule{\isacharunderscore}{\kern0pt}tac\ kernel{\isacharunderscore}{\kern0pt}pair{\isacharunderscore}{\kern0pt}connection{\isacharbrackleft}{\kern0pt}\isakeyword{where}\ Y{\isacharequal}{\kern0pt}Y{\isacharbrackright}{\kern0pt}{\isacharcomma}{\kern0pt}\isanewline
\ \ \ \ \ \ \ \ \ \ \ \ simp{\isacharunderscore}{\kern0pt}all\ add{\isacharcolon}{\kern0pt}\ q{\isacharunderscore}{\kern0pt}epi{\isacharcomma}{\kern0pt}\ metis\ f{\isacharunderscore}{\kern0pt}const\ kernel{\isacharunderscore}{\kern0pt}pair{\isacharunderscore}{\kern0pt}equiv{\isacharunderscore}{\kern0pt}rel\ m{\isacharunderscore}{\kern0pt}def\ q{\isacharunderscore}{\kern0pt}def\ quotient{\isacharunderscore}{\kern0pt}func{\isacharunderscore}{\kern0pt}eq{\isacharparenright}{\kern0pt}\isanewline
\ \ \ \ \ \ \isacommand{then}\isamarkupfalse%
\ \isacommand{obtain}\isamarkupfalse%
\ b\ \isakeyword{where}\ b{\isacharunderscore}{\kern0pt}type{\isacharbrackleft}{\kern0pt}type{\isacharunderscore}{\kern0pt}rule{\isacharbrackright}{\kern0pt}{\isacharcolon}{\kern0pt}\ {\isachardoublequoteopen}b\ {\isacharcolon}{\kern0pt}\ X\ \isactrlbsub f\isactrlesub {\isasymtimes}\isactrlsub c\isactrlbsub f\isactrlesub \ X\ {\isasymrightarrow}\ E\ \isactrlbsub m\isactrlesub {\isasymtimes}\isactrlsub c\isactrlbsub m\isactrlesub \ E{\isachardoublequoteclose}\ \isakeyword{and}\isanewline
\ \ \ \ \ \ \ \ b{\isacharunderscore}{\kern0pt}left{\isacharunderscore}{\kern0pt}eq{\isacharcolon}{\kern0pt}\ {\isachardoublequoteopen}fibered{\isacharunderscore}{\kern0pt}product{\isacharunderscore}{\kern0pt}left{\isacharunderscore}{\kern0pt}proj\ E\ m\ m\ E\ {\isasymcirc}\isactrlsub c\ b\ {\isacharequal}{\kern0pt}\ q\ {\isasymcirc}\isactrlsub c\ fibered{\isacharunderscore}{\kern0pt}product{\isacharunderscore}{\kern0pt}left{\isacharunderscore}{\kern0pt}proj\ X\ f\ f\ X{\isachardoublequoteclose}\ \isakeyword{and}\isanewline
\ \ \ \ \ \ \ \ b{\isacharunderscore}{\kern0pt}right{\isacharunderscore}{\kern0pt}eq{\isacharcolon}{\kern0pt}\ {\isachardoublequoteopen}fibered{\isacharunderscore}{\kern0pt}product{\isacharunderscore}{\kern0pt}right{\isacharunderscore}{\kern0pt}proj\ E\ m\ m\ E\ {\isasymcirc}\isactrlsub c\ b\ {\isacharequal}{\kern0pt}\ q\ {\isasymcirc}\isactrlsub c\ fibered{\isacharunderscore}{\kern0pt}product{\isacharunderscore}{\kern0pt}right{\isacharunderscore}{\kern0pt}proj\ X\ f\ f\ X{\isachardoublequoteclose}\ \isakeyword{and}\isanewline
\ \ \ \ \ \ \ \ b{\isacharunderscore}{\kern0pt}epi{\isacharcolon}{\kern0pt}\ {\isachardoublequoteopen}epimorphism\ b{\isachardoublequoteclose}\isanewline
\ \ \ \ \ \ \ \ \isacommand{by}\isamarkupfalse%
\ auto\isanewline
\isanewline
\ \ \ \ \ \ \isacommand{have}\isamarkupfalse%
\ {\isachardoublequoteopen}fibered{\isacharunderscore}{\kern0pt}product{\isacharunderscore}{\kern0pt}left{\isacharunderscore}{\kern0pt}proj\ E\ m\ m\ E\ {\isasymcirc}\isactrlsub c\ b\ {\isacharequal}{\kern0pt}\ fibered{\isacharunderscore}{\kern0pt}product{\isacharunderscore}{\kern0pt}right{\isacharunderscore}{\kern0pt}proj\ E\ m\ m\ E\ {\isasymcirc}\isactrlsub c\ b{\isachardoublequoteclose}\isanewline
\ \ \ \ \ \ \ \ \isacommand{using}\isamarkupfalse%
\ b{\isacharunderscore}{\kern0pt}left{\isacharunderscore}{\kern0pt}eq\ b{\isacharunderscore}{\kern0pt}right{\isacharunderscore}{\kern0pt}eq\ q{\isacharunderscore}{\kern0pt}eq\ \isacommand{by}\isamarkupfalse%
\ force\isanewline
\ \ \ \ \ \ \isacommand{then}\isamarkupfalse%
\ \isacommand{have}\isamarkupfalse%
\ {\isachardoublequoteopen}fibered{\isacharunderscore}{\kern0pt}product{\isacharunderscore}{\kern0pt}left{\isacharunderscore}{\kern0pt}proj\ E\ m\ m\ E\ {\isacharequal}{\kern0pt}\ fibered{\isacharunderscore}{\kern0pt}product{\isacharunderscore}{\kern0pt}right{\isacharunderscore}{\kern0pt}proj\ E\ m\ m\ E{\isachardoublequoteclose}\isanewline
\ \ \ \ \ \ \ \ \isacommand{using}\isamarkupfalse%
\ b{\isacharunderscore}{\kern0pt}epi\ cfunc{\isacharunderscore}{\kern0pt}type{\isacharunderscore}{\kern0pt}def\ epimorphism{\isacharunderscore}{\kern0pt}def\ \isacommand{by}\isamarkupfalse%
\ {\isacharparenleft}{\kern0pt}typecheck{\isacharunderscore}{\kern0pt}cfuncs{\isacharunderscore}{\kern0pt}prems{\isacharcomma}{\kern0pt}\ auto{\isacharparenright}{\kern0pt}\isanewline
\ \ \ \ \ \ \isacommand{then}\isamarkupfalse%
\ \isacommand{show}\isamarkupfalse%
\ {\isachardoublequoteopen}monomorphism\ m{\isachardoublequoteclose}\isanewline
\ \ \ \ \ \ \ \ \isacommand{using}\isamarkupfalse%
\ kern{\isacharunderscore}{\kern0pt}pair{\isacharunderscore}{\kern0pt}proj{\isacharunderscore}{\kern0pt}iso{\isacharunderscore}{\kern0pt}TFAE{\isadigit{2}}\ m{\isacharunderscore}{\kern0pt}type\ \isacommand{by}\isamarkupfalse%
\ auto\isanewline
\ \ \ \ \isacommand{qed}\isamarkupfalse%
\isanewline
\isanewline
\ \ \ \ \isacommand{show}\isamarkupfalse%
\ f{\isacharunderscore}{\kern0pt}eq{\isacharunderscore}{\kern0pt}m{\isacharunderscore}{\kern0pt}q{\isacharcolon}{\kern0pt}\ {\isachardoublequoteopen}f\ {\isacharequal}{\kern0pt}\ m\ {\isasymcirc}\isactrlsub c\ q{\isachardoublequoteclose}\isanewline
\ \ \ \ \ \ \isacommand{using}\isamarkupfalse%
\ f{\isacharunderscore}{\kern0pt}const\ f{\isacharunderscore}{\kern0pt}type\ kernel{\isacharunderscore}{\kern0pt}pair{\isacharunderscore}{\kern0pt}equiv{\isacharunderscore}{\kern0pt}rel\ m{\isacharunderscore}{\kern0pt}def\ q{\isacharunderscore}{\kern0pt}def\ quotient{\isacharunderscore}{\kern0pt}func{\isacharunderscore}{\kern0pt}eq\ \isacommand{by}\isamarkupfalse%
\ fastforce\isanewline
\isanewline
\ \ \ \ \isacommand{show}\isamarkupfalse%
\ {\isachardoublequoteopen}{\isasymAnd}x{\isachardot}{\kern0pt}\ x\ {\isacharcolon}{\kern0pt}\ E\ {\isasymrightarrow}\ Y\ {\isasymLongrightarrow}\ f\ {\isacharequal}{\kern0pt}\ x\ {\isasymcirc}\isactrlsub c\ q\ {\isasymLongrightarrow}\ x\ {\isacharequal}{\kern0pt}\ m{\isachardoublequoteclose}\isanewline
\ \ \ \ \isacommand{proof}\isamarkupfalse%
\ {\isacharminus}{\kern0pt}\isanewline
\ \ \ \ \ \ \isacommand{fix}\isamarkupfalse%
\ x\isanewline
\ \ \ \ \ \ \isacommand{assume}\isamarkupfalse%
\ x{\isacharunderscore}{\kern0pt}type{\isacharbrackleft}{\kern0pt}type{\isacharunderscore}{\kern0pt}rule{\isacharbrackright}{\kern0pt}{\isacharcolon}{\kern0pt}\ {\isachardoublequoteopen}x\ {\isacharcolon}{\kern0pt}\ E\ {\isasymrightarrow}\ Y{\isachardoublequoteclose}\isanewline
\ \ \ \ \ \ \isacommand{assume}\isamarkupfalse%
\ f{\isacharunderscore}{\kern0pt}eq{\isacharunderscore}{\kern0pt}x{\isacharunderscore}{\kern0pt}q{\isacharcolon}{\kern0pt}\ {\isachardoublequoteopen}f\ {\isacharequal}{\kern0pt}\ x\ {\isasymcirc}\isactrlsub c\ q{\isachardoublequoteclose}\isanewline
\ \ \ \ \ \ \isacommand{have}\isamarkupfalse%
\ {\isachardoublequoteopen}x\ {\isasymcirc}\isactrlsub c\ q\ {\isacharequal}{\kern0pt}\ m\ {\isasymcirc}\isactrlsub c\ q{\isachardoublequoteclose}\isanewline
\ \ \ \ \ \ \ \ \isacommand{using}\isamarkupfalse%
\ f{\isacharunderscore}{\kern0pt}eq{\isacharunderscore}{\kern0pt}m{\isacharunderscore}{\kern0pt}q\ f{\isacharunderscore}{\kern0pt}eq{\isacharunderscore}{\kern0pt}x{\isacharunderscore}{\kern0pt}q\ \isacommand{by}\isamarkupfalse%
\ auto\isanewline
\ \ \ \ \ \ \isacommand{then}\isamarkupfalse%
\ \isacommand{show}\isamarkupfalse%
\ {\isachardoublequoteopen}x\ {\isacharequal}{\kern0pt}\ m{\isachardoublequoteclose}\isanewline
\ \ \ \ \ \ \ \ \isacommand{using}\isamarkupfalse%
\ epimorphism{\isacharunderscore}{\kern0pt}def{\isadigit{2}}\ m{\isacharunderscore}{\kern0pt}type\ q{\isacharunderscore}{\kern0pt}epi\ q{\isacharunderscore}{\kern0pt}type\ x{\isacharunderscore}{\kern0pt}type\ \isacommand{by}\isamarkupfalse%
\ blast\isanewline
\ \ \ \ \isacommand{qed}\isamarkupfalse%
\isanewline
\ \ \isacommand{qed}\isamarkupfalse%
\isanewline
\isacommand{qed}\isamarkupfalse%
%
\endisatagproof
{\isafoldproof}%
%
\isadelimproof
\isanewline
%
\endisadelimproof
\isanewline
\isacommand{lemma}\isamarkupfalse%
\ epi{\isacharunderscore}{\kern0pt}monic{\isacharunderscore}{\kern0pt}factorization{\isadigit{2}}{\isacharcolon}{\kern0pt}\isanewline
\ \ \isakeyword{assumes}\ f{\isacharunderscore}{\kern0pt}type{\isacharbrackleft}{\kern0pt}type{\isacharunderscore}{\kern0pt}rule{\isacharbrackright}{\kern0pt}{\isacharcolon}{\kern0pt}\ {\isachardoublequoteopen}f\ {\isacharcolon}{\kern0pt}\ X\ {\isasymrightarrow}\ Y{\isachardoublequoteclose}\isanewline
\ \ \isakeyword{shows}\ {\isachardoublequoteopen}{\isasymexists}\ g\ m\ E{\isachardot}{\kern0pt}\ g\ {\isacharcolon}{\kern0pt}\ X\ {\isasymrightarrow}\ E\ {\isasymand}\ m\ {\isacharcolon}{\kern0pt}\ E\ {\isasymrightarrow}\ Y\ \isanewline
\ \ \ \ {\isasymand}\ epimorphism\ g\ {\isasymand}\ monomorphism\ m\ {\isasymand}\ f\ {\isacharequal}{\kern0pt}\ m\ {\isasymcirc}\isactrlsub c\ g\isanewline
\ \ \ \ {\isasymand}\ {\isacharparenleft}{\kern0pt}{\isasymforall}x{\isachardot}{\kern0pt}\ x\ {\isacharcolon}{\kern0pt}\ E\ {\isasymrightarrow}\ Y\ {\isasymlongrightarrow}\ f\ {\isacharequal}{\kern0pt}\ x\ {\isasymcirc}\isactrlsub c\ g\ {\isasymlongrightarrow}\ x\ {\isacharequal}{\kern0pt}\ m{\isacharparenright}{\kern0pt}{\isachardoublequoteclose}\isanewline
%
\isadelimproof
\ \ %
\endisadelimproof
%
\isatagproof
\isacommand{using}\isamarkupfalse%
\ epi{\isacharunderscore}{\kern0pt}monic{\isacharunderscore}{\kern0pt}factorization\ coequalizer{\isacharunderscore}{\kern0pt}is{\isacharunderscore}{\kern0pt}epimorphism\ \isacommand{by}\isamarkupfalse%
\ {\isacharparenleft}{\kern0pt}meson\ f{\isacharunderscore}{\kern0pt}type{\isacharparenright}{\kern0pt}%
\endisatagproof
{\isafoldproof}%
%
\isadelimproof
%
\endisadelimproof
%
\isadelimdocument
%
\endisadelimdocument
%
\isatagdocument
%
\isamarkupsection{Image of a Function%
}
\isamarkuptrue%
%
\endisatagdocument
{\isafolddocument}%
%
\isadelimdocument
%
\endisadelimdocument
%
\begin{isamarkuptext}%
The definition below corresponds to Definition 2.3.7 in Halvorson.%
\end{isamarkuptext}\isamarkuptrue%
\isacommand{definition}\isamarkupfalse%
\ image{\isacharunderscore}{\kern0pt}of\ {\isacharcolon}{\kern0pt}{\isacharcolon}{\kern0pt}\ {\isachardoublequoteopen}cfunc\ {\isasymRightarrow}\ cset\ {\isasymRightarrow}\ cfunc\ {\isasymRightarrow}\ cset{\isachardoublequoteclose}\ {\isacharparenleft}{\kern0pt}{\isachardoublequoteopen}{\isacharunderscore}{\kern0pt}{\isasymlparr}{\isacharunderscore}{\kern0pt}{\isasymrparr}\isactrlbsub {\isacharunderscore}{\kern0pt}\isactrlesub {\isachardoublequoteclose}\ {\isacharbrackleft}{\kern0pt}{\isadigit{1}}{\isadigit{0}}{\isadigit{1}}{\isacharcomma}{\kern0pt}{\isadigit{0}}{\isacharcomma}{\kern0pt}{\isadigit{0}}{\isacharbrackright}{\kern0pt}{\isadigit{1}}{\isadigit{0}}{\isadigit{0}}{\isacharparenright}{\kern0pt}\ \isakeyword{where}\isanewline
\ \ {\isachardoublequoteopen}image{\isacharunderscore}{\kern0pt}of\ f\ A\ n\ {\isacharequal}{\kern0pt}\ {\isacharparenleft}{\kern0pt}SOME\ fA{\isachardot}{\kern0pt}\ {\isasymexists}g\ m{\isachardot}{\kern0pt}\isanewline
\ \ \ g\ {\isacharcolon}{\kern0pt}\ A\ {\isasymrightarrow}\ fA\ {\isasymand}\isanewline
\ \ \ m\ {\isacharcolon}{\kern0pt}\ fA\ {\isasymrightarrow}\ codomain\ f\ {\isasymand}\isanewline
\ \ \ coequalizer\ fA\ g\ {\isacharparenleft}{\kern0pt}fibered{\isacharunderscore}{\kern0pt}product{\isacharunderscore}{\kern0pt}left{\isacharunderscore}{\kern0pt}proj\ A\ {\isacharparenleft}{\kern0pt}f\ {\isasymcirc}\isactrlsub c\ n{\isacharparenright}{\kern0pt}\ {\isacharparenleft}{\kern0pt}f\ {\isasymcirc}\isactrlsub c\ n{\isacharparenright}{\kern0pt}\ A{\isacharparenright}{\kern0pt}\ {\isacharparenleft}{\kern0pt}fibered{\isacharunderscore}{\kern0pt}product{\isacharunderscore}{\kern0pt}right{\isacharunderscore}{\kern0pt}proj\ A\ {\isacharparenleft}{\kern0pt}f\ {\isasymcirc}\isactrlsub c\ n{\isacharparenright}{\kern0pt}\ {\isacharparenleft}{\kern0pt}f\ {\isasymcirc}\isactrlsub c\ n{\isacharparenright}{\kern0pt}\ A{\isacharparenright}{\kern0pt}\ {\isasymand}\isanewline
\ \ \ monomorphism\ m\ {\isasymand}\ f\ {\isasymcirc}\isactrlsub c\ n\ {\isacharequal}{\kern0pt}\ m\ {\isasymcirc}\isactrlsub c\ g\ {\isasymand}\ {\isacharparenleft}{\kern0pt}{\isasymforall}x{\isachardot}{\kern0pt}\ x\ {\isacharcolon}{\kern0pt}\ fA\ {\isasymrightarrow}\ codomain\ f\ {\isasymlongrightarrow}\ f\ {\isasymcirc}\isactrlsub c\ n\ {\isacharequal}{\kern0pt}\ x\ {\isasymcirc}\isactrlsub c\ g\ {\isasymlongrightarrow}\ x\ {\isacharequal}{\kern0pt}\ m{\isacharparenright}{\kern0pt}{\isacharparenright}{\kern0pt}{\isachardoublequoteclose}\isanewline
\isanewline
\isacommand{lemma}\isamarkupfalse%
\ image{\isacharunderscore}{\kern0pt}of{\isacharunderscore}{\kern0pt}def{\isadigit{2}}{\isacharcolon}{\kern0pt}\isanewline
\ \ \isakeyword{assumes}\ {\isachardoublequoteopen}f\ {\isacharcolon}{\kern0pt}\ X\ {\isasymrightarrow}\ Y{\isachardoublequoteclose}\ {\isachardoublequoteopen}n\ {\isacharcolon}{\kern0pt}\ A\ {\isasymrightarrow}\ X{\isachardoublequoteclose}\isanewline
\ \ \isakeyword{shows}\ {\isachardoublequoteopen}{\isasymexists}g\ m{\isachardot}{\kern0pt}\isanewline
\ \ \ \ g\ {\isacharcolon}{\kern0pt}\ A\ {\isasymrightarrow}\ f{\isasymlparr}A{\isasymrparr}\isactrlbsub n\isactrlesub \ {\isasymand}\isanewline
\ \ \ \ m\ {\isacharcolon}{\kern0pt}\ f{\isasymlparr}A{\isasymrparr}\isactrlbsub n\isactrlesub \ {\isasymrightarrow}\ Y\ {\isasymand}\isanewline
\ \ \ \ coequalizer\ {\isacharparenleft}{\kern0pt}f{\isasymlparr}A{\isasymrparr}\isactrlbsub n\isactrlesub {\isacharparenright}{\kern0pt}\ g\ {\isacharparenleft}{\kern0pt}fibered{\isacharunderscore}{\kern0pt}product{\isacharunderscore}{\kern0pt}left{\isacharunderscore}{\kern0pt}proj\ A\ {\isacharparenleft}{\kern0pt}f\ {\isasymcirc}\isactrlsub c\ n{\isacharparenright}{\kern0pt}\ {\isacharparenleft}{\kern0pt}f\ {\isasymcirc}\isactrlsub c\ n{\isacharparenright}{\kern0pt}\ A{\isacharparenright}{\kern0pt}\ {\isacharparenleft}{\kern0pt}fibered{\isacharunderscore}{\kern0pt}product{\isacharunderscore}{\kern0pt}right{\isacharunderscore}{\kern0pt}proj\ A\ {\isacharparenleft}{\kern0pt}f\ {\isasymcirc}\isactrlsub c\ n{\isacharparenright}{\kern0pt}\ {\isacharparenleft}{\kern0pt}f\ {\isasymcirc}\isactrlsub c\ n{\isacharparenright}{\kern0pt}\ A{\isacharparenright}{\kern0pt}\ {\isasymand}\isanewline
\ \ \ \ monomorphism\ m\ {\isasymand}\ f\ {\isasymcirc}\isactrlsub c\ n\ {\isacharequal}{\kern0pt}\ m\ {\isasymcirc}\isactrlsub c\ g\ {\isasymand}\ {\isacharparenleft}{\kern0pt}{\isasymforall}x{\isachardot}{\kern0pt}\ x\ {\isacharcolon}{\kern0pt}\ f{\isasymlparr}A{\isasymrparr}\isactrlbsub n\isactrlesub \ {\isasymrightarrow}\ Y\ {\isasymlongrightarrow}\ f\ {\isasymcirc}\isactrlsub c\ n\ {\isacharequal}{\kern0pt}\ x\ {\isasymcirc}\isactrlsub c\ g\ {\isasymlongrightarrow}\ x\ {\isacharequal}{\kern0pt}\ m{\isacharparenright}{\kern0pt}{\isachardoublequoteclose}\isanewline
%
\isadelimproof
%
\endisadelimproof
%
\isatagproof
\isacommand{proof}\isamarkupfalse%
\ {\isacharminus}{\kern0pt}\isanewline
\ \ \isacommand{have}\isamarkupfalse%
\ {\isachardoublequoteopen}{\isasymexists}g\ m{\isachardot}{\kern0pt}\isanewline
\ \ \ \ g\ {\isacharcolon}{\kern0pt}\ A\ {\isasymrightarrow}\ f{\isasymlparr}A{\isasymrparr}\isactrlbsub n\isactrlesub \ {\isasymand}\isanewline
\ \ \ \ m\ {\isacharcolon}{\kern0pt}\ f{\isasymlparr}A{\isasymrparr}\isactrlbsub n\isactrlesub \ {\isasymrightarrow}\ codomain\ f\ {\isasymand}\isanewline
\ \ \ \ coequalizer\ {\isacharparenleft}{\kern0pt}f{\isasymlparr}A{\isasymrparr}\isactrlbsub n\isactrlesub {\isacharparenright}{\kern0pt}\ g\ {\isacharparenleft}{\kern0pt}fibered{\isacharunderscore}{\kern0pt}product{\isacharunderscore}{\kern0pt}left{\isacharunderscore}{\kern0pt}proj\ A\ {\isacharparenleft}{\kern0pt}f\ {\isasymcirc}\isactrlsub c\ n{\isacharparenright}{\kern0pt}\ {\isacharparenleft}{\kern0pt}f\ {\isasymcirc}\isactrlsub c\ n{\isacharparenright}{\kern0pt}\ A{\isacharparenright}{\kern0pt}\ {\isacharparenleft}{\kern0pt}fibered{\isacharunderscore}{\kern0pt}product{\isacharunderscore}{\kern0pt}right{\isacharunderscore}{\kern0pt}proj\ A\ {\isacharparenleft}{\kern0pt}f\ {\isasymcirc}\isactrlsub c\ n{\isacharparenright}{\kern0pt}\ {\isacharparenleft}{\kern0pt}f\ {\isasymcirc}\isactrlsub c\ n{\isacharparenright}{\kern0pt}\ A{\isacharparenright}{\kern0pt}\ {\isasymand}\isanewline
\ \ \ \ monomorphism\ m\ {\isasymand}\ f\ {\isasymcirc}\isactrlsub c\ n\ {\isacharequal}{\kern0pt}\ m\ {\isasymcirc}\isactrlsub c\ g\ {\isasymand}\ {\isacharparenleft}{\kern0pt}{\isasymforall}x{\isachardot}{\kern0pt}\ x\ {\isacharcolon}{\kern0pt}\ f{\isasymlparr}A{\isasymrparr}\isactrlbsub n\isactrlesub \ {\isasymrightarrow}\ codomain\ f\ {\isasymlongrightarrow}\ f\ {\isasymcirc}\isactrlsub c\ n\ {\isacharequal}{\kern0pt}\ x\ {\isasymcirc}\isactrlsub c\ g\ {\isasymlongrightarrow}\ x\ {\isacharequal}{\kern0pt}\ m{\isacharparenright}{\kern0pt}{\isachardoublequoteclose}\isanewline
\ \ \ \ \isacommand{using}\isamarkupfalse%
\ assms\ cfunc{\isacharunderscore}{\kern0pt}type{\isacharunderscore}{\kern0pt}def\ comp{\isacharunderscore}{\kern0pt}type\ epi{\isacharunderscore}{\kern0pt}monic{\isacharunderscore}{\kern0pt}factorization{\isacharbrackleft}{\kern0pt}\isakeyword{where}\ f{\isacharequal}{\kern0pt}{\isachardoublequoteopen}f\ {\isasymcirc}\isactrlsub c\ n{\isachardoublequoteclose}{\isacharcomma}{\kern0pt}\ \isakeyword{where}\ X{\isacharequal}{\kern0pt}A{\isacharcomma}{\kern0pt}\ \isakeyword{where}\ Y{\isacharequal}{\kern0pt}{\isachardoublequoteopen}codomain\ f{\isachardoublequoteclose}{\isacharbrackright}{\kern0pt}\ \isanewline
\ \ \ \ \isacommand{by}\isamarkupfalse%
\ {\isacharparenleft}{\kern0pt}unfold\ image{\isacharunderscore}{\kern0pt}of{\isacharunderscore}{\kern0pt}def{\isacharcomma}{\kern0pt}\ rule{\isacharunderscore}{\kern0pt}tac\ someI{\isacharunderscore}{\kern0pt}ex{\isacharcomma}{\kern0pt}\ auto{\isacharparenright}{\kern0pt}\isanewline
\ \ \isacommand{then}\isamarkupfalse%
\ \isacommand{show}\isamarkupfalse%
\ {\isacharquery}{\kern0pt}thesis\isanewline
\ \ \ \ \isacommand{using}\isamarkupfalse%
\ assms{\isacharparenleft}{\kern0pt}{\isadigit{1}}{\isacharparenright}{\kern0pt}\ cfunc{\isacharunderscore}{\kern0pt}type{\isacharunderscore}{\kern0pt}def\ \isacommand{by}\isamarkupfalse%
\ auto\isanewline
\isacommand{qed}\isamarkupfalse%
%
\endisatagproof
{\isafoldproof}%
%
\isadelimproof
\isanewline
%
\endisadelimproof
\isanewline
\isacommand{definition}\isamarkupfalse%
\ image{\isacharunderscore}{\kern0pt}restriction{\isacharunderscore}{\kern0pt}mapping\ {\isacharcolon}{\kern0pt}{\isacharcolon}{\kern0pt}\ {\isachardoublequoteopen}cfunc\ {\isasymRightarrow}\ cset\ {\isasymtimes}\ cfunc\ {\isasymRightarrow}\ cfunc{\isachardoublequoteclose}\ {\isacharparenleft}{\kern0pt}{\isachardoublequoteopen}{\isacharunderscore}{\kern0pt}{\isasymrestriction}\isactrlbsub {\isacharunderscore}{\kern0pt}\isactrlesub {\isachardoublequoteclose}\ {\isacharbrackleft}{\kern0pt}{\isadigit{1}}{\isadigit{0}}{\isadigit{1}}{\isacharcomma}{\kern0pt}{\isadigit{0}}{\isacharbrackright}{\kern0pt}{\isadigit{1}}{\isadigit{0}}{\isadigit{0}}{\isacharparenright}{\kern0pt}\ \isakeyword{where}\isanewline
\ \ {\isachardoublequoteopen}image{\isacharunderscore}{\kern0pt}restriction{\isacharunderscore}{\kern0pt}mapping\ f\ An\ {\isacharequal}{\kern0pt}\ {\isacharparenleft}{\kern0pt}SOME\ g{\isachardot}{\kern0pt}\ {\isasymexists}\ m{\isachardot}{\kern0pt}\ g\ {\isacharcolon}{\kern0pt}\ fst\ An\ {\isasymrightarrow}\ f{\isasymlparr}fst\ An{\isasymrparr}\isactrlbsub snd\ An\isactrlesub \ {\isasymand}\ m\ {\isacharcolon}{\kern0pt}\ f{\isasymlparr}fst\ An{\isasymrparr}\isactrlbsub snd\ An\isactrlesub \ {\isasymrightarrow}\ codomain\ f\ {\isasymand}\isanewline
\ \ \ \ coequalizer\ {\isacharparenleft}{\kern0pt}f{\isasymlparr}fst\ An{\isasymrparr}\isactrlbsub snd\ An\isactrlesub {\isacharparenright}{\kern0pt}\ g\ {\isacharparenleft}{\kern0pt}fibered{\isacharunderscore}{\kern0pt}product{\isacharunderscore}{\kern0pt}left{\isacharunderscore}{\kern0pt}proj\ {\isacharparenleft}{\kern0pt}fst\ An{\isacharparenright}{\kern0pt}\ {\isacharparenleft}{\kern0pt}f\ {\isasymcirc}\isactrlsub c\ snd\ An{\isacharparenright}{\kern0pt}\ {\isacharparenleft}{\kern0pt}f\ {\isasymcirc}\isactrlsub c\ snd\ An{\isacharparenright}{\kern0pt}\ {\isacharparenleft}{\kern0pt}fst\ An{\isacharparenright}{\kern0pt}{\isacharparenright}{\kern0pt}\ {\isacharparenleft}{\kern0pt}fibered{\isacharunderscore}{\kern0pt}product{\isacharunderscore}{\kern0pt}right{\isacharunderscore}{\kern0pt}proj\ {\isacharparenleft}{\kern0pt}fst\ An{\isacharparenright}{\kern0pt}\ {\isacharparenleft}{\kern0pt}f\ {\isasymcirc}\isactrlsub c\ snd\ An{\isacharparenright}{\kern0pt}\ {\isacharparenleft}{\kern0pt}f\ {\isasymcirc}\isactrlsub c\ snd\ An{\isacharparenright}{\kern0pt}\ {\isacharparenleft}{\kern0pt}fst\ An{\isacharparenright}{\kern0pt}{\isacharparenright}{\kern0pt}\ {\isasymand}\isanewline
\ \ \ \ monomorphism\ m\ {\isasymand}\ f\ {\isasymcirc}\isactrlsub c\ snd\ An\ {\isacharequal}{\kern0pt}\ m\ {\isasymcirc}\isactrlsub c\ g\ {\isasymand}\ {\isacharparenleft}{\kern0pt}{\isasymforall}x{\isachardot}{\kern0pt}\ x\ {\isacharcolon}{\kern0pt}\ f{\isasymlparr}fst\ An{\isasymrparr}\isactrlbsub snd\ An\isactrlesub \ {\isasymrightarrow}\ codomain\ f\ {\isasymlongrightarrow}\ f\ {\isasymcirc}\isactrlsub c\ snd\ An\ {\isacharequal}{\kern0pt}\ x\ {\isasymcirc}\isactrlsub c\ g\ {\isasymlongrightarrow}\ x\ {\isacharequal}{\kern0pt}\ m{\isacharparenright}{\kern0pt}{\isacharparenright}{\kern0pt}{\isachardoublequoteclose}\isanewline
\isanewline
\isacommand{lemma}\isamarkupfalse%
\ image{\isacharunderscore}{\kern0pt}restriction{\isacharunderscore}{\kern0pt}mapping{\isacharunderscore}{\kern0pt}def{\isadigit{2}}{\isacharcolon}{\kern0pt}\isanewline
\ \ \isakeyword{assumes}\ {\isachardoublequoteopen}f\ {\isacharcolon}{\kern0pt}\ X\ {\isasymrightarrow}\ Y{\isachardoublequoteclose}\ {\isachardoublequoteopen}n\ {\isacharcolon}{\kern0pt}\ A\ {\isasymrightarrow}\ X{\isachardoublequoteclose}\isanewline
\ \ \isakeyword{shows}\ {\isachardoublequoteopen}{\isasymexists}\ m{\isachardot}{\kern0pt}\ f{\isasymrestriction}\isactrlbsub {\isacharparenleft}{\kern0pt}A{\isacharcomma}{\kern0pt}\ n{\isacharparenright}{\kern0pt}\isactrlesub \ {\isacharcolon}{\kern0pt}\ A\ {\isasymrightarrow}\ f{\isasymlparr}A{\isasymrparr}\isactrlbsub n\isactrlesub \ {\isasymand}\ m\ {\isacharcolon}{\kern0pt}\ f{\isasymlparr}A{\isasymrparr}\isactrlbsub n\isactrlesub \ {\isasymrightarrow}\ Y\ {\isasymand}\isanewline
\ \ \ \ coequalizer\ {\isacharparenleft}{\kern0pt}f{\isasymlparr}A{\isasymrparr}\isactrlbsub n\isactrlesub {\isacharparenright}{\kern0pt}\ {\isacharparenleft}{\kern0pt}f{\isasymrestriction}\isactrlbsub {\isacharparenleft}{\kern0pt}A{\isacharcomma}{\kern0pt}\ n{\isacharparenright}{\kern0pt}\isactrlesub {\isacharparenright}{\kern0pt}\ {\isacharparenleft}{\kern0pt}fibered{\isacharunderscore}{\kern0pt}product{\isacharunderscore}{\kern0pt}left{\isacharunderscore}{\kern0pt}proj\ A\ {\isacharparenleft}{\kern0pt}f\ {\isasymcirc}\isactrlsub c\ n{\isacharparenright}{\kern0pt}\ {\isacharparenleft}{\kern0pt}f\ {\isasymcirc}\isactrlsub c\ n{\isacharparenright}{\kern0pt}\ A{\isacharparenright}{\kern0pt}\ {\isacharparenleft}{\kern0pt}fibered{\isacharunderscore}{\kern0pt}product{\isacharunderscore}{\kern0pt}right{\isacharunderscore}{\kern0pt}proj\ A\ {\isacharparenleft}{\kern0pt}f\ {\isasymcirc}\isactrlsub c\ n{\isacharparenright}{\kern0pt}\ {\isacharparenleft}{\kern0pt}f\ {\isasymcirc}\isactrlsub c\ n{\isacharparenright}{\kern0pt}\ A{\isacharparenright}{\kern0pt}\ {\isasymand}\isanewline
\ \ \ \ monomorphism\ m\ {\isasymand}\ f\ {\isasymcirc}\isactrlsub c\ n\ {\isacharequal}{\kern0pt}\ m\ {\isasymcirc}\isactrlsub c\ {\isacharparenleft}{\kern0pt}f{\isasymrestriction}\isactrlbsub {\isacharparenleft}{\kern0pt}A{\isacharcomma}{\kern0pt}\ n{\isacharparenright}{\kern0pt}\isactrlesub {\isacharparenright}{\kern0pt}\ {\isasymand}\ {\isacharparenleft}{\kern0pt}{\isasymforall}x{\isachardot}{\kern0pt}\ x\ {\isacharcolon}{\kern0pt}\ f{\isasymlparr}A{\isasymrparr}\isactrlbsub n\isactrlesub \ {\isasymrightarrow}\ Y\ {\isasymlongrightarrow}\ f\ {\isasymcirc}\isactrlsub c\ n\ {\isacharequal}{\kern0pt}\ x\ {\isasymcirc}\isactrlsub c\ {\isacharparenleft}{\kern0pt}f{\isasymrestriction}\isactrlbsub {\isacharparenleft}{\kern0pt}A{\isacharcomma}{\kern0pt}\ n{\isacharparenright}{\kern0pt}\isactrlesub {\isacharparenright}{\kern0pt}\ {\isasymlongrightarrow}\ x\ {\isacharequal}{\kern0pt}\ m{\isacharparenright}{\kern0pt}{\isachardoublequoteclose}\isanewline
%
\isadelimproof
%
\endisadelimproof
%
\isatagproof
\isacommand{proof}\isamarkupfalse%
\ {\isacharminus}{\kern0pt}\isanewline
\ \ \isacommand{have}\isamarkupfalse%
\ codom{\isacharunderscore}{\kern0pt}f{\isacharcolon}{\kern0pt}\ {\isachardoublequoteopen}codomain\ f\ {\isacharequal}{\kern0pt}\ Y{\isachardoublequoteclose}\isanewline
\ \ \ \ \isacommand{using}\isamarkupfalse%
\ assms{\isacharparenleft}{\kern0pt}{\isadigit{1}}{\isacharparenright}{\kern0pt}\ cfunc{\isacharunderscore}{\kern0pt}type{\isacharunderscore}{\kern0pt}def\ \isacommand{by}\isamarkupfalse%
\ auto\isanewline
\ \ \isacommand{have}\isamarkupfalse%
\ {\isachardoublequoteopen}{\isasymexists}\ m{\isachardot}{\kern0pt}\ f{\isasymrestriction}\isactrlbsub {\isacharparenleft}{\kern0pt}A{\isacharcomma}{\kern0pt}\ n{\isacharparenright}{\kern0pt}\isactrlesub \ {\isacharcolon}{\kern0pt}\ fst\ {\isacharparenleft}{\kern0pt}A{\isacharcomma}{\kern0pt}\ n{\isacharparenright}{\kern0pt}\ {\isasymrightarrow}\ f{\isasymlparr}fst\ {\isacharparenleft}{\kern0pt}A{\isacharcomma}{\kern0pt}\ n{\isacharparenright}{\kern0pt}{\isasymrparr}\isactrlbsub snd\ {\isacharparenleft}{\kern0pt}A{\isacharcomma}{\kern0pt}\ n{\isacharparenright}{\kern0pt}\isactrlesub \ {\isasymand}\ m\ {\isacharcolon}{\kern0pt}\ f{\isasymlparr}fst\ {\isacharparenleft}{\kern0pt}A{\isacharcomma}{\kern0pt}\ n{\isacharparenright}{\kern0pt}{\isasymrparr}\isactrlbsub snd\ {\isacharparenleft}{\kern0pt}A{\isacharcomma}{\kern0pt}\ n{\isacharparenright}{\kern0pt}\isactrlesub \ {\isasymrightarrow}\ codomain\ f\ {\isasymand}\isanewline
\ \ \ \ coequalizer\ {\isacharparenleft}{\kern0pt}f{\isasymlparr}fst\ {\isacharparenleft}{\kern0pt}A{\isacharcomma}{\kern0pt}\ n{\isacharparenright}{\kern0pt}{\isasymrparr}\isactrlbsub snd\ {\isacharparenleft}{\kern0pt}A{\isacharcomma}{\kern0pt}\ n{\isacharparenright}{\kern0pt}\isactrlesub {\isacharparenright}{\kern0pt}\ {\isacharparenleft}{\kern0pt}f{\isasymrestriction}\isactrlbsub {\isacharparenleft}{\kern0pt}A{\isacharcomma}{\kern0pt}\ n{\isacharparenright}{\kern0pt}\isactrlesub {\isacharparenright}{\kern0pt}\ {\isacharparenleft}{\kern0pt}fibered{\isacharunderscore}{\kern0pt}product{\isacharunderscore}{\kern0pt}left{\isacharunderscore}{\kern0pt}proj\ {\isacharparenleft}{\kern0pt}fst\ {\isacharparenleft}{\kern0pt}A{\isacharcomma}{\kern0pt}\ n{\isacharparenright}{\kern0pt}{\isacharparenright}{\kern0pt}\ {\isacharparenleft}{\kern0pt}f\ {\isasymcirc}\isactrlsub c\ snd\ {\isacharparenleft}{\kern0pt}A{\isacharcomma}{\kern0pt}\ n{\isacharparenright}{\kern0pt}{\isacharparenright}{\kern0pt}\ {\isacharparenleft}{\kern0pt}f\ {\isasymcirc}\isactrlsub c\ snd\ {\isacharparenleft}{\kern0pt}A{\isacharcomma}{\kern0pt}\ n{\isacharparenright}{\kern0pt}{\isacharparenright}{\kern0pt}\ {\isacharparenleft}{\kern0pt}fst\ {\isacharparenleft}{\kern0pt}A{\isacharcomma}{\kern0pt}\ n{\isacharparenright}{\kern0pt}{\isacharparenright}{\kern0pt}{\isacharparenright}{\kern0pt}\ {\isacharparenleft}{\kern0pt}fibered{\isacharunderscore}{\kern0pt}product{\isacharunderscore}{\kern0pt}right{\isacharunderscore}{\kern0pt}proj\ {\isacharparenleft}{\kern0pt}fst\ {\isacharparenleft}{\kern0pt}A{\isacharcomma}{\kern0pt}\ n{\isacharparenright}{\kern0pt}{\isacharparenright}{\kern0pt}\ {\isacharparenleft}{\kern0pt}f\ {\isasymcirc}\isactrlsub c\ snd\ {\isacharparenleft}{\kern0pt}A{\isacharcomma}{\kern0pt}\ n{\isacharparenright}{\kern0pt}{\isacharparenright}{\kern0pt}\ {\isacharparenleft}{\kern0pt}f\ {\isasymcirc}\isactrlsub c\ snd\ {\isacharparenleft}{\kern0pt}A{\isacharcomma}{\kern0pt}\ n{\isacharparenright}{\kern0pt}{\isacharparenright}{\kern0pt}\ {\isacharparenleft}{\kern0pt}fst\ {\isacharparenleft}{\kern0pt}A{\isacharcomma}{\kern0pt}\ n{\isacharparenright}{\kern0pt}{\isacharparenright}{\kern0pt}{\isacharparenright}{\kern0pt}\ {\isasymand}\isanewline
\ \ \ \ monomorphism\ m\ {\isasymand}\ f\ {\isasymcirc}\isactrlsub c\ snd\ {\isacharparenleft}{\kern0pt}A{\isacharcomma}{\kern0pt}\ n{\isacharparenright}{\kern0pt}\ {\isacharequal}{\kern0pt}\ m\ {\isasymcirc}\isactrlsub c\ {\isacharparenleft}{\kern0pt}f{\isasymrestriction}\isactrlbsub {\isacharparenleft}{\kern0pt}A{\isacharcomma}{\kern0pt}\ n{\isacharparenright}{\kern0pt}\isactrlesub {\isacharparenright}{\kern0pt}\ {\isasymand}\ {\isacharparenleft}{\kern0pt}{\isasymforall}x{\isachardot}{\kern0pt}\ x\ {\isacharcolon}{\kern0pt}\ f{\isasymlparr}fst\ {\isacharparenleft}{\kern0pt}A{\isacharcomma}{\kern0pt}\ n{\isacharparenright}{\kern0pt}{\isasymrparr}\isactrlbsub snd\ {\isacharparenleft}{\kern0pt}A{\isacharcomma}{\kern0pt}\ n{\isacharparenright}{\kern0pt}\isactrlesub \ {\isasymrightarrow}\ codomain\ f\ {\isasymlongrightarrow}\ f\ {\isasymcirc}\isactrlsub c\ snd\ {\isacharparenleft}{\kern0pt}A{\isacharcomma}{\kern0pt}\ n{\isacharparenright}{\kern0pt}\ {\isacharequal}{\kern0pt}\ x\ {\isasymcirc}\isactrlsub c\ {\isacharparenleft}{\kern0pt}f{\isasymrestriction}\isactrlbsub {\isacharparenleft}{\kern0pt}A{\isacharcomma}{\kern0pt}\ n{\isacharparenright}{\kern0pt}\isactrlesub {\isacharparenright}{\kern0pt}\ {\isasymlongrightarrow}\ x\ {\isacharequal}{\kern0pt}\ m{\isacharparenright}{\kern0pt}{\isachardoublequoteclose}\isanewline
\ \ \ \ \isacommand{unfolding}\isamarkupfalse%
\ image{\isacharunderscore}{\kern0pt}restriction{\isacharunderscore}{\kern0pt}mapping{\isacharunderscore}{\kern0pt}def\ \isacommand{by}\isamarkupfalse%
\ {\isacharparenleft}{\kern0pt}rule\ someI{\isacharunderscore}{\kern0pt}ex{\isacharcomma}{\kern0pt}\ insert\ assms\ image{\isacharunderscore}{\kern0pt}of{\isacharunderscore}{\kern0pt}def{\isadigit{2}}\ codom{\isacharunderscore}{\kern0pt}f{\isacharcomma}{\kern0pt}\ auto{\isacharparenright}{\kern0pt}\isanewline
\ \ \isacommand{then}\isamarkupfalse%
\ \isacommand{show}\isamarkupfalse%
\ {\isacharquery}{\kern0pt}thesis\isanewline
\ \ \ \ \isacommand{using}\isamarkupfalse%
\ codom{\isacharunderscore}{\kern0pt}f\ \isacommand{by}\isamarkupfalse%
\ simp\ \isanewline
\isacommand{qed}\isamarkupfalse%
%
\endisatagproof
{\isafoldproof}%
%
\isadelimproof
\isanewline
%
\endisadelimproof
\isanewline
\isacommand{definition}\isamarkupfalse%
\ image{\isacharunderscore}{\kern0pt}subobject{\isacharunderscore}{\kern0pt}mapping\ {\isacharcolon}{\kern0pt}{\isacharcolon}{\kern0pt}\ {\isachardoublequoteopen}cfunc\ {\isasymRightarrow}\ cset\ {\isasymRightarrow}\ cfunc\ {\isasymRightarrow}\ cfunc{\isachardoublequoteclose}\ {\isacharparenleft}{\kern0pt}{\isachardoublequoteopen}{\isacharbrackleft}{\kern0pt}{\isacharunderscore}{\kern0pt}{\isasymlparr}{\isacharunderscore}{\kern0pt}{\isasymrparr}\isactrlbsub {\isacharunderscore}{\kern0pt}\isactrlesub {\isacharbrackright}{\kern0pt}map{\isachardoublequoteclose}\ {\isacharbrackleft}{\kern0pt}{\isadigit{1}}{\isadigit{0}}{\isadigit{1}}{\isacharcomma}{\kern0pt}{\isadigit{0}}{\isacharcomma}{\kern0pt}{\isadigit{0}}{\isacharbrackright}{\kern0pt}{\isadigit{1}}{\isadigit{0}}{\isadigit{0}}{\isacharparenright}{\kern0pt}\ \isakeyword{where}\isanewline
\ \ {\isachardoublequoteopen}{\isacharbrackleft}{\kern0pt}f{\isasymlparr}A{\isasymrparr}\isactrlbsub n\isactrlesub {\isacharbrackright}{\kern0pt}map\ {\isacharequal}{\kern0pt}\ {\isacharparenleft}{\kern0pt}THE\ m{\isachardot}{\kern0pt}\ f{\isasymrestriction}\isactrlbsub {\isacharparenleft}{\kern0pt}A{\isacharcomma}{\kern0pt}\ n{\isacharparenright}{\kern0pt}\isactrlesub \ {\isacharcolon}{\kern0pt}\ A\ {\isasymrightarrow}\ f{\isasymlparr}A{\isasymrparr}\isactrlbsub n\isactrlesub \ {\isasymand}\ m\ {\isacharcolon}{\kern0pt}\ f{\isasymlparr}A{\isasymrparr}\isactrlbsub n\isactrlesub \ {\isasymrightarrow}\ codomain\ f\ {\isasymand}\isanewline
\ \ \ coequalizer\ {\isacharparenleft}{\kern0pt}f{\isasymlparr}A{\isasymrparr}\isactrlbsub n\isactrlesub {\isacharparenright}{\kern0pt}\ {\isacharparenleft}{\kern0pt}f{\isasymrestriction}\isactrlbsub {\isacharparenleft}{\kern0pt}A{\isacharcomma}{\kern0pt}\ n{\isacharparenright}{\kern0pt}\isactrlesub {\isacharparenright}{\kern0pt}\ {\isacharparenleft}{\kern0pt}fibered{\isacharunderscore}{\kern0pt}product{\isacharunderscore}{\kern0pt}left{\isacharunderscore}{\kern0pt}proj\ A\ {\isacharparenleft}{\kern0pt}f\ {\isasymcirc}\isactrlsub c\ n{\isacharparenright}{\kern0pt}\ {\isacharparenleft}{\kern0pt}f\ {\isasymcirc}\isactrlsub c\ n{\isacharparenright}{\kern0pt}\ A{\isacharparenright}{\kern0pt}\ {\isacharparenleft}{\kern0pt}fibered{\isacharunderscore}{\kern0pt}product{\isacharunderscore}{\kern0pt}right{\isacharunderscore}{\kern0pt}proj\ A\ {\isacharparenleft}{\kern0pt}f\ {\isasymcirc}\isactrlsub c\ n{\isacharparenright}{\kern0pt}\ {\isacharparenleft}{\kern0pt}f\ {\isasymcirc}\isactrlsub c\ n{\isacharparenright}{\kern0pt}\ A{\isacharparenright}{\kern0pt}\ {\isasymand}\isanewline
\ \ \ monomorphism\ m\ {\isasymand}\ f\ {\isasymcirc}\isactrlsub c\ n\ {\isacharequal}{\kern0pt}\ m\ {\isasymcirc}\isactrlsub c\ {\isacharparenleft}{\kern0pt}f{\isasymrestriction}\isactrlbsub {\isacharparenleft}{\kern0pt}A{\isacharcomma}{\kern0pt}\ n{\isacharparenright}{\kern0pt}\isactrlesub {\isacharparenright}{\kern0pt}\ {\isasymand}\ {\isacharparenleft}{\kern0pt}{\isasymforall}x{\isachardot}{\kern0pt}\ x\ {\isacharcolon}{\kern0pt}\ {\isacharparenleft}{\kern0pt}f{\isasymlparr}A{\isasymrparr}\isactrlbsub n\isactrlesub {\isacharparenright}{\kern0pt}\ {\isasymrightarrow}\ codomain\ f\ {\isasymlongrightarrow}\ f\ {\isasymcirc}\isactrlsub c\ n\ {\isacharequal}{\kern0pt}\ x\ {\isasymcirc}\isactrlsub c\ {\isacharparenleft}{\kern0pt}f{\isasymrestriction}\isactrlbsub {\isacharparenleft}{\kern0pt}A{\isacharcomma}{\kern0pt}\ n{\isacharparenright}{\kern0pt}\isactrlesub {\isacharparenright}{\kern0pt}\ {\isasymlongrightarrow}\ x\ {\isacharequal}{\kern0pt}\ m{\isacharparenright}{\kern0pt}{\isacharparenright}{\kern0pt}{\isachardoublequoteclose}\isanewline
\isanewline
\isacommand{lemma}\isamarkupfalse%
\ image{\isacharunderscore}{\kern0pt}subobject{\isacharunderscore}{\kern0pt}mapping{\isacharunderscore}{\kern0pt}def{\isadigit{2}}{\isacharcolon}{\kern0pt}\isanewline
\ \ \isakeyword{assumes}\ {\isachardoublequoteopen}f\ {\isacharcolon}{\kern0pt}\ X\ {\isasymrightarrow}\ Y{\isachardoublequoteclose}\ {\isachardoublequoteopen}n\ {\isacharcolon}{\kern0pt}\ A\ {\isasymrightarrow}\ X{\isachardoublequoteclose}\isanewline
\ \ \isakeyword{shows}\ {\isachardoublequoteopen}f{\isasymrestriction}\isactrlbsub {\isacharparenleft}{\kern0pt}A{\isacharcomma}{\kern0pt}\ n{\isacharparenright}{\kern0pt}\isactrlesub \ {\isacharcolon}{\kern0pt}\ A\ {\isasymrightarrow}\ f{\isasymlparr}A{\isasymrparr}\isactrlbsub n\isactrlesub \ {\isasymand}\ {\isacharbrackleft}{\kern0pt}f{\isasymlparr}A{\isasymrparr}\isactrlbsub n\isactrlesub {\isacharbrackright}{\kern0pt}map\ {\isacharcolon}{\kern0pt}\ f{\isasymlparr}A{\isasymrparr}\isactrlbsub n\isactrlesub \ {\isasymrightarrow}\ Y\ {\isasymand}\isanewline
\ \ \ \ coequalizer\ {\isacharparenleft}{\kern0pt}f{\isasymlparr}A{\isasymrparr}\isactrlbsub n\isactrlesub {\isacharparenright}{\kern0pt}\ {\isacharparenleft}{\kern0pt}f{\isasymrestriction}\isactrlbsub {\isacharparenleft}{\kern0pt}A{\isacharcomma}{\kern0pt}\ n{\isacharparenright}{\kern0pt}\isactrlesub {\isacharparenright}{\kern0pt}\ {\isacharparenleft}{\kern0pt}fibered{\isacharunderscore}{\kern0pt}product{\isacharunderscore}{\kern0pt}left{\isacharunderscore}{\kern0pt}proj\ A\ {\isacharparenleft}{\kern0pt}f\ {\isasymcirc}\isactrlsub c\ n{\isacharparenright}{\kern0pt}\ {\isacharparenleft}{\kern0pt}f\ {\isasymcirc}\isactrlsub c\ n{\isacharparenright}{\kern0pt}\ A{\isacharparenright}{\kern0pt}\ {\isacharparenleft}{\kern0pt}fibered{\isacharunderscore}{\kern0pt}product{\isacharunderscore}{\kern0pt}right{\isacharunderscore}{\kern0pt}proj\ A\ {\isacharparenleft}{\kern0pt}f\ {\isasymcirc}\isactrlsub c\ n{\isacharparenright}{\kern0pt}\ {\isacharparenleft}{\kern0pt}f\ {\isasymcirc}\isactrlsub c\ n{\isacharparenright}{\kern0pt}\ A{\isacharparenright}{\kern0pt}\ {\isasymand}\isanewline
\ \ \ \ monomorphism\ {\isacharparenleft}{\kern0pt}{\isacharbrackleft}{\kern0pt}f{\isasymlparr}A{\isasymrparr}\isactrlbsub n\isactrlesub {\isacharbrackright}{\kern0pt}map{\isacharparenright}{\kern0pt}\ {\isasymand}\ f\ {\isasymcirc}\isactrlsub c\ n\ {\isacharequal}{\kern0pt}\ {\isacharbrackleft}{\kern0pt}f{\isasymlparr}A{\isasymrparr}\isactrlbsub n\isactrlesub {\isacharbrackright}{\kern0pt}map\ {\isasymcirc}\isactrlsub c\ {\isacharparenleft}{\kern0pt}f{\isasymrestriction}\isactrlbsub {\isacharparenleft}{\kern0pt}A{\isacharcomma}{\kern0pt}\ n{\isacharparenright}{\kern0pt}\isactrlesub {\isacharparenright}{\kern0pt}\ {\isasymand}\ {\isacharparenleft}{\kern0pt}{\isasymforall}x{\isachardot}{\kern0pt}\ x\ {\isacharcolon}{\kern0pt}\ f{\isasymlparr}A{\isasymrparr}\isactrlbsub n\isactrlesub \ {\isasymrightarrow}\ Y\ {\isasymlongrightarrow}\ f\ {\isasymcirc}\isactrlsub c\ n\ {\isacharequal}{\kern0pt}\ x\ {\isasymcirc}\isactrlsub c\ {\isacharparenleft}{\kern0pt}f{\isasymrestriction}\isactrlbsub {\isacharparenleft}{\kern0pt}A{\isacharcomma}{\kern0pt}\ n{\isacharparenright}{\kern0pt}\isactrlesub {\isacharparenright}{\kern0pt}\ {\isasymlongrightarrow}\ x\ {\isacharequal}{\kern0pt}\ {\isacharbrackleft}{\kern0pt}f{\isasymlparr}A{\isasymrparr}\isactrlbsub n\isactrlesub {\isacharbrackright}{\kern0pt}map{\isacharparenright}{\kern0pt}{\isachardoublequoteclose}\isanewline
%
\isadelimproof
%
\endisadelimproof
%
\isatagproof
\isacommand{proof}\isamarkupfalse%
\ {\isacharminus}{\kern0pt}\isanewline
\ \ \isacommand{have}\isamarkupfalse%
\ codom{\isacharunderscore}{\kern0pt}f{\isacharcolon}{\kern0pt}\ {\isachardoublequoteopen}codomain\ f\ {\isacharequal}{\kern0pt}\ Y{\isachardoublequoteclose}\isanewline
\ \ \ \ \isacommand{using}\isamarkupfalse%
\ assms{\isacharparenleft}{\kern0pt}{\isadigit{1}}{\isacharparenright}{\kern0pt}\ cfunc{\isacharunderscore}{\kern0pt}type{\isacharunderscore}{\kern0pt}def\ \isacommand{by}\isamarkupfalse%
\ auto\isanewline
\ \ \isacommand{have}\isamarkupfalse%
\ {\isachardoublequoteopen}f{\isasymrestriction}\isactrlbsub {\isacharparenleft}{\kern0pt}A{\isacharcomma}{\kern0pt}\ n{\isacharparenright}{\kern0pt}\isactrlesub \ {\isacharcolon}{\kern0pt}\ A\ {\isasymrightarrow}\ f{\isasymlparr}A{\isasymrparr}\isactrlbsub n\isactrlesub \ {\isasymand}\ {\isacharparenleft}{\kern0pt}{\isacharbrackleft}{\kern0pt}f{\isasymlparr}A{\isasymrparr}\isactrlbsub n\isactrlesub {\isacharbrackright}{\kern0pt}map{\isacharparenright}{\kern0pt}\ {\isacharcolon}{\kern0pt}\ f{\isasymlparr}A{\isasymrparr}\isactrlbsub n\isactrlesub \ {\isasymrightarrow}\ codomain\ f\ {\isasymand}\isanewline
\ \ \ coequalizer\ {\isacharparenleft}{\kern0pt}f{\isasymlparr}A{\isasymrparr}\isactrlbsub n\isactrlesub {\isacharparenright}{\kern0pt}\ {\isacharparenleft}{\kern0pt}f{\isasymrestriction}\isactrlbsub {\isacharparenleft}{\kern0pt}A{\isacharcomma}{\kern0pt}\ n{\isacharparenright}{\kern0pt}\isactrlesub {\isacharparenright}{\kern0pt}\ {\isacharparenleft}{\kern0pt}fibered{\isacharunderscore}{\kern0pt}product{\isacharunderscore}{\kern0pt}left{\isacharunderscore}{\kern0pt}proj\ A\ {\isacharparenleft}{\kern0pt}f\ {\isasymcirc}\isactrlsub c\ n{\isacharparenright}{\kern0pt}\ {\isacharparenleft}{\kern0pt}f\ {\isasymcirc}\isactrlsub c\ n{\isacharparenright}{\kern0pt}\ A{\isacharparenright}{\kern0pt}\ {\isacharparenleft}{\kern0pt}fibered{\isacharunderscore}{\kern0pt}product{\isacharunderscore}{\kern0pt}right{\isacharunderscore}{\kern0pt}proj\ A\ {\isacharparenleft}{\kern0pt}f\ {\isasymcirc}\isactrlsub c\ n{\isacharparenright}{\kern0pt}\ {\isacharparenleft}{\kern0pt}f\ {\isasymcirc}\isactrlsub c\ n{\isacharparenright}{\kern0pt}\ A{\isacharparenright}{\kern0pt}\ {\isasymand}\isanewline
\ \ \ monomorphism\ {\isacharparenleft}{\kern0pt}{\isacharbrackleft}{\kern0pt}f{\isasymlparr}A{\isasymrparr}\isactrlbsub n\isactrlesub {\isacharbrackright}{\kern0pt}map{\isacharparenright}{\kern0pt}\ {\isasymand}\ f\ {\isasymcirc}\isactrlsub c\ n\ {\isacharequal}{\kern0pt}\ {\isacharparenleft}{\kern0pt}{\isacharbrackleft}{\kern0pt}f{\isasymlparr}A{\isasymrparr}\isactrlbsub n\isactrlesub {\isacharbrackright}{\kern0pt}map{\isacharparenright}{\kern0pt}\ {\isasymcirc}\isactrlsub c\ {\isacharparenleft}{\kern0pt}f{\isasymrestriction}\isactrlbsub {\isacharparenleft}{\kern0pt}A{\isacharcomma}{\kern0pt}\ n{\isacharparenright}{\kern0pt}\isactrlesub {\isacharparenright}{\kern0pt}\ {\isasymand}\ \isanewline
\ \ \ {\isacharparenleft}{\kern0pt}{\isasymforall}x{\isachardot}{\kern0pt}\ x\ {\isacharcolon}{\kern0pt}\ {\isacharparenleft}{\kern0pt}f{\isasymlparr}A{\isasymrparr}\isactrlbsub n\isactrlesub {\isacharparenright}{\kern0pt}\ {\isasymrightarrow}\ codomain\ f\ {\isasymlongrightarrow}\ f\ {\isasymcirc}\isactrlsub c\ n\ {\isacharequal}{\kern0pt}\ x\ {\isasymcirc}\isactrlsub c\ {\isacharparenleft}{\kern0pt}f{\isasymrestriction}\isactrlbsub {\isacharparenleft}{\kern0pt}A{\isacharcomma}{\kern0pt}\ n{\isacharparenright}{\kern0pt}\isactrlesub {\isacharparenright}{\kern0pt}\ {\isasymlongrightarrow}\ x\ {\isacharequal}{\kern0pt}\ {\isacharparenleft}{\kern0pt}{\isacharbrackleft}{\kern0pt}f{\isasymlparr}A{\isasymrparr}\isactrlbsub n\isactrlesub {\isacharbrackright}{\kern0pt}map{\isacharparenright}{\kern0pt}{\isacharparenright}{\kern0pt}{\isachardoublequoteclose}\isanewline
\ \ \ \ \isacommand{unfolding}\isamarkupfalse%
\ image{\isacharunderscore}{\kern0pt}subobject{\isacharunderscore}{\kern0pt}mapping{\isacharunderscore}{\kern0pt}def\isanewline
\ \ \ \ \isacommand{by}\isamarkupfalse%
\ {\isacharparenleft}{\kern0pt}rule\ theI{\isacharprime}{\kern0pt}{\isacharcomma}{\kern0pt}\ insert\ assms\ codom{\isacharunderscore}{\kern0pt}f\ image{\isacharunderscore}{\kern0pt}restriction{\isacharunderscore}{\kern0pt}mapping{\isacharunderscore}{\kern0pt}def{\isadigit{2}}{\isacharcomma}{\kern0pt}\ blast{\isacharparenright}{\kern0pt}\isanewline
\ \ \isacommand{then}\isamarkupfalse%
\ \isacommand{show}\isamarkupfalse%
\ {\isacharquery}{\kern0pt}thesis\isanewline
\ \ \ \ \isacommand{using}\isamarkupfalse%
\ codom{\isacharunderscore}{\kern0pt}f\ \isacommand{by}\isamarkupfalse%
\ fastforce\isanewline
\isacommand{qed}\isamarkupfalse%
%
\endisatagproof
{\isafoldproof}%
%
\isadelimproof
\isanewline
%
\endisadelimproof
\isanewline
\isacommand{lemma}\isamarkupfalse%
\ image{\isacharunderscore}{\kern0pt}rest{\isacharunderscore}{\kern0pt}map{\isacharunderscore}{\kern0pt}type{\isacharbrackleft}{\kern0pt}type{\isacharunderscore}{\kern0pt}rule{\isacharbrackright}{\kern0pt}{\isacharcolon}{\kern0pt}\isanewline
\ \ \isakeyword{assumes}\ {\isachardoublequoteopen}f\ {\isacharcolon}{\kern0pt}\ X\ {\isasymrightarrow}\ Y{\isachardoublequoteclose}\ {\isachardoublequoteopen}n\ {\isacharcolon}{\kern0pt}\ A\ {\isasymrightarrow}\ X{\isachardoublequoteclose}\isanewline
\ \ \isakeyword{shows}\ {\isachardoublequoteopen}f{\isasymrestriction}\isactrlbsub {\isacharparenleft}{\kern0pt}A{\isacharcomma}{\kern0pt}\ n{\isacharparenright}{\kern0pt}\isactrlesub \ {\isacharcolon}{\kern0pt}\ A\ {\isasymrightarrow}\ f{\isasymlparr}A{\isasymrparr}\isactrlbsub n\isactrlesub {\isachardoublequoteclose}\isanewline
%
\isadelimproof
\ \ %
\endisadelimproof
%
\isatagproof
\isacommand{using}\isamarkupfalse%
\ assms\ image{\isacharunderscore}{\kern0pt}restriction{\isacharunderscore}{\kern0pt}mapping{\isacharunderscore}{\kern0pt}def{\isadigit{2}}\ \isacommand{by}\isamarkupfalse%
\ blast%
\endisatagproof
{\isafoldproof}%
%
\isadelimproof
\isanewline
%
\endisadelimproof
\isanewline
\isacommand{lemma}\isamarkupfalse%
\ image{\isacharunderscore}{\kern0pt}rest{\isacharunderscore}{\kern0pt}map{\isacharunderscore}{\kern0pt}coequalizer{\isacharcolon}{\kern0pt}\isanewline
\ \ \isakeyword{assumes}\ {\isachardoublequoteopen}f\ {\isacharcolon}{\kern0pt}\ X\ {\isasymrightarrow}\ Y{\isachardoublequoteclose}\ {\isachardoublequoteopen}n\ {\isacharcolon}{\kern0pt}\ A\ {\isasymrightarrow}\ X{\isachardoublequoteclose}\isanewline
\ \ \isakeyword{shows}\ {\isachardoublequoteopen}coequalizer\ {\isacharparenleft}{\kern0pt}f{\isasymlparr}A{\isasymrparr}\isactrlbsub n\isactrlesub {\isacharparenright}{\kern0pt}\ {\isacharparenleft}{\kern0pt}f{\isasymrestriction}\isactrlbsub {\isacharparenleft}{\kern0pt}A{\isacharcomma}{\kern0pt}\ n{\isacharparenright}{\kern0pt}\isactrlesub {\isacharparenright}{\kern0pt}\ {\isacharparenleft}{\kern0pt}fibered{\isacharunderscore}{\kern0pt}product{\isacharunderscore}{\kern0pt}left{\isacharunderscore}{\kern0pt}proj\ A\ {\isacharparenleft}{\kern0pt}f\ {\isasymcirc}\isactrlsub c\ n{\isacharparenright}{\kern0pt}\ {\isacharparenleft}{\kern0pt}f\ {\isasymcirc}\isactrlsub c\ n{\isacharparenright}{\kern0pt}\ A{\isacharparenright}{\kern0pt}\ {\isacharparenleft}{\kern0pt}fibered{\isacharunderscore}{\kern0pt}product{\isacharunderscore}{\kern0pt}right{\isacharunderscore}{\kern0pt}proj\ A\ {\isacharparenleft}{\kern0pt}f\ {\isasymcirc}\isactrlsub c\ n{\isacharparenright}{\kern0pt}\ {\isacharparenleft}{\kern0pt}f\ {\isasymcirc}\isactrlsub c\ n{\isacharparenright}{\kern0pt}\ A{\isacharparenright}{\kern0pt}{\isachardoublequoteclose}\isanewline
%
\isadelimproof
\ \ %
\endisadelimproof
%
\isatagproof
\isacommand{using}\isamarkupfalse%
\ assms\ image{\isacharunderscore}{\kern0pt}restriction{\isacharunderscore}{\kern0pt}mapping{\isacharunderscore}{\kern0pt}def{\isadigit{2}}\ \isacommand{by}\isamarkupfalse%
\ blast%
\endisatagproof
{\isafoldproof}%
%
\isadelimproof
\isanewline
%
\endisadelimproof
\isanewline
\isacommand{lemma}\isamarkupfalse%
\ image{\isacharunderscore}{\kern0pt}rest{\isacharunderscore}{\kern0pt}map{\isacharunderscore}{\kern0pt}epi{\isacharcolon}{\kern0pt}\isanewline
\ \ \isakeyword{assumes}\ {\isachardoublequoteopen}f\ {\isacharcolon}{\kern0pt}\ X\ {\isasymrightarrow}\ Y{\isachardoublequoteclose}\ {\isachardoublequoteopen}n\ {\isacharcolon}{\kern0pt}\ A\ {\isasymrightarrow}\ X{\isachardoublequoteclose}\isanewline
\ \ \isakeyword{shows}\ {\isachardoublequoteopen}epimorphism\ {\isacharparenleft}{\kern0pt}f{\isasymrestriction}\isactrlbsub {\isacharparenleft}{\kern0pt}A{\isacharcomma}{\kern0pt}\ n{\isacharparenright}{\kern0pt}\isactrlesub {\isacharparenright}{\kern0pt}{\isachardoublequoteclose}\isanewline
%
\isadelimproof
\ \ %
\endisadelimproof
%
\isatagproof
\isacommand{using}\isamarkupfalse%
\ assms\ image{\isacharunderscore}{\kern0pt}rest{\isacharunderscore}{\kern0pt}map{\isacharunderscore}{\kern0pt}coequalizer\ coequalizer{\isacharunderscore}{\kern0pt}is{\isacharunderscore}{\kern0pt}epimorphism\ \isacommand{by}\isamarkupfalse%
\ blast%
\endisatagproof
{\isafoldproof}%
%
\isadelimproof
\ \isanewline
%
\endisadelimproof
\isanewline
\isacommand{lemma}\isamarkupfalse%
\ image{\isacharunderscore}{\kern0pt}subobj{\isacharunderscore}{\kern0pt}map{\isacharunderscore}{\kern0pt}type{\isacharbrackleft}{\kern0pt}type{\isacharunderscore}{\kern0pt}rule{\isacharbrackright}{\kern0pt}{\isacharcolon}{\kern0pt}\isanewline
\ \ \isakeyword{assumes}\ {\isachardoublequoteopen}f\ {\isacharcolon}{\kern0pt}\ X\ {\isasymrightarrow}\ Y{\isachardoublequoteclose}\ {\isachardoublequoteopen}n\ {\isacharcolon}{\kern0pt}\ A\ {\isasymrightarrow}\ X{\isachardoublequoteclose}\isanewline
\ \ \isakeyword{shows}\ {\isachardoublequoteopen}{\isacharbrackleft}{\kern0pt}f{\isasymlparr}A{\isasymrparr}\isactrlbsub n\isactrlesub {\isacharbrackright}{\kern0pt}map\ {\isacharcolon}{\kern0pt}\ f{\isasymlparr}A{\isasymrparr}\isactrlbsub n\isactrlesub \ {\isasymrightarrow}\ Y{\isachardoublequoteclose}\isanewline
%
\isadelimproof
\ \ %
\endisadelimproof
%
\isatagproof
\isacommand{using}\isamarkupfalse%
\ assms\ image{\isacharunderscore}{\kern0pt}subobject{\isacharunderscore}{\kern0pt}mapping{\isacharunderscore}{\kern0pt}def{\isadigit{2}}\ \isacommand{by}\isamarkupfalse%
\ blast%
\endisatagproof
{\isafoldproof}%
%
\isadelimproof
\isanewline
%
\endisadelimproof
\isanewline
\isacommand{lemma}\isamarkupfalse%
\ image{\isacharunderscore}{\kern0pt}subobj{\isacharunderscore}{\kern0pt}map{\isacharunderscore}{\kern0pt}mono{\isacharcolon}{\kern0pt}\isanewline
\ \ \isakeyword{assumes}\ {\isachardoublequoteopen}f\ {\isacharcolon}{\kern0pt}\ X\ {\isasymrightarrow}\ Y{\isachardoublequoteclose}\ {\isachardoublequoteopen}n\ {\isacharcolon}{\kern0pt}\ A\ {\isasymrightarrow}\ X{\isachardoublequoteclose}\isanewline
\ \ \isakeyword{shows}\ {\isachardoublequoteopen}monomorphism\ {\isacharparenleft}{\kern0pt}{\isacharbrackleft}{\kern0pt}f{\isasymlparr}A{\isasymrparr}\isactrlbsub n\isactrlesub {\isacharbrackright}{\kern0pt}map{\isacharparenright}{\kern0pt}{\isachardoublequoteclose}\isanewline
%
\isadelimproof
\ \ %
\endisadelimproof
%
\isatagproof
\isacommand{using}\isamarkupfalse%
\ assms\ image{\isacharunderscore}{\kern0pt}subobject{\isacharunderscore}{\kern0pt}mapping{\isacharunderscore}{\kern0pt}def{\isadigit{2}}\ \isacommand{by}\isamarkupfalse%
\ blast%
\endisatagproof
{\isafoldproof}%
%
\isadelimproof
\isanewline
%
\endisadelimproof
\isanewline
\isacommand{lemma}\isamarkupfalse%
\ image{\isacharunderscore}{\kern0pt}subobj{\isacharunderscore}{\kern0pt}comp{\isacharunderscore}{\kern0pt}image{\isacharunderscore}{\kern0pt}rest{\isacharcolon}{\kern0pt}\isanewline
\ \ \isakeyword{assumes}\ {\isachardoublequoteopen}f\ {\isacharcolon}{\kern0pt}\ X\ {\isasymrightarrow}\ Y{\isachardoublequoteclose}\ {\isachardoublequoteopen}n\ {\isacharcolon}{\kern0pt}\ A\ {\isasymrightarrow}\ X{\isachardoublequoteclose}\isanewline
\ \ \isakeyword{shows}\ {\isachardoublequoteopen}{\isacharbrackleft}{\kern0pt}f{\isasymlparr}A{\isasymrparr}\isactrlbsub n\isactrlesub {\isacharbrackright}{\kern0pt}map\ {\isasymcirc}\isactrlsub c\ {\isacharparenleft}{\kern0pt}f{\isasymrestriction}\isactrlbsub {\isacharparenleft}{\kern0pt}A{\isacharcomma}{\kern0pt}\ n{\isacharparenright}{\kern0pt}\isactrlesub {\isacharparenright}{\kern0pt}\ {\isacharequal}{\kern0pt}\ f\ {\isasymcirc}\isactrlsub c\ n{\isachardoublequoteclose}\isanewline
%
\isadelimproof
\ \ %
\endisadelimproof
%
\isatagproof
\isacommand{using}\isamarkupfalse%
\ assms\ image{\isacharunderscore}{\kern0pt}subobject{\isacharunderscore}{\kern0pt}mapping{\isacharunderscore}{\kern0pt}def{\isadigit{2}}\ \isacommand{by}\isamarkupfalse%
\ auto%
\endisatagproof
{\isafoldproof}%
%
\isadelimproof
\isanewline
%
\endisadelimproof
\isanewline
\isacommand{lemma}\isamarkupfalse%
\ image{\isacharunderscore}{\kern0pt}subobj{\isacharunderscore}{\kern0pt}map{\isacharunderscore}{\kern0pt}unique{\isacharcolon}{\kern0pt}\isanewline
\ \ \isakeyword{assumes}\ {\isachardoublequoteopen}f\ {\isacharcolon}{\kern0pt}\ X\ {\isasymrightarrow}\ Y{\isachardoublequoteclose}\ {\isachardoublequoteopen}n\ {\isacharcolon}{\kern0pt}\ A\ {\isasymrightarrow}\ X{\isachardoublequoteclose}\isanewline
\ \ \isakeyword{shows}\ {\isachardoublequoteopen}x\ {\isacharcolon}{\kern0pt}\ f{\isasymlparr}A{\isasymrparr}\isactrlbsub n\isactrlesub \ {\isasymrightarrow}\ Y\ {\isasymLongrightarrow}\ f\ {\isasymcirc}\isactrlsub c\ n\ {\isacharequal}{\kern0pt}\ x\ {\isasymcirc}\isactrlsub c\ {\isacharparenleft}{\kern0pt}f{\isasymrestriction}\isactrlbsub {\isacharparenleft}{\kern0pt}A{\isacharcomma}{\kern0pt}\ n{\isacharparenright}{\kern0pt}\isactrlesub {\isacharparenright}{\kern0pt}\ {\isasymLongrightarrow}\ x\ {\isacharequal}{\kern0pt}\ {\isacharbrackleft}{\kern0pt}f{\isasymlparr}A{\isasymrparr}\isactrlbsub n\isactrlesub {\isacharbrackright}{\kern0pt}map{\isachardoublequoteclose}\isanewline
%
\isadelimproof
\ \ %
\endisadelimproof
%
\isatagproof
\isacommand{using}\isamarkupfalse%
\ assms\ image{\isacharunderscore}{\kern0pt}subobject{\isacharunderscore}{\kern0pt}mapping{\isacharunderscore}{\kern0pt}def{\isadigit{2}}\ \isacommand{by}\isamarkupfalse%
\ blast%
\endisatagproof
{\isafoldproof}%
%
\isadelimproof
\isanewline
%
\endisadelimproof
\isanewline
\isacommand{lemma}\isamarkupfalse%
\ image{\isacharunderscore}{\kern0pt}self{\isacharcolon}{\kern0pt}\isanewline
\ \ \isakeyword{assumes}\ {\isachardoublequoteopen}f\ {\isacharcolon}{\kern0pt}\ X\ {\isasymrightarrow}\ Y{\isachardoublequoteclose}\ \isakeyword{and}\ {\isachardoublequoteopen}monomorphism\ f{\isachardoublequoteclose}\isanewline
\ \ \isakeyword{assumes}\ {\isachardoublequoteopen}a\ {\isacharcolon}{\kern0pt}\ A\ {\isasymrightarrow}\ X{\isachardoublequoteclose}\ \isakeyword{and}\ {\isachardoublequoteopen}monomorphism\ a{\isachardoublequoteclose}\isanewline
\ \ \isakeyword{shows}\ {\isachardoublequoteopen}f{\isasymlparr}A{\isasymrparr}\isactrlbsub a\isactrlesub \ {\isasymcong}\ A{\isachardoublequoteclose}\isanewline
%
\isadelimproof
%
\endisadelimproof
%
\isatagproof
\isacommand{proof}\isamarkupfalse%
\ {\isacharminus}{\kern0pt}\isanewline
\ \ \isacommand{have}\isamarkupfalse%
\ {\isachardoublequoteopen}monomorphism\ {\isacharparenleft}{\kern0pt}f\ {\isasymcirc}\isactrlsub c\ a{\isacharparenright}{\kern0pt}{\isachardoublequoteclose}\isanewline
\ \ \ \ \isacommand{using}\isamarkupfalse%
\ assms\ cfunc{\isacharunderscore}{\kern0pt}type{\isacharunderscore}{\kern0pt}def\ composition{\isacharunderscore}{\kern0pt}of{\isacharunderscore}{\kern0pt}monic{\isacharunderscore}{\kern0pt}pair{\isacharunderscore}{\kern0pt}is{\isacharunderscore}{\kern0pt}monic\ \isacommand{by}\isamarkupfalse%
\ auto\isanewline
\ \ \isacommand{then}\isamarkupfalse%
\ \isacommand{have}\isamarkupfalse%
\ {\isachardoublequoteopen}monomorphism\ {\isacharparenleft}{\kern0pt}{\isacharbrackleft}{\kern0pt}f{\isasymlparr}A{\isasymrparr}\isactrlbsub a\isactrlesub {\isacharbrackright}{\kern0pt}map\ {\isasymcirc}\isactrlsub c\ {\isacharparenleft}{\kern0pt}f{\isasymrestriction}\isactrlbsub {\isacharparenleft}{\kern0pt}A{\isacharcomma}{\kern0pt}\ a{\isacharparenright}{\kern0pt}\isactrlesub {\isacharparenright}{\kern0pt}{\isacharparenright}{\kern0pt}{\isachardoublequoteclose}\isanewline
\ \ \ \ \isacommand{using}\isamarkupfalse%
\ assms\ image{\isacharunderscore}{\kern0pt}subobj{\isacharunderscore}{\kern0pt}comp{\isacharunderscore}{\kern0pt}image{\isacharunderscore}{\kern0pt}rest\ \isacommand{by}\isamarkupfalse%
\ auto\isanewline
\ \ \isacommand{then}\isamarkupfalse%
\ \isacommand{have}\isamarkupfalse%
\ {\isachardoublequoteopen}monomorphism\ {\isacharparenleft}{\kern0pt}f{\isasymrestriction}\isactrlbsub {\isacharparenleft}{\kern0pt}A{\isacharcomma}{\kern0pt}\ a{\isacharparenright}{\kern0pt}\isactrlesub {\isacharparenright}{\kern0pt}{\isachardoublequoteclose}\isanewline
\ \ \ \ \isacommand{by}\isamarkupfalse%
\ {\isacharparenleft}{\kern0pt}meson\ assms\ comp{\isacharunderscore}{\kern0pt}monic{\isacharunderscore}{\kern0pt}imp{\isacharunderscore}{\kern0pt}monic{\isacharprime}{\kern0pt}\ image{\isacharunderscore}{\kern0pt}rest{\isacharunderscore}{\kern0pt}map{\isacharunderscore}{\kern0pt}type\ image{\isacharunderscore}{\kern0pt}subobj{\isacharunderscore}{\kern0pt}map{\isacharunderscore}{\kern0pt}type{\isacharparenright}{\kern0pt}\isanewline
\ \ \isacommand{then}\isamarkupfalse%
\ \isacommand{have}\isamarkupfalse%
\ {\isachardoublequoteopen}isomorphism\ {\isacharparenleft}{\kern0pt}f{\isasymrestriction}\isactrlbsub {\isacharparenleft}{\kern0pt}A{\isacharcomma}{\kern0pt}\ a{\isacharparenright}{\kern0pt}\isactrlesub {\isacharparenright}{\kern0pt}{\isachardoublequoteclose}\isanewline
\ \ \ \ \isacommand{using}\isamarkupfalse%
\ assms\ epi{\isacharunderscore}{\kern0pt}mon{\isacharunderscore}{\kern0pt}is{\isacharunderscore}{\kern0pt}iso\ image{\isacharunderscore}{\kern0pt}rest{\isacharunderscore}{\kern0pt}map{\isacharunderscore}{\kern0pt}epi\ \isacommand{by}\isamarkupfalse%
\ blast\isanewline
\ \ \isacommand{then}\isamarkupfalse%
\ \isacommand{have}\isamarkupfalse%
\ {\isachardoublequoteopen}A\ {\isasymcong}\ f{\isasymlparr}A{\isasymrparr}\isactrlbsub a\isactrlesub {\isachardoublequoteclose}\isanewline
\ \ \ \ \isacommand{using}\isamarkupfalse%
\ assms\ \isacommand{unfolding}\isamarkupfalse%
\ is{\isacharunderscore}{\kern0pt}isomorphic{\isacharunderscore}{\kern0pt}def\ \isacommand{by}\isamarkupfalse%
\ {\isacharparenleft}{\kern0pt}rule{\isacharunderscore}{\kern0pt}tac\ x{\isacharequal}{\kern0pt}{\isachardoublequoteopen}f{\isasymrestriction}\isactrlbsub {\isacharparenleft}{\kern0pt}A{\isacharcomma}{\kern0pt}\ a{\isacharparenright}{\kern0pt}\isactrlesub {\isachardoublequoteclose}\ \isakeyword{in}\ exI{\isacharcomma}{\kern0pt}\ typecheck{\isacharunderscore}{\kern0pt}cfuncs{\isacharparenright}{\kern0pt}\isanewline
\ \ \isacommand{then}\isamarkupfalse%
\ \isacommand{show}\isamarkupfalse%
\ {\isacharquery}{\kern0pt}thesis\isanewline
\ \ \ \ \isacommand{by}\isamarkupfalse%
\ {\isacharparenleft}{\kern0pt}simp\ add{\isacharcolon}{\kern0pt}\ isomorphic{\isacharunderscore}{\kern0pt}is{\isacharunderscore}{\kern0pt}symmetric{\isacharparenright}{\kern0pt}\isanewline
\isacommand{qed}\isamarkupfalse%
%
\endisatagproof
{\isafoldproof}%
%
\isadelimproof
%
\endisadelimproof
%
\begin{isamarkuptext}%
The lemma below corresponds to Proposition 2.3.8 in Halvorson.%
\end{isamarkuptext}\isamarkuptrue%
\isacommand{lemma}\isamarkupfalse%
\ image{\isacharunderscore}{\kern0pt}smallest{\isacharunderscore}{\kern0pt}subobject{\isacharcolon}{\kern0pt}\isanewline
\ \ \isakeyword{assumes}\ f{\isacharunderscore}{\kern0pt}type{\isacharbrackleft}{\kern0pt}type{\isacharunderscore}{\kern0pt}rule{\isacharbrackright}{\kern0pt}{\isacharcolon}{\kern0pt}\ {\isachardoublequoteopen}f\ {\isacharcolon}{\kern0pt}\ X\ {\isasymrightarrow}\ Y{\isachardoublequoteclose}\ \isakeyword{and}\ a{\isacharunderscore}{\kern0pt}type{\isacharbrackleft}{\kern0pt}type{\isacharunderscore}{\kern0pt}rule{\isacharbrackright}{\kern0pt}{\isacharcolon}{\kern0pt}\ {\isachardoublequoteopen}a\ {\isacharcolon}{\kern0pt}\ A\ {\isasymrightarrow}\ X{\isachardoublequoteclose}\isanewline
\ \ \isakeyword{shows}\ {\isachardoublequoteopen}{\isacharparenleft}{\kern0pt}B{\isacharcomma}{\kern0pt}\ n{\isacharparenright}{\kern0pt}\ {\isasymsubseteq}\isactrlsub c\ Y\ {\isasymLongrightarrow}\ f\ factorsthru\ n\ {\isasymLongrightarrow}\ {\isacharparenleft}{\kern0pt}f{\isasymlparr}A{\isasymrparr}\isactrlbsub a\isactrlesub {\isacharcomma}{\kern0pt}\ {\isacharbrackleft}{\kern0pt}f{\isasymlparr}A{\isasymrparr}\isactrlbsub a\isactrlesub {\isacharbrackright}{\kern0pt}map{\isacharparenright}{\kern0pt}\ {\isasymsubseteq}\isactrlbsub Y\isactrlesub \ {\isacharparenleft}{\kern0pt}B{\isacharcomma}{\kern0pt}\ n{\isacharparenright}{\kern0pt}{\isachardoublequoteclose}\isanewline
%
\isadelimproof
%
\endisadelimproof
%
\isatagproof
\isacommand{proof}\isamarkupfalse%
\ {\isacharminus}{\kern0pt}\isanewline
\ \ \isacommand{assume}\isamarkupfalse%
\ {\isachardoublequoteopen}{\isacharparenleft}{\kern0pt}B{\isacharcomma}{\kern0pt}\ n{\isacharparenright}{\kern0pt}\ {\isasymsubseteq}\isactrlsub c\ Y{\isachardoublequoteclose}\isanewline
\ \ \isacommand{then}\isamarkupfalse%
\ \isacommand{have}\isamarkupfalse%
\ n{\isacharunderscore}{\kern0pt}type{\isacharbrackleft}{\kern0pt}type{\isacharunderscore}{\kern0pt}rule{\isacharbrackright}{\kern0pt}{\isacharcolon}{\kern0pt}\ {\isachardoublequoteopen}n\ {\isacharcolon}{\kern0pt}\ B\ {\isasymrightarrow}\ Y{\isachardoublequoteclose}\ \isakeyword{and}\ n{\isacharunderscore}{\kern0pt}mono{\isacharcolon}{\kern0pt}\ {\isachardoublequoteopen}monomorphism\ n{\isachardoublequoteclose}\isanewline
\ \ \ \ \isacommand{unfolding}\isamarkupfalse%
\ subobject{\isacharunderscore}{\kern0pt}of{\isacharunderscore}{\kern0pt}def{\isadigit{2}}\ \isacommand{by}\isamarkupfalse%
\ auto\isanewline
\ \ \isacommand{assume}\isamarkupfalse%
\ {\isachardoublequoteopen}f\ factorsthru\ n{\isachardoublequoteclose}\isanewline
\ \ \isacommand{then}\isamarkupfalse%
\ \isacommand{obtain}\isamarkupfalse%
\ g\ \isakeyword{where}\ g{\isacharunderscore}{\kern0pt}type{\isacharbrackleft}{\kern0pt}type{\isacharunderscore}{\kern0pt}rule{\isacharbrackright}{\kern0pt}{\isacharcolon}{\kern0pt}\ {\isachardoublequoteopen}g\ {\isacharcolon}{\kern0pt}\ X\ {\isasymrightarrow}\ B{\isachardoublequoteclose}\ \isakeyword{and}\ f{\isacharunderscore}{\kern0pt}eq{\isacharunderscore}{\kern0pt}ng{\isacharcolon}{\kern0pt}\ {\isachardoublequoteopen}n\ {\isasymcirc}\isactrlsub c\ g\ {\isacharequal}{\kern0pt}\ f{\isachardoublequoteclose}\isanewline
\ \ \ \ \isacommand{using}\isamarkupfalse%
\ factors{\isacharunderscore}{\kern0pt}through{\isacharunderscore}{\kern0pt}def{\isadigit{2}}\ \isacommand{by}\isamarkupfalse%
\ {\isacharparenleft}{\kern0pt}typecheck{\isacharunderscore}{\kern0pt}cfuncs{\isacharcomma}{\kern0pt}\ auto{\isacharparenright}{\kern0pt}\isanewline
\isanewline
\ \ \isacommand{have}\isamarkupfalse%
\ fa{\isacharunderscore}{\kern0pt}type{\isacharbrackleft}{\kern0pt}type{\isacharunderscore}{\kern0pt}rule{\isacharbrackright}{\kern0pt}{\isacharcolon}{\kern0pt}\ {\isachardoublequoteopen}f\ {\isasymcirc}\isactrlsub c\ a\ {\isacharcolon}{\kern0pt}\ A\ {\isasymrightarrow}\ Y{\isachardoublequoteclose}\isanewline
\ \ \ \ \isacommand{by}\isamarkupfalse%
\ {\isacharparenleft}{\kern0pt}typecheck{\isacharunderscore}{\kern0pt}cfuncs{\isacharparenright}{\kern0pt}\isanewline
\isanewline
\ \ \isacommand{obtain}\isamarkupfalse%
\ p{\isadigit{0}}\ \isakeyword{where}\ p{\isadigit{0}}{\isacharunderscore}{\kern0pt}def{\isacharbrackleft}{\kern0pt}simp{\isacharbrackright}{\kern0pt}{\isacharcolon}{\kern0pt}\ {\isachardoublequoteopen}p{\isadigit{0}}\ {\isacharequal}{\kern0pt}\ fibered{\isacharunderscore}{\kern0pt}product{\isacharunderscore}{\kern0pt}left{\isacharunderscore}{\kern0pt}proj\ A\ {\isacharparenleft}{\kern0pt}f{\isasymcirc}\isactrlsub ca{\isacharparenright}{\kern0pt}\ {\isacharparenleft}{\kern0pt}f{\isasymcirc}\isactrlsub ca{\isacharparenright}{\kern0pt}\ A{\isachardoublequoteclose}\isanewline
\ \ \ \ \isacommand{by}\isamarkupfalse%
\ auto\isanewline
\ \ \isacommand{obtain}\isamarkupfalse%
\ p{\isadigit{1}}\ \isakeyword{where}\ p{\isadigit{1}}{\isacharunderscore}{\kern0pt}def{\isacharbrackleft}{\kern0pt}simp{\isacharbrackright}{\kern0pt}{\isacharcolon}{\kern0pt}\ {\isachardoublequoteopen}p{\isadigit{1}}\ {\isacharequal}{\kern0pt}\ fibered{\isacharunderscore}{\kern0pt}product{\isacharunderscore}{\kern0pt}right{\isacharunderscore}{\kern0pt}proj\ A\ {\isacharparenleft}{\kern0pt}f{\isasymcirc}\isactrlsub ca{\isacharparenright}{\kern0pt}\ {\isacharparenleft}{\kern0pt}f{\isasymcirc}\isactrlsub ca{\isacharparenright}{\kern0pt}\ A{\isachardoublequoteclose}\isanewline
\ \ \ \ \isacommand{by}\isamarkupfalse%
\ auto\isanewline
\ \ \isacommand{obtain}\isamarkupfalse%
\ E\ \isakeyword{where}\ E{\isacharunderscore}{\kern0pt}def{\isacharbrackleft}{\kern0pt}simp{\isacharbrackright}{\kern0pt}{\isacharcolon}{\kern0pt}\ {\isachardoublequoteopen}E\ {\isacharequal}{\kern0pt}\ A\ \isactrlbsub f\ {\isasymcirc}\isactrlsub c\ a\isactrlesub {\isasymtimes}\isactrlsub c\isactrlbsub f\ {\isasymcirc}\isactrlsub c\ a\isactrlesub \ A{\isachardoublequoteclose}\isanewline
\ \ \ \ \isacommand{by}\isamarkupfalse%
\ auto\isanewline
\isanewline
\ \ \isacommand{have}\isamarkupfalse%
\ fa{\isacharunderscore}{\kern0pt}coequalizes{\isacharcolon}{\kern0pt}\ {\isachardoublequoteopen}{\isacharparenleft}{\kern0pt}f\ {\isasymcirc}\isactrlsub c\ a{\isacharparenright}{\kern0pt}\ {\isasymcirc}\isactrlsub c\ p{\isadigit{0}}\ {\isacharequal}{\kern0pt}\ {\isacharparenleft}{\kern0pt}f\ {\isasymcirc}\isactrlsub c\ a{\isacharparenright}{\kern0pt}\ {\isasymcirc}\isactrlsub c\ p{\isadigit{1}}{\isachardoublequoteclose}\isanewline
\ \ \ \ \isacommand{using}\isamarkupfalse%
\ fa{\isacharunderscore}{\kern0pt}type\ fibered{\isacharunderscore}{\kern0pt}product{\isacharunderscore}{\kern0pt}proj{\isacharunderscore}{\kern0pt}eq\ \isacommand{by}\isamarkupfalse%
\ auto\isanewline
\ \ \isacommand{have}\isamarkupfalse%
\ ga{\isacharunderscore}{\kern0pt}coequalizes{\isacharcolon}{\kern0pt}\ {\isachardoublequoteopen}{\isacharparenleft}{\kern0pt}g\ {\isasymcirc}\isactrlsub c\ a{\isacharparenright}{\kern0pt}\ {\isasymcirc}\isactrlsub c\ p{\isadigit{0}}\ {\isacharequal}{\kern0pt}\ {\isacharparenleft}{\kern0pt}g\ {\isasymcirc}\isactrlsub c\ a{\isacharparenright}{\kern0pt}\ {\isasymcirc}\isactrlsub c\ p{\isadigit{1}}{\isachardoublequoteclose}\isanewline
\ \ \isacommand{proof}\isamarkupfalse%
\ {\isacharminus}{\kern0pt}\isanewline
\ \ \ \ \isacommand{from}\isamarkupfalse%
\ fa{\isacharunderscore}{\kern0pt}coequalizes\ \isacommand{have}\isamarkupfalse%
\ {\isachardoublequoteopen}n\ {\isasymcirc}\isactrlsub c\ {\isacharparenleft}{\kern0pt}{\isacharparenleft}{\kern0pt}g\ {\isasymcirc}\isactrlsub c\ a{\isacharparenright}{\kern0pt}\ {\isasymcirc}\isactrlsub c\ p{\isadigit{0}}{\isacharparenright}{\kern0pt}\ {\isacharequal}{\kern0pt}\ n\ {\isasymcirc}\isactrlsub c\ {\isacharparenleft}{\kern0pt}{\isacharparenleft}{\kern0pt}g\ {\isasymcirc}\isactrlsub c\ a{\isacharparenright}{\kern0pt}\ {\isasymcirc}\isactrlsub c\ p{\isadigit{1}}{\isacharparenright}{\kern0pt}{\isachardoublequoteclose}\isanewline
\ \ \ \ \ \ \isacommand{by}\isamarkupfalse%
\ {\isacharparenleft}{\kern0pt}auto{\isacharcomma}{\kern0pt}\ typecheck{\isacharunderscore}{\kern0pt}cfuncs{\isacharcomma}{\kern0pt}\ auto\ simp\ add{\isacharcolon}{\kern0pt}\ f{\isacharunderscore}{\kern0pt}eq{\isacharunderscore}{\kern0pt}ng\ comp{\isacharunderscore}{\kern0pt}associative{\isadigit{2}}{\isacharparenright}{\kern0pt}\isanewline
\ \ \ \ \isacommand{then}\isamarkupfalse%
\ \isacommand{show}\isamarkupfalse%
\ {\isachardoublequoteopen}{\isacharparenleft}{\kern0pt}g\ {\isasymcirc}\isactrlsub c\ a{\isacharparenright}{\kern0pt}\ {\isasymcirc}\isactrlsub c\ p{\isadigit{0}}\ {\isacharequal}{\kern0pt}\ {\isacharparenleft}{\kern0pt}g\ {\isasymcirc}\isactrlsub c\ a{\isacharparenright}{\kern0pt}\ {\isasymcirc}\isactrlsub c\ p{\isadigit{1}}{\isachardoublequoteclose}\isanewline
\ \ \ \ \ \ \isacommand{using}\isamarkupfalse%
\ n{\isacharunderscore}{\kern0pt}mono\ \isacommand{unfolding}\isamarkupfalse%
\ monomorphism{\isacharunderscore}{\kern0pt}def{\isadigit{2}}\ \isacommand{by}\isamarkupfalse%
\ {\isacharparenleft}{\kern0pt}auto{\isacharcomma}{\kern0pt}\ typecheck{\isacharunderscore}{\kern0pt}cfuncs{\isacharunderscore}{\kern0pt}prems{\isacharcomma}{\kern0pt}\ meson{\isacharparenright}{\kern0pt}\isanewline
\ \ \isacommand{qed}\isamarkupfalse%
\isanewline
\isanewline
\ \ \isacommand{have}\isamarkupfalse%
\ {\isachardoublequoteopen}{\isasymforall}h\ F{\isachardot}{\kern0pt}\ h\ {\isacharcolon}{\kern0pt}\ A\ {\isasymrightarrow}\ F\ {\isasymand}\ h\ {\isasymcirc}\isactrlsub c\ p{\isadigit{0}}\ {\isacharequal}{\kern0pt}\ h\ {\isasymcirc}\isactrlsub c\ p{\isadigit{1}}\ {\isasymlongrightarrow}\ {\isacharparenleft}{\kern0pt}{\isasymexists}{\isacharbang}{\kern0pt}k{\isachardot}{\kern0pt}\ k\ {\isacharcolon}{\kern0pt}\ f{\isasymlparr}A{\isasymrparr}\isactrlbsub a\isactrlesub \ {\isasymrightarrow}\ F\ {\isasymand}\ k\ {\isasymcirc}\isactrlsub c\ f{\isasymrestriction}\isactrlbsub {\isacharparenleft}{\kern0pt}A{\isacharcomma}{\kern0pt}\ a{\isacharparenright}{\kern0pt}\isactrlesub \ {\isacharequal}{\kern0pt}\ h{\isacharparenright}{\kern0pt}{\isachardoublequoteclose}\isanewline
\ \ \ \ \isacommand{using}\isamarkupfalse%
\ image{\isacharunderscore}{\kern0pt}rest{\isacharunderscore}{\kern0pt}map{\isacharunderscore}{\kern0pt}coequalizer{\isacharbrackleft}{\kern0pt}\isakeyword{where}\ n{\isacharequal}{\kern0pt}a{\isacharbrackright}{\kern0pt}\ \isacommand{unfolding}\isamarkupfalse%
\ coequalizer{\isacharunderscore}{\kern0pt}def\ \isanewline
\ \ \ \ \isacommand{by}\isamarkupfalse%
\ {\isacharparenleft}{\kern0pt}simp{\isacharcomma}{\kern0pt}\ typecheck{\isacharunderscore}{\kern0pt}cfuncs{\isacharcomma}{\kern0pt}\ auto\ simp\ add{\isacharcolon}{\kern0pt}\ cfunc{\isacharunderscore}{\kern0pt}type{\isacharunderscore}{\kern0pt}def{\isacharparenright}{\kern0pt}\isanewline
\ \ \isacommand{then}\isamarkupfalse%
\ \isacommand{obtain}\isamarkupfalse%
\ k\ \isakeyword{where}\ k{\isacharunderscore}{\kern0pt}type{\isacharbrackleft}{\kern0pt}type{\isacharunderscore}{\kern0pt}rule{\isacharbrackright}{\kern0pt}{\isacharcolon}{\kern0pt}\ {\isachardoublequoteopen}k\ {\isacharcolon}{\kern0pt}\ f{\isasymlparr}A{\isasymrparr}\isactrlbsub a\isactrlesub \ {\isasymrightarrow}\ B{\isachardoublequoteclose}\ \isakeyword{and}\ k{\isacharunderscore}{\kern0pt}e{\isacharunderscore}{\kern0pt}eq{\isacharunderscore}{\kern0pt}g{\isacharcolon}{\kern0pt}\ {\isachardoublequoteopen}k\ {\isasymcirc}\isactrlsub c\ f{\isasymrestriction}\isactrlbsub {\isacharparenleft}{\kern0pt}A{\isacharcomma}{\kern0pt}\ a{\isacharparenright}{\kern0pt}\isactrlesub \ {\isacharequal}{\kern0pt}\ g\ {\isasymcirc}\isactrlsub c\ a{\isachardoublequoteclose}\isanewline
\ \ \ \ \isacommand{using}\isamarkupfalse%
\ ga{\isacharunderscore}{\kern0pt}coequalizes\ \isacommand{by}\isamarkupfalse%
\ {\isacharparenleft}{\kern0pt}typecheck{\isacharunderscore}{\kern0pt}cfuncs{\isacharcomma}{\kern0pt}\ blast{\isacharparenright}{\kern0pt}\isanewline
\isanewline
\ \ \isacommand{then}\isamarkupfalse%
\ \isacommand{have}\isamarkupfalse%
\ {\isachardoublequoteopen}n\ {\isasymcirc}\isactrlsub c\ k\ {\isacharequal}{\kern0pt}\ {\isacharbrackleft}{\kern0pt}f{\isasymlparr}A{\isasymrparr}\isactrlbsub a\isactrlesub {\isacharbrackright}{\kern0pt}map{\isachardoublequoteclose}\isanewline
\ \ \ \ \isacommand{by}\isamarkupfalse%
\ {\isacharparenleft}{\kern0pt}typecheck{\isacharunderscore}{\kern0pt}cfuncs{\isacharcomma}{\kern0pt}\ smt\ {\isacharparenleft}{\kern0pt}z{\isadigit{3}}{\isacharparenright}{\kern0pt}\ comp{\isacharunderscore}{\kern0pt}associative{\isadigit{2}}\ f{\isacharunderscore}{\kern0pt}eq{\isacharunderscore}{\kern0pt}ng\ g{\isacharunderscore}{\kern0pt}type\ image{\isacharunderscore}{\kern0pt}rest{\isacharunderscore}{\kern0pt}map{\isacharunderscore}{\kern0pt}type\ image{\isacharunderscore}{\kern0pt}subobj{\isacharunderscore}{\kern0pt}map{\isacharunderscore}{\kern0pt}unique\ k{\isacharunderscore}{\kern0pt}e{\isacharunderscore}{\kern0pt}eq{\isacharunderscore}{\kern0pt}g{\isacharparenright}{\kern0pt}\isanewline
\ \ \isacommand{then}\isamarkupfalse%
\ \isacommand{show}\isamarkupfalse%
\ {\isachardoublequoteopen}{\isacharparenleft}{\kern0pt}f{\isasymlparr}A{\isasymrparr}\isactrlbsub a\isactrlesub {\isacharcomma}{\kern0pt}\ {\isacharbrackleft}{\kern0pt}f{\isasymlparr}A{\isasymrparr}\isactrlbsub a\isactrlesub {\isacharbrackright}{\kern0pt}map{\isacharparenright}{\kern0pt}\ {\isasymsubseteq}\isactrlbsub Y\isactrlesub \ {\isacharparenleft}{\kern0pt}B{\isacharcomma}{\kern0pt}\ n{\isacharparenright}{\kern0pt}{\isachardoublequoteclose}\isanewline
\ \ \ \ \isacommand{unfolding}\isamarkupfalse%
\ relative{\isacharunderscore}{\kern0pt}subset{\isacharunderscore}{\kern0pt}def{\isadigit{2}}\ \isacommand{using}\isamarkupfalse%
\ n{\isacharunderscore}{\kern0pt}mono\ image{\isacharunderscore}{\kern0pt}subobj{\isacharunderscore}{\kern0pt}map{\isacharunderscore}{\kern0pt}mono\isanewline
\ \ \ \ \isacommand{by}\isamarkupfalse%
\ {\isacharparenleft}{\kern0pt}typecheck{\isacharunderscore}{\kern0pt}cfuncs{\isacharcomma}{\kern0pt}\ auto{\isacharcomma}{\kern0pt}\ rule{\isacharunderscore}{\kern0pt}tac\ x{\isacharequal}{\kern0pt}k\ \isakeyword{in}\ exI{\isacharcomma}{\kern0pt}\ typecheck{\isacharunderscore}{\kern0pt}cfuncs{\isacharparenright}{\kern0pt}\isanewline
\isacommand{qed}\isamarkupfalse%
%
\endisatagproof
{\isafoldproof}%
%
\isadelimproof
\isanewline
%
\endisadelimproof
\isanewline
\isacommand{lemma}\isamarkupfalse%
\ images{\isacharunderscore}{\kern0pt}iso{\isacharcolon}{\kern0pt}\isanewline
\ \ \isakeyword{assumes}\ f{\isacharunderscore}{\kern0pt}type{\isacharbrackleft}{\kern0pt}type{\isacharunderscore}{\kern0pt}rule{\isacharbrackright}{\kern0pt}{\isacharcolon}{\kern0pt}\ {\isachardoublequoteopen}f\ {\isacharcolon}{\kern0pt}\ X\ {\isasymrightarrow}\ Y{\isachardoublequoteclose}\isanewline
\ \ \isakeyword{assumes}\ m{\isacharunderscore}{\kern0pt}type{\isacharbrackleft}{\kern0pt}type{\isacharunderscore}{\kern0pt}rule{\isacharbrackright}{\kern0pt}{\isacharcolon}{\kern0pt}\ {\isachardoublequoteopen}m\ {\isacharcolon}{\kern0pt}\ Z\ {\isasymrightarrow}\ X{\isachardoublequoteclose}\ \isakeyword{and}\ n{\isacharunderscore}{\kern0pt}type{\isacharbrackleft}{\kern0pt}type{\isacharunderscore}{\kern0pt}rule{\isacharbrackright}{\kern0pt}{\isacharcolon}{\kern0pt}\ {\isachardoublequoteopen}n\ {\isacharcolon}{\kern0pt}\ A\ {\isasymrightarrow}\ Z{\isachardoublequoteclose}\ \isanewline
\ \ \isakeyword{shows}\ {\isachardoublequoteopen}{\isacharparenleft}{\kern0pt}f\ {\isasymcirc}\isactrlsub c\ m{\isacharparenright}{\kern0pt}{\isasymlparr}A{\isasymrparr}\isactrlbsub n\isactrlesub \ {\isasymcong}\ f{\isasymlparr}A{\isasymrparr}\isactrlbsub m\ {\isasymcirc}\isactrlsub c\ n\isactrlesub {\isachardoublequoteclose}\isanewline
%
\isadelimproof
%
\endisadelimproof
%
\isatagproof
\isacommand{proof}\isamarkupfalse%
\ {\isacharminus}{\kern0pt}\isanewline
\ \ \isacommand{have}\isamarkupfalse%
\ f{\isacharunderscore}{\kern0pt}m{\isacharunderscore}{\kern0pt}image{\isacharunderscore}{\kern0pt}coequalizer{\isacharcolon}{\kern0pt}\isanewline
\ \ \ \ {\isachardoublequoteopen}coequalizer\ {\isacharparenleft}{\kern0pt}{\isacharparenleft}{\kern0pt}f\ {\isasymcirc}\isactrlsub c\ m{\isacharparenright}{\kern0pt}{\isasymlparr}A{\isasymrparr}\isactrlbsub n\isactrlesub {\isacharparenright}{\kern0pt}\ {\isacharparenleft}{\kern0pt}{\isacharparenleft}{\kern0pt}f\ {\isasymcirc}\isactrlsub c\ m{\isacharparenright}{\kern0pt}{\isasymrestriction}\isactrlbsub {\isacharparenleft}{\kern0pt}A{\isacharcomma}{\kern0pt}\ n{\isacharparenright}{\kern0pt}\isactrlesub {\isacharparenright}{\kern0pt}\ \isanewline
\ \ \ \ \ \ {\isacharparenleft}{\kern0pt}fibered{\isacharunderscore}{\kern0pt}product{\isacharunderscore}{\kern0pt}left{\isacharunderscore}{\kern0pt}proj\ A\ {\isacharparenleft}{\kern0pt}f\ {\isasymcirc}\isactrlsub c\ m\ {\isasymcirc}\isactrlsub c\ n{\isacharparenright}{\kern0pt}\ {\isacharparenleft}{\kern0pt}f\ {\isasymcirc}\isactrlsub c\ m\ {\isasymcirc}\isactrlsub c\ n{\isacharparenright}{\kern0pt}\ A{\isacharparenright}{\kern0pt}\ \isanewline
\ \ \ \ \ \ {\isacharparenleft}{\kern0pt}fibered{\isacharunderscore}{\kern0pt}product{\isacharunderscore}{\kern0pt}right{\isacharunderscore}{\kern0pt}proj\ A\ {\isacharparenleft}{\kern0pt}f\ {\isasymcirc}\isactrlsub c\ m\ {\isasymcirc}\isactrlsub c\ n{\isacharparenright}{\kern0pt}\ {\isacharparenleft}{\kern0pt}f\ {\isasymcirc}\isactrlsub c\ m\ {\isasymcirc}\isactrlsub c\ n{\isacharparenright}{\kern0pt}\ A{\isacharparenright}{\kern0pt}{\isachardoublequoteclose}\isanewline
\ \ \ \ \isacommand{by}\isamarkupfalse%
\ {\isacharparenleft}{\kern0pt}typecheck{\isacharunderscore}{\kern0pt}cfuncs{\isacharcomma}{\kern0pt}\ smt\ comp{\isacharunderscore}{\kern0pt}associative{\isadigit{2}}\ image{\isacharunderscore}{\kern0pt}restriction{\isacharunderscore}{\kern0pt}mapping{\isacharunderscore}{\kern0pt}def{\isadigit{2}}{\isacharparenright}{\kern0pt}\isanewline
\isanewline
\ \ \isacommand{have}\isamarkupfalse%
\ f{\isacharunderscore}{\kern0pt}image{\isacharunderscore}{\kern0pt}coequalizer{\isacharcolon}{\kern0pt}\isanewline
\ \ \ \ {\isachardoublequoteopen}coequalizer\ {\isacharparenleft}{\kern0pt}f{\isasymlparr}A{\isasymrparr}\isactrlbsub m\ {\isasymcirc}\isactrlsub c\ n\isactrlesub {\isacharparenright}{\kern0pt}\ {\isacharparenleft}{\kern0pt}f{\isasymrestriction}\isactrlbsub {\isacharparenleft}{\kern0pt}A{\isacharcomma}{\kern0pt}\ m\ {\isasymcirc}\isactrlsub c\ n{\isacharparenright}{\kern0pt}\isactrlesub {\isacharparenright}{\kern0pt}\ \isanewline
\ \ \ \ \ \ {\isacharparenleft}{\kern0pt}fibered{\isacharunderscore}{\kern0pt}product{\isacharunderscore}{\kern0pt}left{\isacharunderscore}{\kern0pt}proj\ A\ {\isacharparenleft}{\kern0pt}f\ {\isasymcirc}\isactrlsub c\ m\ {\isasymcirc}\isactrlsub c\ n{\isacharparenright}{\kern0pt}\ {\isacharparenleft}{\kern0pt}f\ {\isasymcirc}\isactrlsub c\ m\ {\isasymcirc}\isactrlsub c\ n{\isacharparenright}{\kern0pt}\ A{\isacharparenright}{\kern0pt}\ \isanewline
\ \ \ \ \ \ {\isacharparenleft}{\kern0pt}fibered{\isacharunderscore}{\kern0pt}product{\isacharunderscore}{\kern0pt}right{\isacharunderscore}{\kern0pt}proj\ A\ {\isacharparenleft}{\kern0pt}f\ {\isasymcirc}\isactrlsub c\ m\ {\isasymcirc}\isactrlsub c\ n{\isacharparenright}{\kern0pt}\ {\isacharparenleft}{\kern0pt}f\ {\isasymcirc}\isactrlsub c\ m\ {\isasymcirc}\isactrlsub c\ n{\isacharparenright}{\kern0pt}\ A{\isacharparenright}{\kern0pt}{\isachardoublequoteclose}\isanewline
\ \ \ \ \isacommand{by}\isamarkupfalse%
\ {\isacharparenleft}{\kern0pt}typecheck{\isacharunderscore}{\kern0pt}cfuncs{\isacharcomma}{\kern0pt}\ smt\ comp{\isacharunderscore}{\kern0pt}associative{\isadigit{2}}\ image{\isacharunderscore}{\kern0pt}restriction{\isacharunderscore}{\kern0pt}mapping{\isacharunderscore}{\kern0pt}def{\isadigit{2}}{\isacharparenright}{\kern0pt}\isanewline
\isanewline
\ \ \isacommand{from}\isamarkupfalse%
\ f{\isacharunderscore}{\kern0pt}m{\isacharunderscore}{\kern0pt}image{\isacharunderscore}{\kern0pt}coequalizer\ f{\isacharunderscore}{\kern0pt}image{\isacharunderscore}{\kern0pt}coequalizer\isanewline
\ \ \isacommand{show}\isamarkupfalse%
\ {\isachardoublequoteopen}{\isacharparenleft}{\kern0pt}f\ {\isasymcirc}\isactrlsub c\ m{\isacharparenright}{\kern0pt}{\isasymlparr}A{\isasymrparr}\isactrlbsub n\isactrlesub \ {\isasymcong}\ f{\isasymlparr}A{\isasymrparr}\isactrlbsub m\ {\isasymcirc}\isactrlsub c\ n\isactrlesub {\isachardoublequoteclose}\isanewline
\ \ \ \ \isacommand{by}\isamarkupfalse%
\ {\isacharparenleft}{\kern0pt}meson\ coequalizer{\isacharunderscore}{\kern0pt}unique{\isacharparenright}{\kern0pt}\isanewline
\isacommand{qed}\isamarkupfalse%
%
\endisatagproof
{\isafoldproof}%
%
\isadelimproof
\isanewline
%
\endisadelimproof
\isanewline
\isacommand{lemma}\isamarkupfalse%
\ image{\isacharunderscore}{\kern0pt}subset{\isacharunderscore}{\kern0pt}conv{\isacharcolon}{\kern0pt}\isanewline
\ \ \isakeyword{assumes}\ f{\isacharunderscore}{\kern0pt}type{\isacharbrackleft}{\kern0pt}type{\isacharunderscore}{\kern0pt}rule{\isacharbrackright}{\kern0pt}{\isacharcolon}{\kern0pt}\ {\isachardoublequoteopen}f\ {\isacharcolon}{\kern0pt}\ X\ {\isasymrightarrow}\ Y{\isachardoublequoteclose}\isanewline
\ \ \isakeyword{assumes}\ m{\isacharunderscore}{\kern0pt}type{\isacharbrackleft}{\kern0pt}type{\isacharunderscore}{\kern0pt}rule{\isacharbrackright}{\kern0pt}{\isacharcolon}{\kern0pt}\ {\isachardoublequoteopen}m\ {\isacharcolon}{\kern0pt}\ Z\ {\isasymrightarrow}\ X{\isachardoublequoteclose}\ \isakeyword{and}\ n{\isacharunderscore}{\kern0pt}type{\isacharbrackleft}{\kern0pt}type{\isacharunderscore}{\kern0pt}rule{\isacharbrackright}{\kern0pt}{\isacharcolon}{\kern0pt}\ {\isachardoublequoteopen}n\ {\isacharcolon}{\kern0pt}\ A\ {\isasymrightarrow}\ Z{\isachardoublequoteclose}\ \isanewline
\ \ \isakeyword{shows}\ {\isachardoublequoteopen}{\isasymexists}i{\isachardot}{\kern0pt}\ {\isacharparenleft}{\kern0pt}{\isacharparenleft}{\kern0pt}f\ {\isasymcirc}\isactrlsub c\ m{\isacharparenright}{\kern0pt}{\isasymlparr}A{\isasymrparr}\isactrlbsub n\isactrlesub {\isacharcomma}{\kern0pt}\ i{\isacharparenright}{\kern0pt}\ {\isasymsubseteq}\isactrlsub c\ B\ {\isasymLongrightarrow}\ {\isasymexists}j{\isachardot}{\kern0pt}\ {\isacharparenleft}{\kern0pt}f{\isasymlparr}A{\isasymrparr}\isactrlbsub m\ {\isasymcirc}\isactrlsub c\ n\isactrlesub {\isacharcomma}{\kern0pt}\ j{\isacharparenright}{\kern0pt}\ {\isasymsubseteq}\isactrlsub c\ B{\isachardoublequoteclose}\isanewline
%
\isadelimproof
%
\endisadelimproof
%
\isatagproof
\isacommand{proof}\isamarkupfalse%
\ {\isacharminus}{\kern0pt}\isanewline
\ \ \isacommand{assume}\isamarkupfalse%
\ {\isachardoublequoteopen}{\isasymexists}i{\isachardot}{\kern0pt}\ {\isacharparenleft}{\kern0pt}{\isacharparenleft}{\kern0pt}f\ {\isasymcirc}\isactrlsub c\ m{\isacharparenright}{\kern0pt}{\isasymlparr}A{\isasymrparr}\isactrlbsub n\isactrlesub {\isacharcomma}{\kern0pt}\ i{\isacharparenright}{\kern0pt}\ {\isasymsubseteq}\isactrlsub c\ B{\isachardoublequoteclose}\isanewline
\ \ \isacommand{then}\isamarkupfalse%
\ \isacommand{obtain}\isamarkupfalse%
\ i\ \isakeyword{where}\isanewline
\ \ \ \ i{\isacharunderscore}{\kern0pt}type{\isacharbrackleft}{\kern0pt}type{\isacharunderscore}{\kern0pt}rule{\isacharbrackright}{\kern0pt}{\isacharcolon}{\kern0pt}\ {\isachardoublequoteopen}i\ {\isacharcolon}{\kern0pt}\ {\isacharparenleft}{\kern0pt}f\ {\isasymcirc}\isactrlsub c\ m{\isacharparenright}{\kern0pt}{\isasymlparr}A{\isasymrparr}\isactrlbsub n\isactrlesub \ {\isasymrightarrow}\ B{\isachardoublequoteclose}\ \isakeyword{and}\isanewline
\ \ \ \ i{\isacharunderscore}{\kern0pt}mono{\isacharcolon}{\kern0pt}\ {\isachardoublequoteopen}monomorphism\ i{\isachardoublequoteclose}\isanewline
\ \ \ \ \isacommand{unfolding}\isamarkupfalse%
\ subobject{\isacharunderscore}{\kern0pt}of{\isacharunderscore}{\kern0pt}def\ \isacommand{by}\isamarkupfalse%
\ force\isanewline
\isanewline
\ \ \isacommand{have}\isamarkupfalse%
\ {\isachardoublequoteopen}{\isacharparenleft}{\kern0pt}f\ {\isasymcirc}\isactrlsub c\ m{\isacharparenright}{\kern0pt}{\isasymlparr}A{\isasymrparr}\isactrlbsub n\isactrlesub \ {\isasymcong}\ f{\isasymlparr}A{\isasymrparr}\isactrlbsub m\ {\isasymcirc}\isactrlsub c\ n\isactrlesub {\isachardoublequoteclose}\isanewline
\ \ \ \ \isacommand{using}\isamarkupfalse%
\ f{\isacharunderscore}{\kern0pt}type\ images{\isacharunderscore}{\kern0pt}iso\ m{\isacharunderscore}{\kern0pt}type\ n{\isacharunderscore}{\kern0pt}type\ \isacommand{by}\isamarkupfalse%
\ blast\isanewline
\ \ \isacommand{then}\isamarkupfalse%
\ \isacommand{obtain}\isamarkupfalse%
\ k\ \isakeyword{where}\isanewline
\ \ \ \ k{\isacharunderscore}{\kern0pt}type{\isacharbrackleft}{\kern0pt}type{\isacharunderscore}{\kern0pt}rule{\isacharbrackright}{\kern0pt}{\isacharcolon}{\kern0pt}\ {\isachardoublequoteopen}k\ {\isacharcolon}{\kern0pt}\ f{\isasymlparr}A{\isasymrparr}\isactrlbsub m\ {\isasymcirc}\isactrlsub c\ n\isactrlesub \ {\isasymrightarrow}\ {\isacharparenleft}{\kern0pt}f\ {\isasymcirc}\isactrlsub c\ m{\isacharparenright}{\kern0pt}{\isasymlparr}A{\isasymrparr}\isactrlbsub n\isactrlesub {\isachardoublequoteclose}\ \isakeyword{and}\isanewline
\ \ \ \ k{\isacharunderscore}{\kern0pt}mono{\isacharcolon}{\kern0pt}\ {\isachardoublequoteopen}monomorphism\ k{\isachardoublequoteclose}\isanewline
\ \ \ \ \isacommand{by}\isamarkupfalse%
\ {\isacharparenleft}{\kern0pt}meson\ is{\isacharunderscore}{\kern0pt}isomorphic{\isacharunderscore}{\kern0pt}def\ iso{\isacharunderscore}{\kern0pt}imp{\isacharunderscore}{\kern0pt}epi{\isacharunderscore}{\kern0pt}and{\isacharunderscore}{\kern0pt}monic\ isomorphic{\isacharunderscore}{\kern0pt}is{\isacharunderscore}{\kern0pt}symmetric{\isacharparenright}{\kern0pt}\isanewline
\ \ \isacommand{then}\isamarkupfalse%
\ \isacommand{show}\isamarkupfalse%
\ {\isachardoublequoteopen}{\isasymexists}j{\isachardot}{\kern0pt}\ {\isacharparenleft}{\kern0pt}f{\isasymlparr}A{\isasymrparr}\isactrlbsub m\ {\isasymcirc}\isactrlsub c\ n\isactrlesub {\isacharcomma}{\kern0pt}\ j{\isacharparenright}{\kern0pt}\ {\isasymsubseteq}\isactrlsub c\ B{\isachardoublequoteclose}\isanewline
\ \ \ \ \isacommand{unfolding}\isamarkupfalse%
\ subobject{\isacharunderscore}{\kern0pt}of{\isacharunderscore}{\kern0pt}def\ \isacommand{using}\isamarkupfalse%
\ composition{\isacharunderscore}{\kern0pt}of{\isacharunderscore}{\kern0pt}monic{\isacharunderscore}{\kern0pt}pair{\isacharunderscore}{\kern0pt}is{\isacharunderscore}{\kern0pt}monic\ i{\isacharunderscore}{\kern0pt}mono\isanewline
\ \ \ \ \isacommand{by}\isamarkupfalse%
\ {\isacharparenleft}{\kern0pt}rule{\isacharunderscore}{\kern0pt}tac\ x{\isacharequal}{\kern0pt}{\isachardoublequoteopen}i\ {\isasymcirc}\isactrlsub c\ k{\isachardoublequoteclose}\ \isakeyword{in}\ exI{\isacharcomma}{\kern0pt}\ typecheck{\isacharunderscore}{\kern0pt}cfuncs{\isacharcomma}{\kern0pt}\ simp\ add{\isacharcolon}{\kern0pt}\ cfunc{\isacharunderscore}{\kern0pt}type{\isacharunderscore}{\kern0pt}def{\isacharparenright}{\kern0pt}\isanewline
\isacommand{qed}\isamarkupfalse%
%
\endisatagproof
{\isafoldproof}%
%
\isadelimproof
\isanewline
%
\endisadelimproof
\isanewline
\isacommand{lemma}\isamarkupfalse%
\ image{\isacharunderscore}{\kern0pt}rel{\isacharunderscore}{\kern0pt}subset{\isacharunderscore}{\kern0pt}conv{\isacharcolon}{\kern0pt}\isanewline
\ \ \isakeyword{assumes}\ f{\isacharunderscore}{\kern0pt}type{\isacharbrackleft}{\kern0pt}type{\isacharunderscore}{\kern0pt}rule{\isacharbrackright}{\kern0pt}{\isacharcolon}{\kern0pt}\ {\isachardoublequoteopen}f\ {\isacharcolon}{\kern0pt}\ X\ {\isasymrightarrow}\ Y{\isachardoublequoteclose}\isanewline
\ \ \isakeyword{assumes}\ m{\isacharunderscore}{\kern0pt}type{\isacharbrackleft}{\kern0pt}type{\isacharunderscore}{\kern0pt}rule{\isacharbrackright}{\kern0pt}{\isacharcolon}{\kern0pt}\ {\isachardoublequoteopen}m\ {\isacharcolon}{\kern0pt}\ Z\ {\isasymrightarrow}\ X{\isachardoublequoteclose}\ \isakeyword{and}\ n{\isacharunderscore}{\kern0pt}type{\isacharbrackleft}{\kern0pt}type{\isacharunderscore}{\kern0pt}rule{\isacharbrackright}{\kern0pt}{\isacharcolon}{\kern0pt}\ {\isachardoublequoteopen}n\ {\isacharcolon}{\kern0pt}\ A\ {\isasymrightarrow}\ Z{\isachardoublequoteclose}\isanewline
\ \ \isakeyword{assumes}\ rel{\isacharunderscore}{\kern0pt}sub{\isadigit{1}}{\isacharcolon}{\kern0pt}\ {\isachardoublequoteopen}{\isacharparenleft}{\kern0pt}{\isacharparenleft}{\kern0pt}f\ {\isasymcirc}\isactrlsub c\ m{\isacharparenright}{\kern0pt}{\isasymlparr}A{\isasymrparr}\isactrlbsub n\isactrlesub {\isacharcomma}{\kern0pt}\ {\isacharbrackleft}{\kern0pt}{\isacharparenleft}{\kern0pt}f\ {\isasymcirc}\isactrlsub c\ m{\isacharparenright}{\kern0pt}{\isasymlparr}A{\isasymrparr}\isactrlbsub n\isactrlesub {\isacharbrackright}{\kern0pt}map{\isacharparenright}{\kern0pt}\ {\isasymsubseteq}\isactrlbsub Y\isactrlesub \ {\isacharparenleft}{\kern0pt}B{\isacharcomma}{\kern0pt}b{\isacharparenright}{\kern0pt}{\isachardoublequoteclose}\isanewline
\ \ \isakeyword{shows}\ {\isachardoublequoteopen}{\isacharparenleft}{\kern0pt}f{\isasymlparr}A{\isasymrparr}\isactrlbsub m\ {\isasymcirc}\isactrlsub c\ n\isactrlesub {\isacharcomma}{\kern0pt}\ {\isacharbrackleft}{\kern0pt}f{\isasymlparr}A{\isasymrparr}\isactrlbsub m\ {\isasymcirc}\isactrlsub c\ n\isactrlesub {\isacharbrackright}{\kern0pt}map{\isacharparenright}{\kern0pt}\ {\isasymsubseteq}\isactrlbsub Y\isactrlesub \ {\isacharparenleft}{\kern0pt}B{\isacharcomma}{\kern0pt}b{\isacharparenright}{\kern0pt}{\isachardoublequoteclose}\isanewline
%
\isadelimproof
\ \ %
\endisadelimproof
%
\isatagproof
\isacommand{using}\isamarkupfalse%
\ rel{\isacharunderscore}{\kern0pt}sub{\isadigit{1}}\ image{\isacharunderscore}{\kern0pt}subobj{\isacharunderscore}{\kern0pt}map{\isacharunderscore}{\kern0pt}mono\isanewline
\ \ \isacommand{unfolding}\isamarkupfalse%
\ relative{\isacharunderscore}{\kern0pt}subset{\isacharunderscore}{\kern0pt}def{\isadigit{2}}\isanewline
\isacommand{proof}\isamarkupfalse%
\ {\isacharparenleft}{\kern0pt}typecheck{\isacharunderscore}{\kern0pt}cfuncs{\isacharcomma}{\kern0pt}\ auto{\isacharparenright}{\kern0pt}\isanewline
\ \ \isacommand{fix}\isamarkupfalse%
\ k\isanewline
\ \ \isacommand{assume}\isamarkupfalse%
\ k{\isacharunderscore}{\kern0pt}type{\isacharbrackleft}{\kern0pt}type{\isacharunderscore}{\kern0pt}rule{\isacharbrackright}{\kern0pt}{\isacharcolon}{\kern0pt}\ {\isachardoublequoteopen}k\ {\isacharcolon}{\kern0pt}\ {\isacharparenleft}{\kern0pt}f\ {\isasymcirc}\isactrlsub c\ m{\isacharparenright}{\kern0pt}{\isasymlparr}A{\isasymrparr}\isactrlbsub n\isactrlesub \ {\isasymrightarrow}\ B{\isachardoublequoteclose}\isanewline
\ \ \isacommand{assume}\isamarkupfalse%
\ b{\isacharunderscore}{\kern0pt}type{\isacharbrackleft}{\kern0pt}type{\isacharunderscore}{\kern0pt}rule{\isacharbrackright}{\kern0pt}{\isacharcolon}{\kern0pt}\ {\isachardoublequoteopen}b\ {\isacharcolon}{\kern0pt}\ B\ {\isasymrightarrow}\ Y{\isachardoublequoteclose}\isanewline
\ \ \isacommand{assume}\isamarkupfalse%
\ b{\isacharunderscore}{\kern0pt}mono{\isacharcolon}{\kern0pt}\ {\isachardoublequoteopen}monomorphism\ b{\isachardoublequoteclose}\isanewline
\ \ \isacommand{assume}\isamarkupfalse%
\ b{\isacharunderscore}{\kern0pt}k{\isacharunderscore}{\kern0pt}eq{\isacharunderscore}{\kern0pt}map{\isacharcolon}{\kern0pt}\ {\isachardoublequoteopen}b\ {\isasymcirc}\isactrlsub c\ k\ {\isacharequal}{\kern0pt}\ {\isacharbrackleft}{\kern0pt}{\isacharparenleft}{\kern0pt}f\ {\isasymcirc}\isactrlsub c\ m{\isacharparenright}{\kern0pt}{\isasymlparr}A{\isasymrparr}\isactrlbsub n\isactrlesub {\isacharbrackright}{\kern0pt}map{\isachardoublequoteclose}\isanewline
\isanewline
\ \ \isacommand{have}\isamarkupfalse%
\ f{\isacharunderscore}{\kern0pt}m{\isacharunderscore}{\kern0pt}image{\isacharunderscore}{\kern0pt}coequalizer{\isacharcolon}{\kern0pt}\isanewline
\ \ \ \ {\isachardoublequoteopen}coequalizer\ {\isacharparenleft}{\kern0pt}{\isacharparenleft}{\kern0pt}f\ {\isasymcirc}\isactrlsub c\ m{\isacharparenright}{\kern0pt}{\isasymlparr}A{\isasymrparr}\isactrlbsub n\isactrlesub {\isacharparenright}{\kern0pt}\ {\isacharparenleft}{\kern0pt}{\isacharparenleft}{\kern0pt}f\ {\isasymcirc}\isactrlsub c\ m{\isacharparenright}{\kern0pt}{\isasymrestriction}\isactrlbsub {\isacharparenleft}{\kern0pt}A{\isacharcomma}{\kern0pt}\ n{\isacharparenright}{\kern0pt}\isactrlesub {\isacharparenright}{\kern0pt}\ \isanewline
\ \ \ \ \ \ {\isacharparenleft}{\kern0pt}fibered{\isacharunderscore}{\kern0pt}product{\isacharunderscore}{\kern0pt}left{\isacharunderscore}{\kern0pt}proj\ A\ {\isacharparenleft}{\kern0pt}f\ {\isasymcirc}\isactrlsub c\ m\ {\isasymcirc}\isactrlsub c\ n{\isacharparenright}{\kern0pt}\ {\isacharparenleft}{\kern0pt}f\ {\isasymcirc}\isactrlsub c\ m\ {\isasymcirc}\isactrlsub c\ n{\isacharparenright}{\kern0pt}\ A{\isacharparenright}{\kern0pt}\ \isanewline
\ \ \ \ \ \ {\isacharparenleft}{\kern0pt}fibered{\isacharunderscore}{\kern0pt}product{\isacharunderscore}{\kern0pt}right{\isacharunderscore}{\kern0pt}proj\ A\ {\isacharparenleft}{\kern0pt}f\ {\isasymcirc}\isactrlsub c\ m\ {\isasymcirc}\isactrlsub c\ n{\isacharparenright}{\kern0pt}\ {\isacharparenleft}{\kern0pt}f\ {\isasymcirc}\isactrlsub c\ m\ {\isasymcirc}\isactrlsub c\ n{\isacharparenright}{\kern0pt}\ A{\isacharparenright}{\kern0pt}{\isachardoublequoteclose}\isanewline
\ \ \ \ \isacommand{by}\isamarkupfalse%
\ {\isacharparenleft}{\kern0pt}typecheck{\isacharunderscore}{\kern0pt}cfuncs{\isacharcomma}{\kern0pt}\ smt\ comp{\isacharunderscore}{\kern0pt}associative{\isadigit{2}}\ image{\isacharunderscore}{\kern0pt}restriction{\isacharunderscore}{\kern0pt}mapping{\isacharunderscore}{\kern0pt}def{\isadigit{2}}{\isacharparenright}{\kern0pt}\isanewline
\ \ \isacommand{then}\isamarkupfalse%
\ \isacommand{have}\isamarkupfalse%
\ f{\isacharunderscore}{\kern0pt}m{\isacharunderscore}{\kern0pt}image{\isacharunderscore}{\kern0pt}coequalises{\isacharcolon}{\kern0pt}\ \isanewline
\ \ \ \ \ \ {\isachardoublequoteopen}{\isacharparenleft}{\kern0pt}f\ {\isasymcirc}\isactrlsub c\ m{\isacharparenright}{\kern0pt}{\isasymrestriction}\isactrlbsub {\isacharparenleft}{\kern0pt}A{\isacharcomma}{\kern0pt}\ n{\isacharparenright}{\kern0pt}\isactrlesub \ {\isasymcirc}\isactrlsub c\ fibered{\isacharunderscore}{\kern0pt}product{\isacharunderscore}{\kern0pt}left{\isacharunderscore}{\kern0pt}proj\ A\ {\isacharparenleft}{\kern0pt}f\ {\isasymcirc}\isactrlsub c\ m\ {\isasymcirc}\isactrlsub c\ n{\isacharparenright}{\kern0pt}\ {\isacharparenleft}{\kern0pt}f\ {\isasymcirc}\isactrlsub c\ m\ {\isasymcirc}\isactrlsub c\ n{\isacharparenright}{\kern0pt}\ A\isanewline
\ \ \ \ \ \ \ \ {\isacharequal}{\kern0pt}\ {\isacharparenleft}{\kern0pt}f\ {\isasymcirc}\isactrlsub c\ m{\isacharparenright}{\kern0pt}{\isasymrestriction}\isactrlbsub {\isacharparenleft}{\kern0pt}A{\isacharcomma}{\kern0pt}\ n{\isacharparenright}{\kern0pt}\isactrlesub \ {\isasymcirc}\isactrlsub c\ fibered{\isacharunderscore}{\kern0pt}product{\isacharunderscore}{\kern0pt}right{\isacharunderscore}{\kern0pt}proj\ A\ {\isacharparenleft}{\kern0pt}f\ {\isasymcirc}\isactrlsub c\ m\ {\isasymcirc}\isactrlsub c\ n{\isacharparenright}{\kern0pt}\ {\isacharparenleft}{\kern0pt}f\ {\isasymcirc}\isactrlsub c\ m\ {\isasymcirc}\isactrlsub c\ n{\isacharparenright}{\kern0pt}\ A{\isachardoublequoteclose}\isanewline
\ \ \ \ \isacommand{by}\isamarkupfalse%
\ {\isacharparenleft}{\kern0pt}typecheck{\isacharunderscore}{\kern0pt}cfuncs{\isacharunderscore}{\kern0pt}prems{\isacharcomma}{\kern0pt}\ unfold\ coequalizer{\isacharunderscore}{\kern0pt}def{\isadigit{2}}{\isacharcomma}{\kern0pt}\ auto{\isacharparenright}{\kern0pt}\isanewline
\isanewline
\ \ \isacommand{have}\isamarkupfalse%
\ f{\isacharunderscore}{\kern0pt}image{\isacharunderscore}{\kern0pt}coequalizer{\isacharcolon}{\kern0pt}\isanewline
\ \ \ \ {\isachardoublequoteopen}coequalizer\ {\isacharparenleft}{\kern0pt}f{\isasymlparr}A{\isasymrparr}\isactrlbsub m\ {\isasymcirc}\isactrlsub c\ n\isactrlesub {\isacharparenright}{\kern0pt}\ {\isacharparenleft}{\kern0pt}f{\isasymrestriction}\isactrlbsub {\isacharparenleft}{\kern0pt}A{\isacharcomma}{\kern0pt}\ m\ {\isasymcirc}\isactrlsub c\ n{\isacharparenright}{\kern0pt}\isactrlesub {\isacharparenright}{\kern0pt}\ \isanewline
\ \ \ \ \ \ {\isacharparenleft}{\kern0pt}fibered{\isacharunderscore}{\kern0pt}product{\isacharunderscore}{\kern0pt}left{\isacharunderscore}{\kern0pt}proj\ A\ {\isacharparenleft}{\kern0pt}f\ {\isasymcirc}\isactrlsub c\ m\ {\isasymcirc}\isactrlsub c\ n{\isacharparenright}{\kern0pt}\ {\isacharparenleft}{\kern0pt}f\ {\isasymcirc}\isactrlsub c\ m\ {\isasymcirc}\isactrlsub c\ n{\isacharparenright}{\kern0pt}\ A{\isacharparenright}{\kern0pt}\ \isanewline
\ \ \ \ \ \ {\isacharparenleft}{\kern0pt}fibered{\isacharunderscore}{\kern0pt}product{\isacharunderscore}{\kern0pt}right{\isacharunderscore}{\kern0pt}proj\ A\ {\isacharparenleft}{\kern0pt}f\ {\isasymcirc}\isactrlsub c\ m\ {\isasymcirc}\isactrlsub c\ n{\isacharparenright}{\kern0pt}\ {\isacharparenleft}{\kern0pt}f\ {\isasymcirc}\isactrlsub c\ m\ {\isasymcirc}\isactrlsub c\ n{\isacharparenright}{\kern0pt}\ A{\isacharparenright}{\kern0pt}{\isachardoublequoteclose}\isanewline
\ \ \ \ \isacommand{by}\isamarkupfalse%
\ {\isacharparenleft}{\kern0pt}typecheck{\isacharunderscore}{\kern0pt}cfuncs{\isacharcomma}{\kern0pt}\ smt\ comp{\isacharunderscore}{\kern0pt}associative{\isadigit{2}}\ image{\isacharunderscore}{\kern0pt}restriction{\isacharunderscore}{\kern0pt}mapping{\isacharunderscore}{\kern0pt}def{\isadigit{2}}{\isacharparenright}{\kern0pt}\isanewline
\ \ \isacommand{then}\isamarkupfalse%
\ \isacommand{have}\isamarkupfalse%
\ {\isachardoublequoteopen}{\isasymAnd}\ h\ F{\isachardot}{\kern0pt}\ h\ {\isacharcolon}{\kern0pt}\ A\ {\isasymrightarrow}\ F\ {\isasymLongrightarrow}\isanewline
\ \ \ \ \ \ \ \ \ \ \ h\ {\isasymcirc}\isactrlsub c\ fibered{\isacharunderscore}{\kern0pt}product{\isacharunderscore}{\kern0pt}left{\isacharunderscore}{\kern0pt}proj\ A\ {\isacharparenleft}{\kern0pt}f\ {\isasymcirc}\isactrlsub c\ m\ {\isasymcirc}\isactrlsub c\ n{\isacharparenright}{\kern0pt}\ {\isacharparenleft}{\kern0pt}f\ {\isasymcirc}\isactrlsub c\ m\ {\isasymcirc}\isactrlsub c\ n{\isacharparenright}{\kern0pt}\ A\ {\isacharequal}{\kern0pt}\isanewline
\ \ \ \ \ \ \ \ \ \ \ h\ {\isasymcirc}\isactrlsub c\ fibered{\isacharunderscore}{\kern0pt}product{\isacharunderscore}{\kern0pt}right{\isacharunderscore}{\kern0pt}proj\ A\ {\isacharparenleft}{\kern0pt}f\ {\isasymcirc}\isactrlsub c\ m\ {\isasymcirc}\isactrlsub c\ n{\isacharparenright}{\kern0pt}\ {\isacharparenleft}{\kern0pt}f\ {\isasymcirc}\isactrlsub c\ m\ {\isasymcirc}\isactrlsub c\ n{\isacharparenright}{\kern0pt}\ A\ {\isasymLongrightarrow}\isanewline
\ \ \ \ \ \ \ \ \ \ \ {\isacharparenleft}{\kern0pt}{\isasymexists}{\isacharbang}{\kern0pt}k{\isachardot}{\kern0pt}\ k\ {\isacharcolon}{\kern0pt}\ f{\isasymlparr}A{\isasymrparr}\isactrlbsub m\ {\isasymcirc}\isactrlsub c\ n\isactrlesub \ {\isasymrightarrow}\ F\ {\isasymand}\ k\ {\isasymcirc}\isactrlsub c\ f{\isasymrestriction}\isactrlbsub {\isacharparenleft}{\kern0pt}A{\isacharcomma}{\kern0pt}\ m\ {\isasymcirc}\isactrlsub c\ n{\isacharparenright}{\kern0pt}\isactrlesub \ {\isacharequal}{\kern0pt}\ h{\isacharparenright}{\kern0pt}{\isachardoublequoteclose}\isanewline
\ \ \ \ \isacommand{by}\isamarkupfalse%
\ {\isacharparenleft}{\kern0pt}typecheck{\isacharunderscore}{\kern0pt}cfuncs{\isacharunderscore}{\kern0pt}prems{\isacharcomma}{\kern0pt}\ unfold\ coequalizer{\isacharunderscore}{\kern0pt}def{\isadigit{2}}{\isacharcomma}{\kern0pt}\ auto{\isacharparenright}{\kern0pt}\isanewline
\ \ \isacommand{then}\isamarkupfalse%
\ \isacommand{have}\isamarkupfalse%
\ {\isachardoublequoteopen}{\isasymexists}{\isacharbang}{\kern0pt}k{\isachardot}{\kern0pt}\ k\ {\isacharcolon}{\kern0pt}\ f{\isasymlparr}A{\isasymrparr}\isactrlbsub m\ {\isasymcirc}\isactrlsub c\ n\isactrlesub \ {\isasymrightarrow}\ {\isacharparenleft}{\kern0pt}f\ {\isasymcirc}\isactrlsub c\ m{\isacharparenright}{\kern0pt}{\isasymlparr}A{\isasymrparr}\isactrlbsub n\isactrlesub \ {\isasymand}\ k\ {\isasymcirc}\isactrlsub c\ f{\isasymrestriction}\isactrlbsub {\isacharparenleft}{\kern0pt}A{\isacharcomma}{\kern0pt}\ m\ {\isasymcirc}\isactrlsub c\ n{\isacharparenright}{\kern0pt}\isactrlesub \ {\isacharequal}{\kern0pt}\ {\isacharparenleft}{\kern0pt}f\ {\isasymcirc}\isactrlsub c\ m{\isacharparenright}{\kern0pt}{\isasymrestriction}\isactrlbsub {\isacharparenleft}{\kern0pt}A{\isacharcomma}{\kern0pt}\ n{\isacharparenright}{\kern0pt}\isactrlesub {\isachardoublequoteclose}\isanewline
\ \ \ \ \isacommand{using}\isamarkupfalse%
\ f{\isacharunderscore}{\kern0pt}m{\isacharunderscore}{\kern0pt}image{\isacharunderscore}{\kern0pt}coequalises\ \isacommand{by}\isamarkupfalse%
\ {\isacharparenleft}{\kern0pt}typecheck{\isacharunderscore}{\kern0pt}cfuncs{\isacharcomma}{\kern0pt}\ presburger{\isacharparenright}{\kern0pt}\isanewline
\ \ \isacommand{then}\isamarkupfalse%
\ \isacommand{obtain}\isamarkupfalse%
\ k{\isacharprime}{\kern0pt}\ \isakeyword{where}\ \isanewline
\ \ \ \ k{\isacharprime}{\kern0pt}{\isacharunderscore}{\kern0pt}type{\isacharbrackleft}{\kern0pt}type{\isacharunderscore}{\kern0pt}rule{\isacharbrackright}{\kern0pt}{\isacharcolon}{\kern0pt}\ {\isachardoublequoteopen}k{\isacharprime}{\kern0pt}\ {\isacharcolon}{\kern0pt}\ f{\isasymlparr}A{\isasymrparr}\isactrlbsub m\ {\isasymcirc}\isactrlsub c\ n\isactrlesub \ {\isasymrightarrow}\ {\isacharparenleft}{\kern0pt}f\ {\isasymcirc}\isactrlsub c\ m{\isacharparenright}{\kern0pt}{\isasymlparr}A{\isasymrparr}\isactrlbsub n\isactrlesub {\isachardoublequoteclose}\ \isakeyword{and}\isanewline
\ \ \ \ k{\isacharprime}{\kern0pt}{\isacharunderscore}{\kern0pt}eq{\isacharcolon}{\kern0pt}\ {\isachardoublequoteopen}k{\isacharprime}{\kern0pt}\ {\isasymcirc}\isactrlsub c\ f{\isasymrestriction}\isactrlbsub {\isacharparenleft}{\kern0pt}A{\isacharcomma}{\kern0pt}\ m\ {\isasymcirc}\isactrlsub c\ n{\isacharparenright}{\kern0pt}\isactrlesub \ {\isacharequal}{\kern0pt}\ {\isacharparenleft}{\kern0pt}f\ {\isasymcirc}\isactrlsub c\ m{\isacharparenright}{\kern0pt}{\isasymrestriction}\isactrlbsub {\isacharparenleft}{\kern0pt}A{\isacharcomma}{\kern0pt}\ n{\isacharparenright}{\kern0pt}\isactrlesub {\isachardoublequoteclose}\isanewline
\ \ \ \ \isacommand{by}\isamarkupfalse%
\ auto\isanewline
\isanewline
\ \ \isacommand{have}\isamarkupfalse%
\ k{\isacharprime}{\kern0pt}{\isacharunderscore}{\kern0pt}maps{\isacharunderscore}{\kern0pt}eq{\isacharcolon}{\kern0pt}\ {\isachardoublequoteopen}{\isacharbrackleft}{\kern0pt}f{\isasymlparr}A{\isasymrparr}\isactrlbsub m\ {\isasymcirc}\isactrlsub c\ n\isactrlesub {\isacharbrackright}{\kern0pt}map\ {\isacharequal}{\kern0pt}\ {\isacharbrackleft}{\kern0pt}{\isacharparenleft}{\kern0pt}f\ {\isasymcirc}\isactrlsub c\ m{\isacharparenright}{\kern0pt}{\isasymlparr}A{\isasymrparr}\isactrlbsub n\isactrlesub {\isacharbrackright}{\kern0pt}map\ {\isasymcirc}\isactrlsub c\ k{\isacharprime}{\kern0pt}{\isachardoublequoteclose}\isanewline
\ \ \ \ \isacommand{by}\isamarkupfalse%
\ {\isacharparenleft}{\kern0pt}typecheck{\isacharunderscore}{\kern0pt}cfuncs{\isacharcomma}{\kern0pt}\ smt\ {\isacharparenleft}{\kern0pt}z{\isadigit{3}}{\isacharparenright}{\kern0pt}\ comp{\isacharunderscore}{\kern0pt}associative{\isadigit{2}}\ image{\isacharunderscore}{\kern0pt}subobject{\isacharunderscore}{\kern0pt}mapping{\isacharunderscore}{\kern0pt}def{\isadigit{2}}\ k{\isacharprime}{\kern0pt}{\isacharunderscore}{\kern0pt}eq{\isacharparenright}{\kern0pt}\isanewline
\ \ \isacommand{have}\isamarkupfalse%
\ k{\isacharunderscore}{\kern0pt}mono{\isacharcolon}{\kern0pt}\ {\isachardoublequoteopen}monomorphism\ k{\isachardoublequoteclose}\isanewline
\ \ \ \ \isacommand{by}\isamarkupfalse%
\ {\isacharparenleft}{\kern0pt}metis\ b{\isacharunderscore}{\kern0pt}k{\isacharunderscore}{\kern0pt}eq{\isacharunderscore}{\kern0pt}map\ cfunc{\isacharunderscore}{\kern0pt}type{\isacharunderscore}{\kern0pt}def\ comp{\isacharunderscore}{\kern0pt}monic{\isacharunderscore}{\kern0pt}imp{\isacharunderscore}{\kern0pt}monic\ k{\isacharunderscore}{\kern0pt}type\ rel{\isacharunderscore}{\kern0pt}sub{\isadigit{1}}\ relative{\isacharunderscore}{\kern0pt}subset{\isacharunderscore}{\kern0pt}def{\isadigit{2}}{\isacharparenright}{\kern0pt}\isanewline
\ \ \isacommand{have}\isamarkupfalse%
\ k{\isacharprime}{\kern0pt}{\isacharunderscore}{\kern0pt}mono{\isacharcolon}{\kern0pt}\ {\isachardoublequoteopen}monomorphism\ k{\isacharprime}{\kern0pt}{\isachardoublequoteclose}\isanewline
\ \ \ \ \isacommand{by}\isamarkupfalse%
\ {\isacharparenleft}{\kern0pt}smt\ {\isacharparenleft}{\kern0pt}verit{\isacharcomma}{\kern0pt}\ ccfv{\isacharunderscore}{\kern0pt}SIG{\isacharparenright}{\kern0pt}\ cfunc{\isacharunderscore}{\kern0pt}type{\isacharunderscore}{\kern0pt}def\ comp{\isacharunderscore}{\kern0pt}monic{\isacharunderscore}{\kern0pt}imp{\isacharunderscore}{\kern0pt}monic\ comp{\isacharunderscore}{\kern0pt}type\ f{\isacharunderscore}{\kern0pt}type\ image{\isacharunderscore}{\kern0pt}subobject{\isacharunderscore}{\kern0pt}mapping{\isacharunderscore}{\kern0pt}def{\isadigit{2}}\ k{\isacharprime}{\kern0pt}{\isacharunderscore}{\kern0pt}maps{\isacharunderscore}{\kern0pt}eq\ k{\isacharprime}{\kern0pt}{\isacharunderscore}{\kern0pt}type\ m{\isacharunderscore}{\kern0pt}type\ n{\isacharunderscore}{\kern0pt}type{\isacharparenright}{\kern0pt}\isanewline
\isanewline
\ \ \isacommand{show}\isamarkupfalse%
\ {\isachardoublequoteopen}{\isasymexists}k{\isachardot}{\kern0pt}\ k\ {\isacharcolon}{\kern0pt}\ f{\isasymlparr}A{\isasymrparr}\isactrlbsub m\ {\isasymcirc}\isactrlsub c\ n\isactrlesub \ {\isasymrightarrow}\ B\ {\isasymand}\ b\ {\isasymcirc}\isactrlsub c\ k\ {\isacharequal}{\kern0pt}\ {\isacharbrackleft}{\kern0pt}f{\isasymlparr}A{\isasymrparr}\isactrlbsub m\ {\isasymcirc}\isactrlsub c\ n\isactrlesub {\isacharbrackright}{\kern0pt}map{\isachardoublequoteclose}\isanewline
\ \ \ \ \isacommand{by}\isamarkupfalse%
\ {\isacharparenleft}{\kern0pt}rule{\isacharunderscore}{\kern0pt}tac\ x{\isacharequal}{\kern0pt}{\isachardoublequoteopen}k\ {\isasymcirc}\isactrlsub c\ k{\isacharprime}{\kern0pt}{\isachardoublequoteclose}\ \isakeyword{in}\ exI{\isacharcomma}{\kern0pt}\ typecheck{\isacharunderscore}{\kern0pt}cfuncs{\isacharcomma}{\kern0pt}\ simp\ add{\isacharcolon}{\kern0pt}\ b{\isacharunderscore}{\kern0pt}k{\isacharunderscore}{\kern0pt}eq{\isacharunderscore}{\kern0pt}map\ comp{\isacharunderscore}{\kern0pt}associative{\isadigit{2}}\ k{\isacharprime}{\kern0pt}{\isacharunderscore}{\kern0pt}maps{\isacharunderscore}{\kern0pt}eq{\isacharparenright}{\kern0pt}\isanewline
\isacommand{qed}\isamarkupfalse%
%
\endisatagproof
{\isafoldproof}%
%
\isadelimproof
%
\endisadelimproof
%
\begin{isamarkuptext}%
The lemma below corresponds to Proposition 2.3.9 in Halvorson.%
\end{isamarkuptext}\isamarkuptrue%
\isacommand{lemma}\isamarkupfalse%
\ subset{\isacharunderscore}{\kern0pt}inv{\isacharunderscore}{\kern0pt}image{\isacharunderscore}{\kern0pt}iff{\isacharunderscore}{\kern0pt}image{\isacharunderscore}{\kern0pt}subset{\isacharcolon}{\kern0pt}\isanewline
\ \ \isakeyword{assumes}\ {\isachardoublequoteopen}{\isacharparenleft}{\kern0pt}A{\isacharcomma}{\kern0pt}a{\isacharparenright}{\kern0pt}\ {\isasymsubseteq}\isactrlsub c\ X{\isachardoublequoteclose}\ {\isachardoublequoteopen}{\isacharparenleft}{\kern0pt}B{\isacharcomma}{\kern0pt}m{\isacharparenright}{\kern0pt}\ {\isasymsubseteq}\isactrlsub c\ Y{\isachardoublequoteclose}\ \isanewline
\ \ \isakeyword{assumes}{\isacharbrackleft}{\kern0pt}type{\isacharunderscore}{\kern0pt}rule{\isacharbrackright}{\kern0pt}{\isacharcolon}{\kern0pt}\ {\isachardoublequoteopen}f\ {\isacharcolon}{\kern0pt}\ X\ {\isasymrightarrow}\ Y{\isachardoublequoteclose}\isanewline
\ \ \isakeyword{shows}\ {\isachardoublequoteopen}{\isacharparenleft}{\kern0pt}{\isacharparenleft}{\kern0pt}A{\isacharcomma}{\kern0pt}\ a{\isacharparenright}{\kern0pt}\ {\isasymsubseteq}\isactrlbsub X\isactrlesub \ {\isacharparenleft}{\kern0pt}f\isactrlsup {\isacharminus}{\kern0pt}\isactrlsup {\isadigit{1}}{\isasymlparr}B{\isasymrparr}\isactrlbsub m\isactrlesub {\isacharcomma}{\kern0pt}{\isacharbrackleft}{\kern0pt}f\isactrlsup {\isacharminus}{\kern0pt}\isactrlsup {\isadigit{1}}{\isasymlparr}B{\isasymrparr}\isactrlbsub m\isactrlesub {\isacharbrackright}{\kern0pt}map{\isacharparenright}{\kern0pt}{\isacharparenright}{\kern0pt}\ {\isacharequal}{\kern0pt}\ {\isacharparenleft}{\kern0pt}{\isacharparenleft}{\kern0pt}f{\isasymlparr}A{\isasymrparr}\isactrlbsub a\isactrlesub {\isacharcomma}{\kern0pt}\ {\isacharbrackleft}{\kern0pt}f{\isasymlparr}A{\isasymrparr}\isactrlbsub a\isactrlesub {\isacharbrackright}{\kern0pt}map{\isacharparenright}{\kern0pt}\ {\isasymsubseteq}\isactrlbsub Y\isactrlesub \ {\isacharparenleft}{\kern0pt}B{\isacharcomma}{\kern0pt}m{\isacharparenright}{\kern0pt}{\isacharparenright}{\kern0pt}{\isachardoublequoteclose}\isanewline
%
\isadelimproof
%
\endisadelimproof
%
\isatagproof
\isacommand{proof}\isamarkupfalse%
\ auto\isanewline
\ \ \isacommand{have}\isamarkupfalse%
\ b{\isacharunderscore}{\kern0pt}mono{\isacharcolon}{\kern0pt}\ {\isachardoublequoteopen}monomorphism{\isacharparenleft}{\kern0pt}m{\isacharparenright}{\kern0pt}{\isachardoublequoteclose}\isanewline
\ \ \ \ \isacommand{using}\isamarkupfalse%
\ assms{\isacharparenleft}{\kern0pt}{\isadigit{2}}{\isacharparenright}{\kern0pt}\ subobject{\isacharunderscore}{\kern0pt}of{\isacharunderscore}{\kern0pt}def{\isadigit{2}}\ \isacommand{by}\isamarkupfalse%
\ blast\isanewline
\ \ \isacommand{have}\isamarkupfalse%
\ b{\isacharunderscore}{\kern0pt}type{\isacharbrackleft}{\kern0pt}type{\isacharunderscore}{\kern0pt}rule{\isacharbrackright}{\kern0pt}{\isacharcolon}{\kern0pt}\ {\isachardoublequoteopen}m\ {\isacharcolon}{\kern0pt}\ B\ \ {\isasymrightarrow}\ Y{\isachardoublequoteclose}\isanewline
\ \ \ \ \isacommand{using}\isamarkupfalse%
\ assms{\isacharparenleft}{\kern0pt}{\isadigit{2}}{\isacharparenright}{\kern0pt}\ subobject{\isacharunderscore}{\kern0pt}of{\isacharunderscore}{\kern0pt}def{\isadigit{2}}\ \isacommand{by}\isamarkupfalse%
\ blast\isanewline
\ \ \isacommand{obtain}\isamarkupfalse%
\ m{\isacharprime}{\kern0pt}\ \isakeyword{where}\ m{\isacharprime}{\kern0pt}{\isacharunderscore}{\kern0pt}def{\isacharcolon}{\kern0pt}\ {\isachardoublequoteopen}m{\isacharprime}{\kern0pt}\ {\isacharequal}{\kern0pt}\ {\isacharbrackleft}{\kern0pt}f\isactrlsup {\isacharminus}{\kern0pt}\isactrlsup {\isadigit{1}}{\isasymlparr}B{\isasymrparr}\isactrlbsub m\isactrlesub {\isacharbrackright}{\kern0pt}map{\isachardoublequoteclose}\isanewline
\ \ \ \ \isacommand{by}\isamarkupfalse%
\ blast\isanewline
\ \ \isacommand{then}\isamarkupfalse%
\ \isacommand{have}\isamarkupfalse%
\ m{\isacharprime}{\kern0pt}{\isacharunderscore}{\kern0pt}type{\isacharbrackleft}{\kern0pt}type{\isacharunderscore}{\kern0pt}rule{\isacharbrackright}{\kern0pt}{\isacharcolon}{\kern0pt}\ {\isachardoublequoteopen}m{\isacharprime}{\kern0pt}\ {\isacharcolon}{\kern0pt}\ f\isactrlsup {\isacharminus}{\kern0pt}\isactrlsup {\isadigit{1}}{\isasymlparr}B{\isasymrparr}\isactrlbsub m\isactrlesub \ {\isasymrightarrow}\ X{\isachardoublequoteclose}\isanewline
\ \ \ \ \isacommand{using}\isamarkupfalse%
\ assms{\isacharparenleft}{\kern0pt}{\isadigit{3}}{\isacharparenright}{\kern0pt}\ b{\isacharunderscore}{\kern0pt}mono\ inverse{\isacharunderscore}{\kern0pt}image{\isacharunderscore}{\kern0pt}subobject{\isacharunderscore}{\kern0pt}mapping{\isacharunderscore}{\kern0pt}type\ m{\isacharprime}{\kern0pt}{\isacharunderscore}{\kern0pt}def\ \isacommand{by}\isamarkupfalse%
\ {\isacharparenleft}{\kern0pt}typecheck{\isacharunderscore}{\kern0pt}cfuncs{\isacharcomma}{\kern0pt}\ force{\isacharparenright}{\kern0pt}\isanewline
\isanewline
\ \ \isacommand{assume}\isamarkupfalse%
\ {\isachardoublequoteopen}{\isacharparenleft}{\kern0pt}A{\isacharcomma}{\kern0pt}\ a{\isacharparenright}{\kern0pt}\ {\isasymsubseteq}\isactrlbsub X\isactrlesub \ {\isacharparenleft}{\kern0pt}f\isactrlsup {\isacharminus}{\kern0pt}\isactrlsup {\isadigit{1}}{\isasymlparr}B{\isasymrparr}\isactrlbsub m\isactrlesub {\isacharcomma}{\kern0pt}\ {\isacharbrackleft}{\kern0pt}f\isactrlsup {\isacharminus}{\kern0pt}\isactrlsup {\isadigit{1}}{\isasymlparr}B{\isasymrparr}\isactrlbsub m\isactrlesub {\isacharbrackright}{\kern0pt}map{\isacharparenright}{\kern0pt}{\isachardoublequoteclose}\isanewline
\ \ \isacommand{then}\isamarkupfalse%
\ \isacommand{have}\isamarkupfalse%
\ a{\isacharunderscore}{\kern0pt}type{\isacharbrackleft}{\kern0pt}type{\isacharunderscore}{\kern0pt}rule{\isacharbrackright}{\kern0pt}{\isacharcolon}{\kern0pt}\ {\isachardoublequoteopen}a\ {\isacharcolon}{\kern0pt}\ A\ {\isasymrightarrow}\ X{\isachardoublequoteclose}\ \isakeyword{and}\isanewline
\ \ \ \ a{\isacharunderscore}{\kern0pt}mono{\isacharcolon}{\kern0pt}\ {\isachardoublequoteopen}monomorphism\ a{\isachardoublequoteclose}\ \isakeyword{and}\isanewline
\ \ \ \ k{\isacharunderscore}{\kern0pt}exists{\isacharcolon}{\kern0pt}\ {\isachardoublequoteopen}{\isasymexists}k{\isachardot}{\kern0pt}\ k\ {\isacharcolon}{\kern0pt}\ A\ {\isasymrightarrow}\ f\isactrlsup {\isacharminus}{\kern0pt}\isactrlsup {\isadigit{1}}{\isasymlparr}B{\isasymrparr}\isactrlbsub m\isactrlesub \ {\isasymand}\ {\isacharbrackleft}{\kern0pt}f\isactrlsup {\isacharminus}{\kern0pt}\isactrlsup {\isadigit{1}}{\isasymlparr}B{\isasymrparr}\isactrlbsub m\isactrlesub {\isacharbrackright}{\kern0pt}map\ {\isasymcirc}\isactrlsub c\ k\ {\isacharequal}{\kern0pt}\ a{\isachardoublequoteclose}\isanewline
\ \ \ \ \isacommand{unfolding}\isamarkupfalse%
\ relative{\isacharunderscore}{\kern0pt}subset{\isacharunderscore}{\kern0pt}def{\isadigit{2}}\ \isacommand{by}\isamarkupfalse%
\ auto\isanewline
\ \ \isacommand{then}\isamarkupfalse%
\ \isacommand{obtain}\isamarkupfalse%
\ k\ \isakeyword{where}\ k{\isacharunderscore}{\kern0pt}type{\isacharbrackleft}{\kern0pt}type{\isacharunderscore}{\kern0pt}rule{\isacharbrackright}{\kern0pt}{\isacharcolon}{\kern0pt}\ {\isachardoublequoteopen}k\ {\isacharcolon}{\kern0pt}\ A\ {\isasymrightarrow}\ f\isactrlsup {\isacharminus}{\kern0pt}\isactrlsup {\isadigit{1}}{\isasymlparr}B{\isasymrparr}\isactrlbsub m\isactrlesub {\isachardoublequoteclose}\ \isakeyword{and}\ k{\isacharunderscore}{\kern0pt}a{\isacharunderscore}{\kern0pt}eq{\isacharcolon}{\kern0pt}\ {\isachardoublequoteopen}{\isacharbrackleft}{\kern0pt}f\isactrlsup {\isacharminus}{\kern0pt}\isactrlsup {\isadigit{1}}{\isasymlparr}B{\isasymrparr}\isactrlbsub m\isactrlesub {\isacharbrackright}{\kern0pt}map\ {\isasymcirc}\isactrlsub c\ k\ {\isacharequal}{\kern0pt}\ a{\isachardoublequoteclose}\isanewline
\ \ \ \ \isacommand{by}\isamarkupfalse%
\ auto\isanewline
\isanewline
\ \ \isacommand{obtain}\isamarkupfalse%
\ d\ \isakeyword{where}\ d{\isacharunderscore}{\kern0pt}def{\isacharcolon}{\kern0pt}\ {\isachardoublequoteopen}d\ {\isacharequal}{\kern0pt}\ m{\isacharprime}{\kern0pt}\ {\isasymcirc}\isactrlsub c\ k{\isachardoublequoteclose}\isanewline
\ \ \ \ \isacommand{by}\isamarkupfalse%
\ simp\isanewline
\isanewline
\ \ \isacommand{obtain}\isamarkupfalse%
\ j\ \isakeyword{where}\ j{\isacharunderscore}{\kern0pt}def{\isacharcolon}{\kern0pt}\ {\isachardoublequoteopen}j\ {\isacharequal}{\kern0pt}\ {\isacharbrackleft}{\kern0pt}f{\isasymlparr}A{\isasymrparr}\isactrlbsub d\isactrlesub {\isacharbrackright}{\kern0pt}map{\isachardoublequoteclose}\isanewline
\ \ \ \ \isacommand{by}\isamarkupfalse%
\ simp\isanewline
\ \ \isacommand{then}\isamarkupfalse%
\ \isacommand{have}\isamarkupfalse%
\ j{\isacharunderscore}{\kern0pt}type{\isacharbrackleft}{\kern0pt}type{\isacharunderscore}{\kern0pt}rule{\isacharbrackright}{\kern0pt}{\isacharcolon}{\kern0pt}\ {\isachardoublequoteopen}j\ {\isacharcolon}{\kern0pt}\ f{\isasymlparr}A{\isasymrparr}\isactrlbsub d\isactrlesub \ {\isasymrightarrow}\ Y{\isachardoublequoteclose}\isanewline
\ \ \ \ \isacommand{using}\isamarkupfalse%
\ assms{\isacharparenleft}{\kern0pt}{\isadigit{3}}{\isacharparenright}{\kern0pt}\ comp{\isacharunderscore}{\kern0pt}type\ d{\isacharunderscore}{\kern0pt}def\ m{\isacharprime}{\kern0pt}{\isacharunderscore}{\kern0pt}type\ image{\isacharunderscore}{\kern0pt}subobj{\isacharunderscore}{\kern0pt}map{\isacharunderscore}{\kern0pt}type\ k{\isacharunderscore}{\kern0pt}type\ \isacommand{by}\isamarkupfalse%
\ presburger\isanewline
\isanewline
\ \ \isacommand{obtain}\isamarkupfalse%
\ e\ \isakeyword{where}\ e{\isacharunderscore}{\kern0pt}def{\isacharcolon}{\kern0pt}\ {\isachardoublequoteopen}e\ {\isacharequal}{\kern0pt}\ f{\isasymrestriction}\isactrlbsub {\isacharparenleft}{\kern0pt}A{\isacharcomma}{\kern0pt}\ d{\isacharparenright}{\kern0pt}\isactrlesub {\isachardoublequoteclose}\isanewline
\ \ \ \ \isacommand{by}\isamarkupfalse%
\ simp\isanewline
\ \ \isacommand{then}\isamarkupfalse%
\ \isacommand{have}\isamarkupfalse%
\ e{\isacharunderscore}{\kern0pt}type{\isacharbrackleft}{\kern0pt}type{\isacharunderscore}{\kern0pt}rule{\isacharbrackright}{\kern0pt}{\isacharcolon}{\kern0pt}\ {\isachardoublequoteopen}e\ {\isacharcolon}{\kern0pt}\ A\ {\isasymrightarrow}\ f{\isasymlparr}A{\isasymrparr}\isactrlbsub d\isactrlesub {\isachardoublequoteclose}\isanewline
\ \ \ \ \isacommand{using}\isamarkupfalse%
\ assms{\isacharparenleft}{\kern0pt}{\isadigit{3}}{\isacharparenright}{\kern0pt}\ comp{\isacharunderscore}{\kern0pt}type\ d{\isacharunderscore}{\kern0pt}def\ image{\isacharunderscore}{\kern0pt}rest{\isacharunderscore}{\kern0pt}map{\isacharunderscore}{\kern0pt}type\ k{\isacharunderscore}{\kern0pt}type\ m{\isacharprime}{\kern0pt}{\isacharunderscore}{\kern0pt}type\ \isacommand{by}\isamarkupfalse%
\ blast\isanewline
\isanewline
\ \ \isacommand{have}\isamarkupfalse%
\ je{\isacharunderscore}{\kern0pt}equals{\isacharcolon}{\kern0pt}\ {\isachardoublequoteopen}j\ {\isasymcirc}\isactrlsub c\ e\ {\isacharequal}{\kern0pt}\ f\ {\isasymcirc}\isactrlsub c\ m{\isacharprime}{\kern0pt}\ {\isasymcirc}\isactrlsub c\ k{\isachardoublequoteclose}\isanewline
\ \ \ \ \isacommand{by}\isamarkupfalse%
\ {\isacharparenleft}{\kern0pt}typecheck{\isacharunderscore}{\kern0pt}cfuncs{\isacharcomma}{\kern0pt}\ simp\ add{\isacharcolon}{\kern0pt}\ d{\isacharunderscore}{\kern0pt}def\ e{\isacharunderscore}{\kern0pt}def\ image{\isacharunderscore}{\kern0pt}subobj{\isacharunderscore}{\kern0pt}comp{\isacharunderscore}{\kern0pt}image{\isacharunderscore}{\kern0pt}rest\ j{\isacharunderscore}{\kern0pt}def{\isacharparenright}{\kern0pt}\isanewline
\isanewline
\ \ \isacommand{have}\isamarkupfalse%
\ {\isachardoublequoteopen}{\isacharparenleft}{\kern0pt}f\ {\isasymcirc}\isactrlsub c\ m{\isacharprime}{\kern0pt}\ {\isasymcirc}\isactrlsub c\ k{\isacharparenright}{\kern0pt}\ factorsthru\ m{\isachardoublequoteclose}\isanewline
\ \ \isacommand{proof}\isamarkupfalse%
{\isacharparenleft}{\kern0pt}typecheck{\isacharunderscore}{\kern0pt}cfuncs{\isacharcomma}{\kern0pt}\ unfold\ factors{\isacharunderscore}{\kern0pt}through{\isacharunderscore}{\kern0pt}def{\isadigit{2}}{\isacharparenright}{\kern0pt}\ \isanewline
\isanewline
\ \ \ \ \isacommand{obtain}\isamarkupfalse%
\ middle{\isacharunderscore}{\kern0pt}arrow\ \isakeyword{where}\ middle{\isacharunderscore}{\kern0pt}arrow{\isacharunderscore}{\kern0pt}def{\isacharcolon}{\kern0pt}\ \isanewline
\ \ \ \ \ \ {\isachardoublequoteopen}middle{\isacharunderscore}{\kern0pt}arrow\ {\isacharequal}{\kern0pt}\ {\isacharparenleft}{\kern0pt}right{\isacharunderscore}{\kern0pt}cart{\isacharunderscore}{\kern0pt}proj\ X\ B{\isacharparenright}{\kern0pt}\ {\isasymcirc}\isactrlsub c\ {\isacharparenleft}{\kern0pt}inverse{\isacharunderscore}{\kern0pt}image{\isacharunderscore}{\kern0pt}mapping\ f\ B\ m{\isacharparenright}{\kern0pt}{\isachardoublequoteclose}\isanewline
\ \ \ \ \ \ \isacommand{by}\isamarkupfalse%
\ simp\isanewline
\isanewline
\ \ \ \ \isacommand{then}\isamarkupfalse%
\ \isacommand{have}\isamarkupfalse%
\ middle{\isacharunderscore}{\kern0pt}arrow{\isacharunderscore}{\kern0pt}type{\isacharbrackleft}{\kern0pt}type{\isacharunderscore}{\kern0pt}rule{\isacharbrackright}{\kern0pt}{\isacharcolon}{\kern0pt}\ {\isachardoublequoteopen}middle{\isacharunderscore}{\kern0pt}arrow\ {\isacharcolon}{\kern0pt}\ f\isactrlsup {\isacharminus}{\kern0pt}\isactrlsup {\isadigit{1}}{\isasymlparr}B{\isasymrparr}\isactrlbsub m\isactrlesub \ {\isasymrightarrow}\ B{\isachardoublequoteclose}\isanewline
\ \ \ \ \ \ \isacommand{unfolding}\isamarkupfalse%
\ middle{\isacharunderscore}{\kern0pt}arrow{\isacharunderscore}{\kern0pt}def\ \isacommand{using}\isamarkupfalse%
\ b{\isacharunderscore}{\kern0pt}mono\ \isacommand{by}\isamarkupfalse%
\ {\isacharparenleft}{\kern0pt}typecheck{\isacharunderscore}{\kern0pt}cfuncs{\isacharparenright}{\kern0pt}\isanewline
\isanewline
\ \ \ \ \isacommand{show}\isamarkupfalse%
\ {\isachardoublequoteopen}{\isasymexists}h{\isachardot}{\kern0pt}\ h\ {\isacharcolon}{\kern0pt}\ A\ {\isasymrightarrow}\ B\ {\isasymand}\ m\ {\isasymcirc}\isactrlsub c\ h\ {\isacharequal}{\kern0pt}\ f\ {\isasymcirc}\isactrlsub c\ m{\isacharprime}{\kern0pt}\ {\isasymcirc}\isactrlsub c\ k{\isachardoublequoteclose}\isanewline
\ \ \ \ \ \ \isacommand{by}\isamarkupfalse%
\ {\isacharparenleft}{\kern0pt}rule{\isacharunderscore}{\kern0pt}tac\ x{\isacharequal}{\kern0pt}{\isachardoublequoteopen}middle{\isacharunderscore}{\kern0pt}arrow\ {\isasymcirc}\isactrlsub c\ k{\isachardoublequoteclose}\ \isakeyword{in}\ exI{\isacharcomma}{\kern0pt}\ typecheck{\isacharunderscore}{\kern0pt}cfuncs{\isacharcomma}{\kern0pt}\ \isanewline
\ \ \ \ \ \ \ \ \ \ simp\ add{\isacharcolon}{\kern0pt}\ b{\isacharunderscore}{\kern0pt}mono\ cfunc{\isacharunderscore}{\kern0pt}type{\isacharunderscore}{\kern0pt}def\ comp{\isacharunderscore}{\kern0pt}associative{\isadigit{2}}\ inverse{\isacharunderscore}{\kern0pt}image{\isacharunderscore}{\kern0pt}mapping{\isacharunderscore}{\kern0pt}eq\ inverse{\isacharunderscore}{\kern0pt}image{\isacharunderscore}{\kern0pt}subobject{\isacharunderscore}{\kern0pt}mapping{\isacharunderscore}{\kern0pt}def\ m{\isacharprime}{\kern0pt}{\isacharunderscore}{\kern0pt}def\ middle{\isacharunderscore}{\kern0pt}arrow{\isacharunderscore}{\kern0pt}def{\isacharparenright}{\kern0pt}\isanewline
\ \ \isacommand{qed}\isamarkupfalse%
\isanewline
\isanewline
\ \ \isacommand{then}\isamarkupfalse%
\ \isacommand{have}\isamarkupfalse%
\ {\isachardoublequoteopen}{\isacharparenleft}{\kern0pt}{\isacharparenleft}{\kern0pt}f\ {\isasymcirc}\isactrlsub c\ m{\isacharprime}{\kern0pt}\ {\isasymcirc}\isactrlsub c\ k{\isacharparenright}{\kern0pt}{\isasymlparr}A{\isasymrparr}\isactrlbsub id\isactrlsub c\ A\isactrlesub {\isacharcomma}{\kern0pt}\ {\isacharbrackleft}{\kern0pt}{\isacharparenleft}{\kern0pt}f\ {\isasymcirc}\isactrlsub c\ m{\isacharprime}{\kern0pt}\ {\isasymcirc}\isactrlsub c\ k{\isacharparenright}{\kern0pt}{\isasymlparr}A{\isasymrparr}\isactrlbsub id\isactrlsub c\ A\isactrlesub {\isacharbrackright}{\kern0pt}map{\isacharparenright}{\kern0pt}\ {\isasymsubseteq}\isactrlbsub Y\isactrlesub \ {\isacharparenleft}{\kern0pt}B{\isacharcomma}{\kern0pt}\ m{\isacharparenright}{\kern0pt}{\isachardoublequoteclose}\isanewline
\ \ \ \ \isacommand{by}\isamarkupfalse%
\ {\isacharparenleft}{\kern0pt}typecheck{\isacharunderscore}{\kern0pt}cfuncs{\isacharcomma}{\kern0pt}\ meson\ assms{\isacharparenleft}{\kern0pt}{\isadigit{2}}{\isacharparenright}{\kern0pt}\ image{\isacharunderscore}{\kern0pt}smallest{\isacharunderscore}{\kern0pt}subobject{\isacharparenright}{\kern0pt}\isanewline
\ \ \isacommand{then}\isamarkupfalse%
\ \isacommand{have}\isamarkupfalse%
\ {\isachardoublequoteopen}{\isacharparenleft}{\kern0pt}{\isacharparenleft}{\kern0pt}f\ {\isasymcirc}\isactrlsub c\ a{\isacharparenright}{\kern0pt}{\isasymlparr}A{\isasymrparr}\isactrlbsub id\isactrlsub c\ A\isactrlesub {\isacharcomma}{\kern0pt}\ {\isacharbrackleft}{\kern0pt}{\isacharparenleft}{\kern0pt}f\ {\isasymcirc}\isactrlsub c\ a{\isacharparenright}{\kern0pt}{\isasymlparr}A{\isasymrparr}\isactrlbsub id\isactrlsub c\ A\isactrlesub {\isacharbrackright}{\kern0pt}map{\isacharparenright}{\kern0pt}\ {\isasymsubseteq}\isactrlbsub Y\isactrlesub \ {\isacharparenleft}{\kern0pt}B{\isacharcomma}{\kern0pt}\ m{\isacharparenright}{\kern0pt}{\isachardoublequoteclose}\isanewline
\ \ \ \ \isacommand{by}\isamarkupfalse%
\ {\isacharparenleft}{\kern0pt}simp\ add{\isacharcolon}{\kern0pt}\ k{\isacharunderscore}{\kern0pt}a{\isacharunderscore}{\kern0pt}eq\ m{\isacharprime}{\kern0pt}{\isacharunderscore}{\kern0pt}def{\isacharparenright}{\kern0pt}\ \ \ \isanewline
\ \ \isacommand{then}\isamarkupfalse%
\ \isacommand{show}\isamarkupfalse%
\ {\isachardoublequoteopen}{\isacharparenleft}{\kern0pt}f{\isasymlparr}A{\isasymrparr}\isactrlbsub a\isactrlesub {\isacharcomma}{\kern0pt}\ {\isacharbrackleft}{\kern0pt}f{\isasymlparr}A{\isasymrparr}\isactrlbsub a\isactrlesub {\isacharbrackright}{\kern0pt}map{\isacharparenright}{\kern0pt}{\isasymsubseteq}\isactrlbsub Y\isactrlesub {\isacharparenleft}{\kern0pt}B{\isacharcomma}{\kern0pt}\ m{\isacharparenright}{\kern0pt}{\isachardoublequoteclose}\isanewline
\ \ \ \ \isacommand{by}\isamarkupfalse%
\ {\isacharparenleft}{\kern0pt}typecheck{\isacharunderscore}{\kern0pt}cfuncs{\isacharcomma}{\kern0pt}\ metis\ id{\isacharunderscore}{\kern0pt}right{\isacharunderscore}{\kern0pt}unit{\isadigit{2}}\ id{\isacharunderscore}{\kern0pt}type\ image{\isacharunderscore}{\kern0pt}rel{\isacharunderscore}{\kern0pt}subset{\isacharunderscore}{\kern0pt}conv{\isacharparenright}{\kern0pt}\isanewline
\isacommand{next}\isamarkupfalse%
\isanewline
\ \ \isacommand{have}\isamarkupfalse%
\ m{\isacharunderscore}{\kern0pt}mono{\isacharcolon}{\kern0pt}\ {\isachardoublequoteopen}monomorphism{\isacharparenleft}{\kern0pt}m{\isacharparenright}{\kern0pt}{\isachardoublequoteclose}\isanewline
\ \ \ \ \isacommand{using}\isamarkupfalse%
\ assms{\isacharparenleft}{\kern0pt}{\isadigit{2}}{\isacharparenright}{\kern0pt}\ subobject{\isacharunderscore}{\kern0pt}of{\isacharunderscore}{\kern0pt}def{\isadigit{2}}\ \isacommand{by}\isamarkupfalse%
\ blast\isanewline
\ \ \isacommand{have}\isamarkupfalse%
\ m{\isacharunderscore}{\kern0pt}type{\isacharbrackleft}{\kern0pt}type{\isacharunderscore}{\kern0pt}rule{\isacharbrackright}{\kern0pt}{\isacharcolon}{\kern0pt}\ {\isachardoublequoteopen}m\ {\isacharcolon}{\kern0pt}\ B\ \ {\isasymrightarrow}\ Y{\isachardoublequoteclose}\isanewline
\ \ \ \ \isacommand{using}\isamarkupfalse%
\ assms{\isacharparenleft}{\kern0pt}{\isadigit{2}}{\isacharparenright}{\kern0pt}\ subobject{\isacharunderscore}{\kern0pt}of{\isacharunderscore}{\kern0pt}def{\isadigit{2}}\ \isacommand{by}\isamarkupfalse%
\ blast\isanewline
\isanewline
\ \ \isacommand{assume}\isamarkupfalse%
\ {\isachardoublequoteopen}{\isacharparenleft}{\kern0pt}f{\isasymlparr}A{\isasymrparr}\isactrlbsub a\isactrlesub {\isacharcomma}{\kern0pt}\ {\isacharbrackleft}{\kern0pt}f{\isasymlparr}A{\isasymrparr}\isactrlbsub a\isactrlesub {\isacharbrackright}{\kern0pt}map{\isacharparenright}{\kern0pt}\ {\isasymsubseteq}\isactrlbsub Y\isactrlesub \ {\isacharparenleft}{\kern0pt}B{\isacharcomma}{\kern0pt}\ m{\isacharparenright}{\kern0pt}{\isachardoublequoteclose}\isanewline
\ \ \isacommand{then}\isamarkupfalse%
\ \isacommand{obtain}\isamarkupfalse%
\ s\ \isakeyword{where}\ \ \ \ \ \ \ \ \ \ \ \ \ \ \ \ \ \ \ \ \ \ \ \ \ \ \ \ \ \ \ \ \ \ \ \ \ \ \ \ \ \ \ \ \ \isanewline
\ \ \ \ \ \ s{\isacharunderscore}{\kern0pt}type{\isacharbrackleft}{\kern0pt}type{\isacharunderscore}{\kern0pt}rule{\isacharbrackright}{\kern0pt}{\isacharcolon}{\kern0pt}\ {\isachardoublequoteopen}s\ {\isacharcolon}{\kern0pt}\ f{\isasymlparr}A{\isasymrparr}\isactrlbsub a\isactrlesub \ {\isasymrightarrow}\ B{\isachardoublequoteclose}\ \isakeyword{and}\isanewline
\ \ \ \ \ \ m{\isacharunderscore}{\kern0pt}s{\isacharunderscore}{\kern0pt}eq{\isacharunderscore}{\kern0pt}subobj{\isacharunderscore}{\kern0pt}map{\isacharcolon}{\kern0pt}\ {\isachardoublequoteopen}m\ {\isasymcirc}\isactrlsub c\ s\ {\isacharequal}{\kern0pt}\ {\isacharbrackleft}{\kern0pt}f{\isasymlparr}A{\isasymrparr}\isactrlbsub a\isactrlesub {\isacharbrackright}{\kern0pt}map{\isachardoublequoteclose}\isanewline
\ \ \ \ \isacommand{unfolding}\isamarkupfalse%
\ relative{\isacharunderscore}{\kern0pt}subset{\isacharunderscore}{\kern0pt}def{\isadigit{2}}\ \isacommand{by}\isamarkupfalse%
\ auto\isanewline
\isanewline
\ \ \isacommand{have}\isamarkupfalse%
\ a{\isacharunderscore}{\kern0pt}mono{\isacharcolon}{\kern0pt}\ {\isachardoublequoteopen}monomorphism\ a{\isachardoublequoteclose}\isanewline
\ \ \ \ \isacommand{using}\isamarkupfalse%
\ assms{\isacharparenleft}{\kern0pt}{\isadigit{1}}{\isacharparenright}{\kern0pt}\ \isacommand{unfolding}\isamarkupfalse%
\ subobject{\isacharunderscore}{\kern0pt}of{\isacharunderscore}{\kern0pt}def{\isadigit{2}}\ \isacommand{by}\isamarkupfalse%
\ auto\isanewline
\isanewline
\ \ \isacommand{have}\isamarkupfalse%
\ pullback{\isacharunderscore}{\kern0pt}map{\isadigit{1}}{\isacharunderscore}{\kern0pt}type{\isacharbrackleft}{\kern0pt}type{\isacharunderscore}{\kern0pt}rule{\isacharbrackright}{\kern0pt}{\isacharcolon}{\kern0pt}\ {\isachardoublequoteopen}s\ {\isasymcirc}\isactrlsub c\ f{\isasymrestriction}\isactrlbsub {\isacharparenleft}{\kern0pt}A{\isacharcomma}{\kern0pt}\ a{\isacharparenright}{\kern0pt}\isactrlesub \ {\isacharcolon}{\kern0pt}\ A\ {\isasymrightarrow}\ B{\isachardoublequoteclose}\isanewline
\ \ \ \ \isacommand{using}\isamarkupfalse%
\ assms{\isacharparenleft}{\kern0pt}{\isadigit{1}}{\isacharparenright}{\kern0pt}\ \isacommand{unfolding}\isamarkupfalse%
\ subobject{\isacharunderscore}{\kern0pt}of{\isacharunderscore}{\kern0pt}def{\isadigit{2}}\ \isacommand{by}\isamarkupfalse%
\ {\isacharparenleft}{\kern0pt}auto{\isacharcomma}{\kern0pt}\ typecheck{\isacharunderscore}{\kern0pt}cfuncs{\isacharparenright}{\kern0pt}\isanewline
\ \ \isacommand{have}\isamarkupfalse%
\ pullback{\isacharunderscore}{\kern0pt}map{\isadigit{2}}{\isacharunderscore}{\kern0pt}type{\isacharbrackleft}{\kern0pt}type{\isacharunderscore}{\kern0pt}rule{\isacharbrackright}{\kern0pt}{\isacharcolon}{\kern0pt}\ {\isachardoublequoteopen}a\ {\isacharcolon}{\kern0pt}\ A\ {\isasymrightarrow}\ X{\isachardoublequoteclose}\isanewline
\ \ \ \ \isacommand{using}\isamarkupfalse%
\ assms{\isacharparenleft}{\kern0pt}{\isadigit{1}}{\isacharparenright}{\kern0pt}\ \isacommand{unfolding}\isamarkupfalse%
\ subobject{\isacharunderscore}{\kern0pt}of{\isacharunderscore}{\kern0pt}def{\isadigit{2}}\ \isacommand{by}\isamarkupfalse%
\ auto\isanewline
\ \ \isacommand{have}\isamarkupfalse%
\ pullback{\isacharunderscore}{\kern0pt}maps{\isacharunderscore}{\kern0pt}commute{\isacharcolon}{\kern0pt}\ {\isachardoublequoteopen}m\ {\isasymcirc}\isactrlsub c\ s\ {\isasymcirc}\isactrlsub c\ f{\isasymrestriction}\isactrlbsub {\isacharparenleft}{\kern0pt}A{\isacharcomma}{\kern0pt}\ a{\isacharparenright}{\kern0pt}\isactrlesub \ {\isacharequal}{\kern0pt}\ f\ {\isasymcirc}\isactrlsub c\ a{\isachardoublequoteclose}\isanewline
\ \ \ \ \isacommand{by}\isamarkupfalse%
\ {\isacharparenleft}{\kern0pt}typecheck{\isacharunderscore}{\kern0pt}cfuncs{\isacharcomma}{\kern0pt}\ simp\ add{\isacharcolon}{\kern0pt}\ comp{\isacharunderscore}{\kern0pt}associative{\isadigit{2}}\ image{\isacharunderscore}{\kern0pt}subobj{\isacharunderscore}{\kern0pt}comp{\isacharunderscore}{\kern0pt}image{\isacharunderscore}{\kern0pt}rest\ m{\isacharunderscore}{\kern0pt}s{\isacharunderscore}{\kern0pt}eq{\isacharunderscore}{\kern0pt}subobj{\isacharunderscore}{\kern0pt}map{\isacharparenright}{\kern0pt}\isanewline
\isanewline
\ \ \isacommand{have}\isamarkupfalse%
\ {\isachardoublequoteopen}{\isasymAnd}Z\ k\ h{\isachardot}{\kern0pt}\ k\ {\isacharcolon}{\kern0pt}\ Z\ {\isasymrightarrow}\ B\ {\isasymLongrightarrow}\ h\ {\isacharcolon}{\kern0pt}\ Z\ {\isasymrightarrow}\ X\ {\isasymLongrightarrow}\ m\ {\isasymcirc}\isactrlsub c\ k\ {\isacharequal}{\kern0pt}\ f\ {\isasymcirc}\isactrlsub c\ h\ {\isasymLongrightarrow}\isanewline
\ \ \ \ \ {\isacharparenleft}{\kern0pt}{\isasymexists}{\isacharbang}{\kern0pt}j{\isachardot}{\kern0pt}\ j\ {\isacharcolon}{\kern0pt}\ Z\ {\isasymrightarrow}\ f\isactrlsup {\isacharminus}{\kern0pt}\isactrlsup {\isadigit{1}}{\isasymlparr}B{\isasymrparr}\isactrlbsub m\isactrlesub \ {\isasymand}\isanewline
\ \ \ \ \ \ \ \ \ \ \ {\isacharparenleft}{\kern0pt}right{\isacharunderscore}{\kern0pt}cart{\isacharunderscore}{\kern0pt}proj\ X\ B\ {\isasymcirc}\isactrlsub c\ inverse{\isacharunderscore}{\kern0pt}image{\isacharunderscore}{\kern0pt}mapping\ f\ B\ m{\isacharparenright}{\kern0pt}\ {\isasymcirc}\isactrlsub c\ j\ {\isacharequal}{\kern0pt}\ k\ {\isasymand}\isanewline
\ \ \ \ \ \ \ \ \ \ \ {\isacharparenleft}{\kern0pt}left{\isacharunderscore}{\kern0pt}cart{\isacharunderscore}{\kern0pt}proj\ X\ B\ {\isasymcirc}\isactrlsub c\ inverse{\isacharunderscore}{\kern0pt}image{\isacharunderscore}{\kern0pt}mapping\ f\ B\ m{\isacharparenright}{\kern0pt}\ {\isasymcirc}\isactrlsub c\ j\ {\isacharequal}{\kern0pt}\ h{\isacharparenright}{\kern0pt}{\isachardoublequoteclose}\isanewline
\ \ \ \ \isacommand{using}\isamarkupfalse%
\ inverse{\isacharunderscore}{\kern0pt}image{\isacharunderscore}{\kern0pt}pullback\ assms{\isacharparenleft}{\kern0pt}{\isadigit{3}}{\isacharparenright}{\kern0pt}\ m{\isacharunderscore}{\kern0pt}mono\ m{\isacharunderscore}{\kern0pt}type\ \isacommand{unfolding}\isamarkupfalse%
\ is{\isacharunderscore}{\kern0pt}pullback{\isacharunderscore}{\kern0pt}def\ \isacommand{by}\isamarkupfalse%
\ simp\isanewline
\ \ \isacommand{then}\isamarkupfalse%
\ \isacommand{obtain}\isamarkupfalse%
\ k\ \isakeyword{where}\ k{\isacharunderscore}{\kern0pt}type{\isacharbrackleft}{\kern0pt}type{\isacharunderscore}{\kern0pt}rule{\isacharbrackright}{\kern0pt}{\isacharcolon}{\kern0pt}\ {\isachardoublequoteopen}k\ {\isacharcolon}{\kern0pt}\ A\ {\isasymrightarrow}\ f\isactrlsup {\isacharminus}{\kern0pt}\isactrlsup {\isadigit{1}}{\isasymlparr}B{\isasymrparr}\isactrlbsub m\isactrlesub {\isachardoublequoteclose}\ \isakeyword{and}\isanewline
\ \ \ \ k{\isacharunderscore}{\kern0pt}right{\isacharunderscore}{\kern0pt}eq{\isacharcolon}{\kern0pt}\ {\isachardoublequoteopen}{\isacharparenleft}{\kern0pt}right{\isacharunderscore}{\kern0pt}cart{\isacharunderscore}{\kern0pt}proj\ X\ B\ {\isasymcirc}\isactrlsub c\ inverse{\isacharunderscore}{\kern0pt}image{\isacharunderscore}{\kern0pt}mapping\ f\ B\ m{\isacharparenright}{\kern0pt}\ {\isasymcirc}\isactrlsub c\ k\ {\isacharequal}{\kern0pt}\ s\ {\isasymcirc}\isactrlsub c\ f{\isasymrestriction}\isactrlbsub {\isacharparenleft}{\kern0pt}A{\isacharcomma}{\kern0pt}\ a{\isacharparenright}{\kern0pt}\isactrlesub {\isachardoublequoteclose}\ \isakeyword{and}\isanewline
\ \ \ \ k{\isacharunderscore}{\kern0pt}left{\isacharunderscore}{\kern0pt}eq{\isacharcolon}{\kern0pt}\ {\isachardoublequoteopen}{\isacharparenleft}{\kern0pt}left{\isacharunderscore}{\kern0pt}cart{\isacharunderscore}{\kern0pt}proj\ X\ B\ {\isasymcirc}\isactrlsub c\ inverse{\isacharunderscore}{\kern0pt}image{\isacharunderscore}{\kern0pt}mapping\ f\ B\ m{\isacharparenright}{\kern0pt}\ {\isasymcirc}\isactrlsub c\ k\ {\isacharequal}{\kern0pt}\ a{\isachardoublequoteclose}\isanewline
\ \ \ \ \isacommand{using}\isamarkupfalse%
\ pullback{\isacharunderscore}{\kern0pt}map{\isadigit{1}}{\isacharunderscore}{\kern0pt}type\ pullback{\isacharunderscore}{\kern0pt}map{\isadigit{2}}{\isacharunderscore}{\kern0pt}type\ pullback{\isacharunderscore}{\kern0pt}maps{\isacharunderscore}{\kern0pt}commute\ \isacommand{by}\isamarkupfalse%
\ blast\isanewline
\isanewline
\ \ \isacommand{have}\isamarkupfalse%
\ {\isachardoublequoteopen}monomorphism\ {\isacharparenleft}{\kern0pt}{\isacharparenleft}{\kern0pt}left{\isacharunderscore}{\kern0pt}cart{\isacharunderscore}{\kern0pt}proj\ X\ B\ {\isasymcirc}\isactrlsub c\ inverse{\isacharunderscore}{\kern0pt}image{\isacharunderscore}{\kern0pt}mapping\ f\ B\ m{\isacharparenright}{\kern0pt}\ {\isasymcirc}\isactrlsub c\ k{\isacharparenright}{\kern0pt}\ {\isasymLongrightarrow}\ monomorphism\ k{\isachardoublequoteclose}\isanewline
\ \ \ \ \isacommand{using}\isamarkupfalse%
\ comp{\isacharunderscore}{\kern0pt}monic{\isacharunderscore}{\kern0pt}imp{\isacharunderscore}{\kern0pt}monic{\isacharprime}{\kern0pt}\ m{\isacharunderscore}{\kern0pt}mono\ \isacommand{by}\isamarkupfalse%
\ {\isacharparenleft}{\kern0pt}typecheck{\isacharunderscore}{\kern0pt}cfuncs{\isacharcomma}{\kern0pt}\ blast{\isacharparenright}{\kern0pt}\isanewline
\ \ \isacommand{then}\isamarkupfalse%
\ \isacommand{have}\isamarkupfalse%
\ {\isachardoublequoteopen}monomorphism\ k{\isachardoublequoteclose}\isanewline
\ \ \ \ \isacommand{by}\isamarkupfalse%
\ {\isacharparenleft}{\kern0pt}simp\ add{\isacharcolon}{\kern0pt}\ a{\isacharunderscore}{\kern0pt}mono\ k{\isacharunderscore}{\kern0pt}left{\isacharunderscore}{\kern0pt}eq{\isacharparenright}{\kern0pt}\isanewline
\ \ \isacommand{then}\isamarkupfalse%
\ \isacommand{show}\isamarkupfalse%
\ {\isachardoublequoteopen}{\isacharparenleft}{\kern0pt}A{\isacharcomma}{\kern0pt}\ a{\isacharparenright}{\kern0pt}{\isasymsubseteq}\isactrlbsub X\isactrlesub {\isacharparenleft}{\kern0pt}f\isactrlsup {\isacharminus}{\kern0pt}\isactrlsup {\isadigit{1}}{\isasymlparr}B{\isasymrparr}\isactrlbsub m\isactrlesub {\isacharcomma}{\kern0pt}\ {\isacharbrackleft}{\kern0pt}f\isactrlsup {\isacharminus}{\kern0pt}\isactrlsup {\isadigit{1}}{\isasymlparr}B{\isasymrparr}\isactrlbsub m\isactrlesub {\isacharbrackright}{\kern0pt}map{\isacharparenright}{\kern0pt}{\isachardoublequoteclose}\isanewline
\ \ \ \ \isacommand{unfolding}\isamarkupfalse%
\ relative{\isacharunderscore}{\kern0pt}subset{\isacharunderscore}{\kern0pt}def{\isadigit{2}}\ \isanewline
\ \ \ \ \isacommand{using}\isamarkupfalse%
\ assms\ a{\isacharunderscore}{\kern0pt}mono\ m{\isacharunderscore}{\kern0pt}mono\ inverse{\isacharunderscore}{\kern0pt}image{\isacharunderscore}{\kern0pt}subobject{\isacharunderscore}{\kern0pt}mapping{\isacharunderscore}{\kern0pt}mono\isanewline
\ \ \isacommand{proof}\isamarkupfalse%
\ {\isacharparenleft}{\kern0pt}typecheck{\isacharunderscore}{\kern0pt}cfuncs{\isacharcomma}{\kern0pt}\ auto{\isacharparenright}{\kern0pt}\isanewline
\ \ \ \ \isacommand{assume}\isamarkupfalse%
\ {\isachardoublequoteopen}monomorphism\ k{\isachardoublequoteclose}\isanewline
\ \ \ \ \isacommand{then}\isamarkupfalse%
\ \isacommand{show}\isamarkupfalse%
\ {\isachardoublequoteopen}{\isasymexists}k{\isachardot}{\kern0pt}\ k\ {\isacharcolon}{\kern0pt}\ A\ {\isasymrightarrow}\ f\isactrlsup {\isacharminus}{\kern0pt}\isactrlsup {\isadigit{1}}{\isasymlparr}B{\isasymrparr}\isactrlbsub m\isactrlesub \ {\isasymand}\ {\isacharbrackleft}{\kern0pt}f\isactrlsup {\isacharminus}{\kern0pt}\isactrlsup {\isadigit{1}}{\isasymlparr}B{\isasymrparr}\isactrlbsub m\isactrlesub {\isacharbrackright}{\kern0pt}map\ {\isasymcirc}\isactrlsub c\ k\ {\isacharequal}{\kern0pt}\ a{\isachardoublequoteclose}\isanewline
\ \ \ \ \ \ \isacommand{using}\isamarkupfalse%
\ assms{\isacharparenleft}{\kern0pt}{\isadigit{3}}{\isacharparenright}{\kern0pt}\ inverse{\isacharunderscore}{\kern0pt}image{\isacharunderscore}{\kern0pt}subobject{\isacharunderscore}{\kern0pt}mapping{\isacharunderscore}{\kern0pt}def{\isadigit{2}}\ k{\isacharunderscore}{\kern0pt}left{\isacharunderscore}{\kern0pt}eq\ k{\isacharunderscore}{\kern0pt}type\ \isanewline
\ \ \ \ \ \ \isacommand{by}\isamarkupfalse%
\ {\isacharparenleft}{\kern0pt}rule{\isacharunderscore}{\kern0pt}tac\ x{\isacharequal}{\kern0pt}k\ \isakeyword{in}\ exI{\isacharcomma}{\kern0pt}\ force{\isacharparenright}{\kern0pt}\isanewline
\ \ \isacommand{qed}\isamarkupfalse%
\isanewline
\isacommand{qed}\isamarkupfalse%
%
\endisatagproof
{\isafoldproof}%
%
\isadelimproof
%
\endisadelimproof
%
\begin{isamarkuptext}%
The lemma below corresponds to Exercise 2.3.10 in Halvorson.%
\end{isamarkuptext}\isamarkuptrue%
\isacommand{lemma}\isamarkupfalse%
\ in{\isacharunderscore}{\kern0pt}inv{\isacharunderscore}{\kern0pt}image{\isacharunderscore}{\kern0pt}of{\isacharunderscore}{\kern0pt}image{\isacharcolon}{\kern0pt}\isanewline
\ \ \isakeyword{assumes}\ {\isachardoublequoteopen}{\isacharparenleft}{\kern0pt}A{\isacharcomma}{\kern0pt}m{\isacharparenright}{\kern0pt}\ {\isasymsubseteq}\isactrlsub c\ X{\isachardoublequoteclose}\ \isanewline
\ \ \isakeyword{assumes}{\isacharbrackleft}{\kern0pt}type{\isacharunderscore}{\kern0pt}rule{\isacharbrackright}{\kern0pt}{\isacharcolon}{\kern0pt}\ {\isachardoublequoteopen}f\ {\isacharcolon}{\kern0pt}\ X\ {\isasymrightarrow}\ Y{\isachardoublequoteclose}\isanewline
\ \ \isakeyword{shows}\ {\isachardoublequoteopen}{\isacharparenleft}{\kern0pt}A{\isacharcomma}{\kern0pt}m{\isacharparenright}{\kern0pt}\ {\isasymsubseteq}\isactrlbsub X\isactrlesub \ {\isacharparenleft}{\kern0pt}f\isactrlsup {\isacharminus}{\kern0pt}\isactrlsup {\isadigit{1}}{\isasymlparr}f{\isasymlparr}A{\isasymrparr}\isactrlbsub m\isactrlesub {\isasymrparr}\isactrlbsub {\isacharbrackleft}{\kern0pt}f{\isasymlparr}A{\isasymrparr}\isactrlbsub m\isactrlesub {\isacharbrackright}{\kern0pt}map\isactrlesub {\isacharcomma}{\kern0pt}\ {\isacharbrackleft}{\kern0pt}f\isactrlsup {\isacharminus}{\kern0pt}\isactrlsup {\isadigit{1}}{\isasymlparr}f{\isasymlparr}A{\isasymrparr}\isactrlbsub m\isactrlesub {\isasymrparr}\isactrlbsub {\isacharbrackleft}{\kern0pt}f{\isasymlparr}A{\isasymrparr}\isactrlbsub m\isactrlesub {\isacharbrackright}{\kern0pt}map\isactrlesub {\isacharbrackright}{\kern0pt}map{\isacharparenright}{\kern0pt}{\isachardoublequoteclose}\isanewline
%
\isadelimproof
%
\endisadelimproof
%
\isatagproof
\isacommand{proof}\isamarkupfalse%
\ {\isacharminus}{\kern0pt}\isanewline
\ \ \isacommand{have}\isamarkupfalse%
\ m{\isacharunderscore}{\kern0pt}type{\isacharbrackleft}{\kern0pt}type{\isacharunderscore}{\kern0pt}rule{\isacharbrackright}{\kern0pt}{\isacharcolon}{\kern0pt}\ {\isachardoublequoteopen}m\ {\isacharcolon}{\kern0pt}\ A\ {\isasymrightarrow}\ X{\isachardoublequoteclose}\isanewline
\ \ \ \ \isacommand{using}\isamarkupfalse%
\ assms{\isacharparenleft}{\kern0pt}{\isadigit{1}}{\isacharparenright}{\kern0pt}\ \isacommand{unfolding}\isamarkupfalse%
\ subobject{\isacharunderscore}{\kern0pt}of{\isacharunderscore}{\kern0pt}def{\isadigit{2}}\ \isacommand{by}\isamarkupfalse%
\ auto\isanewline
\ \ \isacommand{have}\isamarkupfalse%
\ m{\isacharunderscore}{\kern0pt}mono{\isacharcolon}{\kern0pt}\ {\isachardoublequoteopen}monomorphism\ m{\isachardoublequoteclose}\isanewline
\ \ \ \ \isacommand{using}\isamarkupfalse%
\ assms{\isacharparenleft}{\kern0pt}{\isadigit{1}}{\isacharparenright}{\kern0pt}\ \isacommand{unfolding}\isamarkupfalse%
\ subobject{\isacharunderscore}{\kern0pt}of{\isacharunderscore}{\kern0pt}def{\isadigit{2}}\ \isacommand{by}\isamarkupfalse%
\ auto\isanewline
\isanewline
\ \ \isacommand{have}\isamarkupfalse%
\ {\isachardoublequoteopen}{\isacharparenleft}{\kern0pt}{\isacharparenleft}{\kern0pt}f{\isasymlparr}A{\isasymrparr}\isactrlbsub m\isactrlesub {\isacharcomma}{\kern0pt}\ {\isacharbrackleft}{\kern0pt}f{\isasymlparr}A{\isasymrparr}\isactrlbsub m\isactrlesub {\isacharbrackright}{\kern0pt}map{\isacharparenright}{\kern0pt}\ {\isasymsubseteq}\isactrlbsub Y\isactrlesub \ {\isacharparenleft}{\kern0pt}f{\isasymlparr}A{\isasymrparr}\isactrlbsub m\isactrlesub {\isacharcomma}{\kern0pt}\ {\isacharbrackleft}{\kern0pt}f{\isasymlparr}A{\isasymrparr}\isactrlbsub m\isactrlesub {\isacharbrackright}{\kern0pt}map{\isacharparenright}{\kern0pt}{\isacharparenright}{\kern0pt}{\isachardoublequoteclose}\isanewline
\ \ \ \ \isacommand{unfolding}\isamarkupfalse%
\ relative{\isacharunderscore}{\kern0pt}subset{\isacharunderscore}{\kern0pt}def{\isadigit{2}}\isanewline
\ \ \ \ \isacommand{using}\isamarkupfalse%
\ m{\isacharunderscore}{\kern0pt}mono\ image{\isacharunderscore}{\kern0pt}subobj{\isacharunderscore}{\kern0pt}map{\isacharunderscore}{\kern0pt}mono\ id{\isacharunderscore}{\kern0pt}right{\isacharunderscore}{\kern0pt}unit{\isadigit{2}}\ id{\isacharunderscore}{\kern0pt}type\ \isacommand{by}\isamarkupfalse%
\ {\isacharparenleft}{\kern0pt}typecheck{\isacharunderscore}{\kern0pt}cfuncs{\isacharcomma}{\kern0pt}\ blast{\isacharparenright}{\kern0pt}\isanewline
\ \ \isacommand{then}\isamarkupfalse%
\ \isacommand{show}\isamarkupfalse%
\ {\isachardoublequoteopen}{\isacharparenleft}{\kern0pt}A{\isacharcomma}{\kern0pt}m{\isacharparenright}{\kern0pt}\ {\isasymsubseteq}\isactrlbsub X\isactrlesub \ {\isacharparenleft}{\kern0pt}f\isactrlsup {\isacharminus}{\kern0pt}\isactrlsup {\isadigit{1}}{\isasymlparr}f{\isasymlparr}A{\isasymrparr}\isactrlbsub m\isactrlesub {\isasymrparr}\isactrlbsub {\isacharbrackleft}{\kern0pt}f{\isasymlparr}A{\isasymrparr}\isactrlbsub m\isactrlesub {\isacharbrackright}{\kern0pt}map\isactrlesub {\isacharcomma}{\kern0pt}\ {\isacharbrackleft}{\kern0pt}f\isactrlsup {\isacharminus}{\kern0pt}\isactrlsup {\isadigit{1}}{\isasymlparr}f{\isasymlparr}A{\isasymrparr}\isactrlbsub m\isactrlesub {\isasymrparr}\isactrlbsub {\isacharbrackleft}{\kern0pt}f{\isasymlparr}A{\isasymrparr}\isactrlbsub m\isactrlesub {\isacharbrackright}{\kern0pt}map\isactrlesub {\isacharbrackright}{\kern0pt}map{\isacharparenright}{\kern0pt}{\isachardoublequoteclose}\isanewline
\ \ \ \ \isacommand{by}\isamarkupfalse%
\ {\isacharparenleft}{\kern0pt}meson\ assms\ relative{\isacharunderscore}{\kern0pt}subset{\isacharunderscore}{\kern0pt}def{\isadigit{2}}\ subobject{\isacharunderscore}{\kern0pt}of{\isacharunderscore}{\kern0pt}def{\isadigit{2}}\ subset{\isacharunderscore}{\kern0pt}inv{\isacharunderscore}{\kern0pt}image{\isacharunderscore}{\kern0pt}iff{\isacharunderscore}{\kern0pt}image{\isacharunderscore}{\kern0pt}subset{\isacharparenright}{\kern0pt}\isanewline
\isacommand{qed}\isamarkupfalse%
%
\endisatagproof
{\isafoldproof}%
%
\isadelimproof
%
\endisadelimproof
%
\isadelimdocument
%
\endisadelimdocument
%
\isatagdocument
%
\isamarkupsection{\isa{distribute{\isacharunderscore}{\kern0pt}left} and \isa{distribute{\isacharunderscore}{\kern0pt}right} as Equivalence Relations%
}
\isamarkuptrue%
%
\endisatagdocument
{\isafolddocument}%
%
\isadelimdocument
%
\endisadelimdocument
\isacommand{lemma}\isamarkupfalse%
\ left{\isacharunderscore}{\kern0pt}pair{\isacharunderscore}{\kern0pt}subset{\isacharcolon}{\kern0pt}\isanewline
\ \ \isakeyword{assumes}\ {\isachardoublequoteopen}m\ {\isacharcolon}{\kern0pt}\ Y\ {\isasymrightarrow}\ X\ {\isasymtimes}\isactrlsub c\ X{\isachardoublequoteclose}\ {\isachardoublequoteopen}monomorphism\ m{\isachardoublequoteclose}\isanewline
\ \ \isakeyword{shows}\ {\isachardoublequoteopen}{\isacharparenleft}{\kern0pt}Y\ {\isasymtimes}\isactrlsub c\ Z{\isacharcomma}{\kern0pt}\ distribute{\isacharunderscore}{\kern0pt}right\ X\ X\ Z\ {\isasymcirc}\isactrlsub c\ {\isacharparenleft}{\kern0pt}m\ {\isasymtimes}\isactrlsub f\ id\isactrlsub c\ Z{\isacharparenright}{\kern0pt}{\isacharparenright}{\kern0pt}\ {\isasymsubseteq}\isactrlsub c\ {\isacharparenleft}{\kern0pt}X\ {\isasymtimes}\isactrlsub c\ Z{\isacharparenright}{\kern0pt}\ {\isasymtimes}\isactrlsub c\ {\isacharparenleft}{\kern0pt}X\ {\isasymtimes}\isactrlsub c\ Z{\isacharparenright}{\kern0pt}{\isachardoublequoteclose}\isanewline
%
\isadelimproof
\ \ %
\endisadelimproof
%
\isatagproof
\isacommand{unfolding}\isamarkupfalse%
\ subobject{\isacharunderscore}{\kern0pt}of{\isacharunderscore}{\kern0pt}def{\isadigit{2}}\ \isacommand{using}\isamarkupfalse%
\ assms\isanewline
\isacommand{proof}\isamarkupfalse%
\ {\isacharparenleft}{\kern0pt}typecheck{\isacharunderscore}{\kern0pt}cfuncs{\isacharcomma}{\kern0pt}\ unfold\ monomorphism{\isacharunderscore}{\kern0pt}def{\isadigit{3}}{\isacharcomma}{\kern0pt}\ auto{\isacharparenright}{\kern0pt}\isanewline
\ \ \isacommand{fix}\isamarkupfalse%
\ g\ h\ A\isanewline
\ \ \isacommand{assume}\isamarkupfalse%
\ g{\isacharunderscore}{\kern0pt}type{\isacharcolon}{\kern0pt}\ {\isachardoublequoteopen}g\ {\isacharcolon}{\kern0pt}\ A\ {\isasymrightarrow}\ Y\ {\isasymtimes}\isactrlsub c\ Z{\isachardoublequoteclose}\isanewline
\ \ \isacommand{assume}\isamarkupfalse%
\ h{\isacharunderscore}{\kern0pt}type{\isacharcolon}{\kern0pt}\ {\isachardoublequoteopen}h\ {\isacharcolon}{\kern0pt}\ A\ {\isasymrightarrow}\ Y\ {\isasymtimes}\isactrlsub c\ Z{\isachardoublequoteclose}\isanewline
\ \ \isacommand{assume}\isamarkupfalse%
\ {\isachardoublequoteopen}{\isacharparenleft}{\kern0pt}distribute{\isacharunderscore}{\kern0pt}right\ X\ X\ Z\ {\isasymcirc}\isactrlsub c\ {\isacharparenleft}{\kern0pt}m\ {\isasymtimes}\isactrlsub f\ id\isactrlsub c\ Z{\isacharparenright}{\kern0pt}{\isacharparenright}{\kern0pt}\ {\isasymcirc}\isactrlsub c\ g\ {\isacharequal}{\kern0pt}\ {\isacharparenleft}{\kern0pt}distribute{\isacharunderscore}{\kern0pt}right\ X\ X\ Z\ {\isasymcirc}\isactrlsub c\ m\ {\isasymtimes}\isactrlsub f\ id\isactrlsub c\ Z{\isacharparenright}{\kern0pt}\ {\isasymcirc}\isactrlsub c\ h{\isachardoublequoteclose}\isanewline
\ \ \isacommand{then}\isamarkupfalse%
\ \isacommand{have}\isamarkupfalse%
\ {\isachardoublequoteopen}distribute{\isacharunderscore}{\kern0pt}right\ X\ X\ Z\ {\isasymcirc}\isactrlsub c\ {\isacharparenleft}{\kern0pt}m\ {\isasymtimes}\isactrlsub f\ id\isactrlsub c\ Z{\isacharparenright}{\kern0pt}\ {\isasymcirc}\isactrlsub c\ g\ {\isacharequal}{\kern0pt}\ distribute{\isacharunderscore}{\kern0pt}right\ X\ X\ Z\ {\isasymcirc}\isactrlsub c\ {\isacharparenleft}{\kern0pt}m\ {\isasymtimes}\isactrlsub f\ id\isactrlsub c\ Z{\isacharparenright}{\kern0pt}\ {\isasymcirc}\isactrlsub c\ h{\isachardoublequoteclose}\isanewline
\ \ \ \ \isacommand{using}\isamarkupfalse%
\ assms\ g{\isacharunderscore}{\kern0pt}type\ h{\isacharunderscore}{\kern0pt}type\ \isacommand{by}\isamarkupfalse%
\ {\isacharparenleft}{\kern0pt}typecheck{\isacharunderscore}{\kern0pt}cfuncs{\isacharcomma}{\kern0pt}\ simp\ add{\isacharcolon}{\kern0pt}\ comp{\isacharunderscore}{\kern0pt}associative{\isadigit{2}}{\isacharparenright}{\kern0pt}\isanewline
\ \ \isacommand{then}\isamarkupfalse%
\ \isacommand{have}\isamarkupfalse%
\ {\isachardoublequoteopen}{\isacharparenleft}{\kern0pt}m\ {\isasymtimes}\isactrlsub f\ id\isactrlsub c\ Z{\isacharparenright}{\kern0pt}\ {\isasymcirc}\isactrlsub c\ g\ {\isacharequal}{\kern0pt}\ {\isacharparenleft}{\kern0pt}m\ {\isasymtimes}\isactrlsub f\ id\isactrlsub c\ Z{\isacharparenright}{\kern0pt}\ {\isasymcirc}\isactrlsub c\ h{\isachardoublequoteclose}\isanewline
\ \ \ \ \isacommand{using}\isamarkupfalse%
\ assms\ g{\isacharunderscore}{\kern0pt}type\ h{\isacharunderscore}{\kern0pt}type\ distribute{\isacharunderscore}{\kern0pt}right{\isacharunderscore}{\kern0pt}mono\ distribute{\isacharunderscore}{\kern0pt}right{\isacharunderscore}{\kern0pt}type\ monomorphism{\isacharunderscore}{\kern0pt}def{\isadigit{2}}\isanewline
\ \ \ \ \isacommand{by}\isamarkupfalse%
\ {\isacharparenleft}{\kern0pt}typecheck{\isacharunderscore}{\kern0pt}cfuncs{\isacharcomma}{\kern0pt}\ blast{\isacharparenright}{\kern0pt}\isanewline
\ \ \isacommand{then}\isamarkupfalse%
\ \isacommand{show}\isamarkupfalse%
\ {\isachardoublequoteopen}g\ {\isacharequal}{\kern0pt}\ h{\isachardoublequoteclose}\isanewline
\ \ \isacommand{proof}\isamarkupfalse%
\ {\isacharminus}{\kern0pt}\isanewline
\ \ \ \ \isacommand{have}\isamarkupfalse%
\ {\isachardoublequoteopen}monomorphism\ {\isacharparenleft}{\kern0pt}m\ {\isasymtimes}\isactrlsub f\ id\isactrlsub c\ Z{\isacharparenright}{\kern0pt}{\isachardoublequoteclose}\isanewline
\ \ \ \ \ \ \isacommand{using}\isamarkupfalse%
\ assms\ cfunc{\isacharunderscore}{\kern0pt}cross{\isacharunderscore}{\kern0pt}prod{\isacharunderscore}{\kern0pt}mono\ id{\isacharunderscore}{\kern0pt}isomorphism\ iso{\isacharunderscore}{\kern0pt}imp{\isacharunderscore}{\kern0pt}epi{\isacharunderscore}{\kern0pt}and{\isacharunderscore}{\kern0pt}monic\ \isacommand{by}\isamarkupfalse%
\ {\isacharparenleft}{\kern0pt}typecheck{\isacharunderscore}{\kern0pt}cfuncs{\isacharcomma}{\kern0pt}\ blast{\isacharparenright}{\kern0pt}\isanewline
\ \ \ \ \isacommand{then}\isamarkupfalse%
\ \isacommand{show}\isamarkupfalse%
\ {\isachardoublequoteopen}{\isacharparenleft}{\kern0pt}m\ {\isasymtimes}\isactrlsub f\ id\isactrlsub c\ Z{\isacharparenright}{\kern0pt}\ {\isasymcirc}\isactrlsub c\ g\ {\isacharequal}{\kern0pt}\ {\isacharparenleft}{\kern0pt}m\ {\isasymtimes}\isactrlsub f\ id\isactrlsub c\ Z{\isacharparenright}{\kern0pt}\ {\isasymcirc}\isactrlsub c\ h\ {\isasymLongrightarrow}\ g\ {\isacharequal}{\kern0pt}\ h{\isachardoublequoteclose}\isanewline
\ \ \ \ \ \ \isacommand{using}\isamarkupfalse%
\ assms\ g{\isacharunderscore}{\kern0pt}type\ h{\isacharunderscore}{\kern0pt}type\ \isacommand{unfolding}\isamarkupfalse%
\ monomorphism{\isacharunderscore}{\kern0pt}def{\isadigit{2}}\ \isacommand{by}\isamarkupfalse%
\ {\isacharparenleft}{\kern0pt}typecheck{\isacharunderscore}{\kern0pt}cfuncs{\isacharcomma}{\kern0pt}\ blast{\isacharparenright}{\kern0pt}\isanewline
\ \ \isacommand{qed}\isamarkupfalse%
\isanewline
\isacommand{qed}\isamarkupfalse%
%
\endisatagproof
{\isafoldproof}%
%
\isadelimproof
\isanewline
%
\endisadelimproof
\isanewline
\isacommand{lemma}\isamarkupfalse%
\ right{\isacharunderscore}{\kern0pt}pair{\isacharunderscore}{\kern0pt}subset{\isacharcolon}{\kern0pt}\isanewline
\ \ \isakeyword{assumes}\ {\isachardoublequoteopen}m\ {\isacharcolon}{\kern0pt}\ Y\ {\isasymrightarrow}\ X\ {\isasymtimes}\isactrlsub c\ X{\isachardoublequoteclose}\ {\isachardoublequoteopen}monomorphism\ m{\isachardoublequoteclose}\isanewline
\ \ \isakeyword{shows}\ {\isachardoublequoteopen}{\isacharparenleft}{\kern0pt}Z\ {\isasymtimes}\isactrlsub c\ Y{\isacharcomma}{\kern0pt}\ distribute{\isacharunderscore}{\kern0pt}left\ Z\ X\ X\ {\isasymcirc}\isactrlsub c\ {\isacharparenleft}{\kern0pt}id\isactrlsub c\ Z\ {\isasymtimes}\isactrlsub f\ m{\isacharparenright}{\kern0pt}{\isacharparenright}{\kern0pt}\ {\isasymsubseteq}\isactrlsub c\ {\isacharparenleft}{\kern0pt}Z\ {\isasymtimes}\isactrlsub c\ X{\isacharparenright}{\kern0pt}\ {\isasymtimes}\isactrlsub c\ {\isacharparenleft}{\kern0pt}Z\ {\isasymtimes}\isactrlsub c\ X{\isacharparenright}{\kern0pt}{\isachardoublequoteclose}\isanewline
%
\isadelimproof
\ \ %
\endisadelimproof
%
\isatagproof
\isacommand{unfolding}\isamarkupfalse%
\ subobject{\isacharunderscore}{\kern0pt}of{\isacharunderscore}{\kern0pt}def{\isadigit{2}}\ \isacommand{using}\isamarkupfalse%
\ assms\isanewline
\isacommand{proof}\isamarkupfalse%
\ {\isacharparenleft}{\kern0pt}typecheck{\isacharunderscore}{\kern0pt}cfuncs{\isacharcomma}{\kern0pt}\ unfold\ monomorphism{\isacharunderscore}{\kern0pt}def{\isadigit{3}}{\isacharcomma}{\kern0pt}\ auto{\isacharparenright}{\kern0pt}\isanewline
\ \ \isacommand{fix}\isamarkupfalse%
\ g\ h\ A\isanewline
\ \ \isacommand{assume}\isamarkupfalse%
\ g{\isacharunderscore}{\kern0pt}type{\isacharcolon}{\kern0pt}\ {\isachardoublequoteopen}g\ {\isacharcolon}{\kern0pt}\ A\ {\isasymrightarrow}\ Z\ {\isasymtimes}\isactrlsub c\ Y{\isachardoublequoteclose}\isanewline
\ \ \isacommand{assume}\isamarkupfalse%
\ h{\isacharunderscore}{\kern0pt}type{\isacharcolon}{\kern0pt}\ {\isachardoublequoteopen}h\ {\isacharcolon}{\kern0pt}\ A\ {\isasymrightarrow}\ Z\ {\isasymtimes}\isactrlsub c\ Y{\isachardoublequoteclose}\isanewline
\ \ \isacommand{assume}\isamarkupfalse%
\ {\isachardoublequoteopen}{\isacharparenleft}{\kern0pt}distribute{\isacharunderscore}{\kern0pt}left\ Z\ X\ X\ {\isasymcirc}\isactrlsub c\ {\isacharparenleft}{\kern0pt}id\isactrlsub c\ Z\ {\isasymtimes}\isactrlsub f\ m{\isacharparenright}{\kern0pt}{\isacharparenright}{\kern0pt}\ {\isasymcirc}\isactrlsub c\ g\ {\isacharequal}{\kern0pt}\ {\isacharparenleft}{\kern0pt}distribute{\isacharunderscore}{\kern0pt}left\ Z\ X\ X\ {\isasymcirc}\isactrlsub c\ {\isacharparenleft}{\kern0pt}id\isactrlsub c\ Z\ {\isasymtimes}\isactrlsub f\ m{\isacharparenright}{\kern0pt}{\isacharparenright}{\kern0pt}\ {\isasymcirc}\isactrlsub c\ h{\isachardoublequoteclose}\isanewline
\ \ \isacommand{then}\isamarkupfalse%
\ \isacommand{have}\isamarkupfalse%
\ {\isachardoublequoteopen}distribute{\isacharunderscore}{\kern0pt}left\ Z\ X\ X\ {\isasymcirc}\isactrlsub c\ {\isacharparenleft}{\kern0pt}id\isactrlsub c\ Z\ {\isasymtimes}\isactrlsub f\ m{\isacharparenright}{\kern0pt}\ {\isasymcirc}\isactrlsub c\ g\ {\isacharequal}{\kern0pt}\ distribute{\isacharunderscore}{\kern0pt}left\ Z\ X\ X\ {\isasymcirc}\isactrlsub c\ {\isacharparenleft}{\kern0pt}id\isactrlsub c\ Z\ {\isasymtimes}\isactrlsub f\ m{\isacharparenright}{\kern0pt}\ {\isasymcirc}\isactrlsub c\ h{\isachardoublequoteclose}\isanewline
\ \ \ \ \isacommand{using}\isamarkupfalse%
\ assms\ g{\isacharunderscore}{\kern0pt}type\ h{\isacharunderscore}{\kern0pt}type\ \isacommand{by}\isamarkupfalse%
\ {\isacharparenleft}{\kern0pt}typecheck{\isacharunderscore}{\kern0pt}cfuncs{\isacharcomma}{\kern0pt}\ simp\ add{\isacharcolon}{\kern0pt}\ comp{\isacharunderscore}{\kern0pt}associative{\isadigit{2}}{\isacharparenright}{\kern0pt}\isanewline
\ \ \isacommand{then}\isamarkupfalse%
\ \isacommand{have}\isamarkupfalse%
\ {\isachardoublequoteopen}{\isacharparenleft}{\kern0pt}id\isactrlsub c\ Z\ {\isasymtimes}\isactrlsub f\ m{\isacharparenright}{\kern0pt}\ {\isasymcirc}\isactrlsub c\ g\ {\isacharequal}{\kern0pt}\ {\isacharparenleft}{\kern0pt}id\isactrlsub c\ Z\ {\isasymtimes}\isactrlsub f\ m{\isacharparenright}{\kern0pt}\ {\isasymcirc}\isactrlsub c\ h{\isachardoublequoteclose}\isanewline
\ \ \ \ \isacommand{using}\isamarkupfalse%
\ assms\ g{\isacharunderscore}{\kern0pt}type\ h{\isacharunderscore}{\kern0pt}type\ distribute{\isacharunderscore}{\kern0pt}left{\isacharunderscore}{\kern0pt}mono\ distribute{\isacharunderscore}{\kern0pt}left{\isacharunderscore}{\kern0pt}type\ monomorphism{\isacharunderscore}{\kern0pt}def{\isadigit{2}}\isanewline
\ \ \ \ \isacommand{by}\isamarkupfalse%
\ {\isacharparenleft}{\kern0pt}typecheck{\isacharunderscore}{\kern0pt}cfuncs{\isacharcomma}{\kern0pt}\ blast{\isacharparenright}{\kern0pt}\isanewline
\ \ \isacommand{then}\isamarkupfalse%
\ \isacommand{show}\isamarkupfalse%
\ {\isachardoublequoteopen}g\ {\isacharequal}{\kern0pt}\ h{\isachardoublequoteclose}\isanewline
\ \ \isacommand{proof}\isamarkupfalse%
\ {\isacharminus}{\kern0pt}\isanewline
\ \ \ \ \isacommand{have}\isamarkupfalse%
\ {\isachardoublequoteopen}monomorphism\ {\isacharparenleft}{\kern0pt}id\isactrlsub c\ Z\ {\isasymtimes}\isactrlsub f\ m{\isacharparenright}{\kern0pt}{\isachardoublequoteclose}\isanewline
\ \ \ \ \ \ \isacommand{using}\isamarkupfalse%
\ assms\ cfunc{\isacharunderscore}{\kern0pt}cross{\isacharunderscore}{\kern0pt}prod{\isacharunderscore}{\kern0pt}mono\ id{\isacharunderscore}{\kern0pt}isomorphism\ id{\isacharunderscore}{\kern0pt}type\ iso{\isacharunderscore}{\kern0pt}imp{\isacharunderscore}{\kern0pt}epi{\isacharunderscore}{\kern0pt}and{\isacharunderscore}{\kern0pt}monic\ \isacommand{by}\isamarkupfalse%
\ blast\isanewline
\ \ \ \ \isacommand{then}\isamarkupfalse%
\ \isacommand{show}\isamarkupfalse%
\ {\isachardoublequoteopen}{\isacharparenleft}{\kern0pt}id\isactrlsub c\ Z\ {\isasymtimes}\isactrlsub f\ m{\isacharparenright}{\kern0pt}\ {\isasymcirc}\isactrlsub c\ g\ {\isacharequal}{\kern0pt}\ {\isacharparenleft}{\kern0pt}id\isactrlsub c\ Z\ {\isasymtimes}\isactrlsub f\ m{\isacharparenright}{\kern0pt}\ {\isasymcirc}\isactrlsub c\ h\ {\isasymLongrightarrow}\ g\ {\isacharequal}{\kern0pt}\ h{\isachardoublequoteclose}\isanewline
\ \ \ \ \ \ \isacommand{using}\isamarkupfalse%
\ assms\ g{\isacharunderscore}{\kern0pt}type\ h{\isacharunderscore}{\kern0pt}type\ \isacommand{unfolding}\isamarkupfalse%
\ monomorphism{\isacharunderscore}{\kern0pt}def{\isadigit{2}}\ \isacommand{by}\isamarkupfalse%
\ {\isacharparenleft}{\kern0pt}typecheck{\isacharunderscore}{\kern0pt}cfuncs{\isacharcomma}{\kern0pt}\ blast{\isacharparenright}{\kern0pt}\isanewline
\ \ \isacommand{qed}\isamarkupfalse%
\isanewline
\isacommand{qed}\isamarkupfalse%
%
\endisatagproof
{\isafoldproof}%
%
\isadelimproof
\isanewline
%
\endisadelimproof
\isanewline
\isacommand{lemma}\isamarkupfalse%
\ left{\isacharunderscore}{\kern0pt}pair{\isacharunderscore}{\kern0pt}reflexive{\isacharcolon}{\kern0pt}\isanewline
\ \ \isakeyword{assumes}\ {\isachardoublequoteopen}reflexive{\isacharunderscore}{\kern0pt}on\ X\ {\isacharparenleft}{\kern0pt}Y{\isacharcomma}{\kern0pt}\ m{\isacharparenright}{\kern0pt}{\isachardoublequoteclose}\isanewline
\ \ \isakeyword{shows}\ {\isachardoublequoteopen}reflexive{\isacharunderscore}{\kern0pt}on\ {\isacharparenleft}{\kern0pt}X\ {\isasymtimes}\isactrlsub c\ Z{\isacharparenright}{\kern0pt}\ {\isacharparenleft}{\kern0pt}Y\ {\isasymtimes}\isactrlsub c\ Z{\isacharcomma}{\kern0pt}\ distribute{\isacharunderscore}{\kern0pt}right\ X\ X\ Z\ {\isasymcirc}\isactrlsub c\ {\isacharparenleft}{\kern0pt}m\ {\isasymtimes}\isactrlsub f\ id\isactrlsub c\ Z{\isacharparenright}{\kern0pt}{\isacharparenright}{\kern0pt}{\isachardoublequoteclose}\isanewline
%
\isadelimproof
%
\endisadelimproof
%
\isatagproof
\isacommand{proof}\isamarkupfalse%
\ {\isacharparenleft}{\kern0pt}unfold\ reflexive{\isacharunderscore}{\kern0pt}on{\isacharunderscore}{\kern0pt}def{\isacharcomma}{\kern0pt}\ auto{\isacharparenright}{\kern0pt}\isanewline
\ \ \isacommand{have}\isamarkupfalse%
\ {\isachardoublequoteopen}m\ {\isacharcolon}{\kern0pt}\ Y\ {\isasymrightarrow}\ X\ {\isasymtimes}\isactrlsub c\ X\ {\isasymand}\ monomorphism\ m{\isachardoublequoteclose}\isanewline
\ \ \ \ \isacommand{using}\isamarkupfalse%
\ assms\ \isacommand{unfolding}\isamarkupfalse%
\ reflexive{\isacharunderscore}{\kern0pt}on{\isacharunderscore}{\kern0pt}def\ subobject{\isacharunderscore}{\kern0pt}of{\isacharunderscore}{\kern0pt}def{\isadigit{2}}\ \isacommand{by}\isamarkupfalse%
\ auto\isanewline
\ \ \isacommand{then}\isamarkupfalse%
\ \isacommand{show}\isamarkupfalse%
\ {\isachardoublequoteopen}{\isacharparenleft}{\kern0pt}Y\ {\isasymtimes}\isactrlsub c\ Z{\isacharcomma}{\kern0pt}\ distribute{\isacharunderscore}{\kern0pt}right\ X\ X\ Z\ {\isasymcirc}\isactrlsub c\ m\ {\isasymtimes}\isactrlsub f\ id\isactrlsub c\ Z{\isacharparenright}{\kern0pt}\ {\isasymsubseteq}\isactrlsub c\ {\isacharparenleft}{\kern0pt}X\ {\isasymtimes}\isactrlsub c\ Z{\isacharparenright}{\kern0pt}\ {\isasymtimes}\isactrlsub c\ X\ {\isasymtimes}\isactrlsub c\ Z{\isachardoublequoteclose}\isanewline
\ \ \ \ \isacommand{by}\isamarkupfalse%
\ {\isacharparenleft}{\kern0pt}simp\ add{\isacharcolon}{\kern0pt}\ left{\isacharunderscore}{\kern0pt}pair{\isacharunderscore}{\kern0pt}subset{\isacharparenright}{\kern0pt}\isanewline
\isacommand{next}\isamarkupfalse%
\isanewline
\ \ \isacommand{fix}\isamarkupfalse%
\ xz\isanewline
\ \ \isacommand{have}\isamarkupfalse%
\ m{\isacharunderscore}{\kern0pt}type{\isacharcolon}{\kern0pt}\ {\isachardoublequoteopen}m\ {\isacharcolon}{\kern0pt}\ Y\ {\isasymrightarrow}\ X\ {\isasymtimes}\isactrlsub c\ X{\isachardoublequoteclose}\isanewline
\ \ \ \ \isacommand{using}\isamarkupfalse%
\ assms\ \isacommand{unfolding}\isamarkupfalse%
\ reflexive{\isacharunderscore}{\kern0pt}on{\isacharunderscore}{\kern0pt}def\ subobject{\isacharunderscore}{\kern0pt}of{\isacharunderscore}{\kern0pt}def{\isadigit{2}}\ \isacommand{by}\isamarkupfalse%
\ auto\isanewline
\ \ \isacommand{assume}\isamarkupfalse%
\ xz{\isacharunderscore}{\kern0pt}type{\isacharcolon}{\kern0pt}\ {\isachardoublequoteopen}xz\ {\isasymin}\isactrlsub c\ X\ {\isasymtimes}\isactrlsub c\ Z{\isachardoublequoteclose}\isanewline
\ \ \isacommand{then}\isamarkupfalse%
\ \isacommand{obtain}\isamarkupfalse%
\ x\ z\ \isakeyword{where}\ x{\isacharunderscore}{\kern0pt}type{\isacharcolon}{\kern0pt}\ {\isachardoublequoteopen}x\ {\isasymin}\isactrlsub c\ X{\isachardoublequoteclose}\ \isakeyword{and}\ z{\isacharunderscore}{\kern0pt}type{\isacharcolon}{\kern0pt}\ {\isachardoublequoteopen}z\ {\isasymin}\isactrlsub c\ Z{\isachardoublequoteclose}\ \isakeyword{and}\ xz{\isacharunderscore}{\kern0pt}def{\isacharcolon}{\kern0pt}\ {\isachardoublequoteopen}xz\ {\isacharequal}{\kern0pt}\ {\isasymlangle}x{\isacharcomma}{\kern0pt}\ z{\isasymrangle}{\isachardoublequoteclose}\isanewline
\ \ \ \ \isacommand{using}\isamarkupfalse%
\ cart{\isacharunderscore}{\kern0pt}prod{\isacharunderscore}{\kern0pt}decomp\ \isacommand{by}\isamarkupfalse%
\ blast\isanewline
\ \ \isacommand{then}\isamarkupfalse%
\ \isacommand{show}\isamarkupfalse%
\ {\isachardoublequoteopen}{\isasymlangle}xz{\isacharcomma}{\kern0pt}xz{\isasymrangle}\ {\isasymin}\isactrlbsub {\isacharparenleft}{\kern0pt}X\ {\isasymtimes}\isactrlsub c\ Z{\isacharparenright}{\kern0pt}\ {\isasymtimes}\isactrlsub c\ X\ {\isasymtimes}\isactrlsub c\ Z\isactrlesub \ {\isacharparenleft}{\kern0pt}Y\ {\isasymtimes}\isactrlsub c\ Z{\isacharcomma}{\kern0pt}\ distribute{\isacharunderscore}{\kern0pt}right\ X\ X\ Z\ {\isasymcirc}\isactrlsub c\ m\ {\isasymtimes}\isactrlsub f\ id\isactrlsub c\ Z{\isacharparenright}{\kern0pt}{\isachardoublequoteclose}\isanewline
\ \ \ \ \isacommand{using}\isamarkupfalse%
\ m{\isacharunderscore}{\kern0pt}type\isanewline
\ \ \isacommand{proof}\isamarkupfalse%
\ {\isacharparenleft}{\kern0pt}auto{\isacharcomma}{\kern0pt}\ typecheck{\isacharunderscore}{\kern0pt}cfuncs{\isacharcomma}{\kern0pt}\ unfold\ relative{\isacharunderscore}{\kern0pt}member{\isacharunderscore}{\kern0pt}def{\isadigit{2}}{\isacharcomma}{\kern0pt}\ auto{\isacharparenright}{\kern0pt}\isanewline
\ \ \ \ \isacommand{have}\isamarkupfalse%
\ {\isachardoublequoteopen}monomorphism\ m{\isachardoublequoteclose}\isanewline
\ \ \ \ \ \ \isacommand{using}\isamarkupfalse%
\ assms\ \isacommand{unfolding}\isamarkupfalse%
\ reflexive{\isacharunderscore}{\kern0pt}on{\isacharunderscore}{\kern0pt}def\ subobject{\isacharunderscore}{\kern0pt}of{\isacharunderscore}{\kern0pt}def{\isadigit{2}}\ \isacommand{by}\isamarkupfalse%
\ auto\isanewline
\ \ \ \ \isacommand{then}\isamarkupfalse%
\ \isacommand{show}\isamarkupfalse%
\ {\isachardoublequoteopen}monomorphism\ {\isacharparenleft}{\kern0pt}distribute{\isacharunderscore}{\kern0pt}right\ X\ X\ Z\ {\isasymcirc}\isactrlsub c\ m\ {\isasymtimes}\isactrlsub f\ id\isactrlsub c\ Z{\isacharparenright}{\kern0pt}{\isachardoublequoteclose}\isanewline
\ \ \ \ \ \ \isacommand{using}\isamarkupfalse%
\ \ cfunc{\isacharunderscore}{\kern0pt}cross{\isacharunderscore}{\kern0pt}prod{\isacharunderscore}{\kern0pt}mono\ cfunc{\isacharunderscore}{\kern0pt}type{\isacharunderscore}{\kern0pt}def\ composition{\isacharunderscore}{\kern0pt}of{\isacharunderscore}{\kern0pt}monic{\isacharunderscore}{\kern0pt}pair{\isacharunderscore}{\kern0pt}is{\isacharunderscore}{\kern0pt}monic\ distribute{\isacharunderscore}{\kern0pt}right{\isacharunderscore}{\kern0pt}mono\ id{\isacharunderscore}{\kern0pt}isomorphism\ iso{\isacharunderscore}{\kern0pt}imp{\isacharunderscore}{\kern0pt}epi{\isacharunderscore}{\kern0pt}and{\isacharunderscore}{\kern0pt}monic\ m{\isacharunderscore}{\kern0pt}type\ \isacommand{by}\isamarkupfalse%
\ {\isacharparenleft}{\kern0pt}typecheck{\isacharunderscore}{\kern0pt}cfuncs{\isacharcomma}{\kern0pt}\ auto{\isacharparenright}{\kern0pt}\isanewline
\ \ \isacommand{next}\isamarkupfalse%
\isanewline
\ \ \ \ \isacommand{have}\isamarkupfalse%
\ xzxz{\isacharunderscore}{\kern0pt}type{\isacharcolon}{\kern0pt}\ {\isachardoublequoteopen}{\isasymlangle}{\isasymlangle}x{\isacharcomma}{\kern0pt}z{\isasymrangle}{\isacharcomma}{\kern0pt}{\isasymlangle}x{\isacharcomma}{\kern0pt}z{\isasymrangle}{\isasymrangle}\ {\isasymin}\isactrlsub c\ {\isacharparenleft}{\kern0pt}X\ {\isasymtimes}\isactrlsub c\ Z{\isacharparenright}{\kern0pt}\ {\isasymtimes}\isactrlsub c\ X\ {\isasymtimes}\isactrlsub c\ Z{\isachardoublequoteclose}\isanewline
\ \ \ \ \ \ \isacommand{using}\isamarkupfalse%
\ xz{\isacharunderscore}{\kern0pt}type\ cfunc{\isacharunderscore}{\kern0pt}prod{\isacharunderscore}{\kern0pt}type\ xz{\isacharunderscore}{\kern0pt}def\ \isacommand{by}\isamarkupfalse%
\ blast\isanewline
\ \ \ \ \isacommand{obtain}\isamarkupfalse%
\ y\ \isakeyword{where}\ y{\isacharunderscore}{\kern0pt}def{\isacharcolon}{\kern0pt}\ {\isachardoublequoteopen}y\ {\isasymin}\isactrlsub c\ Y{\isachardoublequoteclose}\ {\isachardoublequoteopen}m\ {\isasymcirc}\isactrlsub c\ y\ {\isacharequal}{\kern0pt}\ {\isasymlangle}x{\isacharcomma}{\kern0pt}\ x{\isasymrangle}{\isachardoublequoteclose}\isanewline
\ \ \ \ \ \ \isacommand{using}\isamarkupfalse%
\ assms\ reflexive{\isacharunderscore}{\kern0pt}def{\isadigit{2}}\ x{\isacharunderscore}{\kern0pt}type\ \isacommand{by}\isamarkupfalse%
\ blast\isanewline
\ \ \ \ \isacommand{have}\isamarkupfalse%
\ mid{\isacharunderscore}{\kern0pt}type{\isacharcolon}{\kern0pt}\ {\isachardoublequoteopen}m\ {\isasymtimes}\isactrlsub f\ id\isactrlsub c\ Z\ {\isacharcolon}{\kern0pt}\ Y\ {\isasymtimes}\isactrlsub c\ Z\ {\isasymrightarrow}\ {\isacharparenleft}{\kern0pt}X\ {\isasymtimes}\isactrlsub c\ X{\isacharparenright}{\kern0pt}\ {\isasymtimes}\isactrlsub c\ Z{\isachardoublequoteclose}\isanewline
\ \ \ \ \ \ \isacommand{by}\isamarkupfalse%
\ {\isacharparenleft}{\kern0pt}simp\ add{\isacharcolon}{\kern0pt}\ cfunc{\isacharunderscore}{\kern0pt}cross{\isacharunderscore}{\kern0pt}prod{\isacharunderscore}{\kern0pt}type\ id{\isacharunderscore}{\kern0pt}type\ m{\isacharunderscore}{\kern0pt}type{\isacharparenright}{\kern0pt}\isanewline
\ \ \ \ \isacommand{have}\isamarkupfalse%
\ dist{\isacharunderscore}{\kern0pt}mid{\isacharunderscore}{\kern0pt}type{\isacharcolon}{\kern0pt}{\isachardoublequoteopen}distribute{\isacharunderscore}{\kern0pt}right\ X\ X\ Z\ {\isasymcirc}\isactrlsub c\ m\ {\isasymtimes}\isactrlsub f\ id\isactrlsub c\ Z\ {\isacharcolon}{\kern0pt}\ Y\ {\isasymtimes}\isactrlsub c\ Z\ {\isasymrightarrow}\ {\isacharparenleft}{\kern0pt}X\ {\isasymtimes}\isactrlsub c\ Z{\isacharparenright}{\kern0pt}\ {\isasymtimes}\isactrlsub c\ X\ {\isasymtimes}\isactrlsub c\ Z{\isachardoublequoteclose}\isanewline
\ \ \ \ \ \ \isacommand{using}\isamarkupfalse%
\ comp{\isacharunderscore}{\kern0pt}type\ distribute{\isacharunderscore}{\kern0pt}right{\isacharunderscore}{\kern0pt}type\ mid{\isacharunderscore}{\kern0pt}type\ \isacommand{by}\isamarkupfalse%
\ force\isanewline
\isanewline
\ \ \ \ \isacommand{have}\isamarkupfalse%
\ yz{\isacharunderscore}{\kern0pt}type{\isacharcolon}{\kern0pt}\ {\isachardoublequoteopen}{\isasymlangle}y{\isacharcomma}{\kern0pt}z{\isasymrangle}\ {\isasymin}\isactrlsub c\ Y\ {\isasymtimes}\isactrlsub c\ Z{\isachardoublequoteclose}\isanewline
\ \ \ \ \ \ \isacommand{by}\isamarkupfalse%
\ {\isacharparenleft}{\kern0pt}typecheck{\isacharunderscore}{\kern0pt}cfuncs{\isacharcomma}{\kern0pt}\ simp\ add{\isacharcolon}{\kern0pt}\ {\isacartoucheopen}z\ {\isasymin}\isactrlsub c\ Z{\isacartoucheclose}\ y{\isacharunderscore}{\kern0pt}def{\isacharparenright}{\kern0pt}\isanewline
\ \ \ \ \isacommand{have}\isamarkupfalse%
\ {\isachardoublequoteopen}{\isacharparenleft}{\kern0pt}distribute{\isacharunderscore}{\kern0pt}right\ X\ X\ Z\ {\isasymcirc}\isactrlsub c\ m\ {\isasymtimes}\isactrlsub f\ id\isactrlsub c\ Z{\isacharparenright}{\kern0pt}\ {\isasymcirc}\isactrlsub c\ {\isasymlangle}y{\isacharcomma}{\kern0pt}z{\isasymrangle}\ \ {\isacharequal}{\kern0pt}\ distribute{\isacharunderscore}{\kern0pt}right\ X\ X\ Z\ {\isasymcirc}\isactrlsub c\ {\isacharparenleft}{\kern0pt}m\ {\isasymtimes}\isactrlsub f\ id{\isacharparenleft}{\kern0pt}Z{\isacharparenright}{\kern0pt}{\isacharparenright}{\kern0pt}\ {\isasymcirc}\isactrlsub c\ {\isasymlangle}y{\isacharcomma}{\kern0pt}z{\isasymrangle}{\isachardoublequoteclose}\isanewline
\ \ \ \ \ \ \isacommand{using}\isamarkupfalse%
\ comp{\isacharunderscore}{\kern0pt}associative{\isadigit{2}}\ mid{\isacharunderscore}{\kern0pt}type\ yz{\isacharunderscore}{\kern0pt}type\ \isacommand{by}\isamarkupfalse%
\ {\isacharparenleft}{\kern0pt}typecheck{\isacharunderscore}{\kern0pt}cfuncs{\isacharcomma}{\kern0pt}\ auto{\isacharparenright}{\kern0pt}\isanewline
\ \ \ \ \isacommand{also}\isamarkupfalse%
\ \isacommand{have}\isamarkupfalse%
\ {\isachardoublequoteopen}{\isachardot}{\kern0pt}{\isachardot}{\kern0pt}{\isachardot}{\kern0pt}\ \ {\isacharequal}{\kern0pt}\ \ distribute{\isacharunderscore}{\kern0pt}right\ X\ X\ Z\ {\isasymcirc}\isactrlsub c\ \ {\isasymlangle}m\ {\isasymcirc}\isactrlsub c\ y{\isacharcomma}{\kern0pt}id{\isacharparenleft}{\kern0pt}Z{\isacharparenright}{\kern0pt}\ {\isasymcirc}\isactrlsub c\ z{\isasymrangle}{\isachardoublequoteclose}\isanewline
\ \ \ \ \ \ \isacommand{using}\isamarkupfalse%
\ z{\isacharunderscore}{\kern0pt}type\ cfunc{\isacharunderscore}{\kern0pt}cross{\isacharunderscore}{\kern0pt}prod{\isacharunderscore}{\kern0pt}comp{\isacharunderscore}{\kern0pt}cfunc{\isacharunderscore}{\kern0pt}prod\ m{\isacharunderscore}{\kern0pt}type\ y{\isacharunderscore}{\kern0pt}def\ \isacommand{by}\isamarkupfalse%
\ {\isacharparenleft}{\kern0pt}typecheck{\isacharunderscore}{\kern0pt}cfuncs{\isacharcomma}{\kern0pt}\ auto{\isacharparenright}{\kern0pt}\isanewline
\ \ \ \ \isacommand{also}\isamarkupfalse%
\ \isacommand{have}\isamarkupfalse%
\ distxxz{\isacharcolon}{\kern0pt}\ {\isachardoublequoteopen}{\isachardot}{\kern0pt}{\isachardot}{\kern0pt}{\isachardot}{\kern0pt}\ {\isacharequal}{\kern0pt}\ distribute{\isacharunderscore}{\kern0pt}right\ X\ X\ Z\ {\isasymcirc}\isactrlsub c\ \ {\isasymlangle}\ {\isasymlangle}x{\isacharcomma}{\kern0pt}\ x{\isasymrangle}{\isacharcomma}{\kern0pt}\ z{\isasymrangle}{\isachardoublequoteclose}\isanewline
\ \ \ \ \ \ \isacommand{using}\isamarkupfalse%
\ z{\isacharunderscore}{\kern0pt}type\ id{\isacharunderscore}{\kern0pt}left{\isacharunderscore}{\kern0pt}unit{\isadigit{2}}\ y{\isacharunderscore}{\kern0pt}def\ \isacommand{by}\isamarkupfalse%
\ auto\isanewline
\ \ \ \ \isacommand{also}\isamarkupfalse%
\ \isacommand{have}\isamarkupfalse%
\ {\isachardoublequoteopen}{\isachardot}{\kern0pt}{\isachardot}{\kern0pt}{\isachardot}{\kern0pt}\ {\isacharequal}{\kern0pt}\ {\isasymlangle}{\isasymlangle}x{\isacharcomma}{\kern0pt}z{\isasymrangle}{\isacharcomma}{\kern0pt}{\isasymlangle}x{\isacharcomma}{\kern0pt}z{\isasymrangle}{\isasymrangle}{\isachardoublequoteclose}\isanewline
\ \ \ \ \ \ \isacommand{by}\isamarkupfalse%
\ {\isacharparenleft}{\kern0pt}meson\ z{\isacharunderscore}{\kern0pt}type\ distribute{\isacharunderscore}{\kern0pt}right{\isacharunderscore}{\kern0pt}ap\ x{\isacharunderscore}{\kern0pt}type{\isacharparenright}{\kern0pt}\isanewline
\ \ \ \ \isacommand{then}\isamarkupfalse%
\ \isacommand{have}\isamarkupfalse%
\ {\isachardoublequoteopen}{\isasymexists}h{\isachardot}{\kern0pt}\ {\isasymlangle}{\isasymlangle}x{\isacharcomma}{\kern0pt}z{\isasymrangle}{\isacharcomma}{\kern0pt}{\isasymlangle}x{\isacharcomma}{\kern0pt}z{\isasymrangle}{\isasymrangle}\ {\isacharequal}{\kern0pt}\ {\isacharparenleft}{\kern0pt}distribute{\isacharunderscore}{\kern0pt}right\ X\ X\ Z\ {\isasymcirc}\isactrlsub c\ m\ {\isasymtimes}\isactrlsub f\ id\isactrlsub c\ Z{\isacharparenright}{\kern0pt}\ {\isasymcirc}\isactrlsub c\ h{\isachardoublequoteclose}\isanewline
\ \ \ \ \ \ \isacommand{by}\isamarkupfalse%
\ {\isacharparenleft}{\kern0pt}metis\ \ calculation{\isacharparenright}{\kern0pt}\isanewline
\ \ \ \ \isacommand{then}\isamarkupfalse%
\ \isacommand{show}\isamarkupfalse%
\ {\isachardoublequoteopen}{\isasymlangle}{\isasymlangle}x{\isacharcomma}{\kern0pt}z{\isasymrangle}{\isacharcomma}{\kern0pt}{\isasymlangle}x{\isacharcomma}{\kern0pt}z{\isasymrangle}{\isasymrangle}\ factorsthru\ {\isacharparenleft}{\kern0pt}distribute{\isacharunderscore}{\kern0pt}right\ X\ X\ Z\ {\isasymcirc}\isactrlsub c\ m\ {\isasymtimes}\isactrlsub f\ id\isactrlsub c\ Z{\isacharparenright}{\kern0pt}{\isachardoublequoteclose}\isanewline
\ \ \ \ \ \ \isacommand{using}\isamarkupfalse%
\ \ xzxz{\isacharunderscore}{\kern0pt}type\ z{\isacharunderscore}{\kern0pt}type\ distribute{\isacharunderscore}{\kern0pt}right{\isacharunderscore}{\kern0pt}ap\ x{\isacharunderscore}{\kern0pt}type\ dist{\isacharunderscore}{\kern0pt}mid{\isacharunderscore}{\kern0pt}type\ calculation\ factors{\isacharunderscore}{\kern0pt}through{\isacharunderscore}{\kern0pt}def{\isadigit{2}}\ yz{\isacharunderscore}{\kern0pt}type\ \isacommand{by}\isamarkupfalse%
\ auto\isanewline
\ \ \isacommand{qed}\isamarkupfalse%
\isanewline
\isacommand{qed}\isamarkupfalse%
%
\endisatagproof
{\isafoldproof}%
%
\isadelimproof
\isanewline
%
\endisadelimproof
\isanewline
\isacommand{lemma}\isamarkupfalse%
\ right{\isacharunderscore}{\kern0pt}pair{\isacharunderscore}{\kern0pt}reflexive{\isacharcolon}{\kern0pt}\isanewline
\ \ \isakeyword{assumes}\ {\isachardoublequoteopen}reflexive{\isacharunderscore}{\kern0pt}on\ X\ {\isacharparenleft}{\kern0pt}Y{\isacharcomma}{\kern0pt}\ m{\isacharparenright}{\kern0pt}{\isachardoublequoteclose}\isanewline
\ \ \isakeyword{shows}\ {\isachardoublequoteopen}reflexive{\isacharunderscore}{\kern0pt}on\ {\isacharparenleft}{\kern0pt}Z\ {\isasymtimes}\isactrlsub c\ X{\isacharparenright}{\kern0pt}\ {\isacharparenleft}{\kern0pt}Z\ {\isasymtimes}\isactrlsub c\ Y{\isacharcomma}{\kern0pt}\ distribute{\isacharunderscore}{\kern0pt}left\ Z\ X\ X\ {\isasymcirc}\isactrlsub c\ {\isacharparenleft}{\kern0pt}id\isactrlsub c\ Z\ {\isasymtimes}\isactrlsub f\ m{\isacharparenright}{\kern0pt}{\isacharparenright}{\kern0pt}{\isachardoublequoteclose}\isanewline
%
\isadelimproof
%
\endisadelimproof
%
\isatagproof
\isacommand{proof}\isamarkupfalse%
\ {\isacharparenleft}{\kern0pt}unfold\ reflexive{\isacharunderscore}{\kern0pt}on{\isacharunderscore}{\kern0pt}def{\isacharcomma}{\kern0pt}\ auto{\isacharparenright}{\kern0pt}\isanewline
\ \ \isacommand{have}\isamarkupfalse%
\ {\isachardoublequoteopen}m\ {\isacharcolon}{\kern0pt}\ Y\ {\isasymrightarrow}\ X\ {\isasymtimes}\isactrlsub c\ X\ {\isasymand}\ monomorphism\ m{\isachardoublequoteclose}\isanewline
\ \ \ \ \isacommand{using}\isamarkupfalse%
\ assms\ \isacommand{unfolding}\isamarkupfalse%
\ reflexive{\isacharunderscore}{\kern0pt}on{\isacharunderscore}{\kern0pt}def\ subobject{\isacharunderscore}{\kern0pt}of{\isacharunderscore}{\kern0pt}def{\isadigit{2}}\ \isacommand{by}\isamarkupfalse%
\ auto\isanewline
\ \ \isacommand{then}\isamarkupfalse%
\ \isacommand{show}\isamarkupfalse%
\ {\isachardoublequoteopen}{\isacharparenleft}{\kern0pt}Z\ {\isasymtimes}\isactrlsub c\ Y{\isacharcomma}{\kern0pt}\ distribute{\isacharunderscore}{\kern0pt}left\ Z\ X\ X\ {\isasymcirc}\isactrlsub c\ {\isacharparenleft}{\kern0pt}id\isactrlsub c\ Z\ {\isasymtimes}\isactrlsub f\ m{\isacharparenright}{\kern0pt}{\isacharparenright}{\kern0pt}\ {\isasymsubseteq}\isactrlsub c\ {\isacharparenleft}{\kern0pt}Z\ {\isasymtimes}\isactrlsub c\ X{\isacharparenright}{\kern0pt}\ {\isasymtimes}\isactrlsub c\ Z\ {\isasymtimes}\isactrlsub c\ X{\isachardoublequoteclose}\isanewline
\ \ \ \ \isacommand{by}\isamarkupfalse%
\ {\isacharparenleft}{\kern0pt}simp\ add{\isacharcolon}{\kern0pt}\ right{\isacharunderscore}{\kern0pt}pair{\isacharunderscore}{\kern0pt}subset{\isacharparenright}{\kern0pt}\isanewline
\ \ \isacommand{next}\isamarkupfalse%
\isanewline
\ \ \isacommand{fix}\isamarkupfalse%
\ zx\isanewline
\ \ \isacommand{have}\isamarkupfalse%
\ m{\isacharunderscore}{\kern0pt}type{\isacharcolon}{\kern0pt}\ {\isachardoublequoteopen}m\ {\isacharcolon}{\kern0pt}\ Y\ {\isasymrightarrow}\ X\ {\isasymtimes}\isactrlsub c\ X{\isachardoublequoteclose}\isanewline
\ \ \ \ \isacommand{using}\isamarkupfalse%
\ assms\ \isacommand{unfolding}\isamarkupfalse%
\ reflexive{\isacharunderscore}{\kern0pt}on{\isacharunderscore}{\kern0pt}def\ subobject{\isacharunderscore}{\kern0pt}of{\isacharunderscore}{\kern0pt}def{\isadigit{2}}\ \isacommand{by}\isamarkupfalse%
\ auto\isanewline
\ \ \isacommand{assume}\isamarkupfalse%
\ zx{\isacharunderscore}{\kern0pt}type{\isacharcolon}{\kern0pt}\ {\isachardoublequoteopen}zx\ {\isasymin}\isactrlsub c\ Z\ {\isasymtimes}\isactrlsub c\ X{\isachardoublequoteclose}\isanewline
\ \ \isacommand{then}\isamarkupfalse%
\ \isacommand{obtain}\isamarkupfalse%
\ z\ x\ \isakeyword{where}\ x{\isacharunderscore}{\kern0pt}type{\isacharcolon}{\kern0pt}\ {\isachardoublequoteopen}x\ {\isasymin}\isactrlsub c\ X{\isachardoublequoteclose}\ \isakeyword{and}\ z{\isacharunderscore}{\kern0pt}type{\isacharcolon}{\kern0pt}\ {\isachardoublequoteopen}z\ {\isasymin}\isactrlsub c\ Z{\isachardoublequoteclose}\ \isakeyword{and}\ zx{\isacharunderscore}{\kern0pt}def{\isacharcolon}{\kern0pt}\ {\isachardoublequoteopen}zx\ {\isacharequal}{\kern0pt}\ {\isasymlangle}z{\isacharcomma}{\kern0pt}\ x{\isasymrangle}{\isachardoublequoteclose}\isanewline
\ \ \ \ \isacommand{using}\isamarkupfalse%
\ cart{\isacharunderscore}{\kern0pt}prod{\isacharunderscore}{\kern0pt}decomp\ \isacommand{by}\isamarkupfalse%
\ blast\isanewline
\ \ \isacommand{then}\isamarkupfalse%
\ \isacommand{show}\isamarkupfalse%
\ {\isachardoublequoteopen}{\isasymlangle}zx{\isacharcomma}{\kern0pt}zx{\isasymrangle}\ {\isasymin}\isactrlbsub {\isacharparenleft}{\kern0pt}Z\ {\isasymtimes}\isactrlsub c\ X{\isacharparenright}{\kern0pt}\ {\isasymtimes}\isactrlsub c\ Z\ {\isasymtimes}\isactrlsub c\ X\isactrlesub \ {\isacharparenleft}{\kern0pt}Z\ {\isasymtimes}\isactrlsub c\ Y{\isacharcomma}{\kern0pt}\ distribute{\isacharunderscore}{\kern0pt}left\ Z\ X\ X\ \ {\isasymcirc}\isactrlsub c\ {\isacharparenleft}{\kern0pt}id\isactrlsub c\ Z\ {\isasymtimes}\isactrlsub f\ m{\isacharparenright}{\kern0pt}{\isacharparenright}{\kern0pt}{\isachardoublequoteclose}\isanewline
\ \ \ \ \isacommand{using}\isamarkupfalse%
\ m{\isacharunderscore}{\kern0pt}type\isanewline
\ \ \isacommand{proof}\isamarkupfalse%
\ {\isacharparenleft}{\kern0pt}auto{\isacharcomma}{\kern0pt}\ typecheck{\isacharunderscore}{\kern0pt}cfuncs{\isacharcomma}{\kern0pt}\ unfold\ relative{\isacharunderscore}{\kern0pt}member{\isacharunderscore}{\kern0pt}def{\isadigit{2}}{\isacharcomma}{\kern0pt}\ auto{\isacharparenright}{\kern0pt}\isanewline
\ \ \ \ \isacommand{have}\isamarkupfalse%
\ {\isachardoublequoteopen}monomorphism\ m{\isachardoublequoteclose}\isanewline
\ \ \ \ \ \ \isacommand{using}\isamarkupfalse%
\ assms\ \isacommand{unfolding}\isamarkupfalse%
\ reflexive{\isacharunderscore}{\kern0pt}on{\isacharunderscore}{\kern0pt}def\ subobject{\isacharunderscore}{\kern0pt}of{\isacharunderscore}{\kern0pt}def{\isadigit{2}}\ \isacommand{by}\isamarkupfalse%
\ auto\isanewline
\ \ \ \ \isacommand{then}\isamarkupfalse%
\ \isacommand{show}\isamarkupfalse%
\ {\isachardoublequoteopen}monomorphism\ {\isacharparenleft}{\kern0pt}distribute{\isacharunderscore}{\kern0pt}left\ Z\ X\ X\ \ {\isasymcirc}\isactrlsub c\ {\isacharparenleft}{\kern0pt}id\isactrlsub c\ Z\ {\isasymtimes}\isactrlsub f\ m{\isacharparenright}{\kern0pt}{\isacharparenright}{\kern0pt}{\isachardoublequoteclose}\isanewline
\ \ \ \ \ \ \isacommand{using}\isamarkupfalse%
\ \ cfunc{\isacharunderscore}{\kern0pt}cross{\isacharunderscore}{\kern0pt}prod{\isacharunderscore}{\kern0pt}mono\ cfunc{\isacharunderscore}{\kern0pt}type{\isacharunderscore}{\kern0pt}def\ composition{\isacharunderscore}{\kern0pt}of{\isacharunderscore}{\kern0pt}monic{\isacharunderscore}{\kern0pt}pair{\isacharunderscore}{\kern0pt}is{\isacharunderscore}{\kern0pt}monic\ distribute{\isacharunderscore}{\kern0pt}left{\isacharunderscore}{\kern0pt}mono\ id{\isacharunderscore}{\kern0pt}isomorphism\ iso{\isacharunderscore}{\kern0pt}imp{\isacharunderscore}{\kern0pt}epi{\isacharunderscore}{\kern0pt}and{\isacharunderscore}{\kern0pt}monic\ m{\isacharunderscore}{\kern0pt}type\ \isacommand{by}\isamarkupfalse%
\ {\isacharparenleft}{\kern0pt}typecheck{\isacharunderscore}{\kern0pt}cfuncs{\isacharcomma}{\kern0pt}\ auto{\isacharparenright}{\kern0pt}\isanewline
\ \ \isacommand{next}\isamarkupfalse%
\isanewline
\ \ \ \ \isacommand{have}\isamarkupfalse%
\ zxzx{\isacharunderscore}{\kern0pt}type{\isacharcolon}{\kern0pt}\ {\isachardoublequoteopen}{\isasymlangle}{\isasymlangle}z{\isacharcomma}{\kern0pt}x{\isasymrangle}{\isacharcomma}{\kern0pt}{\isasymlangle}z{\isacharcomma}{\kern0pt}x{\isasymrangle}{\isasymrangle}\ {\isasymin}\isactrlsub c\ {\isacharparenleft}{\kern0pt}Z\ {\isasymtimes}\isactrlsub c\ X{\isacharparenright}{\kern0pt}\ {\isasymtimes}\isactrlsub c\ Z\ {\isasymtimes}\isactrlsub c\ X{\isachardoublequoteclose}\isanewline
\ \ \ \ \ \ \isacommand{using}\isamarkupfalse%
\ zx{\isacharunderscore}{\kern0pt}type\ cfunc{\isacharunderscore}{\kern0pt}prod{\isacharunderscore}{\kern0pt}type\ zx{\isacharunderscore}{\kern0pt}def\ \isacommand{by}\isamarkupfalse%
\ blast\isanewline
\ \ \ \ \isacommand{obtain}\isamarkupfalse%
\ y\ \isakeyword{where}\ y{\isacharunderscore}{\kern0pt}def{\isacharcolon}{\kern0pt}\ {\isachardoublequoteopen}y\ {\isasymin}\isactrlsub c\ Y{\isachardoublequoteclose}\ {\isachardoublequoteopen}m\ {\isasymcirc}\isactrlsub c\ y\ {\isacharequal}{\kern0pt}\ {\isasymlangle}x{\isacharcomma}{\kern0pt}\ x{\isasymrangle}{\isachardoublequoteclose}\isanewline
\ \ \ \ \ \ \isacommand{using}\isamarkupfalse%
\ assms\ reflexive{\isacharunderscore}{\kern0pt}def{\isadigit{2}}\ x{\isacharunderscore}{\kern0pt}type\ \isacommand{by}\isamarkupfalse%
\ blast\isanewline
\ \ \ \ \ \ \ \ \isacommand{have}\isamarkupfalse%
\ mid{\isacharunderscore}{\kern0pt}type{\isacharcolon}{\kern0pt}\ {\isachardoublequoteopen}{\isacharparenleft}{\kern0pt}id\isactrlsub c\ Z\ {\isasymtimes}\isactrlsub f\ m{\isacharparenright}{\kern0pt}\ {\isacharcolon}{\kern0pt}\ Z\ {\isasymtimes}\isactrlsub c\ Y\ {\isasymrightarrow}\ \ \ Z\ {\isasymtimes}\isactrlsub c\ {\isacharparenleft}{\kern0pt}X\ {\isasymtimes}\isactrlsub c\ X{\isacharparenright}{\kern0pt}{\isachardoublequoteclose}\isanewline
\ \ \ \ \ \ \isacommand{by}\isamarkupfalse%
\ {\isacharparenleft}{\kern0pt}simp\ add{\isacharcolon}{\kern0pt}\ cfunc{\isacharunderscore}{\kern0pt}cross{\isacharunderscore}{\kern0pt}prod{\isacharunderscore}{\kern0pt}type\ id{\isacharunderscore}{\kern0pt}type\ m{\isacharunderscore}{\kern0pt}type{\isacharparenright}{\kern0pt}\isanewline
\ \ \ \ \isacommand{have}\isamarkupfalse%
\ dist{\isacharunderscore}{\kern0pt}mid{\isacharunderscore}{\kern0pt}type{\isacharcolon}{\kern0pt}{\isachardoublequoteopen}distribute{\isacharunderscore}{\kern0pt}left\ Z\ X\ X\ \ {\isasymcirc}\isactrlsub c\ {\isacharparenleft}{\kern0pt}id\isactrlsub c\ Z\ {\isasymtimes}\isactrlsub f\ m{\isacharparenright}{\kern0pt}\ {\isacharcolon}{\kern0pt}\ Z\ {\isasymtimes}\isactrlsub c\ Y\ {\isasymrightarrow}\ {\isacharparenleft}{\kern0pt}Z\ {\isasymtimes}\isactrlsub c\ X{\isacharparenright}{\kern0pt}\ {\isasymtimes}\isactrlsub c\ Z\ {\isasymtimes}\isactrlsub c\ X{\isachardoublequoteclose}\isanewline
\ \ \ \ \ \ \isacommand{using}\isamarkupfalse%
\ comp{\isacharunderscore}{\kern0pt}type\ distribute{\isacharunderscore}{\kern0pt}left{\isacharunderscore}{\kern0pt}type\ mid{\isacharunderscore}{\kern0pt}type\ \isacommand{by}\isamarkupfalse%
\ force\isanewline
\ \ \ \ \isacommand{have}\isamarkupfalse%
\ yz{\isacharunderscore}{\kern0pt}type{\isacharcolon}{\kern0pt}\ {\isachardoublequoteopen}{\isasymlangle}z{\isacharcomma}{\kern0pt}y{\isasymrangle}\ {\isasymin}\isactrlsub c\ Z\ {\isasymtimes}\isactrlsub c\ Y{\isachardoublequoteclose}\isanewline
\ \ \ \ \ \ \isacommand{by}\isamarkupfalse%
\ {\isacharparenleft}{\kern0pt}typecheck{\isacharunderscore}{\kern0pt}cfuncs{\isacharcomma}{\kern0pt}\ simp\ add{\isacharcolon}{\kern0pt}\ {\isacartoucheopen}z\ {\isasymin}\isactrlsub c\ Z{\isacartoucheclose}\ y{\isacharunderscore}{\kern0pt}def{\isacharparenright}{\kern0pt}\isanewline
\ \ \ \ \isacommand{have}\isamarkupfalse%
\ {\isachardoublequoteopen}{\isacharparenleft}{\kern0pt}distribute{\isacharunderscore}{\kern0pt}left\ Z\ X\ X\ \ {\isasymcirc}\isactrlsub c\ {\isacharparenleft}{\kern0pt}id\isactrlsub c\ Z\ {\isasymtimes}\isactrlsub f\ m{\isacharparenright}{\kern0pt}{\isacharparenright}{\kern0pt}\ {\isasymcirc}\isactrlsub c\ {\isasymlangle}z{\isacharcomma}{\kern0pt}y{\isasymrangle}\ \ {\isacharequal}{\kern0pt}\ distribute{\isacharunderscore}{\kern0pt}left\ Z\ X\ X\ \ {\isasymcirc}\isactrlsub c\ {\isacharparenleft}{\kern0pt}id\isactrlsub c\ Z\ {\isasymtimes}\isactrlsub f\ m{\isacharparenright}{\kern0pt}\ {\isasymcirc}\isactrlsub c\ {\isasymlangle}z{\isacharcomma}{\kern0pt}y{\isasymrangle}{\isachardoublequoteclose}\isanewline
\ \ \ \ \ \ \isacommand{using}\isamarkupfalse%
\ comp{\isacharunderscore}{\kern0pt}associative{\isadigit{2}}\ mid{\isacharunderscore}{\kern0pt}type\ yz{\isacharunderscore}{\kern0pt}type\ \isacommand{by}\isamarkupfalse%
\ {\isacharparenleft}{\kern0pt}typecheck{\isacharunderscore}{\kern0pt}cfuncs{\isacharcomma}{\kern0pt}\ auto{\isacharparenright}{\kern0pt}\isanewline
\ \ \ \ \isacommand{also}\isamarkupfalse%
\ \isacommand{have}\isamarkupfalse%
\ {\isachardoublequoteopen}{\isachardot}{\kern0pt}{\isachardot}{\kern0pt}{\isachardot}{\kern0pt}\ \ {\isacharequal}{\kern0pt}\ \ distribute{\isacharunderscore}{\kern0pt}left\ Z\ X\ X\ \ {\isasymcirc}\isactrlsub c\ \ {\isasymlangle}id\isactrlsub c\ Z\ {\isasymcirc}\isactrlsub c\ z\ {\isacharcomma}{\kern0pt}\ m\ {\isasymcirc}\isactrlsub c\ y\ {\isasymrangle}{\isachardoublequoteclose}\isanewline
\ \ \ \ \ \ \isacommand{using}\isamarkupfalse%
\ z{\isacharunderscore}{\kern0pt}type\ cfunc{\isacharunderscore}{\kern0pt}cross{\isacharunderscore}{\kern0pt}prod{\isacharunderscore}{\kern0pt}comp{\isacharunderscore}{\kern0pt}cfunc{\isacharunderscore}{\kern0pt}prod\ m{\isacharunderscore}{\kern0pt}type\ y{\isacharunderscore}{\kern0pt}def\ \isacommand{by}\isamarkupfalse%
\ {\isacharparenleft}{\kern0pt}typecheck{\isacharunderscore}{\kern0pt}cfuncs{\isacharcomma}{\kern0pt}\ auto{\isacharparenright}{\kern0pt}\isanewline
\ \ \ \ \isacommand{also}\isamarkupfalse%
\ \isacommand{have}\isamarkupfalse%
\ distxxz{\isacharcolon}{\kern0pt}\ {\isachardoublequoteopen}{\isachardot}{\kern0pt}{\isachardot}{\kern0pt}{\isachardot}{\kern0pt}\ {\isacharequal}{\kern0pt}\ distribute{\isacharunderscore}{\kern0pt}left\ Z\ X\ X\ \ {\isasymcirc}\isactrlsub c\ \ {\isasymlangle}z{\isacharcomma}{\kern0pt}\ {\isasymlangle}x{\isacharcomma}{\kern0pt}\ x{\isasymrangle}{\isasymrangle}{\isachardoublequoteclose}\isanewline
\ \ \ \ \ \ \isacommand{using}\isamarkupfalse%
\ z{\isacharunderscore}{\kern0pt}type\ id{\isacharunderscore}{\kern0pt}left{\isacharunderscore}{\kern0pt}unit{\isadigit{2}}\ y{\isacharunderscore}{\kern0pt}def\ \isacommand{by}\isamarkupfalse%
\ auto\isanewline
\ \ \ \ \isacommand{also}\isamarkupfalse%
\ \isacommand{have}\isamarkupfalse%
\ {\isachardoublequoteopen}{\isachardot}{\kern0pt}{\isachardot}{\kern0pt}{\isachardot}{\kern0pt}\ {\isacharequal}{\kern0pt}\ {\isasymlangle}{\isasymlangle}z{\isacharcomma}{\kern0pt}x{\isasymrangle}{\isacharcomma}{\kern0pt}{\isasymlangle}z{\isacharcomma}{\kern0pt}x{\isasymrangle}{\isasymrangle}{\isachardoublequoteclose}\isanewline
\ \ \ \ \ \ \isacommand{by}\isamarkupfalse%
\ {\isacharparenleft}{\kern0pt}meson\ z{\isacharunderscore}{\kern0pt}type\ distribute{\isacharunderscore}{\kern0pt}left{\isacharunderscore}{\kern0pt}ap\ x{\isacharunderscore}{\kern0pt}type{\isacharparenright}{\kern0pt}\isanewline
\ \ \ \ \isacommand{then}\isamarkupfalse%
\ \isacommand{have}\isamarkupfalse%
\ {\isachardoublequoteopen}{\isasymexists}h{\isachardot}{\kern0pt}\ {\isasymlangle}{\isasymlangle}z{\isacharcomma}{\kern0pt}x{\isasymrangle}{\isacharcomma}{\kern0pt}{\isasymlangle}z{\isacharcomma}{\kern0pt}x{\isasymrangle}{\isasymrangle}\ {\isacharequal}{\kern0pt}\ {\isacharparenleft}{\kern0pt}distribute{\isacharunderscore}{\kern0pt}left\ Z\ X\ X\ \ {\isasymcirc}\isactrlsub c\ {\isacharparenleft}{\kern0pt}id\isactrlsub c\ Z\ {\isasymtimes}\isactrlsub f\ m{\isacharparenright}{\kern0pt}{\isacharparenright}{\kern0pt}\ {\isasymcirc}\isactrlsub c\ h{\isachardoublequoteclose}\isanewline
\ \ \ \ \ \ \isacommand{by}\isamarkupfalse%
\ {\isacharparenleft}{\kern0pt}metis\ \ calculation{\isacharparenright}{\kern0pt}\isanewline
\ \ \ \ \isacommand{then}\isamarkupfalse%
\ \isacommand{show}\isamarkupfalse%
\ {\isachardoublequoteopen}{\isasymlangle}{\isasymlangle}z{\isacharcomma}{\kern0pt}x{\isasymrangle}{\isacharcomma}{\kern0pt}{\isasymlangle}z{\isacharcomma}{\kern0pt}x{\isasymrangle}{\isasymrangle}\ factorsthru\ {\isacharparenleft}{\kern0pt}distribute{\isacharunderscore}{\kern0pt}left\ Z\ X\ X\ \ {\isasymcirc}\isactrlsub c\ {\isacharparenleft}{\kern0pt}id\isactrlsub c\ Z\ {\isasymtimes}\isactrlsub f\ m{\isacharparenright}{\kern0pt}{\isacharparenright}{\kern0pt}{\isachardoublequoteclose}\isanewline
\ \ \ \ \ \ \isacommand{using}\isamarkupfalse%
\ z{\isacharunderscore}{\kern0pt}type\ distribute{\isacharunderscore}{\kern0pt}left{\isacharunderscore}{\kern0pt}ap\ x{\isacharunderscore}{\kern0pt}type\ calculation\ dist{\isacharunderscore}{\kern0pt}mid{\isacharunderscore}{\kern0pt}type\ factors{\isacharunderscore}{\kern0pt}through{\isacharunderscore}{\kern0pt}def{\isadigit{2}}\ yz{\isacharunderscore}{\kern0pt}type\ zxzx{\isacharunderscore}{\kern0pt}type\ \isacommand{by}\isamarkupfalse%
\ auto\isanewline
\ \ \isacommand{qed}\isamarkupfalse%
\isanewline
\isacommand{qed}\isamarkupfalse%
%
\endisatagproof
{\isafoldproof}%
%
\isadelimproof
\isanewline
%
\endisadelimproof
\isanewline
\isacommand{lemma}\isamarkupfalse%
\ left{\isacharunderscore}{\kern0pt}pair{\isacharunderscore}{\kern0pt}symmetric{\isacharcolon}{\kern0pt}\isanewline
\ \ \isakeyword{assumes}\ {\isachardoublequoteopen}symmetric{\isacharunderscore}{\kern0pt}on\ X\ {\isacharparenleft}{\kern0pt}Y{\isacharcomma}{\kern0pt}\ m{\isacharparenright}{\kern0pt}{\isachardoublequoteclose}\isanewline
\ \ \isakeyword{shows}\ {\isachardoublequoteopen}symmetric{\isacharunderscore}{\kern0pt}on\ {\isacharparenleft}{\kern0pt}X\ {\isasymtimes}\isactrlsub c\ Z{\isacharparenright}{\kern0pt}\ {\isacharparenleft}{\kern0pt}Y\ {\isasymtimes}\isactrlsub c\ Z{\isacharcomma}{\kern0pt}\ distribute{\isacharunderscore}{\kern0pt}right\ X\ X\ Z\ {\isasymcirc}\isactrlsub c\ {\isacharparenleft}{\kern0pt}m\ {\isasymtimes}\isactrlsub f\ id\isactrlsub c\ Z{\isacharparenright}{\kern0pt}{\isacharparenright}{\kern0pt}{\isachardoublequoteclose}\isanewline
%
\isadelimproof
%
\endisadelimproof
%
\isatagproof
\isacommand{proof}\isamarkupfalse%
\ {\isacharparenleft}{\kern0pt}unfold\ symmetric{\isacharunderscore}{\kern0pt}on{\isacharunderscore}{\kern0pt}def{\isacharcomma}{\kern0pt}\ auto{\isacharparenright}{\kern0pt}\isanewline
\ \ \isacommand{have}\isamarkupfalse%
\ {\isachardoublequoteopen}m\ {\isacharcolon}{\kern0pt}\ Y\ {\isasymrightarrow}\ X\ {\isasymtimes}\isactrlsub c\ X{\isachardoublequoteclose}\ {\isachardoublequoteopen}monomorphism\ m{\isachardoublequoteclose}\isanewline
\ \ \ \ \isacommand{using}\isamarkupfalse%
\ assms\ subobject{\isacharunderscore}{\kern0pt}of{\isacharunderscore}{\kern0pt}def{\isadigit{2}}\ symmetric{\isacharunderscore}{\kern0pt}on{\isacharunderscore}{\kern0pt}def\ \isacommand{by}\isamarkupfalse%
\ auto\isanewline
\ \ \isacommand{then}\isamarkupfalse%
\ \isacommand{show}\isamarkupfalse%
\ {\isachardoublequoteopen}{\isacharparenleft}{\kern0pt}Y\ {\isasymtimes}\isactrlsub c\ Z{\isacharcomma}{\kern0pt}\ distribute{\isacharunderscore}{\kern0pt}right\ X\ X\ Z\ {\isasymcirc}\isactrlsub c\ m\ {\isasymtimes}\isactrlsub f\ id\isactrlsub c\ Z{\isacharparenright}{\kern0pt}\ {\isasymsubseteq}\isactrlsub c\ {\isacharparenleft}{\kern0pt}X\ {\isasymtimes}\isactrlsub c\ Z{\isacharparenright}{\kern0pt}\ {\isasymtimes}\isactrlsub c\ X\ {\isasymtimes}\isactrlsub c\ Z{\isachardoublequoteclose}\isanewline
\ \ \ \ \isacommand{by}\isamarkupfalse%
\ {\isacharparenleft}{\kern0pt}simp\ add{\isacharcolon}{\kern0pt}\ left{\isacharunderscore}{\kern0pt}pair{\isacharunderscore}{\kern0pt}subset{\isacharparenright}{\kern0pt}\isanewline
\isacommand{next}\isamarkupfalse%
\isanewline
\ \ \isacommand{have}\isamarkupfalse%
\ m{\isacharunderscore}{\kern0pt}def{\isacharbrackleft}{\kern0pt}type{\isacharunderscore}{\kern0pt}rule{\isacharbrackright}{\kern0pt}{\isacharcolon}{\kern0pt}\ {\isachardoublequoteopen}m\ {\isacharcolon}{\kern0pt}\ Y\ {\isasymrightarrow}\ X\ {\isasymtimes}\isactrlsub c\ X{\isachardoublequoteclose}\ {\isachardoublequoteopen}monomorphism\ m{\isachardoublequoteclose}\isanewline
\ \ \ \ \isacommand{using}\isamarkupfalse%
\ assms\ subobject{\isacharunderscore}{\kern0pt}of{\isacharunderscore}{\kern0pt}def{\isadigit{2}}\ symmetric{\isacharunderscore}{\kern0pt}on{\isacharunderscore}{\kern0pt}def\ \isacommand{by}\isamarkupfalse%
\ auto\isanewline
\ \ \isacommand{fix}\isamarkupfalse%
\ s\ t\ \isanewline
\ \ \isacommand{assume}\isamarkupfalse%
\ s{\isacharunderscore}{\kern0pt}type{\isacharbrackleft}{\kern0pt}type{\isacharunderscore}{\kern0pt}rule{\isacharbrackright}{\kern0pt}{\isacharcolon}{\kern0pt}\ {\isachardoublequoteopen}s\ {\isasymin}\isactrlsub c\ X\ {\isasymtimes}\isactrlsub c\ Z{\isachardoublequoteclose}\isanewline
\ \ \isacommand{assume}\isamarkupfalse%
\ t{\isacharunderscore}{\kern0pt}type{\isacharbrackleft}{\kern0pt}type{\isacharunderscore}{\kern0pt}rule{\isacharbrackright}{\kern0pt}{\isacharcolon}{\kern0pt}\ {\isachardoublequoteopen}t\ {\isasymin}\isactrlsub c\ X\ {\isasymtimes}\isactrlsub c\ Z{\isachardoublequoteclose}\isanewline
\ \ \isacommand{assume}\isamarkupfalse%
\ st{\isacharunderscore}{\kern0pt}relation{\isacharcolon}{\kern0pt}\ {\isachardoublequoteopen}{\isasymlangle}s{\isacharcomma}{\kern0pt}t{\isasymrangle}\ {\isasymin}\isactrlbsub {\isacharparenleft}{\kern0pt}X\ {\isasymtimes}\isactrlsub c\ Z{\isacharparenright}{\kern0pt}\ {\isasymtimes}\isactrlsub c\ X\ {\isasymtimes}\isactrlsub c\ Z\isactrlesub \ {\isacharparenleft}{\kern0pt}Y\ {\isasymtimes}\isactrlsub c\ Z{\isacharcomma}{\kern0pt}\ distribute{\isacharunderscore}{\kern0pt}right\ X\ X\ Z\ {\isasymcirc}\isactrlsub c\ m\ {\isasymtimes}\isactrlsub f\ id\isactrlsub c\ Z{\isacharparenright}{\kern0pt}{\isachardoublequoteclose}\isanewline
\ \ \isanewline
\ \ \isacommand{obtain}\isamarkupfalse%
\ sx\ sz\ \isakeyword{where}\ s{\isacharunderscore}{\kern0pt}def{\isacharbrackleft}{\kern0pt}type{\isacharunderscore}{\kern0pt}rule{\isacharbrackright}{\kern0pt}{\isacharcolon}{\kern0pt}\ {\isachardoublequoteopen}\ sx\ {\isasymin}\isactrlsub c\ X{\isachardoublequoteclose}\ {\isachardoublequoteopen}sz\ {\isasymin}\isactrlsub c\ Z{\isachardoublequoteclose}\ {\isachardoublequoteopen}s\ {\isacharequal}{\kern0pt}\ \ {\isasymlangle}sx{\isacharcomma}{\kern0pt}sz{\isasymrangle}{\isachardoublequoteclose}\isanewline
\ \ \ \ \isacommand{using}\isamarkupfalse%
\ cart{\isacharunderscore}{\kern0pt}prod{\isacharunderscore}{\kern0pt}decomp\ s{\isacharunderscore}{\kern0pt}type\ \isacommand{by}\isamarkupfalse%
\ blast\isanewline
\ \ \isacommand{obtain}\isamarkupfalse%
\ tx\ tz\ \isakeyword{where}\ t{\isacharunderscore}{\kern0pt}def{\isacharbrackleft}{\kern0pt}type{\isacharunderscore}{\kern0pt}rule{\isacharbrackright}{\kern0pt}{\isacharcolon}{\kern0pt}\ {\isachardoublequoteopen}tx\ {\isasymin}\isactrlsub c\ X{\isachardoublequoteclose}\ {\isachardoublequoteopen}tz\ {\isasymin}\isactrlsub c\ Z{\isachardoublequoteclose}\ {\isachardoublequoteopen}t\ {\isacharequal}{\kern0pt}\ \ {\isasymlangle}tx{\isacharcomma}{\kern0pt}tz{\isasymrangle}{\isachardoublequoteclose}\isanewline
\ \ \ \ \isacommand{using}\isamarkupfalse%
\ cart{\isacharunderscore}{\kern0pt}prod{\isacharunderscore}{\kern0pt}decomp\ t{\isacharunderscore}{\kern0pt}type\ \isacommand{by}\isamarkupfalse%
\ blast\ \isanewline
\isanewline
\ \ \isacommand{show}\isamarkupfalse%
\ {\isachardoublequoteopen}{\isasymlangle}t{\isacharcomma}{\kern0pt}s{\isasymrangle}\ {\isasymin}\isactrlbsub {\isacharparenleft}{\kern0pt}X\ {\isasymtimes}\isactrlsub c\ Z{\isacharparenright}{\kern0pt}\ {\isasymtimes}\isactrlsub c\ {\isacharparenleft}{\kern0pt}X\ {\isasymtimes}\isactrlsub c\ Z{\isacharparenright}{\kern0pt}\isactrlesub \ {\isacharparenleft}{\kern0pt}Y\ {\isasymtimes}\isactrlsub c\ Z{\isacharcomma}{\kern0pt}\ distribute{\isacharunderscore}{\kern0pt}right\ X\ X\ Z\ {\isasymcirc}\isactrlsub c\ {\isacharparenleft}{\kern0pt}m\ {\isasymtimes}\isactrlsub f\ id\isactrlsub c\ Z{\isacharparenright}{\kern0pt}{\isacharparenright}{\kern0pt}{\isachardoublequoteclose}\ \isanewline
\ \ \ \ \isacommand{using}\isamarkupfalse%
\ s{\isacharunderscore}{\kern0pt}def\ t{\isacharunderscore}{\kern0pt}def\ m{\isacharunderscore}{\kern0pt}def\isanewline
\ \ \isacommand{proof}\isamarkupfalse%
\ {\isacharparenleft}{\kern0pt}simp{\isacharcomma}{\kern0pt}\ typecheck{\isacharunderscore}{\kern0pt}cfuncs{\isacharcomma}{\kern0pt}\ auto{\isacharcomma}{\kern0pt}\ unfold\ relative{\isacharunderscore}{\kern0pt}member{\isacharunderscore}{\kern0pt}def{\isadigit{2}}{\isacharcomma}{\kern0pt}\ auto{\isacharparenright}{\kern0pt}\isanewline
\ \ \ \ \isacommand{show}\isamarkupfalse%
\ {\isachardoublequoteopen}monomorphism\ {\isacharparenleft}{\kern0pt}distribute{\isacharunderscore}{\kern0pt}right\ X\ X\ Z\ {\isasymcirc}\isactrlsub c\ m\ {\isasymtimes}\isactrlsub f\ id\isactrlsub c\ Z{\isacharparenright}{\kern0pt}{\isachardoublequoteclose}\isanewline
\ \ \ \ \ \ \isacommand{using}\isamarkupfalse%
\ relative{\isacharunderscore}{\kern0pt}member{\isacharunderscore}{\kern0pt}def{\isadigit{2}}\ st{\isacharunderscore}{\kern0pt}relation\ \isacommand{by}\isamarkupfalse%
\ blast\isanewline
\isanewline
\ \ \ \ \isacommand{have}\isamarkupfalse%
\ {\isachardoublequoteopen}{\isasymlangle}{\isasymlangle}sx{\isacharcomma}{\kern0pt}sz{\isasymrangle}{\isacharcomma}{\kern0pt}\ {\isasymlangle}tx{\isacharcomma}{\kern0pt}tz{\isasymrangle}{\isasymrangle}\ factorsthru\ {\isacharparenleft}{\kern0pt}distribute{\isacharunderscore}{\kern0pt}right\ X\ X\ Z\ {\isasymcirc}\isactrlsub c\ m\ {\isasymtimes}\isactrlsub f\ id\isactrlsub c\ Z{\isacharparenright}{\kern0pt}{\isachardoublequoteclose}\isanewline
\ \ \ \ \ \ \isacommand{using}\isamarkupfalse%
\ st{\isacharunderscore}{\kern0pt}relation\ s{\isacharunderscore}{\kern0pt}def\ t{\isacharunderscore}{\kern0pt}def\ \isacommand{unfolding}\isamarkupfalse%
\ relative{\isacharunderscore}{\kern0pt}member{\isacharunderscore}{\kern0pt}def{\isadigit{2}}\ \isacommand{by}\isamarkupfalse%
\ auto\isanewline
\ \ \ \ \isacommand{then}\isamarkupfalse%
\ \isacommand{obtain}\isamarkupfalse%
\ yz\ \isakeyword{where}\ yz{\isacharunderscore}{\kern0pt}type{\isacharbrackleft}{\kern0pt}type{\isacharunderscore}{\kern0pt}rule{\isacharbrackright}{\kern0pt}{\isacharcolon}{\kern0pt}\ {\isachardoublequoteopen}yz\ {\isasymin}\isactrlsub c\ Y\ {\isasymtimes}\isactrlsub c\ Z{\isachardoublequoteclose}\isanewline
\ \ \ \ \ \ \isakeyword{and}\ yz{\isacharunderscore}{\kern0pt}def{\isacharcolon}{\kern0pt}\ {\isachardoublequoteopen}{\isacharparenleft}{\kern0pt}distribute{\isacharunderscore}{\kern0pt}right\ X\ X\ Z\ {\isasymcirc}\isactrlsub c\ {\isacharparenleft}{\kern0pt}m\ {\isasymtimes}\isactrlsub f\ id\isactrlsub c\ Z{\isacharparenright}{\kern0pt}{\isacharparenright}{\kern0pt}\ {\isasymcirc}\isactrlsub c\ yz\ {\isacharequal}{\kern0pt}\ {\isasymlangle}{\isasymlangle}sx{\isacharcomma}{\kern0pt}sz{\isasymrangle}{\isacharcomma}{\kern0pt}\ {\isasymlangle}tx{\isacharcomma}{\kern0pt}tz{\isasymrangle}{\isasymrangle}{\isachardoublequoteclose}\isanewline
\ \ \ \ \ \ \isacommand{using}\isamarkupfalse%
\ s{\isacharunderscore}{\kern0pt}def\ t{\isacharunderscore}{\kern0pt}def\ m{\isacharunderscore}{\kern0pt}def\ \isacommand{by}\isamarkupfalse%
\ {\isacharparenleft}{\kern0pt}typecheck{\isacharunderscore}{\kern0pt}cfuncs{\isacharcomma}{\kern0pt}\ unfold\ factors{\isacharunderscore}{\kern0pt}through{\isacharunderscore}{\kern0pt}def{\isadigit{2}}{\isacharcomma}{\kern0pt}\ auto{\isacharparenright}{\kern0pt}\isanewline
\ \ \ \ \isacommand{then}\isamarkupfalse%
\ \isacommand{obtain}\isamarkupfalse%
\ y\ z\ \isakeyword{where}\isanewline
\ \ \ \ \ \ y{\isacharunderscore}{\kern0pt}type{\isacharbrackleft}{\kern0pt}type{\isacharunderscore}{\kern0pt}rule{\isacharbrackright}{\kern0pt}{\isacharcolon}{\kern0pt}\ {\isachardoublequoteopen}y\ {\isasymin}\isactrlsub c\ Y{\isachardoublequoteclose}\ \isakeyword{and}\ z{\isacharunderscore}{\kern0pt}type{\isacharbrackleft}{\kern0pt}type{\isacharunderscore}{\kern0pt}rule{\isacharbrackright}{\kern0pt}{\isacharcolon}{\kern0pt}\ {\isachardoublequoteopen}z\ {\isasymin}\isactrlsub c\ Z{\isachardoublequoteclose}\ \isakeyword{and}\ yz{\isacharunderscore}{\kern0pt}pair{\isacharcolon}{\kern0pt}\ {\isachardoublequoteopen}yz\ {\isacharequal}{\kern0pt}\ {\isasymlangle}y{\isacharcomma}{\kern0pt}\ z{\isasymrangle}{\isachardoublequoteclose}\isanewline
\ \ \ \ \ \ \isacommand{using}\isamarkupfalse%
\ cart{\isacharunderscore}{\kern0pt}prod{\isacharunderscore}{\kern0pt}decomp\ \isacommand{by}\isamarkupfalse%
\ blast\isanewline
\ \ \ \ \isacommand{then}\isamarkupfalse%
\ \isacommand{obtain}\isamarkupfalse%
\ my{\isadigit{1}}\ my{\isadigit{2}}\ \isakeyword{where}\ my{\isacharunderscore}{\kern0pt}types{\isacharbrackleft}{\kern0pt}type{\isacharunderscore}{\kern0pt}rule{\isacharbrackright}{\kern0pt}{\isacharcolon}{\kern0pt}\ {\isachardoublequoteopen}my{\isadigit{1}}\ {\isasymin}\isactrlsub c\ X{\isachardoublequoteclose}\ {\isachardoublequoteopen}my{\isadigit{2}}\ {\isasymin}\isactrlsub c\ X{\isachardoublequoteclose}\ \isakeyword{and}\ my{\isacharunderscore}{\kern0pt}def{\isacharcolon}{\kern0pt}\ {\isachardoublequoteopen}m\ {\isasymcirc}\isactrlsub c\ y\ {\isacharequal}{\kern0pt}\ {\isasymlangle}my{\isadigit{1}}{\isacharcomma}{\kern0pt}my{\isadigit{2}}{\isasymrangle}{\isachardoublequoteclose}\isanewline
\ \ \ \ \ \ \isacommand{by}\isamarkupfalse%
\ {\isacharparenleft}{\kern0pt}metis\ cart{\isacharunderscore}{\kern0pt}prod{\isacharunderscore}{\kern0pt}decomp\ cfunc{\isacharunderscore}{\kern0pt}type{\isacharunderscore}{\kern0pt}def\ codomain{\isacharunderscore}{\kern0pt}comp\ domain{\isacharunderscore}{\kern0pt}comp\ m{\isacharunderscore}{\kern0pt}def{\isacharparenleft}{\kern0pt}{\isadigit{1}}{\isacharparenright}{\kern0pt}{\isacharparenright}{\kern0pt}\isanewline
\ \ \ \ \isacommand{then}\isamarkupfalse%
\ \isacommand{obtain}\isamarkupfalse%
\ y{\isacharprime}{\kern0pt}\ \isakeyword{where}\ y{\isacharprime}{\kern0pt}{\isacharunderscore}{\kern0pt}type{\isacharbrackleft}{\kern0pt}type{\isacharunderscore}{\kern0pt}rule{\isacharbrackright}{\kern0pt}{\isacharcolon}{\kern0pt}\ {\isachardoublequoteopen}y{\isacharprime}{\kern0pt}\ {\isasymin}\isactrlsub c\ Y{\isachardoublequoteclose}\ \isakeyword{and}\ y{\isacharprime}{\kern0pt}{\isacharunderscore}{\kern0pt}def{\isacharcolon}{\kern0pt}\ {\isachardoublequoteopen}m\ {\isasymcirc}\isactrlsub c\ y{\isacharprime}{\kern0pt}\ {\isacharequal}{\kern0pt}\ {\isasymlangle}my{\isadigit{2}}{\isacharcomma}{\kern0pt}my{\isadigit{1}}{\isasymrangle}{\isachardoublequoteclose}\isanewline
\ \ \ \ \ \ \isacommand{using}\isamarkupfalse%
\ assms\ symmetric{\isacharunderscore}{\kern0pt}def{\isadigit{2}}\ y{\isacharunderscore}{\kern0pt}type\ \isacommand{by}\isamarkupfalse%
\ blast\isanewline
\isanewline
\ \ \ \ \isacommand{have}\isamarkupfalse%
\ {\isachardoublequoteopen}{\isacharparenleft}{\kern0pt}distribute{\isacharunderscore}{\kern0pt}right\ X\ X\ Z\ {\isasymcirc}\isactrlsub c\ {\isacharparenleft}{\kern0pt}m\ {\isasymtimes}\isactrlsub f\ id\isactrlsub c\ Z{\isacharparenright}{\kern0pt}{\isacharparenright}{\kern0pt}\ {\isasymcirc}\isactrlsub c\ yz\ {\isacharequal}{\kern0pt}\ {\isasymlangle}{\isasymlangle}my{\isadigit{1}}{\isacharcomma}{\kern0pt}z{\isasymrangle}{\isacharcomma}{\kern0pt}\ {\isasymlangle}my{\isadigit{2}}{\isacharcomma}{\kern0pt}z{\isasymrangle}{\isasymrangle}{\isachardoublequoteclose}\isanewline
\ \ \ \ \isacommand{proof}\isamarkupfalse%
\ {\isacharminus}{\kern0pt}\isanewline
\ \ \ \ \ \ \isacommand{have}\isamarkupfalse%
\ {\isachardoublequoteopen}{\isacharparenleft}{\kern0pt}distribute{\isacharunderscore}{\kern0pt}right\ X\ X\ Z\ {\isasymcirc}\isactrlsub c\ {\isacharparenleft}{\kern0pt}m\ {\isasymtimes}\isactrlsub f\ id\isactrlsub c\ Z{\isacharparenright}{\kern0pt}{\isacharparenright}{\kern0pt}\ {\isasymcirc}\isactrlsub c\ yz\ {\isacharequal}{\kern0pt}\ distribute{\isacharunderscore}{\kern0pt}right\ X\ X\ Z\ {\isasymcirc}\isactrlsub c\ {\isacharparenleft}{\kern0pt}m\ {\isasymtimes}\isactrlsub f\ id\isactrlsub c\ Z{\isacharparenright}{\kern0pt}\ {\isasymcirc}\isactrlsub c\ {\isasymlangle}y{\isacharcomma}{\kern0pt}\ z{\isasymrangle}{\isachardoublequoteclose}\isanewline
\ \ \ \ \ \ \ \ \isacommand{unfolding}\isamarkupfalse%
\ yz{\isacharunderscore}{\kern0pt}pair\ \isacommand{by}\isamarkupfalse%
\ {\isacharparenleft}{\kern0pt}typecheck{\isacharunderscore}{\kern0pt}cfuncs{\isacharcomma}{\kern0pt}\ simp\ add{\isacharcolon}{\kern0pt}\ comp{\isacharunderscore}{\kern0pt}associative{\isadigit{2}}{\isacharparenright}{\kern0pt}\isanewline
\ \ \ \ \ \ \isacommand{also}\isamarkupfalse%
\ \isacommand{have}\isamarkupfalse%
\ {\isachardoublequoteopen}{\isachardot}{\kern0pt}{\isachardot}{\kern0pt}{\isachardot}{\kern0pt}\ {\isacharequal}{\kern0pt}\ distribute{\isacharunderscore}{\kern0pt}right\ X\ X\ Z\ {\isasymcirc}\isactrlsub c\ {\isasymlangle}m\ {\isasymcirc}\isactrlsub c\ y{\isacharcomma}{\kern0pt}\ id\isactrlsub c\ Z\ {\isasymcirc}\isactrlsub c\ z{\isasymrangle}{\isachardoublequoteclose}\isanewline
\ \ \ \ \ \ \ \ \isacommand{by}\isamarkupfalse%
\ {\isacharparenleft}{\kern0pt}typecheck{\isacharunderscore}{\kern0pt}cfuncs{\isacharcomma}{\kern0pt}\ simp\ add{\isacharcolon}{\kern0pt}\ cfunc{\isacharunderscore}{\kern0pt}cross{\isacharunderscore}{\kern0pt}prod{\isacharunderscore}{\kern0pt}comp{\isacharunderscore}{\kern0pt}cfunc{\isacharunderscore}{\kern0pt}prod{\isacharparenright}{\kern0pt}\isanewline
\ \ \ \ \ \ \isacommand{also}\isamarkupfalse%
\ \isacommand{have}\isamarkupfalse%
\ {\isachardoublequoteopen}{\isachardot}{\kern0pt}{\isachardot}{\kern0pt}{\isachardot}{\kern0pt}\ {\isacharequal}{\kern0pt}\ distribute{\isacharunderscore}{\kern0pt}right\ X\ X\ Z\ {\isasymcirc}\isactrlsub c\ {\isasymlangle}{\isasymlangle}my{\isadigit{1}}{\isacharcomma}{\kern0pt}my{\isadigit{2}}{\isasymrangle}{\isacharcomma}{\kern0pt}\ z{\isasymrangle}{\isachardoublequoteclose}\isanewline
\ \ \ \ \ \ \ \ \isacommand{unfolding}\isamarkupfalse%
\ my{\isacharunderscore}{\kern0pt}def\ \isacommand{by}\isamarkupfalse%
\ {\isacharparenleft}{\kern0pt}typecheck{\isacharunderscore}{\kern0pt}cfuncs{\isacharcomma}{\kern0pt}\ simp\ add{\isacharcolon}{\kern0pt}\ id{\isacharunderscore}{\kern0pt}left{\isacharunderscore}{\kern0pt}unit{\isadigit{2}}{\isacharparenright}{\kern0pt}\isanewline
\ \ \ \ \ \ \isacommand{also}\isamarkupfalse%
\ \isacommand{have}\isamarkupfalse%
\ {\isachardoublequoteopen}{\isachardot}{\kern0pt}{\isachardot}{\kern0pt}{\isachardot}{\kern0pt}\ {\isacharequal}{\kern0pt}\ {\isasymlangle}{\isasymlangle}my{\isadigit{1}}{\isacharcomma}{\kern0pt}z{\isasymrangle}{\isacharcomma}{\kern0pt}\ {\isasymlangle}my{\isadigit{2}}{\isacharcomma}{\kern0pt}z{\isasymrangle}{\isasymrangle}{\isachardoublequoteclose}\isanewline
\ \ \ \ \ \ \ \ \isacommand{using}\isamarkupfalse%
\ distribute{\isacharunderscore}{\kern0pt}right{\isacharunderscore}{\kern0pt}ap\ \isacommand{by}\isamarkupfalse%
\ {\isacharparenleft}{\kern0pt}typecheck{\isacharunderscore}{\kern0pt}cfuncs{\isacharcomma}{\kern0pt}\ auto{\isacharparenright}{\kern0pt}\isanewline
\ \ \ \ \ \ \isacommand{then}\isamarkupfalse%
\ \isacommand{show}\isamarkupfalse%
\ {\isacharquery}{\kern0pt}thesis\isanewline
\ \ \ \ \ \ \ \ \isacommand{using}\isamarkupfalse%
\ calculation\ \isacommand{by}\isamarkupfalse%
\ auto\isanewline
\ \ \ \ \isacommand{qed}\isamarkupfalse%
\ \ \ \isanewline
\ \ \ \ \isacommand{then}\isamarkupfalse%
\ \isacommand{have}\isamarkupfalse%
\ {\isachardoublequoteopen}{\isasymlangle}{\isasymlangle}sx{\isacharcomma}{\kern0pt}sz{\isasymrangle}{\isacharcomma}{\kern0pt}{\isasymlangle}tx{\isacharcomma}{\kern0pt}tz{\isasymrangle}{\isasymrangle}\ {\isacharequal}{\kern0pt}\ {\isasymlangle}{\isasymlangle}my{\isadigit{1}}{\isacharcomma}{\kern0pt}z{\isasymrangle}{\isacharcomma}{\kern0pt}{\isasymlangle}my{\isadigit{2}}{\isacharcomma}{\kern0pt}z{\isasymrangle}{\isasymrangle}{\isachardoublequoteclose}\isanewline
\ \ \ \ \ \ \isacommand{using}\isamarkupfalse%
\ yz{\isacharunderscore}{\kern0pt}def\ \isacommand{by}\isamarkupfalse%
\ auto\isanewline
\ \ \ \ \isacommand{then}\isamarkupfalse%
\ \isacommand{have}\isamarkupfalse%
\ {\isachardoublequoteopen}{\isasymlangle}sx{\isacharcomma}{\kern0pt}sz{\isasymrangle}\ {\isacharequal}{\kern0pt}\ {\isasymlangle}my{\isadigit{1}}{\isacharcomma}{\kern0pt}z{\isasymrangle}\ {\isasymand}\ {\isasymlangle}tx{\isacharcomma}{\kern0pt}tz{\isasymrangle}\ {\isacharequal}{\kern0pt}\ {\isasymlangle}my{\isadigit{2}}{\isacharcomma}{\kern0pt}z{\isasymrangle}{\isachardoublequoteclose}\isanewline
\ \ \ \ \ \ \isacommand{using}\isamarkupfalse%
\ element{\isacharunderscore}{\kern0pt}pair{\isacharunderscore}{\kern0pt}eq\ \isacommand{by}\isamarkupfalse%
\ {\isacharparenleft}{\kern0pt}typecheck{\isacharunderscore}{\kern0pt}cfuncs{\isacharcomma}{\kern0pt}\ auto{\isacharparenright}{\kern0pt}\isanewline
\ \ \ \ \isacommand{then}\isamarkupfalse%
\ \isacommand{have}\isamarkupfalse%
\ eqs{\isacharcolon}{\kern0pt}\ {\isachardoublequoteopen}sx\ {\isacharequal}{\kern0pt}\ my{\isadigit{1}}\ {\isasymand}\ sz\ {\isacharequal}{\kern0pt}\ z\ {\isasymand}\ tx\ {\isacharequal}{\kern0pt}\ my{\isadigit{2}}\ {\isasymand}\ tz\ {\isacharequal}{\kern0pt}\ z{\isachardoublequoteclose}\isanewline
\ \ \ \ \ \ \isacommand{using}\isamarkupfalse%
\ element{\isacharunderscore}{\kern0pt}pair{\isacharunderscore}{\kern0pt}eq\ \isacommand{by}\isamarkupfalse%
\ {\isacharparenleft}{\kern0pt}typecheck{\isacharunderscore}{\kern0pt}cfuncs{\isacharcomma}{\kern0pt}\ auto{\isacharparenright}{\kern0pt}\isanewline
\isanewline
\ \ \ \ \isacommand{have}\isamarkupfalse%
\ {\isachardoublequoteopen}{\isacharparenleft}{\kern0pt}distribute{\isacharunderscore}{\kern0pt}right\ X\ X\ Z\ {\isasymcirc}\isactrlsub c\ {\isacharparenleft}{\kern0pt}m\ {\isasymtimes}\isactrlsub f\ id\isactrlsub c\ Z{\isacharparenright}{\kern0pt}{\isacharparenright}{\kern0pt}\ {\isasymcirc}\isactrlsub c\ {\isasymlangle}y{\isacharprime}{\kern0pt}{\isacharcomma}{\kern0pt}z{\isasymrangle}\ {\isacharequal}{\kern0pt}\ {\isasymlangle}{\isasymlangle}tx{\isacharcomma}{\kern0pt}tz{\isasymrangle}{\isacharcomma}{\kern0pt}\ {\isasymlangle}sx{\isacharcomma}{\kern0pt}sz{\isasymrangle}{\isasymrangle}{\isachardoublequoteclose}\isanewline
\ \ \ \ \isacommand{proof}\isamarkupfalse%
\ {\isacharminus}{\kern0pt}\isanewline
\ \ \ \ \ \ \isacommand{have}\isamarkupfalse%
\ {\isachardoublequoteopen}{\isacharparenleft}{\kern0pt}distribute{\isacharunderscore}{\kern0pt}right\ X\ X\ Z\ {\isasymcirc}\isactrlsub c\ {\isacharparenleft}{\kern0pt}m\ {\isasymtimes}\isactrlsub f\ id\isactrlsub c\ Z{\isacharparenright}{\kern0pt}{\isacharparenright}{\kern0pt}\ {\isasymcirc}\isactrlsub c\ {\isasymlangle}y{\isacharprime}{\kern0pt}{\isacharcomma}{\kern0pt}z{\isasymrangle}\ {\isacharequal}{\kern0pt}\ distribute{\isacharunderscore}{\kern0pt}right\ X\ X\ Z\ {\isasymcirc}\isactrlsub c\ {\isacharparenleft}{\kern0pt}m\ {\isasymtimes}\isactrlsub f\ id\isactrlsub c\ Z{\isacharparenright}{\kern0pt}\ {\isasymcirc}\isactrlsub c\ {\isasymlangle}y{\isacharprime}{\kern0pt}{\isacharcomma}{\kern0pt}z{\isasymrangle}{\isachardoublequoteclose}\isanewline
\ \ \ \ \ \ \ \ \isacommand{by}\isamarkupfalse%
\ {\isacharparenleft}{\kern0pt}typecheck{\isacharunderscore}{\kern0pt}cfuncs{\isacharcomma}{\kern0pt}\ simp\ add{\isacharcolon}{\kern0pt}\ comp{\isacharunderscore}{\kern0pt}associative{\isadigit{2}}{\isacharparenright}{\kern0pt}\isanewline
\ \ \ \ \ \ \isacommand{also}\isamarkupfalse%
\ \isacommand{have}\isamarkupfalse%
\ {\isachardoublequoteopen}{\isachardot}{\kern0pt}{\isachardot}{\kern0pt}{\isachardot}{\kern0pt}\ {\isacharequal}{\kern0pt}\ distribute{\isacharunderscore}{\kern0pt}right\ X\ X\ Z\ {\isasymcirc}\isactrlsub c\ {\isasymlangle}m\ {\isasymcirc}\isactrlsub c\ y{\isacharprime}{\kern0pt}{\isacharcomma}{\kern0pt}id\isactrlsub c\ Z\ {\isasymcirc}\isactrlsub c\ z{\isasymrangle}{\isachardoublequoteclose}\isanewline
\ \ \ \ \ \ \ \ \isacommand{by}\isamarkupfalse%
\ {\isacharparenleft}{\kern0pt}typecheck{\isacharunderscore}{\kern0pt}cfuncs{\isacharcomma}{\kern0pt}\ simp\ add{\isacharcolon}{\kern0pt}\ cfunc{\isacharunderscore}{\kern0pt}cross{\isacharunderscore}{\kern0pt}prod{\isacharunderscore}{\kern0pt}comp{\isacharunderscore}{\kern0pt}cfunc{\isacharunderscore}{\kern0pt}prod{\isacharparenright}{\kern0pt}\isanewline
\ \ \ \ \ \ \isacommand{also}\isamarkupfalse%
\ \isacommand{have}\isamarkupfalse%
\ {\isachardoublequoteopen}{\isachardot}{\kern0pt}{\isachardot}{\kern0pt}{\isachardot}{\kern0pt}\ {\isacharequal}{\kern0pt}\ distribute{\isacharunderscore}{\kern0pt}right\ X\ X\ Z\ {\isasymcirc}\isactrlsub c\ {\isasymlangle}{\isasymlangle}my{\isadigit{2}}{\isacharcomma}{\kern0pt}my{\isadigit{1}}{\isasymrangle}{\isacharcomma}{\kern0pt}\ z{\isasymrangle}{\isachardoublequoteclose}\isanewline
\ \ \ \ \ \ \ \ \isacommand{unfolding}\isamarkupfalse%
\ y{\isacharprime}{\kern0pt}{\isacharunderscore}{\kern0pt}def\ \isacommand{by}\isamarkupfalse%
\ {\isacharparenleft}{\kern0pt}typecheck{\isacharunderscore}{\kern0pt}cfuncs{\isacharcomma}{\kern0pt}\ simp\ add{\isacharcolon}{\kern0pt}\ id{\isacharunderscore}{\kern0pt}left{\isacharunderscore}{\kern0pt}unit{\isadigit{2}}{\isacharparenright}{\kern0pt}\isanewline
\ \ \ \ \ \ \isacommand{also}\isamarkupfalse%
\ \isacommand{have}\isamarkupfalse%
\ {\isachardoublequoteopen}{\isachardot}{\kern0pt}{\isachardot}{\kern0pt}{\isachardot}{\kern0pt}\ {\isacharequal}{\kern0pt}\ {\isasymlangle}{\isasymlangle}my{\isadigit{2}}{\isacharcomma}{\kern0pt}z{\isasymrangle}{\isacharcomma}{\kern0pt}\ {\isasymlangle}my{\isadigit{1}}{\isacharcomma}{\kern0pt}z{\isasymrangle}{\isasymrangle}{\isachardoublequoteclose}\isanewline
\ \ \ \ \ \ \ \ \isacommand{using}\isamarkupfalse%
\ distribute{\isacharunderscore}{\kern0pt}right{\isacharunderscore}{\kern0pt}ap\ \isacommand{by}\isamarkupfalse%
\ {\isacharparenleft}{\kern0pt}typecheck{\isacharunderscore}{\kern0pt}cfuncs{\isacharcomma}{\kern0pt}\ auto{\isacharparenright}{\kern0pt}\isanewline
\ \ \ \ \ \ \isacommand{also}\isamarkupfalse%
\ \isacommand{have}\isamarkupfalse%
\ {\isachardoublequoteopen}{\isachardot}{\kern0pt}{\isachardot}{\kern0pt}{\isachardot}{\kern0pt}\ {\isacharequal}{\kern0pt}\ {\isasymlangle}{\isasymlangle}tx{\isacharcomma}{\kern0pt}tz{\isasymrangle}{\isacharcomma}{\kern0pt}\ {\isasymlangle}sx{\isacharcomma}{\kern0pt}sz{\isasymrangle}{\isasymrangle}{\isachardoublequoteclose}\isanewline
\ \ \ \ \ \ \ \ \isacommand{using}\isamarkupfalse%
\ eqs\ \isacommand{by}\isamarkupfalse%
\ auto\isanewline
\ \ \ \ \ \ \isacommand{then}\isamarkupfalse%
\ \isacommand{show}\isamarkupfalse%
\ {\isacharquery}{\kern0pt}thesis\isanewline
\ \ \ \ \ \ \ \ \isacommand{using}\isamarkupfalse%
\ calculation\ \isacommand{by}\isamarkupfalse%
\ auto\isanewline
\ \ \ \ \isacommand{qed}\isamarkupfalse%
\isanewline
\ \ \ \ \isacommand{then}\isamarkupfalse%
\ \isacommand{show}\isamarkupfalse%
\ {\isachardoublequoteopen}{\isasymlangle}{\isasymlangle}tx{\isacharcomma}{\kern0pt}tz{\isasymrangle}{\isacharcomma}{\kern0pt}{\isasymlangle}sx{\isacharcomma}{\kern0pt}sz{\isasymrangle}{\isasymrangle}\ factorsthru\ {\isacharparenleft}{\kern0pt}distribute{\isacharunderscore}{\kern0pt}right\ X\ X\ Z\ {\isasymcirc}\isactrlsub c\ m\ {\isasymtimes}\isactrlsub f\ id\isactrlsub c\ Z{\isacharparenright}{\kern0pt}{\isachardoublequoteclose}\isanewline
\ \ \ \ \ \ \isacommand{by}\isamarkupfalse%
\ {\isacharparenleft}{\kern0pt}typecheck{\isacharunderscore}{\kern0pt}cfuncs{\isacharcomma}{\kern0pt}\ unfold\ factors{\isacharunderscore}{\kern0pt}through{\isacharunderscore}{\kern0pt}def{\isadigit{2}}{\isacharcomma}{\kern0pt}\ rule{\isacharunderscore}{\kern0pt}tac\ x{\isacharequal}{\kern0pt}{\isachardoublequoteopen}{\isasymlangle}y{\isacharprime}{\kern0pt}{\isacharcomma}{\kern0pt}z{\isasymrangle}{\isachardoublequoteclose}\ \isakeyword{in}\ exI{\isacharcomma}{\kern0pt}\ typecheck{\isacharunderscore}{\kern0pt}cfuncs{\isacharparenright}{\kern0pt}\isanewline
\ \ \isacommand{qed}\isamarkupfalse%
\isanewline
\isacommand{qed}\isamarkupfalse%
%
\endisatagproof
{\isafoldproof}%
%
\isadelimproof
\isanewline
%
\endisadelimproof
\isanewline
\isacommand{lemma}\isamarkupfalse%
\ right{\isacharunderscore}{\kern0pt}pair{\isacharunderscore}{\kern0pt}symmetric{\isacharcolon}{\kern0pt}\isanewline
\ \ \isakeyword{assumes}\ {\isachardoublequoteopen}symmetric{\isacharunderscore}{\kern0pt}on\ X\ {\isacharparenleft}{\kern0pt}Y{\isacharcomma}{\kern0pt}\ m{\isacharparenright}{\kern0pt}{\isachardoublequoteclose}\isanewline
\ \ \isakeyword{shows}\ {\isachardoublequoteopen}symmetric{\isacharunderscore}{\kern0pt}on\ {\isacharparenleft}{\kern0pt}Z\ {\isasymtimes}\isactrlsub c\ X{\isacharparenright}{\kern0pt}\ {\isacharparenleft}{\kern0pt}Z\ {\isasymtimes}\isactrlsub c\ Y{\isacharcomma}{\kern0pt}\ distribute{\isacharunderscore}{\kern0pt}left\ Z\ X\ X\ \ {\isasymcirc}\isactrlsub c\ {\isacharparenleft}{\kern0pt}id\isactrlsub c\ Z\ {\isasymtimes}\isactrlsub f\ m{\isacharparenright}{\kern0pt}{\isacharparenright}{\kern0pt}{\isachardoublequoteclose}\isanewline
%
\isadelimproof
%
\endisadelimproof
%
\isatagproof
\isacommand{proof}\isamarkupfalse%
\ {\isacharparenleft}{\kern0pt}unfold\ symmetric{\isacharunderscore}{\kern0pt}on{\isacharunderscore}{\kern0pt}def{\isacharcomma}{\kern0pt}\ auto{\isacharparenright}{\kern0pt}\isanewline
\ \ \isacommand{have}\isamarkupfalse%
\ {\isachardoublequoteopen}m\ {\isacharcolon}{\kern0pt}\ Y\ {\isasymrightarrow}\ X\ {\isasymtimes}\isactrlsub c\ X{\isachardoublequoteclose}\ {\isachardoublequoteopen}monomorphism\ m{\isachardoublequoteclose}\isanewline
\ \ \ \ \isacommand{using}\isamarkupfalse%
\ assms\ subobject{\isacharunderscore}{\kern0pt}of{\isacharunderscore}{\kern0pt}def{\isadigit{2}}\ symmetric{\isacharunderscore}{\kern0pt}on{\isacharunderscore}{\kern0pt}def\ \isacommand{by}\isamarkupfalse%
\ auto\isanewline
\ \ \isacommand{then}\isamarkupfalse%
\ \isacommand{show}\isamarkupfalse%
\ {\isachardoublequoteopen}{\isacharparenleft}{\kern0pt}Z\ {\isasymtimes}\isactrlsub c\ Y{\isacharcomma}{\kern0pt}\ distribute{\isacharunderscore}{\kern0pt}left\ Z\ X\ X\ \ {\isasymcirc}\isactrlsub c\ {\isacharparenleft}{\kern0pt}id\isactrlsub c\ Z\ {\isasymtimes}\isactrlsub f\ m{\isacharparenright}{\kern0pt}{\isacharparenright}{\kern0pt}\ {\isasymsubseteq}\isactrlsub c\ {\isacharparenleft}{\kern0pt}Z\ {\isasymtimes}\isactrlsub c\ X{\isacharparenright}{\kern0pt}\ {\isasymtimes}\isactrlsub c\ Z\ {\isasymtimes}\isactrlsub c\ X{\isachardoublequoteclose}\isanewline
\ \ \ \ \isacommand{by}\isamarkupfalse%
\ {\isacharparenleft}{\kern0pt}simp\ add{\isacharcolon}{\kern0pt}\ right{\isacharunderscore}{\kern0pt}pair{\isacharunderscore}{\kern0pt}subset{\isacharparenright}{\kern0pt}\isanewline
\isacommand{next}\isamarkupfalse%
\isanewline
\ \ \isacommand{have}\isamarkupfalse%
\ m{\isacharunderscore}{\kern0pt}def{\isacharbrackleft}{\kern0pt}type{\isacharunderscore}{\kern0pt}rule{\isacharbrackright}{\kern0pt}{\isacharcolon}{\kern0pt}\ {\isachardoublequoteopen}m\ {\isacharcolon}{\kern0pt}\ Y\ {\isasymrightarrow}\ X\ {\isasymtimes}\isactrlsub c\ X{\isachardoublequoteclose}\ {\isachardoublequoteopen}monomorphism\ m{\isachardoublequoteclose}\isanewline
\ \ \ \ \isacommand{using}\isamarkupfalse%
\ assms\ subobject{\isacharunderscore}{\kern0pt}of{\isacharunderscore}{\kern0pt}def{\isadigit{2}}\ symmetric{\isacharunderscore}{\kern0pt}on{\isacharunderscore}{\kern0pt}def\ \isacommand{by}\isamarkupfalse%
\ auto\isanewline
\isanewline
\ \ \isacommand{fix}\isamarkupfalse%
\ s\ t\ \isanewline
\ \ \isacommand{assume}\isamarkupfalse%
\ s{\isacharunderscore}{\kern0pt}type{\isacharbrackleft}{\kern0pt}type{\isacharunderscore}{\kern0pt}rule{\isacharbrackright}{\kern0pt}{\isacharcolon}{\kern0pt}\ {\isachardoublequoteopen}s\ {\isasymin}\isactrlsub c\ Z\ {\isasymtimes}\isactrlsub c\ X{\isachardoublequoteclose}\isanewline
\ \ \isacommand{assume}\isamarkupfalse%
\ t{\isacharunderscore}{\kern0pt}type{\isacharbrackleft}{\kern0pt}type{\isacharunderscore}{\kern0pt}rule{\isacharbrackright}{\kern0pt}{\isacharcolon}{\kern0pt}\ {\isachardoublequoteopen}t\ {\isasymin}\isactrlsub c\ Z\ {\isasymtimes}\isactrlsub c\ X{\isachardoublequoteclose}\isanewline
\ \ \isacommand{assume}\isamarkupfalse%
\ st{\isacharunderscore}{\kern0pt}relation{\isacharcolon}{\kern0pt}\ {\isachardoublequoteopen}{\isasymlangle}s{\isacharcomma}{\kern0pt}t{\isasymrangle}\ {\isasymin}\isactrlbsub {\isacharparenleft}{\kern0pt}Z\ {\isasymtimes}\isactrlsub c\ X{\isacharparenright}{\kern0pt}\ {\isasymtimes}\isactrlsub c\ Z\ {\isasymtimes}\isactrlsub c\ X\isactrlesub \ {\isacharparenleft}{\kern0pt}Z\ {\isasymtimes}\isactrlsub c\ Y{\isacharcomma}{\kern0pt}\ distribute{\isacharunderscore}{\kern0pt}left\ Z\ X\ X\ \ {\isasymcirc}\isactrlsub c\ {\isacharparenleft}{\kern0pt}id\isactrlsub c\ Z\ {\isasymtimes}\isactrlsub f\ m{\isacharparenright}{\kern0pt}{\isacharparenright}{\kern0pt}{\isachardoublequoteclose}\isanewline
\ \ \isanewline
\ \ \isacommand{obtain}\isamarkupfalse%
\ xs\ zs\ \isakeyword{where}\ s{\isacharunderscore}{\kern0pt}def{\isacharbrackleft}{\kern0pt}type{\isacharunderscore}{\kern0pt}rule{\isacharbrackright}{\kern0pt}{\isacharcolon}{\kern0pt}\ {\isachardoublequoteopen}\ xs\ {\isasymin}\isactrlsub c\ Z{\isachardoublequoteclose}\ {\isachardoublequoteopen}zs\ {\isasymin}\isactrlsub c\ X{\isachardoublequoteclose}\ {\isachardoublequoteopen}s\ {\isacharequal}{\kern0pt}\ \ {\isasymlangle}xs{\isacharcomma}{\kern0pt}zs{\isasymrangle}{\isachardoublequoteclose}\isanewline
\ \ \ \ \isacommand{using}\isamarkupfalse%
\ cart{\isacharunderscore}{\kern0pt}prod{\isacharunderscore}{\kern0pt}decomp\ s{\isacharunderscore}{\kern0pt}type\ \isacommand{by}\isamarkupfalse%
\ blast\isanewline
\ \ \isacommand{obtain}\isamarkupfalse%
\ xt\ zt\ \isakeyword{where}\ t{\isacharunderscore}{\kern0pt}def{\isacharbrackleft}{\kern0pt}type{\isacharunderscore}{\kern0pt}rule{\isacharbrackright}{\kern0pt}{\isacharcolon}{\kern0pt}\ {\isachardoublequoteopen}xt\ {\isasymin}\isactrlsub c\ Z{\isachardoublequoteclose}\ {\isachardoublequoteopen}zt\ {\isasymin}\isactrlsub c\ X{\isachardoublequoteclose}\ {\isachardoublequoteopen}t\ {\isacharequal}{\kern0pt}\ \ {\isasymlangle}xt{\isacharcomma}{\kern0pt}zt{\isasymrangle}{\isachardoublequoteclose}\isanewline
\ \ \ \ \isacommand{using}\isamarkupfalse%
\ cart{\isacharunderscore}{\kern0pt}prod{\isacharunderscore}{\kern0pt}decomp\ t{\isacharunderscore}{\kern0pt}type\ \isacommand{by}\isamarkupfalse%
\ blast\ \isanewline
\isanewline
\ \ \isacommand{show}\isamarkupfalse%
\ {\isachardoublequoteopen}{\isasymlangle}t{\isacharcomma}{\kern0pt}s{\isasymrangle}\ {\isasymin}\isactrlbsub {\isacharparenleft}{\kern0pt}Z\ {\isasymtimes}\isactrlsub c\ X{\isacharparenright}{\kern0pt}\ {\isasymtimes}\isactrlsub c\ {\isacharparenleft}{\kern0pt}Z\ {\isasymtimes}\isactrlsub c\ X{\isacharparenright}{\kern0pt}\isactrlesub \ {\isacharparenleft}{\kern0pt}Z\ {\isasymtimes}\isactrlsub c\ Y{\isacharcomma}{\kern0pt}\ distribute{\isacharunderscore}{\kern0pt}left\ Z\ X\ X\ \ {\isasymcirc}\isactrlsub c\ {\isacharparenleft}{\kern0pt}id\isactrlsub c\ Z\ {\isasymtimes}\isactrlsub f\ m{\isacharparenright}{\kern0pt}{\isacharparenright}{\kern0pt}{\isachardoublequoteclose}\ \isanewline
\ \ \ \ \isacommand{using}\isamarkupfalse%
\ s{\isacharunderscore}{\kern0pt}def\ t{\isacharunderscore}{\kern0pt}def\ m{\isacharunderscore}{\kern0pt}def\isanewline
\ \ \isacommand{proof}\isamarkupfalse%
\ {\isacharparenleft}{\kern0pt}simp{\isacharcomma}{\kern0pt}\ typecheck{\isacharunderscore}{\kern0pt}cfuncs{\isacharcomma}{\kern0pt}\ auto{\isacharcomma}{\kern0pt}\ unfold\ relative{\isacharunderscore}{\kern0pt}member{\isacharunderscore}{\kern0pt}def{\isadigit{2}}{\isacharcomma}{\kern0pt}\ auto{\isacharparenright}{\kern0pt}\isanewline
\ \ \ \ \isacommand{show}\isamarkupfalse%
\ {\isachardoublequoteopen}monomorphism\ {\isacharparenleft}{\kern0pt}distribute{\isacharunderscore}{\kern0pt}left\ Z\ X\ X\ \ {\isasymcirc}\isactrlsub c\ {\isacharparenleft}{\kern0pt}id\isactrlsub c\ Z\ {\isasymtimes}\isactrlsub f\ m{\isacharparenright}{\kern0pt}{\isacharparenright}{\kern0pt}{\isachardoublequoteclose}\isanewline
\ \ \ \ \ \ \isacommand{using}\isamarkupfalse%
\ relative{\isacharunderscore}{\kern0pt}member{\isacharunderscore}{\kern0pt}def{\isadigit{2}}\ st{\isacharunderscore}{\kern0pt}relation\ \isacommand{by}\isamarkupfalse%
\ blast\isanewline
\isanewline
\ \ \ \ \isacommand{have}\isamarkupfalse%
\ {\isachardoublequoteopen}{\isasymlangle}{\isasymlangle}xs{\isacharcomma}{\kern0pt}zs{\isasymrangle}{\isacharcomma}{\kern0pt}\ {\isasymlangle}xt{\isacharcomma}{\kern0pt}zt{\isasymrangle}{\isasymrangle}\ factorsthru\ {\isacharparenleft}{\kern0pt}distribute{\isacharunderscore}{\kern0pt}left\ Z\ X\ X\ \ {\isasymcirc}\isactrlsub c\ {\isacharparenleft}{\kern0pt}id\isactrlsub c\ Z\ {\isasymtimes}\isactrlsub f\ m{\isacharparenright}{\kern0pt}{\isacharparenright}{\kern0pt}{\isachardoublequoteclose}\isanewline
\ \ \ \ \ \ \isacommand{using}\isamarkupfalse%
\ st{\isacharunderscore}{\kern0pt}relation\ s{\isacharunderscore}{\kern0pt}def\ t{\isacharunderscore}{\kern0pt}def\ \isacommand{unfolding}\isamarkupfalse%
\ relative{\isacharunderscore}{\kern0pt}member{\isacharunderscore}{\kern0pt}def{\isadigit{2}}\ \isacommand{by}\isamarkupfalse%
\ auto\isanewline
\ \ \ \ \isacommand{then}\isamarkupfalse%
\ \isacommand{obtain}\isamarkupfalse%
\ zy\ \isakeyword{where}\ zy{\isacharunderscore}{\kern0pt}type{\isacharbrackleft}{\kern0pt}type{\isacharunderscore}{\kern0pt}rule{\isacharbrackright}{\kern0pt}{\isacharcolon}{\kern0pt}\ {\isachardoublequoteopen}zy\ {\isasymin}\isactrlsub c\ Z\ {\isasymtimes}\isactrlsub c\ Y{\isachardoublequoteclose}\isanewline
\ \ \ \ \ \ \isakeyword{and}\ zy{\isacharunderscore}{\kern0pt}def{\isacharcolon}{\kern0pt}\ {\isachardoublequoteopen}{\isacharparenleft}{\kern0pt}distribute{\isacharunderscore}{\kern0pt}left\ Z\ X\ X\ \ {\isasymcirc}\isactrlsub c\ {\isacharparenleft}{\kern0pt}id\isactrlsub c\ Z\ {\isasymtimes}\isactrlsub f\ m{\isacharparenright}{\kern0pt}{\isacharparenright}{\kern0pt}\ {\isasymcirc}\isactrlsub c\ zy\ {\isacharequal}{\kern0pt}\ {\isasymlangle}{\isasymlangle}xs{\isacharcomma}{\kern0pt}zs{\isasymrangle}{\isacharcomma}{\kern0pt}\ {\isasymlangle}xt{\isacharcomma}{\kern0pt}zt{\isasymrangle}{\isasymrangle}{\isachardoublequoteclose}\isanewline
\ \ \ \ \ \ \isacommand{using}\isamarkupfalse%
\ s{\isacharunderscore}{\kern0pt}def\ t{\isacharunderscore}{\kern0pt}def\ m{\isacharunderscore}{\kern0pt}def\ \isacommand{by}\isamarkupfalse%
\ {\isacharparenleft}{\kern0pt}typecheck{\isacharunderscore}{\kern0pt}cfuncs{\isacharcomma}{\kern0pt}\ unfold\ factors{\isacharunderscore}{\kern0pt}through{\isacharunderscore}{\kern0pt}def{\isadigit{2}}{\isacharcomma}{\kern0pt}\ auto{\isacharparenright}{\kern0pt}\isanewline
\ \ \ \ \isacommand{then}\isamarkupfalse%
\ \isacommand{obtain}\isamarkupfalse%
\ y\ z\ \isakeyword{where}\isanewline
\ \ \ \ \ \ y{\isacharunderscore}{\kern0pt}type{\isacharbrackleft}{\kern0pt}type{\isacharunderscore}{\kern0pt}rule{\isacharbrackright}{\kern0pt}{\isacharcolon}{\kern0pt}\ {\isachardoublequoteopen}y\ {\isasymin}\isactrlsub c\ Y{\isachardoublequoteclose}\ \isakeyword{and}\ z{\isacharunderscore}{\kern0pt}type{\isacharbrackleft}{\kern0pt}type{\isacharunderscore}{\kern0pt}rule{\isacharbrackright}{\kern0pt}{\isacharcolon}{\kern0pt}\ {\isachardoublequoteopen}z\ {\isasymin}\isactrlsub c\ Z{\isachardoublequoteclose}\ \isakeyword{and}\ yz{\isacharunderscore}{\kern0pt}pair{\isacharcolon}{\kern0pt}\ {\isachardoublequoteopen}zy\ {\isacharequal}{\kern0pt}\ {\isasymlangle}z{\isacharcomma}{\kern0pt}\ y{\isasymrangle}{\isachardoublequoteclose}\isanewline
\ \ \ \ \ \ \isacommand{using}\isamarkupfalse%
\ cart{\isacharunderscore}{\kern0pt}prod{\isacharunderscore}{\kern0pt}decomp\ \isacommand{by}\isamarkupfalse%
\ blast\isanewline
\ \ \ \ \isacommand{then}\isamarkupfalse%
\ \isacommand{obtain}\isamarkupfalse%
\ my{\isadigit{1}}\ my{\isadigit{2}}\ \isakeyword{where}\ my{\isacharunderscore}{\kern0pt}types{\isacharbrackleft}{\kern0pt}type{\isacharunderscore}{\kern0pt}rule{\isacharbrackright}{\kern0pt}{\isacharcolon}{\kern0pt}\ {\isachardoublequoteopen}my{\isadigit{1}}\ {\isasymin}\isactrlsub c\ X{\isachardoublequoteclose}\ {\isachardoublequoteopen}my{\isadigit{2}}\ {\isasymin}\isactrlsub c\ X{\isachardoublequoteclose}\ \isakeyword{and}\ my{\isacharunderscore}{\kern0pt}def{\isacharcolon}{\kern0pt}\ {\isachardoublequoteopen}m\ {\isasymcirc}\isactrlsub c\ y\ {\isacharequal}{\kern0pt}\ {\isasymlangle}my{\isadigit{2}}{\isacharcomma}{\kern0pt}my{\isadigit{1}}{\isasymrangle}{\isachardoublequoteclose}\isanewline
\ \ \ \ \ \ \isacommand{by}\isamarkupfalse%
\ {\isacharparenleft}{\kern0pt}metis\ cart{\isacharunderscore}{\kern0pt}prod{\isacharunderscore}{\kern0pt}decomp\ cfunc{\isacharunderscore}{\kern0pt}type{\isacharunderscore}{\kern0pt}def\ codomain{\isacharunderscore}{\kern0pt}comp\ domain{\isacharunderscore}{\kern0pt}comp\ m{\isacharunderscore}{\kern0pt}def{\isacharparenleft}{\kern0pt}{\isadigit{1}}{\isacharparenright}{\kern0pt}{\isacharparenright}{\kern0pt}\isanewline
\ \ \ \ \isacommand{then}\isamarkupfalse%
\ \isacommand{obtain}\isamarkupfalse%
\ y{\isacharprime}{\kern0pt}\ \isakeyword{where}\ y{\isacharprime}{\kern0pt}{\isacharunderscore}{\kern0pt}type{\isacharbrackleft}{\kern0pt}type{\isacharunderscore}{\kern0pt}rule{\isacharbrackright}{\kern0pt}{\isacharcolon}{\kern0pt}\ {\isachardoublequoteopen}y{\isacharprime}{\kern0pt}\ {\isasymin}\isactrlsub c\ Y{\isachardoublequoteclose}\ \isakeyword{and}\ y{\isacharprime}{\kern0pt}{\isacharunderscore}{\kern0pt}def{\isacharcolon}{\kern0pt}\ {\isachardoublequoteopen}m\ {\isasymcirc}\isactrlsub c\ y{\isacharprime}{\kern0pt}\ {\isacharequal}{\kern0pt}\ {\isasymlangle}my{\isadigit{1}}{\isacharcomma}{\kern0pt}my{\isadigit{2}}{\isasymrangle}{\isachardoublequoteclose}\isanewline
\ \ \ \ \ \ \isacommand{using}\isamarkupfalse%
\ assms\ symmetric{\isacharunderscore}{\kern0pt}def{\isadigit{2}}\ y{\isacharunderscore}{\kern0pt}type\ \isacommand{by}\isamarkupfalse%
\ blast\isanewline
\isanewline
\ \ \ \ \isacommand{have}\isamarkupfalse%
\ {\isachardoublequoteopen}{\isacharparenleft}{\kern0pt}distribute{\isacharunderscore}{\kern0pt}left\ Z\ X\ X\ \ {\isasymcirc}\isactrlsub c\ {\isacharparenleft}{\kern0pt}id\isactrlsub c\ Z\ {\isasymtimes}\isactrlsub f\ m{\isacharparenright}{\kern0pt}{\isacharparenright}{\kern0pt}\ {\isasymcirc}\isactrlsub c\ zy\ {\isacharequal}{\kern0pt}\ {\isasymlangle}{\isasymlangle}z{\isacharcomma}{\kern0pt}my{\isadigit{2}}{\isasymrangle}{\isacharcomma}{\kern0pt}\ {\isasymlangle}z{\isacharcomma}{\kern0pt}my{\isadigit{1}}{\isasymrangle}{\isasymrangle}{\isachardoublequoteclose}\isanewline
\ \ \ \ \isacommand{proof}\isamarkupfalse%
\ {\isacharminus}{\kern0pt}\isanewline
\ \ \ \ \ \ \isacommand{have}\isamarkupfalse%
\ {\isachardoublequoteopen}{\isacharparenleft}{\kern0pt}distribute{\isacharunderscore}{\kern0pt}left\ Z\ X\ X\ \ {\isasymcirc}\isactrlsub c\ {\isacharparenleft}{\kern0pt}id\isactrlsub c\ Z\ {\isasymtimes}\isactrlsub f\ m{\isacharparenright}{\kern0pt}{\isacharparenright}{\kern0pt}\ {\isasymcirc}\isactrlsub c\ zy\ {\isacharequal}{\kern0pt}\ distribute{\isacharunderscore}{\kern0pt}left\ Z\ X\ X\ \ {\isasymcirc}\isactrlsub c\ {\isacharparenleft}{\kern0pt}id\isactrlsub c\ Z\ {\isasymtimes}\isactrlsub f\ m{\isacharparenright}{\kern0pt}\ {\isasymcirc}\isactrlsub c\ zy{\isachardoublequoteclose}\isanewline
\ \ \ \ \ \ \ \ \isacommand{unfolding}\isamarkupfalse%
\ yz{\isacharunderscore}{\kern0pt}pair\ \isacommand{by}\isamarkupfalse%
\ {\isacharparenleft}{\kern0pt}typecheck{\isacharunderscore}{\kern0pt}cfuncs{\isacharcomma}{\kern0pt}\ simp\ add{\isacharcolon}{\kern0pt}\ comp{\isacharunderscore}{\kern0pt}associative{\isadigit{2}}{\isacharparenright}{\kern0pt}\isanewline
\ \ \ \ \ \ \isacommand{also}\isamarkupfalse%
\ \isacommand{have}\isamarkupfalse%
\ {\isachardoublequoteopen}{\isachardot}{\kern0pt}{\isachardot}{\kern0pt}{\isachardot}{\kern0pt}\ {\isacharequal}{\kern0pt}\ distribute{\isacharunderscore}{\kern0pt}left\ Z\ X\ X\ \ {\isasymcirc}\isactrlsub c\ {\isasymlangle}id\isactrlsub c\ Z\ {\isasymcirc}\isactrlsub c\ z\ {\isacharcomma}{\kern0pt}\ m\ {\isasymcirc}\isactrlsub c\ y{\isasymrangle}{\isachardoublequoteclose}\isanewline
\ \ \ \ \ \ \ \ \isacommand{by}\isamarkupfalse%
\ {\isacharparenleft}{\kern0pt}typecheck{\isacharunderscore}{\kern0pt}cfuncs{\isacharcomma}{\kern0pt}\ simp\ add{\isacharcolon}{\kern0pt}\ cfunc{\isacharunderscore}{\kern0pt}cross{\isacharunderscore}{\kern0pt}prod{\isacharunderscore}{\kern0pt}comp{\isacharunderscore}{\kern0pt}cfunc{\isacharunderscore}{\kern0pt}prod\ yz{\isacharunderscore}{\kern0pt}pair{\isacharparenright}{\kern0pt}\isanewline
\ \ \ \ \ \ \isacommand{also}\isamarkupfalse%
\ \isacommand{have}\isamarkupfalse%
\ {\isachardoublequoteopen}{\isachardot}{\kern0pt}{\isachardot}{\kern0pt}{\isachardot}{\kern0pt}\ {\isacharequal}{\kern0pt}\ distribute{\isacharunderscore}{\kern0pt}left\ Z\ X\ X\ \ {\isasymcirc}\isactrlsub c\ {\isasymlangle}z\ {\isacharcomma}{\kern0pt}\ {\isasymlangle}my{\isadigit{2}}{\isacharcomma}{\kern0pt}my{\isadigit{1}}{\isasymrangle}{\isasymrangle}{\isachardoublequoteclose}\isanewline
\ \ \ \ \ \ \ \ \isacommand{unfolding}\isamarkupfalse%
\ my{\isacharunderscore}{\kern0pt}def\ \isacommand{by}\isamarkupfalse%
\ {\isacharparenleft}{\kern0pt}typecheck{\isacharunderscore}{\kern0pt}cfuncs{\isacharcomma}{\kern0pt}\ simp\ add{\isacharcolon}{\kern0pt}\ id{\isacharunderscore}{\kern0pt}left{\isacharunderscore}{\kern0pt}unit{\isadigit{2}}{\isacharparenright}{\kern0pt}\isanewline
\ \ \ \ \ \ \isacommand{also}\isamarkupfalse%
\ \isacommand{have}\isamarkupfalse%
\ {\isachardoublequoteopen}{\isachardot}{\kern0pt}{\isachardot}{\kern0pt}{\isachardot}{\kern0pt}\ {\isacharequal}{\kern0pt}\ {\isasymlangle}{\isasymlangle}z{\isacharcomma}{\kern0pt}my{\isadigit{2}}{\isasymrangle}{\isacharcomma}{\kern0pt}\ {\isasymlangle}z{\isacharcomma}{\kern0pt}my{\isadigit{1}}{\isasymrangle}{\isasymrangle}{\isachardoublequoteclose}\isanewline
\ \ \ \ \ \ \ \ \isacommand{using}\isamarkupfalse%
\ distribute{\isacharunderscore}{\kern0pt}left{\isacharunderscore}{\kern0pt}ap\ \isacommand{by}\isamarkupfalse%
\ {\isacharparenleft}{\kern0pt}typecheck{\isacharunderscore}{\kern0pt}cfuncs{\isacharcomma}{\kern0pt}\ auto{\isacharparenright}{\kern0pt}\isanewline
\ \ \ \ \ \ \isacommand{then}\isamarkupfalse%
\ \isacommand{show}\isamarkupfalse%
\ {\isacharquery}{\kern0pt}thesis\isanewline
\ \ \ \ \ \ \ \ \isacommand{using}\isamarkupfalse%
\ calculation\ \isacommand{by}\isamarkupfalse%
\ auto\isanewline
\ \ \ \ \isacommand{qed}\isamarkupfalse%
\ \ \ \isanewline
\ \ \ \ \isacommand{then}\isamarkupfalse%
\ \isacommand{have}\isamarkupfalse%
\ {\isachardoublequoteopen}{\isasymlangle}{\isasymlangle}xs{\isacharcomma}{\kern0pt}zs{\isasymrangle}{\isacharcomma}{\kern0pt}{\isasymlangle}xt{\isacharcomma}{\kern0pt}zt{\isasymrangle}{\isasymrangle}\ {\isacharequal}{\kern0pt}\ {\isasymlangle}{\isasymlangle}z{\isacharcomma}{\kern0pt}my{\isadigit{2}}{\isasymrangle}{\isacharcomma}{\kern0pt}{\isasymlangle}z{\isacharcomma}{\kern0pt}my{\isadigit{1}}{\isasymrangle}{\isasymrangle}{\isachardoublequoteclose}\isanewline
\ \ \ \ \ \ \isacommand{using}\isamarkupfalse%
\ zy{\isacharunderscore}{\kern0pt}def\ \isacommand{by}\isamarkupfalse%
\ auto\isanewline
\ \ \ \ \isacommand{then}\isamarkupfalse%
\ \isacommand{have}\isamarkupfalse%
\ {\isachardoublequoteopen}{\isasymlangle}xs{\isacharcomma}{\kern0pt}zs{\isasymrangle}\ {\isacharequal}{\kern0pt}\ {\isasymlangle}z{\isacharcomma}{\kern0pt}my{\isadigit{2}}{\isasymrangle}\ {\isasymand}\ {\isasymlangle}xt{\isacharcomma}{\kern0pt}zt{\isasymrangle}\ {\isacharequal}{\kern0pt}\ {\isasymlangle}z{\isacharcomma}{\kern0pt}\ my{\isadigit{1}}{\isasymrangle}{\isachardoublequoteclose}\isanewline
\ \ \ \ \ \ \isacommand{using}\isamarkupfalse%
\ element{\isacharunderscore}{\kern0pt}pair{\isacharunderscore}{\kern0pt}eq\ \isacommand{by}\isamarkupfalse%
\ {\isacharparenleft}{\kern0pt}typecheck{\isacharunderscore}{\kern0pt}cfuncs{\isacharcomma}{\kern0pt}\ auto{\isacharparenright}{\kern0pt}\isanewline
\ \ \ \ \isacommand{then}\isamarkupfalse%
\ \isacommand{have}\isamarkupfalse%
\ eqs{\isacharcolon}{\kern0pt}\ {\isachardoublequoteopen}xs\ {\isacharequal}{\kern0pt}\ z\ {\isasymand}\ zs\ {\isacharequal}{\kern0pt}\ my{\isadigit{2}}\ {\isasymand}\ xt\ {\isacharequal}{\kern0pt}\ z\ {\isasymand}\ zt\ {\isacharequal}{\kern0pt}\ my{\isadigit{1}}{\isachardoublequoteclose}\isanewline
\ \ \ \ \ \ \isacommand{using}\isamarkupfalse%
\ element{\isacharunderscore}{\kern0pt}pair{\isacharunderscore}{\kern0pt}eq\ \isacommand{by}\isamarkupfalse%
\ {\isacharparenleft}{\kern0pt}typecheck{\isacharunderscore}{\kern0pt}cfuncs{\isacharcomma}{\kern0pt}\ auto{\isacharparenright}{\kern0pt}\isanewline
\isanewline
\ \ \ \ \isacommand{have}\isamarkupfalse%
\ {\isachardoublequoteopen}{\isacharparenleft}{\kern0pt}distribute{\isacharunderscore}{\kern0pt}left\ Z\ X\ X\ \ {\isasymcirc}\isactrlsub c\ {\isacharparenleft}{\kern0pt}id\isactrlsub c\ Z\ {\isasymtimes}\isactrlsub f\ m{\isacharparenright}{\kern0pt}{\isacharparenright}{\kern0pt}\ {\isasymcirc}\isactrlsub c\ {\isasymlangle}z{\isacharcomma}{\kern0pt}y{\isacharprime}{\kern0pt}{\isasymrangle}\ {\isacharequal}{\kern0pt}\ {\isasymlangle}{\isasymlangle}xt{\isacharcomma}{\kern0pt}zt{\isasymrangle}{\isacharcomma}{\kern0pt}\ {\isasymlangle}xs{\isacharcomma}{\kern0pt}zs{\isasymrangle}{\isasymrangle}{\isachardoublequoteclose}\isanewline
\ \ \ \ \isacommand{proof}\isamarkupfalse%
\ {\isacharminus}{\kern0pt}\isanewline
\ \ \ \ \ \ \isacommand{have}\isamarkupfalse%
\ {\isachardoublequoteopen}{\isacharparenleft}{\kern0pt}distribute{\isacharunderscore}{\kern0pt}left\ Z\ X\ X\ \ {\isasymcirc}\isactrlsub c\ {\isacharparenleft}{\kern0pt}id\isactrlsub c\ Z\ {\isasymtimes}\isactrlsub f\ m{\isacharparenright}{\kern0pt}{\isacharparenright}{\kern0pt}\ {\isasymcirc}\isactrlsub c\ {\isasymlangle}z{\isacharcomma}{\kern0pt}y{\isacharprime}{\kern0pt}{\isasymrangle}\ {\isacharequal}{\kern0pt}\ distribute{\isacharunderscore}{\kern0pt}left\ Z\ X\ X\ \ {\isasymcirc}\isactrlsub c\ {\isacharparenleft}{\kern0pt}id\isactrlsub c\ Z\ {\isasymtimes}\isactrlsub f\ m{\isacharparenright}{\kern0pt}\ {\isasymcirc}\isactrlsub c\ {\isasymlangle}z{\isacharcomma}{\kern0pt}y{\isacharprime}{\kern0pt}{\isasymrangle}{\isachardoublequoteclose}\isanewline
\ \ \ \ \ \ \ \ \isacommand{by}\isamarkupfalse%
\ {\isacharparenleft}{\kern0pt}typecheck{\isacharunderscore}{\kern0pt}cfuncs{\isacharcomma}{\kern0pt}\ simp\ add{\isacharcolon}{\kern0pt}\ comp{\isacharunderscore}{\kern0pt}associative{\isadigit{2}}{\isacharparenright}{\kern0pt}\isanewline
\ \ \ \ \ \ \isacommand{also}\isamarkupfalse%
\ \isacommand{have}\isamarkupfalse%
\ {\isachardoublequoteopen}{\isachardot}{\kern0pt}{\isachardot}{\kern0pt}{\isachardot}{\kern0pt}\ {\isacharequal}{\kern0pt}\ distribute{\isacharunderscore}{\kern0pt}left\ Z\ X\ X\ {\isasymcirc}\isactrlsub c\ {\isasymlangle}id\isactrlsub c\ Z\ {\isasymcirc}\isactrlsub c\ z{\isacharcomma}{\kern0pt}\ m\ {\isasymcirc}\isactrlsub c\ y{\isacharprime}{\kern0pt}{\isasymrangle}{\isachardoublequoteclose}\isanewline
\ \ \ \ \ \ \ \ \isacommand{by}\isamarkupfalse%
\ {\isacharparenleft}{\kern0pt}typecheck{\isacharunderscore}{\kern0pt}cfuncs{\isacharcomma}{\kern0pt}\ simp\ add{\isacharcolon}{\kern0pt}\ cfunc{\isacharunderscore}{\kern0pt}cross{\isacharunderscore}{\kern0pt}prod{\isacharunderscore}{\kern0pt}comp{\isacharunderscore}{\kern0pt}cfunc{\isacharunderscore}{\kern0pt}prod{\isacharparenright}{\kern0pt}\isanewline
\ \ \ \ \ \ \isacommand{also}\isamarkupfalse%
\ \isacommand{have}\isamarkupfalse%
\ {\isachardoublequoteopen}{\isachardot}{\kern0pt}{\isachardot}{\kern0pt}{\isachardot}{\kern0pt}\ {\isacharequal}{\kern0pt}\ distribute{\isacharunderscore}{\kern0pt}left\ Z\ X\ X\ {\isasymcirc}\isactrlsub c\ {\isasymlangle}z{\isacharcomma}{\kern0pt}\ {\isasymlangle}my{\isadigit{1}}{\isacharcomma}{\kern0pt}my{\isadigit{2}}{\isasymrangle}{\isasymrangle}{\isachardoublequoteclose}\isanewline
\ \ \ \ \ \ \ \ \isacommand{unfolding}\isamarkupfalse%
\ y{\isacharprime}{\kern0pt}{\isacharunderscore}{\kern0pt}def\ \isacommand{by}\isamarkupfalse%
\ {\isacharparenleft}{\kern0pt}typecheck{\isacharunderscore}{\kern0pt}cfuncs{\isacharcomma}{\kern0pt}\ simp\ add{\isacharcolon}{\kern0pt}\ id{\isacharunderscore}{\kern0pt}left{\isacharunderscore}{\kern0pt}unit{\isadigit{2}}{\isacharparenright}{\kern0pt}\isanewline
\ \ \ \ \ \ \isacommand{also}\isamarkupfalse%
\ \isacommand{have}\isamarkupfalse%
\ {\isachardoublequoteopen}{\isachardot}{\kern0pt}{\isachardot}{\kern0pt}{\isachardot}{\kern0pt}\ {\isacharequal}{\kern0pt}\ {\isasymlangle}{\isasymlangle}z{\isacharcomma}{\kern0pt}my{\isadigit{1}}{\isasymrangle}{\isacharcomma}{\kern0pt}\ {\isasymlangle}z{\isacharcomma}{\kern0pt}my{\isadigit{2}}{\isasymrangle}{\isasymrangle}{\isachardoublequoteclose}\isanewline
\ \ \ \ \ \ \ \ \isacommand{using}\isamarkupfalse%
\ distribute{\isacharunderscore}{\kern0pt}left{\isacharunderscore}{\kern0pt}ap\ \isacommand{by}\isamarkupfalse%
\ {\isacharparenleft}{\kern0pt}typecheck{\isacharunderscore}{\kern0pt}cfuncs{\isacharcomma}{\kern0pt}\ auto{\isacharparenright}{\kern0pt}\isanewline
\ \ \ \ \ \ \isacommand{also}\isamarkupfalse%
\ \isacommand{have}\isamarkupfalse%
\ {\isachardoublequoteopen}{\isachardot}{\kern0pt}{\isachardot}{\kern0pt}{\isachardot}{\kern0pt}\ {\isacharequal}{\kern0pt}\ {\isasymlangle}{\isasymlangle}xt{\isacharcomma}{\kern0pt}zt{\isasymrangle}{\isacharcomma}{\kern0pt}\ {\isasymlangle}xs{\isacharcomma}{\kern0pt}zs{\isasymrangle}{\isasymrangle}{\isachardoublequoteclose}\isanewline
\ \ \ \ \ \ \ \ \isacommand{using}\isamarkupfalse%
\ eqs\ \isacommand{by}\isamarkupfalse%
\ auto\isanewline
\ \ \ \ \ \ \isacommand{then}\isamarkupfalse%
\ \isacommand{show}\isamarkupfalse%
\ {\isacharquery}{\kern0pt}thesis\isanewline
\ \ \ \ \ \ \ \ \isacommand{using}\isamarkupfalse%
\ calculation\ \isacommand{by}\isamarkupfalse%
\ auto\isanewline
\ \ \ \ \isacommand{qed}\isamarkupfalse%
\isanewline
\ \ \ \ \isacommand{then}\isamarkupfalse%
\ \isacommand{show}\isamarkupfalse%
\ {\isachardoublequoteopen}{\isasymlangle}{\isasymlangle}xt{\isacharcomma}{\kern0pt}zt{\isasymrangle}{\isacharcomma}{\kern0pt}{\isasymlangle}xs{\isacharcomma}{\kern0pt}zs{\isasymrangle}{\isasymrangle}\ factorsthru\ {\isacharparenleft}{\kern0pt}distribute{\isacharunderscore}{\kern0pt}left\ Z\ X\ X\ \ {\isasymcirc}\isactrlsub c\ {\isacharparenleft}{\kern0pt}id\isactrlsub c\ Z\ {\isasymtimes}\isactrlsub f\ m{\isacharparenright}{\kern0pt}{\isacharparenright}{\kern0pt}{\isachardoublequoteclose}\isanewline
\ \ \ \ \ \ \isacommand{by}\isamarkupfalse%
\ {\isacharparenleft}{\kern0pt}typecheck{\isacharunderscore}{\kern0pt}cfuncs{\isacharcomma}{\kern0pt}\ unfold\ factors{\isacharunderscore}{\kern0pt}through{\isacharunderscore}{\kern0pt}def{\isadigit{2}}{\isacharcomma}{\kern0pt}\ rule{\isacharunderscore}{\kern0pt}tac\ x{\isacharequal}{\kern0pt}{\isachardoublequoteopen}{\isasymlangle}z{\isacharcomma}{\kern0pt}y{\isacharprime}{\kern0pt}{\isasymrangle}{\isachardoublequoteclose}\ \isakeyword{in}\ exI{\isacharcomma}{\kern0pt}\ typecheck{\isacharunderscore}{\kern0pt}cfuncs{\isacharparenright}{\kern0pt}\isanewline
\ \ \isacommand{qed}\isamarkupfalse%
\isanewline
\isacommand{qed}\isamarkupfalse%
%
\endisatagproof
{\isafoldproof}%
%
\isadelimproof
\isanewline
%
\endisadelimproof
\isanewline
\isacommand{lemma}\isamarkupfalse%
\ left{\isacharunderscore}{\kern0pt}pair{\isacharunderscore}{\kern0pt}transitive{\isacharcolon}{\kern0pt}\isanewline
\ \ \isakeyword{assumes}\ {\isachardoublequoteopen}transitive{\isacharunderscore}{\kern0pt}on\ X\ {\isacharparenleft}{\kern0pt}Y{\isacharcomma}{\kern0pt}\ m{\isacharparenright}{\kern0pt}{\isachardoublequoteclose}\isanewline
\ \ \isakeyword{shows}\ {\isachardoublequoteopen}transitive{\isacharunderscore}{\kern0pt}on\ {\isacharparenleft}{\kern0pt}X\ {\isasymtimes}\isactrlsub c\ Z{\isacharparenright}{\kern0pt}\ {\isacharparenleft}{\kern0pt}Y\ {\isasymtimes}\isactrlsub c\ Z{\isacharcomma}{\kern0pt}\ distribute{\isacharunderscore}{\kern0pt}right\ X\ X\ Z\ {\isasymcirc}\isactrlsub c\ {\isacharparenleft}{\kern0pt}m\ {\isasymtimes}\isactrlsub f\ id\isactrlsub c\ Z{\isacharparenright}{\kern0pt}{\isacharparenright}{\kern0pt}{\isachardoublequoteclose}\isanewline
%
\isadelimproof
%
\endisadelimproof
%
\isatagproof
\isacommand{proof}\isamarkupfalse%
\ {\isacharparenleft}{\kern0pt}unfold\ transitive{\isacharunderscore}{\kern0pt}on{\isacharunderscore}{\kern0pt}def{\isacharcomma}{\kern0pt}\ auto{\isacharparenright}{\kern0pt}\isanewline
\ \ \isacommand{have}\isamarkupfalse%
\ {\isachardoublequoteopen}m\ {\isacharcolon}{\kern0pt}\ Y\ {\isasymrightarrow}\ X\ {\isasymtimes}\isactrlsub c\ X{\isachardoublequoteclose}\ {\isachardoublequoteopen}monomorphism\ m{\isachardoublequoteclose}\isanewline
\ \ \ \ \isacommand{using}\isamarkupfalse%
\ assms\ subobject{\isacharunderscore}{\kern0pt}of{\isacharunderscore}{\kern0pt}def{\isadigit{2}}\ transitive{\isacharunderscore}{\kern0pt}on{\isacharunderscore}{\kern0pt}def\ \isacommand{by}\isamarkupfalse%
\ auto\isanewline
\ \ \isacommand{then}\isamarkupfalse%
\ \isacommand{show}\isamarkupfalse%
\ {\isachardoublequoteopen}{\isacharparenleft}{\kern0pt}Y\ {\isasymtimes}\isactrlsub c\ Z{\isacharcomma}{\kern0pt}\ distribute{\isacharunderscore}{\kern0pt}right\ X\ X\ Z\ {\isasymcirc}\isactrlsub c\ m\ {\isasymtimes}\isactrlsub f\ id\isactrlsub c\ Z{\isacharparenright}{\kern0pt}\ {\isasymsubseteq}\isactrlsub c\ {\isacharparenleft}{\kern0pt}X\ {\isasymtimes}\isactrlsub c\ Z{\isacharparenright}{\kern0pt}\ {\isasymtimes}\isactrlsub c\ X\ {\isasymtimes}\isactrlsub c\ Z{\isachardoublequoteclose}\isanewline
\ \ \ \ \isacommand{by}\isamarkupfalse%
\ {\isacharparenleft}{\kern0pt}simp\ add{\isacharcolon}{\kern0pt}\ left{\isacharunderscore}{\kern0pt}pair{\isacharunderscore}{\kern0pt}subset{\isacharparenright}{\kern0pt}\isanewline
\isacommand{next}\isamarkupfalse%
\isanewline
\ \ \isacommand{have}\isamarkupfalse%
\ m{\isacharunderscore}{\kern0pt}def{\isacharbrackleft}{\kern0pt}type{\isacharunderscore}{\kern0pt}rule{\isacharbrackright}{\kern0pt}{\isacharcolon}{\kern0pt}\ {\isachardoublequoteopen}m\ {\isacharcolon}{\kern0pt}\ Y\ {\isasymrightarrow}\ X\ {\isasymtimes}\isactrlsub c\ X{\isachardoublequoteclose}\ {\isachardoublequoteopen}monomorphism\ m{\isachardoublequoteclose}\isanewline
\ \ \ \ \isacommand{using}\isamarkupfalse%
\ assms\ subobject{\isacharunderscore}{\kern0pt}of{\isacharunderscore}{\kern0pt}def{\isadigit{2}}\ transitive{\isacharunderscore}{\kern0pt}on{\isacharunderscore}{\kern0pt}def\ \isacommand{by}\isamarkupfalse%
\ auto\isanewline
\isanewline
\ \ \isacommand{fix}\isamarkupfalse%
\ s\ t\ u\isanewline
\ \ \isacommand{assume}\isamarkupfalse%
\ s{\isacharunderscore}{\kern0pt}type{\isacharbrackleft}{\kern0pt}type{\isacharunderscore}{\kern0pt}rule{\isacharbrackright}{\kern0pt}{\isacharcolon}{\kern0pt}\ {\isachardoublequoteopen}s\ {\isasymin}\isactrlsub c\ X\ {\isasymtimes}\isactrlsub c\ Z{\isachardoublequoteclose}\isanewline
\ \ \isacommand{assume}\isamarkupfalse%
\ t{\isacharunderscore}{\kern0pt}type{\isacharbrackleft}{\kern0pt}type{\isacharunderscore}{\kern0pt}rule{\isacharbrackright}{\kern0pt}{\isacharcolon}{\kern0pt}\ {\isachardoublequoteopen}t\ {\isasymin}\isactrlsub c\ X\ {\isasymtimes}\isactrlsub c\ Z{\isachardoublequoteclose}\isanewline
\ \ \isacommand{assume}\isamarkupfalse%
\ u{\isacharunderscore}{\kern0pt}type{\isacharbrackleft}{\kern0pt}type{\isacharunderscore}{\kern0pt}rule{\isacharbrackright}{\kern0pt}{\isacharcolon}{\kern0pt}\ {\isachardoublequoteopen}u\ {\isasymin}\isactrlsub c\ X\ {\isasymtimes}\isactrlsub c\ Z{\isachardoublequoteclose}\isanewline
\isanewline
\ \ \isacommand{assume}\isamarkupfalse%
\ st{\isacharunderscore}{\kern0pt}relation{\isacharcolon}{\kern0pt}\ {\isachardoublequoteopen}{\isasymlangle}s{\isacharcomma}{\kern0pt}t{\isasymrangle}\ {\isasymin}\isactrlbsub {\isacharparenleft}{\kern0pt}X\ {\isasymtimes}\isactrlsub c\ Z{\isacharparenright}{\kern0pt}\ {\isasymtimes}\isactrlsub c\ X\ {\isasymtimes}\isactrlsub c\ Z\isactrlesub \ {\isacharparenleft}{\kern0pt}Y\ {\isasymtimes}\isactrlsub c\ Z{\isacharcomma}{\kern0pt}\ distribute{\isacharunderscore}{\kern0pt}right\ X\ X\ Z\ {\isasymcirc}\isactrlsub c\ m\ {\isasymtimes}\isactrlsub f\ id\isactrlsub c\ Z{\isacharparenright}{\kern0pt}{\isachardoublequoteclose}\isanewline
\ \ \isacommand{then}\isamarkupfalse%
\ \isacommand{obtain}\isamarkupfalse%
\ h\ \isakeyword{where}\ h{\isacharunderscore}{\kern0pt}type{\isacharbrackleft}{\kern0pt}type{\isacharunderscore}{\kern0pt}rule{\isacharbrackright}{\kern0pt}{\isacharcolon}{\kern0pt}\ {\isachardoublequoteopen}h\ {\isasymin}\isactrlsub c\ Y\ {\isasymtimes}\isactrlsub c\ Z{\isachardoublequoteclose}\ \isakeyword{and}\ h{\isacharunderscore}{\kern0pt}def{\isacharcolon}{\kern0pt}\ {\isachardoublequoteopen}{\isacharparenleft}{\kern0pt}distribute{\isacharunderscore}{\kern0pt}right\ X\ X\ Z\ {\isasymcirc}\isactrlsub c\ m\ {\isasymtimes}\isactrlsub f\ id\isactrlsub c\ Z{\isacharparenright}{\kern0pt}\ {\isasymcirc}\isactrlsub c\ h\ {\isacharequal}{\kern0pt}\ {\isasymlangle}s{\isacharcomma}{\kern0pt}t{\isasymrangle}{\isachardoublequoteclose}\isanewline
\ \ \ \ \isacommand{by}\isamarkupfalse%
\ {\isacharparenleft}{\kern0pt}typecheck{\isacharunderscore}{\kern0pt}cfuncs{\isacharcomma}{\kern0pt}\ unfold\ relative{\isacharunderscore}{\kern0pt}member{\isacharunderscore}{\kern0pt}def{\isadigit{2}}\ factors{\isacharunderscore}{\kern0pt}through{\isacharunderscore}{\kern0pt}def{\isadigit{2}}{\isacharcomma}{\kern0pt}\ auto{\isacharparenright}{\kern0pt}\isanewline
\ \ \isacommand{then}\isamarkupfalse%
\ \isacommand{obtain}\isamarkupfalse%
\ hy\ hz\ \isakeyword{where}\ h{\isacharunderscore}{\kern0pt}part{\isacharunderscore}{\kern0pt}types{\isacharbrackleft}{\kern0pt}type{\isacharunderscore}{\kern0pt}rule{\isacharbrackright}{\kern0pt}{\isacharcolon}{\kern0pt}\ {\isachardoublequoteopen}hy\ {\isasymin}\isactrlsub c\ Y{\isachardoublequoteclose}\ {\isachardoublequoteopen}hz\ {\isasymin}\isactrlsub c\ Z{\isachardoublequoteclose}\ \isakeyword{and}\ h{\isacharunderscore}{\kern0pt}decomp{\isacharcolon}{\kern0pt}\ {\isachardoublequoteopen}h\ {\isacharequal}{\kern0pt}\ {\isasymlangle}hy{\isacharcomma}{\kern0pt}\ hz{\isasymrangle}{\isachardoublequoteclose}\isanewline
\ \ \ \ \isacommand{using}\isamarkupfalse%
\ cart{\isacharunderscore}{\kern0pt}prod{\isacharunderscore}{\kern0pt}decomp\ \isacommand{by}\isamarkupfalse%
\ blast\isanewline
\ \ \isacommand{then}\isamarkupfalse%
\ \isacommand{obtain}\isamarkupfalse%
\ mhy{\isadigit{1}}\ mhy{\isadigit{2}}\ \isakeyword{where}\ mhy{\isacharunderscore}{\kern0pt}types{\isacharbrackleft}{\kern0pt}type{\isacharunderscore}{\kern0pt}rule{\isacharbrackright}{\kern0pt}{\isacharcolon}{\kern0pt}\ {\isachardoublequoteopen}mhy{\isadigit{1}}\ {\isasymin}\isactrlsub c\ X{\isachardoublequoteclose}\ {\isachardoublequoteopen}mhy{\isadigit{2}}\ {\isasymin}\isactrlsub c\ X{\isachardoublequoteclose}\ \isakeyword{and}\ mhy{\isacharunderscore}{\kern0pt}decomp{\isacharcolon}{\kern0pt}\ \ {\isachardoublequoteopen}m\ {\isasymcirc}\isactrlsub c\ hy\ {\isacharequal}{\kern0pt}\ {\isasymlangle}mhy{\isadigit{1}}{\isacharcomma}{\kern0pt}\ mhy{\isadigit{2}}{\isasymrangle}{\isachardoublequoteclose}\isanewline
\ \ \ \ \isacommand{using}\isamarkupfalse%
\ cart{\isacharunderscore}{\kern0pt}prod{\isacharunderscore}{\kern0pt}decomp\ \isacommand{by}\isamarkupfalse%
\ {\isacharparenleft}{\kern0pt}typecheck{\isacharunderscore}{\kern0pt}cfuncs{\isacharcomma}{\kern0pt}\ blast{\isacharparenright}{\kern0pt}\isanewline
\isanewline
\ \ \isacommand{have}\isamarkupfalse%
\ {\isachardoublequoteopen}{\isasymlangle}s{\isacharcomma}{\kern0pt}t{\isasymrangle}\ {\isacharequal}{\kern0pt}\ {\isasymlangle}{\isasymlangle}mhy{\isadigit{1}}{\isacharcomma}{\kern0pt}\ hz{\isasymrangle}{\isacharcomma}{\kern0pt}\ {\isasymlangle}mhy{\isadigit{2}}{\isacharcomma}{\kern0pt}\ hz{\isasymrangle}{\isasymrangle}{\isachardoublequoteclose}\isanewline
\ \ \isacommand{proof}\isamarkupfalse%
\ {\isacharminus}{\kern0pt}\isanewline
\ \ \ \ \isacommand{have}\isamarkupfalse%
\ {\isachardoublequoteopen}{\isasymlangle}s{\isacharcomma}{\kern0pt}t{\isasymrangle}\ {\isacharequal}{\kern0pt}\ {\isacharparenleft}{\kern0pt}distribute{\isacharunderscore}{\kern0pt}right\ X\ X\ Z\ {\isasymcirc}\isactrlsub c\ m\ {\isasymtimes}\isactrlsub f\ id\isactrlsub c\ Z{\isacharparenright}{\kern0pt}\ {\isasymcirc}\isactrlsub c\ {\isasymlangle}hy{\isacharcomma}{\kern0pt}\ hz{\isasymrangle}{\isachardoublequoteclose}\isanewline
\ \ \ \ \ \ \isacommand{using}\isamarkupfalse%
\ h{\isacharunderscore}{\kern0pt}decomp\ h{\isacharunderscore}{\kern0pt}def\ \isacommand{by}\isamarkupfalse%
\ auto\isanewline
\ \ \ \ \isacommand{also}\isamarkupfalse%
\ \isacommand{have}\isamarkupfalse%
\ {\isachardoublequoteopen}{\isachardot}{\kern0pt}{\isachardot}{\kern0pt}{\isachardot}{\kern0pt}\ {\isacharequal}{\kern0pt}\ distribute{\isacharunderscore}{\kern0pt}right\ X\ X\ Z\ {\isasymcirc}\isactrlsub c\ {\isacharparenleft}{\kern0pt}m\ {\isasymtimes}\isactrlsub f\ id\isactrlsub c\ Z{\isacharparenright}{\kern0pt}\ {\isasymcirc}\isactrlsub c\ {\isasymlangle}hy{\isacharcomma}{\kern0pt}\ hz{\isasymrangle}{\isachardoublequoteclose}\isanewline
\ \ \ \ \ \ \isacommand{by}\isamarkupfalse%
\ {\isacharparenleft}{\kern0pt}typecheck{\isacharunderscore}{\kern0pt}cfuncs{\isacharcomma}{\kern0pt}\ auto\ simp\ add{\isacharcolon}{\kern0pt}\ comp{\isacharunderscore}{\kern0pt}associative{\isadigit{2}}{\isacharparenright}{\kern0pt}\isanewline
\ \ \ \ \isacommand{also}\isamarkupfalse%
\ \isacommand{have}\isamarkupfalse%
\ {\isachardoublequoteopen}{\isachardot}{\kern0pt}{\isachardot}{\kern0pt}{\isachardot}{\kern0pt}\ {\isacharequal}{\kern0pt}\ distribute{\isacharunderscore}{\kern0pt}right\ X\ X\ Z\ {\isasymcirc}\isactrlsub c\ {\isasymlangle}m\ {\isasymcirc}\isactrlsub c\ hy{\isacharcomma}{\kern0pt}\ hz{\isasymrangle}{\isachardoublequoteclose}\isanewline
\ \ \ \ \ \ \isacommand{by}\isamarkupfalse%
\ {\isacharparenleft}{\kern0pt}typecheck{\isacharunderscore}{\kern0pt}cfuncs{\isacharcomma}{\kern0pt}\ simp\ add{\isacharcolon}{\kern0pt}\ cfunc{\isacharunderscore}{\kern0pt}cross{\isacharunderscore}{\kern0pt}prod{\isacharunderscore}{\kern0pt}comp{\isacharunderscore}{\kern0pt}cfunc{\isacharunderscore}{\kern0pt}prod\ id{\isacharunderscore}{\kern0pt}left{\isacharunderscore}{\kern0pt}unit{\isadigit{2}}{\isacharparenright}{\kern0pt}\isanewline
\ \ \ \ \isacommand{also}\isamarkupfalse%
\ \isacommand{have}\isamarkupfalse%
\ {\isachardoublequoteopen}{\isachardot}{\kern0pt}{\isachardot}{\kern0pt}{\isachardot}{\kern0pt}\ {\isacharequal}{\kern0pt}\ {\isasymlangle}{\isasymlangle}mhy{\isadigit{1}}{\isacharcomma}{\kern0pt}\ hz{\isasymrangle}{\isacharcomma}{\kern0pt}\ {\isasymlangle}mhy{\isadigit{2}}{\isacharcomma}{\kern0pt}\ hz{\isasymrangle}{\isasymrangle}{\isachardoublequoteclose}\isanewline
\ \ \ \ \ \ \isacommand{unfolding}\isamarkupfalse%
\ mhy{\isacharunderscore}{\kern0pt}decomp\ \isacommand{by}\isamarkupfalse%
\ {\isacharparenleft}{\kern0pt}typecheck{\isacharunderscore}{\kern0pt}cfuncs{\isacharcomma}{\kern0pt}\ simp\ add{\isacharcolon}{\kern0pt}\ distribute{\isacharunderscore}{\kern0pt}right{\isacharunderscore}{\kern0pt}ap{\isacharparenright}{\kern0pt}\isanewline
\ \ \ \ \isacommand{then}\isamarkupfalse%
\ \isacommand{show}\isamarkupfalse%
\ {\isacharquery}{\kern0pt}thesis\isanewline
\ \ \ \ \ \ \isacommand{using}\isamarkupfalse%
\ calculation\ \isacommand{by}\isamarkupfalse%
\ auto\isanewline
\ \ \isacommand{qed}\isamarkupfalse%
\isanewline
\ \ \isacommand{then}\isamarkupfalse%
\ \isacommand{have}\isamarkupfalse%
\ s{\isacharunderscore}{\kern0pt}def{\isacharcolon}{\kern0pt}\ {\isachardoublequoteopen}s\ {\isacharequal}{\kern0pt}\ {\isasymlangle}mhy{\isadigit{1}}{\isacharcomma}{\kern0pt}\ hz{\isasymrangle}{\isachardoublequoteclose}\ \isakeyword{and}\ t{\isacharunderscore}{\kern0pt}def{\isacharcolon}{\kern0pt}\ {\isachardoublequoteopen}t\ {\isacharequal}{\kern0pt}\ {\isasymlangle}mhy{\isadigit{2}}{\isacharcomma}{\kern0pt}\ hz{\isasymrangle}{\isachardoublequoteclose}\isanewline
\ \ \ \ \isacommand{using}\isamarkupfalse%
\ cart{\isacharunderscore}{\kern0pt}prod{\isacharunderscore}{\kern0pt}eq{\isadigit{2}}\ \isacommand{by}\isamarkupfalse%
\ {\isacharparenleft}{\kern0pt}typecheck{\isacharunderscore}{\kern0pt}cfuncs{\isacharcomma}{\kern0pt}\ auto{\isacharcomma}{\kern0pt}\ presburger{\isacharparenright}{\kern0pt}\isanewline
\isanewline
\ \ \isacommand{assume}\isamarkupfalse%
\ tu{\isacharunderscore}{\kern0pt}relation{\isacharcolon}{\kern0pt}\ {\isachardoublequoteopen}{\isasymlangle}t{\isacharcomma}{\kern0pt}u{\isasymrangle}\ {\isasymin}\isactrlbsub {\isacharparenleft}{\kern0pt}X\ {\isasymtimes}\isactrlsub c\ Z{\isacharparenright}{\kern0pt}\ {\isasymtimes}\isactrlsub c\ X\ {\isasymtimes}\isactrlsub c\ Z\isactrlesub \ {\isacharparenleft}{\kern0pt}Y\ {\isasymtimes}\isactrlsub c\ Z{\isacharcomma}{\kern0pt}\ distribute{\isacharunderscore}{\kern0pt}right\ X\ X\ Z\ {\isasymcirc}\isactrlsub c\ m\ {\isasymtimes}\isactrlsub f\ id\isactrlsub c\ Z{\isacharparenright}{\kern0pt}{\isachardoublequoteclose}\isanewline
\ \ \isacommand{then}\isamarkupfalse%
\ \isacommand{obtain}\isamarkupfalse%
\ g\ \isakeyword{where}\ g{\isacharunderscore}{\kern0pt}type{\isacharbrackleft}{\kern0pt}type{\isacharunderscore}{\kern0pt}rule{\isacharbrackright}{\kern0pt}{\isacharcolon}{\kern0pt}\ {\isachardoublequoteopen}g\ {\isasymin}\isactrlsub c\ Y\ {\isasymtimes}\isactrlsub c\ Z{\isachardoublequoteclose}\ \isakeyword{and}\ g{\isacharunderscore}{\kern0pt}def{\isacharcolon}{\kern0pt}\ {\isachardoublequoteopen}{\isacharparenleft}{\kern0pt}distribute{\isacharunderscore}{\kern0pt}right\ X\ X\ Z\ {\isasymcirc}\isactrlsub c\ m\ {\isasymtimes}\isactrlsub f\ id\isactrlsub c\ Z{\isacharparenright}{\kern0pt}\ {\isasymcirc}\isactrlsub c\ g\ {\isacharequal}{\kern0pt}\ {\isasymlangle}t{\isacharcomma}{\kern0pt}u{\isasymrangle}{\isachardoublequoteclose}\isanewline
\ \ \ \ \isacommand{by}\isamarkupfalse%
\ {\isacharparenleft}{\kern0pt}typecheck{\isacharunderscore}{\kern0pt}cfuncs{\isacharcomma}{\kern0pt}\ unfold\ relative{\isacharunderscore}{\kern0pt}member{\isacharunderscore}{\kern0pt}def{\isadigit{2}}\ factors{\isacharunderscore}{\kern0pt}through{\isacharunderscore}{\kern0pt}def{\isadigit{2}}{\isacharcomma}{\kern0pt}\ auto{\isacharparenright}{\kern0pt}\isanewline
\ \ \isacommand{then}\isamarkupfalse%
\ \isacommand{obtain}\isamarkupfalse%
\ gy\ gz\ \isakeyword{where}\ g{\isacharunderscore}{\kern0pt}part{\isacharunderscore}{\kern0pt}types{\isacharbrackleft}{\kern0pt}type{\isacharunderscore}{\kern0pt}rule{\isacharbrackright}{\kern0pt}{\isacharcolon}{\kern0pt}\ {\isachardoublequoteopen}gy\ {\isasymin}\isactrlsub c\ Y{\isachardoublequoteclose}\ {\isachardoublequoteopen}gz\ {\isasymin}\isactrlsub c\ Z{\isachardoublequoteclose}\ \isakeyword{and}\ g{\isacharunderscore}{\kern0pt}decomp{\isacharcolon}{\kern0pt}\ {\isachardoublequoteopen}g\ {\isacharequal}{\kern0pt}\ {\isasymlangle}gy{\isacharcomma}{\kern0pt}\ gz{\isasymrangle}{\isachardoublequoteclose}\isanewline
\ \ \ \ \isacommand{using}\isamarkupfalse%
\ cart{\isacharunderscore}{\kern0pt}prod{\isacharunderscore}{\kern0pt}decomp\ \isacommand{by}\isamarkupfalse%
\ blast\isanewline
\ \ \isacommand{then}\isamarkupfalse%
\ \isacommand{obtain}\isamarkupfalse%
\ mgy{\isadigit{1}}\ mgy{\isadigit{2}}\ \isakeyword{where}\ mgy{\isacharunderscore}{\kern0pt}types{\isacharbrackleft}{\kern0pt}type{\isacharunderscore}{\kern0pt}rule{\isacharbrackright}{\kern0pt}{\isacharcolon}{\kern0pt}\ {\isachardoublequoteopen}mgy{\isadigit{1}}\ {\isasymin}\isactrlsub c\ X{\isachardoublequoteclose}\ {\isachardoublequoteopen}mgy{\isadigit{2}}\ {\isasymin}\isactrlsub c\ X{\isachardoublequoteclose}\ \isakeyword{and}\ mgy{\isacharunderscore}{\kern0pt}decomp{\isacharcolon}{\kern0pt}\ \ {\isachardoublequoteopen}m\ {\isasymcirc}\isactrlsub c\ gy\ {\isacharequal}{\kern0pt}\ {\isasymlangle}mgy{\isadigit{1}}{\isacharcomma}{\kern0pt}\ mgy{\isadigit{2}}{\isasymrangle}{\isachardoublequoteclose}\isanewline
\ \ \ \ \isacommand{using}\isamarkupfalse%
\ cart{\isacharunderscore}{\kern0pt}prod{\isacharunderscore}{\kern0pt}decomp\ \isacommand{by}\isamarkupfalse%
\ {\isacharparenleft}{\kern0pt}typecheck{\isacharunderscore}{\kern0pt}cfuncs{\isacharcomma}{\kern0pt}\ blast{\isacharparenright}{\kern0pt}\isanewline
\isanewline
\ \ \isacommand{have}\isamarkupfalse%
\ {\isachardoublequoteopen}{\isasymlangle}t{\isacharcomma}{\kern0pt}u{\isasymrangle}\ {\isacharequal}{\kern0pt}\ {\isasymlangle}{\isasymlangle}mgy{\isadigit{1}}{\isacharcomma}{\kern0pt}\ gz{\isasymrangle}{\isacharcomma}{\kern0pt}\ {\isasymlangle}mgy{\isadigit{2}}{\isacharcomma}{\kern0pt}\ gz{\isasymrangle}{\isasymrangle}{\isachardoublequoteclose}\isanewline
\ \ \isacommand{proof}\isamarkupfalse%
\ {\isacharminus}{\kern0pt}\isanewline
\ \ \ \ \isacommand{have}\isamarkupfalse%
\ {\isachardoublequoteopen}{\isasymlangle}t{\isacharcomma}{\kern0pt}u{\isasymrangle}\ {\isacharequal}{\kern0pt}\ {\isacharparenleft}{\kern0pt}distribute{\isacharunderscore}{\kern0pt}right\ X\ X\ Z\ {\isasymcirc}\isactrlsub c\ m\ {\isasymtimes}\isactrlsub f\ id\isactrlsub c\ Z{\isacharparenright}{\kern0pt}\ {\isasymcirc}\isactrlsub c\ {\isasymlangle}gy{\isacharcomma}{\kern0pt}\ gz{\isasymrangle}{\isachardoublequoteclose}\isanewline
\ \ \ \ \ \ \isacommand{using}\isamarkupfalse%
\ g{\isacharunderscore}{\kern0pt}decomp\ g{\isacharunderscore}{\kern0pt}def\ \isacommand{by}\isamarkupfalse%
\ auto\isanewline
\ \ \ \ \isacommand{also}\isamarkupfalse%
\ \isacommand{have}\isamarkupfalse%
\ {\isachardoublequoteopen}{\isachardot}{\kern0pt}{\isachardot}{\kern0pt}{\isachardot}{\kern0pt}\ {\isacharequal}{\kern0pt}\ distribute{\isacharunderscore}{\kern0pt}right\ X\ X\ Z\ {\isasymcirc}\isactrlsub c\ {\isacharparenleft}{\kern0pt}m\ {\isasymtimes}\isactrlsub f\ id\isactrlsub c\ Z{\isacharparenright}{\kern0pt}\ {\isasymcirc}\isactrlsub c\ {\isasymlangle}gy{\isacharcomma}{\kern0pt}\ gz{\isasymrangle}{\isachardoublequoteclose}\isanewline
\ \ \ \ \ \ \isacommand{by}\isamarkupfalse%
\ {\isacharparenleft}{\kern0pt}typecheck{\isacharunderscore}{\kern0pt}cfuncs{\isacharcomma}{\kern0pt}\ auto\ simp\ add{\isacharcolon}{\kern0pt}\ comp{\isacharunderscore}{\kern0pt}associative{\isadigit{2}}{\isacharparenright}{\kern0pt}\isanewline
\ \ \ \ \isacommand{also}\isamarkupfalse%
\ \isacommand{have}\isamarkupfalse%
\ {\isachardoublequoteopen}{\isachardot}{\kern0pt}{\isachardot}{\kern0pt}{\isachardot}{\kern0pt}\ {\isacharequal}{\kern0pt}\ distribute{\isacharunderscore}{\kern0pt}right\ X\ X\ Z\ {\isasymcirc}\isactrlsub c\ {\isasymlangle}m\ {\isasymcirc}\isactrlsub c\ gy{\isacharcomma}{\kern0pt}\ gz{\isasymrangle}{\isachardoublequoteclose}\isanewline
\ \ \ \ \ \ \isacommand{by}\isamarkupfalse%
\ {\isacharparenleft}{\kern0pt}typecheck{\isacharunderscore}{\kern0pt}cfuncs{\isacharcomma}{\kern0pt}\ simp\ add{\isacharcolon}{\kern0pt}\ cfunc{\isacharunderscore}{\kern0pt}cross{\isacharunderscore}{\kern0pt}prod{\isacharunderscore}{\kern0pt}comp{\isacharunderscore}{\kern0pt}cfunc{\isacharunderscore}{\kern0pt}prod\ id{\isacharunderscore}{\kern0pt}left{\isacharunderscore}{\kern0pt}unit{\isadigit{2}}{\isacharparenright}{\kern0pt}\isanewline
\ \ \ \ \isacommand{also}\isamarkupfalse%
\ \isacommand{have}\isamarkupfalse%
\ {\isachardoublequoteopen}{\isachardot}{\kern0pt}{\isachardot}{\kern0pt}{\isachardot}{\kern0pt}\ {\isacharequal}{\kern0pt}\ {\isasymlangle}{\isasymlangle}mgy{\isadigit{1}}{\isacharcomma}{\kern0pt}\ gz{\isasymrangle}{\isacharcomma}{\kern0pt}\ {\isasymlangle}mgy{\isadigit{2}}{\isacharcomma}{\kern0pt}\ gz{\isasymrangle}{\isasymrangle}{\isachardoublequoteclose}\isanewline
\ \ \ \ \ \ \isacommand{unfolding}\isamarkupfalse%
\ mgy{\isacharunderscore}{\kern0pt}decomp\ \isacommand{by}\isamarkupfalse%
\ {\isacharparenleft}{\kern0pt}typecheck{\isacharunderscore}{\kern0pt}cfuncs{\isacharcomma}{\kern0pt}\ simp\ add{\isacharcolon}{\kern0pt}\ distribute{\isacharunderscore}{\kern0pt}right{\isacharunderscore}{\kern0pt}ap{\isacharparenright}{\kern0pt}\isanewline
\ \ \ \ \isacommand{then}\isamarkupfalse%
\ \isacommand{show}\isamarkupfalse%
\ {\isacharquery}{\kern0pt}thesis\isanewline
\ \ \ \ \ \ \isacommand{using}\isamarkupfalse%
\ calculation\ \isacommand{by}\isamarkupfalse%
\ auto\isanewline
\ \ \isacommand{qed}\isamarkupfalse%
\isanewline
\ \ \isacommand{then}\isamarkupfalse%
\ \isacommand{have}\isamarkupfalse%
\ t{\isacharunderscore}{\kern0pt}def{\isadigit{2}}{\isacharcolon}{\kern0pt}\ {\isachardoublequoteopen}t\ {\isacharequal}{\kern0pt}\ {\isasymlangle}mgy{\isadigit{1}}{\isacharcomma}{\kern0pt}\ gz{\isasymrangle}{\isachardoublequoteclose}\ \isakeyword{and}\ u{\isacharunderscore}{\kern0pt}def{\isacharcolon}{\kern0pt}\ {\isachardoublequoteopen}u\ {\isacharequal}{\kern0pt}\ {\isasymlangle}mgy{\isadigit{2}}{\isacharcomma}{\kern0pt}\ gz{\isasymrangle}{\isachardoublequoteclose}\isanewline
\ \ \ \ \isacommand{using}\isamarkupfalse%
\ cart{\isacharunderscore}{\kern0pt}prod{\isacharunderscore}{\kern0pt}eq{\isadigit{2}}\ \isacommand{by}\isamarkupfalse%
\ {\isacharparenleft}{\kern0pt}typecheck{\isacharunderscore}{\kern0pt}cfuncs{\isacharcomma}{\kern0pt}\ auto{\isacharcomma}{\kern0pt}\ presburger{\isacharparenright}{\kern0pt}\isanewline
\isanewline
\ \ \isacommand{have}\isamarkupfalse%
\ mhy{\isadigit{2}}{\isacharunderscore}{\kern0pt}eq{\isacharunderscore}{\kern0pt}mgy{\isadigit{1}}{\isacharcolon}{\kern0pt}\ {\isachardoublequoteopen}mhy{\isadigit{2}}\ {\isacharequal}{\kern0pt}\ mgy{\isadigit{1}}{\isachardoublequoteclose}\isanewline
\ \ \ \ \isacommand{using}\isamarkupfalse%
\ t{\isacharunderscore}{\kern0pt}def{\isadigit{2}}\ t{\isacharunderscore}{\kern0pt}def\ cart{\isacharunderscore}{\kern0pt}prod{\isacharunderscore}{\kern0pt}eq{\isadigit{2}}\ \isacommand{by}\isamarkupfalse%
\ {\isacharparenleft}{\kern0pt}auto{\isacharcomma}{\kern0pt}\ typecheck{\isacharunderscore}{\kern0pt}cfuncs{\isacharparenright}{\kern0pt}\isanewline
\ \ \isacommand{have}\isamarkupfalse%
\ gy{\isacharunderscore}{\kern0pt}eq{\isacharunderscore}{\kern0pt}gz{\isacharcolon}{\kern0pt}\ {\isachardoublequoteopen}hz\ {\isacharequal}{\kern0pt}\ gz{\isachardoublequoteclose}\isanewline
\ \ \ \ \isacommand{using}\isamarkupfalse%
\ t{\isacharunderscore}{\kern0pt}def{\isadigit{2}}\ t{\isacharunderscore}{\kern0pt}def\ cart{\isacharunderscore}{\kern0pt}prod{\isacharunderscore}{\kern0pt}eq{\isadigit{2}}\ \isacommand{by}\isamarkupfalse%
\ {\isacharparenleft}{\kern0pt}auto{\isacharcomma}{\kern0pt}\ typecheck{\isacharunderscore}{\kern0pt}cfuncs{\isacharparenright}{\kern0pt}\isanewline
\isanewline
\ \ \isacommand{have}\isamarkupfalse%
\ mhy{\isacharunderscore}{\kern0pt}in{\isacharunderscore}{\kern0pt}Y{\isacharcolon}{\kern0pt}\ {\isachardoublequoteopen}{\isasymlangle}mhy{\isadigit{1}}{\isacharcomma}{\kern0pt}\ mhy{\isadigit{2}}{\isasymrangle}\ {\isasymin}\isactrlbsub X\ {\isasymtimes}\isactrlsub c\ X\isactrlesub \ {\isacharparenleft}{\kern0pt}Y{\isacharcomma}{\kern0pt}\ m{\isacharparenright}{\kern0pt}{\isachardoublequoteclose}\isanewline
\ \ \ \ \isacommand{using}\isamarkupfalse%
\ m{\isacharunderscore}{\kern0pt}def\ h{\isacharunderscore}{\kern0pt}part{\isacharunderscore}{\kern0pt}types\ mhy{\isacharunderscore}{\kern0pt}decomp\isanewline
\ \ \ \ \isacommand{by}\isamarkupfalse%
\ {\isacharparenleft}{\kern0pt}typecheck{\isacharunderscore}{\kern0pt}cfuncs{\isacharcomma}{\kern0pt}\ unfold\ relative{\isacharunderscore}{\kern0pt}member{\isacharunderscore}{\kern0pt}def{\isadigit{2}}\ factors{\isacharunderscore}{\kern0pt}through{\isacharunderscore}{\kern0pt}def{\isadigit{2}}{\isacharcomma}{\kern0pt}\ auto{\isacharparenright}{\kern0pt}\isanewline
\ \ \isacommand{have}\isamarkupfalse%
\ mgy{\isacharunderscore}{\kern0pt}in{\isacharunderscore}{\kern0pt}Y{\isacharcolon}{\kern0pt}\ {\isachardoublequoteopen}{\isasymlangle}mhy{\isadigit{2}}{\isacharcomma}{\kern0pt}\ mgy{\isadigit{2}}{\isasymrangle}\ {\isasymin}\isactrlbsub X\ {\isasymtimes}\isactrlsub c\ X\isactrlesub \ {\isacharparenleft}{\kern0pt}Y{\isacharcomma}{\kern0pt}\ m{\isacharparenright}{\kern0pt}{\isachardoublequoteclose}\isanewline
\ \ \ \ \isacommand{using}\isamarkupfalse%
\ m{\isacharunderscore}{\kern0pt}def\ g{\isacharunderscore}{\kern0pt}part{\isacharunderscore}{\kern0pt}types\ mgy{\isacharunderscore}{\kern0pt}decomp\ mhy{\isadigit{2}}{\isacharunderscore}{\kern0pt}eq{\isacharunderscore}{\kern0pt}mgy{\isadigit{1}}\isanewline
\ \ \ \ \isacommand{by}\isamarkupfalse%
\ {\isacharparenleft}{\kern0pt}typecheck{\isacharunderscore}{\kern0pt}cfuncs{\isacharcomma}{\kern0pt}\ unfold\ relative{\isacharunderscore}{\kern0pt}member{\isacharunderscore}{\kern0pt}def{\isadigit{2}}\ factors{\isacharunderscore}{\kern0pt}through{\isacharunderscore}{\kern0pt}def{\isadigit{2}}{\isacharcomma}{\kern0pt}\ auto{\isacharparenright}{\kern0pt}\isanewline
\isanewline
\ \ \isacommand{have}\isamarkupfalse%
\ {\isachardoublequoteopen}{\isasymlangle}mhy{\isadigit{1}}{\isacharcomma}{\kern0pt}\ mgy{\isadigit{2}}{\isasymrangle}\ {\isasymin}\isactrlbsub X\ {\isasymtimes}\isactrlsub c\ X\isactrlesub \ {\isacharparenleft}{\kern0pt}Y{\isacharcomma}{\kern0pt}\ m{\isacharparenright}{\kern0pt}{\isachardoublequoteclose}\isanewline
\ \ \ \ \isacommand{using}\isamarkupfalse%
\ assms\ mhy{\isacharunderscore}{\kern0pt}in{\isacharunderscore}{\kern0pt}Y\ mgy{\isacharunderscore}{\kern0pt}in{\isacharunderscore}{\kern0pt}Y\ mgy{\isacharunderscore}{\kern0pt}types\ mhy{\isadigit{2}}{\isacharunderscore}{\kern0pt}eq{\isacharunderscore}{\kern0pt}mgy{\isadigit{1}}\ \isacommand{unfolding}\isamarkupfalse%
\ transitive{\isacharunderscore}{\kern0pt}on{\isacharunderscore}{\kern0pt}def\isanewline
\ \ \ \ \isacommand{by}\isamarkupfalse%
\ {\isacharparenleft}{\kern0pt}typecheck{\isacharunderscore}{\kern0pt}cfuncs{\isacharcomma}{\kern0pt}\ blast{\isacharparenright}{\kern0pt}\isanewline
\ \ \isacommand{then}\isamarkupfalse%
\ \isacommand{obtain}\isamarkupfalse%
\ y\ \isakeyword{where}\ y{\isacharunderscore}{\kern0pt}type{\isacharbrackleft}{\kern0pt}type{\isacharunderscore}{\kern0pt}rule{\isacharbrackright}{\kern0pt}{\isacharcolon}{\kern0pt}\ {\isachardoublequoteopen}y\ {\isasymin}\isactrlsub c\ Y{\isachardoublequoteclose}\ \isakeyword{and}\ y{\isacharunderscore}{\kern0pt}def{\isacharcolon}{\kern0pt}\ {\isachardoublequoteopen}m\ {\isasymcirc}\isactrlsub c\ y\ {\isacharequal}{\kern0pt}\ {\isasymlangle}mhy{\isadigit{1}}{\isacharcomma}{\kern0pt}\ mgy{\isadigit{2}}{\isasymrangle}{\isachardoublequoteclose}\isanewline
\ \ \ \ \isacommand{by}\isamarkupfalse%
\ {\isacharparenleft}{\kern0pt}typecheck{\isacharunderscore}{\kern0pt}cfuncs{\isacharcomma}{\kern0pt}\ unfold\ relative{\isacharunderscore}{\kern0pt}member{\isacharunderscore}{\kern0pt}def{\isadigit{2}}\ factors{\isacharunderscore}{\kern0pt}through{\isacharunderscore}{\kern0pt}def{\isadigit{2}}{\isacharcomma}{\kern0pt}\ auto{\isacharparenright}{\kern0pt}\isanewline
\isanewline
\ \ \isacommand{show}\isamarkupfalse%
\ {\isachardoublequoteopen}\ {\isasymlangle}s{\isacharcomma}{\kern0pt}u{\isasymrangle}\ {\isasymin}\isactrlbsub {\isacharparenleft}{\kern0pt}X\ {\isasymtimes}\isactrlsub c\ Z{\isacharparenright}{\kern0pt}\ {\isasymtimes}\isactrlsub c\ X\ {\isasymtimes}\isactrlsub c\ Z\isactrlesub \ {\isacharparenleft}{\kern0pt}Y\ {\isasymtimes}\isactrlsub c\ Z{\isacharcomma}{\kern0pt}\ distribute{\isacharunderscore}{\kern0pt}right\ X\ X\ Z\ {\isasymcirc}\isactrlsub c\ {\isacharparenleft}{\kern0pt}m\ {\isasymtimes}\isactrlsub f\ id\isactrlsub c\ Z{\isacharparenright}{\kern0pt}{\isacharparenright}{\kern0pt}{\isachardoublequoteclose}\ \isanewline
\ \ \isacommand{proof}\isamarkupfalse%
\ {\isacharparenleft}{\kern0pt}typecheck{\isacharunderscore}{\kern0pt}cfuncs{\isacharcomma}{\kern0pt}\ unfold\ relative{\isacharunderscore}{\kern0pt}member{\isacharunderscore}{\kern0pt}def{\isadigit{2}}\ factors{\isacharunderscore}{\kern0pt}through{\isacharunderscore}{\kern0pt}def{\isadigit{2}}{\isacharcomma}{\kern0pt}\ auto{\isacharparenright}{\kern0pt}\isanewline
\ \ \ \ \isacommand{show}\isamarkupfalse%
\ {\isachardoublequoteopen}monomorphism\ {\isacharparenleft}{\kern0pt}distribute{\isacharunderscore}{\kern0pt}right\ X\ X\ Z\ {\isasymcirc}\isactrlsub c\ m\ {\isasymtimes}\isactrlsub f\ id\isactrlsub c\ Z{\isacharparenright}{\kern0pt}{\isachardoublequoteclose}\isanewline
\ \ \ \ \ \ \isacommand{using}\isamarkupfalse%
\ relative{\isacharunderscore}{\kern0pt}member{\isacharunderscore}{\kern0pt}def{\isadigit{2}}\ st{\isacharunderscore}{\kern0pt}relation\ \isacommand{by}\isamarkupfalse%
\ blast\isanewline
\isanewline
\ \ \ \ \isacommand{show}\isamarkupfalse%
\ {\isachardoublequoteopen}{\isasymexists}h{\isachardot}{\kern0pt}\ h\ {\isasymin}\isactrlsub c\ Y\ {\isasymtimes}\isactrlsub c\ Z\ {\isasymand}\ {\isacharparenleft}{\kern0pt}distribute{\isacharunderscore}{\kern0pt}right\ X\ X\ Z\ {\isasymcirc}\isactrlsub c\ m\ {\isasymtimes}\isactrlsub f\ id\isactrlsub c\ Z{\isacharparenright}{\kern0pt}\ {\isasymcirc}\isactrlsub c\ h\ {\isacharequal}{\kern0pt}\ {\isasymlangle}s{\isacharcomma}{\kern0pt}u{\isasymrangle}{\isachardoublequoteclose}\isanewline
\ \ \ \ \ \ \isacommand{unfolding}\isamarkupfalse%
\ s{\isacharunderscore}{\kern0pt}def\ u{\isacharunderscore}{\kern0pt}def\ gy{\isacharunderscore}{\kern0pt}eq{\isacharunderscore}{\kern0pt}gz\isanewline
\ \ \ \ \isacommand{proof}\isamarkupfalse%
\ {\isacharparenleft}{\kern0pt}rule{\isacharunderscore}{\kern0pt}tac\ x{\isacharequal}{\kern0pt}{\isachardoublequoteopen}{\isasymlangle}y{\isacharcomma}{\kern0pt}gz{\isasymrangle}{\isachardoublequoteclose}\ \isakeyword{in}\ exI{\isacharcomma}{\kern0pt}\ auto{\isacharcomma}{\kern0pt}\ typecheck{\isacharunderscore}{\kern0pt}cfuncs{\isacharparenright}{\kern0pt}\isanewline
\ \ \ \ \ \ \isacommand{have}\isamarkupfalse%
\ {\isachardoublequoteopen}{\isacharparenleft}{\kern0pt}distribute{\isacharunderscore}{\kern0pt}right\ X\ X\ Z\ {\isasymcirc}\isactrlsub c\ m\ {\isasymtimes}\isactrlsub f\ id\isactrlsub c\ Z{\isacharparenright}{\kern0pt}\ {\isasymcirc}\isactrlsub c\ {\isasymlangle}y{\isacharcomma}{\kern0pt}gz{\isasymrangle}\ {\isacharequal}{\kern0pt}\ distribute{\isacharunderscore}{\kern0pt}right\ X\ X\ Z\ {\isasymcirc}\isactrlsub c\ {\isacharparenleft}{\kern0pt}m\ {\isasymtimes}\isactrlsub f\ id\isactrlsub c\ Z{\isacharparenright}{\kern0pt}\ {\isasymcirc}\isactrlsub c\ {\isasymlangle}y{\isacharcomma}{\kern0pt}gz{\isasymrangle}{\isachardoublequoteclose}\isanewline
\ \ \ \ \ \ \ \ \isacommand{by}\isamarkupfalse%
\ {\isacharparenleft}{\kern0pt}typecheck{\isacharunderscore}{\kern0pt}cfuncs{\isacharcomma}{\kern0pt}\ auto\ simp\ add{\isacharcolon}{\kern0pt}\ comp{\isacharunderscore}{\kern0pt}associative{\isadigit{2}}{\isacharparenright}{\kern0pt}\isanewline
\ \ \ \ \ \ \isacommand{also}\isamarkupfalse%
\ \isacommand{have}\isamarkupfalse%
\ {\isachardoublequoteopen}{\isachardot}{\kern0pt}{\isachardot}{\kern0pt}{\isachardot}{\kern0pt}\ {\isacharequal}{\kern0pt}\ distribute{\isacharunderscore}{\kern0pt}right\ X\ X\ Z\ {\isasymcirc}\isactrlsub c\ {\isasymlangle}m\ {\isasymcirc}\isactrlsub c\ y{\isacharcomma}{\kern0pt}\ gz{\isasymrangle}{\isachardoublequoteclose}\isanewline
\ \ \ \ \ \ \ \ \isacommand{by}\isamarkupfalse%
\ {\isacharparenleft}{\kern0pt}typecheck{\isacharunderscore}{\kern0pt}cfuncs{\isacharcomma}{\kern0pt}\ simp\ add{\isacharcolon}{\kern0pt}\ cfunc{\isacharunderscore}{\kern0pt}cross{\isacharunderscore}{\kern0pt}prod{\isacharunderscore}{\kern0pt}comp{\isacharunderscore}{\kern0pt}cfunc{\isacharunderscore}{\kern0pt}prod\ id{\isacharunderscore}{\kern0pt}left{\isacharunderscore}{\kern0pt}unit{\isadigit{2}}{\isacharparenright}{\kern0pt}\isanewline
\ \ \ \ \ \ \isacommand{also}\isamarkupfalse%
\ \isacommand{have}\isamarkupfalse%
\ {\isachardoublequoteopen}{\isachardot}{\kern0pt}{\isachardot}{\kern0pt}{\isachardot}{\kern0pt}\ {\isacharequal}{\kern0pt}\ {\isasymlangle}{\isasymlangle}mhy{\isadigit{1}}{\isacharcomma}{\kern0pt}gz{\isasymrangle}{\isacharcomma}{\kern0pt}{\isasymlangle}mgy{\isadigit{2}}{\isacharcomma}{\kern0pt}gz{\isasymrangle}{\isasymrangle}{\isachardoublequoteclose}\isanewline
\ \ \ \ \ \ \ \ \isacommand{unfolding}\isamarkupfalse%
\ y{\isacharunderscore}{\kern0pt}def\ \isacommand{by}\isamarkupfalse%
\ {\isacharparenleft}{\kern0pt}typecheck{\isacharunderscore}{\kern0pt}cfuncs{\isacharcomma}{\kern0pt}\ simp\ add{\isacharcolon}{\kern0pt}\ distribute{\isacharunderscore}{\kern0pt}right{\isacharunderscore}{\kern0pt}ap{\isacharparenright}{\kern0pt}\isanewline
\ \ \ \ \ \ \isacommand{then}\isamarkupfalse%
\ \isacommand{show}\isamarkupfalse%
\ {\isachardoublequoteopen}{\isacharparenleft}{\kern0pt}distribute{\isacharunderscore}{\kern0pt}right\ X\ X\ Z\ {\isasymcirc}\isactrlsub c\ m\ {\isasymtimes}\isactrlsub f\ id\isactrlsub c\ Z{\isacharparenright}{\kern0pt}\ {\isasymcirc}\isactrlsub c\ {\isasymlangle}y{\isacharcomma}{\kern0pt}gz{\isasymrangle}\ {\isacharequal}{\kern0pt}\ {\isasymlangle}{\isasymlangle}mhy{\isadigit{1}}{\isacharcomma}{\kern0pt}gz{\isasymrangle}{\isacharcomma}{\kern0pt}{\isasymlangle}mgy{\isadigit{2}}{\isacharcomma}{\kern0pt}gz{\isasymrangle}{\isasymrangle}{\isachardoublequoteclose}\isanewline
\ \ \ \ \ \ \ \ \isacommand{using}\isamarkupfalse%
\ calculation\ \isacommand{by}\isamarkupfalse%
\ auto\isanewline
\ \ \ \ \isacommand{qed}\isamarkupfalse%
\isanewline
\ \ \isacommand{qed}\isamarkupfalse%
\isanewline
\isacommand{qed}\isamarkupfalse%
%
\endisatagproof
{\isafoldproof}%
%
\isadelimproof
\isanewline
%
\endisadelimproof
\isanewline
\isacommand{lemma}\isamarkupfalse%
\ right{\isacharunderscore}{\kern0pt}pair{\isacharunderscore}{\kern0pt}transitive{\isacharcolon}{\kern0pt}\isanewline
\ \ \isakeyword{assumes}\ {\isachardoublequoteopen}transitive{\isacharunderscore}{\kern0pt}on\ X\ {\isacharparenleft}{\kern0pt}Y{\isacharcomma}{\kern0pt}\ m{\isacharparenright}{\kern0pt}{\isachardoublequoteclose}\isanewline
\ \ \isakeyword{shows}\ {\isachardoublequoteopen}transitive{\isacharunderscore}{\kern0pt}on\ {\isacharparenleft}{\kern0pt}Z\ {\isasymtimes}\isactrlsub c\ X{\isacharparenright}{\kern0pt}\ {\isacharparenleft}{\kern0pt}Z\ {\isasymtimes}\isactrlsub c\ Y{\isacharcomma}{\kern0pt}\ distribute{\isacharunderscore}{\kern0pt}left\ Z\ X\ X\ {\isasymcirc}\isactrlsub c\ {\isacharparenleft}{\kern0pt}id\isactrlsub c\ Z\ {\isasymtimes}\isactrlsub f\ m{\isacharparenright}{\kern0pt}{\isacharparenright}{\kern0pt}{\isachardoublequoteclose}\isanewline
%
\isadelimproof
%
\endisadelimproof
%
\isatagproof
\isacommand{proof}\isamarkupfalse%
\ {\isacharparenleft}{\kern0pt}unfold\ transitive{\isacharunderscore}{\kern0pt}on{\isacharunderscore}{\kern0pt}def{\isacharcomma}{\kern0pt}\ auto{\isacharparenright}{\kern0pt}\isanewline
\ \ \isacommand{have}\isamarkupfalse%
\ {\isachardoublequoteopen}m\ {\isacharcolon}{\kern0pt}\ Y\ {\isasymrightarrow}\ X\ {\isasymtimes}\isactrlsub c\ X{\isachardoublequoteclose}\ {\isachardoublequoteopen}monomorphism\ m{\isachardoublequoteclose}\isanewline
\ \ \ \ \isacommand{using}\isamarkupfalse%
\ assms\ subobject{\isacharunderscore}{\kern0pt}of{\isacharunderscore}{\kern0pt}def{\isadigit{2}}\ transitive{\isacharunderscore}{\kern0pt}on{\isacharunderscore}{\kern0pt}def\ \isacommand{by}\isamarkupfalse%
\ auto\isanewline
\ \ \isacommand{then}\isamarkupfalse%
\ \isacommand{show}\isamarkupfalse%
\ {\isachardoublequoteopen}{\isacharparenleft}{\kern0pt}Z\ {\isasymtimes}\isactrlsub c\ Y{\isacharcomma}{\kern0pt}\ distribute{\isacharunderscore}{\kern0pt}left\ Z\ X\ X\ {\isasymcirc}\isactrlsub c\ id\isactrlsub c\ Z\ {\isasymtimes}\isactrlsub f\ m{\isacharparenright}{\kern0pt}\ {\isasymsubseteq}\isactrlsub c\ {\isacharparenleft}{\kern0pt}Z\ {\isasymtimes}\isactrlsub c\ X{\isacharparenright}{\kern0pt}\ {\isasymtimes}\isactrlsub c\ Z\ {\isasymtimes}\isactrlsub c\ X{\isachardoublequoteclose}\isanewline
\ \ \ \ \isacommand{by}\isamarkupfalse%
\ {\isacharparenleft}{\kern0pt}simp\ add{\isacharcolon}{\kern0pt}\ right{\isacharunderscore}{\kern0pt}pair{\isacharunderscore}{\kern0pt}subset{\isacharparenright}{\kern0pt}\isanewline
\isacommand{next}\isamarkupfalse%
\isanewline
\ \ \isacommand{have}\isamarkupfalse%
\ m{\isacharunderscore}{\kern0pt}def{\isacharbrackleft}{\kern0pt}type{\isacharunderscore}{\kern0pt}rule{\isacharbrackright}{\kern0pt}{\isacharcolon}{\kern0pt}\ {\isachardoublequoteopen}m\ {\isacharcolon}{\kern0pt}\ Y\ {\isasymrightarrow}\ X\ {\isasymtimes}\isactrlsub c\ X{\isachardoublequoteclose}\ {\isachardoublequoteopen}monomorphism\ m{\isachardoublequoteclose}\isanewline
\ \ \ \ \isacommand{using}\isamarkupfalse%
\ assms\ subobject{\isacharunderscore}{\kern0pt}of{\isacharunderscore}{\kern0pt}def{\isadigit{2}}\ transitive{\isacharunderscore}{\kern0pt}on{\isacharunderscore}{\kern0pt}def\ \isacommand{by}\isamarkupfalse%
\ auto\isanewline
\isanewline
\ \ \isacommand{fix}\isamarkupfalse%
\ s\ t\ u\isanewline
\ \ \isacommand{assume}\isamarkupfalse%
\ s{\isacharunderscore}{\kern0pt}type{\isacharbrackleft}{\kern0pt}type{\isacharunderscore}{\kern0pt}rule{\isacharbrackright}{\kern0pt}{\isacharcolon}{\kern0pt}\ {\isachardoublequoteopen}s\ {\isasymin}\isactrlsub c\ Z\ {\isasymtimes}\isactrlsub c\ X{\isachardoublequoteclose}\isanewline
\ \ \isacommand{assume}\isamarkupfalse%
\ t{\isacharunderscore}{\kern0pt}type{\isacharbrackleft}{\kern0pt}type{\isacharunderscore}{\kern0pt}rule{\isacharbrackright}{\kern0pt}{\isacharcolon}{\kern0pt}\ {\isachardoublequoteopen}t\ {\isasymin}\isactrlsub c\ Z\ {\isasymtimes}\isactrlsub c\ X{\isachardoublequoteclose}\isanewline
\ \ \isacommand{assume}\isamarkupfalse%
\ u{\isacharunderscore}{\kern0pt}type{\isacharbrackleft}{\kern0pt}type{\isacharunderscore}{\kern0pt}rule{\isacharbrackright}{\kern0pt}{\isacharcolon}{\kern0pt}\ {\isachardoublequoteopen}u\ {\isasymin}\isactrlsub c\ Z\ {\isasymtimes}\isactrlsub c\ X{\isachardoublequoteclose}\isanewline
\ \ \isacommand{assume}\isamarkupfalse%
\ st{\isacharunderscore}{\kern0pt}relation{\isacharcolon}{\kern0pt}\ {\isachardoublequoteopen}{\isasymlangle}s{\isacharcomma}{\kern0pt}t{\isasymrangle}\ {\isasymin}\isactrlbsub {\isacharparenleft}{\kern0pt}Z\ {\isasymtimes}\isactrlsub c\ X{\isacharparenright}{\kern0pt}\ {\isasymtimes}\isactrlsub c\ Z\ {\isasymtimes}\isactrlsub c\ X\isactrlesub \ {\isacharparenleft}{\kern0pt}Z\ {\isasymtimes}\isactrlsub c\ Y{\isacharcomma}{\kern0pt}\ distribute{\isacharunderscore}{\kern0pt}left\ Z\ X\ X\ {\isasymcirc}\isactrlsub c\ id\isactrlsub c\ Z\ {\isasymtimes}\isactrlsub f\ m{\isacharparenright}{\kern0pt}{\isachardoublequoteclose}\isanewline
\ \ \isacommand{then}\isamarkupfalse%
\ \isacommand{obtain}\isamarkupfalse%
\ h\ \isakeyword{where}\ h{\isacharunderscore}{\kern0pt}type{\isacharbrackleft}{\kern0pt}type{\isacharunderscore}{\kern0pt}rule{\isacharbrackright}{\kern0pt}{\isacharcolon}{\kern0pt}\ {\isachardoublequoteopen}h\ {\isasymin}\isactrlsub c\ Z\ {\isasymtimes}\isactrlsub c\ Y{\isachardoublequoteclose}\ \isakeyword{and}\ h{\isacharunderscore}{\kern0pt}def{\isacharcolon}{\kern0pt}\ {\isachardoublequoteopen}{\isacharparenleft}{\kern0pt}distribute{\isacharunderscore}{\kern0pt}left\ Z\ X\ X\ \ {\isasymcirc}\isactrlsub c\ id\isactrlsub c\ Z\ {\isasymtimes}\isactrlsub f\ m{\isacharparenright}{\kern0pt}\ {\isasymcirc}\isactrlsub c\ h\ {\isacharequal}{\kern0pt}\ {\isasymlangle}s{\isacharcomma}{\kern0pt}t{\isasymrangle}{\isachardoublequoteclose}\isanewline
\ \ \ \ \isacommand{by}\isamarkupfalse%
\ {\isacharparenleft}{\kern0pt}typecheck{\isacharunderscore}{\kern0pt}cfuncs{\isacharcomma}{\kern0pt}\ unfold\ relative{\isacharunderscore}{\kern0pt}member{\isacharunderscore}{\kern0pt}def{\isadigit{2}}\ factors{\isacharunderscore}{\kern0pt}through{\isacharunderscore}{\kern0pt}def{\isadigit{2}}{\isacharcomma}{\kern0pt}\ auto{\isacharparenright}{\kern0pt}\isanewline
\ \ \isacommand{then}\isamarkupfalse%
\ \isacommand{obtain}\isamarkupfalse%
\ hy\ hz\ \isakeyword{where}\ h{\isacharunderscore}{\kern0pt}part{\isacharunderscore}{\kern0pt}types{\isacharbrackleft}{\kern0pt}type{\isacharunderscore}{\kern0pt}rule{\isacharbrackright}{\kern0pt}{\isacharcolon}{\kern0pt}\ {\isachardoublequoteopen}hy\ {\isasymin}\isactrlsub c\ Y{\isachardoublequoteclose}\ {\isachardoublequoteopen}hz\ {\isasymin}\isactrlsub c\ Z{\isachardoublequoteclose}\ \isakeyword{and}\ h{\isacharunderscore}{\kern0pt}decomp{\isacharcolon}{\kern0pt}\ {\isachardoublequoteopen}h\ {\isacharequal}{\kern0pt}\ {\isasymlangle}hz{\isacharcomma}{\kern0pt}\ hy{\isasymrangle}{\isachardoublequoteclose}\isanewline
\ \ \ \ \isacommand{using}\isamarkupfalse%
\ cart{\isacharunderscore}{\kern0pt}prod{\isacharunderscore}{\kern0pt}decomp\ \isacommand{by}\isamarkupfalse%
\ blast\isanewline
\ \ \isacommand{then}\isamarkupfalse%
\ \isacommand{obtain}\isamarkupfalse%
\ mhy{\isadigit{1}}\ mhy{\isadigit{2}}\ \isakeyword{where}\ mhy{\isacharunderscore}{\kern0pt}types{\isacharbrackleft}{\kern0pt}type{\isacharunderscore}{\kern0pt}rule{\isacharbrackright}{\kern0pt}{\isacharcolon}{\kern0pt}\ {\isachardoublequoteopen}mhy{\isadigit{1}}\ {\isasymin}\isactrlsub c\ X{\isachardoublequoteclose}\ {\isachardoublequoteopen}mhy{\isadigit{2}}\ {\isasymin}\isactrlsub c\ X{\isachardoublequoteclose}\ \isakeyword{and}\ mhy{\isacharunderscore}{\kern0pt}decomp{\isacharcolon}{\kern0pt}\ \ {\isachardoublequoteopen}m\ {\isasymcirc}\isactrlsub c\ hy\ {\isacharequal}{\kern0pt}\ {\isasymlangle}mhy{\isadigit{1}}{\isacharcomma}{\kern0pt}\ mhy{\isadigit{2}}{\isasymrangle}{\isachardoublequoteclose}\isanewline
\ \ \ \ \isacommand{using}\isamarkupfalse%
\ cart{\isacharunderscore}{\kern0pt}prod{\isacharunderscore}{\kern0pt}decomp\ \isacommand{by}\isamarkupfalse%
\ {\isacharparenleft}{\kern0pt}typecheck{\isacharunderscore}{\kern0pt}cfuncs{\isacharcomma}{\kern0pt}\ blast{\isacharparenright}{\kern0pt}\isanewline
\isanewline
\ \ \isacommand{have}\isamarkupfalse%
\ {\isachardoublequoteopen}{\isasymlangle}s{\isacharcomma}{\kern0pt}t{\isasymrangle}\ {\isacharequal}{\kern0pt}\ {\isasymlangle}{\isasymlangle}hz{\isacharcomma}{\kern0pt}\ mhy{\isadigit{1}}{\isasymrangle}{\isacharcomma}{\kern0pt}\ {\isasymlangle}hz{\isacharcomma}{\kern0pt}\ mhy{\isadigit{2}}{\isasymrangle}{\isasymrangle}{\isachardoublequoteclose}\isanewline
\ \ \isacommand{proof}\isamarkupfalse%
\ {\isacharminus}{\kern0pt}\isanewline
\ \ \ \ \isacommand{have}\isamarkupfalse%
\ {\isachardoublequoteopen}{\isasymlangle}s{\isacharcomma}{\kern0pt}t{\isasymrangle}\ {\isacharequal}{\kern0pt}\ {\isacharparenleft}{\kern0pt}distribute{\isacharunderscore}{\kern0pt}left\ Z\ X\ X\ \ {\isasymcirc}\isactrlsub c\ id\isactrlsub c\ Z\ {\isasymtimes}\isactrlsub f\ m{\isacharparenright}{\kern0pt}\ {\isasymcirc}\isactrlsub c\ {\isasymlangle}hz{\isacharcomma}{\kern0pt}\ hy{\isasymrangle}{\isachardoublequoteclose}\isanewline
\ \ \ \ \ \ \isacommand{using}\isamarkupfalse%
\ h{\isacharunderscore}{\kern0pt}decomp\ h{\isacharunderscore}{\kern0pt}def\ \isacommand{by}\isamarkupfalse%
\ auto\isanewline
\ \ \ \ \isacommand{also}\isamarkupfalse%
\ \isacommand{have}\isamarkupfalse%
\ {\isachardoublequoteopen}{\isachardot}{\kern0pt}{\isachardot}{\kern0pt}{\isachardot}{\kern0pt}\ {\isacharequal}{\kern0pt}\ distribute{\isacharunderscore}{\kern0pt}left\ Z\ X\ X\ \ {\isasymcirc}\isactrlsub c\ {\isacharparenleft}{\kern0pt}id\isactrlsub c\ Z\ {\isasymtimes}\isactrlsub f\ m{\isacharparenright}{\kern0pt}\ {\isasymcirc}\isactrlsub c\ {\isasymlangle}hz{\isacharcomma}{\kern0pt}\ hy{\isasymrangle}{\isachardoublequoteclose}\isanewline
\ \ \ \ \ \ \isacommand{by}\isamarkupfalse%
\ {\isacharparenleft}{\kern0pt}typecheck{\isacharunderscore}{\kern0pt}cfuncs{\isacharcomma}{\kern0pt}\ auto\ simp\ add{\isacharcolon}{\kern0pt}\ comp{\isacharunderscore}{\kern0pt}associative{\isadigit{2}}{\isacharparenright}{\kern0pt}\isanewline
\ \ \ \ \isacommand{also}\isamarkupfalse%
\ \isacommand{have}\isamarkupfalse%
\ {\isachardoublequoteopen}{\isachardot}{\kern0pt}{\isachardot}{\kern0pt}{\isachardot}{\kern0pt}\ {\isacharequal}{\kern0pt}\ distribute{\isacharunderscore}{\kern0pt}left\ Z\ X\ X\ \ {\isasymcirc}\isactrlsub c\ {\isasymlangle}\ hz{\isacharcomma}{\kern0pt}\ m\ {\isasymcirc}\isactrlsub c\ hy{\isasymrangle}{\isachardoublequoteclose}\isanewline
\ \ \ \ \ \ \isacommand{by}\isamarkupfalse%
\ {\isacharparenleft}{\kern0pt}typecheck{\isacharunderscore}{\kern0pt}cfuncs{\isacharcomma}{\kern0pt}\ simp\ add{\isacharcolon}{\kern0pt}\ cfunc{\isacharunderscore}{\kern0pt}cross{\isacharunderscore}{\kern0pt}prod{\isacharunderscore}{\kern0pt}comp{\isacharunderscore}{\kern0pt}cfunc{\isacharunderscore}{\kern0pt}prod\ id{\isacharunderscore}{\kern0pt}left{\isacharunderscore}{\kern0pt}unit{\isadigit{2}}{\isacharparenright}{\kern0pt}\isanewline
\ \ \ \ \isacommand{also}\isamarkupfalse%
\ \isacommand{have}\isamarkupfalse%
\ {\isachardoublequoteopen}{\isachardot}{\kern0pt}{\isachardot}{\kern0pt}{\isachardot}{\kern0pt}\ {\isacharequal}{\kern0pt}\ {\isasymlangle}{\isasymlangle}hz{\isacharcomma}{\kern0pt}\ mhy{\isadigit{1}}{\isasymrangle}{\isacharcomma}{\kern0pt}\ {\isasymlangle}hz{\isacharcomma}{\kern0pt}\ mhy{\isadigit{2}}{\isasymrangle}{\isasymrangle}{\isachardoublequoteclose}\isanewline
\ \ \ \ \ \ \isacommand{unfolding}\isamarkupfalse%
\ mhy{\isacharunderscore}{\kern0pt}decomp\ \isacommand{by}\isamarkupfalse%
\ {\isacharparenleft}{\kern0pt}typecheck{\isacharunderscore}{\kern0pt}cfuncs{\isacharcomma}{\kern0pt}\ simp\ add{\isacharcolon}{\kern0pt}\ distribute{\isacharunderscore}{\kern0pt}left{\isacharunderscore}{\kern0pt}ap{\isacharparenright}{\kern0pt}\isanewline
\ \ \ \ \isacommand{then}\isamarkupfalse%
\ \isacommand{show}\isamarkupfalse%
\ {\isacharquery}{\kern0pt}thesis\isanewline
\ \ \ \ \ \ \isacommand{using}\isamarkupfalse%
\ calculation\ \isacommand{by}\isamarkupfalse%
\ auto\isanewline
\ \ \isacommand{qed}\isamarkupfalse%
\isanewline
\ \ \isacommand{then}\isamarkupfalse%
\ \isacommand{have}\isamarkupfalse%
\ s{\isacharunderscore}{\kern0pt}def{\isacharcolon}{\kern0pt}\ {\isachardoublequoteopen}s\ {\isacharequal}{\kern0pt}\ {\isasymlangle}hz{\isacharcomma}{\kern0pt}\ mhy{\isadigit{1}}{\isasymrangle}{\isachardoublequoteclose}\ \isakeyword{and}\ t{\isacharunderscore}{\kern0pt}def{\isacharcolon}{\kern0pt}\ {\isachardoublequoteopen}t\ {\isacharequal}{\kern0pt}\ {\isasymlangle}hz{\isacharcomma}{\kern0pt}\ mhy{\isadigit{2}}{\isasymrangle}{\isachardoublequoteclose}\isanewline
\ \ \ \ \isacommand{using}\isamarkupfalse%
\ cart{\isacharunderscore}{\kern0pt}prod{\isacharunderscore}{\kern0pt}eq{\isadigit{2}}\ \isacommand{by}\isamarkupfalse%
\ {\isacharparenleft}{\kern0pt}typecheck{\isacharunderscore}{\kern0pt}cfuncs{\isacharcomma}{\kern0pt}\ auto{\isacharcomma}{\kern0pt}\ presburger{\isacharparenright}{\kern0pt}\isanewline
\isanewline
\ \ \isacommand{assume}\isamarkupfalse%
\ tu{\isacharunderscore}{\kern0pt}relation{\isacharcolon}{\kern0pt}\ {\isachardoublequoteopen}{\isasymlangle}t{\isacharcomma}{\kern0pt}u{\isasymrangle}\ {\isasymin}\isactrlbsub {\isacharparenleft}{\kern0pt}Z\ {\isasymtimes}\isactrlsub c\ X{\isacharparenright}{\kern0pt}\ {\isasymtimes}\isactrlsub c\isanewline
\ \ \ \ \ \ \ \ \ \ \ \ \ \ \ Z\ {\isasymtimes}\isactrlsub c\ X\isactrlesub \ {\isacharparenleft}{\kern0pt}Z\ {\isasymtimes}\isactrlsub c\ Y{\isacharcomma}{\kern0pt}\ distribute{\isacharunderscore}{\kern0pt}left\ Z\ X\ X\ {\isasymcirc}\isactrlsub c\ id\isactrlsub c\ Z\ {\isasymtimes}\isactrlsub f\ m{\isacharparenright}{\kern0pt}{\isachardoublequoteclose}\isanewline
\ \ \isacommand{then}\isamarkupfalse%
\ \isacommand{obtain}\isamarkupfalse%
\ g\ \isakeyword{where}\ g{\isacharunderscore}{\kern0pt}type{\isacharbrackleft}{\kern0pt}type{\isacharunderscore}{\kern0pt}rule{\isacharbrackright}{\kern0pt}{\isacharcolon}{\kern0pt}\ {\isachardoublequoteopen}g\ {\isasymin}\isactrlsub c\ Z\ {\isasymtimes}\isactrlsub c\ Y{\isachardoublequoteclose}\ \isakeyword{and}\ g{\isacharunderscore}{\kern0pt}def{\isacharcolon}{\kern0pt}\ {\isachardoublequoteopen}{\isacharparenleft}{\kern0pt}distribute{\isacharunderscore}{\kern0pt}left\ Z\ X\ X\ \ {\isasymcirc}\isactrlsub c\ id\isactrlsub c\ Z\ {\isasymtimes}\isactrlsub f\ m{\isacharparenright}{\kern0pt}\ {\isasymcirc}\isactrlsub c\ g\ {\isacharequal}{\kern0pt}\ {\isasymlangle}t{\isacharcomma}{\kern0pt}u{\isasymrangle}{\isachardoublequoteclose}\isanewline
\ \ \ \ \isacommand{by}\isamarkupfalse%
\ {\isacharparenleft}{\kern0pt}typecheck{\isacharunderscore}{\kern0pt}cfuncs{\isacharcomma}{\kern0pt}\ unfold\ relative{\isacharunderscore}{\kern0pt}member{\isacharunderscore}{\kern0pt}def{\isadigit{2}}\ factors{\isacharunderscore}{\kern0pt}through{\isacharunderscore}{\kern0pt}def{\isadigit{2}}{\isacharcomma}{\kern0pt}\ auto{\isacharparenright}{\kern0pt}\isanewline
\ \ \isacommand{then}\isamarkupfalse%
\ \isacommand{obtain}\isamarkupfalse%
\ gy\ gz\ \isakeyword{where}\ g{\isacharunderscore}{\kern0pt}part{\isacharunderscore}{\kern0pt}types{\isacharbrackleft}{\kern0pt}type{\isacharunderscore}{\kern0pt}rule{\isacharbrackright}{\kern0pt}{\isacharcolon}{\kern0pt}\ {\isachardoublequoteopen}gy\ {\isasymin}\isactrlsub c\ Y{\isachardoublequoteclose}\ {\isachardoublequoteopen}gz\ {\isasymin}\isactrlsub c\ Z{\isachardoublequoteclose}\ \isakeyword{and}\ g{\isacharunderscore}{\kern0pt}decomp{\isacharcolon}{\kern0pt}\ {\isachardoublequoteopen}g\ {\isacharequal}{\kern0pt}\ {\isasymlangle}gz{\isacharcomma}{\kern0pt}\ gy{\isasymrangle}{\isachardoublequoteclose}\isanewline
\ \ \ \ \isacommand{using}\isamarkupfalse%
\ cart{\isacharunderscore}{\kern0pt}prod{\isacharunderscore}{\kern0pt}decomp\ \isacommand{by}\isamarkupfalse%
\ blast\isanewline
\ \ \isacommand{then}\isamarkupfalse%
\ \isacommand{obtain}\isamarkupfalse%
\ mgy{\isadigit{1}}\ mgy{\isadigit{2}}\ \isakeyword{where}\ mgy{\isacharunderscore}{\kern0pt}types{\isacharbrackleft}{\kern0pt}type{\isacharunderscore}{\kern0pt}rule{\isacharbrackright}{\kern0pt}{\isacharcolon}{\kern0pt}\ {\isachardoublequoteopen}mgy{\isadigit{1}}\ {\isasymin}\isactrlsub c\ X{\isachardoublequoteclose}\ {\isachardoublequoteopen}mgy{\isadigit{2}}\ {\isasymin}\isactrlsub c\ X{\isachardoublequoteclose}\ \isakeyword{and}\ mgy{\isacharunderscore}{\kern0pt}decomp{\isacharcolon}{\kern0pt}\ \ {\isachardoublequoteopen}m\ {\isasymcirc}\isactrlsub c\ gy\ {\isacharequal}{\kern0pt}\ {\isasymlangle}mgy{\isadigit{2}}{\isacharcomma}{\kern0pt}\ mgy{\isadigit{1}}{\isasymrangle}{\isachardoublequoteclose}\isanewline
\ \ \ \ \isacommand{using}\isamarkupfalse%
\ cart{\isacharunderscore}{\kern0pt}prod{\isacharunderscore}{\kern0pt}decomp\ \isacommand{by}\isamarkupfalse%
\ {\isacharparenleft}{\kern0pt}typecheck{\isacharunderscore}{\kern0pt}cfuncs{\isacharcomma}{\kern0pt}\ blast{\isacharparenright}{\kern0pt}\isanewline
\isanewline
\ \ \isacommand{have}\isamarkupfalse%
\ {\isachardoublequoteopen}{\isasymlangle}t{\isacharcomma}{\kern0pt}u{\isasymrangle}\ {\isacharequal}{\kern0pt}\ {\isasymlangle}{\isasymlangle}gz{\isacharcomma}{\kern0pt}\ mgy{\isadigit{2}}{\isasymrangle}{\isacharcomma}{\kern0pt}\ {\isasymlangle}gz{\isacharcomma}{\kern0pt}\ mgy{\isadigit{1}}{\isasymrangle}{\isasymrangle}{\isachardoublequoteclose}\isanewline
\ \ \isacommand{proof}\isamarkupfalse%
\ {\isacharminus}{\kern0pt}\isanewline
\ \ \ \ \isacommand{have}\isamarkupfalse%
\ {\isachardoublequoteopen}{\isasymlangle}t{\isacharcomma}{\kern0pt}u{\isasymrangle}\ {\isacharequal}{\kern0pt}\ {\isacharparenleft}{\kern0pt}distribute{\isacharunderscore}{\kern0pt}left\ Z\ X\ X\ \ {\isasymcirc}\isactrlsub c\ id\isactrlsub c\ Z\ {\isasymtimes}\isactrlsub f\ m{\isacharparenright}{\kern0pt}\ {\isasymcirc}\isactrlsub c\ {\isasymlangle}gz{\isacharcomma}{\kern0pt}\ gy{\isasymrangle}{\isachardoublequoteclose}\isanewline
\ \ \ \ \ \ \isacommand{using}\isamarkupfalse%
\ g{\isacharunderscore}{\kern0pt}decomp\ g{\isacharunderscore}{\kern0pt}def\ \isacommand{by}\isamarkupfalse%
\ auto\isanewline
\ \ \ \ \isacommand{also}\isamarkupfalse%
\ \isacommand{have}\isamarkupfalse%
\ {\isachardoublequoteopen}{\isachardot}{\kern0pt}{\isachardot}{\kern0pt}{\isachardot}{\kern0pt}\ {\isacharequal}{\kern0pt}\ distribute{\isacharunderscore}{\kern0pt}left\ Z\ X\ X\ \ {\isasymcirc}\isactrlsub c\ {\isacharparenleft}{\kern0pt}id\isactrlsub c\ Z\ {\isasymtimes}\isactrlsub f\ m{\isacharparenright}{\kern0pt}\ {\isasymcirc}\isactrlsub c\ {\isasymlangle}gz{\isacharcomma}{\kern0pt}\ gy{\isasymrangle}{\isachardoublequoteclose}\isanewline
\ \ \ \ \ \ \isacommand{by}\isamarkupfalse%
\ {\isacharparenleft}{\kern0pt}typecheck{\isacharunderscore}{\kern0pt}cfuncs{\isacharcomma}{\kern0pt}\ auto\ simp\ add{\isacharcolon}{\kern0pt}\ comp{\isacharunderscore}{\kern0pt}associative{\isadigit{2}}{\isacharparenright}{\kern0pt}\isanewline
\ \ \ \ \isacommand{also}\isamarkupfalse%
\ \isacommand{have}\isamarkupfalse%
\ {\isachardoublequoteopen}{\isachardot}{\kern0pt}{\isachardot}{\kern0pt}{\isachardot}{\kern0pt}\ {\isacharequal}{\kern0pt}\ distribute{\isacharunderscore}{\kern0pt}left\ Z\ X\ X\ \ {\isasymcirc}\isactrlsub c\ {\isasymlangle}gz{\isacharcomma}{\kern0pt}\ m\ {\isasymcirc}\isactrlsub c\ gy{\isasymrangle}{\isachardoublequoteclose}\isanewline
\ \ \ \ \ \ \isacommand{by}\isamarkupfalse%
\ {\isacharparenleft}{\kern0pt}typecheck{\isacharunderscore}{\kern0pt}cfuncs{\isacharcomma}{\kern0pt}\ simp\ add{\isacharcolon}{\kern0pt}\ cfunc{\isacharunderscore}{\kern0pt}cross{\isacharunderscore}{\kern0pt}prod{\isacharunderscore}{\kern0pt}comp{\isacharunderscore}{\kern0pt}cfunc{\isacharunderscore}{\kern0pt}prod\ id{\isacharunderscore}{\kern0pt}left{\isacharunderscore}{\kern0pt}unit{\isadigit{2}}{\isacharparenright}{\kern0pt}\isanewline
\ \ \ \ \isacommand{also}\isamarkupfalse%
\ \isacommand{have}\isamarkupfalse%
\ {\isachardoublequoteopen}{\isachardot}{\kern0pt}{\isachardot}{\kern0pt}{\isachardot}{\kern0pt}\ {\isacharequal}{\kern0pt}\ {\isasymlangle}{\isasymlangle}gz{\isacharcomma}{\kern0pt}\ mgy{\isadigit{2}}{\isasymrangle}{\isacharcomma}{\kern0pt}\ {\isasymlangle}gz{\isacharcomma}{\kern0pt}\ mgy{\isadigit{1}}{\isasymrangle}{\isasymrangle}{\isachardoublequoteclose}\isanewline
\ \ \ \ \ \ \isacommand{unfolding}\isamarkupfalse%
\ mgy{\isacharunderscore}{\kern0pt}decomp\ \isacommand{by}\isamarkupfalse%
\ {\isacharparenleft}{\kern0pt}typecheck{\isacharunderscore}{\kern0pt}cfuncs{\isacharcomma}{\kern0pt}\ simp\ add{\isacharcolon}{\kern0pt}\ distribute{\isacharunderscore}{\kern0pt}left{\isacharunderscore}{\kern0pt}ap{\isacharparenright}{\kern0pt}\isanewline
\ \ \ \ \isacommand{then}\isamarkupfalse%
\ \isacommand{show}\isamarkupfalse%
\ {\isacharquery}{\kern0pt}thesis\isanewline
\ \ \ \ \ \ \isacommand{using}\isamarkupfalse%
\ calculation\ \isacommand{by}\isamarkupfalse%
\ auto\isanewline
\ \ \isacommand{qed}\isamarkupfalse%
\isanewline
\ \ \isacommand{then}\isamarkupfalse%
\ \isacommand{have}\isamarkupfalse%
\ t{\isacharunderscore}{\kern0pt}def{\isadigit{2}}{\isacharcolon}{\kern0pt}\ {\isachardoublequoteopen}t\ {\isacharequal}{\kern0pt}\ {\isasymlangle}gz{\isacharcomma}{\kern0pt}\ mgy{\isadigit{2}}{\isasymrangle}{\isachardoublequoteclose}\ \isakeyword{and}\ u{\isacharunderscore}{\kern0pt}def{\isacharcolon}{\kern0pt}\ {\isachardoublequoteopen}u\ {\isacharequal}{\kern0pt}\ {\isasymlangle}gz{\isacharcomma}{\kern0pt}\ mgy{\isadigit{1}}{\isasymrangle}{\isachardoublequoteclose}\isanewline
\ \ \ \ \isacommand{using}\isamarkupfalse%
\ cart{\isacharunderscore}{\kern0pt}prod{\isacharunderscore}{\kern0pt}eq{\isadigit{2}}\ \isacommand{by}\isamarkupfalse%
\ {\isacharparenleft}{\kern0pt}typecheck{\isacharunderscore}{\kern0pt}cfuncs{\isacharcomma}{\kern0pt}\ auto{\isacharcomma}{\kern0pt}\ presburger{\isacharparenright}{\kern0pt}\isanewline
\ \ \isacommand{have}\isamarkupfalse%
\ mhy{\isadigit{2}}{\isacharunderscore}{\kern0pt}eq{\isacharunderscore}{\kern0pt}mgy{\isadigit{2}}{\isacharcolon}{\kern0pt}\ {\isachardoublequoteopen}mhy{\isadigit{2}}\ {\isacharequal}{\kern0pt}\ mgy{\isadigit{2}}{\isachardoublequoteclose}\isanewline
\ \ \ \ \isacommand{using}\isamarkupfalse%
\ t{\isacharunderscore}{\kern0pt}def{\isadigit{2}}\ t{\isacharunderscore}{\kern0pt}def\ cart{\isacharunderscore}{\kern0pt}prod{\isacharunderscore}{\kern0pt}eq{\isadigit{2}}\ \isacommand{by}\isamarkupfalse%
\ {\isacharparenleft}{\kern0pt}auto{\isacharcomma}{\kern0pt}\ typecheck{\isacharunderscore}{\kern0pt}cfuncs{\isacharparenright}{\kern0pt}\isanewline
\ \ \isacommand{have}\isamarkupfalse%
\ gy{\isacharunderscore}{\kern0pt}eq{\isacharunderscore}{\kern0pt}gz{\isacharcolon}{\kern0pt}\ {\isachardoublequoteopen}hz\ {\isacharequal}{\kern0pt}\ gz{\isachardoublequoteclose}\isanewline
\ \ \ \ \isacommand{using}\isamarkupfalse%
\ t{\isacharunderscore}{\kern0pt}def{\isadigit{2}}\ t{\isacharunderscore}{\kern0pt}def\ cart{\isacharunderscore}{\kern0pt}prod{\isacharunderscore}{\kern0pt}eq{\isadigit{2}}\ \isacommand{by}\isamarkupfalse%
\ {\isacharparenleft}{\kern0pt}auto{\isacharcomma}{\kern0pt}\ typecheck{\isacharunderscore}{\kern0pt}cfuncs{\isacharparenright}{\kern0pt}\isanewline
\ \ \isacommand{have}\isamarkupfalse%
\ mhy{\isacharunderscore}{\kern0pt}in{\isacharunderscore}{\kern0pt}Y{\isacharcolon}{\kern0pt}\ {\isachardoublequoteopen}{\isasymlangle}mhy{\isadigit{1}}{\isacharcomma}{\kern0pt}\ mhy{\isadigit{2}}{\isasymrangle}\ {\isasymin}\isactrlbsub X\ {\isasymtimes}\isactrlsub c\ X\isactrlesub \ {\isacharparenleft}{\kern0pt}Y{\isacharcomma}{\kern0pt}\ m{\isacharparenright}{\kern0pt}{\isachardoublequoteclose}\isanewline
\ \ \ \ \isacommand{using}\isamarkupfalse%
\ m{\isacharunderscore}{\kern0pt}def\ h{\isacharunderscore}{\kern0pt}part{\isacharunderscore}{\kern0pt}types\ mhy{\isacharunderscore}{\kern0pt}decomp\isanewline
\ \ \ \ \isacommand{by}\isamarkupfalse%
\ {\isacharparenleft}{\kern0pt}typecheck{\isacharunderscore}{\kern0pt}cfuncs{\isacharcomma}{\kern0pt}\ unfold\ relative{\isacharunderscore}{\kern0pt}member{\isacharunderscore}{\kern0pt}def{\isadigit{2}}\ factors{\isacharunderscore}{\kern0pt}through{\isacharunderscore}{\kern0pt}def{\isadigit{2}}{\isacharcomma}{\kern0pt}\ auto{\isacharparenright}{\kern0pt}\isanewline
\ \ \isacommand{have}\isamarkupfalse%
\ mgy{\isacharunderscore}{\kern0pt}in{\isacharunderscore}{\kern0pt}Y{\isacharcolon}{\kern0pt}\ {\isachardoublequoteopen}{\isasymlangle}mhy{\isadigit{2}}{\isacharcomma}{\kern0pt}\ mgy{\isadigit{1}}{\isasymrangle}\ {\isasymin}\isactrlbsub X\ {\isasymtimes}\isactrlsub c\ X\isactrlesub \ {\isacharparenleft}{\kern0pt}Y{\isacharcomma}{\kern0pt}\ m{\isacharparenright}{\kern0pt}{\isachardoublequoteclose}\isanewline
\ \ \ \ \isacommand{using}\isamarkupfalse%
\ m{\isacharunderscore}{\kern0pt}def\ g{\isacharunderscore}{\kern0pt}part{\isacharunderscore}{\kern0pt}types\ mgy{\isacharunderscore}{\kern0pt}decomp\ mhy{\isadigit{2}}{\isacharunderscore}{\kern0pt}eq{\isacharunderscore}{\kern0pt}mgy{\isadigit{2}}\isanewline
\ \ \ \ \isacommand{by}\isamarkupfalse%
\ {\isacharparenleft}{\kern0pt}typecheck{\isacharunderscore}{\kern0pt}cfuncs{\isacharcomma}{\kern0pt}\ unfold\ relative{\isacharunderscore}{\kern0pt}member{\isacharunderscore}{\kern0pt}def{\isadigit{2}}\ factors{\isacharunderscore}{\kern0pt}through{\isacharunderscore}{\kern0pt}def{\isadigit{2}}{\isacharcomma}{\kern0pt}\ auto{\isacharparenright}{\kern0pt}\isanewline
\ \ \isacommand{have}\isamarkupfalse%
\ {\isachardoublequoteopen}{\isasymlangle}mhy{\isadigit{1}}{\isacharcomma}{\kern0pt}\ mgy{\isadigit{1}}{\isasymrangle}\ {\isasymin}\isactrlbsub X\ {\isasymtimes}\isactrlsub c\ X\isactrlesub \ {\isacharparenleft}{\kern0pt}Y{\isacharcomma}{\kern0pt}\ m{\isacharparenright}{\kern0pt}{\isachardoublequoteclose}\isanewline
\ \ \ \ \isacommand{using}\isamarkupfalse%
\ assms\ mhy{\isacharunderscore}{\kern0pt}in{\isacharunderscore}{\kern0pt}Y\ mgy{\isacharunderscore}{\kern0pt}in{\isacharunderscore}{\kern0pt}Y\ mgy{\isacharunderscore}{\kern0pt}types\ mhy{\isadigit{2}}{\isacharunderscore}{\kern0pt}eq{\isacharunderscore}{\kern0pt}mgy{\isadigit{2}}\ \isacommand{unfolding}\isamarkupfalse%
\ transitive{\isacharunderscore}{\kern0pt}on{\isacharunderscore}{\kern0pt}def\isanewline
\ \ \ \ \isacommand{by}\isamarkupfalse%
\ {\isacharparenleft}{\kern0pt}typecheck{\isacharunderscore}{\kern0pt}cfuncs{\isacharcomma}{\kern0pt}\ blast{\isacharparenright}{\kern0pt}\isanewline
\ \ \isacommand{then}\isamarkupfalse%
\ \isacommand{obtain}\isamarkupfalse%
\ y\ \isakeyword{where}\ y{\isacharunderscore}{\kern0pt}type{\isacharbrackleft}{\kern0pt}type{\isacharunderscore}{\kern0pt}rule{\isacharbrackright}{\kern0pt}{\isacharcolon}{\kern0pt}\ {\isachardoublequoteopen}y\ {\isasymin}\isactrlsub c\ Y{\isachardoublequoteclose}\ \isakeyword{and}\ y{\isacharunderscore}{\kern0pt}def{\isacharcolon}{\kern0pt}\ {\isachardoublequoteopen}m\ {\isasymcirc}\isactrlsub c\ y\ {\isacharequal}{\kern0pt}\ {\isasymlangle}mhy{\isadigit{1}}{\isacharcomma}{\kern0pt}\ mgy{\isadigit{1}}{\isasymrangle}{\isachardoublequoteclose}\isanewline
\ \ \ \ \isacommand{by}\isamarkupfalse%
\ {\isacharparenleft}{\kern0pt}typecheck{\isacharunderscore}{\kern0pt}cfuncs{\isacharcomma}{\kern0pt}\ unfold\ relative{\isacharunderscore}{\kern0pt}member{\isacharunderscore}{\kern0pt}def{\isadigit{2}}\ factors{\isacharunderscore}{\kern0pt}through{\isacharunderscore}{\kern0pt}def{\isadigit{2}}{\isacharcomma}{\kern0pt}\ auto{\isacharparenright}{\kern0pt}\isanewline
\ \ \isacommand{show}\isamarkupfalse%
\ {\isachardoublequoteopen}\ {\isasymlangle}s{\isacharcomma}{\kern0pt}u{\isasymrangle}\ {\isasymin}\isactrlbsub {\isacharparenleft}{\kern0pt}Z\ {\isasymtimes}\isactrlsub c\ X{\isacharparenright}{\kern0pt}\ {\isasymtimes}\isactrlsub c\ Z\ {\isasymtimes}\isactrlsub c\ X\isactrlesub \ {\isacharparenleft}{\kern0pt}Z\ {\isasymtimes}\isactrlsub c\ Y{\isacharcomma}{\kern0pt}\ distribute{\isacharunderscore}{\kern0pt}left\ Z\ X\ X\ {\isasymcirc}\isactrlsub c\ id\isactrlsub c\ Z\ {\isasymtimes}\isactrlsub f\ m{\isacharparenright}{\kern0pt}{\isachardoublequoteclose}\ \isanewline
\ \ \isacommand{proof}\isamarkupfalse%
\ {\isacharparenleft}{\kern0pt}typecheck{\isacharunderscore}{\kern0pt}cfuncs{\isacharcomma}{\kern0pt}\ unfold\ relative{\isacharunderscore}{\kern0pt}member{\isacharunderscore}{\kern0pt}def{\isadigit{2}}\ factors{\isacharunderscore}{\kern0pt}through{\isacharunderscore}{\kern0pt}def{\isadigit{2}}{\isacharcomma}{\kern0pt}\ auto{\isacharparenright}{\kern0pt}\isanewline
\ \ \ \ \isacommand{show}\isamarkupfalse%
\ {\isachardoublequoteopen}monomorphism\ {\isacharparenleft}{\kern0pt}distribute{\isacharunderscore}{\kern0pt}left\ Z\ X\ X\ {\isasymcirc}\isactrlsub c\ id\isactrlsub c\ Z\ {\isasymtimes}\isactrlsub f\ m{\isacharparenright}{\kern0pt}{\isachardoublequoteclose}\isanewline
\ \ \ \ \ \ \isacommand{using}\isamarkupfalse%
\ relative{\isacharunderscore}{\kern0pt}member{\isacharunderscore}{\kern0pt}def{\isadigit{2}}\ st{\isacharunderscore}{\kern0pt}relation\ \isacommand{by}\isamarkupfalse%
\ blast\isanewline
\ \ \ \ \isacommand{show}\isamarkupfalse%
\ {\isachardoublequoteopen}{\isasymexists}h{\isachardot}{\kern0pt}\ h\ {\isasymin}\isactrlsub c\ Z\ {\isasymtimes}\isactrlsub c\ Y\ {\isasymand}\ {\isacharparenleft}{\kern0pt}distribute{\isacharunderscore}{\kern0pt}left\ Z\ X\ X\ {\isasymcirc}\isactrlsub c\ id\isactrlsub c\ Z\ {\isasymtimes}\isactrlsub f\ m{\isacharparenright}{\kern0pt}\ {\isasymcirc}\isactrlsub c\ h\ {\isacharequal}{\kern0pt}\ {\isasymlangle}s{\isacharcomma}{\kern0pt}u{\isasymrangle}{\isachardoublequoteclose}\isanewline
\ \ \ \ \ \ \isacommand{unfolding}\isamarkupfalse%
\ s{\isacharunderscore}{\kern0pt}def\ u{\isacharunderscore}{\kern0pt}def\ gy{\isacharunderscore}{\kern0pt}eq{\isacharunderscore}{\kern0pt}gz\isanewline
\ \ \ \ \isacommand{proof}\isamarkupfalse%
\ {\isacharparenleft}{\kern0pt}rule{\isacharunderscore}{\kern0pt}tac\ x{\isacharequal}{\kern0pt}{\isachardoublequoteopen}{\isasymlangle}gz{\isacharcomma}{\kern0pt}y{\isasymrangle}{\isachardoublequoteclose}\ \isakeyword{in}\ exI{\isacharcomma}{\kern0pt}\ auto{\isacharcomma}{\kern0pt}\ typecheck{\isacharunderscore}{\kern0pt}cfuncs{\isacharparenright}{\kern0pt}\isanewline
\ \ \ \ \ \ \isacommand{have}\isamarkupfalse%
\ {\isachardoublequoteopen}{\isacharparenleft}{\kern0pt}distribute{\isacharunderscore}{\kern0pt}left\ Z\ X\ X\ \ {\isasymcirc}\isactrlsub c\ {\isacharparenleft}{\kern0pt}id\isactrlsub c\ Z\ {\isasymtimes}\isactrlsub f\ m{\isacharparenright}{\kern0pt}{\isacharparenright}{\kern0pt}\ {\isasymcirc}\isactrlsub c\ {\isasymlangle}gz{\isacharcomma}{\kern0pt}y{\isasymrangle}\ {\isacharequal}{\kern0pt}\ distribute{\isacharunderscore}{\kern0pt}left\ Z\ X\ X\ \ {\isasymcirc}\isactrlsub c\ {\isacharparenleft}{\kern0pt}id\isactrlsub c\ Z\ {\isasymtimes}\isactrlsub f\ m{\isacharparenright}{\kern0pt}\ {\isasymcirc}\isactrlsub c\ {\isasymlangle}gz{\isacharcomma}{\kern0pt}y{\isasymrangle}{\isachardoublequoteclose}\isanewline
\ \ \ \ \ \ \ \ \isacommand{by}\isamarkupfalse%
\ {\isacharparenleft}{\kern0pt}typecheck{\isacharunderscore}{\kern0pt}cfuncs{\isacharcomma}{\kern0pt}\ auto\ simp\ add{\isacharcolon}{\kern0pt}\ comp{\isacharunderscore}{\kern0pt}associative{\isadigit{2}}{\isacharparenright}{\kern0pt}\isanewline
\ \ \ \ \ \ \isacommand{also}\isamarkupfalse%
\ \isacommand{have}\isamarkupfalse%
\ {\isachardoublequoteopen}{\isachardot}{\kern0pt}{\isachardot}{\kern0pt}{\isachardot}{\kern0pt}\ {\isacharequal}{\kern0pt}\ distribute{\isacharunderscore}{\kern0pt}left\ Z\ X\ X\ \ {\isasymcirc}\isactrlsub c\ {\isasymlangle}gz{\isacharcomma}{\kern0pt}\ m\ {\isasymcirc}\isactrlsub c\ y{\isasymrangle}{\isachardoublequoteclose}\isanewline
\ \ \ \ \ \ \ \ \isacommand{by}\isamarkupfalse%
\ {\isacharparenleft}{\kern0pt}typecheck{\isacharunderscore}{\kern0pt}cfuncs{\isacharcomma}{\kern0pt}\ simp\ add{\isacharcolon}{\kern0pt}\ cfunc{\isacharunderscore}{\kern0pt}cross{\isacharunderscore}{\kern0pt}prod{\isacharunderscore}{\kern0pt}comp{\isacharunderscore}{\kern0pt}cfunc{\isacharunderscore}{\kern0pt}prod\ id{\isacharunderscore}{\kern0pt}left{\isacharunderscore}{\kern0pt}unit{\isadigit{2}}{\isacharparenright}{\kern0pt}\isanewline
\ \ \ \ \ \ \isacommand{also}\isamarkupfalse%
\ \isacommand{have}\isamarkupfalse%
\ {\isachardoublequoteopen}{\isachardot}{\kern0pt}{\isachardot}{\kern0pt}{\isachardot}{\kern0pt}\ {\isacharequal}{\kern0pt}\ {\isasymlangle}{\isasymlangle}gz{\isacharcomma}{\kern0pt}mhy{\isadigit{1}}{\isasymrangle}{\isacharcomma}{\kern0pt}{\isasymlangle}gz{\isacharcomma}{\kern0pt}mgy{\isadigit{1}}{\isasymrangle}{\isasymrangle}{\isachardoublequoteclose}\isanewline
\ \ \ \ \ \ \ \ \isacommand{by}\isamarkupfalse%
\ {\isacharparenleft}{\kern0pt}typecheck{\isacharunderscore}{\kern0pt}cfuncs{\isacharcomma}{\kern0pt}\ simp\ add{\isacharcolon}{\kern0pt}\ distribute{\isacharunderscore}{\kern0pt}left{\isacharunderscore}{\kern0pt}ap\ y{\isacharunderscore}{\kern0pt}def{\isacharparenright}{\kern0pt}\isanewline
\ \ \ \ \ \ \isacommand{then}\isamarkupfalse%
\ \isacommand{show}\isamarkupfalse%
\ {\isachardoublequoteopen}{\isacharparenleft}{\kern0pt}distribute{\isacharunderscore}{\kern0pt}left\ Z\ X\ X\ {\isasymcirc}\isactrlsub c\ id\isactrlsub c\ Z\ {\isasymtimes}\isactrlsub f\ m{\isacharparenright}{\kern0pt}\ {\isasymcirc}\isactrlsub c\ {\isasymlangle}gz{\isacharcomma}{\kern0pt}y{\isasymrangle}\ {\isacharequal}{\kern0pt}\ {\isasymlangle}{\isasymlangle}gz{\isacharcomma}{\kern0pt}mhy{\isadigit{1}}{\isasymrangle}{\isacharcomma}{\kern0pt}{\isasymlangle}gz{\isacharcomma}{\kern0pt}mgy{\isadigit{1}}{\isasymrangle}{\isasymrangle}{\isachardoublequoteclose}\isanewline
\ \ \ \ \ \ \ \ \isacommand{using}\isamarkupfalse%
\ calculation\ \isacommand{by}\isamarkupfalse%
\ auto\isanewline
\ \ \ \ \isacommand{qed}\isamarkupfalse%
\isanewline
\ \ \isacommand{qed}\isamarkupfalse%
\isanewline
\isacommand{qed}\isamarkupfalse%
%
\endisatagproof
{\isafoldproof}%
%
\isadelimproof
\isanewline
%
\endisadelimproof
\isanewline
\isacommand{lemma}\isamarkupfalse%
\ left{\isacharunderscore}{\kern0pt}pair{\isacharunderscore}{\kern0pt}equiv{\isacharunderscore}{\kern0pt}rel{\isacharcolon}{\kern0pt}\isanewline
\ \ \isakeyword{assumes}\ {\isachardoublequoteopen}equiv{\isacharunderscore}{\kern0pt}rel{\isacharunderscore}{\kern0pt}on\ X\ {\isacharparenleft}{\kern0pt}Y{\isacharcomma}{\kern0pt}\ m{\isacharparenright}{\kern0pt}{\isachardoublequoteclose}\isanewline
\ \ \isakeyword{shows}\ {\isachardoublequoteopen}equiv{\isacharunderscore}{\kern0pt}rel{\isacharunderscore}{\kern0pt}on\ {\isacharparenleft}{\kern0pt}X\ {\isasymtimes}\isactrlsub c\ Z{\isacharparenright}{\kern0pt}\ {\isacharparenleft}{\kern0pt}Y\ {\isasymtimes}\isactrlsub c\ Z{\isacharcomma}{\kern0pt}\ distribute{\isacharunderscore}{\kern0pt}right\ X\ X\ Z\ {\isasymcirc}\isactrlsub c\ {\isacharparenleft}{\kern0pt}m\ {\isasymtimes}\isactrlsub f\ id\ Z{\isacharparenright}{\kern0pt}{\isacharparenright}{\kern0pt}{\isachardoublequoteclose}\isanewline
%
\isadelimproof
\ \ %
\endisadelimproof
%
\isatagproof
\isacommand{using}\isamarkupfalse%
\ assms\ left{\isacharunderscore}{\kern0pt}pair{\isacharunderscore}{\kern0pt}reflexive\ left{\isacharunderscore}{\kern0pt}pair{\isacharunderscore}{\kern0pt}symmetric\ left{\isacharunderscore}{\kern0pt}pair{\isacharunderscore}{\kern0pt}transitive\isanewline
\ \ \isacommand{by}\isamarkupfalse%
\ {\isacharparenleft}{\kern0pt}unfold\ equiv{\isacharunderscore}{\kern0pt}rel{\isacharunderscore}{\kern0pt}on{\isacharunderscore}{\kern0pt}def{\isacharcomma}{\kern0pt}\ auto{\isacharparenright}{\kern0pt}%
\endisatagproof
{\isafoldproof}%
%
\isadelimproof
\isanewline
%
\endisadelimproof
\isanewline
\isacommand{lemma}\isamarkupfalse%
\ right{\isacharunderscore}{\kern0pt}pair{\isacharunderscore}{\kern0pt}equiv{\isacharunderscore}{\kern0pt}rel{\isacharcolon}{\kern0pt}\isanewline
\ \ \isakeyword{assumes}\ {\isachardoublequoteopen}equiv{\isacharunderscore}{\kern0pt}rel{\isacharunderscore}{\kern0pt}on\ X\ {\isacharparenleft}{\kern0pt}Y{\isacharcomma}{\kern0pt}\ m{\isacharparenright}{\kern0pt}{\isachardoublequoteclose}\isanewline
\ \ \isakeyword{shows}\ {\isachardoublequoteopen}equiv{\isacharunderscore}{\kern0pt}rel{\isacharunderscore}{\kern0pt}on\ {\isacharparenleft}{\kern0pt}Z\ {\isasymtimes}\isactrlsub c\ X{\isacharparenright}{\kern0pt}\ {\isacharparenleft}{\kern0pt}Z\ {\isasymtimes}\isactrlsub c\ Y{\isacharcomma}{\kern0pt}\ distribute{\isacharunderscore}{\kern0pt}left\ Z\ X\ X\ \ {\isasymcirc}\isactrlsub c\ {\isacharparenleft}{\kern0pt}id\ Z\ {\isasymtimes}\isactrlsub f\ m{\isacharparenright}{\kern0pt}{\isacharparenright}{\kern0pt}{\isachardoublequoteclose}\isanewline
%
\isadelimproof
\ \ %
\endisadelimproof
%
\isatagproof
\isacommand{using}\isamarkupfalse%
\ assms\ right{\isacharunderscore}{\kern0pt}pair{\isacharunderscore}{\kern0pt}reflexive\ right{\isacharunderscore}{\kern0pt}pair{\isacharunderscore}{\kern0pt}symmetric\ right{\isacharunderscore}{\kern0pt}pair{\isacharunderscore}{\kern0pt}transitive\isanewline
\ \ \isacommand{by}\isamarkupfalse%
\ {\isacharparenleft}{\kern0pt}unfold\ equiv{\isacharunderscore}{\kern0pt}rel{\isacharunderscore}{\kern0pt}on{\isacharunderscore}{\kern0pt}def{\isacharcomma}{\kern0pt}\ auto{\isacharparenright}{\kern0pt}%
\endisatagproof
{\isafoldproof}%
%
\isadelimproof
%
\endisadelimproof
%
\isadelimdocument
%
\endisadelimdocument
%
\isatagdocument
%
\isamarkupsection{Graphs%
}
\isamarkuptrue%
%
\endisatagdocument
{\isafolddocument}%
%
\isadelimdocument
%
\endisadelimdocument
\isacommand{definition}\isamarkupfalse%
\ functional{\isacharunderscore}{\kern0pt}on\ {\isacharcolon}{\kern0pt}{\isacharcolon}{\kern0pt}\ {\isachardoublequoteopen}cset\ {\isasymRightarrow}\ cset\ {\isasymRightarrow}\ cset\ {\isasymtimes}\ cfunc\ {\isasymRightarrow}\ bool{\isachardoublequoteclose}\ \isakeyword{where}\isanewline
\ \ {\isachardoublequoteopen}functional{\isacharunderscore}{\kern0pt}on\ X\ Y\ R\ {\isacharequal}{\kern0pt}\ {\isacharparenleft}{\kern0pt}R\ \ {\isasymsubseteq}\isactrlsub c\ X\ {\isasymtimes}\isactrlsub c\ Y\ {\isasymand}\isanewline
\ \ \ \ {\isacharparenleft}{\kern0pt}{\isasymforall}x{\isachardot}{\kern0pt}\ x\ {\isasymin}\isactrlsub c\ X\ {\isasymlongrightarrow}\ {\isacharparenleft}{\kern0pt}{\isasymexists}{\isacharbang}{\kern0pt}\ y{\isachardot}{\kern0pt}\ \ y\ {\isasymin}\isactrlsub c\ Y\ {\isasymand}\ \ \isanewline
\ \ \ \ \ \ {\isasymlangle}x{\isacharcomma}{\kern0pt}y{\isasymrangle}\ {\isasymin}\isactrlbsub X{\isasymtimes}\isactrlsub cY\isactrlesub \ R{\isacharparenright}{\kern0pt}{\isacharparenright}{\kern0pt}{\isacharparenright}{\kern0pt}{\isachardoublequoteclose}%
\begin{isamarkuptext}%
The definition below corresponds to Definition 2.3.12 in Halvorson.%
\end{isamarkuptext}\isamarkuptrue%
\isacommand{definition}\isamarkupfalse%
\ graph\ {\isacharcolon}{\kern0pt}{\isacharcolon}{\kern0pt}\ {\isachardoublequoteopen}cfunc\ {\isasymRightarrow}\ cset{\isachardoublequoteclose}\ \isakeyword{where}\isanewline
\ {\isachardoublequoteopen}graph\ f\ {\isacharequal}{\kern0pt}\ {\isacharparenleft}{\kern0pt}SOME\ E{\isachardot}{\kern0pt}\ {\isasymexists}\ m{\isachardot}{\kern0pt}\ equalizer\ E\ m\ {\isacharparenleft}{\kern0pt}f\ {\isasymcirc}\isactrlsub c\ left{\isacharunderscore}{\kern0pt}cart{\isacharunderscore}{\kern0pt}proj\ {\isacharparenleft}{\kern0pt}domain\ f{\isacharparenright}{\kern0pt}\ {\isacharparenleft}{\kern0pt}codomain\ f{\isacharparenright}{\kern0pt}{\isacharparenright}{\kern0pt}\ {\isacharparenleft}{\kern0pt}right{\isacharunderscore}{\kern0pt}cart{\isacharunderscore}{\kern0pt}proj\ {\isacharparenleft}{\kern0pt}domain\ f{\isacharparenright}{\kern0pt}\ {\isacharparenleft}{\kern0pt}codomain\ f{\isacharparenright}{\kern0pt}{\isacharparenright}{\kern0pt}{\isacharparenright}{\kern0pt}{\isachardoublequoteclose}\isanewline
\isanewline
\isacommand{lemma}\isamarkupfalse%
\ graph{\isacharunderscore}{\kern0pt}equalizer{\isacharcolon}{\kern0pt}\isanewline
\ \ {\isachardoublequoteopen}{\isasymexists}\ m{\isachardot}{\kern0pt}\ equalizer\ {\isacharparenleft}{\kern0pt}graph\ f{\isacharparenright}{\kern0pt}\ m\ {\isacharparenleft}{\kern0pt}f\ {\isasymcirc}\isactrlsub c\ left{\isacharunderscore}{\kern0pt}cart{\isacharunderscore}{\kern0pt}proj\ {\isacharparenleft}{\kern0pt}domain\ f{\isacharparenright}{\kern0pt}\ {\isacharparenleft}{\kern0pt}codomain\ f{\isacharparenright}{\kern0pt}{\isacharparenright}{\kern0pt}\ {\isacharparenleft}{\kern0pt}right{\isacharunderscore}{\kern0pt}cart{\isacharunderscore}{\kern0pt}proj\ {\isacharparenleft}{\kern0pt}domain\ f{\isacharparenright}{\kern0pt}\ {\isacharparenleft}{\kern0pt}codomain\ f{\isacharparenright}{\kern0pt}{\isacharparenright}{\kern0pt}{\isachardoublequoteclose}\isanewline
%
\isadelimproof
\ \ %
\endisadelimproof
%
\isatagproof
\isacommand{by}\isamarkupfalse%
\ {\isacharparenleft}{\kern0pt}unfold\ graph{\isacharunderscore}{\kern0pt}def{\isacharcomma}{\kern0pt}\ typecheck{\isacharunderscore}{\kern0pt}cfuncs{\isacharcomma}{\kern0pt}\ rule{\isacharunderscore}{\kern0pt}tac\ someI{\isacharunderscore}{\kern0pt}ex{\isacharcomma}{\kern0pt}\ simp\ add{\isacharcolon}{\kern0pt}\ cfunc{\isacharunderscore}{\kern0pt}type{\isacharunderscore}{\kern0pt}def\ equalizer{\isacharunderscore}{\kern0pt}exists{\isacharparenright}{\kern0pt}%
\endisatagproof
{\isafoldproof}%
%
\isadelimproof
\isanewline
%
\endisadelimproof
\ \ \isanewline
\isacommand{lemma}\isamarkupfalse%
\ graph{\isacharunderscore}{\kern0pt}equalizer{\isadigit{2}}{\isacharcolon}{\kern0pt}\isanewline
\ \ \isakeyword{assumes}\ {\isachardoublequoteopen}f\ {\isacharcolon}{\kern0pt}\ X\ {\isasymrightarrow}\ Y{\isachardoublequoteclose}\isanewline
\ \ \isakeyword{shows}\ {\isachardoublequoteopen}{\isasymexists}\ m{\isachardot}{\kern0pt}\ equalizer\ {\isacharparenleft}{\kern0pt}graph\ f{\isacharparenright}{\kern0pt}\ m\ {\isacharparenleft}{\kern0pt}f\ {\isasymcirc}\isactrlsub c\ left{\isacharunderscore}{\kern0pt}cart{\isacharunderscore}{\kern0pt}proj\ X\ Y{\isacharparenright}{\kern0pt}\ {\isacharparenleft}{\kern0pt}right{\isacharunderscore}{\kern0pt}cart{\isacharunderscore}{\kern0pt}proj\ X\ Y{\isacharparenright}{\kern0pt}{\isachardoublequoteclose}\isanewline
%
\isadelimproof
\ \ %
\endisadelimproof
%
\isatagproof
\isacommand{using}\isamarkupfalse%
\ assms\ \isacommand{by}\isamarkupfalse%
\ {\isacharparenleft}{\kern0pt}typecheck{\isacharunderscore}{\kern0pt}cfuncs{\isacharcomma}{\kern0pt}\ metis\ cfunc{\isacharunderscore}{\kern0pt}type{\isacharunderscore}{\kern0pt}def\ graph{\isacharunderscore}{\kern0pt}equalizer{\isacharparenright}{\kern0pt}%
\endisatagproof
{\isafoldproof}%
%
\isadelimproof
\isanewline
%
\endisadelimproof
\isanewline
\isacommand{definition}\isamarkupfalse%
\ graph{\isacharunderscore}{\kern0pt}morph\ {\isacharcolon}{\kern0pt}{\isacharcolon}{\kern0pt}\ {\isachardoublequoteopen}cfunc\ {\isasymRightarrow}\ cfunc{\isachardoublequoteclose}\ \isakeyword{where}\isanewline
\ {\isachardoublequoteopen}graph{\isacharunderscore}{\kern0pt}morph\ f\ {\isacharequal}{\kern0pt}\ {\isacharparenleft}{\kern0pt}SOME\ m{\isachardot}{\kern0pt}\ equalizer\ {\isacharparenleft}{\kern0pt}graph\ f{\isacharparenright}{\kern0pt}\ m\ {\isacharparenleft}{\kern0pt}f\ {\isasymcirc}\isactrlsub c\ left{\isacharunderscore}{\kern0pt}cart{\isacharunderscore}{\kern0pt}proj\ {\isacharparenleft}{\kern0pt}domain\ f{\isacharparenright}{\kern0pt}\ {\isacharparenleft}{\kern0pt}codomain\ f{\isacharparenright}{\kern0pt}{\isacharparenright}{\kern0pt}\ {\isacharparenleft}{\kern0pt}right{\isacharunderscore}{\kern0pt}cart{\isacharunderscore}{\kern0pt}proj\ {\isacharparenleft}{\kern0pt}domain\ f{\isacharparenright}{\kern0pt}\ {\isacharparenleft}{\kern0pt}codomain\ f{\isacharparenright}{\kern0pt}{\isacharparenright}{\kern0pt}{\isacharparenright}{\kern0pt}{\isachardoublequoteclose}\isanewline
\isanewline
\isacommand{lemma}\isamarkupfalse%
\ graph{\isacharunderscore}{\kern0pt}equalizer{\isadigit{3}}{\isacharcolon}{\kern0pt}\isanewline
\ \ {\isachardoublequoteopen}equalizer\ {\isacharparenleft}{\kern0pt}graph\ f{\isacharparenright}{\kern0pt}\ {\isacharparenleft}{\kern0pt}graph{\isacharunderscore}{\kern0pt}morph\ f{\isacharparenright}{\kern0pt}\ {\isacharparenleft}{\kern0pt}f\ {\isasymcirc}\isactrlsub c\ left{\isacharunderscore}{\kern0pt}cart{\isacharunderscore}{\kern0pt}proj\ {\isacharparenleft}{\kern0pt}domain\ f{\isacharparenright}{\kern0pt}\ {\isacharparenleft}{\kern0pt}codomain\ f{\isacharparenright}{\kern0pt}{\isacharparenright}{\kern0pt}\ {\isacharparenleft}{\kern0pt}right{\isacharunderscore}{\kern0pt}cart{\isacharunderscore}{\kern0pt}proj\ {\isacharparenleft}{\kern0pt}domain\ f{\isacharparenright}{\kern0pt}\ {\isacharparenleft}{\kern0pt}codomain\ f{\isacharparenright}{\kern0pt}{\isacharparenright}{\kern0pt}{\isachardoublequoteclose}\isanewline
%
\isadelimproof
\ \ \ %
\endisadelimproof
%
\isatagproof
\isacommand{using}\isamarkupfalse%
\ graph{\isacharunderscore}{\kern0pt}equalizer\ \isacommand{by}\isamarkupfalse%
\ {\isacharparenleft}{\kern0pt}unfold\ graph{\isacharunderscore}{\kern0pt}morph{\isacharunderscore}{\kern0pt}def{\isacharcomma}{\kern0pt}\ typecheck{\isacharunderscore}{\kern0pt}cfuncs{\isacharcomma}{\kern0pt}\ rule{\isacharunderscore}{\kern0pt}tac\ someI{\isacharunderscore}{\kern0pt}ex{\isacharcomma}{\kern0pt}\ blast{\isacharparenright}{\kern0pt}%
\endisatagproof
{\isafoldproof}%
%
\isadelimproof
\isanewline
%
\endisadelimproof
\isanewline
\isacommand{lemma}\isamarkupfalse%
\ graph{\isacharunderscore}{\kern0pt}equalizer{\isadigit{4}}{\isacharcolon}{\kern0pt}\isanewline
\ \ \isakeyword{assumes}\ {\isachardoublequoteopen}f\ {\isacharcolon}{\kern0pt}\ X\ {\isasymrightarrow}\ Y{\isachardoublequoteclose}\isanewline
\ \ \isakeyword{shows}\ {\isachardoublequoteopen}equalizer\ {\isacharparenleft}{\kern0pt}graph\ f{\isacharparenright}{\kern0pt}\ {\isacharparenleft}{\kern0pt}graph{\isacharunderscore}{\kern0pt}morph\ f{\isacharparenright}{\kern0pt}\ {\isacharparenleft}{\kern0pt}f\ {\isasymcirc}\isactrlsub c\ left{\isacharunderscore}{\kern0pt}cart{\isacharunderscore}{\kern0pt}proj\ X\ Y{\isacharparenright}{\kern0pt}\ {\isacharparenleft}{\kern0pt}right{\isacharunderscore}{\kern0pt}cart{\isacharunderscore}{\kern0pt}proj\ X\ Y{\isacharparenright}{\kern0pt}{\isachardoublequoteclose}\isanewline
%
\isadelimproof
\ \ %
\endisadelimproof
%
\isatagproof
\isacommand{using}\isamarkupfalse%
\ assms\ cfunc{\isacharunderscore}{\kern0pt}type{\isacharunderscore}{\kern0pt}def\ graph{\isacharunderscore}{\kern0pt}equalizer{\isadigit{3}}\ \isacommand{by}\isamarkupfalse%
\ auto%
\endisatagproof
{\isafoldproof}%
%
\isadelimproof
\isanewline
%
\endisadelimproof
\isanewline
\isacommand{lemma}\isamarkupfalse%
\ graph{\isacharunderscore}{\kern0pt}subobject{\isacharcolon}{\kern0pt}\isanewline
\ \ \isakeyword{assumes}\ {\isachardoublequoteopen}f\ {\isacharcolon}{\kern0pt}\ X\ {\isasymrightarrow}\ Y{\isachardoublequoteclose}\isanewline
\ \ \isakeyword{shows}\ {\isachardoublequoteopen}{\isacharparenleft}{\kern0pt}graph\ f{\isacharcomma}{\kern0pt}\ graph{\isacharunderscore}{\kern0pt}morph\ f{\isacharparenright}{\kern0pt}\ {\isasymsubseteq}\isactrlsub c\ {\isacharparenleft}{\kern0pt}X\ {\isasymtimes}\isactrlsub c\ Y{\isacharparenright}{\kern0pt}{\isachardoublequoteclose}\isanewline
%
\isadelimproof
\ \ %
\endisadelimproof
%
\isatagproof
\isacommand{by}\isamarkupfalse%
\ {\isacharparenleft}{\kern0pt}metis\ assms\ cfunc{\isacharunderscore}{\kern0pt}type{\isacharunderscore}{\kern0pt}def\ equalizer{\isacharunderscore}{\kern0pt}def\ equalizer{\isacharunderscore}{\kern0pt}is{\isacharunderscore}{\kern0pt}monomorphism\ graph{\isacharunderscore}{\kern0pt}equalizer{\isadigit{3}}\ right{\isacharunderscore}{\kern0pt}cart{\isacharunderscore}{\kern0pt}proj{\isacharunderscore}{\kern0pt}type\ subobject{\isacharunderscore}{\kern0pt}of{\isacharunderscore}{\kern0pt}def{\isadigit{2}}{\isacharparenright}{\kern0pt}%
\endisatagproof
{\isafoldproof}%
%
\isadelimproof
\isanewline
%
\endisadelimproof
\isanewline
\isacommand{lemma}\isamarkupfalse%
\ graph{\isacharunderscore}{\kern0pt}morph{\isacharunderscore}{\kern0pt}type{\isacharbrackleft}{\kern0pt}type{\isacharunderscore}{\kern0pt}rule{\isacharbrackright}{\kern0pt}{\isacharcolon}{\kern0pt}\isanewline
\ \ \isakeyword{assumes}\ {\isachardoublequoteopen}f\ {\isacharcolon}{\kern0pt}\ X\ {\isasymrightarrow}\ Y{\isachardoublequoteclose}\isanewline
\ \ \isakeyword{shows}\ {\isachardoublequoteopen}graph{\isacharunderscore}{\kern0pt}morph{\isacharparenleft}{\kern0pt}f{\isacharparenright}{\kern0pt}\ {\isacharcolon}{\kern0pt}\ graph\ f\ {\isasymrightarrow}\ X\ {\isasymtimes}\isactrlsub c\ Y{\isachardoublequoteclose}\isanewline
%
\isadelimproof
\ \ %
\endisadelimproof
%
\isatagproof
\isacommand{using}\isamarkupfalse%
\ graph{\isacharunderscore}{\kern0pt}subobject\ subobject{\isacharunderscore}{\kern0pt}of{\isacharunderscore}{\kern0pt}def{\isadigit{2}}\ assms\ \isacommand{by}\isamarkupfalse%
\ auto%
\endisatagproof
{\isafoldproof}%
%
\isadelimproof
%
\endisadelimproof
%
\begin{isamarkuptext}%
The lemma below corresponds to Exercise 2.3.13 in Halvorson.%
\end{isamarkuptext}\isamarkuptrue%
\isacommand{lemma}\isamarkupfalse%
\ graphs{\isacharunderscore}{\kern0pt}are{\isacharunderscore}{\kern0pt}functional{\isacharcolon}{\kern0pt}\isanewline
\ \ \isakeyword{assumes}\ {\isachardoublequoteopen}f\ {\isacharcolon}{\kern0pt}\ X\ {\isasymrightarrow}\ Y{\isachardoublequoteclose}\isanewline
\ \ \isakeyword{shows}\ {\isachardoublequoteopen}functional{\isacharunderscore}{\kern0pt}on\ X\ Y\ {\isacharparenleft}{\kern0pt}graph\ f{\isacharcomma}{\kern0pt}\ graph{\isacharunderscore}{\kern0pt}morph\ f{\isacharparenright}{\kern0pt}{\isachardoublequoteclose}\isanewline
%
\isadelimproof
%
\endisadelimproof
%
\isatagproof
\isacommand{proof}\isamarkupfalse%
{\isacharparenleft}{\kern0pt}unfold\ functional{\isacharunderscore}{\kern0pt}on{\isacharunderscore}{\kern0pt}def{\isacharcomma}{\kern0pt}\ auto{\isacharparenright}{\kern0pt}\isanewline
\ \ \isacommand{show}\isamarkupfalse%
\ graph{\isacharunderscore}{\kern0pt}subobj{\isacharcolon}{\kern0pt}\ {\isachardoublequoteopen}{\isacharparenleft}{\kern0pt}graph\ f{\isacharcomma}{\kern0pt}\ graph{\isacharunderscore}{\kern0pt}morph\ f{\isacharparenright}{\kern0pt}\ \ {\isasymsubseteq}\isactrlsub c\ {\isacharparenleft}{\kern0pt}X{\isasymtimes}\isactrlsub c\ Y{\isacharparenright}{\kern0pt}{\isachardoublequoteclose}\isanewline
\ \ \ \ \isacommand{by}\isamarkupfalse%
\ {\isacharparenleft}{\kern0pt}simp\ add{\isacharcolon}{\kern0pt}\ assms\ graph{\isacharunderscore}{\kern0pt}subobject{\isacharparenright}{\kern0pt}\isanewline
\ \ \isacommand{show}\isamarkupfalse%
\ {\isachardoublequoteopen}{\isasymAnd}x{\isachardot}{\kern0pt}\ x\ {\isasymin}\isactrlsub c\ X\ {\isasymLongrightarrow}\ {\isasymexists}y{\isachardot}{\kern0pt}\ y\ {\isasymin}\isactrlsub c\ Y\ {\isasymand}\ {\isasymlangle}x{\isacharcomma}{\kern0pt}y{\isasymrangle}\ {\isasymin}\isactrlbsub X\ {\isasymtimes}\isactrlsub c\ Y\isactrlesub \ {\isacharparenleft}{\kern0pt}graph\ f{\isacharcomma}{\kern0pt}\ graph{\isacharunderscore}{\kern0pt}morph\ f{\isacharparenright}{\kern0pt}{\isachardoublequoteclose}\isanewline
\ \ \isacommand{proof}\isamarkupfalse%
\ {\isacharminus}{\kern0pt}\ \isanewline
\ \ \ \ \isacommand{fix}\isamarkupfalse%
\ x\ \isanewline
\ \ \ \ \isacommand{assume}\isamarkupfalse%
\ x{\isacharunderscore}{\kern0pt}type{\isacharbrackleft}{\kern0pt}type{\isacharunderscore}{\kern0pt}rule{\isacharbrackright}{\kern0pt}{\isacharcolon}{\kern0pt}\ {\isachardoublequoteopen}x\ {\isasymin}\isactrlsub c\ X{\isachardoublequoteclose}\isanewline
\ \ \ \ \isacommand{obtain}\isamarkupfalse%
\ y\ \isakeyword{where}\ y{\isacharunderscore}{\kern0pt}def{\isacharcolon}{\kern0pt}\ {\isachardoublequoteopen}y\ {\isacharequal}{\kern0pt}\ f\ {\isasymcirc}\isactrlsub c\ x{\isachardoublequoteclose}\isanewline
\ \ \ \ \ \ \isacommand{by}\isamarkupfalse%
\ simp\isanewline
\ \ \ \ \isacommand{then}\isamarkupfalse%
\ \isacommand{have}\isamarkupfalse%
\ y{\isacharunderscore}{\kern0pt}type{\isacharbrackleft}{\kern0pt}type{\isacharunderscore}{\kern0pt}rule{\isacharbrackright}{\kern0pt}{\isacharcolon}{\kern0pt}\ {\isachardoublequoteopen}y\ {\isasymin}\isactrlsub c\ Y{\isachardoublequoteclose}\isanewline
\ \ \ \ \ \ \isacommand{using}\isamarkupfalse%
\ assms\ comp{\isacharunderscore}{\kern0pt}type\ x{\isacharunderscore}{\kern0pt}type\ y{\isacharunderscore}{\kern0pt}def\ \isacommand{by}\isamarkupfalse%
\ blast\isanewline
\isanewline
\ \ \ \ \isacommand{have}\isamarkupfalse%
\ {\isachardoublequoteopen}{\isasymlangle}x{\isacharcomma}{\kern0pt}y{\isasymrangle}\ {\isasymin}\isactrlbsub X\ {\isasymtimes}\isactrlsub c\ Y\isactrlesub \ {\isacharparenleft}{\kern0pt}graph\ f{\isacharcomma}{\kern0pt}\ graph{\isacharunderscore}{\kern0pt}morph\ f{\isacharparenright}{\kern0pt}{\isachardoublequoteclose}\isanewline
\ \ \ \ \isacommand{proof}\isamarkupfalse%
{\isacharparenleft}{\kern0pt}unfold\ relative{\isacharunderscore}{\kern0pt}member{\isacharunderscore}{\kern0pt}def{\isacharcomma}{\kern0pt}\ auto{\isacharparenright}{\kern0pt}\isanewline
\ \ \ \ \ \ \isacommand{show}\isamarkupfalse%
\ {\isachardoublequoteopen}{\isasymlangle}x{\isacharcomma}{\kern0pt}y{\isasymrangle}\ {\isasymin}\isactrlsub c\ X\ {\isasymtimes}\isactrlsub c\ Y{\isachardoublequoteclose}\isanewline
\ \ \ \ \ \ \ \ \isacommand{by}\isamarkupfalse%
\ typecheck{\isacharunderscore}{\kern0pt}cfuncs\ \isanewline
\ \ \ \ \ \ \isacommand{show}\isamarkupfalse%
\ {\isachardoublequoteopen}monomorphism\ {\isacharparenleft}{\kern0pt}graph{\isacharunderscore}{\kern0pt}morph\ f{\isacharparenright}{\kern0pt}{\isachardoublequoteclose}\isanewline
\ \ \ \ \ \ \ \ \isacommand{using}\isamarkupfalse%
\ graph{\isacharunderscore}{\kern0pt}subobj\ subobject{\isacharunderscore}{\kern0pt}of{\isacharunderscore}{\kern0pt}def{\isadigit{2}}\ \isacommand{by}\isamarkupfalse%
\ blast\isanewline
\ \ \ \ \ \ \isacommand{show}\isamarkupfalse%
\ {\isachardoublequoteopen}graph{\isacharunderscore}{\kern0pt}morph\ f\ {\isacharcolon}{\kern0pt}\ graph\ f\ {\isasymrightarrow}\ X\ {\isasymtimes}\isactrlsub c\ Y{\isachardoublequoteclose}\isanewline
\ \ \ \ \ \ \ \ \isacommand{using}\isamarkupfalse%
\ graph{\isacharunderscore}{\kern0pt}subobj\ subobject{\isacharunderscore}{\kern0pt}of{\isacharunderscore}{\kern0pt}def{\isadigit{2}}\ \isacommand{by}\isamarkupfalse%
\ blast\isanewline
\ \ \ \ \ \ \isacommand{show}\isamarkupfalse%
\ {\isachardoublequoteopen}{\isasymlangle}x{\isacharcomma}{\kern0pt}y{\isasymrangle}\ factorsthru\ graph{\isacharunderscore}{\kern0pt}morph\ f{\isachardoublequoteclose}\isanewline
\ \ \ \ \ \ \isacommand{proof}\isamarkupfalse%
{\isacharparenleft}{\kern0pt}subst\ xfactorthru{\isacharunderscore}{\kern0pt}equalizer{\isacharunderscore}{\kern0pt}iff{\isacharunderscore}{\kern0pt}fx{\isacharunderscore}{\kern0pt}eq{\isacharunderscore}{\kern0pt}gx{\isacharbrackleft}{\kern0pt}\isakeyword{where}\ E\ {\isacharequal}{\kern0pt}\ {\isachardoublequoteopen}graph\ f{\isachardoublequoteclose}{\isacharcomma}{\kern0pt}\ \isakeyword{where}\ m\ {\isacharequal}{\kern0pt}\ {\isachardoublequoteopen}graph{\isacharunderscore}{\kern0pt}morph\ f{\isachardoublequoteclose}{\isacharcomma}{\kern0pt}\ \ \isanewline
\ \ \ \ \ \ \ \ \ \ \ \ \ \ \ \ \ \ \ \ \ \ \ \ \ \ \ \ \ \ \ \ \ \ \ \ \ \ \ \ \ \ \ \ \ \ \ \ \ \ \ \ \ \isakeyword{where}\ f\ {\isacharequal}{\kern0pt}\ {\isachardoublequoteopen}{\isacharparenleft}{\kern0pt}f\ {\isasymcirc}\isactrlsub c\ left{\isacharunderscore}{\kern0pt}cart{\isacharunderscore}{\kern0pt}proj\ X\ Y{\isacharparenright}{\kern0pt}{\isachardoublequoteclose}{\isacharcomma}{\kern0pt}\ \isakeyword{where}\ g\ {\isacharequal}{\kern0pt}\ {\isachardoublequoteopen}right{\isacharunderscore}{\kern0pt}cart{\isacharunderscore}{\kern0pt}proj\ X\ Y{\isachardoublequoteclose}{\isacharcomma}{\kern0pt}\ \isakeyword{where}\ X\ {\isacharequal}{\kern0pt}\ {\isachardoublequoteopen}X\ {\isasymtimes}\isactrlsub c\ Y{\isachardoublequoteclose}{\isacharcomma}{\kern0pt}\ \isakeyword{where}\ Y\ {\isacharequal}{\kern0pt}\ Y{\isacharcomma}{\kern0pt}\isanewline
\ \ \ \ \ \ \ \ \ \ \ \ \ \ \ \ \ \ \ \ \ \ \ \ \ \ \ \ \ \ \ \ \ \ \ \ \ \ \ \ \ \ \ \ \ \ \ \ \ \ \ \ \ \isakeyword{where}\ x\ {\isacharequal}{\kern0pt}{\isachardoublequoteopen}{\isasymlangle}x{\isacharcomma}{\kern0pt}y{\isasymrangle}{\isachardoublequoteclose}{\isacharbrackright}{\kern0pt}{\isacharparenright}{\kern0pt}\isanewline
\ \ \ \ \ \ \ \ \isacommand{show}\isamarkupfalse%
\ {\isachardoublequoteopen}f\ {\isasymcirc}\isactrlsub c\ left{\isacharunderscore}{\kern0pt}cart{\isacharunderscore}{\kern0pt}proj\ X\ Y\ {\isacharcolon}{\kern0pt}\ X\ {\isasymtimes}\isactrlsub c\ Y\ {\isasymrightarrow}\ Y{\isachardoublequoteclose}\isanewline
\ \ \ \ \ \ \ \ \ \ \isacommand{using}\isamarkupfalse%
\ assms\ \isacommand{by}\isamarkupfalse%
\ typecheck{\isacharunderscore}{\kern0pt}cfuncs\isanewline
\ \ \ \ \ \ \ \ \isacommand{show}\isamarkupfalse%
\ {\isachardoublequoteopen}right{\isacharunderscore}{\kern0pt}cart{\isacharunderscore}{\kern0pt}proj\ X\ Y\ {\isacharcolon}{\kern0pt}\ X\ {\isasymtimes}\isactrlsub c\ Y\ {\isasymrightarrow}\ Y{\isachardoublequoteclose}\isanewline
\ \ \ \ \ \ \ \ \ \ \isacommand{by}\isamarkupfalse%
\ \ typecheck{\isacharunderscore}{\kern0pt}cfuncs\isanewline
\ \ \ \ \ \ \ \ \isacommand{show}\isamarkupfalse%
\ {\isachardoublequoteopen}equalizer\ {\isacharparenleft}{\kern0pt}graph\ f{\isacharparenright}{\kern0pt}\ {\isacharparenleft}{\kern0pt}graph{\isacharunderscore}{\kern0pt}morph\ f{\isacharparenright}{\kern0pt}\ {\isacharparenleft}{\kern0pt}f\ {\isasymcirc}\isactrlsub c\ left{\isacharunderscore}{\kern0pt}cart{\isacharunderscore}{\kern0pt}proj\ X\ Y{\isacharparenright}{\kern0pt}\ {\isacharparenleft}{\kern0pt}right{\isacharunderscore}{\kern0pt}cart{\isacharunderscore}{\kern0pt}proj\ X\ Y{\isacharparenright}{\kern0pt}{\isachardoublequoteclose}\isanewline
\ \ \ \ \ \ \ \ \ \ \isacommand{by}\isamarkupfalse%
\ {\isacharparenleft}{\kern0pt}simp\ add{\isacharcolon}{\kern0pt}\ assms\ graph{\isacharunderscore}{\kern0pt}equalizer{\isadigit{4}}{\isacharparenright}{\kern0pt}\isanewline
\ \ \ \ \ \ \ \ \isacommand{show}\isamarkupfalse%
\ {\isachardoublequoteopen}{\isasymlangle}x{\isacharcomma}{\kern0pt}y{\isasymrangle}\ {\isasymin}\isactrlsub c\ X\ {\isasymtimes}\isactrlsub c\ Y{\isachardoublequoteclose}\isanewline
\ \ \ \ \ \ \ \ \ \ \isacommand{by}\isamarkupfalse%
\ typecheck{\isacharunderscore}{\kern0pt}cfuncs\isanewline
\ \ \ \ \ \ \ \ \isacommand{show}\isamarkupfalse%
\ {\isachardoublequoteopen}{\isacharparenleft}{\kern0pt}f\ {\isasymcirc}\isactrlsub c\ left{\isacharunderscore}{\kern0pt}cart{\isacharunderscore}{\kern0pt}proj\ X\ Y{\isacharparenright}{\kern0pt}\ {\isasymcirc}\isactrlsub c\ {\isasymlangle}x{\isacharcomma}{\kern0pt}y{\isasymrangle}\ {\isacharequal}{\kern0pt}\ right{\isacharunderscore}{\kern0pt}cart{\isacharunderscore}{\kern0pt}proj\ X\ Y\ {\isasymcirc}\isactrlsub c\ {\isasymlangle}x{\isacharcomma}{\kern0pt}y{\isasymrangle}{\isachardoublequoteclose}\isanewline
\ \ \ \ \ \ \ \ \ \ \isacommand{using}\isamarkupfalse%
\ assms\ \ \isanewline
\ \ \ \ \ \ \ \ \ \ \isacommand{by}\isamarkupfalse%
\ {\isacharparenleft}{\kern0pt}typecheck{\isacharunderscore}{\kern0pt}cfuncs{\isacharcomma}{\kern0pt}\ smt\ {\isacharparenleft}{\kern0pt}z{\isadigit{3}}{\isacharparenright}{\kern0pt}\ comp{\isacharunderscore}{\kern0pt}associative{\isadigit{2}}\ left{\isacharunderscore}{\kern0pt}cart{\isacharunderscore}{\kern0pt}proj{\isacharunderscore}{\kern0pt}cfunc{\isacharunderscore}{\kern0pt}prod\ right{\isacharunderscore}{\kern0pt}cart{\isacharunderscore}{\kern0pt}proj{\isacharunderscore}{\kern0pt}cfunc{\isacharunderscore}{\kern0pt}prod\ y{\isacharunderscore}{\kern0pt}def{\isacharparenright}{\kern0pt}\isanewline
\ \ \ \ \ \ \isacommand{qed}\isamarkupfalse%
\isanewline
\ \ \ \ \isacommand{qed}\isamarkupfalse%
\isanewline
\ \ \ \ \isacommand{then}\isamarkupfalse%
\ \isacommand{show}\isamarkupfalse%
\ {\isachardoublequoteopen}{\isasymexists}y{\isachardot}{\kern0pt}\ y\ {\isasymin}\isactrlsub c\ Y\ {\isasymand}\ {\isasymlangle}x{\isacharcomma}{\kern0pt}y{\isasymrangle}\ {\isasymin}\isactrlbsub X\ {\isasymtimes}\isactrlsub c\ Y\isactrlesub \ {\isacharparenleft}{\kern0pt}graph\ f{\isacharcomma}{\kern0pt}\ graph{\isacharunderscore}{\kern0pt}morph\ f{\isacharparenright}{\kern0pt}{\isachardoublequoteclose}\isanewline
\ \ \ \ \ \ \isacommand{using}\isamarkupfalse%
\ y{\isacharunderscore}{\kern0pt}type\ \isacommand{by}\isamarkupfalse%
\ blast\isanewline
\ \ \isacommand{qed}\isamarkupfalse%
\isanewline
\ \ \isacommand{show}\isamarkupfalse%
\ {\isachardoublequoteopen}{\isasymAnd}x\ y\ ya{\isachardot}{\kern0pt}\isanewline
\ \ \ \ \ \ \ x\ {\isasymin}\isactrlsub c\ X\ {\isasymLongrightarrow}\isanewline
\ \ \ \ \ \ \ y\ {\isasymin}\isactrlsub c\ Y\ {\isasymLongrightarrow}\isanewline
\ \ \ \ \ \ \ {\isasymlangle}x{\isacharcomma}{\kern0pt}y{\isasymrangle}\ {\isasymin}\isactrlbsub X\ {\isasymtimes}\isactrlsub c\ Y\isactrlesub \ {\isacharparenleft}{\kern0pt}graph\ f{\isacharcomma}{\kern0pt}\ graph{\isacharunderscore}{\kern0pt}morph\ f{\isacharparenright}{\kern0pt}\ {\isasymLongrightarrow}\ \isanewline
\ \ \ \ \ \ \ \ ya\ {\isasymin}\isactrlsub c\ Y\ {\isasymLongrightarrow}\ \isanewline
\ \ \ \ \ \ \ \ {\isasymlangle}x{\isacharcomma}{\kern0pt}ya{\isasymrangle}\ {\isasymin}\isactrlbsub X\ {\isasymtimes}\isactrlsub c\ Y\isactrlesub \ {\isacharparenleft}{\kern0pt}graph\ f{\isacharcomma}{\kern0pt}\ graph{\isacharunderscore}{\kern0pt}morph\ f{\isacharparenright}{\kern0pt}\isanewline
\ \ \ \ \ \ \ \ \ {\isasymLongrightarrow}\ y\ {\isacharequal}{\kern0pt}\ ya{\isachardoublequoteclose}\isanewline
\ \ \ \ \isacommand{using}\isamarkupfalse%
\ assms\ \ \isanewline
\ \ \ \ \isacommand{by}\isamarkupfalse%
\ {\isacharparenleft}{\kern0pt}smt\ {\isacharparenleft}{\kern0pt}z{\isadigit{3}}{\isacharparenright}{\kern0pt}\ comp{\isacharunderscore}{\kern0pt}associative{\isadigit{2}}\ equalizer{\isacharunderscore}{\kern0pt}def\ factors{\isacharunderscore}{\kern0pt}through{\isacharunderscore}{\kern0pt}def{\isadigit{2}}\ graph{\isacharunderscore}{\kern0pt}equalizer{\isadigit{4}}\ left{\isacharunderscore}{\kern0pt}cart{\isacharunderscore}{\kern0pt}proj{\isacharunderscore}{\kern0pt}cfunc{\isacharunderscore}{\kern0pt}prod\ left{\isacharunderscore}{\kern0pt}cart{\isacharunderscore}{\kern0pt}proj{\isacharunderscore}{\kern0pt}type\ relative{\isacharunderscore}{\kern0pt}member{\isacharunderscore}{\kern0pt}def{\isadigit{2}}\ right{\isacharunderscore}{\kern0pt}cart{\isacharunderscore}{\kern0pt}proj{\isacharunderscore}{\kern0pt}cfunc{\isacharunderscore}{\kern0pt}prod{\isacharparenright}{\kern0pt}\isanewline
\isacommand{qed}\isamarkupfalse%
%
\endisatagproof
{\isafoldproof}%
%
\isadelimproof
\isanewline
%
\endisadelimproof
\isanewline
\isacommand{lemma}\isamarkupfalse%
\ functional{\isacharunderscore}{\kern0pt}on{\isacharunderscore}{\kern0pt}isomorphism{\isacharcolon}{\kern0pt}\isanewline
\ \ \isakeyword{assumes}\ {\isachardoublequoteopen}functional{\isacharunderscore}{\kern0pt}on\ X\ Y\ {\isacharparenleft}{\kern0pt}R{\isacharcomma}{\kern0pt}m{\isacharparenright}{\kern0pt}{\isachardoublequoteclose}\isanewline
\ \ \isakeyword{shows}\ {\isachardoublequoteopen}isomorphism{\isacharparenleft}{\kern0pt}left{\isacharunderscore}{\kern0pt}cart{\isacharunderscore}{\kern0pt}proj\ X\ Y\ {\isasymcirc}\isactrlsub c\ m{\isacharparenright}{\kern0pt}{\isachardoublequoteclose}\isanewline
%
\isadelimproof
%
\endisadelimproof
%
\isatagproof
\isacommand{proof}\isamarkupfalse%
{\isacharminus}{\kern0pt}\isanewline
\ \ \isacommand{have}\isamarkupfalse%
\ m{\isacharunderscore}{\kern0pt}mono{\isacharcolon}{\kern0pt}\ {\isachardoublequoteopen}monomorphism{\isacharparenleft}{\kern0pt}m{\isacharparenright}{\kern0pt}{\isachardoublequoteclose}\isanewline
\ \ \ \ \isacommand{using}\isamarkupfalse%
\ assms\ functional{\isacharunderscore}{\kern0pt}on{\isacharunderscore}{\kern0pt}def\ subobject{\isacharunderscore}{\kern0pt}of{\isacharunderscore}{\kern0pt}def{\isadigit{2}}\ \isacommand{by}\isamarkupfalse%
\ blast\isanewline
\ \ \isacommand{have}\isamarkupfalse%
\ pi{\isadigit{0}}{\isacharunderscore}{\kern0pt}m{\isacharunderscore}{\kern0pt}type{\isacharbrackleft}{\kern0pt}type{\isacharunderscore}{\kern0pt}rule{\isacharbrackright}{\kern0pt}{\isacharcolon}{\kern0pt}\ {\isachardoublequoteopen}left{\isacharunderscore}{\kern0pt}cart{\isacharunderscore}{\kern0pt}proj\ X\ Y\ {\isasymcirc}\isactrlsub c\ m\ {\isacharcolon}{\kern0pt}\ R\ {\isasymrightarrow}\ X{\isachardoublequoteclose}\isanewline
\ \ \ \ \isacommand{using}\isamarkupfalse%
\ assms\ functional{\isacharunderscore}{\kern0pt}on{\isacharunderscore}{\kern0pt}def\ subobject{\isacharunderscore}{\kern0pt}of{\isacharunderscore}{\kern0pt}def{\isadigit{2}}\ \isacommand{by}\isamarkupfalse%
\ {\isacharparenleft}{\kern0pt}typecheck{\isacharunderscore}{\kern0pt}cfuncs{\isacharcomma}{\kern0pt}\ blast{\isacharparenright}{\kern0pt}\isanewline
\ \ \isacommand{have}\isamarkupfalse%
\ surj{\isacharcolon}{\kern0pt}\ {\isachardoublequoteopen}surjective{\isacharparenleft}{\kern0pt}left{\isacharunderscore}{\kern0pt}cart{\isacharunderscore}{\kern0pt}proj\ X\ Y\ {\isasymcirc}\isactrlsub c\ m{\isacharparenright}{\kern0pt}{\isachardoublequoteclose}\isanewline
\ \ \isacommand{proof}\isamarkupfalse%
{\isacharparenleft}{\kern0pt}unfold\ surjective{\isacharunderscore}{\kern0pt}def{\isacharcomma}{\kern0pt}\ auto{\isacharparenright}{\kern0pt}\isanewline
\ \ \ \ \isacommand{fix}\isamarkupfalse%
\ x\ \isanewline
\ \ \ \ \isacommand{assume}\isamarkupfalse%
\ {\isachardoublequoteopen}x\ {\isasymin}\isactrlsub c\ codomain\ {\isacharparenleft}{\kern0pt}left{\isacharunderscore}{\kern0pt}cart{\isacharunderscore}{\kern0pt}proj\ X\ Y\ {\isasymcirc}\isactrlsub c\ m{\isacharparenright}{\kern0pt}{\isachardoublequoteclose}\isanewline
\ \ \ \ \isacommand{then}\isamarkupfalse%
\ \isacommand{have}\isamarkupfalse%
\ {\isacharbrackleft}{\kern0pt}type{\isacharunderscore}{\kern0pt}rule{\isacharbrackright}{\kern0pt}{\isacharcolon}{\kern0pt}\ {\isachardoublequoteopen}x\ {\isasymin}\isactrlsub c\ X{\isachardoublequoteclose}\isanewline
\ \ \ \ \ \ \isacommand{using}\isamarkupfalse%
\ cfunc{\isacharunderscore}{\kern0pt}type{\isacharunderscore}{\kern0pt}def\ pi{\isadigit{0}}{\isacharunderscore}{\kern0pt}m{\isacharunderscore}{\kern0pt}type\ \isacommand{by}\isamarkupfalse%
\ force\isanewline
\ \ \ \ \isacommand{then}\isamarkupfalse%
\ \isacommand{have}\isamarkupfalse%
\ {\isachardoublequoteopen}{\isasymexists}{\isacharbang}{\kern0pt}\ y{\isachardot}{\kern0pt}\ {\isacharparenleft}{\kern0pt}y\ {\isasymin}\isactrlsub c\ Y\ {\isasymand}\ \ {\isasymlangle}x{\isacharcomma}{\kern0pt}y{\isasymrangle}\ {\isasymin}\isactrlbsub X{\isasymtimes}\isactrlsub cY\isactrlesub \ {\isacharparenleft}{\kern0pt}R{\isacharcomma}{\kern0pt}m{\isacharparenright}{\kern0pt}{\isacharparenright}{\kern0pt}{\isachardoublequoteclose}\isanewline
\ \ \ \ \ \ \isacommand{using}\isamarkupfalse%
\ assms\ functional{\isacharunderscore}{\kern0pt}on{\isacharunderscore}{\kern0pt}def\ \ \isacommand{by}\isamarkupfalse%
\ force\isanewline
\ \ \ \ \isacommand{then}\isamarkupfalse%
\ \isacommand{show}\isamarkupfalse%
\ {\isachardoublequoteopen}{\isasymexists}z{\isachardot}{\kern0pt}\ z\ {\isasymin}\isactrlsub c\ domain\ {\isacharparenleft}{\kern0pt}left{\isacharunderscore}{\kern0pt}cart{\isacharunderscore}{\kern0pt}proj\ X\ Y\ {\isasymcirc}\isactrlsub c\ m{\isacharparenright}{\kern0pt}\ {\isasymand}\ {\isacharparenleft}{\kern0pt}left{\isacharunderscore}{\kern0pt}cart{\isacharunderscore}{\kern0pt}proj\ X\ Y\ {\isasymcirc}\isactrlsub c\ m{\isacharparenright}{\kern0pt}\ {\isasymcirc}\isactrlsub c\ z\ {\isacharequal}{\kern0pt}\ x{\isachardoublequoteclose}\isanewline
\ \ \ \ \ \ \isacommand{by}\isamarkupfalse%
\ {\isacharparenleft}{\kern0pt}typecheck{\isacharunderscore}{\kern0pt}cfuncs{\isacharcomma}{\kern0pt}\ smt\ {\isacharparenleft}{\kern0pt}verit{\isacharcomma}{\kern0pt}\ best{\isacharparenright}{\kern0pt}\ cfunc{\isacharunderscore}{\kern0pt}type{\isacharunderscore}{\kern0pt}def\ comp{\isacharunderscore}{\kern0pt}associative\ factors{\isacharunderscore}{\kern0pt}through{\isacharunderscore}{\kern0pt}def{\isadigit{2}}\ left{\isacharunderscore}{\kern0pt}cart{\isacharunderscore}{\kern0pt}proj{\isacharunderscore}{\kern0pt}cfunc{\isacharunderscore}{\kern0pt}prod\ relative{\isacharunderscore}{\kern0pt}member{\isacharunderscore}{\kern0pt}def{\isadigit{2}}{\isacharparenright}{\kern0pt}\isanewline
\ \ \isacommand{qed}\isamarkupfalse%
\isanewline
\ \ \isacommand{have}\isamarkupfalse%
\ inj{\isacharcolon}{\kern0pt}\ {\isachardoublequoteopen}injective{\isacharparenleft}{\kern0pt}left{\isacharunderscore}{\kern0pt}cart{\isacharunderscore}{\kern0pt}proj\ X\ Y\ {\isasymcirc}\isactrlsub c\ m{\isacharparenright}{\kern0pt}{\isachardoublequoteclose}\isanewline
\ \ \isacommand{proof}\isamarkupfalse%
{\isacharparenleft}{\kern0pt}unfold\ injective{\isacharunderscore}{\kern0pt}def{\isacharcomma}{\kern0pt}\ auto{\isacharparenright}{\kern0pt}\isanewline
\ \ \ \ \isacommand{fix}\isamarkupfalse%
\ r{\isadigit{1}}\ r{\isadigit{2}}\ \isanewline
\ \ \ \ \isacommand{assume}\isamarkupfalse%
\ {\isachardoublequoteopen}r{\isadigit{1}}\ {\isasymin}\isactrlsub c\ domain\ {\isacharparenleft}{\kern0pt}left{\isacharunderscore}{\kern0pt}cart{\isacharunderscore}{\kern0pt}proj\ X\ Y\ {\isasymcirc}\isactrlsub c\ m{\isacharparenright}{\kern0pt}{\isachardoublequoteclose}\ \isacommand{then}\isamarkupfalse%
\ \isacommand{have}\isamarkupfalse%
\ r{\isadigit{1}}{\isacharunderscore}{\kern0pt}type{\isacharbrackleft}{\kern0pt}type{\isacharunderscore}{\kern0pt}rule{\isacharbrackright}{\kern0pt}{\isacharcolon}{\kern0pt}\ {\isachardoublequoteopen}r{\isadigit{1}}\ {\isasymin}\isactrlsub c\ R{\isachardoublequoteclose}\isanewline
\ \ \ \ \ \ \isacommand{by}\isamarkupfalse%
\ {\isacharparenleft}{\kern0pt}metis\ cfunc{\isacharunderscore}{\kern0pt}type{\isacharunderscore}{\kern0pt}def\ pi{\isadigit{0}}{\isacharunderscore}{\kern0pt}m{\isacharunderscore}{\kern0pt}type{\isacharparenright}{\kern0pt}\isanewline
\ \ \ \ \isacommand{assume}\isamarkupfalse%
\ {\isachardoublequoteopen}r{\isadigit{2}}\ {\isasymin}\isactrlsub c\ domain\ {\isacharparenleft}{\kern0pt}left{\isacharunderscore}{\kern0pt}cart{\isacharunderscore}{\kern0pt}proj\ X\ Y\ {\isasymcirc}\isactrlsub c\ m{\isacharparenright}{\kern0pt}{\isachardoublequoteclose}\ \isacommand{then}\isamarkupfalse%
\ \isacommand{have}\isamarkupfalse%
\ r{\isadigit{2}}{\isacharunderscore}{\kern0pt}type{\isacharbrackleft}{\kern0pt}type{\isacharunderscore}{\kern0pt}rule{\isacharbrackright}{\kern0pt}{\isacharcolon}{\kern0pt}\ {\isachardoublequoteopen}r{\isadigit{2}}\ {\isasymin}\isactrlsub c\ R{\isachardoublequoteclose}\isanewline
\ \ \ \ \ \ \isacommand{by}\isamarkupfalse%
\ {\isacharparenleft}{\kern0pt}metis\ cfunc{\isacharunderscore}{\kern0pt}type{\isacharunderscore}{\kern0pt}def\ pi{\isadigit{0}}{\isacharunderscore}{\kern0pt}m{\isacharunderscore}{\kern0pt}type{\isacharparenright}{\kern0pt}\isanewline
\ \ \ \ \isacommand{assume}\isamarkupfalse%
\ {\isachardoublequoteopen}{\isacharparenleft}{\kern0pt}left{\isacharunderscore}{\kern0pt}cart{\isacharunderscore}{\kern0pt}proj\ X\ Y\ {\isasymcirc}\isactrlsub c\ m{\isacharparenright}{\kern0pt}\ {\isasymcirc}\isactrlsub c\ r{\isadigit{1}}\ {\isacharequal}{\kern0pt}\ {\isacharparenleft}{\kern0pt}left{\isacharunderscore}{\kern0pt}cart{\isacharunderscore}{\kern0pt}proj\ X\ Y\ {\isasymcirc}\isactrlsub c\ m{\isacharparenright}{\kern0pt}\ {\isasymcirc}\isactrlsub c\ r{\isadigit{2}}{\isachardoublequoteclose}\isanewline
\ \ \ \ \isacommand{then}\isamarkupfalse%
\ \isacommand{have}\isamarkupfalse%
\ eq{\isacharcolon}{\kern0pt}\ {\isachardoublequoteopen}left{\isacharunderscore}{\kern0pt}cart{\isacharunderscore}{\kern0pt}proj\ X\ Y\ {\isasymcirc}\isactrlsub c\ m\ {\isasymcirc}\isactrlsub c\ r{\isadigit{1}}\ {\isacharequal}{\kern0pt}\ left{\isacharunderscore}{\kern0pt}cart{\isacharunderscore}{\kern0pt}proj\ X\ Y\ {\isasymcirc}\isactrlsub c\ m\ {\isasymcirc}\isactrlsub c\ r{\isadigit{2}}{\isachardoublequoteclose}\isanewline
\ \ \ \ \ \ \isacommand{using}\isamarkupfalse%
\ assms\ cfunc{\isacharunderscore}{\kern0pt}type{\isacharunderscore}{\kern0pt}def\ comp{\isacharunderscore}{\kern0pt}associative\ functional{\isacharunderscore}{\kern0pt}on{\isacharunderscore}{\kern0pt}def\ subobject{\isacharunderscore}{\kern0pt}of{\isacharunderscore}{\kern0pt}def{\isadigit{2}}\ \isacommand{by}\isamarkupfalse%
\ {\isacharparenleft}{\kern0pt}typecheck{\isacharunderscore}{\kern0pt}cfuncs{\isacharcomma}{\kern0pt}\ auto{\isacharparenright}{\kern0pt}\isanewline
\ \ \ \ \isacommand{have}\isamarkupfalse%
\ mx{\isacharunderscore}{\kern0pt}type{\isacharbrackleft}{\kern0pt}type{\isacharunderscore}{\kern0pt}rule{\isacharbrackright}{\kern0pt}{\isacharcolon}{\kern0pt}\ {\isachardoublequoteopen}m\ {\isasymcirc}\isactrlsub c\ r{\isadigit{1}}\ {\isasymin}\isactrlsub c\ X{\isasymtimes}\isactrlsub cY{\isachardoublequoteclose}\isanewline
\ \ \ \ \ \ \isacommand{using}\isamarkupfalse%
\ assms\ functional{\isacharunderscore}{\kern0pt}on{\isacharunderscore}{\kern0pt}def\ subobject{\isacharunderscore}{\kern0pt}of{\isacharunderscore}{\kern0pt}def{\isadigit{2}}\ \isacommand{by}\isamarkupfalse%
\ {\isacharparenleft}{\kern0pt}typecheck{\isacharunderscore}{\kern0pt}cfuncs{\isacharcomma}{\kern0pt}\ blast{\isacharparenright}{\kern0pt}\isanewline
\ \ \ \ \isacommand{then}\isamarkupfalse%
\ \isacommand{obtain}\isamarkupfalse%
\ x{\isadigit{1}}\ \isakeyword{and}\ y{\isadigit{1}}\ \isakeyword{where}\ m{\isadigit{1}}r{\isadigit{1}}{\isacharunderscore}{\kern0pt}eqs{\isacharcolon}{\kern0pt}\ {\isachardoublequoteopen}m\ {\isasymcirc}\isactrlsub c\ r{\isadigit{1}}\ {\isacharequal}{\kern0pt}\ {\isasymlangle}x{\isadigit{1}}{\isacharcomma}{\kern0pt}\ y{\isadigit{1}}{\isasymrangle}\ {\isasymand}\ x{\isadigit{1}}\ {\isasymin}\isactrlsub c\ X\ {\isasymand}\ y{\isadigit{1}}\ {\isasymin}\isactrlsub c\ Y{\isachardoublequoteclose}\isanewline
\ \ \ \ \ \ \isacommand{using}\isamarkupfalse%
\ cart{\isacharunderscore}{\kern0pt}prod{\isacharunderscore}{\kern0pt}decomp\ \isacommand{by}\isamarkupfalse%
\ presburger\isanewline
\ \ \ \ \isacommand{have}\isamarkupfalse%
\ my{\isacharunderscore}{\kern0pt}type{\isacharbrackleft}{\kern0pt}type{\isacharunderscore}{\kern0pt}rule{\isacharbrackright}{\kern0pt}{\isacharcolon}{\kern0pt}\ {\isachardoublequoteopen}m\ {\isasymcirc}\isactrlsub c\ r{\isadigit{2}}\ {\isasymin}\isactrlsub c\ X{\isasymtimes}\isactrlsub cY{\isachardoublequoteclose}\isanewline
\ \ \ \ \ \ \isacommand{using}\isamarkupfalse%
\ assms\ functional{\isacharunderscore}{\kern0pt}on{\isacharunderscore}{\kern0pt}def\ subobject{\isacharunderscore}{\kern0pt}of{\isacharunderscore}{\kern0pt}def{\isadigit{2}}\ \isacommand{by}\isamarkupfalse%
\ {\isacharparenleft}{\kern0pt}typecheck{\isacharunderscore}{\kern0pt}cfuncs{\isacharcomma}{\kern0pt}\ blast{\isacharparenright}{\kern0pt}\isanewline
\ \ \ \ \isacommand{then}\isamarkupfalse%
\ \isacommand{obtain}\isamarkupfalse%
\ x{\isadigit{2}}\ \isakeyword{and}\ y{\isadigit{2}}\ \isakeyword{where}\ m{\isadigit{2}}r{\isadigit{2}}{\isacharunderscore}{\kern0pt}eqs{\isacharcolon}{\kern0pt}{\isachardoublequoteopen}m\ {\isasymcirc}\isactrlsub c\ r{\isadigit{2}}\ {\isacharequal}{\kern0pt}\ {\isasymlangle}x{\isadigit{2}}{\isacharcomma}{\kern0pt}\ y{\isadigit{2}}{\isasymrangle}\ {\isasymand}\ x{\isadigit{2}}\ {\isasymin}\isactrlsub c\ X\ {\isasymand}\ y{\isadigit{2}}\ {\isasymin}\isactrlsub c\ Y{\isachardoublequoteclose}\isanewline
\ \ \ \ \ \ \isacommand{using}\isamarkupfalse%
\ cart{\isacharunderscore}{\kern0pt}prod{\isacharunderscore}{\kern0pt}decomp\ \isacommand{by}\isamarkupfalse%
\ presburger\isanewline
\ \ \ \ \isacommand{have}\isamarkupfalse%
\ x{\isacharunderscore}{\kern0pt}equal{\isacharcolon}{\kern0pt}\ {\isachardoublequoteopen}x{\isadigit{1}}\ {\isacharequal}{\kern0pt}\ x{\isadigit{2}}{\isachardoublequoteclose}\isanewline
\ \ \ \ \ \ \isacommand{using}\isamarkupfalse%
\ eq\ left{\isacharunderscore}{\kern0pt}cart{\isacharunderscore}{\kern0pt}proj{\isacharunderscore}{\kern0pt}cfunc{\isacharunderscore}{\kern0pt}prod\ m{\isadigit{1}}r{\isadigit{1}}{\isacharunderscore}{\kern0pt}eqs\ m{\isadigit{2}}r{\isadigit{2}}{\isacharunderscore}{\kern0pt}eqs\ \isacommand{by}\isamarkupfalse%
\ force\isanewline
\ \ \ \ \isacommand{have}\isamarkupfalse%
\ functional{\isacharcolon}{\kern0pt}\ {\isachardoublequoteopen}{\isasymexists}{\isacharbang}{\kern0pt}\ y{\isachardot}{\kern0pt}\ {\isacharparenleft}{\kern0pt}y\ {\isasymin}\isactrlsub c\ Y\ {\isasymand}\ \ {\isasymlangle}x{\isadigit{1}}{\isacharcomma}{\kern0pt}y{\isasymrangle}\ {\isasymin}\isactrlbsub X{\isasymtimes}\isactrlsub cY\isactrlesub \ {\isacharparenleft}{\kern0pt}R{\isacharcomma}{\kern0pt}m{\isacharparenright}{\kern0pt}{\isacharparenright}{\kern0pt}{\isachardoublequoteclose}\isanewline
\ \ \ \ \ \ \isacommand{using}\isamarkupfalse%
\ assms\ functional{\isacharunderscore}{\kern0pt}on{\isacharunderscore}{\kern0pt}def\ m{\isadigit{1}}r{\isadigit{1}}{\isacharunderscore}{\kern0pt}eqs\ \isacommand{by}\isamarkupfalse%
\ force\isanewline
\ \ \ \ \isacommand{then}\isamarkupfalse%
\ \isacommand{have}\isamarkupfalse%
\ y{\isacharunderscore}{\kern0pt}equal{\isacharcolon}{\kern0pt}\ {\isachardoublequoteopen}y{\isadigit{1}}\ {\isacharequal}{\kern0pt}\ y{\isadigit{2}}{\isachardoublequoteclose}\isanewline
\ \ \ \ \ \ \isacommand{by}\isamarkupfalse%
\ {\isacharparenleft}{\kern0pt}metis\ prod{\isachardot}{\kern0pt}sel\ factors{\isacharunderscore}{\kern0pt}through{\isacharunderscore}{\kern0pt}def{\isadigit{2}}\ m{\isadigit{1}}r{\isadigit{1}}{\isacharunderscore}{\kern0pt}eqs\ m{\isadigit{2}}r{\isadigit{2}}{\isacharunderscore}{\kern0pt}eqs\ mx{\isacharunderscore}{\kern0pt}type\ my{\isacharunderscore}{\kern0pt}type\ r{\isadigit{1}}{\isacharunderscore}{\kern0pt}type\ r{\isadigit{2}}{\isacharunderscore}{\kern0pt}type\ relative{\isacharunderscore}{\kern0pt}member{\isacharunderscore}{\kern0pt}def\ x{\isacharunderscore}{\kern0pt}equal{\isacharparenright}{\kern0pt}\isanewline
\ \ \ \ \isacommand{then}\isamarkupfalse%
\ \isacommand{show}\isamarkupfalse%
\ {\isachardoublequoteopen}r{\isadigit{1}}\ {\isacharequal}{\kern0pt}\ r{\isadigit{2}}{\isachardoublequoteclose}\isanewline
\ \ \ \ \ \ \isacommand{by}\isamarkupfalse%
\ {\isacharparenleft}{\kern0pt}metis\ functional\ cfunc{\isacharunderscore}{\kern0pt}type{\isacharunderscore}{\kern0pt}def\ m{\isadigit{1}}r{\isadigit{1}}{\isacharunderscore}{\kern0pt}eqs\ m{\isadigit{2}}r{\isadigit{2}}{\isacharunderscore}{\kern0pt}eqs\ monomorphism{\isacharunderscore}{\kern0pt}def\ r{\isadigit{1}}{\isacharunderscore}{\kern0pt}type\ r{\isadigit{2}}{\isacharunderscore}{\kern0pt}type\ relative{\isacharunderscore}{\kern0pt}member{\isacharunderscore}{\kern0pt}def{\isadigit{2}}\ x{\isacharunderscore}{\kern0pt}equal{\isacharparenright}{\kern0pt}\isanewline
\ \ \isacommand{qed}\isamarkupfalse%
\isanewline
\ \ \isacommand{show}\isamarkupfalse%
\ {\isachardoublequoteopen}isomorphism{\isacharparenleft}{\kern0pt}left{\isacharunderscore}{\kern0pt}cart{\isacharunderscore}{\kern0pt}proj\ X\ Y\ {\isasymcirc}\isactrlsub c\ m{\isacharparenright}{\kern0pt}{\isachardoublequoteclose}\isanewline
\ \ \ \ \isacommand{by}\isamarkupfalse%
\ {\isacharparenleft}{\kern0pt}metis\ epi{\isacharunderscore}{\kern0pt}mon{\isacharunderscore}{\kern0pt}is{\isacharunderscore}{\kern0pt}iso\ inj\ injective{\isacharunderscore}{\kern0pt}imp{\isacharunderscore}{\kern0pt}monomorphism\ surj\ surjective{\isacharunderscore}{\kern0pt}is{\isacharunderscore}{\kern0pt}epimorphism{\isacharparenright}{\kern0pt}\isanewline
\isacommand{qed}\isamarkupfalse%
%
\endisatagproof
{\isafoldproof}%
%
\isadelimproof
%
\endisadelimproof
%
\begin{isamarkuptext}%
The lemma below corresponds to Proposition 2.3.14 in Halvorson.%
\end{isamarkuptext}\isamarkuptrue%
\isacommand{lemma}\isamarkupfalse%
\ functional{\isacharunderscore}{\kern0pt}relations{\isacharunderscore}{\kern0pt}are{\isacharunderscore}{\kern0pt}graphs{\isacharcolon}{\kern0pt}\isanewline
\ \ \isakeyword{assumes}\ {\isachardoublequoteopen}functional{\isacharunderscore}{\kern0pt}on\ X\ Y\ {\isacharparenleft}{\kern0pt}R{\isacharcomma}{\kern0pt}m{\isacharparenright}{\kern0pt}{\isachardoublequoteclose}\isanewline
\ \ \isakeyword{shows}\ {\isachardoublequoteopen}{\isasymexists}{\isacharbang}{\kern0pt}\ f{\isachardot}{\kern0pt}\ f\ {\isacharcolon}{\kern0pt}\ X\ {\isasymrightarrow}\ Y\ {\isasymand}\ \isanewline
\ \ \ \ {\isacharparenleft}{\kern0pt}{\isasymexists}\ i{\isachardot}{\kern0pt}\ i\ {\isacharcolon}{\kern0pt}\ R\ {\isasymrightarrow}\ graph{\isacharparenleft}{\kern0pt}f{\isacharparenright}{\kern0pt}\ {\isasymand}\ isomorphism{\isacharparenleft}{\kern0pt}i{\isacharparenright}{\kern0pt}\ {\isasymand}\ m\ {\isacharequal}{\kern0pt}\ graph{\isacharunderscore}{\kern0pt}morph{\isacharparenleft}{\kern0pt}f{\isacharparenright}{\kern0pt}\ {\isasymcirc}\isactrlsub c\ i{\isacharparenright}{\kern0pt}{\isachardoublequoteclose}\isanewline
%
\isadelimproof
%
\endisadelimproof
%
\isatagproof
\isacommand{proof}\isamarkupfalse%
\ auto\isanewline
\ \ \isacommand{have}\isamarkupfalse%
\ m{\isacharunderscore}{\kern0pt}type{\isacharbrackleft}{\kern0pt}type{\isacharunderscore}{\kern0pt}rule{\isacharbrackright}{\kern0pt}{\isacharcolon}{\kern0pt}\ {\isachardoublequoteopen}m\ {\isacharcolon}{\kern0pt}\ R\ {\isasymrightarrow}\ X\ {\isasymtimes}\isactrlsub c\ Y{\isachardoublequoteclose}\isanewline
\ \ \ \ \isacommand{using}\isamarkupfalse%
\ assms\ \isacommand{unfolding}\isamarkupfalse%
\ functional{\isacharunderscore}{\kern0pt}on{\isacharunderscore}{\kern0pt}def\ subobject{\isacharunderscore}{\kern0pt}of{\isacharunderscore}{\kern0pt}def{\isadigit{2}}\ \isacommand{by}\isamarkupfalse%
\ auto\isanewline
\ \ \isacommand{have}\isamarkupfalse%
\ m{\isacharunderscore}{\kern0pt}mono{\isacharbrackleft}{\kern0pt}type{\isacharunderscore}{\kern0pt}rule{\isacharbrackright}{\kern0pt}{\isacharcolon}{\kern0pt}\ {\isachardoublequoteopen}monomorphism{\isacharparenleft}{\kern0pt}m{\isacharparenright}{\kern0pt}{\isachardoublequoteclose}\isanewline
\ \ \ \ \isacommand{using}\isamarkupfalse%
\ assms\ functional{\isacharunderscore}{\kern0pt}on{\isacharunderscore}{\kern0pt}def\ subobject{\isacharunderscore}{\kern0pt}of{\isacharunderscore}{\kern0pt}def{\isadigit{2}}\ \isacommand{by}\isamarkupfalse%
\ blast\isanewline
\ \ \isacommand{have}\isamarkupfalse%
\ isomorphism{\isacharbrackleft}{\kern0pt}type{\isacharunderscore}{\kern0pt}rule{\isacharbrackright}{\kern0pt}{\isacharcolon}{\kern0pt}\ {\isachardoublequoteopen}isomorphism{\isacharparenleft}{\kern0pt}left{\isacharunderscore}{\kern0pt}cart{\isacharunderscore}{\kern0pt}proj\ X\ Y\ {\isasymcirc}\isactrlsub c\ m{\isacharparenright}{\kern0pt}{\isachardoublequoteclose}\isanewline
\ \ \ \ \isacommand{using}\isamarkupfalse%
\ assms\ functional{\isacharunderscore}{\kern0pt}on{\isacharunderscore}{\kern0pt}isomorphism\ \isacommand{by}\isamarkupfalse%
\ force\isanewline
\ \ \isanewline
\ \ \isacommand{obtain}\isamarkupfalse%
\ h\ \isakeyword{where}\ h{\isacharunderscore}{\kern0pt}type{\isacharbrackleft}{\kern0pt}type{\isacharunderscore}{\kern0pt}rule{\isacharbrackright}{\kern0pt}{\isacharcolon}{\kern0pt}\ {\isachardoublequoteopen}h{\isacharcolon}{\kern0pt}\ X\ {\isasymrightarrow}\ R{\isachardoublequoteclose}\ \isakeyword{and}\ h{\isacharunderscore}{\kern0pt}def{\isacharcolon}{\kern0pt}\ {\isachardoublequoteopen}h\ {\isacharequal}{\kern0pt}\ {\isacharparenleft}{\kern0pt}left{\isacharunderscore}{\kern0pt}cart{\isacharunderscore}{\kern0pt}proj\ X\ Y\ {\isasymcirc}\isactrlsub c\ m{\isacharparenright}{\kern0pt}\isactrlbold {\isasyminverse}{\isachardoublequoteclose}\isanewline
\ \ \ \ \isacommand{by}\isamarkupfalse%
\ typecheck{\isacharunderscore}{\kern0pt}cfuncs\isanewline
\ \ \isacommand{obtain}\isamarkupfalse%
\ f\ \isakeyword{where}\ f{\isacharunderscore}{\kern0pt}def{\isacharcolon}{\kern0pt}\ {\isachardoublequoteopen}f\ {\isacharequal}{\kern0pt}\ {\isacharparenleft}{\kern0pt}right{\isacharunderscore}{\kern0pt}cart{\isacharunderscore}{\kern0pt}proj\ X\ Y{\isacharparenright}{\kern0pt}\ {\isasymcirc}\isactrlsub c\ m\ {\isasymcirc}\isactrlsub c\ h{\isachardoublequoteclose}\isanewline
\ \ \ \ \isacommand{by}\isamarkupfalse%
\ auto\isanewline
\ \ \isacommand{then}\isamarkupfalse%
\ \isacommand{have}\isamarkupfalse%
\ f{\isacharunderscore}{\kern0pt}type{\isacharbrackleft}{\kern0pt}type{\isacharunderscore}{\kern0pt}rule{\isacharbrackright}{\kern0pt}{\isacharcolon}{\kern0pt}\ {\isachardoublequoteopen}f\ {\isacharcolon}{\kern0pt}\ X\ {\isasymrightarrow}\ Y{\isachardoublequoteclose}\isanewline
\ \ \ \ \isacommand{by}\isamarkupfalse%
\ {\isacharparenleft}{\kern0pt}metis\ assms\ comp{\isacharunderscore}{\kern0pt}type\ f{\isacharunderscore}{\kern0pt}def\ functional{\isacharunderscore}{\kern0pt}on{\isacharunderscore}{\kern0pt}def\ h{\isacharunderscore}{\kern0pt}type\ right{\isacharunderscore}{\kern0pt}cart{\isacharunderscore}{\kern0pt}proj{\isacharunderscore}{\kern0pt}type\ subobject{\isacharunderscore}{\kern0pt}of{\isacharunderscore}{\kern0pt}def{\isadigit{2}}{\isacharparenright}{\kern0pt}\isanewline
\isanewline
\ \ \isacommand{have}\isamarkupfalse%
\ eq{\isacharcolon}{\kern0pt}\ {\isachardoublequoteopen}f\ {\isasymcirc}\isactrlsub c\ left{\isacharunderscore}{\kern0pt}cart{\isacharunderscore}{\kern0pt}proj\ X\ Y\ {\isasymcirc}\isactrlsub c\ m\ {\isacharequal}{\kern0pt}\ right{\isacharunderscore}{\kern0pt}cart{\isacharunderscore}{\kern0pt}proj\ X\ Y\ {\isasymcirc}\isactrlsub c\ m{\isachardoublequoteclose}\isanewline
\ \ \ \ \isacommand{unfolding}\isamarkupfalse%
\ f{\isacharunderscore}{\kern0pt}def\ h{\isacharunderscore}{\kern0pt}def\ \isacommand{by}\isamarkupfalse%
\ {\isacharparenleft}{\kern0pt}typecheck{\isacharunderscore}{\kern0pt}cfuncs{\isacharcomma}{\kern0pt}\ smt\ comp{\isacharunderscore}{\kern0pt}associative{\isadigit{2}}\ id{\isacharunderscore}{\kern0pt}right{\isacharunderscore}{\kern0pt}unit{\isadigit{2}}\ inv{\isacharunderscore}{\kern0pt}left\ isomorphism{\isacharparenright}{\kern0pt}\isanewline
\isanewline
\ \ \isacommand{show}\isamarkupfalse%
\ {\isachardoublequoteopen}{\isasymexists}f{\isachardot}{\kern0pt}\ f\ {\isacharcolon}{\kern0pt}\ X\ {\isasymrightarrow}\ Y\ {\isasymand}\ {\isacharparenleft}{\kern0pt}{\isasymexists}i{\isachardot}{\kern0pt}\ i\ {\isacharcolon}{\kern0pt}\ R\ {\isasymrightarrow}\ graph\ f\ {\isasymand}\ isomorphism\ i\ {\isasymand}\ m\ {\isacharequal}{\kern0pt}\ graph{\isacharunderscore}{\kern0pt}morph\ f\ {\isasymcirc}\isactrlsub c\ i{\isacharparenright}{\kern0pt}{\isachardoublequoteclose}\isanewline
\ \ \isacommand{proof}\isamarkupfalse%
\ {\isacharparenleft}{\kern0pt}rule{\isacharunderscore}{\kern0pt}tac\ x{\isacharequal}{\kern0pt}f\ \isakeyword{in}\ exI{\isacharcomma}{\kern0pt}\ auto{\isacharcomma}{\kern0pt}\ typecheck{\isacharunderscore}{\kern0pt}cfuncs{\isacharparenright}{\kern0pt}\isanewline
\ \ \ \ \isacommand{have}\isamarkupfalse%
\ graph{\isacharunderscore}{\kern0pt}equalizer{\isacharcolon}{\kern0pt}\ {\isachardoublequoteopen}equalizer\ {\isacharparenleft}{\kern0pt}graph\ f{\isacharparenright}{\kern0pt}\ {\isacharparenleft}{\kern0pt}graph{\isacharunderscore}{\kern0pt}morph\ f{\isacharparenright}{\kern0pt}\ {\isacharparenleft}{\kern0pt}f\ {\isasymcirc}\isactrlsub c\ left{\isacharunderscore}{\kern0pt}cart{\isacharunderscore}{\kern0pt}proj\ X\ Y{\isacharparenright}{\kern0pt}\ {\isacharparenleft}{\kern0pt}right{\isacharunderscore}{\kern0pt}cart{\isacharunderscore}{\kern0pt}proj\ X\ Y{\isacharparenright}{\kern0pt}{\isachardoublequoteclose}\isanewline
\ \ \ \ \ \ \isacommand{by}\isamarkupfalse%
\ {\isacharparenleft}{\kern0pt}simp\ add{\isacharcolon}{\kern0pt}\ f{\isacharunderscore}{\kern0pt}type\ graph{\isacharunderscore}{\kern0pt}equalizer{\isadigit{4}}{\isacharparenright}{\kern0pt}\isanewline
\ \ \ \ \isacommand{then}\isamarkupfalse%
\ \isacommand{have}\isamarkupfalse%
\ {\isachardoublequoteopen}{\isasymforall}h\ F{\isachardot}{\kern0pt}\ h\ {\isacharcolon}{\kern0pt}\ F\ {\isasymrightarrow}\ X\ {\isasymtimes}\isactrlsub c\ Y\ {\isasymand}\ {\isacharparenleft}{\kern0pt}f\ {\isasymcirc}\isactrlsub c\ left{\isacharunderscore}{\kern0pt}cart{\isacharunderscore}{\kern0pt}proj\ X\ Y{\isacharparenright}{\kern0pt}\ {\isasymcirc}\isactrlsub c\ h\ {\isacharequal}{\kern0pt}\ right{\isacharunderscore}{\kern0pt}cart{\isacharunderscore}{\kern0pt}proj\ X\ Y\ {\isasymcirc}\isactrlsub c\ h\ {\isasymlongrightarrow}\isanewline
\ \ \ \ \ \ \ \ \ \ {\isacharparenleft}{\kern0pt}{\isasymexists}{\isacharbang}{\kern0pt}k{\isachardot}{\kern0pt}\ k\ {\isacharcolon}{\kern0pt}\ F\ {\isasymrightarrow}\ graph\ f\ {\isasymand}\ graph{\isacharunderscore}{\kern0pt}morph\ f\ {\isasymcirc}\isactrlsub c\ k\ {\isacharequal}{\kern0pt}\ h{\isacharparenright}{\kern0pt}{\isachardoublequoteclose}\isanewline
\ \ \ \ \ \ \isacommand{unfolding}\isamarkupfalse%
\ equalizer{\isacharunderscore}{\kern0pt}def\ \isacommand{using}\isamarkupfalse%
\ cfunc{\isacharunderscore}{\kern0pt}type{\isacharunderscore}{\kern0pt}def\ \isacommand{by}\isamarkupfalse%
\ {\isacharparenleft}{\kern0pt}typecheck{\isacharunderscore}{\kern0pt}cfuncs{\isacharcomma}{\kern0pt}\ auto{\isacharparenright}{\kern0pt}\isanewline
\ \ \ \ \isacommand{then}\isamarkupfalse%
\ \isacommand{obtain}\isamarkupfalse%
\ i\ \isakeyword{where}\ i{\isacharunderscore}{\kern0pt}type{\isacharbrackleft}{\kern0pt}type{\isacharunderscore}{\kern0pt}rule{\isacharbrackright}{\kern0pt}{\isacharcolon}{\kern0pt}\ {\isachardoublequoteopen}i\ {\isacharcolon}{\kern0pt}\ R\ {\isasymrightarrow}\ graph\ f{\isachardoublequoteclose}\ \isakeyword{and}\ i{\isacharunderscore}{\kern0pt}eq{\isacharcolon}{\kern0pt}\ {\isachardoublequoteopen}graph{\isacharunderscore}{\kern0pt}morph\ f\ {\isasymcirc}\isactrlsub c\ i\ {\isacharequal}{\kern0pt}\ m{\isachardoublequoteclose}\isanewline
\ \ \ \ \ \ \isacommand{by}\isamarkupfalse%
\ {\isacharparenleft}{\kern0pt}typecheck{\isacharunderscore}{\kern0pt}cfuncs{\isacharcomma}{\kern0pt}\ smt\ comp{\isacharunderscore}{\kern0pt}associative{\isadigit{2}}\ eq\ left{\isacharunderscore}{\kern0pt}cart{\isacharunderscore}{\kern0pt}proj{\isacharunderscore}{\kern0pt}type{\isacharparenright}{\kern0pt}\isanewline
\ \ \ \ \isacommand{have}\isamarkupfalse%
\ {\isachardoublequoteopen}surjective\ i{\isachardoublequoteclose}\isanewline
\ \ \ \ \isacommand{proof}\isamarkupfalse%
\ {\isacharparenleft}{\kern0pt}etcs{\isacharunderscore}{\kern0pt}subst\ surjective{\isacharunderscore}{\kern0pt}def{\isadigit{2}}{\isacharcomma}{\kern0pt}\ auto{\isacharparenright}{\kern0pt}\isanewline
\ \ \ \ \ \ \isacommand{fix}\isamarkupfalse%
\ y{\isacharprime}{\kern0pt}\isanewline
\ \ \ \ \ \ \isacommand{assume}\isamarkupfalse%
\ y{\isacharprime}{\kern0pt}{\isacharunderscore}{\kern0pt}type{\isacharbrackleft}{\kern0pt}type{\isacharunderscore}{\kern0pt}rule{\isacharbrackright}{\kern0pt}{\isacharcolon}{\kern0pt}\ {\isachardoublequoteopen}y{\isacharprime}{\kern0pt}\ {\isasymin}\isactrlsub c\ graph\ f{\isachardoublequoteclose}\isanewline
\isanewline
\ \ \ \ \ \ \isacommand{define}\isamarkupfalse%
\ x\ \isakeyword{where}\ {\isachardoublequoteopen}x\ {\isacharequal}{\kern0pt}\ left{\isacharunderscore}{\kern0pt}cart{\isacharunderscore}{\kern0pt}proj\ X\ Y\ {\isasymcirc}\isactrlsub c\ graph{\isacharunderscore}{\kern0pt}morph{\isacharparenleft}{\kern0pt}f{\isacharparenright}{\kern0pt}\ {\isasymcirc}\isactrlsub c\ y{\isacharprime}{\kern0pt}{\isachardoublequoteclose}\isanewline
\ \ \ \ \ \ \isacommand{then}\isamarkupfalse%
\ \isacommand{have}\isamarkupfalse%
\ x{\isacharunderscore}{\kern0pt}type{\isacharbrackleft}{\kern0pt}type{\isacharunderscore}{\kern0pt}rule{\isacharbrackright}{\kern0pt}{\isacharcolon}{\kern0pt}\ {\isachardoublequoteopen}x\ {\isasymin}\isactrlsub c\ X{\isachardoublequoteclose}\isanewline
\ \ \ \ \ \ \ \ \isacommand{unfolding}\isamarkupfalse%
\ x{\isacharunderscore}{\kern0pt}def\ \isacommand{by}\isamarkupfalse%
\ typecheck{\isacharunderscore}{\kern0pt}cfuncs\isanewline
\isanewline
\ \ \ \ \ \ \isacommand{obtain}\isamarkupfalse%
\ y\ \isakeyword{where}\ y{\isacharunderscore}{\kern0pt}type{\isacharbrackleft}{\kern0pt}type{\isacharunderscore}{\kern0pt}rule{\isacharbrackright}{\kern0pt}{\isacharcolon}{\kern0pt}\ {\isachardoublequoteopen}y\ {\isasymin}\isactrlsub c\ Y{\isachardoublequoteclose}\ \isakeyword{and}\ x{\isacharunderscore}{\kern0pt}y{\isacharunderscore}{\kern0pt}in{\isacharunderscore}{\kern0pt}R{\isacharcolon}{\kern0pt}\ {\isachardoublequoteopen}{\isasymlangle}x{\isacharcomma}{\kern0pt}y{\isasymrangle}\ {\isasymin}\isactrlbsub X\ {\isasymtimes}\isactrlsub c\ Y\isactrlesub \ {\isacharparenleft}{\kern0pt}R{\isacharcomma}{\kern0pt}\ m{\isacharparenright}{\kern0pt}{\isachardoublequoteclose}\isanewline
\ \ \ \ \ \ \ \ \isakeyword{and}\ y{\isacharunderscore}{\kern0pt}unique{\isacharcolon}{\kern0pt}\ {\isachardoublequoteopen}{\isasymforall}\ z{\isachardot}{\kern0pt}\ {\isacharparenleft}{\kern0pt}z\ {\isasymin}\isactrlsub c\ Y\ {\isasymand}\ {\isasymlangle}x{\isacharcomma}{\kern0pt}z{\isasymrangle}\ {\isasymin}\isactrlbsub X\ {\isasymtimes}\isactrlsub c\ Y\isactrlesub \ {\isacharparenleft}{\kern0pt}R{\isacharcomma}{\kern0pt}\ m{\isacharparenright}{\kern0pt}{\isacharparenright}{\kern0pt}\ {\isasymlongrightarrow}\ z\ {\isacharequal}{\kern0pt}\ y{\isachardoublequoteclose}\isanewline
\ \ \ \ \ \ \ \ \isacommand{by}\isamarkupfalse%
\ {\isacharparenleft}{\kern0pt}metis\ assms\ functional{\isacharunderscore}{\kern0pt}on{\isacharunderscore}{\kern0pt}def\ x{\isacharunderscore}{\kern0pt}type{\isacharparenright}{\kern0pt}\isanewline
\isanewline
\ \ \ \ \ \ \isacommand{obtain}\isamarkupfalse%
\ x{\isacharprime}{\kern0pt}\ \isakeyword{where}\ x{\isacharprime}{\kern0pt}{\isacharunderscore}{\kern0pt}type{\isacharbrackleft}{\kern0pt}type{\isacharunderscore}{\kern0pt}rule{\isacharbrackright}{\kern0pt}{\isacharcolon}{\kern0pt}\ {\isachardoublequoteopen}x{\isacharprime}{\kern0pt}\ {\isasymin}\isactrlsub c\ R{\isachardoublequoteclose}\ \isakeyword{and}\ x{\isacharprime}{\kern0pt}{\isacharunderscore}{\kern0pt}eq{\isacharcolon}{\kern0pt}\ {\isachardoublequoteopen}m\ {\isasymcirc}\isactrlsub c\ x{\isacharprime}{\kern0pt}\ {\isacharequal}{\kern0pt}\ {\isasymlangle}x{\isacharcomma}{\kern0pt}\ y{\isasymrangle}{\isachardoublequoteclose}\isanewline
\ \ \ \ \ \ \ \ \isacommand{using}\isamarkupfalse%
\ x{\isacharunderscore}{\kern0pt}y{\isacharunderscore}{\kern0pt}in{\isacharunderscore}{\kern0pt}R\ \isacommand{unfolding}\isamarkupfalse%
\ relative{\isacharunderscore}{\kern0pt}member{\isacharunderscore}{\kern0pt}def{\isadigit{2}}\ \isacommand{by}\isamarkupfalse%
\ {\isacharparenleft}{\kern0pt}{\isacharminus}{\kern0pt}{\isacharcomma}{\kern0pt}\ etcs{\isacharunderscore}{\kern0pt}subst{\isacharunderscore}{\kern0pt}asm\ factors{\isacharunderscore}{\kern0pt}through{\isacharunderscore}{\kern0pt}def{\isadigit{2}}{\isacharcomma}{\kern0pt}\ auto{\isacharparenright}{\kern0pt}\isanewline
\isanewline
\ \ \ \ \ \ \isacommand{have}\isamarkupfalse%
\ {\isachardoublequoteopen}graph{\isacharunderscore}{\kern0pt}morph{\isacharparenleft}{\kern0pt}f{\isacharparenright}{\kern0pt}\ {\isasymcirc}\isactrlsub c\ i\ {\isasymcirc}\isactrlsub c\ x{\isacharprime}{\kern0pt}\ {\isacharequal}{\kern0pt}\ graph{\isacharunderscore}{\kern0pt}morph{\isacharparenleft}{\kern0pt}f{\isacharparenright}{\kern0pt}\ {\isasymcirc}\isactrlsub c\ y{\isacharprime}{\kern0pt}{\isachardoublequoteclose}\isanewline
\ \ \ \ \ \ \isacommand{proof}\isamarkupfalse%
\ {\isacharparenleft}{\kern0pt}typecheck{\isacharunderscore}{\kern0pt}cfuncs{\isacharcomma}{\kern0pt}\ rule\ cart{\isacharunderscore}{\kern0pt}prod{\isacharunderscore}{\kern0pt}eqI{\isacharcomma}{\kern0pt}\ auto{\isacharparenright}{\kern0pt}\isanewline
\ \ \ \ \ \ \ \ \isacommand{show}\isamarkupfalse%
\ left{\isacharcolon}{\kern0pt}\ {\isachardoublequoteopen}left{\isacharunderscore}{\kern0pt}cart{\isacharunderscore}{\kern0pt}proj\ X\ Y\ {\isasymcirc}\isactrlsub c\ graph{\isacharunderscore}{\kern0pt}morph\ f\ {\isasymcirc}\isactrlsub c\ i\ {\isasymcirc}\isactrlsub c\ x{\isacharprime}{\kern0pt}\ {\isacharequal}{\kern0pt}\ left{\isacharunderscore}{\kern0pt}cart{\isacharunderscore}{\kern0pt}proj\ X\ Y\ {\isasymcirc}\isactrlsub c\ graph{\isacharunderscore}{\kern0pt}morph\ f\ {\isasymcirc}\isactrlsub c\ y{\isacharprime}{\kern0pt}{\isachardoublequoteclose}\isanewline
\ \ \ \ \ \ \ \ \isacommand{proof}\isamarkupfalse%
\ {\isacharminus}{\kern0pt}\isanewline
\ \ \ \ \ \ \ \ \ \ \isacommand{have}\isamarkupfalse%
\ {\isachardoublequoteopen}left{\isacharunderscore}{\kern0pt}cart{\isacharunderscore}{\kern0pt}proj\ X\ Y\ {\isasymcirc}\isactrlsub c\ graph{\isacharunderscore}{\kern0pt}morph{\isacharparenleft}{\kern0pt}f{\isacharparenright}{\kern0pt}\ {\isasymcirc}\isactrlsub c\ i\ {\isasymcirc}\isactrlsub c\ x{\isacharprime}{\kern0pt}\ {\isacharequal}{\kern0pt}\ left{\isacharunderscore}{\kern0pt}cart{\isacharunderscore}{\kern0pt}proj\ X\ Y\ {\isasymcirc}\isactrlsub c\ m\ {\isasymcirc}\isactrlsub c\ x{\isacharprime}{\kern0pt}{\isachardoublequoteclose}\isanewline
\ \ \ \ \ \ \ \ \ \ \ \ \isacommand{by}\isamarkupfalse%
\ {\isacharparenleft}{\kern0pt}typecheck{\isacharunderscore}{\kern0pt}cfuncs{\isacharcomma}{\kern0pt}\ smt\ comp{\isacharunderscore}{\kern0pt}associative{\isadigit{2}}\ i{\isacharunderscore}{\kern0pt}eq{\isacharparenright}{\kern0pt}\isanewline
\ \ \ \ \ \ \ \ \ \ \isacommand{also}\isamarkupfalse%
\ \isacommand{have}\isamarkupfalse%
\ {\isachardoublequoteopen}{\isachardot}{\kern0pt}{\isachardot}{\kern0pt}{\isachardot}{\kern0pt}\ {\isacharequal}{\kern0pt}\ x{\isachardoublequoteclose}\isanewline
\ \ \ \ \ \ \ \ \ \ \ \ \isacommand{unfolding}\isamarkupfalse%
\ x{\isacharprime}{\kern0pt}{\isacharunderscore}{\kern0pt}eq\ \isacommand{using}\isamarkupfalse%
\ left{\isacharunderscore}{\kern0pt}cart{\isacharunderscore}{\kern0pt}proj{\isacharunderscore}{\kern0pt}cfunc{\isacharunderscore}{\kern0pt}prod\ \isacommand{by}\isamarkupfalse%
\ {\isacharparenleft}{\kern0pt}typecheck{\isacharunderscore}{\kern0pt}cfuncs{\isacharcomma}{\kern0pt}\ blast{\isacharparenright}{\kern0pt}\isanewline
\ \ \ \ \ \ \ \ \ \ \isacommand{also}\isamarkupfalse%
\ \isacommand{have}\isamarkupfalse%
\ {\isachardoublequoteopen}{\isachardot}{\kern0pt}{\isachardot}{\kern0pt}{\isachardot}{\kern0pt}\ {\isacharequal}{\kern0pt}\ left{\isacharunderscore}{\kern0pt}cart{\isacharunderscore}{\kern0pt}proj\ X\ Y\ {\isasymcirc}\isactrlsub c\ graph{\isacharunderscore}{\kern0pt}morph\ f\ {\isasymcirc}\isactrlsub c\ y{\isacharprime}{\kern0pt}{\isachardoublequoteclose}\isanewline
\ \ \ \ \ \ \ \ \ \ \ \ \isacommand{unfolding}\isamarkupfalse%
\ x{\isacharunderscore}{\kern0pt}def\ \isacommand{by}\isamarkupfalse%
\ auto\isanewline
\ \ \ \ \ \ \ \ \ \ \isacommand{then}\isamarkupfalse%
\ \isacommand{show}\isamarkupfalse%
\ {\isacharquery}{\kern0pt}thesis\ \isacommand{using}\isamarkupfalse%
\ calculation\ \isacommand{by}\isamarkupfalse%
\ auto\isanewline
\ \ \ \ \ \ \ \ \isacommand{qed}\isamarkupfalse%
\isanewline
\isanewline
\ \ \ \ \ \ \ \ \isacommand{show}\isamarkupfalse%
\ {\isachardoublequoteopen}right{\isacharunderscore}{\kern0pt}cart{\isacharunderscore}{\kern0pt}proj\ X\ Y\ {\isasymcirc}\isactrlsub c\ graph{\isacharunderscore}{\kern0pt}morph\ f\ {\isasymcirc}\isactrlsub c\ i\ {\isasymcirc}\isactrlsub c\ x{\isacharprime}{\kern0pt}\ {\isacharequal}{\kern0pt}\ right{\isacharunderscore}{\kern0pt}cart{\isacharunderscore}{\kern0pt}proj\ X\ Y\ {\isasymcirc}\isactrlsub c\ graph{\isacharunderscore}{\kern0pt}morph\ f\ {\isasymcirc}\isactrlsub c\ y{\isacharprime}{\kern0pt}{\isachardoublequoteclose}\isanewline
\ \ \ \ \ \ \ \ \isacommand{proof}\isamarkupfalse%
\ {\isacharminus}{\kern0pt}\isanewline
\ \ \ \ \ \ \ \ \ \ \isacommand{have}\isamarkupfalse%
\ {\isachardoublequoteopen}right{\isacharunderscore}{\kern0pt}cart{\isacharunderscore}{\kern0pt}proj\ X\ Y\ {\isasymcirc}\isactrlsub c\ graph{\isacharunderscore}{\kern0pt}morph\ f\ {\isasymcirc}\isactrlsub c\ i\ {\isasymcirc}\isactrlsub c\ x{\isacharprime}{\kern0pt}\ {\isacharequal}{\kern0pt}\ f\ {\isasymcirc}\isactrlsub c\ left{\isacharunderscore}{\kern0pt}cart{\isacharunderscore}{\kern0pt}proj\ X\ Y\ {\isasymcirc}\isactrlsub c\ graph{\isacharunderscore}{\kern0pt}morph\ f\ {\isasymcirc}\isactrlsub c\ i\ {\isasymcirc}\isactrlsub c\ x{\isacharprime}{\kern0pt}{\isachardoublequoteclose}\isanewline
\ \ \ \ \ \ \ \ \ \ \ \ \isacommand{by}\isamarkupfalse%
\ {\isacharparenleft}{\kern0pt}etcs{\isacharunderscore}{\kern0pt}assocl{\isacharcomma}{\kern0pt}\ typecheck{\isacharunderscore}{\kern0pt}cfuncs{\isacharcomma}{\kern0pt}\ metis\ graph{\isacharunderscore}{\kern0pt}equalizer\ equalizer{\isacharunderscore}{\kern0pt}eq{\isacharparenright}{\kern0pt}\isanewline
\ \ \ \ \ \ \ \ \ \ \isacommand{also}\isamarkupfalse%
\ \isacommand{have}\isamarkupfalse%
\ {\isachardoublequoteopen}{\isachardot}{\kern0pt}{\isachardot}{\kern0pt}{\isachardot}{\kern0pt}\ {\isacharequal}{\kern0pt}\ f\ {\isasymcirc}\isactrlsub c\ left{\isacharunderscore}{\kern0pt}cart{\isacharunderscore}{\kern0pt}proj\ X\ Y\ {\isasymcirc}\isactrlsub c\ graph{\isacharunderscore}{\kern0pt}morph\ f\ {\isasymcirc}\isactrlsub c\ y{\isacharprime}{\kern0pt}{\isachardoublequoteclose}\isanewline
\ \ \ \ \ \ \ \ \ \ \ \ \isacommand{by}\isamarkupfalse%
\ {\isacharparenleft}{\kern0pt}subst\ left{\isacharcomma}{\kern0pt}\ simp{\isacharparenright}{\kern0pt}\isanewline
\ \ \ \ \ \ \ \ \ \ \isacommand{also}\isamarkupfalse%
\ \isacommand{have}\isamarkupfalse%
\ {\isachardoublequoteopen}{\isachardot}{\kern0pt}{\isachardot}{\kern0pt}{\isachardot}{\kern0pt}\ {\isacharequal}{\kern0pt}\ right{\isacharunderscore}{\kern0pt}cart{\isacharunderscore}{\kern0pt}proj\ X\ Y\ {\isasymcirc}\isactrlsub c\ graph{\isacharunderscore}{\kern0pt}morph\ f\ {\isasymcirc}\isactrlsub c\ y{\isacharprime}{\kern0pt}{\isachardoublequoteclose}\isanewline
\ \ \ \ \ \ \ \ \ \ \ \ \isacommand{by}\isamarkupfalse%
\ {\isacharparenleft}{\kern0pt}etcs{\isacharunderscore}{\kern0pt}assocl{\isacharcomma}{\kern0pt}\ typecheck{\isacharunderscore}{\kern0pt}cfuncs{\isacharcomma}{\kern0pt}\ metis\ graph{\isacharunderscore}{\kern0pt}equalizer\ equalizer{\isacharunderscore}{\kern0pt}eq{\isacharparenright}{\kern0pt}\isanewline
\ \ \ \ \ \ \ \ \ \ \isacommand{then}\isamarkupfalse%
\ \isacommand{show}\isamarkupfalse%
\ {\isacharquery}{\kern0pt}thesis\ \isacommand{using}\isamarkupfalse%
\ calculation\ \isacommand{by}\isamarkupfalse%
\ auto\isanewline
\ \ \ \ \ \ \ \ \isacommand{qed}\isamarkupfalse%
\isanewline
\ \ \ \ \ \ \isacommand{qed}\isamarkupfalse%
\isanewline
\ \ \ \ \ \ \isacommand{then}\isamarkupfalse%
\ \isacommand{have}\isamarkupfalse%
\ {\isachardoublequoteopen}i\ {\isasymcirc}\isactrlsub c\ x{\isacharprime}{\kern0pt}\ {\isacharequal}{\kern0pt}\ y{\isacharprime}{\kern0pt}{\isachardoublequoteclose}\isanewline
\ \ \ \ \ \ \ \ \isacommand{using}\isamarkupfalse%
\ equalizer{\isacharunderscore}{\kern0pt}is{\isacharunderscore}{\kern0pt}monomorphism\ graph{\isacharunderscore}{\kern0pt}equalizer\ monomorphism{\isacharunderscore}{\kern0pt}def{\isadigit{2}}\ \isacommand{by}\isamarkupfalse%
\ {\isacharparenleft}{\kern0pt}typecheck{\isacharunderscore}{\kern0pt}cfuncs{\isacharunderscore}{\kern0pt}prems{\isacharcomma}{\kern0pt}\ blast{\isacharparenright}{\kern0pt}\isanewline
\ \ \ \ \ \ \isacommand{then}\isamarkupfalse%
\ \isacommand{show}\isamarkupfalse%
\ {\isachardoublequoteopen}{\isasymexists}x{\isacharprime}{\kern0pt}{\isachardot}{\kern0pt}\ x{\isacharprime}{\kern0pt}\ {\isasymin}\isactrlsub c\ R\ {\isasymand}\ i\ {\isasymcirc}\isactrlsub c\ x{\isacharprime}{\kern0pt}\ {\isacharequal}{\kern0pt}\ y{\isacharprime}{\kern0pt}{\isachardoublequoteclose}\isanewline
\ \ \ \ \ \ \ \ \isacommand{by}\isamarkupfalse%
\ {\isacharparenleft}{\kern0pt}rule{\isacharunderscore}{\kern0pt}tac\ x{\isacharequal}{\kern0pt}x{\isacharprime}{\kern0pt}\ \isakeyword{in}\ exI{\isacharcomma}{\kern0pt}\ simp\ add{\isacharcolon}{\kern0pt}\ x{\isacharprime}{\kern0pt}{\isacharunderscore}{\kern0pt}type{\isacharparenright}{\kern0pt}\isanewline
\ \ \ \ \isacommand{qed}\isamarkupfalse%
\isanewline
\ \ \ \ \isacommand{then}\isamarkupfalse%
\ \isacommand{have}\isamarkupfalse%
\ {\isachardoublequoteopen}isomorphism\ i{\isachardoublequoteclose}\isanewline
\ \ \ \ \ \ \isacommand{by}\isamarkupfalse%
\ {\isacharparenleft}{\kern0pt}metis\ comp{\isacharunderscore}{\kern0pt}monic{\isacharunderscore}{\kern0pt}imp{\isacharunderscore}{\kern0pt}monic{\isacharprime}{\kern0pt}\ epi{\isacharunderscore}{\kern0pt}mon{\isacharunderscore}{\kern0pt}is{\isacharunderscore}{\kern0pt}iso\ f{\isacharunderscore}{\kern0pt}type\ graph{\isacharunderscore}{\kern0pt}morph{\isacharunderscore}{\kern0pt}type\ i{\isacharunderscore}{\kern0pt}eq\ i{\isacharunderscore}{\kern0pt}type\ m{\isacharunderscore}{\kern0pt}mono\ surjective{\isacharunderscore}{\kern0pt}is{\isacharunderscore}{\kern0pt}epimorphism{\isacharparenright}{\kern0pt}\isanewline
\ \ \ \ \isacommand{then}\isamarkupfalse%
\ \isacommand{show}\isamarkupfalse%
\ {\isachardoublequoteopen}{\isasymexists}i{\isachardot}{\kern0pt}\ i\ {\isacharcolon}{\kern0pt}\ R\ {\isasymrightarrow}\ graph\ f\ {\isasymand}\ isomorphism\ i\ {\isasymand}\ m\ {\isacharequal}{\kern0pt}\ graph{\isacharunderscore}{\kern0pt}morph\ f\ {\isasymcirc}\isactrlsub c\ i{\isachardoublequoteclose}\isanewline
\ \ \ \ \ \ \isacommand{by}\isamarkupfalse%
\ {\isacharparenleft}{\kern0pt}rule{\isacharunderscore}{\kern0pt}tac\ x{\isacharequal}{\kern0pt}i\ \isakeyword{in}\ exI{\isacharcomma}{\kern0pt}\ simp\ add{\isacharcolon}{\kern0pt}\ i{\isacharunderscore}{\kern0pt}type\ i{\isacharunderscore}{\kern0pt}eq{\isacharparenright}{\kern0pt}\isanewline
\ \ \isacommand{qed}\isamarkupfalse%
\isanewline
\isacommand{next}\isamarkupfalse%
\isanewline
\ \ \isacommand{fix}\isamarkupfalse%
\ f{\isadigit{1}}\ f{\isadigit{2}}\ i{\isadigit{1}}\ i{\isadigit{2}}\isanewline
\ \ \isacommand{assume}\isamarkupfalse%
\ f{\isadigit{1}}{\isacharunderscore}{\kern0pt}type{\isacharbrackleft}{\kern0pt}type{\isacharunderscore}{\kern0pt}rule{\isacharbrackright}{\kern0pt}{\isacharcolon}{\kern0pt}\ {\isachardoublequoteopen}f{\isadigit{1}}\ {\isacharcolon}{\kern0pt}\ X\ {\isasymrightarrow}\ Y{\isachardoublequoteclose}\isanewline
\ \ \isacommand{assume}\isamarkupfalse%
\ f{\isadigit{2}}{\isacharunderscore}{\kern0pt}type{\isacharbrackleft}{\kern0pt}type{\isacharunderscore}{\kern0pt}rule{\isacharbrackright}{\kern0pt}{\isacharcolon}{\kern0pt}\ {\isachardoublequoteopen}f{\isadigit{2}}\ {\isacharcolon}{\kern0pt}\ X\ {\isasymrightarrow}\ Y{\isachardoublequoteclose}\isanewline
\ \ \isacommand{assume}\isamarkupfalse%
\ i{\isadigit{1}}{\isacharunderscore}{\kern0pt}type{\isacharbrackleft}{\kern0pt}type{\isacharunderscore}{\kern0pt}rule{\isacharbrackright}{\kern0pt}{\isacharcolon}{\kern0pt}\ {\isachardoublequoteopen}i{\isadigit{1}}\ {\isacharcolon}{\kern0pt}\ R\ {\isasymrightarrow}\ graph\ f{\isadigit{1}}{\isachardoublequoteclose}\isanewline
\ \ \isacommand{assume}\isamarkupfalse%
\ i{\isadigit{2}}{\isacharunderscore}{\kern0pt}type{\isacharbrackleft}{\kern0pt}type{\isacharunderscore}{\kern0pt}rule{\isacharbrackright}{\kern0pt}{\isacharcolon}{\kern0pt}\ {\isachardoublequoteopen}i{\isadigit{2}}\ {\isacharcolon}{\kern0pt}\ R\ {\isasymrightarrow}\ graph\ f{\isadigit{2}}{\isachardoublequoteclose}\isanewline
\ \ \isacommand{assume}\isamarkupfalse%
\ i{\isadigit{1}}{\isacharunderscore}{\kern0pt}iso{\isacharcolon}{\kern0pt}\ {\isachardoublequoteopen}isomorphism\ i{\isadigit{1}}{\isachardoublequoteclose}\isanewline
\ \ \isacommand{assume}\isamarkupfalse%
\ i{\isadigit{2}}{\isacharunderscore}{\kern0pt}iso{\isacharcolon}{\kern0pt}\ {\isachardoublequoteopen}isomorphism\ i{\isadigit{2}}{\isachardoublequoteclose}\isanewline
\ \ \isacommand{assume}\isamarkupfalse%
\ eq{\isadigit{1}}{\isacharcolon}{\kern0pt}\ {\isachardoublequoteopen}m\ {\isacharequal}{\kern0pt}\ graph{\isacharunderscore}{\kern0pt}morph\ f{\isadigit{2}}\ {\isasymcirc}\isactrlsub c\ i{\isadigit{2}}{\isachardoublequoteclose}\isanewline
\ \ \isacommand{assume}\isamarkupfalse%
\ eq{\isadigit{2}}{\isacharcolon}{\kern0pt}\ {\isachardoublequoteopen}graph{\isacharunderscore}{\kern0pt}morph\ f{\isadigit{1}}\ {\isasymcirc}\isactrlsub c\ i{\isadigit{1}}\ {\isacharequal}{\kern0pt}\ graph{\isacharunderscore}{\kern0pt}morph\ f{\isadigit{2}}\ {\isasymcirc}\isactrlsub c\ i{\isadigit{2}}{\isachardoublequoteclose}\ \isanewline
\isanewline
\ \ \isacommand{have}\isamarkupfalse%
\ m{\isacharunderscore}{\kern0pt}type{\isacharbrackleft}{\kern0pt}type{\isacharunderscore}{\kern0pt}rule{\isacharbrackright}{\kern0pt}{\isacharcolon}{\kern0pt}\ {\isachardoublequoteopen}m\ {\isacharcolon}{\kern0pt}\ R\ {\isasymrightarrow}\ X\ {\isasymtimes}\isactrlsub c\ Y{\isachardoublequoteclose}\isanewline
\ \ \ \ \isacommand{using}\isamarkupfalse%
\ assms\ \isacommand{unfolding}\isamarkupfalse%
\ functional{\isacharunderscore}{\kern0pt}on{\isacharunderscore}{\kern0pt}def\ subobject{\isacharunderscore}{\kern0pt}of{\isacharunderscore}{\kern0pt}def{\isadigit{2}}\ \isacommand{by}\isamarkupfalse%
\ auto\isanewline
\ \ \isacommand{have}\isamarkupfalse%
\ isomorphism{\isacharbrackleft}{\kern0pt}type{\isacharunderscore}{\kern0pt}rule{\isacharbrackright}{\kern0pt}{\isacharcolon}{\kern0pt}\ {\isachardoublequoteopen}isomorphism{\isacharparenleft}{\kern0pt}left{\isacharunderscore}{\kern0pt}cart{\isacharunderscore}{\kern0pt}proj\ X\ Y\ {\isasymcirc}\isactrlsub c\ m{\isacharparenright}{\kern0pt}{\isachardoublequoteclose}\isanewline
\ \ \ \ \isacommand{using}\isamarkupfalse%
\ assms\ functional{\isacharunderscore}{\kern0pt}on{\isacharunderscore}{\kern0pt}isomorphism\ \isacommand{by}\isamarkupfalse%
\ force\ \ \isanewline
\ \ \isacommand{obtain}\isamarkupfalse%
\ h\ \isakeyword{where}\ h{\isacharunderscore}{\kern0pt}type{\isacharbrackleft}{\kern0pt}type{\isacharunderscore}{\kern0pt}rule{\isacharbrackright}{\kern0pt}{\isacharcolon}{\kern0pt}\ {\isachardoublequoteopen}h{\isacharcolon}{\kern0pt}\ X\ {\isasymrightarrow}\ R{\isachardoublequoteclose}\ \isakeyword{and}\ h{\isacharunderscore}{\kern0pt}def{\isacharcolon}{\kern0pt}\ {\isachardoublequoteopen}h\ {\isacharequal}{\kern0pt}\ {\isacharparenleft}{\kern0pt}left{\isacharunderscore}{\kern0pt}cart{\isacharunderscore}{\kern0pt}proj\ X\ Y\ {\isasymcirc}\isactrlsub c\ m{\isacharparenright}{\kern0pt}\isactrlbold {\isasyminverse}{\isachardoublequoteclose}\isanewline
\ \ \ \ \isacommand{by}\isamarkupfalse%
\ typecheck{\isacharunderscore}{\kern0pt}cfuncs\ \ \isanewline
\ \ \isacommand{have}\isamarkupfalse%
\ {\isachardoublequoteopen}f{\isadigit{1}}\ {\isasymcirc}\isactrlsub c\ left{\isacharunderscore}{\kern0pt}cart{\isacharunderscore}{\kern0pt}proj\ X\ Y\ {\isasymcirc}\isactrlsub c\ m\ {\isacharequal}{\kern0pt}\ f{\isadigit{2}}\ {\isasymcirc}\isactrlsub c\ left{\isacharunderscore}{\kern0pt}cart{\isacharunderscore}{\kern0pt}proj\ X\ Y\ {\isasymcirc}\isactrlsub c\ m{\isachardoublequoteclose}\isanewline
\ \ \isacommand{proof}\isamarkupfalse%
\ {\isacharminus}{\kern0pt}\ \isanewline
\ \ \ \ \isacommand{have}\isamarkupfalse%
\ {\isachardoublequoteopen}f{\isadigit{1}}\ {\isasymcirc}\isactrlsub c\ left{\isacharunderscore}{\kern0pt}cart{\isacharunderscore}{\kern0pt}proj\ X\ Y\ {\isasymcirc}\isactrlsub c\ m\ {\isacharequal}{\kern0pt}\ {\isacharparenleft}{\kern0pt}f{\isadigit{1}}\ {\isasymcirc}\isactrlsub c\ left{\isacharunderscore}{\kern0pt}cart{\isacharunderscore}{\kern0pt}proj\ X\ Y{\isacharparenright}{\kern0pt}\ {\isasymcirc}\isactrlsub c\ graph{\isacharunderscore}{\kern0pt}morph\ f{\isadigit{1}}\ {\isasymcirc}\isactrlsub c\ i{\isadigit{1}}{\isachardoublequoteclose}\isanewline
\ \ \ \ \ \ \isacommand{using}\isamarkupfalse%
\ comp{\isacharunderscore}{\kern0pt}associative{\isadigit{2}}\ eq{\isadigit{1}}\ eq{\isadigit{2}}\ \isacommand{by}\isamarkupfalse%
\ {\isacharparenleft}{\kern0pt}typecheck{\isacharunderscore}{\kern0pt}cfuncs{\isacharcomma}{\kern0pt}\ force{\isacharparenright}{\kern0pt}\isanewline
\ \ \ \ \isacommand{also}\isamarkupfalse%
\ \isacommand{have}\isamarkupfalse%
\ {\isachardoublequoteopen}{\isachardot}{\kern0pt}{\isachardot}{\kern0pt}{\isachardot}{\kern0pt}\ {\isacharequal}{\kern0pt}\ {\isacharparenleft}{\kern0pt}right{\isacharunderscore}{\kern0pt}cart{\isacharunderscore}{\kern0pt}proj\ X\ Y{\isacharparenright}{\kern0pt}\ {\isasymcirc}\isactrlsub c\ graph{\isacharunderscore}{\kern0pt}morph\ f{\isadigit{1}}\ {\isasymcirc}\isactrlsub c\ i{\isadigit{1}}{\isachardoublequoteclose}\isanewline
\ \ \ \ \ \ \isacommand{by}\isamarkupfalse%
\ {\isacharparenleft}{\kern0pt}typecheck{\isacharunderscore}{\kern0pt}cfuncs{\isacharcomma}{\kern0pt}\ smt\ comp{\isacharunderscore}{\kern0pt}associative{\isadigit{2}}\ equalizer{\isacharunderscore}{\kern0pt}def\ graph{\isacharunderscore}{\kern0pt}equalizer{\isadigit{4}}{\isacharparenright}{\kern0pt}\isanewline
\ \ \ \ \isacommand{also}\isamarkupfalse%
\ \isacommand{have}\isamarkupfalse%
\ {\isachardoublequoteopen}{\isachardot}{\kern0pt}{\isachardot}{\kern0pt}{\isachardot}{\kern0pt}\ {\isacharequal}{\kern0pt}\ {\isacharparenleft}{\kern0pt}right{\isacharunderscore}{\kern0pt}cart{\isacharunderscore}{\kern0pt}proj\ X\ Y{\isacharparenright}{\kern0pt}\ {\isasymcirc}\isactrlsub c\ graph{\isacharunderscore}{\kern0pt}morph\ f{\isadigit{2}}\ {\isasymcirc}\isactrlsub c\ i{\isadigit{2}}{\isachardoublequoteclose}\isanewline
\ \ \ \ \ \ \isacommand{by}\isamarkupfalse%
\ {\isacharparenleft}{\kern0pt}simp\ add{\isacharcolon}{\kern0pt}\ eq{\isadigit{2}}{\isacharparenright}{\kern0pt}\isanewline
\ \ \ \ \isacommand{also}\isamarkupfalse%
\ \isacommand{have}\isamarkupfalse%
\ {\isachardoublequoteopen}{\isachardot}{\kern0pt}{\isachardot}{\kern0pt}{\isachardot}{\kern0pt}\ {\isacharequal}{\kern0pt}\ {\isacharparenleft}{\kern0pt}f{\isadigit{2}}\ {\isasymcirc}\isactrlsub c\ left{\isacharunderscore}{\kern0pt}cart{\isacharunderscore}{\kern0pt}proj\ X\ Y{\isacharparenright}{\kern0pt}\ {\isasymcirc}\isactrlsub c\ graph{\isacharunderscore}{\kern0pt}morph\ f{\isadigit{2}}\ {\isasymcirc}\isactrlsub c\ i{\isadigit{2}}{\isachardoublequoteclose}\isanewline
\ \ \ \ \ \ \isacommand{by}\isamarkupfalse%
\ {\isacharparenleft}{\kern0pt}typecheck{\isacharunderscore}{\kern0pt}cfuncs{\isacharcomma}{\kern0pt}\ smt\ comp{\isacharunderscore}{\kern0pt}associative{\isadigit{2}}\ equalizer{\isacharunderscore}{\kern0pt}eq\ graph{\isacharunderscore}{\kern0pt}equalizer{\isadigit{4}}{\isacharparenright}{\kern0pt}\isanewline
\ \ \ \ \isacommand{also}\isamarkupfalse%
\ \isacommand{have}\isamarkupfalse%
\ {\isachardoublequoteopen}{\isachardot}{\kern0pt}{\isachardot}{\kern0pt}{\isachardot}{\kern0pt}\ {\isacharequal}{\kern0pt}\ f{\isadigit{2}}\ {\isasymcirc}\isactrlsub c\ left{\isacharunderscore}{\kern0pt}cart{\isacharunderscore}{\kern0pt}proj\ X\ Y\ {\isasymcirc}\isactrlsub c\ m{\isachardoublequoteclose}\isanewline
\ \ \ \ \ \ \isacommand{by}\isamarkupfalse%
\ {\isacharparenleft}{\kern0pt}typecheck{\isacharunderscore}{\kern0pt}cfuncs{\isacharcomma}{\kern0pt}\ metis\ comp{\isacharunderscore}{\kern0pt}associative{\isadigit{2}}\ eq{\isadigit{1}}{\isacharparenright}{\kern0pt}\isanewline
\ \ \ \ \isacommand{then}\isamarkupfalse%
\ \isacommand{show}\isamarkupfalse%
\ {\isacharquery}{\kern0pt}thesis\ \isacommand{using}\isamarkupfalse%
\ calculation\ \isacommand{by}\isamarkupfalse%
\ auto\isanewline
\ \ \isacommand{qed}\isamarkupfalse%
\isanewline
\ \ \isacommand{then}\isamarkupfalse%
\ \isacommand{show}\isamarkupfalse%
\ {\isachardoublequoteopen}f{\isadigit{1}}\ {\isacharequal}{\kern0pt}\ f{\isadigit{2}}{\isachardoublequoteclose}\isanewline
\ \ \ \ \isacommand{by}\isamarkupfalse%
\ {\isacharparenleft}{\kern0pt}typecheck{\isacharunderscore}{\kern0pt}cfuncs{\isacharcomma}{\kern0pt}\ metis\ cfunc{\isacharunderscore}{\kern0pt}type{\isacharunderscore}{\kern0pt}def\ comp{\isacharunderscore}{\kern0pt}associative\ h{\isacharunderscore}{\kern0pt}def\ h{\isacharunderscore}{\kern0pt}type\ id{\isacharunderscore}{\kern0pt}right{\isacharunderscore}{\kern0pt}unit{\isadigit{2}}\ inverse{\isacharunderscore}{\kern0pt}def{\isadigit{2}}\ isomorphism{\isacharparenright}{\kern0pt}\isanewline
\isacommand{qed}\isamarkupfalse%
%
\endisatagproof
{\isafoldproof}%
%
\isadelimproof
\isanewline
%
\endisadelimproof
%
\isadelimtheory
\isanewline
%
\endisadelimtheory
%
\isatagtheory
\isacommand{end}\isamarkupfalse%
%
\endisatagtheory
{\isafoldtheory}%
%
\isadelimtheory
%
\endisadelimtheory
%
\end{isabellebody}%
\endinput
%:%file=~/ETCS/HOL-ETCS/Equivalence.thy%:%
%:%10=1%:%
%:%11=1%:%
%:%12=2%:%
%:%13=3%:%
%:%27=5%:%
%:%37=7%:%
%:%38=7%:%
%:%39=8%:%
%:%40=9%:%
%:%41=10%:%
%:%42=11%:%
%:%43=11%:%
%:%44=12%:%
%:%46=14%:%
%:%47=15%:%
%:%48=16%:%
%:%49=16%:%
%:%50=17%:%
%:%52=19%:%
%:%53=20%:%
%:%54=21%:%
%:%55=21%:%
%:%56=22%:%
%:%57=23%:%
%:%58=24%:%
%:%59=24%:%
%:%60=25%:%
%:%61=26%:%
%:%62=27%:%
%:%63=27%:%
%:%64=28%:%
%:%65=29%:%
%:%66=30%:%
%:%69=31%:%
%:%73=31%:%
%:%74=31%:%
%:%75=31%:%
%:%76=32%:%
%:%77=32%:%
%:%78=33%:%
%:%79=33%:%
%:%80=34%:%
%:%81=34%:%
%:%82=35%:%
%:%83=35%:%
%:%84=36%:%
%:%85=36%:%
%:%86=37%:%
%:%87=37%:%
%:%88=37%:%
%:%89=38%:%
%:%90=38%:%
%:%91=38%:%
%:%92=39%:%
%:%93=39%:%
%:%94=39%:%
%:%95=40%:%
%:%101=40%:%
%:%104=41%:%
%:%105=42%:%
%:%106=42%:%
%:%107=43%:%
%:%108=44%:%
%:%109=45%:%
%:%110=46%:%
%:%111=47%:%
%:%114=48%:%
%:%118=48%:%
%:%119=48%:%
%:%120=48%:%
%:%121=49%:%
%:%122=49%:%
%:%127=49%:%
%:%130=50%:%
%:%131=51%:%
%:%132=51%:%
%:%133=52%:%
%:%134=53%:%
%:%135=54%:%
%:%136=55%:%
%:%137=56%:%
%:%138=57%:%
%:%139=58%:%
%:%142=59%:%
%:%146=59%:%
%:%147=59%:%
%:%148=59%:%
%:%149=60%:%
%:%150=60%:%
%:%159=62%:%
%:%161=63%:%
%:%162=63%:%
%:%163=64%:%
%:%164=65%:%
%:%171=66%:%
%:%172=66%:%
%:%173=67%:%
%:%174=67%:%
%:%175=68%:%
%:%176=68%:%
%:%177=69%:%
%:%178=69%:%
%:%179=70%:%
%:%180=70%:%
%:%181=70%:%
%:%182=71%:%
%:%183=71%:%
%:%184=72%:%
%:%185=72%:%
%:%186=73%:%
%:%187=73%:%
%:%188=74%:%
%:%189=74%:%
%:%190=74%:%
%:%191=75%:%
%:%192=75%:%
%:%193=76%:%
%:%194=77%:%
%:%195=77%:%
%:%196=78%:%
%:%197=79%:%
%:%198=79%:%
%:%199=80%:%
%:%200=80%:%
%:%201=81%:%
%:%202=81%:%
%:%203=82%:%
%:%204=82%:%
%:%205=82%:%
%:%206=83%:%
%:%207=83%:%
%:%208=84%:%
%:%209=84%:%
%:%210=85%:%
%:%211=85%:%
%:%212=86%:%
%:%213=86%:%
%:%214=87%:%
%:%215=87%:%
%:%216=87%:%
%:%217=88%:%
%:%218=88%:%
%:%219=88%:%
%:%220=89%:%
%:%221=90%:%
%:%222=90%:%
%:%223=90%:%
%:%224=91%:%
%:%225=91%:%
%:%226=91%:%
%:%227=92%:%
%:%228=92%:%
%:%229=93%:%
%:%230=94%:%
%:%231=94%:%
%:%232=95%:%
%:%233=95%:%
%:%234=96%:%
%:%235=96%:%
%:%236=97%:%
%:%237=97%:%
%:%238=97%:%
%:%239=98%:%
%:%240=98%:%
%:%241=99%:%
%:%242=99%:%
%:%243=100%:%
%:%244=100%:%
%:%245=101%:%
%:%246=101%:%
%:%247=102%:%
%:%248=102%:%
%:%249=103%:%
%:%250=104%:%
%:%251=104%:%
%:%252=105%:%
%:%253=105%:%
%:%254=105%:%
%:%255=106%:%
%:%256=107%:%
%:%257=107%:%
%:%258=108%:%
%:%259=108%:%
%:%260=108%:%
%:%261=109%:%
%:%262=110%:%
%:%263=110%:%
%:%264=111%:%
%:%265=111%:%
%:%266=111%:%
%:%267=112%:%
%:%268=112%:%
%:%269=113%:%
%:%279=115%:%
%:%281=116%:%
%:%282=116%:%
%:%283=117%:%
%:%284=118%:%
%:%285=119%:%
%:%286=120%:%
%:%287=121%:%
%:%288=122%:%
%:%289=123%:%
%:%290=124%:%
%:%291=125%:%
%:%292=126%:%
%:%293=127%:%
%:%294=128%:%
%:%295=129%:%
%:%298=131%:%
%:%299=132%:%
%:%300=133%:%
%:%302=135%:%
%:%303=135%:%
%:%304=136%:%
%:%311=138%:%
%:%315=140%:%
%:%327=142%:%
%:%329=143%:%
%:%330=143%:%
%:%331=144%:%
%:%333=146%:%
%:%334=147%:%
%:%335=148%:%
%:%336=148%:%
%:%337=149%:%
%:%338=150%:%
%:%340=152%:%
%:%343=153%:%
%:%347=153%:%
%:%348=153%:%
%:%349=153%:%
%:%350=153%:%
%:%359=155%:%
%:%361=156%:%
%:%362=156%:%
%:%363=157%:%
%:%364=158%:%
%:%371=159%:%
%:%372=159%:%
%:%373=160%:%
%:%374=160%:%
%:%375=161%:%
%:%376=161%:%
%:%377=162%:%
%:%378=162%:%
%:%379=163%:%
%:%380=163%:%
%:%381=164%:%
%:%382=164%:%
%:%383=165%:%
%:%384=165%:%
%:%385=166%:%
%:%386=167%:%
%:%387=167%:%
%:%388=168%:%
%:%389=168%:%
%:%390=168%:%
%:%391=169%:%
%:%392=169%:%
%:%393=170%:%
%:%394=170%:%
%:%395=171%:%
%:%396=171%:%
%:%397=172%:%
%:%398=172%:%
%:%399=173%:%
%:%400=174%:%
%:%401=174%:%
%:%402=175%:%
%:%403=175%:%
%:%404=175%:%
%:%405=176%:%
%:%415=178%:%
%:%417=179%:%
%:%418=179%:%
%:%419=180%:%
%:%422=181%:%
%:%426=181%:%
%:%427=181%:%
%:%428=182%:%
%:%429=182%:%
%:%430=183%:%
%:%431=183%:%
%:%432=184%:%
%:%433=184%:%
%:%434=185%:%
%:%435=185%:%
%:%436=186%:%
%:%437=186%:%
%:%438=187%:%
%:%439=187%:%
%:%440=188%:%
%:%441=188%:%
%:%442=189%:%
%:%443=189%:%
%:%444=190%:%
%:%445=190%:%
%:%446=191%:%
%:%447=191%:%
%:%448=192%:%
%:%449=193%:%
%:%450=193%:%
%:%451=194%:%
%:%452=194%:%
%:%453=194%:%
%:%454=195%:%
%:%455=196%:%
%:%456=196%:%
%:%457=196%:%
%:%458=197%:%
%:%459=198%:%
%:%460=198%:%
%:%461=198%:%
%:%462=199%:%
%:%463=200%:%
%:%464=201%:%
%:%465=201%:%
%:%466=201%:%
%:%467=202%:%
%:%468=202%:%
%:%469=203%:%
%:%475=203%:%
%:%478=204%:%
%:%479=205%:%
%:%480=205%:%
%:%481=206%:%
%:%482=207%:%
%:%483=208%:%
%:%486=209%:%
%:%490=209%:%
%:%491=209%:%
%:%492=210%:%
%:%493=210%:%
%:%494=211%:%
%:%495=211%:%
%:%496=212%:%
%:%497=212%:%
%:%498=212%:%
%:%499=213%:%
%:%500=213%:%
%:%501=214%:%
%:%502=214%:%
%:%503=214%:%
%:%504=215%:%
%:%505=215%:%
%:%506=216%:%
%:%507=216%:%
%:%508=216%:%
%:%509=217%:%
%:%510=217%:%
%:%511=218%:%
%:%512=218%:%
%:%513=219%:%
%:%514=219%:%
%:%515=220%:%
%:%516=220%:%
%:%517=221%:%
%:%518=221%:%
%:%519=222%:%
%:%520=222%:%
%:%521=222%:%
%:%522=223%:%
%:%523=223%:%
%:%524=224%:%
%:%525=224%:%
%:%526=224%:%
%:%527=225%:%
%:%528=225%:%
%:%529=226%:%
%:%530=226%:%
%:%531=227%:%
%:%532=227%:%
%:%533=228%:%
%:%534=228%:%
%:%535=228%:%
%:%536=229%:%
%:%537=229%:%
%:%538=229%:%
%:%539=230%:%
%:%540=230%:%
%:%541=230%:%
%:%542=231%:%
%:%543=231%:%
%:%544=231%:%
%:%545=232%:%
%:%546=232%:%
%:%547=232%:%
%:%548=233%:%
%:%549=233%:%
%:%550=233%:%
%:%551=234%:%
%:%552=234%:%
%:%553=234%:%
%:%554=235%:%
%:%555=235%:%
%:%556=235%:%
%:%557=236%:%
%:%558=236%:%
%:%559=236%:%
%:%560=237%:%
%:%561=237%:%
%:%562=237%:%
%:%563=238%:%
%:%564=238%:%
%:%565=238%:%
%:%566=239%:%
%:%567=239%:%
%:%568=240%:%
%:%569=240%:%
%:%570=240%:%
%:%571=241%:%
%:%572=241%:%
%:%573=241%:%
%:%574=242%:%
%:%575=242%:%
%:%576=243%:%
%:%577=243%:%
%:%578=244%:%
%:%579=244%:%
%:%580=245%:%
%:%581=245%:%
%:%582=246%:%
%:%583=246%:%
%:%584=247%:%
%:%585=248%:%
%:%586=248%:%
%:%587=249%:%
%:%588=249%:%
%:%589=249%:%
%:%590=250%:%
%:%591=250%:%
%:%592=251%:%
%:%593=251%:%
%:%594=252%:%
%:%595=252%:%
%:%596=253%:%
%:%597=253%:%
%:%598=254%:%
%:%599=254%:%
%:%600=255%:%
%:%601=255%:%
%:%602=255%:%
%:%603=256%:%
%:%604=256%:%
%:%605=256%:%
%:%606=256%:%
%:%607=257%:%
%:%608=258%:%
%:%609=258%:%
%:%610=259%:%
%:%611=259%:%
%:%612=259%:%
%:%613=260%:%
%:%614=260%:%
%:%615=260%:%
%:%616=261%:%
%:%617=261%:%
%:%618=261%:%
%:%619=262%:%
%:%620=262%:%
%:%621=262%:%
%:%622=263%:%
%:%623=263%:%
%:%624=263%:%
%:%625=264%:%
%:%626=264%:%
%:%627=265%:%
%:%628=265%:%
%:%629=265%:%
%:%630=266%:%
%:%631=266%:%
%:%632=267%:%
%:%633=267%:%
%:%634=268%:%
%:%635=268%:%
%:%636=269%:%
%:%637=269%:%
%:%638=270%:%
%:%639=270%:%
%:%640=271%:%
%:%641=271%:%
%:%642=272%:%
%:%643=272%:%
%:%644=273%:%
%:%645=273%:%
%:%646=274%:%
%:%647=275%:%
%:%648=275%:%
%:%649=276%:%
%:%650=277%:%
%:%651=277%:%
%:%652=278%:%
%:%653=278%:%
%:%654=278%:%
%:%655=279%:%
%:%656=280%:%
%:%657=280%:%
%:%658=281%:%
%:%659=281%:%
%:%660=282%:%
%:%661=282%:%
%:%662=283%:%
%:%663=283%:%
%:%664=284%:%
%:%665=284%:%
%:%666=285%:%
%:%667=285%:%
%:%668=286%:%
%:%669=286%:%
%:%670=286%:%
%:%671=287%:%
%:%672=287%:%
%:%673=287%:%
%:%674=287%:%
%:%675=288%:%
%:%676=289%:%
%:%677=289%:%
%:%678=290%:%
%:%679=291%:%
%:%680=291%:%
%:%681=292%:%
%:%682=292%:%
%:%683=293%:%
%:%684=293%:%
%:%685=293%:%
%:%686=294%:%
%:%687=295%:%
%:%688=295%:%
%:%689=296%:%
%:%690=296%:%
%:%691=296%:%
%:%692=297%:%
%:%693=297%:%
%:%694=297%:%
%:%695=298%:%
%:%696=298%:%
%:%697=299%:%
%:%698=300%:%
%:%699=300%:%
%:%700=301%:%
%:%701=301%:%
%:%702=302%:%
%:%703=302%:%
%:%704=303%:%
%:%705=303%:%
%:%706=304%:%
%:%707=304%:%
%:%708=305%:%
%:%709=305%:%
%:%710=306%:%
%:%711=306%:%
%:%712=306%:%
%:%713=307%:%
%:%714=307%:%
%:%715=307%:%
%:%716=307%:%
%:%717=308%:%
%:%718=309%:%
%:%719=309%:%
%:%720=310%:%
%:%721=311%:%
%:%722=311%:%
%:%723=312%:%
%:%724=312%:%
%:%725=313%:%
%:%726=313%:%
%:%727=313%:%
%:%728=314%:%
%:%729=315%:%
%:%730=315%:%
%:%731=316%:%
%:%732=316%:%
%:%733=316%:%
%:%734=317%:%
%:%735=317%:%
%:%736=317%:%
%:%737=318%:%
%:%738=318%:%
%:%739=319%:%
%:%740=319%:%
%:%741=320%:%
%:%742=320%:%
%:%743=320%:%
%:%744=321%:%
%:%750=321%:%
%:%753=322%:%
%:%754=323%:%
%:%755=323%:%
%:%756=324%:%
%:%757=325%:%
%:%760=326%:%
%:%764=326%:%
%:%765=326%:%
%:%779=328%:%
%:%791=330%:%
%:%793=331%:%
%:%794=331%:%
%:%795=332%:%
%:%797=334%:%
%:%799=335%:%
%:%800=335%:%
%:%801=336%:%
%:%802=337%:%
%:%809=338%:%
%:%810=338%:%
%:%811=339%:%
%:%812=339%:%
%:%813=340%:%
%:%814=340%:%
%:%815=340%:%
%:%816=341%:%
%:%817=341%:%
%:%818=342%:%
%:%819=342%:%
%:%820=342%:%
%:%821=343%:%
%:%822=343%:%
%:%823=344%:%
%:%824=344%:%
%:%825=344%:%
%:%826=345%:%
%:%827=345%:%
%:%828=345%:%
%:%829=346%:%
%:%830=346%:%
%:%831=346%:%
%:%832=347%:%
%:%833=347%:%
%:%834=347%:%
%:%835=348%:%
%:%836=348%:%
%:%837=348%:%
%:%838=349%:%
%:%839=349%:%
%:%840=350%:%
%:%841=350%:%
%:%842=350%:%
%:%843=351%:%
%:%844=351%:%
%:%845=351%:%
%:%846=352%:%
%:%847=352%:%
%:%848=352%:%
%:%849=353%:%
%:%850=353%:%
%:%851=353%:%
%:%852=354%:%
%:%853=354%:%
%:%854=355%:%
%:%855=355%:%
%:%856=355%:%
%:%857=356%:%
%:%858=356%:%
%:%859=356%:%
%:%860=357%:%
%:%870=359%:%
%:%872=360%:%
%:%873=360%:%
%:%874=361%:%
%:%875=362%:%
%:%882=363%:%
%:%883=363%:%
%:%884=364%:%
%:%885=364%:%
%:%886=365%:%
%:%887=365%:%
%:%888=365%:%
%:%889=366%:%
%:%890=366%:%
%:%891=367%:%
%:%892=367%:%
%:%893=367%:%
%:%894=368%:%
%:%895=368%:%
%:%896=369%:%
%:%897=369%:%
%:%898=369%:%
%:%899=370%:%
%:%900=370%:%
%:%901=371%:%
%:%902=371%:%
%:%903=371%:%
%:%904=371%:%
%:%905=372%:%
%:%906=372%:%
%:%907=373%:%
%:%908=373%:%
%:%909=374%:%
%:%910=374%:%
%:%911=375%:%
%:%912=375%:%
%:%913=376%:%
%:%914=377%:%
%:%915=377%:%
%:%916=378%:%
%:%917=378%:%
%:%918=379%:%
%:%919=379%:%
%:%920=380%:%
%:%921=380%:%
%:%922=381%:%
%:%923=381%:%
%:%924=382%:%
%:%925=382%:%
%:%926=382%:%
%:%927=383%:%
%:%928=384%:%
%:%929=384%:%
%:%930=385%:%
%:%931=385%:%
%:%932=386%:%
%:%933=386%:%
%:%934=387%:%
%:%935=387%:%
%:%936=387%:%
%:%937=388%:%
%:%938=388%:%
%:%939=389%:%
%:%940=389%:%
%:%941=390%:%
%:%942=390%:%
%:%943=391%:%
%:%944=391%:%
%:%945=392%:%
%:%946=393%:%
%:%947=394%:%
%:%948=395%:%
%:%949=396%:%
%:%950=397%:%
%:%951=398%:%
%:%952=399%:%
%:%953=400%:%
%:%954=401%:%
%:%955=402%:%
%:%956=402%:%
%:%957=403%:%
%:%958=403%:%
%:%959=404%:%
%:%960=404%:%
%:%961=405%:%
%:%962=405%:%
%:%963=406%:%
%:%964=407%:%
%:%965=407%:%
%:%966=408%:%
%:%967=408%:%
%:%968=408%:%
%:%969=409%:%
%:%970=409%:%
%:%971=410%:%
%:%972=410%:%
%:%973=410%:%
%:%974=411%:%
%:%975=411%:%
%:%976=412%:%
%:%977=413%:%
%:%978=413%:%
%:%981=416%:%
%:%982=417%:%
%:%983=417%:%
%:%984=418%:%
%:%985=418%:%
%:%986=419%:%
%:%987=419%:%
%:%988=419%:%
%:%989=420%:%
%:%990=420%:%
%:%991=421%:%
%:%992=421%:%
%:%993=422%:%
%:%994=422%:%
%:%995=423%:%
%:%996=423%:%
%:%997=423%:%
%:%998=424%:%
%:%999=424%:%
%:%1000=425%:%
%:%1001=425%:%
%:%1002=426%:%
%:%1003=426%:%
%:%1004=427%:%
%:%1005=427%:%
%:%1006=428%:%
%:%1007=428%:%
%:%1008=429%:%
%:%1009=429%:%
%:%1010=430%:%
%:%1011=430%:%
%:%1012=431%:%
%:%1013=432%:%
%:%1014=433%:%
%:%1015=434%:%
%:%1016=434%:%
%:%1017=434%:%
%:%1018=435%:%
%:%1019=436%:%
%:%1020=437%:%
%:%1021=438%:%
%:%1022=438%:%
%:%1023=439%:%
%:%1024=440%:%
%:%1025=441%:%
%:%1026=441%:%
%:%1027=442%:%
%:%1028=442%:%
%:%1029=443%:%
%:%1030=443%:%
%:%1031=444%:%
%:%1032=444%:%
%:%1033=445%:%
%:%1034=445%:%
%:%1035=445%:%
%:%1036=446%:%
%:%1037=446%:%
%:%1038=446%:%
%:%1039=447%:%
%:%1040=447%:%
%:%1041=447%:%
%:%1042=448%:%
%:%1043=448%:%
%:%1044=449%:%
%:%1045=449%:%
%:%1046=449%:%
%:%1047=450%:%
%:%1048=450%:%
%:%1049=451%:%
%:%1050=451%:%
%:%1051=451%:%
%:%1052=452%:%
%:%1053=452%:%
%:%1054=452%:%
%:%1055=453%:%
%:%1056=453%:%
%:%1057=454%:%
%:%1058=455%:%
%:%1059=456%:%
%:%1060=457%:%
%:%1061=457%:%
%:%1062=457%:%
%:%1063=458%:%
%:%1064=459%:%
%:%1065=460%:%
%:%1066=461%:%
%:%1067=461%:%
%:%1068=461%:%
%:%1069=462%:%
%:%1070=463%:%
%:%1071=464%:%
%:%1072=464%:%
%:%1073=465%:%
%:%1074=465%:%
%:%1075=466%:%
%:%1076=466%:%
%:%1077=467%:%
%:%1078=467%:%
%:%1079=468%:%
%:%1080=468%:%
%:%1081=468%:%
%:%1082=469%:%
%:%1083=469%:%
%:%1084=469%:%
%:%1085=470%:%
%:%1086=470%:%
%:%1087=470%:%
%:%1088=471%:%
%:%1089=471%:%
%:%1090=472%:%
%:%1091=472%:%
%:%1092=472%:%
%:%1093=473%:%
%:%1094=473%:%
%:%1095=474%:%
%:%1096=474%:%
%:%1097=474%:%
%:%1098=475%:%
%:%1099=475%:%
%:%1100=475%:%
%:%1101=476%:%
%:%1102=476%:%
%:%1103=477%:%
%:%1104=478%:%
%:%1105=478%:%
%:%1106=478%:%
%:%1107=479%:%
%:%1108=479%:%
%:%1109=480%:%
%:%1110=481%:%
%:%1111=481%:%
%:%1112=482%:%
%:%1113=482%:%
%:%1114=483%:%
%:%1115=484%:%
%:%1116=484%:%
%:%1117=484%:%
%:%1118=485%:%
%:%1119=485%:%
%:%1120=486%:%
%:%1121=487%:%
%:%1122=488%:%
%:%1123=489%:%
%:%1124=490%:%
%:%1125=490%:%
%:%1126=491%:%
%:%1127=492%:%
%:%1128=493%:%
%:%1129=493%:%
%:%1130=493%:%
%:%1131=494%:%
%:%1132=495%:%
%:%1133=495%:%
%:%1134=496%:%
%:%1135=496%:%
%:%1136=497%:%
%:%1137=497%:%
%:%1138=498%:%
%:%1139=498%:%
%:%1140=499%:%
%:%1141=499%:%
%:%1142=500%:%
%:%1143=500%:%
%:%1144=500%:%
%:%1145=501%:%
%:%1146=501%:%
%:%1147=501%:%
%:%1148=502%:%
%:%1149=502%:%
%:%1150=503%:%
%:%1151=503%:%
%:%1152=504%:%
%:%1153=504%:%
%:%1154=504%:%
%:%1155=505%:%
%:%1156=505%:%
%:%1157=506%:%
%:%1158=506%:%
%:%1159=507%:%
%:%1160=507%:%
%:%1163=510%:%
%:%1164=511%:%
%:%1165=511%:%
%:%1166=511%:%
%:%1167=512%:%
%:%1173=512%:%
%:%1176=513%:%
%:%1177=514%:%
%:%1178=514%:%
%:%1179=515%:%
%:%1180=516%:%
%:%1183=517%:%
%:%1187=517%:%
%:%1188=517%:%
%:%1202=519%:%
%:%1212=521%:%
%:%1213=521%:%
%:%1214=522%:%
%:%1215=523%:%
%:%1218=526%:%
%:%1225=527%:%
%:%1226=527%:%
%:%1227=528%:%
%:%1228=528%:%
%:%1229=529%:%
%:%1230=529%:%
%:%1231=530%:%
%:%1232=530%:%
%:%1233=531%:%
%:%1234=531%:%
%:%1235=532%:%
%:%1236=532%:%
%:%1237=533%:%
%:%1238=533%:%
%:%1239=534%:%
%:%1240=534%:%
%:%1243=537%:%
%:%1244=538%:%
%:%1245=538%:%
%:%1246=539%:%
%:%1247=539%:%
%:%1248=540%:%
%:%1249=540%:%
%:%1250=540%:%
%:%1251=540%:%
%:%1252=541%:%
%:%1253=542%:%
%:%1254=542%:%
%:%1255=543%:%
%:%1256=543%:%
%:%1257=543%:%
%:%1258=543%:%
%:%1259=544%:%
%:%1260=544%:%
%:%1261=544%:%
%:%1262=545%:%
%:%1263=545%:%
%:%1264=545%:%
%:%1265=545%:%
%:%1266=546%:%
%:%1267=547%:%
%:%1268=547%:%
%:%1269=548%:%
%:%1270=548%:%
%:%1271=549%:%
%:%1272=549%:%
%:%1273=549%:%
%:%1274=550%:%
%:%1275=550%:%
%:%1276=550%:%
%:%1277=551%:%
%:%1278=551%:%
%:%1279=551%:%
%:%1280=552%:%
%:%1281=553%:%
%:%1282=553%:%
%:%1283=554%:%
%:%1284=554%:%
%:%1285=555%:%
%:%1286=555%:%
%:%1287=556%:%
%:%1288=556%:%
%:%1289=557%:%
%:%1290=557%:%
%:%1291=557%:%
%:%1292=558%:%
%:%1293=558%:%
%:%1294=558%:%
%:%1297=561%:%
%:%1298=562%:%
%:%1299=562%:%
%:%1300=563%:%
%:%1301=564%:%
%:%1302=564%:%
%:%1303=564%:%
%:%1304=565%:%
%:%1305=566%:%
%:%1306=567%:%
%:%1307=568%:%
%:%1308=568%:%
%:%1309=569%:%
%:%1310=570%:%
%:%1311=570%:%
%:%1312=571%:%
%:%1313=571%:%
%:%1314=571%:%
%:%1315=572%:%
%:%1316=572%:%
%:%1317=572%:%
%:%1318=573%:%
%:%1319=573%:%
%:%1320=573%:%
%:%1321=574%:%
%:%1322=574%:%
%:%1323=574%:%
%:%1324=575%:%
%:%1325=575%:%
%:%1326=575%:%
%:%1327=576%:%
%:%1328=576%:%
%:%1329=577%:%
%:%1330=578%:%
%:%1331=578%:%
%:%1332=579%:%
%:%1333=579%:%
%:%1334=579%:%
%:%1335=580%:%
%:%1336=581%:%
%:%1337=581%:%
%:%1338=582%:%
%:%1339=582%:%
%:%1340=583%:%
%:%1341=583%:%
%:%1342=584%:%
%:%1343=584%:%
%:%1344=585%:%
%:%1345=585%:%
%:%1346=586%:%
%:%1347=586%:%
%:%1348=587%:%
%:%1349=587%:%
%:%1350=587%:%
%:%1351=588%:%
%:%1352=588%:%
%:%1353=588%:%
%:%1354=589%:%
%:%1355=589%:%
%:%1356=589%:%
%:%1357=590%:%
%:%1358=590%:%
%:%1359=591%:%
%:%1360=591%:%
%:%1361=592%:%
%:%1367=592%:%
%:%1370=593%:%
%:%1371=594%:%
%:%1372=594%:%
%:%1373=595%:%
%:%1374=596%:%
%:%1376=598%:%
%:%1379=599%:%
%:%1383=599%:%
%:%1384=599%:%
%:%1385=599%:%
%:%1399=601%:%
%:%1411=603%:%
%:%1413=604%:%
%:%1414=604%:%
%:%1415=605%:%
%:%1419=609%:%
%:%1420=610%:%
%:%1421=611%:%
%:%1422=611%:%
%:%1423=612%:%
%:%1424=613%:%
%:%1428=617%:%
%:%1435=618%:%
%:%1436=618%:%
%:%1437=619%:%
%:%1438=619%:%
%:%1442=623%:%
%:%1443=624%:%
%:%1444=624%:%
%:%1445=625%:%
%:%1446=625%:%
%:%1447=626%:%
%:%1448=626%:%
%:%1449=626%:%
%:%1450=627%:%
%:%1451=627%:%
%:%1452=627%:%
%:%1453=628%:%
%:%1459=628%:%
%:%1462=629%:%
%:%1463=630%:%
%:%1464=630%:%
%:%1465=631%:%
%:%1467=633%:%
%:%1468=634%:%
%:%1469=635%:%
%:%1470=635%:%
%:%1471=636%:%
%:%1472=637%:%
%:%1474=639%:%
%:%1481=640%:%
%:%1482=640%:%
%:%1483=641%:%
%:%1484=641%:%
%:%1485=642%:%
%:%1486=642%:%
%:%1487=642%:%
%:%1488=643%:%
%:%1489=643%:%
%:%1491=645%:%
%:%1492=646%:%
%:%1493=646%:%
%:%1494=646%:%
%:%1495=647%:%
%:%1496=647%:%
%:%1497=647%:%
%:%1498=648%:%
%:%1499=648%:%
%:%1500=648%:%
%:%1501=649%:%
%:%1507=649%:%
%:%1510=650%:%
%:%1511=651%:%
%:%1512=651%:%
%:%1513=652%:%
%:%1515=654%:%
%:%1516=655%:%
%:%1517=656%:%
%:%1518=656%:%
%:%1519=657%:%
%:%1520=658%:%
%:%1522=660%:%
%:%1529=661%:%
%:%1530=661%:%
%:%1531=662%:%
%:%1532=662%:%
%:%1533=663%:%
%:%1534=663%:%
%:%1535=663%:%
%:%1536=664%:%
%:%1537=664%:%
%:%1540=667%:%
%:%1541=668%:%
%:%1542=668%:%
%:%1543=669%:%
%:%1544=669%:%
%:%1545=670%:%
%:%1546=670%:%
%:%1547=670%:%
%:%1548=671%:%
%:%1549=671%:%
%:%1550=671%:%
%:%1551=672%:%
%:%1557=672%:%
%:%1560=673%:%
%:%1561=674%:%
%:%1562=674%:%
%:%1563=675%:%
%:%1564=676%:%
%:%1567=677%:%
%:%1571=677%:%
%:%1572=677%:%
%:%1573=677%:%
%:%1578=677%:%
%:%1581=678%:%
%:%1582=679%:%
%:%1583=679%:%
%:%1584=680%:%
%:%1585=681%:%
%:%1588=682%:%
%:%1592=682%:%
%:%1593=682%:%
%:%1594=682%:%
%:%1599=682%:%
%:%1602=683%:%
%:%1603=684%:%
%:%1604=684%:%
%:%1605=685%:%
%:%1606=686%:%
%:%1609=687%:%
%:%1613=687%:%
%:%1614=687%:%
%:%1615=687%:%
%:%1620=687%:%
%:%1623=688%:%
%:%1624=689%:%
%:%1625=689%:%
%:%1626=690%:%
%:%1627=691%:%
%:%1630=692%:%
%:%1634=692%:%
%:%1635=692%:%
%:%1636=692%:%
%:%1641=692%:%
%:%1644=693%:%
%:%1645=694%:%
%:%1646=694%:%
%:%1647=695%:%
%:%1648=696%:%
%:%1651=697%:%
%:%1655=697%:%
%:%1656=697%:%
%:%1657=697%:%
%:%1662=697%:%
%:%1665=698%:%
%:%1666=699%:%
%:%1667=699%:%
%:%1668=700%:%
%:%1669=701%:%
%:%1672=702%:%
%:%1676=702%:%
%:%1677=702%:%
%:%1678=702%:%
%:%1683=702%:%
%:%1686=703%:%
%:%1687=704%:%
%:%1688=704%:%
%:%1689=705%:%
%:%1690=706%:%
%:%1693=707%:%
%:%1697=707%:%
%:%1698=707%:%
%:%1699=707%:%
%:%1704=707%:%
%:%1707=708%:%
%:%1708=709%:%
%:%1709=709%:%
%:%1710=710%:%
%:%1711=711%:%
%:%1712=712%:%
%:%1719=713%:%
%:%1720=713%:%
%:%1721=714%:%
%:%1722=714%:%
%:%1723=715%:%
%:%1724=715%:%
%:%1725=715%:%
%:%1726=716%:%
%:%1727=716%:%
%:%1728=716%:%
%:%1729=717%:%
%:%1730=717%:%
%:%1731=717%:%
%:%1732=718%:%
%:%1733=718%:%
%:%1734=718%:%
%:%1735=719%:%
%:%1736=719%:%
%:%1737=720%:%
%:%1738=720%:%
%:%1739=720%:%
%:%1740=721%:%
%:%1741=721%:%
%:%1742=721%:%
%:%1743=722%:%
%:%1744=722%:%
%:%1745=722%:%
%:%1746=723%:%
%:%1747=723%:%
%:%1748=723%:%
%:%1749=723%:%
%:%1750=724%:%
%:%1751=724%:%
%:%1752=724%:%
%:%1753=725%:%
%:%1754=725%:%
%:%1755=726%:%
%:%1765=728%:%
%:%1767=729%:%
%:%1768=729%:%
%:%1769=730%:%
%:%1770=731%:%
%:%1777=732%:%
%:%1778=732%:%
%:%1779=733%:%
%:%1780=733%:%
%:%1781=734%:%
%:%1782=734%:%
%:%1783=734%:%
%:%1784=735%:%
%:%1785=735%:%
%:%1786=735%:%
%:%1787=736%:%
%:%1788=736%:%
%:%1789=737%:%
%:%1790=737%:%
%:%1791=737%:%
%:%1792=738%:%
%:%1793=738%:%
%:%1794=738%:%
%:%1795=739%:%
%:%1796=740%:%
%:%1797=740%:%
%:%1798=741%:%
%:%1799=741%:%
%:%1800=742%:%
%:%1801=743%:%
%:%1802=743%:%
%:%1803=744%:%
%:%1804=744%:%
%:%1805=745%:%
%:%1806=745%:%
%:%1807=746%:%
%:%1808=746%:%
%:%1809=747%:%
%:%1810=747%:%
%:%1811=748%:%
%:%1812=748%:%
%:%1813=749%:%
%:%1814=750%:%
%:%1815=750%:%
%:%1816=751%:%
%:%1817=751%:%
%:%1818=751%:%
%:%1819=752%:%
%:%1820=752%:%
%:%1821=753%:%
%:%1822=753%:%
%:%1823=754%:%
%:%1824=754%:%
%:%1825=754%:%
%:%1826=755%:%
%:%1827=755%:%
%:%1828=756%:%
%:%1829=756%:%
%:%1830=756%:%
%:%1831=757%:%
%:%1832=757%:%
%:%1833=757%:%
%:%1834=757%:%
%:%1835=758%:%
%:%1836=758%:%
%:%1837=759%:%
%:%1838=760%:%
%:%1839=760%:%
%:%1840=761%:%
%:%1841=761%:%
%:%1842=761%:%
%:%1843=762%:%
%:%1844=762%:%
%:%1845=763%:%
%:%1846=763%:%
%:%1847=763%:%
%:%1848=764%:%
%:%1849=764%:%
%:%1850=764%:%
%:%1851=765%:%
%:%1852=766%:%
%:%1853=766%:%
%:%1854=766%:%
%:%1855=767%:%
%:%1856=767%:%
%:%1857=768%:%
%:%1858=768%:%
%:%1859=768%:%
%:%1860=769%:%
%:%1861=769%:%
%:%1862=769%:%
%:%1863=770%:%
%:%1864=770%:%
%:%1865=771%:%
%:%1871=771%:%
%:%1874=772%:%
%:%1875=773%:%
%:%1876=773%:%
%:%1877=774%:%
%:%1878=775%:%
%:%1879=776%:%
%:%1886=777%:%
%:%1887=777%:%
%:%1888=778%:%
%:%1889=778%:%
%:%1890=779%:%
%:%1892=781%:%
%:%1893=782%:%
%:%1894=782%:%
%:%1895=783%:%
%:%1896=784%:%
%:%1897=784%:%
%:%1898=785%:%
%:%1900=787%:%
%:%1901=788%:%
%:%1902=788%:%
%:%1903=789%:%
%:%1904=790%:%
%:%1905=790%:%
%:%1906=791%:%
%:%1907=791%:%
%:%1908=792%:%
%:%1909=792%:%
%:%1910=793%:%
%:%1916=793%:%
%:%1919=794%:%
%:%1920=795%:%
%:%1921=795%:%
%:%1922=796%:%
%:%1923=797%:%
%:%1924=798%:%
%:%1931=799%:%
%:%1932=799%:%
%:%1933=800%:%
%:%1934=800%:%
%:%1935=801%:%
%:%1936=801%:%
%:%1937=801%:%
%:%1938=802%:%
%:%1939=803%:%
%:%1940=804%:%
%:%1941=804%:%
%:%1942=804%:%
%:%1943=805%:%
%:%1944=806%:%
%:%1945=806%:%
%:%1946=807%:%
%:%1947=807%:%
%:%1948=807%:%
%:%1949=808%:%
%:%1950=808%:%
%:%1951=808%:%
%:%1952=809%:%
%:%1953=810%:%
%:%1954=811%:%
%:%1955=811%:%
%:%1956=812%:%
%:%1957=812%:%
%:%1958=812%:%
%:%1959=813%:%
%:%1960=813%:%
%:%1961=813%:%
%:%1962=814%:%
%:%1963=814%:%
%:%1964=815%:%
%:%1970=815%:%
%:%1973=816%:%
%:%1974=817%:%
%:%1975=817%:%
%:%1976=818%:%
%:%1977=819%:%
%:%1978=820%:%
%:%1979=821%:%
%:%1982=822%:%
%:%1986=822%:%
%:%1987=822%:%
%:%1988=823%:%
%:%1989=823%:%
%:%1990=824%:%
%:%1991=824%:%
%:%1992=825%:%
%:%1993=825%:%
%:%1994=826%:%
%:%1995=826%:%
%:%1996=827%:%
%:%1997=827%:%
%:%1998=828%:%
%:%1999=828%:%
%:%2000=829%:%
%:%2001=829%:%
%:%2002=830%:%
%:%2003=831%:%
%:%2004=831%:%
%:%2005=832%:%
%:%2007=834%:%
%:%2008=835%:%
%:%2009=835%:%
%:%2010=836%:%
%:%2011=836%:%
%:%2012=836%:%
%:%2013=837%:%
%:%2014=838%:%
%:%2015=839%:%
%:%2016=839%:%
%:%2017=840%:%
%:%2018=841%:%
%:%2019=841%:%
%:%2020=842%:%
%:%2022=844%:%
%:%2023=845%:%
%:%2024=845%:%
%:%2025=846%:%
%:%2026=846%:%
%:%2027=846%:%
%:%2030=849%:%
%:%2031=850%:%
%:%2032=850%:%
%:%2033=851%:%
%:%2034=851%:%
%:%2035=851%:%
%:%2036=852%:%
%:%2037=852%:%
%:%2038=852%:%
%:%2039=853%:%
%:%2040=853%:%
%:%2041=853%:%
%:%2042=854%:%
%:%2043=855%:%
%:%2044=856%:%
%:%2045=856%:%
%:%2046=857%:%
%:%2047=858%:%
%:%2048=858%:%
%:%2049=859%:%
%:%2050=859%:%
%:%2051=860%:%
%:%2052=860%:%
%:%2053=861%:%
%:%2054=861%:%
%:%2055=862%:%
%:%2056=862%:%
%:%2057=863%:%
%:%2058=863%:%
%:%2059=864%:%
%:%2060=865%:%
%:%2061=865%:%
%:%2062=866%:%
%:%2063=866%:%
%:%2064=867%:%
%:%2074=869%:%
%:%2076=870%:%
%:%2077=870%:%
%:%2078=871%:%
%:%2079=872%:%
%:%2080=873%:%
%:%2087=874%:%
%:%2088=874%:%
%:%2089=875%:%
%:%2090=875%:%
%:%2091=876%:%
%:%2092=876%:%
%:%2093=876%:%
%:%2094=877%:%
%:%2095=877%:%
%:%2096=878%:%
%:%2097=878%:%
%:%2098=878%:%
%:%2099=879%:%
%:%2100=879%:%
%:%2101=880%:%
%:%2102=880%:%
%:%2103=881%:%
%:%2104=881%:%
%:%2105=881%:%
%:%2106=882%:%
%:%2107=882%:%
%:%2108=882%:%
%:%2109=883%:%
%:%2110=884%:%
%:%2111=884%:%
%:%2112=885%:%
%:%2113=885%:%
%:%2114=885%:%
%:%2115=886%:%
%:%2116=887%:%
%:%2117=888%:%
%:%2118=888%:%
%:%2119=888%:%
%:%2120=889%:%
%:%2121=889%:%
%:%2122=889%:%
%:%2123=890%:%
%:%2124=890%:%
%:%2125=891%:%
%:%2126=892%:%
%:%2127=892%:%
%:%2128=893%:%
%:%2129=893%:%
%:%2130=894%:%
%:%2131=895%:%
%:%2132=895%:%
%:%2133=896%:%
%:%2134=896%:%
%:%2135=897%:%
%:%2136=897%:%
%:%2137=897%:%
%:%2138=898%:%
%:%2139=898%:%
%:%2140=898%:%
%:%2141=899%:%
%:%2142=900%:%
%:%2143=900%:%
%:%2144=901%:%
%:%2145=901%:%
%:%2146=902%:%
%:%2147=902%:%
%:%2148=902%:%
%:%2149=903%:%
%:%2150=903%:%
%:%2151=903%:%
%:%2152=904%:%
%:%2153=905%:%
%:%2154=905%:%
%:%2155=906%:%
%:%2156=906%:%
%:%2157=907%:%
%:%2158=908%:%
%:%2159=908%:%
%:%2160=909%:%
%:%2161=909%:%
%:%2162=910%:%
%:%2163=911%:%
%:%2164=911%:%
%:%2165=912%:%
%:%2166=913%:%
%:%2167=913%:%
%:%2168=914%:%
%:%2169=915%:%
%:%2170=915%:%
%:%2171=915%:%
%:%2172=916%:%
%:%2173=916%:%
%:%2174=916%:%
%:%2175=916%:%
%:%2176=917%:%
%:%2177=918%:%
%:%2178=918%:%
%:%2179=919%:%
%:%2180=919%:%
%:%2181=920%:%
%:%2182=921%:%
%:%2183=921%:%
%:%2184=922%:%
%:%2185=923%:%
%:%2186=923%:%
%:%2187=923%:%
%:%2188=924%:%
%:%2189=924%:%
%:%2190=925%:%
%:%2191=925%:%
%:%2192=925%:%
%:%2193=926%:%
%:%2194=926%:%
%:%2195=927%:%
%:%2196=927%:%
%:%2197=927%:%
%:%2198=928%:%
%:%2199=928%:%
%:%2200=929%:%
%:%2201=929%:%
%:%2202=930%:%
%:%2203=930%:%
%:%2204=931%:%
%:%2205=931%:%
%:%2206=931%:%
%:%2207=932%:%
%:%2208=932%:%
%:%2209=933%:%
%:%2210=933%:%
%:%2211=933%:%
%:%2212=934%:%
%:%2213=935%:%
%:%2214=935%:%
%:%2215=936%:%
%:%2216=936%:%
%:%2217=936%:%
%:%2218=937%:%
%:%2219=938%:%
%:%2220=939%:%
%:%2221=939%:%
%:%2222=939%:%
%:%2223=940%:%
%:%2224=941%:%
%:%2225=941%:%
%:%2226=942%:%
%:%2227=942%:%
%:%2228=942%:%
%:%2229=942%:%
%:%2230=943%:%
%:%2231=944%:%
%:%2232=944%:%
%:%2233=945%:%
%:%2234=945%:%
%:%2235=945%:%
%:%2236=945%:%
%:%2237=946%:%
%:%2238=946%:%
%:%2239=947%:%
%:%2240=947%:%
%:%2241=947%:%
%:%2242=947%:%
%:%2243=948%:%
%:%2244=948%:%
%:%2245=949%:%
%:%2246=949%:%
%:%2247=950%:%
%:%2248=951%:%
%:%2249=951%:%
%:%2252=954%:%
%:%2253=955%:%
%:%2254=955%:%
%:%2255=955%:%
%:%2256=955%:%
%:%2257=956%:%
%:%2258=956%:%
%:%2259=956%:%
%:%2260=957%:%
%:%2261=958%:%
%:%2262=959%:%
%:%2263=959%:%
%:%2264=959%:%
%:%2265=960%:%
%:%2266=961%:%
%:%2267=961%:%
%:%2268=962%:%
%:%2269=962%:%
%:%2270=962%:%
%:%2271=963%:%
%:%2272=963%:%
%:%2273=963%:%
%:%2274=964%:%
%:%2275=964%:%
%:%2276=965%:%
%:%2277=965%:%
%:%2278=965%:%
%:%2279=966%:%
%:%2280=966%:%
%:%2281=967%:%
%:%2282=967%:%
%:%2283=968%:%
%:%2284=968%:%
%:%2285=969%:%
%:%2286=969%:%
%:%2287=970%:%
%:%2288=970%:%
%:%2289=970%:%
%:%2290=971%:%
%:%2291=971%:%
%:%2292=972%:%
%:%2293=972%:%
%:%2294=973%:%
%:%2295=973%:%
%:%2296=974%:%
%:%2306=976%:%
%:%2308=977%:%
%:%2309=977%:%
%:%2310=978%:%
%:%2311=979%:%
%:%2312=980%:%
%:%2319=981%:%
%:%2320=981%:%
%:%2321=982%:%
%:%2322=982%:%
%:%2323=983%:%
%:%2324=983%:%
%:%2325=983%:%
%:%2326=983%:%
%:%2327=984%:%
%:%2328=984%:%
%:%2329=985%:%
%:%2330=985%:%
%:%2331=985%:%
%:%2332=985%:%
%:%2333=986%:%
%:%2334=987%:%
%:%2335=987%:%
%:%2336=988%:%
%:%2337=988%:%
%:%2338=989%:%
%:%2339=989%:%
%:%2340=989%:%
%:%2341=990%:%
%:%2342=990%:%
%:%2343=990%:%
%:%2344=991%:%
%:%2345=991%:%
%:%2346=992%:%
%:%2361=994%:%
%:%2371=996%:%
%:%2372=996%:%
%:%2373=997%:%
%:%2374=998%:%
%:%2377=999%:%
%:%2381=999%:%
%:%2382=999%:%
%:%2383=999%:%
%:%2384=1000%:%
%:%2385=1000%:%
%:%2386=1001%:%
%:%2387=1001%:%
%:%2388=1002%:%
%:%2389=1002%:%
%:%2390=1003%:%
%:%2391=1003%:%
%:%2392=1004%:%
%:%2393=1004%:%
%:%2394=1005%:%
%:%2395=1005%:%
%:%2396=1005%:%
%:%2397=1006%:%
%:%2398=1006%:%
%:%2399=1006%:%
%:%2400=1007%:%
%:%2401=1007%:%
%:%2402=1007%:%
%:%2403=1008%:%
%:%2404=1008%:%
%:%2405=1009%:%
%:%2406=1009%:%
%:%2407=1010%:%
%:%2408=1010%:%
%:%2409=1010%:%
%:%2410=1011%:%
%:%2411=1011%:%
%:%2412=1012%:%
%:%2413=1012%:%
%:%2414=1013%:%
%:%2415=1013%:%
%:%2416=1013%:%
%:%2417=1014%:%
%:%2418=1014%:%
%:%2419=1014%:%
%:%2420=1015%:%
%:%2421=1015%:%
%:%2422=1015%:%
%:%2423=1015%:%
%:%2424=1016%:%
%:%2425=1016%:%
%:%2426=1017%:%
%:%2432=1017%:%
%:%2435=1018%:%
%:%2436=1019%:%
%:%2437=1019%:%
%:%2438=1020%:%
%:%2439=1021%:%
%:%2442=1022%:%
%:%2446=1022%:%
%:%2447=1022%:%
%:%2448=1022%:%
%:%2449=1023%:%
%:%2450=1023%:%
%:%2451=1024%:%
%:%2452=1024%:%
%:%2453=1025%:%
%:%2454=1025%:%
%:%2455=1026%:%
%:%2456=1026%:%
%:%2457=1027%:%
%:%2458=1027%:%
%:%2459=1028%:%
%:%2460=1028%:%
%:%2461=1028%:%
%:%2462=1029%:%
%:%2463=1029%:%
%:%2464=1029%:%
%:%2465=1030%:%
%:%2466=1030%:%
%:%2467=1030%:%
%:%2468=1031%:%
%:%2469=1031%:%
%:%2470=1032%:%
%:%2471=1032%:%
%:%2472=1033%:%
%:%2473=1033%:%
%:%2474=1033%:%
%:%2475=1034%:%
%:%2476=1034%:%
%:%2477=1035%:%
%:%2478=1035%:%
%:%2479=1036%:%
%:%2480=1036%:%
%:%2481=1036%:%
%:%2482=1037%:%
%:%2483=1037%:%
%:%2484=1037%:%
%:%2485=1038%:%
%:%2486=1038%:%
%:%2487=1038%:%
%:%2488=1038%:%
%:%2489=1039%:%
%:%2490=1039%:%
%:%2491=1040%:%
%:%2497=1040%:%
%:%2500=1041%:%
%:%2501=1042%:%
%:%2502=1042%:%
%:%2503=1043%:%
%:%2504=1044%:%
%:%2511=1045%:%
%:%2512=1045%:%
%:%2513=1046%:%
%:%2514=1046%:%
%:%2515=1047%:%
%:%2516=1047%:%
%:%2517=1047%:%
%:%2518=1047%:%
%:%2519=1048%:%
%:%2520=1048%:%
%:%2521=1048%:%
%:%2522=1049%:%
%:%2523=1049%:%
%:%2524=1050%:%
%:%2525=1050%:%
%:%2526=1051%:%
%:%2527=1051%:%
%:%2528=1052%:%
%:%2529=1052%:%
%:%2530=1053%:%
%:%2531=1053%:%
%:%2532=1053%:%
%:%2533=1053%:%
%:%2534=1054%:%
%:%2535=1054%:%
%:%2536=1055%:%
%:%2537=1055%:%
%:%2538=1055%:%
%:%2539=1056%:%
%:%2540=1056%:%
%:%2541=1056%:%
%:%2542=1057%:%
%:%2543=1057%:%
%:%2544=1057%:%
%:%2545=1058%:%
%:%2546=1058%:%
%:%2547=1059%:%
%:%2548=1059%:%
%:%2549=1060%:%
%:%2550=1060%:%
%:%2551=1061%:%
%:%2552=1061%:%
%:%2553=1061%:%
%:%2554=1061%:%
%:%2555=1062%:%
%:%2556=1062%:%
%:%2557=1062%:%
%:%2558=1063%:%
%:%2559=1063%:%
%:%2560=1063%:%
%:%2561=1064%:%
%:%2562=1064%:%
%:%2563=1065%:%
%:%2564=1065%:%
%:%2565=1066%:%
%:%2566=1066%:%
%:%2567=1066%:%
%:%2568=1067%:%
%:%2569=1067%:%
%:%2570=1068%:%
%:%2571=1068%:%
%:%2572=1068%:%
%:%2573=1069%:%
%:%2574=1069%:%
%:%2575=1070%:%
%:%2576=1070%:%
%:%2577=1071%:%
%:%2578=1071%:%
%:%2579=1072%:%
%:%2580=1072%:%
%:%2581=1072%:%
%:%2582=1073%:%
%:%2583=1074%:%
%:%2584=1074%:%
%:%2585=1075%:%
%:%2586=1075%:%
%:%2587=1076%:%
%:%2588=1076%:%
%:%2589=1077%:%
%:%2590=1077%:%
%:%2591=1077%:%
%:%2592=1078%:%
%:%2593=1078%:%
%:%2594=1078%:%
%:%2595=1079%:%
%:%2596=1079%:%
%:%2597=1079%:%
%:%2598=1080%:%
%:%2599=1080%:%
%:%2600=1080%:%
%:%2601=1081%:%
%:%2602=1081%:%
%:%2603=1081%:%
%:%2604=1082%:%
%:%2605=1082%:%
%:%2606=1082%:%
%:%2607=1083%:%
%:%2608=1083%:%
%:%2609=1084%:%
%:%2610=1084%:%
%:%2611=1084%:%
%:%2612=1085%:%
%:%2613=1085%:%
%:%2614=1086%:%
%:%2615=1086%:%
%:%2616=1086%:%
%:%2617=1087%:%
%:%2618=1087%:%
%:%2619=1087%:%
%:%2620=1088%:%
%:%2621=1088%:%
%:%2622=1089%:%
%:%2628=1089%:%
%:%2631=1090%:%
%:%2632=1091%:%
%:%2633=1091%:%
%:%2634=1092%:%
%:%2635=1093%:%
%:%2642=1094%:%
%:%2643=1094%:%
%:%2644=1095%:%
%:%2645=1095%:%
%:%2646=1096%:%
%:%2647=1096%:%
%:%2648=1096%:%
%:%2649=1096%:%
%:%2650=1097%:%
%:%2651=1097%:%
%:%2652=1097%:%
%:%2653=1098%:%
%:%2654=1098%:%
%:%2655=1099%:%
%:%2656=1099%:%
%:%2657=1100%:%
%:%2658=1100%:%
%:%2659=1101%:%
%:%2660=1101%:%
%:%2661=1102%:%
%:%2662=1102%:%
%:%2663=1102%:%
%:%2664=1102%:%
%:%2665=1103%:%
%:%2666=1103%:%
%:%2667=1104%:%
%:%2668=1104%:%
%:%2669=1104%:%
%:%2670=1105%:%
%:%2671=1105%:%
%:%2672=1105%:%
%:%2673=1106%:%
%:%2674=1106%:%
%:%2675=1106%:%
%:%2676=1107%:%
%:%2677=1107%:%
%:%2678=1108%:%
%:%2679=1108%:%
%:%2680=1109%:%
%:%2681=1109%:%
%:%2682=1110%:%
%:%2683=1110%:%
%:%2684=1110%:%
%:%2685=1110%:%
%:%2686=1111%:%
%:%2687=1111%:%
%:%2688=1111%:%
%:%2689=1112%:%
%:%2690=1112%:%
%:%2691=1112%:%
%:%2692=1113%:%
%:%2693=1113%:%
%:%2694=1114%:%
%:%2695=1114%:%
%:%2696=1115%:%
%:%2697=1115%:%
%:%2698=1115%:%
%:%2699=1116%:%
%:%2700=1116%:%
%:%2701=1117%:%
%:%2702=1117%:%
%:%2703=1117%:%
%:%2704=1118%:%
%:%2705=1118%:%
%:%2706=1119%:%
%:%2707=1119%:%
%:%2708=1120%:%
%:%2709=1120%:%
%:%2710=1121%:%
%:%2711=1121%:%
%:%2712=1121%:%
%:%2713=1122%:%
%:%2714=1122%:%
%:%2715=1123%:%
%:%2716=1123%:%
%:%2717=1124%:%
%:%2718=1124%:%
%:%2719=1125%:%
%:%2720=1125%:%
%:%2721=1125%:%
%:%2722=1126%:%
%:%2723=1126%:%
%:%2724=1126%:%
%:%2725=1127%:%
%:%2726=1127%:%
%:%2727=1127%:%
%:%2728=1128%:%
%:%2729=1128%:%
%:%2730=1128%:%
%:%2731=1129%:%
%:%2732=1129%:%
%:%2733=1129%:%
%:%2734=1130%:%
%:%2735=1130%:%
%:%2736=1130%:%
%:%2737=1131%:%
%:%2738=1131%:%
%:%2739=1132%:%
%:%2740=1132%:%
%:%2741=1132%:%
%:%2742=1133%:%
%:%2743=1133%:%
%:%2744=1134%:%
%:%2745=1134%:%
%:%2746=1134%:%
%:%2747=1135%:%
%:%2748=1135%:%
%:%2749=1135%:%
%:%2750=1136%:%
%:%2751=1136%:%
%:%2752=1137%:%
%:%2758=1137%:%
%:%2761=1138%:%
%:%2762=1139%:%
%:%2763=1139%:%
%:%2764=1140%:%
%:%2765=1141%:%
%:%2772=1142%:%
%:%2773=1142%:%
%:%2774=1143%:%
%:%2775=1143%:%
%:%2776=1144%:%
%:%2777=1144%:%
%:%2778=1144%:%
%:%2779=1145%:%
%:%2780=1145%:%
%:%2781=1145%:%
%:%2782=1146%:%
%:%2783=1146%:%
%:%2784=1147%:%
%:%2785=1147%:%
%:%2786=1148%:%
%:%2787=1148%:%
%:%2788=1149%:%
%:%2789=1149%:%
%:%2790=1149%:%
%:%2791=1150%:%
%:%2792=1150%:%
%:%2793=1151%:%
%:%2794=1151%:%
%:%2795=1152%:%
%:%2796=1152%:%
%:%2797=1153%:%
%:%2798=1153%:%
%:%2799=1154%:%
%:%2800=1155%:%
%:%2801=1155%:%
%:%2802=1156%:%
%:%2803=1156%:%
%:%2804=1156%:%
%:%2805=1157%:%
%:%2806=1157%:%
%:%2807=1158%:%
%:%2808=1158%:%
%:%2809=1158%:%
%:%2810=1159%:%
%:%2811=1160%:%
%:%2812=1160%:%
%:%2813=1161%:%
%:%2814=1161%:%
%:%2815=1162%:%
%:%2816=1162%:%
%:%2817=1163%:%
%:%2818=1163%:%
%:%2819=1164%:%
%:%2820=1164%:%
%:%2821=1164%:%
%:%2822=1165%:%
%:%2823=1166%:%
%:%2824=1166%:%
%:%2825=1167%:%
%:%2826=1167%:%
%:%2827=1167%:%
%:%2828=1167%:%
%:%2829=1168%:%
%:%2830=1168%:%
%:%2831=1168%:%
%:%2832=1169%:%
%:%2833=1170%:%
%:%2834=1170%:%
%:%2835=1170%:%
%:%2836=1171%:%
%:%2837=1171%:%
%:%2838=1171%:%
%:%2839=1172%:%
%:%2840=1173%:%
%:%2841=1173%:%
%:%2842=1173%:%
%:%2843=1174%:%
%:%2844=1174%:%
%:%2845=1174%:%
%:%2846=1175%:%
%:%2847=1175%:%
%:%2848=1176%:%
%:%2849=1176%:%
%:%2850=1176%:%
%:%2851=1177%:%
%:%2852=1177%:%
%:%2853=1177%:%
%:%2854=1178%:%
%:%2855=1179%:%
%:%2856=1179%:%
%:%2857=1180%:%
%:%2858=1180%:%
%:%2859=1181%:%
%:%2860=1181%:%
%:%2861=1182%:%
%:%2862=1182%:%
%:%2863=1182%:%
%:%2864=1183%:%
%:%2865=1183%:%
%:%2866=1183%:%
%:%2867=1184%:%
%:%2868=1184%:%
%:%2869=1185%:%
%:%2870=1185%:%
%:%2871=1185%:%
%:%2872=1186%:%
%:%2873=1186%:%
%:%2874=1186%:%
%:%2875=1187%:%
%:%2876=1187%:%
%:%2877=1187%:%
%:%2878=1188%:%
%:%2879=1188%:%
%:%2880=1188%:%
%:%2881=1189%:%
%:%2882=1189%:%
%:%2883=1189%:%
%:%2884=1190%:%
%:%2885=1190%:%
%:%2886=1190%:%
%:%2887=1191%:%
%:%2888=1191%:%
%:%2889=1192%:%
%:%2890=1192%:%
%:%2891=1192%:%
%:%2892=1193%:%
%:%2893=1193%:%
%:%2894=1193%:%
%:%2895=1194%:%
%:%2896=1194%:%
%:%2897=1194%:%
%:%2898=1195%:%
%:%2899=1195%:%
%:%2900=1195%:%
%:%2901=1196%:%
%:%2902=1196%:%
%:%2903=1196%:%
%:%2904=1197%:%
%:%2905=1197%:%
%:%2906=1197%:%
%:%2907=1198%:%
%:%2908=1199%:%
%:%2909=1199%:%
%:%2910=1200%:%
%:%2911=1200%:%
%:%2912=1201%:%
%:%2913=1201%:%
%:%2914=1202%:%
%:%2915=1202%:%
%:%2916=1203%:%
%:%2917=1203%:%
%:%2918=1203%:%
%:%2919=1204%:%
%:%2920=1204%:%
%:%2921=1205%:%
%:%2922=1205%:%
%:%2923=1205%:%
%:%2924=1206%:%
%:%2925=1206%:%
%:%2926=1206%:%
%:%2927=1207%:%
%:%2928=1207%:%
%:%2929=1207%:%
%:%2930=1208%:%
%:%2931=1208%:%
%:%2932=1208%:%
%:%2933=1209%:%
%:%2934=1209%:%
%:%2935=1209%:%
%:%2936=1210%:%
%:%2937=1210%:%
%:%2938=1210%:%
%:%2939=1211%:%
%:%2940=1211%:%
%:%2941=1211%:%
%:%2942=1212%:%
%:%2943=1212%:%
%:%2944=1212%:%
%:%2945=1213%:%
%:%2946=1213%:%
%:%2947=1214%:%
%:%2948=1214%:%
%:%2949=1214%:%
%:%2950=1215%:%
%:%2951=1215%:%
%:%2952=1216%:%
%:%2953=1216%:%
%:%2954=1217%:%
%:%2960=1217%:%
%:%2963=1218%:%
%:%2964=1219%:%
%:%2965=1219%:%
%:%2966=1220%:%
%:%2967=1221%:%
%:%2974=1222%:%
%:%2975=1222%:%
%:%2976=1223%:%
%:%2977=1223%:%
%:%2978=1224%:%
%:%2979=1224%:%
%:%2980=1224%:%
%:%2981=1225%:%
%:%2982=1225%:%
%:%2983=1225%:%
%:%2984=1226%:%
%:%2985=1226%:%
%:%2986=1227%:%
%:%2987=1227%:%
%:%2988=1228%:%
%:%2989=1228%:%
%:%2990=1229%:%
%:%2991=1229%:%
%:%2992=1229%:%
%:%2993=1230%:%
%:%2994=1231%:%
%:%2995=1231%:%
%:%2996=1232%:%
%:%2997=1232%:%
%:%2998=1233%:%
%:%2999=1233%:%
%:%3000=1234%:%
%:%3001=1234%:%
%:%3002=1235%:%
%:%3003=1236%:%
%:%3004=1236%:%
%:%3005=1237%:%
%:%3006=1237%:%
%:%3007=1237%:%
%:%3008=1238%:%
%:%3009=1238%:%
%:%3010=1239%:%
%:%3011=1239%:%
%:%3012=1239%:%
%:%3013=1240%:%
%:%3014=1241%:%
%:%3015=1241%:%
%:%3016=1242%:%
%:%3017=1242%:%
%:%3018=1243%:%
%:%3019=1243%:%
%:%3020=1244%:%
%:%3021=1244%:%
%:%3022=1245%:%
%:%3023=1245%:%
%:%3024=1245%:%
%:%3025=1246%:%
%:%3026=1247%:%
%:%3027=1247%:%
%:%3028=1248%:%
%:%3029=1248%:%
%:%3030=1248%:%
%:%3031=1248%:%
%:%3032=1249%:%
%:%3033=1249%:%
%:%3034=1249%:%
%:%3035=1250%:%
%:%3036=1251%:%
%:%3037=1251%:%
%:%3038=1251%:%
%:%3039=1252%:%
%:%3040=1252%:%
%:%3041=1252%:%
%:%3042=1253%:%
%:%3043=1254%:%
%:%3044=1254%:%
%:%3045=1254%:%
%:%3046=1255%:%
%:%3047=1255%:%
%:%3048=1255%:%
%:%3049=1256%:%
%:%3050=1256%:%
%:%3051=1257%:%
%:%3052=1257%:%
%:%3053=1257%:%
%:%3054=1258%:%
%:%3055=1258%:%
%:%3056=1258%:%
%:%3057=1259%:%
%:%3058=1260%:%
%:%3059=1260%:%
%:%3060=1261%:%
%:%3061=1261%:%
%:%3062=1262%:%
%:%3063=1262%:%
%:%3064=1263%:%
%:%3065=1263%:%
%:%3066=1263%:%
%:%3067=1264%:%
%:%3068=1264%:%
%:%3069=1264%:%
%:%3070=1265%:%
%:%3071=1265%:%
%:%3072=1266%:%
%:%3073=1266%:%
%:%3074=1266%:%
%:%3075=1267%:%
%:%3076=1267%:%
%:%3077=1267%:%
%:%3078=1268%:%
%:%3079=1268%:%
%:%3080=1268%:%
%:%3081=1269%:%
%:%3082=1269%:%
%:%3083=1269%:%
%:%3084=1270%:%
%:%3085=1270%:%
%:%3086=1270%:%
%:%3087=1271%:%
%:%3088=1271%:%
%:%3089=1271%:%
%:%3090=1272%:%
%:%3091=1272%:%
%:%3092=1273%:%
%:%3093=1273%:%
%:%3094=1273%:%
%:%3095=1274%:%
%:%3096=1274%:%
%:%3097=1274%:%
%:%3098=1275%:%
%:%3099=1275%:%
%:%3100=1275%:%
%:%3101=1276%:%
%:%3102=1276%:%
%:%3103=1276%:%
%:%3104=1277%:%
%:%3105=1277%:%
%:%3106=1277%:%
%:%3107=1278%:%
%:%3108=1278%:%
%:%3109=1278%:%
%:%3110=1279%:%
%:%3111=1280%:%
%:%3112=1280%:%
%:%3113=1281%:%
%:%3114=1281%:%
%:%3115=1282%:%
%:%3116=1282%:%
%:%3117=1283%:%
%:%3118=1283%:%
%:%3119=1284%:%
%:%3120=1284%:%
%:%3121=1284%:%
%:%3122=1285%:%
%:%3123=1285%:%
%:%3124=1286%:%
%:%3125=1286%:%
%:%3126=1286%:%
%:%3127=1287%:%
%:%3128=1287%:%
%:%3129=1287%:%
%:%3130=1288%:%
%:%3131=1288%:%
%:%3132=1288%:%
%:%3133=1289%:%
%:%3134=1289%:%
%:%3135=1289%:%
%:%3136=1290%:%
%:%3137=1290%:%
%:%3138=1290%:%
%:%3139=1291%:%
%:%3140=1291%:%
%:%3141=1291%:%
%:%3142=1292%:%
%:%3143=1292%:%
%:%3144=1292%:%
%:%3145=1293%:%
%:%3146=1293%:%
%:%3147=1293%:%
%:%3148=1294%:%
%:%3149=1294%:%
%:%3150=1295%:%
%:%3151=1295%:%
%:%3152=1295%:%
%:%3153=1296%:%
%:%3154=1296%:%
%:%3155=1297%:%
%:%3156=1297%:%
%:%3157=1298%:%
%:%3163=1298%:%
%:%3166=1299%:%
%:%3167=1300%:%
%:%3168=1300%:%
%:%3169=1301%:%
%:%3170=1302%:%
%:%3177=1303%:%
%:%3178=1303%:%
%:%3179=1304%:%
%:%3180=1304%:%
%:%3181=1305%:%
%:%3182=1305%:%
%:%3183=1305%:%
%:%3184=1306%:%
%:%3185=1306%:%
%:%3186=1306%:%
%:%3187=1307%:%
%:%3188=1307%:%
%:%3189=1308%:%
%:%3190=1308%:%
%:%3191=1309%:%
%:%3192=1309%:%
%:%3193=1310%:%
%:%3194=1310%:%
%:%3195=1310%:%
%:%3196=1311%:%
%:%3197=1312%:%
%:%3198=1312%:%
%:%3199=1313%:%
%:%3200=1313%:%
%:%3201=1314%:%
%:%3202=1314%:%
%:%3203=1315%:%
%:%3204=1315%:%
%:%3205=1316%:%
%:%3206=1317%:%
%:%3207=1317%:%
%:%3208=1318%:%
%:%3209=1318%:%
%:%3210=1318%:%
%:%3211=1319%:%
%:%3212=1319%:%
%:%3213=1320%:%
%:%3214=1320%:%
%:%3215=1320%:%
%:%3216=1321%:%
%:%3217=1321%:%
%:%3218=1321%:%
%:%3219=1322%:%
%:%3220=1322%:%
%:%3221=1322%:%
%:%3222=1323%:%
%:%3223=1323%:%
%:%3224=1323%:%
%:%3225=1324%:%
%:%3226=1325%:%
%:%3227=1325%:%
%:%3228=1326%:%
%:%3229=1326%:%
%:%3230=1327%:%
%:%3231=1327%:%
%:%3232=1328%:%
%:%3233=1328%:%
%:%3234=1328%:%
%:%3235=1329%:%
%:%3236=1329%:%
%:%3237=1329%:%
%:%3238=1330%:%
%:%3239=1330%:%
%:%3240=1331%:%
%:%3241=1331%:%
%:%3242=1331%:%
%:%3243=1332%:%
%:%3244=1332%:%
%:%3245=1333%:%
%:%3246=1333%:%
%:%3247=1333%:%
%:%3248=1334%:%
%:%3249=1334%:%
%:%3250=1334%:%
%:%3251=1335%:%
%:%3252=1335%:%
%:%3253=1335%:%
%:%3254=1336%:%
%:%3255=1336%:%
%:%3256=1336%:%
%:%3257=1337%:%
%:%3258=1337%:%
%:%3259=1338%:%
%:%3260=1338%:%
%:%3261=1338%:%
%:%3262=1339%:%
%:%3263=1339%:%
%:%3264=1339%:%
%:%3265=1340%:%
%:%3266=1341%:%
%:%3267=1341%:%
%:%3268=1342%:%
%:%3269=1342%:%
%:%3270=1342%:%
%:%3271=1343%:%
%:%3272=1343%:%
%:%3273=1344%:%
%:%3274=1344%:%
%:%3275=1344%:%
%:%3276=1345%:%
%:%3277=1345%:%
%:%3278=1345%:%
%:%3279=1346%:%
%:%3280=1346%:%
%:%3281=1346%:%
%:%3282=1347%:%
%:%3283=1347%:%
%:%3284=1347%:%
%:%3285=1348%:%
%:%3286=1349%:%
%:%3287=1349%:%
%:%3288=1350%:%
%:%3289=1350%:%
%:%3290=1351%:%
%:%3291=1351%:%
%:%3292=1352%:%
%:%3293=1352%:%
%:%3294=1352%:%
%:%3295=1353%:%
%:%3296=1353%:%
%:%3297=1353%:%
%:%3298=1354%:%
%:%3299=1354%:%
%:%3300=1355%:%
%:%3301=1355%:%
%:%3302=1355%:%
%:%3303=1356%:%
%:%3304=1356%:%
%:%3305=1357%:%
%:%3306=1357%:%
%:%3307=1357%:%
%:%3308=1358%:%
%:%3309=1358%:%
%:%3310=1358%:%
%:%3311=1359%:%
%:%3312=1359%:%
%:%3313=1359%:%
%:%3314=1360%:%
%:%3315=1360%:%
%:%3316=1360%:%
%:%3317=1361%:%
%:%3318=1361%:%
%:%3319=1362%:%
%:%3320=1362%:%
%:%3321=1362%:%
%:%3322=1363%:%
%:%3323=1363%:%
%:%3324=1363%:%
%:%3325=1364%:%
%:%3326=1365%:%
%:%3327=1365%:%
%:%3328=1366%:%
%:%3329=1366%:%
%:%3330=1366%:%
%:%3331=1367%:%
%:%3332=1367%:%
%:%3333=1368%:%
%:%3334=1368%:%
%:%3335=1368%:%
%:%3336=1369%:%
%:%3337=1370%:%
%:%3338=1370%:%
%:%3339=1371%:%
%:%3340=1371%:%
%:%3341=1372%:%
%:%3342=1372%:%
%:%3343=1373%:%
%:%3344=1373%:%
%:%3345=1374%:%
%:%3346=1374%:%
%:%3347=1375%:%
%:%3348=1375%:%
%:%3349=1376%:%
%:%3350=1377%:%
%:%3351=1377%:%
%:%3352=1378%:%
%:%3353=1378%:%
%:%3354=1378%:%
%:%3355=1379%:%
%:%3356=1379%:%
%:%3357=1380%:%
%:%3358=1380%:%
%:%3359=1380%:%
%:%3360=1381%:%
%:%3361=1381%:%
%:%3362=1382%:%
%:%3363=1383%:%
%:%3364=1383%:%
%:%3365=1384%:%
%:%3366=1384%:%
%:%3367=1385%:%
%:%3368=1385%:%
%:%3369=1386%:%
%:%3370=1386%:%
%:%3371=1386%:%
%:%3372=1387%:%
%:%3373=1388%:%
%:%3374=1388%:%
%:%3375=1389%:%
%:%3376=1389%:%
%:%3377=1390%:%
%:%3378=1390%:%
%:%3379=1391%:%
%:%3380=1391%:%
%:%3381=1392%:%
%:%3382=1392%:%
%:%3383=1393%:%
%:%3384=1393%:%
%:%3385=1393%:%
%:%3386=1394%:%
%:%3387=1394%:%
%:%3388=1395%:%
%:%3389=1395%:%
%:%3390=1395%:%
%:%3391=1396%:%
%:%3392=1396%:%
%:%3393=1396%:%
%:%3394=1397%:%
%:%3395=1397%:%
%:%3396=1397%:%
%:%3397=1398%:%
%:%3398=1398%:%
%:%3399=1398%:%
%:%3400=1399%:%
%:%3401=1399%:%
%:%3402=1400%:%
%:%3403=1400%:%
%:%3404=1401%:%
%:%3410=1401%:%
%:%3413=1402%:%
%:%3414=1403%:%
%:%3415=1403%:%
%:%3416=1404%:%
%:%3417=1405%:%
%:%3424=1406%:%
%:%3425=1406%:%
%:%3426=1407%:%
%:%3427=1407%:%
%:%3428=1408%:%
%:%3429=1408%:%
%:%3430=1408%:%
%:%3431=1409%:%
%:%3432=1409%:%
%:%3433=1409%:%
%:%3434=1410%:%
%:%3435=1410%:%
%:%3436=1411%:%
%:%3437=1411%:%
%:%3438=1412%:%
%:%3439=1412%:%
%:%3440=1413%:%
%:%3441=1413%:%
%:%3442=1413%:%
%:%3443=1414%:%
%:%3444=1415%:%
%:%3445=1415%:%
%:%3446=1416%:%
%:%3447=1416%:%
%:%3448=1417%:%
%:%3449=1417%:%
%:%3450=1418%:%
%:%3451=1418%:%
%:%3452=1419%:%
%:%3453=1419%:%
%:%3454=1420%:%
%:%3455=1420%:%
%:%3456=1420%:%
%:%3457=1421%:%
%:%3458=1421%:%
%:%3459=1422%:%
%:%3460=1422%:%
%:%3461=1422%:%
%:%3462=1423%:%
%:%3463=1423%:%
%:%3464=1423%:%
%:%3465=1424%:%
%:%3466=1424%:%
%:%3467=1424%:%
%:%3468=1425%:%
%:%3469=1425%:%
%:%3470=1425%:%
%:%3471=1426%:%
%:%3472=1427%:%
%:%3473=1427%:%
%:%3474=1428%:%
%:%3475=1428%:%
%:%3476=1429%:%
%:%3477=1429%:%
%:%3478=1430%:%
%:%3479=1430%:%
%:%3480=1430%:%
%:%3481=1431%:%
%:%3482=1431%:%
%:%3483=1431%:%
%:%3484=1432%:%
%:%3485=1432%:%
%:%3486=1433%:%
%:%3487=1433%:%
%:%3488=1433%:%
%:%3489=1434%:%
%:%3490=1434%:%
%:%3491=1435%:%
%:%3492=1435%:%
%:%3493=1435%:%
%:%3494=1436%:%
%:%3495=1436%:%
%:%3496=1436%:%
%:%3497=1437%:%
%:%3498=1437%:%
%:%3499=1437%:%
%:%3500=1438%:%
%:%3501=1438%:%
%:%3502=1438%:%
%:%3503=1439%:%
%:%3504=1439%:%
%:%3505=1440%:%
%:%3506=1440%:%
%:%3507=1440%:%
%:%3508=1441%:%
%:%3509=1441%:%
%:%3510=1441%:%
%:%3511=1442%:%
%:%3512=1443%:%
%:%3513=1443%:%
%:%3514=1444%:%
%:%3515=1445%:%
%:%3516=1445%:%
%:%3517=1445%:%
%:%3518=1446%:%
%:%3519=1446%:%
%:%3520=1447%:%
%:%3521=1447%:%
%:%3522=1447%:%
%:%3523=1448%:%
%:%3524=1448%:%
%:%3525=1448%:%
%:%3526=1449%:%
%:%3527=1449%:%
%:%3528=1449%:%
%:%3529=1450%:%
%:%3530=1450%:%
%:%3531=1450%:%
%:%3532=1451%:%
%:%3533=1452%:%
%:%3534=1452%:%
%:%3535=1453%:%
%:%3536=1453%:%
%:%3537=1454%:%
%:%3538=1454%:%
%:%3539=1455%:%
%:%3540=1455%:%
%:%3541=1455%:%
%:%3542=1456%:%
%:%3543=1456%:%
%:%3544=1456%:%
%:%3545=1457%:%
%:%3546=1457%:%
%:%3547=1458%:%
%:%3548=1458%:%
%:%3549=1458%:%
%:%3550=1459%:%
%:%3551=1459%:%
%:%3552=1460%:%
%:%3553=1460%:%
%:%3554=1460%:%
%:%3555=1461%:%
%:%3556=1461%:%
%:%3557=1461%:%
%:%3558=1462%:%
%:%3559=1462%:%
%:%3560=1462%:%
%:%3561=1463%:%
%:%3562=1463%:%
%:%3563=1463%:%
%:%3564=1464%:%
%:%3565=1464%:%
%:%3566=1465%:%
%:%3567=1465%:%
%:%3568=1465%:%
%:%3569=1466%:%
%:%3570=1466%:%
%:%3571=1466%:%
%:%3572=1467%:%
%:%3573=1467%:%
%:%3574=1468%:%
%:%3575=1468%:%
%:%3576=1468%:%
%:%3577=1469%:%
%:%3578=1469%:%
%:%3579=1470%:%
%:%3580=1470%:%
%:%3581=1470%:%
%:%3582=1471%:%
%:%3583=1471%:%
%:%3584=1472%:%
%:%3585=1472%:%
%:%3586=1473%:%
%:%3587=1473%:%
%:%3588=1474%:%
%:%3589=1474%:%
%:%3590=1475%:%
%:%3591=1475%:%
%:%3592=1476%:%
%:%3593=1476%:%
%:%3594=1477%:%
%:%3595=1477%:%
%:%3596=1478%:%
%:%3597=1478%:%
%:%3598=1478%:%
%:%3599=1479%:%
%:%3600=1479%:%
%:%3601=1480%:%
%:%3602=1480%:%
%:%3603=1480%:%
%:%3604=1481%:%
%:%3605=1481%:%
%:%3606=1482%:%
%:%3607=1482%:%
%:%3608=1483%:%
%:%3609=1483%:%
%:%3610=1484%:%
%:%3611=1484%:%
%:%3612=1485%:%
%:%3613=1485%:%
%:%3614=1485%:%
%:%3615=1486%:%
%:%3616=1486%:%
%:%3617=1487%:%
%:%3618=1487%:%
%:%3619=1488%:%
%:%3620=1488%:%
%:%3621=1489%:%
%:%3622=1489%:%
%:%3623=1490%:%
%:%3624=1490%:%
%:%3625=1491%:%
%:%3626=1491%:%
%:%3627=1491%:%
%:%3628=1492%:%
%:%3629=1492%:%
%:%3630=1493%:%
%:%3631=1493%:%
%:%3632=1493%:%
%:%3633=1494%:%
%:%3634=1494%:%
%:%3635=1495%:%
%:%3636=1495%:%
%:%3637=1495%:%
%:%3638=1496%:%
%:%3639=1496%:%
%:%3640=1496%:%
%:%3641=1497%:%
%:%3642=1497%:%
%:%3643=1498%:%
%:%3644=1498%:%
%:%3645=1499%:%
%:%3651=1499%:%
%:%3654=1500%:%
%:%3655=1501%:%
%:%3656=1501%:%
%:%3657=1502%:%
%:%3658=1503%:%
%:%3661=1504%:%
%:%3665=1504%:%
%:%3666=1504%:%
%:%3667=1505%:%
%:%3668=1505%:%
%:%3673=1505%:%
%:%3676=1506%:%
%:%3677=1507%:%
%:%3678=1507%:%
%:%3679=1508%:%
%:%3680=1509%:%
%:%3683=1510%:%
%:%3687=1510%:%
%:%3688=1510%:%
%:%3689=1511%:%
%:%3690=1511%:%
%:%3704=1513%:%
%:%3714=1515%:%
%:%3715=1515%:%
%:%3716=1516%:%
%:%3720=1520%:%
%:%3722=1521%:%
%:%3723=1521%:%
%:%3724=1522%:%
%:%3725=1523%:%
%:%3726=1524%:%
%:%3727=1524%:%
%:%3728=1525%:%
%:%3731=1526%:%
%:%3735=1526%:%
%:%3736=1526%:%
%:%3741=1526%:%
%:%3744=1527%:%
%:%3745=1528%:%
%:%3746=1528%:%
%:%3747=1529%:%
%:%3748=1530%:%
%:%3751=1531%:%
%:%3755=1531%:%
%:%3756=1531%:%
%:%3757=1531%:%
%:%3762=1531%:%
%:%3765=1532%:%
%:%3766=1533%:%
%:%3767=1533%:%
%:%3768=1534%:%
%:%3769=1535%:%
%:%3770=1536%:%
%:%3771=1536%:%
%:%3772=1537%:%
%:%3775=1538%:%
%:%3779=1538%:%
%:%3780=1538%:%
%:%3781=1538%:%
%:%3786=1538%:%
%:%3789=1539%:%
%:%3790=1540%:%
%:%3791=1540%:%
%:%3792=1541%:%
%:%3793=1542%:%
%:%3796=1543%:%
%:%3800=1543%:%
%:%3801=1543%:%
%:%3802=1543%:%
%:%3807=1543%:%
%:%3810=1544%:%
%:%3811=1545%:%
%:%3812=1545%:%
%:%3813=1546%:%
%:%3814=1547%:%
%:%3817=1548%:%
%:%3821=1548%:%
%:%3822=1548%:%
%:%3827=1548%:%
%:%3830=1549%:%
%:%3831=1550%:%
%:%3832=1550%:%
%:%3833=1551%:%
%:%3834=1552%:%
%:%3837=1553%:%
%:%3841=1553%:%
%:%3842=1553%:%
%:%3843=1553%:%
%:%3852=1555%:%
%:%3854=1556%:%
%:%3855=1556%:%
%:%3856=1557%:%
%:%3857=1558%:%
%:%3864=1559%:%
%:%3865=1559%:%
%:%3866=1560%:%
%:%3867=1560%:%
%:%3868=1561%:%
%:%3869=1561%:%
%:%3870=1562%:%
%:%3871=1562%:%
%:%3872=1563%:%
%:%3873=1563%:%
%:%3874=1564%:%
%:%3875=1564%:%
%:%3876=1565%:%
%:%3877=1565%:%
%:%3878=1566%:%
%:%3879=1566%:%
%:%3880=1567%:%
%:%3881=1567%:%
%:%3882=1568%:%
%:%3883=1568%:%
%:%3884=1568%:%
%:%3885=1569%:%
%:%3886=1569%:%
%:%3887=1569%:%
%:%3888=1570%:%
%:%3889=1571%:%
%:%3890=1571%:%
%:%3891=1572%:%
%:%3892=1572%:%
%:%3893=1573%:%
%:%3894=1573%:%
%:%3895=1574%:%
%:%3896=1574%:%
%:%3897=1575%:%
%:%3898=1575%:%
%:%3899=1576%:%
%:%3900=1576%:%
%:%3901=1576%:%
%:%3902=1577%:%
%:%3903=1577%:%
%:%3904=1578%:%
%:%3905=1578%:%
%:%3906=1578%:%
%:%3907=1579%:%
%:%3908=1579%:%
%:%3909=1580%:%
%:%3910=1580%:%
%:%3911=1581%:%
%:%3912=1582%:%
%:%3913=1583%:%
%:%3914=1583%:%
%:%3915=1584%:%
%:%3916=1584%:%
%:%3917=1584%:%
%:%3918=1585%:%
%:%3919=1585%:%
%:%3920=1586%:%
%:%3921=1586%:%
%:%3922=1587%:%
%:%3923=1587%:%
%:%3924=1588%:%
%:%3925=1588%:%
%:%3926=1589%:%
%:%3927=1589%:%
%:%3928=1590%:%
%:%3929=1590%:%
%:%3930=1591%:%
%:%3931=1591%:%
%:%3932=1592%:%
%:%3933=1592%:%
%:%3934=1593%:%
%:%3935=1593%:%
%:%3936=1594%:%
%:%3937=1594%:%
%:%3938=1595%:%
%:%3939=1595%:%
%:%3940=1596%:%
%:%3941=1596%:%
%:%3942=1596%:%
%:%3943=1597%:%
%:%3944=1597%:%
%:%3945=1597%:%
%:%3946=1598%:%
%:%3947=1598%:%
%:%3948=1599%:%
%:%3949=1599%:%
%:%3955=1605%:%
%:%3956=1606%:%
%:%3957=1606%:%
%:%3958=1607%:%
%:%3959=1607%:%
%:%3960=1608%:%
%:%3966=1608%:%
%:%3969=1609%:%
%:%3970=1610%:%
%:%3971=1610%:%
%:%3972=1611%:%
%:%3973=1612%:%
%:%3980=1613%:%
%:%3981=1613%:%
%:%3982=1614%:%
%:%3983=1614%:%
%:%3984=1615%:%
%:%3985=1615%:%
%:%3986=1615%:%
%:%3987=1616%:%
%:%3988=1616%:%
%:%3989=1617%:%
%:%3990=1617%:%
%:%3991=1617%:%
%:%3992=1618%:%
%:%3993=1618%:%
%:%3994=1619%:%
%:%3995=1619%:%
%:%3996=1620%:%
%:%3997=1620%:%
%:%3998=1621%:%
%:%3999=1621%:%
%:%4000=1622%:%
%:%4001=1622%:%
%:%4002=1622%:%
%:%4003=1623%:%
%:%4004=1623%:%
%:%4005=1623%:%
%:%4006=1624%:%
%:%4007=1624%:%
%:%4008=1624%:%
%:%4009=1625%:%
%:%4010=1625%:%
%:%4011=1625%:%
%:%4012=1626%:%
%:%4013=1626%:%
%:%4014=1626%:%
%:%4015=1627%:%
%:%4016=1627%:%
%:%4017=1628%:%
%:%4018=1628%:%
%:%4019=1629%:%
%:%4020=1629%:%
%:%4021=1630%:%
%:%4022=1630%:%
%:%4023=1631%:%
%:%4024=1631%:%
%:%4025=1632%:%
%:%4026=1632%:%
%:%4027=1632%:%
%:%4028=1632%:%
%:%4029=1633%:%
%:%4030=1633%:%
%:%4031=1634%:%
%:%4032=1634%:%
%:%4033=1634%:%
%:%4034=1634%:%
%:%4035=1635%:%
%:%4036=1635%:%
%:%4037=1636%:%
%:%4038=1636%:%
%:%4039=1637%:%
%:%4040=1637%:%
%:%4041=1637%:%
%:%4042=1638%:%
%:%4043=1638%:%
%:%4044=1638%:%
%:%4045=1639%:%
%:%4046=1639%:%
%:%4047=1640%:%
%:%4048=1640%:%
%:%4049=1640%:%
%:%4050=1641%:%
%:%4051=1641%:%
%:%4052=1641%:%
%:%4053=1642%:%
%:%4054=1642%:%
%:%4055=1642%:%
%:%4056=1643%:%
%:%4057=1643%:%
%:%4058=1644%:%
%:%4059=1644%:%
%:%4060=1644%:%
%:%4061=1645%:%
%:%4062=1645%:%
%:%4063=1645%:%
%:%4064=1646%:%
%:%4065=1646%:%
%:%4066=1646%:%
%:%4067=1647%:%
%:%4068=1647%:%
%:%4069=1648%:%
%:%4070=1648%:%
%:%4071=1648%:%
%:%4072=1649%:%
%:%4073=1649%:%
%:%4074=1650%:%
%:%4075=1650%:%
%:%4076=1650%:%
%:%4077=1651%:%
%:%4078=1651%:%
%:%4079=1651%:%
%:%4080=1652%:%
%:%4081=1652%:%
%:%4082=1653%:%
%:%4083=1653%:%
%:%4084=1653%:%
%:%4085=1654%:%
%:%4086=1654%:%
%:%4087=1655%:%
%:%4088=1655%:%
%:%4089=1656%:%
%:%4090=1656%:%
%:%4091=1657%:%
%:%4092=1657%:%
%:%4093=1658%:%
%:%4103=1660%:%
%:%4105=1661%:%
%:%4106=1661%:%
%:%4107=1662%:%
%:%4108=1663%:%
%:%4109=1664%:%
%:%4116=1665%:%
%:%4117=1665%:%
%:%4118=1666%:%
%:%4119=1666%:%
%:%4120=1667%:%
%:%4121=1667%:%
%:%4122=1667%:%
%:%4123=1667%:%
%:%4124=1668%:%
%:%4125=1668%:%
%:%4126=1669%:%
%:%4127=1669%:%
%:%4128=1669%:%
%:%4129=1670%:%
%:%4130=1670%:%
%:%4131=1671%:%
%:%4132=1671%:%
%:%4133=1671%:%
%:%4134=1672%:%
%:%4135=1673%:%
%:%4136=1673%:%
%:%4137=1674%:%
%:%4138=1674%:%
%:%4139=1675%:%
%:%4140=1675%:%
%:%4141=1676%:%
%:%4142=1676%:%
%:%4143=1677%:%
%:%4144=1677%:%
%:%4145=1677%:%
%:%4146=1678%:%
%:%4147=1678%:%
%:%4148=1679%:%
%:%4149=1680%:%
%:%4150=1680%:%
%:%4151=1681%:%
%:%4152=1681%:%
%:%4153=1681%:%
%:%4154=1682%:%
%:%4155=1683%:%
%:%4156=1683%:%
%:%4157=1684%:%
%:%4158=1684%:%
%:%4159=1685%:%
%:%4160=1685%:%
%:%4161=1686%:%
%:%4162=1686%:%
%:%4163=1687%:%
%:%4164=1687%:%
%:%4165=1687%:%
%:%4166=1688%:%
%:%4167=1689%:%
%:%4168=1689%:%
%:%4169=1689%:%
%:%4170=1689%:%
%:%4171=1690%:%
%:%4172=1690%:%
%:%4173=1690%:%
%:%4174=1691%:%
%:%4175=1691%:%
%:%4176=1692%:%
%:%4177=1692%:%
%:%4178=1693%:%
%:%4179=1693%:%
%:%4180=1694%:%
%:%4181=1694%:%
%:%4182=1695%:%
%:%4183=1695%:%
%:%4184=1696%:%
%:%4185=1697%:%
%:%4186=1697%:%
%:%4187=1698%:%
%:%4188=1698%:%
%:%4189=1698%:%
%:%4190=1699%:%
%:%4191=1699%:%
%:%4192=1699%:%
%:%4193=1700%:%
%:%4194=1701%:%
%:%4195=1701%:%
%:%4196=1702%:%
%:%4197=1703%:%
%:%4198=1703%:%
%:%4199=1704%:%
%:%4200=1705%:%
%:%4201=1705%:%
%:%4202=1706%:%
%:%4203=1706%:%
%:%4204=1706%:%
%:%4205=1706%:%
%:%4206=1707%:%
%:%4207=1708%:%
%:%4208=1708%:%
%:%4209=1709%:%
%:%4210=1709%:%
%:%4211=1710%:%
%:%4212=1710%:%
%:%4213=1711%:%
%:%4214=1711%:%
%:%4215=1712%:%
%:%4216=1712%:%
%:%4217=1713%:%
%:%4218=1713%:%
%:%4219=1714%:%
%:%4220=1714%:%
%:%4221=1714%:%
%:%4222=1715%:%
%:%4223=1715%:%
%:%4224=1715%:%
%:%4225=1715%:%
%:%4226=1716%:%
%:%4227=1716%:%
%:%4228=1716%:%
%:%4229=1717%:%
%:%4230=1717%:%
%:%4231=1717%:%
%:%4232=1718%:%
%:%4233=1718%:%
%:%4234=1718%:%
%:%4235=1718%:%
%:%4236=1718%:%
%:%4237=1719%:%
%:%4238=1719%:%
%:%4239=1720%:%
%:%4240=1721%:%
%:%4241=1721%:%
%:%4242=1722%:%
%:%4243=1722%:%
%:%4244=1723%:%
%:%4245=1723%:%
%:%4246=1724%:%
%:%4247=1724%:%
%:%4248=1725%:%
%:%4249=1725%:%
%:%4250=1725%:%
%:%4251=1726%:%
%:%4252=1726%:%
%:%4253=1727%:%
%:%4254=1727%:%
%:%4255=1727%:%
%:%4256=1728%:%
%:%4257=1728%:%
%:%4258=1729%:%
%:%4259=1729%:%
%:%4260=1729%:%
%:%4261=1729%:%
%:%4262=1729%:%
%:%4263=1730%:%
%:%4264=1730%:%
%:%4265=1731%:%
%:%4266=1731%:%
%:%4267=1732%:%
%:%4268=1732%:%
%:%4269=1732%:%
%:%4270=1733%:%
%:%4271=1733%:%
%:%4272=1733%:%
%:%4273=1734%:%
%:%4274=1734%:%
%:%4275=1734%:%
%:%4276=1735%:%
%:%4277=1735%:%
%:%4278=1736%:%
%:%4279=1736%:%
%:%4280=1737%:%
%:%4281=1737%:%
%:%4282=1737%:%
%:%4283=1738%:%
%:%4284=1738%:%
%:%4285=1739%:%
%:%4286=1739%:%
%:%4287=1739%:%
%:%4288=1740%:%
%:%4289=1740%:%
%:%4290=1741%:%
%:%4291=1741%:%
%:%4292=1742%:%
%:%4293=1742%:%
%:%4294=1743%:%
%:%4295=1743%:%
%:%4296=1744%:%
%:%4297=1744%:%
%:%4298=1745%:%
%:%4299=1745%:%
%:%4300=1746%:%
%:%4301=1746%:%
%:%4302=1747%:%
%:%4303=1747%:%
%:%4304=1748%:%
%:%4305=1748%:%
%:%4306=1749%:%
%:%4307=1749%:%
%:%4308=1750%:%
%:%4309=1750%:%
%:%4310=1751%:%
%:%4311=1751%:%
%:%4312=1752%:%
%:%4313=1753%:%
%:%4314=1753%:%
%:%4315=1754%:%
%:%4316=1754%:%
%:%4317=1754%:%
%:%4318=1754%:%
%:%4319=1755%:%
%:%4320=1755%:%
%:%4321=1756%:%
%:%4322=1756%:%
%:%4323=1756%:%
%:%4324=1757%:%
%:%4325=1757%:%
%:%4326=1758%:%
%:%4327=1758%:%
%:%4328=1759%:%
%:%4329=1759%:%
%:%4330=1760%:%
%:%4331=1760%:%
%:%4332=1761%:%
%:%4333=1761%:%
%:%4334=1762%:%
%:%4335=1762%:%
%:%4336=1762%:%
%:%4337=1763%:%
%:%4338=1763%:%
%:%4339=1763%:%
%:%4340=1764%:%
%:%4341=1764%:%
%:%4342=1765%:%
%:%4343=1765%:%
%:%4344=1765%:%
%:%4345=1766%:%
%:%4346=1766%:%
%:%4347=1767%:%
%:%4348=1767%:%
%:%4349=1767%:%
%:%4350=1768%:%
%:%4351=1768%:%
%:%4352=1769%:%
%:%4353=1769%:%
%:%4354=1769%:%
%:%4355=1770%:%
%:%4356=1770%:%
%:%4357=1771%:%
%:%4358=1771%:%
%:%4359=1771%:%
%:%4360=1771%:%
%:%4361=1771%:%
%:%4362=1772%:%
%:%4363=1772%:%
%:%4364=1773%:%
%:%4365=1773%:%
%:%4366=1773%:%
%:%4367=1774%:%
%:%4368=1774%:%
%:%4369=1775%:%
%:%4375=1775%:%
%:%4380=1776%:%
%:%4385=1777%:%

%
\begin{isabellebody}%
\setisabellecontext{Coproduct}%
%
\isadelimtheory
%
\endisadelimtheory
%
\isatagtheory
\isacommand{theory}\isamarkupfalse%
\ Coproduct\isanewline
\ \ \isakeyword{imports}\ Equivalence\isanewline
\isakeyword{begin}%
\endisatagtheory
{\isafoldtheory}%
%
\isadelimtheory
%
\endisadelimtheory
%
\isadelimdocument
%
\endisadelimdocument
%
\isatagdocument
%
\isamarkupsection{Axiom 7: Coproducts%
}
\isamarkuptrue%
%
\endisatagdocument
{\isafolddocument}%
%
\isadelimdocument
%
\endisadelimdocument
\isacommand{hide{\isacharunderscore}{\kern0pt}const}\isamarkupfalse%
\ case{\isacharunderscore}{\kern0pt}bool%
\begin{isamarkuptext}%
The axiomatization below corresponds to Axiom 7 (Coproducts) in Halvorson.%
\end{isamarkuptext}\isamarkuptrue%
\isacommand{axiomatization}\isamarkupfalse%
\isanewline
\ \ coprod\ {\isacharcolon}{\kern0pt}{\isacharcolon}{\kern0pt}\ {\isachardoublequoteopen}cset\ {\isasymRightarrow}\ cset\ {\isasymRightarrow}\ cset{\isachardoublequoteclose}\ {\isacharparenleft}{\kern0pt}\isakeyword{infixr}\ {\isachardoublequoteopen}{\isasymCoprod}{\isachardoublequoteclose}\ {\isadigit{6}}{\isadigit{5}}{\isacharparenright}{\kern0pt}\ \isakeyword{and}\isanewline
\ \ left{\isacharunderscore}{\kern0pt}coproj\ {\isacharcolon}{\kern0pt}{\isacharcolon}{\kern0pt}\ {\isachardoublequoteopen}cset\ {\isasymRightarrow}\ cset\ {\isasymRightarrow}\ cfunc{\isachardoublequoteclose}\ \isakeyword{and}\isanewline
\ \ right{\isacharunderscore}{\kern0pt}coproj\ {\isacharcolon}{\kern0pt}{\isacharcolon}{\kern0pt}\ {\isachardoublequoteopen}cset\ {\isasymRightarrow}\ cset\ {\isasymRightarrow}\ cfunc{\isachardoublequoteclose}\ \isakeyword{and}\isanewline
\ \ cfunc{\isacharunderscore}{\kern0pt}coprod\ {\isacharcolon}{\kern0pt}{\isacharcolon}{\kern0pt}\ {\isachardoublequoteopen}cfunc\ {\isasymRightarrow}\ cfunc\ {\isasymRightarrow}\ cfunc{\isachardoublequoteclose}\ {\isacharparenleft}{\kern0pt}\isakeyword{infixr}\ {\isachardoublequoteopen}{\isasymamalg}{\isachardoublequoteclose}\ {\isadigit{6}}{\isadigit{5}}{\isacharparenright}{\kern0pt}\isanewline
\isakeyword{where}\isanewline
\ \ left{\isacharunderscore}{\kern0pt}proj{\isacharunderscore}{\kern0pt}type{\isacharbrackleft}{\kern0pt}type{\isacharunderscore}{\kern0pt}rule{\isacharbrackright}{\kern0pt}{\isacharcolon}{\kern0pt}\ {\isachardoublequoteopen}left{\isacharunderscore}{\kern0pt}coproj\ X\ Y\ {\isacharcolon}{\kern0pt}\ X\ {\isasymrightarrow}\ X{\isasymCoprod}Y{\isachardoublequoteclose}\ \isakeyword{and}\isanewline
\ \ right{\isacharunderscore}{\kern0pt}proj{\isacharunderscore}{\kern0pt}type{\isacharbrackleft}{\kern0pt}type{\isacharunderscore}{\kern0pt}rule{\isacharbrackright}{\kern0pt}{\isacharcolon}{\kern0pt}\ {\isachardoublequoteopen}right{\isacharunderscore}{\kern0pt}coproj\ X\ Y\ {\isacharcolon}{\kern0pt}\ Y\ {\isasymrightarrow}\ X{\isasymCoprod}Y{\isachardoublequoteclose}\ \isakeyword{and}\isanewline
\ \ cfunc{\isacharunderscore}{\kern0pt}coprod{\isacharunderscore}{\kern0pt}type{\isacharbrackleft}{\kern0pt}type{\isacharunderscore}{\kern0pt}rule{\isacharbrackright}{\kern0pt}{\isacharcolon}{\kern0pt}\ {\isachardoublequoteopen}f\ {\isacharcolon}{\kern0pt}\ X\ {\isasymrightarrow}\ Z\ {\isasymLongrightarrow}\ g\ {\isacharcolon}{\kern0pt}\ Y\ {\isasymrightarrow}\ Z\ {\isasymLongrightarrow}\ f{\isasymamalg}g\ {\isacharcolon}{\kern0pt}\ \ X{\isasymCoprod}Y\ {\isasymrightarrow}\ Z{\isachardoublequoteclose}\ \isakeyword{and}\isanewline
\ \ left{\isacharunderscore}{\kern0pt}coproj{\isacharunderscore}{\kern0pt}cfunc{\isacharunderscore}{\kern0pt}coprod{\isacharcolon}{\kern0pt}\ {\isachardoublequoteopen}f\ {\isacharcolon}{\kern0pt}\ X\ {\isasymrightarrow}\ Z\ {\isasymLongrightarrow}\ g\ {\isacharcolon}{\kern0pt}\ Y\ {\isasymrightarrow}\ Z\ {\isasymLongrightarrow}\ f{\isasymamalg}g\ {\isasymcirc}\isactrlsub c\ {\isacharparenleft}{\kern0pt}left{\isacharunderscore}{\kern0pt}coproj\ X\ Y{\isacharparenright}{\kern0pt}\ \ {\isacharequal}{\kern0pt}\ f{\isachardoublequoteclose}\ \isakeyword{and}\isanewline
\ \ right{\isacharunderscore}{\kern0pt}coproj{\isacharunderscore}{\kern0pt}cfunc{\isacharunderscore}{\kern0pt}coprod{\isacharcolon}{\kern0pt}\ {\isachardoublequoteopen}f\ {\isacharcolon}{\kern0pt}\ X\ {\isasymrightarrow}\ Z\ {\isasymLongrightarrow}\ g\ {\isacharcolon}{\kern0pt}\ Y\ {\isasymrightarrow}\ Z\ {\isasymLongrightarrow}\ f{\isasymamalg}g\ {\isasymcirc}\isactrlsub c\ {\isacharparenleft}{\kern0pt}right{\isacharunderscore}{\kern0pt}coproj\ X\ Y{\isacharparenright}{\kern0pt}\ \ {\isacharequal}{\kern0pt}\ g{\isachardoublequoteclose}\ \isakeyword{and}\isanewline
\ \ cfunc{\isacharunderscore}{\kern0pt}coprod{\isacharunderscore}{\kern0pt}unique{\isacharcolon}{\kern0pt}\ {\isachardoublequoteopen}f\ {\isacharcolon}{\kern0pt}\ X\ {\isasymrightarrow}\ Z\ {\isasymLongrightarrow}\ g\ {\isacharcolon}{\kern0pt}\ Y\ {\isasymrightarrow}\ Z\ {\isasymLongrightarrow}\ h\ {\isacharcolon}{\kern0pt}\ X\ {\isasymCoprod}\ Y\ {\isasymrightarrow}\ Z\ {\isasymLongrightarrow}\ \isanewline
\ \ \ \ h\ {\isasymcirc}\isactrlsub c\ left{\isacharunderscore}{\kern0pt}coproj\ X\ Y\ {\isacharequal}{\kern0pt}\ f\ {\isasymLongrightarrow}\ h\ {\isasymcirc}\isactrlsub c\ right{\isacharunderscore}{\kern0pt}coproj\ X\ Y\ {\isacharequal}{\kern0pt}\ g\ {\isasymLongrightarrow}\ h\ {\isacharequal}{\kern0pt}\ f{\isasymamalg}g{\isachardoublequoteclose}\isanewline
\isanewline
\isacommand{definition}\isamarkupfalse%
\ is{\isacharunderscore}{\kern0pt}coprod\ {\isacharcolon}{\kern0pt}{\isacharcolon}{\kern0pt}\ {\isachardoublequoteopen}cset\ {\isasymRightarrow}\ cfunc\ {\isasymRightarrow}\ cfunc\ {\isasymRightarrow}\ cset\ {\isasymRightarrow}\ cset\ {\isasymRightarrow}\ bool{\isachardoublequoteclose}\ \isakeyword{where}\isanewline
\ \ {\isachardoublequoteopen}is{\isacharunderscore}{\kern0pt}coprod\ W\ i\isactrlsub {\isadigit{0}}\ i\isactrlsub {\isadigit{1}}\ X\ Y\ {\isasymlongleftrightarrow}\ \isanewline
\ \ \ \ {\isacharparenleft}{\kern0pt}i\isactrlsub {\isadigit{0}}\ {\isacharcolon}{\kern0pt}\ X\ {\isasymrightarrow}\ W\ {\isasymand}\ i\isactrlsub {\isadigit{1}}\ {\isacharcolon}{\kern0pt}\ Y\ {\isasymrightarrow}\ W\ {\isasymand}\isanewline
\ \ \ \ {\isacharparenleft}{\kern0pt}{\isasymforall}\ f\ g\ Z{\isachardot}{\kern0pt}\ {\isacharparenleft}{\kern0pt}f\ {\isacharcolon}{\kern0pt}\ X\ {\isasymrightarrow}\ Z\ {\isasymand}\ g\ {\isacharcolon}{\kern0pt}\ Y\ {\isasymrightarrow}\ Z{\isacharparenright}{\kern0pt}\ {\isasymlongrightarrow}\ \isanewline
\ \ \ \ \ \ {\isacharparenleft}{\kern0pt}{\isasymexists}\ h{\isachardot}{\kern0pt}\ h\ {\isacharcolon}{\kern0pt}\ \ W\ {\isasymrightarrow}\ Z\ {\isasymand}\ h\ {\isasymcirc}\isactrlsub c\ i\isactrlsub {\isadigit{0}}\ {\isacharequal}{\kern0pt}\ f\ {\isasymand}\ h\ {\isasymcirc}\isactrlsub c\ i\isactrlsub {\isadigit{1}}\ {\isacharequal}{\kern0pt}\ g\ {\isasymand}\isanewline
\ \ \ \ \ \ \ \ {\isacharparenleft}{\kern0pt}{\isasymforall}\ h{\isadigit{2}}{\isachardot}{\kern0pt}\ {\isacharparenleft}{\kern0pt}h{\isadigit{2}}\ {\isacharcolon}{\kern0pt}\ W\ {\isasymrightarrow}\ Z\ {\isasymand}\ h{\isadigit{2}}\ {\isasymcirc}\isactrlsub c\ i\isactrlsub {\isadigit{0}}\ {\isacharequal}{\kern0pt}\ f\ {\isasymand}\ h{\isadigit{2}}\ {\isasymcirc}\isactrlsub c\ i\isactrlsub {\isadigit{1}}\ {\isacharequal}{\kern0pt}\ g{\isacharparenright}{\kern0pt}\ {\isasymlongrightarrow}\ h{\isadigit{2}}\ {\isacharequal}{\kern0pt}\ h{\isacharparenright}{\kern0pt}{\isacharparenright}{\kern0pt}{\isacharparenright}{\kern0pt}{\isacharparenright}{\kern0pt}{\isachardoublequoteclose}\isanewline
\isanewline
\isacommand{abbreviation}\isamarkupfalse%
\ is{\isacharunderscore}{\kern0pt}coprod{\isacharunderscore}{\kern0pt}triple\ {\isacharcolon}{\kern0pt}{\isacharcolon}{\kern0pt}\ {\isachardoublequoteopen}cset\ {\isasymtimes}\ cfunc\ {\isasymtimes}\ cfunc\ {\isasymRightarrow}\ cset\ {\isasymRightarrow}\ cset\ {\isasymRightarrow}\ bool{\isachardoublequoteclose}\ \isakeyword{where}\isanewline
\ \ {\isachardoublequoteopen}is{\isacharunderscore}{\kern0pt}coprod{\isacharunderscore}{\kern0pt}triple\ Wi\ X\ Y\ {\isasymequiv}\ is{\isacharunderscore}{\kern0pt}coprod\ {\isacharparenleft}{\kern0pt}fst\ Wi{\isacharparenright}{\kern0pt}\ {\isacharparenleft}{\kern0pt}fst\ {\isacharparenleft}{\kern0pt}snd\ Wi{\isacharparenright}{\kern0pt}{\isacharparenright}{\kern0pt}\ {\isacharparenleft}{\kern0pt}snd\ {\isacharparenleft}{\kern0pt}snd\ Wi{\isacharparenright}{\kern0pt}{\isacharparenright}{\kern0pt}\ X\ Y{\isachardoublequoteclose}\isanewline
\isanewline
\isacommand{lemma}\isamarkupfalse%
\ canonical{\isacharunderscore}{\kern0pt}coprod{\isacharunderscore}{\kern0pt}is{\isacharunderscore}{\kern0pt}coprod{\isacharcolon}{\kern0pt}\isanewline
\ {\isachardoublequoteopen}is{\isacharunderscore}{\kern0pt}coprod\ {\isacharparenleft}{\kern0pt}X\ {\isasymCoprod}\ Y{\isacharparenright}{\kern0pt}\ {\isacharparenleft}{\kern0pt}left{\isacharunderscore}{\kern0pt}coproj\ X\ Y{\isacharparenright}{\kern0pt}\ {\isacharparenleft}{\kern0pt}right{\isacharunderscore}{\kern0pt}coproj\ X\ Y{\isacharparenright}{\kern0pt}\ X\ Y{\isachardoublequoteclose}\isanewline
%
\isadelimproof
\ \ %
\endisadelimproof
%
\isatagproof
\isacommand{unfolding}\isamarkupfalse%
\ is{\isacharunderscore}{\kern0pt}coprod{\isacharunderscore}{\kern0pt}def\isanewline
\isacommand{proof}\isamarkupfalse%
\ {\isacharparenleft}{\kern0pt}typecheck{\isacharunderscore}{\kern0pt}cfuncs{\isacharcomma}{\kern0pt}\ auto{\isacharparenright}{\kern0pt}\isanewline
\ \ \isacommand{fix}\isamarkupfalse%
\ f\ g\ Z\isanewline
\ \ \isacommand{assume}\isamarkupfalse%
\ f{\isacharunderscore}{\kern0pt}type{\isacharcolon}{\kern0pt}\ {\isachardoublequoteopen}f\ {\isacharcolon}{\kern0pt}\ X\ {\isasymrightarrow}\ Z{\isachardoublequoteclose}\isanewline
\ \ \isacommand{assume}\isamarkupfalse%
\ g{\isacharunderscore}{\kern0pt}type{\isacharcolon}{\kern0pt}\ {\isachardoublequoteopen}g\ {\isacharcolon}{\kern0pt}\ Y\ {\isasymrightarrow}\ Z{\isachardoublequoteclose}\isanewline
\ \ \isacommand{show}\isamarkupfalse%
\ {\isachardoublequoteopen}{\isasymexists}h{\isachardot}{\kern0pt}\ h\ {\isacharcolon}{\kern0pt}\ X\ {\isasymCoprod}\ Y\ {\isasymrightarrow}\ Z\ {\isasymand}\isanewline
\ \ \ \ \ \ \ \ \ \ \ h\ {\isasymcirc}\isactrlsub c\ left{\isacharunderscore}{\kern0pt}coproj\ X\ Y\ {\isacharequal}{\kern0pt}\ f\ {\isasymand}\isanewline
\ \ \ \ \ \ \ \ \ \ \ h\ {\isasymcirc}\isactrlsub c\ right{\isacharunderscore}{\kern0pt}coproj\ X\ Y\ {\isacharequal}{\kern0pt}\ g\ {\isasymand}\ {\isacharparenleft}{\kern0pt}{\isasymforall}h{\isadigit{2}}{\isachardot}{\kern0pt}\ h{\isadigit{2}}\ {\isacharcolon}{\kern0pt}\ X\ {\isasymCoprod}\ Y\ {\isasymrightarrow}\ Z\ {\isasymand}\ h{\isadigit{2}}\ {\isasymcirc}\isactrlsub c\ left{\isacharunderscore}{\kern0pt}coproj\ X\ Y\ {\isacharequal}{\kern0pt}\ f\ {\isasymand}\ h{\isadigit{2}}\ {\isasymcirc}\isactrlsub c\ right{\isacharunderscore}{\kern0pt}coproj\ X\ Y\ {\isacharequal}{\kern0pt}\ g\ {\isasymlongrightarrow}\ h{\isadigit{2}}\ {\isacharequal}{\kern0pt}\ h{\isacharparenright}{\kern0pt}{\isachardoublequoteclose}\isanewline
\ \ \ \ \isacommand{using}\isamarkupfalse%
\ cfunc{\isacharunderscore}{\kern0pt}coprod{\isacharunderscore}{\kern0pt}type\ cfunc{\isacharunderscore}{\kern0pt}coprod{\isacharunderscore}{\kern0pt}unique\ f{\isacharunderscore}{\kern0pt}type\ g{\isacharunderscore}{\kern0pt}type\ left{\isacharunderscore}{\kern0pt}coproj{\isacharunderscore}{\kern0pt}cfunc{\isacharunderscore}{\kern0pt}coprod\ right{\isacharunderscore}{\kern0pt}coproj{\isacharunderscore}{\kern0pt}cfunc{\isacharunderscore}{\kern0pt}coprod\ \isanewline
\ \ \ \ \isacommand{by}\isamarkupfalse%
{\isacharparenleft}{\kern0pt}rule{\isacharunderscore}{\kern0pt}tac\ x{\isacharequal}{\kern0pt}{\isachardoublequoteopen}f{\isasymamalg}g{\isachardoublequoteclose}\ \isakeyword{in}\ exI{\isacharcomma}{\kern0pt}\ auto{\isacharparenright}{\kern0pt}\isanewline
\isacommand{qed}\isamarkupfalse%
%
\endisatagproof
{\isafoldproof}%
%
\isadelimproof
%
\endisadelimproof
%
\begin{isamarkuptext}%
The lemma below is dual to Proposition 2.1.8 in Halvorson.%
\end{isamarkuptext}\isamarkuptrue%
\isacommand{lemma}\isamarkupfalse%
\ coprods{\isacharunderscore}{\kern0pt}isomorphic{\isacharcolon}{\kern0pt}\isanewline
\ \ \isakeyword{assumes}\ W{\isacharunderscore}{\kern0pt}coprod{\isacharcolon}{\kern0pt}\ \ {\isachardoublequoteopen}is{\isacharunderscore}{\kern0pt}coprod{\isacharunderscore}{\kern0pt}triple\ {\isacharparenleft}{\kern0pt}W{\isacharcomma}{\kern0pt}\ i\isactrlsub {\isadigit{0}}{\isacharcomma}{\kern0pt}\ i\isactrlsub {\isadigit{1}}{\isacharparenright}{\kern0pt}\ X\ Y{\isachardoublequoteclose}\isanewline
\ \ \isakeyword{assumes}\ W{\isacharprime}{\kern0pt}{\isacharunderscore}{\kern0pt}coprod{\isacharcolon}{\kern0pt}\ {\isachardoublequoteopen}is{\isacharunderscore}{\kern0pt}coprod{\isacharunderscore}{\kern0pt}triple\ {\isacharparenleft}{\kern0pt}W{\isacharprime}{\kern0pt}{\isacharcomma}{\kern0pt}\ i{\isacharprime}{\kern0pt}\isactrlsub {\isadigit{0}}{\isacharcomma}{\kern0pt}\ i{\isacharprime}{\kern0pt}\isactrlsub {\isadigit{1}}{\isacharparenright}{\kern0pt}\ X\ Y{\isachardoublequoteclose}\isanewline
\ \ \isakeyword{shows}\ {\isachardoublequoteopen}{\isasymexists}\ g{\isachardot}{\kern0pt}\ g\ {\isacharcolon}{\kern0pt}\ W\ {\isasymrightarrow}\ W{\isacharprime}{\kern0pt}\ {\isasymand}\ isomorphism\ g\ {\isasymand}\ g\ {\isasymcirc}\isactrlsub c\ i\isactrlsub {\isadigit{0}}\ \ {\isacharequal}{\kern0pt}\ i{\isacharprime}{\kern0pt}\isactrlsub {\isadigit{0}}\ {\isasymand}\ g\ {\isasymcirc}\isactrlsub c\ i\isactrlsub {\isadigit{1}}\ {\isacharequal}{\kern0pt}\ i{\isacharprime}{\kern0pt}\isactrlsub {\isadigit{1}}{\isachardoublequoteclose}\isanewline
%
\isadelimproof
%
\endisadelimproof
%
\isatagproof
\isacommand{proof}\isamarkupfalse%
\ {\isacharminus}{\kern0pt}\isanewline
\ \ \isacommand{obtain}\isamarkupfalse%
\ f\ \isakeyword{where}\ f{\isacharunderscore}{\kern0pt}def{\isacharcolon}{\kern0pt}\ {\isachardoublequoteopen}f\ {\isacharcolon}{\kern0pt}\ W{\isacharprime}{\kern0pt}\ {\isasymrightarrow}\ W\ {\isasymand}\ f\ {\isasymcirc}\isactrlsub c\ i{\isacharprime}{\kern0pt}\isactrlsub {\isadigit{0}}\ \ {\isacharequal}{\kern0pt}\ i\isactrlsub {\isadigit{0}}\ {\isasymand}\ f\ {\isasymcirc}\isactrlsub c\ i{\isacharprime}{\kern0pt}\isactrlsub {\isadigit{1}}\ {\isacharequal}{\kern0pt}\ i\isactrlsub {\isadigit{1}}{\isachardoublequoteclose}\isanewline
\ \ \ \ \isacommand{using}\isamarkupfalse%
\ W{\isacharunderscore}{\kern0pt}coprod\ W{\isacharprime}{\kern0pt}{\isacharunderscore}{\kern0pt}coprod\ \isacommand{unfolding}\isamarkupfalse%
\ is{\isacharunderscore}{\kern0pt}coprod{\isacharunderscore}{\kern0pt}def\isanewline
\ \ \ \ \isacommand{by}\isamarkupfalse%
\ {\isacharparenleft}{\kern0pt}metis\ split{\isacharunderscore}{\kern0pt}pairs{\isacharparenright}{\kern0pt}\isanewline
\isanewline
\ \ \isacommand{obtain}\isamarkupfalse%
\ g\ \isakeyword{where}\ g{\isacharunderscore}{\kern0pt}def{\isacharcolon}{\kern0pt}\ {\isachardoublequoteopen}g\ {\isacharcolon}{\kern0pt}\ W\ {\isasymrightarrow}\ W{\isacharprime}{\kern0pt}\ {\isasymand}\ g\ {\isasymcirc}\isactrlsub c\ i\isactrlsub {\isadigit{0}}\ \ {\isacharequal}{\kern0pt}\ i{\isacharprime}{\kern0pt}\isactrlsub {\isadigit{0}}\ {\isasymand}\ g\ {\isasymcirc}\isactrlsub c\ i\isactrlsub {\isadigit{1}}\ {\isacharequal}{\kern0pt}\ i{\isacharprime}{\kern0pt}\isactrlsub {\isadigit{1}}{\isachardoublequoteclose}\isanewline
\ \ \ \ \isacommand{using}\isamarkupfalse%
\ W{\isacharunderscore}{\kern0pt}coprod\ W{\isacharprime}{\kern0pt}{\isacharunderscore}{\kern0pt}coprod\ \isacommand{unfolding}\isamarkupfalse%
\ is{\isacharunderscore}{\kern0pt}coprod{\isacharunderscore}{\kern0pt}def\isanewline
\ \ \ \ \isacommand{by}\isamarkupfalse%
\ {\isacharparenleft}{\kern0pt}metis\ split{\isacharunderscore}{\kern0pt}pairs{\isacharparenright}{\kern0pt}\isanewline
\isanewline
\ \ \isacommand{have}\isamarkupfalse%
\ fg{\isadigit{0}}{\isacharcolon}{\kern0pt}\ {\isachardoublequoteopen}{\isacharparenleft}{\kern0pt}f\ {\isasymcirc}\isactrlsub c\ g{\isacharparenright}{\kern0pt}\ {\isasymcirc}\isactrlsub c\ \ i\isactrlsub {\isadigit{0}}\ \ \ {\isacharequal}{\kern0pt}\ i\isactrlsub {\isadigit{0}}{\isachardoublequoteclose}\isanewline
\ \ \ \ \isacommand{by}\isamarkupfalse%
\ {\isacharparenleft}{\kern0pt}metis\ W{\isacharunderscore}{\kern0pt}coprod\ comp{\isacharunderscore}{\kern0pt}associative{\isadigit{2}}\ f{\isacharunderscore}{\kern0pt}def\ g{\isacharunderscore}{\kern0pt}def\ is{\isacharunderscore}{\kern0pt}coprod{\isacharunderscore}{\kern0pt}def\ split{\isacharunderscore}{\kern0pt}pairs{\isacharparenright}{\kern0pt}\isanewline
\ \ \isacommand{have}\isamarkupfalse%
\ fg{\isadigit{1}}{\isacharcolon}{\kern0pt}\ {\isachardoublequoteopen}{\isacharparenleft}{\kern0pt}f\ {\isasymcirc}\isactrlsub c\ g{\isacharparenright}{\kern0pt}\ {\isasymcirc}\isactrlsub c\ \ i\isactrlsub {\isadigit{1}}\ \ \ {\isacharequal}{\kern0pt}\ i\isactrlsub {\isadigit{1}}{\isachardoublequoteclose}\isanewline
\ \ \ \ \isacommand{by}\isamarkupfalse%
\ {\isacharparenleft}{\kern0pt}metis\ W{\isacharunderscore}{\kern0pt}coprod\ comp{\isacharunderscore}{\kern0pt}associative{\isadigit{2}}\ f{\isacharunderscore}{\kern0pt}def\ g{\isacharunderscore}{\kern0pt}def\ is{\isacharunderscore}{\kern0pt}coprod{\isacharunderscore}{\kern0pt}def\ split{\isacharunderscore}{\kern0pt}pairs{\isacharparenright}{\kern0pt}\isanewline
\ \ \ \ \isanewline
\ \ \isacommand{obtain}\isamarkupfalse%
\ idW\ \isakeyword{where}\ {\isachardoublequoteopen}idW\ {\isacharcolon}{\kern0pt}\ W\ {\isasymrightarrow}\ W\ {\isasymand}\ {\isacharparenleft}{\kern0pt}{\isasymforall}\ h{\isadigit{2}}{\isachardot}{\kern0pt}\ {\isacharparenleft}{\kern0pt}h{\isadigit{2}}\ {\isacharcolon}{\kern0pt}\ W\ {\isasymrightarrow}\ W\ {\isasymand}\ h{\isadigit{2}}\ {\isasymcirc}\isactrlsub c\ i\isactrlsub {\isadigit{0}}\ \ {\isacharequal}{\kern0pt}\ i\isactrlsub {\isadigit{0}}\ {\isasymand}\ h{\isadigit{2}}\ {\isasymcirc}\isactrlsub c\ i\isactrlsub {\isadigit{1}}\ {\isacharequal}{\kern0pt}\ i\isactrlsub {\isadigit{1}}{\isacharparenright}{\kern0pt}\ {\isasymlongrightarrow}\ h{\isadigit{2}}\ {\isacharequal}{\kern0pt}\ idW{\isacharparenright}{\kern0pt}{\isachardoublequoteclose}\isanewline
\ \ \ \ \isacommand{by}\isamarkupfalse%
\ {\isacharparenleft}{\kern0pt}smt\ {\isacharparenleft}{\kern0pt}verit{\isacharcomma}{\kern0pt}\ best{\isacharparenright}{\kern0pt}\ W{\isacharunderscore}{\kern0pt}coprod\ is{\isacharunderscore}{\kern0pt}coprod{\isacharunderscore}{\kern0pt}def\ prod{\isachardot}{\kern0pt}sel{\isacharparenright}{\kern0pt}\isanewline
\ \ \isacommand{then}\isamarkupfalse%
\ \isacommand{have}\isamarkupfalse%
\ fg{\isacharcolon}{\kern0pt}\ {\isachardoublequoteopen}f\ {\isasymcirc}\isactrlsub c\ g\ {\isacharequal}{\kern0pt}\ id\ W{\isachardoublequoteclose}\isanewline
\ \ \isacommand{proof}\isamarkupfalse%
\ auto\isanewline
\ \ \ \ \isacommand{assume}\isamarkupfalse%
\ idW{\isacharunderscore}{\kern0pt}unique{\isacharcolon}{\kern0pt}\ {\isachardoublequoteopen}{\isasymforall}h{\isadigit{2}}{\isachardot}{\kern0pt}\ h{\isadigit{2}}\ {\isacharcolon}{\kern0pt}\ W\ {\isasymrightarrow}\ W\ {\isasymand}\ h{\isadigit{2}}\ {\isasymcirc}\isactrlsub c\ i\isactrlsub {\isadigit{0}}\ {\isacharequal}{\kern0pt}\ i\isactrlsub {\isadigit{0}}\ {\isasymand}\ h{\isadigit{2}}\ {\isasymcirc}\isactrlsub c\ i\isactrlsub {\isadigit{1}}\ {\isacharequal}{\kern0pt}\ i\isactrlsub {\isadigit{1}}\ {\isasymlongrightarrow}\ h{\isadigit{2}}\ {\isacharequal}{\kern0pt}\ idW{\isachardoublequoteclose}\isanewline
\ \ \ \ \isacommand{have}\isamarkupfalse%
\ {\isadigit{1}}{\isacharcolon}{\kern0pt}\ {\isachardoublequoteopen}f\ {\isasymcirc}\isactrlsub c\ g\ {\isacharequal}{\kern0pt}\ idW{\isachardoublequoteclose}\isanewline
\ \ \ \ \ \ \isacommand{using}\isamarkupfalse%
\ comp{\isacharunderscore}{\kern0pt}type\ f{\isacharunderscore}{\kern0pt}def\ fg{\isadigit{0}}\ fg{\isadigit{1}}\ g{\isacharunderscore}{\kern0pt}def\ idW{\isacharunderscore}{\kern0pt}unique\ \isacommand{by}\isamarkupfalse%
\ blast\isanewline
\ \ \ \ \isacommand{have}\isamarkupfalse%
\ {\isadigit{2}}{\isacharcolon}{\kern0pt}\ {\isachardoublequoteopen}id\ W\ {\isacharequal}{\kern0pt}\ idW{\isachardoublequoteclose}\isanewline
\ \ \ \ \ \ \isacommand{using}\isamarkupfalse%
\ W{\isacharunderscore}{\kern0pt}coprod\ idW{\isacharunderscore}{\kern0pt}unique\ id{\isacharunderscore}{\kern0pt}left{\isacharunderscore}{\kern0pt}unit{\isadigit{2}}\ id{\isacharunderscore}{\kern0pt}type\ is{\isacharunderscore}{\kern0pt}coprod{\isacharunderscore}{\kern0pt}def\ \isacommand{by}\isamarkupfalse%
\ auto\isanewline
\ \ \ \ \isacommand{from}\isamarkupfalse%
\ {\isadigit{1}}\ {\isadigit{2}}\ \isacommand{show}\isamarkupfalse%
\ {\isachardoublequoteopen}f\ {\isasymcirc}\isactrlsub c\ g\ {\isacharequal}{\kern0pt}\ id\ W{\isachardoublequoteclose}\isanewline
\ \ \ \ \ \ \isacommand{by}\isamarkupfalse%
\ auto\isanewline
\ \ \isacommand{qed}\isamarkupfalse%
\isanewline
\isanewline
\ \ \isacommand{have}\isamarkupfalse%
\ gf{\isadigit{0}}{\isacharcolon}{\kern0pt}\ {\isachardoublequoteopen}{\isacharparenleft}{\kern0pt}g\ {\isasymcirc}\isactrlsub c\ f{\isacharparenright}{\kern0pt}\ {\isasymcirc}\isactrlsub c\ i{\isacharprime}{\kern0pt}\isactrlsub {\isadigit{0}}{\isacharequal}{\kern0pt}\ i{\isacharprime}{\kern0pt}\isactrlsub {\isadigit{0}}{\isachardoublequoteclose}\isanewline
\ \ \ \ \isacommand{using}\isamarkupfalse%
\ W{\isacharprime}{\kern0pt}{\isacharunderscore}{\kern0pt}coprod\ comp{\isacharunderscore}{\kern0pt}associative{\isadigit{2}}\ f{\isacharunderscore}{\kern0pt}def\ g{\isacharunderscore}{\kern0pt}def\ is{\isacharunderscore}{\kern0pt}coprod{\isacharunderscore}{\kern0pt}def\ \isacommand{by}\isamarkupfalse%
\ auto\isanewline
\ \ \isacommand{have}\isamarkupfalse%
\ gf{\isadigit{1}}{\isacharcolon}{\kern0pt}\ {\isachardoublequoteopen}{\isacharparenleft}{\kern0pt}g\ {\isasymcirc}\isactrlsub c\ f{\isacharparenright}{\kern0pt}\ {\isasymcirc}\isactrlsub c\ i{\isacharprime}{\kern0pt}\isactrlsub {\isadigit{1}}\ {\isacharequal}{\kern0pt}\ i{\isacharprime}{\kern0pt}\isactrlsub {\isadigit{1}}{\isachardoublequoteclose}\isanewline
\ \ \ \ \isacommand{using}\isamarkupfalse%
\ W{\isacharprime}{\kern0pt}{\isacharunderscore}{\kern0pt}coprod\ comp{\isacharunderscore}{\kern0pt}associative{\isadigit{2}}\ f{\isacharunderscore}{\kern0pt}def\ g{\isacharunderscore}{\kern0pt}def\ is{\isacharunderscore}{\kern0pt}coprod{\isacharunderscore}{\kern0pt}def\ \isacommand{by}\isamarkupfalse%
\ auto\isanewline
\isanewline
\ \ \isacommand{obtain}\isamarkupfalse%
\ idW{\isacharprime}{\kern0pt}\ \isakeyword{where}\ {\isachardoublequoteopen}idW{\isacharprime}{\kern0pt}{\isacharcolon}{\kern0pt}\ W{\isacharprime}{\kern0pt}{\isasymrightarrow}\ W{\isacharprime}{\kern0pt}{\isasymand}\ {\isacharparenleft}{\kern0pt}{\isasymforall}\ h{\isadigit{2}}{\isachardot}{\kern0pt}\ {\isacharparenleft}{\kern0pt}h{\isadigit{2}}\ {\isacharcolon}{\kern0pt}\ W{\isacharprime}{\kern0pt}{\isasymrightarrow}\ W{\isacharprime}{\kern0pt}{\isasymand}\ \ h{\isadigit{2}}\ {\isasymcirc}\isactrlsub c\ i{\isacharprime}{\kern0pt}\isactrlsub {\isadigit{0}}{\isacharequal}{\kern0pt}\ i{\isacharprime}{\kern0pt}\isactrlsub {\isadigit{0}}\ {\isasymand}\ h{\isadigit{2}}\ {\isasymcirc}\isactrlsub c\ i{\isacharprime}{\kern0pt}\isactrlsub {\isadigit{1}}{\isacharequal}{\kern0pt}\ i{\isacharprime}{\kern0pt}\isactrlsub {\isadigit{1}}{\isacharparenright}{\kern0pt}\ {\isasymlongrightarrow}\ h{\isadigit{2}}\ {\isacharequal}{\kern0pt}\ idW{\isacharprime}{\kern0pt}{\isacharparenright}{\kern0pt}{\isachardoublequoteclose}\isanewline
\ \ \ \ \isacommand{by}\isamarkupfalse%
\ {\isacharparenleft}{\kern0pt}smt\ {\isacharparenleft}{\kern0pt}verit{\isacharcomma}{\kern0pt}\ best{\isacharparenright}{\kern0pt}\ W{\isacharprime}{\kern0pt}{\isacharunderscore}{\kern0pt}coprod\ is{\isacharunderscore}{\kern0pt}coprod{\isacharunderscore}{\kern0pt}def\ prod{\isachardot}{\kern0pt}sel{\isacharparenright}{\kern0pt}\isanewline
\ \ \isacommand{then}\isamarkupfalse%
\ \isacommand{have}\isamarkupfalse%
\ gf{\isacharcolon}{\kern0pt}\ {\isachardoublequoteopen}g\ {\isasymcirc}\isactrlsub c\ f\ {\isacharequal}{\kern0pt}\ id\ W{\isacharprime}{\kern0pt}{\isachardoublequoteclose}\isanewline
\ \ \isacommand{proof}\isamarkupfalse%
\ auto\isanewline
\ \ \ \ \isacommand{assume}\isamarkupfalse%
\ idW{\isacharprime}{\kern0pt}{\isacharunderscore}{\kern0pt}unique{\isacharcolon}{\kern0pt}\ {\isachardoublequoteopen}{\isasymforall}h{\isadigit{2}}{\isachardot}{\kern0pt}\ h{\isadigit{2}}\ {\isacharcolon}{\kern0pt}\ W{\isacharprime}{\kern0pt}\ {\isasymrightarrow}\ W{\isacharprime}{\kern0pt}\ {\isasymand}\ h{\isadigit{2}}\ {\isasymcirc}\isactrlsub c\ i{\isacharprime}{\kern0pt}\isactrlsub {\isadigit{0}}\ {\isacharequal}{\kern0pt}\ i{\isacharprime}{\kern0pt}\isactrlsub {\isadigit{0}}\ {\isasymand}\ h{\isadigit{2}}\ {\isasymcirc}\isactrlsub c\ i{\isacharprime}{\kern0pt}\isactrlsub {\isadigit{1}}\ {\isacharequal}{\kern0pt}\ i{\isacharprime}{\kern0pt}\isactrlsub {\isadigit{1}}\ {\isasymlongrightarrow}\ h{\isadigit{2}}\ {\isacharequal}{\kern0pt}\ idW{\isacharprime}{\kern0pt}{\isachardoublequoteclose}\isanewline
\ \ \ \ \isacommand{have}\isamarkupfalse%
\ {\isadigit{1}}{\isacharcolon}{\kern0pt}\ {\isachardoublequoteopen}g\ {\isasymcirc}\isactrlsub c\ f\ {\isacharequal}{\kern0pt}\ idW{\isacharprime}{\kern0pt}{\isachardoublequoteclose}\isanewline
\ \ \ \ \ \ \isacommand{using}\isamarkupfalse%
\ comp{\isacharunderscore}{\kern0pt}type\ f{\isacharunderscore}{\kern0pt}def\ g{\isacharunderscore}{\kern0pt}def\ gf{\isadigit{0}}\ gf{\isadigit{1}}\ idW{\isacharprime}{\kern0pt}{\isacharunderscore}{\kern0pt}unique\ \isacommand{by}\isamarkupfalse%
\ blast\isanewline
\ \ \ \ \isacommand{have}\isamarkupfalse%
\ {\isadigit{2}}{\isacharcolon}{\kern0pt}\ {\isachardoublequoteopen}id\ W{\isacharprime}{\kern0pt}\ {\isacharequal}{\kern0pt}\ idW{\isacharprime}{\kern0pt}{\isachardoublequoteclose}\isanewline
\ \ \ \ \ \ \isacommand{using}\isamarkupfalse%
\ W{\isacharprime}{\kern0pt}{\isacharunderscore}{\kern0pt}coprod\ idW{\isacharprime}{\kern0pt}{\isacharunderscore}{\kern0pt}unique\ id{\isacharunderscore}{\kern0pt}left{\isacharunderscore}{\kern0pt}unit{\isadigit{2}}\ id{\isacharunderscore}{\kern0pt}type\ is{\isacharunderscore}{\kern0pt}coprod{\isacharunderscore}{\kern0pt}def\ \isacommand{by}\isamarkupfalse%
\ auto\ \ \ \ \ \ \isanewline
\ \ \ \ \isacommand{from}\isamarkupfalse%
\ {\isadigit{1}}\ {\isadigit{2}}\ \isacommand{show}\isamarkupfalse%
\ {\isachardoublequoteopen}g\ {\isasymcirc}\isactrlsub c\ f\ {\isacharequal}{\kern0pt}\ id\ W{\isacharprime}{\kern0pt}{\isachardoublequoteclose}\isanewline
\ \ \ \ \ \ \isacommand{by}\isamarkupfalse%
\ auto\isanewline
\ \ \isacommand{qed}\isamarkupfalse%
\isanewline
\isanewline
\ \ \isacommand{have}\isamarkupfalse%
\ g{\isacharunderscore}{\kern0pt}iso{\isacharcolon}{\kern0pt}\ {\isachardoublequoteopen}isomorphism\ g{\isachardoublequoteclose}\isanewline
\ \ \ \ \isacommand{using}\isamarkupfalse%
\ f{\isacharunderscore}{\kern0pt}def\ fg\ g{\isacharunderscore}{\kern0pt}def\ gf\ isomorphism{\isacharunderscore}{\kern0pt}def{\isadigit{3}}\ \isacommand{by}\isamarkupfalse%
\ blast\isanewline
\ \ \isacommand{from}\isamarkupfalse%
\ g{\isacharunderscore}{\kern0pt}iso\ g{\isacharunderscore}{\kern0pt}def\ \isacommand{show}\isamarkupfalse%
\ {\isachardoublequoteopen}{\isasymexists}\ g{\isachardot}{\kern0pt}\ g\ {\isacharcolon}{\kern0pt}\ W\ {\isasymrightarrow}\ W{\isacharprime}{\kern0pt}\ {\isasymand}\ isomorphism\ g\ {\isasymand}\ g\ {\isasymcirc}\isactrlsub c\ i\isactrlsub {\isadigit{0}}\ \ {\isacharequal}{\kern0pt}\ i{\isacharprime}{\kern0pt}\isactrlsub {\isadigit{0}}\ {\isasymand}\ g\ {\isasymcirc}\isactrlsub c\ i\isactrlsub {\isadigit{1}}\ {\isacharequal}{\kern0pt}\ i{\isacharprime}{\kern0pt}\isactrlsub {\isadigit{1}}{\isachardoublequoteclose}\isanewline
\ \ \ \ \isacommand{by}\isamarkupfalse%
\ blast\isanewline
\isacommand{qed}\isamarkupfalse%
%
\endisatagproof
{\isafoldproof}%
%
\isadelimproof
%
\endisadelimproof
%
\isadelimdocument
%
\endisadelimdocument
%
\isatagdocument
%
\isamarkupsubsection{Coproduct Function Properities%
}
\isamarkuptrue%
%
\endisatagdocument
{\isafolddocument}%
%
\isadelimdocument
%
\endisadelimdocument
\isacommand{lemma}\isamarkupfalse%
\ cfunc{\isacharunderscore}{\kern0pt}coprod{\isacharunderscore}{\kern0pt}comp{\isacharcolon}{\kern0pt}\isanewline
\ \ \isakeyword{assumes}\ {\isachardoublequoteopen}a\ {\isacharcolon}{\kern0pt}\ Y\ {\isasymrightarrow}\ Z{\isachardoublequoteclose}\ {\isachardoublequoteopen}b\ {\isacharcolon}{\kern0pt}\ X\ {\isasymrightarrow}\ Y{\isachardoublequoteclose}\ {\isachardoublequoteopen}c\ {\isacharcolon}{\kern0pt}\ W\ {\isasymrightarrow}\ Y{\isachardoublequoteclose}\isanewline
\ \ \isakeyword{shows}\ {\isachardoublequoteopen}{\isacharparenleft}{\kern0pt}a\ {\isasymcirc}\isactrlsub c\ b{\isacharparenright}{\kern0pt}\ {\isasymamalg}\ {\isacharparenleft}{\kern0pt}a\ {\isasymcirc}\isactrlsub c\ c{\isacharparenright}{\kern0pt}\ {\isacharequal}{\kern0pt}\ a\ {\isasymcirc}\isactrlsub c\ {\isacharparenleft}{\kern0pt}b\ {\isasymamalg}\ c{\isacharparenright}{\kern0pt}{\isachardoublequoteclose}\isanewline
%
\isadelimproof
%
\endisadelimproof
%
\isatagproof
\isacommand{proof}\isamarkupfalse%
\ {\isacharminus}{\kern0pt}\isanewline
\ \ \isacommand{have}\isamarkupfalse%
\ {\isachardoublequoteopen}{\isacharparenleft}{\kern0pt}{\isacharparenleft}{\kern0pt}a\ {\isasymcirc}\isactrlsub c\ b{\isacharparenright}{\kern0pt}\ {\isasymamalg}\ {\isacharparenleft}{\kern0pt}a\ {\isasymcirc}\isactrlsub c\ c{\isacharparenright}{\kern0pt}{\isacharparenright}{\kern0pt}\ {\isasymcirc}\isactrlsub c\ {\isacharparenleft}{\kern0pt}left{\isacharunderscore}{\kern0pt}coproj\ X\ W{\isacharparenright}{\kern0pt}\ {\isacharequal}{\kern0pt}\ a\ {\isasymcirc}\isactrlsub c\ {\isacharparenleft}{\kern0pt}b\ {\isasymamalg}\ c{\isacharparenright}{\kern0pt}\ {\isasymcirc}\isactrlsub c\ {\isacharparenleft}{\kern0pt}left{\isacharunderscore}{\kern0pt}coproj\ X\ W{\isacharparenright}{\kern0pt}{\isachardoublequoteclose}\isanewline
\ \ \ \ \isacommand{using}\isamarkupfalse%
\ assms\ \isacommand{by}\isamarkupfalse%
\ {\isacharparenleft}{\kern0pt}typecheck{\isacharunderscore}{\kern0pt}cfuncs{\isacharcomma}{\kern0pt}\ simp\ add{\isacharcolon}{\kern0pt}\ left{\isacharunderscore}{\kern0pt}coproj{\isacharunderscore}{\kern0pt}cfunc{\isacharunderscore}{\kern0pt}coprod{\isacharparenright}{\kern0pt}\isanewline
\ \ \isacommand{then}\isamarkupfalse%
\ \isacommand{have}\isamarkupfalse%
\ left{\isacharunderscore}{\kern0pt}coproj{\isacharunderscore}{\kern0pt}eq{\isacharcolon}{\kern0pt}\ {\isachardoublequoteopen}{\isacharparenleft}{\kern0pt}{\isacharparenleft}{\kern0pt}a\ {\isasymcirc}\isactrlsub c\ b{\isacharparenright}{\kern0pt}\ {\isasymamalg}\ {\isacharparenleft}{\kern0pt}a\ {\isasymcirc}\isactrlsub c\ c{\isacharparenright}{\kern0pt}{\isacharparenright}{\kern0pt}\ {\isasymcirc}\isactrlsub c\ {\isacharparenleft}{\kern0pt}left{\isacharunderscore}{\kern0pt}coproj\ X\ W{\isacharparenright}{\kern0pt}\ {\isacharequal}{\kern0pt}\ {\isacharparenleft}{\kern0pt}a\ {\isasymcirc}\isactrlsub c\ {\isacharparenleft}{\kern0pt}b\ {\isasymamalg}\ c{\isacharparenright}{\kern0pt}{\isacharparenright}{\kern0pt}\ {\isasymcirc}\isactrlsub c\ {\isacharparenleft}{\kern0pt}left{\isacharunderscore}{\kern0pt}coproj\ X\ W{\isacharparenright}{\kern0pt}{\isachardoublequoteclose}\isanewline
\ \ \ \ \isacommand{using}\isamarkupfalse%
\ assms\ \isacommand{by}\isamarkupfalse%
\ {\isacharparenleft}{\kern0pt}typecheck{\isacharunderscore}{\kern0pt}cfuncs{\isacharcomma}{\kern0pt}\ simp\ add{\isacharcolon}{\kern0pt}\ comp{\isacharunderscore}{\kern0pt}associative{\isadigit{2}}{\isacharparenright}{\kern0pt}\isanewline
\ \ \isacommand{have}\isamarkupfalse%
\ {\isachardoublequoteopen}{\isacharparenleft}{\kern0pt}{\isacharparenleft}{\kern0pt}a\ {\isasymcirc}\isactrlsub c\ b{\isacharparenright}{\kern0pt}\ {\isasymamalg}\ {\isacharparenleft}{\kern0pt}a\ {\isasymcirc}\isactrlsub c\ c{\isacharparenright}{\kern0pt}{\isacharparenright}{\kern0pt}\ {\isasymcirc}\isactrlsub c\ {\isacharparenleft}{\kern0pt}right{\isacharunderscore}{\kern0pt}coproj\ X\ W{\isacharparenright}{\kern0pt}\ {\isacharequal}{\kern0pt}\ a\ {\isasymcirc}\isactrlsub c\ {\isacharparenleft}{\kern0pt}b\ {\isasymamalg}\ c{\isacharparenright}{\kern0pt}\ {\isasymcirc}\isactrlsub c\ {\isacharparenleft}{\kern0pt}right{\isacharunderscore}{\kern0pt}coproj\ X\ W{\isacharparenright}{\kern0pt}{\isachardoublequoteclose}\isanewline
\ \ \ \ \isacommand{using}\isamarkupfalse%
\ assms\ \isacommand{by}\isamarkupfalse%
\ {\isacharparenleft}{\kern0pt}typecheck{\isacharunderscore}{\kern0pt}cfuncs{\isacharcomma}{\kern0pt}\ simp\ add{\isacharcolon}{\kern0pt}\ right{\isacharunderscore}{\kern0pt}coproj{\isacharunderscore}{\kern0pt}cfunc{\isacharunderscore}{\kern0pt}coprod{\isacharparenright}{\kern0pt}\isanewline
\ \ \isacommand{then}\isamarkupfalse%
\ \isacommand{have}\isamarkupfalse%
\ right{\isacharunderscore}{\kern0pt}coproj{\isacharunderscore}{\kern0pt}eq{\isacharcolon}{\kern0pt}\ {\isachardoublequoteopen}{\isacharparenleft}{\kern0pt}{\isacharparenleft}{\kern0pt}a\ {\isasymcirc}\isactrlsub c\ b{\isacharparenright}{\kern0pt}\ {\isasymamalg}\ {\isacharparenleft}{\kern0pt}a\ {\isasymcirc}\isactrlsub c\ c{\isacharparenright}{\kern0pt}{\isacharparenright}{\kern0pt}\ {\isasymcirc}\isactrlsub c\ {\isacharparenleft}{\kern0pt}right{\isacharunderscore}{\kern0pt}coproj\ X\ W{\isacharparenright}{\kern0pt}\ {\isacharequal}{\kern0pt}\ {\isacharparenleft}{\kern0pt}a\ {\isasymcirc}\isactrlsub c\ {\isacharparenleft}{\kern0pt}b\ {\isasymamalg}\ c{\isacharparenright}{\kern0pt}{\isacharparenright}{\kern0pt}\ {\isasymcirc}\isactrlsub c\ {\isacharparenleft}{\kern0pt}right{\isacharunderscore}{\kern0pt}coproj\ X\ W{\isacharparenright}{\kern0pt}{\isachardoublequoteclose}\isanewline
\ \ \ \ \isacommand{using}\isamarkupfalse%
\ assms\ \isacommand{by}\isamarkupfalse%
\ {\isacharparenleft}{\kern0pt}typecheck{\isacharunderscore}{\kern0pt}cfuncs{\isacharcomma}{\kern0pt}\ simp\ add{\isacharcolon}{\kern0pt}\ comp{\isacharunderscore}{\kern0pt}associative{\isadigit{2}}{\isacharparenright}{\kern0pt}\isanewline
\isanewline
\ \ \isacommand{show}\isamarkupfalse%
\ {\isachardoublequoteopen}{\isacharparenleft}{\kern0pt}a\ {\isasymcirc}\isactrlsub c\ b{\isacharparenright}{\kern0pt}\ {\isasymamalg}\ {\isacharparenleft}{\kern0pt}a\ {\isasymcirc}\isactrlsub c\ c{\isacharparenright}{\kern0pt}\ {\isacharequal}{\kern0pt}\ a\ {\isasymcirc}\isactrlsub c\ {\isacharparenleft}{\kern0pt}b\ {\isasymamalg}\ c{\isacharparenright}{\kern0pt}{\isachardoublequoteclose}\isanewline
\ \ \ \ \isacommand{using}\isamarkupfalse%
\ assms\ left{\isacharunderscore}{\kern0pt}coproj{\isacharunderscore}{\kern0pt}eq\ right{\isacharunderscore}{\kern0pt}coproj{\isacharunderscore}{\kern0pt}eq\isanewline
\ \ \ \ \isacommand{by}\isamarkupfalse%
\ {\isacharparenleft}{\kern0pt}typecheck{\isacharunderscore}{\kern0pt}cfuncs{\isacharcomma}{\kern0pt}\ smt\ cfunc{\isacharunderscore}{\kern0pt}coprod{\isacharunderscore}{\kern0pt}unique\ left{\isacharunderscore}{\kern0pt}coproj{\isacharunderscore}{\kern0pt}cfunc{\isacharunderscore}{\kern0pt}coprod\ right{\isacharunderscore}{\kern0pt}coproj{\isacharunderscore}{\kern0pt}cfunc{\isacharunderscore}{\kern0pt}coprod{\isacharparenright}{\kern0pt}\isanewline
\isacommand{qed}\isamarkupfalse%
%
\endisatagproof
{\isafoldproof}%
%
\isadelimproof
\isanewline
%
\endisadelimproof
\isanewline
\isacommand{lemma}\isamarkupfalse%
\ id{\isacharunderscore}{\kern0pt}coprod{\isacharcolon}{\kern0pt}\isanewline
\ \ {\isachardoublequoteopen}id{\isacharparenleft}{\kern0pt}A\ {\isasymCoprod}\ B{\isacharparenright}{\kern0pt}\ {\isacharequal}{\kern0pt}\ {\isacharparenleft}{\kern0pt}left{\isacharunderscore}{\kern0pt}coproj\ A\ B{\isacharparenright}{\kern0pt}\ {\isasymamalg}\ {\isacharparenleft}{\kern0pt}right{\isacharunderscore}{\kern0pt}coproj\ A\ B{\isacharparenright}{\kern0pt}{\isachardoublequoteclose}\isanewline
%
\isadelimproof
\ \ \ \ %
\endisadelimproof
%
\isatagproof
\isacommand{by}\isamarkupfalse%
\ {\isacharparenleft}{\kern0pt}typecheck{\isacharunderscore}{\kern0pt}cfuncs{\isacharcomma}{\kern0pt}\ simp\ add{\isacharcolon}{\kern0pt}\ cfunc{\isacharunderscore}{\kern0pt}coprod{\isacharunderscore}{\kern0pt}unique\ id{\isacharunderscore}{\kern0pt}left{\isacharunderscore}{\kern0pt}unit{\isadigit{2}}{\isacharparenright}{\kern0pt}%
\endisatagproof
{\isafoldproof}%
%
\isadelimproof
%
\endisadelimproof
%
\begin{isamarkuptext}%
The lemma below corresponds to Proposition 2.4.1 in Halvorson.%
\end{isamarkuptext}\isamarkuptrue%
\isacommand{lemma}\isamarkupfalse%
\ coproducts{\isacharunderscore}{\kern0pt}disjoint{\isacharcolon}{\kern0pt}\isanewline
\ \ {\isachardoublequoteopen}\ x{\isasymin}\isactrlsub c\ X\ {\isasymLongrightarrow}\ \ y\ {\isasymin}\isactrlsub c\ Y\ {\isasymLongrightarrow}\ \ {\isacharparenleft}{\kern0pt}left{\isacharunderscore}{\kern0pt}coproj\ X\ Y{\isacharparenright}{\kern0pt}\ {\isasymcirc}\isactrlsub c\ x\ {\isasymnoteq}\ {\isacharparenleft}{\kern0pt}right{\isacharunderscore}{\kern0pt}coproj\ X\ Y{\isacharparenright}{\kern0pt}\ {\isasymcirc}\isactrlsub c\ y{\isachardoublequoteclose}\isanewline
%
\isadelimproof
%
\endisadelimproof
%
\isatagproof
\isacommand{proof}\isamarkupfalse%
\ {\isacharparenleft}{\kern0pt}rule\ ccontr{\isacharcomma}{\kern0pt}\ auto{\isacharparenright}{\kern0pt}\isanewline
\ \ \isacommand{assume}\isamarkupfalse%
\ x{\isacharunderscore}{\kern0pt}type{\isacharbrackleft}{\kern0pt}type{\isacharunderscore}{\kern0pt}rule{\isacharbrackright}{\kern0pt}{\isacharcolon}{\kern0pt}\ {\isachardoublequoteopen}x{\isasymin}\isactrlsub c\ X{\isachardoublequoteclose}\ \isanewline
\ \ \isacommand{assume}\isamarkupfalse%
\ y{\isacharunderscore}{\kern0pt}type{\isacharbrackleft}{\kern0pt}type{\isacharunderscore}{\kern0pt}rule{\isacharbrackright}{\kern0pt}{\isacharcolon}{\kern0pt}\ {\isachardoublequoteopen}y\ {\isasymin}\isactrlsub c\ Y{\isachardoublequoteclose}\isanewline
\ \ \isacommand{assume}\isamarkupfalse%
\ BWOC{\isacharcolon}{\kern0pt}\ {\isachardoublequoteopen}{\isacharparenleft}{\kern0pt}{\isacharparenleft}{\kern0pt}left{\isacharunderscore}{\kern0pt}coproj\ X\ Y{\isacharparenright}{\kern0pt}\ {\isasymcirc}\isactrlsub c\ x\ {\isacharequal}{\kern0pt}\ {\isacharparenleft}{\kern0pt}right{\isacharunderscore}{\kern0pt}coproj\ X\ Y{\isacharparenright}{\kern0pt}\ {\isasymcirc}\isactrlsub c\ y{\isacharparenright}{\kern0pt}{\isachardoublequoteclose}\isanewline
\ \ \isacommand{obtain}\isamarkupfalse%
\ g\ \isakeyword{where}\ g{\isacharunderscore}{\kern0pt}def{\isacharcolon}{\kern0pt}\ {\isachardoublequoteopen}g\ factorsthru\ \ {\isasymt}{\isachardoublequoteclose}\ \isakeyword{and}\ g{\isacharunderscore}{\kern0pt}type{\isacharbrackleft}{\kern0pt}type{\isacharunderscore}{\kern0pt}rule{\isacharbrackright}{\kern0pt}{\isacharcolon}{\kern0pt}\ {\isachardoublequoteopen}g{\isacharcolon}{\kern0pt}\ X\ {\isasymrightarrow}\ {\isasymOmega}{\isachardoublequoteclose}\isanewline
\ \ \ \ \isacommand{by}\isamarkupfalse%
\ {\isacharparenleft}{\kern0pt}typecheck{\isacharunderscore}{\kern0pt}cfuncs{\isacharcomma}{\kern0pt}\ meson\ comp{\isacharunderscore}{\kern0pt}type\ factors{\isacharunderscore}{\kern0pt}through{\isacharunderscore}{\kern0pt}def{\isadigit{2}}\ terminal{\isacharunderscore}{\kern0pt}func{\isacharunderscore}{\kern0pt}type{\isacharparenright}{\kern0pt}\isanewline
\ \ \isacommand{then}\isamarkupfalse%
\ \isacommand{have}\isamarkupfalse%
\ fact{\isadigit{1}}{\isacharcolon}{\kern0pt}\ {\isachardoublequoteopen}{\isasymt}\ {\isacharequal}{\kern0pt}\ g\ {\isasymcirc}\isactrlsub c\ x{\isachardoublequoteclose}\isanewline
\ \ \ \ \isacommand{by}\isamarkupfalse%
\ {\isacharparenleft}{\kern0pt}metis\ cfunc{\isacharunderscore}{\kern0pt}type{\isacharunderscore}{\kern0pt}def\ comp{\isacharunderscore}{\kern0pt}associative\ factors{\isacharunderscore}{\kern0pt}through{\isacharunderscore}{\kern0pt}def\ id{\isacharunderscore}{\kern0pt}right{\isacharunderscore}{\kern0pt}unit{\isadigit{2}}\ id{\isacharunderscore}{\kern0pt}type\isanewline
\ \ \ \ \ \ \ \ terminal{\isacharunderscore}{\kern0pt}func{\isacharunderscore}{\kern0pt}comp\ terminal{\isacharunderscore}{\kern0pt}func{\isacharunderscore}{\kern0pt}unique\ true{\isacharunderscore}{\kern0pt}func{\isacharunderscore}{\kern0pt}type\ x{\isacharunderscore}{\kern0pt}type{\isacharparenright}{\kern0pt}\isanewline
\ \ \ \ \ \isanewline
\ \ \isacommand{obtain}\isamarkupfalse%
\ h\ \isakeyword{where}\ h{\isacharunderscore}{\kern0pt}def{\isacharcolon}{\kern0pt}\ {\isachardoublequoteopen}h\ factorsthru\ {\isasymf}{\isachardoublequoteclose}\ \isakeyword{and}\ h{\isacharunderscore}{\kern0pt}type{\isacharbrackleft}{\kern0pt}type{\isacharunderscore}{\kern0pt}rule{\isacharbrackright}{\kern0pt}{\isacharcolon}{\kern0pt}\ {\isachardoublequoteopen}h{\isacharcolon}{\kern0pt}\ Y\ {\isasymrightarrow}\ {\isasymOmega}{\isachardoublequoteclose}\isanewline
\ \ \ \ \isacommand{by}\isamarkupfalse%
\ {\isacharparenleft}{\kern0pt}typecheck{\isacharunderscore}{\kern0pt}cfuncs{\isacharcomma}{\kern0pt}\ meson\ comp{\isacharunderscore}{\kern0pt}type\ factors{\isacharunderscore}{\kern0pt}through{\isacharunderscore}{\kern0pt}def{\isadigit{2}}\ one{\isacharunderscore}{\kern0pt}terminal{\isacharunderscore}{\kern0pt}object\ terminal{\isacharunderscore}{\kern0pt}object{\isacharunderscore}{\kern0pt}def{\isacharparenright}{\kern0pt}\isanewline
\ \ \isacommand{then}\isamarkupfalse%
\ \isacommand{have}\isamarkupfalse%
\ gUh{\isacharunderscore}{\kern0pt}type{\isacharbrackleft}{\kern0pt}type{\isacharunderscore}{\kern0pt}rule{\isacharbrackright}{\kern0pt}{\isacharcolon}{\kern0pt}\ {\isachardoublequoteopen}g\ {\isasymamalg}\ h{\isacharcolon}{\kern0pt}\ X\ {\isasymCoprod}\ Y\ {\isasymrightarrow}\ {\isasymOmega}{\isachardoublequoteclose}\ \isakeyword{and}\ \isanewline
\ \ \ \ \ \ \ \ \ \ \ \ \ \ \ \ \ \ \ \ \ \ \ \ gUh{\isacharunderscore}{\kern0pt}def{\isacharcolon}{\kern0pt}\ {\isachardoublequoteopen}{\isacharparenleft}{\kern0pt}g\ {\isasymamalg}\ h{\isacharparenright}{\kern0pt}\ {\isasymcirc}\isactrlsub c\ {\isacharparenleft}{\kern0pt}left{\isacharunderscore}{\kern0pt}coproj\ X\ Y{\isacharparenright}{\kern0pt}\ {\isacharequal}{\kern0pt}\ g\ {\isasymand}\ \ {\isacharparenleft}{\kern0pt}g\ {\isasymamalg}\ h{\isacharparenright}{\kern0pt}\ {\isasymcirc}\isactrlsub c\ {\isacharparenleft}{\kern0pt}right{\isacharunderscore}{\kern0pt}coproj\ X\ Y{\isacharparenright}{\kern0pt}\ {\isacharequal}{\kern0pt}\ h{\isachardoublequoteclose}\isanewline
\ \ \ \ \isacommand{using}\isamarkupfalse%
\ left{\isacharunderscore}{\kern0pt}coproj{\isacharunderscore}{\kern0pt}cfunc{\isacharunderscore}{\kern0pt}coprod\ right{\isacharunderscore}{\kern0pt}coproj{\isacharunderscore}{\kern0pt}cfunc{\isacharunderscore}{\kern0pt}coprod\ \isacommand{by}\isamarkupfalse%
\ {\isacharparenleft}{\kern0pt}typecheck{\isacharunderscore}{\kern0pt}cfuncs{\isacharcomma}{\kern0pt}\ presburger{\isacharparenright}{\kern0pt}\isanewline
\ \ \isacommand{then}\isamarkupfalse%
\ \isacommand{have}\isamarkupfalse%
\ fact{\isadigit{2}}{\isacharcolon}{\kern0pt}\ {\isachardoublequoteopen}{\isasymf}\ {\isacharequal}{\kern0pt}\ {\isacharparenleft}{\kern0pt}{\isacharparenleft}{\kern0pt}g\ {\isasymamalg}\ h{\isacharparenright}{\kern0pt}\ {\isasymcirc}\isactrlsub c\ {\isacharparenleft}{\kern0pt}right{\isacharunderscore}{\kern0pt}coproj\ X\ Y{\isacharparenright}{\kern0pt}{\isacharparenright}{\kern0pt}\ {\isasymcirc}\isactrlsub c\ y{\isachardoublequoteclose}\isanewline
\ \ \ \ \isacommand{by}\isamarkupfalse%
\ {\isacharparenleft}{\kern0pt}typecheck{\isacharunderscore}{\kern0pt}cfuncs{\isacharcomma}{\kern0pt}\ smt\ {\isacharparenleft}{\kern0pt}verit{\isacharcomma}{\kern0pt}\ ccfv{\isacharunderscore}{\kern0pt}SIG{\isacharparenright}{\kern0pt}\ comp{\isacharunderscore}{\kern0pt}associative{\isadigit{2}}\ factors{\isacharunderscore}{\kern0pt}through{\isacharunderscore}{\kern0pt}def{\isadigit{2}}\ gUh{\isacharunderscore}{\kern0pt}def\ h{\isacharunderscore}{\kern0pt}def\ id{\isacharunderscore}{\kern0pt}right{\isacharunderscore}{\kern0pt}unit{\isadigit{2}}\ terminal{\isacharunderscore}{\kern0pt}func{\isacharunderscore}{\kern0pt}comp{\isacharunderscore}{\kern0pt}elem\ terminal{\isacharunderscore}{\kern0pt}func{\isacharunderscore}{\kern0pt}unique{\isacharparenright}{\kern0pt}\isanewline
\ \ \isacommand{also}\isamarkupfalse%
\ \isacommand{have}\isamarkupfalse%
\ {\isachardoublequoteopen}{\isachardot}{\kern0pt}{\isachardot}{\kern0pt}{\isachardot}{\kern0pt}\ {\isacharequal}{\kern0pt}\ {\isacharparenleft}{\kern0pt}{\isacharparenleft}{\kern0pt}g\ {\isasymamalg}\ h{\isacharparenright}{\kern0pt}\ {\isasymcirc}\isactrlsub c\ {\isacharparenleft}{\kern0pt}left{\isacharunderscore}{\kern0pt}coproj\ X\ Y{\isacharparenright}{\kern0pt}{\isacharparenright}{\kern0pt}\ {\isasymcirc}\isactrlsub c\ x{\isachardoublequoteclose}\isanewline
\ \ \ \ \isacommand{by}\isamarkupfalse%
\ {\isacharparenleft}{\kern0pt}smt\ BWOC\ comp{\isacharunderscore}{\kern0pt}associative{\isadigit{2}}\ gUh{\isacharunderscore}{\kern0pt}type\ left{\isacharunderscore}{\kern0pt}proj{\isacharunderscore}{\kern0pt}type\ right{\isacharunderscore}{\kern0pt}proj{\isacharunderscore}{\kern0pt}type\ x{\isacharunderscore}{\kern0pt}type\ y{\isacharunderscore}{\kern0pt}type{\isacharparenright}{\kern0pt}\ \isanewline
\ \ \isacommand{also}\isamarkupfalse%
\ \isacommand{have}\isamarkupfalse%
\ {\isachardoublequoteopen}{\isachardot}{\kern0pt}{\isachardot}{\kern0pt}{\isachardot}{\kern0pt}\ {\isacharequal}{\kern0pt}\ {\isasymt}{\isachardoublequoteclose}\isanewline
\ \ \ \ \isacommand{by}\isamarkupfalse%
\ {\isacharparenleft}{\kern0pt}simp\ add{\isacharcolon}{\kern0pt}\ fact{\isadigit{1}}\ gUh{\isacharunderscore}{\kern0pt}def{\isacharparenright}{\kern0pt}\isanewline
\ \ \isacommand{then}\isamarkupfalse%
\ \isacommand{show}\isamarkupfalse%
\ False\isanewline
\ \ \ \ \isacommand{using}\isamarkupfalse%
\ calculation\ true{\isacharunderscore}{\kern0pt}false{\isacharunderscore}{\kern0pt}distinct\ \isacommand{by}\isamarkupfalse%
\ auto\isanewline
\isacommand{qed}\isamarkupfalse%
%
\endisatagproof
{\isafoldproof}%
%
\isadelimproof
%
\endisadelimproof
%
\begin{isamarkuptext}%
The lemma below corresponds to Proposition 2.4.2 in Halvorson.%
\end{isamarkuptext}\isamarkuptrue%
\isacommand{lemma}\isamarkupfalse%
\ left{\isacharunderscore}{\kern0pt}coproj{\isacharunderscore}{\kern0pt}are{\isacharunderscore}{\kern0pt}monomorphisms{\isacharcolon}{\kern0pt}\isanewline
\ \ {\isachardoublequoteopen}monomorphism{\isacharparenleft}{\kern0pt}left{\isacharunderscore}{\kern0pt}coproj\ X\ Y{\isacharparenright}{\kern0pt}{\isachardoublequoteclose}\isanewline
%
\isadelimproof
%
\endisadelimproof
%
\isatagproof
\isacommand{proof}\isamarkupfalse%
\ {\isacharparenleft}{\kern0pt}cases\ {\isachardoublequoteopen}{\isasymexists}x{\isachardot}{\kern0pt}\ x\ {\isasymin}\isactrlsub c\ X{\isachardoublequoteclose}{\isacharparenright}{\kern0pt}\isanewline
\ \ \isacommand{assume}\isamarkupfalse%
\ X{\isacharunderscore}{\kern0pt}nonempty{\isacharcolon}{\kern0pt}\ {\isachardoublequoteopen}{\isasymexists}x{\isachardot}{\kern0pt}\ x\ {\isasymin}\isactrlsub c\ X{\isachardoublequoteclose}\isanewline
\ \ \isacommand{then}\isamarkupfalse%
\ \isacommand{obtain}\isamarkupfalse%
\ x\ \isakeyword{where}\ x{\isacharunderscore}{\kern0pt}type{\isacharbrackleft}{\kern0pt}type{\isacharunderscore}{\kern0pt}rule{\isacharbrackright}{\kern0pt}{\isacharcolon}{\kern0pt}\ {\isachardoublequoteopen}x\ {\isasymin}\isactrlsub c\ X{\isachardoublequoteclose}\isanewline
\ \ \ \ \isacommand{by}\isamarkupfalse%
\ auto\isanewline
\ \ \isacommand{then}\isamarkupfalse%
\ \isacommand{have}\isamarkupfalse%
\ {\isachardoublequoteopen}{\isacharparenleft}{\kern0pt}id\ X\ {\isasymamalg}\ {\isacharparenleft}{\kern0pt}x\ {\isasymcirc}\isactrlsub c\ {\isasymbeta}\isactrlbsub Y\isactrlesub {\isacharparenright}{\kern0pt}{\isacharparenright}{\kern0pt}\ {\isasymcirc}\isactrlsub c\ left{\isacharunderscore}{\kern0pt}coproj\ X\ Y\ {\isacharequal}{\kern0pt}\ id\ X{\isachardoublequoteclose}\isanewline
\ \ \ \ \isacommand{by}\isamarkupfalse%
\ {\isacharparenleft}{\kern0pt}typecheck{\isacharunderscore}{\kern0pt}cfuncs{\isacharcomma}{\kern0pt}\ simp\ add{\isacharcolon}{\kern0pt}\ left{\isacharunderscore}{\kern0pt}coproj{\isacharunderscore}{\kern0pt}cfunc{\isacharunderscore}{\kern0pt}coprod{\isacharparenright}{\kern0pt}\isanewline
\ \ \isacommand{then}\isamarkupfalse%
\ \isacommand{show}\isamarkupfalse%
\ {\isachardoublequoteopen}monomorphism\ {\isacharparenleft}{\kern0pt}left{\isacharunderscore}{\kern0pt}coproj\ X\ Y{\isacharparenright}{\kern0pt}{\isachardoublequoteclose}\isanewline
\ \ \ \ \isacommand{by}\isamarkupfalse%
\ {\isacharparenleft}{\kern0pt}typecheck{\isacharunderscore}{\kern0pt}cfuncs{\isacharcomma}{\kern0pt}\ metis\ {\isacharparenleft}{\kern0pt}mono{\isacharunderscore}{\kern0pt}tags{\isacharparenright}{\kern0pt}\ cfunc{\isacharunderscore}{\kern0pt}coprod{\isacharunderscore}{\kern0pt}type\ comp{\isacharunderscore}{\kern0pt}monic{\isacharunderscore}{\kern0pt}imp{\isacharunderscore}{\kern0pt}monic{\isacharprime}{\kern0pt}\isanewline
\ \ \ \ \ \ \ \ comp{\isacharunderscore}{\kern0pt}type\ id{\isacharunderscore}{\kern0pt}isomorphism\ id{\isacharunderscore}{\kern0pt}type\ iso{\isacharunderscore}{\kern0pt}imp{\isacharunderscore}{\kern0pt}epi{\isacharunderscore}{\kern0pt}and{\isacharunderscore}{\kern0pt}monic\ terminal{\isacharunderscore}{\kern0pt}func{\isacharunderscore}{\kern0pt}type\ x{\isacharunderscore}{\kern0pt}type{\isacharparenright}{\kern0pt}\isanewline
\isacommand{next}\isamarkupfalse%
\isanewline
\ \ \isacommand{show}\isamarkupfalse%
\ {\isachardoublequoteopen}{\isasymnexists}x{\isachardot}{\kern0pt}\ x\ {\isasymin}\isactrlsub c\ X\ {\isasymLongrightarrow}\ monomorphism\ {\isacharparenleft}{\kern0pt}left{\isacharunderscore}{\kern0pt}coproj\ X\ Y{\isacharparenright}{\kern0pt}{\isachardoublequoteclose}\isanewline
\ \ \ \ \isacommand{by}\isamarkupfalse%
\ {\isacharparenleft}{\kern0pt}typecheck{\isacharunderscore}{\kern0pt}cfuncs{\isacharcomma}{\kern0pt}\ metis\ cfunc{\isacharunderscore}{\kern0pt}type{\isacharunderscore}{\kern0pt}def\ injective{\isacharunderscore}{\kern0pt}def\ injective{\isacharunderscore}{\kern0pt}imp{\isacharunderscore}{\kern0pt}monomorphism{\isacharparenright}{\kern0pt}\isanewline
\isacommand{qed}\isamarkupfalse%
%
\endisatagproof
{\isafoldproof}%
%
\isadelimproof
\isanewline
%
\endisadelimproof
\isanewline
\isacommand{lemma}\isamarkupfalse%
\ right{\isacharunderscore}{\kern0pt}coproj{\isacharunderscore}{\kern0pt}are{\isacharunderscore}{\kern0pt}monomorphisms{\isacharcolon}{\kern0pt}\isanewline
\ \ {\isachardoublequoteopen}monomorphism{\isacharparenleft}{\kern0pt}right{\isacharunderscore}{\kern0pt}coproj\ X\ Y{\isacharparenright}{\kern0pt}{\isachardoublequoteclose}\isanewline
%
\isadelimproof
%
\endisadelimproof
%
\isatagproof
\isacommand{proof}\isamarkupfalse%
\ {\isacharparenleft}{\kern0pt}cases\ {\isachardoublequoteopen}{\isasymexists}y{\isachardot}{\kern0pt}\ y\ {\isasymin}\isactrlsub c\ Y{\isachardoublequoteclose}{\isacharparenright}{\kern0pt}\isanewline
\ \ \isacommand{assume}\isamarkupfalse%
\ Y{\isacharunderscore}{\kern0pt}nonempty{\isacharcolon}{\kern0pt}\ {\isachardoublequoteopen}{\isasymexists}y{\isachardot}{\kern0pt}\ y\ {\isasymin}\isactrlsub c\ Y{\isachardoublequoteclose}\isanewline
\ \ \isacommand{then}\isamarkupfalse%
\ \isacommand{obtain}\isamarkupfalse%
\ y\ \isakeyword{where}\ y{\isacharunderscore}{\kern0pt}type{\isacharbrackleft}{\kern0pt}type{\isacharunderscore}{\kern0pt}rule{\isacharbrackright}{\kern0pt}{\isacharcolon}{\kern0pt}\ \ {\isachardoublequoteopen}y\ {\isasymin}\isactrlsub c\ Y{\isachardoublequoteclose}\isanewline
\ \ \ \ \isacommand{by}\isamarkupfalse%
\ auto\isanewline
\ \ \isacommand{have}\isamarkupfalse%
\ {\isachardoublequoteopen}{\isacharparenleft}{\kern0pt}{\isacharparenleft}{\kern0pt}y\ {\isasymcirc}\isactrlsub c\ {\isasymbeta}\isactrlbsub X\isactrlesub {\isacharparenright}{\kern0pt}\ {\isasymamalg}\ id\ Y{\isacharparenright}{\kern0pt}\ {\isasymcirc}\isactrlsub c\ right{\isacharunderscore}{\kern0pt}coproj\ X\ Y\ {\isacharequal}{\kern0pt}\ id\ Y{\isachardoublequoteclose}\isanewline
\ \ \ \ \isacommand{by}\isamarkupfalse%
\ {\isacharparenleft}{\kern0pt}typecheck{\isacharunderscore}{\kern0pt}cfuncs{\isacharcomma}{\kern0pt}\ simp\ add{\isacharcolon}{\kern0pt}\ right{\isacharunderscore}{\kern0pt}coproj{\isacharunderscore}{\kern0pt}cfunc{\isacharunderscore}{\kern0pt}coprod{\isacharparenright}{\kern0pt}\isanewline
\ \ \isacommand{then}\isamarkupfalse%
\ \isacommand{show}\isamarkupfalse%
\ {\isachardoublequoteopen}monomorphism\ {\isacharparenleft}{\kern0pt}right{\isacharunderscore}{\kern0pt}coproj\ X\ Y{\isacharparenright}{\kern0pt}{\isachardoublequoteclose}\isanewline
\ \ \ \ \isacommand{by}\isamarkupfalse%
\ {\isacharparenleft}{\kern0pt}typecheck{\isacharunderscore}{\kern0pt}cfuncs{\isacharcomma}{\kern0pt}\ metis\ {\isacharparenleft}{\kern0pt}mono{\isacharunderscore}{\kern0pt}tags{\isacharparenright}{\kern0pt}\ cfunc{\isacharunderscore}{\kern0pt}coprod{\isacharunderscore}{\kern0pt}type\ comp{\isacharunderscore}{\kern0pt}monic{\isacharunderscore}{\kern0pt}imp{\isacharunderscore}{\kern0pt}monic{\isacharprime}{\kern0pt}\isanewline
\ \ \ \ \ \ \ \ comp{\isacharunderscore}{\kern0pt}type\ id{\isacharunderscore}{\kern0pt}isomorphism\ id{\isacharunderscore}{\kern0pt}type\ iso{\isacharunderscore}{\kern0pt}imp{\isacharunderscore}{\kern0pt}epi{\isacharunderscore}{\kern0pt}and{\isacharunderscore}{\kern0pt}monic\ terminal{\isacharunderscore}{\kern0pt}func{\isacharunderscore}{\kern0pt}type\ y{\isacharunderscore}{\kern0pt}type{\isacharparenright}{\kern0pt}\isanewline
\isacommand{next}\isamarkupfalse%
\isanewline
\ \ \isacommand{show}\isamarkupfalse%
\ {\isachardoublequoteopen}{\isasymnexists}y{\isachardot}{\kern0pt}\ y\ {\isasymin}\isactrlsub c\ Y\ {\isasymLongrightarrow}\ monomorphism\ {\isacharparenleft}{\kern0pt}right{\isacharunderscore}{\kern0pt}coproj\ X\ Y{\isacharparenright}{\kern0pt}{\isachardoublequoteclose}\isanewline
\ \ \ \ \isacommand{by}\isamarkupfalse%
\ {\isacharparenleft}{\kern0pt}typecheck{\isacharunderscore}{\kern0pt}cfuncs{\isacharcomma}{\kern0pt}\ metis\ cfunc{\isacharunderscore}{\kern0pt}type{\isacharunderscore}{\kern0pt}def\ injective{\isacharunderscore}{\kern0pt}def\ injective{\isacharunderscore}{\kern0pt}imp{\isacharunderscore}{\kern0pt}monomorphism{\isacharparenright}{\kern0pt}\isanewline
\isacommand{qed}\isamarkupfalse%
%
\endisatagproof
{\isafoldproof}%
%
\isadelimproof
%
\endisadelimproof
%
\begin{isamarkuptext}%
The lemma below corresponds to Exercise 2.4.3 in Halvorson.%
\end{isamarkuptext}\isamarkuptrue%
\isacommand{lemma}\isamarkupfalse%
\ coprojs{\isacharunderscore}{\kern0pt}jointly{\isacharunderscore}{\kern0pt}surj{\isacharcolon}{\kern0pt}\isanewline
\ \ \isakeyword{assumes}\ {\isachardoublequoteopen}z\ {\isasymin}\isactrlsub c\ X\ {\isasymCoprod}\ Y{\isachardoublequoteclose}\isanewline
\ \ \isakeyword{shows}\ {\isachardoublequoteopen}{\isacharparenleft}{\kern0pt}{\isasymexists}\ x{\isachardot}{\kern0pt}\ {\isacharparenleft}{\kern0pt}x\ {\isasymin}\isactrlsub c\ X\ {\isasymand}\ z\ {\isacharequal}{\kern0pt}\ {\isacharparenleft}{\kern0pt}left{\isacharunderscore}{\kern0pt}coproj\ X\ Y{\isacharparenright}{\kern0pt}\ {\isasymcirc}\isactrlsub c\ x{\isacharparenright}{\kern0pt}{\isacharparenright}{\kern0pt}\isanewline
\ \ \ \ \ \ {\isasymor}\ \ {\isacharparenleft}{\kern0pt}{\isasymexists}\ y{\isachardot}{\kern0pt}\ {\isacharparenleft}{\kern0pt}y\ {\isasymin}\isactrlsub c\ Y\ {\isasymand}\ z\ {\isacharequal}{\kern0pt}\ {\isacharparenleft}{\kern0pt}right{\isacharunderscore}{\kern0pt}coproj\ X\ Y{\isacharparenright}{\kern0pt}\ {\isasymcirc}\isactrlsub c\ y{\isacharparenright}{\kern0pt}{\isacharparenright}{\kern0pt}{\isachardoublequoteclose}\isanewline
%
\isadelimproof
%
\endisadelimproof
%
\isatagproof
\isacommand{proof}\isamarkupfalse%
\ {\isacharparenleft}{\kern0pt}rule\ ccontr{\isacharcomma}{\kern0pt}\ auto{\isacharparenright}{\kern0pt}\isanewline
\ \ \isacommand{assume}\isamarkupfalse%
\ not{\isacharunderscore}{\kern0pt}in{\isacharunderscore}{\kern0pt}left{\isacharunderscore}{\kern0pt}image{\isacharcolon}{\kern0pt}\ {\isachardoublequoteopen}{\isasymforall}x{\isachardot}{\kern0pt}\ x\ {\isasymin}\isactrlsub c\ X\ {\isasymlongrightarrow}\ z\ {\isasymnoteq}\ left{\isacharunderscore}{\kern0pt}coproj\ X\ Y\ {\isasymcirc}\isactrlsub c\ x{\isachardoublequoteclose}\isanewline
\ \ \isacommand{assume}\isamarkupfalse%
\ not{\isacharunderscore}{\kern0pt}in{\isacharunderscore}{\kern0pt}right{\isacharunderscore}{\kern0pt}image{\isacharcolon}{\kern0pt}\ {\isachardoublequoteopen}{\isasymforall}y{\isachardot}{\kern0pt}\ y\ {\isasymin}\isactrlsub c\ Y\ {\isasymlongrightarrow}\ z\ {\isasymnoteq}\ right{\isacharunderscore}{\kern0pt}coproj\ X\ Y\ {\isasymcirc}\isactrlsub c\ y{\isachardoublequoteclose}\isanewline
\ \ \isanewline
\ \ \isacommand{obtain}\isamarkupfalse%
\ h\ \isakeyword{where}\ h{\isacharunderscore}{\kern0pt}def{\isacharcolon}{\kern0pt}\ {\isachardoublequoteopen}h\ {\isacharequal}{\kern0pt}\ {\isasymf}\ {\isasymcirc}\isactrlsub c\ {\isasymbeta}\isactrlbsub X\ {\isasymCoprod}\ Y\isactrlesub {\isachardoublequoteclose}\ \isakeyword{and}\ h{\isacharunderscore}{\kern0pt}type{\isacharbrackleft}{\kern0pt}type{\isacharunderscore}{\kern0pt}rule{\isacharbrackright}{\kern0pt}{\isacharcolon}{\kern0pt}\ {\isachardoublequoteopen}h\ {\isacharcolon}{\kern0pt}\ X\ {\isasymCoprod}\ Y\ {\isasymrightarrow}\ {\isasymOmega}{\isachardoublequoteclose}\isanewline
\ \ \ \ \isacommand{by}\isamarkupfalse%
\ typecheck{\isacharunderscore}{\kern0pt}cfuncs\isanewline
\isanewline
\ \ \isacommand{have}\isamarkupfalse%
\ fact{\isadigit{1}}{\isacharcolon}{\kern0pt}\ {\isachardoublequoteopen}{\isacharparenleft}{\kern0pt}eq{\isacharunderscore}{\kern0pt}pred\ {\isacharparenleft}{\kern0pt}X\ {\isasymCoprod}\ Y{\isacharparenright}{\kern0pt}\ {\isasymcirc}\isactrlsub c\ {\isasymlangle}z\ {\isasymcirc}\isactrlsub c\ {\isasymbeta}\isactrlbsub X\ {\isasymCoprod}\ Y\isactrlesub {\isacharcomma}{\kern0pt}\ id\ {\isacharparenleft}{\kern0pt}X\ {\isasymCoprod}\ Y{\isacharparenright}{\kern0pt}{\isasymrangle}{\isacharparenright}{\kern0pt}\ {\isasymcirc}\isactrlsub c\ left{\isacharunderscore}{\kern0pt}coproj\ X\ Y\ {\isacharequal}{\kern0pt}\ h\ {\isasymcirc}\isactrlsub c\ left{\isacharunderscore}{\kern0pt}coproj\ X\ Y{\isachardoublequoteclose}\isanewline
\ \ \isacommand{proof}\isamarkupfalse%
{\isacharparenleft}{\kern0pt}rule\ one{\isacharunderscore}{\kern0pt}separator{\isacharbrackleft}{\kern0pt}\isakeyword{where}\ X{\isacharequal}{\kern0pt}X{\isacharcomma}{\kern0pt}\ \isakeyword{where}\ Y\ {\isacharequal}{\kern0pt}\ {\isasymOmega}{\isacharbrackright}{\kern0pt}{\isacharparenright}{\kern0pt}\isanewline
\ \ \ \ \isacommand{show}\isamarkupfalse%
\ {\isachardoublequoteopen}{\isacharparenleft}{\kern0pt}eq{\isacharunderscore}{\kern0pt}pred\ {\isacharparenleft}{\kern0pt}X\ {\isasymCoprod}\ Y{\isacharparenright}{\kern0pt}\ {\isasymcirc}\isactrlsub c\ {\isasymlangle}z\ {\isasymcirc}\isactrlsub c\ {\isasymbeta}\isactrlbsub X\ {\isasymCoprod}\ Y\isactrlesub {\isacharcomma}{\kern0pt}id\isactrlsub c\ {\isacharparenleft}{\kern0pt}X\ {\isasymCoprod}\ Y{\isacharparenright}{\kern0pt}{\isasymrangle}{\isacharparenright}{\kern0pt}\ {\isasymcirc}\isactrlsub c\ left{\isacharunderscore}{\kern0pt}coproj\ X\ Y\ {\isacharcolon}{\kern0pt}\ X\ {\isasymrightarrow}\ {\isasymOmega}{\isachardoublequoteclose}\isanewline
\ \ \ \ \ \ \isacommand{using}\isamarkupfalse%
\ assms\ \isacommand{by}\isamarkupfalse%
\ typecheck{\isacharunderscore}{\kern0pt}cfuncs\isanewline
\ \ \ \ \isacommand{show}\isamarkupfalse%
\ {\isachardoublequoteopen}h\ {\isasymcirc}\isactrlsub c\ left{\isacharunderscore}{\kern0pt}coproj\ X\ Y\ {\isacharcolon}{\kern0pt}\ X\ {\isasymrightarrow}\ {\isasymOmega}{\isachardoublequoteclose}\isanewline
\ \ \ \ \ \ \isacommand{by}\isamarkupfalse%
\ typecheck{\isacharunderscore}{\kern0pt}cfuncs\isanewline
\ \ \ \ \isacommand{show}\isamarkupfalse%
\ {\isachardoublequoteopen}{\isasymAnd}x{\isachardot}{\kern0pt}\ x\ {\isasymin}\isactrlsub c\ X\ {\isasymLongrightarrow}\ {\isacharparenleft}{\kern0pt}{\isacharparenleft}{\kern0pt}eq{\isacharunderscore}{\kern0pt}pred\ {\isacharparenleft}{\kern0pt}X\ {\isasymCoprod}\ Y{\isacharparenright}{\kern0pt}\ {\isasymcirc}\isactrlsub c\ {\isasymlangle}z\ {\isasymcirc}\isactrlsub c\ {\isasymbeta}\isactrlbsub X\ {\isasymCoprod}\ Y\isactrlesub {\isacharcomma}{\kern0pt}id\isactrlsub c\ {\isacharparenleft}{\kern0pt}X\ {\isasymCoprod}\ Y{\isacharparenright}{\kern0pt}{\isasymrangle}{\isacharparenright}{\kern0pt}\ {\isasymcirc}\isactrlsub c\ left{\isacharunderscore}{\kern0pt}coproj\ X\ Y{\isacharparenright}{\kern0pt}\ {\isasymcirc}\isactrlsub c\ x\ {\isacharequal}{\kern0pt}\isanewline
\ \ \ \ \ \ \ \ \ \ \ \ \ \ \ \ \ \ \ \ \ \ \ \ \ \ {\isacharparenleft}{\kern0pt}h\ {\isasymcirc}\isactrlsub c\ left{\isacharunderscore}{\kern0pt}coproj\ X\ Y{\isacharparenright}{\kern0pt}\ {\isasymcirc}\isactrlsub c\ x{\isachardoublequoteclose}\isanewline
\ \ \ \ \isacommand{proof}\isamarkupfalse%
\ {\isacharminus}{\kern0pt}\ \isanewline
\ \ \ \ \ \ \isacommand{fix}\isamarkupfalse%
\ x\isanewline
\ \ \ \ \ \ \isacommand{assume}\isamarkupfalse%
\ x{\isacharunderscore}{\kern0pt}type{\isacharcolon}{\kern0pt}\ {\isachardoublequoteopen}x\ {\isasymin}\isactrlsub c\ X{\isachardoublequoteclose}\isanewline
\ \ \ \ \ \ \isacommand{have}\isamarkupfalse%
\ {\isachardoublequoteopen}{\isacharparenleft}{\kern0pt}{\isacharparenleft}{\kern0pt}eq{\isacharunderscore}{\kern0pt}pred\ {\isacharparenleft}{\kern0pt}X\ {\isasymCoprod}\ Y{\isacharparenright}{\kern0pt}\ {\isasymcirc}\isactrlsub c\ {\isasymlangle}z\ {\isasymcirc}\isactrlsub c\ {\isasymbeta}\isactrlbsub X\ {\isasymCoprod}\ Y\isactrlesub {\isacharcomma}{\kern0pt}id\isactrlsub c\ {\isacharparenleft}{\kern0pt}X\ {\isasymCoprod}\ Y{\isacharparenright}{\kern0pt}{\isasymrangle}{\isacharparenright}{\kern0pt}\ {\isasymcirc}\isactrlsub c\ left{\isacharunderscore}{\kern0pt}coproj\ X\ Y{\isacharparenright}{\kern0pt}\ {\isasymcirc}\isactrlsub c\ x\ {\isacharequal}{\kern0pt}\ \isanewline
\ \ \ \ \ \ \ \ \ \ \ \ \ \ eq{\isacharunderscore}{\kern0pt}pred\ {\isacharparenleft}{\kern0pt}X\ {\isasymCoprod}\ Y{\isacharparenright}{\kern0pt}\ {\isasymcirc}\isactrlsub c\ {\isasymlangle}z\ {\isasymcirc}\isactrlsub c\ {\isasymbeta}\isactrlbsub X\ {\isasymCoprod}\ Y\isactrlesub {\isacharcomma}{\kern0pt}id\isactrlsub c\ {\isacharparenleft}{\kern0pt}X\ {\isasymCoprod}\ Y{\isacharparenright}{\kern0pt}{\isasymrangle}\ {\isasymcirc}\isactrlsub c\ {\isacharparenleft}{\kern0pt}left{\isacharunderscore}{\kern0pt}coproj\ X\ Y\ {\isasymcirc}\isactrlsub c\ \ x{\isacharparenright}{\kern0pt}{\isachardoublequoteclose}\isanewline
\ \ \ \ \ \ \ \ \ \ \ \ \ \isacommand{using}\isamarkupfalse%
\ x{\isacharunderscore}{\kern0pt}type\ \isacommand{by}\isamarkupfalse%
\ {\isacharparenleft}{\kern0pt}typecheck{\isacharunderscore}{\kern0pt}cfuncs{\isacharcomma}{\kern0pt}\ metis\ assms\ cfunc{\isacharunderscore}{\kern0pt}type{\isacharunderscore}{\kern0pt}def\ comp{\isacharunderscore}{\kern0pt}associative{\isacharparenright}{\kern0pt}\isanewline
\ \ \ \ \ \ \isacommand{also}\isamarkupfalse%
\ \isacommand{have}\isamarkupfalse%
\ {\isachardoublequoteopen}{\isachardot}{\kern0pt}{\isachardot}{\kern0pt}{\isachardot}{\kern0pt}\ {\isacharequal}{\kern0pt}\ {\isasymf}{\isachardoublequoteclose}\isanewline
\ \ \ \ \ \ \ \ \ \ \ \ \ \isacommand{using}\isamarkupfalse%
\ x{\isacharunderscore}{\kern0pt}type\ \isacommand{by}\isamarkupfalse%
\ {\isacharparenleft}{\kern0pt}typecheck{\isacharunderscore}{\kern0pt}cfuncs{\isacharcomma}{\kern0pt}\ simp\ add{\isacharcolon}{\kern0pt}\ assms\ \ eq{\isacharunderscore}{\kern0pt}pred{\isacharunderscore}{\kern0pt}false{\isacharunderscore}{\kern0pt}extract{\isacharunderscore}{\kern0pt}right\ not{\isacharunderscore}{\kern0pt}in{\isacharunderscore}{\kern0pt}left{\isacharunderscore}{\kern0pt}image{\isacharparenright}{\kern0pt}\isanewline
\ \ \ \ \ \ \isacommand{also}\isamarkupfalse%
\ \isacommand{have}\isamarkupfalse%
\ {\isachardoublequoteopen}{\isachardot}{\kern0pt}{\isachardot}{\kern0pt}{\isachardot}{\kern0pt}\ {\isacharequal}{\kern0pt}\ h\ {\isasymcirc}\isactrlsub c\ {\isacharparenleft}{\kern0pt}left{\isacharunderscore}{\kern0pt}coproj\ X\ Y\ {\isasymcirc}\isactrlsub c\ x{\isacharparenright}{\kern0pt}{\isachardoublequoteclose}\isanewline
\ \ \ \ \ \ \ \ \ \ \ \ \ \isacommand{using}\isamarkupfalse%
\ x{\isacharunderscore}{\kern0pt}type\ \isacommand{by}\isamarkupfalse%
\ {\isacharparenleft}{\kern0pt}typecheck{\isacharunderscore}{\kern0pt}cfuncs{\isacharcomma}{\kern0pt}\ smt\ comp{\isacharunderscore}{\kern0pt}associative{\isadigit{2}}\ h{\isacharunderscore}{\kern0pt}def\ id{\isacharunderscore}{\kern0pt}right{\isacharunderscore}{\kern0pt}unit{\isadigit{2}}\ id{\isacharunderscore}{\kern0pt}type\ terminal{\isacharunderscore}{\kern0pt}func{\isacharunderscore}{\kern0pt}comp\ terminal{\isacharunderscore}{\kern0pt}func{\isacharunderscore}{\kern0pt}type\ terminal{\isacharunderscore}{\kern0pt}func{\isacharunderscore}{\kern0pt}unique{\isacharparenright}{\kern0pt}\isanewline
\ \ \ \ \ \ \isacommand{also}\isamarkupfalse%
\ \isacommand{have}\isamarkupfalse%
\ {\isachardoublequoteopen}{\isachardot}{\kern0pt}{\isachardot}{\kern0pt}{\isachardot}{\kern0pt}\ {\isacharequal}{\kern0pt}\ {\isacharparenleft}{\kern0pt}h\ {\isasymcirc}\isactrlsub c\ left{\isacharunderscore}{\kern0pt}coproj\ X\ Y{\isacharparenright}{\kern0pt}\ {\isasymcirc}\isactrlsub c\ x{\isachardoublequoteclose}\isanewline
\ \ \ \ \ \ \ \ \ \ \ \ \ \isacommand{using}\isamarkupfalse%
\ x{\isacharunderscore}{\kern0pt}type\ cfunc{\isacharunderscore}{\kern0pt}type{\isacharunderscore}{\kern0pt}def\ comp{\isacharunderscore}{\kern0pt}associative\ comp{\isacharunderscore}{\kern0pt}type\ false{\isacharunderscore}{\kern0pt}func{\isacharunderscore}{\kern0pt}type\ h{\isacharunderscore}{\kern0pt}def\ terminal{\isacharunderscore}{\kern0pt}func{\isacharunderscore}{\kern0pt}type\ \isacommand{by}\isamarkupfalse%
\ {\isacharparenleft}{\kern0pt}typecheck{\isacharunderscore}{\kern0pt}cfuncs{\isacharcomma}{\kern0pt}\ force{\isacharparenright}{\kern0pt}\isanewline
\ \ \ \ \ \ \isacommand{then}\isamarkupfalse%
\ \isacommand{show}\isamarkupfalse%
\ {\isachardoublequoteopen}{\isacharparenleft}{\kern0pt}{\isacharparenleft}{\kern0pt}eq{\isacharunderscore}{\kern0pt}pred\ {\isacharparenleft}{\kern0pt}X\ {\isasymCoprod}\ Y{\isacharparenright}{\kern0pt}\ {\isasymcirc}\isactrlsub c\ {\isasymlangle}z\ {\isasymcirc}\isactrlsub c\ {\isasymbeta}\isactrlbsub X\ {\isasymCoprod}\ Y\isactrlesub {\isacharcomma}{\kern0pt}id\isactrlsub c\ {\isacharparenleft}{\kern0pt}X\ {\isasymCoprod}\ Y{\isacharparenright}{\kern0pt}{\isasymrangle}{\isacharparenright}{\kern0pt}\ {\isasymcirc}\isactrlsub c\ left{\isacharunderscore}{\kern0pt}coproj\ X\ Y{\isacharparenright}{\kern0pt}\ {\isasymcirc}\isactrlsub c\ x\ \ {\isacharequal}{\kern0pt}\ {\isacharparenleft}{\kern0pt}h\ {\isasymcirc}\isactrlsub c\ left{\isacharunderscore}{\kern0pt}coproj\ X\ Y{\isacharparenright}{\kern0pt}\ {\isasymcirc}\isactrlsub c\ x{\isachardoublequoteclose}\isanewline
\ \ \ \ \ \ \ \ \ \ \ \ \ \isacommand{by}\isamarkupfalse%
\ {\isacharparenleft}{\kern0pt}simp\ add{\isacharcolon}{\kern0pt}\ calculation{\isacharparenright}{\kern0pt}\isanewline
\ \ \ \ \isacommand{qed}\isamarkupfalse%
\isanewline
\ \ \isacommand{qed}\isamarkupfalse%
\isanewline
\isanewline
\ \ \isacommand{have}\isamarkupfalse%
\ fact{\isadigit{2}}{\isacharcolon}{\kern0pt}\ {\isachardoublequoteopen}{\isacharparenleft}{\kern0pt}eq{\isacharunderscore}{\kern0pt}pred\ {\isacharparenleft}{\kern0pt}X\ {\isasymCoprod}\ Y{\isacharparenright}{\kern0pt}\ {\isasymcirc}\isactrlsub c\ {\isasymlangle}z\ {\isasymcirc}\isactrlsub c\ {\isasymbeta}\isactrlbsub X\ {\isasymCoprod}\ Y\isactrlesub {\isacharcomma}{\kern0pt}\ id\ {\isacharparenleft}{\kern0pt}X\ {\isasymCoprod}\ Y{\isacharparenright}{\kern0pt}{\isasymrangle}{\isacharparenright}{\kern0pt}\ {\isasymcirc}\isactrlsub c\ right{\isacharunderscore}{\kern0pt}coproj\ X\ Y\ {\isacharequal}{\kern0pt}\ h\ {\isasymcirc}\isactrlsub c\ right{\isacharunderscore}{\kern0pt}coproj\ X\ Y{\isachardoublequoteclose}\isanewline
\ \ \isacommand{proof}\isamarkupfalse%
{\isacharparenleft}{\kern0pt}rule\ one{\isacharunderscore}{\kern0pt}separator{\isacharbrackleft}{\kern0pt}\isakeyword{where}\ X\ {\isacharequal}{\kern0pt}\ Y{\isacharcomma}{\kern0pt}\ \isakeyword{where}\ Y\ {\isacharequal}{\kern0pt}\ {\isasymOmega}{\isacharbrackright}{\kern0pt}{\isacharparenright}{\kern0pt}\isanewline
\ \ \ \ \isacommand{show}\isamarkupfalse%
\ {\isachardoublequoteopen}{\isacharparenleft}{\kern0pt}eq{\isacharunderscore}{\kern0pt}pred\ {\isacharparenleft}{\kern0pt}X\ {\isasymCoprod}\ Y{\isacharparenright}{\kern0pt}\ {\isasymcirc}\isactrlsub c\ {\isasymlangle}z\ {\isasymcirc}\isactrlsub c\ {\isasymbeta}\isactrlbsub X\ {\isasymCoprod}\ Y\isactrlesub {\isacharcomma}{\kern0pt}id\isactrlsub c\ {\isacharparenleft}{\kern0pt}X\ {\isasymCoprod}\ Y{\isacharparenright}{\kern0pt}{\isasymrangle}{\isacharparenright}{\kern0pt}\ {\isasymcirc}\isactrlsub c\ right{\isacharunderscore}{\kern0pt}coproj\ X\ Y\ {\isacharcolon}{\kern0pt}\ Y\ {\isasymrightarrow}\ {\isasymOmega}{\isachardoublequoteclose}\isanewline
\ \ \ \ \ \ \ \isacommand{by}\isamarkupfalse%
\ {\isacharparenleft}{\kern0pt}meson\ assms\ cfunc{\isacharunderscore}{\kern0pt}prod{\isacharunderscore}{\kern0pt}type\ comp{\isacharunderscore}{\kern0pt}type\ eq{\isacharunderscore}{\kern0pt}pred{\isacharunderscore}{\kern0pt}type\ id{\isacharunderscore}{\kern0pt}type\ right{\isacharunderscore}{\kern0pt}proj{\isacharunderscore}{\kern0pt}type\ terminal{\isacharunderscore}{\kern0pt}func{\isacharunderscore}{\kern0pt}type{\isacharparenright}{\kern0pt}\isanewline
\ \ \ \ \isacommand{show}\isamarkupfalse%
\ {\isachardoublequoteopen}h\ {\isasymcirc}\isactrlsub c\ right{\isacharunderscore}{\kern0pt}coproj\ X\ Y\ {\isacharcolon}{\kern0pt}\ Y\ {\isasymrightarrow}\ {\isasymOmega}{\isachardoublequoteclose}\isanewline
\ \ \ \ \ \ \ \isacommand{using}\isamarkupfalse%
\ cfunc{\isacharunderscore}{\kern0pt}type{\isacharunderscore}{\kern0pt}def\ codomain{\isacharunderscore}{\kern0pt}comp\ domain{\isacharunderscore}{\kern0pt}comp\ false{\isacharunderscore}{\kern0pt}func{\isacharunderscore}{\kern0pt}type\ h{\isacharunderscore}{\kern0pt}def\ right{\isacharunderscore}{\kern0pt}proj{\isacharunderscore}{\kern0pt}type\ terminal{\isacharunderscore}{\kern0pt}func{\isacharunderscore}{\kern0pt}type\ \isacommand{by}\isamarkupfalse%
\ presburger\isanewline
\ \ \ \ \isacommand{show}\isamarkupfalse%
\ {\isachardoublequoteopen}{\isasymAnd}x{\isachardot}{\kern0pt}\ x\ {\isasymin}\isactrlsub c\ Y\ {\isasymLongrightarrow}\isanewline
\ \ \ \ \ \ \ \ \ \ \ {\isacharparenleft}{\kern0pt}{\isacharparenleft}{\kern0pt}eq{\isacharunderscore}{\kern0pt}pred\ {\isacharparenleft}{\kern0pt}X\ {\isasymCoprod}\ Y{\isacharparenright}{\kern0pt}\ {\isasymcirc}\isactrlsub c\ {\isasymlangle}z\ {\isasymcirc}\isactrlsub c\ {\isasymbeta}\isactrlbsub X\ {\isasymCoprod}\ Y\isactrlesub {\isacharcomma}{\kern0pt}id\isactrlsub c\ {\isacharparenleft}{\kern0pt}X\ {\isasymCoprod}\ Y{\isacharparenright}{\kern0pt}{\isasymrangle}{\isacharparenright}{\kern0pt}\ {\isasymcirc}\isactrlsub c\ right{\isacharunderscore}{\kern0pt}coproj\ X\ Y{\isacharparenright}{\kern0pt}\ {\isasymcirc}\isactrlsub c\ x\ {\isacharequal}{\kern0pt}\isanewline
\ \ \ \ \ \ \ \ \ \ \ {\isacharparenleft}{\kern0pt}h\ {\isasymcirc}\isactrlsub c\ right{\isacharunderscore}{\kern0pt}coproj\ X\ Y{\isacharparenright}{\kern0pt}\ {\isasymcirc}\isactrlsub c\ x{\isachardoublequoteclose}\isanewline
\ \ \ \ \isacommand{proof}\isamarkupfalse%
\ {\isacharminus}{\kern0pt}\ \isanewline
\ \ \ \ \ \ \isacommand{fix}\isamarkupfalse%
\ x\isanewline
\ \ \ \ \ \ \isacommand{assume}\isamarkupfalse%
\ x{\isacharunderscore}{\kern0pt}type{\isacharbrackleft}{\kern0pt}type{\isacharunderscore}{\kern0pt}rule{\isacharbrackright}{\kern0pt}{\isacharcolon}{\kern0pt}\ {\isachardoublequoteopen}x\ {\isasymin}\isactrlsub c\ Y{\isachardoublequoteclose}\isanewline
\ \ \ \ \ \ \isacommand{have}\isamarkupfalse%
\ {\isachardoublequoteopen}{\isacharparenleft}{\kern0pt}{\isacharparenleft}{\kern0pt}eq{\isacharunderscore}{\kern0pt}pred\ {\isacharparenleft}{\kern0pt}X\ {\isasymCoprod}\ Y{\isacharparenright}{\kern0pt}\ {\isasymcirc}\isactrlsub c\ {\isasymlangle}z\ {\isasymcirc}\isactrlsub c\ {\isasymbeta}\isactrlbsub X\ {\isasymCoprod}\ Y\isactrlesub {\isacharcomma}{\kern0pt}id\isactrlsub c\ {\isacharparenleft}{\kern0pt}X\ {\isasymCoprod}\ Y{\isacharparenright}{\kern0pt}{\isasymrangle}{\isacharparenright}{\kern0pt}\ {\isasymcirc}\isactrlsub c\ right{\isacharunderscore}{\kern0pt}coproj\ X\ Y{\isacharparenright}{\kern0pt}\ {\isasymcirc}\isactrlsub c\ x\ {\isacharequal}{\kern0pt}\ {\isasymf}{\isachardoublequoteclose}\isanewline
\ \ \ \ \ \ \ \ \isacommand{by}\isamarkupfalse%
\ {\isacharparenleft}{\kern0pt}typecheck{\isacharunderscore}{\kern0pt}cfuncs{\isacharcomma}{\kern0pt}\ smt\ {\isacharparenleft}{\kern0pt}verit{\isacharparenright}{\kern0pt}\ assms\ cfunc{\isacharunderscore}{\kern0pt}type{\isacharunderscore}{\kern0pt}def\ eq{\isacharunderscore}{\kern0pt}pred{\isacharunderscore}{\kern0pt}false{\isacharunderscore}{\kern0pt}extract{\isacharunderscore}{\kern0pt}right\ comp{\isacharunderscore}{\kern0pt}associative\ comp{\isacharunderscore}{\kern0pt}type\ not{\isacharunderscore}{\kern0pt}in{\isacharunderscore}{\kern0pt}right{\isacharunderscore}{\kern0pt}image{\isacharparenright}{\kern0pt}\isanewline
\ \ \ \ \ \ \isacommand{also}\isamarkupfalse%
\ \isacommand{have}\isamarkupfalse%
\ {\isachardoublequoteopen}{\isachardot}{\kern0pt}{\isachardot}{\kern0pt}{\isachardot}{\kern0pt}\ {\isacharequal}{\kern0pt}\ {\isacharparenleft}{\kern0pt}h\ {\isasymcirc}\isactrlsub c\ right{\isacharunderscore}{\kern0pt}coproj\ X\ Y{\isacharparenright}{\kern0pt}\ {\isasymcirc}\isactrlsub c\ x{\isachardoublequoteclose}\isanewline
\ \ \ \ \ \ \ \ \isacommand{by}\isamarkupfalse%
\ {\isacharparenleft}{\kern0pt}etcs{\isacharunderscore}{\kern0pt}assocr{\isacharcomma}{\kern0pt}typecheck{\isacharunderscore}{\kern0pt}cfuncs{\isacharcomma}{\kern0pt}\ metis\ cfunc{\isacharunderscore}{\kern0pt}type{\isacharunderscore}{\kern0pt}def\ comp{\isacharunderscore}{\kern0pt}associative\ h{\isacharunderscore}{\kern0pt}def\ id{\isacharunderscore}{\kern0pt}right{\isacharunderscore}{\kern0pt}unit{\isadigit{2}}\ terminal{\isacharunderscore}{\kern0pt}func{\isacharunderscore}{\kern0pt}comp{\isacharunderscore}{\kern0pt}elem\ terminal{\isacharunderscore}{\kern0pt}func{\isacharunderscore}{\kern0pt}type{\isacharparenright}{\kern0pt}\isanewline
\ \ \ \ \ \ \isacommand{then}\isamarkupfalse%
\ \isacommand{show}\isamarkupfalse%
\ {\isachardoublequoteopen}{\isacharparenleft}{\kern0pt}{\isacharparenleft}{\kern0pt}eq{\isacharunderscore}{\kern0pt}pred\ {\isacharparenleft}{\kern0pt}X\ {\isasymCoprod}\ Y{\isacharparenright}{\kern0pt}\ {\isasymcirc}\isactrlsub c\ {\isasymlangle}z\ {\isasymcirc}\isactrlsub c\ {\isasymbeta}\isactrlbsub X\ {\isasymCoprod}\ Y\isactrlesub {\isacharcomma}{\kern0pt}id\isactrlsub c\ {\isacharparenleft}{\kern0pt}X\ {\isasymCoprod}\ Y{\isacharparenright}{\kern0pt}{\isasymrangle}{\isacharparenright}{\kern0pt}\ {\isasymcirc}\isactrlsub c\ right{\isacharunderscore}{\kern0pt}coproj\ X\ Y{\isacharparenright}{\kern0pt}\ {\isasymcirc}\isactrlsub c\ \ x\ {\isacharequal}{\kern0pt}\ {\isacharparenleft}{\kern0pt}h\ {\isasymcirc}\isactrlsub c\ right{\isacharunderscore}{\kern0pt}coproj\ X\ Y{\isacharparenright}{\kern0pt}\ {\isasymcirc}\isactrlsub c\ x{\isachardoublequoteclose}\isanewline
\ \ \ \ \ \ \ \ \ \isacommand{by}\isamarkupfalse%
\ {\isacharparenleft}{\kern0pt}simp\ add{\isacharcolon}{\kern0pt}\ calculation{\isacharparenright}{\kern0pt}\isanewline
\ \ \ \ \isacommand{qed}\isamarkupfalse%
\isanewline
\ \ \isacommand{qed}\isamarkupfalse%
\isanewline
\ \ \isacommand{have}\isamarkupfalse%
\ indicator{\isacharunderscore}{\kern0pt}is{\isacharunderscore}{\kern0pt}false{\isacharcolon}{\kern0pt}{\isachardoublequoteopen}eq{\isacharunderscore}{\kern0pt}pred\ {\isacharparenleft}{\kern0pt}X\ {\isasymCoprod}\ Y{\isacharparenright}{\kern0pt}\ {\isasymcirc}\isactrlsub c\ {\isasymlangle}z\ {\isasymcirc}\isactrlsub c\ {\isasymbeta}\isactrlbsub X\ {\isasymCoprod}\ Y\isactrlesub {\isacharcomma}{\kern0pt}\ id\ {\isacharparenleft}{\kern0pt}X\ {\isasymCoprod}\ Y{\isacharparenright}{\kern0pt}{\isasymrangle}\ {\isacharequal}{\kern0pt}\ h{\isachardoublequoteclose}\isanewline
\ \ \isacommand{proof}\isamarkupfalse%
{\isacharparenleft}{\kern0pt}rule\ one{\isacharunderscore}{\kern0pt}separator{\isacharbrackleft}{\kern0pt}\isakeyword{where}\ X\ {\isacharequal}{\kern0pt}\ {\isachardoublequoteopen}X\ {\isasymCoprod}\ Y{\isachardoublequoteclose}{\isacharcomma}{\kern0pt}\ \isakeyword{where}\ Y\ {\isacharequal}{\kern0pt}\ {\isasymOmega}{\isacharbrackright}{\kern0pt}{\isacharparenright}{\kern0pt}\isanewline
\ \ \ \ \isacommand{show}\isamarkupfalse%
\ {\isachardoublequoteopen}h\ {\isacharcolon}{\kern0pt}\ X\ {\isasymCoprod}\ Y\ {\isasymrightarrow}\ {\isasymOmega}{\isachardoublequoteclose}\isanewline
\ \ \ \ \ \ \isacommand{by}\isamarkupfalse%
\ typecheck{\isacharunderscore}{\kern0pt}cfuncs\isanewline
\ \ \ \ \isacommand{show}\isamarkupfalse%
\ {\isachardoublequoteopen}eq{\isacharunderscore}{\kern0pt}pred\ {\isacharparenleft}{\kern0pt}X\ {\isasymCoprod}\ Y{\isacharparenright}{\kern0pt}\ {\isasymcirc}\isactrlsub c\ {\isasymlangle}z\ {\isasymcirc}\isactrlsub c\ {\isasymbeta}\isactrlbsub X\ {\isasymCoprod}\ Y\isactrlesub {\isacharcomma}{\kern0pt}id\isactrlsub c\ {\isacharparenleft}{\kern0pt}X\ {\isasymCoprod}\ Y{\isacharparenright}{\kern0pt}{\isasymrangle}\ {\isacharcolon}{\kern0pt}\ X\ {\isasymCoprod}\ Y\ {\isasymrightarrow}\ {\isasymOmega}{\isachardoublequoteclose}\isanewline
\ \ \ \ \ \ \isacommand{using}\isamarkupfalse%
\ assms\ \isacommand{by}\isamarkupfalse%
\ typecheck{\isacharunderscore}{\kern0pt}cfuncs\isanewline
\ \ \ \ \isacommand{then}\isamarkupfalse%
\ \isacommand{show}\isamarkupfalse%
\ {\isachardoublequoteopen}{\isasymAnd}x{\isachardot}{\kern0pt}\ x\ {\isasymin}\isactrlsub c\ X\ {\isasymCoprod}\ Y\ {\isasymLongrightarrow}\ {\isacharparenleft}{\kern0pt}eq{\isacharunderscore}{\kern0pt}pred\ {\isacharparenleft}{\kern0pt}X\ {\isasymCoprod}\ Y{\isacharparenright}{\kern0pt}\ {\isasymcirc}\isactrlsub c\ {\isasymlangle}z\ {\isasymcirc}\isactrlsub c\ {\isasymbeta}\isactrlbsub X\ {\isasymCoprod}\ Y\isactrlesub {\isacharcomma}{\kern0pt}id\isactrlsub c\ {\isacharparenleft}{\kern0pt}X\ {\isasymCoprod}\ Y{\isacharparenright}{\kern0pt}{\isasymrangle}{\isacharparenright}{\kern0pt}\ {\isasymcirc}\isactrlsub c\ x\ {\isacharequal}{\kern0pt}\ h\ {\isasymcirc}\isactrlsub c\ x{\isachardoublequoteclose}\isanewline
\ \ \ \ \ \ \isacommand{by}\isamarkupfalse%
\ {\isacharparenleft}{\kern0pt}typecheck{\isacharunderscore}{\kern0pt}cfuncs{\isacharcomma}{\kern0pt}\ smt\ {\isacharparenleft}{\kern0pt}z{\isadigit{3}}{\isacharparenright}{\kern0pt}\ cfunc{\isacharunderscore}{\kern0pt}coprod{\isacharunderscore}{\kern0pt}comp\ fact{\isadigit{1}}\ fact{\isadigit{2}}\ id{\isacharunderscore}{\kern0pt}coprod\ id{\isacharunderscore}{\kern0pt}right{\isacharunderscore}{\kern0pt}unit{\isadigit{2}}\ left{\isacharunderscore}{\kern0pt}proj{\isacharunderscore}{\kern0pt}type\ right{\isacharunderscore}{\kern0pt}proj{\isacharunderscore}{\kern0pt}type{\isacharparenright}{\kern0pt}\isanewline
\ \ \isacommand{qed}\isamarkupfalse%
\isanewline
\isanewline
\ \ \isacommand{have}\isamarkupfalse%
\ hz{\isacharunderscore}{\kern0pt}gives{\isacharunderscore}{\kern0pt}false{\isacharcolon}{\kern0pt}\ {\isachardoublequoteopen}h\ {\isasymcirc}\isactrlsub c\ z\ {\isacharequal}{\kern0pt}\ {\isasymf}{\isachardoublequoteclose}\isanewline
\ \ \ \ \isacommand{using}\isamarkupfalse%
\ assms\ \isacommand{by}\isamarkupfalse%
\ {\isacharparenleft}{\kern0pt}typecheck{\isacharunderscore}{\kern0pt}cfuncs{\isacharcomma}{\kern0pt}\ smt\ comp{\isacharunderscore}{\kern0pt}associative{\isadigit{2}}\ h{\isacharunderscore}{\kern0pt}def\ id{\isacharunderscore}{\kern0pt}right{\isacharunderscore}{\kern0pt}unit{\isadigit{2}}\ id{\isacharunderscore}{\kern0pt}type\ terminal{\isacharunderscore}{\kern0pt}func{\isacharunderscore}{\kern0pt}comp\ terminal{\isacharunderscore}{\kern0pt}func{\isacharunderscore}{\kern0pt}type\ terminal{\isacharunderscore}{\kern0pt}func{\isacharunderscore}{\kern0pt}unique{\isacharparenright}{\kern0pt}\isanewline
\ \ \isacommand{then}\isamarkupfalse%
\ \isacommand{have}\isamarkupfalse%
\ indicator{\isacharunderscore}{\kern0pt}z{\isacharunderscore}{\kern0pt}gives{\isacharunderscore}{\kern0pt}false{\isacharcolon}{\kern0pt}\ {\isachardoublequoteopen}{\isacharparenleft}{\kern0pt}eq{\isacharunderscore}{\kern0pt}pred\ {\isacharparenleft}{\kern0pt}X\ {\isasymCoprod}\ Y{\isacharparenright}{\kern0pt}\ {\isasymcirc}\isactrlsub c\ {\isasymlangle}z\ {\isasymcirc}\isactrlsub c\ {\isasymbeta}\isactrlbsub X\ {\isasymCoprod}\ Y\isactrlesub {\isacharcomma}{\kern0pt}\ id\ {\isacharparenleft}{\kern0pt}X\ {\isasymCoprod}\ Y{\isacharparenright}{\kern0pt}{\isasymrangle}{\isacharparenright}{\kern0pt}\ {\isasymcirc}\isactrlsub c\ z\ {\isacharequal}{\kern0pt}\ {\isasymf}{\isachardoublequoteclose}\isanewline
\ \ \ \ \isacommand{using}\isamarkupfalse%
\ assms\ indicator{\isacharunderscore}{\kern0pt}is{\isacharunderscore}{\kern0pt}false\ \ \isacommand{by}\isamarkupfalse%
\ {\isacharparenleft}{\kern0pt}typecheck{\isacharunderscore}{\kern0pt}cfuncs{\isacharcomma}{\kern0pt}\ blast{\isacharparenright}{\kern0pt}\isanewline
\ \ \isacommand{then}\isamarkupfalse%
\ \isacommand{have}\isamarkupfalse%
\ indicator{\isacharunderscore}{\kern0pt}z{\isacharunderscore}{\kern0pt}gives{\isacharunderscore}{\kern0pt}true{\isacharcolon}{\kern0pt}\ {\isachardoublequoteopen}{\isacharparenleft}{\kern0pt}eq{\isacharunderscore}{\kern0pt}pred\ {\isacharparenleft}{\kern0pt}X\ {\isasymCoprod}\ Y{\isacharparenright}{\kern0pt}\ {\isasymcirc}\isactrlsub c\ {\isasymlangle}z\ {\isasymcirc}\isactrlsub c\ {\isasymbeta}\isactrlbsub X\ {\isasymCoprod}\ Y\isactrlesub {\isacharcomma}{\kern0pt}\ id\ {\isacharparenleft}{\kern0pt}X\ {\isasymCoprod}\ Y{\isacharparenright}{\kern0pt}{\isasymrangle}{\isacharparenright}{\kern0pt}\ {\isasymcirc}\isactrlsub c\ z\ {\isacharequal}{\kern0pt}\ {\isasymt}{\isachardoublequoteclose}\isanewline
\ \ \ \ \isacommand{using}\isamarkupfalse%
\ assms\ \isacommand{by}\isamarkupfalse%
\ {\isacharparenleft}{\kern0pt}typecheck{\isacharunderscore}{\kern0pt}cfuncs{\isacharcomma}{\kern0pt}\ smt\ {\isacharparenleft}{\kern0pt}verit{\isacharcomma}{\kern0pt}\ del{\isacharunderscore}{\kern0pt}insts{\isacharparenright}{\kern0pt}\ comp{\isacharunderscore}{\kern0pt}associative{\isadigit{2}}\ eq{\isacharunderscore}{\kern0pt}pred{\isacharunderscore}{\kern0pt}true{\isacharunderscore}{\kern0pt}extract{\isacharunderscore}{\kern0pt}right{\isacharparenright}{\kern0pt}\isanewline
\ \ \isacommand{then}\isamarkupfalse%
\ \isacommand{show}\isamarkupfalse%
\ False\isanewline
\ \ \ \ \isacommand{using}\isamarkupfalse%
\ indicator{\isacharunderscore}{\kern0pt}z{\isacharunderscore}{\kern0pt}gives{\isacharunderscore}{\kern0pt}false\ true{\isacharunderscore}{\kern0pt}false{\isacharunderscore}{\kern0pt}distinct\ \isacommand{by}\isamarkupfalse%
\ auto\isanewline
\isacommand{qed}\isamarkupfalse%
%
\endisatagproof
{\isafoldproof}%
%
\isadelimproof
\isanewline
%
\endisadelimproof
\isanewline
\isacommand{lemma}\isamarkupfalse%
\ maps{\isacharunderscore}{\kern0pt}into{\isacharunderscore}{\kern0pt}{\isadigit{1}}u{\isadigit{1}}{\isacharcolon}{\kern0pt}\isanewline
\ \ \isakeyword{assumes}\ x{\isacharunderscore}{\kern0pt}type{\isacharcolon}{\kern0pt}\ \ {\isachardoublequoteopen}x{\isasymin}\isactrlsub c\ {\isacharparenleft}{\kern0pt}one\ {\isasymCoprod}\ one{\isacharparenright}{\kern0pt}{\isachardoublequoteclose}\isanewline
\ \ \isakeyword{shows}\ {\isachardoublequoteopen}{\isacharparenleft}{\kern0pt}x\ {\isacharequal}{\kern0pt}\ left{\isacharunderscore}{\kern0pt}coproj\ one\ one{\isacharparenright}{\kern0pt}\ {\isasymor}\ {\isacharparenleft}{\kern0pt}x\ {\isacharequal}{\kern0pt}\ right{\isacharunderscore}{\kern0pt}coproj\ one\ one{\isacharparenright}{\kern0pt}{\isachardoublequoteclose}\isanewline
%
\isadelimproof
\ \ %
\endisadelimproof
%
\isatagproof
\isacommand{using}\isamarkupfalse%
\ assms\ \isacommand{by}\isamarkupfalse%
\ {\isacharparenleft}{\kern0pt}typecheck{\isacharunderscore}{\kern0pt}cfuncs{\isacharcomma}{\kern0pt}\ metis\ coprojs{\isacharunderscore}{\kern0pt}jointly{\isacharunderscore}{\kern0pt}surj\ terminal{\isacharunderscore}{\kern0pt}func{\isacharunderscore}{\kern0pt}unique{\isacharparenright}{\kern0pt}%
\endisatagproof
{\isafoldproof}%
%
\isadelimproof
\isanewline
%
\endisadelimproof
\isanewline
\isacommand{lemma}\isamarkupfalse%
\ coprod{\isacharunderscore}{\kern0pt}preserves{\isacharunderscore}{\kern0pt}left{\isacharunderscore}{\kern0pt}epi{\isacharcolon}{\kern0pt}\isanewline
\ \ \isakeyword{assumes}\ {\isachardoublequoteopen}f{\isacharcolon}{\kern0pt}\ X\ {\isasymrightarrow}\ Z{\isachardoublequoteclose}\ {\isachardoublequoteopen}g{\isacharcolon}{\kern0pt}\ Y\ {\isasymrightarrow}\ Z{\isachardoublequoteclose}\isanewline
\ \ \isakeyword{assumes}\ {\isachardoublequoteopen}surjective{\isacharparenleft}{\kern0pt}f{\isacharparenright}{\kern0pt}{\isachardoublequoteclose}\isanewline
\ \ \isakeyword{shows}\ {\isachardoublequoteopen}surjective{\isacharparenleft}{\kern0pt}f\ {\isasymamalg}\ g{\isacharparenright}{\kern0pt}{\isachardoublequoteclose}\isanewline
%
\isadelimproof
\ \ %
\endisadelimproof
%
\isatagproof
\isacommand{unfolding}\isamarkupfalse%
\ surjective{\isacharunderscore}{\kern0pt}def\isanewline
\isacommand{proof}\isamarkupfalse%
{\isacharparenleft}{\kern0pt}auto{\isacharparenright}{\kern0pt}\isanewline
\ \ \isacommand{fix}\isamarkupfalse%
\ z\isanewline
\ \ \isacommand{assume}\isamarkupfalse%
\ y{\isacharunderscore}{\kern0pt}type{\isacharbrackleft}{\kern0pt}type{\isacharunderscore}{\kern0pt}rule{\isacharbrackright}{\kern0pt}{\isacharcolon}{\kern0pt}\ {\isachardoublequoteopen}z\ {\isasymin}\isactrlsub c\ codomain\ {\isacharparenleft}{\kern0pt}f\ {\isasymamalg}\ g{\isacharparenright}{\kern0pt}{\isachardoublequoteclose}\isanewline
\ \ \isacommand{then}\isamarkupfalse%
\ \isacommand{obtain}\isamarkupfalse%
\ x\ \isakeyword{where}\ x{\isacharunderscore}{\kern0pt}def{\isacharcolon}{\kern0pt}\ {\isachardoublequoteopen}x\ {\isasymin}\isactrlsub c\ X\ {\isasymand}\ f\ {\isasymcirc}\isactrlsub c\ x\ \ {\isacharequal}{\kern0pt}\ z{\isachardoublequoteclose}\isanewline
\ \ \ \ \isacommand{using}\isamarkupfalse%
\ assms\ cfunc{\isacharunderscore}{\kern0pt}coprod{\isacharunderscore}{\kern0pt}type\ cfunc{\isacharunderscore}{\kern0pt}type{\isacharunderscore}{\kern0pt}def\ cfunc{\isacharunderscore}{\kern0pt}type{\isacharunderscore}{\kern0pt}def\ surjective{\isacharunderscore}{\kern0pt}def\ \isacommand{by}\isamarkupfalse%
\ auto\isanewline
\ \ \isacommand{have}\isamarkupfalse%
\ {\isachardoublequoteopen}{\isacharparenleft}{\kern0pt}f\ {\isasymamalg}\ g{\isacharparenright}{\kern0pt}\ {\isasymcirc}\isactrlsub c\ {\isacharparenleft}{\kern0pt}left{\isacharunderscore}{\kern0pt}coproj\ X\ Y\ {\isasymcirc}\isactrlsub c\ x{\isacharparenright}{\kern0pt}\ {\isacharequal}{\kern0pt}\ z{\isachardoublequoteclose}\isanewline
\ \ \ \ \isacommand{by}\isamarkupfalse%
\ {\isacharparenleft}{\kern0pt}typecheck{\isacharunderscore}{\kern0pt}cfuncs{\isacharcomma}{\kern0pt}\ smt\ assms\ comp{\isacharunderscore}{\kern0pt}associative{\isadigit{2}}\ left{\isacharunderscore}{\kern0pt}coproj{\isacharunderscore}{\kern0pt}cfunc{\isacharunderscore}{\kern0pt}coprod\ x{\isacharunderscore}{\kern0pt}def{\isacharparenright}{\kern0pt}\isanewline
\ \ \isacommand{then}\isamarkupfalse%
\ \isacommand{show}\isamarkupfalse%
\ {\isachardoublequoteopen}{\isasymexists}x{\isachardot}{\kern0pt}\ x\ {\isasymin}\isactrlsub c\ domain{\isacharparenleft}{\kern0pt}f\ {\isasymamalg}\ g{\isacharparenright}{\kern0pt}\ {\isasymand}\ f\ {\isasymamalg}\ g\ {\isasymcirc}\isactrlsub c\ x\ {\isacharequal}{\kern0pt}\ z{\isachardoublequoteclose}\isanewline
\ \ \ \ \isacommand{by}\isamarkupfalse%
\ {\isacharparenleft}{\kern0pt}typecheck{\isacharunderscore}{\kern0pt}cfuncs{\isacharcomma}{\kern0pt}\ metis\ assms{\isacharparenleft}{\kern0pt}{\isadigit{1}}{\isacharcomma}{\kern0pt}{\isadigit{2}}{\isacharparenright}{\kern0pt}\ cfunc{\isacharunderscore}{\kern0pt}type{\isacharunderscore}{\kern0pt}def\ codomain{\isacharunderscore}{\kern0pt}comp\ domain{\isacharunderscore}{\kern0pt}comp\ left{\isacharunderscore}{\kern0pt}proj{\isacharunderscore}{\kern0pt}type\ x{\isacharunderscore}{\kern0pt}def{\isacharparenright}{\kern0pt}\isanewline
\isacommand{qed}\isamarkupfalse%
%
\endisatagproof
{\isafoldproof}%
%
\isadelimproof
\isanewline
%
\endisadelimproof
\isanewline
\isacommand{lemma}\isamarkupfalse%
\ coprod{\isacharunderscore}{\kern0pt}preserves{\isacharunderscore}{\kern0pt}right{\isacharunderscore}{\kern0pt}epi{\isacharcolon}{\kern0pt}\isanewline
\ \ \isakeyword{assumes}\ {\isachardoublequoteopen}f{\isacharcolon}{\kern0pt}\ X\ {\isasymrightarrow}\ Z{\isachardoublequoteclose}\ {\isachardoublequoteopen}g{\isacharcolon}{\kern0pt}\ Y\ {\isasymrightarrow}\ Z{\isachardoublequoteclose}\isanewline
\ \ \isakeyword{assumes}\ {\isachardoublequoteopen}surjective{\isacharparenleft}{\kern0pt}g{\isacharparenright}{\kern0pt}{\isachardoublequoteclose}\isanewline
\ \ \isakeyword{shows}\ {\isachardoublequoteopen}surjective{\isacharparenleft}{\kern0pt}f\ {\isasymamalg}\ g{\isacharparenright}{\kern0pt}{\isachardoublequoteclose}\isanewline
%
\isadelimproof
\ \ %
\endisadelimproof
%
\isatagproof
\isacommand{unfolding}\isamarkupfalse%
\ surjective{\isacharunderscore}{\kern0pt}def\isanewline
\isacommand{proof}\isamarkupfalse%
{\isacharparenleft}{\kern0pt}auto{\isacharparenright}{\kern0pt}\isanewline
\ \ \isacommand{fix}\isamarkupfalse%
\ z\isanewline
\ \ \isacommand{assume}\isamarkupfalse%
\ y{\isacharunderscore}{\kern0pt}type{\isacharcolon}{\kern0pt}\ {\isachardoublequoteopen}z\ {\isasymin}\isactrlsub c\ codomain\ {\isacharparenleft}{\kern0pt}f\ {\isasymamalg}\ g{\isacharparenright}{\kern0pt}{\isachardoublequoteclose}\isanewline
\ \ \isacommand{have}\isamarkupfalse%
\ fug{\isacharunderscore}{\kern0pt}type{\isacharcolon}{\kern0pt}\ {\isachardoublequoteopen}{\isacharparenleft}{\kern0pt}f\ {\isasymamalg}\ g{\isacharparenright}{\kern0pt}\ {\isacharcolon}{\kern0pt}\ {\isacharparenleft}{\kern0pt}X\ {\isasymCoprod}\ Y{\isacharparenright}{\kern0pt}\ {\isasymrightarrow}\ Z{\isachardoublequoteclose}\isanewline
\ \ \ \ \isacommand{by}\isamarkupfalse%
\ {\isacharparenleft}{\kern0pt}typecheck{\isacharunderscore}{\kern0pt}cfuncs{\isacharcomma}{\kern0pt}\ simp\ add{\isacharcolon}{\kern0pt}\ assms{\isacharparenright}{\kern0pt}\isanewline
\ \ \isacommand{then}\isamarkupfalse%
\ \isacommand{have}\isamarkupfalse%
\ y{\isacharunderscore}{\kern0pt}type{\isadigit{2}}{\isacharcolon}{\kern0pt}\ {\isachardoublequoteopen}z\ {\isasymin}\isactrlsub c\ Z{\isachardoublequoteclose}\isanewline
\ \ \ \ \isacommand{using}\isamarkupfalse%
\ cfunc{\isacharunderscore}{\kern0pt}type{\isacharunderscore}{\kern0pt}def\ y{\isacharunderscore}{\kern0pt}type\ \isacommand{by}\isamarkupfalse%
\ auto\isanewline
\ \ \isacommand{then}\isamarkupfalse%
\ \isacommand{have}\isamarkupfalse%
\ {\isachardoublequoteopen}{\isasymexists}\ y{\isachardot}{\kern0pt}\ y\ {\isasymin}\isactrlsub c\ Y\ {\isasymand}\ g\ {\isasymcirc}\isactrlsub c\ y\ \ {\isacharequal}{\kern0pt}\ z{\isachardoublequoteclose}\isanewline
\ \ \ \ \isacommand{using}\isamarkupfalse%
\ assms{\isacharparenleft}{\kern0pt}{\isadigit{2}}{\isacharcomma}{\kern0pt}{\isadigit{3}}{\isacharparenright}{\kern0pt}\ cfunc{\isacharunderscore}{\kern0pt}type{\isacharunderscore}{\kern0pt}def\ surjective{\isacharunderscore}{\kern0pt}def\ \isacommand{by}\isamarkupfalse%
\ auto\isanewline
\ \ \isacommand{then}\isamarkupfalse%
\ \isacommand{obtain}\isamarkupfalse%
\ y\ \isakeyword{where}\ y{\isacharunderscore}{\kern0pt}def{\isacharcolon}{\kern0pt}\ {\isachardoublequoteopen}y\ {\isasymin}\isactrlsub c\ Y\ {\isasymand}\ g\ {\isasymcirc}\isactrlsub c\ y\ \ {\isacharequal}{\kern0pt}\ z{\isachardoublequoteclose}\isanewline
\ \ \ \ \isacommand{by}\isamarkupfalse%
\ blast\isanewline
\ \ \isacommand{have}\isamarkupfalse%
\ coproj{\isacharunderscore}{\kern0pt}x{\isacharunderscore}{\kern0pt}type{\isacharcolon}{\kern0pt}\ {\isachardoublequoteopen}right{\isacharunderscore}{\kern0pt}coproj\ X\ Y\ {\isasymcirc}\isactrlsub c\ y\ \ {\isasymin}\isactrlsub c\ X\ {\isasymCoprod}\ Y{\isachardoublequoteclose}\isanewline
\ \ \ \ \isacommand{using}\isamarkupfalse%
\ comp{\isacharunderscore}{\kern0pt}type\ right{\isacharunderscore}{\kern0pt}proj{\isacharunderscore}{\kern0pt}type\ y{\isacharunderscore}{\kern0pt}def\ \isacommand{by}\isamarkupfalse%
\ blast\isanewline
\ \ \isacommand{have}\isamarkupfalse%
\ {\isachardoublequoteopen}{\isacharparenleft}{\kern0pt}f\ {\isasymamalg}\ g{\isacharparenright}{\kern0pt}\ {\isasymcirc}\isactrlsub c\ {\isacharparenleft}{\kern0pt}right{\isacharunderscore}{\kern0pt}coproj\ X\ Y\ {\isasymcirc}\isactrlsub c\ y{\isacharparenright}{\kern0pt}\ {\isacharequal}{\kern0pt}\ z{\isachardoublequoteclose}\isanewline
\ \ \ \ \isacommand{using}\isamarkupfalse%
\ assms{\isacharparenleft}{\kern0pt}{\isadigit{1}}{\isacharparenright}{\kern0pt}\ assms{\isacharparenleft}{\kern0pt}{\isadigit{2}}{\isacharparenright}{\kern0pt}\ cfunc{\isacharunderscore}{\kern0pt}type{\isacharunderscore}{\kern0pt}def\ comp{\isacharunderscore}{\kern0pt}associative\ fug{\isacharunderscore}{\kern0pt}type\ right{\isacharunderscore}{\kern0pt}coproj{\isacharunderscore}{\kern0pt}cfunc{\isacharunderscore}{\kern0pt}coprod\ right{\isacharunderscore}{\kern0pt}proj{\isacharunderscore}{\kern0pt}type\ y{\isacharunderscore}{\kern0pt}def\ \isacommand{by}\isamarkupfalse%
\ auto\isanewline
\ \ \isacommand{then}\isamarkupfalse%
\ \isacommand{show}\isamarkupfalse%
\ {\isachardoublequoteopen}{\isasymexists}y{\isachardot}{\kern0pt}\ y\ {\isasymin}\isactrlsub c\ domain{\isacharparenleft}{\kern0pt}f\ {\isasymamalg}\ g{\isacharparenright}{\kern0pt}\ {\isasymand}\ f\ {\isasymamalg}\ g\ {\isasymcirc}\isactrlsub c\ y\ {\isacharequal}{\kern0pt}\ z{\isachardoublequoteclose}\isanewline
\ \ \ \ \isacommand{using}\isamarkupfalse%
\ cfunc{\isacharunderscore}{\kern0pt}type{\isacharunderscore}{\kern0pt}def\ coproj{\isacharunderscore}{\kern0pt}x{\isacharunderscore}{\kern0pt}type\ fug{\isacharunderscore}{\kern0pt}type\ \isacommand{by}\isamarkupfalse%
\ auto\isanewline
\isacommand{qed}\isamarkupfalse%
%
\endisatagproof
{\isafoldproof}%
%
\isadelimproof
\isanewline
%
\endisadelimproof
\isanewline
\isacommand{lemma}\isamarkupfalse%
\ coprod{\isacharunderscore}{\kern0pt}eq{\isacharcolon}{\kern0pt}\isanewline
\ \ \isakeyword{assumes}\ {\isachardoublequoteopen}a\ {\isacharcolon}{\kern0pt}\ X\ {\isasymCoprod}\ Y\ {\isasymrightarrow}\ Z{\isachardoublequoteclose}\ {\isachardoublequoteopen}b\ {\isacharcolon}{\kern0pt}\ X\ {\isasymCoprod}\ Y\ {\isasymrightarrow}\ \ Z{\isachardoublequoteclose}\isanewline
\ \ \isakeyword{shows}\ {\isachardoublequoteopen}a\ {\isacharequal}{\kern0pt}\ b\ {\isasymlongleftrightarrow}\ \isanewline
\ \ \ \ {\isacharparenleft}{\kern0pt}a\ {\isasymcirc}\isactrlsub c\ left{\isacharunderscore}{\kern0pt}coproj\ X\ Y\ \ \ {\isacharequal}{\kern0pt}\ b\ {\isasymcirc}\isactrlsub c\ left{\isacharunderscore}{\kern0pt}coproj\ X\ Y\ \isanewline
\ \ \ \ \ \ {\isasymand}\ a\ {\isasymcirc}\isactrlsub c\ right{\isacharunderscore}{\kern0pt}coproj\ X\ Y\ \ {\isacharequal}{\kern0pt}\ b\ {\isasymcirc}\isactrlsub c\ right{\isacharunderscore}{\kern0pt}coproj\ X\ Y{\isacharparenright}{\kern0pt}{\isachardoublequoteclose}\isanewline
%
\isadelimproof
\ \ %
\endisadelimproof
%
\isatagproof
\isacommand{by}\isamarkupfalse%
\ {\isacharparenleft}{\kern0pt}smt\ assms\ cfunc{\isacharunderscore}{\kern0pt}coprod{\isacharunderscore}{\kern0pt}unique\ cfunc{\isacharunderscore}{\kern0pt}type{\isacharunderscore}{\kern0pt}def\ codomain{\isacharunderscore}{\kern0pt}comp\ domain{\isacharunderscore}{\kern0pt}comp\ left{\isacharunderscore}{\kern0pt}proj{\isacharunderscore}{\kern0pt}type\ right{\isacharunderscore}{\kern0pt}proj{\isacharunderscore}{\kern0pt}type{\isacharparenright}{\kern0pt}%
\endisatagproof
{\isafoldproof}%
%
\isadelimproof
\isanewline
%
\endisadelimproof
\isanewline
\isacommand{lemma}\isamarkupfalse%
\ coprod{\isacharunderscore}{\kern0pt}eqI{\isacharcolon}{\kern0pt}\isanewline
\ \ \isakeyword{assumes}\ {\isachardoublequoteopen}a\ {\isacharcolon}{\kern0pt}\ X\ {\isasymCoprod}\ Y\ {\isasymrightarrow}\ Z{\isachardoublequoteclose}\ {\isachardoublequoteopen}b\ {\isacharcolon}{\kern0pt}\ X\ {\isasymCoprod}\ Y\ {\isasymrightarrow}\ Z{\isachardoublequoteclose}\isanewline
\ \ \isakeyword{assumes}\ {\isachardoublequoteopen}{\isacharparenleft}{\kern0pt}a\ {\isasymcirc}\isactrlsub c\ left{\isacharunderscore}{\kern0pt}coproj\ X\ Y\ \ \ {\isacharequal}{\kern0pt}\ b\ {\isasymcirc}\isactrlsub c\ left{\isacharunderscore}{\kern0pt}coproj\ X\ Y\ \isanewline
\ \ \ \ \ \ {\isasymand}\ a\ {\isasymcirc}\isactrlsub c\ right{\isacharunderscore}{\kern0pt}coproj\ X\ Y\ \ {\isacharequal}{\kern0pt}\ b\ {\isasymcirc}\isactrlsub c\ right{\isacharunderscore}{\kern0pt}coproj\ X\ Y{\isacharparenright}{\kern0pt}{\isachardoublequoteclose}\isanewline
\ \ \isakeyword{shows}\ {\isachardoublequoteopen}a\ {\isacharequal}{\kern0pt}\ b{\isachardoublequoteclose}\isanewline
%
\isadelimproof
\ \ %
\endisadelimproof
%
\isatagproof
\isacommand{using}\isamarkupfalse%
\ assms\ coprod{\isacharunderscore}{\kern0pt}eq\ \isacommand{by}\isamarkupfalse%
\ blast%
\endisatagproof
{\isafoldproof}%
%
\isadelimproof
\isanewline
%
\endisadelimproof
\isanewline
\isacommand{lemma}\isamarkupfalse%
\ coprod{\isacharunderscore}{\kern0pt}eq{\isadigit{2}}{\isacharcolon}{\kern0pt}\isanewline
\ \ \isakeyword{assumes}\ {\isachardoublequoteopen}a\ {\isacharcolon}{\kern0pt}\ X\ {\isasymrightarrow}\ Z{\isachardoublequoteclose}\ {\isachardoublequoteopen}b\ {\isacharcolon}{\kern0pt}\ Y\ {\isasymrightarrow}\ Z{\isachardoublequoteclose}\ {\isachardoublequoteopen}c\ {\isacharcolon}{\kern0pt}\ X\ {\isasymrightarrow}\ \ Z{\isachardoublequoteclose}\ {\isachardoublequoteopen}d\ {\isacharcolon}{\kern0pt}\ Y\ {\isasymrightarrow}\ \ Z{\isachardoublequoteclose}\isanewline
\ \ \isakeyword{shows}\ {\isachardoublequoteopen}{\isacharparenleft}{\kern0pt}a\ {\isasymamalg}\ b{\isacharparenright}{\kern0pt}\ {\isacharequal}{\kern0pt}\ {\isacharparenleft}{\kern0pt}c\ {\isasymamalg}\ d{\isacharparenright}{\kern0pt}\ {\isasymlongleftrightarrow}\ {\isacharparenleft}{\kern0pt}a\ {\isacharequal}{\kern0pt}\ c\ {\isasymand}\ b\ {\isacharequal}{\kern0pt}\ d{\isacharparenright}{\kern0pt}{\isachardoublequoteclose}\isanewline
%
\isadelimproof
\ \ %
\endisadelimproof
%
\isatagproof
\isacommand{by}\isamarkupfalse%
\ {\isacharparenleft}{\kern0pt}metis\ assms\ left{\isacharunderscore}{\kern0pt}coproj{\isacharunderscore}{\kern0pt}cfunc{\isacharunderscore}{\kern0pt}coprod\ right{\isacharunderscore}{\kern0pt}coproj{\isacharunderscore}{\kern0pt}cfunc{\isacharunderscore}{\kern0pt}coprod{\isacharparenright}{\kern0pt}%
\endisatagproof
{\isafoldproof}%
%
\isadelimproof
\isanewline
%
\endisadelimproof
\isanewline
\isacommand{lemma}\isamarkupfalse%
\ coprod{\isacharunderscore}{\kern0pt}decomp{\isacharcolon}{\kern0pt}\isanewline
\ \ \isakeyword{assumes}\ {\isachardoublequoteopen}a\ {\isacharcolon}{\kern0pt}\ X\ {\isasymCoprod}\ Y\ {\isasymrightarrow}\ A{\isachardoublequoteclose}\isanewline
\ \ \isakeyword{shows}\ {\isachardoublequoteopen}{\isasymexists}\ x\ y{\isachardot}{\kern0pt}\ a\ {\isacharequal}{\kern0pt}\ {\isacharparenleft}{\kern0pt}x\ {\isasymamalg}\ y{\isacharparenright}{\kern0pt}\ {\isasymand}\ x\ {\isacharcolon}{\kern0pt}\ X\ {\isasymrightarrow}\ A\ {\isasymand}\ y\ {\isacharcolon}{\kern0pt}\ Y\ {\isasymrightarrow}\ A{\isachardoublequoteclose}\isanewline
%
\isadelimproof
%
\endisadelimproof
%
\isatagproof
\isacommand{proof}\isamarkupfalse%
\ {\isacharparenleft}{\kern0pt}rule{\isacharunderscore}{\kern0pt}tac\ x{\isacharequal}{\kern0pt}{\isachardoublequoteopen}a\ {\isasymcirc}\isactrlsub c\ left{\isacharunderscore}{\kern0pt}coproj\ X\ Y{\isachardoublequoteclose}\ \isakeyword{in}\ exI{\isacharcomma}{\kern0pt}\ rule{\isacharunderscore}{\kern0pt}tac\ x{\isacharequal}{\kern0pt}{\isachardoublequoteopen}a\ {\isasymcirc}\isactrlsub c\ right{\isacharunderscore}{\kern0pt}coproj\ X\ Y{\isachardoublequoteclose}\ \isakeyword{in}\ exI{\isacharcomma}{\kern0pt}\ auto{\isacharparenright}{\kern0pt}\isanewline
\ \ \isacommand{show}\isamarkupfalse%
\ {\isachardoublequoteopen}a\ {\isacharequal}{\kern0pt}\ {\isacharparenleft}{\kern0pt}a\ {\isasymcirc}\isactrlsub c\ left{\isacharunderscore}{\kern0pt}coproj\ X\ Y{\isacharparenright}{\kern0pt}\ {\isasymamalg}\ {\isacharparenleft}{\kern0pt}a\ {\isasymcirc}\isactrlsub c\ right{\isacharunderscore}{\kern0pt}coproj\ X\ Y{\isacharparenright}{\kern0pt}{\isachardoublequoteclose}\isanewline
\ \ \ \ \isacommand{using}\isamarkupfalse%
\ assms\ cfunc{\isacharunderscore}{\kern0pt}coprod{\isacharunderscore}{\kern0pt}unique\ cfunc{\isacharunderscore}{\kern0pt}type{\isacharunderscore}{\kern0pt}def\ codomain{\isacharunderscore}{\kern0pt}comp\ domain{\isacharunderscore}{\kern0pt}comp\ left{\isacharunderscore}{\kern0pt}proj{\isacharunderscore}{\kern0pt}type\ right{\isacharunderscore}{\kern0pt}proj{\isacharunderscore}{\kern0pt}type\ \isacommand{by}\isamarkupfalse%
\ auto\isanewline
\ \ \isacommand{show}\isamarkupfalse%
\ {\isachardoublequoteopen}a\ {\isasymcirc}\isactrlsub c\ left{\isacharunderscore}{\kern0pt}coproj\ X\ Y\ {\isacharcolon}{\kern0pt}\ X\ {\isasymrightarrow}\ A{\isachardoublequoteclose}\isanewline
\ \ \ \ \isacommand{by}\isamarkupfalse%
\ {\isacharparenleft}{\kern0pt}meson\ assms\ comp{\isacharunderscore}{\kern0pt}type\ left{\isacharunderscore}{\kern0pt}proj{\isacharunderscore}{\kern0pt}type{\isacharparenright}{\kern0pt}\isanewline
\ \ \isacommand{show}\isamarkupfalse%
\ {\isachardoublequoteopen}a\ {\isasymcirc}\isactrlsub c\ right{\isacharunderscore}{\kern0pt}coproj\ X\ Y\ {\isacharcolon}{\kern0pt}\ Y\ {\isasymrightarrow}\ A{\isachardoublequoteclose}\isanewline
\ \ \ \ \isacommand{by}\isamarkupfalse%
\ {\isacharparenleft}{\kern0pt}meson\ assms\ comp{\isacharunderscore}{\kern0pt}type\ right{\isacharunderscore}{\kern0pt}proj{\isacharunderscore}{\kern0pt}type{\isacharparenright}{\kern0pt}\isanewline
\isacommand{qed}\isamarkupfalse%
%
\endisatagproof
{\isafoldproof}%
%
\isadelimproof
%
\endisadelimproof
%
\begin{isamarkuptext}%
The lemma below corresponds to Proposition 2.4.4 in Halvorson.%
\end{isamarkuptext}\isamarkuptrue%
\isacommand{lemma}\isamarkupfalse%
\ truth{\isacharunderscore}{\kern0pt}value{\isacharunderscore}{\kern0pt}set{\isacharunderscore}{\kern0pt}iso{\isacharunderscore}{\kern0pt}{\isadigit{1}}u{\isadigit{1}}{\isacharcolon}{\kern0pt}\isanewline
\ \ {\isachardoublequoteopen}isomorphism{\isacharparenleft}{\kern0pt}{\isasymt}{\isasymamalg}{\isasymf}{\isacharparenright}{\kern0pt}{\isachardoublequoteclose}\isanewline
%
\isadelimproof
\ \ %
\endisadelimproof
%
\isatagproof
\isacommand{by}\isamarkupfalse%
\ {\isacharparenleft}{\kern0pt}typecheck{\isacharunderscore}{\kern0pt}cfuncs{\isacharcomma}{\kern0pt}\ smt\ {\isacharparenleft}{\kern0pt}verit{\isacharcomma}{\kern0pt}\ best{\isacharparenright}{\kern0pt}\ CollectI\ epi{\isacharunderscore}{\kern0pt}mon{\isacharunderscore}{\kern0pt}is{\isacharunderscore}{\kern0pt}iso\ injective{\isacharunderscore}{\kern0pt}def{\isadigit{2}}\isanewline
\ \ \ \ \ \ injective{\isacharunderscore}{\kern0pt}imp{\isacharunderscore}{\kern0pt}monomorphism\ left{\isacharunderscore}{\kern0pt}coproj{\isacharunderscore}{\kern0pt}cfunc{\isacharunderscore}{\kern0pt}coprod\ left{\isacharunderscore}{\kern0pt}proj{\isacharunderscore}{\kern0pt}type\ maps{\isacharunderscore}{\kern0pt}into{\isacharunderscore}{\kern0pt}{\isadigit{1}}u{\isadigit{1}}\isanewline
\ \ \ \ \ \ right{\isacharunderscore}{\kern0pt}coproj{\isacharunderscore}{\kern0pt}cfunc{\isacharunderscore}{\kern0pt}coprod\ right{\isacharunderscore}{\kern0pt}proj{\isacharunderscore}{\kern0pt}type\ surjective{\isacharunderscore}{\kern0pt}def{\isadigit{2}}\ surjective{\isacharunderscore}{\kern0pt}is{\isacharunderscore}{\kern0pt}epimorphism\ \isanewline
\ \ \ \ \ \ true{\isacharunderscore}{\kern0pt}false{\isacharunderscore}{\kern0pt}distinct\ true{\isacharunderscore}{\kern0pt}false{\isacharunderscore}{\kern0pt}only{\isacharunderscore}{\kern0pt}truth{\isacharunderscore}{\kern0pt}values{\isacharparenright}{\kern0pt}%
\endisatagproof
{\isafoldproof}%
%
\isadelimproof
%
\endisadelimproof
%
\isadelimdocument
%
\endisadelimdocument
%
\isatagdocument
%
\isamarkupsubsubsection{Equality Predicate with Coproduct Properities%
}
\isamarkuptrue%
%
\endisatagdocument
{\isafolddocument}%
%
\isadelimdocument
%
\endisadelimdocument
\isacommand{lemma}\isamarkupfalse%
\ eq{\isacharunderscore}{\kern0pt}pred{\isacharunderscore}{\kern0pt}left{\isacharunderscore}{\kern0pt}coproj{\isacharcolon}{\kern0pt}\isanewline
\ \ \isakeyword{assumes}\ u{\isacharunderscore}{\kern0pt}type{\isacharbrackleft}{\kern0pt}type{\isacharunderscore}{\kern0pt}rule{\isacharbrackright}{\kern0pt}{\isacharcolon}{\kern0pt}\ {\isachardoublequoteopen}u\ {\isasymin}\isactrlsub c\ X\ {\isasymCoprod}\ Y{\isachardoublequoteclose}\ \isakeyword{and}\ x{\isacharunderscore}{\kern0pt}type{\isacharbrackleft}{\kern0pt}type{\isacharunderscore}{\kern0pt}rule{\isacharbrackright}{\kern0pt}{\isacharcolon}{\kern0pt}\ {\isachardoublequoteopen}x\ {\isasymin}\isactrlsub c\ X{\isachardoublequoteclose}\isanewline
\ \ \isakeyword{shows}\ {\isachardoublequoteopen}eq{\isacharunderscore}{\kern0pt}pred\ {\isacharparenleft}{\kern0pt}X\ {\isasymCoprod}\ Y{\isacharparenright}{\kern0pt}\ {\isasymcirc}\isactrlsub c\ {\isasymlangle}u{\isacharcomma}{\kern0pt}\ left{\isacharunderscore}{\kern0pt}coproj\ X\ Y\ {\isasymcirc}\isactrlsub c\ x{\isasymrangle}\ {\isacharequal}{\kern0pt}\ {\isacharparenleft}{\kern0pt}{\isacharparenleft}{\kern0pt}eq{\isacharunderscore}{\kern0pt}pred\ X\ {\isasymcirc}\isactrlsub c\ {\isasymlangle}id\ X{\isacharcomma}{\kern0pt}\ x\ {\isasymcirc}\isactrlsub c\ {\isasymbeta}\isactrlbsub X\isactrlesub {\isasymrangle}{\isacharparenright}{\kern0pt}\ {\isasymamalg}\ {\isacharparenleft}{\kern0pt}{\isasymf}\ {\isasymcirc}\isactrlsub c\ {\isasymbeta}\isactrlbsub Y\isactrlesub {\isacharparenright}{\kern0pt}{\isacharparenright}{\kern0pt}\ {\isasymcirc}\isactrlsub c\ u{\isachardoublequoteclose}\isanewline
%
\isadelimproof
%
\endisadelimproof
%
\isatagproof
\isacommand{proof}\isamarkupfalse%
\ {\isacharparenleft}{\kern0pt}cases\ {\isachardoublequoteopen}eq{\isacharunderscore}{\kern0pt}pred\ {\isacharparenleft}{\kern0pt}X\ {\isasymCoprod}\ Y{\isacharparenright}{\kern0pt}\ {\isasymcirc}\isactrlsub c\ {\isasymlangle}u{\isacharcomma}{\kern0pt}\ left{\isacharunderscore}{\kern0pt}coproj\ X\ Y\ {\isasymcirc}\isactrlsub c\ x{\isasymrangle}{\isacharequal}{\kern0pt}\ {\isasymt}{\isachardoublequoteclose}{\isacharcomma}{\kern0pt}\ auto{\isacharparenright}{\kern0pt}\isanewline
\ \ \isacommand{assume}\isamarkupfalse%
\ {\isachardoublequoteopen}eq{\isacharunderscore}{\kern0pt}pred\ {\isacharparenleft}{\kern0pt}X\ {\isasymCoprod}\ Y{\isacharparenright}{\kern0pt}\ {\isasymcirc}\isactrlsub c\ {\isasymlangle}u{\isacharcomma}{\kern0pt}\ left{\isacharunderscore}{\kern0pt}coproj\ X\ Y\ {\isasymcirc}\isactrlsub c\ x{\isasymrangle}\ {\isacharequal}{\kern0pt}\ {\isasymt}{\isachardoublequoteclose}\isanewline
\ \ \isacommand{then}\isamarkupfalse%
\ \isacommand{have}\isamarkupfalse%
\ u{\isacharunderscore}{\kern0pt}is{\isacharunderscore}{\kern0pt}left{\isacharunderscore}{\kern0pt}coproj{\isacharcolon}{\kern0pt}\ {\isachardoublequoteopen}u\ {\isacharequal}{\kern0pt}\ left{\isacharunderscore}{\kern0pt}coproj\ X\ Y\ {\isasymcirc}\isactrlsub c\ x{\isachardoublequoteclose}\isanewline
\ \ \ \ \isacommand{using}\isamarkupfalse%
\ eq{\isacharunderscore}{\kern0pt}pred{\isacharunderscore}{\kern0pt}iff{\isacharunderscore}{\kern0pt}eq\ \isacommand{by}\isamarkupfalse%
\ {\isacharparenleft}{\kern0pt}typecheck{\isacharunderscore}{\kern0pt}cfuncs{\isacharunderscore}{\kern0pt}prems{\isacharcomma}{\kern0pt}\ presburger{\isacharparenright}{\kern0pt}\isanewline
\ \ \isanewline
\ \ \isacommand{show}\isamarkupfalse%
\ {\isachardoublequoteopen}{\isasymt}\ {\isacharequal}{\kern0pt}\ {\isacharparenleft}{\kern0pt}eq{\isacharunderscore}{\kern0pt}pred\ X\ {\isasymcirc}\isactrlsub c\ {\isasymlangle}id\isactrlsub c\ X{\isacharcomma}{\kern0pt}x\ {\isasymcirc}\isactrlsub c\ {\isasymbeta}\isactrlbsub X\isactrlesub {\isasymrangle}{\isacharparenright}{\kern0pt}\ {\isasymamalg}\ {\isacharparenleft}{\kern0pt}{\isasymf}\ {\isasymcirc}\isactrlsub c\ {\isasymbeta}\isactrlbsub Y\isactrlesub {\isacharparenright}{\kern0pt}\ {\isasymcirc}\isactrlsub c\ u{\isachardoublequoteclose}\isanewline
\ \ \isacommand{proof}\isamarkupfalse%
\ {\isacharminus}{\kern0pt}\isanewline
\ \ \ \ \isacommand{have}\isamarkupfalse%
\ {\isachardoublequoteopen}{\isacharparenleft}{\kern0pt}{\isacharparenleft}{\kern0pt}eq{\isacharunderscore}{\kern0pt}pred\ X\ {\isasymcirc}\isactrlsub c\ {\isasymlangle}id\ X{\isacharcomma}{\kern0pt}\ x\ {\isasymcirc}\isactrlsub c\ {\isasymbeta}\isactrlbsub X\isactrlesub {\isasymrangle}{\isacharparenright}{\kern0pt}\ {\isasymamalg}\ {\isacharparenleft}{\kern0pt}{\isasymf}\ {\isasymcirc}\isactrlsub c\ {\isasymbeta}\isactrlbsub Y\isactrlesub {\isacharparenright}{\kern0pt}{\isacharparenright}{\kern0pt}\ {\isasymcirc}\isactrlsub c\ u\isanewline
\ \ \ \ \ \ \ \ {\isacharequal}{\kern0pt}\ {\isacharparenleft}{\kern0pt}{\isacharparenleft}{\kern0pt}eq{\isacharunderscore}{\kern0pt}pred\ X\ {\isasymcirc}\isactrlsub c\ {\isasymlangle}id\ X{\isacharcomma}{\kern0pt}\ x\ {\isasymcirc}\isactrlsub c\ {\isasymbeta}\isactrlbsub X\isactrlesub {\isasymrangle}{\isacharparenright}{\kern0pt}\ {\isasymamalg}\ {\isacharparenleft}{\kern0pt}{\isasymf}\ {\isasymcirc}\isactrlsub c\ {\isasymbeta}\isactrlbsub Y\isactrlesub {\isacharparenright}{\kern0pt}{\isacharparenright}{\kern0pt}\ {\isasymcirc}\isactrlsub c\ left{\isacharunderscore}{\kern0pt}coproj\ X\ Y\ {\isasymcirc}\isactrlsub c\ x{\isachardoublequoteclose}\isanewline
\ \ \ \ \ \ \isacommand{using}\isamarkupfalse%
\ u{\isacharunderscore}{\kern0pt}is{\isacharunderscore}{\kern0pt}left{\isacharunderscore}{\kern0pt}coproj\ \isacommand{by}\isamarkupfalse%
\ auto\isanewline
\ \ \ \ \isacommand{also}\isamarkupfalse%
\ \isacommand{have}\isamarkupfalse%
\ {\isachardoublequoteopen}{\isachardot}{\kern0pt}{\isachardot}{\kern0pt}{\isachardot}{\kern0pt}\ {\isacharequal}{\kern0pt}\ \ {\isacharparenleft}{\kern0pt}eq{\isacharunderscore}{\kern0pt}pred\ X\ {\isasymcirc}\isactrlsub c\ {\isasymlangle}id\ X{\isacharcomma}{\kern0pt}\ x\ {\isasymcirc}\isactrlsub c\ {\isasymbeta}\isactrlbsub X\isactrlesub {\isasymrangle}{\isacharparenright}{\kern0pt}\ {\isasymcirc}\isactrlsub c\ x\ {\isachardoublequoteclose}\isanewline
\ \ \ \ \ \ \isacommand{by}\isamarkupfalse%
\ {\isacharparenleft}{\kern0pt}typecheck{\isacharunderscore}{\kern0pt}cfuncs{\isacharcomma}{\kern0pt}\ simp\ add{\isacharcolon}{\kern0pt}\ comp{\isacharunderscore}{\kern0pt}associative{\isadigit{2}}\ left{\isacharunderscore}{\kern0pt}coproj{\isacharunderscore}{\kern0pt}cfunc{\isacharunderscore}{\kern0pt}coprod{\isacharparenright}{\kern0pt}\isanewline
\ \ \ \ \isacommand{also}\isamarkupfalse%
\ \isacommand{have}\isamarkupfalse%
\ {\isachardoublequoteopen}{\isachardot}{\kern0pt}{\isachardot}{\kern0pt}{\isachardot}{\kern0pt}\ {\isacharequal}{\kern0pt}\ eq{\isacharunderscore}{\kern0pt}pred\ X\ {\isasymcirc}\isactrlsub c\ {\isasymlangle}x{\isacharcomma}{\kern0pt}\ x{\isasymrangle}{\isachardoublequoteclose}\isanewline
\ \ \ \ \ \ \isacommand{by}\isamarkupfalse%
\ {\isacharparenleft}{\kern0pt}typecheck{\isacharunderscore}{\kern0pt}cfuncs{\isacharcomma}{\kern0pt}\ metis\ cart{\isacharunderscore}{\kern0pt}prod{\isacharunderscore}{\kern0pt}extract{\isacharunderscore}{\kern0pt}left\ cfunc{\isacharunderscore}{\kern0pt}type{\isacharunderscore}{\kern0pt}def\ comp{\isacharunderscore}{\kern0pt}associative{\isacharparenright}{\kern0pt}\isanewline
\ \ \ \ \isacommand{also}\isamarkupfalse%
\ \isacommand{have}\isamarkupfalse%
\ {\isachardoublequoteopen}{\isachardot}{\kern0pt}{\isachardot}{\kern0pt}{\isachardot}{\kern0pt}\ {\isacharequal}{\kern0pt}\ {\isasymt}{\isachardoublequoteclose}\isanewline
\ \ \ \ \ \ \isacommand{using}\isamarkupfalse%
\ eq{\isacharunderscore}{\kern0pt}pred{\isacharunderscore}{\kern0pt}iff{\isacharunderscore}{\kern0pt}eq\ \isacommand{by}\isamarkupfalse%
\ {\isacharparenleft}{\kern0pt}typecheck{\isacharunderscore}{\kern0pt}cfuncs{\isacharcomma}{\kern0pt}\ blast{\isacharparenright}{\kern0pt}\isanewline
\ \ \ \ \isacommand{then}\isamarkupfalse%
\ \isacommand{show}\isamarkupfalse%
\ {\isacharquery}{\kern0pt}thesis\isanewline
\ \ \ \ \ \ \isacommand{by}\isamarkupfalse%
\ {\isacharparenleft}{\kern0pt}simp\ add{\isacharcolon}{\kern0pt}\ calculation{\isacharparenright}{\kern0pt}\isanewline
\ \ \isacommand{qed}\isamarkupfalse%
\isanewline
\isacommand{next}\isamarkupfalse%
\isanewline
\ \ \isacommand{assume}\isamarkupfalse%
\ {\isachardoublequoteopen}eq{\isacharunderscore}{\kern0pt}pred\ {\isacharparenleft}{\kern0pt}X\ {\isasymCoprod}\ Y{\isacharparenright}{\kern0pt}\ {\isasymcirc}\isactrlsub c\ {\isasymlangle}u{\isacharcomma}{\kern0pt}left{\isacharunderscore}{\kern0pt}coproj\ X\ Y\ {\isasymcirc}\isactrlsub c\ x{\isasymrangle}\ {\isasymnoteq}\ {\isasymt}{\isachardoublequoteclose}\isanewline
\ \ \isacommand{then}\isamarkupfalse%
\ \isacommand{have}\isamarkupfalse%
\ eq{\isacharunderscore}{\kern0pt}pred{\isacharunderscore}{\kern0pt}false{\isacharcolon}{\kern0pt}\ {\isachardoublequoteopen}eq{\isacharunderscore}{\kern0pt}pred\ {\isacharparenleft}{\kern0pt}X\ {\isasymCoprod}\ Y{\isacharparenright}{\kern0pt}\ {\isasymcirc}\isactrlsub c\ {\isasymlangle}u{\isacharcomma}{\kern0pt}left{\isacharunderscore}{\kern0pt}coproj\ X\ Y\ {\isasymcirc}\isactrlsub c\ x{\isasymrangle}\ {\isacharequal}{\kern0pt}\ {\isasymf}{\isachardoublequoteclose}\isanewline
\ \ \ \ \isacommand{using}\isamarkupfalse%
\ true{\isacharunderscore}{\kern0pt}false{\isacharunderscore}{\kern0pt}only{\isacharunderscore}{\kern0pt}truth{\isacharunderscore}{\kern0pt}values\ \isacommand{by}\isamarkupfalse%
\ {\isacharparenleft}{\kern0pt}typecheck{\isacharunderscore}{\kern0pt}cfuncs{\isacharcomma}{\kern0pt}\ blast{\isacharparenright}{\kern0pt}\isanewline
\ \ \isacommand{then}\isamarkupfalse%
\ \isacommand{have}\isamarkupfalse%
\ u{\isacharunderscore}{\kern0pt}not{\isacharunderscore}{\kern0pt}left{\isacharunderscore}{\kern0pt}coproj{\isacharunderscore}{\kern0pt}x{\isacharcolon}{\kern0pt}\ {\isachardoublequoteopen}u\ \ {\isasymnoteq}\ left{\isacharunderscore}{\kern0pt}coproj\ X\ Y\ {\isasymcirc}\isactrlsub c\ x{\isachardoublequoteclose}\isanewline
\ \ \ \ \isacommand{using}\isamarkupfalse%
\ eq{\isacharunderscore}{\kern0pt}pred{\isacharunderscore}{\kern0pt}iff{\isacharunderscore}{\kern0pt}eq{\isacharunderscore}{\kern0pt}conv\ \isacommand{by}\isamarkupfalse%
\ {\isacharparenleft}{\kern0pt}typecheck{\isacharunderscore}{\kern0pt}cfuncs{\isacharunderscore}{\kern0pt}prems{\isacharcomma}{\kern0pt}\ presburger{\isacharparenright}{\kern0pt}\isanewline
\ \ \isacommand{show}\isamarkupfalse%
\ {\isachardoublequoteopen}eq{\isacharunderscore}{\kern0pt}pred\ {\isacharparenleft}{\kern0pt}X\ {\isasymCoprod}\ Y{\isacharparenright}{\kern0pt}\ {\isasymcirc}\isactrlsub c\ {\isasymlangle}u{\isacharcomma}{\kern0pt}left{\isacharunderscore}{\kern0pt}coproj\ X\ Y\ {\isasymcirc}\isactrlsub c\ x{\isasymrangle}\ {\isacharequal}{\kern0pt}\ {\isacharparenleft}{\kern0pt}eq{\isacharunderscore}{\kern0pt}pred\ X\ {\isasymcirc}\isactrlsub c\ {\isasymlangle}id\isactrlsub c\ X{\isacharcomma}{\kern0pt}x\ {\isasymcirc}\isactrlsub c\ {\isasymbeta}\isactrlbsub X\isactrlesub {\isasymrangle}{\isacharparenright}{\kern0pt}\ {\isasymamalg}\ {\isacharparenleft}{\kern0pt}{\isasymf}\ {\isasymcirc}\isactrlsub c\ {\isasymbeta}\isactrlbsub Y\isactrlesub {\isacharparenright}{\kern0pt}\ {\isasymcirc}\isactrlsub c\ u{\isachardoublequoteclose}\isanewline
\ \ \isacommand{proof}\isamarkupfalse%
\ {\isacharparenleft}{\kern0pt}insert\ eq{\isacharunderscore}{\kern0pt}pred{\isacharunderscore}{\kern0pt}false{\isacharcomma}{\kern0pt}\ cases\ {\isachardoublequoteopen}{\isasymexists}\ g{\isachardot}{\kern0pt}\ g\ {\isacharcolon}{\kern0pt}\ one\ {\isasymrightarrow}\ X\ {\isasymand}\ u\ {\isacharequal}{\kern0pt}\ left{\isacharunderscore}{\kern0pt}coproj\ X\ Y\ {\isasymcirc}\isactrlsub c\ g{\isachardoublequoteclose}{\isacharcomma}{\kern0pt}\ auto{\isacharparenright}{\kern0pt}\ \ \isanewline
\ \ \ \ \isacommand{fix}\isamarkupfalse%
\ g\isanewline
\ \ \ \ \isacommand{assume}\isamarkupfalse%
\ g{\isacharunderscore}{\kern0pt}type{\isacharbrackleft}{\kern0pt}type{\isacharunderscore}{\kern0pt}rule{\isacharbrackright}{\kern0pt}{\isacharcolon}{\kern0pt}\ {\isachardoublequoteopen}g\ {\isasymin}\isactrlsub c\ X{\isachardoublequoteclose}\isanewline
\ \ \ \ \isacommand{assume}\isamarkupfalse%
\ u{\isacharunderscore}{\kern0pt}right{\isacharunderscore}{\kern0pt}coproj{\isacharcolon}{\kern0pt}\ {\isachardoublequoteopen}u\ {\isacharequal}{\kern0pt}\ left{\isacharunderscore}{\kern0pt}coproj\ X\ Y\ {\isasymcirc}\isactrlsub c\ g{\isachardoublequoteclose}\isanewline
\ \ \ \ \isacommand{then}\isamarkupfalse%
\ \isacommand{have}\isamarkupfalse%
\ x{\isacharunderscore}{\kern0pt}not{\isacharunderscore}{\kern0pt}g{\isacharcolon}{\kern0pt}\ {\isachardoublequoteopen}x\ {\isasymnoteq}\ g{\isachardoublequoteclose}\isanewline
\ \ \ \ \ \ \isacommand{using}\isamarkupfalse%
\ u{\isacharunderscore}{\kern0pt}not{\isacharunderscore}{\kern0pt}left{\isacharunderscore}{\kern0pt}coproj{\isacharunderscore}{\kern0pt}x\ \isacommand{by}\isamarkupfalse%
\ auto\isanewline
\ \ \ \ \isacommand{show}\isamarkupfalse%
\ {\isachardoublequoteopen}{\isasymf}\ {\isacharequal}{\kern0pt}\ {\isacharparenleft}{\kern0pt}eq{\isacharunderscore}{\kern0pt}pred\ X\ {\isasymcirc}\isactrlsub c\ {\isasymlangle}id\isactrlsub c\ X{\isacharcomma}{\kern0pt}x\ {\isasymcirc}\isactrlsub c\ {\isasymbeta}\isactrlbsub X\isactrlesub {\isasymrangle}{\isacharparenright}{\kern0pt}\ {\isasymamalg}\ {\isacharparenleft}{\kern0pt}{\isasymf}\ {\isasymcirc}\isactrlsub c\ {\isasymbeta}\isactrlbsub Y\isactrlesub {\isacharparenright}{\kern0pt}\ {\isasymcirc}\isactrlsub c\ left{\isacharunderscore}{\kern0pt}coproj\ X\ Y\ {\isasymcirc}\isactrlsub c\ g{\isachardoublequoteclose}\isanewline
\ \ \ \ \isacommand{proof}\isamarkupfalse%
\ {\isacharminus}{\kern0pt}\isanewline
\ \ \ \ \ \ \isacommand{have}\isamarkupfalse%
\ {\isachardoublequoteopen}{\isacharparenleft}{\kern0pt}eq{\isacharunderscore}{\kern0pt}pred\ X\ {\isasymcirc}\isactrlsub c\ {\isasymlangle}id\isactrlsub c\ X{\isacharcomma}{\kern0pt}x\ {\isasymcirc}\isactrlsub c\ {\isasymbeta}\isactrlbsub X\isactrlesub {\isasymrangle}{\isacharparenright}{\kern0pt}\ {\isasymamalg}\ {\isacharparenleft}{\kern0pt}{\isasymf}\ {\isasymcirc}\isactrlsub c\ {\isasymbeta}\isactrlbsub Y\isactrlesub {\isacharparenright}{\kern0pt}\ {\isasymcirc}\isactrlsub c\ left{\isacharunderscore}{\kern0pt}coproj\ X\ Y\ {\isasymcirc}\isactrlsub c\ g\isanewline
\ \ \ \ \ \ \ \ \ \ {\isacharequal}{\kern0pt}\ {\isacharparenleft}{\kern0pt}eq{\isacharunderscore}{\kern0pt}pred\ X\ {\isasymcirc}\isactrlsub c\ {\isasymlangle}id\isactrlsub c\ X{\isacharcomma}{\kern0pt}x\ {\isasymcirc}\isactrlsub c\ {\isasymbeta}\isactrlbsub X\isactrlesub {\isasymrangle}{\isacharparenright}{\kern0pt}\ {\isasymcirc}\isactrlsub c\ g{\isachardoublequoteclose}\isanewline
\ \ \ \ \ \ \ \ \isacommand{using}\isamarkupfalse%
\ comp{\isacharunderscore}{\kern0pt}associative{\isadigit{2}}\ left{\isacharunderscore}{\kern0pt}coproj{\isacharunderscore}{\kern0pt}cfunc{\isacharunderscore}{\kern0pt}coprod\ \isacommand{by}\isamarkupfalse%
\ {\isacharparenleft}{\kern0pt}typecheck{\isacharunderscore}{\kern0pt}cfuncs{\isacharcomma}{\kern0pt}\ force{\isacharparenright}{\kern0pt}\isanewline
\ \ \ \ \ \ \isacommand{also}\isamarkupfalse%
\ \isacommand{have}\isamarkupfalse%
\ {\isachardoublequoteopen}{\isachardot}{\kern0pt}{\isachardot}{\kern0pt}{\isachardot}{\kern0pt}\ {\isacharequal}{\kern0pt}\ eq{\isacharunderscore}{\kern0pt}pred\ X\ {\isasymcirc}\isactrlsub c\ {\isasymlangle}g{\isacharcomma}{\kern0pt}x{\isasymrangle}{\isachardoublequoteclose}\isanewline
\ \ \ \ \ \ \ \ \isacommand{by}\isamarkupfalse%
\ {\isacharparenleft}{\kern0pt}typecheck{\isacharunderscore}{\kern0pt}cfuncs{\isacharcomma}{\kern0pt}\ simp\ add{\isacharcolon}{\kern0pt}\ cart{\isacharunderscore}{\kern0pt}prod{\isacharunderscore}{\kern0pt}extract{\isacharunderscore}{\kern0pt}left\ comp{\isacharunderscore}{\kern0pt}associative{\isadigit{2}}{\isacharparenright}{\kern0pt}\isanewline
\ \ \ \ \ \ \isacommand{also}\isamarkupfalse%
\ \isacommand{have}\isamarkupfalse%
\ {\isachardoublequoteopen}{\isachardot}{\kern0pt}{\isachardot}{\kern0pt}{\isachardot}{\kern0pt}\ {\isacharequal}{\kern0pt}\ {\isasymf}{\isachardoublequoteclose}\isanewline
\ \ \ \ \ \ \ \ \isacommand{using}\isamarkupfalse%
\ eq{\isacharunderscore}{\kern0pt}pred{\isacharunderscore}{\kern0pt}iff{\isacharunderscore}{\kern0pt}eq{\isacharunderscore}{\kern0pt}conv\ x{\isacharunderscore}{\kern0pt}not{\isacharunderscore}{\kern0pt}g\ \isacommand{by}\isamarkupfalse%
\ {\isacharparenleft}{\kern0pt}typecheck{\isacharunderscore}{\kern0pt}cfuncs{\isacharcomma}{\kern0pt}\ blast{\isacharparenright}{\kern0pt}\isanewline
\ \ \ \ \ \ \isacommand{then}\isamarkupfalse%
\ \isacommand{show}\isamarkupfalse%
\ {\isacharquery}{\kern0pt}thesis\isanewline
\ \ \ \ \ \ \ \ \isacommand{by}\isamarkupfalse%
\ {\isacharparenleft}{\kern0pt}simp\ add{\isacharcolon}{\kern0pt}\ calculation{\isacharparenright}{\kern0pt}\isanewline
\ \ \ \ \isacommand{qed}\isamarkupfalse%
\isanewline
\ \ \isacommand{next}\isamarkupfalse%
\isanewline
\ \ \ \ \isacommand{assume}\isamarkupfalse%
\ {\isachardoublequoteopen}{\isasymforall}g{\isachardot}{\kern0pt}\ g\ {\isasymin}\isactrlsub c\ X\ {\isasymlongrightarrow}\ u\ {\isasymnoteq}\ left{\isacharunderscore}{\kern0pt}coproj\ X\ Y\ {\isasymcirc}\isactrlsub c\ g{\isachardoublequoteclose}\isanewline
\ \ \ \ \isacommand{then}\isamarkupfalse%
\ \isacommand{obtain}\isamarkupfalse%
\ g\ \isakeyword{where}\ g{\isacharunderscore}{\kern0pt}type{\isacharbrackleft}{\kern0pt}type{\isacharunderscore}{\kern0pt}rule{\isacharbrackright}{\kern0pt}{\isacharcolon}{\kern0pt}\ {\isachardoublequoteopen}g\ {\isasymin}\isactrlsub c\ Y{\isachardoublequoteclose}\ \isakeyword{and}\ u{\isacharunderscore}{\kern0pt}right{\isacharunderscore}{\kern0pt}coproj{\isacharcolon}{\kern0pt}\ {\isachardoublequoteopen}u\ {\isacharequal}{\kern0pt}\ right{\isacharunderscore}{\kern0pt}coproj\ X\ Y\ {\isasymcirc}\isactrlsub c\ g{\isachardoublequoteclose}\isanewline
\ \ \ \ \ \ \isacommand{by}\isamarkupfalse%
\ {\isacharparenleft}{\kern0pt}meson\ coprojs{\isacharunderscore}{\kern0pt}jointly{\isacharunderscore}{\kern0pt}surj\ u{\isacharunderscore}{\kern0pt}type{\isacharparenright}{\kern0pt}\isanewline
\isanewline
\ \ \ \ \isacommand{show}\isamarkupfalse%
\ {\isachardoublequoteopen}{\isasymf}\ {\isacharequal}{\kern0pt}\ {\isacharparenleft}{\kern0pt}eq{\isacharunderscore}{\kern0pt}pred\ X\ {\isasymcirc}\isactrlsub c\ {\isasymlangle}id\isactrlsub c\ X{\isacharcomma}{\kern0pt}x\ {\isasymcirc}\isactrlsub c\ {\isasymbeta}\isactrlbsub X\isactrlesub {\isasymrangle}{\isacharparenright}{\kern0pt}\ {\isasymamalg}\ {\isacharparenleft}{\kern0pt}{\isasymf}\ {\isasymcirc}\isactrlsub c\ {\isasymbeta}\isactrlbsub Y\isactrlesub {\isacharparenright}{\kern0pt}\ {\isasymcirc}\isactrlsub c\ u{\isachardoublequoteclose}\ \ \isanewline
\ \ \ \ \isacommand{proof}\isamarkupfalse%
\ {\isacharminus}{\kern0pt}\isanewline
\ \ \ \ \ \ \isacommand{have}\isamarkupfalse%
\ {\isachardoublequoteopen}{\isacharparenleft}{\kern0pt}eq{\isacharunderscore}{\kern0pt}pred\ X\ {\isasymcirc}\isactrlsub c\ {\isasymlangle}id\isactrlsub c\ X{\isacharcomma}{\kern0pt}x\ {\isasymcirc}\isactrlsub c\ {\isasymbeta}\isactrlbsub X\isactrlesub {\isasymrangle}{\isacharparenright}{\kern0pt}\ {\isasymamalg}\ {\isacharparenleft}{\kern0pt}{\isasymf}\ {\isasymcirc}\isactrlsub c\ {\isasymbeta}\isactrlbsub Y\isactrlesub {\isacharparenright}{\kern0pt}\ {\isasymcirc}\isactrlsub c\ u\isanewline
\ \ \ \ \ \ \ \ \ \ {\isacharequal}{\kern0pt}\ {\isacharparenleft}{\kern0pt}eq{\isacharunderscore}{\kern0pt}pred\ X\ {\isasymcirc}\isactrlsub c\ {\isasymlangle}id\isactrlsub c\ X{\isacharcomma}{\kern0pt}x\ {\isasymcirc}\isactrlsub c\ {\isasymbeta}\isactrlbsub X\isactrlesub {\isasymrangle}{\isacharparenright}{\kern0pt}\ {\isasymamalg}\ {\isacharparenleft}{\kern0pt}{\isasymf}\ {\isasymcirc}\isactrlsub c\ {\isasymbeta}\isactrlbsub Y\isactrlesub {\isacharparenright}{\kern0pt}\ \ {\isasymcirc}\isactrlsub c\ right{\isacharunderscore}{\kern0pt}coproj\ X\ Y\ {\isasymcirc}\isactrlsub c\ g{\isachardoublequoteclose}\isanewline
\ \ \ \ \ \ \ \ \isacommand{using}\isamarkupfalse%
\ u{\isacharunderscore}{\kern0pt}right{\isacharunderscore}{\kern0pt}coproj\ \isacommand{by}\isamarkupfalse%
\ auto\isanewline
\ \ \ \ \ \ \isacommand{also}\isamarkupfalse%
\ \isacommand{have}\isamarkupfalse%
\ {\isachardoublequoteopen}{\isachardot}{\kern0pt}{\isachardot}{\kern0pt}{\isachardot}{\kern0pt}\ {\isacharequal}{\kern0pt}\ {\isacharparenleft}{\kern0pt}{\isasymf}\ {\isasymcirc}\isactrlsub c\ {\isasymbeta}\isactrlbsub Y\isactrlesub {\isacharparenright}{\kern0pt}\ {\isasymcirc}\isactrlsub c\ g{\isachardoublequoteclose}\isanewline
\ \ \ \ \ \ \ \ \isacommand{by}\isamarkupfalse%
\ {\isacharparenleft}{\kern0pt}typecheck{\isacharunderscore}{\kern0pt}cfuncs{\isacharcomma}{\kern0pt}\ simp\ add{\isacharcolon}{\kern0pt}\ comp{\isacharunderscore}{\kern0pt}associative{\isadigit{2}}\ right{\isacharunderscore}{\kern0pt}coproj{\isacharunderscore}{\kern0pt}cfunc{\isacharunderscore}{\kern0pt}coprod{\isacharparenright}{\kern0pt}\isanewline
\ \ \ \ \ \ \isacommand{also}\isamarkupfalse%
\ \isacommand{have}\isamarkupfalse%
\ {\isachardoublequoteopen}{\isachardot}{\kern0pt}{\isachardot}{\kern0pt}{\isachardot}{\kern0pt}\ {\isacharequal}{\kern0pt}\ {\isasymf}{\isachardoublequoteclose}\isanewline
\ \ \ \ \ \ \ \ \isacommand{by}\isamarkupfalse%
\ {\isacharparenleft}{\kern0pt}typecheck{\isacharunderscore}{\kern0pt}cfuncs{\isacharcomma}{\kern0pt}\ smt\ {\isacharparenleft}{\kern0pt}z{\isadigit{3}}{\isacharparenright}{\kern0pt}\ comp{\isacharunderscore}{\kern0pt}associative{\isadigit{2}}\ id{\isacharunderscore}{\kern0pt}right{\isacharunderscore}{\kern0pt}unit{\isadigit{2}}\ id{\isacharunderscore}{\kern0pt}type\ terminal{\isacharunderscore}{\kern0pt}func{\isacharunderscore}{\kern0pt}comp\ terminal{\isacharunderscore}{\kern0pt}func{\isacharunderscore}{\kern0pt}unique{\isacharparenright}{\kern0pt}\isanewline
\ \ \ \ \ \ \isacommand{then}\isamarkupfalse%
\ \isacommand{show}\isamarkupfalse%
\ {\isacharquery}{\kern0pt}thesis\isanewline
\ \ \ \ \ \ \ \ \isacommand{using}\isamarkupfalse%
\ calculation\ \isacommand{by}\isamarkupfalse%
\ auto\isanewline
\ \ \ \ \isacommand{qed}\isamarkupfalse%
\isanewline
\ \ \isacommand{qed}\isamarkupfalse%
\isanewline
\isacommand{qed}\isamarkupfalse%
%
\endisatagproof
{\isafoldproof}%
%
\isadelimproof
\isanewline
%
\endisadelimproof
\isanewline
\isacommand{lemma}\isamarkupfalse%
\ eq{\isacharunderscore}{\kern0pt}pred{\isacharunderscore}{\kern0pt}right{\isacharunderscore}{\kern0pt}coproj{\isacharcolon}{\kern0pt}\isanewline
\ \ \isakeyword{assumes}\ u{\isacharunderscore}{\kern0pt}type{\isacharbrackleft}{\kern0pt}type{\isacharunderscore}{\kern0pt}rule{\isacharbrackright}{\kern0pt}{\isacharcolon}{\kern0pt}\ {\isachardoublequoteopen}u\ {\isasymin}\isactrlsub c\ X\ {\isasymCoprod}\ Y{\isachardoublequoteclose}\ \isakeyword{and}\ y{\isacharunderscore}{\kern0pt}type{\isacharbrackleft}{\kern0pt}type{\isacharunderscore}{\kern0pt}rule{\isacharbrackright}{\kern0pt}{\isacharcolon}{\kern0pt}\ {\isachardoublequoteopen}y\ {\isasymin}\isactrlsub c\ Y{\isachardoublequoteclose}\isanewline
\ \ \isakeyword{shows}\ {\isachardoublequoteopen}eq{\isacharunderscore}{\kern0pt}pred\ {\isacharparenleft}{\kern0pt}X\ {\isasymCoprod}\ Y{\isacharparenright}{\kern0pt}\ {\isasymcirc}\isactrlsub c\ {\isasymlangle}u{\isacharcomma}{\kern0pt}\ right{\isacharunderscore}{\kern0pt}coproj\ X\ Y\ {\isasymcirc}\isactrlsub c\ y{\isasymrangle}\ {\isacharequal}{\kern0pt}\ {\isacharparenleft}{\kern0pt}{\isacharparenleft}{\kern0pt}{\isasymf}\ {\isasymcirc}\isactrlsub c\ {\isasymbeta}\isactrlbsub X\isactrlesub {\isacharparenright}{\kern0pt}\ {\isasymamalg}\ {\isacharparenleft}{\kern0pt}eq{\isacharunderscore}{\kern0pt}pred\ Y\ {\isasymcirc}\isactrlsub c\ {\isasymlangle}id\ Y{\isacharcomma}{\kern0pt}\ y\ {\isasymcirc}\isactrlsub c\ {\isasymbeta}\isactrlbsub Y\isactrlesub {\isasymrangle}{\isacharparenright}{\kern0pt}{\isacharparenright}{\kern0pt}\ {\isasymcirc}\isactrlsub c\ u{\isachardoublequoteclose}\isanewline
%
\isadelimproof
%
\endisadelimproof
%
\isatagproof
\isacommand{proof}\isamarkupfalse%
\ {\isacharparenleft}{\kern0pt}cases\ {\isachardoublequoteopen}eq{\isacharunderscore}{\kern0pt}pred\ {\isacharparenleft}{\kern0pt}X\ {\isasymCoprod}\ Y{\isacharparenright}{\kern0pt}\ {\isasymcirc}\isactrlsub c\ {\isasymlangle}u{\isacharcomma}{\kern0pt}\ right{\isacharunderscore}{\kern0pt}coproj\ X\ Y\ {\isasymcirc}\isactrlsub c\ y{\isasymrangle}\ {\isacharequal}{\kern0pt}\ {\isasymt}{\isachardoublequoteclose}{\isacharcomma}{\kern0pt}\ auto{\isacharparenright}{\kern0pt}\isanewline
\ \ \isacommand{assume}\isamarkupfalse%
\ {\isachardoublequoteopen}eq{\isacharunderscore}{\kern0pt}pred\ {\isacharparenleft}{\kern0pt}X\ {\isasymCoprod}\ Y{\isacharparenright}{\kern0pt}\ {\isasymcirc}\isactrlsub c\ {\isasymlangle}u{\isacharcomma}{\kern0pt}right{\isacharunderscore}{\kern0pt}coproj\ X\ Y\ {\isasymcirc}\isactrlsub c\ y{\isasymrangle}\ {\isacharequal}{\kern0pt}\ {\isasymt}{\isachardoublequoteclose}\isanewline
\ \ \isacommand{then}\isamarkupfalse%
\ \isacommand{have}\isamarkupfalse%
\ u{\isacharunderscore}{\kern0pt}is{\isacharunderscore}{\kern0pt}right{\isacharunderscore}{\kern0pt}coproj{\isacharcolon}{\kern0pt}\ {\isachardoublequoteopen}u\ {\isacharequal}{\kern0pt}\ right{\isacharunderscore}{\kern0pt}coproj\ X\ Y\ {\isasymcirc}\isactrlsub c\ y{\isachardoublequoteclose}\isanewline
\ \ \ \ \isacommand{using}\isamarkupfalse%
\ eq{\isacharunderscore}{\kern0pt}pred{\isacharunderscore}{\kern0pt}iff{\isacharunderscore}{\kern0pt}eq\ \isacommand{by}\isamarkupfalse%
\ {\isacharparenleft}{\kern0pt}typecheck{\isacharunderscore}{\kern0pt}cfuncs{\isacharunderscore}{\kern0pt}prems{\isacharcomma}{\kern0pt}\ presburger{\isacharparenright}{\kern0pt}\isanewline
\ \ \isacommand{show}\isamarkupfalse%
\ {\isachardoublequoteopen}{\isasymt}\ {\isacharequal}{\kern0pt}\ {\isacharparenleft}{\kern0pt}{\isasymf}\ {\isasymcirc}\isactrlsub c\ {\isasymbeta}\isactrlbsub X\isactrlesub {\isacharparenright}{\kern0pt}\ {\isasymamalg}\ {\isacharparenleft}{\kern0pt}eq{\isacharunderscore}{\kern0pt}pred\ Y\ {\isasymcirc}\isactrlsub c\ {\isasymlangle}id\isactrlsub c\ Y{\isacharcomma}{\kern0pt}y\ {\isasymcirc}\isactrlsub c\ {\isasymbeta}\isactrlbsub Y\isactrlesub {\isasymrangle}{\isacharparenright}{\kern0pt}\ {\isasymcirc}\isactrlsub c\ u{\isachardoublequoteclose}\isanewline
\ \ \isacommand{proof}\isamarkupfalse%
\ {\isacharminus}{\kern0pt}\isanewline
\ \ \ \ \isacommand{have}\isamarkupfalse%
\ {\isachardoublequoteopen}{\isacharparenleft}{\kern0pt}{\isasymf}\ {\isasymcirc}\isactrlsub c\ {\isasymbeta}\isactrlbsub X\isactrlesub {\isacharparenright}{\kern0pt}\ {\isasymamalg}\ {\isacharparenleft}{\kern0pt}eq{\isacharunderscore}{\kern0pt}pred\ Y\ {\isasymcirc}\isactrlsub c\ {\isasymlangle}id\isactrlsub c\ Y{\isacharcomma}{\kern0pt}y\ {\isasymcirc}\isactrlsub c\ {\isasymbeta}\isactrlbsub Y\isactrlesub {\isasymrangle}{\isacharparenright}{\kern0pt}\ {\isasymcirc}\isactrlsub c\ u\isanewline
\ \ \ \ \ \ \ \ {\isacharequal}{\kern0pt}\ {\isacharparenleft}{\kern0pt}{\isasymf}\ {\isasymcirc}\isactrlsub c\ {\isasymbeta}\isactrlbsub X\isactrlesub {\isacharparenright}{\kern0pt}\ {\isasymamalg}\ {\isacharparenleft}{\kern0pt}eq{\isacharunderscore}{\kern0pt}pred\ Y\ {\isasymcirc}\isactrlsub c\ {\isasymlangle}id\isactrlsub c\ Y{\isacharcomma}{\kern0pt}y\ {\isasymcirc}\isactrlsub c\ {\isasymbeta}\isactrlbsub Y\isactrlesub {\isasymrangle}{\isacharparenright}{\kern0pt}\ {\isasymcirc}\isactrlsub c\ right{\isacharunderscore}{\kern0pt}coproj\ X\ Y\ {\isasymcirc}\isactrlsub c\ y{\isachardoublequoteclose}\isanewline
\ \ \ \ \ \ \isacommand{using}\isamarkupfalse%
\ u{\isacharunderscore}{\kern0pt}is{\isacharunderscore}{\kern0pt}right{\isacharunderscore}{\kern0pt}coproj\ \isacommand{by}\isamarkupfalse%
\ auto\isanewline
\ \ \ \ \isacommand{also}\isamarkupfalse%
\ \isacommand{have}\isamarkupfalse%
\ {\isachardoublequoteopen}{\isachardot}{\kern0pt}{\isachardot}{\kern0pt}{\isachardot}{\kern0pt}\ {\isacharequal}{\kern0pt}\ {\isacharparenleft}{\kern0pt}eq{\isacharunderscore}{\kern0pt}pred\ Y\ {\isasymcirc}\isactrlsub c\ {\isasymlangle}id\isactrlsub c\ Y{\isacharcomma}{\kern0pt}y\ {\isasymcirc}\isactrlsub c\ {\isasymbeta}\isactrlbsub Y\isactrlesub {\isasymrangle}{\isacharparenright}{\kern0pt}\ {\isasymcirc}\isactrlsub c\ y{\isachardoublequoteclose}\isanewline
\ \ \ \ \ \ \isacommand{by}\isamarkupfalse%
\ {\isacharparenleft}{\kern0pt}typecheck{\isacharunderscore}{\kern0pt}cfuncs{\isacharcomma}{\kern0pt}\ simp\ add{\isacharcolon}{\kern0pt}\ comp{\isacharunderscore}{\kern0pt}associative{\isadigit{2}}\ right{\isacharunderscore}{\kern0pt}coproj{\isacharunderscore}{\kern0pt}cfunc{\isacharunderscore}{\kern0pt}coprod{\isacharparenright}{\kern0pt}\isanewline
\ \ \ \ \isacommand{also}\isamarkupfalse%
\ \isacommand{have}\isamarkupfalse%
\ {\isachardoublequoteopen}{\isachardot}{\kern0pt}{\isachardot}{\kern0pt}{\isachardot}{\kern0pt}\ {\isacharequal}{\kern0pt}\ eq{\isacharunderscore}{\kern0pt}pred\ Y\ {\isasymcirc}\isactrlsub c\ {\isasymlangle}y{\isacharcomma}{\kern0pt}y{\isasymrangle}{\isachardoublequoteclose}\isanewline
\ \ \ \ \ \ \isacommand{by}\isamarkupfalse%
\ {\isacharparenleft}{\kern0pt}typecheck{\isacharunderscore}{\kern0pt}cfuncs{\isacharcomma}{\kern0pt}\ smt\ cart{\isacharunderscore}{\kern0pt}prod{\isacharunderscore}{\kern0pt}extract{\isacharunderscore}{\kern0pt}left\ comp{\isacharunderscore}{\kern0pt}associative{\isadigit{2}}{\isacharparenright}{\kern0pt}\isanewline
\ \ \ \ \isacommand{also}\isamarkupfalse%
\ \isacommand{have}\isamarkupfalse%
\ {\isachardoublequoteopen}{\isachardot}{\kern0pt}{\isachardot}{\kern0pt}{\isachardot}{\kern0pt}\ {\isacharequal}{\kern0pt}\ {\isasymt}{\isachardoublequoteclose}\isanewline
\ \ \ \ \ \ \isacommand{using}\isamarkupfalse%
\ eq{\isacharunderscore}{\kern0pt}pred{\isacharunderscore}{\kern0pt}iff{\isacharunderscore}{\kern0pt}eq\ y{\isacharunderscore}{\kern0pt}type\ \isacommand{by}\isamarkupfalse%
\ auto\isanewline
\ \ \ \ \isacommand{then}\isamarkupfalse%
\ \isacommand{show}\isamarkupfalse%
\ {\isacharquery}{\kern0pt}thesis\isanewline
\ \ \ \ \ \ \isacommand{using}\isamarkupfalse%
\ calculation\ \isacommand{by}\isamarkupfalse%
\ auto\isanewline
\ \ \isacommand{qed}\isamarkupfalse%
\isanewline
\isacommand{next}\isamarkupfalse%
\isanewline
\ \ \isacommand{assume}\isamarkupfalse%
\ {\isachardoublequoteopen}eq{\isacharunderscore}{\kern0pt}pred\ {\isacharparenleft}{\kern0pt}X\ {\isasymCoprod}\ Y{\isacharparenright}{\kern0pt}\ {\isasymcirc}\isactrlsub c\ {\isasymlangle}u{\isacharcomma}{\kern0pt}right{\isacharunderscore}{\kern0pt}coproj\ X\ Y\ {\isasymcirc}\isactrlsub c\ y{\isasymrangle}\ {\isasymnoteq}\ {\isasymt}{\isachardoublequoteclose}\isanewline
\ \ \isacommand{then}\isamarkupfalse%
\ \isacommand{have}\isamarkupfalse%
\ eq{\isacharunderscore}{\kern0pt}pred{\isacharunderscore}{\kern0pt}false{\isacharcolon}{\kern0pt}\ {\isachardoublequoteopen}eq{\isacharunderscore}{\kern0pt}pred\ {\isacharparenleft}{\kern0pt}X\ {\isasymCoprod}\ Y{\isacharparenright}{\kern0pt}\ {\isasymcirc}\isactrlsub c\ {\isasymlangle}u{\isacharcomma}{\kern0pt}right{\isacharunderscore}{\kern0pt}coproj\ X\ Y\ {\isasymcirc}\isactrlsub c\ y{\isasymrangle}\ {\isacharequal}{\kern0pt}\ {\isasymf}{\isachardoublequoteclose}\isanewline
\ \ \ \ \isacommand{using}\isamarkupfalse%
\ true{\isacharunderscore}{\kern0pt}false{\isacharunderscore}{\kern0pt}only{\isacharunderscore}{\kern0pt}truth{\isacharunderscore}{\kern0pt}values\ \isacommand{by}\isamarkupfalse%
\ {\isacharparenleft}{\kern0pt}typecheck{\isacharunderscore}{\kern0pt}cfuncs{\isacharcomma}{\kern0pt}\ blast{\isacharparenright}{\kern0pt}\isanewline
\ \ \isacommand{then}\isamarkupfalse%
\ \isacommand{have}\isamarkupfalse%
\ u{\isacharunderscore}{\kern0pt}not{\isacharunderscore}{\kern0pt}right{\isacharunderscore}{\kern0pt}coproj{\isacharunderscore}{\kern0pt}y{\isacharcolon}{\kern0pt}\ {\isachardoublequoteopen}u\ \ {\isasymnoteq}\ right{\isacharunderscore}{\kern0pt}coproj\ X\ Y\ {\isasymcirc}\isactrlsub c\ y{\isachardoublequoteclose}\isanewline
\ \ \ \ \isacommand{using}\isamarkupfalse%
\ eq{\isacharunderscore}{\kern0pt}pred{\isacharunderscore}{\kern0pt}iff{\isacharunderscore}{\kern0pt}eq{\isacharunderscore}{\kern0pt}conv\ \isacommand{by}\isamarkupfalse%
\ {\isacharparenleft}{\kern0pt}typecheck{\isacharunderscore}{\kern0pt}cfuncs{\isacharunderscore}{\kern0pt}prems{\isacharcomma}{\kern0pt}\ presburger{\isacharparenright}{\kern0pt}\isanewline
\isanewline
\ \ \isacommand{show}\isamarkupfalse%
\ {\isachardoublequoteopen}eq{\isacharunderscore}{\kern0pt}pred\ {\isacharparenleft}{\kern0pt}X\ {\isasymCoprod}\ Y{\isacharparenright}{\kern0pt}\ {\isasymcirc}\isactrlsub c\ {\isasymlangle}u{\isacharcomma}{\kern0pt}right{\isacharunderscore}{\kern0pt}coproj\ X\ Y\ {\isasymcirc}\isactrlsub c\ y{\isasymrangle}\ {\isacharequal}{\kern0pt}\ {\isacharparenleft}{\kern0pt}{\isasymf}\ {\isasymcirc}\isactrlsub c\ {\isasymbeta}\isactrlbsub X\isactrlesub {\isacharparenright}{\kern0pt}\ {\isasymamalg}\ {\isacharparenleft}{\kern0pt}eq{\isacharunderscore}{\kern0pt}pred\ Y\ {\isasymcirc}\isactrlsub c\ {\isasymlangle}id\isactrlsub c\ Y{\isacharcomma}{\kern0pt}y\ {\isasymcirc}\isactrlsub c\ {\isasymbeta}\isactrlbsub Y\isactrlesub {\isasymrangle}{\isacharparenright}{\kern0pt}\ {\isasymcirc}\isactrlsub c\ u{\isachardoublequoteclose}\isanewline
\ \ \isacommand{proof}\isamarkupfalse%
\ {\isacharparenleft}{\kern0pt}insert\ eq{\isacharunderscore}{\kern0pt}pred{\isacharunderscore}{\kern0pt}false{\isacharcomma}{\kern0pt}\ cases\ {\isachardoublequoteopen}{\isasymexists}\ g{\isachardot}{\kern0pt}\ g\ {\isacharcolon}{\kern0pt}\ one\ {\isasymrightarrow}\ Y\ {\isasymand}\ u\ {\isacharequal}{\kern0pt}\ right{\isacharunderscore}{\kern0pt}coproj\ X\ Y\ {\isasymcirc}\isactrlsub c\ g{\isachardoublequoteclose}{\isacharcomma}{\kern0pt}\ auto{\isacharparenright}{\kern0pt}\isanewline
\ \ \ \ \isacommand{fix}\isamarkupfalse%
\ g\isanewline
\ \ \ \ \isacommand{assume}\isamarkupfalse%
\ g{\isacharunderscore}{\kern0pt}type{\isacharbrackleft}{\kern0pt}type{\isacharunderscore}{\kern0pt}rule{\isacharbrackright}{\kern0pt}{\isacharcolon}{\kern0pt}\ {\isachardoublequoteopen}g\ {\isasymin}\isactrlsub c\ Y{\isachardoublequoteclose}\isanewline
\ \ \ \ \isacommand{assume}\isamarkupfalse%
\ u{\isacharunderscore}{\kern0pt}right{\isacharunderscore}{\kern0pt}coproj{\isacharcolon}{\kern0pt}\ {\isachardoublequoteopen}u\ {\isacharequal}{\kern0pt}\ right{\isacharunderscore}{\kern0pt}coproj\ X\ Y\ {\isasymcirc}\isactrlsub c\ g{\isachardoublequoteclose}\isanewline
\ \ \ \ \isacommand{then}\isamarkupfalse%
\ \isacommand{have}\isamarkupfalse%
\ y{\isacharunderscore}{\kern0pt}not{\isacharunderscore}{\kern0pt}g{\isacharcolon}{\kern0pt}\ {\isachardoublequoteopen}y\ {\isasymnoteq}\ g{\isachardoublequoteclose}\isanewline
\ \ \ \ \ \ \isacommand{using}\isamarkupfalse%
\ u{\isacharunderscore}{\kern0pt}not{\isacharunderscore}{\kern0pt}right{\isacharunderscore}{\kern0pt}coproj{\isacharunderscore}{\kern0pt}y\ \isacommand{by}\isamarkupfalse%
\ auto\isanewline
\isanewline
\ \ \ \ \isacommand{show}\isamarkupfalse%
\ {\isachardoublequoteopen}{\isasymf}\ {\isacharequal}{\kern0pt}\ {\isacharparenleft}{\kern0pt}{\isasymf}\ {\isasymcirc}\isactrlsub c\ {\isasymbeta}\isactrlbsub X\isactrlesub {\isacharparenright}{\kern0pt}\ {\isasymamalg}\ {\isacharparenleft}{\kern0pt}eq{\isacharunderscore}{\kern0pt}pred\ Y\ {\isasymcirc}\isactrlsub c\ {\isasymlangle}id\isactrlsub c\ Y{\isacharcomma}{\kern0pt}y\ {\isasymcirc}\isactrlsub c\ {\isasymbeta}\isactrlbsub Y\isactrlesub {\isasymrangle}{\isacharparenright}{\kern0pt}\ {\isasymcirc}\isactrlsub c\ right{\isacharunderscore}{\kern0pt}coproj\ X\ Y\ {\isasymcirc}\isactrlsub c\ g{\isachardoublequoteclose}\isanewline
\ \ \ \ \isacommand{proof}\isamarkupfalse%
\ {\isacharminus}{\kern0pt}\isanewline
\ \ \ \ \ \ \isacommand{have}\isamarkupfalse%
\ {\isachardoublequoteopen}{\isacharparenleft}{\kern0pt}{\isasymf}\ {\isasymcirc}\isactrlsub c\ {\isasymbeta}\isactrlbsub X\isactrlesub {\isacharparenright}{\kern0pt}\ {\isasymamalg}\ {\isacharparenleft}{\kern0pt}eq{\isacharunderscore}{\kern0pt}pred\ Y\ {\isasymcirc}\isactrlsub c\ {\isasymlangle}id\isactrlsub c\ Y{\isacharcomma}{\kern0pt}y\ {\isasymcirc}\isactrlsub c\ {\isasymbeta}\isactrlbsub Y\isactrlesub {\isasymrangle}{\isacharparenright}{\kern0pt}\ {\isasymcirc}\isactrlsub c\ right{\isacharunderscore}{\kern0pt}coproj\ X\ Y\ {\isasymcirc}\isactrlsub c\ g\isanewline
\ \ \ \ \ \ \ \ \ \ {\isacharequal}{\kern0pt}\ {\isacharparenleft}{\kern0pt}eq{\isacharunderscore}{\kern0pt}pred\ Y\ {\isasymcirc}\isactrlsub c\ {\isasymlangle}id\isactrlsub c\ Y{\isacharcomma}{\kern0pt}y\ {\isasymcirc}\isactrlsub c\ {\isasymbeta}\isactrlbsub Y\isactrlesub {\isasymrangle}{\isacharparenright}{\kern0pt}\ {\isasymcirc}\isactrlsub c\ g{\isachardoublequoteclose}\isanewline
\ \ \ \ \ \ \ \ \isacommand{by}\isamarkupfalse%
\ {\isacharparenleft}{\kern0pt}typecheck{\isacharunderscore}{\kern0pt}cfuncs{\isacharcomma}{\kern0pt}\ simp\ add{\isacharcolon}{\kern0pt}\ comp{\isacharunderscore}{\kern0pt}associative{\isadigit{2}}\ right{\isacharunderscore}{\kern0pt}coproj{\isacharunderscore}{\kern0pt}cfunc{\isacharunderscore}{\kern0pt}coprod{\isacharparenright}{\kern0pt}\isanewline
\ \ \ \ \ \ \isacommand{also}\isamarkupfalse%
\ \isacommand{have}\isamarkupfalse%
\ {\isachardoublequoteopen}{\isachardot}{\kern0pt}{\isachardot}{\kern0pt}{\isachardot}{\kern0pt}\ {\isacharequal}{\kern0pt}\ eq{\isacharunderscore}{\kern0pt}pred\ Y\ {\isasymcirc}\isactrlsub c\ {\isasymlangle}g{\isacharcomma}{\kern0pt}y{\isasymrangle}{\isachardoublequoteclose}\isanewline
\ \ \ \ \ \ \ \ \isacommand{using}\isamarkupfalse%
\ cart{\isacharunderscore}{\kern0pt}prod{\isacharunderscore}{\kern0pt}extract{\isacharunderscore}{\kern0pt}left\ comp{\isacharunderscore}{\kern0pt}associative{\isadigit{2}}\ \isacommand{by}\isamarkupfalse%
\ {\isacharparenleft}{\kern0pt}typecheck{\isacharunderscore}{\kern0pt}cfuncs{\isacharcomma}{\kern0pt}\ auto{\isacharparenright}{\kern0pt}\isanewline
\ \ \ \ \ \ \isacommand{also}\isamarkupfalse%
\ \isacommand{have}\isamarkupfalse%
\ {\isachardoublequoteopen}{\isachardot}{\kern0pt}{\isachardot}{\kern0pt}{\isachardot}{\kern0pt}\ {\isacharequal}{\kern0pt}\ {\isasymf}{\isachardoublequoteclose}\isanewline
\ \ \ \ \ \ \ \ \isacommand{using}\isamarkupfalse%
\ eq{\isacharunderscore}{\kern0pt}pred{\isacharunderscore}{\kern0pt}iff{\isacharunderscore}{\kern0pt}eq{\isacharunderscore}{\kern0pt}conv\ y{\isacharunderscore}{\kern0pt}not{\isacharunderscore}{\kern0pt}g\ y{\isacharunderscore}{\kern0pt}type\ g{\isacharunderscore}{\kern0pt}type\ \isacommand{by}\isamarkupfalse%
\ blast\isanewline
\ \ \ \ \ \ \isacommand{then}\isamarkupfalse%
\ \isacommand{show}\isamarkupfalse%
\ {\isacharquery}{\kern0pt}thesis\isanewline
\ \ \ \ \ \ \ \ \isacommand{using}\isamarkupfalse%
\ calculation\ \isacommand{by}\isamarkupfalse%
\ auto\isanewline
\ \ \ \ \isacommand{qed}\isamarkupfalse%
\isanewline
\ \ \isacommand{next}\isamarkupfalse%
\isanewline
\ \ \ \ \isacommand{assume}\isamarkupfalse%
\ {\isachardoublequoteopen}{\isasymforall}g{\isachardot}{\kern0pt}\ g\ {\isasymin}\isactrlsub c\ Y\ {\isasymlongrightarrow}\ u\ {\isasymnoteq}\ right{\isacharunderscore}{\kern0pt}coproj\ X\ Y\ {\isasymcirc}\isactrlsub c\ g{\isachardoublequoteclose}\isanewline
\ \ \ \ \isacommand{then}\isamarkupfalse%
\ \isacommand{obtain}\isamarkupfalse%
\ g\ \isakeyword{where}\ g{\isacharunderscore}{\kern0pt}type{\isacharbrackleft}{\kern0pt}type{\isacharunderscore}{\kern0pt}rule{\isacharbrackright}{\kern0pt}{\isacharcolon}{\kern0pt}\ {\isachardoublequoteopen}g\ {\isasymin}\isactrlsub c\ X{\isachardoublequoteclose}\ \isakeyword{and}\ u{\isacharunderscore}{\kern0pt}left{\isacharunderscore}{\kern0pt}coproj{\isacharcolon}{\kern0pt}\ {\isachardoublequoteopen}u\ {\isacharequal}{\kern0pt}\ left{\isacharunderscore}{\kern0pt}coproj\ X\ Y\ {\isasymcirc}\isactrlsub c\ g{\isachardoublequoteclose}\isanewline
\ \ \ \ \ \ \isacommand{by}\isamarkupfalse%
\ {\isacharparenleft}{\kern0pt}meson\ coprojs{\isacharunderscore}{\kern0pt}jointly{\isacharunderscore}{\kern0pt}surj\ u{\isacharunderscore}{\kern0pt}type{\isacharparenright}{\kern0pt}\isanewline
\ \ \ \ \isacommand{show}\isamarkupfalse%
\ {\isachardoublequoteopen}{\isasymf}\ {\isacharequal}{\kern0pt}\ {\isacharparenleft}{\kern0pt}{\isasymf}\ {\isasymcirc}\isactrlsub c\ {\isasymbeta}\isactrlbsub X\isactrlesub {\isacharparenright}{\kern0pt}\ {\isasymamalg}\ {\isacharparenleft}{\kern0pt}eq{\isacharunderscore}{\kern0pt}pred\ Y\ {\isasymcirc}\isactrlsub c\ {\isasymlangle}id\isactrlsub c\ Y{\isacharcomma}{\kern0pt}y\ {\isasymcirc}\isactrlsub c\ {\isasymbeta}\isactrlbsub Y\isactrlesub {\isasymrangle}{\isacharparenright}{\kern0pt}\ {\isasymcirc}\isactrlsub c\ u{\isachardoublequoteclose}\isanewline
\ \ \ \ \isacommand{proof}\isamarkupfalse%
\ {\isacharminus}{\kern0pt}\isanewline
\ \ \ \ \ \ \isacommand{have}\isamarkupfalse%
\ {\isachardoublequoteopen}{\isacharparenleft}{\kern0pt}{\isasymf}\ {\isasymcirc}\isactrlsub c\ {\isasymbeta}\isactrlbsub X\isactrlesub {\isacharparenright}{\kern0pt}\ {\isasymamalg}\ {\isacharparenleft}{\kern0pt}eq{\isacharunderscore}{\kern0pt}pred\ Y\ {\isasymcirc}\isactrlsub c\ {\isasymlangle}id\isactrlsub c\ Y{\isacharcomma}{\kern0pt}y\ {\isasymcirc}\isactrlsub c\ {\isasymbeta}\isactrlbsub Y\isactrlesub {\isasymrangle}{\isacharparenright}{\kern0pt}\ {\isasymcirc}\isactrlsub c\ u\isanewline
\ \ \ \ \ \ \ \ \ \ {\isacharequal}{\kern0pt}\ {\isacharparenleft}{\kern0pt}{\isasymf}\ {\isasymcirc}\isactrlsub c\ {\isasymbeta}\isactrlbsub X\isactrlesub {\isacharparenright}{\kern0pt}\ {\isasymamalg}\ {\isacharparenleft}{\kern0pt}eq{\isacharunderscore}{\kern0pt}pred\ Y\ {\isasymcirc}\isactrlsub c\ {\isasymlangle}id\isactrlsub c\ Y{\isacharcomma}{\kern0pt}y\ {\isasymcirc}\isactrlsub c\ {\isasymbeta}\isactrlbsub Y\isactrlesub {\isasymrangle}{\isacharparenright}{\kern0pt}\ {\isasymcirc}\isactrlsub c\ left{\isacharunderscore}{\kern0pt}coproj\ X\ Y\ {\isasymcirc}\isactrlsub c\ g{\isachardoublequoteclose}\isanewline
\ \ \ \ \ \ \ \ \isacommand{using}\isamarkupfalse%
\ u{\isacharunderscore}{\kern0pt}left{\isacharunderscore}{\kern0pt}coproj\ \isacommand{by}\isamarkupfalse%
\ auto\isanewline
\ \ \ \ \ \ \isacommand{also}\isamarkupfalse%
\ \isacommand{have}\isamarkupfalse%
\ {\isachardoublequoteopen}{\isachardot}{\kern0pt}{\isachardot}{\kern0pt}{\isachardot}{\kern0pt}\ {\isacharequal}{\kern0pt}\ {\isacharparenleft}{\kern0pt}{\isasymf}\ {\isasymcirc}\isactrlsub c\ {\isasymbeta}\isactrlbsub X\isactrlesub {\isacharparenright}{\kern0pt}\ {\isasymcirc}\isactrlsub c\ g{\isachardoublequoteclose}\isanewline
\ \ \ \ \ \ \ \ \isacommand{by}\isamarkupfalse%
\ {\isacharparenleft}{\kern0pt}typecheck{\isacharunderscore}{\kern0pt}cfuncs{\isacharcomma}{\kern0pt}\ simp\ add{\isacharcolon}{\kern0pt}\ comp{\isacharunderscore}{\kern0pt}associative{\isadigit{2}}\ left{\isacharunderscore}{\kern0pt}coproj{\isacharunderscore}{\kern0pt}cfunc{\isacharunderscore}{\kern0pt}coprod{\isacharparenright}{\kern0pt}\isanewline
\ \ \ \ \ \ \isacommand{also}\isamarkupfalse%
\ \isacommand{have}\isamarkupfalse%
\ {\isachardoublequoteopen}{\isachardot}{\kern0pt}{\isachardot}{\kern0pt}{\isachardot}{\kern0pt}\ {\isacharequal}{\kern0pt}\ {\isasymf}{\isachardoublequoteclose}\isanewline
\ \ \ \ \ \ \ \ \isacommand{by}\isamarkupfalse%
\ {\isacharparenleft}{\kern0pt}typecheck{\isacharunderscore}{\kern0pt}cfuncs{\isacharcomma}{\kern0pt}\ smt\ {\isacharparenleft}{\kern0pt}z{\isadigit{3}}{\isacharparenright}{\kern0pt}\ comp{\isacharunderscore}{\kern0pt}associative{\isadigit{2}}\ id{\isacharunderscore}{\kern0pt}right{\isacharunderscore}{\kern0pt}unit{\isadigit{2}}\ id{\isacharunderscore}{\kern0pt}type\ terminal{\isacharunderscore}{\kern0pt}func{\isacharunderscore}{\kern0pt}comp\ terminal{\isacharunderscore}{\kern0pt}func{\isacharunderscore}{\kern0pt}unique{\isacharparenright}{\kern0pt}\isanewline
\ \ \ \ \ \ \isacommand{then}\isamarkupfalse%
\ \isacommand{show}\isamarkupfalse%
\ {\isacharquery}{\kern0pt}thesis\isanewline
\ \ \ \ \ \ \ \ \isacommand{using}\isamarkupfalse%
\ calculation\ \isacommand{by}\isamarkupfalse%
\ auto\isanewline
\ \ \ \ \isacommand{qed}\isamarkupfalse%
\isanewline
\ \ \isacommand{qed}\isamarkupfalse%
\isanewline
\isacommand{qed}\isamarkupfalse%
%
\endisatagproof
{\isafoldproof}%
%
\isadelimproof
%
\endisadelimproof
%
\isadelimdocument
%
\endisadelimdocument
%
\isatagdocument
%
\isamarkupsubsection{Bowtie Product%
}
\isamarkuptrue%
%
\endisatagdocument
{\isafolddocument}%
%
\isadelimdocument
%
\endisadelimdocument
\isacommand{definition}\isamarkupfalse%
\ cfunc{\isacharunderscore}{\kern0pt}bowtie{\isacharunderscore}{\kern0pt}prod\ {\isacharcolon}{\kern0pt}{\isacharcolon}{\kern0pt}\ {\isachardoublequoteopen}cfunc\ {\isasymRightarrow}\ cfunc\ {\isasymRightarrow}\ cfunc{\isachardoublequoteclose}\ {\isacharparenleft}{\kern0pt}\isakeyword{infixr}\ {\isachardoublequoteopen}{\isasymbowtie}\isactrlsub f{\isachardoublequoteclose}\ {\isadigit{5}}{\isadigit{5}}{\isacharparenright}{\kern0pt}\ \isakeyword{where}\isanewline
\ \ {\isachardoublequoteopen}f\ {\isasymbowtie}\isactrlsub f\ g\ {\isacharequal}{\kern0pt}\ {\isacharparenleft}{\kern0pt}{\isacharparenleft}{\kern0pt}left{\isacharunderscore}{\kern0pt}coproj\ {\isacharparenleft}{\kern0pt}codomain\ f{\isacharparenright}{\kern0pt}\ {\isacharparenleft}{\kern0pt}codomain\ g{\isacharparenright}{\kern0pt}{\isacharparenright}{\kern0pt}\ {\isasymcirc}\isactrlsub c\ f{\isacharparenright}{\kern0pt}\ {\isasymamalg}\ {\isacharparenleft}{\kern0pt}{\isacharparenleft}{\kern0pt}right{\isacharunderscore}{\kern0pt}coproj\ {\isacharparenleft}{\kern0pt}codomain\ f{\isacharparenright}{\kern0pt}\ {\isacharparenleft}{\kern0pt}codomain\ g{\isacharparenright}{\kern0pt}{\isacharparenright}{\kern0pt}\ {\isasymcirc}\isactrlsub c\ g{\isacharparenright}{\kern0pt}{\isachardoublequoteclose}\isanewline
\isanewline
\isacommand{lemma}\isamarkupfalse%
\ cfunc{\isacharunderscore}{\kern0pt}bowtie{\isacharunderscore}{\kern0pt}prod{\isacharunderscore}{\kern0pt}def{\isadigit{2}}{\isacharcolon}{\kern0pt}\ \isanewline
\ \ \isakeyword{assumes}\ {\isachardoublequoteopen}f\ {\isacharcolon}{\kern0pt}\ X\ {\isasymrightarrow}\ Y{\isachardoublequoteclose}\ {\isachardoublequoteopen}g\ {\isacharcolon}{\kern0pt}\ V{\isasymrightarrow}\ W{\isachardoublequoteclose}\isanewline
\ \ \isakeyword{shows}\ {\isachardoublequoteopen}f\ {\isasymbowtie}\isactrlsub f\ g\ {\isacharequal}{\kern0pt}\ {\isacharparenleft}{\kern0pt}left{\isacharunderscore}{\kern0pt}coproj\ Y\ W\ {\isasymcirc}\isactrlsub c\ f{\isacharparenright}{\kern0pt}\ {\isasymamalg}\ {\isacharparenleft}{\kern0pt}right{\isacharunderscore}{\kern0pt}coproj\ Y\ W\ {\isasymcirc}\isactrlsub c\ g{\isacharparenright}{\kern0pt}{\isachardoublequoteclose}\isanewline
%
\isadelimproof
\ \ %
\endisadelimproof
%
\isatagproof
\isacommand{using}\isamarkupfalse%
\ assms\ cfunc{\isacharunderscore}{\kern0pt}bowtie{\isacharunderscore}{\kern0pt}prod{\isacharunderscore}{\kern0pt}def\ cfunc{\isacharunderscore}{\kern0pt}type{\isacharunderscore}{\kern0pt}def\ \isacommand{by}\isamarkupfalse%
\ auto%
\endisatagproof
{\isafoldproof}%
%
\isadelimproof
\isanewline
%
\endisadelimproof
\isanewline
\isacommand{lemma}\isamarkupfalse%
\ cfunc{\isacharunderscore}{\kern0pt}bowtie{\isacharunderscore}{\kern0pt}prod{\isacharunderscore}{\kern0pt}type{\isacharbrackleft}{\kern0pt}type{\isacharunderscore}{\kern0pt}rule{\isacharbrackright}{\kern0pt}{\isacharcolon}{\kern0pt}\isanewline
\ \ {\isachardoublequoteopen}f\ {\isacharcolon}{\kern0pt}\ X\ {\isasymrightarrow}\ Y\ {\isasymLongrightarrow}\ g\ {\isacharcolon}{\kern0pt}\ V\ {\isasymrightarrow}\ W\ {\isasymLongrightarrow}\ f\ {\isasymbowtie}\isactrlsub f\ g\ {\isacharcolon}{\kern0pt}\ X\ {\isasymCoprod}\ V\ {\isasymrightarrow}\ Y\ {\isasymCoprod}\ W{\isachardoublequoteclose}\isanewline
%
\isadelimproof
\ \ %
\endisadelimproof
%
\isatagproof
\isacommand{unfolding}\isamarkupfalse%
\ cfunc{\isacharunderscore}{\kern0pt}bowtie{\isacharunderscore}{\kern0pt}prod{\isacharunderscore}{\kern0pt}def\isanewline
\ \ \isacommand{using}\isamarkupfalse%
\ cfunc{\isacharunderscore}{\kern0pt}coprod{\isacharunderscore}{\kern0pt}type\ cfunc{\isacharunderscore}{\kern0pt}type{\isacharunderscore}{\kern0pt}def\ comp{\isacharunderscore}{\kern0pt}type\ left{\isacharunderscore}{\kern0pt}proj{\isacharunderscore}{\kern0pt}type\ right{\isacharunderscore}{\kern0pt}proj{\isacharunderscore}{\kern0pt}type\ \isacommand{by}\isamarkupfalse%
\ auto%
\endisatagproof
{\isafoldproof}%
%
\isadelimproof
\isanewline
%
\endisadelimproof
\isanewline
\isacommand{lemma}\isamarkupfalse%
\ left{\isacharunderscore}{\kern0pt}coproj{\isacharunderscore}{\kern0pt}cfunc{\isacharunderscore}{\kern0pt}bowtie{\isacharunderscore}{\kern0pt}prod{\isacharcolon}{\kern0pt}\isanewline
\ \ {\isachardoublequoteopen}f\ {\isacharcolon}{\kern0pt}\ X\ {\isasymrightarrow}\ Y\ {\isasymLongrightarrow}\ g\ {\isacharcolon}{\kern0pt}\ V\ {\isasymrightarrow}\ W\ {\isasymLongrightarrow}\ {\isacharparenleft}{\kern0pt}f\ {\isasymbowtie}\isactrlsub f\ g{\isacharparenright}{\kern0pt}\ {\isasymcirc}\isactrlsub c\ left{\isacharunderscore}{\kern0pt}coproj\ X\ V\ {\isacharequal}{\kern0pt}\ left{\isacharunderscore}{\kern0pt}coproj\ Y\ W\ {\isasymcirc}\isactrlsub c\ f{\isachardoublequoteclose}\isanewline
%
\isadelimproof
\ \ %
\endisadelimproof
%
\isatagproof
\isacommand{unfolding}\isamarkupfalse%
\ cfunc{\isacharunderscore}{\kern0pt}bowtie{\isacharunderscore}{\kern0pt}prod{\isacharunderscore}{\kern0pt}def{\isadigit{2}}\isanewline
\ \ \isacommand{by}\isamarkupfalse%
\ {\isacharparenleft}{\kern0pt}meson\ comp{\isacharunderscore}{\kern0pt}type\ left{\isacharunderscore}{\kern0pt}coproj{\isacharunderscore}{\kern0pt}cfunc{\isacharunderscore}{\kern0pt}coprod\ left{\isacharunderscore}{\kern0pt}proj{\isacharunderscore}{\kern0pt}type\ right{\isacharunderscore}{\kern0pt}proj{\isacharunderscore}{\kern0pt}type{\isacharparenright}{\kern0pt}%
\endisatagproof
{\isafoldproof}%
%
\isadelimproof
\isanewline
%
\endisadelimproof
\isanewline
\ \isacommand{lemma}\isamarkupfalse%
\ right{\isacharunderscore}{\kern0pt}coproj{\isacharunderscore}{\kern0pt}cfunc{\isacharunderscore}{\kern0pt}bowtie{\isacharunderscore}{\kern0pt}prod{\isacharcolon}{\kern0pt}\isanewline
\ \ {\isachardoublequoteopen}f\ {\isacharcolon}{\kern0pt}\ X\ {\isasymrightarrow}\ Y\ {\isasymLongrightarrow}\ g\ {\isacharcolon}{\kern0pt}\ V\ {\isasymrightarrow}\ W\ {\isasymLongrightarrow}\ {\isacharparenleft}{\kern0pt}f\ {\isasymbowtie}\isactrlsub f\ g{\isacharparenright}{\kern0pt}\ {\isasymcirc}\isactrlsub c\ right{\isacharunderscore}{\kern0pt}coproj\ X\ V\ {\isacharequal}{\kern0pt}\ right{\isacharunderscore}{\kern0pt}coproj\ Y\ W\ {\isasymcirc}\isactrlsub c\ g{\isachardoublequoteclose}\isanewline
%
\isadelimproof
\ \ %
\endisadelimproof
%
\isatagproof
\isacommand{unfolding}\isamarkupfalse%
\ cfunc{\isacharunderscore}{\kern0pt}bowtie{\isacharunderscore}{\kern0pt}prod{\isacharunderscore}{\kern0pt}def{\isadigit{2}}\isanewline
\ \ \isacommand{by}\isamarkupfalse%
\ {\isacharparenleft}{\kern0pt}meson\ comp{\isacharunderscore}{\kern0pt}type\ right{\isacharunderscore}{\kern0pt}coproj{\isacharunderscore}{\kern0pt}cfunc{\isacharunderscore}{\kern0pt}coprod\ right{\isacharunderscore}{\kern0pt}proj{\isacharunderscore}{\kern0pt}type\ left{\isacharunderscore}{\kern0pt}proj{\isacharunderscore}{\kern0pt}type{\isacharparenright}{\kern0pt}%
\endisatagproof
{\isafoldproof}%
%
\isadelimproof
\isanewline
%
\endisadelimproof
\isanewline
\isacommand{lemma}\isamarkupfalse%
\ cfunc{\isacharunderscore}{\kern0pt}bowtie{\isacharunderscore}{\kern0pt}prod{\isacharunderscore}{\kern0pt}unique{\isacharcolon}{\kern0pt}\ {\isachardoublequoteopen}f\ {\isacharcolon}{\kern0pt}\ X\ {\isasymrightarrow}\ Y\ {\isasymLongrightarrow}\ g\ {\isacharcolon}{\kern0pt}\ V\ {\isasymrightarrow}\ W\ {\isasymLongrightarrow}\ h\ {\isacharcolon}{\kern0pt}\ X\ {\isasymCoprod}\ V\ {\isasymrightarrow}\ Y\ {\isasymCoprod}\ W\ {\isasymLongrightarrow}\isanewline
\ \ \ \ h\ {\isasymcirc}\isactrlsub c\ left{\isacharunderscore}{\kern0pt}coproj\ X\ V\ \ \ {\isacharequal}{\kern0pt}\ left{\isacharunderscore}{\kern0pt}coproj\ Y\ W\ {\isasymcirc}\isactrlsub c\ f\ {\isasymLongrightarrow}\isanewline
\ \ \ \ h\ {\isasymcirc}\isactrlsub c\ right{\isacharunderscore}{\kern0pt}coproj\ X\ V\ {\isacharequal}{\kern0pt}\ right{\isacharunderscore}{\kern0pt}coproj\ Y\ W\ {\isasymcirc}\isactrlsub c\ g\ {\isasymLongrightarrow}\ h\ {\isacharequal}{\kern0pt}\ f\ {\isasymbowtie}\isactrlsub f\ g{\isachardoublequoteclose}\isanewline
%
\isadelimproof
\ \ %
\endisadelimproof
%
\isatagproof
\isacommand{unfolding}\isamarkupfalse%
\ cfunc{\isacharunderscore}{\kern0pt}bowtie{\isacharunderscore}{\kern0pt}prod{\isacharunderscore}{\kern0pt}def\isanewline
\ \ \isacommand{using}\isamarkupfalse%
\ cfunc{\isacharunderscore}{\kern0pt}coprod{\isacharunderscore}{\kern0pt}unique\ cfunc{\isacharunderscore}{\kern0pt}type{\isacharunderscore}{\kern0pt}def\ codomain{\isacharunderscore}{\kern0pt}comp\ domain{\isacharunderscore}{\kern0pt}comp\ left{\isacharunderscore}{\kern0pt}proj{\isacharunderscore}{\kern0pt}type\ right{\isacharunderscore}{\kern0pt}proj{\isacharunderscore}{\kern0pt}type\ \isacommand{by}\isamarkupfalse%
\ auto%
\endisatagproof
{\isafoldproof}%
%
\isadelimproof
%
\endisadelimproof
%
\begin{isamarkuptext}%
The lemma below is dual to Proposition 2.1.11 in Halvorson.%
\end{isamarkuptext}\isamarkuptrue%
\isacommand{lemma}\isamarkupfalse%
\ identity{\isacharunderscore}{\kern0pt}distributes{\isacharunderscore}{\kern0pt}across{\isacharunderscore}{\kern0pt}composition{\isacharunderscore}{\kern0pt}dual{\isacharcolon}{\kern0pt}\isanewline
\ \ \isakeyword{assumes}\ f{\isacharunderscore}{\kern0pt}type{\isacharcolon}{\kern0pt}\ {\isachardoublequoteopen}f\ {\isacharcolon}{\kern0pt}\ A\ {\isasymrightarrow}\ B{\isachardoublequoteclose}\ \isakeyword{and}\ g{\isacharunderscore}{\kern0pt}type{\isacharcolon}{\kern0pt}\ {\isachardoublequoteopen}g\ {\isacharcolon}{\kern0pt}\ B\ {\isasymrightarrow}\ C{\isachardoublequoteclose}\isanewline
\ \ \isakeyword{shows}\ {\isachardoublequoteopen}{\isacharparenleft}{\kern0pt}g\ \ {\isasymcirc}\isactrlsub c\ f{\isacharparenright}{\kern0pt}\ {\isasymbowtie}\isactrlsub f\ id\ X\ {\isacharequal}{\kern0pt}\ {\isacharparenleft}{\kern0pt}g\ {\isasymbowtie}\isactrlsub f\ id\ X{\isacharparenright}{\kern0pt}\ {\isasymcirc}\isactrlsub c\ {\isacharparenleft}{\kern0pt}f\ {\isasymbowtie}\isactrlsub f\ id\ X{\isacharparenright}{\kern0pt}{\isachardoublequoteclose}\isanewline
%
\isadelimproof
%
\endisadelimproof
%
\isatagproof
\isacommand{proof}\isamarkupfalse%
\ {\isacharminus}{\kern0pt}\ \isanewline
\ \ \isacommand{from}\isamarkupfalse%
\ cfunc{\isacharunderscore}{\kern0pt}bowtie{\isacharunderscore}{\kern0pt}prod{\isacharunderscore}{\kern0pt}unique\isanewline
\ \ \isacommand{have}\isamarkupfalse%
\ uniqueness{\isacharcolon}{\kern0pt}\ {\isachardoublequoteopen}{\isasymforall}h{\isachardot}{\kern0pt}\ h\ {\isacharcolon}{\kern0pt}\ A\ {\isasymCoprod}\ \ X\ {\isasymrightarrow}\ C\ {\isasymCoprod}\ X\ {\isasymand}\isanewline
\ \ \ \ h\ {\isasymcirc}\isactrlsub c\ left{\isacharunderscore}{\kern0pt}coproj\ A\ X\ \ {\isacharequal}{\kern0pt}\ left{\isacharunderscore}{\kern0pt}coproj\ C\ X\ {\isasymcirc}\isactrlsub c\ {\isacharparenleft}{\kern0pt}g\ {\isasymcirc}\isactrlsub c\ f{\isacharparenright}{\kern0pt}\ {\isasymand}\isanewline
\ \ \ \ h\ {\isasymcirc}\isactrlsub c\ right{\isacharunderscore}{\kern0pt}coproj\ A\ X\ {\isacharequal}{\kern0pt}\ right{\isacharunderscore}{\kern0pt}coproj\ C\ X\ {\isasymcirc}\isactrlsub c\ \ id{\isacharparenleft}{\kern0pt}X{\isacharparenright}{\kern0pt}\ {\isasymlongrightarrow}\isanewline
\ \ \ \ h\ {\isacharequal}{\kern0pt}\ \ {\isacharparenleft}{\kern0pt}g\ {\isasymcirc}\isactrlsub c\ f{\isacharparenright}{\kern0pt}\ {\isasymbowtie}\isactrlsub f\ \ id\isactrlsub c\ X{\isachardoublequoteclose}\isanewline
\ \ \ \ \isacommand{using}\isamarkupfalse%
\ assms\ \isacommand{by}\isamarkupfalse%
\ {\isacharparenleft}{\kern0pt}typecheck{\isacharunderscore}{\kern0pt}cfuncs{\isacharcomma}{\kern0pt}\ simp\ add{\isacharcolon}{\kern0pt}\ cfunc{\isacharunderscore}{\kern0pt}bowtie{\isacharunderscore}{\kern0pt}prod{\isacharunderscore}{\kern0pt}unique{\isacharparenright}{\kern0pt}\isanewline
\isanewline
\ \ \isacommand{have}\isamarkupfalse%
\ left{\isacharunderscore}{\kern0pt}eq{\isacharcolon}{\kern0pt}\ {\isachardoublequoteopen}\ {\isacharparenleft}{\kern0pt}{\isacharparenleft}{\kern0pt}g\ {\isasymbowtie}\isactrlsub f\ id\isactrlsub c\ X{\isacharparenright}{\kern0pt}\ {\isasymcirc}\isactrlsub c\ {\isacharparenleft}{\kern0pt}f\ {\isasymbowtie}\isactrlsub f\ id\isactrlsub c\ X{\isacharparenright}{\kern0pt}{\isacharparenright}{\kern0pt}\ {\isasymcirc}\isactrlsub c\ left{\isacharunderscore}{\kern0pt}coproj\ A\ X\ {\isacharequal}{\kern0pt}\ left{\isacharunderscore}{\kern0pt}coproj\ C\ X\ {\isasymcirc}\isactrlsub c\ {\isacharparenleft}{\kern0pt}g\ \ {\isasymcirc}\isactrlsub c\ f{\isacharparenright}{\kern0pt}{\isachardoublequoteclose}\isanewline
\ \ \ \ \isacommand{by}\isamarkupfalse%
\ {\isacharparenleft}{\kern0pt}typecheck{\isacharunderscore}{\kern0pt}cfuncs{\isacharcomma}{\kern0pt}\ smt\ comp{\isacharunderscore}{\kern0pt}associative{\isadigit{2}}\ left{\isacharunderscore}{\kern0pt}coproj{\isacharunderscore}{\kern0pt}cfunc{\isacharunderscore}{\kern0pt}bowtie{\isacharunderscore}{\kern0pt}prod\ left{\isacharunderscore}{\kern0pt}proj{\isacharunderscore}{\kern0pt}type\ assms{\isacharparenright}{\kern0pt}\isanewline
\ \ \isacommand{have}\isamarkupfalse%
\ right{\isacharunderscore}{\kern0pt}eq{\isacharcolon}{\kern0pt}\ {\isachardoublequoteopen}\ {\isacharparenleft}{\kern0pt}{\isacharparenleft}{\kern0pt}g\ {\isasymbowtie}\isactrlsub f\ id\isactrlsub c\ X{\isacharparenright}{\kern0pt}\ {\isasymcirc}\isactrlsub c\ {\isacharparenleft}{\kern0pt}f\ {\isasymbowtie}\isactrlsub f\ id\isactrlsub c\ X{\isacharparenright}{\kern0pt}{\isacharparenright}{\kern0pt}\ {\isasymcirc}\isactrlsub c\ right{\isacharunderscore}{\kern0pt}coproj\ A\ X\ {\isacharequal}{\kern0pt}\ right{\isacharunderscore}{\kern0pt}coproj\ C\ X\ {\isasymcirc}\isactrlsub c\ id\ X{\isachardoublequoteclose}\isanewline
\ \ \ \ \isacommand{by}\isamarkupfalse%
{\isacharparenleft}{\kern0pt}typecheck{\isacharunderscore}{\kern0pt}cfuncs{\isacharcomma}{\kern0pt}\ smt\ comp{\isacharunderscore}{\kern0pt}associative{\isadigit{2}}\ id{\isacharunderscore}{\kern0pt}right{\isacharunderscore}{\kern0pt}unit{\isadigit{2}}\ right{\isacharunderscore}{\kern0pt}coproj{\isacharunderscore}{\kern0pt}cfunc{\isacharunderscore}{\kern0pt}bowtie{\isacharunderscore}{\kern0pt}prod\ right{\isacharunderscore}{\kern0pt}proj{\isacharunderscore}{\kern0pt}type\ assms{\isacharparenright}{\kern0pt}\isanewline
\isanewline
\ \ \isacommand{show}\isamarkupfalse%
\ {\isacharquery}{\kern0pt}thesis\isanewline
\ \ \ \ \isacommand{using}\isamarkupfalse%
\ assms\ left{\isacharunderscore}{\kern0pt}eq\ right{\isacharunderscore}{\kern0pt}eq\ uniqueness\ \isacommand{by}\isamarkupfalse%
\ {\isacharparenleft}{\kern0pt}typecheck{\isacharunderscore}{\kern0pt}cfuncs{\isacharcomma}{\kern0pt}\ auto{\isacharparenright}{\kern0pt}\isanewline
\isacommand{qed}\isamarkupfalse%
%
\endisatagproof
{\isafoldproof}%
%
\isadelimproof
\isanewline
%
\endisadelimproof
\isanewline
\isacommand{lemma}\isamarkupfalse%
\ coproduct{\isacharunderscore}{\kern0pt}of{\isacharunderscore}{\kern0pt}beta{\isacharcolon}{\kern0pt}\isanewline
\ \ {\isachardoublequoteopen}{\isasymbeta}\isactrlbsub X\isactrlesub \ {\isasymamalg}\ {\isasymbeta}\isactrlbsub Y\isactrlesub \ {\isacharequal}{\kern0pt}\ {\isasymbeta}\isactrlbsub X{\isasymCoprod}Y\isactrlesub {\isachardoublequoteclose}\isanewline
%
\isadelimproof
\ \ %
\endisadelimproof
%
\isatagproof
\isacommand{by}\isamarkupfalse%
\ {\isacharparenleft}{\kern0pt}metis\ {\isacharparenleft}{\kern0pt}full{\isacharunderscore}{\kern0pt}types{\isacharparenright}{\kern0pt}\ cfunc{\isacharunderscore}{\kern0pt}coprod{\isacharunderscore}{\kern0pt}unique\ left{\isacharunderscore}{\kern0pt}proj{\isacharunderscore}{\kern0pt}type\ right{\isacharunderscore}{\kern0pt}proj{\isacharunderscore}{\kern0pt}type\ terminal{\isacharunderscore}{\kern0pt}func{\isacharunderscore}{\kern0pt}comp\ terminal{\isacharunderscore}{\kern0pt}func{\isacharunderscore}{\kern0pt}type{\isacharparenright}{\kern0pt}%
\endisatagproof
{\isafoldproof}%
%
\isadelimproof
\isanewline
%
\endisadelimproof
\isanewline
\isacommand{lemma}\isamarkupfalse%
\ cfunc{\isacharunderscore}{\kern0pt}bowtieprod{\isacharunderscore}{\kern0pt}comp{\isacharunderscore}{\kern0pt}cfunc{\isacharunderscore}{\kern0pt}coprod{\isacharcolon}{\kern0pt}\isanewline
\ \ \isakeyword{assumes}\ a{\isacharunderscore}{\kern0pt}type{\isacharcolon}{\kern0pt}\ {\isachardoublequoteopen}a\ {\isacharcolon}{\kern0pt}\ Y\ {\isasymrightarrow}\ Z{\isachardoublequoteclose}\ \isakeyword{and}\ b{\isacharunderscore}{\kern0pt}type{\isacharcolon}{\kern0pt}\ {\isachardoublequoteopen}b\ {\isacharcolon}{\kern0pt}\ W\ {\isasymrightarrow}\ Z{\isachardoublequoteclose}\isanewline
\ \ \isakeyword{assumes}\ f{\isacharunderscore}{\kern0pt}type{\isacharcolon}{\kern0pt}\ {\isachardoublequoteopen}f\ {\isacharcolon}{\kern0pt}\ X\ {\isasymrightarrow}\ Y{\isachardoublequoteclose}\ \isakeyword{and}\ g{\isacharunderscore}{\kern0pt}type{\isacharcolon}{\kern0pt}\ {\isachardoublequoteopen}g\ {\isacharcolon}{\kern0pt}\ V\ {\isasymrightarrow}\ W{\isachardoublequoteclose}\isanewline
\ \ \isakeyword{shows}\ {\isachardoublequoteopen}{\isacharparenleft}{\kern0pt}a\ {\isasymamalg}\ b{\isacharparenright}{\kern0pt}\ {\isasymcirc}\isactrlsub c\ {\isacharparenleft}{\kern0pt}f\ {\isasymbowtie}\isactrlsub f\ g{\isacharparenright}{\kern0pt}\ {\isacharequal}{\kern0pt}\ {\isacharparenleft}{\kern0pt}a\ {\isasymcirc}\isactrlsub c\ f{\isacharparenright}{\kern0pt}\ {\isasymamalg}\ {\isacharparenleft}{\kern0pt}b\ {\isasymcirc}\isactrlsub c\ g{\isacharparenright}{\kern0pt}{\isachardoublequoteclose}\isanewline
%
\isadelimproof
%
\endisadelimproof
%
\isatagproof
\isacommand{proof}\isamarkupfalse%
\ {\isacharminus}{\kern0pt}\ \isanewline
\ \ \isacommand{from}\isamarkupfalse%
\ cfunc{\isacharunderscore}{\kern0pt}bowtie{\isacharunderscore}{\kern0pt}prod{\isacharunderscore}{\kern0pt}unique\ \isacommand{have}\isamarkupfalse%
\ uniqueness{\isacharcolon}{\kern0pt}\isanewline
\ \ \ \ {\isachardoublequoteopen}{\isasymforall}h{\isachardot}{\kern0pt}\ h\ {\isacharcolon}{\kern0pt}\ X\ {\isasymCoprod}\ V\ {\isasymrightarrow}\ Z\ {\isasymand}\ h\ {\isasymcirc}\isactrlsub c\ left{\isacharunderscore}{\kern0pt}coproj\ X\ V\ \ \ {\isacharequal}{\kern0pt}\ a\ {\isasymcirc}\isactrlsub c\ f\ {\isasymand}\ h\ {\isasymcirc}\isactrlsub c\ right{\isacharunderscore}{\kern0pt}coproj\ X\ V\ \ {\isacharequal}{\kern0pt}\ b\ {\isasymcirc}\isactrlsub c\ g\ {\isasymlongrightarrow}\ \isanewline
\ \ \ \ \ \ h\ {\isacharequal}{\kern0pt}\ {\isacharparenleft}{\kern0pt}a\ {\isasymcirc}\isactrlsub c\ f{\isacharparenright}{\kern0pt}\ {\isasymamalg}\ {\isacharparenleft}{\kern0pt}b\ {\isasymcirc}\isactrlsub c\ g{\isacharparenright}{\kern0pt}{\isachardoublequoteclose}\isanewline
\ \ \ \ \isacommand{using}\isamarkupfalse%
\ assms\ comp{\isacharunderscore}{\kern0pt}type\ \isacommand{by}\isamarkupfalse%
\ {\isacharparenleft}{\kern0pt}metis\ {\isacharparenleft}{\kern0pt}full{\isacharunderscore}{\kern0pt}types{\isacharparenright}{\kern0pt}\ cfunc{\isacharunderscore}{\kern0pt}coprod{\isacharunderscore}{\kern0pt}unique{\isacharparenright}{\kern0pt}\ \isanewline
\isanewline
\ \ \isacommand{have}\isamarkupfalse%
\ left{\isacharunderscore}{\kern0pt}eq{\isacharcolon}{\kern0pt}\ {\isachardoublequoteopen}{\isacharparenleft}{\kern0pt}a\ {\isasymamalg}\ b\ {\isasymcirc}\isactrlsub c\ f\ {\isasymbowtie}\isactrlsub f\ g{\isacharparenright}{\kern0pt}\ {\isasymcirc}\isactrlsub c\ left{\isacharunderscore}{\kern0pt}coproj\ X\ V\ {\isacharequal}{\kern0pt}\ {\isacharparenleft}{\kern0pt}a\ {\isasymcirc}\isactrlsub c\ f{\isacharparenright}{\kern0pt}{\isachardoublequoteclose}\isanewline
\ \ \isacommand{proof}\isamarkupfalse%
\ {\isacharminus}{\kern0pt}\ \isanewline
\ \ \ \ \isacommand{have}\isamarkupfalse%
\ {\isachardoublequoteopen}{\isacharparenleft}{\kern0pt}a\ {\isasymamalg}\ b\ {\isasymcirc}\isactrlsub c\ f\ {\isasymbowtie}\isactrlsub f\ g{\isacharparenright}{\kern0pt}\ {\isasymcirc}\isactrlsub c\ left{\isacharunderscore}{\kern0pt}coproj\ X\ V\ {\isacharequal}{\kern0pt}\ {\isacharparenleft}{\kern0pt}a\ {\isasymamalg}\ b{\isacharparenright}{\kern0pt}\ {\isasymcirc}\isactrlsub c\ {\isacharparenleft}{\kern0pt}f\ {\isasymbowtie}\isactrlsub f\ g{\isacharparenright}{\kern0pt}\ \ {\isasymcirc}\isactrlsub c\ left{\isacharunderscore}{\kern0pt}coproj\ X\ V{\isachardoublequoteclose}\isanewline
\ \ \ \ \ \ \isacommand{using}\isamarkupfalse%
\ assms\ \isacommand{by}\isamarkupfalse%
\ {\isacharparenleft}{\kern0pt}typecheck{\isacharunderscore}{\kern0pt}cfuncs{\isacharcomma}{\kern0pt}\ simp\ add{\isacharcolon}{\kern0pt}\ comp{\isacharunderscore}{\kern0pt}associative{\isadigit{2}}{\isacharparenright}{\kern0pt}\isanewline
\ \ \ \ \isacommand{also}\isamarkupfalse%
\ \isacommand{have}\isamarkupfalse%
\ {\isachardoublequoteopen}{\isachardot}{\kern0pt}{\isachardot}{\kern0pt}{\isachardot}{\kern0pt}\ {\isacharequal}{\kern0pt}\ {\isacharparenleft}{\kern0pt}a\ {\isasymamalg}\ b{\isacharparenright}{\kern0pt}\ \ {\isasymcirc}\isactrlsub c\ left{\isacharunderscore}{\kern0pt}coproj\ Y\ W\ {\isasymcirc}\isactrlsub c\ f{\isachardoublequoteclose}\isanewline
\ \ \ \ \ \ \isacommand{using}\isamarkupfalse%
\ f{\isacharunderscore}{\kern0pt}type\ g{\isacharunderscore}{\kern0pt}type\ left{\isacharunderscore}{\kern0pt}coproj{\isacharunderscore}{\kern0pt}cfunc{\isacharunderscore}{\kern0pt}bowtie{\isacharunderscore}{\kern0pt}prod\ \isacommand{by}\isamarkupfalse%
\ auto\isanewline
\ \ \ \ \isacommand{also}\isamarkupfalse%
\ \isacommand{have}\isamarkupfalse%
\ {\isachardoublequoteopen}{\isachardot}{\kern0pt}{\isachardot}{\kern0pt}{\isachardot}{\kern0pt}\ {\isacharequal}{\kern0pt}\ {\isacharparenleft}{\kern0pt}{\isacharparenleft}{\kern0pt}a\ {\isasymamalg}\ b{\isacharparenright}{\kern0pt}\ \ {\isasymcirc}\isactrlsub c\ left{\isacharunderscore}{\kern0pt}coproj\ Y\ W{\isacharparenright}{\kern0pt}\ {\isasymcirc}\isactrlsub c\ f{\isachardoublequoteclose}\isanewline
\ \ \ \ \ \ \isacommand{using}\isamarkupfalse%
\ a{\isacharunderscore}{\kern0pt}type\ assms{\isacharparenleft}{\kern0pt}{\isadigit{2}}{\isacharparenright}{\kern0pt}\ cfunc{\isacharunderscore}{\kern0pt}type{\isacharunderscore}{\kern0pt}def\ comp{\isacharunderscore}{\kern0pt}associative\ f{\isacharunderscore}{\kern0pt}type\ \isacommand{by}\isamarkupfalse%
\ {\isacharparenleft}{\kern0pt}typecheck{\isacharunderscore}{\kern0pt}cfuncs{\isacharcomma}{\kern0pt}\ auto{\isacharparenright}{\kern0pt}\isanewline
\ \ \ \ \isacommand{also}\isamarkupfalse%
\ \isacommand{have}\isamarkupfalse%
\ {\isachardoublequoteopen}{\isachardot}{\kern0pt}{\isachardot}{\kern0pt}{\isachardot}{\kern0pt}\ {\isacharequal}{\kern0pt}\ {\isacharparenleft}{\kern0pt}a\ {\isasymcirc}\isactrlsub c\ f{\isacharparenright}{\kern0pt}{\isachardoublequoteclose}\isanewline
\ \ \ \ \ \ \isacommand{using}\isamarkupfalse%
\ a{\isacharunderscore}{\kern0pt}type\ b{\isacharunderscore}{\kern0pt}type\ left{\isacharunderscore}{\kern0pt}coproj{\isacharunderscore}{\kern0pt}cfunc{\isacharunderscore}{\kern0pt}coprod\ \isacommand{by}\isamarkupfalse%
\ presburger\isanewline
\ \ \ \ \isacommand{then}\isamarkupfalse%
\ \isacommand{show}\isamarkupfalse%
\ \ {\isachardoublequoteopen}{\isacharparenleft}{\kern0pt}a\ {\isasymamalg}\ b\ {\isasymcirc}\isactrlsub c\ f\ {\isasymbowtie}\isactrlsub f\ g{\isacharparenright}{\kern0pt}\ {\isasymcirc}\isactrlsub c\ left{\isacharunderscore}{\kern0pt}coproj\ X\ V\ {\isacharequal}{\kern0pt}\ {\isacharparenleft}{\kern0pt}a\ {\isasymcirc}\isactrlsub c\ f{\isacharparenright}{\kern0pt}{\isachardoublequoteclose}\isanewline
\ \ \ \ \ \ \isacommand{by}\isamarkupfalse%
\ {\isacharparenleft}{\kern0pt}simp\ add{\isacharcolon}{\kern0pt}\ calculation{\isacharparenright}{\kern0pt}\isanewline
\ \ \isacommand{qed}\isamarkupfalse%
\isanewline
\isanewline
\ \ \isacommand{have}\isamarkupfalse%
\ right{\isacharunderscore}{\kern0pt}eq{\isacharcolon}{\kern0pt}\ {\isachardoublequoteopen}{\isacharparenleft}{\kern0pt}a\ {\isasymamalg}\ b\ {\isasymcirc}\isactrlsub c\ f\ {\isasymbowtie}\isactrlsub f\ g{\isacharparenright}{\kern0pt}\ {\isasymcirc}\isactrlsub c\ right{\isacharunderscore}{\kern0pt}coproj\ X\ V\ {\isacharequal}{\kern0pt}\ {\isacharparenleft}{\kern0pt}b\ {\isasymcirc}\isactrlsub c\ g{\isacharparenright}{\kern0pt}{\isachardoublequoteclose}\isanewline
\ \ \isacommand{proof}\isamarkupfalse%
\ {\isacharminus}{\kern0pt}\ \isanewline
\ \ \ \ \isacommand{have}\isamarkupfalse%
\ {\isachardoublequoteopen}{\isacharparenleft}{\kern0pt}a\ {\isasymamalg}\ b\ {\isasymcirc}\isactrlsub c\ f\ {\isasymbowtie}\isactrlsub f\ g{\isacharparenright}{\kern0pt}\ {\isasymcirc}\isactrlsub c\ right{\isacharunderscore}{\kern0pt}coproj\ X\ V\ {\isacharequal}{\kern0pt}\ {\isacharparenleft}{\kern0pt}a\ {\isasymamalg}\ b{\isacharparenright}{\kern0pt}\ {\isasymcirc}\isactrlsub c\ {\isacharparenleft}{\kern0pt}f\ {\isasymbowtie}\isactrlsub f\ g{\isacharparenright}{\kern0pt}\ {\isasymcirc}\isactrlsub c\ right{\isacharunderscore}{\kern0pt}coproj\ X\ V{\isachardoublequoteclose}\isanewline
\ \ \ \ \ \ \isacommand{using}\isamarkupfalse%
\ assms\ \isacommand{by}\isamarkupfalse%
\ {\isacharparenleft}{\kern0pt}typecheck{\isacharunderscore}{\kern0pt}cfuncs{\isacharcomma}{\kern0pt}\ simp\ add{\isacharcolon}{\kern0pt}\ comp{\isacharunderscore}{\kern0pt}associative{\isadigit{2}}{\isacharparenright}{\kern0pt}\isanewline
\ \ \ \ \isacommand{also}\isamarkupfalse%
\ \isacommand{have}\isamarkupfalse%
\ {\isachardoublequoteopen}{\isachardot}{\kern0pt}{\isachardot}{\kern0pt}{\isachardot}{\kern0pt}\ {\isacharequal}{\kern0pt}\ {\isacharparenleft}{\kern0pt}a\ {\isasymamalg}\ b{\isacharparenright}{\kern0pt}\ {\isasymcirc}\isactrlsub c\ right{\isacharunderscore}{\kern0pt}coproj\ Y\ W\ {\isasymcirc}\isactrlsub c\ g{\isachardoublequoteclose}\isanewline
\ \ \ \ \ \ \isacommand{using}\isamarkupfalse%
\ f{\isacharunderscore}{\kern0pt}type\ g{\isacharunderscore}{\kern0pt}type\ right{\isacharunderscore}{\kern0pt}coproj{\isacharunderscore}{\kern0pt}cfunc{\isacharunderscore}{\kern0pt}bowtie{\isacharunderscore}{\kern0pt}prod\ \isacommand{by}\isamarkupfalse%
\ auto\isanewline
\ \ \ \ \isacommand{also}\isamarkupfalse%
\ \isacommand{have}\isamarkupfalse%
\ {\isachardoublequoteopen}{\isachardot}{\kern0pt}{\isachardot}{\kern0pt}{\isachardot}{\kern0pt}\ {\isacharequal}{\kern0pt}\ {\isacharparenleft}{\kern0pt}{\isacharparenleft}{\kern0pt}a\ {\isasymamalg}\ b{\isacharparenright}{\kern0pt}\ {\isasymcirc}\isactrlsub c\ right{\isacharunderscore}{\kern0pt}coproj\ Y\ W{\isacharparenright}{\kern0pt}\ {\isasymcirc}\isactrlsub c\ g{\isachardoublequoteclose}\isanewline
\ \ \ \ \ \ \isacommand{using}\isamarkupfalse%
\ a{\isacharunderscore}{\kern0pt}type\ assms{\isacharparenleft}{\kern0pt}{\isadigit{2}}{\isacharparenright}{\kern0pt}\ cfunc{\isacharunderscore}{\kern0pt}type{\isacharunderscore}{\kern0pt}def\ comp{\isacharunderscore}{\kern0pt}associative\ g{\isacharunderscore}{\kern0pt}type\ \isacommand{by}\isamarkupfalse%
\ {\isacharparenleft}{\kern0pt}typecheck{\isacharunderscore}{\kern0pt}cfuncs{\isacharcomma}{\kern0pt}\ auto{\isacharparenright}{\kern0pt}\isanewline
\ \ \ \ \isacommand{also}\isamarkupfalse%
\ \isacommand{have}\isamarkupfalse%
\ {\isachardoublequoteopen}{\isachardot}{\kern0pt}{\isachardot}{\kern0pt}{\isachardot}{\kern0pt}\ {\isacharequal}{\kern0pt}\ {\isacharparenleft}{\kern0pt}b\ {\isasymcirc}\isactrlsub c\ g{\isacharparenright}{\kern0pt}{\isachardoublequoteclose}\isanewline
\ \ \ \ \ \ \isacommand{using}\isamarkupfalse%
\ a{\isacharunderscore}{\kern0pt}type\ b{\isacharunderscore}{\kern0pt}type\ right{\isacharunderscore}{\kern0pt}coproj{\isacharunderscore}{\kern0pt}cfunc{\isacharunderscore}{\kern0pt}coprod\ \isacommand{by}\isamarkupfalse%
\ auto\isanewline
\ \ \ \ \isacommand{then}\isamarkupfalse%
\ \isacommand{show}\isamarkupfalse%
\ {\isachardoublequoteopen}{\isacharparenleft}{\kern0pt}a\ {\isasymamalg}\ b\ {\isasymcirc}\isactrlsub c\ f\ {\isasymbowtie}\isactrlsub f\ g{\isacharparenright}{\kern0pt}\ {\isasymcirc}\isactrlsub c\ right{\isacharunderscore}{\kern0pt}coproj\ X\ V\ {\isacharequal}{\kern0pt}\ {\isacharparenleft}{\kern0pt}b\ {\isasymcirc}\isactrlsub c\ g{\isacharparenright}{\kern0pt}{\isachardoublequoteclose}\isanewline
\ \ \ \ \ \ \isacommand{by}\isamarkupfalse%
\ {\isacharparenleft}{\kern0pt}simp\ add{\isacharcolon}{\kern0pt}\ calculation{\isacharparenright}{\kern0pt}\isanewline
\ \ \isacommand{qed}\isamarkupfalse%
\isanewline
\isanewline
\ \ \isacommand{show}\isamarkupfalse%
\ {\isachardoublequoteopen}{\isacharparenleft}{\kern0pt}a\ {\isasymamalg}\ b{\isacharparenright}{\kern0pt}\ {\isasymcirc}\isactrlsub c\ {\isacharparenleft}{\kern0pt}f\ {\isasymbowtie}\isactrlsub f\ g{\isacharparenright}{\kern0pt}\ {\isacharequal}{\kern0pt}\ {\isacharparenleft}{\kern0pt}a\ {\isasymcirc}\isactrlsub c\ f{\isacharparenright}{\kern0pt}\ {\isasymamalg}\ {\isacharparenleft}{\kern0pt}b\ {\isasymcirc}\isactrlsub c\ g{\isacharparenright}{\kern0pt}{\isachardoublequoteclose}\isanewline
\ \ \ \ \isacommand{using}\isamarkupfalse%
\ uniqueness\ left{\isacharunderscore}{\kern0pt}eq\ right{\isacharunderscore}{\kern0pt}eq\ assms\isanewline
\ \ \ \ \isacommand{by}\isamarkupfalse%
\ {\isacharparenleft}{\kern0pt}typecheck{\isacharunderscore}{\kern0pt}cfuncs{\isacharcomma}{\kern0pt}\ erule{\isacharunderscore}{\kern0pt}tac\ x{\isacharequal}{\kern0pt}{\isachardoublequoteopen}{\isacharparenleft}{\kern0pt}a\ {\isasymamalg}\ b{\isacharparenright}{\kern0pt}\ {\isasymcirc}\isactrlsub c\ {\isacharparenleft}{\kern0pt}f\ {\isasymbowtie}\isactrlsub f\ g{\isacharparenright}{\kern0pt}{\isachardoublequoteclose}\ \isakeyword{in}\ allE{\isacharcomma}{\kern0pt}\ auto{\isacharparenright}{\kern0pt}\isanewline
\isacommand{qed}\isamarkupfalse%
%
\endisatagproof
{\isafoldproof}%
%
\isadelimproof
\isanewline
%
\endisadelimproof
\isanewline
\isacommand{lemma}\isamarkupfalse%
\ id{\isacharunderscore}{\kern0pt}bowtie{\isacharunderscore}{\kern0pt}prod{\isacharcolon}{\kern0pt}\ {\isachardoublequoteopen}id{\isacharparenleft}{\kern0pt}X{\isacharparenright}{\kern0pt}\ {\isasymbowtie}\isactrlsub f\ id{\isacharparenleft}{\kern0pt}Y{\isacharparenright}{\kern0pt}\ {\isacharequal}{\kern0pt}\ id{\isacharparenleft}{\kern0pt}X\ {\isasymCoprod}\ Y{\isacharparenright}{\kern0pt}{\isachardoublequoteclose}\isanewline
%
\isadelimproof
\ \ %
\endisadelimproof
%
\isatagproof
\isacommand{by}\isamarkupfalse%
\ {\isacharparenleft}{\kern0pt}metis\ cfunc{\isacharunderscore}{\kern0pt}bowtie{\isacharunderscore}{\kern0pt}prod{\isacharunderscore}{\kern0pt}def\ id{\isacharunderscore}{\kern0pt}codomain\ id{\isacharunderscore}{\kern0pt}coprod\ id{\isacharunderscore}{\kern0pt}right{\isacharunderscore}{\kern0pt}unit{\isadigit{2}}\ left{\isacharunderscore}{\kern0pt}proj{\isacharunderscore}{\kern0pt}type\ right{\isacharunderscore}{\kern0pt}proj{\isacharunderscore}{\kern0pt}type{\isacharparenright}{\kern0pt}%
\endisatagproof
{\isafoldproof}%
%
\isadelimproof
\isanewline
%
\endisadelimproof
\isanewline
\isacommand{lemma}\isamarkupfalse%
\ cfunc{\isacharunderscore}{\kern0pt}bowtie{\isacharunderscore}{\kern0pt}prod{\isacharunderscore}{\kern0pt}comp{\isacharunderscore}{\kern0pt}cfunc{\isacharunderscore}{\kern0pt}bowtie{\isacharunderscore}{\kern0pt}prod{\isacharcolon}{\kern0pt}\isanewline
\ \ \isakeyword{assumes}\ {\isachardoublequoteopen}f\ {\isacharcolon}{\kern0pt}\ X\ {\isasymrightarrow}\ Y{\isachardoublequoteclose}\ {\isachardoublequoteopen}g\ {\isacharcolon}{\kern0pt}\ V\ {\isasymrightarrow}\ W{\isachardoublequoteclose}\ {\isachardoublequoteopen}x\ {\isacharcolon}{\kern0pt}\ Y\ {\isasymrightarrow}\ S{\isachardoublequoteclose}\ {\isachardoublequoteopen}y\ {\isacharcolon}{\kern0pt}\ W\ {\isasymrightarrow}\ T{\isachardoublequoteclose}\isanewline
\ \ \isakeyword{shows}\ {\isachardoublequoteopen}{\isacharparenleft}{\kern0pt}x\ {\isasymbowtie}\isactrlsub f\ y{\isacharparenright}{\kern0pt}\ {\isasymcirc}\isactrlsub c\ {\isacharparenleft}{\kern0pt}f\ {\isasymbowtie}\isactrlsub f\ g{\isacharparenright}{\kern0pt}\ {\isacharequal}{\kern0pt}\ {\isacharparenleft}{\kern0pt}x\ {\isasymcirc}\isactrlsub c\ f{\isacharparenright}{\kern0pt}\ {\isasymbowtie}\isactrlsub f\ {\isacharparenleft}{\kern0pt}y\ {\isasymcirc}\isactrlsub c\ g{\isacharparenright}{\kern0pt}{\isachardoublequoteclose}\isanewline
%
\isadelimproof
%
\endisadelimproof
%
\isatagproof
\isacommand{proof}\isamarkupfalse%
{\isacharminus}{\kern0pt}\ \isanewline
\ \ \isacommand{have}\isamarkupfalse%
\ {\isachardoublequoteopen}{\isacharparenleft}{\kern0pt}x\ {\isasymbowtie}\isactrlsub f\ y{\isacharparenright}{\kern0pt}\ {\isasymcirc}\isactrlsub c\ {\isacharparenleft}{\kern0pt}{\isacharparenleft}{\kern0pt}left{\isacharunderscore}{\kern0pt}coproj\ Y\ W\ {\isasymcirc}\isactrlsub c\ f{\isacharparenright}{\kern0pt}\ {\isasymamalg}\ {\isacharparenleft}{\kern0pt}right{\isacharunderscore}{\kern0pt}coproj\ Y\ W\ {\isasymcirc}\isactrlsub c\ g{\isacharparenright}{\kern0pt}{\isacharparenright}{\kern0pt}\isanewline
\ \ \ \ \ \ {\isacharequal}{\kern0pt}\ {\isacharparenleft}{\kern0pt}{\isacharparenleft}{\kern0pt}x\ {\isasymbowtie}\isactrlsub f\ y{\isacharparenright}{\kern0pt}\ {\isasymcirc}\isactrlsub c\ left{\isacharunderscore}{\kern0pt}coproj\ Y\ W\ {\isasymcirc}\isactrlsub c\ f{\isacharparenright}{\kern0pt}\ {\isasymamalg}\ {\isacharparenleft}{\kern0pt}{\isacharparenleft}{\kern0pt}x\ {\isasymbowtie}\isactrlsub f\ y{\isacharparenright}{\kern0pt}\ {\isasymcirc}\isactrlsub c\ right{\isacharunderscore}{\kern0pt}coproj\ Y\ W\ {\isasymcirc}\isactrlsub c\ g{\isacharparenright}{\kern0pt}{\isachardoublequoteclose}\isanewline
\ \ \ \ \isacommand{using}\isamarkupfalse%
\ assms\ \isacommand{by}\isamarkupfalse%
\ {\isacharparenleft}{\kern0pt}typecheck{\isacharunderscore}{\kern0pt}cfuncs{\isacharcomma}{\kern0pt}\ simp\ add{\isacharcolon}{\kern0pt}\ cfunc{\isacharunderscore}{\kern0pt}coprod{\isacharunderscore}{\kern0pt}comp{\isacharparenright}{\kern0pt}\isanewline
\ \ \isacommand{also}\isamarkupfalse%
\ \isacommand{have}\isamarkupfalse%
\ {\isachardoublequoteopen}{\isachardot}{\kern0pt}{\isachardot}{\kern0pt}{\isachardot}{\kern0pt}\ {\isacharequal}{\kern0pt}\ {\isacharparenleft}{\kern0pt}{\isacharparenleft}{\kern0pt}{\isacharparenleft}{\kern0pt}x\ {\isasymbowtie}\isactrlsub f\ y{\isacharparenright}{\kern0pt}\ {\isasymcirc}\isactrlsub c\ left{\isacharunderscore}{\kern0pt}coproj\ Y\ W{\isacharparenright}{\kern0pt}\ {\isasymcirc}\isactrlsub c\ f{\isacharparenright}{\kern0pt}\ {\isasymamalg}\ {\isacharparenleft}{\kern0pt}{\isacharparenleft}{\kern0pt}{\isacharparenleft}{\kern0pt}x\ {\isasymbowtie}\isactrlsub f\ y{\isacharparenright}{\kern0pt}\ {\isasymcirc}\isactrlsub c\ right{\isacharunderscore}{\kern0pt}coproj\ Y\ W{\isacharparenright}{\kern0pt}\ {\isasymcirc}\isactrlsub c\ g{\isacharparenright}{\kern0pt}{\isachardoublequoteclose}\isanewline
\ \ \ \ \isacommand{using}\isamarkupfalse%
\ assms\ \isacommand{by}\isamarkupfalse%
\ {\isacharparenleft}{\kern0pt}typecheck{\isacharunderscore}{\kern0pt}cfuncs{\isacharcomma}{\kern0pt}\ simp\ add{\isacharcolon}{\kern0pt}\ comp{\isacharunderscore}{\kern0pt}associative{\isadigit{2}}{\isacharparenright}{\kern0pt}\isanewline
\ \ \isacommand{also}\isamarkupfalse%
\ \isacommand{have}\isamarkupfalse%
\ {\isachardoublequoteopen}{\isachardot}{\kern0pt}{\isachardot}{\kern0pt}{\isachardot}{\kern0pt}\ {\isacharequal}{\kern0pt}\ {\isacharparenleft}{\kern0pt}{\isacharparenleft}{\kern0pt}left{\isacharunderscore}{\kern0pt}coproj\ S\ T\ {\isasymcirc}\isactrlsub c\ x{\isacharparenright}{\kern0pt}\ {\isasymcirc}\isactrlsub c\ f{\isacharparenright}{\kern0pt}\ {\isasymamalg}\ {\isacharparenleft}{\kern0pt}{\isacharparenleft}{\kern0pt}right{\isacharunderscore}{\kern0pt}coproj\ S\ T\ {\isasymcirc}\isactrlsub c\ y{\isacharparenright}{\kern0pt}\ {\isasymcirc}\isactrlsub c\ g{\isacharparenright}{\kern0pt}{\isachardoublequoteclose}\isanewline
\ \ \ \ \isacommand{using}\isamarkupfalse%
\ assms{\isacharparenleft}{\kern0pt}{\isadigit{3}}{\isacharparenright}{\kern0pt}\ assms{\isacharparenleft}{\kern0pt}{\isadigit{4}}{\isacharparenright}{\kern0pt}\ left{\isacharunderscore}{\kern0pt}coproj{\isacharunderscore}{\kern0pt}cfunc{\isacharunderscore}{\kern0pt}bowtie{\isacharunderscore}{\kern0pt}prod\ right{\isacharunderscore}{\kern0pt}coproj{\isacharunderscore}{\kern0pt}cfunc{\isacharunderscore}{\kern0pt}bowtie{\isacharunderscore}{\kern0pt}prod\ \isacommand{by}\isamarkupfalse%
\ auto\isanewline
\ \ \isacommand{also}\isamarkupfalse%
\ \isacommand{have}\isamarkupfalse%
\ {\isachardoublequoteopen}{\isachardot}{\kern0pt}{\isachardot}{\kern0pt}{\isachardot}{\kern0pt}\ {\isacharequal}{\kern0pt}\ {\isacharparenleft}{\kern0pt}left{\isacharunderscore}{\kern0pt}coproj\ S\ T\ {\isasymcirc}\isactrlsub c\ x\ {\isasymcirc}\isactrlsub c\ f{\isacharparenright}{\kern0pt}\ {\isasymamalg}\ {\isacharparenleft}{\kern0pt}right{\isacharunderscore}{\kern0pt}coproj\ S\ T\ {\isasymcirc}\isactrlsub c\ y\ {\isasymcirc}\isactrlsub c\ g{\isacharparenright}{\kern0pt}{\isachardoublequoteclose}\isanewline
\ \ \ \ \isacommand{using}\isamarkupfalse%
\ assms\ \isacommand{by}\isamarkupfalse%
\ {\isacharparenleft}{\kern0pt}typecheck{\isacharunderscore}{\kern0pt}cfuncs{\isacharcomma}{\kern0pt}\ simp\ add{\isacharcolon}{\kern0pt}\ comp{\isacharunderscore}{\kern0pt}associative{\isadigit{2}}{\isacharparenright}{\kern0pt}\isanewline
\ \ \isacommand{also}\isamarkupfalse%
\ \isacommand{have}\isamarkupfalse%
\ {\isachardoublequoteopen}{\isachardot}{\kern0pt}{\isachardot}{\kern0pt}{\isachardot}{\kern0pt}\ {\isacharequal}{\kern0pt}\ {\isacharparenleft}{\kern0pt}x\ {\isasymcirc}\isactrlsub c\ f{\isacharparenright}{\kern0pt}\ {\isasymbowtie}\isactrlsub f\ {\isacharparenleft}{\kern0pt}y\ {\isasymcirc}\isactrlsub c\ g{\isacharparenright}{\kern0pt}{\isachardoublequoteclose}\isanewline
\ \ \ \ \isacommand{using}\isamarkupfalse%
\ assms\ cfunc{\isacharunderscore}{\kern0pt}bowtie{\isacharunderscore}{\kern0pt}prod{\isacharunderscore}{\kern0pt}def\ cfunc{\isacharunderscore}{\kern0pt}type{\isacharunderscore}{\kern0pt}def\ codomain{\isacharunderscore}{\kern0pt}comp\ \isacommand{by}\isamarkupfalse%
\ auto\isanewline
\ \ \isacommand{then}\isamarkupfalse%
\ \isacommand{show}\isamarkupfalse%
\ {\isachardoublequoteopen}{\isacharparenleft}{\kern0pt}x\ {\isasymbowtie}\isactrlsub f\ y{\isacharparenright}{\kern0pt}\ {\isasymcirc}\isactrlsub c\ {\isacharparenleft}{\kern0pt}f\ {\isasymbowtie}\isactrlsub f\ g{\isacharparenright}{\kern0pt}\ {\isacharequal}{\kern0pt}\ {\isacharparenleft}{\kern0pt}x\ {\isasymcirc}\isactrlsub c\ f{\isacharparenright}{\kern0pt}\ {\isasymbowtie}\isactrlsub f\ {\isacharparenleft}{\kern0pt}y\ {\isasymcirc}\isactrlsub c\ g{\isacharparenright}{\kern0pt}{\isachardoublequoteclose}\isanewline
\ \ \ \ \isacommand{using}\isamarkupfalse%
\ assms{\isacharparenleft}{\kern0pt}{\isadigit{1}}{\isacharparenright}{\kern0pt}\ assms{\isacharparenleft}{\kern0pt}{\isadigit{2}}{\isacharparenright}{\kern0pt}\ calculation\ cfunc{\isacharunderscore}{\kern0pt}bowtie{\isacharunderscore}{\kern0pt}prod{\isacharunderscore}{\kern0pt}def{\isadigit{2}}\ \isacommand{by}\isamarkupfalse%
\ auto\isanewline
\isacommand{qed}\isamarkupfalse%
%
\endisatagproof
{\isafoldproof}%
%
\isadelimproof
\isanewline
%
\endisadelimproof
\isanewline
\isacommand{lemma}\isamarkupfalse%
\ cfunc{\isacharunderscore}{\kern0pt}bowtieprod{\isacharunderscore}{\kern0pt}epi{\isacharcolon}{\kern0pt}\isanewline
\ \ \isakeyword{assumes}\ type{\isacharunderscore}{\kern0pt}assms{\isacharcolon}{\kern0pt}\ {\isachardoublequoteopen}f\ {\isacharcolon}{\kern0pt}\ X\ {\isasymrightarrow}\ Y{\isachardoublequoteclose}\ {\isachardoublequoteopen}g\ {\isacharcolon}{\kern0pt}\ V\ {\isasymrightarrow}\ W{\isachardoublequoteclose}\isanewline
\ \ \isakeyword{assumes}\ f{\isacharunderscore}{\kern0pt}epi{\isacharcolon}{\kern0pt}\ {\isachardoublequoteopen}epimorphism\ f{\isachardoublequoteclose}\ \isakeyword{and}\ g{\isacharunderscore}{\kern0pt}epi{\isacharcolon}{\kern0pt}\ {\isachardoublequoteopen}epimorphism\ g{\isachardoublequoteclose}\isanewline
\ \ \isakeyword{shows}\ {\isachardoublequoteopen}epimorphism\ {\isacharparenleft}{\kern0pt}f\ {\isasymbowtie}\isactrlsub f\ g{\isacharparenright}{\kern0pt}{\isachardoublequoteclose}\isanewline
%
\isadelimproof
\ \ %
\endisadelimproof
%
\isatagproof
\isacommand{using}\isamarkupfalse%
\ type{\isacharunderscore}{\kern0pt}assms\isanewline
\isacommand{proof}\isamarkupfalse%
\ {\isacharparenleft}{\kern0pt}typecheck{\isacharunderscore}{\kern0pt}cfuncs{\isacharcomma}{\kern0pt}\ unfold\ epimorphism{\isacharunderscore}{\kern0pt}def{\isadigit{3}}{\isacharcomma}{\kern0pt}\ auto{\isacharparenright}{\kern0pt}\isanewline
\ \ \isacommand{fix}\isamarkupfalse%
\ x\ y\ A\isanewline
\ \ \isacommand{assume}\isamarkupfalse%
\ x{\isacharunderscore}{\kern0pt}type{\isacharcolon}{\kern0pt}\ {\isachardoublequoteopen}x{\isacharcolon}{\kern0pt}\ Y\ {\isasymCoprod}\ W\ {\isasymrightarrow}\ A{\isachardoublequoteclose}\isanewline
\ \ \isacommand{assume}\isamarkupfalse%
\ y{\isacharunderscore}{\kern0pt}type{\isacharcolon}{\kern0pt}\ {\isachardoublequoteopen}y{\isacharcolon}{\kern0pt}\ Y\ {\isasymCoprod}\ W\ {\isasymrightarrow}\ A{\isachardoublequoteclose}\isanewline
\ \ \isacommand{assume}\isamarkupfalse%
\ eqs{\isacharcolon}{\kern0pt}\ {\isachardoublequoteopen}x\ {\isasymcirc}\isactrlsub c\ f\ {\isasymbowtie}\isactrlsub f\ g\ {\isacharequal}{\kern0pt}\ y\ {\isasymcirc}\isactrlsub c\ f\ {\isasymbowtie}\isactrlsub f\ g{\isachardoublequoteclose}\isanewline
\isanewline
\ \ \isacommand{obtain}\isamarkupfalse%
\ x{\isadigit{1}}\ x{\isadigit{2}}\ \isakeyword{where}\ x{\isacharunderscore}{\kern0pt}expand{\isacharcolon}{\kern0pt}\ {\isachardoublequoteopen}x\ {\isacharequal}{\kern0pt}\ x{\isadigit{1}}\ {\isasymamalg}\ x{\isadigit{2}}{\isachardoublequoteclose}\ \isakeyword{and}\ x{\isadigit{1}}{\isacharunderscore}{\kern0pt}x{\isadigit{2}}{\isacharunderscore}{\kern0pt}type{\isacharcolon}{\kern0pt}\ {\isachardoublequoteopen}x{\isadigit{1}}\ {\isacharcolon}{\kern0pt}\ Y\ {\isasymrightarrow}\ A{\isachardoublequoteclose}\ {\isachardoublequoteopen}x{\isadigit{2}}\ {\isacharcolon}{\kern0pt}\ W\ {\isasymrightarrow}\ A{\isachardoublequoteclose}\isanewline
\ \ \ \ \isacommand{using}\isamarkupfalse%
\ coprod{\isacharunderscore}{\kern0pt}decomp\ x{\isacharunderscore}{\kern0pt}type\ \isacommand{by}\isamarkupfalse%
\ blast\isanewline
\ \ \isacommand{obtain}\isamarkupfalse%
\ y{\isadigit{1}}\ y{\isadigit{2}}\ \isakeyword{where}\ y{\isacharunderscore}{\kern0pt}expand{\isacharcolon}{\kern0pt}\ {\isachardoublequoteopen}y\ {\isacharequal}{\kern0pt}\ y{\isadigit{1}}\ {\isasymamalg}\ y{\isadigit{2}}{\isachardoublequoteclose}\ \isakeyword{and}\ y{\isadigit{1}}{\isacharunderscore}{\kern0pt}y{\isadigit{2}}{\isacharunderscore}{\kern0pt}type{\isacharcolon}{\kern0pt}\ {\isachardoublequoteopen}y{\isadigit{1}}\ {\isacharcolon}{\kern0pt}\ Y\ {\isasymrightarrow}\ A{\isachardoublequoteclose}\ {\isachardoublequoteopen}y{\isadigit{2}}\ {\isacharcolon}{\kern0pt}\ W\ {\isasymrightarrow}\ A{\isachardoublequoteclose}\isanewline
\ \ \ \ \isacommand{using}\isamarkupfalse%
\ coprod{\isacharunderscore}{\kern0pt}decomp\ y{\isacharunderscore}{\kern0pt}type\ \isacommand{by}\isamarkupfalse%
\ blast\isanewline
\isanewline
\ \ \isacommand{have}\isamarkupfalse%
\ {\isachardoublequoteopen}{\isacharparenleft}{\kern0pt}x{\isadigit{1}}\ {\isacharequal}{\kern0pt}\ y{\isadigit{1}}{\isacharparenright}{\kern0pt}\ {\isasymand}\ {\isacharparenleft}{\kern0pt}x{\isadigit{2}}\ {\isacharequal}{\kern0pt}\ y{\isadigit{2}}{\isacharparenright}{\kern0pt}{\isachardoublequoteclose}\isanewline
\ \ \isacommand{proof}\isamarkupfalse%
{\isacharparenleft}{\kern0pt}auto{\isacharparenright}{\kern0pt}\isanewline
\ \ \ \ \isacommand{have}\isamarkupfalse%
\ {\isachardoublequoteopen}x{\isadigit{1}}\ {\isasymcirc}\isactrlsub c\ f\ {\isacharequal}{\kern0pt}\ {\isacharparenleft}{\kern0pt}{\isacharparenleft}{\kern0pt}x{\isadigit{1}}\ {\isasymamalg}\ x{\isadigit{2}}{\isacharparenright}{\kern0pt}\ {\isasymcirc}\isactrlsub c\ left{\isacharunderscore}{\kern0pt}coproj\ Y\ W{\isacharparenright}{\kern0pt}\ {\isasymcirc}\isactrlsub c\ f{\isachardoublequoteclose}\isanewline
\ \ \ \ \ \ \isacommand{using}\isamarkupfalse%
\ x{\isadigit{1}}{\isacharunderscore}{\kern0pt}x{\isadigit{2}}{\isacharunderscore}{\kern0pt}type\ left{\isacharunderscore}{\kern0pt}coproj{\isacharunderscore}{\kern0pt}cfunc{\isacharunderscore}{\kern0pt}coprod\ \isacommand{by}\isamarkupfalse%
\ auto\ \isanewline
\ \ \ \ \isacommand{also}\isamarkupfalse%
\ \isacommand{have}\isamarkupfalse%
\ {\isachardoublequoteopen}{\isachardot}{\kern0pt}{\isachardot}{\kern0pt}{\isachardot}{\kern0pt}\ {\isacharequal}{\kern0pt}\ {\isacharparenleft}{\kern0pt}x{\isadigit{1}}\ {\isasymamalg}\ x{\isadigit{2}}{\isacharparenright}{\kern0pt}\ {\isasymcirc}\isactrlsub c\ left{\isacharunderscore}{\kern0pt}coproj\ Y\ W\ {\isasymcirc}\isactrlsub c\ f{\isachardoublequoteclose}\isanewline
\ \ \ \ \ \ \isacommand{using}\isamarkupfalse%
\ assms\ comp{\isacharunderscore}{\kern0pt}associative{\isadigit{2}}\ x{\isacharunderscore}{\kern0pt}expand\ x{\isacharunderscore}{\kern0pt}type\ \isacommand{by}\isamarkupfalse%
\ {\isacharparenleft}{\kern0pt}typecheck{\isacharunderscore}{\kern0pt}cfuncs{\isacharcomma}{\kern0pt}\ auto{\isacharparenright}{\kern0pt}\isanewline
\ \ \ \ \isacommand{also}\isamarkupfalse%
\ \isacommand{have}\isamarkupfalse%
\ {\isachardoublequoteopen}{\isachardot}{\kern0pt}{\isachardot}{\kern0pt}{\isachardot}{\kern0pt}\ {\isacharequal}{\kern0pt}\ {\isacharparenleft}{\kern0pt}x{\isadigit{1}}\ {\isasymamalg}\ x{\isadigit{2}}{\isacharparenright}{\kern0pt}\ {\isasymcirc}\isactrlsub c\ {\isacharparenleft}{\kern0pt}f\ {\isasymbowtie}\isactrlsub f\ g{\isacharparenright}{\kern0pt}\ {\isasymcirc}\isactrlsub c\ left{\isacharunderscore}{\kern0pt}coproj\ X\ V{\isachardoublequoteclose}\isanewline
\ \ \ \ \ \ \isacommand{using}\isamarkupfalse%
\ left{\isacharunderscore}{\kern0pt}coproj{\isacharunderscore}{\kern0pt}cfunc{\isacharunderscore}{\kern0pt}bowtie{\isacharunderscore}{\kern0pt}prod\ type{\isacharunderscore}{\kern0pt}assms\ \isacommand{by}\isamarkupfalse%
\ force\isanewline
\ \ \ \ \isacommand{also}\isamarkupfalse%
\ \isacommand{have}\isamarkupfalse%
\ {\isachardoublequoteopen}{\isachardot}{\kern0pt}{\isachardot}{\kern0pt}{\isachardot}{\kern0pt}\ {\isacharequal}{\kern0pt}\ {\isacharparenleft}{\kern0pt}y{\isadigit{1}}\ {\isasymamalg}\ y{\isadigit{2}}{\isacharparenright}{\kern0pt}\ {\isasymcirc}\isactrlsub c\ {\isacharparenleft}{\kern0pt}f\ {\isasymbowtie}\isactrlsub f\ g{\isacharparenright}{\kern0pt}\ {\isasymcirc}\isactrlsub c\ left{\isacharunderscore}{\kern0pt}coproj\ X\ V{\isachardoublequoteclose}\isanewline
\ \ \ \ \ \ \isacommand{using}\isamarkupfalse%
\ assms\ cfunc{\isacharunderscore}{\kern0pt}type{\isacharunderscore}{\kern0pt}def\ comp{\isacharunderscore}{\kern0pt}associative\ eqs\ x{\isacharunderscore}{\kern0pt}expand\ x{\isacharunderscore}{\kern0pt}type\ y{\isacharunderscore}{\kern0pt}expand\ y{\isacharunderscore}{\kern0pt}type\ \isacommand{by}\isamarkupfalse%
\ {\isacharparenleft}{\kern0pt}typecheck{\isacharunderscore}{\kern0pt}cfuncs{\isacharcomma}{\kern0pt}\ auto{\isacharparenright}{\kern0pt}\isanewline
\ \ \ \ \isacommand{also}\isamarkupfalse%
\ \isacommand{have}\isamarkupfalse%
\ {\isachardoublequoteopen}{\isachardot}{\kern0pt}{\isachardot}{\kern0pt}{\isachardot}{\kern0pt}\ {\isacharequal}{\kern0pt}\ {\isacharparenleft}{\kern0pt}y{\isadigit{1}}\ {\isasymamalg}\ y{\isadigit{2}}{\isacharparenright}{\kern0pt}\ {\isasymcirc}\isactrlsub c\ left{\isacharunderscore}{\kern0pt}coproj\ Y\ W\ {\isasymcirc}\isactrlsub c\ f{\isachardoublequoteclose}\isanewline
\ \ \ \ \ \ \isacommand{using}\isamarkupfalse%
\ assms\ \isacommand{by}\isamarkupfalse%
\ {\isacharparenleft}{\kern0pt}typecheck{\isacharunderscore}{\kern0pt}cfuncs{\isacharcomma}{\kern0pt}\ simp\ add{\isacharcolon}{\kern0pt}\ left{\isacharunderscore}{\kern0pt}coproj{\isacharunderscore}{\kern0pt}cfunc{\isacharunderscore}{\kern0pt}bowtie{\isacharunderscore}{\kern0pt}prod{\isacharparenright}{\kern0pt}\isanewline
\ \ \ \ \isacommand{also}\isamarkupfalse%
\ \isacommand{have}\isamarkupfalse%
\ {\isachardoublequoteopen}{\isachardot}{\kern0pt}{\isachardot}{\kern0pt}{\isachardot}{\kern0pt}\ {\isacharequal}{\kern0pt}\ {\isacharparenleft}{\kern0pt}{\isacharparenleft}{\kern0pt}y{\isadigit{1}}\ {\isasymamalg}\ y{\isadigit{2}}{\isacharparenright}{\kern0pt}\ {\isasymcirc}\isactrlsub c\ left{\isacharunderscore}{\kern0pt}coproj\ Y\ W{\isacharparenright}{\kern0pt}\ {\isasymcirc}\isactrlsub c\ f{\isachardoublequoteclose}\isanewline
\ \ \ \ \ \ \isacommand{using}\isamarkupfalse%
\ assms\ comp{\isacharunderscore}{\kern0pt}associative{\isadigit{2}}\ y{\isacharunderscore}{\kern0pt}expand\ y{\isacharunderscore}{\kern0pt}type\ \isacommand{by}\isamarkupfalse%
\ {\isacharparenleft}{\kern0pt}typecheck{\isacharunderscore}{\kern0pt}cfuncs{\isacharcomma}{\kern0pt}\ blast{\isacharparenright}{\kern0pt}\isanewline
\ \ \ \ \isacommand{also}\isamarkupfalse%
\ \isacommand{have}\isamarkupfalse%
\ {\isachardoublequoteopen}{\isachardot}{\kern0pt}{\isachardot}{\kern0pt}{\isachardot}{\kern0pt}\ {\isacharequal}{\kern0pt}\ y{\isadigit{1}}\ {\isasymcirc}\isactrlsub c\ f{\isachardoublequoteclose}\isanewline
\ \ \ \ \ \ \isacommand{using}\isamarkupfalse%
\ y{\isadigit{1}}{\isacharunderscore}{\kern0pt}y{\isadigit{2}}{\isacharunderscore}{\kern0pt}type\ left{\isacharunderscore}{\kern0pt}coproj{\isacharunderscore}{\kern0pt}cfunc{\isacharunderscore}{\kern0pt}coprod\ \isacommand{by}\isamarkupfalse%
\ auto\ \isanewline
\ \ \ \ \isacommand{then}\isamarkupfalse%
\ \isacommand{show}\isamarkupfalse%
\ {\isachardoublequoteopen}x{\isadigit{1}}\ {\isacharequal}{\kern0pt}\ y{\isadigit{1}}{\isachardoublequoteclose}\isanewline
\ \ \ \ \ \ \isacommand{using}\isamarkupfalse%
\ calculation\ epimorphism{\isacharunderscore}{\kern0pt}def{\isadigit{3}}\ f{\isacharunderscore}{\kern0pt}epi\ type{\isacharunderscore}{\kern0pt}assms{\isacharparenleft}{\kern0pt}{\isadigit{1}}{\isacharparenright}{\kern0pt}\ x{\isadigit{1}}{\isacharunderscore}{\kern0pt}x{\isadigit{2}}{\isacharunderscore}{\kern0pt}type{\isacharparenleft}{\kern0pt}{\isadigit{1}}{\isacharparenright}{\kern0pt}\ y{\isadigit{1}}{\isacharunderscore}{\kern0pt}y{\isadigit{2}}{\isacharunderscore}{\kern0pt}type{\isacharparenleft}{\kern0pt}{\isadigit{1}}{\isacharparenright}{\kern0pt}\ \isacommand{by}\isamarkupfalse%
\ fastforce\isanewline
\ \ \isacommand{next}\isamarkupfalse%
\isanewline
\ \ \ \ \isacommand{have}\isamarkupfalse%
\ {\isachardoublequoteopen}x{\isadigit{2}}\ {\isasymcirc}\isactrlsub c\ g\ {\isacharequal}{\kern0pt}\ {\isacharparenleft}{\kern0pt}{\isacharparenleft}{\kern0pt}x{\isadigit{1}}\ {\isasymamalg}\ x{\isadigit{2}}{\isacharparenright}{\kern0pt}\ {\isasymcirc}\isactrlsub c\ right{\isacharunderscore}{\kern0pt}coproj\ Y\ W{\isacharparenright}{\kern0pt}\ {\isasymcirc}\isactrlsub c\ g{\isachardoublequoteclose}\isanewline
\ \ \ \ \ \ \isacommand{using}\isamarkupfalse%
\ x{\isadigit{1}}{\isacharunderscore}{\kern0pt}x{\isadigit{2}}{\isacharunderscore}{\kern0pt}type\ right{\isacharunderscore}{\kern0pt}coproj{\isacharunderscore}{\kern0pt}cfunc{\isacharunderscore}{\kern0pt}coprod\ \isacommand{by}\isamarkupfalse%
\ auto\ \isanewline
\ \ \ \ \isacommand{also}\isamarkupfalse%
\ \isacommand{have}\isamarkupfalse%
\ {\isachardoublequoteopen}{\isachardot}{\kern0pt}{\isachardot}{\kern0pt}{\isachardot}{\kern0pt}\ {\isacharequal}{\kern0pt}\ {\isacharparenleft}{\kern0pt}x{\isadigit{1}}\ {\isasymamalg}\ x{\isadigit{2}}{\isacharparenright}{\kern0pt}\ {\isasymcirc}\isactrlsub c\ right{\isacharunderscore}{\kern0pt}coproj\ Y\ W\ {\isasymcirc}\isactrlsub c\ g{\isachardoublequoteclose}\isanewline
\ \ \ \ \ \ \isacommand{using}\isamarkupfalse%
\ assms\ comp{\isacharunderscore}{\kern0pt}associative{\isadigit{2}}\ x{\isacharunderscore}{\kern0pt}expand\ x{\isacharunderscore}{\kern0pt}type\ \isacommand{by}\isamarkupfalse%
\ {\isacharparenleft}{\kern0pt}typecheck{\isacharunderscore}{\kern0pt}cfuncs{\isacharcomma}{\kern0pt}\ auto{\isacharparenright}{\kern0pt}\isanewline
\ \ \ \ \isacommand{also}\isamarkupfalse%
\ \isacommand{have}\isamarkupfalse%
\ {\isachardoublequoteopen}{\isachardot}{\kern0pt}{\isachardot}{\kern0pt}{\isachardot}{\kern0pt}\ {\isacharequal}{\kern0pt}\ {\isacharparenleft}{\kern0pt}x{\isadigit{1}}\ {\isasymamalg}\ x{\isadigit{2}}{\isacharparenright}{\kern0pt}\ {\isasymcirc}\isactrlsub c\ {\isacharparenleft}{\kern0pt}f\ {\isasymbowtie}\isactrlsub f\ g{\isacharparenright}{\kern0pt}\ {\isasymcirc}\isactrlsub c\ right{\isacharunderscore}{\kern0pt}coproj\ X\ V{\isachardoublequoteclose}\isanewline
\ \ \ \ \ \ \isacommand{using}\isamarkupfalse%
\ right{\isacharunderscore}{\kern0pt}coproj{\isacharunderscore}{\kern0pt}cfunc{\isacharunderscore}{\kern0pt}bowtie{\isacharunderscore}{\kern0pt}prod\ type{\isacharunderscore}{\kern0pt}assms\ \isacommand{by}\isamarkupfalse%
\ force\isanewline
\ \ \ \ \isacommand{also}\isamarkupfalse%
\ \isacommand{have}\isamarkupfalse%
\ {\isachardoublequoteopen}{\isachardot}{\kern0pt}{\isachardot}{\kern0pt}{\isachardot}{\kern0pt}\ {\isacharequal}{\kern0pt}\ {\isacharparenleft}{\kern0pt}y{\isadigit{1}}\ {\isasymamalg}\ y{\isadigit{2}}{\isacharparenright}{\kern0pt}\ {\isasymcirc}\isactrlsub c\ {\isacharparenleft}{\kern0pt}f\ {\isasymbowtie}\isactrlsub f\ g{\isacharparenright}{\kern0pt}\ {\isasymcirc}\isactrlsub c\ right{\isacharunderscore}{\kern0pt}coproj\ X\ V{\isachardoublequoteclose}\isanewline
\ \ \ \ \ \ \isacommand{using}\isamarkupfalse%
\ assms\ cfunc{\isacharunderscore}{\kern0pt}type{\isacharunderscore}{\kern0pt}def\ comp{\isacharunderscore}{\kern0pt}associative\ eqs\ x{\isacharunderscore}{\kern0pt}expand\ x{\isacharunderscore}{\kern0pt}type\ y{\isacharunderscore}{\kern0pt}expand\ y{\isacharunderscore}{\kern0pt}type\ \isacommand{by}\isamarkupfalse%
\ {\isacharparenleft}{\kern0pt}typecheck{\isacharunderscore}{\kern0pt}cfuncs{\isacharcomma}{\kern0pt}\ auto{\isacharparenright}{\kern0pt}\isanewline
\ \ \ \ \isacommand{also}\isamarkupfalse%
\ \isacommand{have}\isamarkupfalse%
\ {\isachardoublequoteopen}{\isachardot}{\kern0pt}{\isachardot}{\kern0pt}{\isachardot}{\kern0pt}\ {\isacharequal}{\kern0pt}\ {\isacharparenleft}{\kern0pt}y{\isadigit{1}}\ {\isasymamalg}\ y{\isadigit{2}}{\isacharparenright}{\kern0pt}\ {\isasymcirc}\isactrlsub c\ right{\isacharunderscore}{\kern0pt}coproj\ Y\ W\ {\isasymcirc}\isactrlsub c\ g{\isachardoublequoteclose}\isanewline
\ \ \ \ \ \ \isacommand{using}\isamarkupfalse%
\ assms\ \isacommand{by}\isamarkupfalse%
\ {\isacharparenleft}{\kern0pt}typecheck{\isacharunderscore}{\kern0pt}cfuncs{\isacharcomma}{\kern0pt}\ simp\ add{\isacharcolon}{\kern0pt}\ right{\isacharunderscore}{\kern0pt}coproj{\isacharunderscore}{\kern0pt}cfunc{\isacharunderscore}{\kern0pt}bowtie{\isacharunderscore}{\kern0pt}prod{\isacharparenright}{\kern0pt}\isanewline
\ \ \ \ \isacommand{also}\isamarkupfalse%
\ \isacommand{have}\isamarkupfalse%
\ {\isachardoublequoteopen}{\isachardot}{\kern0pt}{\isachardot}{\kern0pt}{\isachardot}{\kern0pt}\ {\isacharequal}{\kern0pt}\ {\isacharparenleft}{\kern0pt}{\isacharparenleft}{\kern0pt}y{\isadigit{1}}\ {\isasymamalg}\ y{\isadigit{2}}{\isacharparenright}{\kern0pt}\ {\isasymcirc}\isactrlsub c\ right{\isacharunderscore}{\kern0pt}coproj\ Y\ W{\isacharparenright}{\kern0pt}\ {\isasymcirc}\isactrlsub c\ g{\isachardoublequoteclose}\isanewline
\ \ \ \ \ \ \isacommand{using}\isamarkupfalse%
\ assms\ comp{\isacharunderscore}{\kern0pt}associative{\isadigit{2}}\ y{\isacharunderscore}{\kern0pt}expand\ y{\isacharunderscore}{\kern0pt}type\ \isacommand{by}\isamarkupfalse%
\ {\isacharparenleft}{\kern0pt}typecheck{\isacharunderscore}{\kern0pt}cfuncs{\isacharcomma}{\kern0pt}\ blast{\isacharparenright}{\kern0pt}\isanewline
\ \ \ \ \isacommand{also}\isamarkupfalse%
\ \isacommand{have}\isamarkupfalse%
\ {\isachardoublequoteopen}{\isachardot}{\kern0pt}{\isachardot}{\kern0pt}{\isachardot}{\kern0pt}\ {\isacharequal}{\kern0pt}\ y{\isadigit{2}}\ {\isasymcirc}\isactrlsub c\ g{\isachardoublequoteclose}\isanewline
\ \ \ \ \ \ \isacommand{using}\isamarkupfalse%
\ right{\isacharunderscore}{\kern0pt}coproj{\isacharunderscore}{\kern0pt}cfunc{\isacharunderscore}{\kern0pt}coprod\ y{\isadigit{1}}{\isacharunderscore}{\kern0pt}y{\isadigit{2}}{\isacharunderscore}{\kern0pt}type{\isacharparenleft}{\kern0pt}{\isadigit{1}}{\isacharparenright}{\kern0pt}\ y{\isadigit{1}}{\isacharunderscore}{\kern0pt}y{\isadigit{2}}{\isacharunderscore}{\kern0pt}type{\isacharparenleft}{\kern0pt}{\isadigit{2}}{\isacharparenright}{\kern0pt}\ \isacommand{by}\isamarkupfalse%
\ auto\isanewline
\ \ \ \ \isacommand{then}\isamarkupfalse%
\ \isacommand{show}\isamarkupfalse%
\ {\isachardoublequoteopen}x{\isadigit{2}}\ {\isacharequal}{\kern0pt}\ y{\isadigit{2}}{\isachardoublequoteclose}\isanewline
\ \ \ \ \ \ \isacommand{using}\isamarkupfalse%
\ calculation\ epimorphism{\isacharunderscore}{\kern0pt}def{\isadigit{3}}\ g{\isacharunderscore}{\kern0pt}epi\ type{\isacharunderscore}{\kern0pt}assms{\isacharparenleft}{\kern0pt}{\isadigit{2}}{\isacharparenright}{\kern0pt}\ x{\isadigit{1}}{\isacharunderscore}{\kern0pt}x{\isadigit{2}}{\isacharunderscore}{\kern0pt}type{\isacharparenleft}{\kern0pt}{\isadigit{2}}{\isacharparenright}{\kern0pt}\ y{\isadigit{1}}{\isacharunderscore}{\kern0pt}y{\isadigit{2}}{\isacharunderscore}{\kern0pt}type{\isacharparenleft}{\kern0pt}{\isadigit{2}}{\isacharparenright}{\kern0pt}\ \isacommand{by}\isamarkupfalse%
\ fastforce\isanewline
\ \ \isacommand{qed}\isamarkupfalse%
\isanewline
\ \ \isacommand{then}\isamarkupfalse%
\ \isacommand{show}\isamarkupfalse%
\ {\isachardoublequoteopen}x\ {\isacharequal}{\kern0pt}\ y{\isachardoublequoteclose}\isanewline
\ \ \ \ \isacommand{by}\isamarkupfalse%
\ {\isacharparenleft}{\kern0pt}simp\ add{\isacharcolon}{\kern0pt}\ x{\isacharunderscore}{\kern0pt}expand\ y{\isacharunderscore}{\kern0pt}expand{\isacharparenright}{\kern0pt}\isanewline
\isacommand{qed}\isamarkupfalse%
%
\endisatagproof
{\isafoldproof}%
%
\isadelimproof
\isanewline
%
\endisadelimproof
\isanewline
\isacommand{lemma}\isamarkupfalse%
\ cfunc{\isacharunderscore}{\kern0pt}bowtieprod{\isacharunderscore}{\kern0pt}inj{\isacharcolon}{\kern0pt}\isanewline
\ \ \isakeyword{assumes}\ type{\isacharunderscore}{\kern0pt}assms{\isacharcolon}{\kern0pt}\ {\isachardoublequoteopen}f\ {\isacharcolon}{\kern0pt}\ X\ {\isasymrightarrow}\ Y{\isachardoublequoteclose}\ {\isachardoublequoteopen}g\ {\isacharcolon}{\kern0pt}\ V\ {\isasymrightarrow}\ W{\isachardoublequoteclose}\isanewline
\ \ \isakeyword{assumes}\ f{\isacharunderscore}{\kern0pt}epi{\isacharcolon}{\kern0pt}\ {\isachardoublequoteopen}injective\ f{\isachardoublequoteclose}\ \isakeyword{and}\ g{\isacharunderscore}{\kern0pt}epi{\isacharcolon}{\kern0pt}\ {\isachardoublequoteopen}injective\ g{\isachardoublequoteclose}\isanewline
\ \ \isakeyword{shows}\ {\isachardoublequoteopen}injective\ {\isacharparenleft}{\kern0pt}f\ {\isasymbowtie}\isactrlsub f\ g{\isacharparenright}{\kern0pt}{\isachardoublequoteclose}\isanewline
%
\isadelimproof
\ \ %
\endisadelimproof
%
\isatagproof
\isacommand{unfolding}\isamarkupfalse%
\ injective{\isacharunderscore}{\kern0pt}def\isanewline
\isacommand{proof}\isamarkupfalse%
{\isacharparenleft}{\kern0pt}auto{\isacharparenright}{\kern0pt}\isanewline
\ \ \isacommand{fix}\isamarkupfalse%
\ z{\isadigit{1}}\ z{\isadigit{2}}\ \isanewline
\ \ \isacommand{assume}\isamarkupfalse%
\ x{\isacharunderscore}{\kern0pt}type{\isacharcolon}{\kern0pt}\ {\isachardoublequoteopen}z{\isadigit{1}}\ {\isasymin}\isactrlsub c\ domain\ {\isacharparenleft}{\kern0pt}f\ {\isasymbowtie}\isactrlsub f\ g{\isacharparenright}{\kern0pt}{\isachardoublequoteclose}\isanewline
\ \ \isacommand{assume}\isamarkupfalse%
\ y{\isacharunderscore}{\kern0pt}type{\isacharcolon}{\kern0pt}\ {\isachardoublequoteopen}z{\isadigit{2}}\ {\isasymin}\isactrlsub c\ domain\ {\isacharparenleft}{\kern0pt}f\ {\isasymbowtie}\isactrlsub f\ g{\isacharparenright}{\kern0pt}{\isachardoublequoteclose}\isanewline
\ \ \isacommand{assume}\isamarkupfalse%
\ eqs{\isacharcolon}{\kern0pt}\ {\isachardoublequoteopen}{\isacharparenleft}{\kern0pt}f\ {\isasymbowtie}\isactrlsub f\ g{\isacharparenright}{\kern0pt}\ {\isasymcirc}\isactrlsub c\ z{\isadigit{1}}\ {\isacharequal}{\kern0pt}\ {\isacharparenleft}{\kern0pt}f\ {\isasymbowtie}\isactrlsub f\ g{\isacharparenright}{\kern0pt}\ {\isasymcirc}\isactrlsub c\ z{\isadigit{2}}{\isachardoublequoteclose}\isanewline
\isanewline
\ \ \isacommand{have}\isamarkupfalse%
\ f{\isacharunderscore}{\kern0pt}bowtie{\isacharunderscore}{\kern0pt}g{\isacharunderscore}{\kern0pt}type{\isacharcolon}{\kern0pt}\ {\isachardoublequoteopen}{\isacharparenleft}{\kern0pt}f\ {\isasymbowtie}\isactrlsub f\ g{\isacharparenright}{\kern0pt}\ {\isacharcolon}{\kern0pt}\ X\ {\isasymCoprod}\ V\ {\isasymrightarrow}\ Y\ {\isasymCoprod}\ W{\isachardoublequoteclose}\isanewline
\ \ \ \ \isacommand{by}\isamarkupfalse%
\ {\isacharparenleft}{\kern0pt}simp\ add{\isacharcolon}{\kern0pt}\ cfunc{\isacharunderscore}{\kern0pt}bowtie{\isacharunderscore}{\kern0pt}prod{\isacharunderscore}{\kern0pt}type\ type{\isacharunderscore}{\kern0pt}assms{\isacharparenleft}{\kern0pt}{\isadigit{1}}{\isacharparenright}{\kern0pt}\ type{\isacharunderscore}{\kern0pt}assms{\isacharparenleft}{\kern0pt}{\isadigit{2}}{\isacharparenright}{\kern0pt}{\isacharparenright}{\kern0pt}\isanewline
\isanewline
\ \ \isacommand{have}\isamarkupfalse%
\ x{\isacharunderscore}{\kern0pt}type{\isadigit{2}}{\isacharcolon}{\kern0pt}\ {\isachardoublequoteopen}z{\isadigit{1}}\ {\isasymin}\isactrlsub c\ X\ {\isasymCoprod}\ V{\isachardoublequoteclose}\isanewline
\ \ \ \ \isacommand{using}\isamarkupfalse%
\ cfunc{\isacharunderscore}{\kern0pt}type{\isacharunderscore}{\kern0pt}def\ f{\isacharunderscore}{\kern0pt}bowtie{\isacharunderscore}{\kern0pt}g{\isacharunderscore}{\kern0pt}type\ x{\isacharunderscore}{\kern0pt}type\ \isacommand{by}\isamarkupfalse%
\ auto\isanewline
\ \ \isacommand{have}\isamarkupfalse%
\ y{\isacharunderscore}{\kern0pt}type{\isadigit{2}}{\isacharcolon}{\kern0pt}\ {\isachardoublequoteopen}z{\isadigit{2}}\ {\isasymin}\isactrlsub c\ X\ {\isasymCoprod}\ V{\isachardoublequoteclose}\isanewline
\ \ \ \ \isacommand{using}\isamarkupfalse%
\ cfunc{\isacharunderscore}{\kern0pt}type{\isacharunderscore}{\kern0pt}def\ f{\isacharunderscore}{\kern0pt}bowtie{\isacharunderscore}{\kern0pt}g{\isacharunderscore}{\kern0pt}type\ y{\isacharunderscore}{\kern0pt}type\ \isacommand{by}\isamarkupfalse%
\ auto\isanewline
\isanewline
\ \ \isacommand{have}\isamarkupfalse%
\ z{\isadigit{1}}{\isacharunderscore}{\kern0pt}decomp{\isacharcolon}{\kern0pt}\ {\isachardoublequoteopen}{\isacharparenleft}{\kern0pt}{\isasymexists}\ x{\isadigit{1}}{\isachardot}{\kern0pt}\ {\isacharparenleft}{\kern0pt}x{\isadigit{1}}\ {\isasymin}\isactrlsub c\ X\ {\isasymand}\ z{\isadigit{1}}\ {\isacharequal}{\kern0pt}\ left{\isacharunderscore}{\kern0pt}coproj\ X\ V\ {\isasymcirc}\isactrlsub c\ x{\isadigit{1}}{\isacharparenright}{\kern0pt}{\isacharparenright}{\kern0pt}\isanewline
\ \ \ \ \ \ {\isasymor}\ \ {\isacharparenleft}{\kern0pt}{\isasymexists}\ y{\isadigit{1}}{\isachardot}{\kern0pt}\ {\isacharparenleft}{\kern0pt}y{\isadigit{1}}\ {\isasymin}\isactrlsub c\ V\ {\isasymand}\ z{\isadigit{1}}\ {\isacharequal}{\kern0pt}\ right{\isacharunderscore}{\kern0pt}coproj\ X\ V\ {\isasymcirc}\isactrlsub c\ y{\isadigit{1}}{\isacharparenright}{\kern0pt}{\isacharparenright}{\kern0pt}{\isachardoublequoteclose}\isanewline
\ \ \ \ \isacommand{by}\isamarkupfalse%
\ {\isacharparenleft}{\kern0pt}simp\ add{\isacharcolon}{\kern0pt}\ coprojs{\isacharunderscore}{\kern0pt}jointly{\isacharunderscore}{\kern0pt}surj\ x{\isacharunderscore}{\kern0pt}type{\isadigit{2}}{\isacharparenright}{\kern0pt}\isanewline
\isanewline
\ \ \isacommand{have}\isamarkupfalse%
\ z{\isadigit{2}}{\isacharunderscore}{\kern0pt}decomp{\isacharcolon}{\kern0pt}\ {\isachardoublequoteopen}{\isacharparenleft}{\kern0pt}{\isasymexists}\ x{\isadigit{2}}{\isachardot}{\kern0pt}\ {\isacharparenleft}{\kern0pt}x{\isadigit{2}}\ {\isasymin}\isactrlsub c\ X\ {\isasymand}\ z{\isadigit{2}}\ {\isacharequal}{\kern0pt}\ left{\isacharunderscore}{\kern0pt}coproj\ X\ V\ {\isasymcirc}\isactrlsub c\ x{\isadigit{2}}{\isacharparenright}{\kern0pt}{\isacharparenright}{\kern0pt}\isanewline
\ \ \ \ \ \ {\isasymor}\ \ {\isacharparenleft}{\kern0pt}{\isasymexists}\ y{\isadigit{2}}{\isachardot}{\kern0pt}\ {\isacharparenleft}{\kern0pt}y{\isadigit{2}}\ {\isasymin}\isactrlsub c\ V\ {\isasymand}\ z{\isadigit{2}}\ {\isacharequal}{\kern0pt}\ right{\isacharunderscore}{\kern0pt}coproj\ X\ V\ {\isasymcirc}\isactrlsub c\ y{\isadigit{2}}{\isacharparenright}{\kern0pt}{\isacharparenright}{\kern0pt}{\isachardoublequoteclose}\isanewline
\ \ \ \ \isacommand{by}\isamarkupfalse%
\ {\isacharparenleft}{\kern0pt}simp\ add{\isacharcolon}{\kern0pt}\ coprojs{\isacharunderscore}{\kern0pt}jointly{\isacharunderscore}{\kern0pt}surj\ y{\isacharunderscore}{\kern0pt}type{\isadigit{2}}{\isacharparenright}{\kern0pt}\isanewline
\isanewline
\ \ \isacommand{show}\isamarkupfalse%
\ {\isachardoublequoteopen}z{\isadigit{1}}\ {\isacharequal}{\kern0pt}\ z{\isadigit{2}}{\isachardoublequoteclose}\isanewline
\ \ \isacommand{proof}\isamarkupfalse%
{\isacharparenleft}{\kern0pt}cases\ {\isachardoublequoteopen}{\isasymexists}\ x{\isadigit{1}}{\isachardot}{\kern0pt}\ x{\isadigit{1}}\ {\isasymin}\isactrlsub c\ X\ {\isasymand}\ z{\isadigit{1}}\ {\isacharequal}{\kern0pt}\ left{\isacharunderscore}{\kern0pt}coproj\ X\ V\ {\isasymcirc}\isactrlsub c\ x{\isadigit{1}}{\isachardoublequoteclose}{\isacharparenright}{\kern0pt}\isanewline
\ \ \ \ \isacommand{assume}\isamarkupfalse%
\ case{\isadigit{1}}{\isacharcolon}{\kern0pt}\ {\isachardoublequoteopen}{\isasymexists}x{\isadigit{1}}{\isachardot}{\kern0pt}\ x{\isadigit{1}}\ {\isasymin}\isactrlsub c\ X\ {\isasymand}\ z{\isadigit{1}}\ {\isacharequal}{\kern0pt}\ left{\isacharunderscore}{\kern0pt}coproj\ X\ V\ {\isasymcirc}\isactrlsub c\ x{\isadigit{1}}{\isachardoublequoteclose}\isanewline
\ \ \ \ \isacommand{obtain}\isamarkupfalse%
\ x{\isadigit{1}}\ \isakeyword{where}\ x{\isadigit{1}}{\isacharunderscore}{\kern0pt}def{\isacharcolon}{\kern0pt}\ {\isachardoublequoteopen}x{\isadigit{1}}\ {\isasymin}\isactrlsub c\ X\ {\isasymand}\ z{\isadigit{1}}\ {\isacharequal}{\kern0pt}\ left{\isacharunderscore}{\kern0pt}coproj\ X\ V\ {\isasymcirc}\isactrlsub c\ x{\isadigit{1}}{\isachardoublequoteclose}\isanewline
\ \ \ \ \ \ \ \ \ \ \isacommand{using}\isamarkupfalse%
\ case{\isadigit{1}}\ \isacommand{by}\isamarkupfalse%
\ blast\isanewline
\ \ \ \ \isacommand{show}\isamarkupfalse%
\ {\isachardoublequoteopen}z{\isadigit{1}}\ {\isacharequal}{\kern0pt}\ z{\isadigit{2}}{\isachardoublequoteclose}\isanewline
\ \ \ \ \isacommand{proof}\isamarkupfalse%
{\isacharparenleft}{\kern0pt}cases\ {\isachardoublequoteopen}{\isasymexists}\ x{\isadigit{2}}{\isachardot}{\kern0pt}\ x{\isadigit{2}}\ {\isasymin}\isactrlsub c\ X\ {\isasymand}\ z{\isadigit{2}}\ {\isacharequal}{\kern0pt}\ left{\isacharunderscore}{\kern0pt}coproj\ X\ V\ {\isasymcirc}\isactrlsub c\ x{\isadigit{2}}{\isachardoublequoteclose}{\isacharparenright}{\kern0pt}\isanewline
\ \ \ \ \ \ \isacommand{assume}\isamarkupfalse%
\ caseA{\isacharcolon}{\kern0pt}\ {\isachardoublequoteopen}{\isasymexists}x{\isadigit{2}}{\isachardot}{\kern0pt}\ x{\isadigit{2}}\ {\isasymin}\isactrlsub c\ X\ {\isasymand}\ z{\isadigit{2}}\ {\isacharequal}{\kern0pt}\ left{\isacharunderscore}{\kern0pt}coproj\ X\ V\ {\isasymcirc}\isactrlsub c\ x{\isadigit{2}}{\isachardoublequoteclose}\isanewline
\ \ \ \ \ \ \isacommand{show}\isamarkupfalse%
\ {\isachardoublequoteopen}z{\isadigit{1}}\ {\isacharequal}{\kern0pt}\ z{\isadigit{2}}{\isachardoublequoteclose}\isanewline
\ \ \ \ \ \ \isacommand{proof}\isamarkupfalse%
\ {\isacharminus}{\kern0pt}\ \isanewline
\ \ \ \ \ \ \ \ \isacommand{obtain}\isamarkupfalse%
\ x{\isadigit{2}}\ \isakeyword{where}\ x{\isadigit{2}}{\isacharunderscore}{\kern0pt}def{\isacharcolon}{\kern0pt}\ {\isachardoublequoteopen}x{\isadigit{2}}\ {\isasymin}\isactrlsub c\ X\ {\isasymand}\ z{\isadigit{2}}\ {\isacharequal}{\kern0pt}\ left{\isacharunderscore}{\kern0pt}coproj\ X\ V\ {\isasymcirc}\isactrlsub c\ x{\isadigit{2}}{\isachardoublequoteclose}\isanewline
\ \ \ \ \ \ \ \ \ \ \isacommand{using}\isamarkupfalse%
\ caseA\ \isacommand{by}\isamarkupfalse%
\ blast\isanewline
\ \ \ \ \ \ \ \ \isacommand{have}\isamarkupfalse%
\ {\isachardoublequoteopen}x{\isadigit{1}}\ {\isacharequal}{\kern0pt}\ x{\isadigit{2}}{\isachardoublequoteclose}\isanewline
\ \ \ \ \ \ \ \ \isacommand{proof}\isamarkupfalse%
\ {\isacharminus}{\kern0pt}\ \isanewline
\ \ \ \ \ \ \ \ \ \ \isacommand{have}\isamarkupfalse%
\ {\isachardoublequoteopen}left{\isacharunderscore}{\kern0pt}coproj\ Y\ W\ {\isasymcirc}\isactrlsub c\ f\ \ {\isasymcirc}\isactrlsub c\ x{\isadigit{1}}\ \ {\isacharequal}{\kern0pt}\ {\isacharparenleft}{\kern0pt}left{\isacharunderscore}{\kern0pt}coproj\ Y\ W\ {\isasymcirc}\isactrlsub c\ f{\isacharparenright}{\kern0pt}\ {\isasymcirc}\isactrlsub c\ x{\isadigit{1}}{\isachardoublequoteclose}\isanewline
\ \ \ \ \ \ \ \ \ \ \ \ \isacommand{using}\isamarkupfalse%
\ cfunc{\isacharunderscore}{\kern0pt}type{\isacharunderscore}{\kern0pt}def\ comp{\isacharunderscore}{\kern0pt}associative\ left{\isacharunderscore}{\kern0pt}proj{\isacharunderscore}{\kern0pt}type\ type{\isacharunderscore}{\kern0pt}assms{\isacharparenleft}{\kern0pt}{\isadigit{1}}{\isacharparenright}{\kern0pt}\ x{\isadigit{1}}{\isacharunderscore}{\kern0pt}def\ \isacommand{by}\isamarkupfalse%
\ auto\ \ \ \ \ \ \ \ \ \ \ \ \isanewline
\ \ \ \ \ \ \ \ \ \ \isacommand{also}\isamarkupfalse%
\ \isacommand{have}\isamarkupfalse%
\ {\isachardoublequoteopen}{\isachardot}{\kern0pt}{\isachardot}{\kern0pt}{\isachardot}{\kern0pt}\ {\isacharequal}{\kern0pt}\ \isanewline
\ \ \ \ \ \ \ \ \ \ \ \ \ \ \ \ {\isacharparenleft}{\kern0pt}{\isacharparenleft}{\kern0pt}{\isacharparenleft}{\kern0pt}left{\isacharunderscore}{\kern0pt}coproj\ Y\ W\ {\isasymcirc}\isactrlsub c\ f{\isacharparenright}{\kern0pt}\ {\isasymamalg}\ {\isacharparenleft}{\kern0pt}right{\isacharunderscore}{\kern0pt}coproj\ Y\ W\ {\isasymcirc}\isactrlsub c\ g{\isacharparenright}{\kern0pt}{\isacharparenright}{\kern0pt}\ {\isasymcirc}\isactrlsub c\ left{\isacharunderscore}{\kern0pt}coproj\ X\ V{\isacharparenright}{\kern0pt}\ {\isasymcirc}\isactrlsub c\ x{\isadigit{1}}{\isachardoublequoteclose}\isanewline
\ \ \ \ \ \ \ \ \ \ \ \ \isacommand{using}\isamarkupfalse%
\ cfunc{\isacharunderscore}{\kern0pt}bowtie{\isacharunderscore}{\kern0pt}prod{\isacharunderscore}{\kern0pt}def{\isadigit{2}}\ left{\isacharunderscore}{\kern0pt}coproj{\isacharunderscore}{\kern0pt}cfunc{\isacharunderscore}{\kern0pt}bowtie{\isacharunderscore}{\kern0pt}prod\ type{\isacharunderscore}{\kern0pt}assms\ \isacommand{by}\isamarkupfalse%
\ auto\isanewline
\ \ \ \ \ \ \ \ \ \ \isacommand{also}\isamarkupfalse%
\ \isacommand{have}\isamarkupfalse%
\ {\isachardoublequoteopen}{\isachardot}{\kern0pt}{\isachardot}{\kern0pt}{\isachardot}{\kern0pt}\ {\isacharequal}{\kern0pt}\ {\isacharparenleft}{\kern0pt}{\isacharparenleft}{\kern0pt}left{\isacharunderscore}{\kern0pt}coproj\ Y\ W\ {\isasymcirc}\isactrlsub c\ f{\isacharparenright}{\kern0pt}\ {\isasymamalg}\ {\isacharparenleft}{\kern0pt}right{\isacharunderscore}{\kern0pt}coproj\ Y\ W\ {\isasymcirc}\isactrlsub c\ g{\isacharparenright}{\kern0pt}{\isacharparenright}{\kern0pt}\ {\isasymcirc}\isactrlsub c\ left{\isacharunderscore}{\kern0pt}coproj\ X\ V\ {\isasymcirc}\isactrlsub c\ x{\isadigit{1}}{\isachardoublequoteclose}\isanewline
\ \ \ \ \ \ \ \ \ \ \ \ \isacommand{using}\isamarkupfalse%
\ comp{\isacharunderscore}{\kern0pt}associative{\isadigit{2}}\ type{\isacharunderscore}{\kern0pt}assms\ x{\isadigit{1}}{\isacharunderscore}{\kern0pt}def\ \isacommand{by}\isamarkupfalse%
\ {\isacharparenleft}{\kern0pt}typecheck{\isacharunderscore}{\kern0pt}cfuncs{\isacharcomma}{\kern0pt}\ fastforce{\isacharparenright}{\kern0pt}\isanewline
\ \ \ \ \ \ \ \ \ \ \isacommand{also}\isamarkupfalse%
\ \isacommand{have}\isamarkupfalse%
\ {\isachardoublequoteopen}{\isachardot}{\kern0pt}{\isachardot}{\kern0pt}{\isachardot}{\kern0pt}\ {\isacharequal}{\kern0pt}\ {\isacharparenleft}{\kern0pt}f\ {\isasymbowtie}\isactrlsub f\ g{\isacharparenright}{\kern0pt}\ {\isasymcirc}\isactrlsub c\ z{\isadigit{1}}{\isachardoublequoteclose}\isanewline
\ \ \ \ \ \ \ \ \ \ \ \ \isacommand{using}\isamarkupfalse%
\ cfunc{\isacharunderscore}{\kern0pt}bowtie{\isacharunderscore}{\kern0pt}prod{\isacharunderscore}{\kern0pt}def{\isadigit{2}}\ type{\isacharunderscore}{\kern0pt}assms\ x{\isadigit{1}}{\isacharunderscore}{\kern0pt}def\ \isacommand{by}\isamarkupfalse%
\ auto\isanewline
\ \ \ \ \ \ \ \ \ \ \isacommand{also}\isamarkupfalse%
\ \isacommand{have}\isamarkupfalse%
\ {\isachardoublequoteopen}{\isachardot}{\kern0pt}{\isachardot}{\kern0pt}{\isachardot}{\kern0pt}\ {\isacharequal}{\kern0pt}\ {\isacharparenleft}{\kern0pt}f\ {\isasymbowtie}\isactrlsub f\ g{\isacharparenright}{\kern0pt}\ {\isasymcirc}\isactrlsub c\ z{\isadigit{2}}{\isachardoublequoteclose}\isanewline
\ \ \ \ \ \ \ \ \ \ \ \ \isacommand{by}\isamarkupfalse%
\ {\isacharparenleft}{\kern0pt}meson\ eqs{\isacharparenright}{\kern0pt}\isanewline
\ \ \ \ \ \ \ \ \ \ \isacommand{also}\isamarkupfalse%
\ \isacommand{have}\isamarkupfalse%
\ {\isachardoublequoteopen}{\isachardot}{\kern0pt}{\isachardot}{\kern0pt}{\isachardot}{\kern0pt}\ {\isacharequal}{\kern0pt}\ {\isacharparenleft}{\kern0pt}{\isacharparenleft}{\kern0pt}left{\isacharunderscore}{\kern0pt}coproj\ Y\ W\ {\isasymcirc}\isactrlsub c\ f{\isacharparenright}{\kern0pt}\ {\isasymamalg}\ {\isacharparenleft}{\kern0pt}right{\isacharunderscore}{\kern0pt}coproj\ Y\ W\ {\isasymcirc}\isactrlsub c\ g{\isacharparenright}{\kern0pt}{\isacharparenright}{\kern0pt}\ {\isasymcirc}\isactrlsub c\ left{\isacharunderscore}{\kern0pt}coproj\ X\ V\ {\isasymcirc}\isactrlsub c\ x{\isadigit{2}}{\isachardoublequoteclose}\isanewline
\ \ \ \ \ \ \ \ \ \ \ \ \isacommand{using}\isamarkupfalse%
\ cfunc{\isacharunderscore}{\kern0pt}bowtie{\isacharunderscore}{\kern0pt}prod{\isacharunderscore}{\kern0pt}def{\isadigit{2}}\ type{\isacharunderscore}{\kern0pt}assms{\isacharparenleft}{\kern0pt}{\isadigit{1}}{\isacharparenright}{\kern0pt}\ type{\isacharunderscore}{\kern0pt}assms{\isacharparenleft}{\kern0pt}{\isadigit{2}}{\isacharparenright}{\kern0pt}\ x{\isadigit{2}}{\isacharunderscore}{\kern0pt}def\ \isacommand{by}\isamarkupfalse%
\ auto\isanewline
\ \ \ \ \ \ \ \ \ \ \isacommand{also}\isamarkupfalse%
\ \isacommand{have}\isamarkupfalse%
\ {\isachardoublequoteopen}{\isachardot}{\kern0pt}{\isachardot}{\kern0pt}{\isachardot}{\kern0pt}\ {\isacharequal}{\kern0pt}\ {\isacharparenleft}{\kern0pt}{\isacharparenleft}{\kern0pt}{\isacharparenleft}{\kern0pt}{\isacharparenleft}{\kern0pt}left{\isacharunderscore}{\kern0pt}coproj\ Y\ W{\isacharparenright}{\kern0pt}\ {\isasymcirc}\isactrlsub c\ f{\isacharparenright}{\kern0pt}\ {\isasymamalg}\ {\isacharparenleft}{\kern0pt}right{\isacharunderscore}{\kern0pt}coproj\ Y\ W\ {\isasymcirc}\isactrlsub c\ g{\isacharparenright}{\kern0pt}{\isacharparenright}{\kern0pt}\ {\isasymcirc}\isactrlsub c\ left{\isacharunderscore}{\kern0pt}coproj\ X\ V{\isacharparenright}{\kern0pt}\ {\isasymcirc}\isactrlsub c\ x{\isadigit{2}}{\isachardoublequoteclose}\isanewline
\ \ \ \ \ \ \ \ \ \ \ \ \isacommand{by}\isamarkupfalse%
\ {\isacharparenleft}{\kern0pt}typecheck{\isacharunderscore}{\kern0pt}cfuncs{\isacharcomma}{\kern0pt}\ meson\ comp{\isacharunderscore}{\kern0pt}associative{\isadigit{2}}\ type{\isacharunderscore}{\kern0pt}assms{\isacharparenleft}{\kern0pt}{\isadigit{1}}{\isacharparenright}{\kern0pt}\ type{\isacharunderscore}{\kern0pt}assms{\isacharparenleft}{\kern0pt}{\isadigit{2}}{\isacharparenright}{\kern0pt}\ x{\isadigit{2}}{\isacharunderscore}{\kern0pt}def{\isacharparenright}{\kern0pt}\isanewline
\ \ \ \ \ \ \ \ \ \ \isacommand{also}\isamarkupfalse%
\ \isacommand{have}\isamarkupfalse%
\ {\isachardoublequoteopen}{\isachardot}{\kern0pt}{\isachardot}{\kern0pt}{\isachardot}{\kern0pt}\ {\isacharequal}{\kern0pt}\ {\isacharparenleft}{\kern0pt}left{\isacharunderscore}{\kern0pt}coproj\ Y\ W\ {\isasymcirc}\isactrlsub c\ f{\isacharparenright}{\kern0pt}\ {\isasymcirc}\isactrlsub c\ x{\isadigit{2}}{\isachardoublequoteclose}\isanewline
\ \ \ \ \ \ \ \ \ \ \ \ \isacommand{using}\isamarkupfalse%
\ cfunc{\isacharunderscore}{\kern0pt}bowtie{\isacharunderscore}{\kern0pt}prod{\isacharunderscore}{\kern0pt}def{\isadigit{2}}\ left{\isacharunderscore}{\kern0pt}coproj{\isacharunderscore}{\kern0pt}cfunc{\isacharunderscore}{\kern0pt}bowtie{\isacharunderscore}{\kern0pt}prod\ type{\isacharunderscore}{\kern0pt}assms\ \isacommand{by}\isamarkupfalse%
\ auto\isanewline
\ \ \ \ \ \ \ \ \ \ \isacommand{also}\isamarkupfalse%
\ \isacommand{have}\isamarkupfalse%
\ {\isachardoublequoteopen}{\isachardot}{\kern0pt}{\isachardot}{\kern0pt}{\isachardot}{\kern0pt}\ {\isacharequal}{\kern0pt}\ left{\isacharunderscore}{\kern0pt}coproj\ Y\ W\ {\isasymcirc}\isactrlsub c\ f\ \ {\isasymcirc}\isactrlsub c\ x{\isadigit{2}}{\isachardoublequoteclose}\isanewline
\ \ \ \ \ \ \ \ \ \ \ \ \isacommand{by}\isamarkupfalse%
\ {\isacharparenleft}{\kern0pt}metis\ comp{\isacharunderscore}{\kern0pt}associative{\isadigit{2}}\ left{\isacharunderscore}{\kern0pt}proj{\isacharunderscore}{\kern0pt}type\ type{\isacharunderscore}{\kern0pt}assms{\isacharparenleft}{\kern0pt}{\isadigit{1}}{\isacharparenright}{\kern0pt}\ x{\isadigit{2}}{\isacharunderscore}{\kern0pt}def{\isacharparenright}{\kern0pt}\isanewline
\ \ \ \ \ \ \ \ \ \ \isacommand{then}\isamarkupfalse%
\ \isacommand{have}\isamarkupfalse%
\ {\isachardoublequoteopen}f\ \ {\isasymcirc}\isactrlsub c\ x{\isadigit{1}}\ {\isacharequal}{\kern0pt}\ f\ \ {\isasymcirc}\isactrlsub c\ x{\isadigit{2}}{\isachardoublequoteclose}\isanewline
\ \ \ \ \ \ \ \ \ \ \ \ \isacommand{using}\isamarkupfalse%
\ \ calculation\ cfunc{\isacharunderscore}{\kern0pt}type{\isacharunderscore}{\kern0pt}def\ left{\isacharunderscore}{\kern0pt}coproj{\isacharunderscore}{\kern0pt}are{\isacharunderscore}{\kern0pt}monomorphisms\isanewline
\ \ \ \ \ \ \ \ \ \ \ \ left{\isacharunderscore}{\kern0pt}proj{\isacharunderscore}{\kern0pt}type\ monomorphism{\isacharunderscore}{\kern0pt}def\ type{\isacharunderscore}{\kern0pt}assms{\isacharparenleft}{\kern0pt}{\isadigit{1}}{\isacharparenright}{\kern0pt}\ x{\isadigit{1}}{\isacharunderscore}{\kern0pt}def\ x{\isadigit{2}}{\isacharunderscore}{\kern0pt}def\ \isacommand{by}\isamarkupfalse%
\ {\isacharparenleft}{\kern0pt}typecheck{\isacharunderscore}{\kern0pt}cfuncs{\isacharcomma}{\kern0pt}auto{\isacharparenright}{\kern0pt}\isanewline
\ \ \ \ \ \ \ \ \ \ \isacommand{then}\isamarkupfalse%
\ \isacommand{show}\isamarkupfalse%
\ {\isachardoublequoteopen}x{\isadigit{1}}\ {\isacharequal}{\kern0pt}\ x{\isadigit{2}}{\isachardoublequoteclose}\isanewline
\ \ \ \ \ \ \ \ \ \ \ \ \isacommand{by}\isamarkupfalse%
\ {\isacharparenleft}{\kern0pt}metis\ cfunc{\isacharunderscore}{\kern0pt}type{\isacharunderscore}{\kern0pt}def\ f{\isacharunderscore}{\kern0pt}epi\ injective{\isacharunderscore}{\kern0pt}def\ type{\isacharunderscore}{\kern0pt}assms{\isacharparenleft}{\kern0pt}{\isadigit{1}}{\isacharparenright}{\kern0pt}\ x{\isadigit{1}}{\isacharunderscore}{\kern0pt}def\ x{\isadigit{2}}{\isacharunderscore}{\kern0pt}def{\isacharparenright}{\kern0pt}\isanewline
\ \ \ \ \ \ \ \ \isacommand{qed}\isamarkupfalse%
\isanewline
\ \ \ \ \ \ \ \ \isacommand{then}\isamarkupfalse%
\ \isacommand{show}\isamarkupfalse%
\ {\isachardoublequoteopen}z{\isadigit{1}}\ {\isacharequal}{\kern0pt}\ z{\isadigit{2}}{\isachardoublequoteclose}\isanewline
\ \ \ \ \ \ \ \ \ \ \isacommand{by}\isamarkupfalse%
\ {\isacharparenleft}{\kern0pt}simp\ add{\isacharcolon}{\kern0pt}\ x{\isadigit{1}}{\isacharunderscore}{\kern0pt}def\ x{\isadigit{2}}{\isacharunderscore}{\kern0pt}def{\isacharparenright}{\kern0pt}\isanewline
\ \ \ \ \ \ \isacommand{qed}\isamarkupfalse%
\isanewline
\ \ \ \ \isacommand{next}\isamarkupfalse%
\ \isanewline
\ \ \ \ \ \ \isacommand{assume}\isamarkupfalse%
\ caseB{\isacharcolon}{\kern0pt}\ {\isachardoublequoteopen}{\isasymnexists}x{\isadigit{2}}{\isachardot}{\kern0pt}\ x{\isadigit{2}}\ {\isasymin}\isactrlsub c\ X\ {\isasymand}\ z{\isadigit{2}}\ {\isacharequal}{\kern0pt}\ left{\isacharunderscore}{\kern0pt}coproj\ X\ V\ {\isasymcirc}\isactrlsub c\ x{\isadigit{2}}{\isachardoublequoteclose}\isanewline
\ \ \ \ \ \ \isacommand{then}\isamarkupfalse%
\ \isacommand{obtain}\isamarkupfalse%
\ y{\isadigit{2}}\ \isakeyword{where}\ y{\isadigit{2}}{\isacharunderscore}{\kern0pt}def{\isacharcolon}{\kern0pt}\ {\isachardoublequoteopen}{\isacharparenleft}{\kern0pt}y{\isadigit{2}}\ {\isasymin}\isactrlsub c\ V\ {\isasymand}\ z{\isadigit{2}}\ {\isacharequal}{\kern0pt}\ right{\isacharunderscore}{\kern0pt}coproj\ X\ V\ {\isasymcirc}\isactrlsub c\ y{\isadigit{2}}{\isacharparenright}{\kern0pt}{\isachardoublequoteclose}\isanewline
\ \ \ \ \ \ \ \ \isacommand{using}\isamarkupfalse%
\ z{\isadigit{2}}{\isacharunderscore}{\kern0pt}decomp\ \isacommand{by}\isamarkupfalse%
\ blast\isanewline
\ \ \ \ \ \ \isacommand{have}\isamarkupfalse%
\ {\isachardoublequoteopen}left{\isacharunderscore}{\kern0pt}coproj\ Y\ W\ {\isasymcirc}\isactrlsub c\ f\ \ {\isasymcirc}\isactrlsub c\ x{\isadigit{1}}\ \ {\isacharequal}{\kern0pt}\ {\isacharparenleft}{\kern0pt}left{\isacharunderscore}{\kern0pt}coproj\ Y\ W\ {\isasymcirc}\isactrlsub c\ f{\isacharparenright}{\kern0pt}\ {\isasymcirc}\isactrlsub c\ x{\isadigit{1}}{\isachardoublequoteclose}\isanewline
\ \ \ \ \ \ \ \ \ \ \ \ \isacommand{using}\isamarkupfalse%
\ cfunc{\isacharunderscore}{\kern0pt}type{\isacharunderscore}{\kern0pt}def\ comp{\isacharunderscore}{\kern0pt}associative\ left{\isacharunderscore}{\kern0pt}proj{\isacharunderscore}{\kern0pt}type\ type{\isacharunderscore}{\kern0pt}assms{\isacharparenleft}{\kern0pt}{\isadigit{1}}{\isacharparenright}{\kern0pt}\ x{\isadigit{1}}{\isacharunderscore}{\kern0pt}def\ \isacommand{by}\isamarkupfalse%
\ auto\ \ \ \ \ \ \ \ \ \ \ \ \isanewline
\ \ \ \ \ \ \isacommand{also}\isamarkupfalse%
\ \isacommand{have}\isamarkupfalse%
\ {\isachardoublequoteopen}{\isachardot}{\kern0pt}{\isachardot}{\kern0pt}{\isachardot}{\kern0pt}\ {\isacharequal}{\kern0pt}\ \isanewline
\ \ \ \ \ \ \ \ \ \ \ \ {\isacharparenleft}{\kern0pt}{\isacharparenleft}{\kern0pt}{\isacharparenleft}{\kern0pt}left{\isacharunderscore}{\kern0pt}coproj\ Y\ W\ {\isasymcirc}\isactrlsub c\ f{\isacharparenright}{\kern0pt}\ {\isasymamalg}\ {\isacharparenleft}{\kern0pt}right{\isacharunderscore}{\kern0pt}coproj\ Y\ W\ {\isasymcirc}\isactrlsub c\ g{\isacharparenright}{\kern0pt}{\isacharparenright}{\kern0pt}\ {\isasymcirc}\isactrlsub c\ left{\isacharunderscore}{\kern0pt}coproj\ X\ V{\isacharparenright}{\kern0pt}\ {\isasymcirc}\isactrlsub c\ x{\isadigit{1}}{\isachardoublequoteclose}\isanewline
\ \ \ \ \ \ \ \ \isacommand{using}\isamarkupfalse%
\ cfunc{\isacharunderscore}{\kern0pt}bowtie{\isacharunderscore}{\kern0pt}prod{\isacharunderscore}{\kern0pt}def{\isadigit{2}}\ left{\isacharunderscore}{\kern0pt}coproj{\isacharunderscore}{\kern0pt}cfunc{\isacharunderscore}{\kern0pt}bowtie{\isacharunderscore}{\kern0pt}prod\ type{\isacharunderscore}{\kern0pt}assms{\isacharparenleft}{\kern0pt}{\isadigit{1}}{\isacharparenright}{\kern0pt}\ type{\isacharunderscore}{\kern0pt}assms{\isacharparenleft}{\kern0pt}{\isadigit{2}}{\isacharparenright}{\kern0pt}\ \isacommand{by}\isamarkupfalse%
\ auto\isanewline
\ \ \ \ \ \ \isacommand{also}\isamarkupfalse%
\ \isacommand{have}\isamarkupfalse%
\ {\isachardoublequoteopen}{\isachardot}{\kern0pt}{\isachardot}{\kern0pt}{\isachardot}{\kern0pt}\ {\isacharequal}{\kern0pt}\ {\isacharparenleft}{\kern0pt}{\isacharparenleft}{\kern0pt}left{\isacharunderscore}{\kern0pt}coproj\ Y\ W\ {\isasymcirc}\isactrlsub c\ f{\isacharparenright}{\kern0pt}\ {\isasymamalg}\ {\isacharparenleft}{\kern0pt}right{\isacharunderscore}{\kern0pt}coproj\ Y\ W\ {\isasymcirc}\isactrlsub c\ g{\isacharparenright}{\kern0pt}{\isacharparenright}{\kern0pt}\ {\isasymcirc}\isactrlsub c\ left{\isacharunderscore}{\kern0pt}coproj\ X\ V\ {\isasymcirc}\isactrlsub c\ x{\isadigit{1}}{\isachardoublequoteclose}\isanewline
\ \ \ \ \ \ \ \ \isacommand{using}\isamarkupfalse%
\ comp{\isacharunderscore}{\kern0pt}associative{\isadigit{2}}\ type{\isacharunderscore}{\kern0pt}assms{\isacharparenleft}{\kern0pt}{\isadigit{1}}{\isacharcomma}{\kern0pt}{\isadigit{2}}{\isacharparenright}{\kern0pt}\ x{\isadigit{1}}{\isacharunderscore}{\kern0pt}def\ \isacommand{by}\isamarkupfalse%
\ {\isacharparenleft}{\kern0pt}typecheck{\isacharunderscore}{\kern0pt}cfuncs{\isacharcomma}{\kern0pt}\ fastforce{\isacharparenright}{\kern0pt}\isanewline
\ \ \ \ \ \ \isacommand{also}\isamarkupfalse%
\ \isacommand{have}\isamarkupfalse%
\ {\isachardoublequoteopen}{\isachardot}{\kern0pt}{\isachardot}{\kern0pt}{\isachardot}{\kern0pt}\ {\isacharequal}{\kern0pt}\ {\isacharparenleft}{\kern0pt}f\ {\isasymbowtie}\isactrlsub f\ g{\isacharparenright}{\kern0pt}\ {\isasymcirc}\isactrlsub c\ z{\isadigit{1}}{\isachardoublequoteclose}\isanewline
\ \ \ \ \ \ \ \ \isacommand{using}\isamarkupfalse%
\ cfunc{\isacharunderscore}{\kern0pt}bowtie{\isacharunderscore}{\kern0pt}prod{\isacharunderscore}{\kern0pt}def{\isadigit{2}}\ type{\isacharunderscore}{\kern0pt}assms\ x{\isadigit{1}}{\isacharunderscore}{\kern0pt}def\ \isacommand{by}\isamarkupfalse%
\ auto\isanewline
\ \ \ \ \ \ \isacommand{also}\isamarkupfalse%
\ \isacommand{have}\isamarkupfalse%
\ {\isachardoublequoteopen}{\isachardot}{\kern0pt}{\isachardot}{\kern0pt}{\isachardot}{\kern0pt}\ {\isacharequal}{\kern0pt}\ {\isacharparenleft}{\kern0pt}f\ {\isasymbowtie}\isactrlsub f\ g{\isacharparenright}{\kern0pt}\ {\isasymcirc}\isactrlsub c\ z{\isadigit{2}}{\isachardoublequoteclose}\isanewline
\ \ \ \ \ \ \ \ \isacommand{by}\isamarkupfalse%
\ {\isacharparenleft}{\kern0pt}meson\ eqs{\isacharparenright}{\kern0pt}\isanewline
\ \ \ \ \ \ \isacommand{also}\isamarkupfalse%
\ \isacommand{have}\isamarkupfalse%
\ {\isachardoublequoteopen}{\isachardot}{\kern0pt}{\isachardot}{\kern0pt}{\isachardot}{\kern0pt}\ {\isacharequal}{\kern0pt}\ {\isacharparenleft}{\kern0pt}{\isacharparenleft}{\kern0pt}left{\isacharunderscore}{\kern0pt}coproj\ Y\ W\ {\isasymcirc}\isactrlsub c\ f{\isacharparenright}{\kern0pt}\ {\isasymamalg}\ {\isacharparenleft}{\kern0pt}right{\isacharunderscore}{\kern0pt}coproj\ Y\ W\ {\isasymcirc}\isactrlsub c\ g{\isacharparenright}{\kern0pt}{\isacharparenright}{\kern0pt}\ {\isasymcirc}\isactrlsub c\ right{\isacharunderscore}{\kern0pt}coproj\ X\ V\ {\isasymcirc}\isactrlsub c\ y{\isadigit{2}}{\isachardoublequoteclose}\isanewline
\ \ \ \ \ \ \ \ \isacommand{using}\isamarkupfalse%
\ cfunc{\isacharunderscore}{\kern0pt}bowtie{\isacharunderscore}{\kern0pt}prod{\isacharunderscore}{\kern0pt}def{\isadigit{2}}\ type{\isacharunderscore}{\kern0pt}assms\ y{\isadigit{2}}{\isacharunderscore}{\kern0pt}def\ \isacommand{by}\isamarkupfalse%
\ auto\isanewline
\ \ \ \ \ \ \isacommand{also}\isamarkupfalse%
\ \isacommand{have}\isamarkupfalse%
\ {\isachardoublequoteopen}{\isachardot}{\kern0pt}{\isachardot}{\kern0pt}{\isachardot}{\kern0pt}\ {\isacharequal}{\kern0pt}\ {\isacharparenleft}{\kern0pt}{\isacharparenleft}{\kern0pt}{\isacharparenleft}{\kern0pt}left{\isacharunderscore}{\kern0pt}coproj\ Y\ W\ {\isasymcirc}\isactrlsub c\ f{\isacharparenright}{\kern0pt}\ {\isasymamalg}\ {\isacharparenleft}{\kern0pt}right{\isacharunderscore}{\kern0pt}coproj\ Y\ W\ {\isasymcirc}\isactrlsub c\ g{\isacharparenright}{\kern0pt}{\isacharparenright}{\kern0pt}\ {\isasymcirc}\isactrlsub c\ right{\isacharunderscore}{\kern0pt}coproj\ X\ V{\isacharparenright}{\kern0pt}\ {\isasymcirc}\isactrlsub c\ y{\isadigit{2}}{\isachardoublequoteclose}\isanewline
\ \ \ \ \ \ \ \ \isacommand{by}\isamarkupfalse%
\ {\isacharparenleft}{\kern0pt}typecheck{\isacharunderscore}{\kern0pt}cfuncs{\isacharcomma}{\kern0pt}\ meson\ comp{\isacharunderscore}{\kern0pt}associative{\isadigit{2}}\ type{\isacharunderscore}{\kern0pt}assms\ y{\isadigit{2}}{\isacharunderscore}{\kern0pt}def{\isacharparenright}{\kern0pt}\isanewline
\ \ \ \ \ \ \isacommand{also}\isamarkupfalse%
\ \isacommand{have}\isamarkupfalse%
\ {\isachardoublequoteopen}{\isachardot}{\kern0pt}{\isachardot}{\kern0pt}{\isachardot}{\kern0pt}\ {\isacharequal}{\kern0pt}\ {\isacharparenleft}{\kern0pt}right{\isacharunderscore}{\kern0pt}coproj\ Y\ W\ {\isasymcirc}\isactrlsub c\ g{\isacharparenright}{\kern0pt}\ {\isasymcirc}\isactrlsub c\ y{\isadigit{2}}{\isachardoublequoteclose}\isanewline
\ \ \ \ \ \ \ \ \isacommand{using}\isamarkupfalse%
\ right{\isacharunderscore}{\kern0pt}coproj{\isacharunderscore}{\kern0pt}cfunc{\isacharunderscore}{\kern0pt}coprod\ type{\isacharunderscore}{\kern0pt}assms\ \isacommand{by}\isamarkupfalse%
\ {\isacharparenleft}{\kern0pt}typecheck{\isacharunderscore}{\kern0pt}cfuncs{\isacharcomma}{\kern0pt}\ fastforce{\isacharparenright}{\kern0pt}\isanewline
\ \ \ \ \ \ \isacommand{also}\isamarkupfalse%
\ \isacommand{have}\isamarkupfalse%
\ {\isachardoublequoteopen}{\isachardot}{\kern0pt}{\isachardot}{\kern0pt}{\isachardot}{\kern0pt}\ {\isacharequal}{\kern0pt}\ right{\isacharunderscore}{\kern0pt}coproj\ Y\ W\ {\isasymcirc}\isactrlsub c\ g\ \ {\isasymcirc}\isactrlsub c\ y{\isadigit{2}}{\isachardoublequoteclose}\isanewline
\ \ \ \ \ \ \ \ \isacommand{using}\isamarkupfalse%
\ comp{\isacharunderscore}{\kern0pt}associative{\isadigit{2}}\ type{\isacharunderscore}{\kern0pt}assms{\isacharparenleft}{\kern0pt}{\isadigit{2}}{\isacharparenright}{\kern0pt}\ y{\isadigit{2}}{\isacharunderscore}{\kern0pt}def\ \isacommand{by}\isamarkupfalse%
\ {\isacharparenleft}{\kern0pt}typecheck{\isacharunderscore}{\kern0pt}cfuncs{\isacharcomma}{\kern0pt}\ auto{\isacharparenright}{\kern0pt}\isanewline
\ \ \ \ \ \ \isacommand{then}\isamarkupfalse%
\ \isacommand{have}\isamarkupfalse%
\ False\isanewline
\ \ \ \ \ \ \ \ \isacommand{using}\isamarkupfalse%
\ calculation\ comp{\isacharunderscore}{\kern0pt}type\ coproducts{\isacharunderscore}{\kern0pt}disjoint\ type{\isacharunderscore}{\kern0pt}assms\ x{\isadigit{1}}{\isacharunderscore}{\kern0pt}def\ y{\isadigit{2}}{\isacharunderscore}{\kern0pt}def\ \isacommand{by}\isamarkupfalse%
\ auto\isanewline
\ \ \ \ \ \ \isacommand{then}\isamarkupfalse%
\ \isacommand{show}\isamarkupfalse%
\ {\isachardoublequoteopen}z{\isadigit{1}}\ {\isacharequal}{\kern0pt}\ z{\isadigit{2}}{\isachardoublequoteclose}\isanewline
\ \ \ \ \ \ \ \ \isacommand{by}\isamarkupfalse%
\ simp\isanewline
\ \ \ \ \isacommand{qed}\isamarkupfalse%
\isanewline
\ \ \isacommand{next}\isamarkupfalse%
\isanewline
\ \ \ \ \isacommand{assume}\isamarkupfalse%
\ case{\isadigit{2}}{\isacharcolon}{\kern0pt}\ {\isachardoublequoteopen}{\isasymnexists}x{\isadigit{1}}{\isachardot}{\kern0pt}\ x{\isadigit{1}}\ {\isasymin}\isactrlsub c\ X\ {\isasymand}\ z{\isadigit{1}}\ {\isacharequal}{\kern0pt}\ left{\isacharunderscore}{\kern0pt}coproj\ X\ V\ {\isasymcirc}\isactrlsub c\ x{\isadigit{1}}{\isachardoublequoteclose}\isanewline
\ \ \ \ \isacommand{then}\isamarkupfalse%
\ \isacommand{obtain}\isamarkupfalse%
\ y{\isadigit{1}}\ \isakeyword{where}\ y{\isadigit{1}}{\isacharunderscore}{\kern0pt}def{\isacharcolon}{\kern0pt}\ {\isachardoublequoteopen}y{\isadigit{1}}\ {\isasymin}\isactrlsub c\ V\ {\isasymand}\ z{\isadigit{1}}\ {\isacharequal}{\kern0pt}\ right{\isacharunderscore}{\kern0pt}coproj\ X\ V\ {\isasymcirc}\isactrlsub c\ y{\isadigit{1}}{\isachardoublequoteclose}\isanewline
\ \ \ \ \ \ \isacommand{using}\isamarkupfalse%
\ z{\isadigit{1}}{\isacharunderscore}{\kern0pt}decomp\ \isacommand{by}\isamarkupfalse%
\ blast\isanewline
\ \ \ \ \isacommand{show}\isamarkupfalse%
\ {\isachardoublequoteopen}z{\isadigit{1}}\ {\isacharequal}{\kern0pt}\ z{\isadigit{2}}{\isachardoublequoteclose}\isanewline
\ \ \ \ \isacommand{proof}\isamarkupfalse%
{\isacharparenleft}{\kern0pt}cases\ {\isachardoublequoteopen}{\isasymexists}\ x{\isadigit{2}}{\isachardot}{\kern0pt}\ x{\isadigit{2}}\ {\isasymin}\isactrlsub c\ X\ {\isasymand}\ z{\isadigit{2}}\ {\isacharequal}{\kern0pt}\ left{\isacharunderscore}{\kern0pt}coproj\ X\ V\ {\isasymcirc}\isactrlsub c\ x{\isadigit{2}}{\isachardoublequoteclose}{\isacharparenright}{\kern0pt}\isanewline
\ \ \ \ \ \ \isacommand{assume}\isamarkupfalse%
\ caseA{\isacharcolon}{\kern0pt}\ {\isachardoublequoteopen}{\isasymexists}x{\isadigit{2}}{\isachardot}{\kern0pt}\ x{\isadigit{2}}\ {\isasymin}\isactrlsub c\ X\ {\isasymand}\ z{\isadigit{2}}\ {\isacharequal}{\kern0pt}\ left{\isacharunderscore}{\kern0pt}coproj\ X\ V\ {\isasymcirc}\isactrlsub c\ x{\isadigit{2}}{\isachardoublequoteclose}\isanewline
\ \ \ \ \ \ \isacommand{show}\isamarkupfalse%
\ {\isachardoublequoteopen}z{\isadigit{1}}\ {\isacharequal}{\kern0pt}\ z{\isadigit{2}}{\isachardoublequoteclose}\isanewline
\ \ \ \ \ \ \isacommand{proof}\isamarkupfalse%
\ {\isacharminus}{\kern0pt}\ \isanewline
\ \ \ \ \ \ \ \ \isacommand{obtain}\isamarkupfalse%
\ x{\isadigit{2}}\ \isakeyword{where}\ x{\isadigit{2}}{\isacharunderscore}{\kern0pt}def{\isacharcolon}{\kern0pt}\ {\isachardoublequoteopen}x{\isadigit{2}}\ {\isasymin}\isactrlsub c\ X\ {\isasymand}\ z{\isadigit{2}}\ {\isacharequal}{\kern0pt}\ left{\isacharunderscore}{\kern0pt}coproj\ X\ V\ {\isasymcirc}\isactrlsub c\ x{\isadigit{2}}{\isachardoublequoteclose}\isanewline
\ \ \ \ \ \ \ \ \ \ \isacommand{using}\isamarkupfalse%
\ caseA\ \isacommand{by}\isamarkupfalse%
\ blast\isanewline
\ \ \ \ \ \ \ \ \isacommand{have}\isamarkupfalse%
\ {\isachardoublequoteopen}left{\isacharunderscore}{\kern0pt}coproj\ Y\ W\ {\isasymcirc}\isactrlsub c\ f\ \ {\isasymcirc}\isactrlsub c\ x{\isadigit{2}}\ \ {\isacharequal}{\kern0pt}\ {\isacharparenleft}{\kern0pt}left{\isacharunderscore}{\kern0pt}coproj\ Y\ W\ {\isasymcirc}\isactrlsub c\ f{\isacharparenright}{\kern0pt}\ {\isasymcirc}\isactrlsub c\ x{\isadigit{2}}{\isachardoublequoteclose}\isanewline
\ \ \ \ \ \ \ \ \ \ \isacommand{using}\isamarkupfalse%
\ comp{\isacharunderscore}{\kern0pt}associative{\isadigit{2}}\ type{\isacharunderscore}{\kern0pt}assms{\isacharparenleft}{\kern0pt}{\isadigit{1}}{\isacharparenright}{\kern0pt}\ x{\isadigit{2}}{\isacharunderscore}{\kern0pt}def\ \isacommand{by}\isamarkupfalse%
\ {\isacharparenleft}{\kern0pt}typecheck{\isacharunderscore}{\kern0pt}cfuncs{\isacharcomma}{\kern0pt}\ auto{\isacharparenright}{\kern0pt}\isanewline
\ \ \ \ \ \ \ \ \isacommand{also}\isamarkupfalse%
\ \isacommand{have}\isamarkupfalse%
\ {\isachardoublequoteopen}{\isachardot}{\kern0pt}{\isachardot}{\kern0pt}{\isachardot}{\kern0pt}\ {\isacharequal}{\kern0pt}\isanewline
\ \ \ \ \ \ \ \ \ \ \ \ \ \ {\isacharparenleft}{\kern0pt}{\isacharparenleft}{\kern0pt}{\isacharparenleft}{\kern0pt}left{\isacharunderscore}{\kern0pt}coproj\ Y\ W\ {\isasymcirc}\isactrlsub c\ f{\isacharparenright}{\kern0pt}\ {\isasymamalg}\ {\isacharparenleft}{\kern0pt}right{\isacharunderscore}{\kern0pt}coproj\ Y\ W\ {\isasymcirc}\isactrlsub c\ g{\isacharparenright}{\kern0pt}{\isacharparenright}{\kern0pt}\ {\isasymcirc}\isactrlsub c\ left{\isacharunderscore}{\kern0pt}coproj\ X\ V{\isacharparenright}{\kern0pt}\ {\isasymcirc}\isactrlsub c\ x{\isadigit{2}}{\isachardoublequoteclose}\isanewline
\ \ \ \ \ \ \ \ \ \ \isacommand{using}\isamarkupfalse%
\ cfunc{\isacharunderscore}{\kern0pt}bowtie{\isacharunderscore}{\kern0pt}prod{\isacharunderscore}{\kern0pt}def{\isadigit{2}}\ left{\isacharunderscore}{\kern0pt}coproj{\isacharunderscore}{\kern0pt}cfunc{\isacharunderscore}{\kern0pt}bowtie{\isacharunderscore}{\kern0pt}prod\ type{\isacharunderscore}{\kern0pt}assms\ \isacommand{by}\isamarkupfalse%
\ auto\isanewline
\ \ \ \ \ \ \ \ \isacommand{also}\isamarkupfalse%
\ \isacommand{have}\isamarkupfalse%
\ {\isachardoublequoteopen}{\isachardot}{\kern0pt}{\isachardot}{\kern0pt}{\isachardot}{\kern0pt}\ {\isacharequal}{\kern0pt}\ {\isacharparenleft}{\kern0pt}{\isacharparenleft}{\kern0pt}left{\isacharunderscore}{\kern0pt}coproj\ Y\ W\ {\isasymcirc}\isactrlsub c\ f{\isacharparenright}{\kern0pt}\ {\isasymamalg}\ {\isacharparenleft}{\kern0pt}right{\isacharunderscore}{\kern0pt}coproj\ Y\ W\ {\isasymcirc}\isactrlsub c\ g{\isacharparenright}{\kern0pt}{\isacharparenright}{\kern0pt}\ {\isasymcirc}\isactrlsub c\ left{\isacharunderscore}{\kern0pt}coproj\ X\ V\ {\isasymcirc}\isactrlsub c\ x{\isadigit{2}}{\isachardoublequoteclose}\isanewline
\ \ \ \ \ \ \ \ \ \ \isacommand{using}\isamarkupfalse%
\ comp{\isacharunderscore}{\kern0pt}associative{\isadigit{2}}\ type{\isacharunderscore}{\kern0pt}assms\ x{\isadigit{2}}{\isacharunderscore}{\kern0pt}def\ \isacommand{by}\isamarkupfalse%
\ {\isacharparenleft}{\kern0pt}typecheck{\isacharunderscore}{\kern0pt}cfuncs{\isacharcomma}{\kern0pt}\ fastforce{\isacharparenright}{\kern0pt}\isanewline
\ \ \ \ \ \ \ \ \isacommand{also}\isamarkupfalse%
\ \isacommand{have}\isamarkupfalse%
\ {\isachardoublequoteopen}{\isachardot}{\kern0pt}{\isachardot}{\kern0pt}{\isachardot}{\kern0pt}\ {\isacharequal}{\kern0pt}\ {\isacharparenleft}{\kern0pt}f\ {\isasymbowtie}\isactrlsub f\ g{\isacharparenright}{\kern0pt}\ {\isasymcirc}\isactrlsub c\ z{\isadigit{2}}{\isachardoublequoteclose}\isanewline
\ \ \ \ \ \ \ \ \ \ \isacommand{using}\isamarkupfalse%
\ cfunc{\isacharunderscore}{\kern0pt}bowtie{\isacharunderscore}{\kern0pt}prod{\isacharunderscore}{\kern0pt}def{\isadigit{2}}\ type{\isacharunderscore}{\kern0pt}assms\ x{\isadigit{2}}{\isacharunderscore}{\kern0pt}def\ \isacommand{by}\isamarkupfalse%
\ auto\isanewline
\ \ \ \ \ \ \ \ \isacommand{also}\isamarkupfalse%
\ \isacommand{have}\isamarkupfalse%
\ {\isachardoublequoteopen}{\isachardot}{\kern0pt}{\isachardot}{\kern0pt}{\isachardot}{\kern0pt}\ {\isacharequal}{\kern0pt}\ {\isacharparenleft}{\kern0pt}f\ {\isasymbowtie}\isactrlsub f\ g{\isacharparenright}{\kern0pt}\ {\isasymcirc}\isactrlsub c\ z{\isadigit{1}}{\isachardoublequoteclose}\isanewline
\ \ \ \ \ \ \ \ \ \ \isacommand{by}\isamarkupfalse%
\ {\isacharparenleft}{\kern0pt}simp\ add{\isacharcolon}{\kern0pt}\ eqs{\isacharparenright}{\kern0pt}\isanewline
\ \ \ \ \ \ \ \ \isacommand{also}\isamarkupfalse%
\ \isacommand{have}\isamarkupfalse%
\ {\isachardoublequoteopen}{\isachardot}{\kern0pt}{\isachardot}{\kern0pt}{\isachardot}{\kern0pt}\ {\isacharequal}{\kern0pt}\ {\isacharparenleft}{\kern0pt}{\isacharparenleft}{\kern0pt}left{\isacharunderscore}{\kern0pt}coproj\ Y\ W\ {\isasymcirc}\isactrlsub c\ f{\isacharparenright}{\kern0pt}\ {\isasymamalg}\ {\isacharparenleft}{\kern0pt}right{\isacharunderscore}{\kern0pt}coproj\ Y\ W\ {\isasymcirc}\isactrlsub c\ g{\isacharparenright}{\kern0pt}{\isacharparenright}{\kern0pt}\ {\isasymcirc}\isactrlsub c\ right{\isacharunderscore}{\kern0pt}coproj\ X\ V\ {\isasymcirc}\isactrlsub c\ y{\isadigit{1}}{\isachardoublequoteclose}\isanewline
\ \ \ \ \ \ \ \ \ \ \isacommand{using}\isamarkupfalse%
\ cfunc{\isacharunderscore}{\kern0pt}bowtie{\isacharunderscore}{\kern0pt}prod{\isacharunderscore}{\kern0pt}def{\isadigit{2}}\ type{\isacharunderscore}{\kern0pt}assms\ y{\isadigit{1}}{\isacharunderscore}{\kern0pt}def\ \isacommand{by}\isamarkupfalse%
\ auto\isanewline
\ \ \ \ \ \ \ \ \isacommand{also}\isamarkupfalse%
\ \isacommand{have}\isamarkupfalse%
\ {\isachardoublequoteopen}{\isachardot}{\kern0pt}{\isachardot}{\kern0pt}{\isachardot}{\kern0pt}\ {\isacharequal}{\kern0pt}\ {\isacharparenleft}{\kern0pt}{\isacharparenleft}{\kern0pt}{\isacharparenleft}{\kern0pt}left{\isacharunderscore}{\kern0pt}coproj\ Y\ W\ {\isasymcirc}\isactrlsub c\ f{\isacharparenright}{\kern0pt}\ {\isasymamalg}\ {\isacharparenleft}{\kern0pt}right{\isacharunderscore}{\kern0pt}coproj\ Y\ W\ {\isasymcirc}\isactrlsub c\ g{\isacharparenright}{\kern0pt}{\isacharparenright}{\kern0pt}\ {\isasymcirc}\isactrlsub c\ right{\isacharunderscore}{\kern0pt}coproj\ X\ V{\isacharparenright}{\kern0pt}\ {\isasymcirc}\isactrlsub c\ y{\isadigit{1}}{\isachardoublequoteclose}\isanewline
\ \ \ \ \ \ \ \ \ \ \isacommand{by}\isamarkupfalse%
\ {\isacharparenleft}{\kern0pt}typecheck{\isacharunderscore}{\kern0pt}cfuncs{\isacharcomma}{\kern0pt}\ meson\ comp{\isacharunderscore}{\kern0pt}associative{\isadigit{2}}\ type{\isacharunderscore}{\kern0pt}assms\ y{\isadigit{1}}{\isacharunderscore}{\kern0pt}def{\isacharparenright}{\kern0pt}\isanewline
\ \ \ \ \ \ \ \ \isacommand{also}\isamarkupfalse%
\ \isacommand{have}\isamarkupfalse%
\ {\isachardoublequoteopen}{\isachardot}{\kern0pt}{\isachardot}{\kern0pt}{\isachardot}{\kern0pt}\ {\isacharequal}{\kern0pt}\ {\isacharparenleft}{\kern0pt}right{\isacharunderscore}{\kern0pt}coproj\ Y\ W\ {\isasymcirc}\isactrlsub c\ g{\isacharparenright}{\kern0pt}\ {\isasymcirc}\isactrlsub c\ y{\isadigit{1}}{\isachardoublequoteclose}\isanewline
\ \ \ \ \ \ \ \ \ \ \isacommand{using}\isamarkupfalse%
\ right{\isacharunderscore}{\kern0pt}coproj{\isacharunderscore}{\kern0pt}cfunc{\isacharunderscore}{\kern0pt}coprod\ type{\isacharunderscore}{\kern0pt}assms\ \ \isacommand{by}\isamarkupfalse%
\ {\isacharparenleft}{\kern0pt}typecheck{\isacharunderscore}{\kern0pt}cfuncs{\isacharcomma}{\kern0pt}\ fastforce{\isacharparenright}{\kern0pt}\isanewline
\ \ \ \ \ \ \ \ \isacommand{also}\isamarkupfalse%
\ \isacommand{have}\isamarkupfalse%
\ {\isachardoublequoteopen}{\isachardot}{\kern0pt}{\isachardot}{\kern0pt}{\isachardot}{\kern0pt}\ {\isacharequal}{\kern0pt}\ right{\isacharunderscore}{\kern0pt}coproj\ Y\ W\ {\isasymcirc}\isactrlsub c\ g\ \ {\isasymcirc}\isactrlsub c\ y{\isadigit{1}}{\isachardoublequoteclose}\isanewline
\ \ \ \ \ \ \ \ \ \ \isacommand{using}\isamarkupfalse%
\ comp{\isacharunderscore}{\kern0pt}associative{\isadigit{2}}\ type{\isacharunderscore}{\kern0pt}assms{\isacharparenleft}{\kern0pt}{\isadigit{2}}{\isacharparenright}{\kern0pt}\ y{\isadigit{1}}{\isacharunderscore}{\kern0pt}def\ \isacommand{by}\isamarkupfalse%
\ {\isacharparenleft}{\kern0pt}typecheck{\isacharunderscore}{\kern0pt}cfuncs{\isacharcomma}{\kern0pt}\ auto{\isacharparenright}{\kern0pt}\isanewline
\ \ \ \ \ \ \ \ \isacommand{then}\isamarkupfalse%
\ \isacommand{have}\isamarkupfalse%
\ False\isanewline
\ \ \ \ \ \ \ \ \ \ \isacommand{using}\isamarkupfalse%
\ calculation\ comp{\isacharunderscore}{\kern0pt}type\ coproducts{\isacharunderscore}{\kern0pt}disjoint\ type{\isacharunderscore}{\kern0pt}assms\ x{\isadigit{2}}{\isacharunderscore}{\kern0pt}def\ y{\isadigit{1}}{\isacharunderscore}{\kern0pt}def\ \isacommand{by}\isamarkupfalse%
\ auto\isanewline
\ \ \ \ \ \ \ \ \isacommand{then}\isamarkupfalse%
\ \isacommand{show}\isamarkupfalse%
\ {\isachardoublequoteopen}z{\isadigit{1}}\ {\isacharequal}{\kern0pt}\ z{\isadigit{2}}{\isachardoublequoteclose}\isanewline
\ \ \ \ \ \ \ \ \ \ \isacommand{by}\isamarkupfalse%
\ simp\isanewline
\ \ \ \ \ \ \isacommand{qed}\isamarkupfalse%
\isanewline
\ \ \ \ \isacommand{next}\isamarkupfalse%
\isanewline
\ \ \ \ \ \ \isacommand{assume}\isamarkupfalse%
\ caseB{\isacharcolon}{\kern0pt}\ {\isachardoublequoteopen}{\isasymnexists}x{\isadigit{2}}{\isachardot}{\kern0pt}\ x{\isadigit{2}}\ {\isasymin}\isactrlsub c\ X\ {\isasymand}\ z{\isadigit{2}}\ {\isacharequal}{\kern0pt}\ left{\isacharunderscore}{\kern0pt}coproj\ X\ V\ {\isasymcirc}\isactrlsub c\ x{\isadigit{2}}{\isachardoublequoteclose}\isanewline
\ \ \ \ \ \ \isacommand{then}\isamarkupfalse%
\ \isacommand{obtain}\isamarkupfalse%
\ y{\isadigit{2}}\ \isakeyword{where}\ y{\isadigit{2}}{\isacharunderscore}{\kern0pt}def{\isacharcolon}{\kern0pt}\ {\isachardoublequoteopen}{\isacharparenleft}{\kern0pt}y{\isadigit{2}}\ {\isasymin}\isactrlsub c\ V\ {\isasymand}\ z{\isadigit{2}}\ {\isacharequal}{\kern0pt}\ right{\isacharunderscore}{\kern0pt}coproj\ X\ V\ {\isasymcirc}\isactrlsub c\ y{\isadigit{2}}{\isacharparenright}{\kern0pt}{\isachardoublequoteclose}\isanewline
\ \ \ \ \ \ \ \ \isacommand{using}\isamarkupfalse%
\ z{\isadigit{2}}{\isacharunderscore}{\kern0pt}decomp\ \isacommand{by}\isamarkupfalse%
\ blast\isanewline
\ \ \ \ \ \ \ \ \isacommand{have}\isamarkupfalse%
\ {\isachardoublequoteopen}y{\isadigit{1}}\ {\isacharequal}{\kern0pt}\ y{\isadigit{2}}{\isachardoublequoteclose}\isanewline
\ \ \ \ \ \ \ \ \isacommand{proof}\isamarkupfalse%
\ {\isacharminus}{\kern0pt}\ \isanewline
\ \ \ \ \ \ \ \ \ \ \isacommand{have}\isamarkupfalse%
\ {\isachardoublequoteopen}right{\isacharunderscore}{\kern0pt}coproj\ Y\ W\ {\isasymcirc}\isactrlsub c\ g\ \ {\isasymcirc}\isactrlsub c\ y{\isadigit{1}}\ \ {\isacharequal}{\kern0pt}\ {\isacharparenleft}{\kern0pt}right{\isacharunderscore}{\kern0pt}coproj\ Y\ W\ {\isasymcirc}\isactrlsub c\ g{\isacharparenright}{\kern0pt}\ {\isasymcirc}\isactrlsub c\ y{\isadigit{1}}{\isachardoublequoteclose}\isanewline
\ \ \ \ \ \ \ \ \ \ \ \ \isacommand{using}\isamarkupfalse%
\ comp{\isacharunderscore}{\kern0pt}associative{\isadigit{2}}\ type{\isacharunderscore}{\kern0pt}assms{\isacharparenleft}{\kern0pt}{\isadigit{2}}{\isacharparenright}{\kern0pt}\ y{\isadigit{1}}{\isacharunderscore}{\kern0pt}def\ \isacommand{by}\isamarkupfalse%
\ {\isacharparenleft}{\kern0pt}typecheck{\isacharunderscore}{\kern0pt}cfuncs{\isacharcomma}{\kern0pt}\ auto{\isacharparenright}{\kern0pt}\isanewline
\ \ \ \ \ \ \ \ \ \ \isacommand{also}\isamarkupfalse%
\ \isacommand{have}\isamarkupfalse%
\ {\isachardoublequoteopen}{\isachardot}{\kern0pt}{\isachardot}{\kern0pt}{\isachardot}{\kern0pt}\ {\isacharequal}{\kern0pt}\isanewline
\ \ \ \ \ \ \ \ \ \ \ \ \ \ \ \ {\isacharparenleft}{\kern0pt}{\isacharparenleft}{\kern0pt}{\isacharparenleft}{\kern0pt}left{\isacharunderscore}{\kern0pt}coproj\ Y\ W\ {\isasymcirc}\isactrlsub c\ f{\isacharparenright}{\kern0pt}\ {\isasymamalg}\ {\isacharparenleft}{\kern0pt}right{\isacharunderscore}{\kern0pt}coproj\ Y\ W\ {\isasymcirc}\isactrlsub c\ g{\isacharparenright}{\kern0pt}{\isacharparenright}{\kern0pt}\ {\isasymcirc}\isactrlsub c\ right{\isacharunderscore}{\kern0pt}coproj\ X\ V{\isacharparenright}{\kern0pt}\ {\isasymcirc}\isactrlsub c\ y{\isadigit{1}}{\isachardoublequoteclose}\isanewline
\ \ \ \ \ \ \ \ \ \ \ \ \isacommand{using}\isamarkupfalse%
\ right{\isacharunderscore}{\kern0pt}coproj{\isacharunderscore}{\kern0pt}cfunc{\isacharunderscore}{\kern0pt}coprod\ type{\isacharunderscore}{\kern0pt}assms\ \isacommand{by}\isamarkupfalse%
\ {\isacharparenleft}{\kern0pt}typecheck{\isacharunderscore}{\kern0pt}cfuncs{\isacharcomma}{\kern0pt}\ fastforce{\isacharparenright}{\kern0pt}\isanewline
\ \ \ \ \ \ \ \ \ \ \isacommand{also}\isamarkupfalse%
\ \isacommand{have}\isamarkupfalse%
\ {\isachardoublequoteopen}{\isachardot}{\kern0pt}{\isachardot}{\kern0pt}{\isachardot}{\kern0pt}\ {\isacharequal}{\kern0pt}\ {\isacharparenleft}{\kern0pt}{\isacharparenleft}{\kern0pt}left{\isacharunderscore}{\kern0pt}coproj\ Y\ W\ {\isasymcirc}\isactrlsub c\ f{\isacharparenright}{\kern0pt}\ {\isasymamalg}\ {\isacharparenleft}{\kern0pt}right{\isacharunderscore}{\kern0pt}coproj\ Y\ W\ {\isasymcirc}\isactrlsub c\ g{\isacharparenright}{\kern0pt}{\isacharparenright}{\kern0pt}\ {\isasymcirc}\isactrlsub c\ right{\isacharunderscore}{\kern0pt}coproj\ X\ V\ {\isasymcirc}\isactrlsub c\ y{\isadigit{1}}{\isachardoublequoteclose}\isanewline
\ \ \ \ \ \ \ \ \ \ \ \ \isacommand{using}\isamarkupfalse%
\ comp{\isacharunderscore}{\kern0pt}associative{\isadigit{2}}\ type{\isacharunderscore}{\kern0pt}assms\ \ y{\isadigit{1}}{\isacharunderscore}{\kern0pt}def\ \isacommand{by}\isamarkupfalse%
\ {\isacharparenleft}{\kern0pt}typecheck{\isacharunderscore}{\kern0pt}cfuncs{\isacharcomma}{\kern0pt}\ fastforce{\isacharparenright}{\kern0pt}\isanewline
\ \ \ \ \ \ \ \ \ \ \isacommand{also}\isamarkupfalse%
\ \isacommand{have}\isamarkupfalse%
\ {\isachardoublequoteopen}{\isachardot}{\kern0pt}{\isachardot}{\kern0pt}{\isachardot}{\kern0pt}\ {\isacharequal}{\kern0pt}\ {\isacharparenleft}{\kern0pt}f\ {\isasymbowtie}\isactrlsub f\ g{\isacharparenright}{\kern0pt}\ {\isasymcirc}\isactrlsub c\ z{\isadigit{1}}{\isachardoublequoteclose}\isanewline
\ \ \ \ \ \ \ \ \ \ \ \ \isacommand{using}\isamarkupfalse%
\ cfunc{\isacharunderscore}{\kern0pt}bowtie{\isacharunderscore}{\kern0pt}prod{\isacharunderscore}{\kern0pt}def{\isadigit{2}}\ type{\isacharunderscore}{\kern0pt}assms\ y{\isadigit{1}}{\isacharunderscore}{\kern0pt}def\ \isacommand{by}\isamarkupfalse%
\ auto\isanewline
\ \ \ \ \ \ \ \ \ \ \isacommand{also}\isamarkupfalse%
\ \isacommand{have}\isamarkupfalse%
\ {\isachardoublequoteopen}{\isachardot}{\kern0pt}{\isachardot}{\kern0pt}{\isachardot}{\kern0pt}\ {\isacharequal}{\kern0pt}\ {\isacharparenleft}{\kern0pt}f\ {\isasymbowtie}\isactrlsub f\ g{\isacharparenright}{\kern0pt}\ {\isasymcirc}\isactrlsub c\ z{\isadigit{2}}{\isachardoublequoteclose}\isanewline
\ \ \ \ \ \ \ \ \ \ \ \ \isacommand{by}\isamarkupfalse%
\ {\isacharparenleft}{\kern0pt}meson\ eqs{\isacharparenright}{\kern0pt}\isanewline
\ \ \ \ \ \ \ \ \ \ \isacommand{also}\isamarkupfalse%
\ \isacommand{have}\isamarkupfalse%
\ {\isachardoublequoteopen}{\isachardot}{\kern0pt}{\isachardot}{\kern0pt}{\isachardot}{\kern0pt}\ {\isacharequal}{\kern0pt}\ {\isacharparenleft}{\kern0pt}{\isacharparenleft}{\kern0pt}left{\isacharunderscore}{\kern0pt}coproj\ Y\ W\ {\isasymcirc}\isactrlsub c\ f{\isacharparenright}{\kern0pt}\ {\isasymamalg}\ {\isacharparenleft}{\kern0pt}right{\isacharunderscore}{\kern0pt}coproj\ Y\ W\ {\isasymcirc}\isactrlsub c\ g{\isacharparenright}{\kern0pt}{\isacharparenright}{\kern0pt}\ {\isasymcirc}\isactrlsub c\ right{\isacharunderscore}{\kern0pt}coproj\ X\ V\ {\isasymcirc}\isactrlsub c\ y{\isadigit{2}}{\isachardoublequoteclose}\isanewline
\ \ \ \ \ \ \ \ \ \ \ \ \isacommand{using}\isamarkupfalse%
\ cfunc{\isacharunderscore}{\kern0pt}bowtie{\isacharunderscore}{\kern0pt}prod{\isacharunderscore}{\kern0pt}def{\isadigit{2}}\ type{\isacharunderscore}{\kern0pt}assms\ y{\isadigit{2}}{\isacharunderscore}{\kern0pt}def\ \isacommand{by}\isamarkupfalse%
\ auto\isanewline
\ \ \ \ \ \ \ \ \ \ \isacommand{also}\isamarkupfalse%
\ \isacommand{have}\isamarkupfalse%
\ {\isachardoublequoteopen}{\isachardot}{\kern0pt}{\isachardot}{\kern0pt}{\isachardot}{\kern0pt}\ {\isacharequal}{\kern0pt}\ {\isacharparenleft}{\kern0pt}{\isacharparenleft}{\kern0pt}{\isacharparenleft}{\kern0pt}left{\isacharunderscore}{\kern0pt}coproj\ Y\ W\ {\isasymcirc}\isactrlsub c\ f{\isacharparenright}{\kern0pt}\ {\isasymamalg}\ {\isacharparenleft}{\kern0pt}right{\isacharunderscore}{\kern0pt}coproj\ Y\ W\ {\isasymcirc}\isactrlsub c\ g{\isacharparenright}{\kern0pt}{\isacharparenright}{\kern0pt}\ {\isasymcirc}\isactrlsub c\ right{\isacharunderscore}{\kern0pt}coproj\ X\ V{\isacharparenright}{\kern0pt}\ {\isasymcirc}\isactrlsub c\ y{\isadigit{2}}{\isachardoublequoteclose}\isanewline
\ \ \ \ \ \ \ \ \ \ \ \ \isacommand{by}\isamarkupfalse%
\ {\isacharparenleft}{\kern0pt}typecheck{\isacharunderscore}{\kern0pt}cfuncs{\isacharcomma}{\kern0pt}\ meson\ comp{\isacharunderscore}{\kern0pt}associative{\isadigit{2}}\ type{\isacharunderscore}{\kern0pt}assms\ \ y{\isadigit{2}}{\isacharunderscore}{\kern0pt}def{\isacharparenright}{\kern0pt}\isanewline
\ \ \ \ \ \ \ \ \ \ \isacommand{also}\isamarkupfalse%
\ \isacommand{have}\isamarkupfalse%
\ {\isachardoublequoteopen}{\isachardot}{\kern0pt}{\isachardot}{\kern0pt}{\isachardot}{\kern0pt}\ {\isacharequal}{\kern0pt}\ {\isacharparenleft}{\kern0pt}right{\isacharunderscore}{\kern0pt}coproj\ Y\ W\ {\isasymcirc}\isactrlsub c\ g{\isacharparenright}{\kern0pt}\ {\isasymcirc}\isactrlsub c\ y{\isadigit{2}}{\isachardoublequoteclose}\isanewline
\ \ \ \ \ \ \ \ \ \ \ \ \isacommand{using}\isamarkupfalse%
\ right{\isacharunderscore}{\kern0pt}coproj{\isacharunderscore}{\kern0pt}cfunc{\isacharunderscore}{\kern0pt}coprod\ type{\isacharunderscore}{\kern0pt}assms\ \isacommand{by}\isamarkupfalse%
\ {\isacharparenleft}{\kern0pt}typecheck{\isacharunderscore}{\kern0pt}cfuncs{\isacharcomma}{\kern0pt}\ fastforce{\isacharparenright}{\kern0pt}\isanewline
\ \ \ \ \ \ \ \ \ \ \isacommand{also}\isamarkupfalse%
\ \isacommand{have}\isamarkupfalse%
\ {\isachardoublequoteopen}{\isachardot}{\kern0pt}{\isachardot}{\kern0pt}{\isachardot}{\kern0pt}\ {\isacharequal}{\kern0pt}\ right{\isacharunderscore}{\kern0pt}coproj\ Y\ W\ {\isasymcirc}\isactrlsub c\ g\ \ {\isasymcirc}\isactrlsub c\ y{\isadigit{2}}{\isachardoublequoteclose}\isanewline
\ \ \ \ \ \ \ \ \ \ \ \ \isacommand{using}\isamarkupfalse%
\ comp{\isacharunderscore}{\kern0pt}associative{\isadigit{2}}\ type{\isacharunderscore}{\kern0pt}assms{\isacharparenleft}{\kern0pt}{\isadigit{2}}{\isacharparenright}{\kern0pt}\ y{\isadigit{2}}{\isacharunderscore}{\kern0pt}def\ \isacommand{by}\isamarkupfalse%
\ {\isacharparenleft}{\kern0pt}typecheck{\isacharunderscore}{\kern0pt}cfuncs{\isacharcomma}{\kern0pt}\ auto{\isacharparenright}{\kern0pt}\isanewline
\ \ \ \ \ \ \ \ \ \ \isacommand{then}\isamarkupfalse%
\ \isacommand{have}\isamarkupfalse%
\ {\isachardoublequoteopen}g\ \ {\isasymcirc}\isactrlsub c\ y{\isadigit{1}}\ {\isacharequal}{\kern0pt}\ g\ \ {\isasymcirc}\isactrlsub c\ y{\isadigit{2}}{\isachardoublequoteclose}\isanewline
\ \ \ \ \ \ \ \ \ \ \ \ \isacommand{using}\isamarkupfalse%
\ \ calculation\ cfunc{\isacharunderscore}{\kern0pt}type{\isacharunderscore}{\kern0pt}def\ right{\isacharunderscore}{\kern0pt}coproj{\isacharunderscore}{\kern0pt}are{\isacharunderscore}{\kern0pt}monomorphisms\isanewline
\ \ \ \ \ \ \ \ \ \ \ \ right{\isacharunderscore}{\kern0pt}proj{\isacharunderscore}{\kern0pt}type\ monomorphism{\isacharunderscore}{\kern0pt}def\ type{\isacharunderscore}{\kern0pt}assms{\isacharparenleft}{\kern0pt}{\isadigit{2}}{\isacharparenright}{\kern0pt}\ y{\isadigit{1}}{\isacharunderscore}{\kern0pt}def\ y{\isadigit{2}}{\isacharunderscore}{\kern0pt}def\ \isacommand{by}\isamarkupfalse%
\ {\isacharparenleft}{\kern0pt}typecheck{\isacharunderscore}{\kern0pt}cfuncs{\isacharcomma}{\kern0pt}auto{\isacharparenright}{\kern0pt}\isanewline
\ \ \ \ \ \ \ \ \ \ \isacommand{then}\isamarkupfalse%
\ \isacommand{show}\isamarkupfalse%
\ {\isachardoublequoteopen}y{\isadigit{1}}\ {\isacharequal}{\kern0pt}\ y{\isadigit{2}}{\isachardoublequoteclose}\isanewline
\ \ \ \ \ \ \ \ \ \ \ \ \isacommand{by}\isamarkupfalse%
\ {\isacharparenleft}{\kern0pt}metis\ cfunc{\isacharunderscore}{\kern0pt}type{\isacharunderscore}{\kern0pt}def\ g{\isacharunderscore}{\kern0pt}epi\ injective{\isacharunderscore}{\kern0pt}def\ type{\isacharunderscore}{\kern0pt}assms{\isacharparenleft}{\kern0pt}{\isadigit{2}}{\isacharparenright}{\kern0pt}\ y{\isadigit{1}}{\isacharunderscore}{\kern0pt}def\ y{\isadigit{2}}{\isacharunderscore}{\kern0pt}def{\isacharparenright}{\kern0pt}\isanewline
\ \ \ \ \ \ \ \ \isacommand{qed}\isamarkupfalse%
\isanewline
\ \ \ \ \ \ \ \ \isacommand{then}\isamarkupfalse%
\ \isacommand{show}\isamarkupfalse%
\ {\isachardoublequoteopen}z{\isadigit{1}}\ {\isacharequal}{\kern0pt}\ z{\isadigit{2}}{\isachardoublequoteclose}\isanewline
\ \ \ \ \ \ \ \ \ \ \isacommand{by}\isamarkupfalse%
\ {\isacharparenleft}{\kern0pt}simp\ add{\isacharcolon}{\kern0pt}\ y{\isadigit{1}}{\isacharunderscore}{\kern0pt}def\ y{\isadigit{2}}{\isacharunderscore}{\kern0pt}def{\isacharparenright}{\kern0pt}\isanewline
\ \ \ \ \ \ \isacommand{qed}\isamarkupfalse%
\isanewline
\ \ \ \isacommand{qed}\isamarkupfalse%
\isanewline
\ \isacommand{qed}\isamarkupfalse%
%
\endisatagproof
{\isafoldproof}%
%
\isadelimproof
\isanewline
%
\endisadelimproof
\isanewline
\isacommand{lemma}\isamarkupfalse%
\ cfunc{\isacharunderscore}{\kern0pt}bowtieprod{\isacharunderscore}{\kern0pt}inj{\isacharunderscore}{\kern0pt}converse{\isacharcolon}{\kern0pt}\isanewline
\ \ \isakeyword{assumes}\ type{\isacharunderscore}{\kern0pt}assms{\isacharcolon}{\kern0pt}\ {\isachardoublequoteopen}f\ {\isacharcolon}{\kern0pt}\ X\ {\isasymrightarrow}\ Y{\isachardoublequoteclose}\ {\isachardoublequoteopen}g\ {\isacharcolon}{\kern0pt}\ Z\ {\isasymrightarrow}\ W{\isachardoublequoteclose}\isanewline
\ \ \isakeyword{assumes}\ inj{\isacharunderscore}{\kern0pt}f{\isacharunderscore}{\kern0pt}bowtie{\isacharunderscore}{\kern0pt}g{\isacharcolon}{\kern0pt}\ {\isachardoublequoteopen}injective\ {\isacharparenleft}{\kern0pt}f\ {\isasymbowtie}\isactrlsub f\ g{\isacharparenright}{\kern0pt}{\isachardoublequoteclose}\isanewline
\ \ \isakeyword{shows}\ {\isachardoublequoteopen}injective\ f\ {\isasymand}\ injective\ g{\isachardoublequoteclose}\isanewline
%
\isadelimproof
\ \ %
\endisadelimproof
%
\isatagproof
\isacommand{unfolding}\isamarkupfalse%
\ injective{\isacharunderscore}{\kern0pt}def\isanewline
\isacommand{proof}\isamarkupfalse%
{\isacharparenleft}{\kern0pt}auto{\isacharparenright}{\kern0pt}\isanewline
\ \ \isacommand{fix}\isamarkupfalse%
\ x\ y\ \isanewline
\ \ \isacommand{assume}\isamarkupfalse%
\ x{\isacharunderscore}{\kern0pt}type{\isacharcolon}{\kern0pt}\ {\isachardoublequoteopen}x\ {\isasymin}\isactrlsub c\ domain\ f{\isachardoublequoteclose}\ \isanewline
\ \ \isacommand{assume}\isamarkupfalse%
\ y{\isacharunderscore}{\kern0pt}type{\isacharcolon}{\kern0pt}\ {\isachardoublequoteopen}y\ {\isasymin}\isactrlsub c\ domain\ f{\isachardoublequoteclose}\isanewline
\ \ \isacommand{assume}\isamarkupfalse%
\ eqs{\isacharcolon}{\kern0pt}\ \ \ \ {\isachardoublequoteopen}f\ {\isasymcirc}\isactrlsub c\ x\ {\isacharequal}{\kern0pt}\ f\ {\isasymcirc}\isactrlsub c\ y{\isachardoublequoteclose}\isanewline
\isanewline
\ \ \isacommand{have}\isamarkupfalse%
\ x{\isacharunderscore}{\kern0pt}type{\isadigit{2}}{\isacharcolon}{\kern0pt}\ {\isachardoublequoteopen}x\ {\isasymin}\isactrlsub c\ X{\isachardoublequoteclose}\isanewline
\ \ \ \ \isacommand{using}\isamarkupfalse%
\ cfunc{\isacharunderscore}{\kern0pt}type{\isacharunderscore}{\kern0pt}def\ type{\isacharunderscore}{\kern0pt}assms{\isacharparenleft}{\kern0pt}{\isadigit{1}}{\isacharparenright}{\kern0pt}\ x{\isacharunderscore}{\kern0pt}type\ \isacommand{by}\isamarkupfalse%
\ auto\isanewline
\ \ \isacommand{have}\isamarkupfalse%
\ y{\isacharunderscore}{\kern0pt}type{\isadigit{2}}{\isacharcolon}{\kern0pt}\ {\isachardoublequoteopen}y\ {\isasymin}\isactrlsub c\ X{\isachardoublequoteclose}\isanewline
\ \ \ \ \isacommand{using}\isamarkupfalse%
\ cfunc{\isacharunderscore}{\kern0pt}type{\isacharunderscore}{\kern0pt}def\ type{\isacharunderscore}{\kern0pt}assms{\isacharparenleft}{\kern0pt}{\isadigit{1}}{\isacharparenright}{\kern0pt}\ y{\isacharunderscore}{\kern0pt}type\ \isacommand{by}\isamarkupfalse%
\ auto\isanewline
\ \ \isacommand{have}\isamarkupfalse%
\ fg{\isacharunderscore}{\kern0pt}bowtie{\isacharunderscore}{\kern0pt}tyepe{\isacharcolon}{\kern0pt}\ {\isachardoublequoteopen}{\isacharparenleft}{\kern0pt}f\ {\isasymbowtie}\isactrlsub f\ g{\isacharparenright}{\kern0pt}\ {\isacharcolon}{\kern0pt}\ X\ {\isasymCoprod}\ Z\ {\isasymrightarrow}\ Y\ {\isasymCoprod}\ W{\isachardoublequoteclose}\isanewline
\ \ \ \ \isacommand{using}\isamarkupfalse%
\ assms\ \isacommand{by}\isamarkupfalse%
\ typecheck{\isacharunderscore}{\kern0pt}cfuncs\isanewline
\ \ \isacommand{have}\isamarkupfalse%
\ lift{\isacharcolon}{\kern0pt}\ {\isachardoublequoteopen}{\isacharparenleft}{\kern0pt}f\ {\isasymbowtie}\isactrlsub f\ g{\isacharparenright}{\kern0pt}\ {\isasymcirc}\isactrlsub c\ left{\isacharunderscore}{\kern0pt}coproj\ X\ Z\ {\isasymcirc}\isactrlsub c\ x\ {\isacharequal}{\kern0pt}\ {\isacharparenleft}{\kern0pt}f\ {\isasymbowtie}\isactrlsub f\ g{\isacharparenright}{\kern0pt}\ {\isasymcirc}\isactrlsub c\ left{\isacharunderscore}{\kern0pt}coproj\ X\ Z\ {\isasymcirc}\isactrlsub c\ y{\isachardoublequoteclose}\isanewline
\ \ \isacommand{proof}\isamarkupfalse%
\ {\isacharminus}{\kern0pt}\ \isanewline
\ \ \ \ \isacommand{have}\isamarkupfalse%
\ {\isachardoublequoteopen}{\isacharparenleft}{\kern0pt}f\ {\isasymbowtie}\isactrlsub f\ g{\isacharparenright}{\kern0pt}\ {\isasymcirc}\isactrlsub c\ left{\isacharunderscore}{\kern0pt}coproj\ X\ Z\ {\isasymcirc}\isactrlsub c\ x\ {\isacharequal}{\kern0pt}\ {\isacharparenleft}{\kern0pt}{\isacharparenleft}{\kern0pt}f\ {\isasymbowtie}\isactrlsub f\ g{\isacharparenright}{\kern0pt}\ {\isasymcirc}\isactrlsub c\ left{\isacharunderscore}{\kern0pt}coproj\ X\ Z{\isacharparenright}{\kern0pt}\ {\isasymcirc}\isactrlsub c\ x{\isachardoublequoteclose}\isanewline
\ \ \ \ \ \ \isacommand{using}\isamarkupfalse%
\ x{\isacharunderscore}{\kern0pt}type{\isadigit{2}}\ comp{\isacharunderscore}{\kern0pt}associative{\isadigit{2}}\ fg{\isacharunderscore}{\kern0pt}bowtie{\isacharunderscore}{\kern0pt}tyepe\ \isacommand{by}\isamarkupfalse%
\ {\isacharparenleft}{\kern0pt}typecheck{\isacharunderscore}{\kern0pt}cfuncs{\isacharcomma}{\kern0pt}\ auto{\isacharparenright}{\kern0pt}\isanewline
\ \ \ \ \isacommand{also}\isamarkupfalse%
\ \isacommand{have}\isamarkupfalse%
\ \ {\isachardoublequoteopen}{\isachardot}{\kern0pt}{\isachardot}{\kern0pt}{\isachardot}{\kern0pt}\ {\isacharequal}{\kern0pt}\ \ {\isacharparenleft}{\kern0pt}left{\isacharunderscore}{\kern0pt}coproj\ Y\ W\ {\isasymcirc}\isactrlsub c\ f{\isacharparenright}{\kern0pt}\ {\isasymcirc}\isactrlsub c\ x{\isachardoublequoteclose}\isanewline
\ \ \ \ \ \ \isacommand{using}\isamarkupfalse%
\ left{\isacharunderscore}{\kern0pt}coproj{\isacharunderscore}{\kern0pt}cfunc{\isacharunderscore}{\kern0pt}bowtie{\isacharunderscore}{\kern0pt}prod\ type{\isacharunderscore}{\kern0pt}assms\ \isacommand{by}\isamarkupfalse%
\ auto\isanewline
\ \ \ \ \isacommand{also}\isamarkupfalse%
\ \isacommand{have}\isamarkupfalse%
\ {\isachardoublequoteopen}{\isachardot}{\kern0pt}{\isachardot}{\kern0pt}{\isachardot}{\kern0pt}\ {\isacharequal}{\kern0pt}\ left{\isacharunderscore}{\kern0pt}coproj\ Y\ W\ {\isasymcirc}\isactrlsub c\ f\ {\isasymcirc}\isactrlsub c\ x{\isachardoublequoteclose}\isanewline
\ \ \ \ \ \ \isacommand{using}\isamarkupfalse%
\ x{\isacharunderscore}{\kern0pt}type{\isadigit{2}}\ comp{\isacharunderscore}{\kern0pt}associative{\isadigit{2}}\ type{\isacharunderscore}{\kern0pt}assms{\isacharparenleft}{\kern0pt}{\isadigit{1}}{\isacharparenright}{\kern0pt}\ \isacommand{by}\isamarkupfalse%
\ {\isacharparenleft}{\kern0pt}typecheck{\isacharunderscore}{\kern0pt}cfuncs{\isacharcomma}{\kern0pt}\ auto{\isacharparenright}{\kern0pt}\isanewline
\ \ \ \ \isacommand{also}\isamarkupfalse%
\ \isacommand{have}\isamarkupfalse%
\ {\isachardoublequoteopen}{\isachardot}{\kern0pt}{\isachardot}{\kern0pt}{\isachardot}{\kern0pt}\ {\isacharequal}{\kern0pt}\ left{\isacharunderscore}{\kern0pt}coproj\ Y\ W\ {\isasymcirc}\isactrlsub c\ f\ {\isasymcirc}\isactrlsub c\ y{\isachardoublequoteclose}\isanewline
\ \ \ \ \ \ \isacommand{by}\isamarkupfalse%
\ {\isacharparenleft}{\kern0pt}simp\ add{\isacharcolon}{\kern0pt}\ eqs{\isacharparenright}{\kern0pt}\isanewline
\ \ \ \ \isacommand{also}\isamarkupfalse%
\ \isacommand{have}\isamarkupfalse%
\ {\isachardoublequoteopen}{\isachardot}{\kern0pt}{\isachardot}{\kern0pt}{\isachardot}{\kern0pt}\ {\isacharequal}{\kern0pt}\ {\isacharparenleft}{\kern0pt}left{\isacharunderscore}{\kern0pt}coproj\ Y\ W\ {\isasymcirc}\isactrlsub c\ f{\isacharparenright}{\kern0pt}\ {\isasymcirc}\isactrlsub c\ y{\isachardoublequoteclose}\isanewline
\ \ \ \ \ \ \isacommand{using}\isamarkupfalse%
\ y{\isacharunderscore}{\kern0pt}type{\isadigit{2}}\ comp{\isacharunderscore}{\kern0pt}associative{\isadigit{2}}\ type{\isacharunderscore}{\kern0pt}assms{\isacharparenleft}{\kern0pt}{\isadigit{1}}{\isacharparenright}{\kern0pt}\ \isacommand{by}\isamarkupfalse%
\ {\isacharparenleft}{\kern0pt}typecheck{\isacharunderscore}{\kern0pt}cfuncs{\isacharcomma}{\kern0pt}\ auto{\isacharparenright}{\kern0pt}\isanewline
\ \ \ \ \isacommand{also}\isamarkupfalse%
\ \isacommand{have}\isamarkupfalse%
\ {\isachardoublequoteopen}{\isachardot}{\kern0pt}{\isachardot}{\kern0pt}{\isachardot}{\kern0pt}\ {\isacharequal}{\kern0pt}\ {\isacharparenleft}{\kern0pt}{\isacharparenleft}{\kern0pt}f\ {\isasymbowtie}\isactrlsub f\ g{\isacharparenright}{\kern0pt}\ {\isasymcirc}\isactrlsub c\ left{\isacharunderscore}{\kern0pt}coproj\ X\ Z{\isacharparenright}{\kern0pt}\ {\isasymcirc}\isactrlsub c\ y{\isachardoublequoteclose}\isanewline
\ \ \ \ \ \ \isacommand{using}\isamarkupfalse%
\ left{\isacharunderscore}{\kern0pt}coproj{\isacharunderscore}{\kern0pt}cfunc{\isacharunderscore}{\kern0pt}bowtie{\isacharunderscore}{\kern0pt}prod\ type{\isacharunderscore}{\kern0pt}assms{\isacharparenleft}{\kern0pt}{\isadigit{1}}{\isacharparenright}{\kern0pt}\ type{\isacharunderscore}{\kern0pt}assms{\isacharparenleft}{\kern0pt}{\isadigit{2}}{\isacharparenright}{\kern0pt}\ \isacommand{by}\isamarkupfalse%
\ auto\isanewline
\ \ \ \ \isacommand{also}\isamarkupfalse%
\ \isacommand{have}\isamarkupfalse%
\ {\isachardoublequoteopen}{\isachardot}{\kern0pt}{\isachardot}{\kern0pt}{\isachardot}{\kern0pt}\ {\isacharequal}{\kern0pt}\ {\isacharparenleft}{\kern0pt}f\ {\isasymbowtie}\isactrlsub f\ g{\isacharparenright}{\kern0pt}\ {\isasymcirc}\isactrlsub c\ left{\isacharunderscore}{\kern0pt}coproj\ X\ Z\ {\isasymcirc}\isactrlsub c\ y{\isachardoublequoteclose}\isanewline
\ \ \ \ \ \ \isacommand{using}\isamarkupfalse%
\ y{\isacharunderscore}{\kern0pt}type{\isadigit{2}}\ comp{\isacharunderscore}{\kern0pt}associative{\isadigit{2}}\ fg{\isacharunderscore}{\kern0pt}bowtie{\isacharunderscore}{\kern0pt}tyepe\ \isacommand{by}\isamarkupfalse%
\ {\isacharparenleft}{\kern0pt}typecheck{\isacharunderscore}{\kern0pt}cfuncs{\isacharcomma}{\kern0pt}\ auto{\isacharparenright}{\kern0pt}\isanewline
\ \ \ \ \isacommand{then}\isamarkupfalse%
\ \isacommand{show}\isamarkupfalse%
\ {\isacharquery}{\kern0pt}thesis\ \isacommand{using}\isamarkupfalse%
\ calculation\ \isacommand{by}\isamarkupfalse%
\ auto\isanewline
\ \ \isacommand{qed}\isamarkupfalse%
\isanewline
\ \ \isacommand{then}\isamarkupfalse%
\ \isacommand{have}\isamarkupfalse%
\ {\isachardoublequoteopen}monomorphism\ {\isacharparenleft}{\kern0pt}f\ {\isasymbowtie}\isactrlsub f\ g{\isacharparenright}{\kern0pt}{\isachardoublequoteclose}\isanewline
\ \ \ \ \isacommand{using}\isamarkupfalse%
\ inj{\isacharunderscore}{\kern0pt}f{\isacharunderscore}{\kern0pt}bowtie{\isacharunderscore}{\kern0pt}g\ injective{\isacharunderscore}{\kern0pt}imp{\isacharunderscore}{\kern0pt}monomorphism\ \isacommand{by}\isamarkupfalse%
\ auto\isanewline
\ \ \isacommand{then}\isamarkupfalse%
\ \isacommand{have}\isamarkupfalse%
\ {\isachardoublequoteopen}left{\isacharunderscore}{\kern0pt}coproj\ X\ Z\ \ {\isasymcirc}\isactrlsub c\ x\ {\isacharequal}{\kern0pt}\ left{\isacharunderscore}{\kern0pt}coproj\ X\ Z\ {\isasymcirc}\isactrlsub c\ y{\isachardoublequoteclose}\isanewline
\ \ \ \ \isacommand{by}\isamarkupfalse%
\ {\isacharparenleft}{\kern0pt}typecheck{\isacharunderscore}{\kern0pt}cfuncs{\isacharcomma}{\kern0pt}\ metis\ cfunc{\isacharunderscore}{\kern0pt}type{\isacharunderscore}{\kern0pt}def\ fg{\isacharunderscore}{\kern0pt}bowtie{\isacharunderscore}{\kern0pt}tyepe\ inj{\isacharunderscore}{\kern0pt}f{\isacharunderscore}{\kern0pt}bowtie{\isacharunderscore}{\kern0pt}g\ injective{\isacharunderscore}{\kern0pt}def\ lift\ x{\isacharunderscore}{\kern0pt}type{\isadigit{2}}\ y{\isacharunderscore}{\kern0pt}type{\isadigit{2}}{\isacharparenright}{\kern0pt}\isanewline
\ \ \isacommand{then}\isamarkupfalse%
\ \isacommand{show}\isamarkupfalse%
\ {\isachardoublequoteopen}x\ {\isacharequal}{\kern0pt}\ y{\isachardoublequoteclose}\isanewline
\ \ \ \ \isacommand{using}\isamarkupfalse%
\ x{\isacharunderscore}{\kern0pt}type{\isadigit{2}}\ y{\isacharunderscore}{\kern0pt}type{\isadigit{2}}\ cfunc{\isacharunderscore}{\kern0pt}type{\isacharunderscore}{\kern0pt}def\ left{\isacharunderscore}{\kern0pt}coproj{\isacharunderscore}{\kern0pt}are{\isacharunderscore}{\kern0pt}monomorphisms\ left{\isacharunderscore}{\kern0pt}proj{\isacharunderscore}{\kern0pt}type\ monomorphism{\isacharunderscore}{\kern0pt}def\ \isacommand{by}\isamarkupfalse%
\ auto\isanewline
\isacommand{next}\isamarkupfalse%
\isanewline
\ \ \isacommand{fix}\isamarkupfalse%
\ x\ y\ \isanewline
\ \ \isacommand{assume}\isamarkupfalse%
\ x{\isacharunderscore}{\kern0pt}type{\isacharcolon}{\kern0pt}\ {\isachardoublequoteopen}x\ {\isasymin}\isactrlsub c\ domain\ g{\isachardoublequoteclose}\ \isanewline
\ \ \isacommand{assume}\isamarkupfalse%
\ y{\isacharunderscore}{\kern0pt}type{\isacharcolon}{\kern0pt}\ {\isachardoublequoteopen}y\ {\isasymin}\isactrlsub c\ domain\ g{\isachardoublequoteclose}\isanewline
\ \ \isacommand{assume}\isamarkupfalse%
\ eqs{\isacharcolon}{\kern0pt}\ \ \ \ {\isachardoublequoteopen}g\ {\isasymcirc}\isactrlsub c\ x\ {\isacharequal}{\kern0pt}\ g\ {\isasymcirc}\isactrlsub c\ y{\isachardoublequoteclose}\isanewline
\isanewline
\ \ \isacommand{have}\isamarkupfalse%
\ x{\isacharunderscore}{\kern0pt}type{\isadigit{2}}{\isacharcolon}{\kern0pt}\ {\isachardoublequoteopen}x\ {\isasymin}\isactrlsub c\ Z{\isachardoublequoteclose}\isanewline
\ \ \ \ \isacommand{using}\isamarkupfalse%
\ cfunc{\isacharunderscore}{\kern0pt}type{\isacharunderscore}{\kern0pt}def\ type{\isacharunderscore}{\kern0pt}assms{\isacharparenleft}{\kern0pt}{\isadigit{2}}{\isacharparenright}{\kern0pt}\ x{\isacharunderscore}{\kern0pt}type\ \isacommand{by}\isamarkupfalse%
\ auto\isanewline
\ \ \isacommand{have}\isamarkupfalse%
\ y{\isacharunderscore}{\kern0pt}type{\isadigit{2}}{\isacharcolon}{\kern0pt}\ {\isachardoublequoteopen}y\ {\isasymin}\isactrlsub c\ Z{\isachardoublequoteclose}\isanewline
\ \ \ \ \isacommand{using}\isamarkupfalse%
\ cfunc{\isacharunderscore}{\kern0pt}type{\isacharunderscore}{\kern0pt}def\ type{\isacharunderscore}{\kern0pt}assms{\isacharparenleft}{\kern0pt}{\isadigit{2}}{\isacharparenright}{\kern0pt}\ y{\isacharunderscore}{\kern0pt}type\ \isacommand{by}\isamarkupfalse%
\ auto\isanewline
\ \ \isacommand{have}\isamarkupfalse%
\ fg{\isacharunderscore}{\kern0pt}bowtie{\isacharunderscore}{\kern0pt}tyepe{\isacharcolon}{\kern0pt}\ {\isachardoublequoteopen}f\ {\isasymbowtie}\isactrlsub f\ g\ {\isacharcolon}{\kern0pt}\ X\ {\isasymCoprod}\ Z\ {\isasymrightarrow}\ Y\ {\isasymCoprod}\ W{\isachardoublequoteclose}\isanewline
\ \ \ \ \isacommand{using}\isamarkupfalse%
\ assms\ \isacommand{by}\isamarkupfalse%
\ typecheck{\isacharunderscore}{\kern0pt}cfuncs\isanewline
\ \ \isacommand{have}\isamarkupfalse%
\ lift{\isacharcolon}{\kern0pt}\ {\isachardoublequoteopen}{\isacharparenleft}{\kern0pt}f\ {\isasymbowtie}\isactrlsub f\ g{\isacharparenright}{\kern0pt}\ {\isasymcirc}\isactrlsub c\ right{\isacharunderscore}{\kern0pt}coproj\ X\ Z\ {\isasymcirc}\isactrlsub c\ x\ {\isacharequal}{\kern0pt}\ {\isacharparenleft}{\kern0pt}f\ {\isasymbowtie}\isactrlsub f\ g{\isacharparenright}{\kern0pt}\ {\isasymcirc}\isactrlsub c\ right{\isacharunderscore}{\kern0pt}coproj\ X\ Z\ {\isasymcirc}\isactrlsub c\ y{\isachardoublequoteclose}\isanewline
\ \ \isacommand{proof}\isamarkupfalse%
\ {\isacharminus}{\kern0pt}\ \isanewline
\ \ \ \ \isacommand{have}\isamarkupfalse%
\ {\isachardoublequoteopen}{\isacharparenleft}{\kern0pt}f\ {\isasymbowtie}\isactrlsub f\ g{\isacharparenright}{\kern0pt}\ {\isasymcirc}\isactrlsub c\ right{\isacharunderscore}{\kern0pt}coproj\ X\ Z\ {\isasymcirc}\isactrlsub c\ x\ {\isacharequal}{\kern0pt}\ {\isacharparenleft}{\kern0pt}{\isacharparenleft}{\kern0pt}f\ {\isasymbowtie}\isactrlsub f\ g{\isacharparenright}{\kern0pt}\ {\isasymcirc}\isactrlsub c\ right{\isacharunderscore}{\kern0pt}coproj\ X\ Z{\isacharparenright}{\kern0pt}\ {\isasymcirc}\isactrlsub c\ x{\isachardoublequoteclose}\isanewline
\ \ \ \ \ \ \isacommand{using}\isamarkupfalse%
\ x{\isacharunderscore}{\kern0pt}type{\isadigit{2}}\ comp{\isacharunderscore}{\kern0pt}associative{\isadigit{2}}\ fg{\isacharunderscore}{\kern0pt}bowtie{\isacharunderscore}{\kern0pt}tyepe\ \isacommand{by}\isamarkupfalse%
\ {\isacharparenleft}{\kern0pt}typecheck{\isacharunderscore}{\kern0pt}cfuncs{\isacharcomma}{\kern0pt}\ auto{\isacharparenright}{\kern0pt}\isanewline
\ \ \ \ \isacommand{also}\isamarkupfalse%
\ \isacommand{have}\isamarkupfalse%
\ \ {\isachardoublequoteopen}{\isachardot}{\kern0pt}{\isachardot}{\kern0pt}{\isachardot}{\kern0pt}\ {\isacharequal}{\kern0pt}\ \ {\isacharparenleft}{\kern0pt}right{\isacharunderscore}{\kern0pt}coproj\ Y\ W\ {\isasymcirc}\isactrlsub c\ g{\isacharparenright}{\kern0pt}\ {\isasymcirc}\isactrlsub c\ x{\isachardoublequoteclose}\isanewline
\ \ \ \ \ \ \isacommand{using}\isamarkupfalse%
\ right{\isacharunderscore}{\kern0pt}coproj{\isacharunderscore}{\kern0pt}cfunc{\isacharunderscore}{\kern0pt}bowtie{\isacharunderscore}{\kern0pt}prod\ type{\isacharunderscore}{\kern0pt}assms\ \isacommand{by}\isamarkupfalse%
\ auto\isanewline
\ \ \ \ \isacommand{also}\isamarkupfalse%
\ \isacommand{have}\isamarkupfalse%
\ {\isachardoublequoteopen}{\isachardot}{\kern0pt}{\isachardot}{\kern0pt}{\isachardot}{\kern0pt}\ {\isacharequal}{\kern0pt}\ right{\isacharunderscore}{\kern0pt}coproj\ Y\ W\ {\isasymcirc}\isactrlsub c\ g\ {\isasymcirc}\isactrlsub c\ x{\isachardoublequoteclose}\isanewline
\ \ \ \ \ \ \isacommand{using}\isamarkupfalse%
\ x{\isacharunderscore}{\kern0pt}type{\isadigit{2}}\ comp{\isacharunderscore}{\kern0pt}associative{\isadigit{2}}\ type{\isacharunderscore}{\kern0pt}assms{\isacharparenleft}{\kern0pt}{\isadigit{2}}{\isacharparenright}{\kern0pt}\ \isacommand{by}\isamarkupfalse%
\ {\isacharparenleft}{\kern0pt}typecheck{\isacharunderscore}{\kern0pt}cfuncs{\isacharcomma}{\kern0pt}\ auto{\isacharparenright}{\kern0pt}\isanewline
\ \ \ \ \isacommand{also}\isamarkupfalse%
\ \isacommand{have}\isamarkupfalse%
\ {\isachardoublequoteopen}{\isachardot}{\kern0pt}{\isachardot}{\kern0pt}{\isachardot}{\kern0pt}\ {\isacharequal}{\kern0pt}\ right{\isacharunderscore}{\kern0pt}coproj\ Y\ W\ {\isasymcirc}\isactrlsub c\ g\ {\isasymcirc}\isactrlsub c\ y{\isachardoublequoteclose}\isanewline
\ \ \ \ \ \ \isacommand{by}\isamarkupfalse%
\ {\isacharparenleft}{\kern0pt}simp\ add{\isacharcolon}{\kern0pt}\ eqs{\isacharparenright}{\kern0pt}\isanewline
\ \ \ \ \isacommand{also}\isamarkupfalse%
\ \isacommand{have}\isamarkupfalse%
\ {\isachardoublequoteopen}{\isachardot}{\kern0pt}{\isachardot}{\kern0pt}{\isachardot}{\kern0pt}\ {\isacharequal}{\kern0pt}\ {\isacharparenleft}{\kern0pt}right{\isacharunderscore}{\kern0pt}coproj\ Y\ W\ {\isasymcirc}\isactrlsub c\ g{\isacharparenright}{\kern0pt}\ {\isasymcirc}\isactrlsub c\ y{\isachardoublequoteclose}\isanewline
\ \ \ \ \ \ \isacommand{using}\isamarkupfalse%
\ y{\isacharunderscore}{\kern0pt}type{\isadigit{2}}\ comp{\isacharunderscore}{\kern0pt}associative{\isadigit{2}}\ type{\isacharunderscore}{\kern0pt}assms{\isacharparenleft}{\kern0pt}{\isadigit{2}}{\isacharparenright}{\kern0pt}\ \isacommand{by}\isamarkupfalse%
\ {\isacharparenleft}{\kern0pt}typecheck{\isacharunderscore}{\kern0pt}cfuncs{\isacharcomma}{\kern0pt}\ auto{\isacharparenright}{\kern0pt}\isanewline
\ \ \ \ \isacommand{also}\isamarkupfalse%
\ \isacommand{have}\isamarkupfalse%
\ {\isachardoublequoteopen}{\isachardot}{\kern0pt}{\isachardot}{\kern0pt}{\isachardot}{\kern0pt}\ {\isacharequal}{\kern0pt}\ {\isacharparenleft}{\kern0pt}{\isacharparenleft}{\kern0pt}f\ {\isasymbowtie}\isactrlsub f\ g{\isacharparenright}{\kern0pt}\ {\isasymcirc}\isactrlsub c\ right{\isacharunderscore}{\kern0pt}coproj\ X\ Z{\isacharparenright}{\kern0pt}\ {\isasymcirc}\isactrlsub c\ y{\isachardoublequoteclose}\isanewline
\ \ \ \ \ \ \isacommand{using}\isamarkupfalse%
\ right{\isacharunderscore}{\kern0pt}coproj{\isacharunderscore}{\kern0pt}cfunc{\isacharunderscore}{\kern0pt}bowtie{\isacharunderscore}{\kern0pt}prod\ type{\isacharunderscore}{\kern0pt}assms{\isacharparenleft}{\kern0pt}{\isadigit{1}}{\isacharparenright}{\kern0pt}\ type{\isacharunderscore}{\kern0pt}assms{\isacharparenleft}{\kern0pt}{\isadigit{2}}{\isacharparenright}{\kern0pt}\ \isacommand{by}\isamarkupfalse%
\ auto\isanewline
\ \ \ \ \isacommand{also}\isamarkupfalse%
\ \isacommand{have}\isamarkupfalse%
\ {\isachardoublequoteopen}{\isachardot}{\kern0pt}{\isachardot}{\kern0pt}{\isachardot}{\kern0pt}\ {\isacharequal}{\kern0pt}\ {\isacharparenleft}{\kern0pt}f\ {\isasymbowtie}\isactrlsub f\ g{\isacharparenright}{\kern0pt}\ {\isasymcirc}\isactrlsub c\ right{\isacharunderscore}{\kern0pt}coproj\ X\ Z\ {\isasymcirc}\isactrlsub c\ y{\isachardoublequoteclose}\isanewline
\ \ \ \ \ \ \isacommand{using}\isamarkupfalse%
\ y{\isacharunderscore}{\kern0pt}type{\isadigit{2}}\ comp{\isacharunderscore}{\kern0pt}associative{\isadigit{2}}\ fg{\isacharunderscore}{\kern0pt}bowtie{\isacharunderscore}{\kern0pt}tyepe\ \isacommand{by}\isamarkupfalse%
\ {\isacharparenleft}{\kern0pt}typecheck{\isacharunderscore}{\kern0pt}cfuncs{\isacharcomma}{\kern0pt}\ auto{\isacharparenright}{\kern0pt}\isanewline
\ \ \ \ \isacommand{then}\isamarkupfalse%
\ \isacommand{show}\isamarkupfalse%
\ {\isacharquery}{\kern0pt}thesis\ \isacommand{using}\isamarkupfalse%
\ calculation\ \isacommand{by}\isamarkupfalse%
\ auto\isanewline
\ \ \isacommand{qed}\isamarkupfalse%
\isanewline
\ \ \isacommand{then}\isamarkupfalse%
\ \isacommand{have}\isamarkupfalse%
\ {\isachardoublequoteopen}monomorphism\ {\isacharparenleft}{\kern0pt}f\ {\isasymbowtie}\isactrlsub f\ g{\isacharparenright}{\kern0pt}{\isachardoublequoteclose}\isanewline
\ \ \ \ \isacommand{using}\isamarkupfalse%
\ inj{\isacharunderscore}{\kern0pt}f{\isacharunderscore}{\kern0pt}bowtie{\isacharunderscore}{\kern0pt}g\ injective{\isacharunderscore}{\kern0pt}imp{\isacharunderscore}{\kern0pt}monomorphism\ \isacommand{by}\isamarkupfalse%
\ auto\isanewline
\ \ \isacommand{then}\isamarkupfalse%
\ \isacommand{have}\isamarkupfalse%
\ {\isachardoublequoteopen}right{\isacharunderscore}{\kern0pt}coproj\ X\ Z\ {\isasymcirc}\isactrlsub c\ x\ {\isacharequal}{\kern0pt}\ right{\isacharunderscore}{\kern0pt}coproj\ X\ Z\ {\isasymcirc}\isactrlsub c\ y{\isachardoublequoteclose}\isanewline
\ \ \ \ \isacommand{by}\isamarkupfalse%
\ {\isacharparenleft}{\kern0pt}typecheck{\isacharunderscore}{\kern0pt}cfuncs{\isacharcomma}{\kern0pt}\ metis\ cfunc{\isacharunderscore}{\kern0pt}type{\isacharunderscore}{\kern0pt}def\ fg{\isacharunderscore}{\kern0pt}bowtie{\isacharunderscore}{\kern0pt}tyepe\ inj{\isacharunderscore}{\kern0pt}f{\isacharunderscore}{\kern0pt}bowtie{\isacharunderscore}{\kern0pt}g\ injective{\isacharunderscore}{\kern0pt}def\ lift\ x{\isacharunderscore}{\kern0pt}type{\isadigit{2}}\ y{\isacharunderscore}{\kern0pt}type{\isadigit{2}}{\isacharparenright}{\kern0pt}\isanewline
\ \ \isacommand{then}\isamarkupfalse%
\ \isacommand{show}\isamarkupfalse%
\ {\isachardoublequoteopen}x\ {\isacharequal}{\kern0pt}\ y{\isachardoublequoteclose}\isanewline
\ \ \ \ \isacommand{using}\isamarkupfalse%
\ x{\isacharunderscore}{\kern0pt}type{\isadigit{2}}\ y{\isacharunderscore}{\kern0pt}type{\isadigit{2}}\ cfunc{\isacharunderscore}{\kern0pt}type{\isacharunderscore}{\kern0pt}def\ right{\isacharunderscore}{\kern0pt}coproj{\isacharunderscore}{\kern0pt}are{\isacharunderscore}{\kern0pt}monomorphisms\ right{\isacharunderscore}{\kern0pt}proj{\isacharunderscore}{\kern0pt}type\ monomorphism{\isacharunderscore}{\kern0pt}def\ \isacommand{by}\isamarkupfalse%
\ auto\isanewline
\isacommand{qed}\isamarkupfalse%
%
\endisatagproof
{\isafoldproof}%
%
\isadelimproof
\isanewline
%
\endisadelimproof
\isanewline
\isacommand{lemma}\isamarkupfalse%
\ cfunc{\isacharunderscore}{\kern0pt}bowtieprod{\isacharunderscore}{\kern0pt}iso{\isacharcolon}{\kern0pt}\isanewline
\ \ \isakeyword{assumes}\ type{\isacharunderscore}{\kern0pt}assms{\isacharcolon}{\kern0pt}\ {\isachardoublequoteopen}f\ {\isacharcolon}{\kern0pt}\ X\ {\isasymrightarrow}\ Y{\isachardoublequoteclose}\ {\isachardoublequoteopen}g\ {\isacharcolon}{\kern0pt}\ V\ {\isasymrightarrow}\ W{\isachardoublequoteclose}\isanewline
\ \ \isakeyword{assumes}\ f{\isacharunderscore}{\kern0pt}iso{\isacharcolon}{\kern0pt}\ {\isachardoublequoteopen}isomorphism\ f{\isachardoublequoteclose}\ \isakeyword{and}\ g{\isacharunderscore}{\kern0pt}iso{\isacharcolon}{\kern0pt}\ {\isachardoublequoteopen}isomorphism\ g{\isachardoublequoteclose}\isanewline
\ \ \isakeyword{shows}\ {\isachardoublequoteopen}isomorphism\ {\isacharparenleft}{\kern0pt}f\ {\isasymbowtie}\isactrlsub f\ g{\isacharparenright}{\kern0pt}{\isachardoublequoteclose}\isanewline
%
\isadelimproof
\ \ %
\endisadelimproof
%
\isatagproof
\isacommand{by}\isamarkupfalse%
\ {\isacharparenleft}{\kern0pt}typecheck{\isacharunderscore}{\kern0pt}cfuncs{\isacharcomma}{\kern0pt}\ meson\ cfunc{\isacharunderscore}{\kern0pt}bowtieprod{\isacharunderscore}{\kern0pt}epi\ cfunc{\isacharunderscore}{\kern0pt}bowtieprod{\isacharunderscore}{\kern0pt}inj\ epi{\isacharunderscore}{\kern0pt}mon{\isacharunderscore}{\kern0pt}is{\isacharunderscore}{\kern0pt}iso\ f{\isacharunderscore}{\kern0pt}iso\ g{\isacharunderscore}{\kern0pt}iso\ injective{\isacharunderscore}{\kern0pt}imp{\isacharunderscore}{\kern0pt}monomorphism\ iso{\isacharunderscore}{\kern0pt}imp{\isacharunderscore}{\kern0pt}epi{\isacharunderscore}{\kern0pt}and{\isacharunderscore}{\kern0pt}monic\ monomorphism{\isacharunderscore}{\kern0pt}imp{\isacharunderscore}{\kern0pt}injective\ singletonI\ assms{\isacharparenright}{\kern0pt}%
\endisatagproof
{\isafoldproof}%
%
\isadelimproof
\isanewline
%
\endisadelimproof
\isanewline
\isacommand{lemma}\isamarkupfalse%
\ cfunc{\isacharunderscore}{\kern0pt}bowtieprod{\isacharunderscore}{\kern0pt}surj{\isacharunderscore}{\kern0pt}converse{\isacharcolon}{\kern0pt}\isanewline
\ \ \isakeyword{assumes}\ type{\isacharunderscore}{\kern0pt}assms{\isacharcolon}{\kern0pt}\ {\isachardoublequoteopen}f\ {\isacharcolon}{\kern0pt}\ X\ {\isasymrightarrow}\ Y{\isachardoublequoteclose}\ {\isachardoublequoteopen}g\ {\isacharcolon}{\kern0pt}\ Z\ {\isasymrightarrow}\ W{\isachardoublequoteclose}\isanewline
\ \ \isakeyword{assumes}\ inj{\isacharunderscore}{\kern0pt}f{\isacharunderscore}{\kern0pt}bowtie{\isacharunderscore}{\kern0pt}g{\isacharcolon}{\kern0pt}\ {\isachardoublequoteopen}surjective\ {\isacharparenleft}{\kern0pt}f\ {\isasymbowtie}\isactrlsub f\ g{\isacharparenright}{\kern0pt}{\isachardoublequoteclose}\isanewline
\ \ \isakeyword{shows}\ {\isachardoublequoteopen}surjective\ f\ {\isasymand}\ surjective\ g{\isachardoublequoteclose}\isanewline
%
\isadelimproof
\ \ %
\endisadelimproof
%
\isatagproof
\isacommand{unfolding}\isamarkupfalse%
\ surjective{\isacharunderscore}{\kern0pt}def\isanewline
\isacommand{proof}\isamarkupfalse%
{\isacharparenleft}{\kern0pt}auto{\isacharparenright}{\kern0pt}\isanewline
\ \ \isacommand{fix}\isamarkupfalse%
\ y\ \isanewline
\ \ \isacommand{assume}\isamarkupfalse%
\ y{\isacharunderscore}{\kern0pt}type{\isacharcolon}{\kern0pt}\ {\isachardoublequoteopen}y\ {\isasymin}\isactrlsub c\ codomain\ f{\isachardoublequoteclose}\ \isanewline
\ \ \isacommand{then}\isamarkupfalse%
\ \isacommand{have}\isamarkupfalse%
\ y{\isacharunderscore}{\kern0pt}type{\isadigit{2}}{\isacharcolon}{\kern0pt}\ {\isachardoublequoteopen}y\ {\isasymin}\isactrlsub c\ Y{\isachardoublequoteclose}\isanewline
\ \ \ \ \isacommand{using}\isamarkupfalse%
\ cfunc{\isacharunderscore}{\kern0pt}type{\isacharunderscore}{\kern0pt}def\ type{\isacharunderscore}{\kern0pt}assms{\isacharparenleft}{\kern0pt}{\isadigit{1}}{\isacharparenright}{\kern0pt}\ \isacommand{by}\isamarkupfalse%
\ auto\isanewline
\ \ \isacommand{then}\isamarkupfalse%
\ \isacommand{have}\isamarkupfalse%
\ coproj{\isacharunderscore}{\kern0pt}y{\isacharunderscore}{\kern0pt}type{\isacharcolon}{\kern0pt}\ {\isachardoublequoteopen}left{\isacharunderscore}{\kern0pt}coproj\ Y\ W\ {\isasymcirc}\isactrlsub c\ y\ {\isasymin}\isactrlsub c\ Y\ {\isasymCoprod}\ W{\isachardoublequoteclose}\ \isanewline
\ \ \ \ \isacommand{by}\isamarkupfalse%
\ typecheck{\isacharunderscore}{\kern0pt}cfuncs\isanewline
\ \ \isacommand{have}\isamarkupfalse%
\ fg{\isacharunderscore}{\kern0pt}type{\isacharcolon}{\kern0pt}\ {\isachardoublequoteopen}{\isacharparenleft}{\kern0pt}f\ {\isasymbowtie}\isactrlsub f\ g{\isacharparenright}{\kern0pt}\ {\isacharcolon}{\kern0pt}\ X\ {\isasymCoprod}\ Z\ {\isasymrightarrow}\ Y\ {\isasymCoprod}\ W{\isachardoublequoteclose}\isanewline
\ \ \ \ \isacommand{using}\isamarkupfalse%
\ assms\ \isacommand{by}\isamarkupfalse%
\ typecheck{\isacharunderscore}{\kern0pt}cfuncs\isanewline
\ \ \isacommand{obtain}\isamarkupfalse%
\ xz\ \isakeyword{where}\ xz{\isacharunderscore}{\kern0pt}def{\isacharcolon}{\kern0pt}\ {\isachardoublequoteopen}xz\ {\isasymin}\isactrlsub c\ X\ {\isasymCoprod}\ Z\ {\isasymand}\ {\isacharparenleft}{\kern0pt}f\ {\isasymbowtie}\isactrlsub f\ g{\isacharparenright}{\kern0pt}\ {\isasymcirc}\isactrlsub c\ xz\ {\isacharequal}{\kern0pt}\ \ left{\isacharunderscore}{\kern0pt}coproj\ Y\ W\ {\isasymcirc}\isactrlsub c\ y{\isachardoublequoteclose}\isanewline
\ \ \ \ \isacommand{using}\isamarkupfalse%
\ fg{\isacharunderscore}{\kern0pt}type\ y{\isacharunderscore}{\kern0pt}type{\isadigit{2}}\ cfunc{\isacharunderscore}{\kern0pt}type{\isacharunderscore}{\kern0pt}def\ inj{\isacharunderscore}{\kern0pt}f{\isacharunderscore}{\kern0pt}bowtie{\isacharunderscore}{\kern0pt}g\ surjective{\isacharunderscore}{\kern0pt}def\ \isacommand{by}\isamarkupfalse%
\ {\isacharparenleft}{\kern0pt}typecheck{\isacharunderscore}{\kern0pt}cfuncs{\isacharcomma}{\kern0pt}\ auto{\isacharparenright}{\kern0pt}\isanewline
\ \ \isacommand{then}\isamarkupfalse%
\ \isacommand{have}\isamarkupfalse%
\ xz{\isacharunderscore}{\kern0pt}form{\isacharcolon}{\kern0pt}\ {\isachardoublequoteopen}{\isacharparenleft}{\kern0pt}{\isasymexists}\ x{\isachardot}{\kern0pt}\ x\ {\isasymin}\isactrlsub c\ X\ {\isasymand}\ left{\isacharunderscore}{\kern0pt}coproj\ X\ Z\ {\isasymcirc}\isactrlsub c\ x\ {\isacharequal}{\kern0pt}\ \ \ xz{\isacharparenright}{\kern0pt}\ {\isasymor}\ \ \isanewline
\ \ \ \ \ \ \ \ \ \ \ \ \ \ \ \ \ \ \ \ \ \ {\isacharparenleft}{\kern0pt}{\isasymexists}\ z{\isachardot}{\kern0pt}\ z\ {\isasymin}\isactrlsub c\ Z\ {\isasymand}\ right{\isacharunderscore}{\kern0pt}coproj\ X\ Z\ {\isasymcirc}\isactrlsub c\ z\ {\isacharequal}{\kern0pt}\ \ xz{\isacharparenright}{\kern0pt}{\isachardoublequoteclose}\isanewline
\ \ \ \ \isacommand{using}\isamarkupfalse%
\ coprojs{\isacharunderscore}{\kern0pt}jointly{\isacharunderscore}{\kern0pt}surj\ xz{\isacharunderscore}{\kern0pt}def\ \isacommand{by}\isamarkupfalse%
\ {\isacharparenleft}{\kern0pt}typecheck{\isacharunderscore}{\kern0pt}cfuncs{\isacharcomma}{\kern0pt}\ blast{\isacharparenright}{\kern0pt}\isanewline
\ \ \isacommand{show}\isamarkupfalse%
\ {\isachardoublequoteopen}{\isasymexists}\ x{\isachardot}{\kern0pt}\ x\ {\isasymin}\isactrlsub c\ domain\ f\ {\isasymand}\ f\ {\isasymcirc}\isactrlsub c\ x\ {\isacharequal}{\kern0pt}\ y{\isachardoublequoteclose}\isanewline
\ \ \isacommand{proof}\isamarkupfalse%
{\isacharparenleft}{\kern0pt}cases\ {\isachardoublequoteopen}{\isasymexists}\ x{\isachardot}{\kern0pt}\ x\ {\isasymin}\isactrlsub c\ X\ {\isasymand}\ left{\isacharunderscore}{\kern0pt}coproj\ X\ Z\ {\isasymcirc}\isactrlsub c\ x\ {\isacharequal}{\kern0pt}\ \ \ xz{\isachardoublequoteclose}{\isacharparenright}{\kern0pt}\isanewline
\ \ \ \ \isacommand{assume}\isamarkupfalse%
\ {\isachardoublequoteopen}{\isasymexists}\ x{\isachardot}{\kern0pt}\ x\ {\isasymin}\isactrlsub c\ X\ {\isasymand}\ left{\isacharunderscore}{\kern0pt}coproj\ X\ Z\ {\isasymcirc}\isactrlsub c\ x\ {\isacharequal}{\kern0pt}\ \ \ xz{\isachardoublequoteclose}\isanewline
\ \ \ \ \isacommand{then}\isamarkupfalse%
\ \isacommand{obtain}\isamarkupfalse%
\ x\ \isakeyword{where}\ x{\isacharunderscore}{\kern0pt}def{\isacharcolon}{\kern0pt}\ {\isachardoublequoteopen}x\ {\isasymin}\isactrlsub c\ X\ {\isasymand}\ left{\isacharunderscore}{\kern0pt}coproj\ X\ Z\ {\isasymcirc}\isactrlsub c\ x\ {\isacharequal}{\kern0pt}\ xz{\isachardoublequoteclose}\isanewline
\ \ \ \ \ \ \isacommand{by}\isamarkupfalse%
\ blast\isanewline
\ \ \ \ \isacommand{have}\isamarkupfalse%
\ {\isachardoublequoteopen}f\ {\isasymcirc}\isactrlsub c\ x\ {\isacharequal}{\kern0pt}\ y{\isachardoublequoteclose}\isanewline
\ \ \ \ \isacommand{proof}\isamarkupfalse%
\ {\isacharminus}{\kern0pt}\ \isanewline
\ \ \ \ \ \ \isacommand{have}\isamarkupfalse%
\ {\isachardoublequoteopen}left{\isacharunderscore}{\kern0pt}coproj\ Y\ W\ {\isasymcirc}\isactrlsub c\ y\ {\isacharequal}{\kern0pt}\ {\isacharparenleft}{\kern0pt}f\ {\isasymbowtie}\isactrlsub f\ g{\isacharparenright}{\kern0pt}\ {\isasymcirc}\isactrlsub c\ xz{\isachardoublequoteclose}\isanewline
\ \ \ \ \ \ \ \ \isacommand{by}\isamarkupfalse%
\ {\isacharparenleft}{\kern0pt}simp\ add{\isacharcolon}{\kern0pt}\ xz{\isacharunderscore}{\kern0pt}def{\isacharparenright}{\kern0pt}\isanewline
\ \ \ \ \ \ \isacommand{also}\isamarkupfalse%
\ \isacommand{have}\isamarkupfalse%
\ {\isachardoublequoteopen}{\isachardot}{\kern0pt}{\isachardot}{\kern0pt}{\isachardot}{\kern0pt}\ {\isacharequal}{\kern0pt}\ {\isacharparenleft}{\kern0pt}f\ {\isasymbowtie}\isactrlsub f\ g{\isacharparenright}{\kern0pt}\ {\isasymcirc}\isactrlsub c\ left{\isacharunderscore}{\kern0pt}coproj\ X\ Z\ {\isasymcirc}\isactrlsub c\ x{\isachardoublequoteclose}\isanewline
\ \ \ \ \ \ \ \ \isacommand{by}\isamarkupfalse%
\ {\isacharparenleft}{\kern0pt}simp\ add{\isacharcolon}{\kern0pt}\ x{\isacharunderscore}{\kern0pt}def{\isacharparenright}{\kern0pt}\isanewline
\ \ \ \ \ \ \isacommand{also}\isamarkupfalse%
\ \isacommand{have}\isamarkupfalse%
\ {\isachardoublequoteopen}{\isachardot}{\kern0pt}{\isachardot}{\kern0pt}{\isachardot}{\kern0pt}\ {\isacharequal}{\kern0pt}\ {\isacharparenleft}{\kern0pt}{\isacharparenleft}{\kern0pt}f\ {\isasymbowtie}\isactrlsub f\ g{\isacharparenright}{\kern0pt}\ {\isasymcirc}\isactrlsub c\ left{\isacharunderscore}{\kern0pt}coproj\ X\ Z{\isacharparenright}{\kern0pt}\ {\isasymcirc}\isactrlsub c\ x{\isachardoublequoteclose}\isanewline
\ \ \ \ \ \ \ \ \isacommand{using}\isamarkupfalse%
\ \ comp{\isacharunderscore}{\kern0pt}associative{\isadigit{2}}\ fg{\isacharunderscore}{\kern0pt}type\ x{\isacharunderscore}{\kern0pt}def\ \isacommand{by}\isamarkupfalse%
\ {\isacharparenleft}{\kern0pt}typecheck{\isacharunderscore}{\kern0pt}cfuncs{\isacharcomma}{\kern0pt}\ auto{\isacharparenright}{\kern0pt}\isanewline
\ \ \ \ \ \ \isacommand{also}\isamarkupfalse%
\ \isacommand{have}\isamarkupfalse%
\ {\isachardoublequoteopen}{\isachardot}{\kern0pt}{\isachardot}{\kern0pt}{\isachardot}{\kern0pt}\ {\isacharequal}{\kern0pt}\ {\isacharparenleft}{\kern0pt}left{\isacharunderscore}{\kern0pt}coproj\ Y\ W\ {\isasymcirc}\isactrlsub c\ f{\isacharparenright}{\kern0pt}\ {\isasymcirc}\isactrlsub c\ x{\isachardoublequoteclose}\isanewline
\ \ \ \ \ \ \ \ \isacommand{using}\isamarkupfalse%
\ left{\isacharunderscore}{\kern0pt}coproj{\isacharunderscore}{\kern0pt}cfunc{\isacharunderscore}{\kern0pt}bowtie{\isacharunderscore}{\kern0pt}prod\ type{\isacharunderscore}{\kern0pt}assms\ \isacommand{by}\isamarkupfalse%
\ auto\isanewline
\ \ \ \ \ \ \isacommand{also}\isamarkupfalse%
\ \isacommand{have}\isamarkupfalse%
\ {\isachardoublequoteopen}{\isachardot}{\kern0pt}{\isachardot}{\kern0pt}{\isachardot}{\kern0pt}\ {\isacharequal}{\kern0pt}\ left{\isacharunderscore}{\kern0pt}coproj\ Y\ W\ {\isasymcirc}\isactrlsub c\ f\ {\isasymcirc}\isactrlsub c\ x{\isachardoublequoteclose}\isanewline
\ \ \ \ \ \ \ \ \isacommand{using}\isamarkupfalse%
\ comp{\isacharunderscore}{\kern0pt}associative{\isadigit{2}}\ type{\isacharunderscore}{\kern0pt}assms{\isacharparenleft}{\kern0pt}{\isadigit{1}}{\isacharparenright}{\kern0pt}\ x{\isacharunderscore}{\kern0pt}def\ \isacommand{by}\isamarkupfalse%
\ {\isacharparenleft}{\kern0pt}typecheck{\isacharunderscore}{\kern0pt}cfuncs{\isacharcomma}{\kern0pt}\ auto{\isacharparenright}{\kern0pt}\isanewline
\ \ \ \ \ \ \isacommand{then}\isamarkupfalse%
\ \isacommand{show}\isamarkupfalse%
\ {\isachardoublequoteopen}f\ {\isasymcirc}\isactrlsub c\ x\ {\isacharequal}{\kern0pt}\ y{\isachardoublequoteclose}\isanewline
\ \ \ \ \ \ \ \ \isacommand{using}\isamarkupfalse%
\ type{\isacharunderscore}{\kern0pt}assms{\isacharparenleft}{\kern0pt}{\isadigit{1}}{\isacharparenright}{\kern0pt}\ x{\isacharunderscore}{\kern0pt}def\ y{\isacharunderscore}{\kern0pt}type{\isadigit{2}}\ \ \isanewline
\ \ \ \ \ \ \ \ \isacommand{by}\isamarkupfalse%
\ {\isacharparenleft}{\kern0pt}typecheck{\isacharunderscore}{\kern0pt}cfuncs{\isacharcomma}{\kern0pt}\ metis\ calculation\ cfunc{\isacharunderscore}{\kern0pt}type{\isacharunderscore}{\kern0pt}def\ left{\isacharunderscore}{\kern0pt}coproj{\isacharunderscore}{\kern0pt}are{\isacharunderscore}{\kern0pt}monomorphisms\ left{\isacharunderscore}{\kern0pt}proj{\isacharunderscore}{\kern0pt}type\ monomorphism{\isacharunderscore}{\kern0pt}def\ x{\isacharunderscore}{\kern0pt}def{\isacharparenright}{\kern0pt}\isanewline
\ \ \ \ \isacommand{qed}\isamarkupfalse%
\isanewline
\ \ \ \ \isacommand{then}\isamarkupfalse%
\ \isacommand{show}\isamarkupfalse%
\ {\isacharquery}{\kern0pt}thesis\isanewline
\ \ \ \ \ \ \isacommand{using}\isamarkupfalse%
\ cfunc{\isacharunderscore}{\kern0pt}type{\isacharunderscore}{\kern0pt}def\ type{\isacharunderscore}{\kern0pt}assms{\isacharparenleft}{\kern0pt}{\isadigit{1}}{\isacharparenright}{\kern0pt}\ x{\isacharunderscore}{\kern0pt}def\ \isacommand{by}\isamarkupfalse%
\ auto\isanewline
\ \isacommand{next}\isamarkupfalse%
\isanewline
\ \ \ \isacommand{assume}\isamarkupfalse%
\ {\isachardoublequoteopen}{\isasymnexists}x{\isachardot}{\kern0pt}\ x\ {\isasymin}\isactrlsub c\ X\ {\isasymand}\ left{\isacharunderscore}{\kern0pt}coproj\ X\ Z\ {\isasymcirc}\isactrlsub c\ x\ {\isacharequal}{\kern0pt}\ xz{\isachardoublequoteclose}\isanewline
\ \ \ \isacommand{then}\isamarkupfalse%
\ \isacommand{obtain}\isamarkupfalse%
\ z\ \isakeyword{where}\ z{\isacharunderscore}{\kern0pt}def{\isacharcolon}{\kern0pt}\ {\isachardoublequoteopen}z\ {\isasymin}\isactrlsub c\ Z\ {\isasymand}\ right{\isacharunderscore}{\kern0pt}coproj\ X\ Z\ {\isasymcirc}\isactrlsub c\ z\ {\isacharequal}{\kern0pt}\ xz{\isachardoublequoteclose}\isanewline
\ \ \ \ \ \isacommand{using}\isamarkupfalse%
\ xz{\isacharunderscore}{\kern0pt}form\ \isacommand{by}\isamarkupfalse%
\ blast\isanewline
\ \ \ \isacommand{have}\isamarkupfalse%
\ False\isanewline
\ \ \ \ \isacommand{proof}\isamarkupfalse%
\ {\isacharminus}{\kern0pt}\ \isanewline
\ \ \ \ \ \ \isacommand{have}\isamarkupfalse%
\ {\isachardoublequoteopen}left{\isacharunderscore}{\kern0pt}coproj\ Y\ W\ {\isasymcirc}\isactrlsub c\ y\ {\isacharequal}{\kern0pt}\ {\isacharparenleft}{\kern0pt}f\ {\isasymbowtie}\isactrlsub f\ g{\isacharparenright}{\kern0pt}\ {\isasymcirc}\isactrlsub c\ xz{\isachardoublequoteclose}\isanewline
\ \ \ \ \ \ \ \ \isacommand{by}\isamarkupfalse%
\ {\isacharparenleft}{\kern0pt}simp\ add{\isacharcolon}{\kern0pt}\ xz{\isacharunderscore}{\kern0pt}def{\isacharparenright}{\kern0pt}\ \ \ \ \ \ \ \ \ \isanewline
\ \ \ \ \ \ \isacommand{also}\isamarkupfalse%
\ \isacommand{have}\isamarkupfalse%
\ {\isachardoublequoteopen}{\isachardot}{\kern0pt}{\isachardot}{\kern0pt}{\isachardot}{\kern0pt}\ {\isacharequal}{\kern0pt}\ {\isacharparenleft}{\kern0pt}f\ {\isasymbowtie}\isactrlsub f\ g{\isacharparenright}{\kern0pt}\ {\isasymcirc}\isactrlsub c\ right{\isacharunderscore}{\kern0pt}coproj\ X\ Z\ {\isasymcirc}\isactrlsub c\ z{\isachardoublequoteclose}\isanewline
\ \ \ \ \ \ \ \ \isacommand{by}\isamarkupfalse%
\ {\isacharparenleft}{\kern0pt}simp\ add{\isacharcolon}{\kern0pt}\ z{\isacharunderscore}{\kern0pt}def{\isacharparenright}{\kern0pt}\isanewline
\ \ \ \ \ \ \isacommand{also}\isamarkupfalse%
\ \isacommand{have}\isamarkupfalse%
\ {\isachardoublequoteopen}{\isachardot}{\kern0pt}{\isachardot}{\kern0pt}{\isachardot}{\kern0pt}\ {\isacharequal}{\kern0pt}\ {\isacharparenleft}{\kern0pt}{\isacharparenleft}{\kern0pt}f\ {\isasymbowtie}\isactrlsub f\ g{\isacharparenright}{\kern0pt}\ {\isasymcirc}\isactrlsub c\ right{\isacharunderscore}{\kern0pt}coproj\ X\ Z{\isacharparenright}{\kern0pt}\ {\isasymcirc}\isactrlsub c\ z{\isachardoublequoteclose}\isanewline
\ \ \ \ \ \ \ \ \isacommand{using}\isamarkupfalse%
\ comp{\isacharunderscore}{\kern0pt}associative{\isadigit{2}}\ fg{\isacharunderscore}{\kern0pt}type\ z{\isacharunderscore}{\kern0pt}def\ \isacommand{by}\isamarkupfalse%
\ {\isacharparenleft}{\kern0pt}typecheck{\isacharunderscore}{\kern0pt}cfuncs{\isacharcomma}{\kern0pt}\ auto{\isacharparenright}{\kern0pt}\isanewline
\ \ \ \ \ \ \isacommand{also}\isamarkupfalse%
\ \isacommand{have}\isamarkupfalse%
\ {\isachardoublequoteopen}{\isachardot}{\kern0pt}{\isachardot}{\kern0pt}{\isachardot}{\kern0pt}\ {\isacharequal}{\kern0pt}\ {\isacharparenleft}{\kern0pt}right{\isacharunderscore}{\kern0pt}coproj\ Y\ W\ {\isasymcirc}\isactrlsub c\ g{\isacharparenright}{\kern0pt}\ {\isasymcirc}\isactrlsub c\ z{\isachardoublequoteclose}\isanewline
\ \ \ \ \ \ \ \ \isacommand{using}\isamarkupfalse%
\ right{\isacharunderscore}{\kern0pt}coproj{\isacharunderscore}{\kern0pt}cfunc{\isacharunderscore}{\kern0pt}bowtie{\isacharunderscore}{\kern0pt}prod\ type{\isacharunderscore}{\kern0pt}assms\ \isacommand{by}\isamarkupfalse%
\ auto\isanewline
\ \ \ \ \ \ \isacommand{also}\isamarkupfalse%
\ \isacommand{have}\isamarkupfalse%
\ {\isachardoublequoteopen}{\isachardot}{\kern0pt}{\isachardot}{\kern0pt}{\isachardot}{\kern0pt}\ {\isacharequal}{\kern0pt}\ right{\isacharunderscore}{\kern0pt}coproj\ Y\ W\ {\isasymcirc}\isactrlsub c\ g\ {\isasymcirc}\isactrlsub c\ z{\isachardoublequoteclose}\isanewline
\ \ \ \ \ \ \ \ \isacommand{using}\isamarkupfalse%
\ comp{\isacharunderscore}{\kern0pt}associative{\isadigit{2}}\ type{\isacharunderscore}{\kern0pt}assms{\isacharparenleft}{\kern0pt}{\isadigit{2}}{\isacharparenright}{\kern0pt}\ z{\isacharunderscore}{\kern0pt}def\ \isacommand{by}\isamarkupfalse%
\ {\isacharparenleft}{\kern0pt}typecheck{\isacharunderscore}{\kern0pt}cfuncs{\isacharcomma}{\kern0pt}\ auto{\isacharparenright}{\kern0pt}\isanewline
\ \ \ \ \ \ \isacommand{then}\isamarkupfalse%
\ \isacommand{show}\isamarkupfalse%
\ False\isanewline
\ \ \ \ \ \ \ \ \isacommand{using}\isamarkupfalse%
\ calculation\ comp{\isacharunderscore}{\kern0pt}type\ coproducts{\isacharunderscore}{\kern0pt}disjoint\ type{\isacharunderscore}{\kern0pt}assms{\isacharparenleft}{\kern0pt}{\isadigit{2}}{\isacharparenright}{\kern0pt}\ y{\isacharunderscore}{\kern0pt}type{\isadigit{2}}\ z{\isacharunderscore}{\kern0pt}def\ \isacommand{by}\isamarkupfalse%
\ auto\isanewline
\ \ \ \isacommand{qed}\isamarkupfalse%
\isanewline
\ \ \ \isacommand{then}\isamarkupfalse%
\ \isacommand{show}\isamarkupfalse%
\ {\isacharquery}{\kern0pt}thesis\isanewline
\ \ \ \ \ \isacommand{by}\isamarkupfalse%
\ simp\isanewline
\ \isacommand{qed}\isamarkupfalse%
\isanewline
\isacommand{next}\isamarkupfalse%
\isanewline
\ \ \isacommand{fix}\isamarkupfalse%
\ y\isanewline
\ \ \isacommand{assume}\isamarkupfalse%
\ y{\isacharunderscore}{\kern0pt}type{\isacharcolon}{\kern0pt}\ {\isachardoublequoteopen}y\ {\isasymin}\isactrlsub c\ codomain\ g{\isachardoublequoteclose}\isanewline
\ \ \isacommand{then}\isamarkupfalse%
\ \isacommand{have}\isamarkupfalse%
\ y{\isacharunderscore}{\kern0pt}type{\isadigit{2}}{\isacharcolon}{\kern0pt}\ {\isachardoublequoteopen}y\ {\isasymin}\isactrlsub c\ W{\isachardoublequoteclose}\isanewline
\ \ \ \ \isacommand{using}\isamarkupfalse%
\ cfunc{\isacharunderscore}{\kern0pt}type{\isacharunderscore}{\kern0pt}def\ type{\isacharunderscore}{\kern0pt}assms{\isacharparenleft}{\kern0pt}{\isadigit{2}}{\isacharparenright}{\kern0pt}\ \isacommand{by}\isamarkupfalse%
\ auto\ \isanewline
\ \ \isacommand{then}\isamarkupfalse%
\ \isacommand{have}\isamarkupfalse%
\ coproj{\isacharunderscore}{\kern0pt}y{\isacharunderscore}{\kern0pt}type{\isacharcolon}{\kern0pt}\ {\isachardoublequoteopen}{\isacharparenleft}{\kern0pt}right{\isacharunderscore}{\kern0pt}coproj\ Y\ W{\isacharparenright}{\kern0pt}\ {\isasymcirc}\isactrlsub c\ y\ {\isasymin}\isactrlsub c\ {\isacharparenleft}{\kern0pt}Y\ {\isasymCoprod}\ W{\isacharparenright}{\kern0pt}{\isachardoublequoteclose}\ \isanewline
\ \ \ \ \isacommand{using}\isamarkupfalse%
\ cfunc{\isacharunderscore}{\kern0pt}type{\isacharunderscore}{\kern0pt}def\ comp{\isacharunderscore}{\kern0pt}type\ right{\isacharunderscore}{\kern0pt}proj{\isacharunderscore}{\kern0pt}type\ type{\isacharunderscore}{\kern0pt}assms{\isacharparenleft}{\kern0pt}{\isadigit{2}}{\isacharparenright}{\kern0pt}\ \isacommand{by}\isamarkupfalse%
\ auto\isanewline
\ \ \isacommand{have}\isamarkupfalse%
\ fg{\isacharunderscore}{\kern0pt}type{\isacharcolon}{\kern0pt}\ {\isachardoublequoteopen}{\isacharparenleft}{\kern0pt}f\ {\isasymbowtie}\isactrlsub f\ g{\isacharparenright}{\kern0pt}\ {\isacharcolon}{\kern0pt}\ X\ {\isasymCoprod}\ Z\ {\isasymrightarrow}\ Y\ {\isasymCoprod}\ W{\isachardoublequoteclose}\isanewline
\ \ \ \ \isacommand{by}\isamarkupfalse%
\ {\isacharparenleft}{\kern0pt}simp\ add{\isacharcolon}{\kern0pt}\ cfunc{\isacharunderscore}{\kern0pt}bowtie{\isacharunderscore}{\kern0pt}prod{\isacharunderscore}{\kern0pt}type\ type{\isacharunderscore}{\kern0pt}assms{\isacharparenright}{\kern0pt}\isanewline
\ \ \isacommand{obtain}\isamarkupfalse%
\ xz\ \isakeyword{where}\ xz{\isacharunderscore}{\kern0pt}def{\isacharcolon}{\kern0pt}\ {\isachardoublequoteopen}xz\ {\isasymin}\isactrlsub c\ X\ {\isasymCoprod}\ Z\ {\isasymand}\ {\isacharparenleft}{\kern0pt}f\ {\isasymbowtie}\isactrlsub f\ g{\isacharparenright}{\kern0pt}\ {\isasymcirc}\isactrlsub c\ xz\ {\isacharequal}{\kern0pt}\ \ right{\isacharunderscore}{\kern0pt}coproj\ Y\ W\ {\isasymcirc}\isactrlsub c\ y{\isachardoublequoteclose}\isanewline
\ \ \ \ \isacommand{using}\isamarkupfalse%
\ fg{\isacharunderscore}{\kern0pt}type\ y{\isacharunderscore}{\kern0pt}type{\isadigit{2}}\ cfunc{\isacharunderscore}{\kern0pt}type{\isacharunderscore}{\kern0pt}def\ inj{\isacharunderscore}{\kern0pt}f{\isacharunderscore}{\kern0pt}bowtie{\isacharunderscore}{\kern0pt}g\ surjective{\isacharunderscore}{\kern0pt}def\ \isacommand{by}\isamarkupfalse%
\ {\isacharparenleft}{\kern0pt}typecheck{\isacharunderscore}{\kern0pt}cfuncs{\isacharcomma}{\kern0pt}\ auto{\isacharparenright}{\kern0pt}\isanewline
\ \ \isacommand{then}\isamarkupfalse%
\ \isacommand{have}\isamarkupfalse%
\ xz{\isacharunderscore}{\kern0pt}form{\isacharcolon}{\kern0pt}\ {\isachardoublequoteopen}{\isacharparenleft}{\kern0pt}{\isasymexists}\ x{\isachardot}{\kern0pt}\ x\ {\isasymin}\isactrlsub c\ X\ {\isasymand}\ left{\isacharunderscore}{\kern0pt}coproj\ X\ Z\ {\isasymcirc}\isactrlsub c\ x\ {\isacharequal}{\kern0pt}\ \ \ xz{\isacharparenright}{\kern0pt}\ {\isasymor}\ \ \isanewline
\ \ \ \ \ \ \ \ \ \ \ \ \ \ \ \ \ \ \ \ \ \ {\isacharparenleft}{\kern0pt}{\isasymexists}\ z{\isachardot}{\kern0pt}\ z\ {\isasymin}\isactrlsub c\ Z\ {\isasymand}\ right{\isacharunderscore}{\kern0pt}coproj\ X\ Z\ {\isasymcirc}\isactrlsub c\ z\ {\isacharequal}{\kern0pt}\ \ xz{\isacharparenright}{\kern0pt}{\isachardoublequoteclose}\isanewline
\ \ \ \ \isacommand{using}\isamarkupfalse%
\ coprojs{\isacharunderscore}{\kern0pt}jointly{\isacharunderscore}{\kern0pt}surj\ xz{\isacharunderscore}{\kern0pt}def\ \isacommand{by}\isamarkupfalse%
\ {\isacharparenleft}{\kern0pt}typecheck{\isacharunderscore}{\kern0pt}cfuncs{\isacharcomma}{\kern0pt}\ blast{\isacharparenright}{\kern0pt}\isanewline
\ \ \isacommand{show}\isamarkupfalse%
\ {\isachardoublequoteopen}{\isasymexists}x{\isachardot}{\kern0pt}\ x\ {\isasymin}\isactrlsub c\ domain\ g\ {\isasymand}\ g\ {\isasymcirc}\isactrlsub c\ x\ {\isacharequal}{\kern0pt}\ y{\isachardoublequoteclose}\isanewline
\ \ \isacommand{proof}\isamarkupfalse%
{\isacharparenleft}{\kern0pt}cases\ {\isachardoublequoteopen}{\isasymexists}\ x{\isachardot}{\kern0pt}\ x\ {\isasymin}\isactrlsub c\ X\ {\isasymand}\ left{\isacharunderscore}{\kern0pt}coproj\ X\ Z\ {\isasymcirc}\isactrlsub c\ x\ {\isacharequal}{\kern0pt}\ \ \ xz{\isachardoublequoteclose}{\isacharparenright}{\kern0pt}\isanewline
\ \ \ \ \isacommand{assume}\isamarkupfalse%
\ {\isachardoublequoteopen}{\isasymexists}\ x{\isachardot}{\kern0pt}\ x\ {\isasymin}\isactrlsub c\ X\ {\isasymand}\ left{\isacharunderscore}{\kern0pt}coproj\ X\ Z\ {\isasymcirc}\isactrlsub c\ x\ {\isacharequal}{\kern0pt}\ xz{\isachardoublequoteclose}\isanewline
\ \ \ \ \isacommand{then}\isamarkupfalse%
\ \isacommand{obtain}\isamarkupfalse%
\ x\ \isakeyword{where}\ x{\isacharunderscore}{\kern0pt}def{\isacharcolon}{\kern0pt}\ {\isachardoublequoteopen}x\ {\isasymin}\isactrlsub c\ X\ {\isasymand}\ left{\isacharunderscore}{\kern0pt}coproj\ X\ Z\ {\isasymcirc}\isactrlsub c\ x\ {\isacharequal}{\kern0pt}\ xz{\isachardoublequoteclose}\isanewline
\ \ \ \ \ \ \isacommand{by}\isamarkupfalse%
\ blast\isanewline
\ \ \ \ \isacommand{have}\isamarkupfalse%
\ False\isanewline
\ \ \ \ \isacommand{proof}\isamarkupfalse%
\ {\isacharminus}{\kern0pt}\ \isanewline
\ \ \ \ \ \ \isacommand{have}\isamarkupfalse%
\ {\isachardoublequoteopen}right{\isacharunderscore}{\kern0pt}coproj\ Y\ W\ {\isasymcirc}\isactrlsub c\ y\ {\isacharequal}{\kern0pt}\ {\isacharparenleft}{\kern0pt}f\ {\isasymbowtie}\isactrlsub f\ g{\isacharparenright}{\kern0pt}\ {\isasymcirc}\isactrlsub c\ xz{\isachardoublequoteclose}\isanewline
\ \ \ \ \ \ \ \ \isacommand{by}\isamarkupfalse%
\ {\isacharparenleft}{\kern0pt}simp\ add{\isacharcolon}{\kern0pt}\ xz{\isacharunderscore}{\kern0pt}def{\isacharparenright}{\kern0pt}\isanewline
\ \ \ \ \ \ \isacommand{also}\isamarkupfalse%
\ \isacommand{have}\isamarkupfalse%
\ {\isachardoublequoteopen}{\isachardot}{\kern0pt}{\isachardot}{\kern0pt}{\isachardot}{\kern0pt}\ {\isacharequal}{\kern0pt}\ {\isacharparenleft}{\kern0pt}f\ {\isasymbowtie}\isactrlsub f\ g{\isacharparenright}{\kern0pt}\ {\isasymcirc}\isactrlsub c\ left{\isacharunderscore}{\kern0pt}coproj\ X\ Z\ {\isasymcirc}\isactrlsub c\ x{\isachardoublequoteclose}\isanewline
\ \ \ \ \ \ \ \ \isacommand{by}\isamarkupfalse%
\ {\isacharparenleft}{\kern0pt}simp\ add{\isacharcolon}{\kern0pt}\ x{\isacharunderscore}{\kern0pt}def{\isacharparenright}{\kern0pt}\isanewline
\ \ \ \ \ \ \isacommand{also}\isamarkupfalse%
\ \isacommand{have}\isamarkupfalse%
\ {\isachardoublequoteopen}{\isachardot}{\kern0pt}{\isachardot}{\kern0pt}{\isachardot}{\kern0pt}\ {\isacharequal}{\kern0pt}\ {\isacharparenleft}{\kern0pt}{\isacharparenleft}{\kern0pt}f\ {\isasymbowtie}\isactrlsub f\ g{\isacharparenright}{\kern0pt}\ {\isasymcirc}\isactrlsub c\ left{\isacharunderscore}{\kern0pt}coproj\ X\ Z{\isacharparenright}{\kern0pt}\ {\isasymcirc}\isactrlsub c\ x{\isachardoublequoteclose}\isanewline
\ \ \ \ \ \ \ \ \isacommand{using}\isamarkupfalse%
\ \ comp{\isacharunderscore}{\kern0pt}associative{\isadigit{2}}\ fg{\isacharunderscore}{\kern0pt}type\ x{\isacharunderscore}{\kern0pt}def\ \isacommand{by}\isamarkupfalse%
\ {\isacharparenleft}{\kern0pt}typecheck{\isacharunderscore}{\kern0pt}cfuncs{\isacharcomma}{\kern0pt}\ auto{\isacharparenright}{\kern0pt}\isanewline
\ \ \ \ \ \ \isacommand{also}\isamarkupfalse%
\ \isacommand{have}\isamarkupfalse%
\ {\isachardoublequoteopen}{\isachardot}{\kern0pt}{\isachardot}{\kern0pt}{\isachardot}{\kern0pt}\ {\isacharequal}{\kern0pt}\ {\isacharparenleft}{\kern0pt}left{\isacharunderscore}{\kern0pt}coproj\ Y\ W\ {\isasymcirc}\isactrlsub c\ f{\isacharparenright}{\kern0pt}\ {\isasymcirc}\isactrlsub c\ x{\isachardoublequoteclose}\isanewline
\ \ \ \ \ \ \ \ \isacommand{using}\isamarkupfalse%
\ left{\isacharunderscore}{\kern0pt}coproj{\isacharunderscore}{\kern0pt}cfunc{\isacharunderscore}{\kern0pt}bowtie{\isacharunderscore}{\kern0pt}prod\ type{\isacharunderscore}{\kern0pt}assms\ \isacommand{by}\isamarkupfalse%
\ auto\isanewline
\ \ \ \ \ \ \isacommand{also}\isamarkupfalse%
\ \isacommand{have}\isamarkupfalse%
\ {\isachardoublequoteopen}{\isachardot}{\kern0pt}{\isachardot}{\kern0pt}{\isachardot}{\kern0pt}\ {\isacharequal}{\kern0pt}\ left{\isacharunderscore}{\kern0pt}coproj\ Y\ W\ {\isasymcirc}\isactrlsub c\ f\ {\isasymcirc}\isactrlsub c\ x{\isachardoublequoteclose}\isanewline
\ \ \ \ \ \ \ \ \isacommand{using}\isamarkupfalse%
\ comp{\isacharunderscore}{\kern0pt}associative{\isadigit{2}}\ type{\isacharunderscore}{\kern0pt}assms{\isacharparenleft}{\kern0pt}{\isadigit{1}}{\isacharparenright}{\kern0pt}\ x{\isacharunderscore}{\kern0pt}def\ \isacommand{by}\isamarkupfalse%
\ {\isacharparenleft}{\kern0pt}typecheck{\isacharunderscore}{\kern0pt}cfuncs{\isacharcomma}{\kern0pt}\ auto{\isacharparenright}{\kern0pt}\isanewline
\ \ \ \ \ \ \isacommand{then}\isamarkupfalse%
\ \isacommand{show}\isamarkupfalse%
\ False\isanewline
\ \ \ \ \ \ \ \ \isacommand{by}\isamarkupfalse%
\ {\isacharparenleft}{\kern0pt}metis\ calculation\ comp{\isacharunderscore}{\kern0pt}type\ coproducts{\isacharunderscore}{\kern0pt}disjoint\ type{\isacharunderscore}{\kern0pt}assms{\isacharparenleft}{\kern0pt}{\isadigit{1}}{\isacharparenright}{\kern0pt}\ x{\isacharunderscore}{\kern0pt}def\ y{\isacharunderscore}{\kern0pt}type{\isadigit{2}}{\isacharparenright}{\kern0pt}\isanewline
\ \ \ \ \isacommand{qed}\isamarkupfalse%
\isanewline
\ \ \ \ \isacommand{then}\isamarkupfalse%
\ \isacommand{show}\isamarkupfalse%
\ {\isacharquery}{\kern0pt}thesis\isanewline
\ \ \ \ \ \ \isacommand{by}\isamarkupfalse%
\ simp\isanewline
\isacommand{next}\isamarkupfalse%
\isanewline
\ \ \isacommand{assume}\isamarkupfalse%
\ {\isachardoublequoteopen}{\isasymnexists}x{\isachardot}{\kern0pt}\ x\ {\isasymin}\isactrlsub c\ X\ {\isasymand}\ left{\isacharunderscore}{\kern0pt}coproj\ X\ Z\ {\isasymcirc}\isactrlsub c\ x\ {\isacharequal}{\kern0pt}\ xz{\isachardoublequoteclose}\isanewline
\ \ \isacommand{then}\isamarkupfalse%
\ \isacommand{obtain}\isamarkupfalse%
\ z\ \isakeyword{where}\ z{\isacharunderscore}{\kern0pt}def{\isacharcolon}{\kern0pt}\ {\isachardoublequoteopen}z\ {\isasymin}\isactrlsub c\ Z\ {\isasymand}\ right{\isacharunderscore}{\kern0pt}coproj\ X\ Z\ {\isasymcirc}\isactrlsub c\ z\ {\isacharequal}{\kern0pt}\ xz{\isachardoublequoteclose}\isanewline
\ \ \ \ \isacommand{using}\isamarkupfalse%
\ xz{\isacharunderscore}{\kern0pt}form\ \isacommand{by}\isamarkupfalse%
\ blast\isanewline
\ \ \isacommand{have}\isamarkupfalse%
\ {\isachardoublequoteopen}g\ {\isasymcirc}\isactrlsub c\ z\ {\isacharequal}{\kern0pt}\ y{\isachardoublequoteclose}\isanewline
\ \ \ \ \isacommand{proof}\isamarkupfalse%
\ {\isacharminus}{\kern0pt}\ \isanewline
\ \ \ \ \ \ \isacommand{have}\isamarkupfalse%
\ {\isachardoublequoteopen}right{\isacharunderscore}{\kern0pt}coproj\ Y\ W\ {\isasymcirc}\isactrlsub c\ y\ {\isacharequal}{\kern0pt}\ {\isacharparenleft}{\kern0pt}f\ {\isasymbowtie}\isactrlsub f\ g{\isacharparenright}{\kern0pt}\ {\isasymcirc}\isactrlsub c\ xz{\isachardoublequoteclose}\isanewline
\ \ \ \ \ \ \ \ \isacommand{by}\isamarkupfalse%
\ {\isacharparenleft}{\kern0pt}simp\ add{\isacharcolon}{\kern0pt}\ xz{\isacharunderscore}{\kern0pt}def{\isacharparenright}{\kern0pt}\ \ \ \ \ \ \ \ \ \isanewline
\ \ \ \ \ \ \isacommand{also}\isamarkupfalse%
\ \isacommand{have}\isamarkupfalse%
\ {\isachardoublequoteopen}{\isachardot}{\kern0pt}{\isachardot}{\kern0pt}{\isachardot}{\kern0pt}\ {\isacharequal}{\kern0pt}\ {\isacharparenleft}{\kern0pt}f\ {\isasymbowtie}\isactrlsub f\ g{\isacharparenright}{\kern0pt}\ {\isasymcirc}\isactrlsub c\ right{\isacharunderscore}{\kern0pt}coproj\ X\ Z\ {\isasymcirc}\isactrlsub c\ z{\isachardoublequoteclose}\isanewline
\ \ \ \ \ \ \ \ \isacommand{by}\isamarkupfalse%
\ {\isacharparenleft}{\kern0pt}simp\ add{\isacharcolon}{\kern0pt}\ z{\isacharunderscore}{\kern0pt}def{\isacharparenright}{\kern0pt}\isanewline
\ \ \ \ \ \ \isacommand{also}\isamarkupfalse%
\ \isacommand{have}\isamarkupfalse%
\ {\isachardoublequoteopen}{\isachardot}{\kern0pt}{\isachardot}{\kern0pt}{\isachardot}{\kern0pt}\ {\isacharequal}{\kern0pt}\ {\isacharparenleft}{\kern0pt}{\isacharparenleft}{\kern0pt}f\ {\isasymbowtie}\isactrlsub f\ g{\isacharparenright}{\kern0pt}\ {\isasymcirc}\isactrlsub c\ right{\isacharunderscore}{\kern0pt}coproj\ X\ Z{\isacharparenright}{\kern0pt}\ {\isasymcirc}\isactrlsub c\ z{\isachardoublequoteclose}\isanewline
\ \ \ \ \ \ \ \ \isacommand{using}\isamarkupfalse%
\ comp{\isacharunderscore}{\kern0pt}associative{\isadigit{2}}\ fg{\isacharunderscore}{\kern0pt}type\ z{\isacharunderscore}{\kern0pt}def\ \isacommand{by}\isamarkupfalse%
\ {\isacharparenleft}{\kern0pt}typecheck{\isacharunderscore}{\kern0pt}cfuncs{\isacharcomma}{\kern0pt}\ auto{\isacharparenright}{\kern0pt}\isanewline
\ \ \ \ \ \ \isacommand{also}\isamarkupfalse%
\ \isacommand{have}\isamarkupfalse%
\ {\isachardoublequoteopen}{\isachardot}{\kern0pt}{\isachardot}{\kern0pt}{\isachardot}{\kern0pt}\ {\isacharequal}{\kern0pt}\ {\isacharparenleft}{\kern0pt}right{\isacharunderscore}{\kern0pt}coproj\ Y\ W\ {\isasymcirc}\isactrlsub c\ g{\isacharparenright}{\kern0pt}\ {\isasymcirc}\isactrlsub c\ z{\isachardoublequoteclose}\isanewline
\ \ \ \ \ \ \ \ \isacommand{using}\isamarkupfalse%
\ right{\isacharunderscore}{\kern0pt}coproj{\isacharunderscore}{\kern0pt}cfunc{\isacharunderscore}{\kern0pt}bowtie{\isacharunderscore}{\kern0pt}prod\ type{\isacharunderscore}{\kern0pt}assms\ \isacommand{by}\isamarkupfalse%
\ auto\isanewline
\ \ \ \ \ \ \isacommand{also}\isamarkupfalse%
\ \isacommand{have}\isamarkupfalse%
\ {\isachardoublequoteopen}{\isachardot}{\kern0pt}{\isachardot}{\kern0pt}{\isachardot}{\kern0pt}\ {\isacharequal}{\kern0pt}\ right{\isacharunderscore}{\kern0pt}coproj\ Y\ W\ {\isasymcirc}\isactrlsub c\ g\ {\isasymcirc}\isactrlsub c\ z{\isachardoublequoteclose}\isanewline
\ \ \ \ \ \ \ \ \isacommand{using}\isamarkupfalse%
\ comp{\isacharunderscore}{\kern0pt}associative{\isadigit{2}}\ type{\isacharunderscore}{\kern0pt}assms{\isacharparenleft}{\kern0pt}{\isadigit{2}}{\isacharparenright}{\kern0pt}\ z{\isacharunderscore}{\kern0pt}def\ \isacommand{by}\isamarkupfalse%
\ {\isacharparenleft}{\kern0pt}typecheck{\isacharunderscore}{\kern0pt}cfuncs{\isacharcomma}{\kern0pt}\ auto{\isacharparenright}{\kern0pt}\isanewline
\ \ \ \ \ \ \isacommand{then}\isamarkupfalse%
\ \isacommand{show}\isamarkupfalse%
\ {\isacharquery}{\kern0pt}thesis\isanewline
\ \ \ \ \ \ \ \ \isacommand{by}\isamarkupfalse%
\ {\isacharparenleft}{\kern0pt}metis\ calculation\ cfunc{\isacharunderscore}{\kern0pt}type{\isacharunderscore}{\kern0pt}def\ codomain{\isacharunderscore}{\kern0pt}comp\ monomorphism{\isacharunderscore}{\kern0pt}def\ \isanewline
\ \ \ \ \ \ \ \ \ \ \ right{\isacharunderscore}{\kern0pt}coproj{\isacharunderscore}{\kern0pt}are{\isacharunderscore}{\kern0pt}monomorphisms\ right{\isacharunderscore}{\kern0pt}proj{\isacharunderscore}{\kern0pt}type\ type{\isacharunderscore}{\kern0pt}assms{\isacharparenleft}{\kern0pt}{\isadigit{2}}{\isacharparenright}{\kern0pt}\ y{\isacharunderscore}{\kern0pt}type{\isadigit{2}}\ z{\isacharunderscore}{\kern0pt}def{\isacharparenright}{\kern0pt}\isanewline
\ \ \ \ \isacommand{qed}\isamarkupfalse%
\isanewline
\ \ \ \ \isacommand{then}\isamarkupfalse%
\ \isacommand{show}\isamarkupfalse%
\ {\isacharquery}{\kern0pt}thesis\isanewline
\ \ \ \ \ \ \isacommand{using}\isamarkupfalse%
\ cfunc{\isacharunderscore}{\kern0pt}type{\isacharunderscore}{\kern0pt}def\ type{\isacharunderscore}{\kern0pt}assms{\isacharparenleft}{\kern0pt}{\isadigit{2}}{\isacharparenright}{\kern0pt}\ z{\isacharunderscore}{\kern0pt}def\ \isacommand{by}\isamarkupfalse%
\ auto\isanewline
\ \isacommand{qed}\isamarkupfalse%
\isanewline
\isacommand{qed}\isamarkupfalse%
%
\endisatagproof
{\isafoldproof}%
%
\isadelimproof
%
\endisadelimproof
%
\isadelimdocument
%
\endisadelimdocument
%
\isatagdocument
%
\isamarkupsubsection{Case Bool%
}
\isamarkuptrue%
%
\endisatagdocument
{\isafolddocument}%
%
\isadelimdocument
%
\endisadelimdocument
\isacommand{definition}\isamarkupfalse%
\ case{\isacharunderscore}{\kern0pt}bool\ {\isacharcolon}{\kern0pt}{\isacharcolon}{\kern0pt}\ {\isachardoublequoteopen}cfunc{\isachardoublequoteclose}\ \isakeyword{where}\isanewline
\ \ {\isachardoublequoteopen}case{\isacharunderscore}{\kern0pt}bool\ {\isacharequal}{\kern0pt}\ {\isacharparenleft}{\kern0pt}THE\ f{\isachardot}{\kern0pt}\ f\ {\isacharcolon}{\kern0pt}\ {\isasymOmega}\ {\isasymrightarrow}\ {\isacharparenleft}{\kern0pt}one\ {\isasymCoprod}\ one{\isacharparenright}{\kern0pt}\ {\isasymand}\ \ \isanewline
\ \ \ \ {\isacharparenleft}{\kern0pt}{\isasymt}\ {\isasymamalg}\ {\isasymf}{\isacharparenright}{\kern0pt}\ {\isasymcirc}\isactrlsub c\ f\ {\isacharequal}{\kern0pt}\ id\ {\isasymOmega}\ {\isasymand}\ f\ {\isasymcirc}\isactrlsub c\ {\isacharparenleft}{\kern0pt}{\isasymt}\ {\isasymamalg}\ {\isasymf}{\isacharparenright}{\kern0pt}\ {\isacharequal}{\kern0pt}\ id\ {\isacharparenleft}{\kern0pt}one\ {\isasymCoprod}\ one{\isacharparenright}{\kern0pt}{\isacharparenright}{\kern0pt}{\isachardoublequoteclose}\isanewline
\isanewline
\isacommand{lemma}\isamarkupfalse%
\ case{\isacharunderscore}{\kern0pt}bool{\isacharunderscore}{\kern0pt}def{\isadigit{2}}{\isacharcolon}{\kern0pt}\isanewline
\ \ {\isachardoublequoteopen}case{\isacharunderscore}{\kern0pt}bool\ {\isacharcolon}{\kern0pt}\ {\isasymOmega}\ {\isasymrightarrow}\ {\isacharparenleft}{\kern0pt}one\ {\isasymCoprod}\ one{\isacharparenright}{\kern0pt}\ {\isasymand}\ \ \isanewline
\ \ \ \ {\isacharparenleft}{\kern0pt}{\isasymt}\ {\isasymamalg}\ {\isasymf}{\isacharparenright}{\kern0pt}\ {\isasymcirc}\isactrlsub c\ case{\isacharunderscore}{\kern0pt}bool\ {\isacharequal}{\kern0pt}\ id\ {\isasymOmega}\ {\isasymand}\ case{\isacharunderscore}{\kern0pt}bool\ {\isasymcirc}\isactrlsub c\ {\isacharparenleft}{\kern0pt}{\isasymt}\ {\isasymamalg}\ {\isasymf}{\isacharparenright}{\kern0pt}\ {\isacharequal}{\kern0pt}\ id\ {\isacharparenleft}{\kern0pt}one\ {\isasymCoprod}\ one{\isacharparenright}{\kern0pt}{\isachardoublequoteclose}\isanewline
%
\isadelimproof
%
\endisadelimproof
%
\isatagproof
\isacommand{proof}\isamarkupfalse%
\ {\isacharparenleft}{\kern0pt}unfold\ case{\isacharunderscore}{\kern0pt}bool{\isacharunderscore}{\kern0pt}def{\isacharcomma}{\kern0pt}\ rule\ theI{\isacharprime}{\kern0pt}{\isacharcomma}{\kern0pt}\ auto{\isacharparenright}{\kern0pt}\isanewline
\ \ \isacommand{show}\isamarkupfalse%
\ {\isachardoublequoteopen}{\isasymexists}x{\isachardot}{\kern0pt}\ x\ {\isacharcolon}{\kern0pt}\ {\isasymOmega}\ {\isasymrightarrow}\ one\ {\isasymCoprod}\ one\ {\isasymand}\ {\isasymt}\ {\isasymamalg}\ {\isasymf}\ {\isasymcirc}\isactrlsub c\ x\ {\isacharequal}{\kern0pt}\ id\isactrlsub c\ {\isasymOmega}\ {\isasymand}\ x\ {\isasymcirc}\isactrlsub c\ {\isasymt}\ {\isasymamalg}\ {\isasymf}\ {\isacharequal}{\kern0pt}\ id\isactrlsub c\ {\isacharparenleft}{\kern0pt}one\ {\isasymCoprod}\ one{\isacharparenright}{\kern0pt}{\isachardoublequoteclose}\isanewline
\ \ \ \ \isacommand{using}\isamarkupfalse%
\ truth{\isacharunderscore}{\kern0pt}value{\isacharunderscore}{\kern0pt}set{\isacharunderscore}{\kern0pt}iso{\isacharunderscore}{\kern0pt}{\isadigit{1}}u{\isadigit{1}}\ \isacommand{unfolding}\isamarkupfalse%
\ isomorphism{\isacharunderscore}{\kern0pt}def\isanewline
\ \ \ \ \isacommand{by}\isamarkupfalse%
\ {\isacharparenleft}{\kern0pt}auto{\isacharcomma}{\kern0pt}\ rule{\isacharunderscore}{\kern0pt}tac\ x{\isacharequal}{\kern0pt}g\ \isakeyword{in}\ exI{\isacharcomma}{\kern0pt}\ typecheck{\isacharunderscore}{\kern0pt}cfuncs{\isacharcomma}{\kern0pt}\ simp\ add{\isacharcolon}{\kern0pt}\ cfunc{\isacharunderscore}{\kern0pt}type{\isacharunderscore}{\kern0pt}def{\isacharparenright}{\kern0pt}\isanewline
\isacommand{next}\isamarkupfalse%
\isanewline
\ \ \isacommand{fix}\isamarkupfalse%
\ x\ y\isanewline
\ \ \isacommand{assume}\isamarkupfalse%
\ x{\isacharunderscore}{\kern0pt}type{\isacharbrackleft}{\kern0pt}type{\isacharunderscore}{\kern0pt}rule{\isacharbrackright}{\kern0pt}{\isacharcolon}{\kern0pt}\ {\isachardoublequoteopen}x\ {\isacharcolon}{\kern0pt}\ {\isasymOmega}\ {\isasymrightarrow}\ one\ {\isasymCoprod}\ one{\isachardoublequoteclose}\ \isakeyword{and}\ y{\isacharunderscore}{\kern0pt}type{\isacharbrackleft}{\kern0pt}type{\isacharunderscore}{\kern0pt}rule{\isacharbrackright}{\kern0pt}{\isacharcolon}{\kern0pt}\ {\isachardoublequoteopen}y\ {\isacharcolon}{\kern0pt}\ {\isasymOmega}\ {\isasymrightarrow}\ one\ {\isasymCoprod}\ one{\isachardoublequoteclose}\isanewline
\ \ \isacommand{assume}\isamarkupfalse%
\ x{\isacharunderscore}{\kern0pt}left{\isacharunderscore}{\kern0pt}inv{\isacharcolon}{\kern0pt}\ {\isachardoublequoteopen}{\isasymt}\ {\isasymamalg}\ {\isasymf}\ {\isasymcirc}\isactrlsub c\ x\ {\isacharequal}{\kern0pt}\ id\isactrlsub c\ {\isasymOmega}{\isachardoublequoteclose}\isanewline
\ \ \isacommand{assume}\isamarkupfalse%
\ {\isachardoublequoteopen}x\ {\isasymcirc}\isactrlsub c\ {\isasymt}\ {\isasymamalg}\ {\isasymf}\ {\isacharequal}{\kern0pt}\ id\isactrlsub c\ {\isacharparenleft}{\kern0pt}one\ {\isasymCoprod}\ one{\isacharparenright}{\kern0pt}{\isachardoublequoteclose}\ {\isachardoublequoteopen}y\ {\isasymcirc}\isactrlsub c\ {\isasymt}\ {\isasymamalg}\ {\isasymf}\ {\isacharequal}{\kern0pt}\ id\isactrlsub c\ {\isacharparenleft}{\kern0pt}one\ {\isasymCoprod}\ one{\isacharparenright}{\kern0pt}{\isachardoublequoteclose}\isanewline
\ \ \isacommand{then}\isamarkupfalse%
\ \isacommand{have}\isamarkupfalse%
\ {\isachardoublequoteopen}x\ {\isasymcirc}\isactrlsub c\ {\isasymt}\ {\isasymamalg}\ {\isasymf}\ {\isacharequal}{\kern0pt}\ y\ {\isasymcirc}\isactrlsub c\ {\isasymt}\ {\isasymamalg}\ {\isasymf}{\isachardoublequoteclose}\isanewline
\ \ \ \ \isacommand{by}\isamarkupfalse%
\ auto\isanewline
\ \ \isacommand{then}\isamarkupfalse%
\ \isacommand{have}\isamarkupfalse%
\ {\isachardoublequoteopen}x\ {\isasymcirc}\isactrlsub c\ {\isasymt}\ {\isasymamalg}\ {\isasymf}\ {\isasymcirc}\isactrlsub c\ x\ {\isacharequal}{\kern0pt}\ y\ {\isasymcirc}\isactrlsub c\ {\isasymt}\ {\isasymamalg}\ {\isasymf}\ {\isasymcirc}\isactrlsub c\ x{\isachardoublequoteclose}\isanewline
\ \ \ \ \isacommand{by}\isamarkupfalse%
\ {\isacharparenleft}{\kern0pt}typecheck{\isacharunderscore}{\kern0pt}cfuncs{\isacharcomma}{\kern0pt}\ auto\ simp\ add{\isacharcolon}{\kern0pt}\ comp{\isacharunderscore}{\kern0pt}associative{\isadigit{2}}{\isacharparenright}{\kern0pt}\isanewline
\ \ \isacommand{then}\isamarkupfalse%
\ \isacommand{show}\isamarkupfalse%
\ {\isachardoublequoteopen}x\ {\isacharequal}{\kern0pt}\ y{\isachardoublequoteclose}\isanewline
\ \ \ \ \isacommand{using}\isamarkupfalse%
\ id{\isacharunderscore}{\kern0pt}right{\isacharunderscore}{\kern0pt}unit{\isadigit{2}}\ x{\isacharunderscore}{\kern0pt}left{\isacharunderscore}{\kern0pt}inv\ \isacommand{by}\isamarkupfalse%
\ {\isacharparenleft}{\kern0pt}typecheck{\isacharunderscore}{\kern0pt}cfuncs{\isacharunderscore}{\kern0pt}prems{\isacharcomma}{\kern0pt}\ auto{\isacharparenright}{\kern0pt}\isanewline
\isacommand{qed}\isamarkupfalse%
%
\endisatagproof
{\isafoldproof}%
%
\isadelimproof
\isanewline
%
\endisadelimproof
\isanewline
\isacommand{lemma}\isamarkupfalse%
\ case{\isacharunderscore}{\kern0pt}bool{\isacharunderscore}{\kern0pt}type{\isacharbrackleft}{\kern0pt}type{\isacharunderscore}{\kern0pt}rule{\isacharbrackright}{\kern0pt}{\isacharcolon}{\kern0pt}\ \isanewline
\ \ {\isachardoublequoteopen}case{\isacharunderscore}{\kern0pt}bool\ {\isacharcolon}{\kern0pt}\ {\isasymOmega}\ {\isasymrightarrow}\ one\ {\isasymCoprod}\ one{\isachardoublequoteclose}\isanewline
%
\isadelimproof
\ \ %
\endisadelimproof
%
\isatagproof
\isacommand{using}\isamarkupfalse%
\ case{\isacharunderscore}{\kern0pt}bool{\isacharunderscore}{\kern0pt}def{\isadigit{2}}\ \isacommand{by}\isamarkupfalse%
\ auto%
\endisatagproof
{\isafoldproof}%
%
\isadelimproof
\isanewline
%
\endisadelimproof
\isanewline
\isacommand{lemma}\isamarkupfalse%
\ case{\isacharunderscore}{\kern0pt}bool{\isacharunderscore}{\kern0pt}true{\isacharunderscore}{\kern0pt}coprod{\isacharunderscore}{\kern0pt}false{\isacharcolon}{\kern0pt}\isanewline
\ \ {\isachardoublequoteopen}case{\isacharunderscore}{\kern0pt}bool\ {\isasymcirc}\isactrlsub c\ {\isacharparenleft}{\kern0pt}{\isasymt}\ {\isasymamalg}\ {\isasymf}{\isacharparenright}{\kern0pt}\ {\isacharequal}{\kern0pt}\ id\ {\isacharparenleft}{\kern0pt}one\ {\isasymCoprod}\ one{\isacharparenright}{\kern0pt}{\isachardoublequoteclose}\isanewline
%
\isadelimproof
\ \ %
\endisadelimproof
%
\isatagproof
\isacommand{using}\isamarkupfalse%
\ case{\isacharunderscore}{\kern0pt}bool{\isacharunderscore}{\kern0pt}def{\isadigit{2}}\ \isacommand{by}\isamarkupfalse%
\ auto%
\endisatagproof
{\isafoldproof}%
%
\isadelimproof
\isanewline
%
\endisadelimproof
\isanewline
\isacommand{lemma}\isamarkupfalse%
\ true{\isacharunderscore}{\kern0pt}coprod{\isacharunderscore}{\kern0pt}false{\isacharunderscore}{\kern0pt}case{\isacharunderscore}{\kern0pt}bool{\isacharcolon}{\kern0pt}\isanewline
\ \ {\isachardoublequoteopen}{\isacharparenleft}{\kern0pt}{\isasymt}\ {\isasymamalg}\ {\isasymf}{\isacharparenright}{\kern0pt}\ {\isasymcirc}\isactrlsub c\ case{\isacharunderscore}{\kern0pt}bool\ {\isacharequal}{\kern0pt}\ id\ {\isasymOmega}{\isachardoublequoteclose}\isanewline
%
\isadelimproof
\ \ %
\endisadelimproof
%
\isatagproof
\isacommand{using}\isamarkupfalse%
\ case{\isacharunderscore}{\kern0pt}bool{\isacharunderscore}{\kern0pt}def{\isadigit{2}}\ \isacommand{by}\isamarkupfalse%
\ auto%
\endisatagproof
{\isafoldproof}%
%
\isadelimproof
\isanewline
%
\endisadelimproof
\isanewline
\isacommand{lemma}\isamarkupfalse%
\ case{\isacharunderscore}{\kern0pt}bool{\isacharunderscore}{\kern0pt}iso{\isacharcolon}{\kern0pt}\isanewline
\ \ {\isachardoublequoteopen}isomorphism\ case{\isacharunderscore}{\kern0pt}bool{\isachardoublequoteclose}\isanewline
%
\isadelimproof
\ \ %
\endisadelimproof
%
\isatagproof
\isacommand{using}\isamarkupfalse%
\ case{\isacharunderscore}{\kern0pt}bool{\isacharunderscore}{\kern0pt}def{\isadigit{2}}\ \isacommand{unfolding}\isamarkupfalse%
\ isomorphism{\isacharunderscore}{\kern0pt}def\isanewline
\ \ \isacommand{by}\isamarkupfalse%
\ {\isacharparenleft}{\kern0pt}rule{\isacharunderscore}{\kern0pt}tac\ x{\isacharequal}{\kern0pt}{\isachardoublequoteopen}{\isasymt}\ {\isasymamalg}\ {\isasymf}{\isachardoublequoteclose}\ \isakeyword{in}\ exI{\isacharcomma}{\kern0pt}\ typecheck{\isacharunderscore}{\kern0pt}cfuncs{\isacharcomma}{\kern0pt}\ auto\ simp\ add{\isacharcolon}{\kern0pt}\ cfunc{\isacharunderscore}{\kern0pt}type{\isacharunderscore}{\kern0pt}def{\isacharparenright}{\kern0pt}%
\endisatagproof
{\isafoldproof}%
%
\isadelimproof
\isanewline
%
\endisadelimproof
\isanewline
\isacommand{lemma}\isamarkupfalse%
\ case{\isacharunderscore}{\kern0pt}bool{\isacharunderscore}{\kern0pt}true{\isacharunderscore}{\kern0pt}and{\isacharunderscore}{\kern0pt}false{\isacharcolon}{\kern0pt}\isanewline
\ \ {\isachardoublequoteopen}{\isacharparenleft}{\kern0pt}case{\isacharunderscore}{\kern0pt}bool\ {\isasymcirc}\isactrlsub c\ {\isasymt}\ {\isacharequal}{\kern0pt}\ left{\isacharunderscore}{\kern0pt}coproj\ one\ one{\isacharparenright}{\kern0pt}\ {\isasymand}\ {\isacharparenleft}{\kern0pt}case{\isacharunderscore}{\kern0pt}bool\ {\isasymcirc}\isactrlsub c\ {\isasymf}\ {\isacharequal}{\kern0pt}\ right{\isacharunderscore}{\kern0pt}coproj\ one\ one{\isacharparenright}{\kern0pt}{\isachardoublequoteclose}\isanewline
%
\isadelimproof
%
\endisadelimproof
%
\isatagproof
\isacommand{proof}\isamarkupfalse%
\ {\isacharminus}{\kern0pt}\isanewline
\ \ \isacommand{have}\isamarkupfalse%
\ {\isachardoublequoteopen}{\isacharparenleft}{\kern0pt}left{\isacharunderscore}{\kern0pt}coproj\ one\ one{\isacharparenright}{\kern0pt}\ {\isasymamalg}\ \ {\isacharparenleft}{\kern0pt}right{\isacharunderscore}{\kern0pt}coproj\ one\ one{\isacharparenright}{\kern0pt}\ {\isacharequal}{\kern0pt}\ id{\isacharparenleft}{\kern0pt}one\ {\isasymCoprod}\ one{\isacharparenright}{\kern0pt}{\isachardoublequoteclose}\isanewline
\ \ \ \ \isacommand{by}\isamarkupfalse%
\ {\isacharparenleft}{\kern0pt}simp\ add{\isacharcolon}{\kern0pt}\ id{\isacharunderscore}{\kern0pt}coprod{\isacharparenright}{\kern0pt}\isanewline
\ \ \isacommand{also}\isamarkupfalse%
\ \isacommand{have}\isamarkupfalse%
\ {\isachardoublequoteopen}{\isachardot}{\kern0pt}{\isachardot}{\kern0pt}{\isachardot}{\kern0pt}\ {\isacharequal}{\kern0pt}\ case{\isacharunderscore}{\kern0pt}bool\ {\isasymcirc}\isactrlsub c\ {\isacharparenleft}{\kern0pt}{\isasymt}\ {\isasymamalg}\ {\isasymf}{\isacharparenright}{\kern0pt}{\isachardoublequoteclose}\isanewline
\ \ \ \ \isacommand{by}\isamarkupfalse%
\ {\isacharparenleft}{\kern0pt}simp\ add{\isacharcolon}{\kern0pt}\ case{\isacharunderscore}{\kern0pt}bool{\isacharunderscore}{\kern0pt}def{\isadigit{2}}{\isacharparenright}{\kern0pt}\isanewline
\ \ \isacommand{also}\isamarkupfalse%
\ \isacommand{have}\isamarkupfalse%
\ {\isachardoublequoteopen}{\isachardot}{\kern0pt}{\isachardot}{\kern0pt}{\isachardot}{\kern0pt}\ \ {\isacharequal}{\kern0pt}\ {\isacharparenleft}{\kern0pt}case{\isacharunderscore}{\kern0pt}bool\ {\isasymcirc}\isactrlsub c\ {\isasymt}{\isacharparenright}{\kern0pt}\ {\isasymamalg}\ {\isacharparenleft}{\kern0pt}case{\isacharunderscore}{\kern0pt}bool\ {\isasymcirc}\isactrlsub c\ {\isasymf}{\isacharparenright}{\kern0pt}{\isachardoublequoteclose}\isanewline
\ \ \ \ \isacommand{using}\isamarkupfalse%
\ case{\isacharunderscore}{\kern0pt}bool{\isacharunderscore}{\kern0pt}def{\isadigit{2}}\ cfunc{\isacharunderscore}{\kern0pt}coprod{\isacharunderscore}{\kern0pt}comp\ false{\isacharunderscore}{\kern0pt}func{\isacharunderscore}{\kern0pt}type\ true{\isacharunderscore}{\kern0pt}func{\isacharunderscore}{\kern0pt}type\ \isacommand{by}\isamarkupfalse%
\ auto\isanewline
\ \ \isacommand{then}\isamarkupfalse%
\ \isacommand{show}\isamarkupfalse%
\ {\isacharquery}{\kern0pt}thesis\ \isanewline
\ \ \ \ \isacommand{using}\isamarkupfalse%
\ \ calculation\ coprod{\isacharunderscore}{\kern0pt}eq{\isadigit{2}}\ \isacommand{by}\isamarkupfalse%
\ {\isacharparenleft}{\kern0pt}typecheck{\isacharunderscore}{\kern0pt}cfuncs{\isacharcomma}{\kern0pt}\ auto{\isacharparenright}{\kern0pt}\isanewline
\isacommand{qed}\isamarkupfalse%
%
\endisatagproof
{\isafoldproof}%
%
\isadelimproof
\isanewline
%
\endisadelimproof
\isanewline
\isacommand{lemma}\isamarkupfalse%
\ case{\isacharunderscore}{\kern0pt}bool{\isacharunderscore}{\kern0pt}true{\isacharcolon}{\kern0pt}\isanewline
\ \ {\isachardoublequoteopen}case{\isacharunderscore}{\kern0pt}bool\ {\isasymcirc}\isactrlsub c\ {\isasymt}\ {\isacharequal}{\kern0pt}\ left{\isacharunderscore}{\kern0pt}coproj\ one\ one{\isachardoublequoteclose}\isanewline
%
\isadelimproof
\ \ %
\endisadelimproof
%
\isatagproof
\isacommand{by}\isamarkupfalse%
\ {\isacharparenleft}{\kern0pt}simp\ add{\isacharcolon}{\kern0pt}\ case{\isacharunderscore}{\kern0pt}bool{\isacharunderscore}{\kern0pt}true{\isacharunderscore}{\kern0pt}and{\isacharunderscore}{\kern0pt}false{\isacharparenright}{\kern0pt}%
\endisatagproof
{\isafoldproof}%
%
\isadelimproof
\isanewline
%
\endisadelimproof
\isanewline
\isacommand{lemma}\isamarkupfalse%
\ case{\isacharunderscore}{\kern0pt}bool{\isacharunderscore}{\kern0pt}false{\isacharcolon}{\kern0pt}\isanewline
\ \ {\isachardoublequoteopen}case{\isacharunderscore}{\kern0pt}bool\ {\isasymcirc}\isactrlsub c\ {\isasymf}\ {\isacharequal}{\kern0pt}\ right{\isacharunderscore}{\kern0pt}coproj\ one\ one{\isachardoublequoteclose}\isanewline
%
\isadelimproof
\ \ %
\endisadelimproof
%
\isatagproof
\isacommand{by}\isamarkupfalse%
\ {\isacharparenleft}{\kern0pt}simp\ add{\isacharcolon}{\kern0pt}\ case{\isacharunderscore}{\kern0pt}bool{\isacharunderscore}{\kern0pt}true{\isacharunderscore}{\kern0pt}and{\isacharunderscore}{\kern0pt}false{\isacharparenright}{\kern0pt}%
\endisatagproof
{\isafoldproof}%
%
\isadelimproof
\isanewline
%
\endisadelimproof
\isanewline
\isacommand{lemma}\isamarkupfalse%
\ coprod{\isacharunderscore}{\kern0pt}case{\isacharunderscore}{\kern0pt}bool{\isacharunderscore}{\kern0pt}true{\isacharcolon}{\kern0pt}\isanewline
\ \ \isakeyword{assumes}\ {\isachardoublequoteopen}x{\isadigit{1}}\ {\isasymin}\isactrlsub c\ X{\isachardoublequoteclose}\isanewline
\ \ \isakeyword{assumes}\ {\isachardoublequoteopen}x{\isadigit{2}}\ {\isasymin}\isactrlsub c\ X{\isachardoublequoteclose}\isanewline
\ \ \isakeyword{shows}\ \ \ {\isachardoublequoteopen}{\isacharparenleft}{\kern0pt}x{\isadigit{1}}\ {\isasymamalg}\ x{\isadigit{2}}\ {\isasymcirc}\isactrlsub c\ case{\isacharunderscore}{\kern0pt}bool{\isacharparenright}{\kern0pt}\ {\isasymcirc}\isactrlsub c\ {\isasymt}\ {\isacharequal}{\kern0pt}\ x{\isadigit{1}}{\isachardoublequoteclose}\isanewline
%
\isadelimproof
%
\endisadelimproof
%
\isatagproof
\isacommand{proof}\isamarkupfalse%
\ {\isacharminus}{\kern0pt}\ \isanewline
\ \ \isacommand{have}\isamarkupfalse%
\ {\isachardoublequoteopen}{\isacharparenleft}{\kern0pt}x{\isadigit{1}}\ {\isasymamalg}\ x{\isadigit{2}}\ {\isasymcirc}\isactrlsub c\ case{\isacharunderscore}{\kern0pt}bool{\isacharparenright}{\kern0pt}\ {\isasymcirc}\isactrlsub c\ {\isasymt}\ {\isacharequal}{\kern0pt}\ {\isacharparenleft}{\kern0pt}x{\isadigit{1}}\ {\isasymamalg}\ x{\isadigit{2}}{\isacharparenright}{\kern0pt}\ {\isasymcirc}\isactrlsub c\ case{\isacharunderscore}{\kern0pt}bool\ {\isasymcirc}\isactrlsub c\ {\isasymt}{\isachardoublequoteclose}\isanewline
\ \ \ \ \isacommand{using}\isamarkupfalse%
\ assms\ \isacommand{by}\isamarkupfalse%
\ {\isacharparenleft}{\kern0pt}typecheck{\isacharunderscore}{\kern0pt}cfuncs\ {\isacharcomma}{\kern0pt}\ simp\ add{\isacharcolon}{\kern0pt}\ comp{\isacharunderscore}{\kern0pt}associative{\isadigit{2}}{\isacharparenright}{\kern0pt}\isanewline
\ \ \isacommand{also}\isamarkupfalse%
\ \isacommand{have}\isamarkupfalse%
\ {\isachardoublequoteopen}{\isachardot}{\kern0pt}{\isachardot}{\kern0pt}{\isachardot}{\kern0pt}\ {\isacharequal}{\kern0pt}\ {\isacharparenleft}{\kern0pt}x{\isadigit{1}}\ {\isasymamalg}\ x{\isadigit{2}}{\isacharparenright}{\kern0pt}\ {\isasymcirc}\isactrlsub c\ \ left{\isacharunderscore}{\kern0pt}coproj\ one\ one{\isachardoublequoteclose}\isanewline
\ \ \ \ \isacommand{using}\isamarkupfalse%
\ assms\ case{\isacharunderscore}{\kern0pt}bool{\isacharunderscore}{\kern0pt}true\ \isacommand{by}\isamarkupfalse%
\ presburger\isanewline
\ \ \isacommand{also}\isamarkupfalse%
\ \isacommand{have}\isamarkupfalse%
\ {\isachardoublequoteopen}{\isachardot}{\kern0pt}{\isachardot}{\kern0pt}{\isachardot}{\kern0pt}\ {\isacharequal}{\kern0pt}\ x{\isadigit{1}}{\isachardoublequoteclose}\isanewline
\ \ \ \ \isacommand{using}\isamarkupfalse%
\ assms\ left{\isacharunderscore}{\kern0pt}coproj{\isacharunderscore}{\kern0pt}cfunc{\isacharunderscore}{\kern0pt}coprod\ \isacommand{by}\isamarkupfalse%
\ force\isanewline
\ \ \isacommand{then}\isamarkupfalse%
\ \isacommand{show}\isamarkupfalse%
\ {\isacharquery}{\kern0pt}thesis\isanewline
\ \ \ \ \isacommand{by}\isamarkupfalse%
\ {\isacharparenleft}{\kern0pt}simp\ add{\isacharcolon}{\kern0pt}\ calculation{\isacharparenright}{\kern0pt}\isanewline
\isacommand{qed}\isamarkupfalse%
%
\endisatagproof
{\isafoldproof}%
%
\isadelimproof
\isanewline
%
\endisadelimproof
\isanewline
\isacommand{lemma}\isamarkupfalse%
\ coprod{\isacharunderscore}{\kern0pt}case{\isacharunderscore}{\kern0pt}bool{\isacharunderscore}{\kern0pt}false{\isacharcolon}{\kern0pt}\isanewline
\ \ \isakeyword{assumes}\ {\isachardoublequoteopen}x{\isadigit{1}}\ {\isasymin}\isactrlsub c\ X{\isachardoublequoteclose}\isanewline
\ \ \isakeyword{assumes}\ {\isachardoublequoteopen}x{\isadigit{2}}\ {\isasymin}\isactrlsub c\ X{\isachardoublequoteclose}\isanewline
\ \ \isakeyword{shows}\ \ \ {\isachardoublequoteopen}{\isacharparenleft}{\kern0pt}x{\isadigit{1}}\ {\isasymamalg}\ x{\isadigit{2}}\ {\isasymcirc}\isactrlsub c\ case{\isacharunderscore}{\kern0pt}bool{\isacharparenright}{\kern0pt}\ {\isasymcirc}\isactrlsub c\ {\isasymf}\ {\isacharequal}{\kern0pt}\ x{\isadigit{2}}{\isachardoublequoteclose}\isanewline
%
\isadelimproof
%
\endisadelimproof
%
\isatagproof
\isacommand{proof}\isamarkupfalse%
\ {\isacharminus}{\kern0pt}\ \isanewline
\ \ \isacommand{have}\isamarkupfalse%
\ {\isachardoublequoteopen}{\isacharparenleft}{\kern0pt}x{\isadigit{1}}\ {\isasymamalg}\ x{\isadigit{2}}\ {\isasymcirc}\isactrlsub c\ case{\isacharunderscore}{\kern0pt}bool{\isacharparenright}{\kern0pt}\ {\isasymcirc}\isactrlsub c\ {\isasymf}\ {\isacharequal}{\kern0pt}\ {\isacharparenleft}{\kern0pt}x{\isadigit{1}}\ {\isasymamalg}\ x{\isadigit{2}}{\isacharparenright}{\kern0pt}\ {\isasymcirc}\isactrlsub c\ case{\isacharunderscore}{\kern0pt}bool\ {\isasymcirc}\isactrlsub c\ {\isasymf}{\isachardoublequoteclose}\isanewline
\ \ \ \ \isacommand{using}\isamarkupfalse%
\ assms\ \isacommand{by}\isamarkupfalse%
\ {\isacharparenleft}{\kern0pt}typecheck{\isacharunderscore}{\kern0pt}cfuncs\ {\isacharcomma}{\kern0pt}\ simp\ add{\isacharcolon}{\kern0pt}\ comp{\isacharunderscore}{\kern0pt}associative{\isadigit{2}}{\isacharparenright}{\kern0pt}\isanewline
\ \ \isacommand{also}\isamarkupfalse%
\ \isacommand{have}\isamarkupfalse%
\ {\isachardoublequoteopen}{\isachardot}{\kern0pt}{\isachardot}{\kern0pt}{\isachardot}{\kern0pt}\ {\isacharequal}{\kern0pt}\ {\isacharparenleft}{\kern0pt}x{\isadigit{1}}\ {\isasymamalg}\ x{\isadigit{2}}{\isacharparenright}{\kern0pt}\ {\isasymcirc}\isactrlsub c\ \ right{\isacharunderscore}{\kern0pt}coproj\ one\ one{\isachardoublequoteclose}\isanewline
\ \ \ \ \isacommand{using}\isamarkupfalse%
\ assms\ case{\isacharunderscore}{\kern0pt}bool{\isacharunderscore}{\kern0pt}false\ \isacommand{by}\isamarkupfalse%
\ presburger\isanewline
\ \ \isacommand{also}\isamarkupfalse%
\ \isacommand{have}\isamarkupfalse%
\ {\isachardoublequoteopen}{\isachardot}{\kern0pt}{\isachardot}{\kern0pt}{\isachardot}{\kern0pt}\ {\isacharequal}{\kern0pt}\ x{\isadigit{2}}{\isachardoublequoteclose}\isanewline
\ \ \ \ \isacommand{using}\isamarkupfalse%
\ assms\ right{\isacharunderscore}{\kern0pt}coproj{\isacharunderscore}{\kern0pt}cfunc{\isacharunderscore}{\kern0pt}coprod\ \isacommand{by}\isamarkupfalse%
\ force\isanewline
\ \ \isacommand{then}\isamarkupfalse%
\ \isacommand{show}\isamarkupfalse%
\ {\isacharquery}{\kern0pt}thesis\isanewline
\ \ \ \ \isacommand{by}\isamarkupfalse%
\ {\isacharparenleft}{\kern0pt}simp\ add{\isacharcolon}{\kern0pt}\ calculation{\isacharparenright}{\kern0pt}\isanewline
\isacommand{qed}\isamarkupfalse%
%
\endisatagproof
{\isafoldproof}%
%
\isadelimproof
%
\endisadelimproof
%
\isadelimdocument
%
\endisadelimdocument
%
\isatagdocument
%
\isamarkupsubsection{Distribution of Products over Coproducts%
}
\isamarkuptrue%
%
\isamarkupsubsubsection{Distribute Product Over Coproduct Auxillary Mapping%
}
\isamarkuptrue%
%
\endisatagdocument
{\isafolddocument}%
%
\isadelimdocument
%
\endisadelimdocument
\isacommand{definition}\isamarkupfalse%
\ dist{\isacharunderscore}{\kern0pt}prod{\isacharunderscore}{\kern0pt}coprod\ {\isacharcolon}{\kern0pt}{\isacharcolon}{\kern0pt}\ {\isachardoublequoteopen}cset\ {\isasymRightarrow}\ cset\ {\isasymRightarrow}\ cset\ {\isasymRightarrow}\ cfunc{\isachardoublequoteclose}\ \isakeyword{where}\isanewline
\ \ {\isachardoublequoteopen}dist{\isacharunderscore}{\kern0pt}prod{\isacharunderscore}{\kern0pt}coprod\ A\ B\ C\ {\isacharequal}{\kern0pt}\ {\isacharparenleft}{\kern0pt}id\ A\ {\isasymtimes}\isactrlsub f\ left{\isacharunderscore}{\kern0pt}coproj\ B\ C{\isacharparenright}{\kern0pt}\ {\isasymamalg}\ {\isacharparenleft}{\kern0pt}id\ A\ {\isasymtimes}\isactrlsub f\ right{\isacharunderscore}{\kern0pt}coproj\ B\ C{\isacharparenright}{\kern0pt}{\isachardoublequoteclose}\isanewline
\isanewline
\isacommand{lemma}\isamarkupfalse%
\ dist{\isacharunderscore}{\kern0pt}prod{\isacharunderscore}{\kern0pt}coprod{\isacharunderscore}{\kern0pt}type{\isacharbrackleft}{\kern0pt}type{\isacharunderscore}{\kern0pt}rule{\isacharbrackright}{\kern0pt}{\isacharcolon}{\kern0pt}\isanewline
\ \ {\isachardoublequoteopen}dist{\isacharunderscore}{\kern0pt}prod{\isacharunderscore}{\kern0pt}coprod\ A\ B\ C\ {\isacharcolon}{\kern0pt}\ {\isacharparenleft}{\kern0pt}A\ {\isasymtimes}\isactrlsub c\ B{\isacharparenright}{\kern0pt}\ {\isasymCoprod}\ {\isacharparenleft}{\kern0pt}A\ {\isasymtimes}\isactrlsub c\ C{\isacharparenright}{\kern0pt}\ {\isasymrightarrow}\ A\ {\isasymtimes}\isactrlsub c\ {\isacharparenleft}{\kern0pt}B\ {\isasymCoprod}\ C{\isacharparenright}{\kern0pt}{\isachardoublequoteclose}\isanewline
%
\isadelimproof
\ \ %
\endisadelimproof
%
\isatagproof
\isacommand{unfolding}\isamarkupfalse%
\ dist{\isacharunderscore}{\kern0pt}prod{\isacharunderscore}{\kern0pt}coprod{\isacharunderscore}{\kern0pt}def\ \isacommand{by}\isamarkupfalse%
\ typecheck{\isacharunderscore}{\kern0pt}cfuncs%
\endisatagproof
{\isafoldproof}%
%
\isadelimproof
\isanewline
%
\endisadelimproof
\isanewline
\isacommand{lemma}\isamarkupfalse%
\ dist{\isacharunderscore}{\kern0pt}prod{\isacharunderscore}{\kern0pt}coprod{\isacharunderscore}{\kern0pt}left{\isacharunderscore}{\kern0pt}ap{\isacharcolon}{\kern0pt}\isanewline
\ \ \isakeyword{assumes}\ {\isachardoublequoteopen}a\ {\isasymin}\isactrlsub c\ A{\isachardoublequoteclose}\ {\isachardoublequoteopen}b\ {\isasymin}\isactrlsub c\ B{\isachardoublequoteclose}\isanewline
\ \ \isakeyword{shows}\ {\isachardoublequoteopen}dist{\isacharunderscore}{\kern0pt}prod{\isacharunderscore}{\kern0pt}coprod\ A\ B\ C\ {\isasymcirc}\isactrlsub c\ left{\isacharunderscore}{\kern0pt}coproj\ {\isacharparenleft}{\kern0pt}A\ {\isasymtimes}\isactrlsub c\ B{\isacharparenright}{\kern0pt}\ {\isacharparenleft}{\kern0pt}A\ {\isasymtimes}\isactrlsub c\ C{\isacharparenright}{\kern0pt}\ {\isasymcirc}\isactrlsub c\ {\isasymlangle}a{\isacharcomma}{\kern0pt}\ b{\isasymrangle}\ {\isacharequal}{\kern0pt}\ {\isasymlangle}a{\isacharcomma}{\kern0pt}\ left{\isacharunderscore}{\kern0pt}coproj\ B\ C\ {\isasymcirc}\isactrlsub c\ b{\isasymrangle}{\isachardoublequoteclose}\isanewline
%
\isadelimproof
\ \ %
\endisadelimproof
%
\isatagproof
\isacommand{unfolding}\isamarkupfalse%
\ dist{\isacharunderscore}{\kern0pt}prod{\isacharunderscore}{\kern0pt}coprod{\isacharunderscore}{\kern0pt}def\ \isacommand{using}\isamarkupfalse%
\ assms\ \isanewline
\ \ \isacommand{by}\isamarkupfalse%
\ {\isacharparenleft}{\kern0pt}typecheck{\isacharunderscore}{\kern0pt}cfuncs{\isacharcomma}{\kern0pt}\ simp\ add{\isacharcolon}{\kern0pt}\ cfunc{\isacharunderscore}{\kern0pt}cross{\isacharunderscore}{\kern0pt}prod{\isacharunderscore}{\kern0pt}comp{\isacharunderscore}{\kern0pt}cfunc{\isacharunderscore}{\kern0pt}prod\ comp{\isacharunderscore}{\kern0pt}associative{\isadigit{2}}\ id{\isacharunderscore}{\kern0pt}left{\isacharunderscore}{\kern0pt}unit{\isadigit{2}}\ left{\isacharunderscore}{\kern0pt}coproj{\isacharunderscore}{\kern0pt}cfunc{\isacharunderscore}{\kern0pt}coprod{\isacharparenright}{\kern0pt}%
\endisatagproof
{\isafoldproof}%
%
\isadelimproof
\isanewline
%
\endisadelimproof
\isanewline
\isacommand{lemma}\isamarkupfalse%
\ dist{\isacharunderscore}{\kern0pt}prod{\isacharunderscore}{\kern0pt}coprod{\isacharunderscore}{\kern0pt}right{\isacharunderscore}{\kern0pt}ap{\isacharcolon}{\kern0pt}\isanewline
\ \ \isakeyword{assumes}\ {\isachardoublequoteopen}a\ {\isasymin}\isactrlsub c\ A{\isachardoublequoteclose}\ {\isachardoublequoteopen}c\ {\isasymin}\isactrlsub c\ C{\isachardoublequoteclose}\isanewline
\ \ \isakeyword{shows}\ {\isachardoublequoteopen}dist{\isacharunderscore}{\kern0pt}prod{\isacharunderscore}{\kern0pt}coprod\ A\ B\ C\ {\isasymcirc}\isactrlsub c\ right{\isacharunderscore}{\kern0pt}coproj\ {\isacharparenleft}{\kern0pt}A\ {\isasymtimes}\isactrlsub c\ B{\isacharparenright}{\kern0pt}\ {\isacharparenleft}{\kern0pt}A\ {\isasymtimes}\isactrlsub c\ C{\isacharparenright}{\kern0pt}\ {\isasymcirc}\isactrlsub c\ {\isasymlangle}a{\isacharcomma}{\kern0pt}\ c{\isasymrangle}\ {\isacharequal}{\kern0pt}\ {\isasymlangle}a{\isacharcomma}{\kern0pt}\ right{\isacharunderscore}{\kern0pt}coproj\ B\ C\ {\isasymcirc}\isactrlsub c\ c{\isasymrangle}{\isachardoublequoteclose}\isanewline
%
\isadelimproof
\ \ %
\endisadelimproof
%
\isatagproof
\isacommand{unfolding}\isamarkupfalse%
\ dist{\isacharunderscore}{\kern0pt}prod{\isacharunderscore}{\kern0pt}coprod{\isacharunderscore}{\kern0pt}def\ \isacommand{using}\isamarkupfalse%
\ assms\ \isanewline
\ \ \isacommand{by}\isamarkupfalse%
\ {\isacharparenleft}{\kern0pt}typecheck{\isacharunderscore}{\kern0pt}cfuncs{\isacharcomma}{\kern0pt}\ simp\ add{\isacharcolon}{\kern0pt}\ cfunc{\isacharunderscore}{\kern0pt}cross{\isacharunderscore}{\kern0pt}prod{\isacharunderscore}{\kern0pt}comp{\isacharunderscore}{\kern0pt}cfunc{\isacharunderscore}{\kern0pt}prod\ comp{\isacharunderscore}{\kern0pt}associative{\isadigit{2}}\ id{\isacharunderscore}{\kern0pt}left{\isacharunderscore}{\kern0pt}unit{\isadigit{2}}\ right{\isacharunderscore}{\kern0pt}coproj{\isacharunderscore}{\kern0pt}cfunc{\isacharunderscore}{\kern0pt}coprod{\isacharparenright}{\kern0pt}%
\endisatagproof
{\isafoldproof}%
%
\isadelimproof
\isanewline
%
\endisadelimproof
\isanewline
\isacommand{lemma}\isamarkupfalse%
\ dist{\isacharunderscore}{\kern0pt}prod{\isacharunderscore}{\kern0pt}coprod{\isacharunderscore}{\kern0pt}mono{\isacharcolon}{\kern0pt}\isanewline
\ \ {\isachardoublequoteopen}monomorphism\ {\isacharparenleft}{\kern0pt}dist{\isacharunderscore}{\kern0pt}prod{\isacharunderscore}{\kern0pt}coprod\ A\ B\ C{\isacharparenright}{\kern0pt}{\isachardoublequoteclose}\isanewline
%
\isadelimproof
%
\endisadelimproof
%
\isatagproof
\isacommand{proof}\isamarkupfalse%
\ {\isacharminus}{\kern0pt}\isanewline
\ \ \isacommand{obtain}\isamarkupfalse%
\ {\isasymphi}\ \isakeyword{where}\ {\isasymphi}{\isacharunderscore}{\kern0pt}def{\isacharcolon}{\kern0pt}\ {\isachardoublequoteopen}{\isasymphi}\ {\isacharequal}{\kern0pt}\ {\isacharparenleft}{\kern0pt}id\ A\ \ {\isasymtimes}\isactrlsub f\ left{\isacharunderscore}{\kern0pt}coproj\ B\ C{\isacharparenright}{\kern0pt}\ {\isasymamalg}\ {\isacharparenleft}{\kern0pt}id\ A\ {\isasymtimes}\isactrlsub f\ right{\isacharunderscore}{\kern0pt}coproj\ B\ C{\isacharparenright}{\kern0pt}{\isachardoublequoteclose}\ \isakeyword{and}\isanewline
\ \ \ \ \ \ \ \ \ \ \ \ \ \ \ \ \ {\isasymphi}{\isacharunderscore}{\kern0pt}type{\isacharbrackleft}{\kern0pt}type{\isacharunderscore}{\kern0pt}rule{\isacharbrackright}{\kern0pt}{\isacharcolon}{\kern0pt}\ {\isachardoublequoteopen}{\isasymphi}\ {\isacharcolon}{\kern0pt}\ {\isacharparenleft}{\kern0pt}A\ {\isasymtimes}\isactrlsub c\ B{\isacharparenright}{\kern0pt}\ {\isasymCoprod}\ {\isacharparenleft}{\kern0pt}A\ {\isasymtimes}\isactrlsub c\ C{\isacharparenright}{\kern0pt}\ {\isasymrightarrow}\ A\ {\isasymtimes}\isactrlsub c\ {\isacharparenleft}{\kern0pt}B\ {\isasymCoprod}\ C{\isacharparenright}{\kern0pt}{\isachardoublequoteclose}\isanewline
\ \ \ \ \isacommand{by}\isamarkupfalse%
\ typecheck{\isacharunderscore}{\kern0pt}cfuncs\isanewline
\isanewline
\ \ \isacommand{have}\isamarkupfalse%
\ injective{\isacharcolon}{\kern0pt}\ {\isachardoublequoteopen}injective{\isacharparenleft}{\kern0pt}{\isasymphi}{\isacharparenright}{\kern0pt}{\isachardoublequoteclose}\isanewline
\ \ \ \ \isacommand{unfolding}\isamarkupfalse%
\ injective{\isacharunderscore}{\kern0pt}def\isanewline
\ \ \isacommand{proof}\isamarkupfalse%
{\isacharparenleft}{\kern0pt}auto{\isacharparenright}{\kern0pt}\ \isanewline
\ \ \ \ \isacommand{fix}\isamarkupfalse%
\ x\ y\isanewline
\ \ \ \ \isacommand{assume}\isamarkupfalse%
\ x{\isacharunderscore}{\kern0pt}type{\isacharcolon}{\kern0pt}\ {\isachardoublequoteopen}x\ {\isasymin}\isactrlsub c\ domain\ {\isasymphi}{\isachardoublequoteclose}\isanewline
\ \ \ \ \isacommand{assume}\isamarkupfalse%
\ y{\isacharunderscore}{\kern0pt}type{\isacharcolon}{\kern0pt}\ {\isachardoublequoteopen}y\ {\isasymin}\isactrlsub c\ domain\ {\isasymphi}{\isachardoublequoteclose}\isanewline
\ \ \ \ \isacommand{assume}\isamarkupfalse%
\ equal{\isacharcolon}{\kern0pt}\ {\isachardoublequoteopen}{\isasymphi}\ {\isasymcirc}\isactrlsub c\ x\ {\isacharequal}{\kern0pt}\ {\isasymphi}\ {\isasymcirc}\isactrlsub c\ y{\isachardoublequoteclose}\isanewline
\isanewline
\ \ \ \ \isacommand{have}\isamarkupfalse%
\ x{\isacharunderscore}{\kern0pt}type{\isacharbrackleft}{\kern0pt}type{\isacharunderscore}{\kern0pt}rule{\isacharbrackright}{\kern0pt}{\isacharcolon}{\kern0pt}\ {\isachardoublequoteopen}x\ {\isasymin}\isactrlsub c\ {\isacharparenleft}{\kern0pt}A\ {\isasymtimes}\isactrlsub c\ B{\isacharparenright}{\kern0pt}\ {\isasymCoprod}\ {\isacharparenleft}{\kern0pt}A\ {\isasymtimes}\isactrlsub c\ C{\isacharparenright}{\kern0pt}{\isachardoublequoteclose}\isanewline
\ \ \ \ \ \ \isacommand{using}\isamarkupfalse%
\ cfunc{\isacharunderscore}{\kern0pt}type{\isacharunderscore}{\kern0pt}def\ {\isasymphi}{\isacharunderscore}{\kern0pt}type\ x{\isacharunderscore}{\kern0pt}type\ \isacommand{by}\isamarkupfalse%
\ auto\isanewline
\ \ \ \ \isacommand{then}\isamarkupfalse%
\ \isacommand{have}\isamarkupfalse%
\ x{\isacharunderscore}{\kern0pt}form{\isacharcolon}{\kern0pt}\ {\isachardoublequoteopen}{\isacharparenleft}{\kern0pt}{\isasymexists}\ x{\isacharprime}{\kern0pt}{\isachardot}{\kern0pt}\ x{\isacharprime}{\kern0pt}\ {\isasymin}\isactrlsub c\ A\ {\isasymtimes}\isactrlsub c\ B\ {\isasymand}\ x\ {\isacharequal}{\kern0pt}\ {\isacharparenleft}{\kern0pt}left{\isacharunderscore}{\kern0pt}coproj\ {\isacharparenleft}{\kern0pt}A\ {\isasymtimes}\isactrlsub c\ B{\isacharparenright}{\kern0pt}\ {\isacharparenleft}{\kern0pt}A\ {\isasymtimes}\isactrlsub c\ C{\isacharparenright}{\kern0pt}{\isacharparenright}{\kern0pt}\ {\isasymcirc}\isactrlsub c\ x{\isacharprime}{\kern0pt}{\isacharparenright}{\kern0pt}\isanewline
\ \ \ \ \ \ {\isasymor}\ \ {\isacharparenleft}{\kern0pt}{\isasymexists}\ x{\isacharprime}{\kern0pt}{\isachardot}{\kern0pt}\ x{\isacharprime}{\kern0pt}\ {\isasymin}\isactrlsub c\ A\ {\isasymtimes}\isactrlsub c\ C\ {\isasymand}\ x\ {\isacharequal}{\kern0pt}\ {\isacharparenleft}{\kern0pt}right{\isacharunderscore}{\kern0pt}coproj\ {\isacharparenleft}{\kern0pt}A\ {\isasymtimes}\isactrlsub c\ B{\isacharparenright}{\kern0pt}\ {\isacharparenleft}{\kern0pt}A\ {\isasymtimes}\isactrlsub c\ C{\isacharparenright}{\kern0pt}{\isacharparenright}{\kern0pt}\ {\isasymcirc}\isactrlsub c\ x{\isacharprime}{\kern0pt}{\isacharparenright}{\kern0pt}{\isachardoublequoteclose}\isanewline
\ \ \ \ \ \ \isacommand{by}\isamarkupfalse%
\ {\isacharparenleft}{\kern0pt}simp\ add{\isacharcolon}{\kern0pt}\ coprojs{\isacharunderscore}{\kern0pt}jointly{\isacharunderscore}{\kern0pt}surj{\isacharparenright}{\kern0pt}\isanewline
\ \ \ \ \isacommand{have}\isamarkupfalse%
\ y{\isacharunderscore}{\kern0pt}type{\isacharbrackleft}{\kern0pt}type{\isacharunderscore}{\kern0pt}rule{\isacharbrackright}{\kern0pt}{\isacharcolon}{\kern0pt}\ {\isachardoublequoteopen}y\ {\isasymin}\isactrlsub c\ {\isacharparenleft}{\kern0pt}A\ {\isasymtimes}\isactrlsub c\ B{\isacharparenright}{\kern0pt}\ {\isasymCoprod}\ {\isacharparenleft}{\kern0pt}A\ {\isasymtimes}\isactrlsub c\ C{\isacharparenright}{\kern0pt}{\isachardoublequoteclose}\isanewline
\ \ \ \ \ \ \isacommand{using}\isamarkupfalse%
\ cfunc{\isacharunderscore}{\kern0pt}type{\isacharunderscore}{\kern0pt}def\ {\isasymphi}{\isacharunderscore}{\kern0pt}type\ y{\isacharunderscore}{\kern0pt}type\ \isacommand{by}\isamarkupfalse%
\ auto\isanewline
\ \ \ \ \isacommand{then}\isamarkupfalse%
\ \isacommand{have}\isamarkupfalse%
\ y{\isacharunderscore}{\kern0pt}form{\isacharcolon}{\kern0pt}\ {\isachardoublequoteopen}{\isacharparenleft}{\kern0pt}{\isasymexists}\ y{\isacharprime}{\kern0pt}{\isachardot}{\kern0pt}\ y{\isacharprime}{\kern0pt}\ {\isasymin}\isactrlsub c\ A\ {\isasymtimes}\isactrlsub c\ B\ {\isasymand}\ y\ {\isacharequal}{\kern0pt}\ {\isacharparenleft}{\kern0pt}left{\isacharunderscore}{\kern0pt}coproj\ {\isacharparenleft}{\kern0pt}A\ {\isasymtimes}\isactrlsub c\ B{\isacharparenright}{\kern0pt}\ {\isacharparenleft}{\kern0pt}A\ {\isasymtimes}\isactrlsub c\ C{\isacharparenright}{\kern0pt}{\isacharparenright}{\kern0pt}\ {\isasymcirc}\isactrlsub c\ y{\isacharprime}{\kern0pt}{\isacharparenright}{\kern0pt}\isanewline
\ \ \ \ \ \ {\isasymor}\ \ {\isacharparenleft}{\kern0pt}{\isasymexists}\ y{\isacharprime}{\kern0pt}{\isachardot}{\kern0pt}\ y{\isacharprime}{\kern0pt}\ {\isasymin}\isactrlsub c\ A\ {\isasymtimes}\isactrlsub c\ C\ {\isasymand}\ y\ {\isacharequal}{\kern0pt}\ {\isacharparenleft}{\kern0pt}right{\isacharunderscore}{\kern0pt}coproj\ {\isacharparenleft}{\kern0pt}A\ {\isasymtimes}\isactrlsub c\ B{\isacharparenright}{\kern0pt}\ {\isacharparenleft}{\kern0pt}A\ {\isasymtimes}\isactrlsub c\ C{\isacharparenright}{\kern0pt}{\isacharparenright}{\kern0pt}\ {\isasymcirc}\isactrlsub c\ y{\isacharprime}{\kern0pt}{\isacharparenright}{\kern0pt}{\isachardoublequoteclose}\isanewline
\ \ \ \ \ \ \isacommand{by}\isamarkupfalse%
\ {\isacharparenleft}{\kern0pt}simp\ add{\isacharcolon}{\kern0pt}\ coprojs{\isacharunderscore}{\kern0pt}jointly{\isacharunderscore}{\kern0pt}surj{\isacharparenright}{\kern0pt}\isanewline
\ \ \ \ \isanewline
\ \ \ \ \isacommand{show}\isamarkupfalse%
\ {\isachardoublequoteopen}x\ {\isacharequal}{\kern0pt}\ y{\isachardoublequoteclose}\ \isanewline
\ \ \ \ \isacommand{proof}\isamarkupfalse%
{\isacharparenleft}{\kern0pt}cases\ {\isachardoublequoteopen}{\isacharparenleft}{\kern0pt}{\isasymexists}\ x{\isacharprime}{\kern0pt}{\isachardot}{\kern0pt}\ x{\isacharprime}{\kern0pt}\ {\isasymin}\isactrlsub c\ A\ {\isasymtimes}\isactrlsub c\ B\ {\isasymand}\ x\ {\isacharequal}{\kern0pt}\ {\isacharparenleft}{\kern0pt}left{\isacharunderscore}{\kern0pt}coproj\ {\isacharparenleft}{\kern0pt}A\ {\isasymtimes}\isactrlsub c\ B{\isacharparenright}{\kern0pt}\ {\isacharparenleft}{\kern0pt}A\ {\isasymtimes}\isactrlsub c\ C{\isacharparenright}{\kern0pt}{\isacharparenright}{\kern0pt}\ {\isasymcirc}\isactrlsub c\ x{\isacharprime}{\kern0pt}{\isacharparenright}{\kern0pt}{\isachardoublequoteclose}{\isacharparenright}{\kern0pt}\isanewline
\ \ \ \ \ \ \isacommand{assume}\isamarkupfalse%
\ {\isachardoublequoteopen}{\isasymexists}\ x{\isacharprime}{\kern0pt}{\isachardot}{\kern0pt}\ x{\isacharprime}{\kern0pt}\ {\isasymin}\isactrlsub c\ A\ {\isasymtimes}\isactrlsub c\ B\ {\isasymand}\ x\ {\isacharequal}{\kern0pt}\ {\isacharparenleft}{\kern0pt}left{\isacharunderscore}{\kern0pt}coproj\ {\isacharparenleft}{\kern0pt}A\ {\isasymtimes}\isactrlsub c\ B{\isacharparenright}{\kern0pt}\ {\isacharparenleft}{\kern0pt}A\ {\isasymtimes}\isactrlsub c\ C{\isacharparenright}{\kern0pt}{\isacharparenright}{\kern0pt}\ {\isasymcirc}\isactrlsub c\ x{\isacharprime}{\kern0pt}{\isachardoublequoteclose}\isanewline
\ \ \ \ \ \ \isacommand{then}\isamarkupfalse%
\ \isacommand{obtain}\isamarkupfalse%
\ x{\isacharprime}{\kern0pt}\ \isakeyword{where}\ x{\isacharprime}{\kern0pt}{\isacharunderscore}{\kern0pt}def{\isacharbrackleft}{\kern0pt}type{\isacharunderscore}{\kern0pt}rule{\isacharbrackright}{\kern0pt}{\isacharcolon}{\kern0pt}\ {\isachardoublequoteopen}x{\isacharprime}{\kern0pt}\ {\isasymin}\isactrlsub c\ A\ {\isasymtimes}\isactrlsub c\ B{\isachardoublequoteclose}\ {\isachardoublequoteopen}x\ {\isacharequal}{\kern0pt}\ left{\isacharunderscore}{\kern0pt}coproj\ {\isacharparenleft}{\kern0pt}A\ {\isasymtimes}\isactrlsub c\ B{\isacharparenright}{\kern0pt}\ {\isacharparenleft}{\kern0pt}A\ {\isasymtimes}\isactrlsub c\ C{\isacharparenright}{\kern0pt}\ {\isasymcirc}\isactrlsub c\ x{\isacharprime}{\kern0pt}{\isachardoublequoteclose}\isanewline
\ \ \ \ \ \ \ \ \isacommand{by}\isamarkupfalse%
\ blast\isanewline
\ \ \ \ \ \ \isacommand{then}\isamarkupfalse%
\ \isacommand{have}\isamarkupfalse%
\ ab{\isacharunderscore}{\kern0pt}exists{\isacharcolon}{\kern0pt}\ {\isachardoublequoteopen}{\isasymexists}\ a\ b{\isachardot}{\kern0pt}\ a\ {\isasymin}\isactrlsub c\ A\ {\isasymand}\ b\ {\isasymin}\isactrlsub c\ B\ {\isasymand}\ x{\isacharprime}{\kern0pt}\ {\isacharequal}{\kern0pt}{\isasymlangle}a{\isacharcomma}{\kern0pt}b{\isasymrangle}{\isachardoublequoteclose}\isanewline
\ \ \ \ \ \ \ \ \isacommand{using}\isamarkupfalse%
\ cart{\isacharunderscore}{\kern0pt}prod{\isacharunderscore}{\kern0pt}decomp\ \isacommand{by}\isamarkupfalse%
\ blast\isanewline
\ \ \ \ \ \ \isacommand{then}\isamarkupfalse%
\ \isacommand{obtain}\isamarkupfalse%
\ a\ b\ \isakeyword{where}\ ab{\isacharunderscore}{\kern0pt}def{\isacharbrackleft}{\kern0pt}type{\isacharunderscore}{\kern0pt}rule{\isacharbrackright}{\kern0pt}{\isacharcolon}{\kern0pt}\ {\isachardoublequoteopen}a\ {\isasymin}\isactrlsub c\ A{\isachardoublequoteclose}\ {\isachardoublequoteopen}b\ {\isasymin}\isactrlsub c\ B{\isachardoublequoteclose}\ \ {\isachardoublequoteopen}x{\isacharprime}{\kern0pt}\ {\isacharequal}{\kern0pt}{\isasymlangle}a{\isacharcomma}{\kern0pt}b{\isasymrangle}{\isachardoublequoteclose}\isanewline
\ \ \ \ \ \ \ \ \isacommand{by}\isamarkupfalse%
\ blast\isanewline
\ \ \ \ \ \ \isacommand{show}\isamarkupfalse%
\ {\isachardoublequoteopen}x\ {\isacharequal}{\kern0pt}\ y{\isachardoublequoteclose}\ \ \isanewline
\ \ \ \ \ \ \isacommand{proof}\isamarkupfalse%
{\isacharparenleft}{\kern0pt}cases\ {\isachardoublequoteopen}{\isasymexists}\ y{\isacharprime}{\kern0pt}{\isachardot}{\kern0pt}\ y{\isacharprime}{\kern0pt}\ {\isasymin}\isactrlsub c\ A\ {\isasymtimes}\isactrlsub c\ B\ {\isasymand}\ y\ {\isacharequal}{\kern0pt}\ {\isacharparenleft}{\kern0pt}left{\isacharunderscore}{\kern0pt}coproj\ {\isacharparenleft}{\kern0pt}A\ {\isasymtimes}\isactrlsub c\ B{\isacharparenright}{\kern0pt}\ {\isacharparenleft}{\kern0pt}A\ {\isasymtimes}\isactrlsub c\ C{\isacharparenright}{\kern0pt}{\isacharparenright}{\kern0pt}\ {\isasymcirc}\isactrlsub c\ y{\isacharprime}{\kern0pt}{\isachardoublequoteclose}{\isacharparenright}{\kern0pt}\isanewline
\ \ \ \ \ \ \ \ \isacommand{assume}\isamarkupfalse%
\ {\isachardoublequoteopen}{\isasymexists}\ y{\isacharprime}{\kern0pt}{\isachardot}{\kern0pt}\ y{\isacharprime}{\kern0pt}\ {\isasymin}\isactrlsub c\ A\ {\isasymtimes}\isactrlsub c\ B\ {\isasymand}\ y\ {\isacharequal}{\kern0pt}\ {\isacharparenleft}{\kern0pt}left{\isacharunderscore}{\kern0pt}coproj\ {\isacharparenleft}{\kern0pt}A\ {\isasymtimes}\isactrlsub c\ B{\isacharparenright}{\kern0pt}\ {\isacharparenleft}{\kern0pt}A\ {\isasymtimes}\isactrlsub c\ C{\isacharparenright}{\kern0pt}{\isacharparenright}{\kern0pt}\ {\isasymcirc}\isactrlsub c\ y{\isacharprime}{\kern0pt}{\isachardoublequoteclose}\isanewline
\ \ \ \ \ \ \ \ \isacommand{then}\isamarkupfalse%
\ \isacommand{obtain}\isamarkupfalse%
\ y{\isacharprime}{\kern0pt}\ \isakeyword{where}\ y{\isacharprime}{\kern0pt}{\isacharunderscore}{\kern0pt}def{\isacharcolon}{\kern0pt}\ {\isachardoublequoteopen}y{\isacharprime}{\kern0pt}\ {\isasymin}\isactrlsub c\ A\ {\isasymtimes}\isactrlsub c\ B{\isachardoublequoteclose}\ {\isachardoublequoteopen}y\ {\isacharequal}{\kern0pt}\ left{\isacharunderscore}{\kern0pt}coproj\ {\isacharparenleft}{\kern0pt}A\ {\isasymtimes}\isactrlsub c\ B{\isacharparenright}{\kern0pt}\ {\isacharparenleft}{\kern0pt}A\ {\isasymtimes}\isactrlsub c\ C{\isacharparenright}{\kern0pt}\ {\isasymcirc}\isactrlsub c\ y{\isacharprime}{\kern0pt}{\isachardoublequoteclose}\isanewline
\ \ \ \ \ \ \ \ \ \ \isacommand{by}\isamarkupfalse%
\ blast\isanewline
\ \ \ \ \ \ \ \ \isacommand{then}\isamarkupfalse%
\ \isacommand{have}\isamarkupfalse%
\ ab{\isacharunderscore}{\kern0pt}exists{\isacharcolon}{\kern0pt}\ {\isachardoublequoteopen}{\isasymexists}\ a{\isacharprime}{\kern0pt}\ b{\isacharprime}{\kern0pt}{\isachardot}{\kern0pt}\ a{\isacharprime}{\kern0pt}\ {\isasymin}\isactrlsub c\ A\ {\isasymand}\ b{\isacharprime}{\kern0pt}\ {\isasymin}\isactrlsub c\ B\ {\isasymand}\ y{\isacharprime}{\kern0pt}\ {\isacharequal}{\kern0pt}{\isasymlangle}a{\isacharprime}{\kern0pt}{\isacharcomma}{\kern0pt}b{\isacharprime}{\kern0pt}{\isasymrangle}{\isachardoublequoteclose}\isanewline
\ \ \ \ \ \ \ \ \ \ \isacommand{using}\isamarkupfalse%
\ cart{\isacharunderscore}{\kern0pt}prod{\isacharunderscore}{\kern0pt}decomp\ \isacommand{by}\isamarkupfalse%
\ blast\isanewline
\ \ \ \ \ \ \ \ \isacommand{then}\isamarkupfalse%
\ \isacommand{obtain}\isamarkupfalse%
\ a{\isacharprime}{\kern0pt}\ b{\isacharprime}{\kern0pt}\ \isakeyword{where}\ a{\isacharprime}{\kern0pt}b{\isacharprime}{\kern0pt}{\isacharunderscore}{\kern0pt}def{\isacharbrackleft}{\kern0pt}type{\isacharunderscore}{\kern0pt}rule{\isacharbrackright}{\kern0pt}{\isacharcolon}{\kern0pt}\ {\isachardoublequoteopen}a{\isacharprime}{\kern0pt}\ {\isasymin}\isactrlsub c\ A{\isachardoublequoteclose}\ {\isachardoublequoteopen}b{\isacharprime}{\kern0pt}\ {\isasymin}\isactrlsub c\ B{\isachardoublequoteclose}\ {\isachardoublequoteopen}y{\isacharprime}{\kern0pt}\ {\isacharequal}{\kern0pt}{\isasymlangle}a{\isacharprime}{\kern0pt}{\isacharcomma}{\kern0pt}b{\isacharprime}{\kern0pt}{\isasymrangle}{\isachardoublequoteclose}\isanewline
\ \ \ \ \ \ \ \ \ \ \isacommand{by}\isamarkupfalse%
\ blast\isanewline
\ \ \ \ \ \ \ \ \isacommand{have}\isamarkupfalse%
\ equal{\isacharunderscore}{\kern0pt}pair{\isacharcolon}{\kern0pt}\ {\isachardoublequoteopen}{\isasymlangle}a{\isacharcomma}{\kern0pt}\ left{\isacharunderscore}{\kern0pt}coproj\ B\ C\ {\isasymcirc}\isactrlsub c\ b{\isasymrangle}\ {\isacharequal}{\kern0pt}\ {\isasymlangle}a{\isacharprime}{\kern0pt}{\isacharcomma}{\kern0pt}\ left{\isacharunderscore}{\kern0pt}coproj\ B\ C\ {\isasymcirc}\isactrlsub c\ b{\isacharprime}{\kern0pt}{\isasymrangle}{\isachardoublequoteclose}\isanewline
\ \ \ \ \ \ \ \ \isacommand{proof}\isamarkupfalse%
\ {\isacharminus}{\kern0pt}\ \isanewline
\ \ \ \ \ \ \ \ \ \ \isacommand{have}\isamarkupfalse%
\ {\isachardoublequoteopen}{\isasymlangle}a{\isacharcomma}{\kern0pt}\ left{\isacharunderscore}{\kern0pt}coproj\ B\ C\ {\isasymcirc}\isactrlsub c\ b{\isasymrangle}\ {\isacharequal}{\kern0pt}\ {\isasymlangle}id\ A\ {\isasymcirc}\isactrlsub c\ a{\isacharcomma}{\kern0pt}\ left{\isacharunderscore}{\kern0pt}coproj\ B\ C\ {\isasymcirc}\isactrlsub c\ b{\isasymrangle}{\isachardoublequoteclose}\isanewline
\ \ \ \ \ \ \ \ \ \ \ \ \isacommand{using}\isamarkupfalse%
\ ab{\isacharunderscore}{\kern0pt}def\ id{\isacharunderscore}{\kern0pt}left{\isacharunderscore}{\kern0pt}unit{\isadigit{2}}\ \isacommand{by}\isamarkupfalse%
\ force\isanewline
\ \ \ \ \ \ \ \ \ \ \isacommand{also}\isamarkupfalse%
\ \isacommand{have}\isamarkupfalse%
\ {\isachardoublequoteopen}{\isachardot}{\kern0pt}{\isachardot}{\kern0pt}{\isachardot}{\kern0pt}\ {\isacharequal}{\kern0pt}\ {\isacharparenleft}{\kern0pt}id\ A\ {\isasymtimes}\isactrlsub f\ left{\isacharunderscore}{\kern0pt}coproj\ B\ C{\isacharparenright}{\kern0pt}\ \ {\isasymcirc}\isactrlsub c\ {\isasymlangle}a{\isacharcomma}{\kern0pt}\ b{\isasymrangle}{\isachardoublequoteclose}\isanewline
\ \ \ \ \ \ \ \ \ \ \ \ \isacommand{by}\isamarkupfalse%
\ {\isacharparenleft}{\kern0pt}smt\ ab{\isacharunderscore}{\kern0pt}def\ cfunc{\isacharunderscore}{\kern0pt}cross{\isacharunderscore}{\kern0pt}prod{\isacharunderscore}{\kern0pt}comp{\isacharunderscore}{\kern0pt}cfunc{\isacharunderscore}{\kern0pt}prod\ id{\isacharunderscore}{\kern0pt}type\ left{\isacharunderscore}{\kern0pt}proj{\isacharunderscore}{\kern0pt}type{\isacharparenright}{\kern0pt}\isanewline
\ \ \ \ \ \ \ \ \ \ \isacommand{also}\isamarkupfalse%
\ \isacommand{have}\isamarkupfalse%
\ {\isachardoublequoteopen}{\isachardot}{\kern0pt}{\isachardot}{\kern0pt}{\isachardot}{\kern0pt}\ {\isacharequal}{\kern0pt}\ {\isacharparenleft}{\kern0pt}{\isasymphi}\ {\isasymcirc}\isactrlsub c\ left{\isacharunderscore}{\kern0pt}coproj\ {\isacharparenleft}{\kern0pt}A\ {\isasymtimes}\isactrlsub c\ B{\isacharparenright}{\kern0pt}\ {\isacharparenleft}{\kern0pt}A\ {\isasymtimes}\isactrlsub c\ C{\isacharparenright}{\kern0pt}{\isacharparenright}{\kern0pt}\ {\isasymcirc}\isactrlsub c\ {\isasymlangle}a{\isacharcomma}{\kern0pt}\ b{\isasymrangle}{\isachardoublequoteclose}\isanewline
\ \ \ \ \ \ \ \ \ \ \ \ \isacommand{unfolding}\isamarkupfalse%
\ {\isasymphi}{\isacharunderscore}{\kern0pt}def\ \isacommand{using}\isamarkupfalse%
\ \ left{\isacharunderscore}{\kern0pt}coproj{\isacharunderscore}{\kern0pt}cfunc{\isacharunderscore}{\kern0pt}coprod\ \isacommand{by}\isamarkupfalse%
\ {\isacharparenleft}{\kern0pt}typecheck{\isacharunderscore}{\kern0pt}cfuncs{\isacharcomma}{\kern0pt}\ auto{\isacharparenright}{\kern0pt}\isanewline
\ \ \ \ \ \ \ \ \ \ \isacommand{also}\isamarkupfalse%
\ \isacommand{have}\isamarkupfalse%
\ {\isachardoublequoteopen}{\isachardot}{\kern0pt}{\isachardot}{\kern0pt}{\isachardot}{\kern0pt}\ {\isacharequal}{\kern0pt}\ {\isasymphi}\ {\isasymcirc}\isactrlsub c\ x{\isachardoublequoteclose}\isanewline
\ \ \ \ \ \ \ \ \ \ \ \ \isacommand{using}\isamarkupfalse%
\ ab{\isacharunderscore}{\kern0pt}def\ comp{\isacharunderscore}{\kern0pt}associative{\isadigit{2}}\ x{\isacharprime}{\kern0pt}{\isacharunderscore}{\kern0pt}def\ \isacommand{by}\isamarkupfalse%
\ {\isacharparenleft}{\kern0pt}typecheck{\isacharunderscore}{\kern0pt}cfuncs{\isacharcomma}{\kern0pt}\ fastforce{\isacharparenright}{\kern0pt}\isanewline
\ \ \ \ \ \ \ \ \ \ \isacommand{also}\isamarkupfalse%
\ \isacommand{have}\isamarkupfalse%
\ {\isachardoublequoteopen}{\isachardot}{\kern0pt}{\isachardot}{\kern0pt}{\isachardot}{\kern0pt}\ {\isacharequal}{\kern0pt}\ {\isasymphi}\ {\isasymcirc}\isactrlsub c\ y{\isachardoublequoteclose}\isanewline
\ \ \ \ \ \ \ \ \ \ \ \ \isacommand{by}\isamarkupfalse%
\ {\isacharparenleft}{\kern0pt}simp\ add{\isacharcolon}{\kern0pt}\ local{\isachardot}{\kern0pt}equal{\isacharparenright}{\kern0pt}\isanewline
\ \ \ \ \ \ \ \ \ \ \isacommand{also}\isamarkupfalse%
\ \isacommand{have}\isamarkupfalse%
\ {\isachardoublequoteopen}{\isachardot}{\kern0pt}{\isachardot}{\kern0pt}{\isachardot}{\kern0pt}\ {\isacharequal}{\kern0pt}\ {\isacharparenleft}{\kern0pt}{\isasymphi}\ {\isasymcirc}\isactrlsub c\ left{\isacharunderscore}{\kern0pt}coproj\ {\isacharparenleft}{\kern0pt}A\ {\isasymtimes}\isactrlsub c\ B{\isacharparenright}{\kern0pt}\ {\isacharparenleft}{\kern0pt}A\ {\isasymtimes}\isactrlsub c\ C{\isacharparenright}{\kern0pt}{\isacharparenright}{\kern0pt}\ {\isasymcirc}\isactrlsub c\ {\isasymlangle}a{\isacharprime}{\kern0pt}{\isacharcomma}{\kern0pt}\ b{\isacharprime}{\kern0pt}{\isasymrangle}{\isachardoublequoteclose}\isanewline
\ \ \ \ \ \ \ \ \ \ \ \ \isacommand{using}\isamarkupfalse%
\ a{\isacharprime}{\kern0pt}b{\isacharprime}{\kern0pt}{\isacharunderscore}{\kern0pt}def\ comp{\isacharunderscore}{\kern0pt}associative{\isadigit{2}}\ {\isasymphi}{\isacharunderscore}{\kern0pt}type\ y{\isacharprime}{\kern0pt}{\isacharunderscore}{\kern0pt}def\ \isacommand{by}\isamarkupfalse%
\ {\isacharparenleft}{\kern0pt}typecheck{\isacharunderscore}{\kern0pt}cfuncs{\isacharcomma}{\kern0pt}\ blast{\isacharparenright}{\kern0pt}\isanewline
\ \ \ \ \ \ \ \ \ \ \isacommand{also}\isamarkupfalse%
\ \isacommand{have}\isamarkupfalse%
\ {\isachardoublequoteopen}{\isachardot}{\kern0pt}{\isachardot}{\kern0pt}{\isachardot}{\kern0pt}\ {\isacharequal}{\kern0pt}\ {\isacharparenleft}{\kern0pt}id\ A\ {\isasymtimes}\isactrlsub f\ left{\isacharunderscore}{\kern0pt}coproj\ B\ C{\isacharparenright}{\kern0pt}\ \ {\isasymcirc}\isactrlsub c\ {\isasymlangle}\ a{\isacharprime}{\kern0pt}{\isacharcomma}{\kern0pt}\ \ b{\isacharprime}{\kern0pt}{\isasymrangle}{\isachardoublequoteclose}\isanewline
\ \ \ \ \ \ \ \ \ \ \ \ \isacommand{unfolding}\isamarkupfalse%
\ {\isasymphi}{\isacharunderscore}{\kern0pt}def\ \isacommand{using}\isamarkupfalse%
\ left{\isacharunderscore}{\kern0pt}coproj{\isacharunderscore}{\kern0pt}cfunc{\isacharunderscore}{\kern0pt}coprod\ \isacommand{by}\isamarkupfalse%
\ {\isacharparenleft}{\kern0pt}typecheck{\isacharunderscore}{\kern0pt}cfuncs{\isacharcomma}{\kern0pt}\ auto{\isacharparenright}{\kern0pt}\isanewline
\ \ \ \ \ \ \ \ \ \ \isacommand{also}\isamarkupfalse%
\ \isacommand{have}\isamarkupfalse%
\ {\isachardoublequoteopen}{\isachardot}{\kern0pt}{\isachardot}{\kern0pt}{\isachardot}{\kern0pt}\ {\isacharequal}{\kern0pt}\ {\isasymlangle}id\ A\ {\isasymcirc}\isactrlsub c\ a{\isacharprime}{\kern0pt}{\isacharcomma}{\kern0pt}\ left{\isacharunderscore}{\kern0pt}coproj\ B\ C\ {\isasymcirc}\isactrlsub c\ b{\isacharprime}{\kern0pt}{\isasymrangle}{\isachardoublequoteclose}\isanewline
\ \ \ \ \ \ \ \ \ \ \ \ \isacommand{using}\isamarkupfalse%
\ a{\isacharprime}{\kern0pt}b{\isacharprime}{\kern0pt}{\isacharunderscore}{\kern0pt}def\ cfunc{\isacharunderscore}{\kern0pt}cross{\isacharunderscore}{\kern0pt}prod{\isacharunderscore}{\kern0pt}comp{\isacharunderscore}{\kern0pt}cfunc{\isacharunderscore}{\kern0pt}prod\ \isacommand{by}\isamarkupfalse%
\ {\isacharparenleft}{\kern0pt}typecheck{\isacharunderscore}{\kern0pt}cfuncs{\isacharcomma}{\kern0pt}\ auto{\isacharparenright}{\kern0pt}\isanewline
\ \ \ \ \ \ \ \ \ \ \isacommand{also}\isamarkupfalse%
\ \isacommand{have}\isamarkupfalse%
\ {\isachardoublequoteopen}{\isachardot}{\kern0pt}{\isachardot}{\kern0pt}{\isachardot}{\kern0pt}\ {\isacharequal}{\kern0pt}\ \ {\isasymlangle}a{\isacharprime}{\kern0pt}{\isacharcomma}{\kern0pt}\ left{\isacharunderscore}{\kern0pt}coproj\ B\ C\ {\isasymcirc}\isactrlsub c\ b{\isacharprime}{\kern0pt}{\isasymrangle}{\isachardoublequoteclose}\isanewline
\ \ \ \ \ \ \ \ \ \ \ \ \isacommand{using}\isamarkupfalse%
\ a{\isacharprime}{\kern0pt}b{\isacharprime}{\kern0pt}{\isacharunderscore}{\kern0pt}def\ id{\isacharunderscore}{\kern0pt}left{\isacharunderscore}{\kern0pt}unit{\isadigit{2}}\ \isacommand{by}\isamarkupfalse%
\ force\isanewline
\ \ \ \ \ \ \ \ \ \ \isacommand{then}\isamarkupfalse%
\ \isacommand{show}\isamarkupfalse%
\ {\isachardoublequoteopen}{\isasymlangle}a{\isacharcomma}{\kern0pt}\ left{\isacharunderscore}{\kern0pt}coproj\ B\ C\ {\isasymcirc}\isactrlsub c\ b{\isasymrangle}\ {\isacharequal}{\kern0pt}\ {\isasymlangle}a{\isacharprime}{\kern0pt}{\isacharcomma}{\kern0pt}\ left{\isacharunderscore}{\kern0pt}coproj\ B\ C\ {\isasymcirc}\isactrlsub c\ b{\isacharprime}{\kern0pt}{\isasymrangle}{\isachardoublequoteclose}\isanewline
\ \ \ \ \ \ \ \ \ \ \ \ \isacommand{by}\isamarkupfalse%
\ {\isacharparenleft}{\kern0pt}simp\ add{\isacharcolon}{\kern0pt}\ calculation{\isacharparenright}{\kern0pt}\isanewline
\ \ \ \ \ \ \ \ \isacommand{qed}\isamarkupfalse%
\isanewline
\ \ \ \ \ \ \ \ \isacommand{then}\isamarkupfalse%
\ \isacommand{have}\isamarkupfalse%
\ a{\isacharunderscore}{\kern0pt}equal{\isacharcolon}{\kern0pt}\ {\isachardoublequoteopen}a\ {\isacharequal}{\kern0pt}\ a{\isacharprime}{\kern0pt}\ {\isasymand}\ left{\isacharunderscore}{\kern0pt}coproj\ B\ C\ {\isasymcirc}\isactrlsub c\ b\ {\isacharequal}{\kern0pt}\ left{\isacharunderscore}{\kern0pt}coproj\ B\ C\ {\isasymcirc}\isactrlsub c\ b{\isacharprime}{\kern0pt}{\isachardoublequoteclose}\isanewline
\ \ \ \ \ \ \ \ \ \ \isacommand{using}\isamarkupfalse%
\ a{\isacharprime}{\kern0pt}b{\isacharprime}{\kern0pt}{\isacharunderscore}{\kern0pt}def\ ab{\isacharunderscore}{\kern0pt}def\ cart{\isacharunderscore}{\kern0pt}prod{\isacharunderscore}{\kern0pt}eq{\isadigit{2}}\ equal{\isacharunderscore}{\kern0pt}pair\ \isacommand{by}\isamarkupfalse%
\ {\isacharparenleft}{\kern0pt}typecheck{\isacharunderscore}{\kern0pt}cfuncs{\isacharcomma}{\kern0pt}\ blast{\isacharparenright}{\kern0pt}\isanewline
\ \ \ \ \ \ \ \ \isacommand{then}\isamarkupfalse%
\ \isacommand{have}\isamarkupfalse%
\ b{\isacharunderscore}{\kern0pt}equal{\isacharcolon}{\kern0pt}\ {\isachardoublequoteopen}b\ {\isacharequal}{\kern0pt}\ b{\isacharprime}{\kern0pt}{\isachardoublequoteclose}\isanewline
\ \ \ \ \ \ \ \ \ \ \isacommand{using}\isamarkupfalse%
\ a{\isacharprime}{\kern0pt}b{\isacharprime}{\kern0pt}{\isacharunderscore}{\kern0pt}def\ a{\isacharunderscore}{\kern0pt}equal\ ab{\isacharunderscore}{\kern0pt}def\ left{\isacharunderscore}{\kern0pt}coproj{\isacharunderscore}{\kern0pt}are{\isacharunderscore}{\kern0pt}monomorphisms\ left{\isacharunderscore}{\kern0pt}proj{\isacharunderscore}{\kern0pt}type\ monomorphism{\isacharunderscore}{\kern0pt}def{\isadigit{3}}\ \isacommand{by}\isamarkupfalse%
\ blast\isanewline
\ \ \ \ \ \ \ \ \isacommand{then}\isamarkupfalse%
\ \isacommand{show}\isamarkupfalse%
\ {\isachardoublequoteopen}x\ {\isacharequal}{\kern0pt}\ y{\isachardoublequoteclose}\isanewline
\ \ \ \ \ \ \ \ \ \ \isacommand{by}\isamarkupfalse%
\ {\isacharparenleft}{\kern0pt}simp\ add{\isacharcolon}{\kern0pt}\ a{\isacharprime}{\kern0pt}b{\isacharprime}{\kern0pt}{\isacharunderscore}{\kern0pt}def\ a{\isacharunderscore}{\kern0pt}equal\ ab{\isacharunderscore}{\kern0pt}def\ x{\isacharprime}{\kern0pt}{\isacharunderscore}{\kern0pt}def\ y{\isacharprime}{\kern0pt}{\isacharunderscore}{\kern0pt}def{\isacharparenright}{\kern0pt}\isanewline
\ \ \ \ \isacommand{next}\isamarkupfalse%
\ \isanewline
\ \ \ \ \ \ \isacommand{assume}\isamarkupfalse%
\ {\isachardoublequoteopen}{\isasymnexists}y{\isacharprime}{\kern0pt}{\isachardot}{\kern0pt}\ y{\isacharprime}{\kern0pt}\ {\isasymin}\isactrlsub c\ A\ {\isasymtimes}\isactrlsub c\ B\ {\isasymand}\ y\ {\isacharequal}{\kern0pt}\ left{\isacharunderscore}{\kern0pt}coproj\ {\isacharparenleft}{\kern0pt}A\ {\isasymtimes}\isactrlsub c\ B{\isacharparenright}{\kern0pt}\ {\isacharparenleft}{\kern0pt}A\ {\isasymtimes}\isactrlsub c\ C{\isacharparenright}{\kern0pt}\ {\isasymcirc}\isactrlsub c\ y{\isacharprime}{\kern0pt}{\isachardoublequoteclose}\isanewline
\ \ \ \ \ \ \isacommand{then}\isamarkupfalse%
\ \isacommand{obtain}\isamarkupfalse%
\ y{\isacharprime}{\kern0pt}\ \isakeyword{where}\ y{\isacharprime}{\kern0pt}{\isacharunderscore}{\kern0pt}def{\isacharcolon}{\kern0pt}\ {\isachardoublequoteopen}y{\isacharprime}{\kern0pt}\ {\isasymin}\isactrlsub c\ A\ {\isasymtimes}\isactrlsub c\ C{\isachardoublequoteclose}\ {\isachardoublequoteopen}y\ {\isacharequal}{\kern0pt}\ right{\isacharunderscore}{\kern0pt}coproj\ {\isacharparenleft}{\kern0pt}A\ {\isasymtimes}\isactrlsub c\ B{\isacharparenright}{\kern0pt}\ {\isacharparenleft}{\kern0pt}A\ {\isasymtimes}\isactrlsub c\ C{\isacharparenright}{\kern0pt}\ {\isasymcirc}\isactrlsub c\ y{\isacharprime}{\kern0pt}{\isachardoublequoteclose}\isanewline
\ \ \ \ \ \ \ \ \isacommand{using}\isamarkupfalse%
\ \ y{\isacharunderscore}{\kern0pt}form\ \isacommand{by}\isamarkupfalse%
\ blast\isanewline
\ \ \ \ \ \ \isacommand{then}\isamarkupfalse%
\ \isacommand{obtain}\isamarkupfalse%
\ a{\isacharprime}{\kern0pt}\ c{\isacharprime}{\kern0pt}\ \isakeyword{where}\ a{\isacharprime}{\kern0pt}c{\isacharprime}{\kern0pt}{\isacharunderscore}{\kern0pt}def{\isacharcolon}{\kern0pt}\ {\isachardoublequoteopen}a{\isacharprime}{\kern0pt}\ {\isasymin}\isactrlsub c\ A{\isachardoublequoteclose}\ {\isachardoublequoteopen}c{\isacharprime}{\kern0pt}\ {\isasymin}\isactrlsub c\ C{\isachardoublequoteclose}\ {\isachardoublequoteopen}y{\isacharprime}{\kern0pt}\ {\isacharequal}{\kern0pt}{\isasymlangle}a{\isacharprime}{\kern0pt}{\isacharcomma}{\kern0pt}c{\isacharprime}{\kern0pt}{\isasymrangle}{\isachardoublequoteclose}\isanewline
\ \ \ \ \ \ \ \ \isacommand{by}\isamarkupfalse%
\ {\isacharparenleft}{\kern0pt}meson\ cart{\isacharunderscore}{\kern0pt}prod{\isacharunderscore}{\kern0pt}decomp{\isacharparenright}{\kern0pt}\isanewline
\ \ \ \ \ \ \isacommand{have}\isamarkupfalse%
\ equal{\isacharunderscore}{\kern0pt}pair{\isacharcolon}{\kern0pt}\ {\isachardoublequoteopen}{\isasymlangle}a{\isacharcomma}{\kern0pt}\ {\isacharparenleft}{\kern0pt}left{\isacharunderscore}{\kern0pt}coproj\ B\ C{\isacharparenright}{\kern0pt}\ {\isasymcirc}\isactrlsub c\ b{\isasymrangle}\ {\isacharequal}{\kern0pt}\ {\isasymlangle}a{\isacharprime}{\kern0pt}{\isacharcomma}{\kern0pt}\ right{\isacharunderscore}{\kern0pt}coproj\ B\ C\ {\isasymcirc}\isactrlsub c\ c{\isacharprime}{\kern0pt}{\isasymrangle}{\isachardoublequoteclose}\isanewline
\ \ \ \ \ \ \isacommand{proof}\isamarkupfalse%
\ {\isacharminus}{\kern0pt}\ \isanewline
\ \ \ \ \ \ \ \ \isacommand{have}\isamarkupfalse%
\ {\isachardoublequoteopen}{\isasymlangle}a{\isacharcomma}{\kern0pt}\ left{\isacharunderscore}{\kern0pt}coproj\ B\ C\ {\isasymcirc}\isactrlsub c\ b{\isasymrangle}\ {\isacharequal}{\kern0pt}\ {\isasymlangle}id\ A\ {\isasymcirc}\isactrlsub c\ a{\isacharcomma}{\kern0pt}\ left{\isacharunderscore}{\kern0pt}coproj\ B\ C\ {\isasymcirc}\isactrlsub c\ b{\isasymrangle}{\isachardoublequoteclose}\isanewline
\ \ \ \ \ \ \ \ \ \ \isacommand{using}\isamarkupfalse%
\ ab{\isacharunderscore}{\kern0pt}def\ id{\isacharunderscore}{\kern0pt}left{\isacharunderscore}{\kern0pt}unit{\isadigit{2}}\ \isacommand{by}\isamarkupfalse%
\ force\isanewline
\ \ \ \ \ \ \ \ \isacommand{also}\isamarkupfalse%
\ \isacommand{have}\isamarkupfalse%
\ {\isachardoublequoteopen}{\isachardot}{\kern0pt}{\isachardot}{\kern0pt}{\isachardot}{\kern0pt}\ {\isacharequal}{\kern0pt}\ {\isacharparenleft}{\kern0pt}id\ A\ {\isasymtimes}\isactrlsub f\ left{\isacharunderscore}{\kern0pt}coproj\ B\ C{\isacharparenright}{\kern0pt}\ {\isasymcirc}\isactrlsub c\ {\isasymlangle}a{\isacharcomma}{\kern0pt}\ b{\isasymrangle}{\isachardoublequoteclose}\isanewline
\ \ \ \ \ \ \ \ \ \ \isacommand{by}\isamarkupfalse%
\ {\isacharparenleft}{\kern0pt}smt\ ab{\isacharunderscore}{\kern0pt}def\ cfunc{\isacharunderscore}{\kern0pt}cross{\isacharunderscore}{\kern0pt}prod{\isacharunderscore}{\kern0pt}comp{\isacharunderscore}{\kern0pt}cfunc{\isacharunderscore}{\kern0pt}prod\ id{\isacharunderscore}{\kern0pt}type\ left{\isacharunderscore}{\kern0pt}proj{\isacharunderscore}{\kern0pt}type{\isacharparenright}{\kern0pt}\isanewline
\ \ \ \ \ \ \ \ \isacommand{also}\isamarkupfalse%
\ \isacommand{have}\isamarkupfalse%
\ {\isachardoublequoteopen}{\isachardot}{\kern0pt}{\isachardot}{\kern0pt}{\isachardot}{\kern0pt}\ {\isacharequal}{\kern0pt}\ {\isacharparenleft}{\kern0pt}{\isasymphi}\ {\isasymcirc}\isactrlsub c\ left{\isacharunderscore}{\kern0pt}coproj\ {\isacharparenleft}{\kern0pt}A\ {\isasymtimes}\isactrlsub c\ B{\isacharparenright}{\kern0pt}\ {\isacharparenleft}{\kern0pt}A\ {\isasymtimes}\isactrlsub c\ C{\isacharparenright}{\kern0pt}{\isacharparenright}{\kern0pt}\ {\isasymcirc}\isactrlsub c\ {\isasymlangle}a{\isacharcomma}{\kern0pt}\ b{\isasymrangle}{\isachardoublequoteclose}\isanewline
\ \ \ \ \ \ \ \ \ \ \isacommand{unfolding}\isamarkupfalse%
\ {\isasymphi}{\isacharunderscore}{\kern0pt}def\ \isacommand{using}\isamarkupfalse%
\ left{\isacharunderscore}{\kern0pt}coproj{\isacharunderscore}{\kern0pt}cfunc{\isacharunderscore}{\kern0pt}coprod\ \isacommand{by}\isamarkupfalse%
\ {\isacharparenleft}{\kern0pt}typecheck{\isacharunderscore}{\kern0pt}cfuncs{\isacharcomma}{\kern0pt}\ auto{\isacharparenright}{\kern0pt}\isanewline
\ \ \ \ \ \ \ \ \isacommand{also}\isamarkupfalse%
\ \isacommand{have}\isamarkupfalse%
\ {\isachardoublequoteopen}{\isachardot}{\kern0pt}{\isachardot}{\kern0pt}{\isachardot}{\kern0pt}\ {\isacharequal}{\kern0pt}\ {\isasymphi}\ {\isasymcirc}\isactrlsub c\ x{\isachardoublequoteclose}\isanewline
\ \ \ \ \ \ \ \ \ \ \isacommand{using}\isamarkupfalse%
\ ab{\isacharunderscore}{\kern0pt}def\ comp{\isacharunderscore}{\kern0pt}associative{\isadigit{2}}\ {\isasymphi}{\isacharunderscore}{\kern0pt}type\ x{\isacharprime}{\kern0pt}{\isacharunderscore}{\kern0pt}def\ \isacommand{by}\isamarkupfalse%
\ {\isacharparenleft}{\kern0pt}typecheck{\isacharunderscore}{\kern0pt}cfuncs{\isacharcomma}{\kern0pt}\ fastforce{\isacharparenright}{\kern0pt}\isanewline
\ \ \ \ \ \ \ \ \isacommand{also}\isamarkupfalse%
\ \isacommand{have}\isamarkupfalse%
\ {\isachardoublequoteopen}{\isachardot}{\kern0pt}{\isachardot}{\kern0pt}{\isachardot}{\kern0pt}\ {\isacharequal}{\kern0pt}\ {\isasymphi}\ {\isasymcirc}\isactrlsub c\ y{\isachardoublequoteclose}\isanewline
\ \ \ \ \ \ \ \ \ \ \isacommand{by}\isamarkupfalse%
\ {\isacharparenleft}{\kern0pt}simp\ add{\isacharcolon}{\kern0pt}\ local{\isachardot}{\kern0pt}equal{\isacharparenright}{\kern0pt}\isanewline
\ \ \ \ \ \ \ \ \isacommand{also}\isamarkupfalse%
\ \isacommand{have}\isamarkupfalse%
\ {\isachardoublequoteopen}{\isachardot}{\kern0pt}{\isachardot}{\kern0pt}{\isachardot}{\kern0pt}\ {\isacharequal}{\kern0pt}\ {\isacharparenleft}{\kern0pt}{\isasymphi}\ {\isasymcirc}\isactrlsub c\ right{\isacharunderscore}{\kern0pt}coproj\ {\isacharparenleft}{\kern0pt}A\ {\isasymtimes}\isactrlsub c\ B{\isacharparenright}{\kern0pt}\ {\isacharparenleft}{\kern0pt}A\ {\isasymtimes}\isactrlsub c\ C{\isacharparenright}{\kern0pt}{\isacharparenright}{\kern0pt}\ {\isasymcirc}\isactrlsub c\ {\isasymlangle}a{\isacharprime}{\kern0pt}{\isacharcomma}{\kern0pt}\ c{\isacharprime}{\kern0pt}{\isasymrangle}{\isachardoublequoteclose}\isanewline
\ \ \ \ \ \ \ \ \ \ \isacommand{using}\isamarkupfalse%
\ a{\isacharprime}{\kern0pt}c{\isacharprime}{\kern0pt}{\isacharunderscore}{\kern0pt}def\ comp{\isacharunderscore}{\kern0pt}associative{\isadigit{2}}\ y{\isacharprime}{\kern0pt}{\isacharunderscore}{\kern0pt}def\ \isacommand{by}\isamarkupfalse%
\ {\isacharparenleft}{\kern0pt}typecheck{\isacharunderscore}{\kern0pt}cfuncs{\isacharcomma}{\kern0pt}\ blast{\isacharparenright}{\kern0pt}\isanewline
\ \ \ \ \ \ \ \ \ \ \isacommand{also}\isamarkupfalse%
\ \isacommand{have}\isamarkupfalse%
\ {\isachardoublequoteopen}{\isachardot}{\kern0pt}{\isachardot}{\kern0pt}{\isachardot}{\kern0pt}\ {\isacharequal}{\kern0pt}\ {\isacharparenleft}{\kern0pt}id\ A\ {\isasymtimes}\isactrlsub f\ right{\isacharunderscore}{\kern0pt}coproj\ B\ C{\isacharparenright}{\kern0pt}\ {\isasymcirc}\isactrlsub c\ {\isasymlangle}a{\isacharprime}{\kern0pt}{\isacharcomma}{\kern0pt}\ c{\isacharprime}{\kern0pt}{\isasymrangle}{\isachardoublequoteclose}\isanewline
\ \ \ \ \ \ \ \ \ \ \isacommand{unfolding}\isamarkupfalse%
\ {\isasymphi}{\isacharunderscore}{\kern0pt}def\ \isacommand{using}\isamarkupfalse%
\ right{\isacharunderscore}{\kern0pt}coproj{\isacharunderscore}{\kern0pt}cfunc{\isacharunderscore}{\kern0pt}coprod\ \isacommand{by}\isamarkupfalse%
\ {\isacharparenleft}{\kern0pt}typecheck{\isacharunderscore}{\kern0pt}cfuncs{\isacharcomma}{\kern0pt}\ auto{\isacharparenright}{\kern0pt}\isanewline
\ \ \ \ \ \ \ \ \isacommand{also}\isamarkupfalse%
\ \isacommand{have}\isamarkupfalse%
\ {\isachardoublequoteopen}{\isachardot}{\kern0pt}{\isachardot}{\kern0pt}{\isachardot}{\kern0pt}\ {\isacharequal}{\kern0pt}\ {\isasymlangle}id\ A\ {\isasymcirc}\isactrlsub c\ a{\isacharprime}{\kern0pt}{\isacharcomma}{\kern0pt}\ right{\isacharunderscore}{\kern0pt}coproj\ B\ C\ {\isasymcirc}\isactrlsub c\ c{\isacharprime}{\kern0pt}{\isasymrangle}{\isachardoublequoteclose}\isanewline
\ \ \ \ \ \ \ \ \ \ \isacommand{using}\isamarkupfalse%
\ a{\isacharprime}{\kern0pt}c{\isacharprime}{\kern0pt}{\isacharunderscore}{\kern0pt}def\ cfunc{\isacharunderscore}{\kern0pt}cross{\isacharunderscore}{\kern0pt}prod{\isacharunderscore}{\kern0pt}comp{\isacharunderscore}{\kern0pt}cfunc{\isacharunderscore}{\kern0pt}prod\ \isacommand{by}\isamarkupfalse%
\ {\isacharparenleft}{\kern0pt}typecheck{\isacharunderscore}{\kern0pt}cfuncs{\isacharcomma}{\kern0pt}auto{\isacharparenright}{\kern0pt}\isanewline
\ \ \ \ \ \ \ \ \isacommand{also}\isamarkupfalse%
\ \isacommand{have}\isamarkupfalse%
\ {\isachardoublequoteopen}{\isachardot}{\kern0pt}{\isachardot}{\kern0pt}{\isachardot}{\kern0pt}\ {\isacharequal}{\kern0pt}\ \ {\isasymlangle}a{\isacharprime}{\kern0pt}{\isacharcomma}{\kern0pt}\ right{\isacharunderscore}{\kern0pt}coproj\ B\ C\ {\isasymcirc}\isactrlsub c\ c{\isacharprime}{\kern0pt}{\isasymrangle}{\isachardoublequoteclose}\isanewline
\ \ \ \ \ \ \ \ \ \ \isacommand{using}\isamarkupfalse%
\ a{\isacharprime}{\kern0pt}c{\isacharprime}{\kern0pt}{\isacharunderscore}{\kern0pt}def\ id{\isacharunderscore}{\kern0pt}left{\isacharunderscore}{\kern0pt}unit{\isadigit{2}}\ \isacommand{by}\isamarkupfalse%
\ force\isanewline
\ \ \ \ \ \ \ \ \isacommand{then}\isamarkupfalse%
\ \isacommand{show}\isamarkupfalse%
\ {\isachardoublequoteopen}{\isasymlangle}a{\isacharcomma}{\kern0pt}\ left{\isacharunderscore}{\kern0pt}coproj\ B\ C\ {\isasymcirc}\isactrlsub c\ b{\isasymrangle}\ {\isacharequal}{\kern0pt}\ {\isasymlangle}a{\isacharprime}{\kern0pt}{\isacharcomma}{\kern0pt}\ right{\isacharunderscore}{\kern0pt}coproj\ B\ C\ {\isasymcirc}\isactrlsub c\ c{\isacharprime}{\kern0pt}{\isasymrangle}{\isachardoublequoteclose}\isanewline
\ \ \ \ \ \ \ \ \ \ \isacommand{by}\isamarkupfalse%
\ {\isacharparenleft}{\kern0pt}simp\ add{\isacharcolon}{\kern0pt}\ calculation{\isacharparenright}{\kern0pt}\isanewline
\ \ \ \ \ \ \isacommand{qed}\isamarkupfalse%
\ \ \ \ \ \ \ \ \isanewline
\ \ \ \ \ \ \isacommand{then}\isamarkupfalse%
\ \isacommand{have}\isamarkupfalse%
\ impossible{\isacharcolon}{\kern0pt}\ {\isachardoublequoteopen}left{\isacharunderscore}{\kern0pt}coproj\ B\ C\ {\isasymcirc}\isactrlsub c\ b\ {\isacharequal}{\kern0pt}\ right{\isacharunderscore}{\kern0pt}coproj\ B\ C\ {\isasymcirc}\isactrlsub c\ c{\isacharprime}{\kern0pt}{\isachardoublequoteclose}\isanewline
\ \ \ \ \ \ \ \ \isacommand{using}\isamarkupfalse%
\ a{\isacharprime}{\kern0pt}c{\isacharprime}{\kern0pt}{\isacharunderscore}{\kern0pt}def\ ab{\isacharunderscore}{\kern0pt}def\ element{\isacharunderscore}{\kern0pt}pair{\isacharunderscore}{\kern0pt}eq\ equal{\isacharunderscore}{\kern0pt}pair\ \isacommand{by}\isamarkupfalse%
\ {\isacharparenleft}{\kern0pt}typecheck{\isacharunderscore}{\kern0pt}cfuncs{\isacharcomma}{\kern0pt}\ blast{\isacharparenright}{\kern0pt}\isanewline
\ \ \ \ \ \ \isacommand{then}\isamarkupfalse%
\ \isacommand{show}\isamarkupfalse%
\ {\isachardoublequoteopen}x\ {\isacharequal}{\kern0pt}\ y{\isachardoublequoteclose}\isanewline
\ \ \ \ \ \ \ \ \isacommand{using}\isamarkupfalse%
\ a{\isacharprime}{\kern0pt}c{\isacharprime}{\kern0pt}{\isacharunderscore}{\kern0pt}def\ ab{\isacharunderscore}{\kern0pt}def\ coproducts{\isacharunderscore}{\kern0pt}disjoint\ \ \isacommand{by}\isamarkupfalse%
\ blast\isanewline
\ \ \ \ \isacommand{qed}\isamarkupfalse%
\isanewline
\ \ \isacommand{next}\isamarkupfalse%
\isanewline
\ \ \ \ \isacommand{assume}\isamarkupfalse%
\ {\isachardoublequoteopen}{\isasymnexists}x{\isacharprime}{\kern0pt}{\isachardot}{\kern0pt}\ x{\isacharprime}{\kern0pt}\ {\isasymin}\isactrlsub c\ A\ {\isasymtimes}\isactrlsub c\ B\ {\isasymand}\ x\ {\isacharequal}{\kern0pt}\ left{\isacharunderscore}{\kern0pt}coproj\ {\isacharparenleft}{\kern0pt}A\ {\isasymtimes}\isactrlsub c\ B{\isacharparenright}{\kern0pt}\ {\isacharparenleft}{\kern0pt}A\ {\isasymtimes}\isactrlsub c\ C{\isacharparenright}{\kern0pt}\ {\isasymcirc}\isactrlsub c\ x{\isacharprime}{\kern0pt}{\isachardoublequoteclose}\isanewline
\ \ \ \ \isacommand{then}\isamarkupfalse%
\ \isacommand{obtain}\isamarkupfalse%
\ x{\isacharprime}{\kern0pt}\ \isakeyword{where}\ x{\isacharprime}{\kern0pt}{\isacharunderscore}{\kern0pt}def{\isacharcolon}{\kern0pt}\ {\isachardoublequoteopen}x{\isacharprime}{\kern0pt}\ {\isasymin}\isactrlsub c\ A\ {\isasymtimes}\isactrlsub c\ C{\isachardoublequoteclose}\ {\isachardoublequoteopen}x\ {\isacharequal}{\kern0pt}\ right{\isacharunderscore}{\kern0pt}coproj\ {\isacharparenleft}{\kern0pt}A\ {\isasymtimes}\isactrlsub c\ B{\isacharparenright}{\kern0pt}\ {\isacharparenleft}{\kern0pt}A\ {\isasymtimes}\isactrlsub c\ C{\isacharparenright}{\kern0pt}\ {\isasymcirc}\isactrlsub c\ x{\isacharprime}{\kern0pt}{\isachardoublequoteclose}\isanewline
\ \ \ \ \ \ \isacommand{using}\isamarkupfalse%
\ \ x{\isacharunderscore}{\kern0pt}form\ \isacommand{by}\isamarkupfalse%
\ blast\isanewline
\ \ \ \ \isacommand{then}\isamarkupfalse%
\ \isacommand{have}\isamarkupfalse%
\ ac{\isacharunderscore}{\kern0pt}exists{\isacharcolon}{\kern0pt}\ {\isachardoublequoteopen}{\isasymexists}\ a\ c{\isachardot}{\kern0pt}\ a\ {\isasymin}\isactrlsub c\ A\ {\isasymand}\ c\ {\isasymin}\isactrlsub c\ C\ {\isasymand}\ x{\isacharprime}{\kern0pt}\ {\isacharequal}{\kern0pt}{\isasymlangle}a{\isacharcomma}{\kern0pt}c{\isasymrangle}{\isachardoublequoteclose}\isanewline
\ \ \ \ \ \ \isacommand{using}\isamarkupfalse%
\ cart{\isacharunderscore}{\kern0pt}prod{\isacharunderscore}{\kern0pt}decomp\ \isacommand{by}\isamarkupfalse%
\ blast\isanewline
\ \ \ \ \isacommand{then}\isamarkupfalse%
\ \isacommand{obtain}\isamarkupfalse%
\ a\ c\ \isakeyword{where}\ ac{\isacharunderscore}{\kern0pt}def{\isacharcolon}{\kern0pt}\ {\isachardoublequoteopen}a\ {\isasymin}\isactrlsub c\ A{\isachardoublequoteclose}\ {\isachardoublequoteopen}c\ {\isasymin}\isactrlsub c\ C{\isachardoublequoteclose}\ {\isachardoublequoteopen}x{\isacharprime}{\kern0pt}\ {\isacharequal}{\kern0pt}{\isasymlangle}a{\isacharcomma}{\kern0pt}c{\isasymrangle}{\isachardoublequoteclose}\isanewline
\ \ \ \ \ \ \isacommand{by}\isamarkupfalse%
\ blast\isanewline
\ \ \ \ \isacommand{show}\isamarkupfalse%
\ {\isachardoublequoteopen}x\ {\isacharequal}{\kern0pt}\ y{\isachardoublequoteclose}\ \ \isanewline
\ \ \ \ \isacommand{proof}\isamarkupfalse%
{\isacharparenleft}{\kern0pt}cases\ {\isachardoublequoteopen}{\isasymexists}\ y{\isacharprime}{\kern0pt}{\isachardot}{\kern0pt}\ y{\isacharprime}{\kern0pt}\ {\isasymin}\isactrlsub c\ A\ {\isasymtimes}\isactrlsub c\ B\ {\isasymand}\ y\ {\isacharequal}{\kern0pt}\ left{\isacharunderscore}{\kern0pt}coproj\ {\isacharparenleft}{\kern0pt}A\ {\isasymtimes}\isactrlsub c\ B{\isacharparenright}{\kern0pt}\ {\isacharparenleft}{\kern0pt}A\ {\isasymtimes}\isactrlsub c\ C{\isacharparenright}{\kern0pt}\ {\isasymcirc}\isactrlsub c\ y{\isacharprime}{\kern0pt}{\isachardoublequoteclose}{\isacharparenright}{\kern0pt}\isanewline
\ \ \ \ \ \ \isacommand{assume}\isamarkupfalse%
\ {\isachardoublequoteopen}{\isasymexists}\ y{\isacharprime}{\kern0pt}{\isachardot}{\kern0pt}\ y{\isacharprime}{\kern0pt}\ {\isasymin}\isactrlsub c\ A\ {\isasymtimes}\isactrlsub c\ B\ {\isasymand}\ y\ {\isacharequal}{\kern0pt}\ left{\isacharunderscore}{\kern0pt}coproj\ {\isacharparenleft}{\kern0pt}A\ {\isasymtimes}\isactrlsub c\ B{\isacharparenright}{\kern0pt}\ {\isacharparenleft}{\kern0pt}A\ {\isasymtimes}\isactrlsub c\ C{\isacharparenright}{\kern0pt}\ {\isasymcirc}\isactrlsub c\ y{\isacharprime}{\kern0pt}{\isachardoublequoteclose}\isanewline
\ \ \ \ \ \ \isacommand{then}\isamarkupfalse%
\ \isacommand{obtain}\isamarkupfalse%
\ y{\isacharprime}{\kern0pt}\ \isakeyword{where}\ y{\isacharprime}{\kern0pt}{\isacharunderscore}{\kern0pt}def{\isacharcolon}{\kern0pt}\ {\isachardoublequoteopen}y{\isacharprime}{\kern0pt}\ {\isasymin}\isactrlsub c\ A\ {\isasymtimes}\isactrlsub c\ B\ {\isasymand}\ y\ {\isacharequal}{\kern0pt}\ left{\isacharunderscore}{\kern0pt}coproj\ {\isacharparenleft}{\kern0pt}A\ {\isasymtimes}\isactrlsub c\ B{\isacharparenright}{\kern0pt}\ {\isacharparenleft}{\kern0pt}A\ {\isasymtimes}\isactrlsub c\ C{\isacharparenright}{\kern0pt}\ {\isasymcirc}\isactrlsub c\ y{\isacharprime}{\kern0pt}{\isachardoublequoteclose}\isanewline
\ \ \ \ \ \ \ \ \isacommand{by}\isamarkupfalse%
\ blast\isanewline
\ \ \ \ \ \ \isacommand{then}\isamarkupfalse%
\ \isacommand{obtain}\isamarkupfalse%
\ a{\isacharprime}{\kern0pt}\ b{\isacharprime}{\kern0pt}\ \isakeyword{where}\ a{\isacharprime}{\kern0pt}b{\isacharprime}{\kern0pt}{\isacharunderscore}{\kern0pt}def{\isacharcolon}{\kern0pt}\ {\isachardoublequoteopen}a{\isacharprime}{\kern0pt}\ {\isasymin}\isactrlsub c\ A\ {\isasymand}\ b{\isacharprime}{\kern0pt}\ {\isasymin}\isactrlsub c\ B\ {\isasymand}\ y{\isacharprime}{\kern0pt}\ {\isacharequal}{\kern0pt}{\isasymlangle}a{\isacharprime}{\kern0pt}{\isacharcomma}{\kern0pt}b{\isacharprime}{\kern0pt}{\isasymrangle}{\isachardoublequoteclose}\isanewline
\ \ \ \ \ \ \ \ \isacommand{using}\isamarkupfalse%
\ cart{\isacharunderscore}{\kern0pt}prod{\isacharunderscore}{\kern0pt}decomp\ y{\isacharprime}{\kern0pt}{\isacharunderscore}{\kern0pt}def\ \isacommand{by}\isamarkupfalse%
\ blast\isanewline
\ \ \ \ \ \ \isacommand{have}\isamarkupfalse%
\ equal{\isacharunderscore}{\kern0pt}pair{\isacharcolon}{\kern0pt}\ {\isachardoublequoteopen}{\isasymlangle}a{\isacharcomma}{\kern0pt}\ right{\isacharunderscore}{\kern0pt}coproj\ B\ C\ {\isasymcirc}\isactrlsub c\ c{\isasymrangle}\ {\isacharequal}{\kern0pt}\ {\isasymlangle}a{\isacharprime}{\kern0pt}{\isacharcomma}{\kern0pt}\ left{\isacharunderscore}{\kern0pt}coproj\ B\ C\ {\isasymcirc}\isactrlsub c\ b{\isacharprime}{\kern0pt}{\isasymrangle}{\isachardoublequoteclose}\isanewline
\ \ \ \ \ \ \isacommand{proof}\isamarkupfalse%
\ {\isacharminus}{\kern0pt}\ \isanewline
\ \ \ \ \ \ \ \ \isacommand{have}\isamarkupfalse%
\ {\isachardoublequoteopen}{\isasymlangle}a{\isacharcomma}{\kern0pt}\ right{\isacharunderscore}{\kern0pt}coproj\ B\ C\ {\isasymcirc}\isactrlsub c\ c{\isasymrangle}\ {\isacharequal}{\kern0pt}\ {\isasymlangle}id{\isacharparenleft}{\kern0pt}A{\isacharparenright}{\kern0pt}\ {\isasymcirc}\isactrlsub c\ a{\isacharcomma}{\kern0pt}\ right{\isacharunderscore}{\kern0pt}coproj\ B\ C\ {\isasymcirc}\isactrlsub c\ c{\isasymrangle}{\isachardoublequoteclose}\isanewline
\ \ \ \ \ \ \ \ \ \ \isacommand{using}\isamarkupfalse%
\ ac{\isacharunderscore}{\kern0pt}def\ id{\isacharunderscore}{\kern0pt}left{\isacharunderscore}{\kern0pt}unit{\isadigit{2}}\ \isacommand{by}\isamarkupfalse%
\ force\isanewline
\ \ \ \ \ \ \ \ \isacommand{also}\isamarkupfalse%
\ \isacommand{have}\isamarkupfalse%
\ {\isachardoublequoteopen}{\isachardot}{\kern0pt}{\isachardot}{\kern0pt}{\isachardot}{\kern0pt}\ {\isacharequal}{\kern0pt}\ {\isacharparenleft}{\kern0pt}id\ A\ {\isasymtimes}\isactrlsub f\ right{\isacharunderscore}{\kern0pt}coproj\ B\ C{\isacharparenright}{\kern0pt}\ \ {\isasymcirc}\isactrlsub c\ {\isasymlangle}a{\isacharcomma}{\kern0pt}\ c{\isasymrangle}{\isachardoublequoteclose}\isanewline
\ \ \ \ \ \ \ \ \ \ \isacommand{by}\isamarkupfalse%
\ {\isacharparenleft}{\kern0pt}smt\ ac{\isacharunderscore}{\kern0pt}def\ cfunc{\isacharunderscore}{\kern0pt}cross{\isacharunderscore}{\kern0pt}prod{\isacharunderscore}{\kern0pt}comp{\isacharunderscore}{\kern0pt}cfunc{\isacharunderscore}{\kern0pt}prod\ id{\isacharunderscore}{\kern0pt}type\ right{\isacharunderscore}{\kern0pt}proj{\isacharunderscore}{\kern0pt}type{\isacharparenright}{\kern0pt}\isanewline
\ \ \ \ \ \ \ \ \isacommand{also}\isamarkupfalse%
\ \isacommand{have}\isamarkupfalse%
\ {\isachardoublequoteopen}{\isachardot}{\kern0pt}{\isachardot}{\kern0pt}{\isachardot}{\kern0pt}\ {\isacharequal}{\kern0pt}\ {\isacharparenleft}{\kern0pt}{\isasymphi}\ {\isasymcirc}\isactrlsub c\ right{\isacharunderscore}{\kern0pt}coproj\ {\isacharparenleft}{\kern0pt}A\ {\isasymtimes}\isactrlsub c\ B{\isacharparenright}{\kern0pt}\ {\isacharparenleft}{\kern0pt}A\ {\isasymtimes}\isactrlsub c\ C{\isacharparenright}{\kern0pt}{\isacharparenright}{\kern0pt}\ {\isasymcirc}\isactrlsub c\ {\isasymlangle}a{\isacharcomma}{\kern0pt}\ c{\isasymrangle}{\isachardoublequoteclose}\isanewline
\ \ \ \ \ \ \ \ \ \ \isacommand{unfolding}\isamarkupfalse%
\ {\isasymphi}{\isacharunderscore}{\kern0pt}def\ \isacommand{using}\isamarkupfalse%
\ right{\isacharunderscore}{\kern0pt}coproj{\isacharunderscore}{\kern0pt}cfunc{\isacharunderscore}{\kern0pt}coprod\ \isacommand{by}\isamarkupfalse%
\ {\isacharparenleft}{\kern0pt}typecheck{\isacharunderscore}{\kern0pt}cfuncs{\isacharcomma}{\kern0pt}\ auto{\isacharparenright}{\kern0pt}\isanewline
\ \ \ \ \ \ \ \ \isacommand{also}\isamarkupfalse%
\ \isacommand{have}\isamarkupfalse%
\ {\isachardoublequoteopen}{\isachardot}{\kern0pt}{\isachardot}{\kern0pt}{\isachardot}{\kern0pt}\ {\isacharequal}{\kern0pt}\ {\isasymphi}\ {\isasymcirc}\isactrlsub c\ x{\isachardoublequoteclose}\isanewline
\ \ \ \ \ \ \ \ \ \ \isacommand{using}\isamarkupfalse%
\ ac{\isacharunderscore}{\kern0pt}def\ comp{\isacharunderscore}{\kern0pt}associative{\isadigit{2}}\ {\isasymphi}{\isacharunderscore}{\kern0pt}type\ x{\isacharprime}{\kern0pt}{\isacharunderscore}{\kern0pt}def\ \isacommand{by}\isamarkupfalse%
\ {\isacharparenleft}{\kern0pt}typecheck{\isacharunderscore}{\kern0pt}cfuncs{\isacharcomma}{\kern0pt}\ fastforce{\isacharparenright}{\kern0pt}\isanewline
\ \ \ \ \ \ \ \ \isacommand{also}\isamarkupfalse%
\ \isacommand{have}\isamarkupfalse%
\ {\isachardoublequoteopen}{\isachardot}{\kern0pt}{\isachardot}{\kern0pt}{\isachardot}{\kern0pt}\ {\isacharequal}{\kern0pt}\ {\isasymphi}\ {\isasymcirc}\isactrlsub c\ y{\isachardoublequoteclose}\isanewline
\ \ \ \ \ \ \ \ \ \ \isacommand{by}\isamarkupfalse%
\ {\isacharparenleft}{\kern0pt}simp\ add{\isacharcolon}{\kern0pt}\ local{\isachardot}{\kern0pt}equal{\isacharparenright}{\kern0pt}\isanewline
\ \ \ \ \ \ \ \ \isacommand{also}\isamarkupfalse%
\ \isacommand{have}\isamarkupfalse%
\ {\isachardoublequoteopen}{\isachardot}{\kern0pt}{\isachardot}{\kern0pt}{\isachardot}{\kern0pt}\ {\isacharequal}{\kern0pt}\ {\isacharparenleft}{\kern0pt}{\isasymphi}\ {\isasymcirc}\isactrlsub c\ left{\isacharunderscore}{\kern0pt}coproj\ {\isacharparenleft}{\kern0pt}A\ {\isasymtimes}\isactrlsub c\ B{\isacharparenright}{\kern0pt}\ {\isacharparenleft}{\kern0pt}A\ {\isasymtimes}\isactrlsub c\ C{\isacharparenright}{\kern0pt}{\isacharparenright}{\kern0pt}\ {\isasymcirc}\isactrlsub c\ {\isasymlangle}a{\isacharprime}{\kern0pt}{\isacharcomma}{\kern0pt}\ b{\isacharprime}{\kern0pt}{\isasymrangle}{\isachardoublequoteclose}\isanewline
\ \ \ \ \ \ \ \ \ \ \isacommand{using}\isamarkupfalse%
\ a{\isacharprime}{\kern0pt}b{\isacharprime}{\kern0pt}{\isacharunderscore}{\kern0pt}def\ comp{\isacharunderscore}{\kern0pt}associative{\isadigit{2}}\ {\isasymphi}{\isacharunderscore}{\kern0pt}type\ y{\isacharprime}{\kern0pt}{\isacharunderscore}{\kern0pt}def\ \isacommand{by}\isamarkupfalse%
\ {\isacharparenleft}{\kern0pt}typecheck{\isacharunderscore}{\kern0pt}cfuncs{\isacharcomma}{\kern0pt}\ blast{\isacharparenright}{\kern0pt}\isanewline
\ \ \ \ \ \ \ \ \ \ \isacommand{also}\isamarkupfalse%
\ \isacommand{have}\isamarkupfalse%
\ {\isachardoublequoteopen}{\isachardot}{\kern0pt}{\isachardot}{\kern0pt}{\isachardot}{\kern0pt}\ {\isacharequal}{\kern0pt}\ {\isacharparenleft}{\kern0pt}id\ A\ {\isasymtimes}\isactrlsub f\ left{\isacharunderscore}{\kern0pt}coproj\ B\ C{\isacharparenright}{\kern0pt}\ {\isasymcirc}\isactrlsub c\ {\isasymlangle}a{\isacharprime}{\kern0pt}{\isacharcomma}{\kern0pt}\ b{\isacharprime}{\kern0pt}{\isasymrangle}{\isachardoublequoteclose}\isanewline
\ \ \ \ \ \ \ \ \ \ \isacommand{unfolding}\isamarkupfalse%
\ {\isasymphi}{\isacharunderscore}{\kern0pt}def\ \isacommand{using}\isamarkupfalse%
\ left{\isacharunderscore}{\kern0pt}coproj{\isacharunderscore}{\kern0pt}cfunc{\isacharunderscore}{\kern0pt}coprod\ \isacommand{by}\isamarkupfalse%
\ {\isacharparenleft}{\kern0pt}typecheck{\isacharunderscore}{\kern0pt}cfuncs{\isacharcomma}{\kern0pt}\ auto{\isacharparenright}{\kern0pt}\isanewline
\ \ \ \ \ \ \ \ \isacommand{also}\isamarkupfalse%
\ \isacommand{have}\isamarkupfalse%
\ {\isachardoublequoteopen}{\isachardot}{\kern0pt}{\isachardot}{\kern0pt}{\isachardot}{\kern0pt}\ {\isacharequal}{\kern0pt}\ {\isasymlangle}id\ A\ {\isasymcirc}\isactrlsub c\ a{\isacharprime}{\kern0pt}{\isacharcomma}{\kern0pt}\ left{\isacharunderscore}{\kern0pt}coproj\ B\ C\ {\isasymcirc}\isactrlsub c\ b{\isacharprime}{\kern0pt}{\isasymrangle}{\isachardoublequoteclose}\isanewline
\ \ \ \ \ \ \ \ \ \ \isacommand{using}\isamarkupfalse%
\ a{\isacharprime}{\kern0pt}b{\isacharprime}{\kern0pt}{\isacharunderscore}{\kern0pt}def\ cfunc{\isacharunderscore}{\kern0pt}cross{\isacharunderscore}{\kern0pt}prod{\isacharunderscore}{\kern0pt}comp{\isacharunderscore}{\kern0pt}cfunc{\isacharunderscore}{\kern0pt}prod\ \isacommand{by}\isamarkupfalse%
\ {\isacharparenleft}{\kern0pt}typecheck{\isacharunderscore}{\kern0pt}cfuncs{\isacharcomma}{\kern0pt}auto{\isacharparenright}{\kern0pt}\isanewline
\ \ \ \ \ \ \ \ \isacommand{also}\isamarkupfalse%
\ \isacommand{have}\isamarkupfalse%
\ {\isachardoublequoteopen}{\isachardot}{\kern0pt}{\isachardot}{\kern0pt}{\isachardot}{\kern0pt}\ {\isacharequal}{\kern0pt}\ \ {\isasymlangle}a{\isacharprime}{\kern0pt}{\isacharcomma}{\kern0pt}\ left{\isacharunderscore}{\kern0pt}coproj\ B\ C\ {\isasymcirc}\isactrlsub c\ b{\isacharprime}{\kern0pt}{\isasymrangle}{\isachardoublequoteclose}\isanewline
\ \ \ \ \ \ \ \ \ \ \isacommand{using}\isamarkupfalse%
\ a{\isacharprime}{\kern0pt}b{\isacharprime}{\kern0pt}{\isacharunderscore}{\kern0pt}def\ id{\isacharunderscore}{\kern0pt}left{\isacharunderscore}{\kern0pt}unit{\isadigit{2}}\ \isacommand{by}\isamarkupfalse%
\ force\isanewline
\ \ \ \ \ \ \ \ \isacommand{then}\isamarkupfalse%
\ \isacommand{show}\isamarkupfalse%
\ {\isachardoublequoteopen}{\isasymlangle}a{\isacharcomma}{\kern0pt}\ right{\isacharunderscore}{\kern0pt}coproj\ B\ C\ {\isasymcirc}\isactrlsub c\ c{\isasymrangle}\ {\isacharequal}{\kern0pt}\ {\isasymlangle}a{\isacharprime}{\kern0pt}{\isacharcomma}{\kern0pt}\ left{\isacharunderscore}{\kern0pt}coproj\ B\ C\ {\isasymcirc}\isactrlsub c\ b{\isacharprime}{\kern0pt}{\isasymrangle}{\isachardoublequoteclose}\isanewline
\ \ \ \ \ \ \ \ \ \ \isacommand{by}\isamarkupfalse%
\ {\isacharparenleft}{\kern0pt}simp\ add{\isacharcolon}{\kern0pt}\ calculation{\isacharparenright}{\kern0pt}\isanewline
\ \ \ \ \ \ \isacommand{qed}\isamarkupfalse%
\ \ \ \ \ \ \ \ \isanewline
\ \ \ \ \ \ \isacommand{then}\isamarkupfalse%
\ \isacommand{have}\isamarkupfalse%
\ impossible{\isacharcolon}{\kern0pt}\ \ {\isachardoublequoteopen}right{\isacharunderscore}{\kern0pt}coproj\ B\ C\ {\isasymcirc}\isactrlsub c\ c\ {\isacharequal}{\kern0pt}\ left{\isacharunderscore}{\kern0pt}coproj\ B\ C\ {\isasymcirc}\isactrlsub c\ b{\isacharprime}{\kern0pt}{\isachardoublequoteclose}\isanewline
\ \ \ \ \ \ \ \ \ \ \isacommand{using}\isamarkupfalse%
\ a{\isacharprime}{\kern0pt}b{\isacharprime}{\kern0pt}{\isacharunderscore}{\kern0pt}def\ ac{\isacharunderscore}{\kern0pt}def\ cart{\isacharunderscore}{\kern0pt}prod{\isacharunderscore}{\kern0pt}eq{\isadigit{2}}\ equal{\isacharunderscore}{\kern0pt}pair\ \isacommand{by}\isamarkupfalse%
\ {\isacharparenleft}{\kern0pt}typecheck{\isacharunderscore}{\kern0pt}cfuncs{\isacharcomma}{\kern0pt}\ blast{\isacharparenright}{\kern0pt}\isanewline
\ \ \ \ \ \ \ \ \isacommand{then}\isamarkupfalse%
\ \isacommand{show}\isamarkupfalse%
\ {\isachardoublequoteopen}x\ {\isacharequal}{\kern0pt}\ y{\isachardoublequoteclose}\isanewline
\ \ \ \ \ \ \ \ \ \ \isacommand{using}\isamarkupfalse%
\ a{\isacharprime}{\kern0pt}b{\isacharprime}{\kern0pt}{\isacharunderscore}{\kern0pt}def\ ac{\isacharunderscore}{\kern0pt}def\ coproducts{\isacharunderscore}{\kern0pt}disjoint\ \isacommand{by}\isamarkupfalse%
\ force\isanewline
\ \ \ \ \ \ \isacommand{next}\isamarkupfalse%
\ \isanewline
\ \ \ \ \ \ \ \ \isacommand{assume}\isamarkupfalse%
\ {\isachardoublequoteopen}{\isasymnexists}y{\isacharprime}{\kern0pt}{\isachardot}{\kern0pt}\ y{\isacharprime}{\kern0pt}\ {\isasymin}\isactrlsub c\ A\ {\isasymtimes}\isactrlsub c\ B\ {\isasymand}\ y\ {\isacharequal}{\kern0pt}\ left{\isacharunderscore}{\kern0pt}coproj\ {\isacharparenleft}{\kern0pt}A\ {\isasymtimes}\isactrlsub c\ B{\isacharparenright}{\kern0pt}\ {\isacharparenleft}{\kern0pt}A\ {\isasymtimes}\isactrlsub c\ C{\isacharparenright}{\kern0pt}\ {\isasymcirc}\isactrlsub c\ y{\isacharprime}{\kern0pt}{\isachardoublequoteclose}\isanewline
\ \ \ \ \ \ \ \ \isacommand{then}\isamarkupfalse%
\ \isacommand{obtain}\isamarkupfalse%
\ y{\isacharprime}{\kern0pt}\ \isakeyword{where}\ y{\isacharprime}{\kern0pt}{\isacharunderscore}{\kern0pt}def{\isacharcolon}{\kern0pt}\ {\isachardoublequoteopen}y{\isacharprime}{\kern0pt}\ {\isasymin}\isactrlsub c\ {\isacharparenleft}{\kern0pt}A\ {\isasymtimes}\isactrlsub c\ C{\isacharparenright}{\kern0pt}\ {\isasymand}\ y\ {\isacharequal}{\kern0pt}\ right{\isacharunderscore}{\kern0pt}coproj\ {\isacharparenleft}{\kern0pt}A\ {\isasymtimes}\isactrlsub c\ B{\isacharparenright}{\kern0pt}\ {\isacharparenleft}{\kern0pt}A\ {\isasymtimes}\isactrlsub c\ C{\isacharparenright}{\kern0pt}\ {\isasymcirc}\isactrlsub c\ y{\isacharprime}{\kern0pt}{\isachardoublequoteclose}\isanewline
\ \ \ \ \ \ \ \ \ \ \isacommand{using}\isamarkupfalse%
\ \ y{\isacharunderscore}{\kern0pt}form\ \isacommand{by}\isamarkupfalse%
\ blast\isanewline
\ \ \ \ \ \ \ \ \isacommand{then}\isamarkupfalse%
\ \isacommand{obtain}\isamarkupfalse%
\ a{\isacharprime}{\kern0pt}\ c{\isacharprime}{\kern0pt}\ \isakeyword{where}\ a{\isacharprime}{\kern0pt}c{\isacharprime}{\kern0pt}{\isacharunderscore}{\kern0pt}def{\isacharcolon}{\kern0pt}\ {\isachardoublequoteopen}a{\isacharprime}{\kern0pt}\ {\isasymin}\isactrlsub c\ A{\isachardoublequoteclose}\ {\isachardoublequoteopen}c{\isacharprime}{\kern0pt}\ {\isasymin}\isactrlsub c\ C{\isachardoublequoteclose}\ {\isachardoublequoteopen}y{\isacharprime}{\kern0pt}\ {\isacharequal}{\kern0pt}{\isasymlangle}a{\isacharprime}{\kern0pt}{\isacharcomma}{\kern0pt}c{\isacharprime}{\kern0pt}{\isasymrangle}{\isachardoublequoteclose}\isanewline
\ \ \ \ \ \ \ \ \ \ \isacommand{using}\isamarkupfalse%
\ cart{\isacharunderscore}{\kern0pt}prod{\isacharunderscore}{\kern0pt}decomp\ \isacommand{by}\isamarkupfalse%
\ blast\isanewline
\ \ \ \ \ \ \ \ \isacommand{have}\isamarkupfalse%
\ equal{\isacharunderscore}{\kern0pt}pair{\isacharcolon}{\kern0pt}\ {\isachardoublequoteopen}{\isasymlangle}a{\isacharcomma}{\kern0pt}\ right{\isacharunderscore}{\kern0pt}coproj\ B\ C\ {\isasymcirc}\isactrlsub c\ c{\isasymrangle}\ {\isacharequal}{\kern0pt}\ {\isasymlangle}a{\isacharprime}{\kern0pt}{\isacharcomma}{\kern0pt}\ right{\isacharunderscore}{\kern0pt}coproj\ B\ C\ {\isasymcirc}\isactrlsub c\ c{\isacharprime}{\kern0pt}{\isasymrangle}{\isachardoublequoteclose}\isanewline
\ \ \ \ \ \ \ \ \isacommand{proof}\isamarkupfalse%
\ {\isacharminus}{\kern0pt}\ \isanewline
\ \ \ \ \ \ \ \ \ \ \isacommand{have}\isamarkupfalse%
\ {\isachardoublequoteopen}{\isasymlangle}a{\isacharcomma}{\kern0pt}\ right{\isacharunderscore}{\kern0pt}coproj\ B\ C\ {\isasymcirc}\isactrlsub c\ c{\isasymrangle}\ {\isacharequal}{\kern0pt}\ {\isasymlangle}id\ A\ {\isasymcirc}\isactrlsub c\ a{\isacharcomma}{\kern0pt}\ right{\isacharunderscore}{\kern0pt}coproj\ B\ C\ {\isasymcirc}\isactrlsub c\ c{\isasymrangle}{\isachardoublequoteclose}\isanewline
\ \ \ \ \ \ \ \ \ \ \ \ \isacommand{using}\isamarkupfalse%
\ ac{\isacharunderscore}{\kern0pt}def\ id{\isacharunderscore}{\kern0pt}left{\isacharunderscore}{\kern0pt}unit{\isadigit{2}}\ \isacommand{by}\isamarkupfalse%
\ force\isanewline
\ \ \ \ \ \ \ \ \ \ \isacommand{also}\isamarkupfalse%
\ \isacommand{have}\isamarkupfalse%
\ {\isachardoublequoteopen}{\isachardot}{\kern0pt}{\isachardot}{\kern0pt}{\isachardot}{\kern0pt}\ {\isacharequal}{\kern0pt}\ {\isacharparenleft}{\kern0pt}id\ A\ {\isasymtimes}\isactrlsub f\ right{\isacharunderscore}{\kern0pt}coproj\ B\ C{\isacharparenright}{\kern0pt}\ \ {\isasymcirc}\isactrlsub c\ {\isasymlangle}a{\isacharcomma}{\kern0pt}\ \ c{\isasymrangle}{\isachardoublequoteclose}\isanewline
\ \ \ \ \ \ \ \ \ \ \ \ \isacommand{by}\isamarkupfalse%
\ {\isacharparenleft}{\kern0pt}smt\ ac{\isacharunderscore}{\kern0pt}def\ cfunc{\isacharunderscore}{\kern0pt}cross{\isacharunderscore}{\kern0pt}prod{\isacharunderscore}{\kern0pt}comp{\isacharunderscore}{\kern0pt}cfunc{\isacharunderscore}{\kern0pt}prod\ id{\isacharunderscore}{\kern0pt}type\ right{\isacharunderscore}{\kern0pt}proj{\isacharunderscore}{\kern0pt}type{\isacharparenright}{\kern0pt}\isanewline
\ \ \ \ \ \ \ \ \ \ \isacommand{also}\isamarkupfalse%
\ \isacommand{have}\isamarkupfalse%
\ {\isachardoublequoteopen}{\isachardot}{\kern0pt}{\isachardot}{\kern0pt}{\isachardot}{\kern0pt}\ {\isacharequal}{\kern0pt}\ {\isacharparenleft}{\kern0pt}{\isasymphi}\ {\isasymcirc}\isactrlsub c\ right{\isacharunderscore}{\kern0pt}coproj\ {\isacharparenleft}{\kern0pt}A\ {\isasymtimes}\isactrlsub c\ B{\isacharparenright}{\kern0pt}\ {\isacharparenleft}{\kern0pt}A\ {\isasymtimes}\isactrlsub c\ C{\isacharparenright}{\kern0pt}{\isacharparenright}{\kern0pt}\ {\isasymcirc}\isactrlsub c\ {\isasymlangle}a{\isacharcomma}{\kern0pt}\ c{\isasymrangle}{\isachardoublequoteclose}\isanewline
\ \ \ \ \ \ \ \ \ \ \ \ \isacommand{unfolding}\isamarkupfalse%
\ {\isasymphi}{\isacharunderscore}{\kern0pt}def\ \isacommand{using}\isamarkupfalse%
\ right{\isacharunderscore}{\kern0pt}coproj{\isacharunderscore}{\kern0pt}cfunc{\isacharunderscore}{\kern0pt}coprod\ \isacommand{by}\isamarkupfalse%
\ {\isacharparenleft}{\kern0pt}typecheck{\isacharunderscore}{\kern0pt}cfuncs{\isacharcomma}{\kern0pt}\ auto{\isacharparenright}{\kern0pt}\isanewline
\ \ \ \ \ \ \ \ \ \ \isacommand{also}\isamarkupfalse%
\ \isacommand{have}\isamarkupfalse%
\ {\isachardoublequoteopen}{\isachardot}{\kern0pt}{\isachardot}{\kern0pt}{\isachardot}{\kern0pt}\ {\isacharequal}{\kern0pt}\ {\isasymphi}\ {\isasymcirc}\isactrlsub c\ x{\isachardoublequoteclose}\isanewline
\ \ \ \ \ \ \ \ \ \ \ \ \isacommand{using}\isamarkupfalse%
\ ac{\isacharunderscore}{\kern0pt}def\ comp{\isacharunderscore}{\kern0pt}associative{\isadigit{2}}\ {\isasymphi}{\isacharunderscore}{\kern0pt}type\ x{\isacharprime}{\kern0pt}{\isacharunderscore}{\kern0pt}def\ \isacommand{by}\isamarkupfalse%
\ {\isacharparenleft}{\kern0pt}typecheck{\isacharunderscore}{\kern0pt}cfuncs{\isacharcomma}{\kern0pt}\ fastforce{\isacharparenright}{\kern0pt}\isanewline
\ \ \ \ \ \ \ \ \ \ \isacommand{also}\isamarkupfalse%
\ \isacommand{have}\isamarkupfalse%
\ {\isachardoublequoteopen}{\isachardot}{\kern0pt}{\isachardot}{\kern0pt}{\isachardot}{\kern0pt}\ {\isacharequal}{\kern0pt}\ {\isasymphi}\ {\isasymcirc}\isactrlsub c\ y{\isachardoublequoteclose}\isanewline
\ \ \ \ \ \ \ \ \ \ \ \ \isacommand{by}\isamarkupfalse%
\ {\isacharparenleft}{\kern0pt}simp\ add{\isacharcolon}{\kern0pt}\ local{\isachardot}{\kern0pt}equal{\isacharparenright}{\kern0pt}\isanewline
\ \ \ \ \ \ \ \ \ \ \isacommand{also}\isamarkupfalse%
\ \isacommand{have}\isamarkupfalse%
\ {\isachardoublequoteopen}{\isachardot}{\kern0pt}{\isachardot}{\kern0pt}{\isachardot}{\kern0pt}\ {\isacharequal}{\kern0pt}\ {\isacharparenleft}{\kern0pt}{\isasymphi}\ {\isasymcirc}\isactrlsub c\ right{\isacharunderscore}{\kern0pt}coproj\ {\isacharparenleft}{\kern0pt}A\ {\isasymtimes}\isactrlsub c\ B{\isacharparenright}{\kern0pt}\ {\isacharparenleft}{\kern0pt}A\ {\isasymtimes}\isactrlsub c\ C{\isacharparenright}{\kern0pt}{\isacharparenright}{\kern0pt}\ {\isasymcirc}\isactrlsub c\ {\isasymlangle}a{\isacharprime}{\kern0pt}{\isacharcomma}{\kern0pt}\ c{\isacharprime}{\kern0pt}{\isasymrangle}{\isachardoublequoteclose}\isanewline
\ \ \ \ \ \ \ \ \ \ \ \ \isacommand{using}\isamarkupfalse%
\ a{\isacharprime}{\kern0pt}c{\isacharprime}{\kern0pt}{\isacharunderscore}{\kern0pt}def\ comp{\isacharunderscore}{\kern0pt}associative{\isadigit{2}}\ {\isasymphi}{\isacharunderscore}{\kern0pt}type\ y{\isacharprime}{\kern0pt}{\isacharunderscore}{\kern0pt}def\ \isacommand{by}\isamarkupfalse%
\ {\isacharparenleft}{\kern0pt}typecheck{\isacharunderscore}{\kern0pt}cfuncs{\isacharcomma}{\kern0pt}\ blast{\isacharparenright}{\kern0pt}\isanewline
\ \ \ \ \ \ \ \ \ \ \isacommand{also}\isamarkupfalse%
\ \isacommand{have}\isamarkupfalse%
\ {\isachardoublequoteopen}{\isachardot}{\kern0pt}{\isachardot}{\kern0pt}{\isachardot}{\kern0pt}\ {\isacharequal}{\kern0pt}\ {\isacharparenleft}{\kern0pt}id\ A\ {\isasymtimes}\isactrlsub f\ right{\isacharunderscore}{\kern0pt}coproj\ B\ C{\isacharparenright}{\kern0pt}\ \ {\isasymcirc}\isactrlsub c\ {\isasymlangle}a{\isacharprime}{\kern0pt}{\isacharcomma}{\kern0pt}\ \ c{\isacharprime}{\kern0pt}{\isasymrangle}{\isachardoublequoteclose}\isanewline
\ \ \ \ \ \ \ \ \ \ \ \ \isacommand{unfolding}\isamarkupfalse%
\ {\isasymphi}{\isacharunderscore}{\kern0pt}def\ \isacommand{using}\isamarkupfalse%
\ right{\isacharunderscore}{\kern0pt}coproj{\isacharunderscore}{\kern0pt}cfunc{\isacharunderscore}{\kern0pt}coprod\ \isacommand{by}\isamarkupfalse%
\ {\isacharparenleft}{\kern0pt}typecheck{\isacharunderscore}{\kern0pt}cfuncs{\isacharcomma}{\kern0pt}\ auto{\isacharparenright}{\kern0pt}\isanewline
\ \ \ \ \ \ \ \ \ \ \isacommand{also}\isamarkupfalse%
\ \isacommand{have}\isamarkupfalse%
\ {\isachardoublequoteopen}{\isachardot}{\kern0pt}{\isachardot}{\kern0pt}{\isachardot}{\kern0pt}\ {\isacharequal}{\kern0pt}\ {\isasymlangle}id\ A\ {\isasymcirc}\isactrlsub c\ a{\isacharprime}{\kern0pt}{\isacharcomma}{\kern0pt}\ right{\isacharunderscore}{\kern0pt}coproj\ B\ C\ {\isasymcirc}\isactrlsub c\ c{\isacharprime}{\kern0pt}{\isasymrangle}{\isachardoublequoteclose}\isanewline
\ \ \ \ \ \ \ \ \ \ \ \ \isacommand{using}\isamarkupfalse%
\ a{\isacharprime}{\kern0pt}c{\isacharprime}{\kern0pt}{\isacharunderscore}{\kern0pt}def\ cfunc{\isacharunderscore}{\kern0pt}cross{\isacharunderscore}{\kern0pt}prod{\isacharunderscore}{\kern0pt}comp{\isacharunderscore}{\kern0pt}cfunc{\isacharunderscore}{\kern0pt}prod\ \isacommand{by}\isamarkupfalse%
\ {\isacharparenleft}{\kern0pt}typecheck{\isacharunderscore}{\kern0pt}cfuncs{\isacharcomma}{\kern0pt}auto{\isacharparenright}{\kern0pt}\isanewline
\ \ \ \ \ \ \ \ \ \ \isacommand{also}\isamarkupfalse%
\ \isacommand{have}\isamarkupfalse%
\ {\isachardoublequoteopen}{\isachardot}{\kern0pt}{\isachardot}{\kern0pt}{\isachardot}{\kern0pt}\ {\isacharequal}{\kern0pt}\ \ {\isasymlangle}a{\isacharprime}{\kern0pt}{\isacharcomma}{\kern0pt}\ right{\isacharunderscore}{\kern0pt}coproj\ B\ C\ {\isasymcirc}\isactrlsub c\ c{\isacharprime}{\kern0pt}{\isasymrangle}{\isachardoublequoteclose}\isanewline
\ \ \ \ \ \ \ \ \ \ \ \ \isacommand{using}\isamarkupfalse%
\ a{\isacharprime}{\kern0pt}c{\isacharprime}{\kern0pt}{\isacharunderscore}{\kern0pt}def\ id{\isacharunderscore}{\kern0pt}left{\isacharunderscore}{\kern0pt}unit{\isadigit{2}}\ \isacommand{by}\isamarkupfalse%
\ force\isanewline
\ \ \ \ \ \ \ \ \ \ \isacommand{then}\isamarkupfalse%
\ \isacommand{show}\isamarkupfalse%
\ {\isachardoublequoteopen}{\isasymlangle}a{\isacharcomma}{\kern0pt}\ right{\isacharunderscore}{\kern0pt}coproj\ B\ C\ {\isasymcirc}\isactrlsub c\ c{\isasymrangle}\ {\isacharequal}{\kern0pt}\ {\isasymlangle}a{\isacharprime}{\kern0pt}{\isacharcomma}{\kern0pt}\ right{\isacharunderscore}{\kern0pt}coproj\ B\ C\ {\isasymcirc}\isactrlsub c\ c{\isacharprime}{\kern0pt}{\isasymrangle}{\isachardoublequoteclose}\isanewline
\ \ \ \ \ \ \ \ \ \ \ \ \isacommand{by}\isamarkupfalse%
\ {\isacharparenleft}{\kern0pt}simp\ add{\isacharcolon}{\kern0pt}\ calculation{\isacharparenright}{\kern0pt}\isanewline
\ \ \ \ \ \ \ \ \isacommand{qed}\isamarkupfalse%
\ \ \ \ \ \isanewline
\ \ \ \ \ \ \ \ \isacommand{then}\isamarkupfalse%
\ \isacommand{have}\isamarkupfalse%
\ a{\isacharunderscore}{\kern0pt}equal{\isacharcolon}{\kern0pt}\ {\isachardoublequoteopen}a\ {\isacharequal}{\kern0pt}\ a{\isacharprime}{\kern0pt}\ {\isasymand}\ right{\isacharunderscore}{\kern0pt}coproj\ B\ C\ {\isasymcirc}\isactrlsub c\ c\ {\isacharequal}{\kern0pt}\ right{\isacharunderscore}{\kern0pt}coproj\ B\ C\ {\isasymcirc}\isactrlsub c\ c{\isacharprime}{\kern0pt}{\isachardoublequoteclose}\isanewline
\ \ \ \ \ \ \ \ \ \ \isacommand{using}\isamarkupfalse%
\ a{\isacharprime}{\kern0pt}c{\isacharprime}{\kern0pt}{\isacharunderscore}{\kern0pt}def\ ac{\isacharunderscore}{\kern0pt}def\ element{\isacharunderscore}{\kern0pt}pair{\isacharunderscore}{\kern0pt}eq\ equal{\isacharunderscore}{\kern0pt}pair\ \isacommand{by}\isamarkupfalse%
\ {\isacharparenleft}{\kern0pt}typecheck{\isacharunderscore}{\kern0pt}cfuncs{\isacharcomma}{\kern0pt}\ blast{\isacharparenright}{\kern0pt}\isanewline
\ \ \ \ \ \ \ \ \isacommand{then}\isamarkupfalse%
\ \isacommand{have}\isamarkupfalse%
\ c{\isacharunderscore}{\kern0pt}equal{\isacharcolon}{\kern0pt}\ {\isachardoublequoteopen}c\ {\isacharequal}{\kern0pt}\ c{\isacharprime}{\kern0pt}{\isachardoublequoteclose}\ \isanewline
\ \ \ \ \ \ \ \ \ \ \isacommand{using}\isamarkupfalse%
\ a{\isacharprime}{\kern0pt}c{\isacharprime}{\kern0pt}{\isacharunderscore}{\kern0pt}def\ a{\isacharunderscore}{\kern0pt}equal\ ac{\isacharunderscore}{\kern0pt}def\ right{\isacharunderscore}{\kern0pt}coproj{\isacharunderscore}{\kern0pt}are{\isacharunderscore}{\kern0pt}monomorphisms\ right{\isacharunderscore}{\kern0pt}proj{\isacharunderscore}{\kern0pt}type\ monomorphism{\isacharunderscore}{\kern0pt}def{\isadigit{3}}\ \isacommand{by}\isamarkupfalse%
\ blast\isanewline
\ \ \ \ \ \ \ \ \isacommand{then}\isamarkupfalse%
\ \isacommand{show}\isamarkupfalse%
\ {\isachardoublequoteopen}x\ {\isacharequal}{\kern0pt}\ y{\isachardoublequoteclose}\isanewline
\ \ \ \ \ \ \ \ \ \ \isacommand{by}\isamarkupfalse%
\ {\isacharparenleft}{\kern0pt}simp\ add{\isacharcolon}{\kern0pt}\ a{\isacharprime}{\kern0pt}c{\isacharprime}{\kern0pt}{\isacharunderscore}{\kern0pt}def\ a{\isacharunderscore}{\kern0pt}equal\ ac{\isacharunderscore}{\kern0pt}def\ x{\isacharprime}{\kern0pt}{\isacharunderscore}{\kern0pt}def\ y{\isacharprime}{\kern0pt}{\isacharunderscore}{\kern0pt}def{\isacharparenright}{\kern0pt}\isanewline
\ \ \ \ \ \ \isacommand{qed}\isamarkupfalse%
\isanewline
\ \ \ \ \isacommand{qed}\isamarkupfalse%
\isanewline
\ \ \isacommand{qed}\isamarkupfalse%
\isanewline
\ \ \isacommand{then}\isamarkupfalse%
\ \isacommand{show}\isamarkupfalse%
\ {\isachardoublequoteopen}monomorphism\ {\isacharparenleft}{\kern0pt}dist{\isacharunderscore}{\kern0pt}prod{\isacharunderscore}{\kern0pt}coprod\ A\ B\ C{\isacharparenright}{\kern0pt}{\isachardoublequoteclose}\isanewline
\ \ \ \ \isacommand{using}\isamarkupfalse%
\ {\isasymphi}{\isacharunderscore}{\kern0pt}def\ dist{\isacharunderscore}{\kern0pt}prod{\isacharunderscore}{\kern0pt}coprod{\isacharunderscore}{\kern0pt}def\ injective{\isacharunderscore}{\kern0pt}imp{\isacharunderscore}{\kern0pt}monomorphism\ \isacommand{by}\isamarkupfalse%
\ fastforce\isanewline
\isacommand{qed}\isamarkupfalse%
%
\endisatagproof
{\isafoldproof}%
%
\isadelimproof
\isanewline
%
\endisadelimproof
\isanewline
\isacommand{lemma}\isamarkupfalse%
\ dist{\isacharunderscore}{\kern0pt}prod{\isacharunderscore}{\kern0pt}coprod{\isacharunderscore}{\kern0pt}epi{\isacharcolon}{\kern0pt}\isanewline
\ \ {\isachardoublequoteopen}epimorphism\ {\isacharparenleft}{\kern0pt}dist{\isacharunderscore}{\kern0pt}prod{\isacharunderscore}{\kern0pt}coprod\ A\ B\ C{\isacharparenright}{\kern0pt}{\isachardoublequoteclose}\isanewline
%
\isadelimproof
%
\endisadelimproof
%
\isatagproof
\isacommand{proof}\isamarkupfalse%
\ {\isacharminus}{\kern0pt}\isanewline
\ \ \isacommand{obtain}\isamarkupfalse%
\ {\isasymphi}\ \isakeyword{where}\ {\isasymphi}{\isacharunderscore}{\kern0pt}def{\isacharcolon}{\kern0pt}\ {\isachardoublequoteopen}{\isasymphi}\ {\isacharequal}{\kern0pt}\ {\isacharparenleft}{\kern0pt}id\ A\ {\isasymtimes}\isactrlsub f\ left{\isacharunderscore}{\kern0pt}coproj\ B\ C{\isacharparenright}{\kern0pt}\ {\isasymamalg}\ {\isacharparenleft}{\kern0pt}id\ A\ {\isasymtimes}\isactrlsub f\ right{\isacharunderscore}{\kern0pt}coproj\ B\ C{\isacharparenright}{\kern0pt}{\isachardoublequoteclose}\ \isakeyword{and}\isanewline
\ \ \ \ \ \ \ \ \ \ \ \ \ \ \ \ \ {\isasymphi}{\isacharunderscore}{\kern0pt}type{\isacharbrackleft}{\kern0pt}type{\isacharunderscore}{\kern0pt}rule{\isacharbrackright}{\kern0pt}{\isacharcolon}{\kern0pt}\ {\isachardoublequoteopen}{\isasymphi}\ {\isacharcolon}{\kern0pt}\ {\isacharparenleft}{\kern0pt}A\ {\isasymtimes}\isactrlsub c\ B{\isacharparenright}{\kern0pt}\ {\isasymCoprod}\ {\isacharparenleft}{\kern0pt}A\ {\isasymtimes}\isactrlsub c\ C{\isacharparenright}{\kern0pt}\ {\isasymrightarrow}\ A\ {\isasymtimes}\isactrlsub c\ {\isacharparenleft}{\kern0pt}B\ {\isasymCoprod}\ C{\isacharparenright}{\kern0pt}{\isachardoublequoteclose}\isanewline
\ \ \ \ \isacommand{by}\isamarkupfalse%
\ typecheck{\isacharunderscore}{\kern0pt}cfuncs\isanewline
\ \ \isacommand{have}\isamarkupfalse%
\ surjective{\isacharcolon}{\kern0pt}\ {\isachardoublequoteopen}surjective{\isacharparenleft}{\kern0pt}{\isacharparenleft}{\kern0pt}id\ A\ {\isasymtimes}\isactrlsub f\ left{\isacharunderscore}{\kern0pt}coproj\ B\ C{\isacharparenright}{\kern0pt}\ {\isasymamalg}\ {\isacharparenleft}{\kern0pt}id\ A\ {\isasymtimes}\isactrlsub f\ right{\isacharunderscore}{\kern0pt}coproj\ B\ C{\isacharparenright}{\kern0pt}{\isacharparenright}{\kern0pt}{\isachardoublequoteclose}\isanewline
\ \ \ \ \isacommand{unfolding}\isamarkupfalse%
\ surjective{\isacharunderscore}{\kern0pt}def\isanewline
\ \ \isacommand{proof}\isamarkupfalse%
{\isacharparenleft}{\kern0pt}auto{\isacharparenright}{\kern0pt}\isanewline
\ \ \ \ \isacommand{fix}\isamarkupfalse%
\ y\ \isanewline
\ \ \ \ \isacommand{assume}\isamarkupfalse%
\ y{\isacharunderscore}{\kern0pt}type{\isacharcolon}{\kern0pt}\ {\isachardoublequoteopen}y\ {\isasymin}\isactrlsub c\ codomain\ {\isacharparenleft}{\kern0pt}{\isacharparenleft}{\kern0pt}id\isactrlsub c\ A\ {\isasymtimes}\isactrlsub f\ left{\isacharunderscore}{\kern0pt}coproj\ B\ C{\isacharparenright}{\kern0pt}\ {\isasymamalg}\ {\isacharparenleft}{\kern0pt}id\isactrlsub c\ A\ {\isasymtimes}\isactrlsub f\ right{\isacharunderscore}{\kern0pt}coproj\ B\ C{\isacharparenright}{\kern0pt}{\isacharparenright}{\kern0pt}{\isachardoublequoteclose}\isanewline
\ \ \ \ \isacommand{then}\isamarkupfalse%
\ \isacommand{have}\isamarkupfalse%
\ y{\isacharunderscore}{\kern0pt}type{\isadigit{2}}{\isacharcolon}{\kern0pt}\ {\isachardoublequoteopen}y\ {\isasymin}\isactrlsub c\ A\ {\isasymtimes}\isactrlsub c\ {\isacharparenleft}{\kern0pt}B\ {\isasymCoprod}\ C{\isacharparenright}{\kern0pt}{\isachardoublequoteclose}\isanewline
\ \ \ \ \ \ \isacommand{using}\isamarkupfalse%
\ {\isasymphi}{\isacharunderscore}{\kern0pt}def\ {\isasymphi}{\isacharunderscore}{\kern0pt}type\ cfunc{\isacharunderscore}{\kern0pt}type{\isacharunderscore}{\kern0pt}def\ \isacommand{by}\isamarkupfalse%
\ auto\isanewline
\ \ \ \ \isacommand{then}\isamarkupfalse%
\ \isacommand{obtain}\isamarkupfalse%
\ a\ \isakeyword{where}\ a{\isacharunderscore}{\kern0pt}def{\isacharcolon}{\kern0pt}\ {\isachardoublequoteopen}{\isasymexists}\ bc{\isachardot}{\kern0pt}\ a\ {\isasymin}\isactrlsub c\ A\ {\isasymand}\ bc\ {\isasymin}\isactrlsub c\ B\ {\isasymCoprod}\ C\ {\isasymand}\ y\ {\isacharequal}{\kern0pt}\ {\isasymlangle}a{\isacharcomma}{\kern0pt}bc{\isasymrangle}{\isachardoublequoteclose}\isanewline
\ \ \ \ \ \ \isacommand{by}\isamarkupfalse%
\ {\isacharparenleft}{\kern0pt}meson\ cart{\isacharunderscore}{\kern0pt}prod{\isacharunderscore}{\kern0pt}decomp{\isacharparenright}{\kern0pt}\isanewline
\ \ \ \ \isacommand{then}\isamarkupfalse%
\ \isacommand{obtain}\isamarkupfalse%
\ bc\ \isakeyword{where}\ bc{\isacharunderscore}{\kern0pt}def{\isacharcolon}{\kern0pt}\ {\isachardoublequoteopen}bc\ {\isasymin}\isactrlsub c\ {\isacharparenleft}{\kern0pt}B\ {\isasymCoprod}\ C{\isacharparenright}{\kern0pt}\ {\isasymand}\ y\ {\isacharequal}{\kern0pt}\ {\isasymlangle}a{\isacharcomma}{\kern0pt}bc{\isasymrangle}{\isachardoublequoteclose}\isanewline
\ \ \ \ \ \ \isacommand{by}\isamarkupfalse%
\ blast\isanewline
\ \ \ \ \isacommand{have}\isamarkupfalse%
\ bc{\isacharunderscore}{\kern0pt}form{\isacharcolon}{\kern0pt}\ {\isachardoublequoteopen}{\isacharparenleft}{\kern0pt}{\isasymexists}\ b{\isachardot}{\kern0pt}\ b\ {\isasymin}\isactrlsub c\ B\ {\isasymand}\ bc\ {\isacharequal}{\kern0pt}\ left{\isacharunderscore}{\kern0pt}coproj\ B\ C\ {\isasymcirc}\isactrlsub c\ b{\isacharparenright}{\kern0pt}\ {\isasymor}\ {\isacharparenleft}{\kern0pt}{\isasymexists}\ c{\isachardot}{\kern0pt}\ c\ {\isasymin}\isactrlsub c\ C\ {\isasymand}\ bc\ {\isacharequal}{\kern0pt}\ right{\isacharunderscore}{\kern0pt}coproj\ B\ C\ {\isasymcirc}\isactrlsub c\ c{\isacharparenright}{\kern0pt}{\isachardoublequoteclose}\isanewline
\ \ \ \ \ \ \isacommand{by}\isamarkupfalse%
\ {\isacharparenleft}{\kern0pt}simp\ add{\isacharcolon}{\kern0pt}\ bc{\isacharunderscore}{\kern0pt}def\ coprojs{\isacharunderscore}{\kern0pt}jointly{\isacharunderscore}{\kern0pt}surj{\isacharparenright}{\kern0pt}\isanewline
\ \ \ \ \isacommand{have}\isamarkupfalse%
\ domain{\isacharunderscore}{\kern0pt}is{\isacharcolon}{\kern0pt}\ {\isachardoublequoteopen}{\isacharparenleft}{\kern0pt}A\ {\isasymtimes}\isactrlsub c\ B{\isacharparenright}{\kern0pt}\ {\isasymCoprod}\ {\isacharparenleft}{\kern0pt}A\ {\isasymtimes}\isactrlsub c\ C{\isacharparenright}{\kern0pt}\ {\isacharequal}{\kern0pt}\ domain\ {\isacharparenleft}{\kern0pt}{\isacharparenleft}{\kern0pt}id\isactrlsub c\ A\ {\isasymtimes}\isactrlsub f\ left{\isacharunderscore}{\kern0pt}coproj\ B\ C{\isacharparenright}{\kern0pt}\ {\isasymamalg}\ {\isacharparenleft}{\kern0pt}id\isactrlsub c\ A\ {\isasymtimes}\isactrlsub f\ right{\isacharunderscore}{\kern0pt}coproj\ B\ C{\isacharparenright}{\kern0pt}{\isacharparenright}{\kern0pt}{\isachardoublequoteclose}\isanewline
\ \ \ \ \ \ \isacommand{by}\isamarkupfalse%
\ {\isacharparenleft}{\kern0pt}typecheck{\isacharunderscore}{\kern0pt}cfuncs{\isacharcomma}{\kern0pt}\ simp\ add{\isacharcolon}{\kern0pt}\ cfunc{\isacharunderscore}{\kern0pt}type{\isacharunderscore}{\kern0pt}def{\isacharparenright}{\kern0pt}\isanewline
\ \ \ \ \isacommand{show}\isamarkupfalse%
\ {\isachardoublequoteopen}{\isasymexists}x{\isachardot}{\kern0pt}\ x\ {\isasymin}\isactrlsub c\ domain\ {\isacharparenleft}{\kern0pt}{\isacharparenleft}{\kern0pt}id\isactrlsub c\ A\ {\isasymtimes}\isactrlsub f\ left{\isacharunderscore}{\kern0pt}coproj\ B\ C{\isacharparenright}{\kern0pt}\ {\isasymamalg}\ {\isacharparenleft}{\kern0pt}id\isactrlsub c\ A\ {\isasymtimes}\isactrlsub f\ right{\isacharunderscore}{\kern0pt}coproj\ B\ C{\isacharparenright}{\kern0pt}{\isacharparenright}{\kern0pt}\ {\isasymand}\isanewline
\ \ \ \ \ \ \ \ \ \ \ \ \ {\isacharparenleft}{\kern0pt}id\isactrlsub c\ A\ {\isasymtimes}\isactrlsub f\ left{\isacharunderscore}{\kern0pt}coproj\ B\ C{\isacharparenright}{\kern0pt}\ {\isasymamalg}\ {\isacharparenleft}{\kern0pt}id\isactrlsub c\ A\ {\isasymtimes}\isactrlsub f\ right{\isacharunderscore}{\kern0pt}coproj\ B\ C{\isacharparenright}{\kern0pt}\ {\isasymcirc}\isactrlsub c\ x\ {\isacharequal}{\kern0pt}\ y{\isachardoublequoteclose}\isanewline
\ \ \ \ \isacommand{proof}\isamarkupfalse%
{\isacharparenleft}{\kern0pt}cases\ {\isachardoublequoteopen}{\isasymexists}\ b{\isachardot}{\kern0pt}\ b\ {\isasymin}\isactrlsub c\ B\ {\isasymand}\ bc\ {\isacharequal}{\kern0pt}\ left{\isacharunderscore}{\kern0pt}coproj\ B\ C\ {\isasymcirc}\isactrlsub c\ b{\isachardoublequoteclose}{\isacharparenright}{\kern0pt}\isanewline
\ \ \ \ \ \ \isacommand{assume}\isamarkupfalse%
\ case{\isadigit{1}}{\isacharcolon}{\kern0pt}\ {\isachardoublequoteopen}{\isasymexists}b{\isachardot}{\kern0pt}\ b\ {\isasymin}\isactrlsub c\ B\ {\isasymand}\ bc\ {\isacharequal}{\kern0pt}\ left{\isacharunderscore}{\kern0pt}coproj\ B\ C\ {\isasymcirc}\isactrlsub c\ b{\isachardoublequoteclose}\isanewline
\ \ \ \ \ \ \isacommand{then}\isamarkupfalse%
\ \isacommand{obtain}\isamarkupfalse%
\ b\ \isakeyword{where}\ b{\isacharunderscore}{\kern0pt}def{\isacharcolon}{\kern0pt}\ {\isachardoublequoteopen}b\ {\isasymin}\isactrlsub c\ B\ {\isasymand}\ bc\ {\isacharequal}{\kern0pt}\ left{\isacharunderscore}{\kern0pt}coproj\ B\ C\ {\isasymcirc}\isactrlsub c\ b{\isachardoublequoteclose}\isanewline
\ \ \ \ \ \ \ \ \isacommand{by}\isamarkupfalse%
\ blast\isanewline
\ \ \ \ \ \ \isacommand{then}\isamarkupfalse%
\ \isacommand{have}\isamarkupfalse%
\ ab{\isacharunderscore}{\kern0pt}type{\isacharcolon}{\kern0pt}\ {\isachardoublequoteopen}{\isasymlangle}a{\isacharcomma}{\kern0pt}\ b{\isasymrangle}\ {\isasymin}\isactrlsub c\ {\isacharparenleft}{\kern0pt}A\ {\isasymtimes}\isactrlsub c\ B{\isacharparenright}{\kern0pt}{\isachardoublequoteclose}\isanewline
\ \ \ \ \ \ \ \ \isacommand{using}\isamarkupfalse%
\ a{\isacharunderscore}{\kern0pt}def\ b{\isacharunderscore}{\kern0pt}def\ \isacommand{by}\isamarkupfalse%
\ {\isacharparenleft}{\kern0pt}typecheck{\isacharunderscore}{\kern0pt}cfuncs{\isacharcomma}{\kern0pt}\ blast{\isacharparenright}{\kern0pt}\isanewline
\ \ \ \ \ \ \isacommand{obtain}\isamarkupfalse%
\ x\ \isakeyword{where}\ x{\isacharunderscore}{\kern0pt}def{\isacharcolon}{\kern0pt}\ {\isachardoublequoteopen}x\ {\isacharequal}{\kern0pt}\ left{\isacharunderscore}{\kern0pt}coproj\ {\isacharparenleft}{\kern0pt}A\ {\isasymtimes}\isactrlsub c\ B{\isacharparenright}{\kern0pt}\ {\isacharparenleft}{\kern0pt}A\ {\isasymtimes}\isactrlsub c\ C{\isacharparenright}{\kern0pt}\ {\isasymcirc}\isactrlsub c\ {\isasymlangle}a{\isacharcomma}{\kern0pt}\ b{\isasymrangle}{\isachardoublequoteclose}\isanewline
\ \ \ \ \ \ \ \ \isacommand{by}\isamarkupfalse%
\ simp\isanewline
\ \ \ \ \ \ \isacommand{have}\isamarkupfalse%
\ x{\isacharunderscore}{\kern0pt}type{\isacharcolon}{\kern0pt}\ {\isachardoublequoteopen}x\ {\isasymin}\isactrlsub c\ domain\ {\isacharparenleft}{\kern0pt}{\isacharparenleft}{\kern0pt}id\isactrlsub c\ A\ {\isasymtimes}\isactrlsub f\ left{\isacharunderscore}{\kern0pt}coproj\ B\ C{\isacharparenright}{\kern0pt}\ {\isasymamalg}\ {\isacharparenleft}{\kern0pt}id\isactrlsub c\ A\ {\isasymtimes}\isactrlsub f\ right{\isacharunderscore}{\kern0pt}coproj\ B\ C{\isacharparenright}{\kern0pt}{\isacharparenright}{\kern0pt}{\isachardoublequoteclose}\isanewline
\ \ \ \ \ \ \ \ \isacommand{using}\isamarkupfalse%
\ ab{\isacharunderscore}{\kern0pt}type\ cfunc{\isacharunderscore}{\kern0pt}type{\isacharunderscore}{\kern0pt}def\ codomain{\isacharunderscore}{\kern0pt}comp\ domain{\isacharunderscore}{\kern0pt}comp\ domain{\isacharunderscore}{\kern0pt}is\ left{\isacharunderscore}{\kern0pt}proj{\isacharunderscore}{\kern0pt}type\ x{\isacharunderscore}{\kern0pt}def\ \isacommand{by}\isamarkupfalse%
\ auto\isanewline
\ \ \ \ \ \ \isacommand{have}\isamarkupfalse%
\ y{\isacharunderscore}{\kern0pt}def{\isadigit{2}}{\isacharcolon}{\kern0pt}\ {\isachardoublequoteopen}y\ {\isacharequal}{\kern0pt}\ {\isasymlangle}a{\isacharcomma}{\kern0pt}left{\isacharunderscore}{\kern0pt}coproj\ B\ C\ {\isasymcirc}\isactrlsub c\ b{\isasymrangle}{\isachardoublequoteclose}\isanewline
\ \ \ \ \ \ \ \ \isacommand{by}\isamarkupfalse%
\ {\isacharparenleft}{\kern0pt}simp\ add{\isacharcolon}{\kern0pt}\ b{\isacharunderscore}{\kern0pt}def\ bc{\isacharunderscore}{\kern0pt}def{\isacharparenright}{\kern0pt}\isanewline
\ \ \ \ \ \ \isacommand{have}\isamarkupfalse%
\ {\isachardoublequoteopen}y\ {\isacharequal}{\kern0pt}\ {\isacharparenleft}{\kern0pt}id{\isacharparenleft}{\kern0pt}A{\isacharparenright}{\kern0pt}\ {\isasymtimes}\isactrlsub f\ left{\isacharunderscore}{\kern0pt}coproj\ B\ C{\isacharparenright}{\kern0pt}\ {\isasymcirc}\isactrlsub c\ {\isasymlangle}a{\isacharcomma}{\kern0pt}b{\isasymrangle}{\isachardoublequoteclose}\isanewline
\ \ \ \ \ \ \ \ \isacommand{using}\isamarkupfalse%
\ a{\isacharunderscore}{\kern0pt}def\ b{\isacharunderscore}{\kern0pt}def\ cfunc{\isacharunderscore}{\kern0pt}cross{\isacharunderscore}{\kern0pt}prod{\isacharunderscore}{\kern0pt}comp{\isacharunderscore}{\kern0pt}cfunc{\isacharunderscore}{\kern0pt}prod\ id{\isacharunderscore}{\kern0pt}left{\isacharunderscore}{\kern0pt}unit{\isadigit{2}}\ y{\isacharunderscore}{\kern0pt}def{\isadigit{2}}\ \isacommand{by}\isamarkupfalse%
\ {\isacharparenleft}{\kern0pt}typecheck{\isacharunderscore}{\kern0pt}cfuncs{\isacharcomma}{\kern0pt}\ auto{\isacharparenright}{\kern0pt}\isanewline
\ \ \ \ \ \ \isacommand{also}\isamarkupfalse%
\ \isacommand{have}\isamarkupfalse%
\ {\isachardoublequoteopen}{\isachardot}{\kern0pt}{\isachardot}{\kern0pt}{\isachardot}{\kern0pt}\ {\isacharequal}{\kern0pt}\ {\isacharparenleft}{\kern0pt}{\isasymphi}\ {\isasymcirc}\isactrlsub c\ left{\isacharunderscore}{\kern0pt}coproj\ {\isacharparenleft}{\kern0pt}A\ {\isasymtimes}\isactrlsub c\ B{\isacharparenright}{\kern0pt}\ {\isacharparenleft}{\kern0pt}A\ {\isasymtimes}\isactrlsub c\ C{\isacharparenright}{\kern0pt}{\isacharparenright}{\kern0pt}\ {\isasymcirc}\isactrlsub c\ {\isasymlangle}a{\isacharcomma}{\kern0pt}\ b{\isasymrangle}{\isachardoublequoteclose}\isanewline
\ \ \ \ \ \ \ \ \isacommand{unfolding}\isamarkupfalse%
\ {\isasymphi}{\isacharunderscore}{\kern0pt}def\ \isacommand{by}\isamarkupfalse%
\ {\isacharparenleft}{\kern0pt}typecheck{\isacharunderscore}{\kern0pt}cfuncs{\isacharcomma}{\kern0pt}\ simp\ add{\isacharcolon}{\kern0pt}\ left{\isacharunderscore}{\kern0pt}coproj{\isacharunderscore}{\kern0pt}cfunc{\isacharunderscore}{\kern0pt}coprod{\isacharparenright}{\kern0pt}\isanewline
\ \ \ \ \ \ \isacommand{also}\isamarkupfalse%
\ \isacommand{have}\isamarkupfalse%
\ {\isachardoublequoteopen}{\isachardot}{\kern0pt}{\isachardot}{\kern0pt}{\isachardot}{\kern0pt}\ {\isacharequal}{\kern0pt}\ {\isasymphi}\ {\isasymcirc}\isactrlsub c\ x{\isachardoublequoteclose}\isanewline
\ \ \ \ \ \ \ \ \isacommand{using}\isamarkupfalse%
\ {\isasymphi}{\isacharunderscore}{\kern0pt}type\ x{\isacharunderscore}{\kern0pt}def\ ab{\isacharunderscore}{\kern0pt}type\ comp{\isacharunderscore}{\kern0pt}associative{\isadigit{2}}\ \isacommand{by}\isamarkupfalse%
\ {\isacharparenleft}{\kern0pt}typecheck{\isacharunderscore}{\kern0pt}cfuncs{\isacharcomma}{\kern0pt}\ auto{\isacharparenright}{\kern0pt}\isanewline
\ \ \ \ \ \ \isacommand{then}\isamarkupfalse%
\ \isacommand{show}\isamarkupfalse%
\ {\isachardoublequoteopen}{\isasymexists}x{\isachardot}{\kern0pt}\ x\ {\isasymin}\isactrlsub c\ domain\ {\isacharparenleft}{\kern0pt}{\isacharparenleft}{\kern0pt}id\isactrlsub c\ A\ {\isasymtimes}\isactrlsub f\ left{\isacharunderscore}{\kern0pt}coproj\ B\ C{\isacharparenright}{\kern0pt}\ {\isasymamalg}\ {\isacharparenleft}{\kern0pt}id\isactrlsub c\ A\ {\isasymtimes}\isactrlsub f\ right{\isacharunderscore}{\kern0pt}coproj\ B\ C{\isacharparenright}{\kern0pt}{\isacharparenright}{\kern0pt}\ {\isasymand}\isanewline
\ \ \ \ \ \ \ \ {\isacharparenleft}{\kern0pt}id\isactrlsub c\ A\ {\isasymtimes}\isactrlsub f\ left{\isacharunderscore}{\kern0pt}coproj\ B\ C{\isacharparenright}{\kern0pt}\ {\isasymamalg}\ {\isacharparenleft}{\kern0pt}id\isactrlsub c\ A\ {\isasymtimes}\isactrlsub f\ right{\isacharunderscore}{\kern0pt}coproj\ B\ C{\isacharparenright}{\kern0pt}\ {\isasymcirc}\isactrlsub c\ x\ {\isacharequal}{\kern0pt}\ y{\isachardoublequoteclose}\isanewline
\ \ \ \ \ \ \ \ \isacommand{using}\isamarkupfalse%
\ {\isasymphi}{\isacharunderscore}{\kern0pt}def\ calculation\ x{\isacharunderscore}{\kern0pt}type\ \isacommand{by}\isamarkupfalse%
\ auto\isanewline
\ \ \ \ \isacommand{next}\isamarkupfalse%
\isanewline
\ \ \ \ \ \ \isacommand{assume}\isamarkupfalse%
\ {\isachardoublequoteopen}{\isasymnexists}b{\isachardot}{\kern0pt}\ b\ {\isasymin}\isactrlsub c\ B\ {\isasymand}\ bc\ {\isacharequal}{\kern0pt}\ left{\isacharunderscore}{\kern0pt}coproj\ B\ C\ {\isasymcirc}\isactrlsub c\ b{\isachardoublequoteclose}\isanewline
\ \ \ \ \ \ \isacommand{then}\isamarkupfalse%
\ \isacommand{have}\isamarkupfalse%
\ case{\isadigit{2}}{\isacharcolon}{\kern0pt}\ {\isachardoublequoteopen}{\isasymexists}\ c{\isachardot}{\kern0pt}\ c\ {\isasymin}\isactrlsub c\ C\ {\isasymand}\ bc\ {\isacharequal}{\kern0pt}\ {\isacharparenleft}{\kern0pt}right{\isacharunderscore}{\kern0pt}coproj\ B\ C\ \ {\isasymcirc}\isactrlsub c\ c{\isacharparenright}{\kern0pt}{\isachardoublequoteclose}\isanewline
\ \ \ \ \ \ \ \ \isacommand{using}\isamarkupfalse%
\ bc{\isacharunderscore}{\kern0pt}form\ \isacommand{by}\isamarkupfalse%
\ blast\isanewline
\ \ \ \ \ \ \isacommand{then}\isamarkupfalse%
\ \isacommand{obtain}\isamarkupfalse%
\ c\ \isakeyword{where}\ c{\isacharunderscore}{\kern0pt}def{\isacharcolon}{\kern0pt}\ {\isachardoublequoteopen}c\ {\isasymin}\isactrlsub c\ C\ {\isasymand}\ bc\ {\isacharequal}{\kern0pt}\ right{\isacharunderscore}{\kern0pt}coproj\ B\ C\ \ {\isasymcirc}\isactrlsub c\ c{\isachardoublequoteclose}\isanewline
\ \ \ \ \ \ \ \ \isacommand{by}\isamarkupfalse%
\ blast\isanewline
\ \ \ \ \ \ \isacommand{then}\isamarkupfalse%
\ \isacommand{have}\isamarkupfalse%
\ ac{\isacharunderscore}{\kern0pt}type{\isacharcolon}{\kern0pt}\ {\isachardoublequoteopen}{\isasymlangle}a{\isacharcomma}{\kern0pt}\ c{\isasymrangle}\ {\isasymin}\isactrlsub c\ {\isacharparenleft}{\kern0pt}A\ {\isasymtimes}\isactrlsub c\ C{\isacharparenright}{\kern0pt}{\isachardoublequoteclose}\isanewline
\ \ \ \ \ \ \ \ \isacommand{using}\isamarkupfalse%
\ a{\isacharunderscore}{\kern0pt}def\ c{\isacharunderscore}{\kern0pt}def\ \isacommand{by}\isamarkupfalse%
\ {\isacharparenleft}{\kern0pt}typecheck{\isacharunderscore}{\kern0pt}cfuncs{\isacharcomma}{\kern0pt}\ blast{\isacharparenright}{\kern0pt}\isanewline
\ \ \ \ \ \ \isacommand{obtain}\isamarkupfalse%
\ x\ \isakeyword{where}\ x{\isacharunderscore}{\kern0pt}def{\isacharcolon}{\kern0pt}\ {\isachardoublequoteopen}x\ {\isacharequal}{\kern0pt}\ right{\isacharunderscore}{\kern0pt}coproj\ {\isacharparenleft}{\kern0pt}A\ {\isasymtimes}\isactrlsub c\ B{\isacharparenright}{\kern0pt}\ {\isacharparenleft}{\kern0pt}A\ {\isasymtimes}\isactrlsub c\ C{\isacharparenright}{\kern0pt}\ {\isasymcirc}\isactrlsub c\ {\isasymlangle}a{\isacharcomma}{\kern0pt}\ c{\isasymrangle}{\isachardoublequoteclose}\isanewline
\ \ \ \ \ \ \ \ \isacommand{by}\isamarkupfalse%
\ simp\isanewline
\ \ \ \ \ \ \isacommand{have}\isamarkupfalse%
\ x{\isacharunderscore}{\kern0pt}type{\isacharcolon}{\kern0pt}\ {\isachardoublequoteopen}x\ {\isasymin}\isactrlsub c\ domain\ {\isacharparenleft}{\kern0pt}{\isacharparenleft}{\kern0pt}id\isactrlsub c\ A\ {\isasymtimes}\isactrlsub f\ left{\isacharunderscore}{\kern0pt}coproj\ B\ C{\isacharparenright}{\kern0pt}\ {\isasymamalg}\ {\isacharparenleft}{\kern0pt}id\isactrlsub c\ A\ {\isasymtimes}\isactrlsub f\ right{\isacharunderscore}{\kern0pt}coproj\ B\ C{\isacharparenright}{\kern0pt}{\isacharparenright}{\kern0pt}{\isachardoublequoteclose}\isanewline
\ \ \ \ \ \ \ \ \isacommand{using}\isamarkupfalse%
\ ac{\isacharunderscore}{\kern0pt}type\ cfunc{\isacharunderscore}{\kern0pt}type{\isacharunderscore}{\kern0pt}def\ codomain{\isacharunderscore}{\kern0pt}comp\ domain{\isacharunderscore}{\kern0pt}comp\ domain{\isacharunderscore}{\kern0pt}is\ right{\isacharunderscore}{\kern0pt}proj{\isacharunderscore}{\kern0pt}type\ x{\isacharunderscore}{\kern0pt}def\ \isacommand{by}\isamarkupfalse%
\ auto\isanewline
\ \ \ \ \ \ \isacommand{have}\isamarkupfalse%
\ y{\isacharunderscore}{\kern0pt}def{\isadigit{2}}{\isacharcolon}{\kern0pt}\ {\isachardoublequoteopen}y\ {\isacharequal}{\kern0pt}\ {\isasymlangle}a{\isacharcomma}{\kern0pt}right{\isacharunderscore}{\kern0pt}coproj\ B\ C\ {\isasymcirc}\isactrlsub c\ c{\isasymrangle}{\isachardoublequoteclose}\isanewline
\ \ \ \ \ \ \ \ \isacommand{by}\isamarkupfalse%
\ {\isacharparenleft}{\kern0pt}simp\ add{\isacharcolon}{\kern0pt}\ c{\isacharunderscore}{\kern0pt}def\ bc{\isacharunderscore}{\kern0pt}def{\isacharparenright}{\kern0pt}\isanewline
\ \ \ \ \ \ \isacommand{have}\isamarkupfalse%
\ {\isachardoublequoteopen}y\ {\isacharequal}{\kern0pt}\ {\isacharparenleft}{\kern0pt}id{\isacharparenleft}{\kern0pt}A{\isacharparenright}{\kern0pt}\ {\isasymtimes}\isactrlsub f\ right{\isacharunderscore}{\kern0pt}coproj\ B\ C{\isacharparenright}{\kern0pt}\ {\isasymcirc}\isactrlsub c\ {\isasymlangle}a{\isacharcomma}{\kern0pt}c{\isasymrangle}{\isachardoublequoteclose}\isanewline
\ \ \ \ \ \ \ \ \isacommand{using}\isamarkupfalse%
\ a{\isacharunderscore}{\kern0pt}def\ c{\isacharunderscore}{\kern0pt}def\ cfunc{\isacharunderscore}{\kern0pt}cross{\isacharunderscore}{\kern0pt}prod{\isacharunderscore}{\kern0pt}comp{\isacharunderscore}{\kern0pt}cfunc{\isacharunderscore}{\kern0pt}prod\ id{\isacharunderscore}{\kern0pt}left{\isacharunderscore}{\kern0pt}unit{\isadigit{2}}\ y{\isacharunderscore}{\kern0pt}def{\isadigit{2}}\ \isacommand{by}\isamarkupfalse%
\ {\isacharparenleft}{\kern0pt}typecheck{\isacharunderscore}{\kern0pt}cfuncs{\isacharcomma}{\kern0pt}\ auto{\isacharparenright}{\kern0pt}\isanewline
\ \ \ \ \ \ \isacommand{also}\isamarkupfalse%
\ \isacommand{have}\isamarkupfalse%
\ {\isachardoublequoteopen}{\isachardot}{\kern0pt}{\isachardot}{\kern0pt}{\isachardot}{\kern0pt}\ {\isacharequal}{\kern0pt}\ {\isacharparenleft}{\kern0pt}{\isasymphi}\ {\isasymcirc}\isactrlsub c\ right{\isacharunderscore}{\kern0pt}coproj\ {\isacharparenleft}{\kern0pt}A\ {\isasymtimes}\isactrlsub c\ B{\isacharparenright}{\kern0pt}\ {\isacharparenleft}{\kern0pt}A\ {\isasymtimes}\isactrlsub c\ C{\isacharparenright}{\kern0pt}{\isacharparenright}{\kern0pt}\ {\isasymcirc}\isactrlsub c\ {\isasymlangle}a{\isacharcomma}{\kern0pt}\ c{\isasymrangle}{\isachardoublequoteclose}\isanewline
\ \ \ \ \ \ \ \ \isacommand{unfolding}\isamarkupfalse%
\ {\isasymphi}{\isacharunderscore}{\kern0pt}def\ \isacommand{using}\isamarkupfalse%
\ right{\isacharunderscore}{\kern0pt}coproj{\isacharunderscore}{\kern0pt}cfunc{\isacharunderscore}{\kern0pt}coprod\ \isacommand{by}\isamarkupfalse%
\ {\isacharparenleft}{\kern0pt}typecheck{\isacharunderscore}{\kern0pt}cfuncs{\isacharcomma}{\kern0pt}\ auto{\isacharparenright}{\kern0pt}\isanewline
\ \ \ \ \ \ \isacommand{also}\isamarkupfalse%
\ \isacommand{have}\isamarkupfalse%
\ {\isachardoublequoteopen}{\isachardot}{\kern0pt}{\isachardot}{\kern0pt}{\isachardot}{\kern0pt}\ {\isacharequal}{\kern0pt}\ {\isasymphi}\ {\isasymcirc}\isactrlsub c\ x{\isachardoublequoteclose}\isanewline
\ \ \ \ \ \ \ \ \isacommand{using}\isamarkupfalse%
\ {\isasymphi}{\isacharunderscore}{\kern0pt}type\ x{\isacharunderscore}{\kern0pt}def\ ac{\isacharunderscore}{\kern0pt}type\ comp{\isacharunderscore}{\kern0pt}associative{\isadigit{2}}\ \isacommand{by}\isamarkupfalse%
\ {\isacharparenleft}{\kern0pt}typecheck{\isacharunderscore}{\kern0pt}cfuncs{\isacharcomma}{\kern0pt}\ auto{\isacharparenright}{\kern0pt}\isanewline
\ \ \ \ \ \ \isacommand{then}\isamarkupfalse%
\ \isacommand{show}\isamarkupfalse%
\ {\isachardoublequoteopen}{\isasymexists}x{\isachardot}{\kern0pt}\ x\ {\isasymin}\isactrlsub c\ domain\ {\isacharparenleft}{\kern0pt}{\isacharparenleft}{\kern0pt}id\isactrlsub c\ A\ {\isasymtimes}\isactrlsub f\ left{\isacharunderscore}{\kern0pt}coproj\ B\ C{\isacharparenright}{\kern0pt}\ {\isasymamalg}\ {\isacharparenleft}{\kern0pt}id\isactrlsub c\ A\ {\isasymtimes}\isactrlsub f\ right{\isacharunderscore}{\kern0pt}coproj\ B\ C{\isacharparenright}{\kern0pt}{\isacharparenright}{\kern0pt}\ {\isasymand}\isanewline
\ \ \ \ \ \ \ \ {\isacharparenleft}{\kern0pt}id\isactrlsub c\ A\ {\isasymtimes}\isactrlsub f\ left{\isacharunderscore}{\kern0pt}coproj\ B\ C{\isacharparenright}{\kern0pt}\ {\isasymamalg}\ {\isacharparenleft}{\kern0pt}id\isactrlsub c\ A\ {\isasymtimes}\isactrlsub f\ right{\isacharunderscore}{\kern0pt}coproj\ B\ C{\isacharparenright}{\kern0pt}\ {\isasymcirc}\isactrlsub c\ x\ {\isacharequal}{\kern0pt}\ y{\isachardoublequoteclose}\isanewline
\ \ \ \ \ \ \ \ \isacommand{using}\isamarkupfalse%
\ {\isasymphi}{\isacharunderscore}{\kern0pt}def\ calculation\ x{\isacharunderscore}{\kern0pt}type\ \isacommand{by}\isamarkupfalse%
\ auto\isanewline
\ \ \ \ \isacommand{qed}\isamarkupfalse%
\isanewline
\ \ \isacommand{qed}\isamarkupfalse%
\isanewline
\ \ \isacommand{then}\isamarkupfalse%
\ \isacommand{show}\isamarkupfalse%
\ {\isachardoublequoteopen}epimorphism\ {\isacharparenleft}{\kern0pt}dist{\isacharunderscore}{\kern0pt}prod{\isacharunderscore}{\kern0pt}coprod\ A\ B\ C{\isacharparenright}{\kern0pt}{\isachardoublequoteclose}\isanewline
\ \ \ \ \isacommand{by}\isamarkupfalse%
\ {\isacharparenleft}{\kern0pt}simp\ add{\isacharcolon}{\kern0pt}\ dist{\isacharunderscore}{\kern0pt}prod{\isacharunderscore}{\kern0pt}coprod{\isacharunderscore}{\kern0pt}def\ surjective{\isacharunderscore}{\kern0pt}is{\isacharunderscore}{\kern0pt}epimorphism{\isacharparenright}{\kern0pt}\isanewline
\isacommand{qed}\isamarkupfalse%
%
\endisatagproof
{\isafoldproof}%
%
\isadelimproof
\isanewline
%
\endisadelimproof
\isanewline
\isacommand{lemma}\isamarkupfalse%
\ dist{\isacharunderscore}{\kern0pt}prod{\isacharunderscore}{\kern0pt}coprod{\isacharunderscore}{\kern0pt}iso{\isacharcolon}{\kern0pt}\isanewline
\ \ {\isachardoublequoteopen}isomorphism{\isacharparenleft}{\kern0pt}dist{\isacharunderscore}{\kern0pt}prod{\isacharunderscore}{\kern0pt}coprod\ A\ B\ C{\isacharparenright}{\kern0pt}{\isachardoublequoteclose}\isanewline
%
\isadelimproof
\ \ %
\endisadelimproof
%
\isatagproof
\isacommand{by}\isamarkupfalse%
\ {\isacharparenleft}{\kern0pt}simp\ add{\isacharcolon}{\kern0pt}\ dist{\isacharunderscore}{\kern0pt}prod{\isacharunderscore}{\kern0pt}coprod{\isacharunderscore}{\kern0pt}epi\ dist{\isacharunderscore}{\kern0pt}prod{\isacharunderscore}{\kern0pt}coprod{\isacharunderscore}{\kern0pt}mono\ epi{\isacharunderscore}{\kern0pt}mon{\isacharunderscore}{\kern0pt}is{\isacharunderscore}{\kern0pt}iso{\isacharparenright}{\kern0pt}%
\endisatagproof
{\isafoldproof}%
%
\isadelimproof
%
\endisadelimproof
%
\begin{isamarkuptext}%
The lemma below corresponds to Proposition 2.5.10 in Halvorson.%
\end{isamarkuptext}\isamarkuptrue%
\isacommand{lemma}\isamarkupfalse%
\ prod{\isacharunderscore}{\kern0pt}distribute{\isacharunderscore}{\kern0pt}coprod{\isacharcolon}{\kern0pt}\isanewline
\ \ {\isachardoublequoteopen}A\ {\isasymtimes}\isactrlsub c\ {\isacharparenleft}{\kern0pt}X\ {\isasymCoprod}\ Y{\isacharparenright}{\kern0pt}\ {\isasymcong}\ {\isacharparenleft}{\kern0pt}A\ {\isasymtimes}\isactrlsub c\ X{\isacharparenright}{\kern0pt}\ {\isasymCoprod}\ {\isacharparenleft}{\kern0pt}A\ {\isasymtimes}\isactrlsub c\ Y{\isacharparenright}{\kern0pt}{\isachardoublequoteclose}\isanewline
%
\isadelimproof
\ \ %
\endisadelimproof
%
\isatagproof
\isacommand{using}\isamarkupfalse%
\ dist{\isacharunderscore}{\kern0pt}prod{\isacharunderscore}{\kern0pt}coprod{\isacharunderscore}{\kern0pt}iso\ dist{\isacharunderscore}{\kern0pt}prod{\isacharunderscore}{\kern0pt}coprod{\isacharunderscore}{\kern0pt}type\ is{\isacharunderscore}{\kern0pt}isomorphic{\isacharunderscore}{\kern0pt}def\ isomorphic{\isacharunderscore}{\kern0pt}is{\isacharunderscore}{\kern0pt}symmetric\ \isacommand{by}\isamarkupfalse%
\ blast%
\endisatagproof
{\isafoldproof}%
%
\isadelimproof
%
\endisadelimproof
%
\isadelimdocument
%
\endisadelimdocument
%
\isatagdocument
%
\isamarkupsubsubsection{Inverse Distribute Product Over Coproduct Auxillary Mapping%
}
\isamarkuptrue%
%
\endisatagdocument
{\isafolddocument}%
%
\isadelimdocument
%
\endisadelimdocument
\isacommand{definition}\isamarkupfalse%
\ dist{\isacharunderscore}{\kern0pt}prod{\isacharunderscore}{\kern0pt}coprod{\isacharunderscore}{\kern0pt}inv\ {\isacharcolon}{\kern0pt}{\isacharcolon}{\kern0pt}\ {\isachardoublequoteopen}cset\ {\isasymRightarrow}\ cset\ {\isasymRightarrow}\ cset\ {\isasymRightarrow}\ cfunc{\isachardoublequoteclose}\ \isakeyword{where}\isanewline
\ \ {\isachardoublequoteopen}dist{\isacharunderscore}{\kern0pt}prod{\isacharunderscore}{\kern0pt}coprod{\isacharunderscore}{\kern0pt}inv\ A\ B\ C\ {\isacharequal}{\kern0pt}\ {\isacharparenleft}{\kern0pt}THE\ f{\isachardot}{\kern0pt}\ f\ {\isacharcolon}{\kern0pt}\ A\ {\isasymtimes}\isactrlsub c\ {\isacharparenleft}{\kern0pt}B\ {\isasymCoprod}\ C{\isacharparenright}{\kern0pt}\ {\isasymrightarrow}\ {\isacharparenleft}{\kern0pt}A\ {\isasymtimes}\isactrlsub c\ B{\isacharparenright}{\kern0pt}\ {\isasymCoprod}\ {\isacharparenleft}{\kern0pt}A\ {\isasymtimes}\isactrlsub c\ C{\isacharparenright}{\kern0pt}\isanewline
\ \ \ \ {\isasymand}\ f\ {\isasymcirc}\isactrlsub c\ dist{\isacharunderscore}{\kern0pt}prod{\isacharunderscore}{\kern0pt}coprod\ A\ B\ C\ {\isacharequal}{\kern0pt}\ id\ {\isacharparenleft}{\kern0pt}{\isacharparenleft}{\kern0pt}A\ {\isasymtimes}\isactrlsub c\ B{\isacharparenright}{\kern0pt}\ {\isasymCoprod}\ {\isacharparenleft}{\kern0pt}A\ {\isasymtimes}\isactrlsub c\ C{\isacharparenright}{\kern0pt}{\isacharparenright}{\kern0pt}\isanewline
\ \ \ \ {\isasymand}\ dist{\isacharunderscore}{\kern0pt}prod{\isacharunderscore}{\kern0pt}coprod\ A\ B\ C\ {\isasymcirc}\isactrlsub c\ f\ {\isacharequal}{\kern0pt}\ id\ {\isacharparenleft}{\kern0pt}A\ {\isasymtimes}\isactrlsub c\ {\isacharparenleft}{\kern0pt}B\ {\isasymCoprod}\ C{\isacharparenright}{\kern0pt}{\isacharparenright}{\kern0pt}{\isacharparenright}{\kern0pt}{\isachardoublequoteclose}\isanewline
\isanewline
\isacommand{lemma}\isamarkupfalse%
\ dist{\isacharunderscore}{\kern0pt}prod{\isacharunderscore}{\kern0pt}coprod{\isacharunderscore}{\kern0pt}inv{\isacharunderscore}{\kern0pt}def{\isadigit{2}}{\isacharcolon}{\kern0pt}\isanewline
\ \ \isakeyword{shows}\ {\isachardoublequoteopen}dist{\isacharunderscore}{\kern0pt}prod{\isacharunderscore}{\kern0pt}coprod{\isacharunderscore}{\kern0pt}inv\ A\ B\ C\ {\isacharcolon}{\kern0pt}\ A\ {\isasymtimes}\isactrlsub c\ {\isacharparenleft}{\kern0pt}B\ {\isasymCoprod}\ C{\isacharparenright}{\kern0pt}\ {\isasymrightarrow}\ {\isacharparenleft}{\kern0pt}A\ {\isasymtimes}\isactrlsub c\ B{\isacharparenright}{\kern0pt}\ {\isasymCoprod}\ {\isacharparenleft}{\kern0pt}A\ {\isasymtimes}\isactrlsub c\ C{\isacharparenright}{\kern0pt}\isanewline
\ \ \ \ {\isasymand}\ dist{\isacharunderscore}{\kern0pt}prod{\isacharunderscore}{\kern0pt}coprod{\isacharunderscore}{\kern0pt}inv\ A\ B\ C\ {\isasymcirc}\isactrlsub c\ dist{\isacharunderscore}{\kern0pt}prod{\isacharunderscore}{\kern0pt}coprod\ A\ B\ C\ {\isacharequal}{\kern0pt}\ id\ {\isacharparenleft}{\kern0pt}{\isacharparenleft}{\kern0pt}A\ {\isasymtimes}\isactrlsub c\ B{\isacharparenright}{\kern0pt}\ {\isasymCoprod}\ {\isacharparenleft}{\kern0pt}A\ {\isasymtimes}\isactrlsub c\ C{\isacharparenright}{\kern0pt}{\isacharparenright}{\kern0pt}\isanewline
\ \ \ \ {\isasymand}\ dist{\isacharunderscore}{\kern0pt}prod{\isacharunderscore}{\kern0pt}coprod\ A\ B\ C\ {\isasymcirc}\isactrlsub c\ dist{\isacharunderscore}{\kern0pt}prod{\isacharunderscore}{\kern0pt}coprod{\isacharunderscore}{\kern0pt}inv\ A\ B\ C\ {\isacharequal}{\kern0pt}\ id\ {\isacharparenleft}{\kern0pt}A\ {\isasymtimes}\isactrlsub c\ {\isacharparenleft}{\kern0pt}B\ {\isasymCoprod}\ C{\isacharparenright}{\kern0pt}{\isacharparenright}{\kern0pt}{\isachardoublequoteclose}\isanewline
%
\isadelimproof
\ \ %
\endisadelimproof
%
\isatagproof
\isacommand{unfolding}\isamarkupfalse%
\ dist{\isacharunderscore}{\kern0pt}prod{\isacharunderscore}{\kern0pt}coprod{\isacharunderscore}{\kern0pt}inv{\isacharunderscore}{\kern0pt}def\isanewline
\isacommand{proof}\isamarkupfalse%
\ {\isacharparenleft}{\kern0pt}rule\ theI{\isacharprime}{\kern0pt}{\isacharcomma}{\kern0pt}\ auto{\isacharparenright}{\kern0pt}\isanewline
\ \ \isacommand{show}\isamarkupfalse%
\ {\isachardoublequoteopen}{\isasymexists}x{\isachardot}{\kern0pt}\ x\ {\isacharcolon}{\kern0pt}\ A\ {\isasymtimes}\isactrlsub c\ B\ {\isasymCoprod}\ C\ {\isasymrightarrow}\ {\isacharparenleft}{\kern0pt}A\ {\isasymtimes}\isactrlsub c\ B{\isacharparenright}{\kern0pt}\ {\isasymCoprod}\ A\ {\isasymtimes}\isactrlsub c\ C\ {\isasymand}\isanewline
\ \ \ \ \ \ \ \ x\ {\isasymcirc}\isactrlsub c\ dist{\isacharunderscore}{\kern0pt}prod{\isacharunderscore}{\kern0pt}coprod\ A\ B\ C\ {\isacharequal}{\kern0pt}\ id\isactrlsub c\ {\isacharparenleft}{\kern0pt}{\isacharparenleft}{\kern0pt}A\ {\isasymtimes}\isactrlsub c\ B{\isacharparenright}{\kern0pt}\ {\isasymCoprod}\ A\ {\isasymtimes}\isactrlsub c\ C{\isacharparenright}{\kern0pt}\ {\isasymand}\isanewline
\ \ \ \ \ \ \ \ dist{\isacharunderscore}{\kern0pt}prod{\isacharunderscore}{\kern0pt}coprod\ A\ B\ C\ {\isasymcirc}\isactrlsub c\ x\ {\isacharequal}{\kern0pt}\ id\isactrlsub c\ {\isacharparenleft}{\kern0pt}A\ {\isasymtimes}\isactrlsub c\ B\ {\isasymCoprod}\ C{\isacharparenright}{\kern0pt}{\isachardoublequoteclose}\isanewline
\ \ \ \ \isacommand{using}\isamarkupfalse%
\ dist{\isacharunderscore}{\kern0pt}prod{\isacharunderscore}{\kern0pt}coprod{\isacharunderscore}{\kern0pt}iso{\isacharbrackleft}{\kern0pt}\isakeyword{where}\ A{\isacharequal}{\kern0pt}A{\isacharcomma}{\kern0pt}\ \isakeyword{where}\ B{\isacharequal}{\kern0pt}B{\isacharcomma}{\kern0pt}\ \isakeyword{where}\ C{\isacharequal}{\kern0pt}C{\isacharbrackright}{\kern0pt}\ \isacommand{unfolding}\isamarkupfalse%
\ isomorphism{\isacharunderscore}{\kern0pt}def\isanewline
\ \ \ \ \isacommand{by}\isamarkupfalse%
\ {\isacharparenleft}{\kern0pt}typecheck{\isacharunderscore}{\kern0pt}cfuncs{\isacharcomma}{\kern0pt}\ auto\ simp\ add{\isacharcolon}{\kern0pt}\ cfunc{\isacharunderscore}{\kern0pt}type{\isacharunderscore}{\kern0pt}def{\isacharparenright}{\kern0pt}\isanewline
\ \ \isacommand{then}\isamarkupfalse%
\ \isacommand{obtain}\isamarkupfalse%
\ inv\ \isakeyword{where}\ inv{\isacharunderscore}{\kern0pt}type{\isacharcolon}{\kern0pt}\ {\isachardoublequoteopen}inv\ {\isacharcolon}{\kern0pt}\ A\ {\isasymtimes}\isactrlsub c\ B\ {\isasymCoprod}\ C\ {\isasymrightarrow}\ {\isacharparenleft}{\kern0pt}A\ {\isasymtimes}\isactrlsub c\ B{\isacharparenright}{\kern0pt}\ {\isasymCoprod}\ A\ {\isasymtimes}\isactrlsub c\ C{\isachardoublequoteclose}\ \isakeyword{and}\isanewline
\ \ \ \ \ \ \ \ inv{\isacharunderscore}{\kern0pt}left{\isacharcolon}{\kern0pt}\ {\isachardoublequoteopen}inv\ {\isasymcirc}\isactrlsub c\ dist{\isacharunderscore}{\kern0pt}prod{\isacharunderscore}{\kern0pt}coprod\ A\ B\ C\ {\isacharequal}{\kern0pt}\ id\isactrlsub c\ {\isacharparenleft}{\kern0pt}{\isacharparenleft}{\kern0pt}A\ {\isasymtimes}\isactrlsub c\ B{\isacharparenright}{\kern0pt}\ {\isasymCoprod}\ A\ {\isasymtimes}\isactrlsub c\ C{\isacharparenright}{\kern0pt}{\isachardoublequoteclose}\ \isakeyword{and}\isanewline
\ \ \ \ \ \ \ \ inv{\isacharunderscore}{\kern0pt}right{\isacharcolon}{\kern0pt}\ {\isachardoublequoteopen}dist{\isacharunderscore}{\kern0pt}prod{\isacharunderscore}{\kern0pt}coprod\ A\ B\ C\ {\isasymcirc}\isactrlsub c\ inv\ {\isacharequal}{\kern0pt}\ id\isactrlsub c\ {\isacharparenleft}{\kern0pt}A\ {\isasymtimes}\isactrlsub c\ B\ {\isasymCoprod}\ C{\isacharparenright}{\kern0pt}{\isachardoublequoteclose}\isanewline
\ \ \ \ \isacommand{by}\isamarkupfalse%
\ auto\isanewline
\isanewline
\ \ \isacommand{fix}\isamarkupfalse%
\ x\ y\isanewline
\ \ \isacommand{assume}\isamarkupfalse%
\ x{\isacharunderscore}{\kern0pt}type{\isacharcolon}{\kern0pt}\ {\isachardoublequoteopen}x\ {\isacharcolon}{\kern0pt}\ A\ {\isasymtimes}\isactrlsub c\ B\ {\isasymCoprod}\ C\ {\isasymrightarrow}\ {\isacharparenleft}{\kern0pt}A\ {\isasymtimes}\isactrlsub c\ B{\isacharparenright}{\kern0pt}\ {\isasymCoprod}\ A\ {\isasymtimes}\isactrlsub c\ C{\isachardoublequoteclose}\isanewline
\ \ \isacommand{assume}\isamarkupfalse%
\ y{\isacharunderscore}{\kern0pt}type{\isacharcolon}{\kern0pt}\ {\isachardoublequoteopen}y\ {\isacharcolon}{\kern0pt}\ A\ {\isasymtimes}\isactrlsub c\ B\ {\isasymCoprod}\ C\ {\isasymrightarrow}\ {\isacharparenleft}{\kern0pt}A\ {\isasymtimes}\isactrlsub c\ B{\isacharparenright}{\kern0pt}\ {\isasymCoprod}\ A\ {\isasymtimes}\isactrlsub c\ C{\isachardoublequoteclose}\isanewline
\isanewline
\ \ \isacommand{assume}\isamarkupfalse%
\ {\isachardoublequoteopen}x\ {\isasymcirc}\isactrlsub c\ dist{\isacharunderscore}{\kern0pt}prod{\isacharunderscore}{\kern0pt}coprod\ A\ B\ C\ {\isacharequal}{\kern0pt}\ id\isactrlsub c\ {\isacharparenleft}{\kern0pt}{\isacharparenleft}{\kern0pt}A\ {\isasymtimes}\isactrlsub c\ B{\isacharparenright}{\kern0pt}\ {\isasymCoprod}\ A\ {\isasymtimes}\isactrlsub c\ C{\isacharparenright}{\kern0pt}{\isachardoublequoteclose}\isanewline
\ \ \ \ \isakeyword{and}\ {\isachardoublequoteopen}y\ {\isasymcirc}\isactrlsub c\ dist{\isacharunderscore}{\kern0pt}prod{\isacharunderscore}{\kern0pt}coprod\ A\ B\ C\ {\isacharequal}{\kern0pt}\ id\isactrlsub c\ {\isacharparenleft}{\kern0pt}{\isacharparenleft}{\kern0pt}A\ {\isasymtimes}\isactrlsub c\ B{\isacharparenright}{\kern0pt}\ {\isasymCoprod}\ A\ {\isasymtimes}\isactrlsub c\ C{\isacharparenright}{\kern0pt}{\isachardoublequoteclose}\isanewline
\ \ \isacommand{then}\isamarkupfalse%
\ \isacommand{have}\isamarkupfalse%
\ {\isachardoublequoteopen}x\ {\isasymcirc}\isactrlsub c\ dist{\isacharunderscore}{\kern0pt}prod{\isacharunderscore}{\kern0pt}coprod\ A\ B\ C\ {\isacharequal}{\kern0pt}\ y\ {\isasymcirc}\isactrlsub c\ dist{\isacharunderscore}{\kern0pt}prod{\isacharunderscore}{\kern0pt}coprod\ A\ B\ C{\isachardoublequoteclose}\isanewline
\ \ \ \ \isacommand{by}\isamarkupfalse%
\ auto\isanewline
\ \ \isacommand{then}\isamarkupfalse%
\ \isacommand{have}\isamarkupfalse%
\ {\isachardoublequoteopen}{\isacharparenleft}{\kern0pt}x\ {\isasymcirc}\isactrlsub c\ dist{\isacharunderscore}{\kern0pt}prod{\isacharunderscore}{\kern0pt}coprod\ A\ B\ C{\isacharparenright}{\kern0pt}\ {\isasymcirc}\isactrlsub c\ inv\ {\isacharequal}{\kern0pt}\ {\isacharparenleft}{\kern0pt}y\ {\isasymcirc}\isactrlsub c\ dist{\isacharunderscore}{\kern0pt}prod{\isacharunderscore}{\kern0pt}coprod\ A\ B\ C{\isacharparenright}{\kern0pt}\ {\isasymcirc}\isactrlsub c\ inv{\isachardoublequoteclose}\isanewline
\ \ \ \ \isacommand{by}\isamarkupfalse%
\ auto\isanewline
\ \ \isacommand{then}\isamarkupfalse%
\ \isacommand{have}\isamarkupfalse%
\ {\isachardoublequoteopen}x\ {\isasymcirc}\isactrlsub c\ dist{\isacharunderscore}{\kern0pt}prod{\isacharunderscore}{\kern0pt}coprod\ A\ B\ C\ {\isasymcirc}\isactrlsub c\ inv\ {\isacharequal}{\kern0pt}\ y\ {\isasymcirc}\isactrlsub c\ dist{\isacharunderscore}{\kern0pt}prod{\isacharunderscore}{\kern0pt}coprod\ A\ B\ C\ {\isasymcirc}\isactrlsub c\ inv{\isachardoublequoteclose}\isanewline
\ \ \ \ \isacommand{using}\isamarkupfalse%
\ inv{\isacharunderscore}{\kern0pt}type\ x{\isacharunderscore}{\kern0pt}type\ y{\isacharunderscore}{\kern0pt}type\ \isacommand{by}\isamarkupfalse%
\ {\isacharparenleft}{\kern0pt}typecheck{\isacharunderscore}{\kern0pt}cfuncs{\isacharcomma}{\kern0pt}\ auto\ simp\ add{\isacharcolon}{\kern0pt}\ comp{\isacharunderscore}{\kern0pt}associative{\isadigit{2}}{\isacharparenright}{\kern0pt}\isanewline
\ \ \isacommand{then}\isamarkupfalse%
\ \isacommand{have}\isamarkupfalse%
\ {\isachardoublequoteopen}x\ {\isasymcirc}\isactrlsub c\ id\isactrlsub c\ {\isacharparenleft}{\kern0pt}A\ {\isasymtimes}\isactrlsub c\ B\ {\isasymCoprod}\ C{\isacharparenright}{\kern0pt}\ {\isacharequal}{\kern0pt}\ y\ {\isasymcirc}\isactrlsub c\ id\isactrlsub c\ {\isacharparenleft}{\kern0pt}A\ {\isasymtimes}\isactrlsub c\ B\ {\isasymCoprod}\ C{\isacharparenright}{\kern0pt}{\isachardoublequoteclose}\isanewline
\ \ \ \ \isacommand{by}\isamarkupfalse%
\ {\isacharparenleft}{\kern0pt}simp\ add{\isacharcolon}{\kern0pt}\ inv{\isacharunderscore}{\kern0pt}right{\isacharparenright}{\kern0pt}\isanewline
\ \ \isacommand{then}\isamarkupfalse%
\ \isacommand{show}\isamarkupfalse%
\ {\isachardoublequoteopen}x\ {\isacharequal}{\kern0pt}\ y{\isachardoublequoteclose}\isanewline
\ \ \ \ \isacommand{using}\isamarkupfalse%
\ id{\isacharunderscore}{\kern0pt}right{\isacharunderscore}{\kern0pt}unit{\isadigit{2}}\ x{\isacharunderscore}{\kern0pt}type\ y{\isacharunderscore}{\kern0pt}type\ \isacommand{by}\isamarkupfalse%
\ auto\isanewline
\isacommand{qed}\isamarkupfalse%
%
\endisatagproof
{\isafoldproof}%
%
\isadelimproof
\isanewline
%
\endisadelimproof
\isanewline
\isacommand{lemma}\isamarkupfalse%
\ dist{\isacharunderscore}{\kern0pt}prod{\isacharunderscore}{\kern0pt}coprod{\isacharunderscore}{\kern0pt}inv{\isacharunderscore}{\kern0pt}type{\isacharbrackleft}{\kern0pt}type{\isacharunderscore}{\kern0pt}rule{\isacharbrackright}{\kern0pt}{\isacharcolon}{\kern0pt}\isanewline
\ \ {\isachardoublequoteopen}dist{\isacharunderscore}{\kern0pt}prod{\isacharunderscore}{\kern0pt}coprod{\isacharunderscore}{\kern0pt}inv\ A\ B\ C\ {\isacharcolon}{\kern0pt}\ A\ {\isasymtimes}\isactrlsub c\ {\isacharparenleft}{\kern0pt}B\ {\isasymCoprod}\ C{\isacharparenright}{\kern0pt}\ {\isasymrightarrow}\ {\isacharparenleft}{\kern0pt}A\ {\isasymtimes}\isactrlsub c\ B{\isacharparenright}{\kern0pt}\ {\isasymCoprod}\ {\isacharparenleft}{\kern0pt}A\ {\isasymtimes}\isactrlsub c\ C{\isacharparenright}{\kern0pt}{\isachardoublequoteclose}\isanewline
%
\isadelimproof
\ \ %
\endisadelimproof
%
\isatagproof
\isacommand{by}\isamarkupfalse%
\ {\isacharparenleft}{\kern0pt}simp\ add{\isacharcolon}{\kern0pt}\ dist{\isacharunderscore}{\kern0pt}prod{\isacharunderscore}{\kern0pt}coprod{\isacharunderscore}{\kern0pt}inv{\isacharunderscore}{\kern0pt}def{\isadigit{2}}{\isacharparenright}{\kern0pt}%
\endisatagproof
{\isafoldproof}%
%
\isadelimproof
\isanewline
%
\endisadelimproof
\isanewline
\isacommand{lemma}\isamarkupfalse%
\ dist{\isacharunderscore}{\kern0pt}prod{\isacharunderscore}{\kern0pt}coprod{\isacharunderscore}{\kern0pt}inv{\isacharunderscore}{\kern0pt}left{\isacharcolon}{\kern0pt}\isanewline
\ \ {\isachardoublequoteopen}dist{\isacharunderscore}{\kern0pt}prod{\isacharunderscore}{\kern0pt}coprod{\isacharunderscore}{\kern0pt}inv\ A\ B\ C\ {\isasymcirc}\isactrlsub c\ dist{\isacharunderscore}{\kern0pt}prod{\isacharunderscore}{\kern0pt}coprod\ A\ B\ C\ {\isacharequal}{\kern0pt}\ id\ {\isacharparenleft}{\kern0pt}{\isacharparenleft}{\kern0pt}A\ {\isasymtimes}\isactrlsub c\ B{\isacharparenright}{\kern0pt}\ {\isasymCoprod}\ {\isacharparenleft}{\kern0pt}A\ {\isasymtimes}\isactrlsub c\ C{\isacharparenright}{\kern0pt}{\isacharparenright}{\kern0pt}{\isachardoublequoteclose}\isanewline
%
\isadelimproof
\ \ %
\endisadelimproof
%
\isatagproof
\isacommand{by}\isamarkupfalse%
\ {\isacharparenleft}{\kern0pt}simp\ add{\isacharcolon}{\kern0pt}\ dist{\isacharunderscore}{\kern0pt}prod{\isacharunderscore}{\kern0pt}coprod{\isacharunderscore}{\kern0pt}inv{\isacharunderscore}{\kern0pt}def{\isadigit{2}}{\isacharparenright}{\kern0pt}%
\endisatagproof
{\isafoldproof}%
%
\isadelimproof
\isanewline
%
\endisadelimproof
\isanewline
\isacommand{lemma}\isamarkupfalse%
\ dist{\isacharunderscore}{\kern0pt}prod{\isacharunderscore}{\kern0pt}coprod{\isacharunderscore}{\kern0pt}inv{\isacharunderscore}{\kern0pt}right{\isacharcolon}{\kern0pt}\isanewline
\ \ {\isachardoublequoteopen}dist{\isacharunderscore}{\kern0pt}prod{\isacharunderscore}{\kern0pt}coprod\ A\ B\ C\ {\isasymcirc}\isactrlsub c\ dist{\isacharunderscore}{\kern0pt}prod{\isacharunderscore}{\kern0pt}coprod{\isacharunderscore}{\kern0pt}inv\ A\ B\ C\ {\isacharequal}{\kern0pt}\ id\ {\isacharparenleft}{\kern0pt}A\ {\isasymtimes}\isactrlsub c\ {\isacharparenleft}{\kern0pt}B\ {\isasymCoprod}\ C{\isacharparenright}{\kern0pt}{\isacharparenright}{\kern0pt}{\isachardoublequoteclose}\isanewline
%
\isadelimproof
\ \ %
\endisadelimproof
%
\isatagproof
\isacommand{by}\isamarkupfalse%
\ {\isacharparenleft}{\kern0pt}simp\ add{\isacharcolon}{\kern0pt}\ dist{\isacharunderscore}{\kern0pt}prod{\isacharunderscore}{\kern0pt}coprod{\isacharunderscore}{\kern0pt}inv{\isacharunderscore}{\kern0pt}def{\isadigit{2}}{\isacharparenright}{\kern0pt}%
\endisatagproof
{\isafoldproof}%
%
\isadelimproof
\isanewline
%
\endisadelimproof
\isanewline
\isacommand{lemma}\isamarkupfalse%
\ dist{\isacharunderscore}{\kern0pt}prod{\isacharunderscore}{\kern0pt}coprod{\isacharunderscore}{\kern0pt}inv{\isacharunderscore}{\kern0pt}iso{\isacharcolon}{\kern0pt}\isanewline
\ \ {\isachardoublequoteopen}isomorphism{\isacharparenleft}{\kern0pt}dist{\isacharunderscore}{\kern0pt}prod{\isacharunderscore}{\kern0pt}coprod{\isacharunderscore}{\kern0pt}inv\ A\ B\ C{\isacharparenright}{\kern0pt}{\isachardoublequoteclose}\isanewline
%
\isadelimproof
\ \ %
\endisadelimproof
%
\isatagproof
\isacommand{by}\isamarkupfalse%
\ {\isacharparenleft}{\kern0pt}metis\ dist{\isacharunderscore}{\kern0pt}prod{\isacharunderscore}{\kern0pt}coprod{\isacharunderscore}{\kern0pt}inv{\isacharunderscore}{\kern0pt}right\ dist{\isacharunderscore}{\kern0pt}prod{\isacharunderscore}{\kern0pt}coprod{\isacharunderscore}{\kern0pt}inv{\isacharunderscore}{\kern0pt}type\ dist{\isacharunderscore}{\kern0pt}prod{\isacharunderscore}{\kern0pt}coprod{\isacharunderscore}{\kern0pt}iso\ dist{\isacharunderscore}{\kern0pt}prod{\isacharunderscore}{\kern0pt}coprod{\isacharunderscore}{\kern0pt}type\ id{\isacharunderscore}{\kern0pt}isomorphism\ id{\isacharunderscore}{\kern0pt}right{\isacharunderscore}{\kern0pt}unit{\isadigit{2}}\ id{\isacharunderscore}{\kern0pt}type\ isomorphism{\isacharunderscore}{\kern0pt}sandwich{\isacharparenright}{\kern0pt}%
\endisatagproof
{\isafoldproof}%
%
\isadelimproof
\isanewline
%
\endisadelimproof
\isanewline
\isacommand{lemma}\isamarkupfalse%
\ dist{\isacharunderscore}{\kern0pt}prod{\isacharunderscore}{\kern0pt}coprod{\isacharunderscore}{\kern0pt}inv{\isacharunderscore}{\kern0pt}left{\isacharunderscore}{\kern0pt}ap{\isacharcolon}{\kern0pt}\isanewline
\ \ \isakeyword{assumes}\ {\isachardoublequoteopen}a\ {\isasymin}\isactrlsub c\ A{\isachardoublequoteclose}\ {\isachardoublequoteopen}b\ {\isasymin}\isactrlsub c\ B{\isachardoublequoteclose}\isanewline
\ \ \isakeyword{shows}\ {\isachardoublequoteopen}dist{\isacharunderscore}{\kern0pt}prod{\isacharunderscore}{\kern0pt}coprod{\isacharunderscore}{\kern0pt}inv\ A\ B\ C\ {\isasymcirc}\isactrlsub c\ {\isasymlangle}a{\isacharcomma}{\kern0pt}left{\isacharunderscore}{\kern0pt}coproj\ B\ C\ {\isasymcirc}\isactrlsub c\ b{\isasymrangle}\ {\isacharequal}{\kern0pt}\ left{\isacharunderscore}{\kern0pt}coproj\ {\isacharparenleft}{\kern0pt}A\ {\isasymtimes}\isactrlsub c\ B{\isacharparenright}{\kern0pt}\ {\isacharparenleft}{\kern0pt}A\ {\isasymtimes}\isactrlsub c\ C{\isacharparenright}{\kern0pt}\ {\isasymcirc}\isactrlsub c\ {\isasymlangle}a{\isacharcomma}{\kern0pt}b{\isasymrangle}{\isachardoublequoteclose}\isanewline
%
\isadelimproof
\ \ %
\endisadelimproof
%
\isatagproof
\isacommand{using}\isamarkupfalse%
\ assms\ \isacommand{by}\isamarkupfalse%
\ {\isacharparenleft}{\kern0pt}typecheck{\isacharunderscore}{\kern0pt}cfuncs{\isacharcomma}{\kern0pt}\ smt\ comp{\isacharunderscore}{\kern0pt}associative{\isadigit{2}}\ dist{\isacharunderscore}{\kern0pt}prod{\isacharunderscore}{\kern0pt}coprod{\isacharunderscore}{\kern0pt}inv{\isacharunderscore}{\kern0pt}def{\isadigit{2}}\ dist{\isacharunderscore}{\kern0pt}prod{\isacharunderscore}{\kern0pt}coprod{\isacharunderscore}{\kern0pt}left{\isacharunderscore}{\kern0pt}ap\ dist{\isacharunderscore}{\kern0pt}prod{\isacharunderscore}{\kern0pt}coprod{\isacharunderscore}{\kern0pt}type\ id{\isacharunderscore}{\kern0pt}left{\isacharunderscore}{\kern0pt}unit{\isadigit{2}}{\isacharparenright}{\kern0pt}%
\endisatagproof
{\isafoldproof}%
%
\isadelimproof
\isanewline
%
\endisadelimproof
\isanewline
\isacommand{lemma}\isamarkupfalse%
\ dist{\isacharunderscore}{\kern0pt}prod{\isacharunderscore}{\kern0pt}coprod{\isacharunderscore}{\kern0pt}inv{\isacharunderscore}{\kern0pt}right{\isacharunderscore}{\kern0pt}ap{\isacharcolon}{\kern0pt}\isanewline
\ \ \isakeyword{assumes}\ {\isachardoublequoteopen}a\ {\isasymin}\isactrlsub c\ A{\isachardoublequoteclose}\ {\isachardoublequoteopen}c\ {\isasymin}\isactrlsub c\ C{\isachardoublequoteclose}\isanewline
\ \ \isakeyword{shows}\ {\isachardoublequoteopen}dist{\isacharunderscore}{\kern0pt}prod{\isacharunderscore}{\kern0pt}coprod{\isacharunderscore}{\kern0pt}inv\ A\ B\ C\ {\isasymcirc}\isactrlsub c\ {\isasymlangle}a{\isacharcomma}{\kern0pt}right{\isacharunderscore}{\kern0pt}coproj\ B\ C\ {\isasymcirc}\isactrlsub c\ c{\isasymrangle}\ {\isacharequal}{\kern0pt}\ right{\isacharunderscore}{\kern0pt}coproj\ {\isacharparenleft}{\kern0pt}A\ {\isasymtimes}\isactrlsub c\ B{\isacharparenright}{\kern0pt}\ {\isacharparenleft}{\kern0pt}A\ {\isasymtimes}\isactrlsub c\ C{\isacharparenright}{\kern0pt}\ {\isasymcirc}\isactrlsub c\ {\isasymlangle}a{\isacharcomma}{\kern0pt}c{\isasymrangle}{\isachardoublequoteclose}\isanewline
%
\isadelimproof
\ \ %
\endisadelimproof
%
\isatagproof
\isacommand{using}\isamarkupfalse%
\ assms\ \isacommand{by}\isamarkupfalse%
\ {\isacharparenleft}{\kern0pt}typecheck{\isacharunderscore}{\kern0pt}cfuncs{\isacharcomma}{\kern0pt}\ smt\ comp{\isacharunderscore}{\kern0pt}associative{\isadigit{2}}\ dist{\isacharunderscore}{\kern0pt}prod{\isacharunderscore}{\kern0pt}coprod{\isacharunderscore}{\kern0pt}inv{\isacharunderscore}{\kern0pt}def{\isadigit{2}}\ dist{\isacharunderscore}{\kern0pt}prod{\isacharunderscore}{\kern0pt}coprod{\isacharunderscore}{\kern0pt}right{\isacharunderscore}{\kern0pt}ap\ dist{\isacharunderscore}{\kern0pt}prod{\isacharunderscore}{\kern0pt}coprod{\isacharunderscore}{\kern0pt}type\ id{\isacharunderscore}{\kern0pt}left{\isacharunderscore}{\kern0pt}unit{\isadigit{2}}{\isacharparenright}{\kern0pt}%
\endisatagproof
{\isafoldproof}%
%
\isadelimproof
%
\endisadelimproof
%
\isadelimdocument
%
\endisadelimdocument
%
\isatagdocument
%
\isamarkupsubsubsection{Distribute Product Over Coproduct Auxillary Mapping 2%
}
\isamarkuptrue%
%
\endisatagdocument
{\isafolddocument}%
%
\isadelimdocument
%
\endisadelimdocument
\isacommand{definition}\isamarkupfalse%
\ dist{\isacharunderscore}{\kern0pt}prod{\isacharunderscore}{\kern0pt}coprod{\isadigit{2}}\ {\isacharcolon}{\kern0pt}{\isacharcolon}{\kern0pt}\ {\isachardoublequoteopen}cset\ {\isasymRightarrow}\ cset\ {\isasymRightarrow}\ cset\ {\isasymRightarrow}\ cfunc{\isachardoublequoteclose}\ \isakeyword{where}\isanewline
\ \ {\isachardoublequoteopen}dist{\isacharunderscore}{\kern0pt}prod{\isacharunderscore}{\kern0pt}coprod{\isadigit{2}}\ A\ B\ C\ {\isacharequal}{\kern0pt}\ swap\ C\ {\isacharparenleft}{\kern0pt}A\ {\isasymCoprod}\ B{\isacharparenright}{\kern0pt}\ {\isasymcirc}\isactrlsub c\ dist{\isacharunderscore}{\kern0pt}prod{\isacharunderscore}{\kern0pt}coprod\ C\ A\ B\ {\isasymcirc}\isactrlsub c\ {\isacharparenleft}{\kern0pt}swap\ A\ C\ {\isasymbowtie}\isactrlsub f\ swap\ B\ C{\isacharparenright}{\kern0pt}{\isachardoublequoteclose}\isanewline
\isanewline
\isacommand{lemma}\isamarkupfalse%
\ dist{\isacharunderscore}{\kern0pt}prod{\isacharunderscore}{\kern0pt}coprod{\isadigit{2}}{\isacharunderscore}{\kern0pt}type{\isacharbrackleft}{\kern0pt}type{\isacharunderscore}{\kern0pt}rule{\isacharbrackright}{\kern0pt}{\isacharcolon}{\kern0pt}\isanewline
\ \ {\isachardoublequoteopen}dist{\isacharunderscore}{\kern0pt}prod{\isacharunderscore}{\kern0pt}coprod{\isadigit{2}}\ A\ B\ C\ {\isacharcolon}{\kern0pt}\ {\isacharparenleft}{\kern0pt}A\ {\isasymtimes}\isactrlsub c\ C{\isacharparenright}{\kern0pt}\ {\isasymCoprod}\ {\isacharparenleft}{\kern0pt}B\ {\isasymtimes}\isactrlsub c\ C{\isacharparenright}{\kern0pt}\ {\isasymrightarrow}\ {\isacharparenleft}{\kern0pt}A\ {\isasymCoprod}\ B{\isacharparenright}{\kern0pt}\ {\isasymtimes}\isactrlsub c\ C{\isachardoublequoteclose}\isanewline
%
\isadelimproof
\ \ %
\endisadelimproof
%
\isatagproof
\isacommand{unfolding}\isamarkupfalse%
\ dist{\isacharunderscore}{\kern0pt}prod{\isacharunderscore}{\kern0pt}coprod{\isadigit{2}}{\isacharunderscore}{\kern0pt}def\ \isacommand{by}\isamarkupfalse%
\ typecheck{\isacharunderscore}{\kern0pt}cfuncs%
\endisatagproof
{\isafoldproof}%
%
\isadelimproof
\isanewline
%
\endisadelimproof
\isanewline
\isacommand{lemma}\isamarkupfalse%
\ dist{\isacharunderscore}{\kern0pt}prod{\isacharunderscore}{\kern0pt}coprod{\isadigit{2}}{\isacharunderscore}{\kern0pt}left{\isacharunderscore}{\kern0pt}ap{\isacharcolon}{\kern0pt}\isanewline
\ \ \isakeyword{assumes}\ {\isachardoublequoteopen}a\ {\isasymin}\isactrlsub c\ A{\isachardoublequoteclose}\ {\isachardoublequoteopen}c\ {\isasymin}\isactrlsub c\ C{\isachardoublequoteclose}\isanewline
\ \ \isakeyword{shows}\ {\isachardoublequoteopen}dist{\isacharunderscore}{\kern0pt}prod{\isacharunderscore}{\kern0pt}coprod{\isadigit{2}}\ A\ B\ C\ {\isasymcirc}\isactrlsub c\ {\isacharparenleft}{\kern0pt}left{\isacharunderscore}{\kern0pt}coproj\ {\isacharparenleft}{\kern0pt}A\ {\isasymtimes}\isactrlsub c\ C{\isacharparenright}{\kern0pt}\ {\isacharparenleft}{\kern0pt}B\ {\isasymtimes}\isactrlsub c\ C{\isacharparenright}{\kern0pt}\ {\isasymcirc}\isactrlsub c\ {\isasymlangle}a{\isacharcomma}{\kern0pt}\ c{\isasymrangle}{\isacharparenright}{\kern0pt}\ {\isacharequal}{\kern0pt}\ {\isasymlangle}left{\isacharunderscore}{\kern0pt}coproj\ A\ B\ {\isasymcirc}\isactrlsub c\ a{\isacharcomma}{\kern0pt}\ c{\isasymrangle}{\isachardoublequoteclose}\isanewline
%
\isadelimproof
%
\endisadelimproof
%
\isatagproof
\isacommand{proof}\isamarkupfalse%
\ {\isacharminus}{\kern0pt}\isanewline
\ \ \isacommand{have}\isamarkupfalse%
\ {\isachardoublequoteopen}dist{\isacharunderscore}{\kern0pt}prod{\isacharunderscore}{\kern0pt}coprod{\isadigit{2}}\ A\ B\ C\ {\isasymcirc}\isactrlsub c\ {\isacharparenleft}{\kern0pt}left{\isacharunderscore}{\kern0pt}coproj\ {\isacharparenleft}{\kern0pt}A\ {\isasymtimes}\isactrlsub c\ C{\isacharparenright}{\kern0pt}\ {\isacharparenleft}{\kern0pt}B\ {\isasymtimes}\isactrlsub c\ C{\isacharparenright}{\kern0pt}\ {\isasymcirc}\isactrlsub c\ {\isasymlangle}a{\isacharcomma}{\kern0pt}\ c{\isasymrangle}{\isacharparenright}{\kern0pt}\isanewline
\ \ \ \ {\isacharequal}{\kern0pt}\ {\isacharparenleft}{\kern0pt}swap\ C\ {\isacharparenleft}{\kern0pt}A\ {\isasymCoprod}\ B{\isacharparenright}{\kern0pt}\ {\isasymcirc}\isactrlsub c\ dist{\isacharunderscore}{\kern0pt}prod{\isacharunderscore}{\kern0pt}coprod\ C\ A\ B\ {\isasymcirc}\isactrlsub c\ {\isacharparenleft}{\kern0pt}swap\ A\ C\ {\isasymbowtie}\isactrlsub f\ swap\ B\ C{\isacharparenright}{\kern0pt}{\isacharparenright}{\kern0pt}\ {\isasymcirc}\isactrlsub c\ {\isacharparenleft}{\kern0pt}left{\isacharunderscore}{\kern0pt}coproj\ {\isacharparenleft}{\kern0pt}A\ {\isasymtimes}\isactrlsub c\ C{\isacharparenright}{\kern0pt}\ {\isacharparenleft}{\kern0pt}B\ {\isasymtimes}\isactrlsub c\ C{\isacharparenright}{\kern0pt}\ {\isasymcirc}\isactrlsub c\ {\isasymlangle}a{\isacharcomma}{\kern0pt}\ c{\isasymrangle}{\isacharparenright}{\kern0pt}{\isachardoublequoteclose}\isanewline
\ \ \ \ \isacommand{unfolding}\isamarkupfalse%
\ dist{\isacharunderscore}{\kern0pt}prod{\isacharunderscore}{\kern0pt}coprod{\isadigit{2}}{\isacharunderscore}{\kern0pt}def\ \isacommand{by}\isamarkupfalse%
\ auto\isanewline
\ \ \isacommand{also}\isamarkupfalse%
\ \isacommand{have}\isamarkupfalse%
\ {\isachardoublequoteopen}{\isachardot}{\kern0pt}{\isachardot}{\kern0pt}{\isachardot}{\kern0pt}\ {\isacharequal}{\kern0pt}\ swap\ C\ {\isacharparenleft}{\kern0pt}A\ {\isasymCoprod}\ B{\isacharparenright}{\kern0pt}\ {\isasymcirc}\isactrlsub c\ dist{\isacharunderscore}{\kern0pt}prod{\isacharunderscore}{\kern0pt}coprod\ C\ A\ B\ {\isasymcirc}\isactrlsub c\ {\isacharparenleft}{\kern0pt}{\isacharparenleft}{\kern0pt}swap\ A\ C\ {\isasymbowtie}\isactrlsub f\ swap\ B\ C{\isacharparenright}{\kern0pt}\ {\isasymcirc}\isactrlsub c\ left{\isacharunderscore}{\kern0pt}coproj\ {\isacharparenleft}{\kern0pt}A\ {\isasymtimes}\isactrlsub c\ C{\isacharparenright}{\kern0pt}\ {\isacharparenleft}{\kern0pt}B\ {\isasymtimes}\isactrlsub c\ C{\isacharparenright}{\kern0pt}{\isacharparenright}{\kern0pt}\ {\isasymcirc}\isactrlsub c\ {\isasymlangle}a{\isacharcomma}{\kern0pt}\ c{\isasymrangle}{\isachardoublequoteclose}\isanewline
\ \ \ \ \isacommand{using}\isamarkupfalse%
\ assms\ \isacommand{by}\isamarkupfalse%
\ {\isacharparenleft}{\kern0pt}typecheck{\isacharunderscore}{\kern0pt}cfuncs{\isacharcomma}{\kern0pt}\ smt\ comp{\isacharunderscore}{\kern0pt}associative{\isadigit{2}}{\isacharparenright}{\kern0pt}\isanewline
\ \ \isacommand{also}\isamarkupfalse%
\ \isacommand{have}\isamarkupfalse%
\ {\isachardoublequoteopen}{\isachardot}{\kern0pt}{\isachardot}{\kern0pt}{\isachardot}{\kern0pt}\ {\isacharequal}{\kern0pt}\ swap\ C\ {\isacharparenleft}{\kern0pt}A\ {\isasymCoprod}\ B{\isacharparenright}{\kern0pt}\ {\isasymcirc}\isactrlsub c\ dist{\isacharunderscore}{\kern0pt}prod{\isacharunderscore}{\kern0pt}coprod\ C\ A\ B\ {\isasymcirc}\isactrlsub c\ {\isacharparenleft}{\kern0pt}left{\isacharunderscore}{\kern0pt}coproj\ {\isacharparenleft}{\kern0pt}C\ {\isasymtimes}\isactrlsub c\ A{\isacharparenright}{\kern0pt}\ {\isacharparenleft}{\kern0pt}C\ {\isasymtimes}\isactrlsub c\ B{\isacharparenright}{\kern0pt}\ {\isasymcirc}\isactrlsub c\ swap\ A\ C{\isacharparenright}{\kern0pt}\ {\isasymcirc}\isactrlsub c\ {\isasymlangle}a{\isacharcomma}{\kern0pt}\ c{\isasymrangle}{\isachardoublequoteclose}\isanewline
\ \ \ \ \isacommand{using}\isamarkupfalse%
\ assms\ \isacommand{by}\isamarkupfalse%
\ {\isacharparenleft}{\kern0pt}typecheck{\isacharunderscore}{\kern0pt}cfuncs{\isacharcomma}{\kern0pt}\ auto\ simp\ add{\isacharcolon}{\kern0pt}\ left{\isacharunderscore}{\kern0pt}coproj{\isacharunderscore}{\kern0pt}cfunc{\isacharunderscore}{\kern0pt}bowtie{\isacharunderscore}{\kern0pt}prod{\isacharparenright}{\kern0pt}\isanewline
\ \ \isacommand{also}\isamarkupfalse%
\ \isacommand{have}\isamarkupfalse%
\ {\isachardoublequoteopen}{\isachardot}{\kern0pt}{\isachardot}{\kern0pt}{\isachardot}{\kern0pt}\ {\isacharequal}{\kern0pt}\ swap\ C\ {\isacharparenleft}{\kern0pt}A\ {\isasymCoprod}\ B{\isacharparenright}{\kern0pt}\ {\isasymcirc}\isactrlsub c\ dist{\isacharunderscore}{\kern0pt}prod{\isacharunderscore}{\kern0pt}coprod\ C\ A\ B\ {\isasymcirc}\isactrlsub c\ left{\isacharunderscore}{\kern0pt}coproj\ {\isacharparenleft}{\kern0pt}C\ {\isasymtimes}\isactrlsub c\ A{\isacharparenright}{\kern0pt}\ {\isacharparenleft}{\kern0pt}C\ {\isasymtimes}\isactrlsub c\ B{\isacharparenright}{\kern0pt}\ {\isasymcirc}\isactrlsub c\ swap\ A\ C\ {\isasymcirc}\isactrlsub c\ {\isasymlangle}a{\isacharcomma}{\kern0pt}\ c{\isasymrangle}{\isachardoublequoteclose}\isanewline
\ \ \ \ \isacommand{using}\isamarkupfalse%
\ assms\ \isacommand{by}\isamarkupfalse%
\ {\isacharparenleft}{\kern0pt}typecheck{\isacharunderscore}{\kern0pt}cfuncs{\isacharcomma}{\kern0pt}\ auto\ simp\ add{\isacharcolon}{\kern0pt}\ comp{\isacharunderscore}{\kern0pt}associative{\isadigit{2}}{\isacharparenright}{\kern0pt}\isanewline
\ \ \isacommand{also}\isamarkupfalse%
\ \isacommand{have}\isamarkupfalse%
\ {\isachardoublequoteopen}{\isachardot}{\kern0pt}{\isachardot}{\kern0pt}{\isachardot}{\kern0pt}\ {\isacharequal}{\kern0pt}\ swap\ C\ {\isacharparenleft}{\kern0pt}A\ {\isasymCoprod}\ B{\isacharparenright}{\kern0pt}\ {\isasymcirc}\isactrlsub c\ dist{\isacharunderscore}{\kern0pt}prod{\isacharunderscore}{\kern0pt}coprod\ C\ A\ B\ {\isasymcirc}\isactrlsub c\ left{\isacharunderscore}{\kern0pt}coproj\ {\isacharparenleft}{\kern0pt}C\ {\isasymtimes}\isactrlsub c\ A{\isacharparenright}{\kern0pt}\ {\isacharparenleft}{\kern0pt}C\ {\isasymtimes}\isactrlsub c\ B{\isacharparenright}{\kern0pt}\ {\isasymcirc}\isactrlsub c\ {\isasymlangle}c{\isacharcomma}{\kern0pt}\ a{\isasymrangle}{\isachardoublequoteclose}\isanewline
\ \ \ \ \isacommand{using}\isamarkupfalse%
\ assms\ swap{\isacharunderscore}{\kern0pt}ap\ \isacommand{by}\isamarkupfalse%
\ {\isacharparenleft}{\kern0pt}typecheck{\isacharunderscore}{\kern0pt}cfuncs{\isacharcomma}{\kern0pt}\ auto{\isacharparenright}{\kern0pt}\isanewline
\ \ \isacommand{also}\isamarkupfalse%
\ \isacommand{have}\isamarkupfalse%
\ {\isachardoublequoteopen}{\isachardot}{\kern0pt}{\isachardot}{\kern0pt}{\isachardot}{\kern0pt}\ {\isacharequal}{\kern0pt}\ swap\ C\ {\isacharparenleft}{\kern0pt}A\ {\isasymCoprod}\ B{\isacharparenright}{\kern0pt}\ {\isasymcirc}\isactrlsub c\ {\isasymlangle}c{\isacharcomma}{\kern0pt}\ left{\isacharunderscore}{\kern0pt}coproj\ A\ B\ {\isasymcirc}\isactrlsub c\ a{\isasymrangle}{\isachardoublequoteclose}\isanewline
\ \ \ \ \isacommand{using}\isamarkupfalse%
\ assms\ \isacommand{by}\isamarkupfalse%
\ {\isacharparenleft}{\kern0pt}typecheck{\isacharunderscore}{\kern0pt}cfuncs{\isacharcomma}{\kern0pt}\ simp\ add{\isacharcolon}{\kern0pt}\ dist{\isacharunderscore}{\kern0pt}prod{\isacharunderscore}{\kern0pt}coprod{\isacharunderscore}{\kern0pt}left{\isacharunderscore}{\kern0pt}ap{\isacharparenright}{\kern0pt}\isanewline
\ \ \isacommand{also}\isamarkupfalse%
\ \isacommand{have}\isamarkupfalse%
\ {\isachardoublequoteopen}{\isachardot}{\kern0pt}{\isachardot}{\kern0pt}{\isachardot}{\kern0pt}\ {\isacharequal}{\kern0pt}\ {\isasymlangle}left{\isacharunderscore}{\kern0pt}coproj\ A\ B\ {\isasymcirc}\isactrlsub c\ a{\isacharcomma}{\kern0pt}\ c{\isasymrangle}{\isachardoublequoteclose}\isanewline
\ \ \ \ \isacommand{using}\isamarkupfalse%
\ assms\ swap{\isacharunderscore}{\kern0pt}ap\ \isacommand{by}\isamarkupfalse%
\ {\isacharparenleft}{\kern0pt}typecheck{\isacharunderscore}{\kern0pt}cfuncs{\isacharcomma}{\kern0pt}\ auto{\isacharparenright}{\kern0pt}\isanewline
\ \ \isacommand{then}\isamarkupfalse%
\ \isacommand{show}\isamarkupfalse%
\ {\isacharquery}{\kern0pt}thesis\isanewline
\ \ \ \ \isacommand{using}\isamarkupfalse%
\ calculation\ \isacommand{by}\isamarkupfalse%
\ auto\isanewline
\isacommand{qed}\isamarkupfalse%
%
\endisatagproof
{\isafoldproof}%
%
\isadelimproof
\isanewline
%
\endisadelimproof
\isanewline
\isacommand{lemma}\isamarkupfalse%
\ dist{\isacharunderscore}{\kern0pt}prod{\isacharunderscore}{\kern0pt}coprod{\isadigit{2}}{\isacharunderscore}{\kern0pt}right{\isacharunderscore}{\kern0pt}ap{\isacharcolon}{\kern0pt}\isanewline
\ \ \isakeyword{assumes}\ {\isachardoublequoteopen}b\ {\isasymin}\isactrlsub c\ B{\isachardoublequoteclose}\ {\isachardoublequoteopen}c\ {\isasymin}\isactrlsub c\ C{\isachardoublequoteclose}\isanewline
\ \ \isakeyword{shows}\ {\isachardoublequoteopen}dist{\isacharunderscore}{\kern0pt}prod{\isacharunderscore}{\kern0pt}coprod{\isadigit{2}}\ A\ B\ C\ {\isasymcirc}\isactrlsub c\ right{\isacharunderscore}{\kern0pt}coproj\ {\isacharparenleft}{\kern0pt}A\ {\isasymtimes}\isactrlsub c\ C{\isacharparenright}{\kern0pt}\ {\isacharparenleft}{\kern0pt}B\ {\isasymtimes}\isactrlsub c\ C{\isacharparenright}{\kern0pt}\ {\isasymcirc}\isactrlsub c\ {\isasymlangle}b{\isacharcomma}{\kern0pt}\ c{\isasymrangle}\ {\isacharequal}{\kern0pt}\ {\isasymlangle}right{\isacharunderscore}{\kern0pt}coproj\ A\ B\ {\isasymcirc}\isactrlsub c\ b{\isacharcomma}{\kern0pt}\ c{\isasymrangle}{\isachardoublequoteclose}\isanewline
%
\isadelimproof
%
\endisadelimproof
%
\isatagproof
\isacommand{proof}\isamarkupfalse%
\ {\isacharminus}{\kern0pt}\isanewline
\ \ \isacommand{have}\isamarkupfalse%
\ {\isachardoublequoteopen}dist{\isacharunderscore}{\kern0pt}prod{\isacharunderscore}{\kern0pt}coprod{\isadigit{2}}\ A\ B\ C\ {\isasymcirc}\isactrlsub c\ right{\isacharunderscore}{\kern0pt}coproj\ {\isacharparenleft}{\kern0pt}A\ {\isasymtimes}\isactrlsub c\ C{\isacharparenright}{\kern0pt}\ {\isacharparenleft}{\kern0pt}B\ {\isasymtimes}\isactrlsub c\ C{\isacharparenright}{\kern0pt}\ {\isasymcirc}\isactrlsub c\ {\isasymlangle}b{\isacharcomma}{\kern0pt}\ c{\isasymrangle}\isanewline
\ \ \ \ {\isacharequal}{\kern0pt}\ {\isacharparenleft}{\kern0pt}swap\ C\ {\isacharparenleft}{\kern0pt}A\ {\isasymCoprod}\ B{\isacharparenright}{\kern0pt}\ {\isasymcirc}\isactrlsub c\ dist{\isacharunderscore}{\kern0pt}prod{\isacharunderscore}{\kern0pt}coprod\ C\ A\ B\ {\isasymcirc}\isactrlsub c\ {\isacharparenleft}{\kern0pt}swap\ A\ C\ {\isasymbowtie}\isactrlsub f\ swap\ B\ C{\isacharparenright}{\kern0pt}{\isacharparenright}{\kern0pt}\ {\isasymcirc}\isactrlsub c\ {\isacharparenleft}{\kern0pt}right{\isacharunderscore}{\kern0pt}coproj\ {\isacharparenleft}{\kern0pt}A\ {\isasymtimes}\isactrlsub c\ C{\isacharparenright}{\kern0pt}\ {\isacharparenleft}{\kern0pt}B\ {\isasymtimes}\isactrlsub c\ C{\isacharparenright}{\kern0pt}\ {\isasymcirc}\isactrlsub c\ {\isasymlangle}b{\isacharcomma}{\kern0pt}\ c{\isasymrangle}{\isacharparenright}{\kern0pt}{\isachardoublequoteclose}\isanewline
\ \ \ \ \isacommand{unfolding}\isamarkupfalse%
\ dist{\isacharunderscore}{\kern0pt}prod{\isacharunderscore}{\kern0pt}coprod{\isadigit{2}}{\isacharunderscore}{\kern0pt}def\ \isacommand{by}\isamarkupfalse%
\ auto\isanewline
\ \ \isacommand{also}\isamarkupfalse%
\ \isacommand{have}\isamarkupfalse%
\ {\isachardoublequoteopen}{\isachardot}{\kern0pt}{\isachardot}{\kern0pt}{\isachardot}{\kern0pt}\ {\isacharequal}{\kern0pt}\ swap\ C\ {\isacharparenleft}{\kern0pt}A\ {\isasymCoprod}\ B{\isacharparenright}{\kern0pt}\ {\isasymcirc}\isactrlsub c\ dist{\isacharunderscore}{\kern0pt}prod{\isacharunderscore}{\kern0pt}coprod\ C\ A\ B\ {\isasymcirc}\isactrlsub c\ {\isacharparenleft}{\kern0pt}{\isacharparenleft}{\kern0pt}swap\ A\ C\ {\isasymbowtie}\isactrlsub f\ swap\ B\ C{\isacharparenright}{\kern0pt}\ {\isasymcirc}\isactrlsub c\ right{\isacharunderscore}{\kern0pt}coproj\ {\isacharparenleft}{\kern0pt}A\ {\isasymtimes}\isactrlsub c\ C{\isacharparenright}{\kern0pt}\ {\isacharparenleft}{\kern0pt}B\ {\isasymtimes}\isactrlsub c\ C{\isacharparenright}{\kern0pt}{\isacharparenright}{\kern0pt}\ {\isasymcirc}\isactrlsub c\ {\isasymlangle}b{\isacharcomma}{\kern0pt}\ c{\isasymrangle}{\isachardoublequoteclose}\isanewline
\ \ \ \ \isacommand{using}\isamarkupfalse%
\ assms\ \isacommand{by}\isamarkupfalse%
\ {\isacharparenleft}{\kern0pt}typecheck{\isacharunderscore}{\kern0pt}cfuncs{\isacharcomma}{\kern0pt}\ smt\ comp{\isacharunderscore}{\kern0pt}associative{\isadigit{2}}{\isacharparenright}{\kern0pt}\isanewline
\ \ \isacommand{also}\isamarkupfalse%
\ \isacommand{have}\isamarkupfalse%
\ {\isachardoublequoteopen}{\isachardot}{\kern0pt}{\isachardot}{\kern0pt}{\isachardot}{\kern0pt}\ {\isacharequal}{\kern0pt}\ swap\ C\ {\isacharparenleft}{\kern0pt}A\ {\isasymCoprod}\ B{\isacharparenright}{\kern0pt}\ {\isasymcirc}\isactrlsub c\ dist{\isacharunderscore}{\kern0pt}prod{\isacharunderscore}{\kern0pt}coprod\ C\ A\ B\ {\isasymcirc}\isactrlsub c\ {\isacharparenleft}{\kern0pt}right{\isacharunderscore}{\kern0pt}coproj\ {\isacharparenleft}{\kern0pt}C\ {\isasymtimes}\isactrlsub c\ A{\isacharparenright}{\kern0pt}\ {\isacharparenleft}{\kern0pt}C\ {\isasymtimes}\isactrlsub c\ B{\isacharparenright}{\kern0pt}\ {\isasymcirc}\isactrlsub c\ swap\ B\ C{\isacharparenright}{\kern0pt}\ {\isasymcirc}\isactrlsub c\ {\isasymlangle}b{\isacharcomma}{\kern0pt}\ c{\isasymrangle}{\isachardoublequoteclose}\isanewline
\ \ \ \ \isacommand{using}\isamarkupfalse%
\ assms\ \isacommand{by}\isamarkupfalse%
\ {\isacharparenleft}{\kern0pt}typecheck{\isacharunderscore}{\kern0pt}cfuncs{\isacharcomma}{\kern0pt}\ auto\ simp\ add{\isacharcolon}{\kern0pt}\ right{\isacharunderscore}{\kern0pt}coproj{\isacharunderscore}{\kern0pt}cfunc{\isacharunderscore}{\kern0pt}bowtie{\isacharunderscore}{\kern0pt}prod{\isacharparenright}{\kern0pt}\isanewline
\ \ \isacommand{also}\isamarkupfalse%
\ \isacommand{have}\isamarkupfalse%
\ {\isachardoublequoteopen}{\isachardot}{\kern0pt}{\isachardot}{\kern0pt}{\isachardot}{\kern0pt}\ {\isacharequal}{\kern0pt}\ swap\ C\ {\isacharparenleft}{\kern0pt}A\ {\isasymCoprod}\ B{\isacharparenright}{\kern0pt}\ {\isasymcirc}\isactrlsub c\ dist{\isacharunderscore}{\kern0pt}prod{\isacharunderscore}{\kern0pt}coprod\ C\ A\ B\ {\isasymcirc}\isactrlsub c\ right{\isacharunderscore}{\kern0pt}coproj\ {\isacharparenleft}{\kern0pt}C\ {\isasymtimes}\isactrlsub c\ A{\isacharparenright}{\kern0pt}\ {\isacharparenleft}{\kern0pt}C\ {\isasymtimes}\isactrlsub c\ B{\isacharparenright}{\kern0pt}\ {\isasymcirc}\isactrlsub c\ swap\ B\ C\ {\isasymcirc}\isactrlsub c\ {\isasymlangle}b{\isacharcomma}{\kern0pt}\ c{\isasymrangle}{\isachardoublequoteclose}\isanewline
\ \ \ \ \isacommand{using}\isamarkupfalse%
\ assms\ \isacommand{by}\isamarkupfalse%
\ {\isacharparenleft}{\kern0pt}typecheck{\isacharunderscore}{\kern0pt}cfuncs{\isacharcomma}{\kern0pt}\ auto\ simp\ add{\isacharcolon}{\kern0pt}\ comp{\isacharunderscore}{\kern0pt}associative{\isadigit{2}}{\isacharparenright}{\kern0pt}\isanewline
\ \ \isacommand{also}\isamarkupfalse%
\ \isacommand{have}\isamarkupfalse%
\ {\isachardoublequoteopen}{\isachardot}{\kern0pt}{\isachardot}{\kern0pt}{\isachardot}{\kern0pt}\ {\isacharequal}{\kern0pt}\ swap\ C\ {\isacharparenleft}{\kern0pt}A\ {\isasymCoprod}\ B{\isacharparenright}{\kern0pt}\ {\isasymcirc}\isactrlsub c\ dist{\isacharunderscore}{\kern0pt}prod{\isacharunderscore}{\kern0pt}coprod\ C\ A\ B\ {\isasymcirc}\isactrlsub c\ right{\isacharunderscore}{\kern0pt}coproj\ {\isacharparenleft}{\kern0pt}C\ {\isasymtimes}\isactrlsub c\ A{\isacharparenright}{\kern0pt}\ {\isacharparenleft}{\kern0pt}C\ {\isasymtimes}\isactrlsub c\ B{\isacharparenright}{\kern0pt}\ {\isasymcirc}\isactrlsub c\ {\isasymlangle}c{\isacharcomma}{\kern0pt}\ b{\isasymrangle}{\isachardoublequoteclose}\isanewline
\ \ \ \ \isacommand{using}\isamarkupfalse%
\ assms\ swap{\isacharunderscore}{\kern0pt}ap\ \isacommand{by}\isamarkupfalse%
\ {\isacharparenleft}{\kern0pt}typecheck{\isacharunderscore}{\kern0pt}cfuncs{\isacharcomma}{\kern0pt}\ auto{\isacharparenright}{\kern0pt}\isanewline
\ \ \isacommand{also}\isamarkupfalse%
\ \isacommand{have}\isamarkupfalse%
\ {\isachardoublequoteopen}{\isachardot}{\kern0pt}{\isachardot}{\kern0pt}{\isachardot}{\kern0pt}\ {\isacharequal}{\kern0pt}\ swap\ C\ {\isacharparenleft}{\kern0pt}A\ {\isasymCoprod}\ B{\isacharparenright}{\kern0pt}\ {\isasymcirc}\isactrlsub c\ {\isasymlangle}c{\isacharcomma}{\kern0pt}\ right{\isacharunderscore}{\kern0pt}coproj\ A\ B\ {\isasymcirc}\isactrlsub c\ b{\isasymrangle}{\isachardoublequoteclose}\isanewline
\ \ \ \ \isacommand{using}\isamarkupfalse%
\ assms\ \isacommand{by}\isamarkupfalse%
\ {\isacharparenleft}{\kern0pt}typecheck{\isacharunderscore}{\kern0pt}cfuncs{\isacharcomma}{\kern0pt}\ simp\ add{\isacharcolon}{\kern0pt}\ dist{\isacharunderscore}{\kern0pt}prod{\isacharunderscore}{\kern0pt}coprod{\isacharunderscore}{\kern0pt}right{\isacharunderscore}{\kern0pt}ap{\isacharparenright}{\kern0pt}\isanewline
\ \ \isacommand{also}\isamarkupfalse%
\ \isacommand{have}\isamarkupfalse%
\ {\isachardoublequoteopen}{\isachardot}{\kern0pt}{\isachardot}{\kern0pt}{\isachardot}{\kern0pt}\ {\isacharequal}{\kern0pt}\ {\isasymlangle}right{\isacharunderscore}{\kern0pt}coproj\ A\ B\ {\isasymcirc}\isactrlsub c\ b{\isacharcomma}{\kern0pt}\ c{\isasymrangle}{\isachardoublequoteclose}\isanewline
\ \ \ \ \isacommand{using}\isamarkupfalse%
\ assms\ swap{\isacharunderscore}{\kern0pt}ap\ \isacommand{by}\isamarkupfalse%
\ {\isacharparenleft}{\kern0pt}typecheck{\isacharunderscore}{\kern0pt}cfuncs{\isacharcomma}{\kern0pt}\ auto{\isacharparenright}{\kern0pt}\isanewline
\ \ \isacommand{then}\isamarkupfalse%
\ \isacommand{show}\isamarkupfalse%
\ {\isacharquery}{\kern0pt}thesis\isanewline
\ \ \ \ \isacommand{using}\isamarkupfalse%
\ calculation\ \isacommand{by}\isamarkupfalse%
\ auto\isanewline
\isacommand{qed}\isamarkupfalse%
%
\endisatagproof
{\isafoldproof}%
%
\isadelimproof
%
\endisadelimproof
%
\isadelimdocument
%
\endisadelimdocument
%
\isatagdocument
%
\isamarkupsubsubsection{Inverse Distribute Product Over Coproduct Auxillary Mapping 2%
}
\isamarkuptrue%
%
\endisatagdocument
{\isafolddocument}%
%
\isadelimdocument
%
\endisadelimdocument
\isacommand{definition}\isamarkupfalse%
\ dist{\isacharunderscore}{\kern0pt}prod{\isacharunderscore}{\kern0pt}coprod{\isacharunderscore}{\kern0pt}inv{\isadigit{2}}\ {\isacharcolon}{\kern0pt}{\isacharcolon}{\kern0pt}\ {\isachardoublequoteopen}cset\ {\isasymRightarrow}\ cset\ {\isasymRightarrow}\ cset\ {\isasymRightarrow}\ cfunc{\isachardoublequoteclose}\ \isakeyword{where}\isanewline
\ \ {\isachardoublequoteopen}dist{\isacharunderscore}{\kern0pt}prod{\isacharunderscore}{\kern0pt}coprod{\isacharunderscore}{\kern0pt}inv{\isadigit{2}}\ A\ B\ C\ {\isacharequal}{\kern0pt}\ {\isacharparenleft}{\kern0pt}swap\ C\ A\ {\isasymbowtie}\isactrlsub f\ swap\ C\ B{\isacharparenright}{\kern0pt}\ {\isasymcirc}\isactrlsub c\ dist{\isacharunderscore}{\kern0pt}prod{\isacharunderscore}{\kern0pt}coprod{\isacharunderscore}{\kern0pt}inv\ C\ A\ B\ {\isasymcirc}\isactrlsub c\ swap\ {\isacharparenleft}{\kern0pt}A\ {\isasymCoprod}\ B{\isacharparenright}{\kern0pt}\ C{\isachardoublequoteclose}\isanewline
\isanewline
\isacommand{lemma}\isamarkupfalse%
\ dist{\isacharunderscore}{\kern0pt}prod{\isacharunderscore}{\kern0pt}coprod{\isacharunderscore}{\kern0pt}inv{\isadigit{2}}{\isacharunderscore}{\kern0pt}type{\isacharbrackleft}{\kern0pt}type{\isacharunderscore}{\kern0pt}rule{\isacharbrackright}{\kern0pt}{\isacharcolon}{\kern0pt}\isanewline
\ \ {\isachardoublequoteopen}dist{\isacharunderscore}{\kern0pt}prod{\isacharunderscore}{\kern0pt}coprod{\isacharunderscore}{\kern0pt}inv{\isadigit{2}}\ A\ B\ C\ {\isacharcolon}{\kern0pt}\ {\isacharparenleft}{\kern0pt}A\ {\isasymCoprod}\ B{\isacharparenright}{\kern0pt}\ {\isasymtimes}\isactrlsub c\ C\ {\isasymrightarrow}\ {\isacharparenleft}{\kern0pt}A\ {\isasymtimes}\isactrlsub c\ C{\isacharparenright}{\kern0pt}\ {\isasymCoprod}\ {\isacharparenleft}{\kern0pt}B\ {\isasymtimes}\isactrlsub c\ C{\isacharparenright}{\kern0pt}{\isachardoublequoteclose}\isanewline
%
\isadelimproof
\ \ %
\endisadelimproof
%
\isatagproof
\isacommand{unfolding}\isamarkupfalse%
\ dist{\isacharunderscore}{\kern0pt}prod{\isacharunderscore}{\kern0pt}coprod{\isacharunderscore}{\kern0pt}inv{\isadigit{2}}{\isacharunderscore}{\kern0pt}def\ \isacommand{by}\isamarkupfalse%
\ typecheck{\isacharunderscore}{\kern0pt}cfuncs%
\endisatagproof
{\isafoldproof}%
%
\isadelimproof
\isanewline
%
\endisadelimproof
\isanewline
\isacommand{lemma}\isamarkupfalse%
\ dist{\isacharunderscore}{\kern0pt}prod{\isacharunderscore}{\kern0pt}coprod{\isacharunderscore}{\kern0pt}inv{\isadigit{2}}{\isacharunderscore}{\kern0pt}left{\isacharunderscore}{\kern0pt}ap{\isacharcolon}{\kern0pt}\isanewline
\ \ \isakeyword{assumes}\ {\isachardoublequoteopen}a\ {\isasymin}\isactrlsub c\ A{\isachardoublequoteclose}\ {\isachardoublequoteopen}c\ {\isasymin}\isactrlsub c\ C{\isachardoublequoteclose}\isanewline
\ \ \isakeyword{shows}\ {\isachardoublequoteopen}dist{\isacharunderscore}{\kern0pt}prod{\isacharunderscore}{\kern0pt}coprod{\isacharunderscore}{\kern0pt}inv{\isadigit{2}}\ A\ B\ C\ {\isasymcirc}\isactrlsub c\ {\isasymlangle}left{\isacharunderscore}{\kern0pt}coproj\ A\ B\ {\isasymcirc}\isactrlsub c\ a{\isacharcomma}{\kern0pt}\ c{\isasymrangle}\ {\isacharequal}{\kern0pt}\ left{\isacharunderscore}{\kern0pt}coproj\ {\isacharparenleft}{\kern0pt}A\ {\isasymtimes}\isactrlsub c\ C{\isacharparenright}{\kern0pt}\ {\isacharparenleft}{\kern0pt}B\ {\isasymtimes}\isactrlsub c\ C{\isacharparenright}{\kern0pt}\ {\isasymcirc}\isactrlsub c\ {\isasymlangle}a{\isacharcomma}{\kern0pt}\ c{\isasymrangle}{\isachardoublequoteclose}\isanewline
%
\isadelimproof
%
\endisadelimproof
%
\isatagproof
\isacommand{proof}\isamarkupfalse%
\ {\isacharminus}{\kern0pt}\isanewline
\ \ \isacommand{have}\isamarkupfalse%
\ {\isachardoublequoteopen}dist{\isacharunderscore}{\kern0pt}prod{\isacharunderscore}{\kern0pt}coprod{\isacharunderscore}{\kern0pt}inv{\isadigit{2}}\ A\ B\ C\ {\isasymcirc}\isactrlsub c\ {\isasymlangle}left{\isacharunderscore}{\kern0pt}coproj\ A\ B\ {\isasymcirc}\isactrlsub c\ a{\isacharcomma}{\kern0pt}\ c{\isasymrangle}\isanewline
\ \ \ \ {\isacharequal}{\kern0pt}\ {\isacharparenleft}{\kern0pt}{\isacharparenleft}{\kern0pt}swap\ C\ A\ {\isasymbowtie}\isactrlsub f\ swap\ C\ B{\isacharparenright}{\kern0pt}\ {\isasymcirc}\isactrlsub c\ dist{\isacharunderscore}{\kern0pt}prod{\isacharunderscore}{\kern0pt}coprod{\isacharunderscore}{\kern0pt}inv\ C\ A\ B\ {\isasymcirc}\isactrlsub c\ swap\ {\isacharparenleft}{\kern0pt}A\ {\isasymCoprod}\ B{\isacharparenright}{\kern0pt}\ C{\isacharparenright}{\kern0pt}\ {\isasymcirc}\isactrlsub c\ {\isasymlangle}left{\isacharunderscore}{\kern0pt}coproj\ A\ B\ {\isasymcirc}\isactrlsub c\ a{\isacharcomma}{\kern0pt}\ c{\isasymrangle}{\isachardoublequoteclose}\isanewline
\ \ \ \ \isacommand{unfolding}\isamarkupfalse%
\ dist{\isacharunderscore}{\kern0pt}prod{\isacharunderscore}{\kern0pt}coprod{\isacharunderscore}{\kern0pt}inv{\isadigit{2}}{\isacharunderscore}{\kern0pt}def\ \isacommand{by}\isamarkupfalse%
\ auto\isanewline
\ \ \isacommand{also}\isamarkupfalse%
\ \isacommand{have}\isamarkupfalse%
\ {\isachardoublequoteopen}{\isachardot}{\kern0pt}{\isachardot}{\kern0pt}{\isachardot}{\kern0pt}\ {\isacharequal}{\kern0pt}\ {\isacharparenleft}{\kern0pt}swap\ C\ A\ {\isasymbowtie}\isactrlsub f\ swap\ C\ B{\isacharparenright}{\kern0pt}\ {\isasymcirc}\isactrlsub c\ dist{\isacharunderscore}{\kern0pt}prod{\isacharunderscore}{\kern0pt}coprod{\isacharunderscore}{\kern0pt}inv\ C\ A\ B\ {\isasymcirc}\isactrlsub c\ swap\ {\isacharparenleft}{\kern0pt}A\ {\isasymCoprod}\ B{\isacharparenright}{\kern0pt}\ C\ {\isasymcirc}\isactrlsub c\ {\isasymlangle}left{\isacharunderscore}{\kern0pt}coproj\ A\ B\ {\isasymcirc}\isactrlsub c\ a{\isacharcomma}{\kern0pt}\ c{\isasymrangle}{\isachardoublequoteclose}\isanewline
\ \ \ \ \isacommand{using}\isamarkupfalse%
\ assms\ \isacommand{by}\isamarkupfalse%
\ {\isacharparenleft}{\kern0pt}typecheck{\isacharunderscore}{\kern0pt}cfuncs{\isacharcomma}{\kern0pt}\ smt\ comp{\isacharunderscore}{\kern0pt}associative{\isadigit{2}}{\isacharparenright}{\kern0pt}\isanewline
\ \ \isacommand{also}\isamarkupfalse%
\ \isacommand{have}\isamarkupfalse%
\ {\isachardoublequoteopen}{\isachardot}{\kern0pt}{\isachardot}{\kern0pt}{\isachardot}{\kern0pt}\ {\isacharequal}{\kern0pt}\ {\isacharparenleft}{\kern0pt}swap\ C\ A\ {\isasymbowtie}\isactrlsub f\ swap\ C\ B{\isacharparenright}{\kern0pt}\ {\isasymcirc}\isactrlsub c\ dist{\isacharunderscore}{\kern0pt}prod{\isacharunderscore}{\kern0pt}coprod{\isacharunderscore}{\kern0pt}inv\ C\ A\ B\ {\isasymcirc}\isactrlsub c\ {\isasymlangle}c{\isacharcomma}{\kern0pt}\ left{\isacharunderscore}{\kern0pt}coproj\ A\ B\ {\isasymcirc}\isactrlsub c\ a{\isasymrangle}{\isachardoublequoteclose}\isanewline
\ \ \ \ \isacommand{using}\isamarkupfalse%
\ assms\ swap{\isacharunderscore}{\kern0pt}ap\ \isacommand{by}\isamarkupfalse%
\ {\isacharparenleft}{\kern0pt}typecheck{\isacharunderscore}{\kern0pt}cfuncs{\isacharcomma}{\kern0pt}\ auto{\isacharparenright}{\kern0pt}\isanewline
\ \ \isacommand{also}\isamarkupfalse%
\ \isacommand{have}\isamarkupfalse%
\ {\isachardoublequoteopen}{\isachardot}{\kern0pt}{\isachardot}{\kern0pt}{\isachardot}{\kern0pt}\ {\isacharequal}{\kern0pt}\ {\isacharparenleft}{\kern0pt}swap\ C\ A\ {\isasymbowtie}\isactrlsub f\ swap\ C\ B{\isacharparenright}{\kern0pt}\ {\isasymcirc}\isactrlsub c\ left{\isacharunderscore}{\kern0pt}coproj\ {\isacharparenleft}{\kern0pt}C\ {\isasymtimes}\isactrlsub c\ A{\isacharparenright}{\kern0pt}\ {\isacharparenleft}{\kern0pt}C\ {\isasymtimes}\isactrlsub c\ B{\isacharparenright}{\kern0pt}\ {\isasymcirc}\isactrlsub c\ {\isasymlangle}c{\isacharcomma}{\kern0pt}\ a{\isasymrangle}{\isachardoublequoteclose}\isanewline
\ \ \ \ \isacommand{using}\isamarkupfalse%
\ assms\ \isacommand{by}\isamarkupfalse%
\ {\isacharparenleft}{\kern0pt}typecheck{\isacharunderscore}{\kern0pt}cfuncs{\isacharcomma}{\kern0pt}\ simp\ add{\isacharcolon}{\kern0pt}\ dist{\isacharunderscore}{\kern0pt}prod{\isacharunderscore}{\kern0pt}coprod{\isacharunderscore}{\kern0pt}inv{\isacharunderscore}{\kern0pt}left{\isacharunderscore}{\kern0pt}ap{\isacharparenright}{\kern0pt}\isanewline
\ \ \isacommand{also}\isamarkupfalse%
\ \isacommand{have}\isamarkupfalse%
\ {\isachardoublequoteopen}{\isachardot}{\kern0pt}{\isachardot}{\kern0pt}{\isachardot}{\kern0pt}\ {\isacharequal}{\kern0pt}\ {\isacharparenleft}{\kern0pt}{\isacharparenleft}{\kern0pt}swap\ C\ A\ {\isasymbowtie}\isactrlsub f\ swap\ C\ B{\isacharparenright}{\kern0pt}\ {\isasymcirc}\isactrlsub c\ left{\isacharunderscore}{\kern0pt}coproj\ {\isacharparenleft}{\kern0pt}C\ {\isasymtimes}\isactrlsub c\ A{\isacharparenright}{\kern0pt}\ {\isacharparenleft}{\kern0pt}C\ {\isasymtimes}\isactrlsub c\ B{\isacharparenright}{\kern0pt}{\isacharparenright}{\kern0pt}\ {\isasymcirc}\isactrlsub c\ {\isasymlangle}c{\isacharcomma}{\kern0pt}\ a{\isasymrangle}{\isachardoublequoteclose}\isanewline
\ \ \ \ \isacommand{using}\isamarkupfalse%
\ assms\ \isacommand{by}\isamarkupfalse%
\ {\isacharparenleft}{\kern0pt}typecheck{\isacharunderscore}{\kern0pt}cfuncs{\isacharcomma}{\kern0pt}\ smt\ comp{\isacharunderscore}{\kern0pt}associative{\isadigit{2}}{\isacharparenright}{\kern0pt}\isanewline
\ \ \isacommand{also}\isamarkupfalse%
\ \isacommand{have}\isamarkupfalse%
\ {\isachardoublequoteopen}{\isachardot}{\kern0pt}{\isachardot}{\kern0pt}{\isachardot}{\kern0pt}\ {\isacharequal}{\kern0pt}\ {\isacharparenleft}{\kern0pt}left{\isacharunderscore}{\kern0pt}coproj\ {\isacharparenleft}{\kern0pt}A\ {\isasymtimes}\isactrlsub c\ C{\isacharparenright}{\kern0pt}\ {\isacharparenleft}{\kern0pt}B\ {\isasymtimes}\isactrlsub c\ C{\isacharparenright}{\kern0pt}\ {\isasymcirc}\isactrlsub c\ swap\ C\ A{\isacharparenright}{\kern0pt}\ {\isasymcirc}\isactrlsub c\ {\isasymlangle}c{\isacharcomma}{\kern0pt}\ a{\isasymrangle}{\isachardoublequoteclose}\isanewline
\ \ \ \ \isacommand{using}\isamarkupfalse%
\ assms\ left{\isacharunderscore}{\kern0pt}coproj{\isacharunderscore}{\kern0pt}cfunc{\isacharunderscore}{\kern0pt}bowtie{\isacharunderscore}{\kern0pt}prod\ \isacommand{by}\isamarkupfalse%
\ {\isacharparenleft}{\kern0pt}typecheck{\isacharunderscore}{\kern0pt}cfuncs{\isacharcomma}{\kern0pt}\ auto{\isacharparenright}{\kern0pt}\isanewline
\ \ \isacommand{also}\isamarkupfalse%
\ \isacommand{have}\isamarkupfalse%
\ {\isachardoublequoteopen}{\isachardot}{\kern0pt}{\isachardot}{\kern0pt}{\isachardot}{\kern0pt}\ {\isacharequal}{\kern0pt}\ left{\isacharunderscore}{\kern0pt}coproj\ {\isacharparenleft}{\kern0pt}A\ {\isasymtimes}\isactrlsub c\ C{\isacharparenright}{\kern0pt}\ {\isacharparenleft}{\kern0pt}B\ {\isasymtimes}\isactrlsub c\ C{\isacharparenright}{\kern0pt}\ {\isasymcirc}\isactrlsub c\ swap\ C\ A\ {\isasymcirc}\isactrlsub c\ {\isasymlangle}c{\isacharcomma}{\kern0pt}\ a{\isasymrangle}{\isachardoublequoteclose}\isanewline
\ \ \ \ \isacommand{using}\isamarkupfalse%
\ assms\ \isacommand{by}\isamarkupfalse%
\ {\isacharparenleft}{\kern0pt}typecheck{\isacharunderscore}{\kern0pt}cfuncs{\isacharcomma}{\kern0pt}\ smt\ comp{\isacharunderscore}{\kern0pt}associative{\isadigit{2}}{\isacharparenright}{\kern0pt}\isanewline
\ \ \isacommand{also}\isamarkupfalse%
\ \isacommand{have}\isamarkupfalse%
\ {\isachardoublequoteopen}{\isachardot}{\kern0pt}{\isachardot}{\kern0pt}{\isachardot}{\kern0pt}\ {\isacharequal}{\kern0pt}\ left{\isacharunderscore}{\kern0pt}coproj\ {\isacharparenleft}{\kern0pt}A\ {\isasymtimes}\isactrlsub c\ C{\isacharparenright}{\kern0pt}\ {\isacharparenleft}{\kern0pt}B\ {\isasymtimes}\isactrlsub c\ C{\isacharparenright}{\kern0pt}\ {\isasymcirc}\isactrlsub c\ {\isasymlangle}a{\isacharcomma}{\kern0pt}\ c{\isasymrangle}{\isachardoublequoteclose}\isanewline
\ \ \ \ \isacommand{using}\isamarkupfalse%
\ assms\ swap{\isacharunderscore}{\kern0pt}ap\ \isacommand{by}\isamarkupfalse%
\ {\isacharparenleft}{\kern0pt}typecheck{\isacharunderscore}{\kern0pt}cfuncs{\isacharcomma}{\kern0pt}\ auto{\isacharparenright}{\kern0pt}\isanewline
\ \ \isacommand{then}\isamarkupfalse%
\ \isacommand{show}\isamarkupfalse%
\ {\isacharquery}{\kern0pt}thesis\isanewline
\ \ \ \ \isacommand{using}\isamarkupfalse%
\ calculation\ \isacommand{by}\isamarkupfalse%
\ auto\isanewline
\isacommand{qed}\isamarkupfalse%
%
\endisatagproof
{\isafoldproof}%
%
\isadelimproof
\isanewline
%
\endisadelimproof
\isanewline
\isacommand{lemma}\isamarkupfalse%
\ dist{\isacharunderscore}{\kern0pt}prod{\isacharunderscore}{\kern0pt}coprod{\isacharunderscore}{\kern0pt}inv{\isadigit{2}}{\isacharunderscore}{\kern0pt}right{\isacharunderscore}{\kern0pt}ap{\isacharcolon}{\kern0pt}\isanewline
\ \ \isakeyword{assumes}\ {\isachardoublequoteopen}b\ {\isasymin}\isactrlsub c\ B{\isachardoublequoteclose}\ {\isachardoublequoteopen}c\ {\isasymin}\isactrlsub c\ C{\isachardoublequoteclose}\isanewline
\ \ \isakeyword{shows}\ {\isachardoublequoteopen}dist{\isacharunderscore}{\kern0pt}prod{\isacharunderscore}{\kern0pt}coprod{\isacharunderscore}{\kern0pt}inv{\isadigit{2}}\ A\ B\ C\ {\isasymcirc}\isactrlsub c\ {\isasymlangle}right{\isacharunderscore}{\kern0pt}coproj\ A\ B\ {\isasymcirc}\isactrlsub c\ b{\isacharcomma}{\kern0pt}\ c{\isasymrangle}\ {\isacharequal}{\kern0pt}\ right{\isacharunderscore}{\kern0pt}coproj\ {\isacharparenleft}{\kern0pt}A\ {\isasymtimes}\isactrlsub c\ C{\isacharparenright}{\kern0pt}\ {\isacharparenleft}{\kern0pt}B\ {\isasymtimes}\isactrlsub c\ C{\isacharparenright}{\kern0pt}\ {\isasymcirc}\isactrlsub c\ {\isasymlangle}b{\isacharcomma}{\kern0pt}\ c{\isasymrangle}{\isachardoublequoteclose}\isanewline
%
\isadelimproof
%
\endisadelimproof
%
\isatagproof
\isacommand{proof}\isamarkupfalse%
\ {\isacharminus}{\kern0pt}\isanewline
\ \ \isacommand{have}\isamarkupfalse%
\ {\isachardoublequoteopen}dist{\isacharunderscore}{\kern0pt}prod{\isacharunderscore}{\kern0pt}coprod{\isacharunderscore}{\kern0pt}inv{\isadigit{2}}\ A\ B\ C\ {\isasymcirc}\isactrlsub c\ {\isasymlangle}right{\isacharunderscore}{\kern0pt}coproj\ A\ B\ {\isasymcirc}\isactrlsub c\ b{\isacharcomma}{\kern0pt}\ c{\isasymrangle}\isanewline
\ \ \ \ {\isacharequal}{\kern0pt}\ {\isacharparenleft}{\kern0pt}{\isacharparenleft}{\kern0pt}swap\ C\ A\ {\isasymbowtie}\isactrlsub f\ swap\ C\ B{\isacharparenright}{\kern0pt}\ {\isasymcirc}\isactrlsub c\ dist{\isacharunderscore}{\kern0pt}prod{\isacharunderscore}{\kern0pt}coprod{\isacharunderscore}{\kern0pt}inv\ C\ A\ B\ {\isasymcirc}\isactrlsub c\ swap\ {\isacharparenleft}{\kern0pt}A\ {\isasymCoprod}\ B{\isacharparenright}{\kern0pt}\ C{\isacharparenright}{\kern0pt}\ {\isasymcirc}\isactrlsub c\ {\isasymlangle}right{\isacharunderscore}{\kern0pt}coproj\ A\ B\ {\isasymcirc}\isactrlsub c\ b{\isacharcomma}{\kern0pt}\ c{\isasymrangle}{\isachardoublequoteclose}\isanewline
\ \ \ \ \isacommand{unfolding}\isamarkupfalse%
\ dist{\isacharunderscore}{\kern0pt}prod{\isacharunderscore}{\kern0pt}coprod{\isacharunderscore}{\kern0pt}inv{\isadigit{2}}{\isacharunderscore}{\kern0pt}def\ \isacommand{by}\isamarkupfalse%
\ auto\isanewline
\ \ \isacommand{also}\isamarkupfalse%
\ \isacommand{have}\isamarkupfalse%
\ {\isachardoublequoteopen}{\isachardot}{\kern0pt}{\isachardot}{\kern0pt}{\isachardot}{\kern0pt}\ {\isacharequal}{\kern0pt}\ {\isacharparenleft}{\kern0pt}swap\ C\ A\ {\isasymbowtie}\isactrlsub f\ swap\ C\ B{\isacharparenright}{\kern0pt}\ {\isasymcirc}\isactrlsub c\ dist{\isacharunderscore}{\kern0pt}prod{\isacharunderscore}{\kern0pt}coprod{\isacharunderscore}{\kern0pt}inv\ C\ A\ B\ {\isasymcirc}\isactrlsub c\ swap\ {\isacharparenleft}{\kern0pt}A\ {\isasymCoprod}\ B{\isacharparenright}{\kern0pt}\ C\ {\isasymcirc}\isactrlsub c\ {\isasymlangle}right{\isacharunderscore}{\kern0pt}coproj\ A\ B\ {\isasymcirc}\isactrlsub c\ b{\isacharcomma}{\kern0pt}\ c{\isasymrangle}{\isachardoublequoteclose}\isanewline
\ \ \ \ \isacommand{using}\isamarkupfalse%
\ assms\ \isacommand{by}\isamarkupfalse%
\ {\isacharparenleft}{\kern0pt}typecheck{\isacharunderscore}{\kern0pt}cfuncs{\isacharcomma}{\kern0pt}\ smt\ comp{\isacharunderscore}{\kern0pt}associative{\isadigit{2}}{\isacharparenright}{\kern0pt}\isanewline
\ \ \isacommand{also}\isamarkupfalse%
\ \isacommand{have}\isamarkupfalse%
\ {\isachardoublequoteopen}{\isachardot}{\kern0pt}{\isachardot}{\kern0pt}{\isachardot}{\kern0pt}\ {\isacharequal}{\kern0pt}\ {\isacharparenleft}{\kern0pt}swap\ C\ A\ {\isasymbowtie}\isactrlsub f\ swap\ C\ B{\isacharparenright}{\kern0pt}\ {\isasymcirc}\isactrlsub c\ dist{\isacharunderscore}{\kern0pt}prod{\isacharunderscore}{\kern0pt}coprod{\isacharunderscore}{\kern0pt}inv\ C\ A\ B\ {\isasymcirc}\isactrlsub c\ {\isasymlangle}c{\isacharcomma}{\kern0pt}\ right{\isacharunderscore}{\kern0pt}coproj\ A\ B\ {\isasymcirc}\isactrlsub c\ b{\isasymrangle}{\isachardoublequoteclose}\isanewline
\ \ \ \ \isacommand{using}\isamarkupfalse%
\ assms\ swap{\isacharunderscore}{\kern0pt}ap\ \isacommand{by}\isamarkupfalse%
\ {\isacharparenleft}{\kern0pt}typecheck{\isacharunderscore}{\kern0pt}cfuncs{\isacharcomma}{\kern0pt}\ auto{\isacharparenright}{\kern0pt}\isanewline
\ \ \isacommand{also}\isamarkupfalse%
\ \isacommand{have}\isamarkupfalse%
\ {\isachardoublequoteopen}{\isachardot}{\kern0pt}{\isachardot}{\kern0pt}{\isachardot}{\kern0pt}\ {\isacharequal}{\kern0pt}\ {\isacharparenleft}{\kern0pt}swap\ C\ A\ {\isasymbowtie}\isactrlsub f\ swap\ C\ B{\isacharparenright}{\kern0pt}\ {\isasymcirc}\isactrlsub c\ right{\isacharunderscore}{\kern0pt}coproj\ {\isacharparenleft}{\kern0pt}C\ {\isasymtimes}\isactrlsub c\ A{\isacharparenright}{\kern0pt}\ {\isacharparenleft}{\kern0pt}C\ {\isasymtimes}\isactrlsub c\ B{\isacharparenright}{\kern0pt}\ {\isasymcirc}\isactrlsub c\ {\isasymlangle}c{\isacharcomma}{\kern0pt}\ b{\isasymrangle}{\isachardoublequoteclose}\isanewline
\ \ \ \ \isacommand{using}\isamarkupfalse%
\ assms\ \isacommand{by}\isamarkupfalse%
\ {\isacharparenleft}{\kern0pt}typecheck{\isacharunderscore}{\kern0pt}cfuncs{\isacharcomma}{\kern0pt}\ simp\ add{\isacharcolon}{\kern0pt}\ dist{\isacharunderscore}{\kern0pt}prod{\isacharunderscore}{\kern0pt}coprod{\isacharunderscore}{\kern0pt}inv{\isacharunderscore}{\kern0pt}right{\isacharunderscore}{\kern0pt}ap{\isacharparenright}{\kern0pt}\isanewline
\ \ \isacommand{also}\isamarkupfalse%
\ \isacommand{have}\isamarkupfalse%
\ {\isachardoublequoteopen}{\isachardot}{\kern0pt}{\isachardot}{\kern0pt}{\isachardot}{\kern0pt}\ {\isacharequal}{\kern0pt}\ {\isacharparenleft}{\kern0pt}{\isacharparenleft}{\kern0pt}swap\ C\ A\ {\isasymbowtie}\isactrlsub f\ swap\ C\ B{\isacharparenright}{\kern0pt}\ {\isasymcirc}\isactrlsub c\ right{\isacharunderscore}{\kern0pt}coproj\ {\isacharparenleft}{\kern0pt}C\ {\isasymtimes}\isactrlsub c\ A{\isacharparenright}{\kern0pt}\ {\isacharparenleft}{\kern0pt}C\ {\isasymtimes}\isactrlsub c\ B{\isacharparenright}{\kern0pt}{\isacharparenright}{\kern0pt}\ {\isasymcirc}\isactrlsub c\ {\isasymlangle}c{\isacharcomma}{\kern0pt}\ b{\isasymrangle}{\isachardoublequoteclose}\isanewline
\ \ \ \ \isacommand{using}\isamarkupfalse%
\ assms\ \isacommand{by}\isamarkupfalse%
\ {\isacharparenleft}{\kern0pt}typecheck{\isacharunderscore}{\kern0pt}cfuncs{\isacharcomma}{\kern0pt}\ auto\ simp\ add{\isacharcolon}{\kern0pt}\ comp{\isacharunderscore}{\kern0pt}associative{\isadigit{2}}{\isacharparenright}{\kern0pt}\isanewline
\ \ \isacommand{also}\isamarkupfalse%
\ \isacommand{have}\isamarkupfalse%
\ {\isachardoublequoteopen}{\isachardot}{\kern0pt}{\isachardot}{\kern0pt}{\isachardot}{\kern0pt}\ {\isacharequal}{\kern0pt}\ {\isacharparenleft}{\kern0pt}right{\isacharunderscore}{\kern0pt}coproj\ {\isacharparenleft}{\kern0pt}A\ {\isasymtimes}\isactrlsub c\ C{\isacharparenright}{\kern0pt}\ {\isacharparenleft}{\kern0pt}B\ {\isasymtimes}\isactrlsub c\ C{\isacharparenright}{\kern0pt}\ {\isasymcirc}\isactrlsub c\ swap\ C\ B{\isacharparenright}{\kern0pt}\ {\isasymcirc}\isactrlsub c\ {\isasymlangle}c{\isacharcomma}{\kern0pt}\ b{\isasymrangle}{\isachardoublequoteclose}\isanewline
\ \ \ \ \isacommand{using}\isamarkupfalse%
\ assms\ \isacommand{by}\isamarkupfalse%
\ {\isacharparenleft}{\kern0pt}typecheck{\isacharunderscore}{\kern0pt}cfuncs{\isacharcomma}{\kern0pt}\ auto\ simp\ add{\isacharcolon}{\kern0pt}\ right{\isacharunderscore}{\kern0pt}coproj{\isacharunderscore}{\kern0pt}cfunc{\isacharunderscore}{\kern0pt}bowtie{\isacharunderscore}{\kern0pt}prod{\isacharparenright}{\kern0pt}\isanewline
\ \ \isacommand{also}\isamarkupfalse%
\ \isacommand{have}\isamarkupfalse%
\ {\isachardoublequoteopen}{\isachardot}{\kern0pt}{\isachardot}{\kern0pt}{\isachardot}{\kern0pt}\ {\isacharequal}{\kern0pt}\ right{\isacharunderscore}{\kern0pt}coproj\ {\isacharparenleft}{\kern0pt}A\ {\isasymtimes}\isactrlsub c\ C{\isacharparenright}{\kern0pt}\ {\isacharparenleft}{\kern0pt}B\ {\isasymtimes}\isactrlsub c\ C{\isacharparenright}{\kern0pt}\ {\isasymcirc}\isactrlsub c\ swap\ C\ B\ {\isasymcirc}\isactrlsub c\ {\isasymlangle}c{\isacharcomma}{\kern0pt}\ b{\isasymrangle}{\isachardoublequoteclose}\isanewline
\ \ \ \ \isacommand{using}\isamarkupfalse%
\ assms\ \isacommand{by}\isamarkupfalse%
\ {\isacharparenleft}{\kern0pt}typecheck{\isacharunderscore}{\kern0pt}cfuncs{\isacharcomma}{\kern0pt}\ auto\ simp\ add{\isacharcolon}{\kern0pt}\ comp{\isacharunderscore}{\kern0pt}associative{\isadigit{2}}{\isacharparenright}{\kern0pt}\isanewline
\ \ \isacommand{also}\isamarkupfalse%
\ \isacommand{have}\isamarkupfalse%
\ {\isachardoublequoteopen}{\isachardot}{\kern0pt}{\isachardot}{\kern0pt}{\isachardot}{\kern0pt}\ {\isacharequal}{\kern0pt}\ right{\isacharunderscore}{\kern0pt}coproj\ {\isacharparenleft}{\kern0pt}A\ {\isasymtimes}\isactrlsub c\ C{\isacharparenright}{\kern0pt}\ {\isacharparenleft}{\kern0pt}B\ {\isasymtimes}\isactrlsub c\ C{\isacharparenright}{\kern0pt}\ {\isasymcirc}\isactrlsub c\ {\isasymlangle}b{\isacharcomma}{\kern0pt}\ c{\isasymrangle}{\isachardoublequoteclose}\isanewline
\ \ \ \ \isacommand{using}\isamarkupfalse%
\ assms\ swap{\isacharunderscore}{\kern0pt}ap\ \isacommand{by}\isamarkupfalse%
\ {\isacharparenleft}{\kern0pt}typecheck{\isacharunderscore}{\kern0pt}cfuncs{\isacharcomma}{\kern0pt}\ auto{\isacharparenright}{\kern0pt}\isanewline
\ \ \isacommand{then}\isamarkupfalse%
\ \isacommand{show}\isamarkupfalse%
\ {\isacharquery}{\kern0pt}thesis\isanewline
\ \ \ \ \isacommand{using}\isamarkupfalse%
\ calculation\ \isacommand{by}\isamarkupfalse%
\ auto\isanewline
\isacommand{qed}\isamarkupfalse%
%
\endisatagproof
{\isafoldproof}%
%
\isadelimproof
\isanewline
%
\endisadelimproof
\isanewline
\isacommand{lemma}\isamarkupfalse%
\ dist{\isacharunderscore}{\kern0pt}prod{\isacharunderscore}{\kern0pt}coprod{\isacharunderscore}{\kern0pt}inv{\isadigit{2}}{\isacharunderscore}{\kern0pt}left{\isacharunderscore}{\kern0pt}coproj{\isacharcolon}{\kern0pt}\isanewline
\ \ {\isachardoublequoteopen}dist{\isacharunderscore}{\kern0pt}prod{\isacharunderscore}{\kern0pt}coprod{\isacharunderscore}{\kern0pt}inv{\isadigit{2}}\ X\ Y\ H\ {\isasymcirc}\isactrlsub c\ {\isacharparenleft}{\kern0pt}left{\isacharunderscore}{\kern0pt}coproj\ X\ Y\ {\isasymtimes}\isactrlsub f\ id\ H{\isacharparenright}{\kern0pt}\ {\isacharequal}{\kern0pt}\ left{\isacharunderscore}{\kern0pt}coproj\ {\isacharparenleft}{\kern0pt}X\ {\isasymtimes}\isactrlsub c\ H{\isacharparenright}{\kern0pt}\ {\isacharparenleft}{\kern0pt}Y\ {\isasymtimes}\isactrlsub c\ H{\isacharparenright}{\kern0pt}{\isachardoublequoteclose}\isanewline
%
\isadelimproof
\ \ %
\endisadelimproof
%
\isatagproof
\isacommand{by}\isamarkupfalse%
\ {\isacharparenleft}{\kern0pt}typecheck{\isacharunderscore}{\kern0pt}cfuncs{\isacharcomma}{\kern0pt}\ smt\ {\isacharparenleft}{\kern0pt}z{\isadigit{3}}{\isacharparenright}{\kern0pt}\ one{\isacharunderscore}{\kern0pt}separator\ cart{\isacharunderscore}{\kern0pt}prod{\isacharunderscore}{\kern0pt}decomp\ cfunc{\isacharunderscore}{\kern0pt}cross{\isacharunderscore}{\kern0pt}prod{\isacharunderscore}{\kern0pt}comp{\isacharunderscore}{\kern0pt}cfunc{\isacharunderscore}{\kern0pt}prod\ comp{\isacharunderscore}{\kern0pt}associative{\isadigit{2}}\ dist{\isacharunderscore}{\kern0pt}prod{\isacharunderscore}{\kern0pt}coprod{\isacharunderscore}{\kern0pt}inv{\isadigit{2}}{\isacharunderscore}{\kern0pt}left{\isacharunderscore}{\kern0pt}ap\ id{\isacharunderscore}{\kern0pt}left{\isacharunderscore}{\kern0pt}unit{\isadigit{2}}{\isacharparenright}{\kern0pt}%
\endisatagproof
{\isafoldproof}%
%
\isadelimproof
\isanewline
%
\endisadelimproof
\isanewline
\isacommand{lemma}\isamarkupfalse%
\ dist{\isacharunderscore}{\kern0pt}prod{\isacharunderscore}{\kern0pt}coprod{\isacharunderscore}{\kern0pt}inv{\isadigit{2}}{\isacharunderscore}{\kern0pt}right{\isacharunderscore}{\kern0pt}coproj{\isacharcolon}{\kern0pt}\isanewline
\ \ {\isachardoublequoteopen}dist{\isacharunderscore}{\kern0pt}prod{\isacharunderscore}{\kern0pt}coprod{\isacharunderscore}{\kern0pt}inv{\isadigit{2}}\ X\ Y\ H\ {\isasymcirc}\isactrlsub c\ {\isacharparenleft}{\kern0pt}right{\isacharunderscore}{\kern0pt}coproj\ X\ Y\ {\isasymtimes}\isactrlsub f\ id\ H{\isacharparenright}{\kern0pt}\ {\isacharequal}{\kern0pt}\ right{\isacharunderscore}{\kern0pt}coproj\ {\isacharparenleft}{\kern0pt}X\ {\isasymtimes}\isactrlsub c\ H{\isacharparenright}{\kern0pt}\ {\isacharparenleft}{\kern0pt}Y\ {\isasymtimes}\isactrlsub c\ H{\isacharparenright}{\kern0pt}{\isachardoublequoteclose}\isanewline
%
\isadelimproof
\ \ %
\endisadelimproof
%
\isatagproof
\isacommand{by}\isamarkupfalse%
\ {\isacharparenleft}{\kern0pt}typecheck{\isacharunderscore}{\kern0pt}cfuncs{\isacharcomma}{\kern0pt}\ smt\ {\isacharparenleft}{\kern0pt}z{\isadigit{3}}{\isacharparenright}{\kern0pt}\ one{\isacharunderscore}{\kern0pt}separator\ cart{\isacharunderscore}{\kern0pt}prod{\isacharunderscore}{\kern0pt}decomp\ cfunc{\isacharunderscore}{\kern0pt}cross{\isacharunderscore}{\kern0pt}prod{\isacharunderscore}{\kern0pt}comp{\isacharunderscore}{\kern0pt}cfunc{\isacharunderscore}{\kern0pt}prod\ comp{\isacharunderscore}{\kern0pt}associative{\isadigit{2}}\ dist{\isacharunderscore}{\kern0pt}prod{\isacharunderscore}{\kern0pt}coprod{\isacharunderscore}{\kern0pt}inv{\isadigit{2}}{\isacharunderscore}{\kern0pt}right{\isacharunderscore}{\kern0pt}ap\ id{\isacharunderscore}{\kern0pt}left{\isacharunderscore}{\kern0pt}unit{\isadigit{2}}{\isacharparenright}{\kern0pt}%
\endisatagproof
{\isafoldproof}%
%
\isadelimproof
\isanewline
%
\endisadelimproof
\isanewline
\isacommand{lemma}\isamarkupfalse%
\ dist{\isacharunderscore}{\kern0pt}prod{\isacharunderscore}{\kern0pt}coprod{\isadigit{2}}{\isacharunderscore}{\kern0pt}inv{\isadigit{2}}{\isacharunderscore}{\kern0pt}id{\isacharcolon}{\kern0pt}\isanewline
{\isachardoublequoteopen}dist{\isacharunderscore}{\kern0pt}prod{\isacharunderscore}{\kern0pt}coprod{\isadigit{2}}\ A\ B\ C\ {\isasymcirc}\isactrlsub c\ dist{\isacharunderscore}{\kern0pt}prod{\isacharunderscore}{\kern0pt}coprod{\isacharunderscore}{\kern0pt}inv{\isadigit{2}}\ A\ B\ C\ {\isacharequal}{\kern0pt}\ id\ {\isacharparenleft}{\kern0pt}{\isacharparenleft}{\kern0pt}A\ {\isasymCoprod}\ B{\isacharparenright}{\kern0pt}\ {\isasymtimes}\isactrlsub c\ C{\isacharparenright}{\kern0pt}{\isachardoublequoteclose}\isanewline
%
\isadelimproof
\ \ %
\endisadelimproof
%
\isatagproof
\isacommand{unfolding}\isamarkupfalse%
\ dist{\isacharunderscore}{\kern0pt}prod{\isacharunderscore}{\kern0pt}coprod{\isadigit{2}}{\isacharunderscore}{\kern0pt}def\ dist{\isacharunderscore}{\kern0pt}prod{\isacharunderscore}{\kern0pt}coprod{\isacharunderscore}{\kern0pt}inv{\isadigit{2}}{\isacharunderscore}{\kern0pt}def\ \isacommand{by}\isamarkupfalse%
{\isacharparenleft}{\kern0pt}{\isacharminus}{\kern0pt}{\isacharcomma}{\kern0pt}typecheck{\isacharunderscore}{\kern0pt}cfuncs{\isacharcomma}{\kern0pt}\isanewline
\ \ smt\ {\isacharparenleft}{\kern0pt}z{\isadigit{3}}{\isacharparenright}{\kern0pt}\ cfunc{\isacharunderscore}{\kern0pt}bowtie{\isacharunderscore}{\kern0pt}prod{\isacharunderscore}{\kern0pt}comp{\isacharunderscore}{\kern0pt}cfunc{\isacharunderscore}{\kern0pt}bowtie{\isacharunderscore}{\kern0pt}prod\ comp{\isacharunderscore}{\kern0pt}associative{\isadigit{2}}\ dist{\isacharunderscore}{\kern0pt}prod{\isacharunderscore}{\kern0pt}coprod{\isacharunderscore}{\kern0pt}inv{\isacharunderscore}{\kern0pt}right\ id{\isacharunderscore}{\kern0pt}bowtie{\isacharunderscore}{\kern0pt}prod\ id{\isacharunderscore}{\kern0pt}right{\isacharunderscore}{\kern0pt}unit{\isadigit{2}}\ swap{\isacharunderscore}{\kern0pt}idempotent{\isacharparenright}{\kern0pt}%
\endisatagproof
{\isafoldproof}%
%
\isadelimproof
\isanewline
%
\endisadelimproof
\ \ \ \isanewline
\isacommand{lemma}\isamarkupfalse%
\ dist{\isacharunderscore}{\kern0pt}prod{\isacharunderscore}{\kern0pt}coprod{\isacharunderscore}{\kern0pt}inv{\isadigit{2}}{\isacharunderscore}{\kern0pt}inv{\isacharunderscore}{\kern0pt}id{\isacharcolon}{\kern0pt}\isanewline
{\isachardoublequoteopen}dist{\isacharunderscore}{\kern0pt}prod{\isacharunderscore}{\kern0pt}coprod{\isacharunderscore}{\kern0pt}inv{\isadigit{2}}\ A\ B\ C\ {\isasymcirc}\isactrlsub c\ dist{\isacharunderscore}{\kern0pt}prod{\isacharunderscore}{\kern0pt}coprod{\isadigit{2}}\ A\ B\ C\ {\isacharequal}{\kern0pt}\ id\ {\isacharparenleft}{\kern0pt}{\isacharparenleft}{\kern0pt}A\ {\isasymtimes}\isactrlsub c\ C{\isacharparenright}{\kern0pt}\ {\isasymCoprod}\ {\isacharparenleft}{\kern0pt}B\ {\isasymtimes}\isactrlsub c\ C{\isacharparenright}{\kern0pt}{\isacharparenright}{\kern0pt}{\isachardoublequoteclose}\isanewline
%
\isadelimproof
\ \ %
\endisadelimproof
%
\isatagproof
\isacommand{unfolding}\isamarkupfalse%
\ dist{\isacharunderscore}{\kern0pt}prod{\isacharunderscore}{\kern0pt}coprod{\isadigit{2}}{\isacharunderscore}{\kern0pt}def\ dist{\isacharunderscore}{\kern0pt}prod{\isacharunderscore}{\kern0pt}coprod{\isacharunderscore}{\kern0pt}inv{\isadigit{2}}{\isacharunderscore}{\kern0pt}def\ \isacommand{by}\isamarkupfalse%
{\isacharparenleft}{\kern0pt}{\isacharminus}{\kern0pt}{\isacharcomma}{\kern0pt}typecheck{\isacharunderscore}{\kern0pt}cfuncs{\isacharcomma}{\kern0pt}\isanewline
\ \ smt\ {\isacharparenleft}{\kern0pt}z{\isadigit{3}}{\isacharparenright}{\kern0pt}\ cfunc{\isacharunderscore}{\kern0pt}bowtie{\isacharunderscore}{\kern0pt}prod{\isacharunderscore}{\kern0pt}comp{\isacharunderscore}{\kern0pt}cfunc{\isacharunderscore}{\kern0pt}bowtie{\isacharunderscore}{\kern0pt}prod\ comp{\isacharunderscore}{\kern0pt}associative{\isadigit{2}}\ dist{\isacharunderscore}{\kern0pt}prod{\isacharunderscore}{\kern0pt}coprod{\isacharunderscore}{\kern0pt}inv{\isacharunderscore}{\kern0pt}left\ id{\isacharunderscore}{\kern0pt}bowtie{\isacharunderscore}{\kern0pt}prod\ id{\isacharunderscore}{\kern0pt}right{\isacharunderscore}{\kern0pt}unit{\isadigit{2}}\ swap{\isacharunderscore}{\kern0pt}idempotent{\isacharparenright}{\kern0pt}%
\endisatagproof
{\isafoldproof}%
%
\isadelimproof
\isanewline
%
\endisadelimproof
\isanewline
\isacommand{lemma}\isamarkupfalse%
\ dist{\isacharunderscore}{\kern0pt}prod{\isacharunderscore}{\kern0pt}coprod{\isadigit{2}}{\isacharunderscore}{\kern0pt}iso{\isacharcolon}{\kern0pt}\isanewline
\ \ {\isachardoublequoteopen}isomorphism{\isacharparenleft}{\kern0pt}dist{\isacharunderscore}{\kern0pt}prod{\isacharunderscore}{\kern0pt}coprod{\isadigit{2}}\ A\ B\ C{\isacharparenright}{\kern0pt}{\isachardoublequoteclose}\isanewline
%
\isadelimproof
\ \ %
\endisadelimproof
%
\isatagproof
\isacommand{by}\isamarkupfalse%
\ {\isacharparenleft}{\kern0pt}metis\ cfunc{\isacharunderscore}{\kern0pt}type{\isacharunderscore}{\kern0pt}def\ dist{\isacharunderscore}{\kern0pt}prod{\isacharunderscore}{\kern0pt}coprod{\isadigit{2}}{\isacharunderscore}{\kern0pt}inv{\isadigit{2}}{\isacharunderscore}{\kern0pt}id\ dist{\isacharunderscore}{\kern0pt}prod{\isacharunderscore}{\kern0pt}coprod{\isadigit{2}}{\isacharunderscore}{\kern0pt}type\ dist{\isacharunderscore}{\kern0pt}prod{\isacharunderscore}{\kern0pt}coprod{\isacharunderscore}{\kern0pt}inv{\isadigit{2}}{\isacharunderscore}{\kern0pt}inv{\isacharunderscore}{\kern0pt}id\ dist{\isacharunderscore}{\kern0pt}prod{\isacharunderscore}{\kern0pt}coprod{\isacharunderscore}{\kern0pt}inv{\isadigit{2}}{\isacharunderscore}{\kern0pt}type\ isomorphism{\isacharunderscore}{\kern0pt}def{\isacharparenright}{\kern0pt}%
\endisatagproof
{\isafoldproof}%
%
\isadelimproof
%
\endisadelimproof
%
\isadelimdocument
%
\endisadelimdocument
%
\isatagdocument
%
\isamarkupsubsection{Casting between sets%
}
\isamarkuptrue%
%
\isamarkupsubsubsection{Going from a set or its complement to the superset%
}
\isamarkuptrue%
%
\endisatagdocument
{\isafolddocument}%
%
\isadelimdocument
%
\endisadelimdocument
%
\begin{isamarkuptext}%
This subsection corresponds to Proposition 2.4.5 in Halvorson.%
\end{isamarkuptext}\isamarkuptrue%
\isacommand{definition}\isamarkupfalse%
\ into{\isacharunderscore}{\kern0pt}super\ {\isacharcolon}{\kern0pt}{\isacharcolon}{\kern0pt}\ {\isachardoublequoteopen}cfunc\ {\isasymRightarrow}\ cfunc{\isachardoublequoteclose}\ \isakeyword{where}\isanewline
\ \ {\isachardoublequoteopen}into{\isacharunderscore}{\kern0pt}super\ m\ {\isacharequal}{\kern0pt}\ m\ {\isasymamalg}\ m\isactrlsup c{\isachardoublequoteclose}\isanewline
\isanewline
\isacommand{lemma}\isamarkupfalse%
\ into{\isacharunderscore}{\kern0pt}super{\isacharunderscore}{\kern0pt}type{\isacharbrackleft}{\kern0pt}type{\isacharunderscore}{\kern0pt}rule{\isacharbrackright}{\kern0pt}{\isacharcolon}{\kern0pt}\isanewline
\ \ {\isachardoublequoteopen}monomorphism\ m\ {\isasymLongrightarrow}\ m\ {\isacharcolon}{\kern0pt}\ X\ {\isasymrightarrow}\ Y\ {\isasymLongrightarrow}\ into{\isacharunderscore}{\kern0pt}super\ m\ {\isacharcolon}{\kern0pt}\ X\ {\isasymCoprod}\ {\isacharparenleft}{\kern0pt}Y\ {\isasymsetminus}\ {\isacharparenleft}{\kern0pt}X{\isacharcomma}{\kern0pt}m{\isacharparenright}{\kern0pt}{\isacharparenright}{\kern0pt}\ {\isasymrightarrow}\ Y{\isachardoublequoteclose}\isanewline
%
\isadelimproof
\ \ %
\endisadelimproof
%
\isatagproof
\isacommand{unfolding}\isamarkupfalse%
\ into{\isacharunderscore}{\kern0pt}super{\isacharunderscore}{\kern0pt}def\ \isacommand{by}\isamarkupfalse%
\ typecheck{\isacharunderscore}{\kern0pt}cfuncs%
\endisatagproof
{\isafoldproof}%
%
\isadelimproof
\isanewline
%
\endisadelimproof
\isanewline
\isacommand{lemma}\isamarkupfalse%
\ into{\isacharunderscore}{\kern0pt}super{\isacharunderscore}{\kern0pt}mono{\isacharcolon}{\kern0pt}\isanewline
\ \ \isakeyword{assumes}\ {\isachardoublequoteopen}monomorphism\ m{\isachardoublequoteclose}\ {\isachardoublequoteopen}m\ {\isacharcolon}{\kern0pt}\ X\ {\isasymrightarrow}\ Y{\isachardoublequoteclose}\isanewline
\ \ \isakeyword{shows}\ {\isachardoublequoteopen}monomorphism\ {\isacharparenleft}{\kern0pt}into{\isacharunderscore}{\kern0pt}super\ m{\isacharparenright}{\kern0pt}{\isachardoublequoteclose}\isanewline
%
\isadelimproof
%
\endisadelimproof
%
\isatagproof
\isacommand{proof}\isamarkupfalse%
\ {\isacharparenleft}{\kern0pt}rule\ injective{\isacharunderscore}{\kern0pt}imp{\isacharunderscore}{\kern0pt}monomorphism{\isacharcomma}{\kern0pt}\ unfold\ injective{\isacharunderscore}{\kern0pt}def{\isacharcomma}{\kern0pt}\ auto{\isacharparenright}{\kern0pt}\isanewline
\ \ \isacommand{fix}\isamarkupfalse%
\ x\ y\isanewline
\ \ \isacommand{assume}\isamarkupfalse%
\ {\isachardoublequoteopen}x\ {\isasymin}\isactrlsub c\ domain\ {\isacharparenleft}{\kern0pt}into{\isacharunderscore}{\kern0pt}super\ m{\isacharparenright}{\kern0pt}{\isachardoublequoteclose}\ \ \isacommand{then}\isamarkupfalse%
\ \isacommand{have}\isamarkupfalse%
\ x{\isacharunderscore}{\kern0pt}type{\isacharcolon}{\kern0pt}\ {\isachardoublequoteopen}x\ {\isasymin}\isactrlsub c\ X\ {\isasymCoprod}\ {\isacharparenleft}{\kern0pt}Y\ {\isasymsetminus}\ {\isacharparenleft}{\kern0pt}X{\isacharcomma}{\kern0pt}m{\isacharparenright}{\kern0pt}{\isacharparenright}{\kern0pt}{\isachardoublequoteclose}\isanewline
\ \ \ \ \isacommand{using}\isamarkupfalse%
\ assms\ cfunc{\isacharunderscore}{\kern0pt}type{\isacharunderscore}{\kern0pt}def\ into{\isacharunderscore}{\kern0pt}super{\isacharunderscore}{\kern0pt}type\ \isacommand{by}\isamarkupfalse%
\ auto\isanewline
\ \ \isanewline
\ \ \isacommand{assume}\isamarkupfalse%
\ {\isachardoublequoteopen}y\ {\isasymin}\isactrlsub c\ domain\ {\isacharparenleft}{\kern0pt}into{\isacharunderscore}{\kern0pt}super\ m{\isacharparenright}{\kern0pt}{\isachardoublequoteclose}\ \ \isacommand{then}\isamarkupfalse%
\ \isacommand{have}\isamarkupfalse%
\ y{\isacharunderscore}{\kern0pt}type{\isacharcolon}{\kern0pt}\ {\isachardoublequoteopen}y\ {\isasymin}\isactrlsub c\ X\ {\isasymCoprod}\ {\isacharparenleft}{\kern0pt}Y\ {\isasymsetminus}\ {\isacharparenleft}{\kern0pt}X{\isacharcomma}{\kern0pt}m{\isacharparenright}{\kern0pt}{\isacharparenright}{\kern0pt}{\isachardoublequoteclose}\isanewline
\ \ \ \ \isacommand{using}\isamarkupfalse%
\ assms\ cfunc{\isacharunderscore}{\kern0pt}type{\isacharunderscore}{\kern0pt}def\ into{\isacharunderscore}{\kern0pt}super{\isacharunderscore}{\kern0pt}type\ \isacommand{by}\isamarkupfalse%
\ auto\isanewline
\isanewline
\ \ \isacommand{assume}\isamarkupfalse%
\ into{\isacharunderscore}{\kern0pt}super{\isacharunderscore}{\kern0pt}eq{\isacharcolon}{\kern0pt}\ {\isachardoublequoteopen}into{\isacharunderscore}{\kern0pt}super\ m\ {\isasymcirc}\isactrlsub c\ x\ {\isacharequal}{\kern0pt}\ into{\isacharunderscore}{\kern0pt}super\ m\ {\isasymcirc}\isactrlsub c\ y{\isachardoublequoteclose}\isanewline
\isanewline
\ \ \isacommand{have}\isamarkupfalse%
\ x{\isacharunderscore}{\kern0pt}cases{\isacharcolon}{\kern0pt}\ {\isachardoublequoteopen}{\isacharparenleft}{\kern0pt}{\isasymexists}\ x{\isacharprime}{\kern0pt}{\isachardot}{\kern0pt}\ x{\isacharprime}{\kern0pt}\ {\isasymin}\isactrlsub c\ X\ {\isasymand}\ x\ {\isacharequal}{\kern0pt}\ left{\isacharunderscore}{\kern0pt}coproj\ X\ {\isacharparenleft}{\kern0pt}Y\ {\isasymsetminus}\ {\isacharparenleft}{\kern0pt}X{\isacharcomma}{\kern0pt}m{\isacharparenright}{\kern0pt}{\isacharparenright}{\kern0pt}\ {\isasymcirc}\isactrlsub c\ x{\isacharprime}{\kern0pt}{\isacharparenright}{\kern0pt}\isanewline
\ \ \ \ {\isasymor}\ \ {\isacharparenleft}{\kern0pt}{\isasymexists}\ x{\isacharprime}{\kern0pt}{\isachardot}{\kern0pt}\ x{\isacharprime}{\kern0pt}\ {\isasymin}\isactrlsub c\ Y\ {\isasymsetminus}\ {\isacharparenleft}{\kern0pt}X{\isacharcomma}{\kern0pt}m{\isacharparenright}{\kern0pt}\ {\isasymand}\ x\ {\isacharequal}{\kern0pt}\ right{\isacharunderscore}{\kern0pt}coproj\ X\ {\isacharparenleft}{\kern0pt}Y\ {\isasymsetminus}\ {\isacharparenleft}{\kern0pt}X{\isacharcomma}{\kern0pt}m{\isacharparenright}{\kern0pt}{\isacharparenright}{\kern0pt}\ {\isasymcirc}\isactrlsub c\ x{\isacharprime}{\kern0pt}{\isacharparenright}{\kern0pt}{\isachardoublequoteclose}\isanewline
\ \ \ \ \isacommand{by}\isamarkupfalse%
\ {\isacharparenleft}{\kern0pt}simp\ add{\isacharcolon}{\kern0pt}\ coprojs{\isacharunderscore}{\kern0pt}jointly{\isacharunderscore}{\kern0pt}surj\ x{\isacharunderscore}{\kern0pt}type{\isacharparenright}{\kern0pt}\isanewline
\isanewline
\ \ \isacommand{have}\isamarkupfalse%
\ y{\isacharunderscore}{\kern0pt}cases{\isacharcolon}{\kern0pt}\ {\isachardoublequoteopen}{\isacharparenleft}{\kern0pt}{\isasymexists}\ y{\isacharprime}{\kern0pt}{\isachardot}{\kern0pt}\ y{\isacharprime}{\kern0pt}\ {\isasymin}\isactrlsub c\ X\ {\isasymand}\ y\ {\isacharequal}{\kern0pt}\ left{\isacharunderscore}{\kern0pt}coproj\ X\ {\isacharparenleft}{\kern0pt}Y\ {\isasymsetminus}\ {\isacharparenleft}{\kern0pt}X{\isacharcomma}{\kern0pt}m{\isacharparenright}{\kern0pt}{\isacharparenright}{\kern0pt}\ {\isasymcirc}\isactrlsub c\ y{\isacharprime}{\kern0pt}{\isacharparenright}{\kern0pt}\isanewline
\ \ \ \ {\isasymor}\ \ {\isacharparenleft}{\kern0pt}{\isasymexists}\ y{\isacharprime}{\kern0pt}{\isachardot}{\kern0pt}\ y{\isacharprime}{\kern0pt}\ {\isasymin}\isactrlsub c\ Y\ {\isasymsetminus}\ {\isacharparenleft}{\kern0pt}X{\isacharcomma}{\kern0pt}m{\isacharparenright}{\kern0pt}\ {\isasymand}\ y\ {\isacharequal}{\kern0pt}\ right{\isacharunderscore}{\kern0pt}coproj\ X\ {\isacharparenleft}{\kern0pt}Y\ {\isasymsetminus}\ {\isacharparenleft}{\kern0pt}X{\isacharcomma}{\kern0pt}m{\isacharparenright}{\kern0pt}{\isacharparenright}{\kern0pt}\ {\isasymcirc}\isactrlsub c\ y{\isacharprime}{\kern0pt}{\isacharparenright}{\kern0pt}{\isachardoublequoteclose}\isanewline
\ \ \ \ \isacommand{by}\isamarkupfalse%
\ {\isacharparenleft}{\kern0pt}simp\ add{\isacharcolon}{\kern0pt}\ coprojs{\isacharunderscore}{\kern0pt}jointly{\isacharunderscore}{\kern0pt}surj\ y{\isacharunderscore}{\kern0pt}type{\isacharparenright}{\kern0pt}\isanewline
\isanewline
\ \ \isacommand{show}\isamarkupfalse%
\ {\isachardoublequoteopen}x\ {\isacharequal}{\kern0pt}\ y{\isachardoublequoteclose}\isanewline
\ \ \ \ \isacommand{using}\isamarkupfalse%
\ x{\isacharunderscore}{\kern0pt}cases\ y{\isacharunderscore}{\kern0pt}cases\isanewline
\ \ \isacommand{proof}\isamarkupfalse%
\ auto\isanewline
\ \ \ \ \isacommand{fix}\isamarkupfalse%
\ x{\isacharprime}{\kern0pt}\ y{\isacharprime}{\kern0pt}\isanewline
\ \ \ \ \isacommand{assume}\isamarkupfalse%
\ x{\isacharprime}{\kern0pt}{\isacharunderscore}{\kern0pt}type{\isacharcolon}{\kern0pt}\ {\isachardoublequoteopen}x{\isacharprime}{\kern0pt}\ {\isasymin}\isactrlsub c\ X{\isachardoublequoteclose}\ \isakeyword{and}\ x{\isacharunderscore}{\kern0pt}def{\isacharcolon}{\kern0pt}\ {\isachardoublequoteopen}x\ {\isacharequal}{\kern0pt}\ left{\isacharunderscore}{\kern0pt}coproj\ X\ {\isacharparenleft}{\kern0pt}Y\ {\isasymsetminus}\ {\isacharparenleft}{\kern0pt}X{\isacharcomma}{\kern0pt}\ m{\isacharparenright}{\kern0pt}{\isacharparenright}{\kern0pt}\ {\isasymcirc}\isactrlsub c\ x{\isacharprime}{\kern0pt}{\isachardoublequoteclose}\isanewline
\ \ \ \ \isacommand{assume}\isamarkupfalse%
\ y{\isacharprime}{\kern0pt}{\isacharunderscore}{\kern0pt}type{\isacharcolon}{\kern0pt}\ {\isachardoublequoteopen}y{\isacharprime}{\kern0pt}\ {\isasymin}\isactrlsub c\ X{\isachardoublequoteclose}\ \isakeyword{and}\ y{\isacharunderscore}{\kern0pt}def{\isacharcolon}{\kern0pt}\ {\isachardoublequoteopen}y\ {\isacharequal}{\kern0pt}\ left{\isacharunderscore}{\kern0pt}coproj\ X\ {\isacharparenleft}{\kern0pt}Y\ {\isasymsetminus}\ {\isacharparenleft}{\kern0pt}X{\isacharcomma}{\kern0pt}\ m{\isacharparenright}{\kern0pt}{\isacharparenright}{\kern0pt}\ {\isasymcirc}\isactrlsub c\ y{\isacharprime}{\kern0pt}{\isachardoublequoteclose}\isanewline
\isanewline
\ \ \ \ \isacommand{have}\isamarkupfalse%
\ {\isachardoublequoteopen}into{\isacharunderscore}{\kern0pt}super\ m\ {\isasymcirc}\isactrlsub c\ left{\isacharunderscore}{\kern0pt}coproj\ X\ {\isacharparenleft}{\kern0pt}Y\ {\isasymsetminus}\ {\isacharparenleft}{\kern0pt}X{\isacharcomma}{\kern0pt}\ m{\isacharparenright}{\kern0pt}{\isacharparenright}{\kern0pt}\ {\isasymcirc}\isactrlsub c\ x{\isacharprime}{\kern0pt}\ {\isacharequal}{\kern0pt}\ into{\isacharunderscore}{\kern0pt}super\ m\ {\isasymcirc}\isactrlsub c\ left{\isacharunderscore}{\kern0pt}coproj\ X\ {\isacharparenleft}{\kern0pt}Y\ {\isasymsetminus}\ {\isacharparenleft}{\kern0pt}X{\isacharcomma}{\kern0pt}\ m{\isacharparenright}{\kern0pt}{\isacharparenright}{\kern0pt}\ {\isasymcirc}\isactrlsub c\ y{\isacharprime}{\kern0pt}{\isachardoublequoteclose}\isanewline
\ \ \ \ \ \ \isacommand{using}\isamarkupfalse%
\ into{\isacharunderscore}{\kern0pt}super{\isacharunderscore}{\kern0pt}eq\ \isacommand{unfolding}\isamarkupfalse%
\ x{\isacharunderscore}{\kern0pt}def\ y{\isacharunderscore}{\kern0pt}def\ \isacommand{by}\isamarkupfalse%
\ auto\isanewline
\ \ \ \ \isacommand{then}\isamarkupfalse%
\ \isacommand{have}\isamarkupfalse%
\ {\isachardoublequoteopen}{\isacharparenleft}{\kern0pt}into{\isacharunderscore}{\kern0pt}super\ m\ {\isasymcirc}\isactrlsub c\ left{\isacharunderscore}{\kern0pt}coproj\ X\ {\isacharparenleft}{\kern0pt}Y\ {\isasymsetminus}\ {\isacharparenleft}{\kern0pt}X{\isacharcomma}{\kern0pt}\ m{\isacharparenright}{\kern0pt}{\isacharparenright}{\kern0pt}{\isacharparenright}{\kern0pt}\ {\isasymcirc}\isactrlsub c\ x{\isacharprime}{\kern0pt}\ {\isacharequal}{\kern0pt}\ {\isacharparenleft}{\kern0pt}into{\isacharunderscore}{\kern0pt}super\ m\ {\isasymcirc}\isactrlsub c\ left{\isacharunderscore}{\kern0pt}coproj\ X\ {\isacharparenleft}{\kern0pt}Y\ {\isasymsetminus}\ {\isacharparenleft}{\kern0pt}X{\isacharcomma}{\kern0pt}\ m{\isacharparenright}{\kern0pt}{\isacharparenright}{\kern0pt}{\isacharparenright}{\kern0pt}\ {\isasymcirc}\isactrlsub c\ y{\isacharprime}{\kern0pt}{\isachardoublequoteclose}\isanewline
\ \ \ \ \ \ \isacommand{using}\isamarkupfalse%
\ assms\ x{\isacharprime}{\kern0pt}{\isacharunderscore}{\kern0pt}type\ y{\isacharprime}{\kern0pt}{\isacharunderscore}{\kern0pt}type\ comp{\isacharunderscore}{\kern0pt}associative{\isadigit{2}}\ \isacommand{by}\isamarkupfalse%
\ {\isacharparenleft}{\kern0pt}typecheck{\isacharunderscore}{\kern0pt}cfuncs{\isacharcomma}{\kern0pt}\ auto{\isacharparenright}{\kern0pt}\isanewline
\ \ \ \ \isacommand{then}\isamarkupfalse%
\ \isacommand{have}\isamarkupfalse%
\ {\isachardoublequoteopen}m\ {\isasymcirc}\isactrlsub c\ x{\isacharprime}{\kern0pt}\ {\isacharequal}{\kern0pt}\ m\ {\isasymcirc}\isactrlsub c\ y{\isacharprime}{\kern0pt}{\isachardoublequoteclose}\isanewline
\ \ \ \ \ \ \isacommand{using}\isamarkupfalse%
\ assms\ \isacommand{unfolding}\isamarkupfalse%
\ into{\isacharunderscore}{\kern0pt}super{\isacharunderscore}{\kern0pt}def\isanewline
\ \ \ \ \ \ \isacommand{by}\isamarkupfalse%
\ {\isacharparenleft}{\kern0pt}simp\ add{\isacharcolon}{\kern0pt}\ complement{\isacharunderscore}{\kern0pt}morphism{\isacharunderscore}{\kern0pt}type\ left{\isacharunderscore}{\kern0pt}coproj{\isacharunderscore}{\kern0pt}cfunc{\isacharunderscore}{\kern0pt}coprod{\isacharparenright}{\kern0pt}\isanewline
\ \ \ \ \isacommand{then}\isamarkupfalse%
\ \isacommand{have}\isamarkupfalse%
\ {\isachardoublequoteopen}x{\isacharprime}{\kern0pt}\ {\isacharequal}{\kern0pt}\ y{\isacharprime}{\kern0pt}{\isachardoublequoteclose}\isanewline
\ \ \ \ \ \ \isacommand{using}\isamarkupfalse%
\ assms\ cfunc{\isacharunderscore}{\kern0pt}type{\isacharunderscore}{\kern0pt}def\ monomorphism{\isacharunderscore}{\kern0pt}def\ x{\isacharprime}{\kern0pt}{\isacharunderscore}{\kern0pt}type\ y{\isacharprime}{\kern0pt}{\isacharunderscore}{\kern0pt}type\ \isacommand{by}\isamarkupfalse%
\ auto\isanewline
\ \ \ \ \isacommand{then}\isamarkupfalse%
\ \isacommand{show}\isamarkupfalse%
\ {\isachardoublequoteopen}left{\isacharunderscore}{\kern0pt}coproj\ X\ {\isacharparenleft}{\kern0pt}Y\ {\isasymsetminus}\ {\isacharparenleft}{\kern0pt}X{\isacharcomma}{\kern0pt}\ m{\isacharparenright}{\kern0pt}{\isacharparenright}{\kern0pt}\ {\isasymcirc}\isactrlsub c\ x{\isacharprime}{\kern0pt}\ {\isacharequal}{\kern0pt}\ left{\isacharunderscore}{\kern0pt}coproj\ X\ {\isacharparenleft}{\kern0pt}Y\ {\isasymsetminus}\ {\isacharparenleft}{\kern0pt}X{\isacharcomma}{\kern0pt}\ m{\isacharparenright}{\kern0pt}{\isacharparenright}{\kern0pt}\ {\isasymcirc}\isactrlsub c\ y{\isacharprime}{\kern0pt}{\isachardoublequoteclose}\isanewline
\ \ \ \ \ \ \isacommand{by}\isamarkupfalse%
\ simp\isanewline
\ \ \isacommand{next}\isamarkupfalse%
\isanewline
\ \ \ \ \isacommand{fix}\isamarkupfalse%
\ x{\isacharprime}{\kern0pt}\ y{\isacharprime}{\kern0pt}\isanewline
\ \ \ \ \isacommand{assume}\isamarkupfalse%
\ x{\isacharprime}{\kern0pt}{\isacharunderscore}{\kern0pt}type{\isacharcolon}{\kern0pt}\ {\isachardoublequoteopen}x{\isacharprime}{\kern0pt}\ {\isasymin}\isactrlsub c\ X{\isachardoublequoteclose}\ \isakeyword{and}\ x{\isacharunderscore}{\kern0pt}def{\isacharcolon}{\kern0pt}\ {\isachardoublequoteopen}x\ {\isacharequal}{\kern0pt}\ left{\isacharunderscore}{\kern0pt}coproj\ X\ {\isacharparenleft}{\kern0pt}Y\ {\isasymsetminus}\ {\isacharparenleft}{\kern0pt}X{\isacharcomma}{\kern0pt}\ m{\isacharparenright}{\kern0pt}{\isacharparenright}{\kern0pt}\ {\isasymcirc}\isactrlsub c\ x{\isacharprime}{\kern0pt}{\isachardoublequoteclose}\isanewline
\ \ \ \ \isacommand{assume}\isamarkupfalse%
\ y{\isacharprime}{\kern0pt}{\isacharunderscore}{\kern0pt}type{\isacharcolon}{\kern0pt}\ {\isachardoublequoteopen}y{\isacharprime}{\kern0pt}\ {\isasymin}\isactrlsub c\ Y\ {\isasymsetminus}\ {\isacharparenleft}{\kern0pt}X{\isacharcomma}{\kern0pt}\ m{\isacharparenright}{\kern0pt}{\isachardoublequoteclose}\ \isakeyword{and}\ y{\isacharunderscore}{\kern0pt}def{\isacharcolon}{\kern0pt}\ {\isachardoublequoteopen}y\ {\isacharequal}{\kern0pt}\ right{\isacharunderscore}{\kern0pt}coproj\ X\ {\isacharparenleft}{\kern0pt}Y\ {\isasymsetminus}\ {\isacharparenleft}{\kern0pt}X{\isacharcomma}{\kern0pt}\ m{\isacharparenright}{\kern0pt}{\isacharparenright}{\kern0pt}\ {\isasymcirc}\isactrlsub c\ y{\isacharprime}{\kern0pt}{\isachardoublequoteclose}\isanewline
\isanewline
\ \ \ \ \isacommand{have}\isamarkupfalse%
\ {\isachardoublequoteopen}into{\isacharunderscore}{\kern0pt}super\ m\ {\isasymcirc}\isactrlsub c\ left{\isacharunderscore}{\kern0pt}coproj\ X\ {\isacharparenleft}{\kern0pt}Y\ {\isasymsetminus}\ {\isacharparenleft}{\kern0pt}X{\isacharcomma}{\kern0pt}\ m{\isacharparenright}{\kern0pt}{\isacharparenright}{\kern0pt}\ {\isasymcirc}\isactrlsub c\ x{\isacharprime}{\kern0pt}\ {\isacharequal}{\kern0pt}\ into{\isacharunderscore}{\kern0pt}super\ m\ {\isasymcirc}\isactrlsub c\ right{\isacharunderscore}{\kern0pt}coproj\ X\ {\isacharparenleft}{\kern0pt}Y\ {\isasymsetminus}\ {\isacharparenleft}{\kern0pt}X{\isacharcomma}{\kern0pt}\ m{\isacharparenright}{\kern0pt}{\isacharparenright}{\kern0pt}\ {\isasymcirc}\isactrlsub c\ y{\isacharprime}{\kern0pt}{\isachardoublequoteclose}\isanewline
\ \ \ \ \ \ \isacommand{using}\isamarkupfalse%
\ into{\isacharunderscore}{\kern0pt}super{\isacharunderscore}{\kern0pt}eq\ \isacommand{unfolding}\isamarkupfalse%
\ x{\isacharunderscore}{\kern0pt}def\ y{\isacharunderscore}{\kern0pt}def\ \isacommand{by}\isamarkupfalse%
\ auto\isanewline
\ \ \ \ \isacommand{then}\isamarkupfalse%
\ \isacommand{have}\isamarkupfalse%
\ {\isachardoublequoteopen}{\isacharparenleft}{\kern0pt}into{\isacharunderscore}{\kern0pt}super\ m\ {\isasymcirc}\isactrlsub c\ left{\isacharunderscore}{\kern0pt}coproj\ X\ {\isacharparenleft}{\kern0pt}Y\ {\isasymsetminus}\ {\isacharparenleft}{\kern0pt}X{\isacharcomma}{\kern0pt}\ m{\isacharparenright}{\kern0pt}{\isacharparenright}{\kern0pt}{\isacharparenright}{\kern0pt}\ {\isasymcirc}\isactrlsub c\ x{\isacharprime}{\kern0pt}\ {\isacharequal}{\kern0pt}\ {\isacharparenleft}{\kern0pt}into{\isacharunderscore}{\kern0pt}super\ m\ {\isasymcirc}\isactrlsub c\ right{\isacharunderscore}{\kern0pt}coproj\ X\ {\isacharparenleft}{\kern0pt}Y\ {\isasymsetminus}\ {\isacharparenleft}{\kern0pt}X{\isacharcomma}{\kern0pt}\ m{\isacharparenright}{\kern0pt}{\isacharparenright}{\kern0pt}{\isacharparenright}{\kern0pt}\ {\isasymcirc}\isactrlsub c\ y{\isacharprime}{\kern0pt}{\isachardoublequoteclose}\isanewline
\ \ \ \ \ \ \isacommand{using}\isamarkupfalse%
\ assms\ x{\isacharprime}{\kern0pt}{\isacharunderscore}{\kern0pt}type\ y{\isacharprime}{\kern0pt}{\isacharunderscore}{\kern0pt}type\ comp{\isacharunderscore}{\kern0pt}associative{\isadigit{2}}\ \isacommand{by}\isamarkupfalse%
\ {\isacharparenleft}{\kern0pt}typecheck{\isacharunderscore}{\kern0pt}cfuncs{\isacharcomma}{\kern0pt}\ auto{\isacharparenright}{\kern0pt}\isanewline
\ \ \ \ \isacommand{then}\isamarkupfalse%
\ \isacommand{have}\isamarkupfalse%
\ {\isachardoublequoteopen}m\ {\isasymcirc}\isactrlsub c\ x{\isacharprime}{\kern0pt}\ {\isacharequal}{\kern0pt}\ m\isactrlsup c\ {\isasymcirc}\isactrlsub c\ y{\isacharprime}{\kern0pt}{\isachardoublequoteclose}\isanewline
\ \ \ \ \ \ \isacommand{using}\isamarkupfalse%
\ assms\ \isacommand{unfolding}\isamarkupfalse%
\ into{\isacharunderscore}{\kern0pt}super{\isacharunderscore}{\kern0pt}def\isanewline
\ \ \ \ \ \ \isacommand{by}\isamarkupfalse%
\ {\isacharparenleft}{\kern0pt}simp\ add{\isacharcolon}{\kern0pt}\ complement{\isacharunderscore}{\kern0pt}morphism{\isacharunderscore}{\kern0pt}type\ left{\isacharunderscore}{\kern0pt}coproj{\isacharunderscore}{\kern0pt}cfunc{\isacharunderscore}{\kern0pt}coprod\ right{\isacharunderscore}{\kern0pt}coproj{\isacharunderscore}{\kern0pt}cfunc{\isacharunderscore}{\kern0pt}coprod{\isacharparenright}{\kern0pt}\isanewline
\ \ \ \ \isacommand{then}\isamarkupfalse%
\ \isacommand{have}\isamarkupfalse%
\ False\isanewline
\ \ \ \ \ \ \isacommand{using}\isamarkupfalse%
\ assms{\isacharparenleft}{\kern0pt}{\isadigit{1}}{\isacharparenright}{\kern0pt}\ assms{\isacharparenleft}{\kern0pt}{\isadigit{2}}{\isacharparenright}{\kern0pt}\ complement{\isacharunderscore}{\kern0pt}disjoint\ x{\isacharprime}{\kern0pt}{\isacharunderscore}{\kern0pt}type\ y{\isacharprime}{\kern0pt}{\isacharunderscore}{\kern0pt}type\ \isacommand{by}\isamarkupfalse%
\ blast\isanewline
\ \ \ \ \isacommand{then}\isamarkupfalse%
\ \isacommand{show}\isamarkupfalse%
\ {\isachardoublequoteopen}left{\isacharunderscore}{\kern0pt}coproj\ X\ {\isacharparenleft}{\kern0pt}Y\ {\isasymsetminus}\ {\isacharparenleft}{\kern0pt}X{\isacharcomma}{\kern0pt}\ m{\isacharparenright}{\kern0pt}{\isacharparenright}{\kern0pt}\ {\isasymcirc}\isactrlsub c\ x{\isacharprime}{\kern0pt}\ {\isacharequal}{\kern0pt}\ right{\isacharunderscore}{\kern0pt}coproj\ X\ {\isacharparenleft}{\kern0pt}Y\ {\isasymsetminus}\ {\isacharparenleft}{\kern0pt}X{\isacharcomma}{\kern0pt}\ m{\isacharparenright}{\kern0pt}{\isacharparenright}{\kern0pt}\ {\isasymcirc}\isactrlsub c\ y{\isacharprime}{\kern0pt}{\isachardoublequoteclose}\isanewline
\ \ \ \ \ \ \isacommand{by}\isamarkupfalse%
\ auto\isanewline
\ \ \isacommand{next}\isamarkupfalse%
\isanewline
\ \ \ \ \isacommand{fix}\isamarkupfalse%
\ x{\isacharprime}{\kern0pt}\ y{\isacharprime}{\kern0pt}\isanewline
\ \ \ \ \isacommand{assume}\isamarkupfalse%
\ x{\isacharprime}{\kern0pt}{\isacharunderscore}{\kern0pt}type{\isacharcolon}{\kern0pt}\ {\isachardoublequoteopen}x{\isacharprime}{\kern0pt}\ {\isasymin}\isactrlsub c\ Y\ {\isasymsetminus}\ {\isacharparenleft}{\kern0pt}X{\isacharcomma}{\kern0pt}\ m{\isacharparenright}{\kern0pt}{\isachardoublequoteclose}\ \isakeyword{and}\ x{\isacharunderscore}{\kern0pt}def{\isacharcolon}{\kern0pt}\ {\isachardoublequoteopen}x\ {\isacharequal}{\kern0pt}\ right{\isacharunderscore}{\kern0pt}coproj\ X\ {\isacharparenleft}{\kern0pt}Y\ {\isasymsetminus}\ {\isacharparenleft}{\kern0pt}X{\isacharcomma}{\kern0pt}\ m{\isacharparenright}{\kern0pt}{\isacharparenright}{\kern0pt}\ {\isasymcirc}\isactrlsub c\ x{\isacharprime}{\kern0pt}{\isachardoublequoteclose}\isanewline
\ \ \ \ \isacommand{assume}\isamarkupfalse%
\ y{\isacharprime}{\kern0pt}{\isacharunderscore}{\kern0pt}type{\isacharcolon}{\kern0pt}\ {\isachardoublequoteopen}y{\isacharprime}{\kern0pt}\ {\isasymin}\isactrlsub c\ X{\isachardoublequoteclose}\ \isakeyword{and}\ y{\isacharunderscore}{\kern0pt}def{\isacharcolon}{\kern0pt}\ {\isachardoublequoteopen}y\ {\isacharequal}{\kern0pt}\ left{\isacharunderscore}{\kern0pt}coproj\ X\ {\isacharparenleft}{\kern0pt}Y\ {\isasymsetminus}\ {\isacharparenleft}{\kern0pt}X{\isacharcomma}{\kern0pt}\ m{\isacharparenright}{\kern0pt}{\isacharparenright}{\kern0pt}\ {\isasymcirc}\isactrlsub c\ y{\isacharprime}{\kern0pt}{\isachardoublequoteclose}\isanewline
\isanewline
\ \ \ \ \isacommand{have}\isamarkupfalse%
\ {\isachardoublequoteopen}into{\isacharunderscore}{\kern0pt}super\ m\ {\isasymcirc}\isactrlsub c\ right{\isacharunderscore}{\kern0pt}coproj\ X\ {\isacharparenleft}{\kern0pt}Y\ {\isasymsetminus}\ {\isacharparenleft}{\kern0pt}X{\isacharcomma}{\kern0pt}\ m{\isacharparenright}{\kern0pt}{\isacharparenright}{\kern0pt}\ {\isasymcirc}\isactrlsub c\ x{\isacharprime}{\kern0pt}\ {\isacharequal}{\kern0pt}\ into{\isacharunderscore}{\kern0pt}super\ m\ {\isasymcirc}\isactrlsub c\ left{\isacharunderscore}{\kern0pt}coproj\ X\ {\isacharparenleft}{\kern0pt}Y\ {\isasymsetminus}\ {\isacharparenleft}{\kern0pt}X{\isacharcomma}{\kern0pt}\ m{\isacharparenright}{\kern0pt}{\isacharparenright}{\kern0pt}\ {\isasymcirc}\isactrlsub c\ y{\isacharprime}{\kern0pt}{\isachardoublequoteclose}\isanewline
\ \ \ \ \ \ \isacommand{using}\isamarkupfalse%
\ into{\isacharunderscore}{\kern0pt}super{\isacharunderscore}{\kern0pt}eq\ \isacommand{unfolding}\isamarkupfalse%
\ x{\isacharunderscore}{\kern0pt}def\ y{\isacharunderscore}{\kern0pt}def\ \isacommand{by}\isamarkupfalse%
\ auto\isanewline
\ \ \ \ \isacommand{then}\isamarkupfalse%
\ \isacommand{have}\isamarkupfalse%
\ {\isachardoublequoteopen}{\isacharparenleft}{\kern0pt}into{\isacharunderscore}{\kern0pt}super\ m\ {\isasymcirc}\isactrlsub c\ right{\isacharunderscore}{\kern0pt}coproj\ X\ {\isacharparenleft}{\kern0pt}Y\ {\isasymsetminus}\ {\isacharparenleft}{\kern0pt}X{\isacharcomma}{\kern0pt}\ m{\isacharparenright}{\kern0pt}{\isacharparenright}{\kern0pt}{\isacharparenright}{\kern0pt}\ {\isasymcirc}\isactrlsub c\ x{\isacharprime}{\kern0pt}\ {\isacharequal}{\kern0pt}\ {\isacharparenleft}{\kern0pt}into{\isacharunderscore}{\kern0pt}super\ m\ {\isasymcirc}\isactrlsub c\ left{\isacharunderscore}{\kern0pt}coproj\ X\ {\isacharparenleft}{\kern0pt}Y\ {\isasymsetminus}\ {\isacharparenleft}{\kern0pt}X{\isacharcomma}{\kern0pt}\ m{\isacharparenright}{\kern0pt}{\isacharparenright}{\kern0pt}{\isacharparenright}{\kern0pt}\ {\isasymcirc}\isactrlsub c\ y{\isacharprime}{\kern0pt}{\isachardoublequoteclose}\isanewline
\ \ \ \ \ \ \isacommand{using}\isamarkupfalse%
\ assms\ x{\isacharprime}{\kern0pt}{\isacharunderscore}{\kern0pt}type\ y{\isacharprime}{\kern0pt}{\isacharunderscore}{\kern0pt}type\ comp{\isacharunderscore}{\kern0pt}associative{\isadigit{2}}\ \isacommand{by}\isamarkupfalse%
\ {\isacharparenleft}{\kern0pt}typecheck{\isacharunderscore}{\kern0pt}cfuncs{\isacharcomma}{\kern0pt}\ auto{\isacharparenright}{\kern0pt}\isanewline
\ \ \ \ \isacommand{then}\isamarkupfalse%
\ \isacommand{have}\isamarkupfalse%
\ {\isachardoublequoteopen}m\isactrlsup c\ {\isasymcirc}\isactrlsub c\ x{\isacharprime}{\kern0pt}\ {\isacharequal}{\kern0pt}\ m\ {\isasymcirc}\isactrlsub c\ y{\isacharprime}{\kern0pt}{\isachardoublequoteclose}\isanewline
\ \ \ \ \ \ \isacommand{using}\isamarkupfalse%
\ assms\ \isacommand{unfolding}\isamarkupfalse%
\ into{\isacharunderscore}{\kern0pt}super{\isacharunderscore}{\kern0pt}def\isanewline
\ \ \ \ \ \ \isacommand{by}\isamarkupfalse%
\ {\isacharparenleft}{\kern0pt}simp\ add{\isacharcolon}{\kern0pt}\ complement{\isacharunderscore}{\kern0pt}morphism{\isacharunderscore}{\kern0pt}type\ left{\isacharunderscore}{\kern0pt}coproj{\isacharunderscore}{\kern0pt}cfunc{\isacharunderscore}{\kern0pt}coprod\ right{\isacharunderscore}{\kern0pt}coproj{\isacharunderscore}{\kern0pt}cfunc{\isacharunderscore}{\kern0pt}coprod{\isacharparenright}{\kern0pt}\isanewline
\ \ \ \ \isacommand{then}\isamarkupfalse%
\ \isacommand{have}\isamarkupfalse%
\ False\isanewline
\ \ \ \ \ \ \isacommand{using}\isamarkupfalse%
\ assms{\isacharparenleft}{\kern0pt}{\isadigit{1}}{\isacharparenright}{\kern0pt}\ assms{\isacharparenleft}{\kern0pt}{\isadigit{2}}{\isacharparenright}{\kern0pt}\ complement{\isacharunderscore}{\kern0pt}disjoint\ x{\isacharprime}{\kern0pt}{\isacharunderscore}{\kern0pt}type\ y{\isacharprime}{\kern0pt}{\isacharunderscore}{\kern0pt}type\ \isacommand{by}\isamarkupfalse%
\ fastforce\isanewline
\ \ \ \ \isacommand{then}\isamarkupfalse%
\ \isacommand{show}\isamarkupfalse%
\ {\isachardoublequoteopen}right{\isacharunderscore}{\kern0pt}coproj\ X\ {\isacharparenleft}{\kern0pt}Y\ {\isasymsetminus}\ {\isacharparenleft}{\kern0pt}X{\isacharcomma}{\kern0pt}\ m{\isacharparenright}{\kern0pt}{\isacharparenright}{\kern0pt}\ {\isasymcirc}\isactrlsub c\ x{\isacharprime}{\kern0pt}\ {\isacharequal}{\kern0pt}\ left{\isacharunderscore}{\kern0pt}coproj\ X\ {\isacharparenleft}{\kern0pt}Y\ {\isasymsetminus}\ {\isacharparenleft}{\kern0pt}X{\isacharcomma}{\kern0pt}\ m{\isacharparenright}{\kern0pt}{\isacharparenright}{\kern0pt}\ {\isasymcirc}\isactrlsub c\ y{\isacharprime}{\kern0pt}{\isachardoublequoteclose}\isanewline
\ \ \ \ \ \ \isacommand{by}\isamarkupfalse%
\ auto\isanewline
\ \ \isacommand{next}\isamarkupfalse%
\isanewline
\ \ \ \ \isacommand{fix}\isamarkupfalse%
\ x{\isacharprime}{\kern0pt}\ y{\isacharprime}{\kern0pt}\isanewline
\ \ \ \ \isacommand{assume}\isamarkupfalse%
\ x{\isacharprime}{\kern0pt}{\isacharunderscore}{\kern0pt}type{\isacharcolon}{\kern0pt}\ {\isachardoublequoteopen}x{\isacharprime}{\kern0pt}\ {\isasymin}\isactrlsub c\ Y\ {\isasymsetminus}\ {\isacharparenleft}{\kern0pt}X{\isacharcomma}{\kern0pt}\ m{\isacharparenright}{\kern0pt}{\isachardoublequoteclose}\ \isakeyword{and}\ x{\isacharunderscore}{\kern0pt}def{\isacharcolon}{\kern0pt}\ {\isachardoublequoteopen}x\ {\isacharequal}{\kern0pt}\ right{\isacharunderscore}{\kern0pt}coproj\ X\ {\isacharparenleft}{\kern0pt}Y\ {\isasymsetminus}\ {\isacharparenleft}{\kern0pt}X{\isacharcomma}{\kern0pt}\ m{\isacharparenright}{\kern0pt}{\isacharparenright}{\kern0pt}\ {\isasymcirc}\isactrlsub c\ x{\isacharprime}{\kern0pt}{\isachardoublequoteclose}\isanewline
\ \ \ \ \isacommand{assume}\isamarkupfalse%
\ y{\isacharprime}{\kern0pt}{\isacharunderscore}{\kern0pt}type{\isacharcolon}{\kern0pt}\ {\isachardoublequoteopen}y{\isacharprime}{\kern0pt}\ {\isasymin}\isactrlsub c\ Y\ {\isasymsetminus}\ {\isacharparenleft}{\kern0pt}X{\isacharcomma}{\kern0pt}\ m{\isacharparenright}{\kern0pt}{\isachardoublequoteclose}\ \isakeyword{and}\ y{\isacharunderscore}{\kern0pt}def{\isacharcolon}{\kern0pt}\ {\isachardoublequoteopen}y\ {\isacharequal}{\kern0pt}\ right{\isacharunderscore}{\kern0pt}coproj\ X\ {\isacharparenleft}{\kern0pt}Y\ {\isasymsetminus}\ {\isacharparenleft}{\kern0pt}X{\isacharcomma}{\kern0pt}\ m{\isacharparenright}{\kern0pt}{\isacharparenright}{\kern0pt}\ {\isasymcirc}\isactrlsub c\ y{\isacharprime}{\kern0pt}{\isachardoublequoteclose}\isanewline
\isanewline
\ \ \ \ \isacommand{have}\isamarkupfalse%
\ {\isachardoublequoteopen}into{\isacharunderscore}{\kern0pt}super\ m\ {\isasymcirc}\isactrlsub c\ right{\isacharunderscore}{\kern0pt}coproj\ X\ {\isacharparenleft}{\kern0pt}Y\ {\isasymsetminus}\ {\isacharparenleft}{\kern0pt}X{\isacharcomma}{\kern0pt}\ m{\isacharparenright}{\kern0pt}{\isacharparenright}{\kern0pt}\ {\isasymcirc}\isactrlsub c\ x{\isacharprime}{\kern0pt}\ {\isacharequal}{\kern0pt}\ into{\isacharunderscore}{\kern0pt}super\ m\ {\isasymcirc}\isactrlsub c\ right{\isacharunderscore}{\kern0pt}coproj\ X\ {\isacharparenleft}{\kern0pt}Y\ {\isasymsetminus}\ {\isacharparenleft}{\kern0pt}X{\isacharcomma}{\kern0pt}\ m{\isacharparenright}{\kern0pt}{\isacharparenright}{\kern0pt}\ {\isasymcirc}\isactrlsub c\ y{\isacharprime}{\kern0pt}{\isachardoublequoteclose}\isanewline
\ \ \ \ \ \ \isacommand{using}\isamarkupfalse%
\ into{\isacharunderscore}{\kern0pt}super{\isacharunderscore}{\kern0pt}eq\ \isacommand{unfolding}\isamarkupfalse%
\ x{\isacharunderscore}{\kern0pt}def\ y{\isacharunderscore}{\kern0pt}def\ \isacommand{by}\isamarkupfalse%
\ auto\isanewline
\ \ \ \ \isacommand{then}\isamarkupfalse%
\ \isacommand{have}\isamarkupfalse%
\ {\isachardoublequoteopen}{\isacharparenleft}{\kern0pt}into{\isacharunderscore}{\kern0pt}super\ m\ {\isasymcirc}\isactrlsub c\ right{\isacharunderscore}{\kern0pt}coproj\ X\ {\isacharparenleft}{\kern0pt}Y\ {\isasymsetminus}\ {\isacharparenleft}{\kern0pt}X{\isacharcomma}{\kern0pt}\ m{\isacharparenright}{\kern0pt}{\isacharparenright}{\kern0pt}{\isacharparenright}{\kern0pt}\ {\isasymcirc}\isactrlsub c\ x{\isacharprime}{\kern0pt}\ {\isacharequal}{\kern0pt}\ {\isacharparenleft}{\kern0pt}into{\isacharunderscore}{\kern0pt}super\ m\ {\isasymcirc}\isactrlsub c\ right{\isacharunderscore}{\kern0pt}coproj\ X\ {\isacharparenleft}{\kern0pt}Y\ {\isasymsetminus}\ {\isacharparenleft}{\kern0pt}X{\isacharcomma}{\kern0pt}\ m{\isacharparenright}{\kern0pt}{\isacharparenright}{\kern0pt}{\isacharparenright}{\kern0pt}\ {\isasymcirc}\isactrlsub c\ y{\isacharprime}{\kern0pt}{\isachardoublequoteclose}\isanewline
\ \ \ \ \ \ \isacommand{using}\isamarkupfalse%
\ assms\ x{\isacharprime}{\kern0pt}{\isacharunderscore}{\kern0pt}type\ y{\isacharprime}{\kern0pt}{\isacharunderscore}{\kern0pt}type\ comp{\isacharunderscore}{\kern0pt}associative{\isadigit{2}}\ \isacommand{by}\isamarkupfalse%
\ {\isacharparenleft}{\kern0pt}typecheck{\isacharunderscore}{\kern0pt}cfuncs{\isacharcomma}{\kern0pt}\ auto{\isacharparenright}{\kern0pt}\isanewline
\ \ \ \ \isacommand{then}\isamarkupfalse%
\ \isacommand{have}\isamarkupfalse%
\ {\isachardoublequoteopen}m\isactrlsup c\ {\isasymcirc}\isactrlsub c\ x{\isacharprime}{\kern0pt}\ {\isacharequal}{\kern0pt}\ m\isactrlsup c\ {\isasymcirc}\isactrlsub c\ y{\isacharprime}{\kern0pt}{\isachardoublequoteclose}\isanewline
\ \ \ \ \ \ \isacommand{using}\isamarkupfalse%
\ assms\ \isacommand{unfolding}\isamarkupfalse%
\ into{\isacharunderscore}{\kern0pt}super{\isacharunderscore}{\kern0pt}def\isanewline
\ \ \ \ \ \ \isacommand{by}\isamarkupfalse%
\ {\isacharparenleft}{\kern0pt}simp\ add{\isacharcolon}{\kern0pt}\ complement{\isacharunderscore}{\kern0pt}morphism{\isacharunderscore}{\kern0pt}type\ right{\isacharunderscore}{\kern0pt}coproj{\isacharunderscore}{\kern0pt}cfunc{\isacharunderscore}{\kern0pt}coprod{\isacharparenright}{\kern0pt}\isanewline
\ \ \ \ \isacommand{then}\isamarkupfalse%
\ \isacommand{have}\isamarkupfalse%
\ {\isachardoublequoteopen}x{\isacharprime}{\kern0pt}\ {\isacharequal}{\kern0pt}\ y{\isacharprime}{\kern0pt}{\isachardoublequoteclose}\isanewline
\ \ \ \ \ \ \isacommand{using}\isamarkupfalse%
\ assms\ complement{\isacharunderscore}{\kern0pt}morphism{\isacharunderscore}{\kern0pt}mono\ complement{\isacharunderscore}{\kern0pt}morphism{\isacharunderscore}{\kern0pt}type\ monomorphism{\isacharunderscore}{\kern0pt}def{\isadigit{2}}\ x{\isacharprime}{\kern0pt}{\isacharunderscore}{\kern0pt}type\ y{\isacharprime}{\kern0pt}{\isacharunderscore}{\kern0pt}type\ \isacommand{by}\isamarkupfalse%
\ blast\isanewline
\ \ \ \ \isacommand{then}\isamarkupfalse%
\ \isacommand{show}\isamarkupfalse%
\ {\isachardoublequoteopen}right{\isacharunderscore}{\kern0pt}coproj\ X\ {\isacharparenleft}{\kern0pt}Y\ {\isasymsetminus}\ {\isacharparenleft}{\kern0pt}X{\isacharcomma}{\kern0pt}\ m{\isacharparenright}{\kern0pt}{\isacharparenright}{\kern0pt}\ {\isasymcirc}\isactrlsub c\ x{\isacharprime}{\kern0pt}\ {\isacharequal}{\kern0pt}\ right{\isacharunderscore}{\kern0pt}coproj\ X\ {\isacharparenleft}{\kern0pt}Y\ {\isasymsetminus}\ {\isacharparenleft}{\kern0pt}X{\isacharcomma}{\kern0pt}\ m{\isacharparenright}{\kern0pt}{\isacharparenright}{\kern0pt}\ {\isasymcirc}\isactrlsub c\ y{\isacharprime}{\kern0pt}{\isachardoublequoteclose}\isanewline
\ \ \ \ \ \ \isacommand{by}\isamarkupfalse%
\ simp\isanewline
\ \ \isacommand{qed}\isamarkupfalse%
\isanewline
\isacommand{qed}\isamarkupfalse%
%
\endisatagproof
{\isafoldproof}%
%
\isadelimproof
\isanewline
%
\endisadelimproof
\isanewline
\isacommand{lemma}\isamarkupfalse%
\ into{\isacharunderscore}{\kern0pt}super{\isacharunderscore}{\kern0pt}epi{\isacharcolon}{\kern0pt}\isanewline
\ \ \isakeyword{assumes}\ {\isachardoublequoteopen}monomorphism\ m{\isachardoublequoteclose}\ {\isachardoublequoteopen}m\ {\isacharcolon}{\kern0pt}\ X\ {\isasymrightarrow}\ Y{\isachardoublequoteclose}\isanewline
\ \ \isakeyword{shows}\ {\isachardoublequoteopen}epimorphism\ {\isacharparenleft}{\kern0pt}into{\isacharunderscore}{\kern0pt}super\ m{\isacharparenright}{\kern0pt}{\isachardoublequoteclose}\isanewline
%
\isadelimproof
%
\endisadelimproof
%
\isatagproof
\isacommand{proof}\isamarkupfalse%
\ {\isacharparenleft}{\kern0pt}rule\ surjective{\isacharunderscore}{\kern0pt}is{\isacharunderscore}{\kern0pt}epimorphism{\isacharcomma}{\kern0pt}\ unfold\ surjective{\isacharunderscore}{\kern0pt}def{\isacharcomma}{\kern0pt}\ auto{\isacharparenright}{\kern0pt}\isanewline
\ \ \isacommand{fix}\isamarkupfalse%
\ y\isanewline
\ \ \isacommand{assume}\isamarkupfalse%
\ {\isachardoublequoteopen}y\ {\isasymin}\isactrlsub c\ codomain\ {\isacharparenleft}{\kern0pt}into{\isacharunderscore}{\kern0pt}super\ m{\isacharparenright}{\kern0pt}{\isachardoublequoteclose}\isanewline
\ \ \isacommand{then}\isamarkupfalse%
\ \isacommand{have}\isamarkupfalse%
\ y{\isacharunderscore}{\kern0pt}type{\isacharcolon}{\kern0pt}\ {\isachardoublequoteopen}y\ {\isasymin}\isactrlsub c\ Y{\isachardoublequoteclose}\isanewline
\ \ \ \ \isacommand{using}\isamarkupfalse%
\ assms\ cfunc{\isacharunderscore}{\kern0pt}type{\isacharunderscore}{\kern0pt}def\ into{\isacharunderscore}{\kern0pt}super{\isacharunderscore}{\kern0pt}type\ \isacommand{by}\isamarkupfalse%
\ auto\isanewline
\isanewline
\ \ \isacommand{have}\isamarkupfalse%
\ y{\isacharunderscore}{\kern0pt}cases{\isacharcolon}{\kern0pt}\ {\isachardoublequoteopen}{\isacharparenleft}{\kern0pt}characteristic{\isacharunderscore}{\kern0pt}func\ m\ {\isasymcirc}\isactrlsub c\ y\ {\isacharequal}{\kern0pt}\ {\isasymt}{\isacharparenright}{\kern0pt}\ {\isasymor}\ {\isacharparenleft}{\kern0pt}characteristic{\isacharunderscore}{\kern0pt}func\ m\ {\isasymcirc}\isactrlsub c\ y\ {\isacharequal}{\kern0pt}\ {\isasymf}{\isacharparenright}{\kern0pt}{\isachardoublequoteclose}\isanewline
\ \ \ \ \isacommand{using}\isamarkupfalse%
\ y{\isacharunderscore}{\kern0pt}type\ assms\ true{\isacharunderscore}{\kern0pt}false{\isacharunderscore}{\kern0pt}only{\isacharunderscore}{\kern0pt}truth{\isacharunderscore}{\kern0pt}values\ \isacommand{by}\isamarkupfalse%
\ {\isacharparenleft}{\kern0pt}typecheck{\isacharunderscore}{\kern0pt}cfuncs{\isacharcomma}{\kern0pt}\ blast{\isacharparenright}{\kern0pt}\isanewline
\ \ \isacommand{then}\isamarkupfalse%
\ \isacommand{show}\isamarkupfalse%
\ {\isachardoublequoteopen}{\isasymexists}x{\isachardot}{\kern0pt}\ x\ {\isasymin}\isactrlsub c\ domain\ {\isacharparenleft}{\kern0pt}into{\isacharunderscore}{\kern0pt}super\ m{\isacharparenright}{\kern0pt}\ {\isasymand}\ into{\isacharunderscore}{\kern0pt}super\ m\ {\isasymcirc}\isactrlsub c\ x\ {\isacharequal}{\kern0pt}\ y{\isachardoublequoteclose}\isanewline
\ \ \isacommand{proof}\isamarkupfalse%
\ auto\isanewline
\ \ \ \ \isacommand{assume}\isamarkupfalse%
\ {\isachardoublequoteopen}characteristic{\isacharunderscore}{\kern0pt}func\ m\ {\isasymcirc}\isactrlsub c\ y\ {\isacharequal}{\kern0pt}\ {\isasymt}{\isachardoublequoteclose}\isanewline
\ \ \ \ \isacommand{then}\isamarkupfalse%
\ \isacommand{have}\isamarkupfalse%
\ {\isachardoublequoteopen}y\ {\isasymin}\isactrlbsub Y\isactrlesub \ {\isacharparenleft}{\kern0pt}X{\isacharcomma}{\kern0pt}\ m{\isacharparenright}{\kern0pt}{\isachardoublequoteclose}\isanewline
\ \ \ \ \ \ \isacommand{by}\isamarkupfalse%
\ {\isacharparenleft}{\kern0pt}simp\ add{\isacharcolon}{\kern0pt}\ assms\ characteristic{\isacharunderscore}{\kern0pt}func{\isacharunderscore}{\kern0pt}true{\isacharunderscore}{\kern0pt}relative{\isacharunderscore}{\kern0pt}member\ y{\isacharunderscore}{\kern0pt}type{\isacharparenright}{\kern0pt}\isanewline
\ \ \ \ \isacommand{then}\isamarkupfalse%
\ \isacommand{obtain}\isamarkupfalse%
\ x\ \isakeyword{where}\ x{\isacharunderscore}{\kern0pt}type{\isacharcolon}{\kern0pt}\ {\isachardoublequoteopen}x\ {\isasymin}\isactrlsub c\ X{\isachardoublequoteclose}\ \isakeyword{and}\ x{\isacharunderscore}{\kern0pt}def{\isacharcolon}{\kern0pt}\ {\isachardoublequoteopen}y\ {\isacharequal}{\kern0pt}\ m\ {\isasymcirc}\isactrlsub c\ x{\isachardoublequoteclose}\isanewline
\ \ \ \ \ \ \isacommand{by}\isamarkupfalse%
\ {\isacharparenleft}{\kern0pt}unfold\ relative{\isacharunderscore}{\kern0pt}member{\isacharunderscore}{\kern0pt}def{\isadigit{2}}{\isacharcomma}{\kern0pt}\ auto{\isacharcomma}{\kern0pt}\ unfold\ factors{\isacharunderscore}{\kern0pt}through{\isacharunderscore}{\kern0pt}def{\isadigit{2}}{\isacharcomma}{\kern0pt}\ auto{\isacharparenright}{\kern0pt}\isanewline
\ \ \ \ \isacommand{then}\isamarkupfalse%
\ \isacommand{show}\isamarkupfalse%
\ {\isachardoublequoteopen}{\isasymexists}x{\isachardot}{\kern0pt}\ x\ {\isasymin}\isactrlsub c\ domain\ {\isacharparenleft}{\kern0pt}into{\isacharunderscore}{\kern0pt}super\ m{\isacharparenright}{\kern0pt}\ {\isasymand}\ into{\isacharunderscore}{\kern0pt}super\ m\ {\isasymcirc}\isactrlsub c\ x\ {\isacharequal}{\kern0pt}\ y{\isachardoublequoteclose}\isanewline
\ \ \ \ \ \ \isacommand{unfolding}\isamarkupfalse%
\ into{\isacharunderscore}{\kern0pt}super{\isacharunderscore}{\kern0pt}def\ \isacommand{using}\isamarkupfalse%
\ assms\ cfunc{\isacharunderscore}{\kern0pt}type{\isacharunderscore}{\kern0pt}def\ comp{\isacharunderscore}{\kern0pt}associative\ left{\isacharunderscore}{\kern0pt}coproj{\isacharunderscore}{\kern0pt}cfunc{\isacharunderscore}{\kern0pt}coprod\isanewline
\ \ \ \ \ \ \isacommand{by}\isamarkupfalse%
\ {\isacharparenleft}{\kern0pt}rule{\isacharunderscore}{\kern0pt}tac\ x{\isacharequal}{\kern0pt}{\isachardoublequoteopen}left{\isacharunderscore}{\kern0pt}coproj\ X\ {\isacharparenleft}{\kern0pt}Y\ {\isasymsetminus}\ {\isacharparenleft}{\kern0pt}X{\isacharcomma}{\kern0pt}\ m{\isacharparenright}{\kern0pt}{\isacharparenright}{\kern0pt}\ {\isasymcirc}\isactrlsub c\ x{\isachardoublequoteclose}\ \isakeyword{in}\ exI{\isacharcomma}{\kern0pt}\ typecheck{\isacharunderscore}{\kern0pt}cfuncs{\isacharcomma}{\kern0pt}\ metis{\isacharparenright}{\kern0pt}\isanewline
\ \ \isacommand{next}\isamarkupfalse%
\isanewline
\ \ \ \ \isacommand{assume}\isamarkupfalse%
\ {\isachardoublequoteopen}characteristic{\isacharunderscore}{\kern0pt}func\ m\ {\isasymcirc}\isactrlsub c\ y\ {\isacharequal}{\kern0pt}\ {\isasymf}{\isachardoublequoteclose}\isanewline
\ \ \ \ \isacommand{then}\isamarkupfalse%
\ \isacommand{have}\isamarkupfalse%
\ {\isachardoublequoteopen}{\isasymnot}\ y\ {\isasymin}\isactrlbsub Y\isactrlesub \ {\isacharparenleft}{\kern0pt}X{\isacharcomma}{\kern0pt}\ m{\isacharparenright}{\kern0pt}{\isachardoublequoteclose}\isanewline
\ \ \ \ \ \ \isacommand{by}\isamarkupfalse%
\ {\isacharparenleft}{\kern0pt}simp\ add{\isacharcolon}{\kern0pt}\ assms\ characteristic{\isacharunderscore}{\kern0pt}func{\isacharunderscore}{\kern0pt}false{\isacharunderscore}{\kern0pt}not{\isacharunderscore}{\kern0pt}relative{\isacharunderscore}{\kern0pt}member\ y{\isacharunderscore}{\kern0pt}type{\isacharparenright}{\kern0pt}\isanewline
\ \ \ \ \isacommand{then}\isamarkupfalse%
\ \isacommand{have}\isamarkupfalse%
\ {\isachardoublequoteopen}y\ {\isasymin}\isactrlbsub Y\isactrlesub \ {\isacharparenleft}{\kern0pt}Y\ {\isasymsetminus}\ {\isacharparenleft}{\kern0pt}X{\isacharcomma}{\kern0pt}\ m{\isacharparenright}{\kern0pt}{\isacharcomma}{\kern0pt}\ m\isactrlsup c{\isacharparenright}{\kern0pt}{\isachardoublequoteclose}\isanewline
\ \ \ \ \ \ \isacommand{by}\isamarkupfalse%
\ {\isacharparenleft}{\kern0pt}simp\ add{\isacharcolon}{\kern0pt}\ assms\ not{\isacharunderscore}{\kern0pt}in{\isacharunderscore}{\kern0pt}subset{\isacharunderscore}{\kern0pt}in{\isacharunderscore}{\kern0pt}complement\ y{\isacharunderscore}{\kern0pt}type{\isacharparenright}{\kern0pt}\isanewline
\ \ \ \ \isacommand{then}\isamarkupfalse%
\ \isacommand{obtain}\isamarkupfalse%
\ x{\isacharprime}{\kern0pt}\ \isakeyword{where}\ x{\isacharprime}{\kern0pt}{\isacharunderscore}{\kern0pt}type{\isacharcolon}{\kern0pt}\ {\isachardoublequoteopen}x{\isacharprime}{\kern0pt}\ {\isasymin}\isactrlsub c\ Y\ {\isasymsetminus}\ {\isacharparenleft}{\kern0pt}X{\isacharcomma}{\kern0pt}\ m{\isacharparenright}{\kern0pt}{\isachardoublequoteclose}\ \isakeyword{and}\ x{\isacharprime}{\kern0pt}{\isacharunderscore}{\kern0pt}def{\isacharcolon}{\kern0pt}\ {\isachardoublequoteopen}y\ {\isacharequal}{\kern0pt}\ m\isactrlsup c\ {\isasymcirc}\isactrlsub c\ x{\isacharprime}{\kern0pt}{\isachardoublequoteclose}\isanewline
\ \ \ \ \ \ \isacommand{by}\isamarkupfalse%
\ {\isacharparenleft}{\kern0pt}unfold\ relative{\isacharunderscore}{\kern0pt}member{\isacharunderscore}{\kern0pt}def{\isadigit{2}}{\isacharcomma}{\kern0pt}\ auto{\isacharcomma}{\kern0pt}\ unfold\ factors{\isacharunderscore}{\kern0pt}through{\isacharunderscore}{\kern0pt}def{\isadigit{2}}{\isacharcomma}{\kern0pt}\ auto{\isacharparenright}{\kern0pt}\isanewline
\ \ \ \ \isacommand{then}\isamarkupfalse%
\ \isacommand{show}\isamarkupfalse%
\ {\isachardoublequoteopen}{\isasymexists}x{\isachardot}{\kern0pt}\ x\ {\isasymin}\isactrlsub c\ domain\ {\isacharparenleft}{\kern0pt}into{\isacharunderscore}{\kern0pt}super\ m{\isacharparenright}{\kern0pt}\ {\isasymand}\ into{\isacharunderscore}{\kern0pt}super\ m\ {\isasymcirc}\isactrlsub c\ x\ {\isacharequal}{\kern0pt}\ y{\isachardoublequoteclose}\isanewline
\ \ \ \ \ \ \isacommand{unfolding}\isamarkupfalse%
\ into{\isacharunderscore}{\kern0pt}super{\isacharunderscore}{\kern0pt}def\ \isacommand{using}\isamarkupfalse%
\ assms\ cfunc{\isacharunderscore}{\kern0pt}type{\isacharunderscore}{\kern0pt}def\ comp{\isacharunderscore}{\kern0pt}associative\ right{\isacharunderscore}{\kern0pt}coproj{\isacharunderscore}{\kern0pt}cfunc{\isacharunderscore}{\kern0pt}coprod\isanewline
\ \ \ \ \ \ \isacommand{by}\isamarkupfalse%
\ {\isacharparenleft}{\kern0pt}rule{\isacharunderscore}{\kern0pt}tac\ x{\isacharequal}{\kern0pt}{\isachardoublequoteopen}right{\isacharunderscore}{\kern0pt}coproj\ X\ {\isacharparenleft}{\kern0pt}Y\ {\isasymsetminus}\ {\isacharparenleft}{\kern0pt}X{\isacharcomma}{\kern0pt}\ m{\isacharparenright}{\kern0pt}{\isacharparenright}{\kern0pt}\ {\isasymcirc}\isactrlsub c\ x{\isacharprime}{\kern0pt}{\isachardoublequoteclose}\ \isakeyword{in}\ exI{\isacharcomma}{\kern0pt}\ typecheck{\isacharunderscore}{\kern0pt}cfuncs{\isacharcomma}{\kern0pt}\ metis{\isacharparenright}{\kern0pt}\isanewline
\ \ \isacommand{qed}\isamarkupfalse%
\isanewline
\isacommand{qed}\isamarkupfalse%
%
\endisatagproof
{\isafoldproof}%
%
\isadelimproof
\isanewline
%
\endisadelimproof
\isanewline
\isacommand{lemma}\isamarkupfalse%
\ into{\isacharunderscore}{\kern0pt}super{\isacharunderscore}{\kern0pt}iso{\isacharcolon}{\kern0pt}\isanewline
\ \ \isakeyword{assumes}\ {\isachardoublequoteopen}monomorphism\ m{\isachardoublequoteclose}\ {\isachardoublequoteopen}m\ {\isacharcolon}{\kern0pt}\ X\ {\isasymrightarrow}\ Y{\isachardoublequoteclose}\isanewline
\ \ \isakeyword{shows}\ {\isachardoublequoteopen}isomorphism\ {\isacharparenleft}{\kern0pt}into{\isacharunderscore}{\kern0pt}super\ m{\isacharparenright}{\kern0pt}{\isachardoublequoteclose}\isanewline
%
\isadelimproof
\ \ %
\endisadelimproof
%
\isatagproof
\isacommand{using}\isamarkupfalse%
\ assms\ epi{\isacharunderscore}{\kern0pt}mon{\isacharunderscore}{\kern0pt}is{\isacharunderscore}{\kern0pt}iso\ into{\isacharunderscore}{\kern0pt}super{\isacharunderscore}{\kern0pt}epi\ into{\isacharunderscore}{\kern0pt}super{\isacharunderscore}{\kern0pt}mono\ \isacommand{by}\isamarkupfalse%
\ auto%
\endisatagproof
{\isafoldproof}%
%
\isadelimproof
%
\endisadelimproof
%
\isadelimdocument
%
\endisadelimdocument
%
\isatagdocument
%
\isamarkupsubsubsection{Going from a set to a subset or its complement%
}
\isamarkuptrue%
%
\endisatagdocument
{\isafolddocument}%
%
\isadelimdocument
%
\endisadelimdocument
\isacommand{definition}\isamarkupfalse%
\ try{\isacharunderscore}{\kern0pt}cast\ {\isacharcolon}{\kern0pt}{\isacharcolon}{\kern0pt}\ {\isachardoublequoteopen}cfunc\ {\isasymRightarrow}\ cfunc{\isachardoublequoteclose}\ \isakeyword{where}\isanewline
\ \ {\isachardoublequoteopen}try{\isacharunderscore}{\kern0pt}cast\ m\ {\isacharequal}{\kern0pt}\ {\isacharparenleft}{\kern0pt}THE\ m{\isacharprime}{\kern0pt}{\isachardot}{\kern0pt}\ m{\isacharprime}{\kern0pt}\ {\isacharcolon}{\kern0pt}\ codomain\ m\ {\isasymrightarrow}\ domain\ m\ {\isasymCoprod}\ {\isacharparenleft}{\kern0pt}{\isacharparenleft}{\kern0pt}codomain\ m{\isacharparenright}{\kern0pt}\ {\isasymsetminus}\ {\isacharparenleft}{\kern0pt}{\isacharparenleft}{\kern0pt}domain\ m{\isacharparenright}{\kern0pt}{\isacharcomma}{\kern0pt}m{\isacharparenright}{\kern0pt}{\isacharparenright}{\kern0pt}\isanewline
\ \ \ \ {\isasymand}\ m{\isacharprime}{\kern0pt}\ {\isasymcirc}\isactrlsub c\ into{\isacharunderscore}{\kern0pt}super\ m\ {\isacharequal}{\kern0pt}\ id\ {\isacharparenleft}{\kern0pt}domain\ m\ {\isasymCoprod}\ {\isacharparenleft}{\kern0pt}codomain\ m\ {\isasymsetminus}\ {\isacharparenleft}{\kern0pt}{\isacharparenleft}{\kern0pt}domain\ m{\isacharparenright}{\kern0pt}{\isacharcomma}{\kern0pt}m{\isacharparenright}{\kern0pt}{\isacharparenright}{\kern0pt}{\isacharparenright}{\kern0pt}\isanewline
\ \ \ \ {\isasymand}\ into{\isacharunderscore}{\kern0pt}super\ m\ {\isasymcirc}\isactrlsub c\ m{\isacharprime}{\kern0pt}\ {\isacharequal}{\kern0pt}\ id\ {\isacharparenleft}{\kern0pt}codomain\ m{\isacharparenright}{\kern0pt}{\isacharparenright}{\kern0pt}{\isachardoublequoteclose}\isanewline
\isanewline
\isacommand{lemma}\isamarkupfalse%
\ try{\isacharunderscore}{\kern0pt}cast{\isacharunderscore}{\kern0pt}def{\isadigit{2}}{\isacharcolon}{\kern0pt}\isanewline
\ \ \isakeyword{assumes}\ {\isachardoublequoteopen}monomorphism\ m{\isachardoublequoteclose}\ {\isachardoublequoteopen}m\ {\isacharcolon}{\kern0pt}\ X\ {\isasymrightarrow}\ Y{\isachardoublequoteclose}\isanewline
\ \ \isakeyword{shows}\ {\isachardoublequoteopen}try{\isacharunderscore}{\kern0pt}cast\ m\ {\isacharcolon}{\kern0pt}\ codomain\ m\ {\isasymrightarrow}\ {\isacharparenleft}{\kern0pt}domain\ m{\isacharparenright}{\kern0pt}\ {\isasymCoprod}\ {\isacharparenleft}{\kern0pt}{\isacharparenleft}{\kern0pt}codomain\ m{\isacharparenright}{\kern0pt}\ {\isasymsetminus}\ {\isacharparenleft}{\kern0pt}{\isacharparenleft}{\kern0pt}domain\ m{\isacharparenright}{\kern0pt}{\isacharcomma}{\kern0pt}m{\isacharparenright}{\kern0pt}{\isacharparenright}{\kern0pt}\isanewline
\ \ \ \ {\isasymand}\ try{\isacharunderscore}{\kern0pt}cast\ m\ {\isasymcirc}\isactrlsub c\ into{\isacharunderscore}{\kern0pt}super\ m\ {\isacharequal}{\kern0pt}\ id\ {\isacharparenleft}{\kern0pt}{\isacharparenleft}{\kern0pt}domain\ m{\isacharparenright}{\kern0pt}\ {\isasymCoprod}\ {\isacharparenleft}{\kern0pt}{\isacharparenleft}{\kern0pt}codomain\ m{\isacharparenright}{\kern0pt}\ {\isasymsetminus}\ {\isacharparenleft}{\kern0pt}{\isacharparenleft}{\kern0pt}domain\ m{\isacharparenright}{\kern0pt}{\isacharcomma}{\kern0pt}m{\isacharparenright}{\kern0pt}{\isacharparenright}{\kern0pt}{\isacharparenright}{\kern0pt}\isanewline
\ \ \ \ {\isasymand}\ into{\isacharunderscore}{\kern0pt}super\ m\ {\isasymcirc}\isactrlsub c\ try{\isacharunderscore}{\kern0pt}cast\ m\ {\isacharequal}{\kern0pt}\ id\ {\isacharparenleft}{\kern0pt}codomain\ m{\isacharparenright}{\kern0pt}{\isachardoublequoteclose}\isanewline
%
\isadelimproof
\ \ %
\endisadelimproof
%
\isatagproof
\isacommand{unfolding}\isamarkupfalse%
\ try{\isacharunderscore}{\kern0pt}cast{\isacharunderscore}{\kern0pt}def\isanewline
\isacommand{proof}\isamarkupfalse%
\ {\isacharparenleft}{\kern0pt}rule\ theI{\isacharprime}{\kern0pt}{\isacharcomma}{\kern0pt}\ auto{\isacharparenright}{\kern0pt}\isanewline
\ \ \isacommand{show}\isamarkupfalse%
\ {\isachardoublequoteopen}{\isasymexists}x{\isachardot}{\kern0pt}\ x\ {\isacharcolon}{\kern0pt}\ codomain\ m\ {\isasymrightarrow}\ domain\ m\ {\isasymCoprod}\ {\isacharparenleft}{\kern0pt}codomain\ m\ {\isasymsetminus}\ {\isacharparenleft}{\kern0pt}domain\ m{\isacharcomma}{\kern0pt}\ m{\isacharparenright}{\kern0pt}{\isacharparenright}{\kern0pt}\ {\isasymand}\isanewline
\ \ \ \ \ \ \ \ x\ {\isasymcirc}\isactrlsub c\ into{\isacharunderscore}{\kern0pt}super\ m\ {\isacharequal}{\kern0pt}\ id\isactrlsub c\ {\isacharparenleft}{\kern0pt}domain\ m\ {\isasymCoprod}\ {\isacharparenleft}{\kern0pt}codomain\ m\ {\isasymsetminus}\ {\isacharparenleft}{\kern0pt}domain\ m{\isacharcomma}{\kern0pt}\ m{\isacharparenright}{\kern0pt}{\isacharparenright}{\kern0pt}{\isacharparenright}{\kern0pt}\ {\isasymand}\isanewline
\ \ \ \ \ \ \ \ into{\isacharunderscore}{\kern0pt}super\ m\ {\isasymcirc}\isactrlsub c\ x\ {\isacharequal}{\kern0pt}\ id\isactrlsub c\ {\isacharparenleft}{\kern0pt}codomain\ m{\isacharparenright}{\kern0pt}{\isachardoublequoteclose}\isanewline
\ \ \ \ \isacommand{using}\isamarkupfalse%
\ assms\ into{\isacharunderscore}{\kern0pt}super{\isacharunderscore}{\kern0pt}iso\ cfunc{\isacharunderscore}{\kern0pt}type{\isacharunderscore}{\kern0pt}def\ into{\isacharunderscore}{\kern0pt}super{\isacharunderscore}{\kern0pt}type\ \isacommand{unfolding}\isamarkupfalse%
\ isomorphism{\isacharunderscore}{\kern0pt}def\ \isacommand{by}\isamarkupfalse%
\ fastforce\isanewline
\isacommand{next}\isamarkupfalse%
\isanewline
\ \ \isacommand{fix}\isamarkupfalse%
\ x\ y\isanewline
\ \ \isacommand{assume}\isamarkupfalse%
\ x{\isacharunderscore}{\kern0pt}type{\isacharcolon}{\kern0pt}\ {\isachardoublequoteopen}x\ {\isacharcolon}{\kern0pt}\ codomain\ m\ {\isasymrightarrow}\ domain\ m\ {\isasymCoprod}\ {\isacharparenleft}{\kern0pt}codomain\ m\ {\isasymsetminus}\ {\isacharparenleft}{\kern0pt}domain\ m{\isacharcomma}{\kern0pt}\ m{\isacharparenright}{\kern0pt}{\isacharparenright}{\kern0pt}{\isachardoublequoteclose}\isanewline
\ \ \isacommand{assume}\isamarkupfalse%
\ y{\isacharunderscore}{\kern0pt}type{\isacharcolon}{\kern0pt}\ {\isachardoublequoteopen}y\ {\isacharcolon}{\kern0pt}\ codomain\ m\ {\isasymrightarrow}\ domain\ m\ {\isasymCoprod}\ {\isacharparenleft}{\kern0pt}codomain\ m\ {\isasymsetminus}\ {\isacharparenleft}{\kern0pt}domain\ m{\isacharcomma}{\kern0pt}\ m{\isacharparenright}{\kern0pt}{\isacharparenright}{\kern0pt}{\isachardoublequoteclose}\isanewline
\ \ \isacommand{assume}\isamarkupfalse%
\ {\isachardoublequoteopen}into{\isacharunderscore}{\kern0pt}super\ m\ {\isasymcirc}\isactrlsub c\ x\ {\isacharequal}{\kern0pt}\ id\isactrlsub c\ {\isacharparenleft}{\kern0pt}codomain\ m{\isacharparenright}{\kern0pt}{\isachardoublequoteclose}\ \isakeyword{and}\ {\isachardoublequoteopen}into{\isacharunderscore}{\kern0pt}super\ m\ {\isasymcirc}\isactrlsub c\ y\ {\isacharequal}{\kern0pt}\ id\isactrlsub c\ {\isacharparenleft}{\kern0pt}codomain\ m{\isacharparenright}{\kern0pt}{\isachardoublequoteclose}\isanewline
\ \ \isacommand{then}\isamarkupfalse%
\ \isacommand{have}\isamarkupfalse%
\ {\isachardoublequoteopen}into{\isacharunderscore}{\kern0pt}super\ m\ {\isasymcirc}\isactrlsub c\ x\ {\isacharequal}{\kern0pt}\ into{\isacharunderscore}{\kern0pt}super\ m\ {\isasymcirc}\isactrlsub c\ y{\isachardoublequoteclose}\isanewline
\ \ \ \ \isacommand{by}\isamarkupfalse%
\ auto\isanewline
\ \ \isacommand{then}\isamarkupfalse%
\ \isacommand{show}\isamarkupfalse%
\ {\isachardoublequoteopen}x\ {\isacharequal}{\kern0pt}\ y{\isachardoublequoteclose}\isanewline
\ \ \ \ \isacommand{using}\isamarkupfalse%
\ into{\isacharunderscore}{\kern0pt}super{\isacharunderscore}{\kern0pt}mono\ \isacommand{unfolding}\isamarkupfalse%
\ monomorphism{\isacharunderscore}{\kern0pt}def\isanewline
\ \ \ \ \isacommand{by}\isamarkupfalse%
\ {\isacharparenleft}{\kern0pt}metis\ assms{\isacharparenleft}{\kern0pt}{\isadigit{1}}{\isacharparenright}{\kern0pt}\ cfunc{\isacharunderscore}{\kern0pt}type{\isacharunderscore}{\kern0pt}def\ into{\isacharunderscore}{\kern0pt}super{\isacharunderscore}{\kern0pt}type\ monomorphism{\isacharunderscore}{\kern0pt}def\ x{\isacharunderscore}{\kern0pt}type\ y{\isacharunderscore}{\kern0pt}type{\isacharparenright}{\kern0pt}\isanewline
\isacommand{qed}\isamarkupfalse%
%
\endisatagproof
{\isafoldproof}%
%
\isadelimproof
\isanewline
%
\endisadelimproof
\isanewline
\isacommand{lemma}\isamarkupfalse%
\ try{\isacharunderscore}{\kern0pt}cast{\isacharunderscore}{\kern0pt}type{\isacharbrackleft}{\kern0pt}type{\isacharunderscore}{\kern0pt}rule{\isacharbrackright}{\kern0pt}{\isacharcolon}{\kern0pt}\isanewline
\ \ \isakeyword{assumes}\ {\isachardoublequoteopen}monomorphism\ m{\isachardoublequoteclose}\ {\isachardoublequoteopen}m\ {\isacharcolon}{\kern0pt}\ X\ {\isasymrightarrow}\ Y{\isachardoublequoteclose}\isanewline
\ \ \isakeyword{shows}\ {\isachardoublequoteopen}try{\isacharunderscore}{\kern0pt}cast\ m\ {\isacharcolon}{\kern0pt}\ Y\ {\isasymrightarrow}\ X\ {\isasymCoprod}\ {\isacharparenleft}{\kern0pt}Y\ {\isasymsetminus}\ {\isacharparenleft}{\kern0pt}X{\isacharcomma}{\kern0pt}m{\isacharparenright}{\kern0pt}{\isacharparenright}{\kern0pt}{\isachardoublequoteclose}\isanewline
%
\isadelimproof
\ \ %
\endisadelimproof
%
\isatagproof
\isacommand{using}\isamarkupfalse%
\ assms\ cfunc{\isacharunderscore}{\kern0pt}type{\isacharunderscore}{\kern0pt}def\ try{\isacharunderscore}{\kern0pt}cast{\isacharunderscore}{\kern0pt}def{\isadigit{2}}\ \isacommand{by}\isamarkupfalse%
\ auto%
\endisatagproof
{\isafoldproof}%
%
\isadelimproof
\ \isanewline
%
\endisadelimproof
\isanewline
\isacommand{lemma}\isamarkupfalse%
\ try{\isacharunderscore}{\kern0pt}cast{\isacharunderscore}{\kern0pt}into{\isacharunderscore}{\kern0pt}super{\isacharcolon}{\kern0pt}\isanewline
\ \ \isakeyword{assumes}\ {\isachardoublequoteopen}monomorphism\ m{\isachardoublequoteclose}\ {\isachardoublequoteopen}m\ {\isacharcolon}{\kern0pt}\ X\ {\isasymrightarrow}\ Y{\isachardoublequoteclose}\isanewline
\ \ \isakeyword{shows}\ {\isachardoublequoteopen}try{\isacharunderscore}{\kern0pt}cast\ m\ {\isasymcirc}\isactrlsub c\ into{\isacharunderscore}{\kern0pt}super\ m\ {\isacharequal}{\kern0pt}\ id\ {\isacharparenleft}{\kern0pt}X\ {\isasymCoprod}\ {\isacharparenleft}{\kern0pt}Y\ {\isasymsetminus}\ {\isacharparenleft}{\kern0pt}X{\isacharcomma}{\kern0pt}m{\isacharparenright}{\kern0pt}{\isacharparenright}{\kern0pt}{\isacharparenright}{\kern0pt}{\isachardoublequoteclose}\isanewline
%
\isadelimproof
\ \ %
\endisadelimproof
%
\isatagproof
\isacommand{using}\isamarkupfalse%
\ assms\ cfunc{\isacharunderscore}{\kern0pt}type{\isacharunderscore}{\kern0pt}def\ try{\isacharunderscore}{\kern0pt}cast{\isacharunderscore}{\kern0pt}def{\isadigit{2}}\ \isacommand{by}\isamarkupfalse%
\ auto%
\endisatagproof
{\isafoldproof}%
%
\isadelimproof
\isanewline
%
\endisadelimproof
\isanewline
\isacommand{lemma}\isamarkupfalse%
\ into{\isacharunderscore}{\kern0pt}super{\isacharunderscore}{\kern0pt}try{\isacharunderscore}{\kern0pt}cast{\isacharcolon}{\kern0pt}\isanewline
\ \ \isakeyword{assumes}\ {\isachardoublequoteopen}monomorphism\ m{\isachardoublequoteclose}\ {\isachardoublequoteopen}m\ {\isacharcolon}{\kern0pt}\ X\ {\isasymrightarrow}\ Y{\isachardoublequoteclose}\isanewline
\ \ \isakeyword{shows}\ {\isachardoublequoteopen}into{\isacharunderscore}{\kern0pt}super\ m\ {\isasymcirc}\isactrlsub c\ \ try{\isacharunderscore}{\kern0pt}cast\ m\ {\isacharequal}{\kern0pt}\ id\ Y{\isachardoublequoteclose}\isanewline
%
\isadelimproof
\ \ %
\endisadelimproof
%
\isatagproof
\isacommand{using}\isamarkupfalse%
\ assms\ cfunc{\isacharunderscore}{\kern0pt}type{\isacharunderscore}{\kern0pt}def\ try{\isacharunderscore}{\kern0pt}cast{\isacharunderscore}{\kern0pt}def{\isadigit{2}}\ \isacommand{by}\isamarkupfalse%
\ auto%
\endisatagproof
{\isafoldproof}%
%
\isadelimproof
\isanewline
%
\endisadelimproof
\isanewline
\isacommand{lemma}\isamarkupfalse%
\ try{\isacharunderscore}{\kern0pt}cast{\isacharunderscore}{\kern0pt}in{\isacharunderscore}{\kern0pt}X{\isacharcolon}{\kern0pt}\isanewline
\ \ \isakeyword{assumes}\ m{\isacharunderscore}{\kern0pt}type{\isacharcolon}{\kern0pt}\ {\isachardoublequoteopen}monomorphism\ m{\isachardoublequoteclose}\ {\isachardoublequoteopen}m\ {\isacharcolon}{\kern0pt}\ X\ {\isasymrightarrow}\ Y{\isachardoublequoteclose}\isanewline
\ \ \isakeyword{assumes}\ y{\isacharunderscore}{\kern0pt}in{\isacharunderscore}{\kern0pt}X{\isacharcolon}{\kern0pt}\ {\isachardoublequoteopen}y\ {\isasymin}\isactrlbsub Y\isactrlesub \ {\isacharparenleft}{\kern0pt}X{\isacharcomma}{\kern0pt}\ m{\isacharparenright}{\kern0pt}{\isachardoublequoteclose}\isanewline
\ \ \isakeyword{shows}\ {\isachardoublequoteopen}{\isasymexists}\ x{\isachardot}{\kern0pt}\ x\ {\isasymin}\isactrlsub c\ X\ {\isasymand}\ try{\isacharunderscore}{\kern0pt}cast\ m\ {\isasymcirc}\isactrlsub c\ y\ {\isacharequal}{\kern0pt}\ left{\isacharunderscore}{\kern0pt}coproj\ X\ {\isacharparenleft}{\kern0pt}Y\ {\isasymsetminus}\ {\isacharparenleft}{\kern0pt}X{\isacharcomma}{\kern0pt}m{\isacharparenright}{\kern0pt}{\isacharparenright}{\kern0pt}\ {\isasymcirc}\isactrlsub c\ x{\isachardoublequoteclose}\isanewline
%
\isadelimproof
%
\endisadelimproof
%
\isatagproof
\isacommand{proof}\isamarkupfalse%
\ {\isacharminus}{\kern0pt}\isanewline
\ \ \isacommand{have}\isamarkupfalse%
\ y{\isacharunderscore}{\kern0pt}type{\isacharcolon}{\kern0pt}\ {\isachardoublequoteopen}y\ {\isasymin}\isactrlsub c\ Y{\isachardoublequoteclose}\isanewline
\ \ \ \ \isacommand{using}\isamarkupfalse%
\ y{\isacharunderscore}{\kern0pt}in{\isacharunderscore}{\kern0pt}X\ \isacommand{unfolding}\isamarkupfalse%
\ relative{\isacharunderscore}{\kern0pt}member{\isacharunderscore}{\kern0pt}def{\isadigit{2}}\ \isacommand{by}\isamarkupfalse%
\ auto\isanewline
\ \ \isacommand{obtain}\isamarkupfalse%
\ x\ \isakeyword{where}\ x{\isacharunderscore}{\kern0pt}type{\isacharcolon}{\kern0pt}\ {\isachardoublequoteopen}x\ {\isasymin}\isactrlsub c\ X{\isachardoublequoteclose}\ \isakeyword{and}\ x{\isacharunderscore}{\kern0pt}def{\isacharcolon}{\kern0pt}\ {\isachardoublequoteopen}y\ {\isacharequal}{\kern0pt}\ m\ {\isasymcirc}\isactrlsub c\ x{\isachardoublequoteclose}\isanewline
\ \ \ \ \isacommand{using}\isamarkupfalse%
\ y{\isacharunderscore}{\kern0pt}in{\isacharunderscore}{\kern0pt}X\ \isacommand{unfolding}\isamarkupfalse%
\ relative{\isacharunderscore}{\kern0pt}member{\isacharunderscore}{\kern0pt}def{\isadigit{2}}\ factors{\isacharunderscore}{\kern0pt}through{\isacharunderscore}{\kern0pt}def\ \isacommand{by}\isamarkupfalse%
\ {\isacharparenleft}{\kern0pt}auto\ simp\ add{\isacharcolon}{\kern0pt}\ cfunc{\isacharunderscore}{\kern0pt}type{\isacharunderscore}{\kern0pt}def{\isacharparenright}{\kern0pt}\isanewline
\ \ \isacommand{then}\isamarkupfalse%
\ \isacommand{have}\isamarkupfalse%
\ {\isachardoublequoteopen}y\ {\isacharequal}{\kern0pt}\ {\isacharparenleft}{\kern0pt}into{\isacharunderscore}{\kern0pt}super\ m\ {\isasymcirc}\isactrlsub c\ left{\isacharunderscore}{\kern0pt}coproj\ X\ {\isacharparenleft}{\kern0pt}Y\ {\isasymsetminus}\ {\isacharparenleft}{\kern0pt}X{\isacharcomma}{\kern0pt}m{\isacharparenright}{\kern0pt}{\isacharparenright}{\kern0pt}{\isacharparenright}{\kern0pt}\ {\isasymcirc}\isactrlsub c\ x{\isachardoublequoteclose}\isanewline
\ \ \ \ \isacommand{unfolding}\isamarkupfalse%
\ into{\isacharunderscore}{\kern0pt}super{\isacharunderscore}{\kern0pt}def\ \isacommand{using}\isamarkupfalse%
\ complement{\isacharunderscore}{\kern0pt}morphism{\isacharunderscore}{\kern0pt}type\ left{\isacharunderscore}{\kern0pt}coproj{\isacharunderscore}{\kern0pt}cfunc{\isacharunderscore}{\kern0pt}coprod\ m{\isacharunderscore}{\kern0pt}type\ \isacommand{by}\isamarkupfalse%
\ auto\isanewline
\ \ \isacommand{then}\isamarkupfalse%
\ \isacommand{have}\isamarkupfalse%
\ {\isachardoublequoteopen}y\ {\isacharequal}{\kern0pt}\ into{\isacharunderscore}{\kern0pt}super\ m\ {\isasymcirc}\isactrlsub c\ left{\isacharunderscore}{\kern0pt}coproj\ X\ {\isacharparenleft}{\kern0pt}Y\ {\isasymsetminus}\ {\isacharparenleft}{\kern0pt}X{\isacharcomma}{\kern0pt}m{\isacharparenright}{\kern0pt}{\isacharparenright}{\kern0pt}\ {\isasymcirc}\isactrlsub c\ x{\isachardoublequoteclose}\isanewline
\ \ \ \ \isacommand{using}\isamarkupfalse%
\ x{\isacharunderscore}{\kern0pt}type\ m{\isacharunderscore}{\kern0pt}type\ \isacommand{by}\isamarkupfalse%
\ {\isacharparenleft}{\kern0pt}typecheck{\isacharunderscore}{\kern0pt}cfuncs{\isacharcomma}{\kern0pt}\ simp\ add{\isacharcolon}{\kern0pt}\ \ comp{\isacharunderscore}{\kern0pt}associative{\isadigit{2}}{\isacharparenright}{\kern0pt}\isanewline
\ \ \isacommand{then}\isamarkupfalse%
\ \isacommand{have}\isamarkupfalse%
\ {\isachardoublequoteopen}try{\isacharunderscore}{\kern0pt}cast\ m\ {\isasymcirc}\isactrlsub c\ y\ {\isacharequal}{\kern0pt}\ {\isacharparenleft}{\kern0pt}try{\isacharunderscore}{\kern0pt}cast\ m\ {\isasymcirc}\isactrlsub c\ into{\isacharunderscore}{\kern0pt}super\ m{\isacharparenright}{\kern0pt}\ {\isasymcirc}\isactrlsub c\ left{\isacharunderscore}{\kern0pt}coproj\ X\ {\isacharparenleft}{\kern0pt}Y\ {\isasymsetminus}\ {\isacharparenleft}{\kern0pt}X{\isacharcomma}{\kern0pt}m{\isacharparenright}{\kern0pt}{\isacharparenright}{\kern0pt}\ {\isasymcirc}\isactrlsub c\ x{\isachardoublequoteclose}\isanewline
\ \ \ \ \isacommand{using}\isamarkupfalse%
\ x{\isacharunderscore}{\kern0pt}type\ m{\isacharunderscore}{\kern0pt}type\ \isacommand{by}\isamarkupfalse%
\ {\isacharparenleft}{\kern0pt}typecheck{\isacharunderscore}{\kern0pt}cfuncs{\isacharcomma}{\kern0pt}\ smt\ comp{\isacharunderscore}{\kern0pt}associative{\isadigit{2}}{\isacharparenright}{\kern0pt}\isanewline
\ \ \isacommand{then}\isamarkupfalse%
\ \isacommand{have}\isamarkupfalse%
\ {\isachardoublequoteopen}try{\isacharunderscore}{\kern0pt}cast\ m\ {\isasymcirc}\isactrlsub c\ y\ {\isacharequal}{\kern0pt}\ left{\isacharunderscore}{\kern0pt}coproj\ X\ {\isacharparenleft}{\kern0pt}Y\ {\isasymsetminus}\ {\isacharparenleft}{\kern0pt}X{\isacharcomma}{\kern0pt}m{\isacharparenright}{\kern0pt}{\isacharparenright}{\kern0pt}\ {\isasymcirc}\isactrlsub c\ x{\isachardoublequoteclose}\isanewline
\ \ \ \ \isacommand{using}\isamarkupfalse%
\ m{\isacharunderscore}{\kern0pt}type\ x{\isacharunderscore}{\kern0pt}type\ \isacommand{by}\isamarkupfalse%
\ {\isacharparenleft}{\kern0pt}typecheck{\isacharunderscore}{\kern0pt}cfuncs{\isacharcomma}{\kern0pt}\ simp\ add{\isacharcolon}{\kern0pt}\ id{\isacharunderscore}{\kern0pt}left{\isacharunderscore}{\kern0pt}unit{\isadigit{2}}\ try{\isacharunderscore}{\kern0pt}cast{\isacharunderscore}{\kern0pt}into{\isacharunderscore}{\kern0pt}super{\isacharparenright}{\kern0pt}\isanewline
\ \ \isacommand{then}\isamarkupfalse%
\ \isacommand{show}\isamarkupfalse%
\ {\isacharquery}{\kern0pt}thesis\isanewline
\ \ \ \ \isacommand{using}\isamarkupfalse%
\ x{\isacharunderscore}{\kern0pt}type\ \isacommand{by}\isamarkupfalse%
\ blast\isanewline
\isacommand{qed}\isamarkupfalse%
%
\endisatagproof
{\isafoldproof}%
%
\isadelimproof
\isanewline
%
\endisadelimproof
\isanewline
\isacommand{lemma}\isamarkupfalse%
\ try{\isacharunderscore}{\kern0pt}cast{\isacharunderscore}{\kern0pt}not{\isacharunderscore}{\kern0pt}in{\isacharunderscore}{\kern0pt}X{\isacharcolon}{\kern0pt}\isanewline
\ \ \isakeyword{assumes}\ m{\isacharunderscore}{\kern0pt}type{\isacharcolon}{\kern0pt}\ {\isachardoublequoteopen}monomorphism\ m{\isachardoublequoteclose}\ {\isachardoublequoteopen}m\ {\isacharcolon}{\kern0pt}\ X\ {\isasymrightarrow}\ Y{\isachardoublequoteclose}\isanewline
\ \ \isakeyword{assumes}\ y{\isacharunderscore}{\kern0pt}in{\isacharunderscore}{\kern0pt}X{\isacharcolon}{\kern0pt}\ {\isachardoublequoteopen}{\isasymnot}\ y\ {\isasymin}\isactrlbsub Y\isactrlesub \ {\isacharparenleft}{\kern0pt}X{\isacharcomma}{\kern0pt}\ m{\isacharparenright}{\kern0pt}{\isachardoublequoteclose}\ \isakeyword{and}\ y{\isacharunderscore}{\kern0pt}type{\isacharcolon}{\kern0pt}\ {\isachardoublequoteopen}y\ {\isasymin}\isactrlsub c\ Y{\isachardoublequoteclose}\ \ \isanewline
\ \ \isakeyword{shows}\ {\isachardoublequoteopen}{\isasymexists}\ x{\isachardot}{\kern0pt}\ x\ {\isasymin}\isactrlsub c\ Y\ {\isasymsetminus}\ {\isacharparenleft}{\kern0pt}X{\isacharcomma}{\kern0pt}m{\isacharparenright}{\kern0pt}\ {\isasymand}\ try{\isacharunderscore}{\kern0pt}cast\ m\ {\isasymcirc}\isactrlsub c\ y\ {\isacharequal}{\kern0pt}\ right{\isacharunderscore}{\kern0pt}coproj\ X\ {\isacharparenleft}{\kern0pt}Y\ {\isasymsetminus}\ {\isacharparenleft}{\kern0pt}X{\isacharcomma}{\kern0pt}m{\isacharparenright}{\kern0pt}{\isacharparenright}{\kern0pt}\ {\isasymcirc}\isactrlsub c\ x{\isachardoublequoteclose}\isanewline
%
\isadelimproof
%
\endisadelimproof
%
\isatagproof
\isacommand{proof}\isamarkupfalse%
\ {\isacharminus}{\kern0pt}\isanewline
\ \ \isacommand{have}\isamarkupfalse%
\ y{\isacharunderscore}{\kern0pt}in{\isacharunderscore}{\kern0pt}complement{\isacharcolon}{\kern0pt}\ {\isachardoublequoteopen}y\ {\isasymin}\isactrlbsub Y\isactrlesub \ {\isacharparenleft}{\kern0pt}Y\ {\isasymsetminus}\ {\isacharparenleft}{\kern0pt}X{\isacharcomma}{\kern0pt}m{\isacharparenright}{\kern0pt}{\isacharcomma}{\kern0pt}\ m\isactrlsup c{\isacharparenright}{\kern0pt}{\isachardoublequoteclose}\isanewline
\ \ \ \ \isacommand{by}\isamarkupfalse%
\ {\isacharparenleft}{\kern0pt}simp\ add{\isacharcolon}{\kern0pt}\ assms\ not{\isacharunderscore}{\kern0pt}in{\isacharunderscore}{\kern0pt}subset{\isacharunderscore}{\kern0pt}in{\isacharunderscore}{\kern0pt}complement{\isacharparenright}{\kern0pt}\isanewline
\ \ \isacommand{then}\isamarkupfalse%
\ \isacommand{obtain}\isamarkupfalse%
\ x\ \isakeyword{where}\ x{\isacharunderscore}{\kern0pt}type{\isacharcolon}{\kern0pt}\ {\isachardoublequoteopen}x\ {\isasymin}\isactrlsub c\ Y\ {\isasymsetminus}\ {\isacharparenleft}{\kern0pt}X{\isacharcomma}{\kern0pt}m{\isacharparenright}{\kern0pt}{\isachardoublequoteclose}\ \isakeyword{and}\ x{\isacharunderscore}{\kern0pt}def{\isacharcolon}{\kern0pt}\ {\isachardoublequoteopen}y\ {\isacharequal}{\kern0pt}\ m\isactrlsup c\ {\isasymcirc}\isactrlsub c\ x{\isachardoublequoteclose}\isanewline
\ \ \ \ \isacommand{unfolding}\isamarkupfalse%
\ relative{\isacharunderscore}{\kern0pt}member{\isacharunderscore}{\kern0pt}def{\isadigit{2}}\ factors{\isacharunderscore}{\kern0pt}through{\isacharunderscore}{\kern0pt}def\ \isacommand{by}\isamarkupfalse%
\ {\isacharparenleft}{\kern0pt}auto\ simp\ add{\isacharcolon}{\kern0pt}\ cfunc{\isacharunderscore}{\kern0pt}type{\isacharunderscore}{\kern0pt}def{\isacharparenright}{\kern0pt}\isanewline
\ \ \isacommand{then}\isamarkupfalse%
\ \isacommand{have}\isamarkupfalse%
\ {\isachardoublequoteopen}y\ {\isacharequal}{\kern0pt}\ {\isacharparenleft}{\kern0pt}into{\isacharunderscore}{\kern0pt}super\ m\ {\isasymcirc}\isactrlsub c\ right{\isacharunderscore}{\kern0pt}coproj\ X\ {\isacharparenleft}{\kern0pt}Y\ {\isasymsetminus}\ {\isacharparenleft}{\kern0pt}X{\isacharcomma}{\kern0pt}m{\isacharparenright}{\kern0pt}{\isacharparenright}{\kern0pt}{\isacharparenright}{\kern0pt}\ {\isasymcirc}\isactrlsub c\ x{\isachardoublequoteclose}\isanewline
\ \ \ \ \isacommand{unfolding}\isamarkupfalse%
\ into{\isacharunderscore}{\kern0pt}super{\isacharunderscore}{\kern0pt}def\ \isacommand{using}\isamarkupfalse%
\ complement{\isacharunderscore}{\kern0pt}morphism{\isacharunderscore}{\kern0pt}type\ m{\isacharunderscore}{\kern0pt}type\ right{\isacharunderscore}{\kern0pt}coproj{\isacharunderscore}{\kern0pt}cfunc{\isacharunderscore}{\kern0pt}coprod\ \isacommand{by}\isamarkupfalse%
\ auto\ \isanewline
\ \ \isacommand{then}\isamarkupfalse%
\ \isacommand{have}\isamarkupfalse%
\ {\isachardoublequoteopen}y\ {\isacharequal}{\kern0pt}\ into{\isacharunderscore}{\kern0pt}super\ m\ {\isasymcirc}\isactrlsub c\ right{\isacharunderscore}{\kern0pt}coproj\ X\ {\isacharparenleft}{\kern0pt}Y\ {\isasymsetminus}\ {\isacharparenleft}{\kern0pt}X{\isacharcomma}{\kern0pt}m{\isacharparenright}{\kern0pt}{\isacharparenright}{\kern0pt}\ {\isasymcirc}\isactrlsub c\ x{\isachardoublequoteclose}\isanewline
\ \ \ \ \isacommand{using}\isamarkupfalse%
\ x{\isacharunderscore}{\kern0pt}type\ m{\isacharunderscore}{\kern0pt}type\ \isacommand{by}\isamarkupfalse%
\ {\isacharparenleft}{\kern0pt}typecheck{\isacharunderscore}{\kern0pt}cfuncs{\isacharcomma}{\kern0pt}\ simp\ add{\isacharcolon}{\kern0pt}\ \ comp{\isacharunderscore}{\kern0pt}associative{\isadigit{2}}{\isacharparenright}{\kern0pt}\isanewline
\ \ \isacommand{then}\isamarkupfalse%
\ \isacommand{have}\isamarkupfalse%
\ {\isachardoublequoteopen}try{\isacharunderscore}{\kern0pt}cast\ m\ {\isasymcirc}\isactrlsub c\ y\ {\isacharequal}{\kern0pt}\ {\isacharparenleft}{\kern0pt}try{\isacharunderscore}{\kern0pt}cast\ m\ {\isasymcirc}\isactrlsub c\ into{\isacharunderscore}{\kern0pt}super\ m{\isacharparenright}{\kern0pt}\ {\isasymcirc}\isactrlsub c\ right{\isacharunderscore}{\kern0pt}coproj\ X\ {\isacharparenleft}{\kern0pt}Y\ {\isasymsetminus}\ {\isacharparenleft}{\kern0pt}X{\isacharcomma}{\kern0pt}m{\isacharparenright}{\kern0pt}{\isacharparenright}{\kern0pt}\ {\isasymcirc}\isactrlsub c\ x{\isachardoublequoteclose}\isanewline
\ \ \ \ \isacommand{using}\isamarkupfalse%
\ x{\isacharunderscore}{\kern0pt}type\ m{\isacharunderscore}{\kern0pt}type\ \isacommand{by}\isamarkupfalse%
\ {\isacharparenleft}{\kern0pt}typecheck{\isacharunderscore}{\kern0pt}cfuncs{\isacharcomma}{\kern0pt}\ smt\ comp{\isacharunderscore}{\kern0pt}associative{\isadigit{2}}{\isacharparenright}{\kern0pt}\isanewline
\ \ \isacommand{then}\isamarkupfalse%
\ \isacommand{have}\isamarkupfalse%
\ {\isachardoublequoteopen}try{\isacharunderscore}{\kern0pt}cast\ m\ {\isasymcirc}\isactrlsub c\ y\ {\isacharequal}{\kern0pt}\ right{\isacharunderscore}{\kern0pt}coproj\ X\ {\isacharparenleft}{\kern0pt}Y\ {\isasymsetminus}\ {\isacharparenleft}{\kern0pt}X{\isacharcomma}{\kern0pt}m{\isacharparenright}{\kern0pt}{\isacharparenright}{\kern0pt}\ {\isasymcirc}\isactrlsub c\ x{\isachardoublequoteclose}\isanewline
\ \ \ \ \isacommand{using}\isamarkupfalse%
\ m{\isacharunderscore}{\kern0pt}type\ x{\isacharunderscore}{\kern0pt}type\ \isacommand{by}\isamarkupfalse%
\ {\isacharparenleft}{\kern0pt}typecheck{\isacharunderscore}{\kern0pt}cfuncs{\isacharcomma}{\kern0pt}\ simp\ add{\isacharcolon}{\kern0pt}\ id{\isacharunderscore}{\kern0pt}left{\isacharunderscore}{\kern0pt}unit{\isadigit{2}}\ try{\isacharunderscore}{\kern0pt}cast{\isacharunderscore}{\kern0pt}into{\isacharunderscore}{\kern0pt}super{\isacharparenright}{\kern0pt}\isanewline
\ \ \isacommand{then}\isamarkupfalse%
\ \isacommand{show}\isamarkupfalse%
\ {\isacharquery}{\kern0pt}thesis\isanewline
\ \ \ \ \isacommand{using}\isamarkupfalse%
\ x{\isacharunderscore}{\kern0pt}type\ \isacommand{by}\isamarkupfalse%
\ blast\isanewline
\isacommand{qed}\isamarkupfalse%
%
\endisatagproof
{\isafoldproof}%
%
\isadelimproof
\isanewline
%
\endisadelimproof
\isanewline
\isacommand{lemma}\isamarkupfalse%
\ try{\isacharunderscore}{\kern0pt}cast{\isacharunderscore}{\kern0pt}m{\isacharunderscore}{\kern0pt}m{\isacharcolon}{\kern0pt}\isanewline
\ \ \isakeyword{assumes}\ m{\isacharunderscore}{\kern0pt}type{\isacharcolon}{\kern0pt}\ {\isachardoublequoteopen}monomorphism\ m{\isachardoublequoteclose}\ {\isachardoublequoteopen}m\ {\isacharcolon}{\kern0pt}\ X\ {\isasymrightarrow}\ Y{\isachardoublequoteclose}\isanewline
\ \ \isakeyword{shows}\ {\isachardoublequoteopen}{\isacharparenleft}{\kern0pt}try{\isacharunderscore}{\kern0pt}cast\ m{\isacharparenright}{\kern0pt}\ {\isasymcirc}\isactrlsub c\ m\ {\isacharequal}{\kern0pt}\ left{\isacharunderscore}{\kern0pt}coproj\ X\ {\isacharparenleft}{\kern0pt}Y\ {\isasymsetminus}\ {\isacharparenleft}{\kern0pt}X{\isacharcomma}{\kern0pt}m{\isacharparenright}{\kern0pt}{\isacharparenright}{\kern0pt}{\isachardoublequoteclose}\isanewline
%
\isadelimproof
\ \ %
\endisadelimproof
%
\isatagproof
\isacommand{by}\isamarkupfalse%
\ {\isacharparenleft}{\kern0pt}smt\ comp{\isacharunderscore}{\kern0pt}associative{\isadigit{2}}\ complement{\isacharunderscore}{\kern0pt}morphism{\isacharunderscore}{\kern0pt}type\ id{\isacharunderscore}{\kern0pt}left{\isacharunderscore}{\kern0pt}unit{\isadigit{2}}\ into{\isacharunderscore}{\kern0pt}super{\isacharunderscore}{\kern0pt}def\ into{\isacharunderscore}{\kern0pt}super{\isacharunderscore}{\kern0pt}type\ left{\isacharunderscore}{\kern0pt}coproj{\isacharunderscore}{\kern0pt}cfunc{\isacharunderscore}{\kern0pt}coprod\ left{\isacharunderscore}{\kern0pt}proj{\isacharunderscore}{\kern0pt}type\ m{\isacharunderscore}{\kern0pt}type\ try{\isacharunderscore}{\kern0pt}cast{\isacharunderscore}{\kern0pt}into{\isacharunderscore}{\kern0pt}super\ try{\isacharunderscore}{\kern0pt}cast{\isacharunderscore}{\kern0pt}type{\isacharparenright}{\kern0pt}%
\endisatagproof
{\isafoldproof}%
%
\isadelimproof
\isanewline
%
\endisadelimproof
\isanewline
\isacommand{lemma}\isamarkupfalse%
\ try{\isacharunderscore}{\kern0pt}cast{\isacharunderscore}{\kern0pt}m{\isacharunderscore}{\kern0pt}m{\isacharprime}{\kern0pt}{\isacharcolon}{\kern0pt}\isanewline
\ \ \isakeyword{assumes}\ m{\isacharunderscore}{\kern0pt}type{\isacharcolon}{\kern0pt}\ {\isachardoublequoteopen}monomorphism\ m{\isachardoublequoteclose}\ {\isachardoublequoteopen}m\ {\isacharcolon}{\kern0pt}\ X\ {\isasymrightarrow}\ Y{\isachardoublequoteclose}\isanewline
\ \ \isakeyword{shows}\ {\isachardoublequoteopen}{\isacharparenleft}{\kern0pt}try{\isacharunderscore}{\kern0pt}cast\ m{\isacharparenright}{\kern0pt}\ {\isasymcirc}\isactrlsub c\ m\isactrlsup c\ {\isacharequal}{\kern0pt}\ right{\isacharunderscore}{\kern0pt}coproj\ X\ {\isacharparenleft}{\kern0pt}Y\ {\isasymsetminus}\ {\isacharparenleft}{\kern0pt}X{\isacharcomma}{\kern0pt}m{\isacharparenright}{\kern0pt}{\isacharparenright}{\kern0pt}{\isachardoublequoteclose}\isanewline
%
\isadelimproof
\ \ %
\endisadelimproof
%
\isatagproof
\isacommand{by}\isamarkupfalse%
\ {\isacharparenleft}{\kern0pt}smt\ comp{\isacharunderscore}{\kern0pt}associative{\isadigit{2}}\ complement{\isacharunderscore}{\kern0pt}morphism{\isacharunderscore}{\kern0pt}type\ id{\isacharunderscore}{\kern0pt}left{\isacharunderscore}{\kern0pt}unit{\isadigit{2}}\ into{\isacharunderscore}{\kern0pt}super{\isacharunderscore}{\kern0pt}def\ into{\isacharunderscore}{\kern0pt}super{\isacharunderscore}{\kern0pt}type\ m{\isacharunderscore}{\kern0pt}type{\isacharparenleft}{\kern0pt}{\isadigit{1}}{\isacharparenright}{\kern0pt}\ m{\isacharunderscore}{\kern0pt}type{\isacharparenleft}{\kern0pt}{\isadigit{2}}{\isacharparenright}{\kern0pt}\ right{\isacharunderscore}{\kern0pt}coproj{\isacharunderscore}{\kern0pt}cfunc{\isacharunderscore}{\kern0pt}coprod\ right{\isacharunderscore}{\kern0pt}proj{\isacharunderscore}{\kern0pt}type\ try{\isacharunderscore}{\kern0pt}cast{\isacharunderscore}{\kern0pt}into{\isacharunderscore}{\kern0pt}super\ try{\isacharunderscore}{\kern0pt}cast{\isacharunderscore}{\kern0pt}type{\isacharparenright}{\kern0pt}%
\endisatagproof
{\isafoldproof}%
%
\isadelimproof
\isanewline
%
\endisadelimproof
\isanewline
\isacommand{lemma}\isamarkupfalse%
\ try{\isacharunderscore}{\kern0pt}cast{\isacharunderscore}{\kern0pt}mono{\isacharcolon}{\kern0pt}\isanewline
\ \ \isakeyword{assumes}\ m{\isacharunderscore}{\kern0pt}type{\isacharcolon}{\kern0pt}\ {\isachardoublequoteopen}monomorphism\ m{\isachardoublequoteclose}\ {\isachardoublequoteopen}m\ {\isacharcolon}{\kern0pt}\ X\ {\isasymrightarrow}\ Y{\isachardoublequoteclose}\isanewline
\ \ \isakeyword{shows}\ {\isachardoublequoteopen}monomorphism{\isacharparenleft}{\kern0pt}try{\isacharunderscore}{\kern0pt}cast\ m{\isacharparenright}{\kern0pt}{\isachardoublequoteclose}\isanewline
%
\isadelimproof
\ \ %
\endisadelimproof
%
\isatagproof
\isacommand{by}\isamarkupfalse%
\ {\isacharparenleft}{\kern0pt}smt\ cfunc{\isacharunderscore}{\kern0pt}type{\isacharunderscore}{\kern0pt}def\ comp{\isacharunderscore}{\kern0pt}monic{\isacharunderscore}{\kern0pt}imp{\isacharunderscore}{\kern0pt}monic{\isacharprime}{\kern0pt}\ id{\isacharunderscore}{\kern0pt}isomorphism\ into{\isacharunderscore}{\kern0pt}super{\isacharunderscore}{\kern0pt}type\ iso{\isacharunderscore}{\kern0pt}imp{\isacharunderscore}{\kern0pt}epi{\isacharunderscore}{\kern0pt}and{\isacharunderscore}{\kern0pt}monic\ try{\isacharunderscore}{\kern0pt}cast{\isacharunderscore}{\kern0pt}def{\isadigit{2}}\ assms{\isacharparenright}{\kern0pt}%
\endisatagproof
{\isafoldproof}%
%
\isadelimproof
%
\endisadelimproof
%
\isadelimdocument
%
\endisadelimdocument
%
\isatagdocument
%
\isamarkupsubsection{Coproduct Set Properities%
}
\isamarkuptrue%
%
\endisatagdocument
{\isafolddocument}%
%
\isadelimdocument
%
\endisadelimdocument
\isacommand{lemma}\isamarkupfalse%
\ coproduct{\isacharunderscore}{\kern0pt}commutes{\isacharcolon}{\kern0pt}\isanewline
\ \ {\isachardoublequoteopen}A\ {\isasymCoprod}\ B\ {\isasymcong}\ B\ {\isasymCoprod}\ A{\isachardoublequoteclose}\isanewline
%
\isadelimproof
%
\endisadelimproof
%
\isatagproof
\isacommand{proof}\isamarkupfalse%
\ {\isacharminus}{\kern0pt}\isanewline
\ \ \isacommand{have}\isamarkupfalse%
\ id{\isacharunderscore}{\kern0pt}AB{\isacharcolon}{\kern0pt}\ {\isachardoublequoteopen}{\isacharparenleft}{\kern0pt}{\isacharparenleft}{\kern0pt}right{\isacharunderscore}{\kern0pt}coproj\ A\ B{\isacharparenright}{\kern0pt}\ \ {\isasymamalg}\ {\isacharparenleft}{\kern0pt}left{\isacharunderscore}{\kern0pt}coproj\ A\ B{\isacharparenright}{\kern0pt}{\isacharparenright}{\kern0pt}\ {\isasymcirc}\isactrlsub c\ {\isacharparenleft}{\kern0pt}{\isacharparenleft}{\kern0pt}right{\isacharunderscore}{\kern0pt}coproj\ B\ A{\isacharparenright}{\kern0pt}\ {\isasymamalg}\ {\isacharparenleft}{\kern0pt}left{\isacharunderscore}{\kern0pt}coproj\ B\ A{\isacharparenright}{\kern0pt}{\isacharparenright}{\kern0pt}\ {\isacharequal}{\kern0pt}\ id{\isacharparenleft}{\kern0pt}A\ {\isasymCoprod}\ B{\isacharparenright}{\kern0pt}{\isachardoublequoteclose}\isanewline
\ \ \ \ \isacommand{by}\isamarkupfalse%
\ {\isacharparenleft}{\kern0pt}typecheck{\isacharunderscore}{\kern0pt}cfuncs{\isacharcomma}{\kern0pt}\ smt\ {\isacharparenleft}{\kern0pt}z{\isadigit{3}}{\isacharparenright}{\kern0pt}\ cfunc{\isacharunderscore}{\kern0pt}coprod{\isacharunderscore}{\kern0pt}comp\ id{\isacharunderscore}{\kern0pt}coprod\ left{\isacharunderscore}{\kern0pt}coproj{\isacharunderscore}{\kern0pt}cfunc{\isacharunderscore}{\kern0pt}coprod\ right{\isacharunderscore}{\kern0pt}coproj{\isacharunderscore}{\kern0pt}cfunc{\isacharunderscore}{\kern0pt}coprod{\isacharparenright}{\kern0pt}\isanewline
\ \ \isacommand{have}\isamarkupfalse%
\ id{\isacharunderscore}{\kern0pt}BA{\isacharcolon}{\kern0pt}\ {\isachardoublequoteopen}\ {\isacharparenleft}{\kern0pt}{\isacharparenleft}{\kern0pt}right{\isacharunderscore}{\kern0pt}coproj\ B\ A{\isacharparenright}{\kern0pt}\ {\isasymamalg}\ {\isacharparenleft}{\kern0pt}left{\isacharunderscore}{\kern0pt}coproj\ B\ A{\isacharparenright}{\kern0pt}{\isacharparenright}{\kern0pt}\ {\isasymcirc}\isactrlsub c\ {\isacharparenleft}{\kern0pt}{\isacharparenleft}{\kern0pt}right{\isacharunderscore}{\kern0pt}coproj\ A\ B{\isacharparenright}{\kern0pt}\ \ {\isasymamalg}\ {\isacharparenleft}{\kern0pt}left{\isacharunderscore}{\kern0pt}coproj\ A\ B{\isacharparenright}{\kern0pt}{\isacharparenright}{\kern0pt}\ {\isacharequal}{\kern0pt}\ id{\isacharparenleft}{\kern0pt}B\ {\isasymCoprod}\ A{\isacharparenright}{\kern0pt}{\isachardoublequoteclose}\isanewline
\ \ \ \ \isacommand{by}\isamarkupfalse%
\ {\isacharparenleft}{\kern0pt}typecheck{\isacharunderscore}{\kern0pt}cfuncs{\isacharcomma}{\kern0pt}\ smt\ {\isacharparenleft}{\kern0pt}z{\isadigit{3}}{\isacharparenright}{\kern0pt}\ cfunc{\isacharunderscore}{\kern0pt}coprod{\isacharunderscore}{\kern0pt}comp\ id{\isacharunderscore}{\kern0pt}coprod\ right{\isacharunderscore}{\kern0pt}coproj{\isacharunderscore}{\kern0pt}cfunc{\isacharunderscore}{\kern0pt}coprod\ left{\isacharunderscore}{\kern0pt}coproj{\isacharunderscore}{\kern0pt}cfunc{\isacharunderscore}{\kern0pt}coprod{\isacharparenright}{\kern0pt}\isanewline
\ \ \isacommand{show}\isamarkupfalse%
\ {\isachardoublequoteopen}A\ {\isasymCoprod}\ B\ {\isasymcong}\ B\ {\isasymCoprod}\ A{\isachardoublequoteclose}\isanewline
\ \ \ \ \isacommand{by}\isamarkupfalse%
\ {\isacharparenleft}{\kern0pt}smt\ {\isacharparenleft}{\kern0pt}verit{\isacharcomma}{\kern0pt}\ ccfv{\isacharunderscore}{\kern0pt}threshold{\isacharparenright}{\kern0pt}\ cfunc{\isacharunderscore}{\kern0pt}coprod{\isacharunderscore}{\kern0pt}type\ cfunc{\isacharunderscore}{\kern0pt}type{\isacharunderscore}{\kern0pt}def\ id{\isacharunderscore}{\kern0pt}AB\ id{\isacharunderscore}{\kern0pt}BA\ is{\isacharunderscore}{\kern0pt}isomorphic{\isacharunderscore}{\kern0pt}def\ isomorphism{\isacharunderscore}{\kern0pt}def\ left{\isacharunderscore}{\kern0pt}proj{\isacharunderscore}{\kern0pt}type\ right{\isacharunderscore}{\kern0pt}proj{\isacharunderscore}{\kern0pt}type{\isacharparenright}{\kern0pt}\isanewline
\isacommand{qed}\isamarkupfalse%
%
\endisatagproof
{\isafoldproof}%
%
\isadelimproof
\isanewline
%
\endisadelimproof
\isanewline
\isacommand{lemma}\isamarkupfalse%
\ coproduct{\isacharunderscore}{\kern0pt}associates{\isacharcolon}{\kern0pt}\isanewline
\ \ {\isachardoublequoteopen}A\ {\isasymCoprod}\ {\isacharparenleft}{\kern0pt}B\ {\isasymCoprod}\ C{\isacharparenright}{\kern0pt}\ \ {\isasymcong}\ {\isacharparenleft}{\kern0pt}A\ {\isasymCoprod}\ B{\isacharparenright}{\kern0pt}\ {\isasymCoprod}\ C{\isachardoublequoteclose}\isanewline
%
\isadelimproof
%
\endisadelimproof
%
\isatagproof
\isacommand{proof}\isamarkupfalse%
\ {\isacharminus}{\kern0pt}\isanewline
\ \ \isacommand{obtain}\isamarkupfalse%
\ q\ \isakeyword{where}\ q{\isacharunderscore}{\kern0pt}def{\isacharcolon}{\kern0pt}\ {\isachardoublequoteopen}q\ {\isacharequal}{\kern0pt}\ {\isacharparenleft}{\kern0pt}left{\isacharunderscore}{\kern0pt}coproj\ {\isacharparenleft}{\kern0pt}A\ {\isasymCoprod}\ B{\isacharparenright}{\kern0pt}\ C\ {\isacharparenright}{\kern0pt}\ {\isasymcirc}\isactrlsub c\ {\isacharparenleft}{\kern0pt}right{\isacharunderscore}{\kern0pt}coproj\ A\ B{\isacharparenright}{\kern0pt}{\isachardoublequoteclose}\ \isakeyword{and}\ q{\isacharunderscore}{\kern0pt}type{\isacharbrackleft}{\kern0pt}type{\isacharunderscore}{\kern0pt}rule{\isacharbrackright}{\kern0pt}{\isacharcolon}{\kern0pt}\ {\isachardoublequoteopen}q{\isacharcolon}{\kern0pt}\ B\ {\isasymrightarrow}\ {\isacharparenleft}{\kern0pt}A\ {\isasymCoprod}\ B{\isacharparenright}{\kern0pt}\ {\isasymCoprod}\ C{\isachardoublequoteclose}\isanewline
\ \ \ \ \isacommand{by}\isamarkupfalse%
\ typecheck{\isacharunderscore}{\kern0pt}cfuncs\ \ \isanewline
\ \ \isacommand{obtain}\isamarkupfalse%
\ f\ \isakeyword{where}\ f{\isacharunderscore}{\kern0pt}def{\isacharcolon}{\kern0pt}\ {\isachardoublequoteopen}f\ {\isacharequal}{\kern0pt}\ q\ {\isasymamalg}\ {\isacharparenleft}{\kern0pt}right{\isacharunderscore}{\kern0pt}coproj\ {\isacharparenleft}{\kern0pt}A\ {\isasymCoprod}\ B{\isacharparenright}{\kern0pt}\ C{\isacharparenright}{\kern0pt}{\isachardoublequoteclose}\ \isakeyword{and}\ f{\isacharunderscore}{\kern0pt}type{\isacharbrackleft}{\kern0pt}type{\isacharunderscore}{\kern0pt}rule{\isacharbrackright}{\kern0pt}{\isacharcolon}{\kern0pt}\ {\isachardoublequoteopen}{\isacharparenleft}{\kern0pt}f{\isacharcolon}{\kern0pt}\ {\isacharparenleft}{\kern0pt}B\ {\isasymCoprod}\ C{\isacharparenright}{\kern0pt}\ {\isasymrightarrow}\ {\isacharparenleft}{\kern0pt}{\isacharparenleft}{\kern0pt}A\ {\isasymCoprod}\ B{\isacharparenright}{\kern0pt}\ {\isasymCoprod}\ C{\isacharparenright}{\kern0pt}{\isacharparenright}{\kern0pt}{\isachardoublequoteclose}\isanewline
\ \ \ \ \isacommand{by}\isamarkupfalse%
\ typecheck{\isacharunderscore}{\kern0pt}cfuncs\isanewline
\ \ \isacommand{have}\isamarkupfalse%
\ f{\isacharunderscore}{\kern0pt}prop{\isacharcolon}{\kern0pt}\ {\isachardoublequoteopen}{\isacharparenleft}{\kern0pt}f\ {\isasymcirc}\isactrlsub c\ left{\isacharunderscore}{\kern0pt}coproj\ B\ C\ {\isacharequal}{\kern0pt}\ q{\isacharparenright}{\kern0pt}\ {\isasymand}\ {\isacharparenleft}{\kern0pt}f\ {\isasymcirc}\isactrlsub c\ right{\isacharunderscore}{\kern0pt}coproj\ B\ C\ {\isacharequal}{\kern0pt}\ right{\isacharunderscore}{\kern0pt}coproj\ {\isacharparenleft}{\kern0pt}A\ {\isasymCoprod}\ B{\isacharparenright}{\kern0pt}\ C{\isacharparenright}{\kern0pt}{\isachardoublequoteclose}\isanewline
\ \ \ \ \isacommand{by}\isamarkupfalse%
\ {\isacharparenleft}{\kern0pt}typecheck{\isacharunderscore}{\kern0pt}cfuncs{\isacharcomma}{\kern0pt}\ simp\ add{\isacharcolon}{\kern0pt}\ f{\isacharunderscore}{\kern0pt}def\ left{\isacharunderscore}{\kern0pt}coproj{\isacharunderscore}{\kern0pt}cfunc{\isacharunderscore}{\kern0pt}coprod\ right{\isacharunderscore}{\kern0pt}coproj{\isacharunderscore}{\kern0pt}cfunc{\isacharunderscore}{\kern0pt}coprod{\isacharparenright}{\kern0pt}\isanewline
\ \ \isacommand{then}\isamarkupfalse%
\ \isacommand{have}\isamarkupfalse%
\ f{\isacharunderscore}{\kern0pt}unique{\isacharcolon}{\kern0pt}\ {\isachardoublequoteopen}{\isacharparenleft}{\kern0pt}{\isasymexists}{\isacharbang}{\kern0pt}f{\isachardot}{\kern0pt}\ {\isacharparenleft}{\kern0pt}f{\isacharcolon}{\kern0pt}\ {\isacharparenleft}{\kern0pt}B\ {\isasymCoprod}\ C{\isacharparenright}{\kern0pt}\ {\isasymrightarrow}\ {\isacharparenleft}{\kern0pt}{\isacharparenleft}{\kern0pt}A\ {\isasymCoprod}\ B{\isacharparenright}{\kern0pt}\ {\isasymCoprod}\ C{\isacharparenright}{\kern0pt}{\isacharparenright}{\kern0pt}\ {\isasymand}\ {\isacharparenleft}{\kern0pt}f\ {\isasymcirc}\isactrlsub c\ left{\isacharunderscore}{\kern0pt}coproj\ B\ C\ {\isacharequal}{\kern0pt}\ q{\isacharparenright}{\kern0pt}\ {\isasymand}\ {\isacharparenleft}{\kern0pt}f\ {\isasymcirc}\isactrlsub c\ right{\isacharunderscore}{\kern0pt}coproj\ B\ C\ {\isacharequal}{\kern0pt}\ right{\isacharunderscore}{\kern0pt}coproj\ {\isacharparenleft}{\kern0pt}A\ {\isasymCoprod}\ B{\isacharparenright}{\kern0pt}\ C{\isacharparenright}{\kern0pt}{\isacharparenright}{\kern0pt}{\isachardoublequoteclose}\isanewline
\ \ \ \ \isacommand{by}\isamarkupfalse%
\ {\isacharparenleft}{\kern0pt}typecheck{\isacharunderscore}{\kern0pt}cfuncs{\isacharcomma}{\kern0pt}\ metis\ cfunc{\isacharunderscore}{\kern0pt}coprod{\isacharunderscore}{\kern0pt}unique\ f{\isacharunderscore}{\kern0pt}prop\ f{\isacharunderscore}{\kern0pt}type{\isacharparenright}{\kern0pt}\isanewline
\isanewline
\ \ \isacommand{obtain}\isamarkupfalse%
\ m\ \isakeyword{where}\ m{\isacharunderscore}{\kern0pt}def{\isacharcolon}{\kern0pt}\ {\isachardoublequoteopen}m\ {\isacharequal}{\kern0pt}\ {\isacharparenleft}{\kern0pt}left{\isacharunderscore}{\kern0pt}coproj\ {\isacharparenleft}{\kern0pt}A\ {\isasymCoprod}\ B{\isacharparenright}{\kern0pt}\ C\ {\isacharparenright}{\kern0pt}\ {\isasymcirc}\isactrlsub c\ {\isacharparenleft}{\kern0pt}left{\isacharunderscore}{\kern0pt}coproj\ A\ B{\isacharparenright}{\kern0pt}{\isachardoublequoteclose}\ \isakeyword{and}\ m{\isacharunderscore}{\kern0pt}type{\isacharbrackleft}{\kern0pt}type{\isacharunderscore}{\kern0pt}rule{\isacharbrackright}{\kern0pt}{\isacharcolon}{\kern0pt}\ {\isachardoublequoteopen}m\ {\isacharcolon}{\kern0pt}\ A\ {\isasymrightarrow}\ {\isacharparenleft}{\kern0pt}A\ {\isasymCoprod}\ B{\isacharparenright}{\kern0pt}\ {\isasymCoprod}\ C{\isachardoublequoteclose}\isanewline
\ \ \ \ \isacommand{by}\isamarkupfalse%
\ typecheck{\isacharunderscore}{\kern0pt}cfuncs\isanewline
\ \ \isacommand{obtain}\isamarkupfalse%
\ g\ \isakeyword{where}\ g{\isacharunderscore}{\kern0pt}def{\isacharcolon}{\kern0pt}\ {\isachardoublequoteopen}g\ {\isacharequal}{\kern0pt}\ m\ {\isasymamalg}\ f{\isachardoublequoteclose}\ \isakeyword{and}\ g{\isacharunderscore}{\kern0pt}type{\isacharbrackleft}{\kern0pt}type{\isacharunderscore}{\kern0pt}rule{\isacharbrackright}{\kern0pt}{\isacharcolon}{\kern0pt}\ {\isachardoublequoteopen}g{\isacharcolon}{\kern0pt}\ A\ {\isasymCoprod}\ {\isacharparenleft}{\kern0pt}B\ {\isasymCoprod}\ C{\isacharparenright}{\kern0pt}\ \ {\isasymrightarrow}\ {\isacharparenleft}{\kern0pt}A\ {\isasymCoprod}\ B{\isacharparenright}{\kern0pt}\ {\isasymCoprod}\ C{\isachardoublequoteclose}\isanewline
\ \ \ \ \isacommand{by}\isamarkupfalse%
\ typecheck{\isacharunderscore}{\kern0pt}cfuncs\isanewline
\ \ \isacommand{have}\isamarkupfalse%
\ g{\isacharunderscore}{\kern0pt}prop{\isacharcolon}{\kern0pt}\ {\isachardoublequoteopen}{\isacharparenleft}{\kern0pt}g\ {\isasymcirc}\isactrlsub c\ {\isacharparenleft}{\kern0pt}left{\isacharunderscore}{\kern0pt}coproj\ A\ {\isacharparenleft}{\kern0pt}B\ {\isasymCoprod}\ C{\isacharparenright}{\kern0pt}{\isacharparenright}{\kern0pt}\ {\isacharequal}{\kern0pt}\ m{\isacharparenright}{\kern0pt}\ {\isasymand}\ {\isacharparenleft}{\kern0pt}g\ {\isasymcirc}\isactrlsub c\ {\isacharparenleft}{\kern0pt}right{\isacharunderscore}{\kern0pt}coproj\ A\ {\isacharparenleft}{\kern0pt}B\ {\isasymCoprod}\ C{\isacharparenright}{\kern0pt}{\isacharparenright}{\kern0pt}\ {\isacharequal}{\kern0pt}\ f{\isacharparenright}{\kern0pt}{\isachardoublequoteclose}\isanewline
\ \ \ \ \isacommand{by}\isamarkupfalse%
\ {\isacharparenleft}{\kern0pt}typecheck{\isacharunderscore}{\kern0pt}cfuncs{\isacharcomma}{\kern0pt}\ simp\ add{\isacharcolon}{\kern0pt}\ g{\isacharunderscore}{\kern0pt}def\ left{\isacharunderscore}{\kern0pt}coproj{\isacharunderscore}{\kern0pt}cfunc{\isacharunderscore}{\kern0pt}coprod\ right{\isacharunderscore}{\kern0pt}coproj{\isacharunderscore}{\kern0pt}cfunc{\isacharunderscore}{\kern0pt}coprod{\isacharparenright}{\kern0pt}\ \isanewline
\ \ \isacommand{have}\isamarkupfalse%
\ g{\isacharunderscore}{\kern0pt}unique{\isacharcolon}{\kern0pt}\ {\isachardoublequoteopen}{\isasymexists}{\isacharbang}{\kern0pt}\ g{\isachardot}{\kern0pt}\ {\isacharparenleft}{\kern0pt}{\isacharparenleft}{\kern0pt}g{\isacharcolon}{\kern0pt}\ A\ {\isasymCoprod}\ {\isacharparenleft}{\kern0pt}B\ {\isasymCoprod}\ C{\isacharparenright}{\kern0pt}\ \ {\isasymrightarrow}\ {\isacharparenleft}{\kern0pt}A\ {\isasymCoprod}\ B{\isacharparenright}{\kern0pt}\ {\isasymCoprod}\ C{\isacharparenright}{\kern0pt}\ {\isasymand}\ {\isacharparenleft}{\kern0pt}g\ {\isasymcirc}\isactrlsub c\ {\isacharparenleft}{\kern0pt}left{\isacharunderscore}{\kern0pt}coproj\ A\ {\isacharparenleft}{\kern0pt}B\ {\isasymCoprod}\ C{\isacharparenright}{\kern0pt}{\isacharparenright}{\kern0pt}\ {\isacharequal}{\kern0pt}\ m{\isacharparenright}{\kern0pt}\ {\isasymand}\ {\isacharparenleft}{\kern0pt}g\ {\isasymcirc}\isactrlsub c\ {\isacharparenleft}{\kern0pt}right{\isacharunderscore}{\kern0pt}coproj\ A\ {\isacharparenleft}{\kern0pt}B\ {\isasymCoprod}\ C{\isacharparenright}{\kern0pt}{\isacharparenright}{\kern0pt}\ {\isacharequal}{\kern0pt}\ f{\isacharparenright}{\kern0pt}{\isacharparenright}{\kern0pt}{\isachardoublequoteclose}\isanewline
\ \ \ \ \isacommand{by}\isamarkupfalse%
\ {\isacharparenleft}{\kern0pt}typecheck{\isacharunderscore}{\kern0pt}cfuncs{\isacharcomma}{\kern0pt}\ metis\ cfunc{\isacharunderscore}{\kern0pt}coprod{\isacharunderscore}{\kern0pt}unique\ g{\isacharunderscore}{\kern0pt}prop\ g{\isacharunderscore}{\kern0pt}type{\isacharparenright}{\kern0pt}\isanewline
\isanewline
\ \ \isacommand{obtain}\isamarkupfalse%
\ p\ \isakeyword{where}\ p{\isacharunderscore}{\kern0pt}def{\isacharcolon}{\kern0pt}\ {\isachardoublequoteopen}p\ {\isacharequal}{\kern0pt}\ {\isacharparenleft}{\kern0pt}right{\isacharunderscore}{\kern0pt}coproj\ A\ {\isacharparenleft}{\kern0pt}B\ {\isasymCoprod}\ C{\isacharparenright}{\kern0pt}{\isacharparenright}{\kern0pt}\ {\isasymcirc}\isactrlsub c\ \ {\isacharparenleft}{\kern0pt}left{\isacharunderscore}{\kern0pt}coproj\ B\ C{\isacharparenright}{\kern0pt}{\isachardoublequoteclose}\ \isakeyword{and}\ p{\isacharunderscore}{\kern0pt}type{\isacharbrackleft}{\kern0pt}type{\isacharunderscore}{\kern0pt}rule{\isacharbrackright}{\kern0pt}{\isacharcolon}{\kern0pt}\ {\isachardoublequoteopen}p{\isacharcolon}{\kern0pt}\ B\ {\isasymrightarrow}\ A\ {\isasymCoprod}\ {\isacharparenleft}{\kern0pt}B\ {\isasymCoprod}\ C{\isacharparenright}{\kern0pt}{\isachardoublequoteclose}\isanewline
\ \ \ \ \isacommand{by}\isamarkupfalse%
\ typecheck{\isacharunderscore}{\kern0pt}cfuncs\isanewline
\ \ \isacommand{obtain}\isamarkupfalse%
\ h\ \isakeyword{where}\ h{\isacharunderscore}{\kern0pt}def{\isacharcolon}{\kern0pt}\ {\isachardoublequoteopen}h\ {\isacharequal}{\kern0pt}\ {\isacharparenleft}{\kern0pt}left{\isacharunderscore}{\kern0pt}coproj\ A\ {\isacharparenleft}{\kern0pt}B\ {\isasymCoprod}\ C{\isacharparenright}{\kern0pt}{\isacharparenright}{\kern0pt}\ {\isasymamalg}\ p{\isachardoublequoteclose}\ \isakeyword{and}\ h{\isacharunderscore}{\kern0pt}type{\isacharbrackleft}{\kern0pt}type{\isacharunderscore}{\kern0pt}rule{\isacharbrackright}{\kern0pt}{\isacharcolon}{\kern0pt}\ {\isachardoublequoteopen}h{\isacharcolon}{\kern0pt}\ {\isacharparenleft}{\kern0pt}A\ {\isasymCoprod}\ B{\isacharparenright}{\kern0pt}\ {\isasymrightarrow}\ A\ {\isasymCoprod}\ {\isacharparenleft}{\kern0pt}B\ {\isasymCoprod}\ C{\isacharparenright}{\kern0pt}{\isachardoublequoteclose}\isanewline
\ \ \ \ \isacommand{by}\isamarkupfalse%
\ typecheck{\isacharunderscore}{\kern0pt}cfuncs\isanewline
\ \ \isacommand{have}\isamarkupfalse%
\ h{\isacharunderscore}{\kern0pt}prop{\isadigit{1}}{\isacharcolon}{\kern0pt}\ {\isachardoublequoteopen}h\ {\isasymcirc}\isactrlsub c\ {\isacharparenleft}{\kern0pt}left{\isacharunderscore}{\kern0pt}coproj\ A\ B{\isacharparenright}{\kern0pt}\ \ {\isacharequal}{\kern0pt}\ {\isacharparenleft}{\kern0pt}left{\isacharunderscore}{\kern0pt}coproj\ A\ {\isacharparenleft}{\kern0pt}B\ {\isasymCoprod}\ C{\isacharparenright}{\kern0pt}{\isacharparenright}{\kern0pt}{\isachardoublequoteclose}\isanewline
\ \ \ \ \isacommand{by}\isamarkupfalse%
\ {\isacharparenleft}{\kern0pt}typecheck{\isacharunderscore}{\kern0pt}cfuncs{\isacharcomma}{\kern0pt}\ simp\ add{\isacharcolon}{\kern0pt}\ h{\isacharunderscore}{\kern0pt}def\ left{\isacharunderscore}{\kern0pt}coproj{\isacharunderscore}{\kern0pt}cfunc{\isacharunderscore}{\kern0pt}coprod\ p{\isacharunderscore}{\kern0pt}type{\isacharparenright}{\kern0pt}\isanewline
\ \ \isacommand{have}\isamarkupfalse%
\ h{\isacharunderscore}{\kern0pt}prop{\isadigit{2}}{\isacharcolon}{\kern0pt}\ {\isachardoublequoteopen}h\ {\isasymcirc}\isactrlsub c\ {\isacharparenleft}{\kern0pt}right{\isacharunderscore}{\kern0pt}coproj\ A\ B{\isacharparenright}{\kern0pt}\ {\isacharequal}{\kern0pt}\ p{\isachardoublequoteclose}\isanewline
\ \ \ \ \isacommand{using}\isamarkupfalse%
\ h{\isacharunderscore}{\kern0pt}def\ left{\isacharunderscore}{\kern0pt}proj{\isacharunderscore}{\kern0pt}type\ right{\isacharunderscore}{\kern0pt}coproj{\isacharunderscore}{\kern0pt}cfunc{\isacharunderscore}{\kern0pt}coprod\ \isacommand{by}\isamarkupfalse%
\ {\isacharparenleft}{\kern0pt}typecheck{\isacharunderscore}{\kern0pt}cfuncs{\isacharcomma}{\kern0pt}\ blast{\isacharparenright}{\kern0pt}\isanewline
\ \ \isacommand{have}\isamarkupfalse%
\ h{\isacharunderscore}{\kern0pt}unique{\isacharcolon}{\kern0pt}\ {\isachardoublequoteopen}{\isasymexists}{\isacharbang}{\kern0pt}\ h{\isachardot}{\kern0pt}\ {\isacharparenleft}{\kern0pt}{\isacharparenleft}{\kern0pt}h{\isacharcolon}{\kern0pt}\ {\isacharparenleft}{\kern0pt}A\ {\isasymCoprod}\ B{\isacharparenright}{\kern0pt}\ {\isasymrightarrow}\ A\ {\isasymCoprod}\ {\isacharparenleft}{\kern0pt}B\ {\isasymCoprod}\ C{\isacharparenright}{\kern0pt}{\isacharparenright}{\kern0pt}\ {\isasymand}\ {\isacharparenleft}{\kern0pt}h\ {\isasymcirc}\isactrlsub c\ {\isacharparenleft}{\kern0pt}left{\isacharunderscore}{\kern0pt}coproj\ A\ B{\isacharparenright}{\kern0pt}\ \ {\isacharequal}{\kern0pt}\ {\isacharparenleft}{\kern0pt}left{\isacharunderscore}{\kern0pt}coproj\ A\ {\isacharparenleft}{\kern0pt}B\ {\isasymCoprod}\ C{\isacharparenright}{\kern0pt}{\isacharparenright}{\kern0pt}{\isacharparenright}{\kern0pt}\ {\isasymand}\ {\isacharparenleft}{\kern0pt}h\ {\isasymcirc}\isactrlsub c\ {\isacharparenleft}{\kern0pt}right{\isacharunderscore}{\kern0pt}coproj\ A\ B{\isacharparenright}{\kern0pt}\ {\isacharequal}{\kern0pt}p{\isacharparenright}{\kern0pt}{\isacharparenright}{\kern0pt}{\isachardoublequoteclose}\isanewline
\ \ \ \ \isacommand{by}\isamarkupfalse%
\ {\isacharparenleft}{\kern0pt}typecheck{\isacharunderscore}{\kern0pt}cfuncs{\isacharcomma}{\kern0pt}\ metis\ cfunc{\isacharunderscore}{\kern0pt}coprod{\isacharunderscore}{\kern0pt}unique\ h{\isacharunderscore}{\kern0pt}prop{\isadigit{1}}\ h{\isacharunderscore}{\kern0pt}prop{\isadigit{2}}\ h{\isacharunderscore}{\kern0pt}type{\isacharparenright}{\kern0pt}\isanewline
\isanewline
\ \ \isacommand{obtain}\isamarkupfalse%
\ j\ \isakeyword{where}\ j{\isacharunderscore}{\kern0pt}def{\isacharcolon}{\kern0pt}\ {\isachardoublequoteopen}j\ {\isacharequal}{\kern0pt}\ {\isacharparenleft}{\kern0pt}right{\isacharunderscore}{\kern0pt}coproj\ A\ {\isacharparenleft}{\kern0pt}B\ {\isasymCoprod}\ C{\isacharparenright}{\kern0pt}{\isacharparenright}{\kern0pt}\ {\isasymcirc}\isactrlsub c\ \ {\isacharparenleft}{\kern0pt}right{\isacharunderscore}{\kern0pt}coproj\ B\ C{\isacharparenright}{\kern0pt}{\isachardoublequoteclose}\ \isakeyword{and}\ j{\isacharunderscore}{\kern0pt}type{\isacharbrackleft}{\kern0pt}type{\isacharunderscore}{\kern0pt}rule{\isacharbrackright}{\kern0pt}{\isacharcolon}{\kern0pt}\ {\isachardoublequoteopen}j\ {\isacharcolon}{\kern0pt}\ C\ {\isasymrightarrow}\ A\ {\isasymCoprod}\ {\isacharparenleft}{\kern0pt}B\ {\isasymCoprod}\ C{\isacharparenright}{\kern0pt}{\isachardoublequoteclose}\isanewline
\ \ \ \ \isacommand{by}\isamarkupfalse%
\ typecheck{\isacharunderscore}{\kern0pt}cfuncs\isanewline
\ \ \isacommand{obtain}\isamarkupfalse%
\ k\ \isakeyword{where}\ k{\isacharunderscore}{\kern0pt}def{\isacharcolon}{\kern0pt}\ {\isachardoublequoteopen}k\ {\isacharequal}{\kern0pt}\ h\ {\isasymamalg}\ j{\isachardoublequoteclose}\ \isakeyword{and}\ k{\isacharunderscore}{\kern0pt}type{\isacharbrackleft}{\kern0pt}type{\isacharunderscore}{\kern0pt}rule{\isacharbrackright}{\kern0pt}{\isacharcolon}{\kern0pt}\ {\isachardoublequoteopen}k{\isacharcolon}{\kern0pt}\ {\isacharparenleft}{\kern0pt}A\ {\isasymCoprod}\ B{\isacharparenright}{\kern0pt}\ {\isasymCoprod}\ C\ {\isasymrightarrow}\ A\ {\isasymCoprod}\ {\isacharparenleft}{\kern0pt}B\ {\isasymCoprod}\ C{\isacharparenright}{\kern0pt}{\isachardoublequoteclose}\isanewline
\ \ \ \ \isacommand{by}\isamarkupfalse%
\ typecheck{\isacharunderscore}{\kern0pt}cfuncs\isanewline
\isanewline
\ \ \isacommand{have}\isamarkupfalse%
\ fact{\isadigit{1}}{\isacharcolon}{\kern0pt}\ {\isachardoublequoteopen}{\isacharparenleft}{\kern0pt}k\ {\isasymcirc}\isactrlsub c\ g{\isacharparenright}{\kern0pt}\ {\isasymcirc}\isactrlsub c\ {\isacharparenleft}{\kern0pt}left{\isacharunderscore}{\kern0pt}coproj\ A\ {\isacharparenleft}{\kern0pt}B\ {\isasymCoprod}\ C{\isacharparenright}{\kern0pt}{\isacharparenright}{\kern0pt}\ {\isacharequal}{\kern0pt}\ {\isacharparenleft}{\kern0pt}left{\isacharunderscore}{\kern0pt}coproj\ A\ {\isacharparenleft}{\kern0pt}B\ {\isasymCoprod}\ C{\isacharparenright}{\kern0pt}{\isacharparenright}{\kern0pt}{\isachardoublequoteclose}\isanewline
\ \ \ \ \isacommand{by}\isamarkupfalse%
\ {\isacharparenleft}{\kern0pt}typecheck{\isacharunderscore}{\kern0pt}cfuncs{\isacharcomma}{\kern0pt}\ smt\ {\isacharparenleft}{\kern0pt}z{\isadigit{3}}{\isacharparenright}{\kern0pt}\ comp{\isacharunderscore}{\kern0pt}associative{\isadigit{2}}\ g{\isacharunderscore}{\kern0pt}prop\ h{\isacharunderscore}{\kern0pt}prop{\isadigit{1}}\ h{\isacharunderscore}{\kern0pt}type\ j{\isacharunderscore}{\kern0pt}type\ k{\isacharunderscore}{\kern0pt}def\ left{\isacharunderscore}{\kern0pt}coproj{\isacharunderscore}{\kern0pt}cfunc{\isacharunderscore}{\kern0pt}coprod\ left{\isacharunderscore}{\kern0pt}proj{\isacharunderscore}{\kern0pt}type\ m{\isacharunderscore}{\kern0pt}def{\isacharparenright}{\kern0pt}\isanewline
\ \ \isacommand{have}\isamarkupfalse%
\ fact{\isadigit{2}}{\isacharcolon}{\kern0pt}\ {\isachardoublequoteopen}{\isacharparenleft}{\kern0pt}g\ {\isasymcirc}\isactrlsub c\ k{\isacharparenright}{\kern0pt}\ {\isasymcirc}\isactrlsub c\ {\isacharparenleft}{\kern0pt}left{\isacharunderscore}{\kern0pt}coproj\ {\isacharparenleft}{\kern0pt}A\ {\isasymCoprod}\ B{\isacharparenright}{\kern0pt}\ C{\isacharparenright}{\kern0pt}\ {\isacharequal}{\kern0pt}\ {\isacharparenleft}{\kern0pt}left{\isacharunderscore}{\kern0pt}coproj\ {\isacharparenleft}{\kern0pt}A\ {\isasymCoprod}\ B{\isacharparenright}{\kern0pt}\ C{\isacharparenright}{\kern0pt}{\isachardoublequoteclose}\isanewline
\ \ \ \ \isacommand{by}\isamarkupfalse%
\ {\isacharparenleft}{\kern0pt}typecheck{\isacharunderscore}{\kern0pt}cfuncs{\isacharcomma}{\kern0pt}\ smt\ {\isacharparenleft}{\kern0pt}verit{\isacharparenright}{\kern0pt}\ cfunc{\isacharunderscore}{\kern0pt}coprod{\isacharunderscore}{\kern0pt}comp\ cfunc{\isacharunderscore}{\kern0pt}coprod{\isacharunderscore}{\kern0pt}unique\ comp{\isacharunderscore}{\kern0pt}associative{\isadigit{2}}\ comp{\isacharunderscore}{\kern0pt}type\ f{\isacharunderscore}{\kern0pt}prop\ g{\isacharunderscore}{\kern0pt}prop\ g{\isacharunderscore}{\kern0pt}type\ h{\isacharunderscore}{\kern0pt}def\ h{\isacharunderscore}{\kern0pt}type\ j{\isacharunderscore}{\kern0pt}def\ k{\isacharunderscore}{\kern0pt}def\ k{\isacharunderscore}{\kern0pt}type\ left{\isacharunderscore}{\kern0pt}coproj{\isacharunderscore}{\kern0pt}cfunc{\isacharunderscore}{\kern0pt}coprod\ left{\isacharunderscore}{\kern0pt}proj{\isacharunderscore}{\kern0pt}type\ m{\isacharunderscore}{\kern0pt}def\ p{\isacharunderscore}{\kern0pt}def\ p{\isacharunderscore}{\kern0pt}type\ q{\isacharunderscore}{\kern0pt}def\ right{\isacharunderscore}{\kern0pt}proj{\isacharunderscore}{\kern0pt}type{\isacharparenright}{\kern0pt}\isanewline
\ \ \isacommand{have}\isamarkupfalse%
\ fact{\isadigit{3}}{\isacharcolon}{\kern0pt}\ {\isachardoublequoteopen}{\isacharparenleft}{\kern0pt}g\ {\isasymcirc}\isactrlsub c\ k{\isacharparenright}{\kern0pt}\ {\isasymcirc}\isactrlsub c\ {\isacharparenleft}{\kern0pt}right{\isacharunderscore}{\kern0pt}coproj\ {\isacharparenleft}{\kern0pt}A\ {\isasymCoprod}\ B{\isacharparenright}{\kern0pt}\ C{\isacharparenright}{\kern0pt}\ {\isacharequal}{\kern0pt}\ {\isacharparenleft}{\kern0pt}right{\isacharunderscore}{\kern0pt}coproj\ {\isacharparenleft}{\kern0pt}A\ {\isasymCoprod}\ B{\isacharparenright}{\kern0pt}\ C{\isacharparenright}{\kern0pt}{\isachardoublequoteclose}\isanewline
\ \ \ \ \isacommand{by}\isamarkupfalse%
\ {\isacharparenleft}{\kern0pt}smt\ comp{\isacharunderscore}{\kern0pt}associative{\isadigit{2}}\ comp{\isacharunderscore}{\kern0pt}type\ f{\isacharunderscore}{\kern0pt}def\ g{\isacharunderscore}{\kern0pt}prop\ g{\isacharunderscore}{\kern0pt}type\ h{\isacharunderscore}{\kern0pt}type\ j{\isacharunderscore}{\kern0pt}def\ k{\isacharunderscore}{\kern0pt}def\ k{\isacharunderscore}{\kern0pt}type\ q{\isacharunderscore}{\kern0pt}type\ right{\isacharunderscore}{\kern0pt}coproj{\isacharunderscore}{\kern0pt}cfunc{\isacharunderscore}{\kern0pt}coprod\ right{\isacharunderscore}{\kern0pt}proj{\isacharunderscore}{\kern0pt}type{\isacharparenright}{\kern0pt}\isanewline
\ \ \isacommand{have}\isamarkupfalse%
\ fact{\isadigit{4}}{\isacharcolon}{\kern0pt}\ {\isachardoublequoteopen}{\isacharparenleft}{\kern0pt}k\ {\isasymcirc}\isactrlsub c\ g{\isacharparenright}{\kern0pt}\ {\isasymcirc}\isactrlsub c\ {\isacharparenleft}{\kern0pt}right{\isacharunderscore}{\kern0pt}coproj\ A\ {\isacharparenleft}{\kern0pt}B\ {\isasymCoprod}\ C{\isacharparenright}{\kern0pt}{\isacharparenright}{\kern0pt}\ {\isacharequal}{\kern0pt}\ {\isacharparenleft}{\kern0pt}right{\isacharunderscore}{\kern0pt}coproj\ A\ {\isacharparenleft}{\kern0pt}B\ {\isasymCoprod}\ C{\isacharparenright}{\kern0pt}{\isacharparenright}{\kern0pt}{\isachardoublequoteclose}\isanewline
\ \ \ \ \isacommand{by}\isamarkupfalse%
\ {\isacharparenleft}{\kern0pt}typecheck{\isacharunderscore}{\kern0pt}cfuncs{\isacharcomma}{\kern0pt}\ smt\ {\isacharparenleft}{\kern0pt}verit{\isacharcomma}{\kern0pt}\ ccfv{\isacharunderscore}{\kern0pt}threshold{\isacharparenright}{\kern0pt}\ cfunc{\isacharunderscore}{\kern0pt}coprod{\isacharunderscore}{\kern0pt}unique\ cfunc{\isacharunderscore}{\kern0pt}type{\isacharunderscore}{\kern0pt}def\ comp{\isacharunderscore}{\kern0pt}associative\ comp{\isacharunderscore}{\kern0pt}type\ f{\isacharunderscore}{\kern0pt}prop\ g{\isacharunderscore}{\kern0pt}prop\ h{\isacharunderscore}{\kern0pt}prop{\isadigit{2}}\ h{\isacharunderscore}{\kern0pt}type\ j{\isacharunderscore}{\kern0pt}def\ k{\isacharunderscore}{\kern0pt}def\ left{\isacharunderscore}{\kern0pt}coproj{\isacharunderscore}{\kern0pt}cfunc{\isacharunderscore}{\kern0pt}coprod\ left{\isacharunderscore}{\kern0pt}proj{\isacharunderscore}{\kern0pt}type\ p{\isacharunderscore}{\kern0pt}def\ q{\isacharunderscore}{\kern0pt}def\ right{\isacharunderscore}{\kern0pt}coproj{\isacharunderscore}{\kern0pt}cfunc{\isacharunderscore}{\kern0pt}coprod\ right{\isacharunderscore}{\kern0pt}proj{\isacharunderscore}{\kern0pt}type{\isacharparenright}{\kern0pt}\isanewline
\ \ \isacommand{have}\isamarkupfalse%
\ fact{\isadigit{5}}{\isacharcolon}{\kern0pt}\ {\isachardoublequoteopen}{\isacharparenleft}{\kern0pt}k\ {\isasymcirc}\isactrlsub c\ g{\isacharparenright}{\kern0pt}\ {\isacharequal}{\kern0pt}\ id{\isacharparenleft}{\kern0pt}\ A\ {\isasymCoprod}\ {\isacharparenleft}{\kern0pt}B\ {\isasymCoprod}\ C{\isacharparenright}{\kern0pt}{\isacharparenright}{\kern0pt}{\isachardoublequoteclose}\isanewline
\ \ \ \ \isacommand{by}\isamarkupfalse%
\ {\isacharparenleft}{\kern0pt}typecheck{\isacharunderscore}{\kern0pt}cfuncs{\isacharcomma}{\kern0pt}\ metis\ cfunc{\isacharunderscore}{\kern0pt}coprod{\isacharunderscore}{\kern0pt}unique\ fact{\isadigit{1}}\ fact{\isadigit{4}}\ id{\isacharunderscore}{\kern0pt}coprod\ left{\isacharunderscore}{\kern0pt}proj{\isacharunderscore}{\kern0pt}type\ right{\isacharunderscore}{\kern0pt}proj{\isacharunderscore}{\kern0pt}type{\isacharparenright}{\kern0pt}\isanewline
\ \ \isacommand{have}\isamarkupfalse%
\ fact{\isadigit{6}}{\isacharcolon}{\kern0pt}\ {\isachardoublequoteopen}{\isacharparenleft}{\kern0pt}g\ {\isasymcirc}\isactrlsub c\ k{\isacharparenright}{\kern0pt}\ {\isacharequal}{\kern0pt}\ id{\isacharparenleft}{\kern0pt}{\isacharparenleft}{\kern0pt}A\ {\isasymCoprod}\ B{\isacharparenright}{\kern0pt}\ {\isasymCoprod}\ C{\isacharparenright}{\kern0pt}{\isachardoublequoteclose}\isanewline
\ \ \ \ \isacommand{by}\isamarkupfalse%
\ {\isacharparenleft}{\kern0pt}typecheck{\isacharunderscore}{\kern0pt}cfuncs{\isacharcomma}{\kern0pt}\ metis\ cfunc{\isacharunderscore}{\kern0pt}coprod{\isacharunderscore}{\kern0pt}unique\ fact{\isadigit{2}}\ fact{\isadigit{3}}\ id{\isacharunderscore}{\kern0pt}coprod\ left{\isacharunderscore}{\kern0pt}proj{\isacharunderscore}{\kern0pt}type\ right{\isacharunderscore}{\kern0pt}proj{\isacharunderscore}{\kern0pt}type{\isacharparenright}{\kern0pt}\isanewline
\ \ \isacommand{show}\isamarkupfalse%
\ {\isacharquery}{\kern0pt}thesis\isanewline
\ \ \ \ \isacommand{by}\isamarkupfalse%
\ {\isacharparenleft}{\kern0pt}metis\ cfunc{\isacharunderscore}{\kern0pt}type{\isacharunderscore}{\kern0pt}def\ fact{\isadigit{5}}\ fact{\isadigit{6}}\ g{\isacharunderscore}{\kern0pt}type\ is{\isacharunderscore}{\kern0pt}isomorphic{\isacharunderscore}{\kern0pt}def\ isomorphism{\isacharunderscore}{\kern0pt}def\ k{\isacharunderscore}{\kern0pt}type{\isacharparenright}{\kern0pt}\isanewline
\isacommand{qed}\isamarkupfalse%
%
\endisatagproof
{\isafoldproof}%
%
\isadelimproof
%
\endisadelimproof
%
\begin{isamarkuptext}%
The lemma below corresponds to Proposition 2.5.10.%
\end{isamarkuptext}\isamarkuptrue%
\isacommand{lemma}\isamarkupfalse%
\ product{\isacharunderscore}{\kern0pt}distribute{\isacharunderscore}{\kern0pt}over{\isacharunderscore}{\kern0pt}coproduct{\isacharunderscore}{\kern0pt}left{\isacharcolon}{\kern0pt}\isanewline
\ \ {\isachardoublequoteopen}A\ {\isasymtimes}\isactrlsub c\ {\isacharparenleft}{\kern0pt}X\ {\isasymCoprod}\ Y{\isacharparenright}{\kern0pt}\ {\isasymcong}\ {\isacharparenleft}{\kern0pt}A\ {\isasymtimes}\isactrlsub c\ X{\isacharparenright}{\kern0pt}\ {\isasymCoprod}\ {\isacharparenleft}{\kern0pt}A\ {\isasymtimes}\isactrlsub c\ Y{\isacharparenright}{\kern0pt}{\isachardoublequoteclose}\isanewline
%
\isadelimproof
\ \ %
\endisadelimproof
%
\isatagproof
\isacommand{using}\isamarkupfalse%
\ dist{\isacharunderscore}{\kern0pt}prod{\isacharunderscore}{\kern0pt}coprod{\isacharunderscore}{\kern0pt}type\ dist{\isacharunderscore}{\kern0pt}prod{\isacharunderscore}{\kern0pt}coprod{\isacharunderscore}{\kern0pt}iso\ is{\isacharunderscore}{\kern0pt}isomorphic{\isacharunderscore}{\kern0pt}def\ isomorphic{\isacharunderscore}{\kern0pt}is{\isacharunderscore}{\kern0pt}symmetric\ \isacommand{by}\isamarkupfalse%
\ blast%
\endisatagproof
{\isafoldproof}%
%
\isadelimproof
\isanewline
%
\endisadelimproof
\isanewline
\isacommand{lemma}\isamarkupfalse%
\ prod{\isacharunderscore}{\kern0pt}pres{\isacharunderscore}{\kern0pt}iso{\isacharcolon}{\kern0pt}\isanewline
\ \ \isakeyword{assumes}\ {\isachardoublequoteopen}A\ {\isasymcong}\ \ C{\isachardoublequoteclose}\ \ {\isachardoublequoteopen}B\ {\isasymcong}\ D{\isachardoublequoteclose}\isanewline
\ \ \isakeyword{shows}\ {\isachardoublequoteopen}A\ {\isasymtimes}\isactrlsub c\ B\ {\isasymcong}\ \ C\ {\isasymtimes}\isactrlsub c\ D{\isachardoublequoteclose}\isanewline
%
\isadelimproof
%
\endisadelimproof
%
\isatagproof
\isacommand{proof}\isamarkupfalse%
\ {\isacharminus}{\kern0pt}\ \isanewline
\ \ \isacommand{obtain}\isamarkupfalse%
\ f\ \isakeyword{where}\ f{\isacharunderscore}{\kern0pt}def{\isacharcolon}{\kern0pt}\ {\isachardoublequoteopen}f{\isacharcolon}{\kern0pt}\ A\ {\isasymrightarrow}\ C\ {\isasymand}\ isomorphism{\isacharparenleft}{\kern0pt}f{\isacharparenright}{\kern0pt}{\isachardoublequoteclose}\isanewline
\ \ \ \ \isacommand{using}\isamarkupfalse%
\ assms{\isacharparenleft}{\kern0pt}{\isadigit{1}}{\isacharparenright}{\kern0pt}\ is{\isacharunderscore}{\kern0pt}isomorphic{\isacharunderscore}{\kern0pt}def\ \isacommand{by}\isamarkupfalse%
\ blast\isanewline
\ \ \isacommand{obtain}\isamarkupfalse%
\ g\ \isakeyword{where}\ g{\isacharunderscore}{\kern0pt}def{\isacharcolon}{\kern0pt}\ {\isachardoublequoteopen}g{\isacharcolon}{\kern0pt}\ B\ {\isasymrightarrow}\ D\ {\isasymand}\ isomorphism{\isacharparenleft}{\kern0pt}g{\isacharparenright}{\kern0pt}{\isachardoublequoteclose}\isanewline
\ \ \ \ \isacommand{using}\isamarkupfalse%
\ assms{\isacharparenleft}{\kern0pt}{\isadigit{2}}{\isacharparenright}{\kern0pt}\ is{\isacharunderscore}{\kern0pt}isomorphic{\isacharunderscore}{\kern0pt}def\ \isacommand{by}\isamarkupfalse%
\ blast\isanewline
\ \ \isacommand{have}\isamarkupfalse%
\ {\isachardoublequoteopen}isomorphism{\isacharparenleft}{\kern0pt}f{\isasymtimes}\isactrlsub fg{\isacharparenright}{\kern0pt}{\isachardoublequoteclose}\isanewline
\ \ \ \ \isacommand{by}\isamarkupfalse%
\ {\isacharparenleft}{\kern0pt}meson\ cfunc{\isacharunderscore}{\kern0pt}cross{\isacharunderscore}{\kern0pt}prod{\isacharunderscore}{\kern0pt}mono\ cfunc{\isacharunderscore}{\kern0pt}cross{\isacharunderscore}{\kern0pt}prod{\isacharunderscore}{\kern0pt}surj\ epi{\isacharunderscore}{\kern0pt}is{\isacharunderscore}{\kern0pt}surj\ epi{\isacharunderscore}{\kern0pt}mon{\isacharunderscore}{\kern0pt}is{\isacharunderscore}{\kern0pt}iso\ f{\isacharunderscore}{\kern0pt}def\ g{\isacharunderscore}{\kern0pt}def\ iso{\isacharunderscore}{\kern0pt}imp{\isacharunderscore}{\kern0pt}epi{\isacharunderscore}{\kern0pt}and{\isacharunderscore}{\kern0pt}monic\ surjective{\isacharunderscore}{\kern0pt}is{\isacharunderscore}{\kern0pt}epimorphism{\isacharparenright}{\kern0pt}\isanewline
\ \ \isacommand{then}\isamarkupfalse%
\ \isacommand{show}\isamarkupfalse%
\ {\isachardoublequoteopen}A\ {\isasymtimes}\isactrlsub c\ B\ {\isasymcong}\ \ C\ {\isasymtimes}\isactrlsub c\ D{\isachardoublequoteclose}\isanewline
\ \ \ \ \isacommand{by}\isamarkupfalse%
\ {\isacharparenleft}{\kern0pt}meson\ cfunc{\isacharunderscore}{\kern0pt}cross{\isacharunderscore}{\kern0pt}prod{\isacharunderscore}{\kern0pt}type\ f{\isacharunderscore}{\kern0pt}def\ g{\isacharunderscore}{\kern0pt}def\ is{\isacharunderscore}{\kern0pt}isomorphic{\isacharunderscore}{\kern0pt}def{\isacharparenright}{\kern0pt}\isanewline
\isacommand{qed}\isamarkupfalse%
%
\endisatagproof
{\isafoldproof}%
%
\isadelimproof
\isanewline
%
\endisadelimproof
\isanewline
\isacommand{lemma}\isamarkupfalse%
\ coprod{\isacharunderscore}{\kern0pt}pres{\isacharunderscore}{\kern0pt}iso{\isacharcolon}{\kern0pt}\isanewline
\ \ \isakeyword{assumes}\ {\isachardoublequoteopen}A\ {\isasymcong}\ \ C{\isachardoublequoteclose}\ \ {\isachardoublequoteopen}B\ {\isasymcong}\ D{\isachardoublequoteclose}\isanewline
\ \ \isakeyword{shows}\ {\isachardoublequoteopen}A\ {\isasymCoprod}\ B\ {\isasymcong}\ \ C\ {\isasymCoprod}\ D{\isachardoublequoteclose}\isanewline
%
\isadelimproof
%
\endisadelimproof
%
\isatagproof
\isacommand{proof}\isamarkupfalse%
{\isacharminus}{\kern0pt}\ \isanewline
\ \ \isacommand{obtain}\isamarkupfalse%
\ f\ \isakeyword{where}\ f{\isacharunderscore}{\kern0pt}def{\isacharcolon}{\kern0pt}\ {\isachardoublequoteopen}f{\isacharcolon}{\kern0pt}\ A\ {\isasymrightarrow}\ C{\isachardoublequoteclose}\ {\isachardoublequoteopen}isomorphism{\isacharparenleft}{\kern0pt}f{\isacharparenright}{\kern0pt}{\isachardoublequoteclose}\isanewline
\ \ \ \ \isacommand{using}\isamarkupfalse%
\ assms{\isacharparenleft}{\kern0pt}{\isadigit{1}}{\isacharparenright}{\kern0pt}\ is{\isacharunderscore}{\kern0pt}isomorphic{\isacharunderscore}{\kern0pt}def\ \isacommand{by}\isamarkupfalse%
\ blast\isanewline
\ \ \isacommand{obtain}\isamarkupfalse%
\ g\ \isakeyword{where}\ g{\isacharunderscore}{\kern0pt}def{\isacharcolon}{\kern0pt}\ {\isachardoublequoteopen}g{\isacharcolon}{\kern0pt}\ B\ {\isasymrightarrow}\ D{\isachardoublequoteclose}\ {\isachardoublequoteopen}isomorphism{\isacharparenleft}{\kern0pt}g{\isacharparenright}{\kern0pt}{\isachardoublequoteclose}\isanewline
\ \ \ \ \isacommand{using}\isamarkupfalse%
\ assms{\isacharparenleft}{\kern0pt}{\isadigit{2}}{\isacharparenright}{\kern0pt}\ is{\isacharunderscore}{\kern0pt}isomorphic{\isacharunderscore}{\kern0pt}def\ \isacommand{by}\isamarkupfalse%
\ blast\isanewline
\isanewline
\ \ \isacommand{have}\isamarkupfalse%
\ surj{\isacharunderscore}{\kern0pt}f{\isacharcolon}{\kern0pt}\ {\isachardoublequoteopen}surjective{\isacharparenleft}{\kern0pt}f{\isacharparenright}{\kern0pt}{\isachardoublequoteclose}\isanewline
\ \ \ \ \isacommand{using}\isamarkupfalse%
\ epi{\isacharunderscore}{\kern0pt}is{\isacharunderscore}{\kern0pt}surj\ f{\isacharunderscore}{\kern0pt}def\ iso{\isacharunderscore}{\kern0pt}imp{\isacharunderscore}{\kern0pt}epi{\isacharunderscore}{\kern0pt}and{\isacharunderscore}{\kern0pt}monic\ \isacommand{by}\isamarkupfalse%
\ blast\isanewline
\ \ \isacommand{have}\isamarkupfalse%
\ surj{\isacharunderscore}{\kern0pt}g{\isacharcolon}{\kern0pt}\ {\isachardoublequoteopen}surjective{\isacharparenleft}{\kern0pt}g{\isacharparenright}{\kern0pt}{\isachardoublequoteclose}\isanewline
\ \ \ \ \isacommand{using}\isamarkupfalse%
\ epi{\isacharunderscore}{\kern0pt}is{\isacharunderscore}{\kern0pt}surj\ g{\isacharunderscore}{\kern0pt}def\ iso{\isacharunderscore}{\kern0pt}imp{\isacharunderscore}{\kern0pt}epi{\isacharunderscore}{\kern0pt}and{\isacharunderscore}{\kern0pt}monic\ \isacommand{by}\isamarkupfalse%
\ blast\isanewline
\isanewline
\ \ \isacommand{have}\isamarkupfalse%
\ coproj{\isacharunderscore}{\kern0pt}f{\isacharunderscore}{\kern0pt}inject{\isacharcolon}{\kern0pt}\ {\isachardoublequoteopen}injective{\isacharparenleft}{\kern0pt}{\isacharparenleft}{\kern0pt}{\isacharparenleft}{\kern0pt}left{\isacharunderscore}{\kern0pt}coproj\ C\ D{\isacharparenright}{\kern0pt}\ {\isasymcirc}\isactrlsub c\ f{\isacharparenright}{\kern0pt}{\isacharparenright}{\kern0pt}{\isachardoublequoteclose}\isanewline
\ \ \ \ \isacommand{using}\isamarkupfalse%
\ cfunc{\isacharunderscore}{\kern0pt}type{\isacharunderscore}{\kern0pt}def\ composition{\isacharunderscore}{\kern0pt}of{\isacharunderscore}{\kern0pt}monic{\isacharunderscore}{\kern0pt}pair{\isacharunderscore}{\kern0pt}is{\isacharunderscore}{\kern0pt}monic\ f{\isacharunderscore}{\kern0pt}def\ iso{\isacharunderscore}{\kern0pt}imp{\isacharunderscore}{\kern0pt}epi{\isacharunderscore}{\kern0pt}and{\isacharunderscore}{\kern0pt}monic\ left{\isacharunderscore}{\kern0pt}coproj{\isacharunderscore}{\kern0pt}are{\isacharunderscore}{\kern0pt}monomorphisms\ left{\isacharunderscore}{\kern0pt}proj{\isacharunderscore}{\kern0pt}type\ monomorphism{\isacharunderscore}{\kern0pt}imp{\isacharunderscore}{\kern0pt}injective\ \isacommand{by}\isamarkupfalse%
\ auto\isanewline
\ \ \ \ \isanewline
\ \ \isacommand{have}\isamarkupfalse%
\ coproj{\isacharunderscore}{\kern0pt}g{\isacharunderscore}{\kern0pt}inject{\isacharcolon}{\kern0pt}\ {\isachardoublequoteopen}injective{\isacharparenleft}{\kern0pt}{\isacharparenleft}{\kern0pt}{\isacharparenleft}{\kern0pt}right{\isacharunderscore}{\kern0pt}coproj\ C\ D{\isacharparenright}{\kern0pt}\ {\isasymcirc}\isactrlsub c\ g{\isacharparenright}{\kern0pt}{\isacharparenright}{\kern0pt}{\isachardoublequoteclose}\isanewline
\ \ \ \ \isacommand{using}\isamarkupfalse%
\ cfunc{\isacharunderscore}{\kern0pt}type{\isacharunderscore}{\kern0pt}def\ composition{\isacharunderscore}{\kern0pt}of{\isacharunderscore}{\kern0pt}monic{\isacharunderscore}{\kern0pt}pair{\isacharunderscore}{\kern0pt}is{\isacharunderscore}{\kern0pt}monic\ g{\isacharunderscore}{\kern0pt}def\ iso{\isacharunderscore}{\kern0pt}imp{\isacharunderscore}{\kern0pt}epi{\isacharunderscore}{\kern0pt}and{\isacharunderscore}{\kern0pt}monic\ right{\isacharunderscore}{\kern0pt}coproj{\isacharunderscore}{\kern0pt}are{\isacharunderscore}{\kern0pt}monomorphisms\ right{\isacharunderscore}{\kern0pt}proj{\isacharunderscore}{\kern0pt}type\ monomorphism{\isacharunderscore}{\kern0pt}imp{\isacharunderscore}{\kern0pt}injective\ \isacommand{by}\isamarkupfalse%
\ auto\isanewline
\isanewline
\ \ \isacommand{obtain}\isamarkupfalse%
\ {\isasymphi}\ \isakeyword{where}\ {\isasymphi}{\isacharunderscore}{\kern0pt}def{\isacharcolon}{\kern0pt}\ {\isachardoublequoteopen}{\isasymphi}\ {\isacharequal}{\kern0pt}\ {\isacharparenleft}{\kern0pt}left{\isacharunderscore}{\kern0pt}coproj\ C\ D\ {\isasymcirc}\isactrlsub c\ f{\isacharparenright}{\kern0pt}\ \ {\isasymamalg}\ {\isacharparenleft}{\kern0pt}right{\isacharunderscore}{\kern0pt}coproj\ C\ D\ {\isasymcirc}\isactrlsub c\ g{\isacharparenright}{\kern0pt}{\isachardoublequoteclose}\isanewline
\ \ \ \ \isacommand{by}\isamarkupfalse%
\ simp\isanewline
\ \ \isacommand{then}\isamarkupfalse%
\ \isacommand{have}\isamarkupfalse%
\ {\isasymphi}{\isacharunderscore}{\kern0pt}type{\isacharcolon}{\kern0pt}\ {\isachardoublequoteopen}{\isasymphi}{\isacharcolon}{\kern0pt}\ A\ {\isasymCoprod}\ B\ {\isasymrightarrow}\ \ C\ {\isasymCoprod}\ D{\isachardoublequoteclose}\isanewline
\ \ \ \ \isacommand{using}\isamarkupfalse%
\ cfunc{\isacharunderscore}{\kern0pt}coprod{\isacharunderscore}{\kern0pt}type\ cfunc{\isacharunderscore}{\kern0pt}type{\isacharunderscore}{\kern0pt}def\ codomain{\isacharunderscore}{\kern0pt}comp\ domain{\isacharunderscore}{\kern0pt}comp\ f{\isacharunderscore}{\kern0pt}def\ g{\isacharunderscore}{\kern0pt}def\ left{\isacharunderscore}{\kern0pt}proj{\isacharunderscore}{\kern0pt}type\ right{\isacharunderscore}{\kern0pt}proj{\isacharunderscore}{\kern0pt}type\ \isacommand{by}\isamarkupfalse%
\ auto\isanewline
\isanewline
\ \ \isacommand{have}\isamarkupfalse%
\ {\isachardoublequoteopen}surjective{\isacharparenleft}{\kern0pt}{\isasymphi}{\isacharparenright}{\kern0pt}{\isachardoublequoteclose}\isanewline
\ \ \ \ \isacommand{unfolding}\isamarkupfalse%
\ surjective{\isacharunderscore}{\kern0pt}def\isanewline
\ \ \isacommand{proof}\isamarkupfalse%
{\isacharparenleft}{\kern0pt}auto{\isacharparenright}{\kern0pt}\ \isanewline
\ \ \ \ \isacommand{fix}\isamarkupfalse%
\ y\ \isanewline
\ \ \ \ \isacommand{assume}\isamarkupfalse%
\ y{\isacharunderscore}{\kern0pt}type{\isacharcolon}{\kern0pt}\ {\isachardoublequoteopen}y\ {\isasymin}\isactrlsub c\ codomain\ {\isasymphi}{\isachardoublequoteclose}\isanewline
\ \ \ \ \isacommand{then}\isamarkupfalse%
\ \isacommand{have}\isamarkupfalse%
\ y{\isacharunderscore}{\kern0pt}type{\isadigit{2}}{\isacharcolon}{\kern0pt}\ {\isachardoublequoteopen}y\ {\isasymin}\isactrlsub c\ C\ {\isasymCoprod}\ D{\isachardoublequoteclose}\isanewline
\ \ \ \ \ \ \isacommand{using}\isamarkupfalse%
\ {\isasymphi}{\isacharunderscore}{\kern0pt}type\ cfunc{\isacharunderscore}{\kern0pt}type{\isacharunderscore}{\kern0pt}def\ \isacommand{by}\isamarkupfalse%
\ auto\isanewline
\ \ \ \ \isacommand{then}\isamarkupfalse%
\ \isacommand{have}\isamarkupfalse%
\ y{\isacharunderscore}{\kern0pt}form{\isacharcolon}{\kern0pt}\ {\isachardoublequoteopen}{\isacharparenleft}{\kern0pt}{\isasymexists}\ c{\isachardot}{\kern0pt}\ c\ {\isasymin}\isactrlsub c\ C\ {\isasymand}\ y\ {\isacharequal}{\kern0pt}\ left{\isacharunderscore}{\kern0pt}coproj\ C\ D\ {\isasymcirc}\isactrlsub c\ c{\isacharparenright}{\kern0pt}\isanewline
\ \ \ \ \ \ {\isasymor}\ \ {\isacharparenleft}{\kern0pt}{\isasymexists}\ d{\isachardot}{\kern0pt}\ d\ {\isasymin}\isactrlsub c\ D\ {\isasymand}\ y\ {\isacharequal}{\kern0pt}\ right{\isacharunderscore}{\kern0pt}coproj\ C\ D\ {\isasymcirc}\isactrlsub c\ d{\isacharparenright}{\kern0pt}{\isachardoublequoteclose}\isanewline
\ \ \ \ \ \ \isacommand{using}\isamarkupfalse%
\ coprojs{\isacharunderscore}{\kern0pt}jointly{\isacharunderscore}{\kern0pt}surj\ \isacommand{by}\isamarkupfalse%
\ auto\isanewline
\ \ \ \ \isacommand{show}\isamarkupfalse%
\ {\isachardoublequoteopen}{\isasymexists}x{\isachardot}{\kern0pt}\ x\ {\isasymin}\isactrlsub c\ domain\ {\isasymphi}\ {\isasymand}\ {\isasymphi}\ {\isasymcirc}\isactrlsub c\ x\ {\isacharequal}{\kern0pt}\ y{\isachardoublequoteclose}\isanewline
\ \ \ \ \isacommand{proof}\isamarkupfalse%
{\isacharparenleft}{\kern0pt}cases\ {\isachardoublequoteopen}{\isasymexists}\ c{\isachardot}{\kern0pt}\ c\ {\isasymin}\isactrlsub c\ C\ {\isasymand}\ y\ {\isacharequal}{\kern0pt}\ left{\isacharunderscore}{\kern0pt}coproj\ C\ D\ {\isasymcirc}\isactrlsub c\ c{\isachardoublequoteclose}{\isacharparenright}{\kern0pt}\isanewline
\ \ \ \ \ \ \isacommand{assume}\isamarkupfalse%
\ {\isachardoublequoteopen}{\isasymexists}\ c{\isachardot}{\kern0pt}\ c\ {\isasymin}\isactrlsub c\ C\ {\isasymand}\ y\ {\isacharequal}{\kern0pt}\ left{\isacharunderscore}{\kern0pt}coproj\ C\ D\ {\isasymcirc}\isactrlsub c\ c{\isachardoublequoteclose}\isanewline
\ \ \ \ \ \ \isacommand{then}\isamarkupfalse%
\ \isacommand{obtain}\isamarkupfalse%
\ c\ \isakeyword{where}\ c{\isacharunderscore}{\kern0pt}def{\isacharcolon}{\kern0pt}\ {\isachardoublequoteopen}c\ {\isasymin}\isactrlsub c\ C\ {\isasymand}\ y\ {\isacharequal}{\kern0pt}\ left{\isacharunderscore}{\kern0pt}coproj\ C\ D\ {\isasymcirc}\isactrlsub c\ c{\isachardoublequoteclose}\isanewline
\ \ \ \ \ \ \ \ \isacommand{by}\isamarkupfalse%
\ blast\isanewline
\ \ \ \ \ \ \isacommand{then}\isamarkupfalse%
\ \isacommand{have}\isamarkupfalse%
\ {\isachardoublequoteopen}{\isasymexists}\ a{\isachardot}{\kern0pt}\ a\ {\isasymin}\isactrlsub c\ A\ {\isasymand}\ f\ {\isasymcirc}\isactrlsub c\ a\ {\isacharequal}{\kern0pt}\ c{\isachardoublequoteclose}\isanewline
\ \ \ \ \ \ \ \ \isacommand{using}\isamarkupfalse%
\ cfunc{\isacharunderscore}{\kern0pt}type{\isacharunderscore}{\kern0pt}def\ f{\isacharunderscore}{\kern0pt}def\ surj{\isacharunderscore}{\kern0pt}f\ surjective{\isacharunderscore}{\kern0pt}def\ \isacommand{by}\isamarkupfalse%
\ auto\isanewline
\ \ \ \ \ \ \isacommand{then}\isamarkupfalse%
\ \isacommand{obtain}\isamarkupfalse%
\ a\ \isakeyword{where}\ a{\isacharunderscore}{\kern0pt}def{\isacharcolon}{\kern0pt}\ {\isachardoublequoteopen}a\ {\isasymin}\isactrlsub c\ A\ {\isasymand}\ f\ {\isasymcirc}\isactrlsub c\ a\ {\isacharequal}{\kern0pt}\ c{\isachardoublequoteclose}\isanewline
\ \ \ \ \ \ \ \ \isacommand{by}\isamarkupfalse%
\ blast\isanewline
\ \ \ \ \ \ \isacommand{obtain}\isamarkupfalse%
\ x\ \isakeyword{where}\ x{\isacharunderscore}{\kern0pt}def{\isacharcolon}{\kern0pt}\ {\isachardoublequoteopen}x\ {\isacharequal}{\kern0pt}\ left{\isacharunderscore}{\kern0pt}coproj\ A\ B\ {\isasymcirc}\isactrlsub c\ a{\isachardoublequoteclose}\isanewline
\ \ \ \ \ \ \ \ \isacommand{by}\isamarkupfalse%
\ blast\isanewline
\ \ \ \ \ \ \isacommand{have}\isamarkupfalse%
\ x{\isacharunderscore}{\kern0pt}type{\isacharcolon}{\kern0pt}\ {\isachardoublequoteopen}x\ {\isasymin}\isactrlsub c\ A\ {\isasymCoprod}\ B{\isachardoublequoteclose}\isanewline
\ \ \ \ \ \ \ \ \isacommand{using}\isamarkupfalse%
\ a{\isacharunderscore}{\kern0pt}def\ comp{\isacharunderscore}{\kern0pt}type\ left{\isacharunderscore}{\kern0pt}proj{\isacharunderscore}{\kern0pt}type\ x{\isacharunderscore}{\kern0pt}def\ \isacommand{by}\isamarkupfalse%
\ blast\isanewline
\ \ \ \ \ \ \isacommand{have}\isamarkupfalse%
\ {\isachardoublequoteopen}{\isasymphi}\ {\isasymcirc}\isactrlsub c\ x\ {\isacharequal}{\kern0pt}\ y{\isachardoublequoteclose}\isanewline
\ \ \ \ \ \ \ \ \isacommand{using}\isamarkupfalse%
\ {\isasymphi}{\isacharunderscore}{\kern0pt}def\ {\isasymphi}{\isacharunderscore}{\kern0pt}type\ a{\isacharunderscore}{\kern0pt}def\ c{\isacharunderscore}{\kern0pt}def\ cfunc{\isacharunderscore}{\kern0pt}type{\isacharunderscore}{\kern0pt}def\ comp{\isacharunderscore}{\kern0pt}associative\ comp{\isacharunderscore}{\kern0pt}type\ f{\isacharunderscore}{\kern0pt}def\ g{\isacharunderscore}{\kern0pt}def\ left{\isacharunderscore}{\kern0pt}coproj{\isacharunderscore}{\kern0pt}cfunc{\isacharunderscore}{\kern0pt}coprod\ left{\isacharunderscore}{\kern0pt}proj{\isacharunderscore}{\kern0pt}type\ right{\isacharunderscore}{\kern0pt}proj{\isacharunderscore}{\kern0pt}type\ x{\isacharunderscore}{\kern0pt}def\ \isacommand{by}\isamarkupfalse%
\ {\isacharparenleft}{\kern0pt}smt\ {\isacharparenleft}{\kern0pt}verit{\isacharparenright}{\kern0pt}{\isacharparenright}{\kern0pt}\isanewline
\ \ \ \ \ \ \isacommand{then}\isamarkupfalse%
\ \isacommand{show}\isamarkupfalse%
\ {\isachardoublequoteopen}{\isasymexists}x{\isachardot}{\kern0pt}\ x\ {\isasymin}\isactrlsub c\ domain\ {\isasymphi}\ {\isasymand}\ {\isasymphi}\ {\isasymcirc}\isactrlsub c\ x\ {\isacharequal}{\kern0pt}\ y{\isachardoublequoteclose}\isanewline
\ \ \ \ \ \ \ \ \isacommand{using}\isamarkupfalse%
\ {\isasymphi}{\isacharunderscore}{\kern0pt}type\ cfunc{\isacharunderscore}{\kern0pt}type{\isacharunderscore}{\kern0pt}def\ x{\isacharunderscore}{\kern0pt}type\ \isacommand{by}\isamarkupfalse%
\ auto\isanewline
\ \ \ \ \isacommand{next}\isamarkupfalse%
\isanewline
\ \ \ \ \ \ \isacommand{assume}\isamarkupfalse%
\ {\isachardoublequoteopen}{\isasymnexists}c{\isachardot}{\kern0pt}\ c\ {\isasymin}\isactrlsub c\ C\ {\isasymand}\ y\ {\isacharequal}{\kern0pt}\ left{\isacharunderscore}{\kern0pt}coproj\ C\ D\ {\isasymcirc}\isactrlsub c\ c{\isachardoublequoteclose}\isanewline
\ \ \ \ \ \ \isacommand{then}\isamarkupfalse%
\ \isacommand{have}\isamarkupfalse%
\ y{\isacharunderscore}{\kern0pt}def{\isadigit{2}}{\isacharcolon}{\kern0pt}\ {\isachardoublequoteopen}{\isasymexists}\ d{\isachardot}{\kern0pt}\ d\ {\isasymin}\isactrlsub c\ D\ {\isasymand}\ y\ {\isacharequal}{\kern0pt}\ right{\isacharunderscore}{\kern0pt}coproj\ C\ D\ {\isasymcirc}\isactrlsub c\ d{\isachardoublequoteclose}\isanewline
\ \ \ \ \ \ \ \ \isacommand{using}\isamarkupfalse%
\ y{\isacharunderscore}{\kern0pt}form\ \isacommand{by}\isamarkupfalse%
\ blast\isanewline
\ \ \ \ \ \ \isacommand{then}\isamarkupfalse%
\ \isacommand{obtain}\isamarkupfalse%
\ d\ \isakeyword{where}\ d{\isacharunderscore}{\kern0pt}def{\isacharcolon}{\kern0pt}\ {\isachardoublequoteopen}d\ {\isasymin}\isactrlsub c\ D{\isachardoublequoteclose}\ {\isachardoublequoteopen}y\ {\isacharequal}{\kern0pt}\ right{\isacharunderscore}{\kern0pt}coproj\ C\ D\ {\isasymcirc}\isactrlsub c\ d{\isachardoublequoteclose}\isanewline
\ \ \ \ \ \ \ \ \isacommand{by}\isamarkupfalse%
\ blast\isanewline
\ \ \ \ \ \ \isacommand{then}\isamarkupfalse%
\ \isacommand{have}\isamarkupfalse%
\ {\isachardoublequoteopen}{\isasymexists}\ b{\isachardot}{\kern0pt}\ b\ {\isasymin}\isactrlsub c\ B\ {\isasymand}\ g\ {\isasymcirc}\isactrlsub c\ b\ {\isacharequal}{\kern0pt}\ d{\isachardoublequoteclose}\isanewline
\ \ \ \ \ \ \ \ \isacommand{using}\isamarkupfalse%
\ cfunc{\isacharunderscore}{\kern0pt}type{\isacharunderscore}{\kern0pt}def\ g{\isacharunderscore}{\kern0pt}def\ surj{\isacharunderscore}{\kern0pt}g\ surjective{\isacharunderscore}{\kern0pt}def\ \isacommand{by}\isamarkupfalse%
\ auto\isanewline
\ \ \ \ \ \ \isacommand{then}\isamarkupfalse%
\ \isacommand{obtain}\isamarkupfalse%
\ b\ \isakeyword{where}\ b{\isacharunderscore}{\kern0pt}def{\isacharcolon}{\kern0pt}\ {\isachardoublequoteopen}b\ {\isasymin}\isactrlsub c\ B{\isachardoublequoteclose}\ {\isachardoublequoteopen}g\ {\isasymcirc}\isactrlsub c\ b\ {\isacharequal}{\kern0pt}\ d{\isachardoublequoteclose}\isanewline
\ \ \ \ \ \ \ \ \isacommand{by}\isamarkupfalse%
\ blast\isanewline
\ \ \ \ \ \ \isacommand{obtain}\isamarkupfalse%
\ x\ \isakeyword{where}\ x{\isacharunderscore}{\kern0pt}def{\isacharcolon}{\kern0pt}\ {\isachardoublequoteopen}x\ {\isacharequal}{\kern0pt}\ right{\isacharunderscore}{\kern0pt}coproj\ A\ B\ {\isasymcirc}\isactrlsub c\ b{\isachardoublequoteclose}\isanewline
\ \ \ \ \ \ \ \ \isacommand{by}\isamarkupfalse%
\ blast\isanewline
\ \ \ \ \ \ \isacommand{have}\isamarkupfalse%
\ x{\isacharunderscore}{\kern0pt}type{\isacharcolon}{\kern0pt}\ {\isachardoublequoteopen}x\ {\isasymin}\isactrlsub c\ A\ {\isasymCoprod}\ B{\isachardoublequoteclose}\isanewline
\ \ \ \ \ \ \ \ \isacommand{using}\isamarkupfalse%
\ b{\isacharunderscore}{\kern0pt}def\ comp{\isacharunderscore}{\kern0pt}type\ right{\isacharunderscore}{\kern0pt}proj{\isacharunderscore}{\kern0pt}type\ x{\isacharunderscore}{\kern0pt}def\ \isacommand{by}\isamarkupfalse%
\ blast\isanewline
\ \ \ \ \ \ \isacommand{have}\isamarkupfalse%
\ {\isachardoublequoteopen}{\isasymphi}\ {\isasymcirc}\isactrlsub c\ x\ {\isacharequal}{\kern0pt}\ y{\isachardoublequoteclose}\isanewline
\ \ \ \ \ \ \ \ \isacommand{using}\isamarkupfalse%
\ {\isasymphi}{\isacharunderscore}{\kern0pt}def\ {\isasymphi}{\isacharunderscore}{\kern0pt}type\ b{\isacharunderscore}{\kern0pt}def\ cfunc{\isacharunderscore}{\kern0pt}type{\isacharunderscore}{\kern0pt}def\ comp{\isacharunderscore}{\kern0pt}associative\ comp{\isacharunderscore}{\kern0pt}type\ d{\isacharunderscore}{\kern0pt}def\ f{\isacharunderscore}{\kern0pt}def\ g{\isacharunderscore}{\kern0pt}def\ left{\isacharunderscore}{\kern0pt}proj{\isacharunderscore}{\kern0pt}type\ right{\isacharunderscore}{\kern0pt}coproj{\isacharunderscore}{\kern0pt}cfunc{\isacharunderscore}{\kern0pt}coprod\ right{\isacharunderscore}{\kern0pt}proj{\isacharunderscore}{\kern0pt}type\ x{\isacharunderscore}{\kern0pt}def\ \isacommand{by}\isamarkupfalse%
\ {\isacharparenleft}{\kern0pt}smt\ {\isacharparenleft}{\kern0pt}verit{\isacharparenright}{\kern0pt}{\isacharparenright}{\kern0pt}\isanewline
\ \ \ \ \ \ \isacommand{then}\isamarkupfalse%
\ \isacommand{show}\isamarkupfalse%
\ {\isachardoublequoteopen}{\isasymexists}x{\isachardot}{\kern0pt}\ x\ {\isasymin}\isactrlsub c\ domain\ {\isasymphi}\ {\isasymand}\ {\isasymphi}\ {\isasymcirc}\isactrlsub c\ x\ {\isacharequal}{\kern0pt}\ y{\isachardoublequoteclose}\isanewline
\ \ \ \ \ \ \ \ \isacommand{using}\isamarkupfalse%
\ {\isasymphi}{\isacharunderscore}{\kern0pt}type\ cfunc{\isacharunderscore}{\kern0pt}type{\isacharunderscore}{\kern0pt}def\ x{\isacharunderscore}{\kern0pt}type\ \isacommand{by}\isamarkupfalse%
\ auto\isanewline
\ \ \ \ \isacommand{qed}\isamarkupfalse%
\isanewline
\ \ \isacommand{qed}\isamarkupfalse%
\isanewline
\isanewline
\ \ \isacommand{have}\isamarkupfalse%
\ {\isachardoublequoteopen}injective{\isacharparenleft}{\kern0pt}{\isasymphi}{\isacharparenright}{\kern0pt}{\isachardoublequoteclose}\isanewline
\ \ \ \ \isacommand{unfolding}\isamarkupfalse%
\ injective{\isacharunderscore}{\kern0pt}def\isanewline
\ \ \isacommand{proof}\isamarkupfalse%
{\isacharparenleft}{\kern0pt}auto{\isacharparenright}{\kern0pt}\isanewline
\ \ \ \ \isacommand{fix}\isamarkupfalse%
\ x\ y\ \ \ \isanewline
\ \ \ \ \isacommand{assume}\isamarkupfalse%
\ x{\isacharunderscore}{\kern0pt}type{\isacharcolon}{\kern0pt}\ {\isachardoublequoteopen}x\ {\isasymin}\isactrlsub c\ domain\ {\isasymphi}{\isachardoublequoteclose}\isanewline
\ \ \ \ \isacommand{assume}\isamarkupfalse%
\ y{\isacharunderscore}{\kern0pt}type{\isacharcolon}{\kern0pt}\ {\isachardoublequoteopen}y\ {\isasymin}\isactrlsub c\ domain\ {\isasymphi}{\isachardoublequoteclose}\isanewline
\ \ \ \ \isacommand{assume}\isamarkupfalse%
\ equals{\isacharcolon}{\kern0pt}\ {\isachardoublequoteopen}{\isasymphi}\ {\isasymcirc}\isactrlsub c\ x\ {\isacharequal}{\kern0pt}\ {\isasymphi}\ {\isasymcirc}\isactrlsub c\ y{\isachardoublequoteclose}\isanewline
\ \ \ \ \isacommand{have}\isamarkupfalse%
\ x{\isacharunderscore}{\kern0pt}type{\isadigit{2}}{\isacharcolon}{\kern0pt}\ {\isachardoublequoteopen}x\ {\isasymin}\isactrlsub c\ A\ {\isasymCoprod}\ B{\isachardoublequoteclose}\isanewline
\ \ \ \ \ \ \isacommand{using}\isamarkupfalse%
\ {\isasymphi}{\isacharunderscore}{\kern0pt}type\ cfunc{\isacharunderscore}{\kern0pt}type{\isacharunderscore}{\kern0pt}def\ x{\isacharunderscore}{\kern0pt}type\ \isacommand{by}\isamarkupfalse%
\ auto\isanewline
\ \ \ \ \isacommand{have}\isamarkupfalse%
\ y{\isacharunderscore}{\kern0pt}type{\isadigit{2}}{\isacharcolon}{\kern0pt}\ {\isachardoublequoteopen}y\ {\isasymin}\isactrlsub c\ A\ {\isasymCoprod}\ B{\isachardoublequoteclose}\isanewline
\ \ \ \ \ \ \isacommand{using}\isamarkupfalse%
\ {\isasymphi}{\isacharunderscore}{\kern0pt}type\ cfunc{\isacharunderscore}{\kern0pt}type{\isacharunderscore}{\kern0pt}def\ y{\isacharunderscore}{\kern0pt}type\ \isacommand{by}\isamarkupfalse%
\ auto\isanewline
\isanewline
\ \ \ \ \isacommand{have}\isamarkupfalse%
\ phix{\isacharunderscore}{\kern0pt}type{\isacharcolon}{\kern0pt}\ {\isachardoublequoteopen}{\isasymphi}\ {\isasymcirc}\isactrlsub c\ x\ {\isasymin}\isactrlsub c\ C\ {\isasymCoprod}\ D{\isachardoublequoteclose}\isanewline
\ \ \ \ \ \ \isacommand{using}\isamarkupfalse%
\ {\isasymphi}{\isacharunderscore}{\kern0pt}type\ comp{\isacharunderscore}{\kern0pt}type\ x{\isacharunderscore}{\kern0pt}type{\isadigit{2}}\ \isacommand{by}\isamarkupfalse%
\ blast\isanewline
\ \ \ \ \isacommand{have}\isamarkupfalse%
\ phiy{\isacharunderscore}{\kern0pt}type{\isacharcolon}{\kern0pt}\ {\isachardoublequoteopen}{\isasymphi}\ {\isasymcirc}\isactrlsub c\ y\ {\isasymin}\isactrlsub c\ C\ {\isasymCoprod}\ D{\isachardoublequoteclose}\isanewline
\ \ \ \ \ \ \isacommand{using}\isamarkupfalse%
\ equals\ phix{\isacharunderscore}{\kern0pt}type\ \isacommand{by}\isamarkupfalse%
\ auto\isanewline
\isanewline
\ \ \ \ \isacommand{have}\isamarkupfalse%
\ x{\isacharunderscore}{\kern0pt}form{\isacharcolon}{\kern0pt}\ {\isachardoublequoteopen}{\isacharparenleft}{\kern0pt}{\isasymexists}\ a{\isachardot}{\kern0pt}\ a\ {\isasymin}\isactrlsub c\ A\ \ {\isasymand}\ x\ {\isacharequal}{\kern0pt}\ left{\isacharunderscore}{\kern0pt}coproj\ A\ B\ {\isasymcirc}\isactrlsub c\ a{\isacharparenright}{\kern0pt}\isanewline
\ \ \ \ \ \ {\isasymor}\ \ {\isacharparenleft}{\kern0pt}{\isasymexists}\ b{\isachardot}{\kern0pt}\ b\ {\isasymin}\isactrlsub c\ B\ {\isasymand}\ x\ {\isacharequal}{\kern0pt}\ right{\isacharunderscore}{\kern0pt}coproj\ A\ B\ {\isasymcirc}\isactrlsub c\ b{\isacharparenright}{\kern0pt}{\isachardoublequoteclose}\isanewline
\ \ \ \ \ \ \isacommand{using}\isamarkupfalse%
\ cfunc{\isacharunderscore}{\kern0pt}type{\isacharunderscore}{\kern0pt}def\ coprojs{\isacharunderscore}{\kern0pt}jointly{\isacharunderscore}{\kern0pt}surj\ x{\isacharunderscore}{\kern0pt}type\ x{\isacharunderscore}{\kern0pt}type{\isadigit{2}}\ y{\isacharunderscore}{\kern0pt}type\ \isacommand{by}\isamarkupfalse%
\ auto\isanewline
\ \ \ \ \isanewline
\ \ \ \ \isacommand{have}\isamarkupfalse%
\ y{\isacharunderscore}{\kern0pt}form{\isacharcolon}{\kern0pt}\ {\isachardoublequoteopen}{\isacharparenleft}{\kern0pt}{\isasymexists}\ a{\isachardot}{\kern0pt}\ a\ {\isasymin}\isactrlsub c\ A\ \ {\isasymand}\ y\ {\isacharequal}{\kern0pt}\ left{\isacharunderscore}{\kern0pt}coproj\ A\ B\ {\isasymcirc}\isactrlsub c\ a{\isacharparenright}{\kern0pt}\isanewline
\ \ \ \ \ \ {\isasymor}\ \ {\isacharparenleft}{\kern0pt}{\isasymexists}\ b{\isachardot}{\kern0pt}\ b\ {\isasymin}\isactrlsub c\ B\ {\isasymand}\ y\ {\isacharequal}{\kern0pt}\ right{\isacharunderscore}{\kern0pt}coproj\ A\ B\ {\isasymcirc}\isactrlsub c\ b{\isacharparenright}{\kern0pt}{\isachardoublequoteclose}\isanewline
\ \ \ \ \ \ \isacommand{using}\isamarkupfalse%
\ cfunc{\isacharunderscore}{\kern0pt}type{\isacharunderscore}{\kern0pt}def\ coprojs{\isacharunderscore}{\kern0pt}jointly{\isacharunderscore}{\kern0pt}surj\ x{\isacharunderscore}{\kern0pt}type\ x{\isacharunderscore}{\kern0pt}type{\isadigit{2}}\ y{\isacharunderscore}{\kern0pt}type\ \isacommand{by}\isamarkupfalse%
\ auto\isanewline
\isanewline
\ \ \ \ \isacommand{show}\isamarkupfalse%
\ {\isachardoublequoteopen}x{\isacharequal}{\kern0pt}y{\isachardoublequoteclose}\isanewline
\ \ \ \ \isacommand{proof}\isamarkupfalse%
{\isacharparenleft}{\kern0pt}cases\ {\isachardoublequoteopen}{\isasymexists}\ a{\isachardot}{\kern0pt}\ a\ {\isasymin}\isactrlsub c\ A\ \ {\isasymand}\ x\ {\isacharequal}{\kern0pt}\ left{\isacharunderscore}{\kern0pt}coproj\ A\ B\ {\isasymcirc}\isactrlsub c\ a{\isachardoublequoteclose}{\isacharparenright}{\kern0pt}\isanewline
\ \ \ \ \ \ \isacommand{assume}\isamarkupfalse%
\ {\isachardoublequoteopen}{\isasymexists}\ a{\isachardot}{\kern0pt}\ a\ {\isasymin}\isactrlsub c\ A\ \ {\isasymand}\ x\ {\isacharequal}{\kern0pt}\ left{\isacharunderscore}{\kern0pt}coproj\ A\ B\ {\isasymcirc}\isactrlsub c\ a{\isachardoublequoteclose}\isanewline
\ \ \ \ \ \ \isacommand{then}\isamarkupfalse%
\ \isacommand{obtain}\isamarkupfalse%
\ a\ \isakeyword{where}\ a{\isacharunderscore}{\kern0pt}def{\isacharcolon}{\kern0pt}\ {\isachardoublequoteopen}a\ {\isasymin}\isactrlsub c\ A{\isachardoublequoteclose}\ {\isachardoublequoteopen}x\ {\isacharequal}{\kern0pt}\ left{\isacharunderscore}{\kern0pt}coproj\ A\ B\ {\isasymcirc}\isactrlsub c\ a{\isachardoublequoteclose}\isanewline
\ \ \ \ \ \ \ \ \isacommand{by}\isamarkupfalse%
\ blast\isanewline
\ \ \ \ \ \ \isacommand{show}\isamarkupfalse%
\ {\isachardoublequoteopen}x\ {\isacharequal}{\kern0pt}\ y{\isachardoublequoteclose}\isanewline
\ \ \ \ \ \ \isacommand{proof}\isamarkupfalse%
{\isacharparenleft}{\kern0pt}cases\ {\isachardoublequoteopen}{\isasymexists}\ a{\isachardot}{\kern0pt}\ a\ {\isasymin}\isactrlsub c\ A\ \ {\isasymand}\ y\ {\isacharequal}{\kern0pt}\ left{\isacharunderscore}{\kern0pt}coproj\ A\ B\ {\isasymcirc}\isactrlsub c\ a{\isachardoublequoteclose}{\isacharparenright}{\kern0pt}\isanewline
\ \ \ \ \ \ \ \ \isacommand{assume}\isamarkupfalse%
\ {\isachardoublequoteopen}{\isasymexists}\ a{\isachardot}{\kern0pt}\ a\ {\isasymin}\isactrlsub c\ A\ \ {\isasymand}\ y\ {\isacharequal}{\kern0pt}\ left{\isacharunderscore}{\kern0pt}coproj\ A\ B\ {\isasymcirc}\isactrlsub c\ a{\isachardoublequoteclose}\isanewline
\ \ \ \ \ \ \ \ \isacommand{then}\isamarkupfalse%
\ \isacommand{obtain}\isamarkupfalse%
\ a{\isacharprime}{\kern0pt}\ \isakeyword{where}\ a{\isacharprime}{\kern0pt}{\isacharunderscore}{\kern0pt}def{\isacharcolon}{\kern0pt}\ {\isachardoublequoteopen}a{\isacharprime}{\kern0pt}\ {\isasymin}\isactrlsub c\ A{\isachardoublequoteclose}\ {\isachardoublequoteopen}y\ {\isacharequal}{\kern0pt}\ left{\isacharunderscore}{\kern0pt}coproj\ A\ B\ {\isasymcirc}\isactrlsub c\ a{\isacharprime}{\kern0pt}{\isachardoublequoteclose}\isanewline
\ \ \ \ \ \ \ \ \ \ \isacommand{by}\isamarkupfalse%
\ blast\isanewline
\ \ \ \ \ \ \ \ \isacommand{then}\isamarkupfalse%
\ \isacommand{have}\isamarkupfalse%
\ {\isachardoublequoteopen}a\ {\isacharequal}{\kern0pt}\ a{\isacharprime}{\kern0pt}{\isachardoublequoteclose}\isanewline
\ \ \ \ \ \ \ \ \isacommand{proof}\isamarkupfalse%
\ {\isacharminus}{\kern0pt}\ \isanewline
\ \ \ \ \ \ \ \ \ \ \isacommand{have}\isamarkupfalse%
\ {\isachardoublequoteopen}{\isacharparenleft}{\kern0pt}left{\isacharunderscore}{\kern0pt}coproj\ C\ D\ {\isasymcirc}\isactrlsub c\ f{\isacharparenright}{\kern0pt}\ {\isasymcirc}\isactrlsub c\ a\ {\isacharequal}{\kern0pt}\ {\isasymphi}\ {\isasymcirc}\isactrlsub c\ x{\isachardoublequoteclose}\isanewline
\ \ \ \ \ \ \ \ \ \ \ \ \isacommand{using}\isamarkupfalse%
\ {\isasymphi}{\isacharunderscore}{\kern0pt}def\ a{\isacharunderscore}{\kern0pt}def\ cfunc{\isacharunderscore}{\kern0pt}type{\isacharunderscore}{\kern0pt}def\ comp{\isacharunderscore}{\kern0pt}associative\ comp{\isacharunderscore}{\kern0pt}type\ f{\isacharunderscore}{\kern0pt}def\ g{\isacharunderscore}{\kern0pt}def\ left{\isacharunderscore}{\kern0pt}coproj{\isacharunderscore}{\kern0pt}cfunc{\isacharunderscore}{\kern0pt}coprod\ left{\isacharunderscore}{\kern0pt}proj{\isacharunderscore}{\kern0pt}type\ right{\isacharunderscore}{\kern0pt}proj{\isacharunderscore}{\kern0pt}type\ x{\isacharunderscore}{\kern0pt}type\ \isacommand{by}\isamarkupfalse%
\ {\isacharparenleft}{\kern0pt}smt\ {\isacharparenleft}{\kern0pt}verit{\isacharparenright}{\kern0pt}{\isacharparenright}{\kern0pt}\isanewline
\ \ \ \ \ \ \ \ \ \ \isacommand{also}\isamarkupfalse%
\ \isacommand{have}\isamarkupfalse%
\ {\isachardoublequoteopen}{\isachardot}{\kern0pt}{\isachardot}{\kern0pt}{\isachardot}{\kern0pt}\ {\isacharequal}{\kern0pt}\ {\isasymphi}\ {\isasymcirc}\isactrlsub c\ y{\isachardoublequoteclose}\isanewline
\ \ \ \ \ \ \ \ \ \ \ \ \isacommand{by}\isamarkupfalse%
\ {\isacharparenleft}{\kern0pt}meson\ equals{\isacharparenright}{\kern0pt}\isanewline
\ \ \ \ \ \ \ \ \ \ \isacommand{also}\isamarkupfalse%
\ \isacommand{have}\isamarkupfalse%
\ {\isachardoublequoteopen}{\isachardot}{\kern0pt}{\isachardot}{\kern0pt}{\isachardot}{\kern0pt}\ {\isacharequal}{\kern0pt}\ {\isacharparenleft}{\kern0pt}{\isasymphi}\ {\isasymcirc}\isactrlsub c\ left{\isacharunderscore}{\kern0pt}coproj\ A\ B{\isacharparenright}{\kern0pt}\ {\isasymcirc}\isactrlsub c\ a{\isacharprime}{\kern0pt}{\isachardoublequoteclose}\isanewline
\ \ \ \ \ \ \ \ \ \ \ \ \isacommand{using}\isamarkupfalse%
\ {\isasymphi}{\isacharunderscore}{\kern0pt}type\ a{\isacharprime}{\kern0pt}{\isacharunderscore}{\kern0pt}def\ comp{\isacharunderscore}{\kern0pt}associative{\isadigit{2}}\ \isacommand{by}\isamarkupfalse%
\ {\isacharparenleft}{\kern0pt}typecheck{\isacharunderscore}{\kern0pt}cfuncs{\isacharcomma}{\kern0pt}\ blast{\isacharparenright}{\kern0pt}\isanewline
\ \ \ \ \ \ \ \ \ \ \isacommand{also}\isamarkupfalse%
\ \isacommand{have}\isamarkupfalse%
\ {\isachardoublequoteopen}{\isachardot}{\kern0pt}{\isachardot}{\kern0pt}{\isachardot}{\kern0pt}\ {\isacharequal}{\kern0pt}\ {\isacharparenleft}{\kern0pt}left{\isacharunderscore}{\kern0pt}coproj\ C\ D\ {\isasymcirc}\isactrlsub c\ f{\isacharparenright}{\kern0pt}\ {\isasymcirc}\isactrlsub c\ a{\isacharprime}{\kern0pt}{\isachardoublequoteclose}\isanewline
\ \ \ \ \ \ \ \ \ \ \ \ \isacommand{unfolding}\isamarkupfalse%
\ {\isasymphi}{\isacharunderscore}{\kern0pt}def\ \isacommand{using}\isamarkupfalse%
\ f{\isacharunderscore}{\kern0pt}def\ g{\isacharunderscore}{\kern0pt}def\ a{\isacharprime}{\kern0pt}{\isacharunderscore}{\kern0pt}def\ left{\isacharunderscore}{\kern0pt}coproj{\isacharunderscore}{\kern0pt}cfunc{\isacharunderscore}{\kern0pt}coprod\ \isacommand{by}\isamarkupfalse%
\ {\isacharparenleft}{\kern0pt}typecheck{\isacharunderscore}{\kern0pt}cfuncs{\isacharcomma}{\kern0pt}\ auto{\isacharparenright}{\kern0pt}\isanewline
\ \ \ \ \ \ \ \ \ \ \isacommand{then}\isamarkupfalse%
\ \isacommand{show}\isamarkupfalse%
\ {\isachardoublequoteopen}a\ {\isacharequal}{\kern0pt}\ a{\isacharprime}{\kern0pt}{\isachardoublequoteclose}\isanewline
\ \ \ \ \ \ \ \ \ \ \ \ \isacommand{by}\isamarkupfalse%
\ {\isacharparenleft}{\kern0pt}smt\ a{\isacharprime}{\kern0pt}{\isacharunderscore}{\kern0pt}def\ a{\isacharunderscore}{\kern0pt}def\ calculation\ cfunc{\isacharunderscore}{\kern0pt}type{\isacharunderscore}{\kern0pt}def\ coproj{\isacharunderscore}{\kern0pt}f{\isacharunderscore}{\kern0pt}inject\ domain{\isacharunderscore}{\kern0pt}comp\ f{\isacharunderscore}{\kern0pt}def\ injective{\isacharunderscore}{\kern0pt}def\ left{\isacharunderscore}{\kern0pt}proj{\isacharunderscore}{\kern0pt}type{\isacharparenright}{\kern0pt}\isanewline
\ \ \ \ \ \ \ \ \isacommand{qed}\isamarkupfalse%
\isanewline
\ \ \ \ \ \ \ \ \isacommand{then}\isamarkupfalse%
\ \isacommand{show}\isamarkupfalse%
\ {\isachardoublequoteopen}x{\isacharequal}{\kern0pt}y{\isachardoublequoteclose}\isanewline
\ \ \ \ \ \ \ \ \ \ \isacommand{by}\isamarkupfalse%
\ {\isacharparenleft}{\kern0pt}simp\ add{\isacharcolon}{\kern0pt}\ \ a{\isacharprime}{\kern0pt}{\isacharunderscore}{\kern0pt}def{\isacharparenleft}{\kern0pt}{\isadigit{2}}{\isacharparenright}{\kern0pt}\ a{\isacharunderscore}{\kern0pt}def{\isacharparenleft}{\kern0pt}{\isadigit{2}}{\isacharparenright}{\kern0pt}{\isacharparenright}{\kern0pt}\isanewline
\ \ \ \ \ \ \isacommand{next}\isamarkupfalse%
\isanewline
\ \ \ \ \ \ \ \ \isacommand{assume}\isamarkupfalse%
\ {\isachardoublequoteopen}{\isasymnexists}a{\isachardot}{\kern0pt}\ a\ {\isasymin}\isactrlsub c\ A\ {\isasymand}\ y\ {\isacharequal}{\kern0pt}\ left{\isacharunderscore}{\kern0pt}coproj\ A\ B\ {\isasymcirc}\isactrlsub c\ a{\isachardoublequoteclose}\isanewline
\ \ \ \ \ \ \ \ \isacommand{then}\isamarkupfalse%
\ \isacommand{have}\isamarkupfalse%
\ {\isachardoublequoteopen}{\isasymexists}\ b{\isachardot}{\kern0pt}\ b\ {\isasymin}\isactrlsub c\ B\ {\isasymand}\ y\ {\isacharequal}{\kern0pt}\ right{\isacharunderscore}{\kern0pt}coproj\ A\ B\ {\isasymcirc}\isactrlsub c\ b{\isachardoublequoteclose}\isanewline
\ \ \ \ \ \ \ \ \ \ \isacommand{using}\isamarkupfalse%
\ y{\isacharunderscore}{\kern0pt}form\ \isacommand{by}\isamarkupfalse%
\ blast\isanewline
\ \ \ \ \ \ \ \ \isacommand{then}\isamarkupfalse%
\ \isacommand{obtain}\isamarkupfalse%
\ b{\isacharprime}{\kern0pt}\ \isakeyword{where}\ b{\isacharprime}{\kern0pt}{\isacharunderscore}{\kern0pt}def{\isacharcolon}{\kern0pt}\ {\isachardoublequoteopen}b{\isacharprime}{\kern0pt}\ {\isasymin}\isactrlsub c\ B{\isachardoublequoteclose}\ {\isachardoublequoteopen}y\ {\isacharequal}{\kern0pt}\ right{\isacharunderscore}{\kern0pt}coproj\ A\ B\ {\isasymcirc}\isactrlsub c\ b{\isacharprime}{\kern0pt}{\isachardoublequoteclose}\isanewline
\ \ \ \ \ \ \ \ \ \ \isacommand{by}\isamarkupfalse%
\ blast\isanewline
\ \ \ \ \ \ \ \ \isacommand{show}\isamarkupfalse%
\ {\isachardoublequoteopen}x\ {\isacharequal}{\kern0pt}\ y{\isachardoublequoteclose}\isanewline
\ \ \ \ \ \ \ \ \isacommand{proof}\isamarkupfalse%
\ {\isacharminus}{\kern0pt}\ \isanewline
\ \ \ \ \ \ \ \ \ \ \isacommand{have}\isamarkupfalse%
\ {\isachardoublequoteopen}left{\isacharunderscore}{\kern0pt}coproj\ C\ D\ {\isasymcirc}\isactrlsub c\ {\isacharparenleft}{\kern0pt}f\ {\isasymcirc}\isactrlsub c\ a{\isacharparenright}{\kern0pt}\ {\isacharequal}{\kern0pt}\ {\isacharparenleft}{\kern0pt}left{\isacharunderscore}{\kern0pt}coproj\ C\ D\ {\isasymcirc}\isactrlsub c\ f{\isacharparenright}{\kern0pt}\ {\isasymcirc}\isactrlsub c\ a{\isachardoublequoteclose}\isanewline
\ \ \ \ \ \ \ \ \ \ \ \ \isacommand{using}\isamarkupfalse%
\ a{\isacharunderscore}{\kern0pt}def\ cfunc{\isacharunderscore}{\kern0pt}type{\isacharunderscore}{\kern0pt}def\ comp{\isacharunderscore}{\kern0pt}associative\ f{\isacharunderscore}{\kern0pt}def\ left{\isacharunderscore}{\kern0pt}proj{\isacharunderscore}{\kern0pt}type\ \isacommand{by}\isamarkupfalse%
\ auto\isanewline
\ \ \ \ \ \ \ \ \ \ \isacommand{also}\isamarkupfalse%
\ \isacommand{have}\isamarkupfalse%
\ {\isachardoublequoteopen}{\isachardot}{\kern0pt}{\isachardot}{\kern0pt}{\isachardot}{\kern0pt}\ \ {\isacharequal}{\kern0pt}\ {\isasymphi}\ {\isasymcirc}\isactrlsub c\ x{\isachardoublequoteclose}\isanewline
\ \ \ \ \ \ \ \ \ \ \ \ \ \ \isacommand{using}\isamarkupfalse%
\ {\isasymphi}{\isacharunderscore}{\kern0pt}def\ a{\isacharunderscore}{\kern0pt}def\ cfunc{\isacharunderscore}{\kern0pt}type{\isacharunderscore}{\kern0pt}def\ comp{\isacharunderscore}{\kern0pt}associative\ comp{\isacharunderscore}{\kern0pt}type\ f{\isacharunderscore}{\kern0pt}def\ g{\isacharunderscore}{\kern0pt}def\ left{\isacharunderscore}{\kern0pt}coproj{\isacharunderscore}{\kern0pt}cfunc{\isacharunderscore}{\kern0pt}coprod\ left{\isacharunderscore}{\kern0pt}proj{\isacharunderscore}{\kern0pt}type\ right{\isacharunderscore}{\kern0pt}proj{\isacharunderscore}{\kern0pt}type\ x{\isacharunderscore}{\kern0pt}type\ \isacommand{by}\isamarkupfalse%
\ {\isacharparenleft}{\kern0pt}smt\ {\isacharparenleft}{\kern0pt}verit{\isacharparenright}{\kern0pt}{\isacharparenright}{\kern0pt}\isanewline
\ \ \ \ \ \ \ \ \ \ \isacommand{also}\isamarkupfalse%
\ \isacommand{have}\isamarkupfalse%
\ {\isachardoublequoteopen}{\isachardot}{\kern0pt}{\isachardot}{\kern0pt}{\isachardot}{\kern0pt}\ {\isacharequal}{\kern0pt}\ {\isasymphi}\ {\isasymcirc}\isactrlsub c\ y{\isachardoublequoteclose}\isanewline
\ \ \ \ \ \ \ \ \ \ \ \ \isacommand{by}\isamarkupfalse%
\ {\isacharparenleft}{\kern0pt}meson\ equals{\isacharparenright}{\kern0pt}\isanewline
\ \ \ \ \ \ \ \ \ \ \isacommand{also}\isamarkupfalse%
\ \isacommand{have}\isamarkupfalse%
\ {\isachardoublequoteopen}{\isachardot}{\kern0pt}{\isachardot}{\kern0pt}{\isachardot}{\kern0pt}\ {\isacharequal}{\kern0pt}\ {\isacharparenleft}{\kern0pt}{\isasymphi}\ {\isasymcirc}\isactrlsub c\ right{\isacharunderscore}{\kern0pt}coproj\ A\ B{\isacharparenright}{\kern0pt}\ {\isasymcirc}\isactrlsub c\ b{\isacharprime}{\kern0pt}{\isachardoublequoteclose}\isanewline
\ \ \ \ \ \ \ \ \ \ \ \ \isacommand{using}\isamarkupfalse%
\ {\isasymphi}{\isacharunderscore}{\kern0pt}type\ b{\isacharprime}{\kern0pt}{\isacharunderscore}{\kern0pt}def\ comp{\isacharunderscore}{\kern0pt}associative{\isadigit{2}}\ \isacommand{by}\isamarkupfalse%
\ {\isacharparenleft}{\kern0pt}typecheck{\isacharunderscore}{\kern0pt}cfuncs{\isacharcomma}{\kern0pt}\ blast{\isacharparenright}{\kern0pt}\isanewline
\ \ \ \ \ \ \ \ \ \ \isacommand{also}\isamarkupfalse%
\ \isacommand{have}\isamarkupfalse%
\ {\isachardoublequoteopen}{\isachardot}{\kern0pt}{\isachardot}{\kern0pt}{\isachardot}{\kern0pt}\ {\isacharequal}{\kern0pt}\ {\isacharparenleft}{\kern0pt}right{\isacharunderscore}{\kern0pt}coproj\ C\ D\ {\isasymcirc}\isactrlsub c\ g{\isacharparenright}{\kern0pt}\ {\isasymcirc}\isactrlsub c\ b{\isacharprime}{\kern0pt}\ {\isachardoublequoteclose}\isanewline
\ \ \ \ \ \ \ \ \ \ \ \ \isacommand{unfolding}\isamarkupfalse%
\ {\isasymphi}{\isacharunderscore}{\kern0pt}def\ \isacommand{using}\isamarkupfalse%
\ f{\isacharunderscore}{\kern0pt}def\ g{\isacharunderscore}{\kern0pt}def\ b{\isacharprime}{\kern0pt}{\isacharunderscore}{\kern0pt}def\ right{\isacharunderscore}{\kern0pt}coproj{\isacharunderscore}{\kern0pt}cfunc{\isacharunderscore}{\kern0pt}coprod\ \isacommand{by}\isamarkupfalse%
\ {\isacharparenleft}{\kern0pt}typecheck{\isacharunderscore}{\kern0pt}cfuncs{\isacharcomma}{\kern0pt}\ auto{\isacharparenright}{\kern0pt}\isanewline
\ \ \ \ \ \ \ \ \ \ \isacommand{also}\isamarkupfalse%
\ \isacommand{have}\isamarkupfalse%
\ {\isachardoublequoteopen}{\isachardot}{\kern0pt}{\isachardot}{\kern0pt}{\isachardot}{\kern0pt}\ {\isacharequal}{\kern0pt}\ right{\isacharunderscore}{\kern0pt}coproj\ C\ D\ {\isasymcirc}\isactrlsub c\ {\isacharparenleft}{\kern0pt}g\ {\isasymcirc}\isactrlsub c\ b{\isacharprime}{\kern0pt}{\isacharparenright}{\kern0pt}{\isachardoublequoteclose}\isanewline
\ \ \ \ \ \ \ \ \ \ \ \ \ \ \isacommand{using}\isamarkupfalse%
\ g{\isacharunderscore}{\kern0pt}def\ b{\isacharprime}{\kern0pt}{\isacharunderscore}{\kern0pt}def\ \isacommand{by}\isamarkupfalse%
\ {\isacharparenleft}{\kern0pt}typecheck{\isacharunderscore}{\kern0pt}cfuncs{\isacharcomma}{\kern0pt}\ simp\ add{\isacharcolon}{\kern0pt}\ comp{\isacharunderscore}{\kern0pt}associative{\isadigit{2}}{\isacharparenright}{\kern0pt}\isanewline
\ \ \ \ \ \ \ \ \ \ \isacommand{then}\isamarkupfalse%
\ \isacommand{show}\isamarkupfalse%
\ {\isachardoublequoteopen}x\ {\isacharequal}{\kern0pt}\ y{\isachardoublequoteclose}\isanewline
\ \ \ \ \ \ \ \ \ \ \ \ \ \isacommand{using}\isamarkupfalse%
\ \ a{\isacharunderscore}{\kern0pt}def{\isacharparenleft}{\kern0pt}{\isadigit{1}}{\isacharparenright}{\kern0pt}\ b{\isacharprime}{\kern0pt}{\isacharunderscore}{\kern0pt}def{\isacharparenleft}{\kern0pt}{\isadigit{1}}{\isacharparenright}{\kern0pt}\ calculation\ comp{\isacharunderscore}{\kern0pt}type\ coproducts{\isacharunderscore}{\kern0pt}disjoint\ f{\isacharunderscore}{\kern0pt}def{\isacharparenleft}{\kern0pt}{\isadigit{1}}{\isacharparenright}{\kern0pt}\ g{\isacharunderscore}{\kern0pt}def{\isacharparenleft}{\kern0pt}{\isadigit{1}}{\isacharparenright}{\kern0pt}\ \isacommand{by}\isamarkupfalse%
\ auto\isanewline
\ \ \ \ \ \ \ \ \ \isacommand{qed}\isamarkupfalse%
\isanewline
\ \ \ \ \ \ \ \isacommand{qed}\isamarkupfalse%
\isanewline
\ \ \ \ \ \isacommand{next}\isamarkupfalse%
\isanewline
\ \ \ \ \ \ \ \ \ \isacommand{assume}\isamarkupfalse%
\ {\isachardoublequoteopen}{\isasymnexists}a{\isachardot}{\kern0pt}\ a\ {\isasymin}\isactrlsub c\ A\ {\isasymand}\ x\ {\isacharequal}{\kern0pt}\ left{\isacharunderscore}{\kern0pt}coproj\ A\ B\ {\isasymcirc}\isactrlsub c\ a{\isachardoublequoteclose}\isanewline
\ \ \ \ \ \ \ \ \ \isacommand{then}\isamarkupfalse%
\ \isacommand{have}\isamarkupfalse%
\ {\isachardoublequoteopen}{\isasymexists}\ b{\isachardot}{\kern0pt}\ b\ {\isasymin}\isactrlsub c\ B\ {\isasymand}\ x\ {\isacharequal}{\kern0pt}\ right{\isacharunderscore}{\kern0pt}coproj\ A\ B\ {\isasymcirc}\isactrlsub c\ b{\isachardoublequoteclose}\isanewline
\ \ \ \ \ \ \ \ \ \ \ \isacommand{using}\isamarkupfalse%
\ x{\isacharunderscore}{\kern0pt}form\ \isacommand{by}\isamarkupfalse%
\ blast\isanewline
\ \ \ \ \ \ \ \ \ \isacommand{then}\isamarkupfalse%
\ \isacommand{obtain}\isamarkupfalse%
\ b\ \isakeyword{where}\ b{\isacharunderscore}{\kern0pt}def{\isacharcolon}{\kern0pt}\ {\isachardoublequoteopen}b\ {\isasymin}\isactrlsub c\ B\ {\isasymand}\ x\ {\isacharequal}{\kern0pt}\ right{\isacharunderscore}{\kern0pt}coproj\ A\ B\ {\isasymcirc}\isactrlsub c\ b{\isachardoublequoteclose}\isanewline
\ \ \ \ \ \ \ \ \ \ \ \isacommand{by}\isamarkupfalse%
\ blast\isanewline
\ \ \ \ \ \ \ \ \ \ \ \ \ \ \isacommand{show}\isamarkupfalse%
\ {\isachardoublequoteopen}x\ {\isacharequal}{\kern0pt}\ y{\isachardoublequoteclose}\isanewline
\ \ \ \ \ \ \ \ \ \ \ \ \ \ \isacommand{proof}\isamarkupfalse%
{\isacharparenleft}{\kern0pt}cases\ {\isachardoublequoteopen}{\isasymexists}\ a{\isachardot}{\kern0pt}\ a\ {\isasymin}\isactrlsub c\ A\ \ {\isasymand}\ y\ {\isacharequal}{\kern0pt}\ left{\isacharunderscore}{\kern0pt}coproj\ A\ B\ {\isasymcirc}\isactrlsub c\ a{\isachardoublequoteclose}{\isacharparenright}{\kern0pt}\isanewline
\ \ \ \ \ \ \ \ \ \ \ \ \ \ \ \ \ \isacommand{assume}\isamarkupfalse%
\ {\isachardoublequoteopen}{\isasymexists}\ a{\isachardot}{\kern0pt}\ a\ {\isasymin}\isactrlsub c\ A\ \ {\isasymand}\ y\ {\isacharequal}{\kern0pt}\ left{\isacharunderscore}{\kern0pt}coproj\ A\ B\ {\isasymcirc}\isactrlsub c\ a{\isachardoublequoteclose}\isanewline
\ \ \ \ \ \ \ \ \ \ \ \ \ \ \ \ \ \isacommand{then}\isamarkupfalse%
\ \isacommand{obtain}\isamarkupfalse%
\ a{\isacharprime}{\kern0pt}\ \isakeyword{where}\ a{\isacharprime}{\kern0pt}{\isacharunderscore}{\kern0pt}def{\isacharcolon}{\kern0pt}\ {\isachardoublequoteopen}a{\isacharprime}{\kern0pt}\ {\isasymin}\isactrlsub c\ A{\isachardoublequoteclose}\ {\isachardoublequoteopen}y\ {\isacharequal}{\kern0pt}\ left{\isacharunderscore}{\kern0pt}coproj\ A\ B\ {\isasymcirc}\isactrlsub c\ a{\isacharprime}{\kern0pt}{\isachardoublequoteclose}\isanewline
\ \ \ \ \ \ \ \ \ \ \ \ \ \ \ \ \ \ \ \isacommand{by}\isamarkupfalse%
\ blast\isanewline
\ \ \ \ \ \ \ \ \ \ \ \ \ \ \ \ \ \isacommand{show}\isamarkupfalse%
\ {\isachardoublequoteopen}x\ {\isacharequal}{\kern0pt}\ y{\isachardoublequoteclose}\isanewline
\ \ \ \ \ \ \ \ \ \ \ \ \ \ \ \ \ \isacommand{proof}\isamarkupfalse%
\ {\isacharminus}{\kern0pt}\ \isanewline
\ \ \ \ \ \ \ \ \ \ \ \ \ \ \ \ \ \ \isacommand{have}\isamarkupfalse%
\ {\isachardoublequoteopen}right{\isacharunderscore}{\kern0pt}coproj\ C\ D\ {\isasymcirc}\isactrlsub c\ {\isacharparenleft}{\kern0pt}g\ {\isasymcirc}\isactrlsub c\ b{\isacharparenright}{\kern0pt}\ {\isacharequal}{\kern0pt}\ {\isacharparenleft}{\kern0pt}right{\isacharunderscore}{\kern0pt}coproj\ C\ D\ {\isasymcirc}\isactrlsub c\ g{\isacharparenright}{\kern0pt}\ {\isasymcirc}\isactrlsub c\ b{\isachardoublequoteclose}\isanewline
\ \ \ \ \ \ \ \ \ \ \ \ \ \ \ \ \ \ \ \ \isacommand{using}\isamarkupfalse%
\ b{\isacharunderscore}{\kern0pt}def\ cfunc{\isacharunderscore}{\kern0pt}type{\isacharunderscore}{\kern0pt}def\ comp{\isacharunderscore}{\kern0pt}associative\ g{\isacharunderscore}{\kern0pt}def\ right{\isacharunderscore}{\kern0pt}proj{\isacharunderscore}{\kern0pt}type\ \isacommand{by}\isamarkupfalse%
\ auto\isanewline
\ \ \ \ \ \ \ \ \ \ \ \ \ \ \ \ \ \ \isacommand{also}\isamarkupfalse%
\ \isacommand{have}\isamarkupfalse%
\ {\isachardoublequoteopen}{\isachardot}{\kern0pt}{\isachardot}{\kern0pt}{\isachardot}{\kern0pt}\ \ {\isacharequal}{\kern0pt}\ {\isasymphi}\ {\isasymcirc}\isactrlsub c\ x{\isachardoublequoteclose}\isanewline
\ \ \ \ \ \ \ \ \ \ \ \ \ \ \ \ \ \ \ \ \isacommand{by}\isamarkupfalse%
\ {\isacharparenleft}{\kern0pt}smt\ {\isasymphi}{\isacharunderscore}{\kern0pt}def\ {\isasymphi}{\isacharunderscore}{\kern0pt}type\ b{\isacharunderscore}{\kern0pt}def\ comp{\isacharunderscore}{\kern0pt}associative{\isadigit{2}}\ comp{\isacharunderscore}{\kern0pt}type\ f{\isacharunderscore}{\kern0pt}def{\isacharparenleft}{\kern0pt}{\isadigit{1}}{\isacharparenright}{\kern0pt}\ g{\isacharunderscore}{\kern0pt}def{\isacharparenleft}{\kern0pt}{\isadigit{1}}{\isacharparenright}{\kern0pt}\ left{\isacharunderscore}{\kern0pt}proj{\isacharunderscore}{\kern0pt}type\ right{\isacharunderscore}{\kern0pt}coproj{\isacharunderscore}{\kern0pt}cfunc{\isacharunderscore}{\kern0pt}coprod\ right{\isacharunderscore}{\kern0pt}proj{\isacharunderscore}{\kern0pt}type{\isacharparenright}{\kern0pt}\isanewline
\ \ \ \ \ \ \ \ \ \ \ \ \ \ \ \ \ \ \isacommand{also}\isamarkupfalse%
\ \isacommand{have}\isamarkupfalse%
\ {\isachardoublequoteopen}{\isachardot}{\kern0pt}{\isachardot}{\kern0pt}{\isachardot}{\kern0pt}\ {\isacharequal}{\kern0pt}\ {\isasymphi}\ {\isasymcirc}\isactrlsub c\ y{\isachardoublequoteclose}\isanewline
\ \ \ \ \ \ \ \ \ \ \ \ \ \ \ \ \ \ \ \ \isacommand{by}\isamarkupfalse%
\ {\isacharparenleft}{\kern0pt}meson\ equals{\isacharparenright}{\kern0pt}\isanewline
\ \ \ \ \ \ \ \ \ \ \ \ \ \ \ \ \ \ \isacommand{also}\isamarkupfalse%
\ \isacommand{have}\isamarkupfalse%
\ {\isachardoublequoteopen}{\isachardot}{\kern0pt}{\isachardot}{\kern0pt}{\isachardot}{\kern0pt}\ {\isacharequal}{\kern0pt}\ {\isacharparenleft}{\kern0pt}{\isasymphi}\ {\isasymcirc}\isactrlsub c\ left{\isacharunderscore}{\kern0pt}coproj\ A\ B{\isacharparenright}{\kern0pt}\ {\isasymcirc}\isactrlsub c\ a{\isacharprime}{\kern0pt}{\isachardoublequoteclose}\isanewline
\ \ \ \ \ \ \ \ \ \ \ \ \ \ \ \ \ \ \ \ \isacommand{using}\isamarkupfalse%
\ {\isasymphi}{\isacharunderscore}{\kern0pt}type\ a{\isacharprime}{\kern0pt}{\isacharunderscore}{\kern0pt}def\ comp{\isacharunderscore}{\kern0pt}associative{\isadigit{2}}\ \isacommand{by}\isamarkupfalse%
\ {\isacharparenleft}{\kern0pt}typecheck{\isacharunderscore}{\kern0pt}cfuncs{\isacharcomma}{\kern0pt}\ blast{\isacharparenright}{\kern0pt}\isanewline
\ \ \ \ \ \ \ \ \ \ \ \ \ \ \ \ \ \ \isacommand{also}\isamarkupfalse%
\ \isacommand{have}\isamarkupfalse%
\ {\isachardoublequoteopen}{\isachardot}{\kern0pt}{\isachardot}{\kern0pt}{\isachardot}{\kern0pt}\ {\isacharequal}{\kern0pt}\ {\isacharparenleft}{\kern0pt}left{\isacharunderscore}{\kern0pt}coproj\ C\ D\ {\isasymcirc}\isactrlsub c\ f{\isacharparenright}{\kern0pt}\ {\isasymcirc}\isactrlsub c\ a{\isacharprime}{\kern0pt}\ {\isachardoublequoteclose}\isanewline
\ \ \ \ \ \ \ \ \ \ \ \ \ \ \ \ \ \ \ \ \isacommand{unfolding}\isamarkupfalse%
\ {\isasymphi}{\isacharunderscore}{\kern0pt}def\ \isacommand{using}\isamarkupfalse%
\ f{\isacharunderscore}{\kern0pt}def\ g{\isacharunderscore}{\kern0pt}def\ a{\isacharprime}{\kern0pt}{\isacharunderscore}{\kern0pt}def\ left{\isacharunderscore}{\kern0pt}coproj{\isacharunderscore}{\kern0pt}cfunc{\isacharunderscore}{\kern0pt}coprod\ \isacommand{by}\isamarkupfalse%
\ {\isacharparenleft}{\kern0pt}typecheck{\isacharunderscore}{\kern0pt}cfuncs{\isacharcomma}{\kern0pt}\ auto{\isacharparenright}{\kern0pt}\isanewline
\ \ \ \ \ \ \ \ \ \ \ \ \ \ \ \ \ \ \isacommand{also}\isamarkupfalse%
\ \isacommand{have}\isamarkupfalse%
\ {\isachardoublequoteopen}{\isachardot}{\kern0pt}{\isachardot}{\kern0pt}{\isachardot}{\kern0pt}\ {\isacharequal}{\kern0pt}\ left{\isacharunderscore}{\kern0pt}coproj\ C\ D\ {\isasymcirc}\isactrlsub c\ {\isacharparenleft}{\kern0pt}f\ {\isasymcirc}\isactrlsub c\ a{\isacharprime}{\kern0pt}{\isacharparenright}{\kern0pt}{\isachardoublequoteclose}\isanewline
\ \ \ \ \ \ \ \ \ \ \ \ \ \ \ \ \ \ \ \ \ \ \isacommand{using}\isamarkupfalse%
\ f{\isacharunderscore}{\kern0pt}def\ a{\isacharprime}{\kern0pt}{\isacharunderscore}{\kern0pt}def\ \isacommand{by}\isamarkupfalse%
\ {\isacharparenleft}{\kern0pt}typecheck{\isacharunderscore}{\kern0pt}cfuncs{\isacharcomma}{\kern0pt}\ simp\ add{\isacharcolon}{\kern0pt}\ comp{\isacharunderscore}{\kern0pt}associative{\isadigit{2}}{\isacharparenright}{\kern0pt}\isanewline
\ \ \ \ \ \ \ \ \ \ \ \ \ \ \ \ \ \ \isacommand{then}\isamarkupfalse%
\ \isacommand{show}\isamarkupfalse%
\ {\isachardoublequoteopen}x\ {\isacharequal}{\kern0pt}\ y{\isachardoublequoteclose}\isanewline
\ \ \ \ \ \ \ \ \ \ \ \ \ \ \ \ \ \ \ \ \isacommand{by}\isamarkupfalse%
\ {\isacharparenleft}{\kern0pt}metis\ a{\isacharprime}{\kern0pt}{\isacharunderscore}{\kern0pt}def{\isacharparenleft}{\kern0pt}{\isadigit{1}}{\isacharparenright}{\kern0pt}\ b{\isacharunderscore}{\kern0pt}def\ calculation\ comp{\isacharunderscore}{\kern0pt}type\ coproducts{\isacharunderscore}{\kern0pt}disjoint\ f{\isacharunderscore}{\kern0pt}def{\isacharparenleft}{\kern0pt}{\isadigit{1}}{\isacharparenright}{\kern0pt}\ g{\isacharunderscore}{\kern0pt}def{\isacharparenleft}{\kern0pt}{\isadigit{1}}{\isacharparenright}{\kern0pt}{\isacharparenright}{\kern0pt}\isanewline
\ \ \ \ \ \ \ \ \ \ \ \ \ \ \ \ \isacommand{qed}\isamarkupfalse%
\isanewline
\ \ \ \ \ \ \ \ \isacommand{next}\isamarkupfalse%
\isanewline
\ \ \ \ \ \ \ \ \ \ \isacommand{assume}\isamarkupfalse%
\ {\isachardoublequoteopen}{\isasymnexists}a{\isachardot}{\kern0pt}\ a\ {\isasymin}\isactrlsub c\ A\ {\isasymand}\ y\ {\isacharequal}{\kern0pt}\ left{\isacharunderscore}{\kern0pt}coproj\ A\ B\ {\isasymcirc}\isactrlsub c\ a{\isachardoublequoteclose}\isanewline
\ \ \ \ \ \ \ \ \ \ \isacommand{then}\isamarkupfalse%
\ \isacommand{have}\isamarkupfalse%
\ {\isachardoublequoteopen}{\isasymexists}\ b{\isachardot}{\kern0pt}\ b\ {\isasymin}\isactrlsub c\ B\ {\isasymand}\ y\ {\isacharequal}{\kern0pt}\ right{\isacharunderscore}{\kern0pt}coproj\ A\ B\ {\isasymcirc}\isactrlsub c\ b{\isachardoublequoteclose}\isanewline
\ \ \ \ \ \ \ \ \ \ \ \ \isacommand{using}\isamarkupfalse%
\ y{\isacharunderscore}{\kern0pt}form\ \isacommand{by}\isamarkupfalse%
\ blast\isanewline
\ \ \ \ \ \ \ \ \isacommand{then}\isamarkupfalse%
\ \isacommand{obtain}\isamarkupfalse%
\ b{\isacharprime}{\kern0pt}\ \isakeyword{where}\ b{\isacharprime}{\kern0pt}{\isacharunderscore}{\kern0pt}def{\isacharcolon}{\kern0pt}\ {\isachardoublequoteopen}b{\isacharprime}{\kern0pt}\ {\isasymin}\isactrlsub c\ B{\isachardoublequoteclose}\ {\isachardoublequoteopen}y\ {\isacharequal}{\kern0pt}\ right{\isacharunderscore}{\kern0pt}coproj\ A\ B\ {\isasymcirc}\isactrlsub c\ b{\isacharprime}{\kern0pt}{\isachardoublequoteclose}\isanewline
\ \ \ \ \ \ \ \ \ \ \isacommand{by}\isamarkupfalse%
\ blast\isanewline
\ \ \ \ \ \ \ \ \isacommand{then}\isamarkupfalse%
\ \isacommand{have}\isamarkupfalse%
\ {\isachardoublequoteopen}b\ {\isacharequal}{\kern0pt}\ b{\isacharprime}{\kern0pt}{\isachardoublequoteclose}\isanewline
\ \ \ \ \ \ \ \ \isacommand{proof}\isamarkupfalse%
\ {\isacharminus}{\kern0pt}\ \isanewline
\ \ \ \ \ \ \ \ \ \ \isacommand{have}\isamarkupfalse%
\ {\isachardoublequoteopen}{\isacharparenleft}{\kern0pt}right{\isacharunderscore}{\kern0pt}coproj\ C\ D\ {\isasymcirc}\isactrlsub c\ g{\isacharparenright}{\kern0pt}\ {\isasymcirc}\isactrlsub c\ b\ {\isacharequal}{\kern0pt}\ {\isasymphi}\ {\isasymcirc}\isactrlsub c\ x{\isachardoublequoteclose}\isanewline
\ \ \ \ \ \ \ \ \ \ \ \ \isacommand{by}\isamarkupfalse%
\ {\isacharparenleft}{\kern0pt}smt\ {\isasymphi}{\isacharunderscore}{\kern0pt}def\ {\isasymphi}{\isacharunderscore}{\kern0pt}type\ b{\isacharunderscore}{\kern0pt}def\ comp{\isacharunderscore}{\kern0pt}associative{\isadigit{2}}\ comp{\isacharunderscore}{\kern0pt}type\ f{\isacharunderscore}{\kern0pt}def{\isacharparenleft}{\kern0pt}{\isadigit{1}}{\isacharparenright}{\kern0pt}\ g{\isacharunderscore}{\kern0pt}def{\isacharparenleft}{\kern0pt}{\isadigit{1}}{\isacharparenright}{\kern0pt}\ left{\isacharunderscore}{\kern0pt}proj{\isacharunderscore}{\kern0pt}type\ right{\isacharunderscore}{\kern0pt}coproj{\isacharunderscore}{\kern0pt}cfunc{\isacharunderscore}{\kern0pt}coprod\ right{\isacharunderscore}{\kern0pt}proj{\isacharunderscore}{\kern0pt}type{\isacharparenright}{\kern0pt}\isanewline
\ \ \ \ \ \ \ \ \ \ \isacommand{also}\isamarkupfalse%
\ \isacommand{have}\isamarkupfalse%
\ {\isachardoublequoteopen}{\isachardot}{\kern0pt}{\isachardot}{\kern0pt}{\isachardot}{\kern0pt}\ {\isacharequal}{\kern0pt}\ {\isasymphi}\ {\isasymcirc}\isactrlsub c\ y{\isachardoublequoteclose}\isanewline
\ \ \ \ \ \ \ \ \ \ \ \ \isacommand{by}\isamarkupfalse%
\ {\isacharparenleft}{\kern0pt}meson\ equals{\isacharparenright}{\kern0pt}\isanewline
\ \ \ \ \ \ \ \ \ \ \isacommand{also}\isamarkupfalse%
\ \isacommand{have}\isamarkupfalse%
\ {\isachardoublequoteopen}{\isachardot}{\kern0pt}{\isachardot}{\kern0pt}{\isachardot}{\kern0pt}\ {\isacharequal}{\kern0pt}\ {\isacharparenleft}{\kern0pt}{\isasymphi}\ {\isasymcirc}\isactrlsub c\ right{\isacharunderscore}{\kern0pt}coproj\ A\ B{\isacharparenright}{\kern0pt}\ {\isasymcirc}\isactrlsub c\ b{\isacharprime}{\kern0pt}{\isachardoublequoteclose}\isanewline
\ \ \ \ \ \ \ \ \ \ \ \ \isacommand{using}\isamarkupfalse%
\ {\isasymphi}{\isacharunderscore}{\kern0pt}type\ b{\isacharprime}{\kern0pt}{\isacharunderscore}{\kern0pt}def\ comp{\isacharunderscore}{\kern0pt}associative{\isadigit{2}}\ \isacommand{by}\isamarkupfalse%
\ {\isacharparenleft}{\kern0pt}typecheck{\isacharunderscore}{\kern0pt}cfuncs{\isacharcomma}{\kern0pt}\ blast{\isacharparenright}{\kern0pt}\isanewline
\ \ \ \ \ \ \ \ \ \ \isacommand{also}\isamarkupfalse%
\ \isacommand{have}\isamarkupfalse%
\ {\isachardoublequoteopen}{\isachardot}{\kern0pt}{\isachardot}{\kern0pt}{\isachardot}{\kern0pt}\ {\isacharequal}{\kern0pt}\ {\isacharparenleft}{\kern0pt}right{\isacharunderscore}{\kern0pt}coproj\ C\ D\ {\isasymcirc}\isactrlsub c\ g{\isacharparenright}{\kern0pt}\ {\isasymcirc}\isactrlsub c\ b{\isacharprime}{\kern0pt}{\isachardoublequoteclose}\isanewline
\ \ \ \ \ \ \ \ \ \ \ \ \isacommand{unfolding}\isamarkupfalse%
\ {\isasymphi}{\isacharunderscore}{\kern0pt}def\ \isacommand{using}\isamarkupfalse%
\ f{\isacharunderscore}{\kern0pt}def\ g{\isacharunderscore}{\kern0pt}def\ b{\isacharprime}{\kern0pt}{\isacharunderscore}{\kern0pt}def\ right{\isacharunderscore}{\kern0pt}coproj{\isacharunderscore}{\kern0pt}cfunc{\isacharunderscore}{\kern0pt}coprod\ \isacommand{by}\isamarkupfalse%
\ {\isacharparenleft}{\kern0pt}typecheck{\isacharunderscore}{\kern0pt}cfuncs{\isacharcomma}{\kern0pt}\ auto{\isacharparenright}{\kern0pt}\isanewline
\ \ \ \ \ \ \ \ \ \ \isacommand{then}\isamarkupfalse%
\ \isacommand{show}\isamarkupfalse%
\ {\isachardoublequoteopen}b\ {\isacharequal}{\kern0pt}\ b{\isacharprime}{\kern0pt}{\isachardoublequoteclose}\isanewline
\ \ \ \ \ \ \ \ \ \ \ \ \isacommand{by}\isamarkupfalse%
\ {\isacharparenleft}{\kern0pt}smt\ b{\isacharprime}{\kern0pt}{\isacharunderscore}{\kern0pt}def\ b{\isacharunderscore}{\kern0pt}def\ calculation\ cfunc{\isacharunderscore}{\kern0pt}type{\isacharunderscore}{\kern0pt}def\ coproj{\isacharunderscore}{\kern0pt}g{\isacharunderscore}{\kern0pt}inject\ domain{\isacharunderscore}{\kern0pt}comp\ g{\isacharunderscore}{\kern0pt}def\ injective{\isacharunderscore}{\kern0pt}def\ right{\isacharunderscore}{\kern0pt}proj{\isacharunderscore}{\kern0pt}type{\isacharparenright}{\kern0pt}\isanewline
\ \ \ \ \ \ \ \ \isacommand{qed}\isamarkupfalse%
\isanewline
\ \ \ \ \ \ \ \ \isacommand{then}\isamarkupfalse%
\ \isacommand{show}\isamarkupfalse%
\ {\isachardoublequoteopen}x\ {\isacharequal}{\kern0pt}\ y{\isachardoublequoteclose}\isanewline
\ \ \ \ \ \ \ \ \ \ \isacommand{by}\isamarkupfalse%
\ {\isacharparenleft}{\kern0pt}simp\ add{\isacharcolon}{\kern0pt}\ b{\isacharprime}{\kern0pt}{\isacharunderscore}{\kern0pt}def{\isacharparenleft}{\kern0pt}{\isadigit{2}}{\isacharparenright}{\kern0pt}\ b{\isacharunderscore}{\kern0pt}def{\isacharparenright}{\kern0pt}\isanewline
\ \ \ \ \ \ \isacommand{qed}\isamarkupfalse%
\isanewline
\ \ \ \ \isacommand{qed}\isamarkupfalse%
\isanewline
\ \ \isacommand{qed}\isamarkupfalse%
\isanewline
\isanewline
\ \ \isacommand{have}\isamarkupfalse%
\ {\isachardoublequoteopen}monomorphism\ {\isasymphi}{\isachardoublequoteclose}\isanewline
\ \ \ \ \isacommand{using}\isamarkupfalse%
\ {\isacartoucheopen}injective\ {\isasymphi}{\isacartoucheclose}\ injective{\isacharunderscore}{\kern0pt}imp{\isacharunderscore}{\kern0pt}monomorphism\ \isacommand{by}\isamarkupfalse%
\ blast\isanewline
\ \ \isacommand{have}\isamarkupfalse%
\ {\isachardoublequoteopen}epimorphism\ {\isasymphi}{\isachardoublequoteclose}\isanewline
\ \ \ \ \isacommand{by}\isamarkupfalse%
\ {\isacharparenleft}{\kern0pt}simp\ add{\isacharcolon}{\kern0pt}\ {\isacartoucheopen}surjective\ {\isasymphi}{\isacartoucheclose}\ surjective{\isacharunderscore}{\kern0pt}is{\isacharunderscore}{\kern0pt}epimorphism{\isacharparenright}{\kern0pt}\isanewline
\ \ \isacommand{have}\isamarkupfalse%
\ {\isachardoublequoteopen}isomorphism\ {\isasymphi}{\isachardoublequoteclose}\isanewline
\ \ \ \ \isacommand{using}\isamarkupfalse%
\ {\isacartoucheopen}epimorphism\ {\isasymphi}{\isacartoucheclose}\ {\isacartoucheopen}monomorphism\ {\isasymphi}{\isacartoucheclose}\ epi{\isacharunderscore}{\kern0pt}mon{\isacharunderscore}{\kern0pt}is{\isacharunderscore}{\kern0pt}iso\ \isacommand{by}\isamarkupfalse%
\ blast\isanewline
\ \ \isacommand{then}\isamarkupfalse%
\ \isacommand{show}\isamarkupfalse%
\ {\isacharquery}{\kern0pt}thesis\isanewline
\ \ \ \ \isacommand{using}\isamarkupfalse%
\ {\isasymphi}{\isacharunderscore}{\kern0pt}type\ is{\isacharunderscore}{\kern0pt}isomorphic{\isacharunderscore}{\kern0pt}def\ \isacommand{by}\isamarkupfalse%
\ blast\isanewline
\isacommand{qed}\isamarkupfalse%
%
\endisatagproof
{\isafoldproof}%
%
\isadelimproof
\ \isanewline
%
\endisadelimproof
\isanewline
\isacommand{lemma}\isamarkupfalse%
\ product{\isacharunderscore}{\kern0pt}distribute{\isacharunderscore}{\kern0pt}over{\isacharunderscore}{\kern0pt}coproduct{\isacharunderscore}{\kern0pt}right{\isacharcolon}{\kern0pt}\isanewline
\ \ {\isachardoublequoteopen}{\isacharparenleft}{\kern0pt}A\ {\isasymCoprod}\ B{\isacharparenright}{\kern0pt}\ {\isasymtimes}\isactrlsub c\ C\ \ {\isasymcong}\ {\isacharparenleft}{\kern0pt}A\ {\isasymtimes}\isactrlsub c\ C{\isacharparenright}{\kern0pt}\ {\isasymCoprod}\ {\isacharparenleft}{\kern0pt}B\ {\isasymtimes}\isactrlsub c\ C{\isacharparenright}{\kern0pt}{\isachardoublequoteclose}\isanewline
%
\isadelimproof
\ \ %
\endisadelimproof
%
\isatagproof
\isacommand{by}\isamarkupfalse%
\ {\isacharparenleft}{\kern0pt}meson\ coprod{\isacharunderscore}{\kern0pt}pres{\isacharunderscore}{\kern0pt}iso\ isomorphic{\isacharunderscore}{\kern0pt}is{\isacharunderscore}{\kern0pt}transitive\ product{\isacharunderscore}{\kern0pt}commutes\ product{\isacharunderscore}{\kern0pt}distribute{\isacharunderscore}{\kern0pt}over{\isacharunderscore}{\kern0pt}coproduct{\isacharunderscore}{\kern0pt}left{\isacharparenright}{\kern0pt}%
\endisatagproof
{\isafoldproof}%
%
\isadelimproof
\isanewline
%
\endisadelimproof
\isanewline
\isacommand{lemma}\isamarkupfalse%
\ coproduct{\isacharunderscore}{\kern0pt}with{\isacharunderscore}{\kern0pt}self{\isacharunderscore}{\kern0pt}iso{\isacharcolon}{\kern0pt}\isanewline
\ \ {\isachardoublequoteopen}X\ {\isasymCoprod}\ X\ {\isasymcong}\ X\ {\isasymtimes}\isactrlsub c\ {\isasymOmega}{\isachardoublequoteclose}\isanewline
%
\isadelimproof
%
\endisadelimproof
%
\isatagproof
\isacommand{proof}\isamarkupfalse%
\ {\isacharminus}{\kern0pt}\ \isanewline
\ \ \isacommand{obtain}\isamarkupfalse%
\ {\isasymrho}\ \isakeyword{where}\ {\isasymrho}{\isacharunderscore}{\kern0pt}def{\isacharcolon}{\kern0pt}\ {\isachardoublequoteopen}{\isasymrho}\ {\isacharequal}{\kern0pt}\ {\isasymlangle}id\ X{\isacharcomma}{\kern0pt}\ {\isasymt}\ {\isasymcirc}\isactrlsub c\ {\isasymbeta}\isactrlbsub X\isactrlesub {\isasymrangle}\ {\isasymamalg}\ {\isasymlangle}id\ X{\isacharcomma}{\kern0pt}\ {\isasymf}\ {\isasymcirc}\isactrlsub c\ {\isasymbeta}\isactrlbsub X\isactrlesub {\isasymrangle}{\isachardoublequoteclose}\ \isakeyword{and}\ {\isasymrho}{\isacharunderscore}{\kern0pt}type{\isacharbrackleft}{\kern0pt}type{\isacharunderscore}{\kern0pt}rule{\isacharbrackright}{\kern0pt}{\isacharcolon}{\kern0pt}\ {\isachardoublequoteopen}{\isasymrho}\ {\isacharcolon}{\kern0pt}\ X\ {\isasymCoprod}\ X\ {\isasymrightarrow}\ X\ {\isasymtimes}\isactrlsub c\ {\isasymOmega}{\isachardoublequoteclose}\isanewline
\ \ \ \ \isacommand{by}\isamarkupfalse%
\ typecheck{\isacharunderscore}{\kern0pt}cfuncs\isanewline
\ \ \isacommand{have}\isamarkupfalse%
\ {\isasymrho}{\isacharunderscore}{\kern0pt}inj{\isacharcolon}{\kern0pt}\ {\isachardoublequoteopen}injective\ {\isasymrho}{\isachardoublequoteclose}\isanewline
\ \ \ \ \isacommand{unfolding}\isamarkupfalse%
\ injective{\isacharunderscore}{\kern0pt}def\isanewline
\ \ \isacommand{proof}\isamarkupfalse%
{\isacharparenleft}{\kern0pt}auto{\isacharparenright}{\kern0pt}\isanewline
\ \ \ \ \isacommand{fix}\isamarkupfalse%
\ x\ y\ \isanewline
\ \ \ \ \isacommand{assume}\isamarkupfalse%
\ {\isachardoublequoteopen}x\ {\isasymin}\isactrlsub c\ domain\ {\isasymrho}{\isachardoublequoteclose}\ \isacommand{then}\isamarkupfalse%
\ \isacommand{have}\isamarkupfalse%
\ x{\isacharunderscore}{\kern0pt}type{\isacharbrackleft}{\kern0pt}type{\isacharunderscore}{\kern0pt}rule{\isacharbrackright}{\kern0pt}{\isacharcolon}{\kern0pt}\ {\isachardoublequoteopen}x\ {\isasymin}\isactrlsub c\ X\ {\isasymCoprod}\ X{\isachardoublequoteclose}\isanewline
\ \ \ \ \ \ \isacommand{using}\isamarkupfalse%
\ {\isasymrho}{\isacharunderscore}{\kern0pt}type\ cfunc{\isacharunderscore}{\kern0pt}type{\isacharunderscore}{\kern0pt}def\ \isacommand{by}\isamarkupfalse%
\ auto\isanewline
\ \ \ \ \isacommand{assume}\isamarkupfalse%
\ {\isachardoublequoteopen}y\ {\isasymin}\isactrlsub c\ domain\ {\isasymrho}{\isachardoublequoteclose}\ \isacommand{then}\isamarkupfalse%
\ \isacommand{have}\isamarkupfalse%
\ y{\isacharunderscore}{\kern0pt}type{\isacharbrackleft}{\kern0pt}type{\isacharunderscore}{\kern0pt}rule{\isacharbrackright}{\kern0pt}{\isacharcolon}{\kern0pt}\ {\isachardoublequoteopen}y\ {\isasymin}\isactrlsub c\ X\ {\isasymCoprod}\ X{\isachardoublequoteclose}\isanewline
\ \ \ \ \ \ \isacommand{using}\isamarkupfalse%
\ {\isasymrho}{\isacharunderscore}{\kern0pt}type\ cfunc{\isacharunderscore}{\kern0pt}type{\isacharunderscore}{\kern0pt}def\ \isacommand{by}\isamarkupfalse%
\ auto\isanewline
\ \ \ \ \isacommand{assume}\isamarkupfalse%
\ equals{\isacharcolon}{\kern0pt}\ {\isachardoublequoteopen}{\isasymrho}\ {\isasymcirc}\isactrlsub c\ x\ {\isacharequal}{\kern0pt}\ {\isasymrho}\ {\isasymcirc}\isactrlsub c\ y{\isachardoublequoteclose}\isanewline
\ \ \ \ \isacommand{show}\isamarkupfalse%
\ {\isachardoublequoteopen}x\ {\isacharequal}{\kern0pt}\ y{\isachardoublequoteclose}\isanewline
\ \ \ \ \isacommand{proof}\isamarkupfalse%
{\isacharparenleft}{\kern0pt}cases\ {\isachardoublequoteopen}{\isasymexists}\ lx{\isachardot}{\kern0pt}\ x\ {\isacharequal}{\kern0pt}\ left{\isacharunderscore}{\kern0pt}coproj\ X\ X\ {\isasymcirc}\isactrlsub c\ lx\ {\isasymand}\ lx\ {\isasymin}\isactrlsub c\ X{\isachardoublequoteclose}{\isacharparenright}{\kern0pt}\isanewline
\ \ \ \ \ \ \isacommand{assume}\isamarkupfalse%
\ {\isachardoublequoteopen}{\isasymexists}lx{\isachardot}{\kern0pt}\ x\ {\isacharequal}{\kern0pt}\ left{\isacharunderscore}{\kern0pt}coproj\ X\ X\ {\isasymcirc}\isactrlsub c\ lx\ {\isasymand}\ lx\ {\isasymin}\isactrlsub c\ X{\isachardoublequoteclose}\isanewline
\ \ \ \ \ \ \isacommand{then}\isamarkupfalse%
\ \isacommand{obtain}\isamarkupfalse%
\ lx\ \isakeyword{where}\ lx{\isacharunderscore}{\kern0pt}def{\isacharcolon}{\kern0pt}\ {\isachardoublequoteopen}x\ {\isacharequal}{\kern0pt}\ left{\isacharunderscore}{\kern0pt}coproj\ X\ X\ {\isasymcirc}\isactrlsub c\ lx\ {\isasymand}\ lx\ {\isasymin}\isactrlsub c\ X{\isachardoublequoteclose}\isanewline
\ \ \ \ \ \ \ \ \isacommand{by}\isamarkupfalse%
\ blast\isanewline
\ \ \ \ \ \ \isacommand{have}\isamarkupfalse%
\ {\isasymrho}x{\isacharcolon}{\kern0pt}\ {\isachardoublequoteopen}{\isasymrho}\ {\isasymcirc}\isactrlsub c\ x\ {\isacharequal}{\kern0pt}\ {\isasymlangle}lx{\isacharcomma}{\kern0pt}\ {\isasymt}{\isasymrangle}{\isachardoublequoteclose}\isanewline
\ \ \ \ \ \ \isacommand{proof}\isamarkupfalse%
\ {\isacharminus}{\kern0pt}\ \isanewline
\ \ \ \ \ \ \ \ \isacommand{have}\isamarkupfalse%
\ {\isachardoublequoteopen}{\isasymrho}\ {\isasymcirc}\isactrlsub c\ x\ {\isacharequal}{\kern0pt}\ {\isacharparenleft}{\kern0pt}{\isasymrho}\ {\isasymcirc}\isactrlsub c\ left{\isacharunderscore}{\kern0pt}coproj\ X\ X{\isacharparenright}{\kern0pt}\ {\isasymcirc}\isactrlsub c\ lx{\isachardoublequoteclose}\isanewline
\ \ \ \ \ \ \ \ \ \ \isacommand{using}\isamarkupfalse%
\ comp{\isacharunderscore}{\kern0pt}associative{\isadigit{2}}\ lx{\isacharunderscore}{\kern0pt}def\ \isacommand{by}\isamarkupfalse%
\ {\isacharparenleft}{\kern0pt}typecheck{\isacharunderscore}{\kern0pt}cfuncs{\isacharcomma}{\kern0pt}\ blast{\isacharparenright}{\kern0pt}\isanewline
\ \ \ \ \ \ \ \ \isacommand{also}\isamarkupfalse%
\ \isacommand{have}\isamarkupfalse%
\ {\isachardoublequoteopen}{\isachardot}{\kern0pt}{\isachardot}{\kern0pt}{\isachardot}{\kern0pt}\ {\isacharequal}{\kern0pt}\ {\isasymlangle}id\ X{\isacharcomma}{\kern0pt}\ {\isasymt}\ {\isasymcirc}\isactrlsub c\ {\isasymbeta}\isactrlbsub X\isactrlesub {\isasymrangle}\ \ {\isasymcirc}\isactrlsub c\ lx{\isachardoublequoteclose}\isanewline
\ \ \ \ \ \ \ \ \ \ \isacommand{unfolding}\isamarkupfalse%
\ {\isasymrho}{\isacharunderscore}{\kern0pt}def\ \ \isacommand{using}\isamarkupfalse%
\ left{\isacharunderscore}{\kern0pt}coproj{\isacharunderscore}{\kern0pt}cfunc{\isacharunderscore}{\kern0pt}coprod\ \isacommand{by}\isamarkupfalse%
\ {\isacharparenleft}{\kern0pt}typecheck{\isacharunderscore}{\kern0pt}cfuncs{\isacharcomma}{\kern0pt}\ presburger{\isacharparenright}{\kern0pt}\isanewline
\ \ \ \ \ \ \ \ \isacommand{also}\isamarkupfalse%
\ \isacommand{have}\isamarkupfalse%
\ {\isachardoublequoteopen}{\isachardot}{\kern0pt}{\isachardot}{\kern0pt}{\isachardot}{\kern0pt}\ {\isacharequal}{\kern0pt}\ {\isasymlangle}lx{\isacharcomma}{\kern0pt}\ {\isasymt}{\isasymrangle}{\isachardoublequoteclose}\isanewline
\ \ \ \ \ \ \ \ \ \ \isacommand{by}\isamarkupfalse%
\ {\isacharparenleft}{\kern0pt}typecheck{\isacharunderscore}{\kern0pt}cfuncs{\isacharcomma}{\kern0pt}\ metis\ cart{\isacharunderscore}{\kern0pt}prod{\isacharunderscore}{\kern0pt}extract{\isacharunderscore}{\kern0pt}left\ lx{\isacharunderscore}{\kern0pt}def{\isacharparenright}{\kern0pt}\isanewline
\ \ \ \ \ \ \ \ \isacommand{then}\isamarkupfalse%
\ \isacommand{show}\isamarkupfalse%
\ {\isacharquery}{\kern0pt}thesis\isanewline
\ \ \ \ \ \ \ \ \ \ \isacommand{by}\isamarkupfalse%
\ {\isacharparenleft}{\kern0pt}simp\ add{\isacharcolon}{\kern0pt}\ calculation{\isacharparenright}{\kern0pt}\isanewline
\ \ \ \ \ \ \isacommand{qed}\isamarkupfalse%
\isanewline
\ \ \ \ \ \ \isacommand{show}\isamarkupfalse%
\ {\isachardoublequoteopen}x\ {\isacharequal}{\kern0pt}\ y{\isachardoublequoteclose}\isanewline
\ \ \ \ \ \ \isacommand{proof}\isamarkupfalse%
{\isacharparenleft}{\kern0pt}cases\ {\isachardoublequoteopen}{\isasymexists}\ ly{\isachardot}{\kern0pt}\ y\ {\isacharequal}{\kern0pt}\ left{\isacharunderscore}{\kern0pt}coproj\ X\ X\ {\isasymcirc}\isactrlsub c\ ly\ {\isasymand}\ ly\ {\isasymin}\isactrlsub c\ X{\isachardoublequoteclose}{\isacharparenright}{\kern0pt}\isanewline
\ \ \ \ \ \ \ \ \isacommand{assume}\isamarkupfalse%
\ {\isachardoublequoteopen}{\isasymexists}ly{\isachardot}{\kern0pt}\ y\ {\isacharequal}{\kern0pt}\ left{\isacharunderscore}{\kern0pt}coproj\ X\ X\ {\isasymcirc}\isactrlsub c\ ly\ {\isasymand}\ ly\ {\isasymin}\isactrlsub c\ X{\isachardoublequoteclose}\isanewline
\ \ \ \ \ \ \ \ \isacommand{then}\isamarkupfalse%
\ \isacommand{obtain}\isamarkupfalse%
\ ly\ \isakeyword{where}\ ly{\isacharunderscore}{\kern0pt}def{\isacharcolon}{\kern0pt}\ {\isachardoublequoteopen}y\ {\isacharequal}{\kern0pt}\ left{\isacharunderscore}{\kern0pt}coproj\ X\ X\ {\isasymcirc}\isactrlsub c\ ly\ {\isasymand}\ ly\ {\isasymin}\isactrlsub c\ X{\isachardoublequoteclose}\isanewline
\ \ \ \ \ \ \ \ \ \ \isacommand{by}\isamarkupfalse%
\ blast\isanewline
\ \ \ \ \ \ \ \ \isacommand{have}\isamarkupfalse%
\ {\isachardoublequoteopen}{\isasymrho}\ {\isasymcirc}\isactrlsub c\ y\ {\isacharequal}{\kern0pt}\ {\isasymlangle}ly{\isacharcomma}{\kern0pt}\ {\isasymt}{\isasymrangle}{\isachardoublequoteclose}\isanewline
\ \ \ \ \ \ \ \ \isacommand{proof}\isamarkupfalse%
\ {\isacharminus}{\kern0pt}\ \isanewline
\ \ \ \ \ \ \ \ \ \ \isacommand{have}\isamarkupfalse%
\ {\isachardoublequoteopen}{\isasymrho}\ {\isasymcirc}\isactrlsub c\ y\ {\isacharequal}{\kern0pt}\ {\isacharparenleft}{\kern0pt}{\isasymrho}\ {\isasymcirc}\isactrlsub c\ left{\isacharunderscore}{\kern0pt}coproj\ X\ X{\isacharparenright}{\kern0pt}\ {\isasymcirc}\isactrlsub c\ ly{\isachardoublequoteclose}\isanewline
\ \ \ \ \ \ \ \ \ \ \ \ \isacommand{using}\isamarkupfalse%
\ comp{\isacharunderscore}{\kern0pt}associative{\isadigit{2}}\ ly{\isacharunderscore}{\kern0pt}def\ \isacommand{by}\isamarkupfalse%
\ {\isacharparenleft}{\kern0pt}typecheck{\isacharunderscore}{\kern0pt}cfuncs{\isacharcomma}{\kern0pt}\ blast{\isacharparenright}{\kern0pt}\isanewline
\ \ \ \ \ \ \ \ \ \ \isacommand{also}\isamarkupfalse%
\ \isacommand{have}\isamarkupfalse%
\ {\isachardoublequoteopen}{\isachardot}{\kern0pt}{\isachardot}{\kern0pt}{\isachardot}{\kern0pt}\ {\isacharequal}{\kern0pt}\ {\isasymlangle}id\ X{\isacharcomma}{\kern0pt}\ {\isasymt}\ {\isasymcirc}\isactrlsub c\ {\isasymbeta}\isactrlbsub X\isactrlesub {\isasymrangle}\ \ {\isasymcirc}\isactrlsub c\ ly{\isachardoublequoteclose}\isanewline
\ \ \ \ \ \ \ \ \ \ \ \ \isacommand{unfolding}\isamarkupfalse%
\ {\isasymrho}{\isacharunderscore}{\kern0pt}def\ \ \isacommand{using}\isamarkupfalse%
\ left{\isacharunderscore}{\kern0pt}coproj{\isacharunderscore}{\kern0pt}cfunc{\isacharunderscore}{\kern0pt}coprod\ \isacommand{by}\isamarkupfalse%
\ {\isacharparenleft}{\kern0pt}typecheck{\isacharunderscore}{\kern0pt}cfuncs{\isacharcomma}{\kern0pt}\ presburger{\isacharparenright}{\kern0pt}\isanewline
\ \ \ \ \ \ \ \ \ \ \isacommand{also}\isamarkupfalse%
\ \isacommand{have}\isamarkupfalse%
\ {\isachardoublequoteopen}{\isachardot}{\kern0pt}{\isachardot}{\kern0pt}{\isachardot}{\kern0pt}\ {\isacharequal}{\kern0pt}\ {\isasymlangle}ly{\isacharcomma}{\kern0pt}\ {\isasymt}{\isasymrangle}{\isachardoublequoteclose}\isanewline
\ \ \ \ \ \ \ \ \ \ \ \ \isacommand{by}\isamarkupfalse%
\ {\isacharparenleft}{\kern0pt}typecheck{\isacharunderscore}{\kern0pt}cfuncs{\isacharcomma}{\kern0pt}\ metis\ cart{\isacharunderscore}{\kern0pt}prod{\isacharunderscore}{\kern0pt}extract{\isacharunderscore}{\kern0pt}left\ ly{\isacharunderscore}{\kern0pt}def{\isacharparenright}{\kern0pt}\isanewline
\ \ \ \ \ \ \ \ \ \ \isacommand{then}\isamarkupfalse%
\ \isacommand{show}\isamarkupfalse%
\ {\isacharquery}{\kern0pt}thesis\isanewline
\ \ \ \ \ \ \ \ \ \ \ \ \isacommand{by}\isamarkupfalse%
\ {\isacharparenleft}{\kern0pt}simp\ add{\isacharcolon}{\kern0pt}\ calculation{\isacharparenright}{\kern0pt}\isanewline
\ \ \ \ \ \ \ \ \isacommand{qed}\isamarkupfalse%
\isanewline
\ \ \ \ \ \ \ \ \isacommand{then}\isamarkupfalse%
\ \isacommand{show}\isamarkupfalse%
\ {\isachardoublequoteopen}x\ {\isacharequal}{\kern0pt}\ y{\isachardoublequoteclose}\isanewline
\ \ \ \ \ \ \ \ \ \ \isacommand{using}\isamarkupfalse%
\ {\isasymrho}x\ cart{\isacharunderscore}{\kern0pt}prod{\isacharunderscore}{\kern0pt}eq{\isadigit{2}}\ equals\ lx{\isacharunderscore}{\kern0pt}def\ ly{\isacharunderscore}{\kern0pt}def\ true{\isacharunderscore}{\kern0pt}func{\isacharunderscore}{\kern0pt}type\ \isacommand{by}\isamarkupfalse%
\ auto\isanewline
\ \ \ \ \ \ \isacommand{next}\isamarkupfalse%
\isanewline
\ \ \ \ \ \ \ \ \isacommand{assume}\isamarkupfalse%
\ {\isachardoublequoteopen}{\isasymnexists}ly{\isachardot}{\kern0pt}\ y\ {\isacharequal}{\kern0pt}\ left{\isacharunderscore}{\kern0pt}coproj\ X\ X\ {\isasymcirc}\isactrlsub c\ ly\ {\isasymand}\ ly\ {\isasymin}\isactrlsub c\ X{\isachardoublequoteclose}\isanewline
\ \ \ \ \ \ \ \ \isacommand{then}\isamarkupfalse%
\ \isacommand{obtain}\isamarkupfalse%
\ ry\ \isakeyword{where}\ ry{\isacharunderscore}{\kern0pt}def{\isacharcolon}{\kern0pt}\ {\isachardoublequoteopen}y\ {\isacharequal}{\kern0pt}\ right{\isacharunderscore}{\kern0pt}coproj\ X\ X\ {\isasymcirc}\isactrlsub c\ ry{\isachardoublequoteclose}\ \isakeyword{and}\ ry{\isacharunderscore}{\kern0pt}type{\isacharbrackleft}{\kern0pt}type{\isacharunderscore}{\kern0pt}rule{\isacharbrackright}{\kern0pt}{\isacharcolon}{\kern0pt}\ {\isachardoublequoteopen}ry\ {\isasymin}\isactrlsub c\ X{\isachardoublequoteclose}\isanewline
\ \ \ \ \ \ \ \ \ \ \isacommand{by}\isamarkupfalse%
\ {\isacharparenleft}{\kern0pt}meson\ y{\isacharunderscore}{\kern0pt}type\ coprojs{\isacharunderscore}{\kern0pt}jointly{\isacharunderscore}{\kern0pt}surj{\isacharparenright}{\kern0pt}\isanewline
\ \ \ \ \ \ \ \ \isacommand{have}\isamarkupfalse%
\ {\isasymrho}y{\isacharcolon}{\kern0pt}\ {\isachardoublequoteopen}{\isasymrho}\ {\isasymcirc}\isactrlsub c\ y\ {\isacharequal}{\kern0pt}\ {\isasymlangle}ry{\isacharcomma}{\kern0pt}\ {\isasymf}{\isasymrangle}{\isachardoublequoteclose}\isanewline
\ \ \ \ \ \ \ \ \isacommand{proof}\isamarkupfalse%
\ {\isacharminus}{\kern0pt}\ \isanewline
\ \ \ \ \ \ \ \ \ \ \isacommand{have}\isamarkupfalse%
\ {\isachardoublequoteopen}{\isasymrho}\ {\isasymcirc}\isactrlsub c\ y\ {\isacharequal}{\kern0pt}\ {\isacharparenleft}{\kern0pt}{\isasymrho}\ {\isasymcirc}\isactrlsub c\ right{\isacharunderscore}{\kern0pt}coproj\ X\ X{\isacharparenright}{\kern0pt}\ {\isasymcirc}\isactrlsub c\ ry{\isachardoublequoteclose}\isanewline
\ \ \ \ \ \ \ \ \ \ \ \ \isacommand{using}\isamarkupfalse%
\ comp{\isacharunderscore}{\kern0pt}associative{\isadigit{2}}\ ry{\isacharunderscore}{\kern0pt}def\ \isacommand{by}\isamarkupfalse%
\ {\isacharparenleft}{\kern0pt}typecheck{\isacharunderscore}{\kern0pt}cfuncs{\isacharcomma}{\kern0pt}\ blast{\isacharparenright}{\kern0pt}\isanewline
\ \ \ \ \ \ \ \ \ \ \isacommand{also}\isamarkupfalse%
\ \isacommand{have}\isamarkupfalse%
\ {\isachardoublequoteopen}{\isachardot}{\kern0pt}{\isachardot}{\kern0pt}{\isachardot}{\kern0pt}\ {\isacharequal}{\kern0pt}\ {\isasymlangle}id\ X{\isacharcomma}{\kern0pt}\ {\isasymf}\ {\isasymcirc}\isactrlsub c\ {\isasymbeta}\isactrlbsub X\isactrlesub {\isasymrangle}\ \ {\isasymcirc}\isactrlsub c\ ry{\isachardoublequoteclose}\isanewline
\ \ \ \ \ \ \ \ \ \ \ \ \isacommand{unfolding}\isamarkupfalse%
\ {\isasymrho}{\isacharunderscore}{\kern0pt}def\ \ \isacommand{using}\isamarkupfalse%
\ right{\isacharunderscore}{\kern0pt}coproj{\isacharunderscore}{\kern0pt}cfunc{\isacharunderscore}{\kern0pt}coprod\ \isacommand{by}\isamarkupfalse%
\ {\isacharparenleft}{\kern0pt}typecheck{\isacharunderscore}{\kern0pt}cfuncs{\isacharcomma}{\kern0pt}\ presburger{\isacharparenright}{\kern0pt}\isanewline
\ \ \ \ \ \ \ \ \ \ \isacommand{also}\isamarkupfalse%
\ \isacommand{have}\isamarkupfalse%
\ {\isachardoublequoteopen}{\isachardot}{\kern0pt}{\isachardot}{\kern0pt}{\isachardot}{\kern0pt}\ {\isacharequal}{\kern0pt}\ {\isasymlangle}ry{\isacharcomma}{\kern0pt}\ {\isasymf}{\isasymrangle}{\isachardoublequoteclose}\isanewline
\ \ \ \ \ \ \ \ \ \ \ \ \isacommand{by}\isamarkupfalse%
\ {\isacharparenleft}{\kern0pt}typecheck{\isacharunderscore}{\kern0pt}cfuncs{\isacharcomma}{\kern0pt}\ metis\ cart{\isacharunderscore}{\kern0pt}prod{\isacharunderscore}{\kern0pt}extract{\isacharunderscore}{\kern0pt}left{\isacharparenright}{\kern0pt}\isanewline
\ \ \ \ \ \ \ \ \ \ \isacommand{then}\isamarkupfalse%
\ \isacommand{show}\isamarkupfalse%
\ {\isacharquery}{\kern0pt}thesis\isanewline
\ \ \ \ \ \ \ \ \ \ \ \ \isacommand{by}\isamarkupfalse%
\ {\isacharparenleft}{\kern0pt}simp\ add{\isacharcolon}{\kern0pt}\ calculation{\isacharparenright}{\kern0pt}\isanewline
\ \ \ \ \ \ \ \ \isacommand{qed}\isamarkupfalse%
\isanewline
\ \ \ \ \ \ \ \ \isacommand{then}\isamarkupfalse%
\ \isacommand{show}\isamarkupfalse%
\ {\isacharquery}{\kern0pt}thesis\isanewline
\ \ \ \ \ \ \ \ \ \ \isacommand{using}\isamarkupfalse%
\ {\isasymrho}x\ {\isasymrho}y\ cart{\isacharunderscore}{\kern0pt}prod{\isacharunderscore}{\kern0pt}eq{\isadigit{2}}\ equals\ false{\isacharunderscore}{\kern0pt}func{\isacharunderscore}{\kern0pt}type\ lx{\isacharunderscore}{\kern0pt}def\ ry{\isacharunderscore}{\kern0pt}type\ true{\isacharunderscore}{\kern0pt}false{\isacharunderscore}{\kern0pt}distinct\ true{\isacharunderscore}{\kern0pt}func{\isacharunderscore}{\kern0pt}type\ \isacommand{by}\isamarkupfalse%
\ force\isanewline
\ \ \ \ \ \ \isacommand{qed}\isamarkupfalse%
\isanewline
\ \ \ \ \isacommand{next}\isamarkupfalse%
\isanewline
\ \ \ \ \ \ \isacommand{assume}\isamarkupfalse%
\ {\isachardoublequoteopen}{\isasymnexists}lx{\isachardot}{\kern0pt}\ x\ {\isacharequal}{\kern0pt}\ left{\isacharunderscore}{\kern0pt}coproj\ X\ X\ {\isasymcirc}\isactrlsub c\ lx\ {\isasymand}\ lx\ {\isasymin}\isactrlsub c\ X{\isachardoublequoteclose}\isanewline
\ \ \ \ \ \ \isacommand{then}\isamarkupfalse%
\ \isacommand{obtain}\isamarkupfalse%
\ rx\ \isakeyword{where}\ rx{\isacharunderscore}{\kern0pt}def{\isacharcolon}{\kern0pt}\ {\isachardoublequoteopen}x\ {\isacharequal}{\kern0pt}\ right{\isacharunderscore}{\kern0pt}coproj\ X\ X\ {\isasymcirc}\isactrlsub c\ rx\ {\isasymand}\ rx\ {\isasymin}\isactrlsub c\ X{\isachardoublequoteclose}\isanewline
\ \ \ \ \ \ \ \ \isacommand{by}\isamarkupfalse%
\ {\isacharparenleft}{\kern0pt}typecheck{\isacharunderscore}{\kern0pt}cfuncs{\isacharcomma}{\kern0pt}\ meson\ coprojs{\isacharunderscore}{\kern0pt}jointly{\isacharunderscore}{\kern0pt}surj{\isacharparenright}{\kern0pt}\isanewline
\ \ \ \ \ \ \isacommand{have}\isamarkupfalse%
\ {\isasymrho}x{\isacharcolon}{\kern0pt}\ {\isachardoublequoteopen}{\isasymrho}\ {\isasymcirc}\isactrlsub c\ x\ {\isacharequal}{\kern0pt}\ {\isasymlangle}rx{\isacharcomma}{\kern0pt}\ {\isasymf}{\isasymrangle}{\isachardoublequoteclose}\isanewline
\ \ \ \ \ \ \isacommand{proof}\isamarkupfalse%
\ {\isacharminus}{\kern0pt}\ \isanewline
\ \ \ \ \ \ \ \ \isacommand{have}\isamarkupfalse%
\ {\isachardoublequoteopen}{\isasymrho}\ {\isasymcirc}\isactrlsub c\ x\ {\isacharequal}{\kern0pt}\ {\isacharparenleft}{\kern0pt}{\isasymrho}\ {\isasymcirc}\isactrlsub c\ right{\isacharunderscore}{\kern0pt}coproj\ X\ X{\isacharparenright}{\kern0pt}\ {\isasymcirc}\isactrlsub c\ rx{\isachardoublequoteclose}\isanewline
\ \ \ \ \ \ \ \ \ \ \isacommand{using}\isamarkupfalse%
\ comp{\isacharunderscore}{\kern0pt}associative{\isadigit{2}}\ rx{\isacharunderscore}{\kern0pt}def\ \isacommand{by}\isamarkupfalse%
\ {\isacharparenleft}{\kern0pt}typecheck{\isacharunderscore}{\kern0pt}cfuncs{\isacharcomma}{\kern0pt}\ blast{\isacharparenright}{\kern0pt}\isanewline
\ \ \ \ \ \ \ \ \isacommand{also}\isamarkupfalse%
\ \isacommand{have}\isamarkupfalse%
\ {\isachardoublequoteopen}{\isachardot}{\kern0pt}{\isachardot}{\kern0pt}{\isachardot}{\kern0pt}\ {\isacharequal}{\kern0pt}\ {\isasymlangle}id\ X{\isacharcomma}{\kern0pt}\ {\isasymf}\ {\isasymcirc}\isactrlsub c\ {\isasymbeta}\isactrlbsub X\isactrlesub {\isasymrangle}\ \ {\isasymcirc}\isactrlsub c\ rx{\isachardoublequoteclose}\isanewline
\ \ \ \ \ \ \ \ \ \ \isacommand{unfolding}\isamarkupfalse%
\ {\isasymrho}{\isacharunderscore}{\kern0pt}def\ \ \isacommand{using}\isamarkupfalse%
\ right{\isacharunderscore}{\kern0pt}coproj{\isacharunderscore}{\kern0pt}cfunc{\isacharunderscore}{\kern0pt}coprod\ \isacommand{by}\isamarkupfalse%
\ {\isacharparenleft}{\kern0pt}typecheck{\isacharunderscore}{\kern0pt}cfuncs{\isacharcomma}{\kern0pt}\ presburger{\isacharparenright}{\kern0pt}\isanewline
\ \ \ \ \ \ \ \ \isacommand{also}\isamarkupfalse%
\ \isacommand{have}\isamarkupfalse%
\ {\isachardoublequoteopen}{\isachardot}{\kern0pt}{\isachardot}{\kern0pt}{\isachardot}{\kern0pt}\ {\isacharequal}{\kern0pt}\ {\isasymlangle}rx{\isacharcomma}{\kern0pt}\ {\isasymf}{\isasymrangle}{\isachardoublequoteclose}\isanewline
\ \ \ \ \ \ \ \ \ \ \isacommand{by}\isamarkupfalse%
\ {\isacharparenleft}{\kern0pt}typecheck{\isacharunderscore}{\kern0pt}cfuncs{\isacharcomma}{\kern0pt}\ metis\ cart{\isacharunderscore}{\kern0pt}prod{\isacharunderscore}{\kern0pt}extract{\isacharunderscore}{\kern0pt}left\ rx{\isacharunderscore}{\kern0pt}def{\isacharparenright}{\kern0pt}\isanewline
\ \ \ \ \ \ \ \ \isacommand{then}\isamarkupfalse%
\ \isacommand{show}\isamarkupfalse%
\ {\isacharquery}{\kern0pt}thesis\isanewline
\ \ \ \ \ \ \ \ \ \ \isacommand{by}\isamarkupfalse%
\ {\isacharparenleft}{\kern0pt}simp\ add{\isacharcolon}{\kern0pt}\ calculation{\isacharparenright}{\kern0pt}\isanewline
\ \ \ \ \ \ \isacommand{qed}\isamarkupfalse%
\isanewline
\ \ \ \ \ \ \isacommand{show}\isamarkupfalse%
\ {\isachardoublequoteopen}x\ {\isacharequal}{\kern0pt}\ y{\isachardoublequoteclose}\isanewline
\ \ \ \ \ \ \isacommand{proof}\isamarkupfalse%
{\isacharparenleft}{\kern0pt}cases\ {\isachardoublequoteopen}{\isasymexists}\ ly{\isachardot}{\kern0pt}\ y\ {\isacharequal}{\kern0pt}\ left{\isacharunderscore}{\kern0pt}coproj\ X\ X\ {\isasymcirc}\isactrlsub c\ ly\ {\isasymand}\ ly\ {\isasymin}\isactrlsub c\ X{\isachardoublequoteclose}{\isacharparenright}{\kern0pt}\isanewline
\ \ \ \ \ \ \ \ \isacommand{assume}\isamarkupfalse%
\ {\isachardoublequoteopen}{\isasymexists}ly{\isachardot}{\kern0pt}\ y\ {\isacharequal}{\kern0pt}\ left{\isacharunderscore}{\kern0pt}coproj\ X\ X\ {\isasymcirc}\isactrlsub c\ ly\ {\isasymand}\ ly\ {\isasymin}\isactrlsub c\ X{\isachardoublequoteclose}\isanewline
\ \ \ \ \ \ \ \ \isacommand{then}\isamarkupfalse%
\ \isacommand{obtain}\isamarkupfalse%
\ ly\ \isakeyword{where}\ ly{\isacharunderscore}{\kern0pt}def{\isacharcolon}{\kern0pt}\ {\isachardoublequoteopen}y\ {\isacharequal}{\kern0pt}\ left{\isacharunderscore}{\kern0pt}coproj\ X\ X\ {\isasymcirc}\isactrlsub c\ ly\ {\isasymand}\ ly\ {\isasymin}\isactrlsub c\ X{\isachardoublequoteclose}\isanewline
\ \ \ \ \ \ \ \ \ \ \isacommand{by}\isamarkupfalse%
\ blast\isanewline
\ \ \ \ \ \ \ \ \isacommand{have}\isamarkupfalse%
\ {\isachardoublequoteopen}{\isasymrho}\ {\isasymcirc}\isactrlsub c\ y\ {\isacharequal}{\kern0pt}\ {\isasymlangle}ly{\isacharcomma}{\kern0pt}\ {\isasymt}{\isasymrangle}{\isachardoublequoteclose}\isanewline
\ \ \ \ \ \ \ \ \isacommand{proof}\isamarkupfalse%
\ {\isacharminus}{\kern0pt}\ \isanewline
\ \ \ \ \ \ \ \ \ \ \isacommand{have}\isamarkupfalse%
\ {\isachardoublequoteopen}{\isasymrho}\ {\isasymcirc}\isactrlsub c\ y\ {\isacharequal}{\kern0pt}\ {\isacharparenleft}{\kern0pt}{\isasymrho}\ {\isasymcirc}\isactrlsub c\ left{\isacharunderscore}{\kern0pt}coproj\ X\ X{\isacharparenright}{\kern0pt}\ {\isasymcirc}\isactrlsub c\ ly{\isachardoublequoteclose}\isanewline
\ \ \ \ \ \ \ \ \ \ \ \ \isacommand{using}\isamarkupfalse%
\ comp{\isacharunderscore}{\kern0pt}associative{\isadigit{2}}\ ly{\isacharunderscore}{\kern0pt}def\ \isacommand{by}\isamarkupfalse%
\ {\isacharparenleft}{\kern0pt}typecheck{\isacharunderscore}{\kern0pt}cfuncs{\isacharcomma}{\kern0pt}\ blast{\isacharparenright}{\kern0pt}\isanewline
\ \ \ \ \ \ \ \ \ \ \isacommand{also}\isamarkupfalse%
\ \isacommand{have}\isamarkupfalse%
\ {\isachardoublequoteopen}{\isachardot}{\kern0pt}{\isachardot}{\kern0pt}{\isachardot}{\kern0pt}\ {\isacharequal}{\kern0pt}\ {\isasymlangle}id\ X{\isacharcomma}{\kern0pt}\ {\isasymt}\ {\isasymcirc}\isactrlsub c\ {\isasymbeta}\isactrlbsub X\isactrlesub {\isasymrangle}\ {\isasymcirc}\isactrlsub c\ ly{\isachardoublequoteclose}\isanewline
\ \ \ \ \ \ \ \ \ \ \ \ \isacommand{unfolding}\isamarkupfalse%
\ {\isasymrho}{\isacharunderscore}{\kern0pt}def\ \ \isacommand{using}\isamarkupfalse%
\ left{\isacharunderscore}{\kern0pt}coproj{\isacharunderscore}{\kern0pt}cfunc{\isacharunderscore}{\kern0pt}coprod\ \isacommand{by}\isamarkupfalse%
\ {\isacharparenleft}{\kern0pt}typecheck{\isacharunderscore}{\kern0pt}cfuncs{\isacharcomma}{\kern0pt}\ presburger{\isacharparenright}{\kern0pt}\isanewline
\ \ \ \ \ \ \ \ \ \ \isacommand{also}\isamarkupfalse%
\ \isacommand{have}\isamarkupfalse%
\ {\isachardoublequoteopen}{\isachardot}{\kern0pt}{\isachardot}{\kern0pt}{\isachardot}{\kern0pt}\ {\isacharequal}{\kern0pt}\ {\isasymlangle}ly{\isacharcomma}{\kern0pt}\ {\isasymt}{\isasymrangle}{\isachardoublequoteclose}\isanewline
\ \ \ \ \ \ \ \ \ \ \ \ \isacommand{by}\isamarkupfalse%
\ {\isacharparenleft}{\kern0pt}typecheck{\isacharunderscore}{\kern0pt}cfuncs{\isacharcomma}{\kern0pt}\ metis\ cart{\isacharunderscore}{\kern0pt}prod{\isacharunderscore}{\kern0pt}extract{\isacharunderscore}{\kern0pt}left\ ly{\isacharunderscore}{\kern0pt}def{\isacharparenright}{\kern0pt}\isanewline
\ \ \ \ \ \ \ \ \ \ \isacommand{then}\isamarkupfalse%
\ \isacommand{show}\isamarkupfalse%
\ {\isacharquery}{\kern0pt}thesis\isanewline
\ \ \ \ \ \ \ \ \ \ \ \ \isacommand{by}\isamarkupfalse%
\ {\isacharparenleft}{\kern0pt}simp\ add{\isacharcolon}{\kern0pt}\ calculation{\isacharparenright}{\kern0pt}\isanewline
\ \ \ \ \ \ \ \ \isacommand{qed}\isamarkupfalse%
\isanewline
\ \ \ \ \ \ \ \ \isacommand{then}\isamarkupfalse%
\ \isacommand{show}\isamarkupfalse%
\ {\isachardoublequoteopen}x\ {\isacharequal}{\kern0pt}\ y{\isachardoublequoteclose}\isanewline
\ \ \ \ \ \ \ \ \ \ \isacommand{using}\isamarkupfalse%
\ {\isasymrho}x\ cart{\isacharunderscore}{\kern0pt}prod{\isacharunderscore}{\kern0pt}eq{\isadigit{2}}\ equals\ false{\isacharunderscore}{\kern0pt}func{\isacharunderscore}{\kern0pt}type\ ly{\isacharunderscore}{\kern0pt}def\ rx{\isacharunderscore}{\kern0pt}def\ true{\isacharunderscore}{\kern0pt}false{\isacharunderscore}{\kern0pt}distinct\ true{\isacharunderscore}{\kern0pt}func{\isacharunderscore}{\kern0pt}type\ \isacommand{by}\isamarkupfalse%
\ force\isanewline
\ \ \ \ \ \ \isacommand{next}\isamarkupfalse%
\isanewline
\ \ \ \ \ \ \ \ \isacommand{assume}\isamarkupfalse%
\ {\isachardoublequoteopen}{\isasymnexists}ly{\isachardot}{\kern0pt}\ y\ {\isacharequal}{\kern0pt}\ left{\isacharunderscore}{\kern0pt}coproj\ X\ X\ {\isasymcirc}\isactrlsub c\ ly\ {\isasymand}\ ly\ {\isasymin}\isactrlsub c\ X{\isachardoublequoteclose}\isanewline
\ \ \ \ \ \ \ \ \isacommand{then}\isamarkupfalse%
\ \isacommand{obtain}\isamarkupfalse%
\ ry\ \isakeyword{where}\ ry{\isacharunderscore}{\kern0pt}def{\isacharcolon}{\kern0pt}\ {\isachardoublequoteopen}y\ {\isacharequal}{\kern0pt}\ right{\isacharunderscore}{\kern0pt}coproj\ X\ X\ {\isasymcirc}\isactrlsub c\ ry\ {\isasymand}\ ry\ {\isasymin}\isactrlsub c\ X{\isachardoublequoteclose}\isanewline
\ \ \ \ \ \ \ \ \ \ \isacommand{using}\isamarkupfalse%
\ \ coprojs{\isacharunderscore}{\kern0pt}jointly{\isacharunderscore}{\kern0pt}surj\ \isacommand{by}\isamarkupfalse%
\ {\isacharparenleft}{\kern0pt}typecheck{\isacharunderscore}{\kern0pt}cfuncs{\isacharcomma}{\kern0pt}\ blast{\isacharparenright}{\kern0pt}\isanewline
\ \ \ \ \ \ \ \ \isacommand{have}\isamarkupfalse%
\ {\isasymrho}y{\isacharcolon}{\kern0pt}\ {\isachardoublequoteopen}{\isasymrho}\ {\isasymcirc}\isactrlsub c\ y\ {\isacharequal}{\kern0pt}\ {\isasymlangle}ry{\isacharcomma}{\kern0pt}\ {\isasymf}{\isasymrangle}{\isachardoublequoteclose}\isanewline
\ \ \ \ \ \ \ \ \isacommand{proof}\isamarkupfalse%
\ {\isacharminus}{\kern0pt}\ \isanewline
\ \ \ \ \ \ \ \ \ \ \isacommand{have}\isamarkupfalse%
\ {\isachardoublequoteopen}{\isasymrho}\ {\isasymcirc}\isactrlsub c\ y\ {\isacharequal}{\kern0pt}\ {\isacharparenleft}{\kern0pt}{\isasymrho}\ {\isasymcirc}\isactrlsub c\ right{\isacharunderscore}{\kern0pt}coproj\ X\ X{\isacharparenright}{\kern0pt}\ {\isasymcirc}\isactrlsub c\ ry{\isachardoublequoteclose}\isanewline
\ \ \ \ \ \ \ \ \ \ \ \ \isacommand{using}\isamarkupfalse%
\ comp{\isacharunderscore}{\kern0pt}associative{\isadigit{2}}\ ry{\isacharunderscore}{\kern0pt}def\ \isacommand{by}\isamarkupfalse%
\ {\isacharparenleft}{\kern0pt}typecheck{\isacharunderscore}{\kern0pt}cfuncs{\isacharcomma}{\kern0pt}\ blast{\isacharparenright}{\kern0pt}\isanewline
\ \ \ \ \ \ \ \ \ \ \isacommand{also}\isamarkupfalse%
\ \isacommand{have}\isamarkupfalse%
\ {\isachardoublequoteopen}{\isachardot}{\kern0pt}{\isachardot}{\kern0pt}{\isachardot}{\kern0pt}\ {\isacharequal}{\kern0pt}\ {\isasymlangle}id\ X{\isacharcomma}{\kern0pt}\ {\isasymf}\ {\isasymcirc}\isactrlsub c\ {\isasymbeta}\isactrlbsub X\isactrlesub {\isasymrangle}\ \ {\isasymcirc}\isactrlsub c\ ry{\isachardoublequoteclose}\isanewline
\ \ \ \ \ \ \ \ \ \ \ \ \isacommand{unfolding}\isamarkupfalse%
\ {\isasymrho}{\isacharunderscore}{\kern0pt}def\ \ \isacommand{using}\isamarkupfalse%
\ right{\isacharunderscore}{\kern0pt}coproj{\isacharunderscore}{\kern0pt}cfunc{\isacharunderscore}{\kern0pt}coprod\ \isacommand{by}\isamarkupfalse%
\ {\isacharparenleft}{\kern0pt}typecheck{\isacharunderscore}{\kern0pt}cfuncs{\isacharcomma}{\kern0pt}\ presburger{\isacharparenright}{\kern0pt}\isanewline
\ \ \ \ \ \ \ \ \ \ \isacommand{also}\isamarkupfalse%
\ \isacommand{have}\isamarkupfalse%
\ {\isachardoublequoteopen}{\isachardot}{\kern0pt}{\isachardot}{\kern0pt}{\isachardot}{\kern0pt}\ {\isacharequal}{\kern0pt}\ {\isasymlangle}ry{\isacharcomma}{\kern0pt}\ {\isasymf}{\isasymrangle}{\isachardoublequoteclose}\isanewline
\ \ \ \ \ \ \ \ \ \ \ \ \isacommand{by}\isamarkupfalse%
\ {\isacharparenleft}{\kern0pt}typecheck{\isacharunderscore}{\kern0pt}cfuncs{\isacharcomma}{\kern0pt}\ metis\ cart{\isacharunderscore}{\kern0pt}prod{\isacharunderscore}{\kern0pt}extract{\isacharunderscore}{\kern0pt}left\ ry{\isacharunderscore}{\kern0pt}def{\isacharparenright}{\kern0pt}\isanewline
\ \ \ \ \ \ \ \ \ \ \isacommand{then}\isamarkupfalse%
\ \isacommand{show}\isamarkupfalse%
\ {\isacharquery}{\kern0pt}thesis\isanewline
\ \ \ \ \ \ \ \ \ \ \ \ \isacommand{by}\isamarkupfalse%
\ {\isacharparenleft}{\kern0pt}simp\ add{\isacharcolon}{\kern0pt}\ calculation{\isacharparenright}{\kern0pt}\isanewline
\ \ \ \ \ \ \ \ \isacommand{qed}\isamarkupfalse%
\isanewline
\ \ \ \ \ \ \ \ \isacommand{show}\isamarkupfalse%
\ {\isachardoublequoteopen}x\ {\isacharequal}{\kern0pt}\ y{\isachardoublequoteclose}\isanewline
\ \ \ \ \ \ \ \ \ \ \isacommand{using}\isamarkupfalse%
\ {\isasymrho}x\ {\isasymrho}y\ cart{\isacharunderscore}{\kern0pt}prod{\isacharunderscore}{\kern0pt}eq{\isadigit{2}}\ equals\ false{\isacharunderscore}{\kern0pt}func{\isacharunderscore}{\kern0pt}type\ rx{\isacharunderscore}{\kern0pt}def\ ry{\isacharunderscore}{\kern0pt}def\ \isacommand{by}\isamarkupfalse%
\ auto\isanewline
\ \ \ \ \ \ \isacommand{qed}\isamarkupfalse%
\isanewline
\ \ \ \ \isacommand{qed}\isamarkupfalse%
\isanewline
\ \ \isacommand{qed}\isamarkupfalse%
\isanewline
\ \ \isacommand{have}\isamarkupfalse%
\ {\isachardoublequoteopen}surjective\ {\isasymrho}{\isachardoublequoteclose}\isanewline
\ \ \ \ \isacommand{unfolding}\isamarkupfalse%
\ surjective{\isacharunderscore}{\kern0pt}def\isanewline
\ \ \isacommand{proof}\isamarkupfalse%
{\isacharparenleft}{\kern0pt}auto{\isacharparenright}{\kern0pt}\isanewline
\ \ \ \ \isacommand{fix}\isamarkupfalse%
\ y\isanewline
\ \ \ \ \isacommand{assume}\isamarkupfalse%
\ {\isachardoublequoteopen}y\ {\isasymin}\isactrlsub c\ codomain\ {\isasymrho}{\isachardoublequoteclose}\ \isacommand{then}\isamarkupfalse%
\ \isacommand{have}\isamarkupfalse%
\ y{\isacharunderscore}{\kern0pt}type{\isacharbrackleft}{\kern0pt}type{\isacharunderscore}{\kern0pt}rule{\isacharbrackright}{\kern0pt}{\isacharcolon}{\kern0pt}\ {\isachardoublequoteopen}y\ {\isasymin}\isactrlsub c\ X\ {\isasymtimes}\isactrlsub c\ {\isasymOmega}{\isachardoublequoteclose}\isanewline
\ \ \ \ \ \ \isacommand{using}\isamarkupfalse%
\ {\isasymrho}{\isacharunderscore}{\kern0pt}type\ cfunc{\isacharunderscore}{\kern0pt}type{\isacharunderscore}{\kern0pt}def\ \isacommand{by}\isamarkupfalse%
\ fastforce\isanewline
\ \ \ \ \isacommand{then}\isamarkupfalse%
\ \isacommand{obtain}\isamarkupfalse%
\ x\ w\ \isakeyword{where}\ y{\isacharunderscore}{\kern0pt}def{\isacharcolon}{\kern0pt}\ {\isachardoublequoteopen}y\ {\isacharequal}{\kern0pt}\ {\isasymlangle}x{\isacharcomma}{\kern0pt}w{\isasymrangle}\ {\isasymand}\ x\ {\isasymin}\isactrlsub c\ X\ {\isasymand}\ w\ {\isasymin}\isactrlsub c\ {\isasymOmega}{\isachardoublequoteclose}\isanewline
\ \ \ \ \ \ \isacommand{using}\isamarkupfalse%
\ cart{\isacharunderscore}{\kern0pt}prod{\isacharunderscore}{\kern0pt}decomp\ \isacommand{by}\isamarkupfalse%
\ fastforce\isanewline
\ \ \ \ \isacommand{show}\isamarkupfalse%
\ {\isachardoublequoteopen}{\isasymexists}x{\isachardot}{\kern0pt}\ x\ {\isasymin}\isactrlsub c\ domain\ {\isasymrho}\ {\isasymand}\ {\isasymrho}\ {\isasymcirc}\isactrlsub c\ x\ {\isacharequal}{\kern0pt}\ y{\isachardoublequoteclose}\isanewline
\ \ \ \ \isacommand{proof}\isamarkupfalse%
{\isacharparenleft}{\kern0pt}cases\ {\isachardoublequoteopen}w\ {\isacharequal}{\kern0pt}\ {\isasymt}{\isachardoublequoteclose}{\isacharparenright}{\kern0pt}\isanewline
\ \ \ \ \ \ \isacommand{assume}\isamarkupfalse%
\ {\isachardoublequoteopen}w\ {\isacharequal}{\kern0pt}\ {\isasymt}{\isachardoublequoteclose}\isanewline
\ \ \ \ \ \ \isacommand{obtain}\isamarkupfalse%
\ z\ \isakeyword{where}\ z{\isacharunderscore}{\kern0pt}def{\isacharcolon}{\kern0pt}\ {\isachardoublequoteopen}z\ {\isacharequal}{\kern0pt}\ left{\isacharunderscore}{\kern0pt}coproj\ X\ X\ {\isasymcirc}\isactrlsub c\ x{\isachardoublequoteclose}\isanewline
\ \ \ \ \ \ \ \ \isacommand{by}\isamarkupfalse%
\ simp\isanewline
\ \ \ \ \ \ \isacommand{have}\isamarkupfalse%
\ {\isachardoublequoteopen}{\isasymrho}\ {\isasymcirc}\isactrlsub c\ z\ {\isacharequal}{\kern0pt}\ y{\isachardoublequoteclose}\isanewline
\ \ \ \ \ \ \isacommand{proof}\isamarkupfalse%
\ {\isacharminus}{\kern0pt}\ \isanewline
\ \ \ \ \ \ \ \ \isacommand{have}\isamarkupfalse%
\ {\isachardoublequoteopen}{\isasymrho}\ {\isasymcirc}\isactrlsub c\ z\ {\isacharequal}{\kern0pt}\ {\isacharparenleft}{\kern0pt}{\isasymrho}\ {\isasymcirc}\isactrlsub c\ left{\isacharunderscore}{\kern0pt}coproj\ X\ X{\isacharparenright}{\kern0pt}\ {\isasymcirc}\isactrlsub c\ x{\isachardoublequoteclose}\isanewline
\ \ \ \ \ \ \ \ \ \ \isacommand{using}\isamarkupfalse%
\ comp{\isacharunderscore}{\kern0pt}associative{\isadigit{2}}\ y{\isacharunderscore}{\kern0pt}def\ z{\isacharunderscore}{\kern0pt}def\ \isacommand{by}\isamarkupfalse%
\ {\isacharparenleft}{\kern0pt}typecheck{\isacharunderscore}{\kern0pt}cfuncs{\isacharcomma}{\kern0pt}\ blast{\isacharparenright}{\kern0pt}\isanewline
\ \ \ \ \ \ \ \ \isacommand{also}\isamarkupfalse%
\ \isacommand{have}\isamarkupfalse%
\ {\isachardoublequoteopen}{\isachardot}{\kern0pt}{\isachardot}{\kern0pt}{\isachardot}{\kern0pt}\ {\isacharequal}{\kern0pt}\ {\isasymlangle}id\ X{\isacharcomma}{\kern0pt}\ {\isasymt}\ {\isasymcirc}\isactrlsub c\ {\isasymbeta}\isactrlbsub X\isactrlesub {\isasymrangle}\ \ {\isasymcirc}\isactrlsub c\ x{\isachardoublequoteclose}\isanewline
\ \ \ \ \ \ \ \ \ \ \isacommand{unfolding}\isamarkupfalse%
\ {\isasymrho}{\isacharunderscore}{\kern0pt}def\ \ \isacommand{using}\isamarkupfalse%
\ left{\isacharunderscore}{\kern0pt}coproj{\isacharunderscore}{\kern0pt}cfunc{\isacharunderscore}{\kern0pt}coprod\ \isacommand{by}\isamarkupfalse%
\ {\isacharparenleft}{\kern0pt}typecheck{\isacharunderscore}{\kern0pt}cfuncs{\isacharcomma}{\kern0pt}\ presburger{\isacharparenright}{\kern0pt}\ \ \ \ \ \ \ \ \isanewline
\ \ \ \ \ \ \ \ \isacommand{also}\isamarkupfalse%
\ \isacommand{have}\isamarkupfalse%
\ {\isachardoublequoteopen}{\isachardot}{\kern0pt}{\isachardot}{\kern0pt}{\isachardot}{\kern0pt}\ {\isacharequal}{\kern0pt}\ y{\isachardoublequoteclose}\isanewline
\ \ \ \ \ \ \ \ \ \ \isacommand{using}\isamarkupfalse%
\ {\isacartoucheopen}w\ {\isacharequal}{\kern0pt}\ {\isasymt}{\isacartoucheclose}\ cart{\isacharunderscore}{\kern0pt}prod{\isacharunderscore}{\kern0pt}extract{\isacharunderscore}{\kern0pt}left\ y{\isacharunderscore}{\kern0pt}def\ \isacommand{by}\isamarkupfalse%
\ auto\isanewline
\ \ \ \ \ \ \ \ \isacommand{then}\isamarkupfalse%
\ \isacommand{show}\isamarkupfalse%
\ {\isacharquery}{\kern0pt}thesis\isanewline
\ \ \ \ \ \ \ \ \ \ \isacommand{by}\isamarkupfalse%
\ {\isacharparenleft}{\kern0pt}simp\ add{\isacharcolon}{\kern0pt}\ calculation{\isacharparenright}{\kern0pt}\isanewline
\ \ \ \ \ \ \isacommand{qed}\isamarkupfalse%
\isanewline
\ \ \ \ \ \ \isacommand{then}\isamarkupfalse%
\ \isacommand{show}\isamarkupfalse%
\ {\isacharquery}{\kern0pt}thesis\isanewline
\ \ \ \ \ \ \ \ \isacommand{by}\isamarkupfalse%
\ {\isacharparenleft}{\kern0pt}metis\ {\isasymrho}{\isacharunderscore}{\kern0pt}type\ cfunc{\isacharunderscore}{\kern0pt}type{\isacharunderscore}{\kern0pt}def\ codomain{\isacharunderscore}{\kern0pt}comp\ domain{\isacharunderscore}{\kern0pt}comp\ left{\isacharunderscore}{\kern0pt}proj{\isacharunderscore}{\kern0pt}type\ y{\isacharunderscore}{\kern0pt}def\ z{\isacharunderscore}{\kern0pt}def{\isacharparenright}{\kern0pt}\isanewline
\ \ \ \ \isacommand{next}\isamarkupfalse%
\isanewline
\ \ \ \ \ \ \isacommand{assume}\isamarkupfalse%
\ {\isachardoublequoteopen}w\ {\isasymnoteq}\ {\isasymt}{\isachardoublequoteclose}\ \isacommand{then}\isamarkupfalse%
\ \isacommand{have}\isamarkupfalse%
\ {\isachardoublequoteopen}w\ {\isacharequal}{\kern0pt}\ {\isasymf}{\isachardoublequoteclose}\ \ \isanewline
\ \ \ \ \ \ \ \ \isacommand{by}\isamarkupfalse%
\ {\isacharparenleft}{\kern0pt}typecheck{\isacharunderscore}{\kern0pt}cfuncs{\isacharcomma}{\kern0pt}\ meson\ true{\isacharunderscore}{\kern0pt}false{\isacharunderscore}{\kern0pt}only{\isacharunderscore}{\kern0pt}truth{\isacharunderscore}{\kern0pt}values\ y{\isacharunderscore}{\kern0pt}def{\isacharparenright}{\kern0pt}\isanewline
\ \ \ \ \ \ \isacommand{obtain}\isamarkupfalse%
\ z\ \isakeyword{where}\ z{\isacharunderscore}{\kern0pt}def{\isacharcolon}{\kern0pt}\ {\isachardoublequoteopen}z\ {\isacharequal}{\kern0pt}\ right{\isacharunderscore}{\kern0pt}coproj\ X\ X\ {\isasymcirc}\isactrlsub c\ x{\isachardoublequoteclose}\isanewline
\ \ \ \ \ \ \ \ \isacommand{by}\isamarkupfalse%
\ simp\isanewline
\ \ \ \ \ \ \isacommand{have}\isamarkupfalse%
\ {\isachardoublequoteopen}{\isasymrho}\ {\isasymcirc}\isactrlsub c\ z\ {\isacharequal}{\kern0pt}\ y{\isachardoublequoteclose}\isanewline
\ \ \ \ \ \ \isacommand{proof}\isamarkupfalse%
\ {\isacharminus}{\kern0pt}\ \isanewline
\ \ \ \ \ \ \ \ \isacommand{have}\isamarkupfalse%
\ {\isachardoublequoteopen}{\isasymrho}\ {\isasymcirc}\isactrlsub c\ z\ {\isacharequal}{\kern0pt}\ {\isacharparenleft}{\kern0pt}{\isasymrho}\ {\isasymcirc}\isactrlsub c\ right{\isacharunderscore}{\kern0pt}coproj\ X\ X{\isacharparenright}{\kern0pt}\ {\isasymcirc}\isactrlsub c\ x{\isachardoublequoteclose}\isanewline
\ \ \ \ \ \ \ \ \ \ \isacommand{using}\isamarkupfalse%
\ comp{\isacharunderscore}{\kern0pt}associative{\isadigit{2}}\ y{\isacharunderscore}{\kern0pt}def\ z{\isacharunderscore}{\kern0pt}def\ \isacommand{by}\isamarkupfalse%
\ {\isacharparenleft}{\kern0pt}typecheck{\isacharunderscore}{\kern0pt}cfuncs{\isacharcomma}{\kern0pt}\ blast{\isacharparenright}{\kern0pt}\isanewline
\ \ \ \ \ \ \ \ \isacommand{also}\isamarkupfalse%
\ \isacommand{have}\isamarkupfalse%
\ {\isachardoublequoteopen}{\isachardot}{\kern0pt}{\isachardot}{\kern0pt}{\isachardot}{\kern0pt}\ {\isacharequal}{\kern0pt}\ {\isasymlangle}id\ X{\isacharcomma}{\kern0pt}\ {\isasymf}\ {\isasymcirc}\isactrlsub c\ {\isasymbeta}\isactrlbsub X\isactrlesub {\isasymrangle}\ \ {\isasymcirc}\isactrlsub c\ x{\isachardoublequoteclose}\isanewline
\ \ \ \ \ \ \ \ \ \ \isacommand{unfolding}\isamarkupfalse%
\ {\isasymrho}{\isacharunderscore}{\kern0pt}def\ \ \isacommand{using}\isamarkupfalse%
\ right{\isacharunderscore}{\kern0pt}coproj{\isacharunderscore}{\kern0pt}cfunc{\isacharunderscore}{\kern0pt}coprod\ \isacommand{by}\isamarkupfalse%
\ {\isacharparenleft}{\kern0pt}typecheck{\isacharunderscore}{\kern0pt}cfuncs{\isacharcomma}{\kern0pt}\ presburger{\isacharparenright}{\kern0pt}\ \ \ \ \ \ \ \ \isanewline
\ \ \ \ \ \ \ \ \isacommand{also}\isamarkupfalse%
\ \isacommand{have}\isamarkupfalse%
\ {\isachardoublequoteopen}{\isachardot}{\kern0pt}{\isachardot}{\kern0pt}{\isachardot}{\kern0pt}\ {\isacharequal}{\kern0pt}\ y{\isachardoublequoteclose}\isanewline
\ \ \ \ \ \ \ \ \ \ \isacommand{using}\isamarkupfalse%
\ {\isacartoucheopen}w\ {\isacharequal}{\kern0pt}\ {\isasymf}{\isacartoucheclose}\ cart{\isacharunderscore}{\kern0pt}prod{\isacharunderscore}{\kern0pt}extract{\isacharunderscore}{\kern0pt}left\ y{\isacharunderscore}{\kern0pt}def\ \isacommand{by}\isamarkupfalse%
\ auto\isanewline
\ \ \ \ \ \ \ \ \isacommand{then}\isamarkupfalse%
\ \isacommand{show}\isamarkupfalse%
\ {\isacharquery}{\kern0pt}thesis\isanewline
\ \ \ \ \ \ \ \ \ \ \isacommand{by}\isamarkupfalse%
\ {\isacharparenleft}{\kern0pt}simp\ add{\isacharcolon}{\kern0pt}\ calculation{\isacharparenright}{\kern0pt}\isanewline
\ \ \ \ \ \ \isacommand{qed}\isamarkupfalse%
\isanewline
\ \ \ \ \ \ \isacommand{then}\isamarkupfalse%
\ \isacommand{show}\isamarkupfalse%
\ {\isacharquery}{\kern0pt}thesis\isanewline
\ \ \ \ \ \ \ \ \isacommand{by}\isamarkupfalse%
\ {\isacharparenleft}{\kern0pt}metis\ {\isasymrho}{\isacharunderscore}{\kern0pt}type\ cfunc{\isacharunderscore}{\kern0pt}type{\isacharunderscore}{\kern0pt}def\ codomain{\isacharunderscore}{\kern0pt}comp\ domain{\isacharunderscore}{\kern0pt}comp\ right{\isacharunderscore}{\kern0pt}proj{\isacharunderscore}{\kern0pt}type\ y{\isacharunderscore}{\kern0pt}def\ z{\isacharunderscore}{\kern0pt}def{\isacharparenright}{\kern0pt}\isanewline
\ \ \ \ \isacommand{qed}\isamarkupfalse%
\isanewline
\ \ \isacommand{qed}\isamarkupfalse%
\isanewline
\ \ \isacommand{then}\isamarkupfalse%
\ \isacommand{show}\isamarkupfalse%
\ {\isacharquery}{\kern0pt}thesis\isanewline
\ \ \ \ \isacommand{by}\isamarkupfalse%
\ {\isacharparenleft}{\kern0pt}metis\ {\isasymrho}{\isacharunderscore}{\kern0pt}inj\ {\isasymrho}{\isacharunderscore}{\kern0pt}type\ epi{\isacharunderscore}{\kern0pt}mon{\isacharunderscore}{\kern0pt}is{\isacharunderscore}{\kern0pt}iso\ injective{\isacharunderscore}{\kern0pt}imp{\isacharunderscore}{\kern0pt}monomorphism\ is{\isacharunderscore}{\kern0pt}isomorphic{\isacharunderscore}{\kern0pt}def\ surjective{\isacharunderscore}{\kern0pt}is{\isacharunderscore}{\kern0pt}epimorphism{\isacharparenright}{\kern0pt}\isanewline
\isacommand{qed}\isamarkupfalse%
%
\endisatagproof
{\isafoldproof}%
%
\isadelimproof
\isanewline
%
\endisadelimproof
\isanewline
\isacommand{lemma}\isamarkupfalse%
\ oneUone{\isacharunderscore}{\kern0pt}iso{\isacharunderscore}{\kern0pt}{\isasymOmega}{\isacharcolon}{\kern0pt}\isanewline
\ \ {\isachardoublequoteopen}one\ {\isasymCoprod}\ one\ {\isasymcong}\ {\isasymOmega}{\isachardoublequoteclose}\isanewline
%
\isadelimproof
\ \ %
\endisadelimproof
%
\isatagproof
\isacommand{by}\isamarkupfalse%
\ {\isacharparenleft}{\kern0pt}meson\ truth{\isacharunderscore}{\kern0pt}value{\isacharunderscore}{\kern0pt}set{\isacharunderscore}{\kern0pt}iso{\isacharunderscore}{\kern0pt}{\isadigit{1}}u{\isadigit{1}}\ cfunc{\isacharunderscore}{\kern0pt}coprod{\isacharunderscore}{\kern0pt}type\ false{\isacharunderscore}{\kern0pt}func{\isacharunderscore}{\kern0pt}type\ is{\isacharunderscore}{\kern0pt}isomorphic{\isacharunderscore}{\kern0pt}def\ true{\isacharunderscore}{\kern0pt}func{\isacharunderscore}{\kern0pt}type{\isacharparenright}{\kern0pt}%
\endisatagproof
{\isafoldproof}%
%
\isadelimproof
%
\endisadelimproof
%
\begin{isamarkuptext}%
The lemma below is dual to Proposition 2.2.2 in Halvorson.%
\end{isamarkuptext}\isamarkuptrue%
\isacommand{lemma}\isamarkupfalse%
\ {\isachardoublequoteopen}card\ {\isacharbraceleft}{\kern0pt}x{\isachardot}{\kern0pt}\ x\ {\isasymin}\isactrlsub c\ {\isasymOmega}\ {\isasymCoprod}\ {\isasymOmega}{\isacharbraceright}{\kern0pt}\ {\isacharequal}{\kern0pt}\ {\isadigit{4}}{\isachardoublequoteclose}\isanewline
%
\isadelimproof
%
\endisadelimproof
%
\isatagproof
\isacommand{proof}\isamarkupfalse%
\ {\isacharminus}{\kern0pt}\isanewline
\ \ \isanewline
\ \ \isacommand{have}\isamarkupfalse%
\ f{\isadigit{1}}{\isacharcolon}{\kern0pt}\ {\isachardoublequoteopen}{\isacharparenleft}{\kern0pt}left{\isacharunderscore}{\kern0pt}coproj\ {\isasymOmega}\ {\isasymOmega}{\isacharparenright}{\kern0pt}\ {\isasymcirc}\isactrlsub c\ {\isasymt}\ {\isasymnoteq}\ {\isacharparenleft}{\kern0pt}right{\isacharunderscore}{\kern0pt}coproj\ {\isasymOmega}\ {\isasymOmega}{\isacharparenright}{\kern0pt}\ {\isasymcirc}\isactrlsub c\ {\isasymt}{\isachardoublequoteclose}\isanewline
\ \ \ \ \isacommand{by}\isamarkupfalse%
\ {\isacharparenleft}{\kern0pt}typecheck{\isacharunderscore}{\kern0pt}cfuncs{\isacharcomma}{\kern0pt}\ simp\ add{\isacharcolon}{\kern0pt}\ coproducts{\isacharunderscore}{\kern0pt}disjoint{\isacharparenright}{\kern0pt}\isanewline
\ \ \isacommand{have}\isamarkupfalse%
\ f{\isadigit{2}}{\isacharcolon}{\kern0pt}\ {\isachardoublequoteopen}{\isacharparenleft}{\kern0pt}left{\isacharunderscore}{\kern0pt}coproj\ {\isasymOmega}\ {\isasymOmega}{\isacharparenright}{\kern0pt}\ {\isasymcirc}\isactrlsub c\ {\isasymt}\ {\isasymnoteq}\ {\isacharparenleft}{\kern0pt}left{\isacharunderscore}{\kern0pt}coproj\ {\isasymOmega}\ {\isasymOmega}{\isacharparenright}{\kern0pt}\ {\isasymcirc}\isactrlsub c\ {\isasymf}{\isachardoublequoteclose}\isanewline
\ \ \ \ \isacommand{by}\isamarkupfalse%
\ {\isacharparenleft}{\kern0pt}typecheck{\isacharunderscore}{\kern0pt}cfuncs{\isacharcomma}{\kern0pt}\ metis\ cfunc{\isacharunderscore}{\kern0pt}type{\isacharunderscore}{\kern0pt}def\ left{\isacharunderscore}{\kern0pt}coproj{\isacharunderscore}{\kern0pt}are{\isacharunderscore}{\kern0pt}monomorphisms\ monomorphism{\isacharunderscore}{\kern0pt}def\ true{\isacharunderscore}{\kern0pt}false{\isacharunderscore}{\kern0pt}distinct{\isacharparenright}{\kern0pt}\isanewline
\ \ \isacommand{have}\isamarkupfalse%
\ f{\isadigit{3}}{\isacharcolon}{\kern0pt}\ {\isachardoublequoteopen}{\isacharparenleft}{\kern0pt}left{\isacharunderscore}{\kern0pt}coproj\ {\isasymOmega}\ {\isasymOmega}{\isacharparenright}{\kern0pt}\ {\isasymcirc}\isactrlsub c\ {\isasymt}\ {\isasymnoteq}\ {\isacharparenleft}{\kern0pt}right{\isacharunderscore}{\kern0pt}coproj\ {\isasymOmega}\ {\isasymOmega}{\isacharparenright}{\kern0pt}\ {\isasymcirc}\isactrlsub c\ {\isasymf}{\isachardoublequoteclose}\isanewline
\ \ \ \ \isacommand{by}\isamarkupfalse%
\ {\isacharparenleft}{\kern0pt}typecheck{\isacharunderscore}{\kern0pt}cfuncs{\isacharcomma}{\kern0pt}\ simp\ add{\isacharcolon}{\kern0pt}\ coproducts{\isacharunderscore}{\kern0pt}disjoint{\isacharparenright}{\kern0pt}\isanewline
\ \ \isacommand{have}\isamarkupfalse%
\ f{\isadigit{4}}{\isacharcolon}{\kern0pt}\ {\isachardoublequoteopen}{\isacharparenleft}{\kern0pt}right{\isacharunderscore}{\kern0pt}coproj\ {\isasymOmega}\ {\isasymOmega}{\isacharparenright}{\kern0pt}\ {\isasymcirc}\isactrlsub c\ {\isasymt}\ {\isasymnoteq}\ {\isacharparenleft}{\kern0pt}left{\isacharunderscore}{\kern0pt}coproj\ {\isasymOmega}\ {\isasymOmega}{\isacharparenright}{\kern0pt}\ {\isasymcirc}\isactrlsub c\ {\isasymf}{\isachardoublequoteclose}\isanewline
\ \ \ \ \isacommand{by}\isamarkupfalse%
\ {\isacharparenleft}{\kern0pt}typecheck{\isacharunderscore}{\kern0pt}cfuncs{\isacharcomma}{\kern0pt}\ metis\ {\isacharparenleft}{\kern0pt}no{\isacharunderscore}{\kern0pt}types{\isacharparenright}{\kern0pt}\ coproducts{\isacharunderscore}{\kern0pt}disjoint{\isacharparenright}{\kern0pt}\isanewline
\ \ \isacommand{have}\isamarkupfalse%
\ f{\isadigit{5}}{\isacharcolon}{\kern0pt}\ {\isachardoublequoteopen}{\isacharparenleft}{\kern0pt}right{\isacharunderscore}{\kern0pt}coproj\ {\isasymOmega}\ {\isasymOmega}{\isacharparenright}{\kern0pt}\ {\isasymcirc}\isactrlsub c\ {\isasymt}\ {\isasymnoteq}\ {\isacharparenleft}{\kern0pt}right{\isacharunderscore}{\kern0pt}coproj\ {\isasymOmega}\ {\isasymOmega}{\isacharparenright}{\kern0pt}\ {\isasymcirc}\isactrlsub c\ {\isasymf}{\isachardoublequoteclose}\isanewline
\ \ \ \ \isacommand{by}\isamarkupfalse%
\ {\isacharparenleft}{\kern0pt}typecheck{\isacharunderscore}{\kern0pt}cfuncs{\isacharcomma}{\kern0pt}\ metis\ cfunc{\isacharunderscore}{\kern0pt}type{\isacharunderscore}{\kern0pt}def\ monomorphism{\isacharunderscore}{\kern0pt}def\ right{\isacharunderscore}{\kern0pt}coproj{\isacharunderscore}{\kern0pt}are{\isacharunderscore}{\kern0pt}monomorphisms\ true{\isacharunderscore}{\kern0pt}false{\isacharunderscore}{\kern0pt}distinct{\isacharparenright}{\kern0pt}\isanewline
\ \ \isacommand{have}\isamarkupfalse%
\ f{\isadigit{6}}{\isacharcolon}{\kern0pt}\ {\isachardoublequoteopen}{\isacharparenleft}{\kern0pt}left{\isacharunderscore}{\kern0pt}coproj\ {\isasymOmega}\ {\isasymOmega}{\isacharparenright}{\kern0pt}\ {\isasymcirc}\isactrlsub c\ {\isasymf}\ {\isasymnoteq}\ {\isacharparenleft}{\kern0pt}right{\isacharunderscore}{\kern0pt}coproj\ {\isasymOmega}\ {\isasymOmega}{\isacharparenright}{\kern0pt}\ {\isasymcirc}\isactrlsub c\ {\isasymf}{\isachardoublequoteclose}\isanewline
\ \ \ \ \isacommand{by}\isamarkupfalse%
\ {\isacharparenleft}{\kern0pt}typecheck{\isacharunderscore}{\kern0pt}cfuncs{\isacharcomma}{\kern0pt}\ simp\ add{\isacharcolon}{\kern0pt}\ coproducts{\isacharunderscore}{\kern0pt}disjoint{\isacharparenright}{\kern0pt}\isanewline
\ \ \isanewline
\ \ \isacommand{have}\isamarkupfalse%
\ {\isachardoublequoteopen}{\isacharbraceleft}{\kern0pt}x{\isachardot}{\kern0pt}\ x\ {\isasymin}\isactrlsub c\ {\isasymOmega}\ {\isasymCoprod}\ {\isasymOmega}{\isacharbraceright}{\kern0pt}\ {\isacharequal}{\kern0pt}\ {\isacharbraceleft}{\kern0pt}{\isacharparenleft}{\kern0pt}left{\isacharunderscore}{\kern0pt}coproj\ {\isasymOmega}\ {\isasymOmega}{\isacharparenright}{\kern0pt}\ {\isasymcirc}\isactrlsub c\ {\isasymt}\ {\isacharcomma}{\kern0pt}\ {\isacharparenleft}{\kern0pt}right{\isacharunderscore}{\kern0pt}coproj\ {\isasymOmega}\ {\isasymOmega}{\isacharparenright}{\kern0pt}\ {\isasymcirc}\isactrlsub c\ {\isasymt}{\isacharcomma}{\kern0pt}\ {\isacharparenleft}{\kern0pt}left{\isacharunderscore}{\kern0pt}coproj\ {\isasymOmega}\ {\isasymOmega}{\isacharparenright}{\kern0pt}\ {\isasymcirc}\isactrlsub c\ {\isasymf}{\isacharcomma}{\kern0pt}\ {\isacharparenleft}{\kern0pt}right{\isacharunderscore}{\kern0pt}coproj\ {\isasymOmega}\ {\isasymOmega}{\isacharparenright}{\kern0pt}\ {\isasymcirc}\isactrlsub c\ {\isasymf}{\isacharbraceright}{\kern0pt}{\isachardoublequoteclose}\isanewline
\ \ \ \ \isacommand{using}\isamarkupfalse%
\ coprojs{\isacharunderscore}{\kern0pt}jointly{\isacharunderscore}{\kern0pt}surj\ true{\isacharunderscore}{\kern0pt}false{\isacharunderscore}{\kern0pt}only{\isacharunderscore}{\kern0pt}truth{\isacharunderscore}{\kern0pt}values\ \isanewline
\ \ \ \ \isacommand{by}\isamarkupfalse%
\ {\isacharparenleft}{\kern0pt}typecheck{\isacharunderscore}{\kern0pt}cfuncs{\isacharcomma}{\kern0pt}\ auto{\isacharparenright}{\kern0pt}\ \isanewline
\ \ \isacommand{then}\isamarkupfalse%
\ \isacommand{show}\isamarkupfalse%
\ {\isachardoublequoteopen}card\ {\isacharbraceleft}{\kern0pt}x{\isachardot}{\kern0pt}\ x\ {\isasymin}\isactrlsub c\ {\isasymOmega}\ {\isasymCoprod}\ {\isasymOmega}{\isacharbraceright}{\kern0pt}\ {\isacharequal}{\kern0pt}\ {\isadigit{4}}{\isachardoublequoteclose}\isanewline
\ \ \ \ \isacommand{by}\isamarkupfalse%
\ {\isacharparenleft}{\kern0pt}simp\ add{\isacharcolon}{\kern0pt}\ f{\isadigit{1}}\ f{\isadigit{2}}\ f{\isadigit{3}}\ f{\isadigit{4}}\ f{\isadigit{5}}\ f{\isadigit{6}}{\isacharparenright}{\kern0pt}\isanewline
\isacommand{qed}\isamarkupfalse%
%
\endisatagproof
{\isafoldproof}%
%
\isadelimproof
\isanewline
%
\endisadelimproof
%
\isadelimtheory
\isanewline
%
\endisadelimtheory
%
\isatagtheory
\isacommand{end}\isamarkupfalse%
%
\endisatagtheory
{\isafoldtheory}%
%
\isadelimtheory
%
\endisadelimtheory
%
\end{isabellebody}%
\endinput
%:%file=~/ETCS/HOL-ETCS/Coproduct.thy%:%
%:%10=1%:%
%:%11=1%:%
%:%12=2%:%
%:%13=3%:%
%:%27=5%:%
%:%37=8%:%
%:%38=8%:%
%:%40=10%:%
%:%42=11%:%
%:%43=11%:%
%:%44=12%:%
%:%45=13%:%
%:%46=14%:%
%:%47=15%:%
%:%48=16%:%
%:%49=17%:%
%:%50=18%:%
%:%51=19%:%
%:%52=20%:%
%:%53=21%:%
%:%54=22%:%
%:%55=23%:%
%:%56=24%:%
%:%57=25%:%
%:%58=25%:%
%:%59=26%:%
%:%63=30%:%
%:%64=31%:%
%:%65=32%:%
%:%66=32%:%
%:%67=33%:%
%:%68=34%:%
%:%69=35%:%
%:%70=35%:%
%:%71=36%:%
%:%74=37%:%
%:%78=37%:%
%:%79=37%:%
%:%80=38%:%
%:%81=38%:%
%:%82=39%:%
%:%83=39%:%
%:%84=40%:%
%:%85=40%:%
%:%86=41%:%
%:%87=41%:%
%:%88=42%:%
%:%89=42%:%
%:%91=44%:%
%:%92=45%:%
%:%93=45%:%
%:%94=46%:%
%:%95=46%:%
%:%96=47%:%
%:%106=49%:%
%:%108=50%:%
%:%109=50%:%
%:%110=51%:%
%:%111=52%:%
%:%112=53%:%
%:%119=54%:%
%:%120=54%:%
%:%121=55%:%
%:%122=55%:%
%:%123=56%:%
%:%124=56%:%
%:%125=56%:%
%:%126=57%:%
%:%127=57%:%
%:%128=58%:%
%:%129=59%:%
%:%130=59%:%
%:%131=60%:%
%:%132=60%:%
%:%133=60%:%
%:%134=61%:%
%:%135=61%:%
%:%136=62%:%
%:%137=63%:%
%:%138=63%:%
%:%139=64%:%
%:%140=64%:%
%:%141=65%:%
%:%142=65%:%
%:%143=66%:%
%:%144=66%:%
%:%145=67%:%
%:%146=68%:%
%:%147=68%:%
%:%148=69%:%
%:%149=69%:%
%:%150=70%:%
%:%151=70%:%
%:%152=70%:%
%:%153=71%:%
%:%154=71%:%
%:%155=72%:%
%:%156=72%:%
%:%157=73%:%
%:%158=73%:%
%:%159=74%:%
%:%160=74%:%
%:%161=74%:%
%:%162=75%:%
%:%163=75%:%
%:%164=76%:%
%:%165=76%:%
%:%166=76%:%
%:%167=77%:%
%:%168=77%:%
%:%169=77%:%
%:%170=78%:%
%:%171=78%:%
%:%172=79%:%
%:%173=79%:%
%:%174=80%:%
%:%175=81%:%
%:%176=81%:%
%:%177=82%:%
%:%178=82%:%
%:%179=82%:%
%:%180=83%:%
%:%181=83%:%
%:%182=84%:%
%:%183=84%:%
%:%184=84%:%
%:%185=85%:%
%:%186=86%:%
%:%187=86%:%
%:%188=87%:%
%:%189=87%:%
%:%190=88%:%
%:%191=88%:%
%:%192=88%:%
%:%193=89%:%
%:%194=89%:%
%:%195=90%:%
%:%196=90%:%
%:%197=91%:%
%:%198=91%:%
%:%199=92%:%
%:%200=92%:%
%:%201=92%:%
%:%202=93%:%
%:%203=93%:%
%:%204=94%:%
%:%205=94%:%
%:%206=94%:%
%:%207=95%:%
%:%208=95%:%
%:%209=95%:%
%:%210=96%:%
%:%211=96%:%
%:%212=97%:%
%:%213=97%:%
%:%214=98%:%
%:%215=99%:%
%:%216=99%:%
%:%217=100%:%
%:%218=100%:%
%:%219=100%:%
%:%220=101%:%
%:%221=101%:%
%:%222=101%:%
%:%223=102%:%
%:%224=102%:%
%:%225=103%:%
%:%240=105%:%
%:%250=107%:%
%:%251=107%:%
%:%252=108%:%
%:%253=109%:%
%:%260=110%:%
%:%261=110%:%
%:%262=111%:%
%:%263=111%:%
%:%264=112%:%
%:%265=112%:%
%:%266=112%:%
%:%267=113%:%
%:%268=113%:%
%:%269=113%:%
%:%270=114%:%
%:%271=114%:%
%:%272=114%:%
%:%273=115%:%
%:%274=115%:%
%:%275=116%:%
%:%276=116%:%
%:%277=116%:%
%:%278=117%:%
%:%279=117%:%
%:%280=117%:%
%:%281=118%:%
%:%282=118%:%
%:%283=118%:%
%:%284=119%:%
%:%285=120%:%
%:%286=120%:%
%:%287=121%:%
%:%288=121%:%
%:%289=122%:%
%:%290=122%:%
%:%291=123%:%
%:%297=123%:%
%:%300=124%:%
%:%301=125%:%
%:%302=125%:%
%:%303=126%:%
%:%306=127%:%
%:%310=127%:%
%:%311=127%:%
%:%320=129%:%
%:%322=130%:%
%:%323=130%:%
%:%324=131%:%
%:%331=132%:%
%:%332=132%:%
%:%333=133%:%
%:%334=133%:%
%:%335=134%:%
%:%336=134%:%
%:%337=135%:%
%:%338=135%:%
%:%339=136%:%
%:%340=136%:%
%:%341=137%:%
%:%342=137%:%
%:%343=138%:%
%:%344=138%:%
%:%345=138%:%
%:%346=139%:%
%:%347=139%:%
%:%348=140%:%
%:%349=141%:%
%:%350=142%:%
%:%351=142%:%
%:%352=143%:%
%:%353=143%:%
%:%354=144%:%
%:%355=144%:%
%:%356=144%:%
%:%357=145%:%
%:%358=146%:%
%:%359=146%:%
%:%360=146%:%
%:%361=147%:%
%:%362=147%:%
%:%363=147%:%
%:%364=148%:%
%:%365=148%:%
%:%366=149%:%
%:%367=149%:%
%:%368=149%:%
%:%369=150%:%
%:%370=150%:%
%:%371=151%:%
%:%372=151%:%
%:%373=151%:%
%:%374=152%:%
%:%375=152%:%
%:%376=153%:%
%:%377=153%:%
%:%378=153%:%
%:%379=154%:%
%:%380=154%:%
%:%381=154%:%
%:%382=155%:%
%:%392=157%:%
%:%394=158%:%
%:%395=158%:%
%:%396=159%:%
%:%403=160%:%
%:%404=160%:%
%:%405=161%:%
%:%406=161%:%
%:%407=162%:%
%:%408=162%:%
%:%409=162%:%
%:%410=163%:%
%:%411=163%:%
%:%412=164%:%
%:%413=164%:%
%:%414=164%:%
%:%415=165%:%
%:%416=165%:%
%:%417=166%:%
%:%418=166%:%
%:%419=166%:%
%:%420=167%:%
%:%421=167%:%
%:%422=168%:%
%:%423=169%:%
%:%424=169%:%
%:%425=170%:%
%:%426=170%:%
%:%427=171%:%
%:%428=171%:%
%:%429=172%:%
%:%435=172%:%
%:%438=173%:%
%:%439=174%:%
%:%440=174%:%
%:%441=175%:%
%:%448=176%:%
%:%449=176%:%
%:%450=177%:%
%:%451=177%:%
%:%452=178%:%
%:%453=178%:%
%:%454=178%:%
%:%455=179%:%
%:%456=179%:%
%:%457=180%:%
%:%458=180%:%
%:%459=181%:%
%:%460=181%:%
%:%461=182%:%
%:%462=182%:%
%:%463=182%:%
%:%464=183%:%
%:%465=183%:%
%:%466=184%:%
%:%467=185%:%
%:%468=185%:%
%:%469=186%:%
%:%470=186%:%
%:%471=187%:%
%:%472=187%:%
%:%473=188%:%
%:%483=190%:%
%:%485=191%:%
%:%486=191%:%
%:%487=192%:%
%:%488=193%:%
%:%489=194%:%
%:%496=195%:%
%:%497=195%:%
%:%498=196%:%
%:%499=196%:%
%:%500=197%:%
%:%501=197%:%
%:%502=198%:%
%:%503=199%:%
%:%504=199%:%
%:%505=200%:%
%:%506=200%:%
%:%507=201%:%
%:%508=202%:%
%:%509=202%:%
%:%510=203%:%
%:%511=203%:%
%:%512=204%:%
%:%513=204%:%
%:%514=205%:%
%:%515=205%:%
%:%516=205%:%
%:%517=206%:%
%:%518=206%:%
%:%519=207%:%
%:%520=207%:%
%:%521=208%:%
%:%522=208%:%
%:%523=209%:%
%:%524=210%:%
%:%525=210%:%
%:%526=211%:%
%:%527=211%:%
%:%528=212%:%
%:%529=212%:%
%:%530=213%:%
%:%531=213%:%
%:%532=214%:%
%:%533=215%:%
%:%534=215%:%
%:%535=215%:%
%:%536=216%:%
%:%537=216%:%
%:%538=216%:%
%:%539=217%:%
%:%540=217%:%
%:%541=217%:%
%:%542=218%:%
%:%543=218%:%
%:%544=218%:%
%:%545=219%:%
%:%546=219%:%
%:%547=219%:%
%:%548=220%:%
%:%549=220%:%
%:%550=220%:%
%:%551=221%:%
%:%552=221%:%
%:%553=221%:%
%:%554=222%:%
%:%555=222%:%
%:%556=222%:%
%:%557=223%:%
%:%558=223%:%
%:%559=224%:%
%:%560=224%:%
%:%561=225%:%
%:%562=225%:%
%:%563=226%:%
%:%564=227%:%
%:%565=227%:%
%:%566=228%:%
%:%567=228%:%
%:%568=229%:%
%:%569=229%:%
%:%570=230%:%
%:%571=230%:%
%:%572=231%:%
%:%573=231%:%
%:%574=232%:%
%:%575=232%:%
%:%576=232%:%
%:%577=233%:%
%:%578=233%:%
%:%580=235%:%
%:%581=236%:%
%:%582=236%:%
%:%583=237%:%
%:%584=237%:%
%:%585=238%:%
%:%586=238%:%
%:%587=239%:%
%:%588=239%:%
%:%589=240%:%
%:%590=240%:%
%:%591=241%:%
%:%592=241%:%
%:%593=241%:%
%:%594=242%:%
%:%595=242%:%
%:%596=243%:%
%:%597=243%:%
%:%598=243%:%
%:%599=244%:%
%:%600=244%:%
%:%601=245%:%
%:%602=245%:%
%:%603=246%:%
%:%604=246%:%
%:%605=247%:%
%:%606=247%:%
%:%607=248%:%
%:%608=248%:%
%:%609=249%:%
%:%610=249%:%
%:%611=250%:%
%:%612=250%:%
%:%613=251%:%
%:%614=251%:%
%:%615=252%:%
%:%616=252%:%
%:%617=252%:%
%:%618=253%:%
%:%619=253%:%
%:%620=253%:%
%:%621=254%:%
%:%622=254%:%
%:%623=255%:%
%:%624=255%:%
%:%625=256%:%
%:%626=257%:%
%:%627=257%:%
%:%628=258%:%
%:%629=258%:%
%:%630=258%:%
%:%631=259%:%
%:%632=259%:%
%:%633=259%:%
%:%634=260%:%
%:%635=260%:%
%:%636=260%:%
%:%637=261%:%
%:%638=261%:%
%:%639=261%:%
%:%640=262%:%
%:%641=262%:%
%:%642=262%:%
%:%643=263%:%
%:%644=263%:%
%:%645=263%:%
%:%646=264%:%
%:%647=264%:%
%:%648=264%:%
%:%649=265%:%
%:%655=265%:%
%:%658=266%:%
%:%659=267%:%
%:%660=267%:%
%:%661=268%:%
%:%662=269%:%
%:%665=270%:%
%:%669=270%:%
%:%670=270%:%
%:%671=270%:%
%:%676=270%:%
%:%679=271%:%
%:%680=272%:%
%:%681=272%:%
%:%682=273%:%
%:%683=274%:%
%:%684=275%:%
%:%687=276%:%
%:%691=276%:%
%:%692=276%:%
%:%693=277%:%
%:%694=277%:%
%:%695=278%:%
%:%696=278%:%
%:%697=279%:%
%:%698=279%:%
%:%699=280%:%
%:%700=280%:%
%:%701=280%:%
%:%702=281%:%
%:%703=281%:%
%:%704=281%:%
%:%705=282%:%
%:%706=282%:%
%:%707=283%:%
%:%708=283%:%
%:%709=284%:%
%:%710=284%:%
%:%711=284%:%
%:%712=285%:%
%:%713=285%:%
%:%714=286%:%
%:%720=286%:%
%:%723=287%:%
%:%724=288%:%
%:%725=288%:%
%:%726=289%:%
%:%727=290%:%
%:%728=291%:%
%:%731=292%:%
%:%735=292%:%
%:%736=292%:%
%:%737=293%:%
%:%738=293%:%
%:%739=294%:%
%:%740=294%:%
%:%741=295%:%
%:%742=295%:%
%:%743=296%:%
%:%744=296%:%
%:%745=297%:%
%:%746=297%:%
%:%747=298%:%
%:%748=298%:%
%:%749=298%:%
%:%750=299%:%
%:%751=299%:%
%:%752=299%:%
%:%753=300%:%
%:%754=300%:%
%:%755=300%:%
%:%756=301%:%
%:%757=301%:%
%:%758=301%:%
%:%759=302%:%
%:%760=302%:%
%:%761=302%:%
%:%762=303%:%
%:%763=303%:%
%:%764=304%:%
%:%765=304%:%
%:%766=305%:%
%:%767=305%:%
%:%768=305%:%
%:%769=306%:%
%:%770=306%:%
%:%771=307%:%
%:%772=307%:%
%:%773=307%:%
%:%774=308%:%
%:%775=308%:%
%:%776=308%:%
%:%777=309%:%
%:%778=309%:%
%:%779=309%:%
%:%780=310%:%
%:%786=310%:%
%:%789=311%:%
%:%790=312%:%
%:%791=312%:%
%:%792=313%:%
%:%793=314%:%
%:%795=316%:%
%:%798=317%:%
%:%802=317%:%
%:%803=317%:%
%:%808=317%:%
%:%811=318%:%
%:%812=319%:%
%:%813=319%:%
%:%814=320%:%
%:%815=321%:%
%:%816=322%:%
%:%817=323%:%
%:%820=324%:%
%:%824=324%:%
%:%825=324%:%
%:%826=324%:%
%:%831=324%:%
%:%834=325%:%
%:%835=326%:%
%:%836=326%:%
%:%837=327%:%
%:%838=328%:%
%:%841=329%:%
%:%845=329%:%
%:%846=329%:%
%:%851=329%:%
%:%854=330%:%
%:%855=331%:%
%:%856=331%:%
%:%857=332%:%
%:%858=333%:%
%:%865=334%:%
%:%866=334%:%
%:%867=335%:%
%:%868=335%:%
%:%869=336%:%
%:%870=336%:%
%:%871=336%:%
%:%872=337%:%
%:%873=337%:%
%:%874=338%:%
%:%875=338%:%
%:%876=339%:%
%:%877=339%:%
%:%878=340%:%
%:%879=340%:%
%:%880=341%:%
%:%890=343%:%
%:%892=344%:%
%:%893=344%:%
%:%894=345%:%
%:%897=346%:%
%:%901=346%:%
%:%902=346%:%
%:%903=347%:%
%:%904=348%:%
%:%905=349%:%
%:%919=351%:%
%:%929=353%:%
%:%930=353%:%
%:%931=354%:%
%:%932=355%:%
%:%939=356%:%
%:%940=356%:%
%:%941=357%:%
%:%942=357%:%
%:%943=358%:%
%:%944=358%:%
%:%945=358%:%
%:%946=359%:%
%:%947=359%:%
%:%948=359%:%
%:%949=360%:%
%:%950=361%:%
%:%951=361%:%
%:%952=362%:%
%:%953=362%:%
%:%954=363%:%
%:%955=363%:%
%:%956=364%:%
%:%957=365%:%
%:%958=365%:%
%:%959=365%:%
%:%960=366%:%
%:%961=366%:%
%:%962=366%:%
%:%963=367%:%
%:%964=367%:%
%:%965=368%:%
%:%966=368%:%
%:%967=368%:%
%:%968=369%:%
%:%969=369%:%
%:%970=370%:%
%:%971=370%:%
%:%972=370%:%
%:%973=371%:%
%:%974=371%:%
%:%975=371%:%
%:%976=372%:%
%:%977=372%:%
%:%978=372%:%
%:%979=373%:%
%:%980=373%:%
%:%981=374%:%
%:%982=374%:%
%:%983=375%:%
%:%984=375%:%
%:%985=376%:%
%:%986=376%:%
%:%987=377%:%
%:%988=377%:%
%:%989=377%:%
%:%990=378%:%
%:%991=378%:%
%:%992=378%:%
%:%993=379%:%
%:%994=379%:%
%:%995=379%:%
%:%996=380%:%
%:%997=380%:%
%:%998=380%:%
%:%999=381%:%
%:%1000=381%:%
%:%1001=382%:%
%:%1002=382%:%
%:%1003=383%:%
%:%1004=383%:%
%:%1005=384%:%
%:%1006=384%:%
%:%1007=385%:%
%:%1008=385%:%
%:%1009=386%:%
%:%1010=386%:%
%:%1011=386%:%
%:%1012=387%:%
%:%1013=387%:%
%:%1014=387%:%
%:%1015=388%:%
%:%1016=388%:%
%:%1017=389%:%
%:%1018=389%:%
%:%1019=390%:%
%:%1020=390%:%
%:%1021=391%:%
%:%1022=392%:%
%:%1023=392%:%
%:%1024=392%:%
%:%1025=393%:%
%:%1026=393%:%
%:%1027=393%:%
%:%1028=394%:%
%:%1029=394%:%
%:%1030=395%:%
%:%1031=395%:%
%:%1032=395%:%
%:%1033=396%:%
%:%1034=396%:%
%:%1035=396%:%
%:%1036=397%:%
%:%1037=397%:%
%:%1038=397%:%
%:%1039=398%:%
%:%1040=398%:%
%:%1041=399%:%
%:%1042=399%:%
%:%1043=400%:%
%:%1044=400%:%
%:%1045=401%:%
%:%1046=401%:%
%:%1047=402%:%
%:%1048=402%:%
%:%1049=402%:%
%:%1050=403%:%
%:%1051=403%:%
%:%1052=404%:%
%:%1053=405%:%
%:%1054=405%:%
%:%1055=406%:%
%:%1056=406%:%
%:%1057=407%:%
%:%1058=407%:%
%:%1059=408%:%
%:%1060=409%:%
%:%1061=409%:%
%:%1062=409%:%
%:%1063=410%:%
%:%1064=410%:%
%:%1065=410%:%
%:%1066=411%:%
%:%1067=411%:%
%:%1068=412%:%
%:%1069=412%:%
%:%1070=412%:%
%:%1071=413%:%
%:%1072=413%:%
%:%1073=414%:%
%:%1074=414%:%
%:%1075=414%:%
%:%1076=415%:%
%:%1077=415%:%
%:%1078=415%:%
%:%1079=416%:%
%:%1080=416%:%
%:%1081=417%:%
%:%1082=417%:%
%:%1083=418%:%
%:%1089=418%:%
%:%1092=419%:%
%:%1093=420%:%
%:%1094=420%:%
%:%1095=421%:%
%:%1096=422%:%
%:%1103=423%:%
%:%1104=423%:%
%:%1105=424%:%
%:%1106=424%:%
%:%1107=425%:%
%:%1108=425%:%
%:%1109=425%:%
%:%1110=426%:%
%:%1111=426%:%
%:%1112=426%:%
%:%1113=427%:%
%:%1114=427%:%
%:%1115=428%:%
%:%1116=428%:%
%:%1117=429%:%
%:%1118=429%:%
%:%1119=430%:%
%:%1120=431%:%
%:%1121=431%:%
%:%1122=431%:%
%:%1123=432%:%
%:%1124=432%:%
%:%1125=432%:%
%:%1126=433%:%
%:%1127=433%:%
%:%1128=434%:%
%:%1129=434%:%
%:%1130=434%:%
%:%1131=435%:%
%:%1132=435%:%
%:%1133=436%:%
%:%1134=436%:%
%:%1135=436%:%
%:%1136=437%:%
%:%1137=437%:%
%:%1138=437%:%
%:%1139=438%:%
%:%1140=438%:%
%:%1141=438%:%
%:%1142=439%:%
%:%1143=439%:%
%:%1144=439%:%
%:%1145=440%:%
%:%1146=440%:%
%:%1147=441%:%
%:%1148=441%:%
%:%1149=442%:%
%:%1150=442%:%
%:%1151=443%:%
%:%1152=443%:%
%:%1153=443%:%
%:%1154=444%:%
%:%1155=444%:%
%:%1156=444%:%
%:%1157=445%:%
%:%1158=445%:%
%:%1159=445%:%
%:%1160=446%:%
%:%1161=446%:%
%:%1162=446%:%
%:%1163=447%:%
%:%1164=448%:%
%:%1165=448%:%
%:%1166=449%:%
%:%1167=449%:%
%:%1168=450%:%
%:%1169=450%:%
%:%1170=451%:%
%:%1171=451%:%
%:%1172=452%:%
%:%1173=452%:%
%:%1174=453%:%
%:%1175=453%:%
%:%1176=453%:%
%:%1177=454%:%
%:%1178=454%:%
%:%1179=454%:%
%:%1180=455%:%
%:%1181=456%:%
%:%1182=456%:%
%:%1183=457%:%
%:%1184=457%:%
%:%1185=458%:%
%:%1186=458%:%
%:%1187=459%:%
%:%1188=460%:%
%:%1189=460%:%
%:%1190=461%:%
%:%1191=461%:%
%:%1192=461%:%
%:%1193=462%:%
%:%1194=462%:%
%:%1195=462%:%
%:%1196=463%:%
%:%1197=463%:%
%:%1198=463%:%
%:%1199=464%:%
%:%1200=464%:%
%:%1201=464%:%
%:%1202=465%:%
%:%1203=465%:%
%:%1204=465%:%
%:%1205=466%:%
%:%1206=466%:%
%:%1207=466%:%
%:%1208=467%:%
%:%1209=467%:%
%:%1210=468%:%
%:%1211=468%:%
%:%1212=469%:%
%:%1213=469%:%
%:%1214=470%:%
%:%1215=470%:%
%:%1216=470%:%
%:%1217=471%:%
%:%1218=471%:%
%:%1219=472%:%
%:%1220=472%:%
%:%1221=473%:%
%:%1222=473%:%
%:%1223=474%:%
%:%1224=474%:%
%:%1225=475%:%
%:%1226=476%:%
%:%1227=476%:%
%:%1228=476%:%
%:%1229=477%:%
%:%1230=477%:%
%:%1231=477%:%
%:%1232=478%:%
%:%1233=478%:%
%:%1234=479%:%
%:%1235=479%:%
%:%1236=479%:%
%:%1237=480%:%
%:%1238=480%:%
%:%1239=481%:%
%:%1240=481%:%
%:%1241=481%:%
%:%1242=482%:%
%:%1243=482%:%
%:%1244=482%:%
%:%1245=483%:%
%:%1246=483%:%
%:%1247=484%:%
%:%1248=484%:%
%:%1249=485%:%
%:%1264=487%:%
%:%1274=489%:%
%:%1275=489%:%
%:%1276=490%:%
%:%1277=491%:%
%:%1278=492%:%
%:%1279=492%:%
%:%1280=493%:%
%:%1281=494%:%
%:%1284=495%:%
%:%1288=495%:%
%:%1289=495%:%
%:%1290=495%:%
%:%1295=495%:%
%:%1298=496%:%
%:%1299=497%:%
%:%1300=497%:%
%:%1301=498%:%
%:%1304=499%:%
%:%1308=499%:%
%:%1309=499%:%
%:%1310=500%:%
%:%1311=500%:%
%:%1312=500%:%
%:%1317=500%:%
%:%1320=501%:%
%:%1321=502%:%
%:%1322=502%:%
%:%1323=503%:%
%:%1326=504%:%
%:%1330=504%:%
%:%1331=504%:%
%:%1332=505%:%
%:%1333=505%:%
%:%1338=505%:%
%:%1341=506%:%
%:%1342=507%:%
%:%1343=507%:%
%:%1344=508%:%
%:%1347=509%:%
%:%1351=509%:%
%:%1352=509%:%
%:%1353=510%:%
%:%1354=510%:%
%:%1359=510%:%
%:%1362=511%:%
%:%1363=512%:%
%:%1364=512%:%
%:%1366=514%:%
%:%1369=515%:%
%:%1373=515%:%
%:%1374=515%:%
%:%1375=516%:%
%:%1376=516%:%
%:%1377=516%:%
%:%1386=518%:%
%:%1388=519%:%
%:%1389=519%:%
%:%1390=520%:%
%:%1391=521%:%
%:%1398=522%:%
%:%1399=522%:%
%:%1400=523%:%
%:%1401=523%:%
%:%1402=524%:%
%:%1403=524%:%
%:%1406=527%:%
%:%1407=528%:%
%:%1408=528%:%
%:%1409=528%:%
%:%1410=529%:%
%:%1411=530%:%
%:%1412=530%:%
%:%1413=531%:%
%:%1414=531%:%
%:%1415=532%:%
%:%1416=532%:%
%:%1417=533%:%
%:%1418=533%:%
%:%1419=534%:%
%:%1420=535%:%
%:%1421=535%:%
%:%1422=536%:%
%:%1423=536%:%
%:%1424=536%:%
%:%1425=537%:%
%:%1431=537%:%
%:%1434=538%:%
%:%1435=539%:%
%:%1436=539%:%
%:%1437=540%:%
%:%1440=541%:%
%:%1444=541%:%
%:%1445=541%:%
%:%1450=541%:%
%:%1453=542%:%
%:%1454=543%:%
%:%1455=543%:%
%:%1456=544%:%
%:%1457=545%:%
%:%1458=546%:%
%:%1465=547%:%
%:%1466=547%:%
%:%1467=548%:%
%:%1468=548%:%
%:%1469=548%:%
%:%1470=549%:%
%:%1471=550%:%
%:%1472=551%:%
%:%1473=551%:%
%:%1474=551%:%
%:%1475=552%:%
%:%1476=553%:%
%:%1477=553%:%
%:%1478=554%:%
%:%1479=554%:%
%:%1480=555%:%
%:%1481=555%:%
%:%1482=556%:%
%:%1483=556%:%
%:%1484=556%:%
%:%1485=557%:%
%:%1486=557%:%
%:%1487=557%:%
%:%1488=558%:%
%:%1489=558%:%
%:%1490=558%:%
%:%1491=559%:%
%:%1492=559%:%
%:%1493=559%:%
%:%1494=560%:%
%:%1495=560%:%
%:%1496=560%:%
%:%1497=561%:%
%:%1498=561%:%
%:%1499=561%:%
%:%1500=562%:%
%:%1501=562%:%
%:%1502=562%:%
%:%1503=563%:%
%:%1504=563%:%
%:%1505=563%:%
%:%1506=564%:%
%:%1507=564%:%
%:%1508=565%:%
%:%1509=565%:%
%:%1510=566%:%
%:%1511=567%:%
%:%1512=567%:%
%:%1513=568%:%
%:%1514=568%:%
%:%1515=569%:%
%:%1516=569%:%
%:%1517=570%:%
%:%1518=570%:%
%:%1519=570%:%
%:%1520=571%:%
%:%1521=571%:%
%:%1522=571%:%
%:%1523=572%:%
%:%1524=572%:%
%:%1525=572%:%
%:%1526=573%:%
%:%1527=573%:%
%:%1528=573%:%
%:%1529=574%:%
%:%1530=574%:%
%:%1531=574%:%
%:%1532=575%:%
%:%1533=575%:%
%:%1534=575%:%
%:%1535=576%:%
%:%1536=576%:%
%:%1537=576%:%
%:%1538=577%:%
%:%1539=577%:%
%:%1540=577%:%
%:%1541=578%:%
%:%1542=578%:%
%:%1543=579%:%
%:%1544=579%:%
%:%1545=580%:%
%:%1546=581%:%
%:%1547=581%:%
%:%1548=582%:%
%:%1549=582%:%
%:%1550=583%:%
%:%1551=583%:%
%:%1552=584%:%
%:%1558=584%:%
%:%1561=585%:%
%:%1562=586%:%
%:%1563=586%:%
%:%1566=587%:%
%:%1570=587%:%
%:%1571=587%:%
%:%1576=587%:%
%:%1579=588%:%
%:%1580=589%:%
%:%1581=589%:%
%:%1582=590%:%
%:%1583=591%:%
%:%1590=592%:%
%:%1591=592%:%
%:%1592=593%:%
%:%1593=593%:%
%:%1594=594%:%
%:%1595=595%:%
%:%1596=595%:%
%:%1597=595%:%
%:%1598=596%:%
%:%1599=596%:%
%:%1600=596%:%
%:%1601=597%:%
%:%1602=597%:%
%:%1603=597%:%
%:%1604=598%:%
%:%1605=598%:%
%:%1606=598%:%
%:%1607=599%:%
%:%1608=599%:%
%:%1609=599%:%
%:%1610=600%:%
%:%1611=600%:%
%:%1612=600%:%
%:%1613=601%:%
%:%1614=601%:%
%:%1615=601%:%
%:%1616=602%:%
%:%1617=602%:%
%:%1618=602%:%
%:%1619=603%:%
%:%1620=603%:%
%:%1621=603%:%
%:%1622=604%:%
%:%1623=604%:%
%:%1624=604%:%
%:%1625=605%:%
%:%1626=605%:%
%:%1627=605%:%
%:%1628=606%:%
%:%1634=606%:%
%:%1637=607%:%
%:%1638=608%:%
%:%1639=608%:%
%:%1640=609%:%
%:%1641=610%:%
%:%1642=611%:%
%:%1645=612%:%
%:%1649=612%:%
%:%1650=612%:%
%:%1651=613%:%
%:%1652=613%:%
%:%1653=614%:%
%:%1654=614%:%
%:%1655=615%:%
%:%1656=615%:%
%:%1657=616%:%
%:%1658=616%:%
%:%1659=617%:%
%:%1660=617%:%
%:%1661=618%:%
%:%1662=619%:%
%:%1663=619%:%
%:%1664=620%:%
%:%1665=620%:%
%:%1666=620%:%
%:%1667=621%:%
%:%1668=621%:%
%:%1669=622%:%
%:%1670=622%:%
%:%1671=622%:%
%:%1672=623%:%
%:%1673=624%:%
%:%1674=624%:%
%:%1675=625%:%
%:%1676=625%:%
%:%1677=626%:%
%:%1678=626%:%
%:%1679=627%:%
%:%1680=627%:%
%:%1681=627%:%
%:%1682=628%:%
%:%1683=628%:%
%:%1684=628%:%
%:%1685=629%:%
%:%1686=629%:%
%:%1687=629%:%
%:%1688=630%:%
%:%1689=630%:%
%:%1690=630%:%
%:%1691=631%:%
%:%1692=631%:%
%:%1693=631%:%
%:%1694=632%:%
%:%1695=632%:%
%:%1696=632%:%
%:%1697=633%:%
%:%1698=633%:%
%:%1699=633%:%
%:%1700=634%:%
%:%1701=634%:%
%:%1702=634%:%
%:%1703=635%:%
%:%1704=635%:%
%:%1705=635%:%
%:%1706=636%:%
%:%1707=636%:%
%:%1708=636%:%
%:%1709=637%:%
%:%1710=637%:%
%:%1711=637%:%
%:%1712=638%:%
%:%1713=638%:%
%:%1714=638%:%
%:%1715=639%:%
%:%1716=639%:%
%:%1717=639%:%
%:%1718=640%:%
%:%1719=640%:%
%:%1720=640%:%
%:%1721=641%:%
%:%1722=641%:%
%:%1723=641%:%
%:%1724=642%:%
%:%1725=642%:%
%:%1726=643%:%
%:%1727=643%:%
%:%1728=644%:%
%:%1729=644%:%
%:%1730=644%:%
%:%1731=645%:%
%:%1732=645%:%
%:%1733=645%:%
%:%1734=646%:%
%:%1735=646%:%
%:%1736=646%:%
%:%1737=647%:%
%:%1738=647%:%
%:%1739=647%:%
%:%1740=648%:%
%:%1741=648%:%
%:%1742=648%:%
%:%1743=649%:%
%:%1744=649%:%
%:%1745=649%:%
%:%1746=650%:%
%:%1747=650%:%
%:%1748=650%:%
%:%1749=651%:%
%:%1750=651%:%
%:%1751=651%:%
%:%1752=652%:%
%:%1753=652%:%
%:%1754=652%:%
%:%1755=653%:%
%:%1756=653%:%
%:%1757=653%:%
%:%1758=654%:%
%:%1759=654%:%
%:%1760=654%:%
%:%1761=655%:%
%:%1762=655%:%
%:%1763=655%:%
%:%1764=656%:%
%:%1765=656%:%
%:%1766=656%:%
%:%1767=657%:%
%:%1768=657%:%
%:%1769=657%:%
%:%1770=658%:%
%:%1771=658%:%
%:%1772=658%:%
%:%1773=659%:%
%:%1774=659%:%
%:%1775=660%:%
%:%1776=660%:%
%:%1777=660%:%
%:%1778=661%:%
%:%1779=661%:%
%:%1780=662%:%
%:%1786=662%:%
%:%1789=663%:%
%:%1790=664%:%
%:%1791=664%:%
%:%1792=665%:%
%:%1793=666%:%
%:%1794=667%:%
%:%1797=668%:%
%:%1801=668%:%
%:%1802=668%:%
%:%1803=669%:%
%:%1804=669%:%
%:%1805=670%:%
%:%1806=670%:%
%:%1807=671%:%
%:%1808=671%:%
%:%1809=672%:%
%:%1810=672%:%
%:%1811=673%:%
%:%1812=673%:%
%:%1813=674%:%
%:%1814=675%:%
%:%1815=675%:%
%:%1816=676%:%
%:%1817=676%:%
%:%1818=677%:%
%:%1819=678%:%
%:%1820=678%:%
%:%1821=679%:%
%:%1822=679%:%
%:%1823=679%:%
%:%1824=680%:%
%:%1825=680%:%
%:%1826=681%:%
%:%1827=681%:%
%:%1828=681%:%
%:%1829=682%:%
%:%1830=683%:%
%:%1831=683%:%
%:%1832=684%:%
%:%1833=685%:%
%:%1834=685%:%
%:%1835=686%:%
%:%1836=687%:%
%:%1837=687%:%
%:%1838=688%:%
%:%1839=689%:%
%:%1840=689%:%
%:%1841=690%:%
%:%1842=691%:%
%:%1843=691%:%
%:%1844=692%:%
%:%1845=692%:%
%:%1846=693%:%
%:%1847=693%:%
%:%1848=694%:%
%:%1849=694%:%
%:%1850=695%:%
%:%1851=695%:%
%:%1852=695%:%
%:%1853=696%:%
%:%1854=696%:%
%:%1855=697%:%
%:%1856=697%:%
%:%1857=698%:%
%:%1858=698%:%
%:%1859=699%:%
%:%1860=699%:%
%:%1861=700%:%
%:%1862=700%:%
%:%1863=701%:%
%:%1864=701%:%
%:%1865=702%:%
%:%1866=702%:%
%:%1867=702%:%
%:%1868=703%:%
%:%1869=703%:%
%:%1870=704%:%
%:%1871=704%:%
%:%1872=705%:%
%:%1873=705%:%
%:%1874=706%:%
%:%1875=706%:%
%:%1876=706%:%
%:%1877=707%:%
%:%1878=707%:%
%:%1879=707%:%
%:%1880=708%:%
%:%1881=709%:%
%:%1882=709%:%
%:%1883=709%:%
%:%1884=710%:%
%:%1885=710%:%
%:%1886=710%:%
%:%1887=711%:%
%:%1888=711%:%
%:%1889=711%:%
%:%1890=712%:%
%:%1891=712%:%
%:%1892=712%:%
%:%1893=713%:%
%:%1894=713%:%
%:%1895=713%:%
%:%1896=714%:%
%:%1897=714%:%
%:%1898=714%:%
%:%1899=715%:%
%:%1900=715%:%
%:%1901=716%:%
%:%1902=716%:%
%:%1903=716%:%
%:%1904=717%:%
%:%1905=717%:%
%:%1906=717%:%
%:%1907=718%:%
%:%1908=718%:%
%:%1909=718%:%
%:%1910=719%:%
%:%1911=719%:%
%:%1912=720%:%
%:%1913=720%:%
%:%1914=720%:%
%:%1915=721%:%
%:%1916=721%:%
%:%1917=721%:%
%:%1918=722%:%
%:%1919=722%:%
%:%1920=722%:%
%:%1921=723%:%
%:%1922=723%:%
%:%1923=724%:%
%:%1924=724%:%
%:%1925=724%:%
%:%1926=725%:%
%:%1927=725%:%
%:%1928=726%:%
%:%1929=726%:%
%:%1930=727%:%
%:%1931=727%:%
%:%1932=727%:%
%:%1933=728%:%
%:%1934=728%:%
%:%1935=729%:%
%:%1936=729%:%
%:%1937=730%:%
%:%1938=730%:%
%:%1939=730%:%
%:%1940=731%:%
%:%1941=731%:%
%:%1942=732%:%
%:%1943=732%:%
%:%1944=733%:%
%:%1945=733%:%
%:%1946=734%:%
%:%1947=734%:%
%:%1948=735%:%
%:%1949=735%:%
%:%1950=735%:%
%:%1951=736%:%
%:%1952=736%:%
%:%1953=736%:%
%:%1954=737%:%
%:%1955=737%:%
%:%1956=738%:%
%:%1957=738%:%
%:%1958=738%:%
%:%1959=739%:%
%:%1960=739%:%
%:%1961=739%:%
%:%1962=740%:%
%:%1963=741%:%
%:%1964=741%:%
%:%1965=741%:%
%:%1966=742%:%
%:%1967=742%:%
%:%1968=742%:%
%:%1969=743%:%
%:%1970=743%:%
%:%1971=743%:%
%:%1972=744%:%
%:%1973=744%:%
%:%1974=744%:%
%:%1975=745%:%
%:%1976=745%:%
%:%1977=745%:%
%:%1978=746%:%
%:%1979=746%:%
%:%1980=746%:%
%:%1981=747%:%
%:%1982=747%:%
%:%1983=748%:%
%:%1984=748%:%
%:%1985=748%:%
%:%1986=749%:%
%:%1987=749%:%
%:%1988=749%:%
%:%1989=750%:%
%:%1990=750%:%
%:%1991=750%:%
%:%1992=751%:%
%:%1993=751%:%
%:%1994=752%:%
%:%1995=752%:%
%:%1996=752%:%
%:%1997=753%:%
%:%1998=753%:%
%:%1999=753%:%
%:%2000=754%:%
%:%2001=754%:%
%:%2002=754%:%
%:%2003=755%:%
%:%2004=755%:%
%:%2005=755%:%
%:%2006=756%:%
%:%2007=756%:%
%:%2008=756%:%
%:%2009=757%:%
%:%2010=757%:%
%:%2011=757%:%
%:%2012=758%:%
%:%2013=758%:%
%:%2014=758%:%
%:%2015=759%:%
%:%2016=759%:%
%:%2017=760%:%
%:%2018=760%:%
%:%2019=761%:%
%:%2020=761%:%
%:%2021=762%:%
%:%2022=762%:%
%:%2023=763%:%
%:%2024=763%:%
%:%2025=763%:%
%:%2026=764%:%
%:%2027=764%:%
%:%2028=764%:%
%:%2029=765%:%
%:%2030=765%:%
%:%2031=766%:%
%:%2032=766%:%
%:%2033=767%:%
%:%2034=767%:%
%:%2035=768%:%
%:%2036=768%:%
%:%2037=769%:%
%:%2038=769%:%
%:%2039=770%:%
%:%2040=770%:%
%:%2041=771%:%
%:%2042=771%:%
%:%2043=771%:%
%:%2044=772%:%
%:%2045=772%:%
%:%2046=773%:%
%:%2047=773%:%
%:%2048=773%:%
%:%2049=774%:%
%:%2050=774%:%
%:%2051=774%:%
%:%2052=775%:%
%:%2053=776%:%
%:%2054=776%:%
%:%2055=776%:%
%:%2056=777%:%
%:%2057=777%:%
%:%2058=777%:%
%:%2059=778%:%
%:%2060=778%:%
%:%2061=778%:%
%:%2062=779%:%
%:%2063=779%:%
%:%2064=779%:%
%:%2065=780%:%
%:%2066=780%:%
%:%2067=780%:%
%:%2068=781%:%
%:%2069=781%:%
%:%2070=781%:%
%:%2071=782%:%
%:%2072=782%:%
%:%2073=783%:%
%:%2074=783%:%
%:%2075=783%:%
%:%2076=784%:%
%:%2077=784%:%
%:%2078=784%:%
%:%2079=785%:%
%:%2080=785%:%
%:%2081=785%:%
%:%2082=786%:%
%:%2083=786%:%
%:%2084=787%:%
%:%2085=787%:%
%:%2086=787%:%
%:%2087=788%:%
%:%2088=788%:%
%:%2089=788%:%
%:%2090=789%:%
%:%2091=789%:%
%:%2092=789%:%
%:%2093=790%:%
%:%2094=790%:%
%:%2095=790%:%
%:%2096=791%:%
%:%2097=791%:%
%:%2098=791%:%
%:%2099=792%:%
%:%2100=792%:%
%:%2101=792%:%
%:%2102=793%:%
%:%2103=793%:%
%:%2104=793%:%
%:%2105=794%:%
%:%2106=794%:%
%:%2107=795%:%
%:%2108=795%:%
%:%2109=796%:%
%:%2110=796%:%
%:%2111=797%:%
%:%2112=797%:%
%:%2113=798%:%
%:%2114=798%:%
%:%2115=798%:%
%:%2116=799%:%
%:%2117=799%:%
%:%2118=799%:%
%:%2119=800%:%
%:%2120=800%:%
%:%2121=801%:%
%:%2122=801%:%
%:%2123=802%:%
%:%2124=802%:%
%:%2125=803%:%
%:%2126=803%:%
%:%2127=803%:%
%:%2128=804%:%
%:%2129=804%:%
%:%2130=804%:%
%:%2131=805%:%
%:%2132=806%:%
%:%2133=806%:%
%:%2134=806%:%
%:%2135=807%:%
%:%2136=807%:%
%:%2137=807%:%
%:%2138=808%:%
%:%2139=808%:%
%:%2140=808%:%
%:%2141=809%:%
%:%2142=809%:%
%:%2143=809%:%
%:%2144=810%:%
%:%2145=810%:%
%:%2146=810%:%
%:%2147=811%:%
%:%2148=811%:%
%:%2149=811%:%
%:%2150=812%:%
%:%2151=812%:%
%:%2152=813%:%
%:%2153=813%:%
%:%2154=813%:%
%:%2155=814%:%
%:%2156=814%:%
%:%2157=814%:%
%:%2158=815%:%
%:%2159=815%:%
%:%2160=815%:%
%:%2161=816%:%
%:%2162=816%:%
%:%2163=817%:%
%:%2164=817%:%
%:%2165=817%:%
%:%2166=818%:%
%:%2167=818%:%
%:%2168=818%:%
%:%2169=819%:%
%:%2170=819%:%
%:%2171=819%:%
%:%2172=820%:%
%:%2173=820%:%
%:%2174=820%:%
%:%2175=821%:%
%:%2176=821%:%
%:%2177=821%:%
%:%2178=822%:%
%:%2179=822%:%
%:%2180=823%:%
%:%2181=823%:%
%:%2182=824%:%
%:%2183=824%:%
%:%2184=824%:%
%:%2185=825%:%
%:%2186=825%:%
%:%2187=826%:%
%:%2188=826%:%
%:%2189=827%:%
%:%2190=827%:%
%:%2191=827%:%
%:%2192=828%:%
%:%2193=828%:%
%:%2194=829%:%
%:%2195=829%:%
%:%2196=830%:%
%:%2197=830%:%
%:%2198=831%:%
%:%2204=831%:%
%:%2207=832%:%
%:%2208=833%:%
%:%2209=833%:%
%:%2210=834%:%
%:%2211=835%:%
%:%2212=836%:%
%:%2215=837%:%
%:%2219=837%:%
%:%2220=837%:%
%:%2221=838%:%
%:%2222=838%:%
%:%2223=839%:%
%:%2224=839%:%
%:%2225=840%:%
%:%2226=840%:%
%:%2227=841%:%
%:%2228=841%:%
%:%2229=842%:%
%:%2230=842%:%
%:%2231=843%:%
%:%2232=844%:%
%:%2233=844%:%
%:%2234=845%:%
%:%2235=845%:%
%:%2236=845%:%
%:%2237=846%:%
%:%2238=846%:%
%:%2239=847%:%
%:%2240=847%:%
%:%2241=847%:%
%:%2242=848%:%
%:%2243=848%:%
%:%2244=849%:%
%:%2245=849%:%
%:%2246=849%:%
%:%2247=850%:%
%:%2248=850%:%
%:%2249=851%:%
%:%2250=851%:%
%:%2251=852%:%
%:%2252=852%:%
%:%2253=853%:%
%:%2254=853%:%
%:%2255=853%:%
%:%2256=854%:%
%:%2257=854%:%
%:%2258=854%:%
%:%2259=855%:%
%:%2260=855%:%
%:%2261=855%:%
%:%2262=856%:%
%:%2263=856%:%
%:%2264=856%:%
%:%2265=857%:%
%:%2266=857%:%
%:%2267=857%:%
%:%2268=858%:%
%:%2269=858%:%
%:%2270=858%:%
%:%2271=859%:%
%:%2272=859%:%
%:%2273=860%:%
%:%2274=860%:%
%:%2275=860%:%
%:%2276=861%:%
%:%2277=861%:%
%:%2278=861%:%
%:%2279=862%:%
%:%2280=862%:%
%:%2281=862%:%
%:%2282=863%:%
%:%2283=863%:%
%:%2284=863%:%
%:%2285=864%:%
%:%2286=864%:%
%:%2287=864%:%
%:%2288=865%:%
%:%2289=865%:%
%:%2290=865%:%
%:%2291=866%:%
%:%2292=866%:%
%:%2293=866%:%
%:%2294=866%:%
%:%2295=866%:%
%:%2296=867%:%
%:%2297=867%:%
%:%2298=868%:%
%:%2299=868%:%
%:%2300=868%:%
%:%2301=869%:%
%:%2302=869%:%
%:%2303=869%:%
%:%2304=870%:%
%:%2305=870%:%
%:%2306=870%:%
%:%2307=871%:%
%:%2308=871%:%
%:%2309=872%:%
%:%2310=872%:%
%:%2311=872%:%
%:%2312=873%:%
%:%2313=873%:%
%:%2314=873%:%
%:%2315=874%:%
%:%2316=874%:%
%:%2317=875%:%
%:%2318=875%:%
%:%2319=876%:%
%:%2320=876%:%
%:%2321=877%:%
%:%2322=877%:%
%:%2323=878%:%
%:%2324=878%:%
%:%2325=879%:%
%:%2326=880%:%
%:%2327=880%:%
%:%2328=881%:%
%:%2329=881%:%
%:%2330=881%:%
%:%2331=882%:%
%:%2332=882%:%
%:%2333=883%:%
%:%2334=883%:%
%:%2335=883%:%
%:%2336=884%:%
%:%2337=884%:%
%:%2338=885%:%
%:%2339=885%:%
%:%2340=885%:%
%:%2341=886%:%
%:%2342=886%:%
%:%2343=887%:%
%:%2344=887%:%
%:%2345=888%:%
%:%2346=888%:%
%:%2347=889%:%
%:%2348=889%:%
%:%2349=889%:%
%:%2350=890%:%
%:%2351=890%:%
%:%2352=890%:%
%:%2353=891%:%
%:%2354=891%:%
%:%2355=891%:%
%:%2356=892%:%
%:%2357=892%:%
%:%2358=892%:%
%:%2359=893%:%
%:%2360=893%:%
%:%2361=893%:%
%:%2362=894%:%
%:%2363=894%:%
%:%2364=894%:%
%:%2365=895%:%
%:%2366=895%:%
%:%2367=896%:%
%:%2368=896%:%
%:%2369=896%:%
%:%2370=897%:%
%:%2371=897%:%
%:%2372=897%:%
%:%2373=898%:%
%:%2374=898%:%
%:%2375=898%:%
%:%2376=899%:%
%:%2377=899%:%
%:%2378=899%:%
%:%2379=900%:%
%:%2380=900%:%
%:%2381=900%:%
%:%2382=901%:%
%:%2383=901%:%
%:%2384=901%:%
%:%2385=902%:%
%:%2386=902%:%
%:%2387=902%:%
%:%2388=902%:%
%:%2389=902%:%
%:%2390=903%:%
%:%2391=903%:%
%:%2392=904%:%
%:%2393=904%:%
%:%2394=904%:%
%:%2395=905%:%
%:%2396=905%:%
%:%2397=905%:%
%:%2398=906%:%
%:%2399=906%:%
%:%2400=906%:%
%:%2401=907%:%
%:%2402=907%:%
%:%2403=908%:%
%:%2404=908%:%
%:%2405=908%:%
%:%2406=909%:%
%:%2407=909%:%
%:%2408=909%:%
%:%2409=910%:%
%:%2415=910%:%
%:%2418=911%:%
%:%2419=912%:%
%:%2420=912%:%
%:%2421=913%:%
%:%2422=914%:%
%:%2423=915%:%
%:%2426=916%:%
%:%2430=916%:%
%:%2431=916%:%
%:%2436=916%:%
%:%2439=917%:%
%:%2440=918%:%
%:%2441=918%:%
%:%2442=919%:%
%:%2443=920%:%
%:%2444=921%:%
%:%2447=922%:%
%:%2451=922%:%
%:%2452=922%:%
%:%2453=923%:%
%:%2454=923%:%
%:%2455=924%:%
%:%2456=924%:%
%:%2457=925%:%
%:%2458=925%:%
%:%2459=926%:%
%:%2460=926%:%
%:%2461=926%:%
%:%2462=927%:%
%:%2463=927%:%
%:%2464=927%:%
%:%2465=928%:%
%:%2466=928%:%
%:%2467=928%:%
%:%2468=929%:%
%:%2469=929%:%
%:%2470=930%:%
%:%2471=930%:%
%:%2472=931%:%
%:%2473=931%:%
%:%2474=931%:%
%:%2475=932%:%
%:%2476=932%:%
%:%2477=933%:%
%:%2478=933%:%
%:%2479=933%:%
%:%2480=934%:%
%:%2481=934%:%
%:%2482=934%:%
%:%2483=935%:%
%:%2484=936%:%
%:%2485=936%:%
%:%2486=936%:%
%:%2487=937%:%
%:%2488=937%:%
%:%2489=938%:%
%:%2490=938%:%
%:%2491=939%:%
%:%2492=939%:%
%:%2493=940%:%
%:%2494=940%:%
%:%2495=940%:%
%:%2496=941%:%
%:%2497=941%:%
%:%2498=942%:%
%:%2499=942%:%
%:%2500=943%:%
%:%2501=943%:%
%:%2502=944%:%
%:%2503=944%:%
%:%2504=945%:%
%:%2505=945%:%
%:%2506=946%:%
%:%2507=946%:%
%:%2508=946%:%
%:%2509=947%:%
%:%2510=947%:%
%:%2511=948%:%
%:%2512=948%:%
%:%2513=948%:%
%:%2514=949%:%
%:%2515=949%:%
%:%2516=949%:%
%:%2517=950%:%
%:%2518=950%:%
%:%2519=950%:%
%:%2520=951%:%
%:%2521=951%:%
%:%2522=951%:%
%:%2523=952%:%
%:%2524=952%:%
%:%2525=952%:%
%:%2526=953%:%
%:%2527=953%:%
%:%2528=953%:%
%:%2529=954%:%
%:%2530=954%:%
%:%2531=954%:%
%:%2532=955%:%
%:%2533=955%:%
%:%2534=956%:%
%:%2535=956%:%
%:%2536=957%:%
%:%2537=957%:%
%:%2538=958%:%
%:%2539=958%:%
%:%2540=958%:%
%:%2541=959%:%
%:%2542=959%:%
%:%2543=959%:%
%:%2544=960%:%
%:%2545=960%:%
%:%2546=961%:%
%:%2547=961%:%
%:%2548=962%:%
%:%2549=962%:%
%:%2550=962%:%
%:%2551=963%:%
%:%2552=963%:%
%:%2553=963%:%
%:%2554=964%:%
%:%2555=964%:%
%:%2556=965%:%
%:%2557=965%:%
%:%2558=966%:%
%:%2559=966%:%
%:%2560=967%:%
%:%2561=967%:%
%:%2562=968%:%
%:%2563=968%:%
%:%2564=968%:%
%:%2565=969%:%
%:%2566=969%:%
%:%2567=970%:%
%:%2568=970%:%
%:%2569=970%:%
%:%2570=971%:%
%:%2571=971%:%
%:%2572=971%:%
%:%2573=972%:%
%:%2574=972%:%
%:%2575=972%:%
%:%2576=973%:%
%:%2577=973%:%
%:%2578=973%:%
%:%2579=974%:%
%:%2580=974%:%
%:%2581=974%:%
%:%2582=975%:%
%:%2583=975%:%
%:%2584=975%:%
%:%2585=976%:%
%:%2586=976%:%
%:%2587=976%:%
%:%2588=977%:%
%:%2589=977%:%
%:%2590=977%:%
%:%2591=978%:%
%:%2592=978%:%
%:%2593=979%:%
%:%2594=979%:%
%:%2595=979%:%
%:%2596=980%:%
%:%2597=980%:%
%:%2598=981%:%
%:%2599=981%:%
%:%2600=982%:%
%:%2601=982%:%
%:%2602=983%:%
%:%2603=983%:%
%:%2604=984%:%
%:%2605=984%:%
%:%2606=985%:%
%:%2607=985%:%
%:%2608=985%:%
%:%2609=986%:%
%:%2610=986%:%
%:%2611=986%:%
%:%2612=987%:%
%:%2613=987%:%
%:%2614=987%:%
%:%2615=988%:%
%:%2616=988%:%
%:%2617=988%:%
%:%2618=989%:%
%:%2619=989%:%
%:%2620=990%:%
%:%2621=990%:%
%:%2622=991%:%
%:%2623=991%:%
%:%2624=992%:%
%:%2625=992%:%
%:%2626=992%:%
%:%2627=993%:%
%:%2628=993%:%
%:%2629=993%:%
%:%2630=994%:%
%:%2631=995%:%
%:%2632=995%:%
%:%2633=995%:%
%:%2634=996%:%
%:%2635=996%:%
%:%2636=997%:%
%:%2637=997%:%
%:%2638=998%:%
%:%2639=998%:%
%:%2640=999%:%
%:%2641=999%:%
%:%2642=999%:%
%:%2643=1000%:%
%:%2644=1000%:%
%:%2645=1001%:%
%:%2646=1001%:%
%:%2647=1002%:%
%:%2648=1002%:%
%:%2649=1003%:%
%:%2650=1003%:%
%:%2651=1004%:%
%:%2652=1004%:%
%:%2653=1005%:%
%:%2654=1005%:%
%:%2655=1005%:%
%:%2656=1006%:%
%:%2657=1006%:%
%:%2658=1007%:%
%:%2659=1007%:%
%:%2660=1007%:%
%:%2661=1008%:%
%:%2662=1008%:%
%:%2663=1008%:%
%:%2664=1009%:%
%:%2665=1009%:%
%:%2666=1009%:%
%:%2667=1010%:%
%:%2668=1010%:%
%:%2669=1010%:%
%:%2670=1011%:%
%:%2671=1011%:%
%:%2672=1011%:%
%:%2673=1012%:%
%:%2674=1012%:%
%:%2675=1012%:%
%:%2676=1013%:%
%:%2677=1013%:%
%:%2678=1013%:%
%:%2679=1014%:%
%:%2680=1014%:%
%:%2681=1015%:%
%:%2682=1015%:%
%:%2683=1016%:%
%:%2684=1016%:%
%:%2685=1016%:%
%:%2686=1017%:%
%:%2687=1017%:%
%:%2688=1018%:%
%:%2689=1018%:%
%:%2690=1019%:%
%:%2691=1019%:%
%:%2692=1020%:%
%:%2693=1020%:%
%:%2694=1020%:%
%:%2695=1021%:%
%:%2696=1021%:%
%:%2697=1021%:%
%:%2698=1022%:%
%:%2699=1022%:%
%:%2700=1023%:%
%:%2701=1023%:%
%:%2702=1024%:%
%:%2703=1024%:%
%:%2704=1025%:%
%:%2705=1025%:%
%:%2706=1026%:%
%:%2707=1026%:%
%:%2708=1026%:%
%:%2709=1027%:%
%:%2710=1027%:%
%:%2711=1028%:%
%:%2712=1028%:%
%:%2713=1028%:%
%:%2714=1029%:%
%:%2715=1029%:%
%:%2716=1029%:%
%:%2717=1030%:%
%:%2718=1030%:%
%:%2719=1030%:%
%:%2720=1031%:%
%:%2721=1031%:%
%:%2722=1031%:%
%:%2723=1032%:%
%:%2724=1032%:%
%:%2725=1032%:%
%:%2726=1033%:%
%:%2727=1033%:%
%:%2728=1033%:%
%:%2729=1034%:%
%:%2730=1034%:%
%:%2731=1034%:%
%:%2732=1035%:%
%:%2733=1035%:%
%:%2734=1036%:%
%:%2735=1037%:%
%:%2736=1037%:%
%:%2737=1038%:%
%:%2738=1038%:%
%:%2739=1038%:%
%:%2740=1039%:%
%:%2741=1039%:%
%:%2742=1039%:%
%:%2743=1040%:%
%:%2744=1040%:%
%:%2745=1041%:%
%:%2760=1043%:%
%:%2770=1045%:%
%:%2771=1045%:%
%:%2772=1046%:%
%:%2773=1047%:%
%:%2774=1048%:%
%:%2775=1049%:%
%:%2776=1049%:%
%:%2777=1050%:%
%:%2778=1051%:%
%:%2785=1052%:%
%:%2786=1052%:%
%:%2787=1053%:%
%:%2788=1053%:%
%:%2789=1054%:%
%:%2790=1054%:%
%:%2791=1054%:%
%:%2792=1055%:%
%:%2793=1055%:%
%:%2794=1056%:%
%:%2795=1056%:%
%:%2796=1057%:%
%:%2797=1057%:%
%:%2798=1058%:%
%:%2799=1058%:%
%:%2800=1059%:%
%:%2801=1059%:%
%:%2802=1060%:%
%:%2803=1060%:%
%:%2804=1061%:%
%:%2805=1061%:%
%:%2806=1061%:%
%:%2807=1062%:%
%:%2808=1062%:%
%:%2809=1063%:%
%:%2810=1063%:%
%:%2811=1063%:%
%:%2812=1064%:%
%:%2813=1064%:%
%:%2814=1065%:%
%:%2815=1065%:%
%:%2816=1065%:%
%:%2817=1066%:%
%:%2818=1066%:%
%:%2819=1066%:%
%:%2820=1067%:%
%:%2826=1067%:%
%:%2829=1068%:%
%:%2830=1069%:%
%:%2831=1069%:%
%:%2832=1070%:%
%:%2835=1071%:%
%:%2839=1071%:%
%:%2840=1071%:%
%:%2841=1071%:%
%:%2846=1071%:%
%:%2849=1072%:%
%:%2850=1073%:%
%:%2851=1073%:%
%:%2852=1074%:%
%:%2855=1075%:%
%:%2859=1075%:%
%:%2860=1075%:%
%:%2861=1075%:%
%:%2866=1075%:%
%:%2869=1076%:%
%:%2870=1077%:%
%:%2871=1077%:%
%:%2872=1078%:%
%:%2875=1079%:%
%:%2879=1079%:%
%:%2880=1079%:%
%:%2881=1079%:%
%:%2886=1079%:%
%:%2889=1080%:%
%:%2890=1081%:%
%:%2891=1081%:%
%:%2892=1082%:%
%:%2895=1083%:%
%:%2899=1083%:%
%:%2900=1083%:%
%:%2901=1083%:%
%:%2902=1084%:%
%:%2903=1084%:%
%:%2908=1084%:%
%:%2911=1085%:%
%:%2912=1086%:%
%:%2913=1086%:%
%:%2914=1087%:%
%:%2921=1088%:%
%:%2922=1088%:%
%:%2923=1089%:%
%:%2924=1089%:%
%:%2925=1090%:%
%:%2926=1090%:%
%:%2927=1091%:%
%:%2928=1091%:%
%:%2929=1091%:%
%:%2930=1092%:%
%:%2931=1092%:%
%:%2932=1093%:%
%:%2933=1093%:%
%:%2934=1093%:%
%:%2935=1094%:%
%:%2936=1094%:%
%:%2937=1094%:%
%:%2938=1095%:%
%:%2939=1095%:%
%:%2940=1095%:%
%:%2941=1096%:%
%:%2942=1096%:%
%:%2943=1096%:%
%:%2944=1097%:%
%:%2950=1097%:%
%:%2953=1098%:%
%:%2954=1099%:%
%:%2955=1099%:%
%:%2956=1100%:%
%:%2959=1101%:%
%:%2963=1101%:%
%:%2964=1101%:%
%:%2969=1101%:%
%:%2972=1102%:%
%:%2973=1103%:%
%:%2974=1103%:%
%:%2975=1104%:%
%:%2978=1105%:%
%:%2982=1105%:%
%:%2983=1105%:%
%:%2988=1105%:%
%:%2991=1106%:%
%:%2992=1107%:%
%:%2993=1107%:%
%:%2994=1108%:%
%:%2995=1109%:%
%:%2996=1110%:%
%:%3003=1111%:%
%:%3004=1111%:%
%:%3005=1112%:%
%:%3006=1112%:%
%:%3007=1113%:%
%:%3008=1113%:%
%:%3009=1113%:%
%:%3010=1114%:%
%:%3011=1114%:%
%:%3012=1114%:%
%:%3013=1115%:%
%:%3014=1115%:%
%:%3015=1115%:%
%:%3016=1116%:%
%:%3017=1116%:%
%:%3018=1116%:%
%:%3019=1117%:%
%:%3020=1117%:%
%:%3021=1117%:%
%:%3022=1118%:%
%:%3023=1118%:%
%:%3024=1118%:%
%:%3025=1119%:%
%:%3026=1119%:%
%:%3027=1120%:%
%:%3033=1120%:%
%:%3036=1121%:%
%:%3037=1122%:%
%:%3038=1122%:%
%:%3039=1123%:%
%:%3040=1124%:%
%:%3041=1125%:%
%:%3048=1126%:%
%:%3049=1126%:%
%:%3050=1127%:%
%:%3051=1127%:%
%:%3052=1128%:%
%:%3053=1128%:%
%:%3054=1128%:%
%:%3055=1129%:%
%:%3056=1129%:%
%:%3057=1129%:%
%:%3058=1130%:%
%:%3059=1130%:%
%:%3060=1130%:%
%:%3061=1131%:%
%:%3062=1131%:%
%:%3063=1131%:%
%:%3064=1132%:%
%:%3065=1132%:%
%:%3066=1132%:%
%:%3067=1133%:%
%:%3068=1133%:%
%:%3069=1133%:%
%:%3070=1134%:%
%:%3071=1134%:%
%:%3072=1135%:%
%:%3087=1137%:%
%:%3091=1139%:%
%:%3101=1141%:%
%:%3102=1141%:%
%:%3103=1142%:%
%:%3104=1143%:%
%:%3105=1144%:%
%:%3106=1144%:%
%:%3107=1145%:%
%:%3110=1146%:%
%:%3114=1146%:%
%:%3115=1146%:%
%:%3116=1146%:%
%:%3121=1146%:%
%:%3124=1147%:%
%:%3125=1148%:%
%:%3126=1148%:%
%:%3127=1149%:%
%:%3128=1150%:%
%:%3131=1151%:%
%:%3135=1151%:%
%:%3136=1151%:%
%:%3137=1151%:%
%:%3138=1152%:%
%:%3139=1152%:%
%:%3144=1152%:%
%:%3147=1153%:%
%:%3148=1154%:%
%:%3149=1154%:%
%:%3150=1155%:%
%:%3151=1156%:%
%:%3154=1157%:%
%:%3158=1157%:%
%:%3159=1157%:%
%:%3160=1157%:%
%:%3161=1158%:%
%:%3162=1158%:%
%:%3167=1158%:%
%:%3170=1159%:%
%:%3171=1160%:%
%:%3172=1160%:%
%:%3173=1161%:%
%:%3180=1162%:%
%:%3181=1162%:%
%:%3182=1163%:%
%:%3183=1163%:%
%:%3184=1164%:%
%:%3185=1165%:%
%:%3186=1165%:%
%:%3187=1166%:%
%:%3188=1167%:%
%:%3189=1167%:%
%:%3190=1168%:%
%:%3191=1168%:%
%:%3192=1169%:%
%:%3193=1169%:%
%:%3194=1170%:%
%:%3195=1170%:%
%:%3196=1171%:%
%:%3197=1171%:%
%:%3198=1172%:%
%:%3199=1172%:%
%:%3200=1173%:%
%:%3201=1173%:%
%:%3202=1174%:%
%:%3203=1175%:%
%:%3204=1175%:%
%:%3205=1176%:%
%:%3206=1176%:%
%:%3207=1176%:%
%:%3208=1177%:%
%:%3209=1177%:%
%:%3210=1177%:%
%:%3211=1178%:%
%:%3212=1179%:%
%:%3213=1179%:%
%:%3214=1180%:%
%:%3215=1180%:%
%:%3216=1181%:%
%:%3217=1181%:%
%:%3218=1181%:%
%:%3219=1182%:%
%:%3220=1182%:%
%:%3221=1182%:%
%:%3222=1183%:%
%:%3223=1184%:%
%:%3224=1184%:%
%:%3225=1185%:%
%:%3226=1186%:%
%:%3227=1186%:%
%:%3228=1187%:%
%:%3229=1187%:%
%:%3230=1188%:%
%:%3231=1188%:%
%:%3232=1189%:%
%:%3233=1189%:%
%:%3234=1189%:%
%:%3235=1190%:%
%:%3236=1190%:%
%:%3237=1191%:%
%:%3238=1191%:%
%:%3239=1191%:%
%:%3240=1192%:%
%:%3241=1192%:%
%:%3242=1192%:%
%:%3243=1193%:%
%:%3244=1193%:%
%:%3245=1193%:%
%:%3246=1194%:%
%:%3247=1194%:%
%:%3248=1195%:%
%:%3249=1195%:%
%:%3250=1196%:%
%:%3251=1196%:%
%:%3252=1197%:%
%:%3253=1197%:%
%:%3254=1198%:%
%:%3255=1198%:%
%:%3256=1198%:%
%:%3257=1199%:%
%:%3258=1199%:%
%:%3259=1200%:%
%:%3260=1200%:%
%:%3261=1200%:%
%:%3262=1201%:%
%:%3263=1201%:%
%:%3264=1201%:%
%:%3265=1202%:%
%:%3266=1202%:%
%:%3267=1202%:%
%:%3268=1203%:%
%:%3269=1203%:%
%:%3270=1204%:%
%:%3271=1204%:%
%:%3272=1205%:%
%:%3273=1205%:%
%:%3274=1206%:%
%:%3275=1206%:%
%:%3276=1207%:%
%:%3277=1207%:%
%:%3278=1207%:%
%:%3279=1208%:%
%:%3280=1208%:%
%:%3281=1208%:%
%:%3282=1209%:%
%:%3283=1209%:%
%:%3284=1210%:%
%:%3285=1210%:%
%:%3286=1210%:%
%:%3287=1211%:%
%:%3288=1211%:%
%:%3289=1211%:%
%:%3290=1211%:%
%:%3291=1212%:%
%:%3292=1212%:%
%:%3293=1212%:%
%:%3294=1213%:%
%:%3295=1213%:%
%:%3296=1213%:%
%:%3297=1214%:%
%:%3298=1214%:%
%:%3299=1214%:%
%:%3300=1215%:%
%:%3301=1215%:%
%:%3302=1216%:%
%:%3303=1216%:%
%:%3304=1216%:%
%:%3305=1217%:%
%:%3306=1217%:%
%:%3307=1217%:%
%:%3308=1218%:%
%:%3309=1218%:%
%:%3310=1218%:%
%:%3311=1219%:%
%:%3312=1219%:%
%:%3313=1219%:%
%:%3314=1219%:%
%:%3315=1220%:%
%:%3316=1220%:%
%:%3317=1220%:%
%:%3318=1221%:%
%:%3319=1221%:%
%:%3320=1221%:%
%:%3321=1222%:%
%:%3322=1222%:%
%:%3323=1222%:%
%:%3324=1223%:%
%:%3325=1223%:%
%:%3326=1223%:%
%:%3327=1224%:%
%:%3328=1224%:%
%:%3329=1224%:%
%:%3330=1225%:%
%:%3331=1225%:%
%:%3332=1226%:%
%:%3333=1226%:%
%:%3334=1227%:%
%:%3335=1227%:%
%:%3336=1227%:%
%:%3337=1228%:%
%:%3338=1228%:%
%:%3339=1228%:%
%:%3340=1229%:%
%:%3341=1229%:%
%:%3342=1229%:%
%:%3343=1230%:%
%:%3344=1230%:%
%:%3345=1230%:%
%:%3346=1231%:%
%:%3347=1231%:%
%:%3348=1231%:%
%:%3349=1232%:%
%:%3350=1232%:%
%:%3351=1233%:%
%:%3352=1233%:%
%:%3353=1234%:%
%:%3354=1234%:%
%:%3355=1235%:%
%:%3356=1235%:%
%:%3357=1235%:%
%:%3358=1236%:%
%:%3359=1236%:%
%:%3360=1236%:%
%:%3361=1237%:%
%:%3362=1237%:%
%:%3363=1237%:%
%:%3364=1238%:%
%:%3365=1238%:%
%:%3366=1239%:%
%:%3367=1239%:%
%:%3368=1240%:%
%:%3369=1240%:%
%:%3370=1241%:%
%:%3371=1241%:%
%:%3372=1242%:%
%:%3373=1242%:%
%:%3374=1242%:%
%:%3375=1243%:%
%:%3376=1243%:%
%:%3377=1243%:%
%:%3378=1244%:%
%:%3379=1244%:%
%:%3380=1245%:%
%:%3381=1245%:%
%:%3382=1245%:%
%:%3383=1246%:%
%:%3384=1246%:%
%:%3385=1246%:%
%:%3386=1246%:%
%:%3387=1247%:%
%:%3388=1247%:%
%:%3389=1247%:%
%:%3390=1248%:%
%:%3391=1248%:%
%:%3392=1248%:%
%:%3393=1249%:%
%:%3394=1249%:%
%:%3395=1249%:%
%:%3396=1250%:%
%:%3397=1250%:%
%:%3398=1251%:%
%:%3399=1251%:%
%:%3400=1251%:%
%:%3401=1252%:%
%:%3402=1252%:%
%:%3403=1252%:%
%:%3404=1253%:%
%:%3405=1253%:%
%:%3406=1253%:%
%:%3407=1254%:%
%:%3408=1254%:%
%:%3409=1254%:%
%:%3410=1254%:%
%:%3411=1255%:%
%:%3412=1255%:%
%:%3413=1255%:%
%:%3414=1256%:%
%:%3415=1256%:%
%:%3416=1256%:%
%:%3417=1257%:%
%:%3418=1257%:%
%:%3419=1257%:%
%:%3420=1258%:%
%:%3421=1258%:%
%:%3422=1258%:%
%:%3423=1259%:%
%:%3424=1259%:%
%:%3425=1259%:%
%:%3426=1260%:%
%:%3427=1260%:%
%:%3428=1261%:%
%:%3429=1261%:%
%:%3430=1262%:%
%:%3431=1262%:%
%:%3432=1262%:%
%:%3433=1263%:%
%:%3434=1263%:%
%:%3435=1263%:%
%:%3436=1264%:%
%:%3437=1264%:%
%:%3438=1264%:%
%:%3439=1265%:%
%:%3440=1265%:%
%:%3441=1265%:%
%:%3442=1266%:%
%:%3443=1266%:%
%:%3444=1267%:%
%:%3445=1267%:%
%:%3446=1268%:%
%:%3447=1268%:%
%:%3448=1269%:%
%:%3449=1269%:%
%:%3450=1269%:%
%:%3451=1270%:%
%:%3452=1270%:%
%:%3453=1270%:%
%:%3454=1271%:%
%:%3455=1271%:%
%:%3456=1271%:%
%:%3457=1272%:%
%:%3458=1272%:%
%:%3459=1272%:%
%:%3460=1273%:%
%:%3461=1273%:%
%:%3462=1273%:%
%:%3463=1274%:%
%:%3464=1274%:%
%:%3465=1275%:%
%:%3466=1275%:%
%:%3467=1276%:%
%:%3468=1276%:%
%:%3469=1277%:%
%:%3470=1277%:%
%:%3471=1278%:%
%:%3472=1278%:%
%:%3473=1278%:%
%:%3474=1279%:%
%:%3475=1279%:%
%:%3476=1280%:%
%:%3477=1280%:%
%:%3478=1280%:%
%:%3479=1281%:%
%:%3480=1281%:%
%:%3481=1281%:%
%:%3482=1282%:%
%:%3483=1282%:%
%:%3484=1283%:%
%:%3485=1283%:%
%:%3486=1284%:%
%:%3487=1284%:%
%:%3488=1285%:%
%:%3489=1285%:%
%:%3490=1285%:%
%:%3491=1286%:%
%:%3492=1286%:%
%:%3493=1286%:%
%:%3494=1287%:%
%:%3495=1287%:%
%:%3496=1288%:%
%:%3497=1288%:%
%:%3498=1288%:%
%:%3499=1289%:%
%:%3500=1289%:%
%:%3501=1289%:%
%:%3502=1289%:%
%:%3503=1290%:%
%:%3504=1290%:%
%:%3505=1290%:%
%:%3506=1291%:%
%:%3507=1291%:%
%:%3508=1291%:%
%:%3509=1292%:%
%:%3510=1292%:%
%:%3511=1292%:%
%:%3512=1293%:%
%:%3513=1293%:%
%:%3514=1294%:%
%:%3515=1294%:%
%:%3516=1294%:%
%:%3517=1295%:%
%:%3518=1295%:%
%:%3519=1295%:%
%:%3520=1296%:%
%:%3521=1296%:%
%:%3522=1296%:%
%:%3523=1297%:%
%:%3524=1297%:%
%:%3525=1297%:%
%:%3526=1297%:%
%:%3527=1298%:%
%:%3528=1298%:%
%:%3529=1298%:%
%:%3530=1299%:%
%:%3531=1299%:%
%:%3532=1299%:%
%:%3533=1300%:%
%:%3534=1300%:%
%:%3535=1300%:%
%:%3536=1301%:%
%:%3537=1301%:%
%:%3538=1301%:%
%:%3539=1302%:%
%:%3540=1302%:%
%:%3541=1302%:%
%:%3542=1303%:%
%:%3543=1303%:%
%:%3544=1304%:%
%:%3545=1304%:%
%:%3546=1305%:%
%:%3547=1305%:%
%:%3548=1305%:%
%:%3549=1306%:%
%:%3550=1306%:%
%:%3551=1306%:%
%:%3552=1307%:%
%:%3553=1307%:%
%:%3554=1307%:%
%:%3555=1308%:%
%:%3556=1308%:%
%:%3557=1308%:%
%:%3558=1309%:%
%:%3559=1309%:%
%:%3560=1310%:%
%:%3561=1310%:%
%:%3562=1311%:%
%:%3563=1311%:%
%:%3564=1311%:%
%:%3565=1312%:%
%:%3566=1312%:%
%:%3567=1312%:%
%:%3568=1313%:%
%:%3569=1313%:%
%:%3570=1313%:%
%:%3571=1314%:%
%:%3572=1314%:%
%:%3573=1314%:%
%:%3574=1315%:%
%:%3575=1315%:%
%:%3576=1316%:%
%:%3577=1316%:%
%:%3578=1317%:%
%:%3579=1317%:%
%:%3580=1318%:%
%:%3581=1318%:%
%:%3582=1318%:%
%:%3583=1319%:%
%:%3584=1319%:%
%:%3585=1319%:%
%:%3586=1320%:%
%:%3587=1320%:%
%:%3588=1321%:%
%:%3589=1321%:%
%:%3590=1321%:%
%:%3591=1322%:%
%:%3592=1322%:%
%:%3593=1322%:%
%:%3594=1322%:%
%:%3595=1323%:%
%:%3596=1323%:%
%:%3597=1323%:%
%:%3598=1324%:%
%:%3599=1324%:%
%:%3600=1324%:%
%:%3601=1325%:%
%:%3602=1325%:%
%:%3603=1325%:%
%:%3604=1326%:%
%:%3605=1326%:%
%:%3606=1327%:%
%:%3607=1327%:%
%:%3608=1327%:%
%:%3609=1328%:%
%:%3610=1328%:%
%:%3611=1328%:%
%:%3612=1329%:%
%:%3613=1329%:%
%:%3614=1329%:%
%:%3615=1330%:%
%:%3616=1330%:%
%:%3617=1330%:%
%:%3618=1330%:%
%:%3619=1331%:%
%:%3620=1331%:%
%:%3621=1331%:%
%:%3622=1332%:%
%:%3623=1332%:%
%:%3624=1332%:%
%:%3625=1333%:%
%:%3626=1333%:%
%:%3627=1333%:%
%:%3628=1334%:%
%:%3629=1334%:%
%:%3630=1334%:%
%:%3631=1335%:%
%:%3632=1335%:%
%:%3633=1335%:%
%:%3634=1336%:%
%:%3635=1336%:%
%:%3636=1337%:%
%:%3637=1337%:%
%:%3638=1338%:%
%:%3639=1338%:%
%:%3640=1338%:%
%:%3641=1339%:%
%:%3642=1339%:%
%:%3643=1339%:%
%:%3644=1340%:%
%:%3645=1340%:%
%:%3646=1340%:%
%:%3647=1341%:%
%:%3648=1341%:%
%:%3649=1341%:%
%:%3650=1342%:%
%:%3651=1342%:%
%:%3652=1342%:%
%:%3653=1343%:%
%:%3654=1343%:%
%:%3655=1344%:%
%:%3656=1344%:%
%:%3657=1345%:%
%:%3658=1345%:%
%:%3659=1346%:%
%:%3660=1346%:%
%:%3661=1347%:%
%:%3662=1347%:%
%:%3663=1347%:%
%:%3664=1348%:%
%:%3665=1348%:%
%:%3666=1348%:%
%:%3667=1349%:%
%:%3673=1349%:%
%:%3676=1350%:%
%:%3677=1351%:%
%:%3678=1351%:%
%:%3679=1352%:%
%:%3686=1353%:%
%:%3687=1353%:%
%:%3688=1354%:%
%:%3689=1354%:%
%:%3690=1355%:%
%:%3691=1356%:%
%:%3692=1356%:%
%:%3693=1357%:%
%:%3694=1357%:%
%:%3695=1358%:%
%:%3696=1358%:%
%:%3697=1359%:%
%:%3698=1359%:%
%:%3699=1360%:%
%:%3700=1360%:%
%:%3701=1361%:%
%:%3702=1361%:%
%:%3703=1362%:%
%:%3704=1362%:%
%:%3705=1362%:%
%:%3706=1363%:%
%:%3707=1363%:%
%:%3708=1363%:%
%:%3709=1364%:%
%:%3710=1364%:%
%:%3711=1364%:%
%:%3712=1365%:%
%:%3713=1365%:%
%:%3714=1366%:%
%:%3715=1366%:%
%:%3716=1366%:%
%:%3717=1367%:%
%:%3718=1367%:%
%:%3719=1368%:%
%:%3720=1368%:%
%:%3721=1369%:%
%:%3722=1369%:%
%:%3723=1370%:%
%:%3724=1370%:%
%:%3725=1371%:%
%:%3726=1371%:%
%:%3727=1372%:%
%:%3728=1372%:%
%:%3729=1373%:%
%:%3730=1374%:%
%:%3731=1374%:%
%:%3732=1375%:%
%:%3733=1375%:%
%:%3734=1376%:%
%:%3735=1376%:%
%:%3736=1376%:%
%:%3737=1377%:%
%:%3738=1377%:%
%:%3739=1378%:%
%:%3740=1378%:%
%:%3741=1378%:%
%:%3742=1379%:%
%:%3743=1379%:%
%:%3744=1379%:%
%:%3745=1380%:%
%:%3746=1380%:%
%:%3747=1381%:%
%:%3748=1381%:%
%:%3749=1382%:%
%:%3750=1382%:%
%:%3751=1383%:%
%:%3752=1383%:%
%:%3753=1383%:%
%:%3754=1384%:%
%:%3755=1384%:%
%:%3756=1385%:%
%:%3757=1385%:%
%:%3758=1386%:%
%:%3759=1386%:%
%:%3760=1387%:%
%:%3761=1387%:%
%:%3762=1387%:%
%:%3763=1388%:%
%:%3764=1388%:%
%:%3765=1388%:%
%:%3766=1389%:%
%:%3767=1389%:%
%:%3768=1389%:%
%:%3769=1390%:%
%:%3770=1390%:%
%:%3771=1390%:%
%:%3772=1391%:%
%:%3773=1391%:%
%:%3774=1391%:%
%:%3775=1392%:%
%:%3776=1392%:%
%:%3777=1392%:%
%:%3778=1393%:%
%:%3779=1394%:%
%:%3780=1394%:%
%:%3781=1394%:%
%:%3782=1395%:%
%:%3783=1395%:%
%:%3784=1396%:%
%:%3785=1396%:%
%:%3786=1397%:%
%:%3787=1397%:%
%:%3788=1397%:%
%:%3789=1398%:%
%:%3790=1398%:%
%:%3791=1398%:%
%:%3792=1399%:%
%:%3793=1399%:%
%:%3794=1399%:%
%:%3795=1400%:%
%:%3796=1400%:%
%:%3797=1401%:%
%:%3798=1401%:%
%:%3799=1401%:%
%:%3800=1402%:%
%:%3801=1402%:%
%:%3802=1402%:%
%:%3803=1403%:%
%:%3804=1403%:%
%:%3805=1404%:%
%:%3806=1404%:%
%:%3807=1405%:%
%:%3808=1405%:%
%:%3809=1406%:%
%:%3810=1406%:%
%:%3811=1406%:%
%:%3812=1407%:%
%:%3813=1407%:%
%:%3814=1408%:%
%:%3815=1408%:%
%:%3816=1409%:%
%:%3817=1409%:%
%:%3818=1410%:%
%:%3819=1410%:%
%:%3820=1410%:%
%:%3821=1411%:%
%:%3822=1411%:%
%:%3823=1411%:%
%:%3824=1412%:%
%:%3825=1412%:%
%:%3826=1412%:%
%:%3827=1412%:%
%:%3828=1413%:%
%:%3829=1413%:%
%:%3830=1413%:%
%:%3831=1414%:%
%:%3832=1414%:%
%:%3833=1414%:%
%:%3834=1415%:%
%:%3835=1415%:%
%:%3836=1415%:%
%:%3837=1416%:%
%:%3838=1417%:%
%:%3839=1417%:%
%:%3840=1417%:%
%:%3841=1418%:%
%:%3842=1418%:%
%:%3843=1419%:%
%:%3844=1419%:%
%:%3845=1420%:%
%:%3846=1420%:%
%:%3847=1420%:%
%:%3848=1421%:%
%:%3849=1421%:%
%:%3850=1422%:%
%:%3856=1422%:%
%:%3859=1423%:%
%:%3860=1424%:%
%:%3861=1424%:%
%:%3862=1425%:%
%:%3865=1426%:%
%:%3869=1426%:%
%:%3870=1426%:%
%:%3879=1428%:%
%:%3881=1429%:%
%:%3882=1429%:%
%:%3883=1430%:%
%:%3886=1431%:%
%:%3890=1431%:%
%:%3891=1431%:%
%:%3892=1431%:%
%:%3906=1433%:%
%:%3916=1435%:%
%:%3917=1435%:%
%:%3918=1436%:%
%:%3920=1438%:%
%:%3921=1439%:%
%:%3922=1440%:%
%:%3923=1440%:%
%:%3924=1441%:%
%:%3926=1443%:%
%:%3929=1444%:%
%:%3933=1444%:%
%:%3934=1444%:%
%:%3935=1445%:%
%:%3936=1445%:%
%:%3937=1446%:%
%:%3938=1446%:%
%:%3940=1448%:%
%:%3941=1449%:%
%:%3942=1449%:%
%:%3943=1449%:%
%:%3944=1450%:%
%:%3945=1450%:%
%:%3946=1451%:%
%:%3947=1451%:%
%:%3948=1451%:%
%:%3949=1452%:%
%:%3950=1453%:%
%:%3951=1454%:%
%:%3952=1454%:%
%:%3953=1455%:%
%:%3954=1456%:%
%:%3955=1456%:%
%:%3956=1457%:%
%:%3957=1457%:%
%:%3958=1458%:%
%:%3959=1458%:%
%:%3960=1459%:%
%:%3961=1460%:%
%:%3962=1460%:%
%:%3963=1461%:%
%:%3964=1462%:%
%:%3965=1462%:%
%:%3966=1462%:%
%:%3967=1463%:%
%:%3968=1463%:%
%:%3969=1464%:%
%:%3970=1464%:%
%:%3971=1464%:%
%:%3972=1465%:%
%:%3973=1465%:%
%:%3974=1466%:%
%:%3975=1466%:%
%:%3976=1466%:%
%:%3977=1467%:%
%:%3978=1467%:%
%:%3979=1467%:%
%:%3980=1468%:%
%:%3981=1468%:%
%:%3982=1468%:%
%:%3983=1469%:%
%:%3984=1469%:%
%:%3985=1470%:%
%:%3986=1470%:%
%:%3987=1470%:%
%:%3988=1471%:%
%:%3989=1471%:%
%:%3990=1471%:%
%:%3991=1472%:%
%:%3997=1472%:%
%:%4000=1473%:%
%:%4001=1474%:%
%:%4002=1474%:%
%:%4003=1475%:%
%:%4006=1476%:%
%:%4010=1476%:%
%:%4011=1476%:%
%:%4016=1476%:%
%:%4019=1477%:%
%:%4020=1478%:%
%:%4021=1478%:%
%:%4022=1479%:%
%:%4025=1480%:%
%:%4029=1480%:%
%:%4030=1480%:%
%:%4035=1480%:%
%:%4038=1481%:%
%:%4039=1482%:%
%:%4040=1482%:%
%:%4041=1483%:%
%:%4044=1484%:%
%:%4048=1484%:%
%:%4049=1484%:%
%:%4054=1484%:%
%:%4057=1485%:%
%:%4058=1486%:%
%:%4059=1486%:%
%:%4060=1487%:%
%:%4063=1488%:%
%:%4067=1488%:%
%:%4068=1488%:%
%:%4073=1488%:%
%:%4076=1489%:%
%:%4077=1490%:%
%:%4078=1490%:%
%:%4079=1491%:%
%:%4080=1492%:%
%:%4083=1493%:%
%:%4087=1493%:%
%:%4088=1493%:%
%:%4089=1493%:%
%:%4094=1493%:%
%:%4097=1494%:%
%:%4098=1495%:%
%:%4099=1495%:%
%:%4100=1496%:%
%:%4101=1497%:%
%:%4104=1498%:%
%:%4108=1498%:%
%:%4109=1498%:%
%:%4110=1498%:%
%:%4124=1500%:%
%:%4134=1501%:%
%:%4135=1501%:%
%:%4136=1502%:%
%:%4137=1503%:%
%:%4138=1504%:%
%:%4139=1504%:%
%:%4140=1505%:%
%:%4143=1506%:%
%:%4147=1506%:%
%:%4148=1506%:%
%:%4149=1506%:%
%:%4154=1506%:%
%:%4157=1507%:%
%:%4158=1508%:%
%:%4159=1508%:%
%:%4160=1509%:%
%:%4161=1510%:%
%:%4168=1511%:%
%:%4169=1511%:%
%:%4170=1512%:%
%:%4171=1512%:%
%:%4172=1513%:%
%:%4173=1514%:%
%:%4174=1514%:%
%:%4175=1514%:%
%:%4176=1515%:%
%:%4177=1515%:%
%:%4178=1515%:%
%:%4179=1516%:%
%:%4180=1516%:%
%:%4181=1516%:%
%:%4182=1517%:%
%:%4183=1517%:%
%:%4184=1517%:%
%:%4185=1518%:%
%:%4186=1518%:%
%:%4187=1518%:%
%:%4188=1519%:%
%:%4189=1519%:%
%:%4190=1519%:%
%:%4191=1520%:%
%:%4192=1520%:%
%:%4193=1520%:%
%:%4194=1521%:%
%:%4195=1521%:%
%:%4196=1521%:%
%:%4197=1522%:%
%:%4198=1522%:%
%:%4199=1522%:%
%:%4200=1523%:%
%:%4201=1523%:%
%:%4202=1523%:%
%:%4203=1524%:%
%:%4204=1524%:%
%:%4205=1524%:%
%:%4206=1525%:%
%:%4207=1525%:%
%:%4208=1525%:%
%:%4209=1526%:%
%:%4210=1526%:%
%:%4211=1526%:%
%:%4212=1527%:%
%:%4213=1527%:%
%:%4214=1527%:%
%:%4215=1528%:%
%:%4216=1528%:%
%:%4217=1528%:%
%:%4218=1529%:%
%:%4224=1529%:%
%:%4227=1530%:%
%:%4228=1531%:%
%:%4229=1531%:%
%:%4230=1532%:%
%:%4231=1533%:%
%:%4238=1534%:%
%:%4239=1534%:%
%:%4240=1535%:%
%:%4241=1535%:%
%:%4242=1536%:%
%:%4243=1537%:%
%:%4244=1537%:%
%:%4245=1537%:%
%:%4246=1538%:%
%:%4247=1538%:%
%:%4248=1538%:%
%:%4249=1539%:%
%:%4250=1539%:%
%:%4251=1539%:%
%:%4252=1540%:%
%:%4253=1540%:%
%:%4254=1540%:%
%:%4255=1541%:%
%:%4256=1541%:%
%:%4257=1541%:%
%:%4258=1542%:%
%:%4259=1542%:%
%:%4260=1542%:%
%:%4261=1543%:%
%:%4262=1543%:%
%:%4263=1543%:%
%:%4264=1544%:%
%:%4265=1544%:%
%:%4266=1544%:%
%:%4267=1545%:%
%:%4268=1545%:%
%:%4269=1545%:%
%:%4270=1546%:%
%:%4271=1546%:%
%:%4272=1546%:%
%:%4273=1547%:%
%:%4274=1547%:%
%:%4275=1547%:%
%:%4276=1548%:%
%:%4277=1548%:%
%:%4278=1548%:%
%:%4279=1549%:%
%:%4280=1549%:%
%:%4281=1549%:%
%:%4282=1550%:%
%:%4283=1550%:%
%:%4284=1550%:%
%:%4285=1551%:%
%:%4286=1551%:%
%:%4287=1551%:%
%:%4288=1552%:%
%:%4303=1554%:%
%:%4313=1556%:%
%:%4314=1556%:%
%:%4315=1557%:%
%:%4316=1558%:%
%:%4317=1559%:%
%:%4318=1559%:%
%:%4319=1560%:%
%:%4322=1561%:%
%:%4326=1561%:%
%:%4327=1561%:%
%:%4328=1561%:%
%:%4333=1561%:%
%:%4336=1562%:%
%:%4337=1563%:%
%:%4338=1563%:%
%:%4339=1564%:%
%:%4340=1565%:%
%:%4347=1566%:%
%:%4348=1566%:%
%:%4349=1567%:%
%:%4350=1567%:%
%:%4351=1568%:%
%:%4352=1569%:%
%:%4353=1569%:%
%:%4354=1569%:%
%:%4355=1570%:%
%:%4356=1570%:%
%:%4357=1570%:%
%:%4358=1571%:%
%:%4359=1571%:%
%:%4360=1571%:%
%:%4361=1572%:%
%:%4362=1572%:%
%:%4363=1572%:%
%:%4364=1573%:%
%:%4365=1573%:%
%:%4366=1573%:%
%:%4367=1574%:%
%:%4368=1574%:%
%:%4369=1574%:%
%:%4370=1575%:%
%:%4371=1575%:%
%:%4372=1575%:%
%:%4373=1576%:%
%:%4374=1576%:%
%:%4375=1576%:%
%:%4376=1577%:%
%:%4377=1577%:%
%:%4378=1577%:%
%:%4379=1578%:%
%:%4380=1578%:%
%:%4381=1578%:%
%:%4382=1579%:%
%:%4383=1579%:%
%:%4384=1579%:%
%:%4385=1580%:%
%:%4386=1580%:%
%:%4387=1580%:%
%:%4388=1581%:%
%:%4389=1581%:%
%:%4390=1581%:%
%:%4391=1582%:%
%:%4392=1582%:%
%:%4393=1582%:%
%:%4394=1583%:%
%:%4395=1583%:%
%:%4396=1583%:%
%:%4397=1584%:%
%:%4398=1584%:%
%:%4399=1584%:%
%:%4400=1585%:%
%:%4401=1585%:%
%:%4402=1585%:%
%:%4403=1586%:%
%:%4409=1586%:%
%:%4412=1587%:%
%:%4413=1588%:%
%:%4414=1588%:%
%:%4415=1589%:%
%:%4416=1590%:%
%:%4423=1591%:%
%:%4424=1591%:%
%:%4425=1592%:%
%:%4426=1592%:%
%:%4427=1593%:%
%:%4428=1594%:%
%:%4429=1594%:%
%:%4430=1594%:%
%:%4431=1595%:%
%:%4432=1595%:%
%:%4433=1595%:%
%:%4434=1596%:%
%:%4435=1596%:%
%:%4436=1596%:%
%:%4437=1597%:%
%:%4438=1597%:%
%:%4439=1597%:%
%:%4440=1598%:%
%:%4441=1598%:%
%:%4442=1598%:%
%:%4443=1599%:%
%:%4444=1599%:%
%:%4445=1599%:%
%:%4446=1600%:%
%:%4447=1600%:%
%:%4448=1600%:%
%:%4449=1601%:%
%:%4450=1601%:%
%:%4451=1601%:%
%:%4452=1602%:%
%:%4453=1602%:%
%:%4454=1602%:%
%:%4455=1603%:%
%:%4456=1603%:%
%:%4457=1603%:%
%:%4458=1604%:%
%:%4459=1604%:%
%:%4460=1604%:%
%:%4461=1605%:%
%:%4462=1605%:%
%:%4463=1605%:%
%:%4464=1606%:%
%:%4465=1606%:%
%:%4466=1606%:%
%:%4467=1607%:%
%:%4468=1607%:%
%:%4469=1607%:%
%:%4470=1608%:%
%:%4471=1608%:%
%:%4472=1608%:%
%:%4473=1609%:%
%:%4474=1609%:%
%:%4475=1609%:%
%:%4476=1610%:%
%:%4477=1610%:%
%:%4478=1610%:%
%:%4479=1611%:%
%:%4485=1611%:%
%:%4488=1612%:%
%:%4489=1613%:%
%:%4490=1613%:%
%:%4491=1614%:%
%:%4494=1615%:%
%:%4498=1615%:%
%:%4499=1615%:%
%:%4504=1615%:%
%:%4507=1616%:%
%:%4508=1617%:%
%:%4509=1617%:%
%:%4510=1618%:%
%:%4513=1619%:%
%:%4517=1619%:%
%:%4518=1619%:%
%:%4523=1619%:%
%:%4526=1620%:%
%:%4527=1621%:%
%:%4528=1621%:%
%:%4529=1622%:%
%:%4532=1623%:%
%:%4536=1623%:%
%:%4537=1623%:%
%:%4538=1623%:%
%:%4539=1624%:%
%:%4544=1624%:%
%:%4547=1625%:%
%:%4548=1626%:%
%:%4549=1626%:%
%:%4550=1627%:%
%:%4553=1628%:%
%:%4557=1628%:%
%:%4558=1628%:%
%:%4559=1628%:%
%:%4560=1629%:%
%:%4565=1629%:%
%:%4568=1630%:%
%:%4569=1631%:%
%:%4570=1631%:%
%:%4571=1632%:%
%:%4574=1633%:%
%:%4578=1633%:%
%:%4579=1633%:%
%:%4593=1635%:%
%:%4597=1637%:%
%:%4609=1639%:%
%:%4611=1640%:%
%:%4612=1640%:%
%:%4613=1641%:%
%:%4614=1642%:%
%:%4615=1643%:%
%:%4616=1643%:%
%:%4617=1644%:%
%:%4620=1645%:%
%:%4624=1645%:%
%:%4625=1645%:%
%:%4626=1645%:%
%:%4631=1645%:%
%:%4634=1646%:%
%:%4635=1647%:%
%:%4636=1647%:%
%:%4637=1648%:%
%:%4638=1649%:%
%:%4645=1650%:%
%:%4646=1650%:%
%:%4647=1651%:%
%:%4648=1651%:%
%:%4649=1652%:%
%:%4650=1652%:%
%:%4651=1652%:%
%:%4652=1652%:%
%:%4653=1653%:%
%:%4654=1653%:%
%:%4655=1653%:%
%:%4656=1654%:%
%:%4657=1655%:%
%:%4658=1655%:%
%:%4659=1655%:%
%:%4660=1655%:%
%:%4661=1656%:%
%:%4662=1656%:%
%:%4663=1656%:%
%:%4664=1657%:%
%:%4665=1658%:%
%:%4666=1658%:%
%:%4667=1659%:%
%:%4668=1660%:%
%:%4669=1660%:%
%:%4670=1661%:%
%:%4671=1662%:%
%:%4672=1662%:%
%:%4673=1663%:%
%:%4674=1664%:%
%:%4675=1664%:%
%:%4676=1665%:%
%:%4677=1666%:%
%:%4678=1666%:%
%:%4679=1667%:%
%:%4680=1668%:%
%:%4681=1668%:%
%:%4682=1669%:%
%:%4683=1669%:%
%:%4684=1670%:%
%:%4685=1670%:%
%:%4686=1671%:%
%:%4687=1671%:%
%:%4688=1672%:%
%:%4689=1672%:%
%:%4690=1673%:%
%:%4691=1673%:%
%:%4692=1674%:%
%:%4693=1675%:%
%:%4694=1675%:%
%:%4695=1676%:%
%:%4696=1676%:%
%:%4697=1676%:%
%:%4698=1676%:%
%:%4699=1677%:%
%:%4700=1677%:%
%:%4701=1677%:%
%:%4702=1678%:%
%:%4703=1678%:%
%:%4704=1678%:%
%:%4705=1679%:%
%:%4706=1679%:%
%:%4707=1679%:%
%:%4708=1680%:%
%:%4709=1680%:%
%:%4710=1680%:%
%:%4711=1681%:%
%:%4712=1681%:%
%:%4713=1682%:%
%:%4714=1682%:%
%:%4715=1682%:%
%:%4716=1683%:%
%:%4717=1683%:%
%:%4718=1683%:%
%:%4719=1684%:%
%:%4720=1684%:%
%:%4721=1684%:%
%:%4722=1685%:%
%:%4723=1685%:%
%:%4724=1686%:%
%:%4725=1686%:%
%:%4726=1687%:%
%:%4727=1687%:%
%:%4728=1688%:%
%:%4729=1688%:%
%:%4730=1689%:%
%:%4731=1689%:%
%:%4732=1690%:%
%:%4733=1691%:%
%:%4734=1691%:%
%:%4735=1692%:%
%:%4736=1692%:%
%:%4737=1692%:%
%:%4738=1692%:%
%:%4739=1693%:%
%:%4740=1693%:%
%:%4741=1693%:%
%:%4742=1694%:%
%:%4743=1694%:%
%:%4744=1694%:%
%:%4745=1695%:%
%:%4746=1695%:%
%:%4747=1695%:%
%:%4748=1696%:%
%:%4749=1696%:%
%:%4750=1696%:%
%:%4751=1697%:%
%:%4752=1697%:%
%:%4753=1698%:%
%:%4754=1698%:%
%:%4755=1698%:%
%:%4756=1699%:%
%:%4757=1699%:%
%:%4758=1699%:%
%:%4759=1700%:%
%:%4760=1700%:%
%:%4761=1700%:%
%:%4762=1701%:%
%:%4763=1701%:%
%:%4764=1702%:%
%:%4765=1702%:%
%:%4766=1703%:%
%:%4767=1703%:%
%:%4768=1704%:%
%:%4769=1704%:%
%:%4770=1705%:%
%:%4771=1705%:%
%:%4772=1706%:%
%:%4773=1707%:%
%:%4774=1707%:%
%:%4775=1708%:%
%:%4776=1708%:%
%:%4777=1708%:%
%:%4778=1708%:%
%:%4779=1709%:%
%:%4780=1709%:%
%:%4781=1709%:%
%:%4782=1710%:%
%:%4783=1710%:%
%:%4784=1710%:%
%:%4785=1711%:%
%:%4786=1711%:%
%:%4787=1711%:%
%:%4788=1712%:%
%:%4789=1712%:%
%:%4790=1712%:%
%:%4791=1713%:%
%:%4792=1713%:%
%:%4793=1714%:%
%:%4794=1714%:%
%:%4795=1714%:%
%:%4796=1715%:%
%:%4797=1715%:%
%:%4798=1715%:%
%:%4799=1716%:%
%:%4800=1716%:%
%:%4801=1716%:%
%:%4802=1717%:%
%:%4803=1717%:%
%:%4804=1718%:%
%:%4805=1718%:%
%:%4806=1719%:%
%:%4807=1719%:%
%:%4808=1720%:%
%:%4809=1720%:%
%:%4810=1721%:%
%:%4811=1721%:%
%:%4812=1722%:%
%:%4813=1723%:%
%:%4814=1723%:%
%:%4815=1724%:%
%:%4816=1724%:%
%:%4817=1724%:%
%:%4818=1724%:%
%:%4819=1725%:%
%:%4820=1725%:%
%:%4821=1725%:%
%:%4822=1726%:%
%:%4823=1726%:%
%:%4824=1726%:%
%:%4825=1727%:%
%:%4826=1727%:%
%:%4827=1727%:%
%:%4828=1728%:%
%:%4829=1728%:%
%:%4830=1728%:%
%:%4831=1729%:%
%:%4832=1729%:%
%:%4833=1730%:%
%:%4834=1730%:%
%:%4835=1730%:%
%:%4836=1731%:%
%:%4837=1731%:%
%:%4838=1731%:%
%:%4839=1732%:%
%:%4840=1732%:%
%:%4841=1732%:%
%:%4842=1733%:%
%:%4843=1733%:%
%:%4844=1734%:%
%:%4845=1734%:%
%:%4846=1735%:%
%:%4852=1735%:%
%:%4855=1736%:%
%:%4856=1737%:%
%:%4857=1737%:%
%:%4858=1738%:%
%:%4859=1739%:%
%:%4866=1740%:%
%:%4867=1740%:%
%:%4868=1741%:%
%:%4869=1741%:%
%:%4870=1742%:%
%:%4871=1742%:%
%:%4872=1743%:%
%:%4873=1743%:%
%:%4874=1743%:%
%:%4875=1744%:%
%:%4876=1744%:%
%:%4877=1744%:%
%:%4878=1745%:%
%:%4879=1746%:%
%:%4880=1746%:%
%:%4881=1747%:%
%:%4882=1747%:%
%:%4883=1747%:%
%:%4884=1748%:%
%:%4885=1748%:%
%:%4886=1748%:%
%:%4887=1749%:%
%:%4888=1749%:%
%:%4889=1750%:%
%:%4890=1750%:%
%:%4891=1751%:%
%:%4892=1751%:%
%:%4893=1751%:%
%:%4894=1752%:%
%:%4895=1752%:%
%:%4896=1753%:%
%:%4897=1753%:%
%:%4898=1753%:%
%:%4899=1754%:%
%:%4900=1754%:%
%:%4901=1755%:%
%:%4902=1755%:%
%:%4903=1755%:%
%:%4904=1756%:%
%:%4905=1756%:%
%:%4906=1756%:%
%:%4907=1757%:%
%:%4908=1757%:%
%:%4909=1758%:%
%:%4910=1758%:%
%:%4911=1759%:%
%:%4912=1759%:%
%:%4913=1760%:%
%:%4914=1760%:%
%:%4915=1760%:%
%:%4916=1761%:%
%:%4917=1761%:%
%:%4918=1762%:%
%:%4919=1762%:%
%:%4920=1762%:%
%:%4921=1763%:%
%:%4922=1763%:%
%:%4923=1764%:%
%:%4924=1764%:%
%:%4925=1764%:%
%:%4926=1765%:%
%:%4927=1765%:%
%:%4928=1766%:%
%:%4929=1766%:%
%:%4930=1766%:%
%:%4931=1767%:%
%:%4932=1767%:%
%:%4933=1767%:%
%:%4934=1768%:%
%:%4935=1768%:%
%:%4936=1769%:%
%:%4937=1769%:%
%:%4938=1770%:%
%:%4944=1770%:%
%:%4947=1771%:%
%:%4948=1772%:%
%:%4949=1772%:%
%:%4950=1773%:%
%:%4951=1774%:%
%:%4954=1775%:%
%:%4958=1775%:%
%:%4959=1775%:%
%:%4960=1775%:%
%:%4974=1777%:%
%:%4984=1779%:%
%:%4985=1779%:%
%:%4986=1780%:%
%:%4988=1782%:%
%:%4989=1783%:%
%:%4990=1784%:%
%:%4991=1784%:%
%:%4992=1785%:%
%:%4993=1786%:%
%:%4995=1788%:%
%:%4998=1789%:%
%:%5002=1789%:%
%:%5003=1789%:%
%:%5004=1790%:%
%:%5005=1790%:%
%:%5006=1791%:%
%:%5007=1791%:%
%:%5009=1793%:%
%:%5010=1794%:%
%:%5011=1794%:%
%:%5012=1794%:%
%:%5013=1794%:%
%:%5014=1795%:%
%:%5015=1795%:%
%:%5016=1796%:%
%:%5017=1796%:%
%:%5018=1797%:%
%:%5019=1797%:%
%:%5020=1798%:%
%:%5021=1798%:%
%:%5022=1799%:%
%:%5023=1799%:%
%:%5024=1800%:%
%:%5025=1800%:%
%:%5026=1800%:%
%:%5027=1801%:%
%:%5028=1801%:%
%:%5029=1802%:%
%:%5030=1802%:%
%:%5031=1802%:%
%:%5032=1803%:%
%:%5033=1803%:%
%:%5034=1803%:%
%:%5035=1804%:%
%:%5036=1804%:%
%:%5037=1805%:%
%:%5043=1805%:%
%:%5046=1806%:%
%:%5047=1807%:%
%:%5048=1807%:%
%:%5049=1808%:%
%:%5050=1809%:%
%:%5053=1810%:%
%:%5057=1810%:%
%:%5058=1810%:%
%:%5059=1810%:%
%:%5064=1810%:%
%:%5067=1811%:%
%:%5068=1812%:%
%:%5069=1812%:%
%:%5070=1813%:%
%:%5071=1814%:%
%:%5074=1815%:%
%:%5078=1815%:%
%:%5079=1815%:%
%:%5080=1815%:%
%:%5085=1815%:%
%:%5088=1816%:%
%:%5089=1817%:%
%:%5090=1817%:%
%:%5091=1818%:%
%:%5092=1819%:%
%:%5095=1820%:%
%:%5099=1820%:%
%:%5100=1820%:%
%:%5101=1820%:%
%:%5106=1820%:%
%:%5109=1821%:%
%:%5110=1822%:%
%:%5111=1822%:%
%:%5112=1823%:%
%:%5113=1824%:%
%:%5114=1825%:%
%:%5121=1826%:%
%:%5122=1826%:%
%:%5123=1827%:%
%:%5124=1827%:%
%:%5125=1828%:%
%:%5126=1828%:%
%:%5127=1828%:%
%:%5128=1828%:%
%:%5129=1829%:%
%:%5130=1829%:%
%:%5131=1830%:%
%:%5132=1830%:%
%:%5133=1830%:%
%:%5134=1830%:%
%:%5135=1831%:%
%:%5136=1831%:%
%:%5137=1831%:%
%:%5138=1832%:%
%:%5139=1832%:%
%:%5140=1832%:%
%:%5141=1832%:%
%:%5142=1833%:%
%:%5143=1833%:%
%:%5144=1833%:%
%:%5145=1834%:%
%:%5146=1834%:%
%:%5147=1834%:%
%:%5148=1835%:%
%:%5149=1835%:%
%:%5150=1835%:%
%:%5151=1836%:%
%:%5152=1836%:%
%:%5153=1836%:%
%:%5154=1837%:%
%:%5155=1837%:%
%:%5156=1837%:%
%:%5157=1838%:%
%:%5158=1838%:%
%:%5159=1838%:%
%:%5160=1839%:%
%:%5161=1839%:%
%:%5162=1839%:%
%:%5163=1840%:%
%:%5164=1840%:%
%:%5165=1840%:%
%:%5166=1841%:%
%:%5172=1841%:%
%:%5175=1842%:%
%:%5176=1843%:%
%:%5177=1843%:%
%:%5178=1844%:%
%:%5179=1845%:%
%:%5180=1846%:%
%:%5187=1847%:%
%:%5188=1847%:%
%:%5189=1848%:%
%:%5190=1848%:%
%:%5191=1849%:%
%:%5192=1849%:%
%:%5193=1850%:%
%:%5194=1850%:%
%:%5195=1850%:%
%:%5196=1851%:%
%:%5197=1851%:%
%:%5198=1851%:%
%:%5199=1852%:%
%:%5200=1852%:%
%:%5201=1852%:%
%:%5202=1853%:%
%:%5203=1853%:%
%:%5204=1853%:%
%:%5205=1853%:%
%:%5206=1854%:%
%:%5207=1854%:%
%:%5208=1854%:%
%:%5209=1855%:%
%:%5210=1855%:%
%:%5211=1855%:%
%:%5212=1856%:%
%:%5213=1856%:%
%:%5214=1856%:%
%:%5215=1857%:%
%:%5216=1857%:%
%:%5217=1857%:%
%:%5218=1858%:%
%:%5219=1858%:%
%:%5220=1858%:%
%:%5221=1859%:%
%:%5222=1859%:%
%:%5223=1859%:%
%:%5224=1860%:%
%:%5225=1860%:%
%:%5226=1860%:%
%:%5227=1861%:%
%:%5228=1861%:%
%:%5229=1861%:%
%:%5230=1862%:%
%:%5236=1862%:%
%:%5239=1863%:%
%:%5240=1864%:%
%:%5241=1864%:%
%:%5242=1865%:%
%:%5243=1866%:%
%:%5246=1867%:%
%:%5250=1867%:%
%:%5251=1867%:%
%:%5256=1867%:%
%:%5259=1868%:%
%:%5260=1869%:%
%:%5261=1869%:%
%:%5262=1870%:%
%:%5263=1871%:%
%:%5266=1872%:%
%:%5270=1872%:%
%:%5271=1872%:%
%:%5276=1872%:%
%:%5279=1873%:%
%:%5280=1874%:%
%:%5281=1874%:%
%:%5282=1875%:%
%:%5283=1876%:%
%:%5286=1877%:%
%:%5290=1877%:%
%:%5291=1877%:%
%:%5305=1879%:%
%:%5315=1881%:%
%:%5316=1881%:%
%:%5317=1882%:%
%:%5324=1883%:%
%:%5325=1883%:%
%:%5326=1884%:%
%:%5327=1884%:%
%:%5328=1885%:%
%:%5329=1885%:%
%:%5330=1886%:%
%:%5331=1886%:%
%:%5332=1887%:%
%:%5333=1887%:%
%:%5334=1888%:%
%:%5335=1888%:%
%:%5336=1889%:%
%:%5337=1889%:%
%:%5338=1890%:%
%:%5344=1890%:%
%:%5347=1891%:%
%:%5348=1892%:%
%:%5349=1892%:%
%:%5350=1893%:%
%:%5357=1894%:%
%:%5358=1894%:%
%:%5359=1895%:%
%:%5360=1895%:%
%:%5361=1896%:%
%:%5362=1896%:%
%:%5363=1897%:%
%:%5364=1897%:%
%:%5365=1898%:%
%:%5366=1898%:%
%:%5367=1899%:%
%:%5368=1899%:%
%:%5369=1900%:%
%:%5370=1900%:%
%:%5371=1901%:%
%:%5372=1901%:%
%:%5373=1901%:%
%:%5374=1902%:%
%:%5375=1902%:%
%:%5376=1903%:%
%:%5377=1904%:%
%:%5378=1904%:%
%:%5379=1905%:%
%:%5380=1905%:%
%:%5381=1906%:%
%:%5382=1906%:%
%:%5383=1907%:%
%:%5384=1907%:%
%:%5385=1908%:%
%:%5386=1908%:%
%:%5387=1909%:%
%:%5388=1909%:%
%:%5389=1910%:%
%:%5390=1910%:%
%:%5391=1911%:%
%:%5392=1911%:%
%:%5393=1912%:%
%:%5394=1913%:%
%:%5395=1913%:%
%:%5396=1914%:%
%:%5397=1914%:%
%:%5398=1915%:%
%:%5399=1915%:%
%:%5400=1916%:%
%:%5401=1916%:%
%:%5402=1917%:%
%:%5403=1917%:%
%:%5404=1918%:%
%:%5405=1918%:%
%:%5406=1919%:%
%:%5407=1919%:%
%:%5408=1920%:%
%:%5409=1920%:%
%:%5410=1920%:%
%:%5411=1921%:%
%:%5412=1921%:%
%:%5413=1922%:%
%:%5414=1922%:%
%:%5415=1923%:%
%:%5416=1924%:%
%:%5417=1924%:%
%:%5418=1925%:%
%:%5419=1925%:%
%:%5420=1926%:%
%:%5421=1926%:%
%:%5422=1927%:%
%:%5423=1927%:%
%:%5424=1928%:%
%:%5425=1929%:%
%:%5426=1929%:%
%:%5427=1930%:%
%:%5428=1930%:%
%:%5429=1931%:%
%:%5430=1931%:%
%:%5431=1932%:%
%:%5432=1932%:%
%:%5433=1933%:%
%:%5434=1933%:%
%:%5435=1934%:%
%:%5436=1934%:%
%:%5437=1935%:%
%:%5438=1935%:%
%:%5439=1936%:%
%:%5440=1936%:%
%:%5441=1937%:%
%:%5442=1937%:%
%:%5443=1938%:%
%:%5444=1938%:%
%:%5445=1939%:%
%:%5446=1939%:%
%:%5447=1940%:%
%:%5448=1940%:%
%:%5449=1941%:%
%:%5450=1941%:%
%:%5451=1942%:%
%:%5452=1942%:%
%:%5453=1943%:%
%:%5463=1945%:%
%:%5465=1946%:%
%:%5466=1946%:%
%:%5467=1947%:%
%:%5470=1948%:%
%:%5474=1948%:%
%:%5475=1948%:%
%:%5476=1948%:%
%:%5481=1948%:%
%:%5484=1949%:%
%:%5485=1950%:%
%:%5486=1950%:%
%:%5487=1951%:%
%:%5488=1952%:%
%:%5495=1953%:%
%:%5496=1953%:%
%:%5497=1954%:%
%:%5498=1954%:%
%:%5499=1955%:%
%:%5500=1955%:%
%:%5501=1955%:%
%:%5502=1956%:%
%:%5503=1956%:%
%:%5504=1957%:%
%:%5505=1957%:%
%:%5506=1957%:%
%:%5507=1958%:%
%:%5508=1958%:%
%:%5509=1959%:%
%:%5510=1959%:%
%:%5511=1960%:%
%:%5512=1960%:%
%:%5513=1960%:%
%:%5514=1961%:%
%:%5515=1961%:%
%:%5516=1962%:%
%:%5522=1962%:%
%:%5525=1963%:%
%:%5526=1964%:%
%:%5527=1964%:%
%:%5528=1965%:%
%:%5529=1966%:%
%:%5536=1967%:%
%:%5537=1967%:%
%:%5538=1968%:%
%:%5539=1968%:%
%:%5540=1969%:%
%:%5541=1969%:%
%:%5542=1969%:%
%:%5543=1970%:%
%:%5544=1970%:%
%:%5545=1971%:%
%:%5546=1971%:%
%:%5547=1971%:%
%:%5548=1972%:%
%:%5549=1973%:%
%:%5550=1973%:%
%:%5551=1974%:%
%:%5552=1974%:%
%:%5553=1974%:%
%:%5554=1975%:%
%:%5555=1975%:%
%:%5556=1976%:%
%:%5557=1976%:%
%:%5558=1976%:%
%:%5559=1977%:%
%:%5560=1978%:%
%:%5561=1978%:%
%:%5562=1979%:%
%:%5563=1979%:%
%:%5564=1979%:%
%:%5565=1980%:%
%:%5566=1981%:%
%:%5567=1981%:%
%:%5568=1982%:%
%:%5569=1982%:%
%:%5570=1982%:%
%:%5571=1983%:%
%:%5572=1984%:%
%:%5573=1984%:%
%:%5574=1985%:%
%:%5575=1985%:%
%:%5576=1986%:%
%:%5577=1986%:%
%:%5578=1986%:%
%:%5579=1987%:%
%:%5580=1987%:%
%:%5581=1987%:%
%:%5582=1988%:%
%:%5583=1989%:%
%:%5584=1989%:%
%:%5585=1990%:%
%:%5586=1990%:%
%:%5587=1991%:%
%:%5588=1991%:%
%:%5589=1992%:%
%:%5590=1992%:%
%:%5591=1993%:%
%:%5592=1993%:%
%:%5593=1994%:%
%:%5594=1994%:%
%:%5595=1994%:%
%:%5596=1995%:%
%:%5597=1995%:%
%:%5598=1995%:%
%:%5599=1996%:%
%:%5600=1996%:%
%:%5601=1996%:%
%:%5602=1997%:%
%:%5603=1998%:%
%:%5604=1998%:%
%:%5605=1998%:%
%:%5606=1999%:%
%:%5607=1999%:%
%:%5608=2000%:%
%:%5609=2000%:%
%:%5610=2001%:%
%:%5611=2001%:%
%:%5612=2002%:%
%:%5613=2002%:%
%:%5614=2002%:%
%:%5615=2003%:%
%:%5616=2003%:%
%:%5617=2004%:%
%:%5618=2004%:%
%:%5619=2004%:%
%:%5620=2005%:%
%:%5621=2005%:%
%:%5622=2005%:%
%:%5623=2006%:%
%:%5624=2006%:%
%:%5625=2006%:%
%:%5626=2007%:%
%:%5627=2007%:%
%:%5628=2008%:%
%:%5629=2008%:%
%:%5630=2009%:%
%:%5631=2009%:%
%:%5632=2010%:%
%:%5633=2010%:%
%:%5634=2011%:%
%:%5635=2011%:%
%:%5636=2011%:%
%:%5637=2012%:%
%:%5638=2012%:%
%:%5639=2013%:%
%:%5640=2013%:%
%:%5641=2013%:%
%:%5642=2014%:%
%:%5643=2014%:%
%:%5644=2014%:%
%:%5645=2015%:%
%:%5646=2015%:%
%:%5647=2015%:%
%:%5648=2016%:%
%:%5649=2016%:%
%:%5650=2017%:%
%:%5651=2017%:%
%:%5652=2018%:%
%:%5653=2018%:%
%:%5654=2018%:%
%:%5655=2019%:%
%:%5656=2019%:%
%:%5657=2019%:%
%:%5658=2020%:%
%:%5659=2020%:%
%:%5660=2020%:%
%:%5661=2021%:%
%:%5662=2021%:%
%:%5663=2022%:%
%:%5664=2022%:%
%:%5665=2022%:%
%:%5666=2023%:%
%:%5667=2023%:%
%:%5668=2023%:%
%:%5669=2024%:%
%:%5670=2024%:%
%:%5671=2024%:%
%:%5672=2025%:%
%:%5673=2025%:%
%:%5674=2026%:%
%:%5675=2026%:%
%:%5676=2027%:%
%:%5677=2027%:%
%:%5678=2028%:%
%:%5679=2028%:%
%:%5680=2029%:%
%:%5681=2029%:%
%:%5682=2029%:%
%:%5683=2030%:%
%:%5684=2030%:%
%:%5685=2031%:%
%:%5686=2031%:%
%:%5687=2031%:%
%:%5688=2032%:%
%:%5689=2032%:%
%:%5690=2032%:%
%:%5691=2033%:%
%:%5692=2033%:%
%:%5693=2033%:%
%:%5694=2034%:%
%:%5695=2034%:%
%:%5696=2035%:%
%:%5697=2035%:%
%:%5698=2036%:%
%:%5699=2037%:%
%:%5700=2037%:%
%:%5701=2038%:%
%:%5702=2038%:%
%:%5703=2039%:%
%:%5704=2039%:%
%:%5705=2040%:%
%:%5706=2040%:%
%:%5707=2041%:%
%:%5708=2041%:%
%:%5709=2042%:%
%:%5710=2042%:%
%:%5711=2043%:%
%:%5712=2043%:%
%:%5713=2044%:%
%:%5714=2044%:%
%:%5715=2045%:%
%:%5716=2045%:%
%:%5717=2045%:%
%:%5718=2046%:%
%:%5719=2046%:%
%:%5720=2047%:%
%:%5721=2047%:%
%:%5722=2047%:%
%:%5723=2048%:%
%:%5724=2049%:%
%:%5725=2049%:%
%:%5726=2050%:%
%:%5727=2050%:%
%:%5728=2050%:%
%:%5729=2051%:%
%:%5730=2051%:%
%:%5731=2052%:%
%:%5732=2052%:%
%:%5733=2052%:%
%:%5734=2053%:%
%:%5735=2054%:%
%:%5736=2054%:%
%:%5737=2055%:%
%:%5738=2056%:%
%:%5739=2056%:%
%:%5740=2056%:%
%:%5741=2057%:%
%:%5742=2058%:%
%:%5743=2058%:%
%:%5744=2059%:%
%:%5745=2060%:%
%:%5746=2060%:%
%:%5747=2060%:%
%:%5748=2061%:%
%:%5749=2062%:%
%:%5750=2062%:%
%:%5751=2063%:%
%:%5752=2063%:%
%:%5753=2064%:%
%:%5754=2064%:%
%:%5755=2065%:%
%:%5756=2065%:%
%:%5757=2065%:%
%:%5758=2066%:%
%:%5759=2066%:%
%:%5760=2067%:%
%:%5761=2067%:%
%:%5762=2068%:%
%:%5763=2068%:%
%:%5764=2069%:%
%:%5765=2069%:%
%:%5766=2070%:%
%:%5767=2070%:%
%:%5768=2070%:%
%:%5769=2071%:%
%:%5770=2071%:%
%:%5771=2072%:%
%:%5772=2072%:%
%:%5773=2072%:%
%:%5774=2073%:%
%:%5775=2073%:%
%:%5776=2074%:%
%:%5777=2074%:%
%:%5778=2075%:%
%:%5779=2075%:%
%:%5780=2075%:%
%:%5781=2076%:%
%:%5782=2076%:%
%:%5783=2076%:%
%:%5784=2077%:%
%:%5785=2077%:%
%:%5786=2078%:%
%:%5787=2078%:%
%:%5788=2078%:%
%:%5789=2079%:%
%:%5790=2079%:%
%:%5791=2079%:%
%:%5792=2080%:%
%:%5793=2080%:%
%:%5794=2080%:%
%:%5795=2081%:%
%:%5796=2081%:%
%:%5797=2081%:%
%:%5798=2081%:%
%:%5799=2082%:%
%:%5800=2082%:%
%:%5801=2082%:%
%:%5802=2083%:%
%:%5803=2083%:%
%:%5804=2084%:%
%:%5805=2084%:%
%:%5806=2085%:%
%:%5807=2085%:%
%:%5808=2085%:%
%:%5809=2086%:%
%:%5810=2086%:%
%:%5811=2087%:%
%:%5812=2087%:%
%:%5813=2088%:%
%:%5814=2088%:%
%:%5815=2089%:%
%:%5816=2089%:%
%:%5817=2089%:%
%:%5818=2090%:%
%:%5819=2090%:%
%:%5820=2090%:%
%:%5821=2091%:%
%:%5822=2091%:%
%:%5823=2091%:%
%:%5824=2092%:%
%:%5825=2092%:%
%:%5826=2093%:%
%:%5827=2093%:%
%:%5828=2094%:%
%:%5829=2094%:%
%:%5830=2095%:%
%:%5831=2095%:%
%:%5832=2096%:%
%:%5833=2096%:%
%:%5834=2096%:%
%:%5835=2097%:%
%:%5836=2097%:%
%:%5837=2097%:%
%:%5838=2098%:%
%:%5839=2098%:%
%:%5840=2098%:%
%:%5841=2099%:%
%:%5842=2099%:%
%:%5843=2099%:%
%:%5844=2100%:%
%:%5845=2100%:%
%:%5846=2101%:%
%:%5847=2101%:%
%:%5848=2101%:%
%:%5849=2102%:%
%:%5850=2102%:%
%:%5851=2102%:%
%:%5852=2103%:%
%:%5853=2103%:%
%:%5854=2103%:%
%:%5855=2104%:%
%:%5856=2104%:%
%:%5857=2104%:%
%:%5858=2104%:%
%:%5859=2105%:%
%:%5860=2105%:%
%:%5861=2105%:%
%:%5862=2106%:%
%:%5863=2106%:%
%:%5864=2106%:%
%:%5865=2107%:%
%:%5866=2107%:%
%:%5867=2107%:%
%:%5868=2108%:%
%:%5869=2108%:%
%:%5870=2108%:%
%:%5871=2109%:%
%:%5872=2109%:%
%:%5873=2110%:%
%:%5874=2110%:%
%:%5875=2111%:%
%:%5876=2111%:%
%:%5877=2112%:%
%:%5878=2112%:%
%:%5879=2113%:%
%:%5880=2113%:%
%:%5881=2113%:%
%:%5882=2114%:%
%:%5883=2114%:%
%:%5884=2114%:%
%:%5885=2115%:%
%:%5886=2115%:%
%:%5887=2115%:%
%:%5888=2116%:%
%:%5889=2116%:%
%:%5890=2117%:%
%:%5891=2117%:%
%:%5892=2118%:%
%:%5893=2118%:%
%:%5894=2119%:%
%:%5895=2119%:%
%:%5896=2120%:%
%:%5897=2120%:%
%:%5898=2120%:%
%:%5899=2121%:%
%:%5900=2121%:%
%:%5901=2122%:%
%:%5902=2122%:%
%:%5903=2123%:%
%:%5904=2123%:%
%:%5905=2124%:%
%:%5906=2124%:%
%:%5907=2125%:%
%:%5908=2125%:%
%:%5909=2125%:%
%:%5910=2126%:%
%:%5911=2126%:%
%:%5912=2126%:%
%:%5913=2127%:%
%:%5914=2127%:%
%:%5915=2128%:%
%:%5916=2128%:%
%:%5917=2128%:%
%:%5918=2129%:%
%:%5919=2129%:%
%:%5920=2130%:%
%:%5921=2130%:%
%:%5922=2130%:%
%:%5923=2131%:%
%:%5924=2131%:%
%:%5925=2131%:%
%:%5926=2132%:%
%:%5927=2132%:%
%:%5928=2132%:%
%:%5929=2133%:%
%:%5930=2133%:%
%:%5931=2133%:%
%:%5932=2133%:%
%:%5933=2134%:%
%:%5934=2134%:%
%:%5935=2134%:%
%:%5936=2135%:%
%:%5937=2135%:%
%:%5938=2135%:%
%:%5939=2136%:%
%:%5940=2136%:%
%:%5941=2136%:%
%:%5942=2137%:%
%:%5943=2137%:%
%:%5944=2138%:%
%:%5945=2138%:%
%:%5946=2139%:%
%:%5947=2139%:%
%:%5948=2140%:%
%:%5949=2140%:%
%:%5950=2141%:%
%:%5951=2141%:%
%:%5952=2141%:%
%:%5953=2142%:%
%:%5954=2142%:%
%:%5955=2142%:%
%:%5956=2143%:%
%:%5957=2143%:%
%:%5958=2143%:%
%:%5959=2144%:%
%:%5960=2144%:%
%:%5961=2145%:%
%:%5962=2145%:%
%:%5963=2145%:%
%:%5964=2146%:%
%:%5965=2146%:%
%:%5966=2147%:%
%:%5967=2147%:%
%:%5968=2148%:%
%:%5969=2148%:%
%:%5970=2149%:%
%:%5971=2149%:%
%:%5972=2149%:%
%:%5973=2150%:%
%:%5974=2150%:%
%:%5975=2151%:%
%:%5976=2151%:%
%:%5977=2151%:%
%:%5978=2152%:%
%:%5979=2152%:%
%:%5980=2152%:%
%:%5981=2153%:%
%:%5982=2153%:%
%:%5983=2153%:%
%:%5984=2154%:%
%:%5985=2154%:%
%:%5986=2154%:%
%:%5987=2154%:%
%:%5988=2155%:%
%:%5989=2155%:%
%:%5990=2155%:%
%:%5991=2156%:%
%:%5992=2156%:%
%:%5993=2157%:%
%:%5994=2157%:%
%:%5995=2158%:%
%:%5996=2158%:%
%:%5997=2158%:%
%:%5998=2159%:%
%:%5999=2159%:%
%:%6000=2160%:%
%:%6001=2160%:%
%:%6002=2161%:%
%:%6003=2161%:%
%:%6004=2162%:%
%:%6005=2162%:%
%:%6006=2163%:%
%:%6007=2164%:%
%:%6008=2164%:%
%:%6009=2165%:%
%:%6010=2165%:%
%:%6011=2165%:%
%:%6012=2166%:%
%:%6013=2166%:%
%:%6014=2167%:%
%:%6015=2167%:%
%:%6016=2168%:%
%:%6017=2168%:%
%:%6018=2169%:%
%:%6019=2169%:%
%:%6020=2169%:%
%:%6021=2170%:%
%:%6022=2170%:%
%:%6023=2170%:%
%:%6024=2171%:%
%:%6025=2171%:%
%:%6026=2171%:%
%:%6027=2172%:%
%:%6033=2172%:%
%:%6036=2173%:%
%:%6037=2174%:%
%:%6038=2174%:%
%:%6039=2175%:%
%:%6042=2176%:%
%:%6046=2176%:%
%:%6047=2176%:%
%:%6052=2176%:%
%:%6055=2177%:%
%:%6056=2178%:%
%:%6057=2178%:%
%:%6058=2179%:%
%:%6065=2180%:%
%:%6066=2180%:%
%:%6067=2181%:%
%:%6068=2181%:%
%:%6069=2182%:%
%:%6070=2182%:%
%:%6071=2183%:%
%:%6072=2183%:%
%:%6073=2184%:%
%:%6074=2184%:%
%:%6075=2185%:%
%:%6076=2185%:%
%:%6077=2186%:%
%:%6078=2186%:%
%:%6079=2187%:%
%:%6080=2187%:%
%:%6081=2187%:%
%:%6082=2187%:%
%:%6083=2188%:%
%:%6084=2188%:%
%:%6085=2188%:%
%:%6086=2189%:%
%:%6087=2189%:%
%:%6088=2189%:%
%:%6089=2189%:%
%:%6090=2190%:%
%:%6091=2190%:%
%:%6092=2190%:%
%:%6093=2191%:%
%:%6094=2191%:%
%:%6095=2192%:%
%:%6096=2192%:%
%:%6097=2193%:%
%:%6098=2193%:%
%:%6099=2194%:%
%:%6100=2194%:%
%:%6101=2195%:%
%:%6102=2195%:%
%:%6103=2195%:%
%:%6104=2196%:%
%:%6105=2196%:%
%:%6106=2197%:%
%:%6107=2197%:%
%:%6108=2198%:%
%:%6109=2198%:%
%:%6110=2199%:%
%:%6111=2199%:%
%:%6112=2200%:%
%:%6113=2200%:%
%:%6114=2200%:%
%:%6115=2201%:%
%:%6116=2201%:%
%:%6117=2201%:%
%:%6118=2202%:%
%:%6119=2202%:%
%:%6120=2202%:%
%:%6121=2202%:%
%:%6122=2203%:%
%:%6123=2203%:%
%:%6124=2203%:%
%:%6125=2204%:%
%:%6126=2204%:%
%:%6127=2205%:%
%:%6128=2205%:%
%:%6129=2205%:%
%:%6130=2206%:%
%:%6131=2206%:%
%:%6132=2207%:%
%:%6133=2207%:%
%:%6134=2208%:%
%:%6135=2208%:%
%:%6136=2209%:%
%:%6137=2209%:%
%:%6138=2210%:%
%:%6139=2210%:%
%:%6140=2211%:%
%:%6141=2211%:%
%:%6142=2211%:%
%:%6143=2212%:%
%:%6144=2212%:%
%:%6145=2213%:%
%:%6146=2213%:%
%:%6147=2214%:%
%:%6148=2214%:%
%:%6149=2215%:%
%:%6150=2215%:%
%:%6151=2216%:%
%:%6152=2216%:%
%:%6153=2216%:%
%:%6154=2217%:%
%:%6155=2217%:%
%:%6156=2217%:%
%:%6157=2218%:%
%:%6158=2218%:%
%:%6159=2218%:%
%:%6160=2218%:%
%:%6161=2219%:%
%:%6162=2219%:%
%:%6163=2219%:%
%:%6164=2220%:%
%:%6165=2220%:%
%:%6166=2221%:%
%:%6167=2221%:%
%:%6168=2221%:%
%:%6169=2222%:%
%:%6170=2222%:%
%:%6171=2223%:%
%:%6172=2223%:%
%:%6173=2224%:%
%:%6174=2224%:%
%:%6175=2224%:%
%:%6176=2225%:%
%:%6177=2225%:%
%:%6178=2225%:%
%:%6179=2226%:%
%:%6180=2226%:%
%:%6181=2227%:%
%:%6182=2227%:%
%:%6183=2228%:%
%:%6184=2228%:%
%:%6185=2228%:%
%:%6186=2229%:%
%:%6187=2229%:%
%:%6188=2230%:%
%:%6189=2230%:%
%:%6190=2231%:%
%:%6191=2231%:%
%:%6192=2232%:%
%:%6193=2232%:%
%:%6194=2233%:%
%:%6195=2233%:%
%:%6196=2233%:%
%:%6197=2234%:%
%:%6198=2234%:%
%:%6199=2234%:%
%:%6200=2235%:%
%:%6201=2235%:%
%:%6202=2235%:%
%:%6203=2235%:%
%:%6204=2236%:%
%:%6205=2236%:%
%:%6206=2236%:%
%:%6207=2237%:%
%:%6208=2237%:%
%:%6209=2238%:%
%:%6210=2238%:%
%:%6211=2238%:%
%:%6212=2239%:%
%:%6213=2239%:%
%:%6214=2240%:%
%:%6215=2240%:%
%:%6216=2241%:%
%:%6217=2241%:%
%:%6218=2241%:%
%:%6219=2242%:%
%:%6220=2242%:%
%:%6221=2242%:%
%:%6222=2243%:%
%:%6223=2243%:%
%:%6224=2244%:%
%:%6225=2244%:%
%:%6226=2245%:%
%:%6227=2245%:%
%:%6228=2246%:%
%:%6229=2246%:%
%:%6230=2246%:%
%:%6231=2247%:%
%:%6232=2247%:%
%:%6233=2248%:%
%:%6234=2248%:%
%:%6235=2249%:%
%:%6236=2249%:%
%:%6237=2250%:%
%:%6238=2250%:%
%:%6239=2251%:%
%:%6240=2251%:%
%:%6241=2251%:%
%:%6242=2252%:%
%:%6243=2252%:%
%:%6244=2252%:%
%:%6245=2253%:%
%:%6246=2253%:%
%:%6247=2253%:%
%:%6248=2253%:%
%:%6249=2254%:%
%:%6250=2254%:%
%:%6251=2254%:%
%:%6252=2255%:%
%:%6253=2255%:%
%:%6254=2256%:%
%:%6255=2256%:%
%:%6256=2256%:%
%:%6257=2257%:%
%:%6258=2257%:%
%:%6259=2258%:%
%:%6260=2258%:%
%:%6261=2259%:%
%:%6262=2259%:%
%:%6263=2260%:%
%:%6264=2260%:%
%:%6265=2261%:%
%:%6266=2261%:%
%:%6267=2262%:%
%:%6268=2262%:%
%:%6269=2262%:%
%:%6270=2263%:%
%:%6271=2263%:%
%:%6272=2264%:%
%:%6273=2264%:%
%:%6274=2265%:%
%:%6275=2265%:%
%:%6276=2266%:%
%:%6277=2266%:%
%:%6278=2267%:%
%:%6279=2267%:%
%:%6280=2267%:%
%:%6281=2268%:%
%:%6282=2268%:%
%:%6283=2268%:%
%:%6284=2269%:%
%:%6285=2269%:%
%:%6286=2269%:%
%:%6287=2269%:%
%:%6288=2270%:%
%:%6289=2270%:%
%:%6290=2270%:%
%:%6291=2271%:%
%:%6292=2271%:%
%:%6293=2272%:%
%:%6294=2272%:%
%:%6295=2272%:%
%:%6296=2273%:%
%:%6297=2273%:%
%:%6298=2274%:%
%:%6299=2274%:%
%:%6300=2275%:%
%:%6301=2275%:%
%:%6302=2275%:%
%:%6303=2276%:%
%:%6304=2276%:%
%:%6305=2276%:%
%:%6306=2277%:%
%:%6307=2277%:%
%:%6308=2278%:%
%:%6309=2278%:%
%:%6310=2279%:%
%:%6311=2279%:%
%:%6312=2279%:%
%:%6313=2280%:%
%:%6314=2280%:%
%:%6315=2280%:%
%:%6316=2281%:%
%:%6317=2281%:%
%:%6318=2282%:%
%:%6319=2282%:%
%:%6320=2283%:%
%:%6321=2283%:%
%:%6322=2284%:%
%:%6323=2284%:%
%:%6324=2284%:%
%:%6325=2285%:%
%:%6326=2285%:%
%:%6327=2285%:%
%:%6328=2286%:%
%:%6329=2286%:%
%:%6330=2286%:%
%:%6331=2286%:%
%:%6332=2287%:%
%:%6333=2287%:%
%:%6334=2287%:%
%:%6335=2288%:%
%:%6336=2288%:%
%:%6337=2289%:%
%:%6338=2289%:%
%:%6339=2289%:%
%:%6340=2290%:%
%:%6341=2290%:%
%:%6342=2291%:%
%:%6343=2291%:%
%:%6344=2292%:%
%:%6345=2292%:%
%:%6346=2293%:%
%:%6347=2293%:%
%:%6348=2293%:%
%:%6349=2294%:%
%:%6350=2294%:%
%:%6351=2295%:%
%:%6352=2295%:%
%:%6353=2296%:%
%:%6354=2296%:%
%:%6355=2297%:%
%:%6356=2297%:%
%:%6357=2298%:%
%:%6358=2298%:%
%:%6359=2299%:%
%:%6360=2299%:%
%:%6361=2300%:%
%:%6362=2300%:%
%:%6363=2301%:%
%:%6364=2301%:%
%:%6365=2301%:%
%:%6366=2301%:%
%:%6367=2302%:%
%:%6368=2302%:%
%:%6369=2302%:%
%:%6370=2303%:%
%:%6371=2303%:%
%:%6372=2303%:%
%:%6373=2304%:%
%:%6374=2304%:%
%:%6375=2304%:%
%:%6376=2305%:%
%:%6377=2305%:%
%:%6378=2306%:%
%:%6379=2306%:%
%:%6380=2307%:%
%:%6381=2307%:%
%:%6382=2308%:%
%:%6383=2308%:%
%:%6384=2309%:%
%:%6385=2309%:%
%:%6386=2310%:%
%:%6387=2310%:%
%:%6388=2311%:%
%:%6389=2311%:%
%:%6390=2312%:%
%:%6391=2312%:%
%:%6392=2313%:%
%:%6393=2313%:%
%:%6394=2313%:%
%:%6395=2314%:%
%:%6396=2314%:%
%:%6397=2314%:%
%:%6398=2315%:%
%:%6399=2315%:%
%:%6400=2315%:%
%:%6401=2315%:%
%:%6402=2316%:%
%:%6403=2316%:%
%:%6404=2316%:%
%:%6405=2317%:%
%:%6406=2317%:%
%:%6407=2317%:%
%:%6408=2318%:%
%:%6409=2318%:%
%:%6410=2318%:%
%:%6411=2319%:%
%:%6412=2319%:%
%:%6413=2320%:%
%:%6414=2320%:%
%:%6415=2321%:%
%:%6416=2321%:%
%:%6417=2321%:%
%:%6418=2322%:%
%:%6419=2322%:%
%:%6420=2323%:%
%:%6421=2323%:%
%:%6422=2324%:%
%:%6423=2324%:%
%:%6424=2324%:%
%:%6425=2324%:%
%:%6426=2325%:%
%:%6427=2325%:%
%:%6428=2326%:%
%:%6429=2326%:%
%:%6430=2327%:%
%:%6431=2327%:%
%:%6432=2328%:%
%:%6433=2328%:%
%:%6434=2329%:%
%:%6435=2329%:%
%:%6436=2330%:%
%:%6437=2330%:%
%:%6438=2331%:%
%:%6439=2331%:%
%:%6440=2331%:%
%:%6441=2332%:%
%:%6442=2332%:%
%:%6443=2332%:%
%:%6444=2333%:%
%:%6445=2333%:%
%:%6446=2333%:%
%:%6447=2333%:%
%:%6448=2334%:%
%:%6449=2334%:%
%:%6450=2334%:%
%:%6451=2335%:%
%:%6452=2335%:%
%:%6453=2335%:%
%:%6454=2336%:%
%:%6455=2336%:%
%:%6456=2336%:%
%:%6457=2337%:%
%:%6458=2337%:%
%:%6459=2338%:%
%:%6460=2338%:%
%:%6461=2339%:%
%:%6462=2339%:%
%:%6463=2339%:%
%:%6464=2340%:%
%:%6465=2340%:%
%:%6466=2341%:%
%:%6467=2341%:%
%:%6468=2342%:%
%:%6469=2342%:%
%:%6470=2343%:%
%:%6471=2343%:%
%:%6472=2343%:%
%:%6473=2344%:%
%:%6474=2344%:%
%:%6475=2345%:%
%:%6481=2345%:%
%:%6484=2346%:%
%:%6485=2347%:%
%:%6486=2347%:%
%:%6487=2348%:%
%:%6490=2349%:%
%:%6494=2349%:%
%:%6495=2349%:%
%:%6504=2351%:%
%:%6506=2352%:%
%:%6507=2352%:%
%:%6514=2353%:%
%:%6515=2353%:%
%:%6516=2354%:%
%:%6517=2355%:%
%:%6518=2355%:%
%:%6519=2356%:%
%:%6520=2356%:%
%:%6521=2357%:%
%:%6522=2357%:%
%:%6523=2358%:%
%:%6524=2358%:%
%:%6525=2359%:%
%:%6526=2359%:%
%:%6527=2360%:%
%:%6528=2360%:%
%:%6529=2361%:%
%:%6530=2361%:%
%:%6531=2362%:%
%:%6532=2362%:%
%:%6533=2363%:%
%:%6534=2363%:%
%:%6535=2364%:%
%:%6536=2364%:%
%:%6537=2365%:%
%:%6538=2365%:%
%:%6539=2366%:%
%:%6540=2366%:%
%:%6541=2367%:%
%:%6542=2368%:%
%:%6543=2368%:%
%:%6544=2369%:%
%:%6545=2369%:%
%:%6546=2370%:%
%:%6547=2370%:%
%:%6548=2371%:%
%:%6549=2371%:%
%:%6550=2371%:%
%:%6551=2372%:%
%:%6552=2372%:%
%:%6553=2373%:%
%:%6559=2373%:%
%:%6564=2374%:%
%:%6569=2375%:%

%
\begin{isabellebody}%
\setisabellecontext{Axiom{\isacharunderscore}{\kern0pt}Of{\isacharunderscore}{\kern0pt}Choice}%
%
\isadelimdocument
%
\endisadelimdocument
%
\isatagdocument
%
\isamarkupsection{Axiom of Choice%
}
\isamarkuptrue%
%
\endisatagdocument
{\isafolddocument}%
%
\isadelimdocument
%
\endisadelimdocument
%
\isadelimtheory
%
\endisadelimtheory
%
\isatagtheory
\isacommand{theory}\isamarkupfalse%
\ Axiom{\isacharunderscore}{\kern0pt}Of{\isacharunderscore}{\kern0pt}Choice\isanewline
\ \ \isakeyword{imports}\ Coproduct\isanewline
\isakeyword{begin}%
\endisatagtheory
{\isafoldtheory}%
%
\isadelimtheory
%
\endisadelimtheory
%
\begin{isamarkuptext}%
The two definitions below correspond to Definition 2.7.1 in Halvorson.%
\end{isamarkuptext}\isamarkuptrue%
\isacommand{definition}\isamarkupfalse%
\ section{\isacharunderscore}{\kern0pt}of\ {\isacharcolon}{\kern0pt}{\isacharcolon}{\kern0pt}\ {\isachardoublequoteopen}cfunc\ {\isasymRightarrow}\ cfunc\ {\isasymRightarrow}\ bool{\isachardoublequoteclose}\ {\isacharparenleft}{\kern0pt}\isakeyword{infix}\ {\isachardoublequoteopen}sectionof{\isachardoublequoteclose}\ {\isadigit{9}}{\isadigit{0}}{\isacharparenright}{\kern0pt}\isanewline
\ \ \isakeyword{where}\ {\isachardoublequoteopen}s\ sectionof\ f\ {\isasymlongleftrightarrow}\ s\ {\isacharcolon}{\kern0pt}\ codomain\ f\ {\isasymrightarrow}\ domain\ f\ {\isasymand}\ f\ {\isasymcirc}\isactrlsub c\ s\ {\isacharequal}{\kern0pt}\ id\ {\isacharparenleft}{\kern0pt}codomain\ f{\isacharparenright}{\kern0pt}{\isachardoublequoteclose}\isanewline
\isanewline
\isacommand{definition}\isamarkupfalse%
\ split{\isacharunderscore}{\kern0pt}epimorphism\ {\isacharcolon}{\kern0pt}{\isacharcolon}{\kern0pt}\ {\isachardoublequoteopen}cfunc\ {\isasymRightarrow}\ bool{\isachardoublequoteclose}\isanewline
\ \ \isakeyword{where}\ {\isachardoublequoteopen}split{\isacharunderscore}{\kern0pt}epimorphism\ f\ {\isasymlongleftrightarrow}\ {\isacharparenleft}{\kern0pt}{\isasymexists}\ s{\isachardot}{\kern0pt}\ \ s\ {\isacharcolon}{\kern0pt}\ codomain\ f\ {\isasymrightarrow}\ domain\ f\ {\isasymand}\ f\ {\isasymcirc}\isactrlsub c\ s\ {\isacharequal}{\kern0pt}\ id\ {\isacharparenleft}{\kern0pt}codomain\ f{\isacharparenright}{\kern0pt}{\isacharparenright}{\kern0pt}{\isachardoublequoteclose}\isanewline
\isanewline
\isacommand{lemma}\isamarkupfalse%
\ split{\isacharunderscore}{\kern0pt}epimorphism{\isacharunderscore}{\kern0pt}def{\isadigit{2}}{\isacharcolon}{\kern0pt}\ \isanewline
\ \ \isakeyword{assumes}\ f{\isacharunderscore}{\kern0pt}type{\isacharcolon}{\kern0pt}\ {\isachardoublequoteopen}f\ {\isacharcolon}{\kern0pt}\ X\ {\isasymrightarrow}\ Y{\isachardoublequoteclose}\isanewline
\ \ \isakeyword{assumes}\ f{\isacharunderscore}{\kern0pt}split{\isacharunderscore}{\kern0pt}epic{\isacharcolon}{\kern0pt}\ {\isachardoublequoteopen}split{\isacharunderscore}{\kern0pt}epimorphism\ f{\isachardoublequoteclose}\isanewline
\ \ \isakeyword{shows}\ {\isachardoublequoteopen}{\isasymexists}\ s{\isachardot}{\kern0pt}\ {\isacharparenleft}{\kern0pt}f\ {\isasymcirc}\isactrlsub c\ s\ {\isacharequal}{\kern0pt}\ id\ Y{\isacharparenright}{\kern0pt}\ {\isasymand}\ {\isacharparenleft}{\kern0pt}s{\isacharcolon}{\kern0pt}\ Y\ {\isasymrightarrow}\ X{\isacharparenright}{\kern0pt}{\isachardoublequoteclose}\isanewline
%
\isadelimproof
\ \ %
\endisadelimproof
%
\isatagproof
\isacommand{using}\isamarkupfalse%
\ cfunc{\isacharunderscore}{\kern0pt}type{\isacharunderscore}{\kern0pt}def\ f{\isacharunderscore}{\kern0pt}split{\isacharunderscore}{\kern0pt}epic\ f{\isacharunderscore}{\kern0pt}type\ split{\isacharunderscore}{\kern0pt}epimorphism{\isacharunderscore}{\kern0pt}def\ \isacommand{by}\isamarkupfalse%
\ auto%
\endisatagproof
{\isafoldproof}%
%
\isadelimproof
\isanewline
%
\endisadelimproof
\isanewline
\isacommand{lemma}\isamarkupfalse%
\ sections{\isacharunderscore}{\kern0pt}define{\isacharunderscore}{\kern0pt}splits{\isacharcolon}{\kern0pt}\isanewline
\ \ \isakeyword{assumes}\ {\isachardoublequoteopen}s\ sectionof\ f{\isachardoublequoteclose}\isanewline
\ \ \isakeyword{assumes}\ {\isachardoublequoteopen}s\ {\isacharcolon}{\kern0pt}\ Y\ {\isasymrightarrow}\ X{\isachardoublequoteclose}\isanewline
\ \ \isakeyword{shows}\ {\isachardoublequoteopen}f\ {\isacharcolon}{\kern0pt}\ X\ {\isasymrightarrow}\ Y\ {\isasymand}\ split{\isacharunderscore}{\kern0pt}epimorphism{\isacharparenleft}{\kern0pt}f{\isacharparenright}{\kern0pt}{\isachardoublequoteclose}\isanewline
%
\isadelimproof
\ \ %
\endisadelimproof
%
\isatagproof
\isacommand{using}\isamarkupfalse%
\ assms\ cfunc{\isacharunderscore}{\kern0pt}type{\isacharunderscore}{\kern0pt}def\ section{\isacharunderscore}{\kern0pt}of{\isacharunderscore}{\kern0pt}def\ split{\isacharunderscore}{\kern0pt}epimorphism{\isacharunderscore}{\kern0pt}def\ \isacommand{by}\isamarkupfalse%
\ auto%
\endisatagproof
{\isafoldproof}%
%
\isadelimproof
%
\endisadelimproof
%
\begin{isamarkuptext}%
The axiomatization below corresponds to Axiom 11 (Axiom of Choice) in Halvorson.%
\end{isamarkuptext}\isamarkuptrue%
\isacommand{axiomatization}\isamarkupfalse%
\isanewline
\ \ \isakeyword{where}\isanewline
\ \ axiom{\isacharunderscore}{\kern0pt}of{\isacharunderscore}{\kern0pt}choice{\isacharcolon}{\kern0pt}\ {\isachardoublequoteopen}epimorphism\ f\ {\isasymlongrightarrow}\ {\isacharparenleft}{\kern0pt}{\isasymexists}\ g\ {\isachardot}{\kern0pt}\ g\ sectionof\ f{\isacharparenright}{\kern0pt}{\isachardoublequoteclose}\isanewline
\isanewline
\isacommand{lemma}\isamarkupfalse%
\ epis{\isacharunderscore}{\kern0pt}give{\isacharunderscore}{\kern0pt}monos{\isacharcolon}{\kern0pt}\ \ \isanewline
\ \ \isakeyword{assumes}\ f{\isacharunderscore}{\kern0pt}type{\isacharcolon}{\kern0pt}\ {\isachardoublequoteopen}f\ {\isacharcolon}{\kern0pt}\ X\ {\isasymrightarrow}\ Y{\isachardoublequoteclose}\isanewline
\ \ \isakeyword{assumes}\ f{\isacharunderscore}{\kern0pt}epi{\isacharcolon}{\kern0pt}\ {\isachardoublequoteopen}epimorphism\ f{\isachardoublequoteclose}\isanewline
\ \ \isakeyword{shows}\ {\isachardoublequoteopen}{\isasymexists}g{\isachardot}{\kern0pt}\ g{\isacharcolon}{\kern0pt}\ Y\ {\isasymrightarrow}\ X\ {\isasymand}\ monomorphism\ g\ {\isasymand}\ f\ {\isasymcirc}\isactrlsub c\ g\ {\isacharequal}{\kern0pt}\ id\ Y{\isachardoublequoteclose}\isanewline
%
\isadelimproof
\ \ %
\endisadelimproof
%
\isatagproof
\isacommand{using}\isamarkupfalse%
\ assms\ \ \isanewline
\ \ \isacommand{by}\isamarkupfalse%
\ {\isacharparenleft}{\kern0pt}typecheck{\isacharunderscore}{\kern0pt}cfuncs{\isacharunderscore}{\kern0pt}prems{\isacharcomma}{\kern0pt}\ metis\ axiom{\isacharunderscore}{\kern0pt}of{\isacharunderscore}{\kern0pt}choice\ cfunc{\isacharunderscore}{\kern0pt}type{\isacharunderscore}{\kern0pt}def\ comp{\isacharunderscore}{\kern0pt}monic{\isacharunderscore}{\kern0pt}imp{\isacharunderscore}{\kern0pt}monic\ f{\isacharunderscore}{\kern0pt}epi\ id{\isacharunderscore}{\kern0pt}isomorphism\ iso{\isacharunderscore}{\kern0pt}imp{\isacharunderscore}{\kern0pt}epi{\isacharunderscore}{\kern0pt}and{\isacharunderscore}{\kern0pt}monic\ section{\isacharunderscore}{\kern0pt}of{\isacharunderscore}{\kern0pt}def{\isacharparenright}{\kern0pt}%
\endisatagproof
{\isafoldproof}%
%
\isadelimproof
\isanewline
%
\endisadelimproof
\isanewline
\isacommand{corollary}\isamarkupfalse%
\ epis{\isacharunderscore}{\kern0pt}are{\isacharunderscore}{\kern0pt}split{\isacharcolon}{\kern0pt}\isanewline
\ \ \isakeyword{assumes}\ f{\isacharunderscore}{\kern0pt}type{\isacharcolon}{\kern0pt}\ {\isachardoublequoteopen}f\ {\isacharcolon}{\kern0pt}\ X\ {\isasymrightarrow}\ Y{\isachardoublequoteclose}\isanewline
\ \ \isakeyword{assumes}\ f{\isacharunderscore}{\kern0pt}epi{\isacharcolon}{\kern0pt}\ {\isachardoublequoteopen}epimorphism\ f{\isachardoublequoteclose}\isanewline
\ \ \isakeyword{shows}\ {\isachardoublequoteopen}split{\isacharunderscore}{\kern0pt}epimorphism\ f{\isachardoublequoteclose}\isanewline
%
\isadelimproof
\ \ %
\endisadelimproof
%
\isatagproof
\isacommand{using}\isamarkupfalse%
\ epis{\isacharunderscore}{\kern0pt}give{\isacharunderscore}{\kern0pt}monos\ cfunc{\isacharunderscore}{\kern0pt}type{\isacharunderscore}{\kern0pt}def\ \ f{\isacharunderscore}{\kern0pt}epi\ split{\isacharunderscore}{\kern0pt}epimorphism{\isacharunderscore}{\kern0pt}def\ \isacommand{by}\isamarkupfalse%
\ blast%
\endisatagproof
{\isafoldproof}%
%
\isadelimproof
%
\endisadelimproof
%
\begin{isamarkuptext}%
The lemma below corresponds to Proposition 2.6.8 in Halvorson.%
\end{isamarkuptext}\isamarkuptrue%
\isacommand{lemma}\isamarkupfalse%
\ monos{\isacharunderscore}{\kern0pt}give{\isacharunderscore}{\kern0pt}epis{\isacharcolon}{\kern0pt}\isanewline
\ \ \isakeyword{assumes}\ f{\isacharunderscore}{\kern0pt}type{\isacharbrackleft}{\kern0pt}type{\isacharunderscore}{\kern0pt}rule{\isacharbrackright}{\kern0pt}{\isacharcolon}{\kern0pt}\ {\isachardoublequoteopen}f\ {\isacharcolon}{\kern0pt}\ X\ {\isasymrightarrow}\ Y{\isachardoublequoteclose}\isanewline
\ \ \isakeyword{assumes}\ f{\isacharunderscore}{\kern0pt}mono{\isacharcolon}{\kern0pt}\ {\isachardoublequoteopen}monomorphism\ f{\isachardoublequoteclose}\isanewline
\ \ \isakeyword{assumes}\ X{\isacharunderscore}{\kern0pt}nonempty{\isacharcolon}{\kern0pt}\ {\isachardoublequoteopen}nonempty\ X{\isachardoublequoteclose}\isanewline
\ \ \isakeyword{shows}\ {\isachardoublequoteopen}{\isasymexists}g{\isachardot}{\kern0pt}\ g{\isacharcolon}{\kern0pt}\ Y\ {\isasymrightarrow}\ X\ {\isasymand}\ epimorphism\ g\ {\isasymand}\ g\ {\isasymcirc}\isactrlsub c\ f\ {\isacharequal}{\kern0pt}\ id\ X{\isachardoublequoteclose}\isanewline
%
\isadelimproof
%
\endisadelimproof
%
\isatagproof
\isacommand{proof}\isamarkupfalse%
\ {\isacharminus}{\kern0pt}\isanewline
\ \ \isacommand{obtain}\isamarkupfalse%
\ g\ m\ E\ \isakeyword{where}\ g{\isacharunderscore}{\kern0pt}type{\isacharbrackleft}{\kern0pt}type{\isacharunderscore}{\kern0pt}rule{\isacharbrackright}{\kern0pt}{\isacharcolon}{\kern0pt}\ {\isachardoublequoteopen}g\ {\isacharcolon}{\kern0pt}\ X\ {\isasymrightarrow}\ E{\isachardoublequoteclose}\ \isakeyword{and}\ m{\isacharunderscore}{\kern0pt}type{\isacharbrackleft}{\kern0pt}type{\isacharunderscore}{\kern0pt}rule{\isacharbrackright}{\kern0pt}{\isacharcolon}{\kern0pt}\ {\isachardoublequoteopen}m\ {\isacharcolon}{\kern0pt}\ E\ {\isasymrightarrow}\ Y{\isachardoublequoteclose}\ \isakeyword{and}\isanewline
\ \ \ \ \ \ g{\isacharunderscore}{\kern0pt}epi{\isacharcolon}{\kern0pt}\ {\isachardoublequoteopen}epimorphism\ g{\isachardoublequoteclose}\ \isakeyword{and}\ m{\isacharunderscore}{\kern0pt}mono{\isacharbrackleft}{\kern0pt}type{\isacharunderscore}{\kern0pt}rule{\isacharbrackright}{\kern0pt}{\isacharcolon}{\kern0pt}\ {\isachardoublequoteopen}monomorphism\ m{\isachardoublequoteclose}\ \isakeyword{and}\ f{\isacharunderscore}{\kern0pt}eq{\isacharcolon}{\kern0pt}\ {\isachardoublequoteopen}f\ {\isacharequal}{\kern0pt}\ m\ {\isasymcirc}\isactrlsub c\ g{\isachardoublequoteclose}\isanewline
\ \ \ \ \isacommand{using}\isamarkupfalse%
\ epi{\isacharunderscore}{\kern0pt}monic{\isacharunderscore}{\kern0pt}factorization{\isadigit{2}}\ f{\isacharunderscore}{\kern0pt}type\ \isacommand{by}\isamarkupfalse%
\ blast\isanewline
\isanewline
\ \ \isacommand{have}\isamarkupfalse%
\ g{\isacharunderscore}{\kern0pt}mono{\isacharcolon}{\kern0pt}\ {\isachardoublequoteopen}monomorphism\ g{\isachardoublequoteclose}\isanewline
\ \ \isacommand{proof}\isamarkupfalse%
\ {\isacharparenleft}{\kern0pt}typecheck{\isacharunderscore}{\kern0pt}cfuncs{\isacharcomma}{\kern0pt}\ unfold\ monomorphism{\isacharunderscore}{\kern0pt}def{\isadigit{3}}{\isacharcomma}{\kern0pt}\ clarify{\isacharparenright}{\kern0pt}\isanewline
\ \ \ \ \isacommand{fix}\isamarkupfalse%
\ x\ y\ A\isanewline
\ \ \ \ \isacommand{assume}\isamarkupfalse%
\ x{\isacharunderscore}{\kern0pt}type{\isacharbrackleft}{\kern0pt}type{\isacharunderscore}{\kern0pt}rule{\isacharbrackright}{\kern0pt}{\isacharcolon}{\kern0pt}\ {\isachardoublequoteopen}x\ {\isacharcolon}{\kern0pt}\ A\ {\isasymrightarrow}\ X{\isachardoublequoteclose}\ \isakeyword{and}\ y{\isacharunderscore}{\kern0pt}type{\isacharbrackleft}{\kern0pt}type{\isacharunderscore}{\kern0pt}rule{\isacharbrackright}{\kern0pt}{\isacharcolon}{\kern0pt}\ {\isachardoublequoteopen}y\ {\isacharcolon}{\kern0pt}\ A\ {\isasymrightarrow}\ X{\isachardoublequoteclose}\isanewline
\ \ \ \ \isacommand{assume}\isamarkupfalse%
\ {\isachardoublequoteopen}g\ {\isasymcirc}\isactrlsub c\ x\ {\isacharequal}{\kern0pt}\ g\ {\isasymcirc}\isactrlsub c\ y{\isachardoublequoteclose}\isanewline
\ \ \ \ \isacommand{then}\isamarkupfalse%
\ \isacommand{have}\isamarkupfalse%
\ {\isachardoublequoteopen}{\isacharparenleft}{\kern0pt}m\ {\isasymcirc}\isactrlsub c\ g{\isacharparenright}{\kern0pt}\ {\isasymcirc}\isactrlsub c\ x\ {\isacharequal}{\kern0pt}\ {\isacharparenleft}{\kern0pt}m\ {\isasymcirc}\isactrlsub c\ g{\isacharparenright}{\kern0pt}\ {\isasymcirc}\isactrlsub c\ y{\isachardoublequoteclose}\isanewline
\ \ \ \ \ \ \isacommand{by}\isamarkupfalse%
\ {\isacharparenleft}{\kern0pt}typecheck{\isacharunderscore}{\kern0pt}cfuncs{\isacharcomma}{\kern0pt}\ smt\ comp{\isacharunderscore}{\kern0pt}associative{\isadigit{2}}{\isacharparenright}{\kern0pt}\isanewline
\ \ \ \ \isacommand{then}\isamarkupfalse%
\ \isacommand{have}\isamarkupfalse%
\ {\isachardoublequoteopen}f\ {\isasymcirc}\isactrlsub c\ x\ {\isacharequal}{\kern0pt}\ f\ {\isasymcirc}\isactrlsub c\ y{\isachardoublequoteclose}\isanewline
\ \ \ \ \ \ \isacommand{unfolding}\isamarkupfalse%
\ f{\isacharunderscore}{\kern0pt}eq\ \isacommand{by}\isamarkupfalse%
\ auto\isanewline
\ \ \ \ \isacommand{then}\isamarkupfalse%
\ \isacommand{show}\isamarkupfalse%
\ {\isachardoublequoteopen}x\ {\isacharequal}{\kern0pt}\ y{\isachardoublequoteclose}\isanewline
\ \ \ \ \ \ \isacommand{using}\isamarkupfalse%
\ f{\isacharunderscore}{\kern0pt}mono\ f{\isacharunderscore}{\kern0pt}type\ monomorphism{\isacharunderscore}{\kern0pt}def{\isadigit{2}}\ x{\isacharunderscore}{\kern0pt}type\ y{\isacharunderscore}{\kern0pt}type\ \isacommand{by}\isamarkupfalse%
\ blast\isanewline
\ \ \isacommand{qed}\isamarkupfalse%
\isanewline
\isanewline
\ \ \isacommand{have}\isamarkupfalse%
\ g{\isacharunderscore}{\kern0pt}iso{\isacharcolon}{\kern0pt}\ {\isachardoublequoteopen}isomorphism\ g{\isachardoublequoteclose}\isanewline
\ \ \ \ \isacommand{by}\isamarkupfalse%
\ {\isacharparenleft}{\kern0pt}simp\ add{\isacharcolon}{\kern0pt}\ epi{\isacharunderscore}{\kern0pt}mon{\isacharunderscore}{\kern0pt}is{\isacharunderscore}{\kern0pt}iso\ g{\isacharunderscore}{\kern0pt}epi\ g{\isacharunderscore}{\kern0pt}mono{\isacharparenright}{\kern0pt}\isanewline
\ \ \isacommand{then}\isamarkupfalse%
\ \isacommand{obtain}\isamarkupfalse%
\ g{\isacharunderscore}{\kern0pt}inv\ \isakeyword{where}\ g{\isacharunderscore}{\kern0pt}inv{\isacharunderscore}{\kern0pt}type{\isacharbrackleft}{\kern0pt}type{\isacharunderscore}{\kern0pt}rule{\isacharbrackright}{\kern0pt}{\isacharcolon}{\kern0pt}\ {\isachardoublequoteopen}g{\isacharunderscore}{\kern0pt}inv\ {\isacharcolon}{\kern0pt}\ E\ {\isasymrightarrow}\ X{\isachardoublequoteclose}\ \isakeyword{and}\isanewline
\ \ \ \ \ \ g{\isacharunderscore}{\kern0pt}g{\isacharunderscore}{\kern0pt}inv{\isacharcolon}{\kern0pt}\ {\isachardoublequoteopen}g\ {\isasymcirc}\isactrlsub c\ g{\isacharunderscore}{\kern0pt}inv\ {\isacharequal}{\kern0pt}\ id\ E{\isachardoublequoteclose}\ \isakeyword{and}\ g{\isacharunderscore}{\kern0pt}inv{\isacharunderscore}{\kern0pt}g{\isacharcolon}{\kern0pt}\ {\isachardoublequoteopen}g{\isacharunderscore}{\kern0pt}inv\ {\isasymcirc}\isactrlsub c\ g\ {\isacharequal}{\kern0pt}\ id\ X{\isachardoublequoteclose}\isanewline
\ \ \ \ \isacommand{using}\isamarkupfalse%
\ cfunc{\isacharunderscore}{\kern0pt}type{\isacharunderscore}{\kern0pt}def\ g{\isacharunderscore}{\kern0pt}type\ isomorphism{\isacharunderscore}{\kern0pt}def\ \isacommand{by}\isamarkupfalse%
\ auto\isanewline
\isanewline
\ \ \isacommand{obtain}\isamarkupfalse%
\ x\ \isakeyword{where}\ x{\isacharunderscore}{\kern0pt}type{\isacharbrackleft}{\kern0pt}type{\isacharunderscore}{\kern0pt}rule{\isacharbrackright}{\kern0pt}{\isacharcolon}{\kern0pt}\ {\isachardoublequoteopen}x\ {\isasymin}\isactrlsub c\ X{\isachardoublequoteclose}\isanewline
\ \ \ \ \isacommand{using}\isamarkupfalse%
\ X{\isacharunderscore}{\kern0pt}nonempty\ nonempty{\isacharunderscore}{\kern0pt}def\ \isacommand{by}\isamarkupfalse%
\ blast\isanewline
\isanewline
\ \ \isacommand{show}\isamarkupfalse%
\ {\isachardoublequoteopen}{\isasymexists}g{\isachardot}{\kern0pt}\ g{\isacharcolon}{\kern0pt}\ Y\ {\isasymrightarrow}\ X\ {\isasymand}\ epimorphism\ g\ {\isasymand}\ g\ {\isasymcirc}\isactrlsub c\ f\ {\isacharequal}{\kern0pt}\ id\isactrlsub c\ X{\isachardoublequoteclose}\isanewline
\ \ \isacommand{proof}\isamarkupfalse%
\ {\isacharparenleft}{\kern0pt}rule{\isacharunderscore}{\kern0pt}tac\ x{\isacharequal}{\kern0pt}{\isachardoublequoteopen}{\isacharparenleft}{\kern0pt}g{\isacharunderscore}{\kern0pt}inv\ {\isasymamalg}\ {\isacharparenleft}{\kern0pt}x\ {\isasymcirc}\isactrlsub c\ {\isasymbeta}\isactrlbsub Y\ {\isasymsetminus}\ {\isacharparenleft}{\kern0pt}E{\isacharcomma}{\kern0pt}\ m{\isacharparenright}{\kern0pt}\isactrlesub {\isacharparenright}{\kern0pt}{\isacharparenright}{\kern0pt}\ {\isasymcirc}\isactrlsub c\ try{\isacharunderscore}{\kern0pt}cast\ m{\isachardoublequoteclose}\ \isakeyword{in}\ exI{\isacharcomma}{\kern0pt}\ safe{\isacharcomma}{\kern0pt}\ typecheck{\isacharunderscore}{\kern0pt}cfuncs{\isacharparenright}{\kern0pt}\isanewline
\ \ \ \ \isacommand{have}\isamarkupfalse%
\ func{\isacharunderscore}{\kern0pt}f{\isacharunderscore}{\kern0pt}elem{\isacharunderscore}{\kern0pt}eq{\isacharcolon}{\kern0pt}\ {\isachardoublequoteopen}{\isasymAnd}\ y{\isachardot}{\kern0pt}\ y\ {\isasymin}\isactrlsub c\ X\ {\isasymLongrightarrow}\ {\isacharparenleft}{\kern0pt}g{\isacharunderscore}{\kern0pt}inv\ {\isasymamalg}\ {\isacharparenleft}{\kern0pt}x\ {\isasymcirc}\isactrlsub c\ {\isasymbeta}\isactrlbsub Y\ {\isasymsetminus}\ {\isacharparenleft}{\kern0pt}E{\isacharcomma}{\kern0pt}\ m{\isacharparenright}{\kern0pt}\isactrlesub {\isacharparenright}{\kern0pt}\ {\isasymcirc}\isactrlsub c\ try{\isacharunderscore}{\kern0pt}cast\ m{\isacharparenright}{\kern0pt}\ {\isasymcirc}\isactrlsub c\ f\ {\isasymcirc}\isactrlsub c\ y\ {\isacharequal}{\kern0pt}\ y{\isachardoublequoteclose}\isanewline
\ \ \ \ \isacommand{proof}\isamarkupfalse%
\ {\isacharminus}{\kern0pt}\isanewline
\ \ \ \ \ \ \isacommand{fix}\isamarkupfalse%
\ y\isanewline
\ \ \ \ \ \ \isacommand{assume}\isamarkupfalse%
\ y{\isacharunderscore}{\kern0pt}type{\isacharbrackleft}{\kern0pt}type{\isacharunderscore}{\kern0pt}rule{\isacharbrackright}{\kern0pt}{\isacharcolon}{\kern0pt}\ {\isachardoublequoteopen}y\ {\isasymin}\isactrlsub c\ X{\isachardoublequoteclose}\isanewline
\isanewline
\ \ \ \ \ \ \isacommand{have}\isamarkupfalse%
\ {\isachardoublequoteopen}{\isacharparenleft}{\kern0pt}g{\isacharunderscore}{\kern0pt}inv\ {\isasymamalg}\ {\isacharparenleft}{\kern0pt}x\ {\isasymcirc}\isactrlsub c\ {\isasymbeta}\isactrlbsub Y\ {\isasymsetminus}\ {\isacharparenleft}{\kern0pt}E{\isacharcomma}{\kern0pt}\ m{\isacharparenright}{\kern0pt}\isactrlesub {\isacharparenright}{\kern0pt}\ {\isasymcirc}\isactrlsub c\ try{\isacharunderscore}{\kern0pt}cast\ m{\isacharparenright}{\kern0pt}\ {\isasymcirc}\isactrlsub c\ f\ {\isasymcirc}\isactrlsub c\ y\isanewline
\ \ \ \ \ \ \ \ \ \ {\isacharequal}{\kern0pt}\ g{\isacharunderscore}{\kern0pt}inv\ {\isasymamalg}\ {\isacharparenleft}{\kern0pt}x\ {\isasymcirc}\isactrlsub c\ {\isasymbeta}\isactrlbsub Y\ {\isasymsetminus}\ {\isacharparenleft}{\kern0pt}E{\isacharcomma}{\kern0pt}\ m{\isacharparenright}{\kern0pt}\isactrlesub {\isacharparenright}{\kern0pt}\ {\isasymcirc}\isactrlsub c\ {\isacharparenleft}{\kern0pt}try{\isacharunderscore}{\kern0pt}cast\ m\ {\isasymcirc}\isactrlsub c\ m{\isacharparenright}{\kern0pt}\ {\isasymcirc}\isactrlsub c\ g\ {\isasymcirc}\isactrlsub c\ y{\isachardoublequoteclose}\isanewline
\ \ \ \ \ \ \ \ \isacommand{unfolding}\isamarkupfalse%
\ f{\isacharunderscore}{\kern0pt}eq\ \isacommand{by}\isamarkupfalse%
\ {\isacharparenleft}{\kern0pt}typecheck{\isacharunderscore}{\kern0pt}cfuncs{\isacharcomma}{\kern0pt}\ smt\ comp{\isacharunderscore}{\kern0pt}associative{\isadigit{2}}{\isacharparenright}{\kern0pt}\isanewline
\ \ \ \ \ \ \isacommand{also}\isamarkupfalse%
\ \isacommand{have}\isamarkupfalse%
\ {\isachardoublequoteopen}{\isachardot}{\kern0pt}{\isachardot}{\kern0pt}{\isachardot}{\kern0pt}\ {\isacharequal}{\kern0pt}\ {\isacharparenleft}{\kern0pt}g{\isacharunderscore}{\kern0pt}inv\ {\isasymamalg}\ {\isacharparenleft}{\kern0pt}x\ {\isasymcirc}\isactrlsub c\ {\isasymbeta}\isactrlbsub Y\ {\isasymsetminus}\ {\isacharparenleft}{\kern0pt}E{\isacharcomma}{\kern0pt}\ m{\isacharparenright}{\kern0pt}\isactrlesub {\isacharparenright}{\kern0pt}\ {\isasymcirc}\isactrlsub c\ left{\isacharunderscore}{\kern0pt}coproj\ E\ {\isacharparenleft}{\kern0pt}Y\ {\isasymsetminus}\ {\isacharparenleft}{\kern0pt}E{\isacharcomma}{\kern0pt}m{\isacharparenright}{\kern0pt}{\isacharparenright}{\kern0pt}{\isacharparenright}{\kern0pt}\ {\isasymcirc}\isactrlsub c\ g\ {\isasymcirc}\isactrlsub c\ y{\isachardoublequoteclose}\isanewline
\ \ \ \ \ \ \ \ \isacommand{by}\isamarkupfalse%
\ {\isacharparenleft}{\kern0pt}typecheck{\isacharunderscore}{\kern0pt}cfuncs{\isacharcomma}{\kern0pt}\ smt\ comp{\isacharunderscore}{\kern0pt}associative{\isadigit{2}}\ m{\isacharunderscore}{\kern0pt}mono\ try{\isacharunderscore}{\kern0pt}cast{\isacharunderscore}{\kern0pt}m{\isacharunderscore}{\kern0pt}m{\isacharparenright}{\kern0pt}\isanewline
\ \ \ \ \ \ \isacommand{also}\isamarkupfalse%
\ \isacommand{have}\isamarkupfalse%
\ {\isachardoublequoteopen}{\isachardot}{\kern0pt}{\isachardot}{\kern0pt}{\isachardot}{\kern0pt}\ {\isacharequal}{\kern0pt}\ {\isacharparenleft}{\kern0pt}g{\isacharunderscore}{\kern0pt}inv\ {\isasymcirc}\isactrlsub c\ g{\isacharparenright}{\kern0pt}\ {\isasymcirc}\isactrlsub c\ y{\isachardoublequoteclose}\isanewline
\ \ \ \ \ \ \ \ \isacommand{by}\isamarkupfalse%
\ {\isacharparenleft}{\kern0pt}typecheck{\isacharunderscore}{\kern0pt}cfuncs{\isacharcomma}{\kern0pt}\ simp\ add{\isacharcolon}{\kern0pt}\ comp{\isacharunderscore}{\kern0pt}associative{\isadigit{2}}\ left{\isacharunderscore}{\kern0pt}coproj{\isacharunderscore}{\kern0pt}cfunc{\isacharunderscore}{\kern0pt}coprod{\isacharparenright}{\kern0pt}\isanewline
\ \ \ \ \ \ \isacommand{also}\isamarkupfalse%
\ \isacommand{have}\isamarkupfalse%
\ {\isachardoublequoteopen}{\isachardot}{\kern0pt}{\isachardot}{\kern0pt}{\isachardot}{\kern0pt}\ {\isacharequal}{\kern0pt}\ y{\isachardoublequoteclose}\isanewline
\ \ \ \ \ \ \ \ \isacommand{by}\isamarkupfalse%
\ {\isacharparenleft}{\kern0pt}typecheck{\isacharunderscore}{\kern0pt}cfuncs{\isacharcomma}{\kern0pt}\ simp\ add{\isacharcolon}{\kern0pt}\ g{\isacharunderscore}{\kern0pt}inv{\isacharunderscore}{\kern0pt}g\ id{\isacharunderscore}{\kern0pt}left{\isacharunderscore}{\kern0pt}unit{\isadigit{2}}{\isacharparenright}{\kern0pt}\isanewline
\ \ \ \ \ \ \isacommand{then}\isamarkupfalse%
\ \isacommand{show}\isamarkupfalse%
\ {\isachardoublequoteopen}{\isacharparenleft}{\kern0pt}g{\isacharunderscore}{\kern0pt}inv\ {\isasymamalg}\ {\isacharparenleft}{\kern0pt}x\ {\isasymcirc}\isactrlsub c\ {\isasymbeta}\isactrlbsub Y\ {\isasymsetminus}\ {\isacharparenleft}{\kern0pt}E{\isacharcomma}{\kern0pt}\ m{\isacharparenright}{\kern0pt}\isactrlesub {\isacharparenright}{\kern0pt}\ {\isasymcirc}\isactrlsub c\ try{\isacharunderscore}{\kern0pt}cast\ m{\isacharparenright}{\kern0pt}\ {\isasymcirc}\isactrlsub c\ f\ {\isasymcirc}\isactrlsub c\ y\ {\isacharequal}{\kern0pt}\ y{\isachardoublequoteclose}\isanewline
\ \ \ \ \ \ \ \ \isacommand{using}\isamarkupfalse%
\ calculation\ \isacommand{by}\isamarkupfalse%
\ auto\isanewline
\ \ \ \ \isacommand{qed}\isamarkupfalse%
\isanewline
\ \ \ \ \isacommand{show}\isamarkupfalse%
\ {\isachardoublequoteopen}epimorphism\ {\isacharparenleft}{\kern0pt}g{\isacharunderscore}{\kern0pt}inv\ {\isasymamalg}\ {\isacharparenleft}{\kern0pt}x\ {\isasymcirc}\isactrlsub c\ {\isasymbeta}\isactrlbsub Y\ {\isasymsetminus}\ {\isacharparenleft}{\kern0pt}E{\isacharcomma}{\kern0pt}\ m{\isacharparenright}{\kern0pt}\isactrlesub {\isacharparenright}{\kern0pt}\ {\isasymcirc}\isactrlsub c\ try{\isacharunderscore}{\kern0pt}cast\ m{\isacharparenright}{\kern0pt}{\isachardoublequoteclose}\isanewline
\ \ \ \ \isacommand{proof}\isamarkupfalse%
\ {\isacharparenleft}{\kern0pt}rule\ surjective{\isacharunderscore}{\kern0pt}is{\isacharunderscore}{\kern0pt}epimorphism{\isacharcomma}{\kern0pt}\ etcs{\isacharunderscore}{\kern0pt}subst\ surjective{\isacharunderscore}{\kern0pt}def{\isadigit{2}}{\isacharcomma}{\kern0pt}\ clarify{\isacharparenright}{\kern0pt}\isanewline
\ \ \ \ \ \ \isacommand{fix}\isamarkupfalse%
\ y\isanewline
\ \ \ \ \ \ \isacommand{assume}\isamarkupfalse%
\ y{\isacharunderscore}{\kern0pt}type{\isacharbrackleft}{\kern0pt}type{\isacharunderscore}{\kern0pt}rule{\isacharbrackright}{\kern0pt}{\isacharcolon}{\kern0pt}\ {\isachardoublequoteopen}y\ {\isasymin}\isactrlsub c\ X{\isachardoublequoteclose}\isanewline
\ \ \ \ \ \ \isacommand{show}\isamarkupfalse%
\ {\isachardoublequoteopen}{\isasymexists}xa{\isachardot}{\kern0pt}\ xa\ {\isasymin}\isactrlsub c\ Y\ {\isasymand}\ {\isacharparenleft}{\kern0pt}g{\isacharunderscore}{\kern0pt}inv\ {\isasymamalg}\ {\isacharparenleft}{\kern0pt}x\ {\isasymcirc}\isactrlsub c\ {\isasymbeta}\isactrlbsub Y\ {\isasymsetminus}\ {\isacharparenleft}{\kern0pt}E{\isacharcomma}{\kern0pt}\ m{\isacharparenright}{\kern0pt}\isactrlesub {\isacharparenright}{\kern0pt}\ {\isasymcirc}\isactrlsub c\ try{\isacharunderscore}{\kern0pt}cast\ m{\isacharparenright}{\kern0pt}\ {\isasymcirc}\isactrlsub c\ xa\ {\isacharequal}{\kern0pt}\ y{\isachardoublequoteclose}\isanewline
\ \ \ \ \ \ \ \ \isacommand{by}\isamarkupfalse%
\ {\isacharparenleft}{\kern0pt}rule\ exI{\isacharbrackleft}{\kern0pt}\isakeyword{where}\ x{\isacharequal}{\kern0pt}{\isachardoublequoteopen}f\ {\isasymcirc}\isactrlsub c\ y{\isachardoublequoteclose}{\isacharbrackright}{\kern0pt}{\isacharcomma}{\kern0pt}\ typecheck{\isacharunderscore}{\kern0pt}cfuncs{\isacharcomma}{\kern0pt}\ smt\ func{\isacharunderscore}{\kern0pt}f{\isacharunderscore}{\kern0pt}elem{\isacharunderscore}{\kern0pt}eq{\isacharparenright}{\kern0pt}\isanewline
\ \ \ \ \isacommand{qed}\isamarkupfalse%
\isanewline
\ \ \ \ \isacommand{show}\isamarkupfalse%
\ {\isachardoublequoteopen}{\isacharparenleft}{\kern0pt}g{\isacharunderscore}{\kern0pt}inv\ {\isasymamalg}\ {\isacharparenleft}{\kern0pt}x\ {\isasymcirc}\isactrlsub c\ {\isasymbeta}\isactrlbsub Y\ {\isasymsetminus}\ {\isacharparenleft}{\kern0pt}E{\isacharcomma}{\kern0pt}\ m{\isacharparenright}{\kern0pt}\isactrlesub {\isacharparenright}{\kern0pt}\ {\isasymcirc}\isactrlsub c\ try{\isacharunderscore}{\kern0pt}cast\ m{\isacharparenright}{\kern0pt}\ {\isasymcirc}\isactrlsub c\ f\ {\isacharequal}{\kern0pt}\ id\isactrlsub c\ X{\isachardoublequoteclose}\isanewline
\ \ \ \ \ \ \isacommand{by}\isamarkupfalse%
\ {\isacharparenleft}{\kern0pt}insert\ comp{\isacharunderscore}{\kern0pt}associative{\isadigit{2}}\ func{\isacharunderscore}{\kern0pt}f{\isacharunderscore}{\kern0pt}elem{\isacharunderscore}{\kern0pt}eq\ id{\isacharunderscore}{\kern0pt}left{\isacharunderscore}{\kern0pt}unit{\isadigit{2}}{\isacharcomma}{\kern0pt}\ typecheck{\isacharunderscore}{\kern0pt}cfuncs{\isacharcomma}{\kern0pt}\ rule\ one{\isacharunderscore}{\kern0pt}separator{\isacharcomma}{\kern0pt}\ auto{\isacharparenright}{\kern0pt}\isanewline
\ \ \isacommand{qed}\isamarkupfalse%
\isanewline
\isacommand{qed}\isamarkupfalse%
%
\endisatagproof
{\isafoldproof}%
%
\isadelimproof
%
\endisadelimproof
%
\begin{isamarkuptext}%
The lemma below corresponds to Exercise 2.7.2(i) in Halvorson.%
\end{isamarkuptext}\isamarkuptrue%
\isacommand{lemma}\isamarkupfalse%
\ split{\isacharunderscore}{\kern0pt}epis{\isacharunderscore}{\kern0pt}are{\isacharunderscore}{\kern0pt}regular{\isacharcolon}{\kern0pt}\ \isanewline
\ \ \isakeyword{assumes}\ f{\isacharunderscore}{\kern0pt}type{\isacharbrackleft}{\kern0pt}type{\isacharunderscore}{\kern0pt}rule{\isacharbrackright}{\kern0pt}{\isacharcolon}{\kern0pt}\ {\isachardoublequoteopen}f\ {\isacharcolon}{\kern0pt}\ X\ {\isasymrightarrow}\ Y{\isachardoublequoteclose}\isanewline
\ \ \isakeyword{assumes}\ {\isachardoublequoteopen}split{\isacharunderscore}{\kern0pt}epimorphism\ f{\isachardoublequoteclose}\isanewline
\ \ \isakeyword{shows}\ {\isachardoublequoteopen}regular{\isacharunderscore}{\kern0pt}epimorphism\ f{\isachardoublequoteclose}\isanewline
%
\isadelimproof
%
\endisadelimproof
%
\isatagproof
\isacommand{proof}\isamarkupfalse%
\ {\isacharminus}{\kern0pt}\ \isanewline
\ \ \isacommand{obtain}\isamarkupfalse%
\ s\ \isakeyword{where}\ s{\isacharunderscore}{\kern0pt}type{\isacharbrackleft}{\kern0pt}type{\isacharunderscore}{\kern0pt}rule{\isacharbrackright}{\kern0pt}{\isacharcolon}{\kern0pt}\ {\isachardoublequoteopen}s\ {\isacharcolon}{\kern0pt}\ Y\ {\isasymrightarrow}\ X{\isachardoublequoteclose}\ \isakeyword{and}\ s{\isacharunderscore}{\kern0pt}splits{\isacharcolon}{\kern0pt}\ \ {\isachardoublequoteopen}f\ {\isasymcirc}\isactrlsub c\ s\ {\isacharequal}{\kern0pt}\ id\ Y{\isachardoublequoteclose}\isanewline
\ \ \ \ \isacommand{by}\isamarkupfalse%
\ {\isacharparenleft}{\kern0pt}meson\ assms{\isacharparenleft}{\kern0pt}{\isadigit{2}}{\isacharparenright}{\kern0pt}\ f{\isacharunderscore}{\kern0pt}type\ split{\isacharunderscore}{\kern0pt}epimorphism{\isacharunderscore}{\kern0pt}def{\isadigit{2}}{\isacharparenright}{\kern0pt}\isanewline
\ \ \isacommand{then}\isamarkupfalse%
\ \isacommand{have}\isamarkupfalse%
\ {\isachardoublequoteopen}coequalizer\ Y\ f\ {\isacharparenleft}{\kern0pt}s\ {\isasymcirc}\isactrlsub c\ f{\isacharparenright}{\kern0pt}\ {\isacharparenleft}{\kern0pt}id\ X{\isacharparenright}{\kern0pt}{\isachardoublequoteclose}\isanewline
\ \ \ \ \isacommand{unfolding}\isamarkupfalse%
\ coequalizer{\isacharunderscore}{\kern0pt}def\ \isanewline
\ \ \ \ \isacommand{by}\isamarkupfalse%
\ {\isacharparenleft}{\kern0pt}rule{\isacharunderscore}{\kern0pt}tac\ x{\isacharequal}{\kern0pt}{\isachardoublequoteopen}X{\isachardoublequoteclose}\ \isakeyword{in}\ exI{\isacharcomma}{\kern0pt}\ rule{\isacharunderscore}{\kern0pt}tac\ x{\isacharequal}{\kern0pt}{\isachardoublequoteopen}X{\isachardoublequoteclose}\ \isakeyword{in}\ exI{\isacharcomma}{\kern0pt}\ typecheck{\isacharunderscore}{\kern0pt}cfuncs{\isacharcomma}{\kern0pt}\isanewline
\ \ \ \ \ \ \ \ smt\ {\isacharparenleft}{\kern0pt}verit{\isacharcomma}{\kern0pt}\ ccfv{\isacharunderscore}{\kern0pt}threshold{\isacharparenright}{\kern0pt}\ cfunc{\isacharunderscore}{\kern0pt}type{\isacharunderscore}{\kern0pt}def\ comp{\isacharunderscore}{\kern0pt}associative\ comp{\isacharunderscore}{\kern0pt}type\ id{\isacharunderscore}{\kern0pt}left{\isacharunderscore}{\kern0pt}unit{\isadigit{2}}\ id{\isacharunderscore}{\kern0pt}right{\isacharunderscore}{\kern0pt}unit{\isadigit{2}}{\isacharparenright}{\kern0pt}\isanewline
\ \ \isacommand{then}\isamarkupfalse%
\ \isacommand{show}\isamarkupfalse%
\ {\isacharquery}{\kern0pt}thesis\isanewline
\ \ \ \ \isacommand{using}\isamarkupfalse%
\ assms\ coequalizer{\isacharunderscore}{\kern0pt}is{\isacharunderscore}{\kern0pt}epimorphism\ epimorphisms{\isacharunderscore}{\kern0pt}are{\isacharunderscore}{\kern0pt}regular\ \isacommand{by}\isamarkupfalse%
\ blast\isanewline
\isacommand{qed}\isamarkupfalse%
%
\endisatagproof
{\isafoldproof}%
%
\isadelimproof
%
\endisadelimproof
%
\begin{isamarkuptext}%
The lemma below corresponds to Exercise 2.7.2(ii) in Halvorson.%
\end{isamarkuptext}\isamarkuptrue%
\isacommand{lemma}\isamarkupfalse%
\ sections{\isacharunderscore}{\kern0pt}are{\isacharunderscore}{\kern0pt}regular{\isacharunderscore}{\kern0pt}monos{\isacharcolon}{\kern0pt}\ \isanewline
\ \ \isakeyword{assumes}\ s{\isacharunderscore}{\kern0pt}type{\isacharcolon}{\kern0pt}\ \ {\isachardoublequoteopen}s\ {\isacharcolon}{\kern0pt}\ Y\ {\isasymrightarrow}\ X{\isachardoublequoteclose}\isanewline
\ \ \isakeyword{assumes}\ {\isachardoublequoteopen}s\ sectionof\ f{\isachardoublequoteclose}\isanewline
\ \ \isakeyword{shows}\ {\isachardoublequoteopen}regular{\isacharunderscore}{\kern0pt}monomorphism\ s{\isachardoublequoteclose}\isanewline
%
\isadelimproof
%
\endisadelimproof
%
\isatagproof
\isacommand{proof}\isamarkupfalse%
\ {\isacharminus}{\kern0pt}\ \ \ \isanewline
\ \ \isacommand{have}\isamarkupfalse%
\ {\isachardoublequoteopen}coequalizer\ Y\ f\ {\isacharparenleft}{\kern0pt}s\ {\isasymcirc}\isactrlsub c\ f{\isacharparenright}{\kern0pt}\ {\isacharparenleft}{\kern0pt}id\ X{\isacharparenright}{\kern0pt}{\isachardoublequoteclose}\isanewline
\ \ \ \ \isacommand{unfolding}\isamarkupfalse%
\ coequalizer{\isacharunderscore}{\kern0pt}def\ \isanewline
\ \ \ \ \isacommand{by}\isamarkupfalse%
\ {\isacharparenleft}{\kern0pt}rule{\isacharunderscore}{\kern0pt}tac\ x{\isacharequal}{\kern0pt}{\isachardoublequoteopen}X{\isachardoublequoteclose}\ \isakeyword{in}\ exI{\isacharcomma}{\kern0pt}\ rule{\isacharunderscore}{\kern0pt}tac\ x{\isacharequal}{\kern0pt}{\isachardoublequoteopen}X{\isachardoublequoteclose}\ \isakeyword{in}\ exI{\isacharcomma}{\kern0pt}\ typecheck{\isacharunderscore}{\kern0pt}cfuncs{\isacharcomma}{\kern0pt}\isanewline
\ \ \ \ \ \ \ \ smt\ {\isacharparenleft}{\kern0pt}z{\isadigit{3}}{\isacharparenright}{\kern0pt}\ assms\ cfunc{\isacharunderscore}{\kern0pt}type{\isacharunderscore}{\kern0pt}def\ comp{\isacharunderscore}{\kern0pt}associative{\isadigit{2}}\ comp{\isacharunderscore}{\kern0pt}type\ id{\isacharunderscore}{\kern0pt}left{\isacharunderscore}{\kern0pt}unit\ id{\isacharunderscore}{\kern0pt}right{\isacharunderscore}{\kern0pt}unit{\isadigit{2}}\ section{\isacharunderscore}{\kern0pt}of{\isacharunderscore}{\kern0pt}def{\isacharparenright}{\kern0pt}\isanewline
\ \ \isacommand{then}\isamarkupfalse%
\ \isacommand{show}\isamarkupfalse%
\ {\isacharquery}{\kern0pt}thesis\isanewline
\ \ \ \ \isacommand{by}\isamarkupfalse%
\ {\isacharparenleft}{\kern0pt}metis\ assms{\isacharparenleft}{\kern0pt}{\isadigit{2}}{\isacharparenright}{\kern0pt}\ cfunc{\isacharunderscore}{\kern0pt}type{\isacharunderscore}{\kern0pt}def\ comp{\isacharunderscore}{\kern0pt}monic{\isacharunderscore}{\kern0pt}imp{\isacharunderscore}{\kern0pt}monic{\isacharprime}{\kern0pt}\ id{\isacharunderscore}{\kern0pt}isomorphism\ iso{\isacharunderscore}{\kern0pt}imp{\isacharunderscore}{\kern0pt}epi{\isacharunderscore}{\kern0pt}and{\isacharunderscore}{\kern0pt}monic\ mono{\isacharunderscore}{\kern0pt}is{\isacharunderscore}{\kern0pt}regmono\ section{\isacharunderscore}{\kern0pt}of{\isacharunderscore}{\kern0pt}def{\isacharparenright}{\kern0pt}\isanewline
\isacommand{qed}\isamarkupfalse%
%
\endisatagproof
{\isafoldproof}%
%
\isadelimproof
\isanewline
%
\endisadelimproof
%
\isadelimtheory
\isanewline
%
\endisadelimtheory
%
\isatagtheory
\isacommand{end}\isamarkupfalse%
%
\endisatagtheory
{\isafoldtheory}%
%
\isadelimtheory
%
\endisadelimtheory
%
\end{isabellebody}%
\endinput
%:%file=~/ETCS/HOL-ETCS/Axiom_Of_Choice.thy%:%
%:%11=1%:%
%:%27=3%:%
%:%28=3%:%
%:%29=4%:%
%:%30=5%:%
%:%39=7%:%
%:%41=8%:%
%:%42=8%:%
%:%43=9%:%
%:%44=10%:%
%:%45=11%:%
%:%46=11%:%
%:%47=12%:%
%:%48=13%:%
%:%49=14%:%
%:%50=14%:%
%:%51=15%:%
%:%52=16%:%
%:%53=17%:%
%:%56=18%:%
%:%60=18%:%
%:%61=18%:%
%:%62=18%:%
%:%67=18%:%
%:%70=19%:%
%:%71=20%:%
%:%72=20%:%
%:%73=21%:%
%:%74=22%:%
%:%75=23%:%
%:%78=24%:%
%:%82=24%:%
%:%83=24%:%
%:%84=24%:%
%:%93=26%:%
%:%95=27%:%
%:%96=27%:%
%:%97=28%:%
%:%98=29%:%
%:%99=30%:%
%:%100=31%:%
%:%101=31%:%
%:%102=32%:%
%:%103=33%:%
%:%104=34%:%
%:%107=35%:%
%:%111=35%:%
%:%112=35%:%
%:%113=36%:%
%:%114=36%:%
%:%119=36%:%
%:%122=37%:%
%:%123=38%:%
%:%124=38%:%
%:%125=39%:%
%:%126=40%:%
%:%127=41%:%
%:%130=42%:%
%:%134=42%:%
%:%135=42%:%
%:%136=42%:%
%:%145=44%:%
%:%147=45%:%
%:%148=45%:%
%:%149=46%:%
%:%150=47%:%
%:%151=48%:%
%:%152=49%:%
%:%159=50%:%
%:%160=50%:%
%:%161=51%:%
%:%162=51%:%
%:%163=52%:%
%:%164=53%:%
%:%165=53%:%
%:%166=53%:%
%:%167=54%:%
%:%168=55%:%
%:%169=55%:%
%:%170=56%:%
%:%171=56%:%
%:%172=57%:%
%:%173=57%:%
%:%174=58%:%
%:%175=58%:%
%:%176=59%:%
%:%177=59%:%
%:%178=60%:%
%:%179=60%:%
%:%180=60%:%
%:%181=61%:%
%:%182=61%:%
%:%183=62%:%
%:%184=62%:%
%:%185=62%:%
%:%186=63%:%
%:%187=63%:%
%:%188=63%:%
%:%189=64%:%
%:%190=64%:%
%:%191=64%:%
%:%192=65%:%
%:%193=65%:%
%:%194=65%:%
%:%195=66%:%
%:%196=66%:%
%:%197=67%:%
%:%198=68%:%
%:%199=68%:%
%:%200=69%:%
%:%201=69%:%
%:%202=70%:%
%:%203=70%:%
%:%204=70%:%
%:%205=71%:%
%:%206=72%:%
%:%207=72%:%
%:%208=72%:%
%:%209=73%:%
%:%210=74%:%
%:%211=74%:%
%:%212=75%:%
%:%213=75%:%
%:%214=75%:%
%:%215=76%:%
%:%216=77%:%
%:%217=77%:%
%:%218=78%:%
%:%219=78%:%
%:%220=79%:%
%:%221=79%:%
%:%222=80%:%
%:%223=80%:%
%:%224=81%:%
%:%225=81%:%
%:%226=82%:%
%:%227=82%:%
%:%228=83%:%
%:%229=84%:%
%:%230=84%:%
%:%231=85%:%
%:%232=86%:%
%:%233=86%:%
%:%234=86%:%
%:%235=87%:%
%:%236=87%:%
%:%237=87%:%
%:%238=88%:%
%:%239=88%:%
%:%240=89%:%
%:%241=89%:%
%:%242=89%:%
%:%243=90%:%
%:%244=90%:%
%:%245=91%:%
%:%246=91%:%
%:%247=91%:%
%:%248=92%:%
%:%249=92%:%
%:%250=93%:%
%:%251=93%:%
%:%252=93%:%
%:%253=94%:%
%:%254=94%:%
%:%255=94%:%
%:%256=95%:%
%:%257=95%:%
%:%258=96%:%
%:%259=96%:%
%:%260=97%:%
%:%261=97%:%
%:%262=98%:%
%:%263=98%:%
%:%264=99%:%
%:%265=99%:%
%:%266=100%:%
%:%267=100%:%
%:%268=101%:%
%:%269=101%:%
%:%270=102%:%
%:%271=102%:%
%:%272=103%:%
%:%273=103%:%
%:%274=104%:%
%:%275=104%:%
%:%276=105%:%
%:%277=105%:%
%:%278=106%:%
%:%288=108%:%
%:%290=109%:%
%:%291=109%:%
%:%292=110%:%
%:%293=111%:%
%:%294=112%:%
%:%301=113%:%
%:%302=113%:%
%:%303=114%:%
%:%304=114%:%
%:%305=115%:%
%:%306=115%:%
%:%307=116%:%
%:%308=116%:%
%:%309=116%:%
%:%310=117%:%
%:%311=117%:%
%:%312=118%:%
%:%313=118%:%
%:%314=119%:%
%:%315=120%:%
%:%316=120%:%
%:%317=120%:%
%:%318=121%:%
%:%319=121%:%
%:%320=121%:%
%:%321=122%:%
%:%331=124%:%
%:%333=125%:%
%:%334=125%:%
%:%335=126%:%
%:%336=127%:%
%:%337=128%:%
%:%344=129%:%
%:%345=129%:%
%:%346=130%:%
%:%347=130%:%
%:%348=131%:%
%:%349=131%:%
%:%350=132%:%
%:%351=132%:%
%:%352=133%:%
%:%353=134%:%
%:%354=134%:%
%:%355=134%:%
%:%356=135%:%
%:%357=135%:%
%:%358=136%:%
%:%364=136%:%
%:%369=137%:%
%:%374=138%:%

%
\begin{isabellebody}%
\setisabellecontext{Initial}%
%
\isadelimdocument
%
\endisadelimdocument
%
\isatagdocument
%
\isamarkupsection{Empty Set and Initial Objects%
}
\isamarkuptrue%
%
\endisatagdocument
{\isafolddocument}%
%
\isadelimdocument
%
\endisadelimdocument
%
\isadelimtheory
%
\endisadelimtheory
%
\isatagtheory
\isacommand{theory}\isamarkupfalse%
\ Initial\isanewline
\ \ \isakeyword{imports}\ Coproduct\isanewline
\isakeyword{begin}%
\endisatagtheory
{\isafoldtheory}%
%
\isadelimtheory
%
\endisadelimtheory
%
\begin{isamarkuptext}%
The axiomatization below corresponds to Axiom 8 (Empty Set) in Halvorson.%
\end{isamarkuptext}\isamarkuptrue%
\isacommand{axiomatization}\isamarkupfalse%
\isanewline
\ \ initial{\isacharunderscore}{\kern0pt}func\ {\isacharcolon}{\kern0pt}{\isacharcolon}{\kern0pt}\ {\isachardoublequoteopen}cset\ {\isasymRightarrow}\ cfunc{\isachardoublequoteclose}\ {\isacharparenleft}{\kern0pt}{\isachardoublequoteopen}{\isasymalpha}\isactrlbsub {\isacharunderscore}{\kern0pt}\isactrlesub {\isachardoublequoteclose}\ {\isadigit{1}}{\isadigit{0}}{\isadigit{0}}{\isacharparenright}{\kern0pt}\ \isakeyword{and}\isanewline
\ \ emptyset\ {\isacharcolon}{\kern0pt}{\isacharcolon}{\kern0pt}\ {\isachardoublequoteopen}cset{\isachardoublequoteclose}\ {\isacharparenleft}{\kern0pt}{\isachardoublequoteopen}{\isasymemptyset}{\isachardoublequoteclose}{\isacharparenright}{\kern0pt}\isanewline
\isakeyword{where}\isanewline
\ \ initial{\isacharunderscore}{\kern0pt}func{\isacharunderscore}{\kern0pt}type{\isacharbrackleft}{\kern0pt}type{\isacharunderscore}{\kern0pt}rule{\isacharbrackright}{\kern0pt}{\isacharcolon}{\kern0pt}\ {\isachardoublequoteopen}initial{\isacharunderscore}{\kern0pt}func\ X\ {\isacharcolon}{\kern0pt}\ \ {\isasymemptyset}\ {\isasymrightarrow}\ X{\isachardoublequoteclose}\ \isakeyword{and}\isanewline
\ \ initial{\isacharunderscore}{\kern0pt}func{\isacharunderscore}{\kern0pt}unique{\isacharcolon}{\kern0pt}\ {\isachardoublequoteopen}h\ {\isacharcolon}{\kern0pt}\ {\isasymemptyset}\ {\isasymrightarrow}\ X\ {\isasymLongrightarrow}\ h\ {\isacharequal}{\kern0pt}\ initial{\isacharunderscore}{\kern0pt}func\ X{\isachardoublequoteclose}\ \isakeyword{and}\isanewline
\ \ emptyset{\isacharunderscore}{\kern0pt}is{\isacharunderscore}{\kern0pt}empty{\isacharcolon}{\kern0pt}\ {\isachardoublequoteopen}{\isasymnot}{\isacharparenleft}{\kern0pt}x\ {\isasymin}\isactrlsub c\ {\isasymemptyset}{\isacharparenright}{\kern0pt}{\isachardoublequoteclose}\isanewline
\isanewline
\isacommand{definition}\isamarkupfalse%
\ initial{\isacharunderscore}{\kern0pt}object\ {\isacharcolon}{\kern0pt}{\isacharcolon}{\kern0pt}\ {\isachardoublequoteopen}cset\ {\isasymRightarrow}\ bool{\isachardoublequoteclose}\ \isakeyword{where}\isanewline
\ \ {\isachardoublequoteopen}initial{\isacharunderscore}{\kern0pt}object{\isacharparenleft}{\kern0pt}X{\isacharparenright}{\kern0pt}\ {\isasymlongleftrightarrow}\ {\isacharparenleft}{\kern0pt}{\isasymforall}\ Y{\isachardot}{\kern0pt}\ {\isasymexists}{\isacharbang}{\kern0pt}\ f{\isachardot}{\kern0pt}\ f\ {\isacharcolon}{\kern0pt}\ X\ {\isasymrightarrow}\ Y{\isacharparenright}{\kern0pt}{\isachardoublequoteclose}\isanewline
\isanewline
\isacommand{lemma}\isamarkupfalse%
\ emptyset{\isacharunderscore}{\kern0pt}is{\isacharunderscore}{\kern0pt}initial{\isacharcolon}{\kern0pt}\isanewline
\ \ {\isachardoublequoteopen}initial{\isacharunderscore}{\kern0pt}object{\isacharparenleft}{\kern0pt}{\isasymemptyset}{\isacharparenright}{\kern0pt}{\isachardoublequoteclose}\isanewline
%
\isadelimproof
\ \ %
\endisadelimproof
%
\isatagproof
\isacommand{using}\isamarkupfalse%
\ initial{\isacharunderscore}{\kern0pt}func{\isacharunderscore}{\kern0pt}type\ initial{\isacharunderscore}{\kern0pt}func{\isacharunderscore}{\kern0pt}unique\ initial{\isacharunderscore}{\kern0pt}object{\isacharunderscore}{\kern0pt}def\ \isacommand{by}\isamarkupfalse%
\ blast%
\endisatagproof
{\isafoldproof}%
%
\isadelimproof
\isanewline
%
\endisadelimproof
\isanewline
\isacommand{lemma}\isamarkupfalse%
\ initial{\isacharunderscore}{\kern0pt}iso{\isacharunderscore}{\kern0pt}empty{\isacharcolon}{\kern0pt}\isanewline
\ \ \isakeyword{assumes}\ {\isachardoublequoteopen}initial{\isacharunderscore}{\kern0pt}object{\isacharparenleft}{\kern0pt}X{\isacharparenright}{\kern0pt}{\isachardoublequoteclose}\isanewline
\ \ \isakeyword{shows}\ {\isachardoublequoteopen}X\ {\isasymcong}\ {\isasymemptyset}{\isachardoublequoteclose}\isanewline
%
\isadelimproof
\ \ %
\endisadelimproof
%
\isatagproof
\isacommand{by}\isamarkupfalse%
\ {\isacharparenleft}{\kern0pt}metis\ assms\ cfunc{\isacharunderscore}{\kern0pt}type{\isacharunderscore}{\kern0pt}def\ comp{\isacharunderscore}{\kern0pt}type\ emptyset{\isacharunderscore}{\kern0pt}is{\isacharunderscore}{\kern0pt}empty\ epi{\isacharunderscore}{\kern0pt}mon{\isacharunderscore}{\kern0pt}is{\isacharunderscore}{\kern0pt}iso\ initial{\isacharunderscore}{\kern0pt}object{\isacharunderscore}{\kern0pt}def\ injective{\isacharunderscore}{\kern0pt}def\ injective{\isacharunderscore}{\kern0pt}imp{\isacharunderscore}{\kern0pt}monomorphism\ is{\isacharunderscore}{\kern0pt}isomorphic{\isacharunderscore}{\kern0pt}def\ surjective{\isacharunderscore}{\kern0pt}def\ surjective{\isacharunderscore}{\kern0pt}is{\isacharunderscore}{\kern0pt}epimorphism{\isacharparenright}{\kern0pt}%
\endisatagproof
{\isafoldproof}%
%
\isadelimproof
%
\endisadelimproof
%
\begin{isamarkuptext}%
The lemma below corresponds to Exercise 2.4.6 in Halvorson.%
\end{isamarkuptext}\isamarkuptrue%
\isacommand{lemma}\isamarkupfalse%
\ coproduct{\isacharunderscore}{\kern0pt}with{\isacharunderscore}{\kern0pt}empty{\isacharcolon}{\kern0pt}\isanewline
\ \ \isakeyword{shows}\ {\isachardoublequoteopen}X\ {\isasymCoprod}\ {\isasymemptyset}\ {\isasymcong}\ X{\isachardoublequoteclose}\isanewline
%
\isadelimproof
%
\endisadelimproof
%
\isatagproof
\isacommand{proof}\isamarkupfalse%
\ {\isacharminus}{\kern0pt}\isanewline
\ \ \isacommand{have}\isamarkupfalse%
\ comp{\isadigit{1}}{\isacharcolon}{\kern0pt}\ {\isachardoublequoteopen}{\isacharparenleft}{\kern0pt}left{\isacharunderscore}{\kern0pt}coproj\ X\ {\isasymemptyset}\ {\isasymcirc}\isactrlsub c\ {\isacharparenleft}{\kern0pt}id\ X\ {\isasymamalg}\ {\isasymalpha}\isactrlbsub X\isactrlesub {\isacharparenright}{\kern0pt}{\isacharparenright}{\kern0pt}\ {\isasymcirc}\isactrlsub c\ left{\isacharunderscore}{\kern0pt}coproj\ X\ {\isasymemptyset}\ {\isacharequal}{\kern0pt}\ left{\isacharunderscore}{\kern0pt}coproj\ X\ {\isasymemptyset}{\isachardoublequoteclose}\isanewline
\ \ \isacommand{proof}\isamarkupfalse%
\ {\isacharminus}{\kern0pt}\isanewline
\ \ \ \ \isacommand{have}\isamarkupfalse%
\ {\isachardoublequoteopen}{\isacharparenleft}{\kern0pt}left{\isacharunderscore}{\kern0pt}coproj\ X\ {\isasymemptyset}\ {\isasymcirc}\isactrlsub c\ {\isacharparenleft}{\kern0pt}id\ X\ {\isasymamalg}\ {\isasymalpha}\isactrlbsub X\isactrlesub {\isacharparenright}{\kern0pt}{\isacharparenright}{\kern0pt}\ {\isasymcirc}\isactrlsub c\ left{\isacharunderscore}{\kern0pt}coproj\ X\ {\isasymemptyset}\ {\isacharequal}{\kern0pt}\isanewline
\ \ \ \ \ \ \ \ \ \ \ \ left{\isacharunderscore}{\kern0pt}coproj\ X\ {\isasymemptyset}\ {\isasymcirc}\isactrlsub c\ {\isacharparenleft}{\kern0pt}id\ X\ {\isasymamalg}\ {\isasymalpha}\isactrlbsub X\isactrlesub \ {\isasymcirc}\isactrlsub c\ left{\isacharunderscore}{\kern0pt}coproj\ X\ {\isasymemptyset}{\isacharparenright}{\kern0pt}{\isachardoublequoteclose}\isanewline
\ \ \ \ \ \ \isacommand{by}\isamarkupfalse%
\ {\isacharparenleft}{\kern0pt}typecheck{\isacharunderscore}{\kern0pt}cfuncs{\isacharcomma}{\kern0pt}\ simp\ add{\isacharcolon}{\kern0pt}\ comp{\isacharunderscore}{\kern0pt}associative{\isadigit{2}}{\isacharparenright}{\kern0pt}\isanewline
\ \ \ \ \isacommand{also}\isamarkupfalse%
\ \isacommand{have}\isamarkupfalse%
\ {\isachardoublequoteopen}{\isachardot}{\kern0pt}{\isachardot}{\kern0pt}{\isachardot}{\kern0pt}\ {\isacharequal}{\kern0pt}\ left{\isacharunderscore}{\kern0pt}coproj\ X\ {\isasymemptyset}\ {\isasymcirc}\isactrlsub c\ id{\isacharparenleft}{\kern0pt}X{\isacharparenright}{\kern0pt}{\isachardoublequoteclose}\isanewline
\ \ \ \ \ \ \isacommand{by}\isamarkupfalse%
\ {\isacharparenleft}{\kern0pt}typecheck{\isacharunderscore}{\kern0pt}cfuncs{\isacharcomma}{\kern0pt}\ metis\ left{\isacharunderscore}{\kern0pt}coproj{\isacharunderscore}{\kern0pt}cfunc{\isacharunderscore}{\kern0pt}coprod{\isacharparenright}{\kern0pt}\isanewline
\ \ \ \ \isacommand{also}\isamarkupfalse%
\ \isacommand{have}\isamarkupfalse%
\ {\isachardoublequoteopen}{\isachardot}{\kern0pt}{\isachardot}{\kern0pt}{\isachardot}{\kern0pt}\ {\isacharequal}{\kern0pt}\ left{\isacharunderscore}{\kern0pt}coproj\ X\ {\isasymemptyset}{\isachardoublequoteclose}\isanewline
\ \ \ \ \ \ \isacommand{by}\isamarkupfalse%
\ {\isacharparenleft}{\kern0pt}typecheck{\isacharunderscore}{\kern0pt}cfuncs{\isacharcomma}{\kern0pt}\ metis\ id{\isacharunderscore}{\kern0pt}right{\isacharunderscore}{\kern0pt}unit{\isadigit{2}}{\isacharparenright}{\kern0pt}\isanewline
\ \ \ \ \isacommand{then}\isamarkupfalse%
\ \isacommand{show}\isamarkupfalse%
\ {\isacharquery}{\kern0pt}thesis\ \isacommand{using}\isamarkupfalse%
\ calculation\ \isacommand{by}\isamarkupfalse%
\ auto\isanewline
\ \ \isacommand{qed}\isamarkupfalse%
\isanewline
\ \ \isacommand{have}\isamarkupfalse%
\ comp{\isadigit{2}}{\isacharcolon}{\kern0pt}\ {\isachardoublequoteopen}{\isacharparenleft}{\kern0pt}left{\isacharunderscore}{\kern0pt}coproj\ X\ {\isasymemptyset}\ {\isasymcirc}\isactrlsub c\ {\isacharparenleft}{\kern0pt}id{\isacharparenleft}{\kern0pt}X{\isacharparenright}{\kern0pt}\ {\isasymamalg}\ {\isasymalpha}\isactrlbsub X\isactrlesub {\isacharparenright}{\kern0pt}{\isacharparenright}{\kern0pt}\ {\isasymcirc}\isactrlsub c\ right{\isacharunderscore}{\kern0pt}coproj\ X\ {\isasymemptyset}\ {\isacharequal}{\kern0pt}\ right{\isacharunderscore}{\kern0pt}coproj\ X\ {\isasymemptyset}{\isachardoublequoteclose}\isanewline
\ \ \isacommand{proof}\isamarkupfalse%
\ {\isacharminus}{\kern0pt}\isanewline
\ \ \ \ \isacommand{have}\isamarkupfalse%
\ {\isachardoublequoteopen}{\isacharparenleft}{\kern0pt}{\isacharparenleft}{\kern0pt}left{\isacharunderscore}{\kern0pt}coproj\ X\ {\isasymemptyset}{\isacharparenright}{\kern0pt}\ {\isasymcirc}\isactrlsub c\ {\isacharparenleft}{\kern0pt}id{\isacharparenleft}{\kern0pt}X{\isacharparenright}{\kern0pt}\ {\isasymamalg}\ {\isasymalpha}\isactrlbsub X\isactrlesub {\isacharparenright}{\kern0pt}{\isacharparenright}{\kern0pt}\ {\isasymcirc}\isactrlsub c\ {\isacharparenleft}{\kern0pt}right{\isacharunderscore}{\kern0pt}coproj\ X\ {\isasymemptyset}{\isacharparenright}{\kern0pt}\ {\isacharequal}{\kern0pt}\ \isanewline
\ \ \ \ \ \ \ \ \ \ \ \ \ {\isacharparenleft}{\kern0pt}left{\isacharunderscore}{\kern0pt}coproj\ X\ {\isasymemptyset}{\isacharparenright}{\kern0pt}\ {\isasymcirc}\isactrlsub c\ {\isacharparenleft}{\kern0pt}{\isacharparenleft}{\kern0pt}id{\isacharparenleft}{\kern0pt}X{\isacharparenright}{\kern0pt}\ {\isasymamalg}\ {\isasymalpha}\isactrlbsub X\isactrlesub {\isacharparenright}{\kern0pt}\ {\isasymcirc}\isactrlsub c\ {\isacharparenleft}{\kern0pt}right{\isacharunderscore}{\kern0pt}coproj\ X\ {\isasymemptyset}{\isacharparenright}{\kern0pt}{\isacharparenright}{\kern0pt}{\isachardoublequoteclose}\isanewline
\ \ \ \ \ \ \isacommand{by}\isamarkupfalse%
\ {\isacharparenleft}{\kern0pt}typecheck{\isacharunderscore}{\kern0pt}cfuncs{\isacharcomma}{\kern0pt}\ simp\ add{\isacharcolon}{\kern0pt}\ comp{\isacharunderscore}{\kern0pt}associative{\isadigit{2}}{\isacharparenright}{\kern0pt}\isanewline
\ \ \ \ \isacommand{also}\isamarkupfalse%
\ \isacommand{have}\isamarkupfalse%
\ {\isachardoublequoteopen}{\isachardot}{\kern0pt}{\isachardot}{\kern0pt}{\isachardot}{\kern0pt}\ {\isacharequal}{\kern0pt}\ {\isacharparenleft}{\kern0pt}left{\isacharunderscore}{\kern0pt}coproj\ X\ {\isasymemptyset}{\isacharparenright}{\kern0pt}\ {\isasymcirc}\isactrlsub c\ {\isasymalpha}\isactrlbsub X\isactrlesub {\isachardoublequoteclose}\isanewline
\ \ \ \ \ \ \isacommand{by}\isamarkupfalse%
\ {\isacharparenleft}{\kern0pt}typecheck{\isacharunderscore}{\kern0pt}cfuncs{\isacharcomma}{\kern0pt}\ metis\ right{\isacharunderscore}{\kern0pt}coproj{\isacharunderscore}{\kern0pt}cfunc{\isacharunderscore}{\kern0pt}coprod{\isacharparenright}{\kern0pt}\isanewline
\ \ \ \ \isacommand{also}\isamarkupfalse%
\ \isacommand{have}\isamarkupfalse%
\ {\isachardoublequoteopen}{\isachardot}{\kern0pt}{\isachardot}{\kern0pt}{\isachardot}{\kern0pt}\ {\isacharequal}{\kern0pt}\ right{\isacharunderscore}{\kern0pt}coproj\ X\ {\isasymemptyset}{\isachardoublequoteclose}\isanewline
\ \ \ \ \ \ \isacommand{by}\isamarkupfalse%
\ {\isacharparenleft}{\kern0pt}typecheck{\isacharunderscore}{\kern0pt}cfuncs{\isacharcomma}{\kern0pt}\ metis\ initial{\isacharunderscore}{\kern0pt}func{\isacharunderscore}{\kern0pt}unique{\isacharparenright}{\kern0pt}\isanewline
\ \ \ \ \isacommand{then}\isamarkupfalse%
\ \isacommand{show}\isamarkupfalse%
\ {\isacharquery}{\kern0pt}thesis\ \isacommand{using}\isamarkupfalse%
\ calculation\ \isacommand{by}\isamarkupfalse%
\ auto\isanewline
\ \ \isacommand{qed}\isamarkupfalse%
\isanewline
\ \ \isacommand{then}\isamarkupfalse%
\ \isacommand{have}\isamarkupfalse%
\ fact{\isadigit{1}}{\isacharcolon}{\kern0pt}\ {\isachardoublequoteopen}{\isacharparenleft}{\kern0pt}left{\isacharunderscore}{\kern0pt}coproj\ X\ {\isasymemptyset}{\isacharparenright}{\kern0pt}{\isasymamalg}{\isacharparenleft}{\kern0pt}right{\isacharunderscore}{\kern0pt}coproj\ X\ {\isasymemptyset}{\isacharparenright}{\kern0pt}\ {\isasymcirc}\isactrlsub c\ left{\isacharunderscore}{\kern0pt}coproj\ X\ {\isasymemptyset}\ {\isacharequal}{\kern0pt}\ left{\isacharunderscore}{\kern0pt}coproj\ X\ {\isasymemptyset}{\isachardoublequoteclose}\isanewline
\ \ \ \ \isacommand{using}\isamarkupfalse%
\ left{\isacharunderscore}{\kern0pt}coproj{\isacharunderscore}{\kern0pt}cfunc{\isacharunderscore}{\kern0pt}coprod\ \isacommand{by}\isamarkupfalse%
\ {\isacharparenleft}{\kern0pt}typecheck{\isacharunderscore}{\kern0pt}cfuncs{\isacharcomma}{\kern0pt}\ blast{\isacharparenright}{\kern0pt}\isanewline
\ \ \isacommand{then}\isamarkupfalse%
\ \isacommand{have}\isamarkupfalse%
\ fact{\isadigit{2}}{\isacharcolon}{\kern0pt}\ {\isachardoublequoteopen}{\isacharparenleft}{\kern0pt}{\isacharparenleft}{\kern0pt}left{\isacharunderscore}{\kern0pt}coproj\ X\ {\isasymemptyset}{\isacharparenright}{\kern0pt}{\isasymamalg}{\isacharparenleft}{\kern0pt}right{\isacharunderscore}{\kern0pt}coproj\ X\ {\isasymemptyset}{\isacharparenright}{\kern0pt}{\isacharparenright}{\kern0pt}\ {\isasymcirc}\isactrlsub c\ {\isacharparenleft}{\kern0pt}right{\isacharunderscore}{\kern0pt}coproj\ X\ {\isasymemptyset}{\isacharparenright}{\kern0pt}\ {\isacharequal}{\kern0pt}\ right{\isacharunderscore}{\kern0pt}coproj\ X\ {\isasymemptyset}{\isachardoublequoteclose}\isanewline
\ \ \ \ \isacommand{using}\isamarkupfalse%
\ right{\isacharunderscore}{\kern0pt}coproj{\isacharunderscore}{\kern0pt}cfunc{\isacharunderscore}{\kern0pt}coprod\ \isacommand{by}\isamarkupfalse%
\ {\isacharparenleft}{\kern0pt}typecheck{\isacharunderscore}{\kern0pt}cfuncs{\isacharcomma}{\kern0pt}\ blast{\isacharparenright}{\kern0pt}\isanewline
\ \ \isacommand{then}\isamarkupfalse%
\ \isacommand{have}\isamarkupfalse%
\ concl{\isacharcolon}{\kern0pt}\ {\isachardoublequoteopen}{\isacharparenleft}{\kern0pt}left{\isacharunderscore}{\kern0pt}coproj\ X\ {\isasymemptyset}{\isacharparenright}{\kern0pt}\ {\isasymcirc}\isactrlsub c\ {\isacharparenleft}{\kern0pt}id{\isacharparenleft}{\kern0pt}X{\isacharparenright}{\kern0pt}\ {\isasymamalg}\ {\isasymalpha}\isactrlbsub X\isactrlesub {\isacharparenright}{\kern0pt}\ {\isacharequal}{\kern0pt}\ {\isacharparenleft}{\kern0pt}{\isacharparenleft}{\kern0pt}left{\isacharunderscore}{\kern0pt}coproj\ X\ {\isasymemptyset}{\isacharparenright}{\kern0pt}{\isasymamalg}{\isacharparenleft}{\kern0pt}right{\isacharunderscore}{\kern0pt}coproj\ X\ {\isasymemptyset}{\isacharparenright}{\kern0pt}{\isacharparenright}{\kern0pt}{\isachardoublequoteclose}\isanewline
\ \ \ \ \isacommand{using}\isamarkupfalse%
\ cfunc{\isacharunderscore}{\kern0pt}coprod{\isacharunderscore}{\kern0pt}unique\ comp{\isadigit{1}}\ comp{\isadigit{2}}\ \isacommand{by}\isamarkupfalse%
\ {\isacharparenleft}{\kern0pt}typecheck{\isacharunderscore}{\kern0pt}cfuncs{\isacharcomma}{\kern0pt}\ blast{\isacharparenright}{\kern0pt}\isanewline
\ \ \isacommand{also}\isamarkupfalse%
\ \isacommand{have}\isamarkupfalse%
\ {\isachardoublequoteopen}{\isachardot}{\kern0pt}{\isachardot}{\kern0pt}{\isachardot}{\kern0pt}\ {\isacharequal}{\kern0pt}\ id{\isacharparenleft}{\kern0pt}X{\isasymCoprod}{\isasymemptyset}{\isacharparenright}{\kern0pt}{\isachardoublequoteclose}\isanewline
\ \ \ \ \isacommand{using}\isamarkupfalse%
\ cfunc{\isacharunderscore}{\kern0pt}coprod{\isacharunderscore}{\kern0pt}unique\ id{\isacharunderscore}{\kern0pt}left{\isacharunderscore}{\kern0pt}unit{\isadigit{2}}\ \isacommand{by}\isamarkupfalse%
\ {\isacharparenleft}{\kern0pt}typecheck{\isacharunderscore}{\kern0pt}cfuncs{\isacharcomma}{\kern0pt}\ auto{\isacharparenright}{\kern0pt}\isanewline
\ \ \isacommand{then}\isamarkupfalse%
\ \isacommand{have}\isamarkupfalse%
\ {\isachardoublequoteopen}isomorphism{\isacharparenleft}{\kern0pt}id{\isacharparenleft}{\kern0pt}X{\isacharparenright}{\kern0pt}\ {\isasymamalg}\ {\isasymalpha}\isactrlbsub X\isactrlesub {\isacharparenright}{\kern0pt}{\isachardoublequoteclose}\isanewline
\ \ \ \ \isacommand{unfolding}\isamarkupfalse%
\ isomorphism{\isacharunderscore}{\kern0pt}def\ \isanewline
\ \ \ \ \isacommand{by}\isamarkupfalse%
\ {\isacharparenleft}{\kern0pt}rule{\isacharunderscore}{\kern0pt}tac\ x{\isacharequal}{\kern0pt}{\isachardoublequoteopen}left{\isacharunderscore}{\kern0pt}coproj\ X\ {\isasymemptyset}{\isachardoublequoteclose}\ \isakeyword{in}\ exI{\isacharcomma}{\kern0pt}\ typecheck{\isacharunderscore}{\kern0pt}cfuncs{\isacharcomma}{\kern0pt}\ simp\ add{\isacharcolon}{\kern0pt}\ cfunc{\isacharunderscore}{\kern0pt}type{\isacharunderscore}{\kern0pt}def\ concl\ left{\isacharunderscore}{\kern0pt}coproj{\isacharunderscore}{\kern0pt}cfunc{\isacharunderscore}{\kern0pt}coprod{\isacharparenright}{\kern0pt}\isanewline
\ \ \isacommand{then}\isamarkupfalse%
\ \isacommand{show}\isamarkupfalse%
\ {\isachardoublequoteopen}X{\isasymCoprod}{\isasymemptyset}\ {\isasymcong}\ X{\isachardoublequoteclose}\isanewline
\ \ \ \ \isacommand{using}\isamarkupfalse%
\ cfunc{\isacharunderscore}{\kern0pt}coprod{\isacharunderscore}{\kern0pt}type\ id{\isacharunderscore}{\kern0pt}type\ initial{\isacharunderscore}{\kern0pt}func{\isacharunderscore}{\kern0pt}type\ is{\isacharunderscore}{\kern0pt}isomorphic{\isacharunderscore}{\kern0pt}def\ \isacommand{by}\isamarkupfalse%
\ blast\isanewline
\isacommand{qed}\isamarkupfalse%
%
\endisatagproof
{\isafoldproof}%
%
\isadelimproof
%
\endisadelimproof
%
\begin{isamarkuptext}%
The lemma below corresponds to Proposition 2.4.7 in Halvorson.%
\end{isamarkuptext}\isamarkuptrue%
\isacommand{lemma}\isamarkupfalse%
\ function{\isacharunderscore}{\kern0pt}to{\isacharunderscore}{\kern0pt}empty{\isacharunderscore}{\kern0pt}is{\isacharunderscore}{\kern0pt}iso{\isacharcolon}{\kern0pt}\isanewline
\ \ \isakeyword{assumes}\ {\isachardoublequoteopen}f{\isacharcolon}{\kern0pt}\ X\ {\isasymrightarrow}\ {\isasymemptyset}{\isachardoublequoteclose}\isanewline
\ \ \isakeyword{shows}\ {\isachardoublequoteopen}isomorphism{\isacharparenleft}{\kern0pt}f{\isacharparenright}{\kern0pt}{\isachardoublequoteclose}\isanewline
%
\isadelimproof
\ \ %
\endisadelimproof
%
\isatagproof
\isacommand{by}\isamarkupfalse%
\ {\isacharparenleft}{\kern0pt}metis\ assms\ cfunc{\isacharunderscore}{\kern0pt}type{\isacharunderscore}{\kern0pt}def\ comp{\isacharunderscore}{\kern0pt}type\ emptyset{\isacharunderscore}{\kern0pt}is{\isacharunderscore}{\kern0pt}empty\ epi{\isacharunderscore}{\kern0pt}mon{\isacharunderscore}{\kern0pt}is{\isacharunderscore}{\kern0pt}iso\ injective{\isacharunderscore}{\kern0pt}def\ injective{\isacharunderscore}{\kern0pt}imp{\isacharunderscore}{\kern0pt}monomorphism\ \ surjective{\isacharunderscore}{\kern0pt}def\ surjective{\isacharunderscore}{\kern0pt}is{\isacharunderscore}{\kern0pt}epimorphism{\isacharparenright}{\kern0pt}%
\endisatagproof
{\isafoldproof}%
%
\isadelimproof
\isanewline
%
\endisadelimproof
\isanewline
\isacommand{lemma}\isamarkupfalse%
\ empty{\isacharunderscore}{\kern0pt}prod{\isacharunderscore}{\kern0pt}X{\isacharcolon}{\kern0pt}\isanewline
\ \ {\isachardoublequoteopen}{\isasymemptyset}\ {\isasymtimes}\isactrlsub c\ X\ {\isasymcong}\ {\isasymemptyset}{\isachardoublequoteclose}\isanewline
%
\isadelimproof
\ \ %
\endisadelimproof
%
\isatagproof
\isacommand{using}\isamarkupfalse%
\ cfunc{\isacharunderscore}{\kern0pt}type{\isacharunderscore}{\kern0pt}def\ function{\isacharunderscore}{\kern0pt}to{\isacharunderscore}{\kern0pt}empty{\isacharunderscore}{\kern0pt}is{\isacharunderscore}{\kern0pt}iso\ is{\isacharunderscore}{\kern0pt}isomorphic{\isacharunderscore}{\kern0pt}def\ left{\isacharunderscore}{\kern0pt}cart{\isacharunderscore}{\kern0pt}proj{\isacharunderscore}{\kern0pt}type\ \isacommand{by}\isamarkupfalse%
\ blast%
\endisatagproof
{\isafoldproof}%
%
\isadelimproof
\isanewline
%
\endisadelimproof
\isanewline
\isacommand{lemma}\isamarkupfalse%
\ X{\isacharunderscore}{\kern0pt}prod{\isacharunderscore}{\kern0pt}empty{\isacharcolon}{\kern0pt}\ \isanewline
\ \ {\isachardoublequoteopen}X\ {\isasymtimes}\isactrlsub c\ {\isasymemptyset}\ {\isasymcong}\ {\isasymemptyset}{\isachardoublequoteclose}\isanewline
%
\isadelimproof
\ \ %
\endisadelimproof
%
\isatagproof
\isacommand{using}\isamarkupfalse%
\ cfunc{\isacharunderscore}{\kern0pt}type{\isacharunderscore}{\kern0pt}def\ function{\isacharunderscore}{\kern0pt}to{\isacharunderscore}{\kern0pt}empty{\isacharunderscore}{\kern0pt}is{\isacharunderscore}{\kern0pt}iso\ is{\isacharunderscore}{\kern0pt}isomorphic{\isacharunderscore}{\kern0pt}def\ right{\isacharunderscore}{\kern0pt}cart{\isacharunderscore}{\kern0pt}proj{\isacharunderscore}{\kern0pt}type\ \isacommand{by}\isamarkupfalse%
\ blast%
\endisatagproof
{\isafoldproof}%
%
\isadelimproof
%
\endisadelimproof
%
\begin{isamarkuptext}%
The lemma below corresponds to Proposition 2.4.8 in Halvorson.%
\end{isamarkuptext}\isamarkuptrue%
\isacommand{lemma}\isamarkupfalse%
\ no{\isacharunderscore}{\kern0pt}el{\isacharunderscore}{\kern0pt}iff{\isacharunderscore}{\kern0pt}iso{\isacharunderscore}{\kern0pt}empty{\isacharcolon}{\kern0pt}\isanewline
\ \ {\isachardoublequoteopen}is{\isacharunderscore}{\kern0pt}empty\ X\ {\isasymlongleftrightarrow}\ X\ {\isasymcong}\ {\isasymemptyset}{\isachardoublequoteclose}\isanewline
%
\isadelimproof
%
\endisadelimproof
%
\isatagproof
\isacommand{proof}\isamarkupfalse%
\ safe\isanewline
\ \ \isacommand{show}\isamarkupfalse%
\ {\isachardoublequoteopen}X\ {\isasymcong}\ {\isasymemptyset}\ {\isasymLongrightarrow}\ is{\isacharunderscore}{\kern0pt}empty\ X{\isachardoublequoteclose}\isanewline
\ \ \ \ \isacommand{by}\isamarkupfalse%
\ {\isacharparenleft}{\kern0pt}meson\ is{\isacharunderscore}{\kern0pt}empty{\isacharunderscore}{\kern0pt}def\ comp{\isacharunderscore}{\kern0pt}type\ emptyset{\isacharunderscore}{\kern0pt}is{\isacharunderscore}{\kern0pt}empty\ is{\isacharunderscore}{\kern0pt}isomorphic{\isacharunderscore}{\kern0pt}def{\isacharparenright}{\kern0pt}\isanewline
\isacommand{next}\isamarkupfalse%
\ \isanewline
\ \ \isacommand{assume}\isamarkupfalse%
\ {\isachardoublequoteopen}is{\isacharunderscore}{\kern0pt}empty\ X{\isachardoublequoteclose}\isanewline
\ \ \isacommand{obtain}\isamarkupfalse%
\ f\ \isakeyword{where}\ f{\isacharunderscore}{\kern0pt}type{\isacharcolon}{\kern0pt}\ {\isachardoublequoteopen}f{\isacharcolon}{\kern0pt}\ {\isasymemptyset}\ {\isasymrightarrow}\ X{\isachardoublequoteclose}\isanewline
\ \ \ \ \isacommand{using}\isamarkupfalse%
\ initial{\isacharunderscore}{\kern0pt}func{\isacharunderscore}{\kern0pt}type\ \isacommand{by}\isamarkupfalse%
\ blast\isanewline
\ \isanewline
\ \ \isacommand{have}\isamarkupfalse%
\ \ f{\isacharunderscore}{\kern0pt}inj{\isacharcolon}{\kern0pt}\ {\isachardoublequoteopen}injective{\isacharparenleft}{\kern0pt}f{\isacharparenright}{\kern0pt}{\isachardoublequoteclose}\isanewline
\ \ \ \ \isacommand{using}\isamarkupfalse%
\ cfunc{\isacharunderscore}{\kern0pt}type{\isacharunderscore}{\kern0pt}def\ emptyset{\isacharunderscore}{\kern0pt}is{\isacharunderscore}{\kern0pt}empty\ f{\isacharunderscore}{\kern0pt}type\ injective{\isacharunderscore}{\kern0pt}def\ \isacommand{by}\isamarkupfalse%
\ auto\isanewline
\ \ \isacommand{then}\isamarkupfalse%
\ \isacommand{have}\isamarkupfalse%
\ f{\isacharunderscore}{\kern0pt}mono{\isacharcolon}{\kern0pt}\ {\isachardoublequoteopen}monomorphism{\isacharparenleft}{\kern0pt}f{\isacharparenright}{\kern0pt}{\isachardoublequoteclose}\isanewline
\ \ \ \ \isacommand{using}\isamarkupfalse%
\ \ cfunc{\isacharunderscore}{\kern0pt}type{\isacharunderscore}{\kern0pt}def\ f{\isacharunderscore}{\kern0pt}type\ injective{\isacharunderscore}{\kern0pt}imp{\isacharunderscore}{\kern0pt}monomorphism\ \isacommand{by}\isamarkupfalse%
\ blast\isanewline
\ \ \isacommand{have}\isamarkupfalse%
\ f{\isacharunderscore}{\kern0pt}surj{\isacharcolon}{\kern0pt}\ {\isachardoublequoteopen}surjective{\isacharparenleft}{\kern0pt}f{\isacharparenright}{\kern0pt}{\isachardoublequoteclose}\isanewline
\ \ \ \ \isacommand{using}\isamarkupfalse%
\ is{\isacharunderscore}{\kern0pt}empty{\isacharunderscore}{\kern0pt}def\ {\isacartoucheopen}is{\isacharunderscore}{\kern0pt}empty\ X{\isacartoucheclose}\ f{\isacharunderscore}{\kern0pt}type\ surjective{\isacharunderscore}{\kern0pt}def{\isadigit{2}}\ \isacommand{by}\isamarkupfalse%
\ presburger\isanewline
\ \ \isacommand{then}\isamarkupfalse%
\ \isacommand{have}\isamarkupfalse%
\ epi{\isacharunderscore}{\kern0pt}f{\isacharcolon}{\kern0pt}\ {\isachardoublequoteopen}epimorphism{\isacharparenleft}{\kern0pt}f{\isacharparenright}{\kern0pt}{\isachardoublequoteclose}\isanewline
\ \ \ \ \isacommand{using}\isamarkupfalse%
\ surjective{\isacharunderscore}{\kern0pt}is{\isacharunderscore}{\kern0pt}epimorphism\ \isacommand{by}\isamarkupfalse%
\ blast\isanewline
\ \ \isacommand{then}\isamarkupfalse%
\ \isacommand{have}\isamarkupfalse%
\ iso{\isacharunderscore}{\kern0pt}f{\isacharcolon}{\kern0pt}\ {\isachardoublequoteopen}isomorphism{\isacharparenleft}{\kern0pt}f{\isacharparenright}{\kern0pt}{\isachardoublequoteclose}\isanewline
\ \ \ \ \isacommand{using}\isamarkupfalse%
\ cfunc{\isacharunderscore}{\kern0pt}type{\isacharunderscore}{\kern0pt}def\ epi{\isacharunderscore}{\kern0pt}mon{\isacharunderscore}{\kern0pt}is{\isacharunderscore}{\kern0pt}iso\ f{\isacharunderscore}{\kern0pt}mono\ f{\isacharunderscore}{\kern0pt}type\ \isacommand{by}\isamarkupfalse%
\ blast\isanewline
\ \ \isacommand{then}\isamarkupfalse%
\ \isacommand{show}\isamarkupfalse%
\ {\isachardoublequoteopen}X\ {\isasymcong}\ {\isasymemptyset}{\isachardoublequoteclose}\isanewline
\ \ \ \ \isacommand{using}\isamarkupfalse%
\ f{\isacharunderscore}{\kern0pt}type\ is{\isacharunderscore}{\kern0pt}isomorphic{\isacharunderscore}{\kern0pt}def\ isomorphic{\isacharunderscore}{\kern0pt}is{\isacharunderscore}{\kern0pt}symmetric\ \isacommand{by}\isamarkupfalse%
\ blast\isanewline
\isacommand{qed}\isamarkupfalse%
%
\endisatagproof
{\isafoldproof}%
%
\isadelimproof
\isanewline
%
\endisadelimproof
\isanewline
\isacommand{lemma}\isamarkupfalse%
\ initial{\isacharunderscore}{\kern0pt}maps{\isacharunderscore}{\kern0pt}mono{\isacharcolon}{\kern0pt}\isanewline
\ \ \isakeyword{assumes}\ {\isachardoublequoteopen}initial{\isacharunderscore}{\kern0pt}object{\isacharparenleft}{\kern0pt}X{\isacharparenright}{\kern0pt}{\isachardoublequoteclose}\isanewline
\ \ \isakeyword{assumes}\ {\isachardoublequoteopen}f\ {\isacharcolon}{\kern0pt}\ X\ {\isasymrightarrow}\ Y{\isachardoublequoteclose}\isanewline
\ \ \isakeyword{shows}\ {\isachardoublequoteopen}monomorphism{\isacharparenleft}{\kern0pt}f{\isacharparenright}{\kern0pt}{\isachardoublequoteclose}\isanewline
%
\isadelimproof
\ \ %
\endisadelimproof
%
\isatagproof
\isacommand{by}\isamarkupfalse%
\ {\isacharparenleft}{\kern0pt}metis\ assms\ cfunc{\isacharunderscore}{\kern0pt}type{\isacharunderscore}{\kern0pt}def\ initial{\isacharunderscore}{\kern0pt}iso{\isacharunderscore}{\kern0pt}empty\ injective{\isacharunderscore}{\kern0pt}def\ injective{\isacharunderscore}{\kern0pt}imp{\isacharunderscore}{\kern0pt}monomorphism\ no{\isacharunderscore}{\kern0pt}el{\isacharunderscore}{\kern0pt}iff{\isacharunderscore}{\kern0pt}iso{\isacharunderscore}{\kern0pt}empty\ is{\isacharunderscore}{\kern0pt}empty{\isacharunderscore}{\kern0pt}def{\isacharparenright}{\kern0pt}%
\endisatagproof
{\isafoldproof}%
%
\isadelimproof
\isanewline
%
\endisadelimproof
\isanewline
\isacommand{lemma}\isamarkupfalse%
\ iso{\isacharunderscore}{\kern0pt}empty{\isacharunderscore}{\kern0pt}initial{\isacharcolon}{\kern0pt}\isanewline
\ \ \isakeyword{assumes}\ {\isachardoublequoteopen}X\ {\isasymcong}\ {\isasymemptyset}{\isachardoublequoteclose}\isanewline
\ \ \isakeyword{shows}\ {\isachardoublequoteopen}initial{\isacharunderscore}{\kern0pt}object\ X{\isachardoublequoteclose}\isanewline
%
\isadelimproof
\ \ %
\endisadelimproof
%
\isatagproof
\isacommand{unfolding}\isamarkupfalse%
\ initial{\isacharunderscore}{\kern0pt}object{\isacharunderscore}{\kern0pt}def\isanewline
\ \ \isacommand{by}\isamarkupfalse%
\ {\isacharparenleft}{\kern0pt}meson\ assms\ comp{\isacharunderscore}{\kern0pt}type\ is{\isacharunderscore}{\kern0pt}isomorphic{\isacharunderscore}{\kern0pt}def\ isomorphic{\isacharunderscore}{\kern0pt}is{\isacharunderscore}{\kern0pt}symmetric\ isomorphic{\isacharunderscore}{\kern0pt}is{\isacharunderscore}{\kern0pt}transitive\ no{\isacharunderscore}{\kern0pt}el{\isacharunderscore}{\kern0pt}iff{\isacharunderscore}{\kern0pt}iso{\isacharunderscore}{\kern0pt}empty\ is{\isacharunderscore}{\kern0pt}empty{\isacharunderscore}{\kern0pt}def\ one{\isacharunderscore}{\kern0pt}separator\ terminal{\isacharunderscore}{\kern0pt}func{\isacharunderscore}{\kern0pt}type{\isacharparenright}{\kern0pt}%
\endisatagproof
{\isafoldproof}%
%
\isadelimproof
\isanewline
%
\endisadelimproof
\isanewline
\isacommand{lemma}\isamarkupfalse%
\ function{\isacharunderscore}{\kern0pt}to{\isacharunderscore}{\kern0pt}empty{\isacharunderscore}{\kern0pt}set{\isacharunderscore}{\kern0pt}is{\isacharunderscore}{\kern0pt}iso{\isacharcolon}{\kern0pt}\isanewline
\ \ \isakeyword{assumes}\ {\isachardoublequoteopen}f{\isacharcolon}{\kern0pt}\ X\ {\isasymrightarrow}\ Y{\isachardoublequoteclose}\isanewline
\ \ \isakeyword{assumes}\ {\isachardoublequoteopen}is{\isacharunderscore}{\kern0pt}empty\ Y{\isachardoublequoteclose}\isanewline
\ \ \isakeyword{shows}\ {\isachardoublequoteopen}isomorphism\ f{\isachardoublequoteclose}\isanewline
%
\isadelimproof
\ \ %
\endisadelimproof
%
\isatagproof
\isacommand{by}\isamarkupfalse%
\ {\isacharparenleft}{\kern0pt}metis\ assms\ cfunc{\isacharunderscore}{\kern0pt}type{\isacharunderscore}{\kern0pt}def\ comp{\isacharunderscore}{\kern0pt}type\ epi{\isacharunderscore}{\kern0pt}mon{\isacharunderscore}{\kern0pt}is{\isacharunderscore}{\kern0pt}iso\ injective{\isacharunderscore}{\kern0pt}def\ injective{\isacharunderscore}{\kern0pt}imp{\isacharunderscore}{\kern0pt}monomorphism\ is{\isacharunderscore}{\kern0pt}empty{\isacharunderscore}{\kern0pt}def\ surjective{\isacharunderscore}{\kern0pt}def\ surjective{\isacharunderscore}{\kern0pt}is{\isacharunderscore}{\kern0pt}epimorphism{\isacharparenright}{\kern0pt}%
\endisatagproof
{\isafoldproof}%
%
\isadelimproof
\isanewline
%
\endisadelimproof
\isanewline
\isacommand{lemma}\isamarkupfalse%
\ prod{\isacharunderscore}{\kern0pt}iso{\isacharunderscore}{\kern0pt}to{\isacharunderscore}{\kern0pt}empty{\isacharunderscore}{\kern0pt}right{\isacharcolon}{\kern0pt}\isanewline
\ \ \isakeyword{assumes}\ {\isachardoublequoteopen}nonempty\ X{\isachardoublequoteclose}\isanewline
\ \ \isakeyword{assumes}\ {\isachardoublequoteopen}X\ {\isasymtimes}\isactrlsub c\ Y\ {\isasymcong}\ {\isasymemptyset}{\isachardoublequoteclose}\isanewline
\ \ \isakeyword{shows}\ {\isachardoublequoteopen}is{\isacharunderscore}{\kern0pt}empty\ Y{\isachardoublequoteclose}\isanewline
%
\isadelimproof
\ \ %
\endisadelimproof
%
\isatagproof
\isacommand{by}\isamarkupfalse%
\ {\isacharparenleft}{\kern0pt}metis\ emptyset{\isacharunderscore}{\kern0pt}is{\isacharunderscore}{\kern0pt}empty\ is{\isacharunderscore}{\kern0pt}empty{\isacharunderscore}{\kern0pt}def\ cfunc{\isacharunderscore}{\kern0pt}prod{\isacharunderscore}{\kern0pt}type\ epi{\isacharunderscore}{\kern0pt}is{\isacharunderscore}{\kern0pt}surj\ is{\isacharunderscore}{\kern0pt}isomorphic{\isacharunderscore}{\kern0pt}def\ iso{\isacharunderscore}{\kern0pt}imp{\isacharunderscore}{\kern0pt}epi{\isacharunderscore}{\kern0pt}and{\isacharunderscore}{\kern0pt}monic\ isomorphic{\isacharunderscore}{\kern0pt}is{\isacharunderscore}{\kern0pt}symmetric\ nonempty{\isacharunderscore}{\kern0pt}def\ surjective{\isacharunderscore}{\kern0pt}def{\isadigit{2}}\ assms{\isacharparenright}{\kern0pt}%
\endisatagproof
{\isafoldproof}%
%
\isadelimproof
\isanewline
%
\endisadelimproof
\isanewline
\isacommand{lemma}\isamarkupfalse%
\ prod{\isacharunderscore}{\kern0pt}iso{\isacharunderscore}{\kern0pt}to{\isacharunderscore}{\kern0pt}empty{\isacharunderscore}{\kern0pt}left{\isacharcolon}{\kern0pt}\isanewline
\ \ \isakeyword{assumes}\ {\isachardoublequoteopen}nonempty\ Y{\isachardoublequoteclose}\isanewline
\ \ \isakeyword{assumes}\ {\isachardoublequoteopen}X\ {\isasymtimes}\isactrlsub c\ Y\ {\isasymcong}\ {\isasymemptyset}{\isachardoublequoteclose}\isanewline
\ \ \isakeyword{shows}\ {\isachardoublequoteopen}is{\isacharunderscore}{\kern0pt}empty\ X{\isachardoublequoteclose}\isanewline
%
\isadelimproof
\ \ %
\endisadelimproof
%
\isatagproof
\isacommand{by}\isamarkupfalse%
\ {\isacharparenleft}{\kern0pt}meson\ is{\isacharunderscore}{\kern0pt}empty{\isacharunderscore}{\kern0pt}def\ nonempty{\isacharunderscore}{\kern0pt}def\ prod{\isacharunderscore}{\kern0pt}iso{\isacharunderscore}{\kern0pt}to{\isacharunderscore}{\kern0pt}empty{\isacharunderscore}{\kern0pt}right\ assms{\isacharparenright}{\kern0pt}%
\endisatagproof
{\isafoldproof}%
%
\isadelimproof
\isanewline
%
\endisadelimproof
\isanewline
\isacommand{lemma}\isamarkupfalse%
\ empty{\isacharunderscore}{\kern0pt}subset{\isacharcolon}{\kern0pt}\ {\isachardoublequoteopen}{\isacharparenleft}{\kern0pt}{\isasymemptyset}{\isacharcomma}{\kern0pt}\ {\isasymalpha}\isactrlbsub X\isactrlesub {\isacharparenright}{\kern0pt}\ {\isasymsubseteq}\isactrlsub c\ X{\isachardoublequoteclose}\isanewline
%
\isadelimproof
\ \ %
\endisadelimproof
%
\isatagproof
\isacommand{by}\isamarkupfalse%
\ {\isacharparenleft}{\kern0pt}metis\ cfunc{\isacharunderscore}{\kern0pt}type{\isacharunderscore}{\kern0pt}def\ emptyset{\isacharunderscore}{\kern0pt}is{\isacharunderscore}{\kern0pt}empty\ initial{\isacharunderscore}{\kern0pt}func{\isacharunderscore}{\kern0pt}type\ injective{\isacharunderscore}{\kern0pt}def\ injective{\isacharunderscore}{\kern0pt}imp{\isacharunderscore}{\kern0pt}monomorphism\ subobject{\isacharunderscore}{\kern0pt}of{\isacharunderscore}{\kern0pt}def{\isadigit{2}}{\isacharparenright}{\kern0pt}%
\endisatagproof
{\isafoldproof}%
%
\isadelimproof
%
\endisadelimproof
%
\begin{isamarkuptext}%
The lemma below corresponds to Proposition 2.2.1 in Halvorson.%
\end{isamarkuptext}\isamarkuptrue%
\isacommand{lemma}\isamarkupfalse%
\ one{\isacharunderscore}{\kern0pt}has{\isacharunderscore}{\kern0pt}two{\isacharunderscore}{\kern0pt}subsets{\isacharcolon}{\kern0pt}\isanewline
\ \ {\isachardoublequoteopen}card\ {\isacharparenleft}{\kern0pt}{\isacharbraceleft}{\kern0pt}{\isacharparenleft}{\kern0pt}X{\isacharcomma}{\kern0pt}m{\isacharparenright}{\kern0pt}{\isachardot}{\kern0pt}\ {\isacharparenleft}{\kern0pt}X{\isacharcomma}{\kern0pt}m{\isacharparenright}{\kern0pt}\ {\isasymsubseteq}\isactrlsub c\ {\isasymone}{\isacharbraceright}{\kern0pt}{\isacharslash}{\kern0pt}{\isacharslash}{\kern0pt}{\isacharbraceleft}{\kern0pt}{\isacharparenleft}{\kern0pt}{\isacharparenleft}{\kern0pt}X{\isadigit{1}}{\isacharcomma}{\kern0pt}m{\isadigit{1}}{\isacharparenright}{\kern0pt}{\isacharcomma}{\kern0pt}{\isacharparenleft}{\kern0pt}X{\isadigit{2}}{\isacharcomma}{\kern0pt}m{\isadigit{2}}{\isacharparenright}{\kern0pt}{\isacharparenright}{\kern0pt}{\isachardot}{\kern0pt}\ X{\isadigit{1}}\ {\isasymcong}\ X{\isadigit{2}}{\isacharbraceright}{\kern0pt}{\isacharparenright}{\kern0pt}\ {\isacharequal}{\kern0pt}\ {\isadigit{2}}{\isachardoublequoteclose}\isanewline
%
\isadelimproof
%
\endisadelimproof
%
\isatagproof
\isacommand{proof}\isamarkupfalse%
\ {\isacharminus}{\kern0pt}\isanewline
\ \ \isacommand{have}\isamarkupfalse%
\ one{\isacharunderscore}{\kern0pt}subobject{\isacharcolon}{\kern0pt}\ {\isachardoublequoteopen}{\isacharparenleft}{\kern0pt}{\isasymone}{\isacharcomma}{\kern0pt}\ id\ {\isasymone}{\isacharparenright}{\kern0pt}\ {\isasymsubseteq}\isactrlsub c\ {\isasymone}{\isachardoublequoteclose}\isanewline
\ \ \ \ \isacommand{using}\isamarkupfalse%
\ element{\isacharunderscore}{\kern0pt}monomorphism\ id{\isacharunderscore}{\kern0pt}type\ subobject{\isacharunderscore}{\kern0pt}of{\isacharunderscore}{\kern0pt}def{\isadigit{2}}\ \isacommand{by}\isamarkupfalse%
\ blast\isanewline
\ \ \isacommand{have}\isamarkupfalse%
\ empty{\isacharunderscore}{\kern0pt}subobject{\isacharcolon}{\kern0pt}\ {\isachardoublequoteopen}{\isacharparenleft}{\kern0pt}{\isasymemptyset}{\isacharcomma}{\kern0pt}\ {\isasymalpha}\isactrlbsub {\isasymone}\isactrlesub {\isacharparenright}{\kern0pt}\ {\isasymsubseteq}\isactrlsub c\ {\isasymone}{\isachardoublequoteclose}\isanewline
\ \ \ \ \isacommand{by}\isamarkupfalse%
\ {\isacharparenleft}{\kern0pt}simp\ add{\isacharcolon}{\kern0pt}\ empty{\isacharunderscore}{\kern0pt}subset{\isacharparenright}{\kern0pt}\isanewline
\isanewline
\ \ \isacommand{have}\isamarkupfalse%
\ subobject{\isacharunderscore}{\kern0pt}one{\isacharunderscore}{\kern0pt}or{\isacharunderscore}{\kern0pt}empty{\isacharcolon}{\kern0pt}\ {\isachardoublequoteopen}{\isasymAnd}X\ m{\isachardot}{\kern0pt}\ {\isacharparenleft}{\kern0pt}X{\isacharcomma}{\kern0pt}m{\isacharparenright}{\kern0pt}\ {\isasymsubseteq}\isactrlsub c\ {\isasymone}\ {\isasymLongrightarrow}\ X\ {\isasymcong}\ {\isasymone}\ {\isasymor}\ X\ {\isasymcong}\ {\isasymemptyset}{\isachardoublequoteclose}\isanewline
\ \ \isacommand{proof}\isamarkupfalse%
\ {\isacharminus}{\kern0pt}\isanewline
\ \ \ \ \isacommand{fix}\isamarkupfalse%
\ X\ m\isanewline
\ \ \ \ \isacommand{assume}\isamarkupfalse%
\ X{\isacharunderscore}{\kern0pt}m{\isacharunderscore}{\kern0pt}subobject{\isacharcolon}{\kern0pt}\ {\isachardoublequoteopen}{\isacharparenleft}{\kern0pt}X{\isacharcomma}{\kern0pt}\ m{\isacharparenright}{\kern0pt}\ {\isasymsubseteq}\isactrlsub c\ {\isasymone}{\isachardoublequoteclose}\isanewline
\isanewline
\ \ \ \ \isacommand{obtain}\isamarkupfalse%
\ {\isasymchi}\ \isakeyword{where}\ {\isasymchi}{\isacharunderscore}{\kern0pt}pullback{\isacharcolon}{\kern0pt}\ {\isachardoublequoteopen}is{\isacharunderscore}{\kern0pt}pullback\ X\ {\isasymone}\ {\isasymone}\ {\isasymOmega}\ {\isacharparenleft}{\kern0pt}{\isasymbeta}\isactrlbsub X\isactrlesub {\isacharparenright}{\kern0pt}\ {\isasymt}\ m\ {\isasymchi}{\isachardoublequoteclose}\isanewline
\ \ \ \ \ \ \isacommand{using}\isamarkupfalse%
\ X{\isacharunderscore}{\kern0pt}m{\isacharunderscore}{\kern0pt}subobject\ characteristic{\isacharunderscore}{\kern0pt}function{\isacharunderscore}{\kern0pt}exists\ subobject{\isacharunderscore}{\kern0pt}of{\isacharunderscore}{\kern0pt}def{\isadigit{2}}\ \isacommand{by}\isamarkupfalse%
\ blast\isanewline
\ \ \ \ \isacommand{then}\isamarkupfalse%
\ \isacommand{have}\isamarkupfalse%
\ {\isasymchi}{\isacharunderscore}{\kern0pt}true{\isacharunderscore}{\kern0pt}or{\isacharunderscore}{\kern0pt}false{\isacharcolon}{\kern0pt}\ {\isachardoublequoteopen}{\isasymchi}\ {\isacharequal}{\kern0pt}\ {\isasymt}\ {\isasymor}\ {\isasymchi}\ {\isacharequal}{\kern0pt}\ {\isasymf}{\isachardoublequoteclose}\isanewline
\ \ \ \ \ \ \isacommand{unfolding}\isamarkupfalse%
\ is{\isacharunderscore}{\kern0pt}pullback{\isacharunderscore}{\kern0pt}def\ \ \isacommand{using}\isamarkupfalse%
\ true{\isacharunderscore}{\kern0pt}false{\isacharunderscore}{\kern0pt}only{\isacharunderscore}{\kern0pt}truth{\isacharunderscore}{\kern0pt}values\ \isacommand{by}\isamarkupfalse%
\ auto\isanewline
\isanewline
\ \ \ \ \isacommand{have}\isamarkupfalse%
\ true{\isacharunderscore}{\kern0pt}iso{\isacharunderscore}{\kern0pt}one{\isacharcolon}{\kern0pt}\ {\isachardoublequoteopen}{\isasymchi}\ {\isacharequal}{\kern0pt}\ {\isasymt}\ {\isasymLongrightarrow}\ X\ {\isasymcong}\ {\isasymone}{\isachardoublequoteclose}\isanewline
\ \ \ \ \isacommand{proof}\isamarkupfalse%
\ {\isacharminus}{\kern0pt}\isanewline
\ \ \ \ \ \ \isacommand{assume}\isamarkupfalse%
\ {\isasymchi}{\isacharunderscore}{\kern0pt}true{\isacharcolon}{\kern0pt}\ {\isachardoublequoteopen}{\isasymchi}\ {\isacharequal}{\kern0pt}\ {\isasymt}{\isachardoublequoteclose}\isanewline
\ \ \ \ \ \ \isacommand{then}\isamarkupfalse%
\ \isacommand{have}\isamarkupfalse%
\ {\isachardoublequoteopen}{\isasymexists}{\isacharbang}{\kern0pt}j{\isachardot}{\kern0pt}\ j\ {\isasymin}\isactrlsub c\ X\ {\isasymand}\ {\isasymbeta}\isactrlbsub X\isactrlesub \ {\isasymcirc}\isactrlsub c\ j\ {\isacharequal}{\kern0pt}\ id\isactrlsub c\ {\isasymone}\ {\isasymand}\ m\ {\isasymcirc}\isactrlsub c\ j\ {\isacharequal}{\kern0pt}\ id\isactrlsub c\ {\isasymone}{\isachardoublequoteclose}\isanewline
\ \ \ \ \ \ \ \ \isacommand{using}\isamarkupfalse%
\ {\isasymchi}{\isacharunderscore}{\kern0pt}pullback\ {\isasymchi}{\isacharunderscore}{\kern0pt}true\ is{\isacharunderscore}{\kern0pt}pullback{\isacharunderscore}{\kern0pt}def\ \isacommand{by}\isamarkupfalse%
\ {\isacharparenleft}{\kern0pt}typecheck{\isacharunderscore}{\kern0pt}cfuncs{\isacharcomma}{\kern0pt}\ auto{\isacharparenright}{\kern0pt}\isanewline
\ \ \ \ \ \ \isacommand{then}\isamarkupfalse%
\ \isacommand{show}\isamarkupfalse%
\ {\isachardoublequoteopen}X\ {\isasymcong}\ {\isasymone}{\isachardoublequoteclose}\isanewline
\ \ \ \ \ \ \ \ \isacommand{using}\isamarkupfalse%
\ single{\isacharunderscore}{\kern0pt}elem{\isacharunderscore}{\kern0pt}iso{\isacharunderscore}{\kern0pt}one\isanewline
\ \ \ \ \ \ \ \ \isacommand{by}\isamarkupfalse%
\ {\isacharparenleft}{\kern0pt}metis\ X{\isacharunderscore}{\kern0pt}m{\isacharunderscore}{\kern0pt}subobject\ subobject{\isacharunderscore}{\kern0pt}of{\isacharunderscore}{\kern0pt}def{\isadigit{2}}\ terminal{\isacharunderscore}{\kern0pt}func{\isacharunderscore}{\kern0pt}comp{\isacharunderscore}{\kern0pt}elem\ terminal{\isacharunderscore}{\kern0pt}func{\isacharunderscore}{\kern0pt}unique{\isacharparenright}{\kern0pt}\ \isanewline
\ \ \ \ \isacommand{qed}\isamarkupfalse%
\isanewline
\isanewline
\ \ \ \ \isacommand{have}\isamarkupfalse%
\ false{\isacharunderscore}{\kern0pt}iso{\isacharunderscore}{\kern0pt}one{\isacharcolon}{\kern0pt}\ {\isachardoublequoteopen}{\isasymchi}\ {\isacharequal}{\kern0pt}\ {\isasymf}\ {\isasymLongrightarrow}\ X\ {\isasymcong}\ {\isasymemptyset}{\isachardoublequoteclose}\isanewline
\ \ \ \ \isacommand{proof}\isamarkupfalse%
\ {\isacharminus}{\kern0pt}\isanewline
\ \ \ \ \ \ \isacommand{assume}\isamarkupfalse%
\ {\isasymchi}{\isacharunderscore}{\kern0pt}false{\isacharcolon}{\kern0pt}\ {\isachardoublequoteopen}{\isasymchi}\ {\isacharequal}{\kern0pt}\ {\isasymf}{\isachardoublequoteclose}\isanewline
\ \ \ \ \ \ \isacommand{have}\isamarkupfalse%
\ {\isachardoublequoteopen}{\isasymnexists}\ x{\isachardot}{\kern0pt}\ x\ {\isasymin}\isactrlsub c\ X{\isachardoublequoteclose}\isanewline
\ \ \ \ \ \ \isacommand{proof}\isamarkupfalse%
\ clarify\isanewline
\ \ \ \ \ \ \ \ \isacommand{fix}\isamarkupfalse%
\ x\isanewline
\ \ \ \ \ \ \ \ \isacommand{assume}\isamarkupfalse%
\ x{\isacharunderscore}{\kern0pt}in{\isacharunderscore}{\kern0pt}X{\isacharcolon}{\kern0pt}\ {\isachardoublequoteopen}x\ {\isasymin}\isactrlsub c\ X{\isachardoublequoteclose}\isanewline
\ \ \ \ \ \ \ \ \isacommand{have}\isamarkupfalse%
\ {\isachardoublequoteopen}{\isasymt}\ {\isasymcirc}\isactrlsub c\ {\isasymbeta}\isactrlbsub X\isactrlesub \ {\isacharequal}{\kern0pt}\ {\isasymf}\ {\isasymcirc}\isactrlsub c\ m{\isachardoublequoteclose}\isanewline
\ \ \ \ \ \ \ \ \ \ \isacommand{using}\isamarkupfalse%
\ {\isasymchi}{\isacharunderscore}{\kern0pt}false\ {\isasymchi}{\isacharunderscore}{\kern0pt}pullback\ is{\isacharunderscore}{\kern0pt}pullback{\isacharunderscore}{\kern0pt}def\ \isacommand{by}\isamarkupfalse%
\ auto\isanewline
\ \ \ \ \ \ \ \ \isacommand{then}\isamarkupfalse%
\ \isacommand{have}\isamarkupfalse%
\ {\isachardoublequoteopen}{\isasymt}\ {\isasymcirc}\isactrlsub c\ {\isacharparenleft}{\kern0pt}{\isasymbeta}\isactrlbsub X\isactrlesub \ {\isasymcirc}\isactrlsub c\ x{\isacharparenright}{\kern0pt}\ {\isacharequal}{\kern0pt}\ {\isasymf}\ {\isasymcirc}\isactrlsub c\ {\isacharparenleft}{\kern0pt}m\ {\isasymcirc}\isactrlsub c\ x{\isacharparenright}{\kern0pt}{\isachardoublequoteclose}\isanewline
\ \ \ \ \ \ \ \ \ \ \isacommand{by}\isamarkupfalse%
\ {\isacharparenleft}{\kern0pt}smt\ X{\isacharunderscore}{\kern0pt}m{\isacharunderscore}{\kern0pt}subobject\ comp{\isacharunderscore}{\kern0pt}associative{\isadigit{2}}\ false{\isacharunderscore}{\kern0pt}func{\isacharunderscore}{\kern0pt}type\ subobject{\isacharunderscore}{\kern0pt}of{\isacharunderscore}{\kern0pt}def{\isadigit{2}}\isanewline
\ \ \ \ \ \ \ \ \ \ \ \ \ \ terminal{\isacharunderscore}{\kern0pt}func{\isacharunderscore}{\kern0pt}type\ true{\isacharunderscore}{\kern0pt}func{\isacharunderscore}{\kern0pt}type\ x{\isacharunderscore}{\kern0pt}in{\isacharunderscore}{\kern0pt}X{\isacharparenright}{\kern0pt}\isanewline
\ \ \ \ \ \ \ \ \isacommand{then}\isamarkupfalse%
\ \isacommand{have}\isamarkupfalse%
\ {\isachardoublequoteopen}{\isasymt}\ {\isacharequal}{\kern0pt}\ {\isasymf}{\isachardoublequoteclose}\isanewline
\ \ \ \ \ \ \ \ \ \ \isacommand{by}\isamarkupfalse%
\ {\isacharparenleft}{\kern0pt}smt\ X{\isacharunderscore}{\kern0pt}m{\isacharunderscore}{\kern0pt}subobject\ cfunc{\isacharunderscore}{\kern0pt}type{\isacharunderscore}{\kern0pt}def\ comp{\isacharunderscore}{\kern0pt}type\ false{\isacharunderscore}{\kern0pt}func{\isacharunderscore}{\kern0pt}type\ id{\isacharunderscore}{\kern0pt}right{\isacharunderscore}{\kern0pt}unit\ id{\isacharunderscore}{\kern0pt}type\isanewline
\ \ \ \ \ \ \ \ \ \ \ \ \ \ subobject{\isacharunderscore}{\kern0pt}of{\isacharunderscore}{\kern0pt}def{\isadigit{2}}\ terminal{\isacharunderscore}{\kern0pt}func{\isacharunderscore}{\kern0pt}unique\ true{\isacharunderscore}{\kern0pt}func{\isacharunderscore}{\kern0pt}type\ x{\isacharunderscore}{\kern0pt}in{\isacharunderscore}{\kern0pt}X{\isacharparenright}{\kern0pt}\isanewline
\ \ \ \ \ \ \ \ \isacommand{then}\isamarkupfalse%
\ \isacommand{show}\isamarkupfalse%
\ False\isanewline
\ \ \ \ \ \ \ \ \ \ \isacommand{using}\isamarkupfalse%
\ true{\isacharunderscore}{\kern0pt}false{\isacharunderscore}{\kern0pt}distinct\ \isacommand{by}\isamarkupfalse%
\ auto\isanewline
\ \ \ \ \ \ \isacommand{qed}\isamarkupfalse%
\isanewline
\ \ \ \ \ \ \isacommand{then}\isamarkupfalse%
\ \isacommand{show}\isamarkupfalse%
\ {\isachardoublequoteopen}X\ {\isasymcong}\ {\isasymemptyset}{\isachardoublequoteclose}\isanewline
\ \ \ \ \ \ \ \ \isacommand{using}\isamarkupfalse%
\ is{\isacharunderscore}{\kern0pt}empty{\isacharunderscore}{\kern0pt}def\ {\isacartoucheopen}{\isasymnexists}x{\isachardot}{\kern0pt}\ x\ {\isasymin}\isactrlsub c\ X{\isacartoucheclose}\ no{\isacharunderscore}{\kern0pt}el{\isacharunderscore}{\kern0pt}iff{\isacharunderscore}{\kern0pt}iso{\isacharunderscore}{\kern0pt}empty\ \isacommand{by}\isamarkupfalse%
\ blast\isanewline
\ \ \ \ \isacommand{qed}\isamarkupfalse%
\isanewline
\isanewline
\ \ \ \ \isacommand{show}\isamarkupfalse%
\ {\isachardoublequoteopen}X\ {\isasymcong}\ {\isasymone}\ {\isasymor}\ X\ {\isasymcong}\ {\isasymemptyset}{\isachardoublequoteclose}\isanewline
\ \ \ \ \ \ \isacommand{using}\isamarkupfalse%
\ {\isasymchi}{\isacharunderscore}{\kern0pt}true{\isacharunderscore}{\kern0pt}or{\isacharunderscore}{\kern0pt}false\ false{\isacharunderscore}{\kern0pt}iso{\isacharunderscore}{\kern0pt}one\ true{\isacharunderscore}{\kern0pt}iso{\isacharunderscore}{\kern0pt}one\ \isacommand{by}\isamarkupfalse%
\ blast\isanewline
\ \ \isacommand{qed}\isamarkupfalse%
\isanewline
\isanewline
\ \ \isacommand{have}\isamarkupfalse%
\ classes{\isacharunderscore}{\kern0pt}distinct{\isacharcolon}{\kern0pt}\ {\isachardoublequoteopen}{\isacharbraceleft}{\kern0pt}{\isacharparenleft}{\kern0pt}X{\isacharcomma}{\kern0pt}\ m{\isacharparenright}{\kern0pt}{\isachardot}{\kern0pt}\ X\ {\isasymcong}\ {\isasymemptyset}{\isacharbraceright}{\kern0pt}\ {\isasymnoteq}\ {\isacharbraceleft}{\kern0pt}{\isacharparenleft}{\kern0pt}X{\isacharcomma}{\kern0pt}\ m{\isacharparenright}{\kern0pt}{\isachardot}{\kern0pt}\ X\ {\isasymcong}\ {\isasymone}{\isacharbraceright}{\kern0pt}{\isachardoublequoteclose}\isanewline
\ \ \ \ \isacommand{by}\isamarkupfalse%
\ {\isacharparenleft}{\kern0pt}metis\ case{\isacharunderscore}{\kern0pt}prod{\isacharunderscore}{\kern0pt}eta\ curry{\isacharunderscore}{\kern0pt}case{\isacharunderscore}{\kern0pt}prod\ emptyset{\isacharunderscore}{\kern0pt}is{\isacharunderscore}{\kern0pt}empty\ id{\isacharunderscore}{\kern0pt}isomorphism\ id{\isacharunderscore}{\kern0pt}type\ is{\isacharunderscore}{\kern0pt}isomorphic{\isacharunderscore}{\kern0pt}def\ mem{\isacharunderscore}{\kern0pt}Collect{\isacharunderscore}{\kern0pt}eq{\isacharparenright}{\kern0pt}\isanewline
\isanewline
\ \ \isacommand{have}\isamarkupfalse%
\ {\isachardoublequoteopen}{\isacharbraceleft}{\kern0pt}{\isacharparenleft}{\kern0pt}X{\isacharcomma}{\kern0pt}\ m{\isacharparenright}{\kern0pt}{\isachardot}{\kern0pt}\ {\isacharparenleft}{\kern0pt}X{\isacharcomma}{\kern0pt}\ m{\isacharparenright}{\kern0pt}\ {\isasymsubseteq}\isactrlsub c\ {\isasymone}{\isacharbraceright}{\kern0pt}\ {\isacharslash}{\kern0pt}{\isacharslash}{\kern0pt}\ {\isacharbraceleft}{\kern0pt}{\isacharparenleft}{\kern0pt}{\isacharparenleft}{\kern0pt}X{\isadigit{1}}{\isacharcomma}{\kern0pt}\ m{\isadigit{1}}{\isacharparenright}{\kern0pt}{\isacharcomma}{\kern0pt}\ {\isacharparenleft}{\kern0pt}X{\isadigit{2}}{\isacharcomma}{\kern0pt}\ m{\isadigit{2}}{\isacharparenright}{\kern0pt}{\isacharparenright}{\kern0pt}{\isachardot}{\kern0pt}\ X{\isadigit{1}}\ {\isasymcong}\ X{\isadigit{2}}{\isacharbraceright}{\kern0pt}\ {\isacharequal}{\kern0pt}\ {\isacharbraceleft}{\kern0pt}{\isacharbraceleft}{\kern0pt}{\isacharparenleft}{\kern0pt}X{\isacharcomma}{\kern0pt}\ m{\isacharparenright}{\kern0pt}{\isachardot}{\kern0pt}\ X\ {\isasymcong}\ {\isasymemptyset}{\isacharbraceright}{\kern0pt}{\isacharcomma}{\kern0pt}\ {\isacharbraceleft}{\kern0pt}{\isacharparenleft}{\kern0pt}X{\isacharcomma}{\kern0pt}\ m{\isacharparenright}{\kern0pt}{\isachardot}{\kern0pt}\ X\ {\isasymcong}\ {\isasymone}{\isacharbraceright}{\kern0pt}{\isacharbraceright}{\kern0pt}{\isachardoublequoteclose}\isanewline
\ \ \isacommand{proof}\isamarkupfalse%
\isanewline
\ \ \ \ \isacommand{show}\isamarkupfalse%
\ {\isachardoublequoteopen}{\isacharbraceleft}{\kern0pt}{\isacharparenleft}{\kern0pt}X{\isacharcomma}{\kern0pt}\ m{\isacharparenright}{\kern0pt}{\isachardot}{\kern0pt}\ {\isacharparenleft}{\kern0pt}X{\isacharcomma}{\kern0pt}\ m{\isacharparenright}{\kern0pt}\ {\isasymsubseteq}\isactrlsub c\ {\isasymone}{\isacharbraceright}{\kern0pt}\ {\isacharslash}{\kern0pt}{\isacharslash}{\kern0pt}\ {\isacharbraceleft}{\kern0pt}{\isacharparenleft}{\kern0pt}{\isacharparenleft}{\kern0pt}X{\isadigit{1}}{\isacharcomma}{\kern0pt}\ m{\isadigit{1}}{\isacharparenright}{\kern0pt}{\isacharcomma}{\kern0pt}\ {\isacharparenleft}{\kern0pt}X{\isadigit{2}}{\isacharcomma}{\kern0pt}\ m{\isadigit{2}}{\isacharparenright}{\kern0pt}{\isacharparenright}{\kern0pt}{\isachardot}{\kern0pt}\ X{\isadigit{1}}\ {\isasymcong}\ X{\isadigit{2}}{\isacharbraceright}{\kern0pt}\ {\isasymsubseteq}\ {\isacharbraceleft}{\kern0pt}{\isacharbraceleft}{\kern0pt}{\isacharparenleft}{\kern0pt}X{\isacharcomma}{\kern0pt}\ m{\isacharparenright}{\kern0pt}{\isachardot}{\kern0pt}\ X\ {\isasymcong}\ {\isasymemptyset}{\isacharbraceright}{\kern0pt}{\isacharcomma}{\kern0pt}\ {\isacharbraceleft}{\kern0pt}{\isacharparenleft}{\kern0pt}X{\isacharcomma}{\kern0pt}\ m{\isacharparenright}{\kern0pt}{\isachardot}{\kern0pt}\ X\ {\isasymcong}\ {\isasymone}{\isacharbraceright}{\kern0pt}{\isacharbraceright}{\kern0pt}{\isachardoublequoteclose}\isanewline
\ \ \ \ \ \ \isacommand{by}\isamarkupfalse%
\ {\isacharparenleft}{\kern0pt}unfold\ quotient{\isacharunderscore}{\kern0pt}def{\isacharcomma}{\kern0pt}\ auto{\isacharcomma}{\kern0pt}\ insert\ isomorphic{\isacharunderscore}{\kern0pt}is{\isacharunderscore}{\kern0pt}symmetric\ isomorphic{\isacharunderscore}{\kern0pt}is{\isacharunderscore}{\kern0pt}transitive\ subobject{\isacharunderscore}{\kern0pt}one{\isacharunderscore}{\kern0pt}or{\isacharunderscore}{\kern0pt}empty{\isacharcomma}{\kern0pt}\ blast{\isacharplus}{\kern0pt}{\isacharparenright}{\kern0pt}\isanewline
\ \ \isacommand{next}\isamarkupfalse%
\isanewline
\ \ \ \ \isacommand{show}\isamarkupfalse%
\ {\isachardoublequoteopen}{\isacharbraceleft}{\kern0pt}{\isacharbraceleft}{\kern0pt}{\isacharparenleft}{\kern0pt}X{\isacharcomma}{\kern0pt}\ m{\isacharparenright}{\kern0pt}{\isachardot}{\kern0pt}\ X\ {\isasymcong}\ {\isasymemptyset}{\isacharbraceright}{\kern0pt}{\isacharcomma}{\kern0pt}\ {\isacharbraceleft}{\kern0pt}{\isacharparenleft}{\kern0pt}X{\isacharcomma}{\kern0pt}\ m{\isacharparenright}{\kern0pt}{\isachardot}{\kern0pt}\ X\ {\isasymcong}\ {\isasymone}{\isacharbraceright}{\kern0pt}{\isacharbraceright}{\kern0pt}\ {\isasymsubseteq}\ {\isacharbraceleft}{\kern0pt}{\isacharparenleft}{\kern0pt}X{\isacharcomma}{\kern0pt}\ m{\isacharparenright}{\kern0pt}{\isachardot}{\kern0pt}\ {\isacharparenleft}{\kern0pt}X{\isacharcomma}{\kern0pt}\ m{\isacharparenright}{\kern0pt}\ {\isasymsubseteq}\isactrlsub c\ {\isasymone}{\isacharbraceright}{\kern0pt}\ {\isacharslash}{\kern0pt}{\isacharslash}{\kern0pt}\ {\isacharbraceleft}{\kern0pt}{\isacharparenleft}{\kern0pt}{\isacharparenleft}{\kern0pt}X{\isadigit{1}}{\isacharcomma}{\kern0pt}\ m{\isadigit{1}}{\isacharparenright}{\kern0pt}{\isacharcomma}{\kern0pt}\ X{\isadigit{2}}{\isacharcomma}{\kern0pt}\ m{\isadigit{2}}{\isacharparenright}{\kern0pt}{\isachardot}{\kern0pt}\ X{\isadigit{1}}\ {\isasymcong}\ X{\isadigit{2}}{\isacharbraceright}{\kern0pt}{\isachardoublequoteclose}\isanewline
\ \ \ \ \ \ \isacommand{by}\isamarkupfalse%
\ {\isacharparenleft}{\kern0pt}unfold\ quotient{\isacharunderscore}{\kern0pt}def{\isacharcomma}{\kern0pt}\ insert\ empty{\isacharunderscore}{\kern0pt}subobject\ one{\isacharunderscore}{\kern0pt}subobject{\isacharcomma}{\kern0pt}\ auto\ simp\ add{\isacharcolon}{\kern0pt}\ isomorphic{\isacharunderscore}{\kern0pt}is{\isacharunderscore}{\kern0pt}symmetric{\isacharparenright}{\kern0pt}\isanewline
\ \ \isacommand{qed}\isamarkupfalse%
\isanewline
\ \ \isacommand{then}\isamarkupfalse%
\ \isacommand{show}\isamarkupfalse%
\ {\isachardoublequoteopen}card\ {\isacharparenleft}{\kern0pt}{\isacharbraceleft}{\kern0pt}{\isacharparenleft}{\kern0pt}X{\isacharcomma}{\kern0pt}\ m{\isacharparenright}{\kern0pt}{\isachardot}{\kern0pt}\ {\isacharparenleft}{\kern0pt}X{\isacharcomma}{\kern0pt}\ m{\isacharparenright}{\kern0pt}\ {\isasymsubseteq}\isactrlsub c\ {\isasymone}{\isacharbraceright}{\kern0pt}\ {\isacharslash}{\kern0pt}{\isacharslash}{\kern0pt}\ {\isacharbraceleft}{\kern0pt}{\isacharparenleft}{\kern0pt}{\isacharparenleft}{\kern0pt}X{\isacharcomma}{\kern0pt}\ m{\isadigit{1}}{\isacharparenright}{\kern0pt}{\isacharcomma}{\kern0pt}\ {\isacharparenleft}{\kern0pt}Y{\isacharcomma}{\kern0pt}\ m{\isadigit{2}}{\isacharparenright}{\kern0pt}{\isacharparenright}{\kern0pt}{\isachardot}{\kern0pt}\ X\ {\isasymcong}\ Y{\isacharbraceright}{\kern0pt}{\isacharparenright}{\kern0pt}\ {\isacharequal}{\kern0pt}\ {\isadigit{2}}{\isachardoublequoteclose}\isanewline
\ \ \ \ \isacommand{by}\isamarkupfalse%
\ {\isacharparenleft}{\kern0pt}simp\ add{\isacharcolon}{\kern0pt}\ classes{\isacharunderscore}{\kern0pt}distinct{\isacharparenright}{\kern0pt}\isanewline
\isacommand{qed}\isamarkupfalse%
%
\endisatagproof
{\isafoldproof}%
%
\isadelimproof
\isanewline
%
\endisadelimproof
\isanewline
\isacommand{lemma}\isamarkupfalse%
\ coprod{\isacharunderscore}{\kern0pt}with{\isacharunderscore}{\kern0pt}init{\isacharunderscore}{\kern0pt}obj{\isadigit{1}}{\isacharcolon}{\kern0pt}\ \isanewline
\ \ \isakeyword{assumes}\ {\isachardoublequoteopen}initial{\isacharunderscore}{\kern0pt}object\ Y{\isachardoublequoteclose}\isanewline
\ \ \isakeyword{shows}\ {\isachardoublequoteopen}X\ {\isasymCoprod}\ Y\ {\isasymcong}\ X{\isachardoublequoteclose}\isanewline
%
\isadelimproof
\ \ %
\endisadelimproof
%
\isatagproof
\isacommand{by}\isamarkupfalse%
\ {\isacharparenleft}{\kern0pt}meson\ assms\ coprod{\isacharunderscore}{\kern0pt}pres{\isacharunderscore}{\kern0pt}iso\ coproduct{\isacharunderscore}{\kern0pt}with{\isacharunderscore}{\kern0pt}empty\ initial{\isacharunderscore}{\kern0pt}iso{\isacharunderscore}{\kern0pt}empty\ isomorphic{\isacharunderscore}{\kern0pt}is{\isacharunderscore}{\kern0pt}reflexive\ isomorphic{\isacharunderscore}{\kern0pt}is{\isacharunderscore}{\kern0pt}transitive{\isacharparenright}{\kern0pt}%
\endisatagproof
{\isafoldproof}%
%
\isadelimproof
\isanewline
%
\endisadelimproof
\isanewline
\isacommand{lemma}\isamarkupfalse%
\ coprod{\isacharunderscore}{\kern0pt}with{\isacharunderscore}{\kern0pt}init{\isacharunderscore}{\kern0pt}obj{\isadigit{2}}{\isacharcolon}{\kern0pt}\ \isanewline
\ \ \isakeyword{assumes}\ {\isachardoublequoteopen}initial{\isacharunderscore}{\kern0pt}object\ X{\isachardoublequoteclose}\isanewline
\ \ \isakeyword{shows}\ {\isachardoublequoteopen}X\ {\isasymCoprod}\ Y\ {\isasymcong}\ Y{\isachardoublequoteclose}\isanewline
%
\isadelimproof
\ \ %
\endisadelimproof
%
\isatagproof
\isacommand{using}\isamarkupfalse%
\ assms\ coprod{\isacharunderscore}{\kern0pt}with{\isacharunderscore}{\kern0pt}init{\isacharunderscore}{\kern0pt}obj{\isadigit{1}}\ coproduct{\isacharunderscore}{\kern0pt}commutes\ isomorphic{\isacharunderscore}{\kern0pt}is{\isacharunderscore}{\kern0pt}transitive\ \isacommand{by}\isamarkupfalse%
\ blast%
\endisatagproof
{\isafoldproof}%
%
\isadelimproof
\isanewline
%
\endisadelimproof
\isanewline
\isacommand{lemma}\isamarkupfalse%
\ prod{\isacharunderscore}{\kern0pt}with{\isacharunderscore}{\kern0pt}term{\isacharunderscore}{\kern0pt}obj{\isadigit{1}}{\isacharcolon}{\kern0pt}\isanewline
\ \ \isakeyword{assumes}\ {\isachardoublequoteopen}terminal{\isacharunderscore}{\kern0pt}object{\isacharparenleft}{\kern0pt}X{\isacharparenright}{\kern0pt}{\isachardoublequoteclose}\ \isanewline
\ \ \isakeyword{shows}\ \ {\isachardoublequoteopen}X\ {\isasymtimes}\isactrlsub c\ Y\ {\isasymcong}\ Y{\isachardoublequoteclose}\ \isanewline
%
\isadelimproof
\ \ %
\endisadelimproof
%
\isatagproof
\isacommand{by}\isamarkupfalse%
\ {\isacharparenleft}{\kern0pt}meson\ assms\ isomorphic{\isacharunderscore}{\kern0pt}is{\isacharunderscore}{\kern0pt}reflexive\ isomorphic{\isacharunderscore}{\kern0pt}is{\isacharunderscore}{\kern0pt}transitive\ one{\isacharunderscore}{\kern0pt}terminal{\isacharunderscore}{\kern0pt}object\ one{\isacharunderscore}{\kern0pt}x{\isacharunderscore}{\kern0pt}A{\isacharunderscore}{\kern0pt}iso{\isacharunderscore}{\kern0pt}A\ prod{\isacharunderscore}{\kern0pt}pres{\isacharunderscore}{\kern0pt}iso\ terminal{\isacharunderscore}{\kern0pt}objects{\isacharunderscore}{\kern0pt}isomorphic{\isacharparenright}{\kern0pt}%
\endisatagproof
{\isafoldproof}%
%
\isadelimproof
\isanewline
%
\endisadelimproof
\isanewline
\isacommand{lemma}\isamarkupfalse%
\ prod{\isacharunderscore}{\kern0pt}with{\isacharunderscore}{\kern0pt}term{\isacharunderscore}{\kern0pt}obj{\isadigit{2}}{\isacharcolon}{\kern0pt}\isanewline
\ \ \isakeyword{assumes}\ {\isachardoublequoteopen}terminal{\isacharunderscore}{\kern0pt}object{\isacharparenleft}{\kern0pt}Y{\isacharparenright}{\kern0pt}{\isachardoublequoteclose}\ \isanewline
\ \ \isakeyword{shows}\ \ {\isachardoublequoteopen}X\ {\isasymtimes}\isactrlsub c\ Y\ {\isasymcong}\ X{\isachardoublequoteclose}\isanewline
%
\isadelimproof
\ \ %
\endisadelimproof
%
\isatagproof
\isacommand{by}\isamarkupfalse%
\ {\isacharparenleft}{\kern0pt}meson\ assms\ isomorphic{\isacharunderscore}{\kern0pt}is{\isacharunderscore}{\kern0pt}transitive\ prod{\isacharunderscore}{\kern0pt}with{\isacharunderscore}{\kern0pt}term{\isacharunderscore}{\kern0pt}obj{\isadigit{1}}\ product{\isacharunderscore}{\kern0pt}commutes{\isacharparenright}{\kern0pt}%
\endisatagproof
{\isafoldproof}%
%
\isadelimproof
\isanewline
%
\endisadelimproof
%
\isadelimtheory
\isanewline
%
\endisadelimtheory
%
\isatagtheory
\isacommand{end}\isamarkupfalse%
%
\endisatagtheory
{\isafoldtheory}%
%
\isadelimtheory
%
\endisadelimtheory
%
\end{isabellebody}%
\endinput
%:%file=~/ETCS/HOL-ETCS/Initial.thy%:%
%:%11=1%:%
%:%27=3%:%
%:%28=3%:%
%:%29=4%:%
%:%30=5%:%
%:%39=7%:%
%:%41=8%:%
%:%42=8%:%
%:%43=9%:%
%:%44=10%:%
%:%45=11%:%
%:%46=12%:%
%:%47=13%:%
%:%48=14%:%
%:%49=15%:%
%:%50=16%:%
%:%51=16%:%
%:%52=17%:%
%:%53=18%:%
%:%54=19%:%
%:%55=19%:%
%:%56=20%:%
%:%59=21%:%
%:%63=21%:%
%:%64=21%:%
%:%65=21%:%
%:%70=21%:%
%:%73=22%:%
%:%74=23%:%
%:%75=23%:%
%:%76=24%:%
%:%77=25%:%
%:%80=26%:%
%:%84=26%:%
%:%85=26%:%
%:%94=28%:%
%:%96=29%:%
%:%97=29%:%
%:%98=30%:%
%:%105=31%:%
%:%106=31%:%
%:%107=32%:%
%:%108=32%:%
%:%109=33%:%
%:%110=33%:%
%:%111=34%:%
%:%112=34%:%
%:%113=35%:%
%:%114=36%:%
%:%115=36%:%
%:%116=37%:%
%:%117=37%:%
%:%118=37%:%
%:%119=38%:%
%:%120=38%:%
%:%121=39%:%
%:%122=39%:%
%:%123=39%:%
%:%124=40%:%
%:%125=40%:%
%:%126=41%:%
%:%127=41%:%
%:%128=41%:%
%:%129=41%:%
%:%130=41%:%
%:%131=42%:%
%:%132=42%:%
%:%133=43%:%
%:%134=43%:%
%:%135=44%:%
%:%136=44%:%
%:%137=45%:%
%:%138=45%:%
%:%139=46%:%
%:%140=47%:%
%:%141=47%:%
%:%142=48%:%
%:%143=48%:%
%:%144=48%:%
%:%145=49%:%
%:%146=49%:%
%:%147=50%:%
%:%148=50%:%
%:%149=50%:%
%:%150=51%:%
%:%151=51%:%
%:%152=52%:%
%:%153=52%:%
%:%154=52%:%
%:%155=52%:%
%:%156=52%:%
%:%157=53%:%
%:%158=53%:%
%:%159=54%:%
%:%160=54%:%
%:%161=54%:%
%:%162=55%:%
%:%163=55%:%
%:%164=55%:%
%:%165=56%:%
%:%166=56%:%
%:%167=56%:%
%:%168=57%:%
%:%169=57%:%
%:%170=57%:%
%:%171=58%:%
%:%172=58%:%
%:%173=58%:%
%:%174=59%:%
%:%175=59%:%
%:%176=59%:%
%:%177=60%:%
%:%178=60%:%
%:%179=60%:%
%:%180=61%:%
%:%181=61%:%
%:%182=61%:%
%:%183=62%:%
%:%184=62%:%
%:%185=62%:%
%:%186=63%:%
%:%187=63%:%
%:%188=64%:%
%:%189=64%:%
%:%190=65%:%
%:%191=65%:%
%:%192=65%:%
%:%193=66%:%
%:%194=66%:%
%:%195=66%:%
%:%196=67%:%
%:%206=69%:%
%:%208=70%:%
%:%209=70%:%
%:%210=71%:%
%:%211=72%:%
%:%214=73%:%
%:%218=73%:%
%:%219=73%:%
%:%224=73%:%
%:%227=74%:%
%:%228=75%:%
%:%229=75%:%
%:%230=76%:%
%:%233=77%:%
%:%237=77%:%
%:%238=77%:%
%:%239=77%:%
%:%244=77%:%
%:%247=78%:%
%:%248=79%:%
%:%249=79%:%
%:%250=80%:%
%:%253=81%:%
%:%257=81%:%
%:%258=81%:%
%:%259=81%:%
%:%268=83%:%
%:%270=84%:%
%:%271=84%:%
%:%272=85%:%
%:%279=86%:%
%:%280=86%:%
%:%281=87%:%
%:%282=87%:%
%:%283=88%:%
%:%284=88%:%
%:%285=89%:%
%:%286=89%:%
%:%287=90%:%
%:%288=90%:%
%:%289=91%:%
%:%290=91%:%
%:%291=92%:%
%:%292=92%:%
%:%293=92%:%
%:%294=93%:%
%:%295=94%:%
%:%296=94%:%
%:%297=95%:%
%:%298=95%:%
%:%299=95%:%
%:%300=96%:%
%:%301=96%:%
%:%302=96%:%
%:%303=97%:%
%:%304=97%:%
%:%305=97%:%
%:%306=98%:%
%:%307=98%:%
%:%308=99%:%
%:%309=99%:%
%:%310=99%:%
%:%311=100%:%
%:%312=100%:%
%:%313=100%:%
%:%314=101%:%
%:%315=101%:%
%:%316=101%:%
%:%317=102%:%
%:%318=102%:%
%:%319=102%:%
%:%320=103%:%
%:%321=103%:%
%:%322=103%:%
%:%323=104%:%
%:%324=104%:%
%:%325=104%:%
%:%326=105%:%
%:%327=105%:%
%:%328=105%:%
%:%329=106%:%
%:%335=106%:%
%:%338=107%:%
%:%339=108%:%
%:%340=108%:%
%:%341=109%:%
%:%342=110%:%
%:%343=111%:%
%:%346=112%:%
%:%350=112%:%
%:%351=112%:%
%:%356=112%:%
%:%359=113%:%
%:%360=114%:%
%:%361=114%:%
%:%362=115%:%
%:%363=116%:%
%:%366=117%:%
%:%370=117%:%
%:%371=117%:%
%:%372=118%:%
%:%373=118%:%
%:%378=118%:%
%:%381=119%:%
%:%382=120%:%
%:%383=120%:%
%:%384=121%:%
%:%385=122%:%
%:%386=123%:%
%:%389=124%:%
%:%393=124%:%
%:%394=124%:%
%:%399=124%:%
%:%402=125%:%
%:%403=126%:%
%:%404=126%:%
%:%405=127%:%
%:%406=128%:%
%:%407=129%:%
%:%410=130%:%
%:%414=130%:%
%:%415=130%:%
%:%420=130%:%
%:%423=131%:%
%:%424=132%:%
%:%425=132%:%
%:%426=133%:%
%:%427=134%:%
%:%428=135%:%
%:%431=136%:%
%:%435=136%:%
%:%436=136%:%
%:%441=136%:%
%:%444=137%:%
%:%445=138%:%
%:%446=138%:%
%:%449=139%:%
%:%453=139%:%
%:%454=139%:%
%:%463=141%:%
%:%465=142%:%
%:%466=142%:%
%:%467=143%:%
%:%474=144%:%
%:%475=144%:%
%:%476=145%:%
%:%477=145%:%
%:%478=146%:%
%:%479=146%:%
%:%480=146%:%
%:%481=147%:%
%:%482=147%:%
%:%483=148%:%
%:%484=148%:%
%:%485=149%:%
%:%486=150%:%
%:%487=150%:%
%:%488=151%:%
%:%489=151%:%
%:%490=152%:%
%:%491=152%:%
%:%492=153%:%
%:%493=153%:%
%:%494=154%:%
%:%495=155%:%
%:%496=155%:%
%:%497=156%:%
%:%498=156%:%
%:%499=156%:%
%:%500=157%:%
%:%501=157%:%
%:%502=157%:%
%:%503=158%:%
%:%504=158%:%
%:%505=158%:%
%:%506=158%:%
%:%507=159%:%
%:%508=160%:%
%:%509=160%:%
%:%510=161%:%
%:%511=161%:%
%:%512=162%:%
%:%513=162%:%
%:%514=163%:%
%:%515=163%:%
%:%516=163%:%
%:%517=164%:%
%:%518=164%:%
%:%519=164%:%
%:%520=165%:%
%:%521=165%:%
%:%522=165%:%
%:%523=166%:%
%:%524=166%:%
%:%525=167%:%
%:%526=167%:%
%:%527=168%:%
%:%528=168%:%
%:%529=169%:%
%:%530=170%:%
%:%531=170%:%
%:%532=171%:%
%:%533=171%:%
%:%534=172%:%
%:%535=172%:%
%:%536=173%:%
%:%537=173%:%
%:%538=174%:%
%:%539=174%:%
%:%540=175%:%
%:%541=175%:%
%:%542=176%:%
%:%543=176%:%
%:%544=177%:%
%:%545=177%:%
%:%546=178%:%
%:%547=178%:%
%:%548=178%:%
%:%549=179%:%
%:%550=179%:%
%:%551=179%:%
%:%552=180%:%
%:%553=180%:%
%:%554=181%:%
%:%555=182%:%
%:%556=182%:%
%:%557=182%:%
%:%558=183%:%
%:%559=183%:%
%:%560=184%:%
%:%561=185%:%
%:%562=185%:%
%:%563=185%:%
%:%564=186%:%
%:%565=186%:%
%:%566=186%:%
%:%567=187%:%
%:%568=187%:%
%:%569=188%:%
%:%570=188%:%
%:%571=188%:%
%:%572=189%:%
%:%573=189%:%
%:%574=189%:%
%:%575=190%:%
%:%576=190%:%
%:%577=191%:%
%:%578=192%:%
%:%579=192%:%
%:%580=193%:%
%:%581=193%:%
%:%582=193%:%
%:%583=194%:%
%:%584=194%:%
%:%585=195%:%
%:%586=196%:%
%:%587=196%:%
%:%588=197%:%
%:%589=197%:%
%:%590=198%:%
%:%591=199%:%
%:%592=199%:%
%:%593=200%:%
%:%594=200%:%
%:%595=201%:%
%:%596=201%:%
%:%597=202%:%
%:%598=202%:%
%:%599=203%:%
%:%600=203%:%
%:%601=204%:%
%:%602=204%:%
%:%603=205%:%
%:%604=205%:%
%:%605=206%:%
%:%606=206%:%
%:%607=207%:%
%:%608=207%:%
%:%609=207%:%
%:%610=208%:%
%:%611=208%:%
%:%612=209%:%
%:%618=209%:%
%:%621=210%:%
%:%622=211%:%
%:%623=211%:%
%:%624=212%:%
%:%625=213%:%
%:%628=214%:%
%:%632=214%:%
%:%633=214%:%
%:%638=214%:%
%:%641=215%:%
%:%642=216%:%
%:%643=216%:%
%:%644=217%:%
%:%645=218%:%
%:%648=219%:%
%:%652=219%:%
%:%653=219%:%
%:%654=219%:%
%:%659=219%:%
%:%662=220%:%
%:%663=221%:%
%:%664=221%:%
%:%665=222%:%
%:%666=223%:%
%:%669=224%:%
%:%673=224%:%
%:%674=224%:%
%:%679=224%:%
%:%682=225%:%
%:%683=226%:%
%:%684=226%:%
%:%685=227%:%
%:%686=228%:%
%:%689=229%:%
%:%693=229%:%
%:%694=229%:%
%:%699=229%:%
%:%704=230%:%
%:%709=231%:%

%
\begin{isabellebody}%
\setisabellecontext{Exponential{\isacharunderscore}{\kern0pt}Objects}%
%
\isadelimdocument
%
\endisadelimdocument
%
\isatagdocument
%
\isamarkupsection{Exponential Objects, Transposes and Evaluation%
}
\isamarkuptrue%
%
\endisatagdocument
{\isafolddocument}%
%
\isadelimdocument
%
\endisadelimdocument
%
\isadelimtheory
%
\endisadelimtheory
%
\isatagtheory
\isacommand{theory}\isamarkupfalse%
\ Exponential{\isacharunderscore}{\kern0pt}Objects\isanewline
\ \ \isakeyword{imports}\ Initial\isanewline
\isakeyword{begin}%
\endisatagtheory
{\isafoldtheory}%
%
\isadelimtheory
%
\endisadelimtheory
%
\begin{isamarkuptext}%
The axiomatization below corresponds to Axiom 9 (Exponential Objects) in Halvorson.%
\end{isamarkuptext}\isamarkuptrue%
\isacommand{axiomatization}\isamarkupfalse%
\isanewline
\ \ exp{\isacharunderscore}{\kern0pt}set\ {\isacharcolon}{\kern0pt}{\isacharcolon}{\kern0pt}\ {\isachardoublequoteopen}cset\ {\isasymRightarrow}\ cset\ {\isasymRightarrow}\ cset{\isachardoublequoteclose}\ {\isacharparenleft}{\kern0pt}{\isachardoublequoteopen}{\isacharunderscore}{\kern0pt}\isactrlbsup {\isacharunderscore}{\kern0pt}\isactrlesup {\isachardoublequoteclose}\ {\isacharbrackleft}{\kern0pt}{\isadigit{1}}{\isadigit{0}}{\isadigit{0}}{\isacharcomma}{\kern0pt}{\isadigit{1}}{\isadigit{0}}{\isadigit{0}}{\isacharbrackright}{\kern0pt}{\isadigit{1}}{\isadigit{0}}{\isadigit{0}}{\isacharparenright}{\kern0pt}\ \isakeyword{and}\isanewline
\ \ eval{\isacharunderscore}{\kern0pt}func\ \ {\isacharcolon}{\kern0pt}{\isacharcolon}{\kern0pt}\ {\isachardoublequoteopen}cset\ {\isasymRightarrow}\ cset\ {\isasymRightarrow}\ cfunc{\isachardoublequoteclose}\ \isakeyword{and}\isanewline
\ \ transpose{\isacharunderscore}{\kern0pt}func\ {\isacharcolon}{\kern0pt}{\isacharcolon}{\kern0pt}\ {\isachardoublequoteopen}cfunc\ {\isasymRightarrow}\ cfunc{\isachardoublequoteclose}\ {\isacharparenleft}{\kern0pt}{\isachardoublequoteopen}{\isacharunderscore}{\kern0pt}\isactrlsup {\isasymsharp}{\isachardoublequoteclose}\ {\isacharbrackleft}{\kern0pt}{\isadigit{1}}{\isadigit{0}}{\isadigit{0}}{\isacharbrackright}{\kern0pt}{\isadigit{1}}{\isadigit{0}}{\isadigit{0}}{\isacharparenright}{\kern0pt}\isanewline
\isakeyword{where}\isanewline
\ \ exp{\isacharunderscore}{\kern0pt}set{\isacharunderscore}{\kern0pt}inj{\isacharcolon}{\kern0pt}\ {\isachardoublequoteopen}X\isactrlbsup A\isactrlesup \ {\isacharequal}{\kern0pt}\ Y\isactrlbsup B\isactrlesup \ {\isasymLongrightarrow}\ X\ {\isacharequal}{\kern0pt}\ Y\ {\isasymand}\ A\ {\isacharequal}{\kern0pt}\ B{\isachardoublequoteclose}\ \isakeyword{and}\isanewline
\ \ eval{\isacharunderscore}{\kern0pt}func{\isacharunderscore}{\kern0pt}type{\isacharbrackleft}{\kern0pt}type{\isacharunderscore}{\kern0pt}rule{\isacharbrackright}{\kern0pt}{\isacharcolon}{\kern0pt}\ {\isachardoublequoteopen}eval{\isacharunderscore}{\kern0pt}func\ X\ A\ {\isacharcolon}{\kern0pt}\ A{\isasymtimes}\isactrlsub c\ X\isactrlbsup A\isactrlesup \ {\isasymrightarrow}\ X{\isachardoublequoteclose}\ \isakeyword{and}\isanewline
\ \ transpose{\isacharunderscore}{\kern0pt}func{\isacharunderscore}{\kern0pt}type{\isacharbrackleft}{\kern0pt}type{\isacharunderscore}{\kern0pt}rule{\isacharbrackright}{\kern0pt}{\isacharcolon}{\kern0pt}\ {\isachardoublequoteopen}f\ {\isacharcolon}{\kern0pt}\ A\ {\isasymtimes}\isactrlsub c\ Z\ {\isasymrightarrow}\ X\ {\isasymLongrightarrow}\ f\isactrlsup {\isasymsharp}\ {\isacharcolon}{\kern0pt}\ Z\ {\isasymrightarrow}\ X\isactrlbsup A\isactrlesup {\isachardoublequoteclose}\ \isakeyword{and}\isanewline
\ \ transpose{\isacharunderscore}{\kern0pt}func{\isacharunderscore}{\kern0pt}def{\isacharcolon}{\kern0pt}\ {\isachardoublequoteopen}f\ {\isacharcolon}{\kern0pt}\ A\ {\isasymtimes}\isactrlsub c\ Z\ {\isasymrightarrow}\ X\ {\isasymLongrightarrow}\ {\isacharparenleft}{\kern0pt}eval{\isacharunderscore}{\kern0pt}func\ X\ A{\isacharparenright}{\kern0pt}\ {\isasymcirc}\isactrlsub c\ {\isacharparenleft}{\kern0pt}id\ A\ {\isasymtimes}\isactrlsub f\ f\isactrlsup {\isasymsharp}{\isacharparenright}{\kern0pt}\ {\isacharequal}{\kern0pt}\ f{\isachardoublequoteclose}\ \isakeyword{and}\isanewline
\ \ transpose{\isacharunderscore}{\kern0pt}func{\isacharunderscore}{\kern0pt}unique{\isacharcolon}{\kern0pt}\ \isanewline
\ \ \ \ {\isachardoublequoteopen}f\ {\isacharcolon}{\kern0pt}\ A{\isasymtimes}\isactrlsub cZ\ {\isasymrightarrow}\ X\ {\isasymLongrightarrow}\ g{\isacharcolon}{\kern0pt}\ Z\ {\isasymrightarrow}\ X\isactrlbsup A\isactrlesup \ {\isasymLongrightarrow}\ {\isacharparenleft}{\kern0pt}eval{\isacharunderscore}{\kern0pt}func\ X\ A{\isacharparenright}{\kern0pt}\ {\isasymcirc}\isactrlsub c\ {\isacharparenleft}{\kern0pt}id\ A\ {\isasymtimes}\isactrlsub f\ g{\isacharparenright}{\kern0pt}\ {\isacharequal}{\kern0pt}\ f\ {\isasymLongrightarrow}\ g\ {\isacharequal}{\kern0pt}\ f\isactrlsup {\isasymsharp}{\isachardoublequoteclose}\isanewline
\isanewline
\isacommand{lemma}\isamarkupfalse%
\ eval{\isacharunderscore}{\kern0pt}func{\isacharunderscore}{\kern0pt}surj{\isacharcolon}{\kern0pt}\isanewline
\ \ \isakeyword{assumes}\ {\isachardoublequoteopen}nonempty{\isacharparenleft}{\kern0pt}A{\isacharparenright}{\kern0pt}{\isachardoublequoteclose}\isanewline
\ \ \isakeyword{shows}\ {\isachardoublequoteopen}surjective{\isacharparenleft}{\kern0pt}{\isacharparenleft}{\kern0pt}eval{\isacharunderscore}{\kern0pt}func\ X\ A{\isacharparenright}{\kern0pt}{\isacharparenright}{\kern0pt}{\isachardoublequoteclose}\isanewline
%
\isadelimproof
\ \ %
\endisadelimproof
%
\isatagproof
\isacommand{unfolding}\isamarkupfalse%
\ surjective{\isacharunderscore}{\kern0pt}def\isanewline
\isacommand{proof}\isamarkupfalse%
{\isacharparenleft}{\kern0pt}clarify{\isacharparenright}{\kern0pt}\isanewline
\ \ \isacommand{fix}\isamarkupfalse%
\ x\isanewline
\ \ \isacommand{assume}\isamarkupfalse%
\ x{\isacharunderscore}{\kern0pt}type{\isacharcolon}{\kern0pt}\ {\isachardoublequoteopen}x\ {\isasymin}\isactrlsub c\ codomain\ {\isacharparenleft}{\kern0pt}eval{\isacharunderscore}{\kern0pt}func\ X\ A{\isacharparenright}{\kern0pt}{\isachardoublequoteclose}\isanewline
\ \ \isacommand{then}\isamarkupfalse%
\ \isacommand{have}\isamarkupfalse%
\ x{\isacharunderscore}{\kern0pt}type{\isadigit{2}}{\isacharbrackleft}{\kern0pt}type{\isacharunderscore}{\kern0pt}rule{\isacharbrackright}{\kern0pt}{\isacharcolon}{\kern0pt}\ {\isachardoublequoteopen}x\ {\isasymin}\isactrlsub c\ X{\isachardoublequoteclose}\isanewline
\ \ \ \ \isacommand{using}\isamarkupfalse%
\ cfunc{\isacharunderscore}{\kern0pt}type{\isacharunderscore}{\kern0pt}def\ eval{\isacharunderscore}{\kern0pt}func{\isacharunderscore}{\kern0pt}type\ \isacommand{by}\isamarkupfalse%
\ auto\isanewline
\ \ \isacommand{obtain}\isamarkupfalse%
\ a\ \isakeyword{where}\ a{\isacharunderscore}{\kern0pt}def{\isacharbrackleft}{\kern0pt}type{\isacharunderscore}{\kern0pt}rule{\isacharbrackright}{\kern0pt}{\isacharcolon}{\kern0pt}\ {\isachardoublequoteopen}a\ {\isasymin}\isactrlsub c\ A{\isachardoublequoteclose}\isanewline
\ \ \ \ \isacommand{using}\isamarkupfalse%
\ assms\ nonempty{\isacharunderscore}{\kern0pt}def\ \isacommand{by}\isamarkupfalse%
\ auto\isanewline
\ \ \isacommand{have}\isamarkupfalse%
\ needed{\isacharunderscore}{\kern0pt}type{\isacharcolon}{\kern0pt}\ {\isachardoublequoteopen}{\isasymlangle}a{\isacharcomma}{\kern0pt}\ {\isacharparenleft}{\kern0pt}x\ {\isasymcirc}\isactrlsub c\ right{\isacharunderscore}{\kern0pt}cart{\isacharunderscore}{\kern0pt}proj\ A\ {\isasymone}{\isacharparenright}{\kern0pt}\isactrlsup {\isasymsharp}{\isasymrangle}\ {\isasymin}\isactrlsub c\ domain\ {\isacharparenleft}{\kern0pt}eval{\isacharunderscore}{\kern0pt}func\ X\ A{\isacharparenright}{\kern0pt}{\isachardoublequoteclose}\isanewline
\ \ \ \ \isacommand{using}\isamarkupfalse%
\ cfunc{\isacharunderscore}{\kern0pt}type{\isacharunderscore}{\kern0pt}def\ \isacommand{by}\isamarkupfalse%
\ {\isacharparenleft}{\kern0pt}typecheck{\isacharunderscore}{\kern0pt}cfuncs{\isacharcomma}{\kern0pt}\ auto{\isacharparenright}{\kern0pt}\isanewline
\ \ \isacommand{have}\isamarkupfalse%
\ {\isachardoublequoteopen}{\isacharparenleft}{\kern0pt}eval{\isacharunderscore}{\kern0pt}func\ X\ A{\isacharparenright}{\kern0pt}\ {\isasymcirc}\isactrlsub c\ \ {\isasymlangle}a{\isacharcomma}{\kern0pt}\ {\isacharparenleft}{\kern0pt}x\ {\isasymcirc}\isactrlsub c\ right{\isacharunderscore}{\kern0pt}cart{\isacharunderscore}{\kern0pt}proj\ A\ {\isasymone}{\isacharparenright}{\kern0pt}\isactrlsup {\isasymsharp}{\isasymrangle}\ {\isacharequal}{\kern0pt}\ \ \ \ \isanewline
\ \ \ \ \ \ \ \ {\isacharparenleft}{\kern0pt}eval{\isacharunderscore}{\kern0pt}func\ X\ A{\isacharparenright}{\kern0pt}\ {\isasymcirc}\isactrlsub c\ {\isacharparenleft}{\kern0pt}{\isacharparenleft}{\kern0pt}id{\isacharparenleft}{\kern0pt}A{\isacharparenright}{\kern0pt}\ {\isasymtimes}\isactrlsub f\ {\isacharparenleft}{\kern0pt}x\ {\isasymcirc}\isactrlsub c\ right{\isacharunderscore}{\kern0pt}cart{\isacharunderscore}{\kern0pt}proj\ A\ {\isasymone}{\isacharparenright}{\kern0pt}\isactrlsup {\isasymsharp}{\isacharparenright}{\kern0pt}\ {\isasymcirc}\isactrlsub c\ {\isasymlangle}a{\isacharcomma}{\kern0pt}\ id{\isacharparenleft}{\kern0pt}{\isasymone}{\isacharparenright}{\kern0pt}{\isasymrangle}{\isacharparenright}{\kern0pt}{\isachardoublequoteclose}\isanewline
\ \ \ \ \isacommand{by}\isamarkupfalse%
\ {\isacharparenleft}{\kern0pt}typecheck{\isacharunderscore}{\kern0pt}cfuncs{\isacharcomma}{\kern0pt}\ smt\ a{\isacharunderscore}{\kern0pt}def\ cfunc{\isacharunderscore}{\kern0pt}cross{\isacharunderscore}{\kern0pt}prod{\isacharunderscore}{\kern0pt}comp{\isacharunderscore}{\kern0pt}cfunc{\isacharunderscore}{\kern0pt}prod\ id{\isacharunderscore}{\kern0pt}left{\isacharunderscore}{\kern0pt}unit{\isadigit{2}}\ id{\isacharunderscore}{\kern0pt}right{\isacharunderscore}{\kern0pt}unit{\isadigit{2}}\ x{\isacharunderscore}{\kern0pt}type{\isadigit{2}}{\isacharparenright}{\kern0pt}\isanewline
\ \ \isacommand{also}\isamarkupfalse%
\ \isacommand{have}\isamarkupfalse%
\ {\isachardoublequoteopen}{\isachardot}{\kern0pt}{\isachardot}{\kern0pt}{\isachardot}{\kern0pt}\ {\isacharequal}{\kern0pt}\ {\isacharparenleft}{\kern0pt}{\isacharparenleft}{\kern0pt}eval{\isacharunderscore}{\kern0pt}func\ X\ A{\isacharparenright}{\kern0pt}\ {\isasymcirc}\isactrlsub c\ {\isacharparenleft}{\kern0pt}id{\isacharparenleft}{\kern0pt}A{\isacharparenright}{\kern0pt}\ {\isasymtimes}\isactrlsub f\ {\isacharparenleft}{\kern0pt}x\ {\isasymcirc}\isactrlsub c\ right{\isacharunderscore}{\kern0pt}cart{\isacharunderscore}{\kern0pt}proj\ A\ {\isasymone}{\isacharparenright}{\kern0pt}\isactrlsup {\isasymsharp}{\isacharparenright}{\kern0pt}{\isacharparenright}{\kern0pt}\ {\isasymcirc}\isactrlsub c\ {\isasymlangle}a{\isacharcomma}{\kern0pt}\ id{\isacharparenleft}{\kern0pt}{\isasymone}{\isacharparenright}{\kern0pt}{\isasymrangle}{\isachardoublequoteclose}\isanewline
\ \ \ \ \isacommand{by}\isamarkupfalse%
\ {\isacharparenleft}{\kern0pt}typecheck{\isacharunderscore}{\kern0pt}cfuncs{\isacharcomma}{\kern0pt}\ meson\ a{\isacharunderscore}{\kern0pt}def\ comp{\isacharunderscore}{\kern0pt}associative{\isadigit{2}}\ x{\isacharunderscore}{\kern0pt}type{\isadigit{2}}{\isacharparenright}{\kern0pt}\isanewline
\ \ \isacommand{also}\isamarkupfalse%
\ \isacommand{have}\isamarkupfalse%
\ {\isachardoublequoteopen}{\isachardot}{\kern0pt}{\isachardot}{\kern0pt}{\isachardot}{\kern0pt}\ {\isacharequal}{\kern0pt}\ {\isacharparenleft}{\kern0pt}x\ {\isasymcirc}\isactrlsub c\ right{\isacharunderscore}{\kern0pt}cart{\isacharunderscore}{\kern0pt}proj\ A\ {\isasymone}{\isacharparenright}{\kern0pt}\ {\isasymcirc}\isactrlsub c\ {\isasymlangle}a{\isacharcomma}{\kern0pt}\ id{\isacharparenleft}{\kern0pt}{\isasymone}{\isacharparenright}{\kern0pt}{\isasymrangle}{\isachardoublequoteclose}\isanewline
\ \ \ \ \isacommand{by}\isamarkupfalse%
\ {\isacharparenleft}{\kern0pt}metis\ comp{\isacharunderscore}{\kern0pt}type\ right{\isacharunderscore}{\kern0pt}cart{\isacharunderscore}{\kern0pt}proj{\isacharunderscore}{\kern0pt}type\ transpose{\isacharunderscore}{\kern0pt}func{\isacharunderscore}{\kern0pt}def\ x{\isacharunderscore}{\kern0pt}type{\isadigit{2}}{\isacharparenright}{\kern0pt}\ \isanewline
\ \ \isacommand{also}\isamarkupfalse%
\ \isacommand{have}\isamarkupfalse%
\ {\isachardoublequoteopen}{\isachardot}{\kern0pt}{\isachardot}{\kern0pt}{\isachardot}{\kern0pt}\ {\isacharequal}{\kern0pt}\ x\ {\isasymcirc}\isactrlsub c\ {\isacharparenleft}{\kern0pt}right{\isacharunderscore}{\kern0pt}cart{\isacharunderscore}{\kern0pt}proj\ A\ {\isasymone}\ {\isasymcirc}\isactrlsub c\ {\isasymlangle}a{\isacharcomma}{\kern0pt}\ id{\isacharparenleft}{\kern0pt}{\isasymone}{\isacharparenright}{\kern0pt}{\isasymrangle}{\isacharparenright}{\kern0pt}{\isachardoublequoteclose}\isanewline
\ \ \ \ \isacommand{using}\isamarkupfalse%
\ a{\isacharunderscore}{\kern0pt}def\ cfunc{\isacharunderscore}{\kern0pt}type{\isacharunderscore}{\kern0pt}def\ comp{\isacharunderscore}{\kern0pt}associative\ x{\isacharunderscore}{\kern0pt}type{\isadigit{2}}\ \isacommand{by}\isamarkupfalse%
\ {\isacharparenleft}{\kern0pt}typecheck{\isacharunderscore}{\kern0pt}cfuncs{\isacharcomma}{\kern0pt}\ auto{\isacharparenright}{\kern0pt}\isanewline
\ \ \isacommand{also}\isamarkupfalse%
\ \isacommand{have}\isamarkupfalse%
\ {\isachardoublequoteopen}{\isachardot}{\kern0pt}{\isachardot}{\kern0pt}{\isachardot}{\kern0pt}\ {\isacharequal}{\kern0pt}\ x{\isachardoublequoteclose}\isanewline
\ \ \ \ \isacommand{using}\isamarkupfalse%
\ a{\isacharunderscore}{\kern0pt}def\ id{\isacharunderscore}{\kern0pt}right{\isacharunderscore}{\kern0pt}unit{\isadigit{2}}\ right{\isacharunderscore}{\kern0pt}cart{\isacharunderscore}{\kern0pt}proj{\isacharunderscore}{\kern0pt}cfunc{\isacharunderscore}{\kern0pt}prod\ x{\isacharunderscore}{\kern0pt}type{\isadigit{2}}\ \isacommand{by}\isamarkupfalse%
\ {\isacharparenleft}{\kern0pt}typecheck{\isacharunderscore}{\kern0pt}cfuncs{\isacharcomma}{\kern0pt}\ auto{\isacharparenright}{\kern0pt}\isanewline
\ \ \isacommand{then}\isamarkupfalse%
\ \isacommand{show}\isamarkupfalse%
\ {\isachardoublequoteopen}{\isasymexists}y{\isachardot}{\kern0pt}\ y\ {\isasymin}\isactrlsub c\ domain\ {\isacharparenleft}{\kern0pt}eval{\isacharunderscore}{\kern0pt}func\ X\ A{\isacharparenright}{\kern0pt}\ {\isasymand}\ eval{\isacharunderscore}{\kern0pt}func\ X\ A\ {\isasymcirc}\isactrlsub c\ y\ {\isacharequal}{\kern0pt}\ x{\isachardoublequoteclose}\isanewline
\ \ \ \ \isacommand{using}\isamarkupfalse%
\ calculation\ needed{\isacharunderscore}{\kern0pt}type\ \isacommand{by}\isamarkupfalse%
\ {\isacharparenleft}{\kern0pt}typecheck{\isacharunderscore}{\kern0pt}cfuncs{\isacharcomma}{\kern0pt}\ auto{\isacharparenright}{\kern0pt}\isanewline
\isacommand{qed}\isamarkupfalse%
%
\endisatagproof
{\isafoldproof}%
%
\isadelimproof
%
\endisadelimproof
%
\begin{isamarkuptext}%
The lemma below corresponds to a note above Definition 2.5.1 in Halvorson.%
\end{isamarkuptext}\isamarkuptrue%
\isacommand{lemma}\isamarkupfalse%
\ exponential{\isacharunderscore}{\kern0pt}object{\isacharunderscore}{\kern0pt}identity{\isacharcolon}{\kern0pt}\isanewline
\ \ {\isachardoublequoteopen}{\isacharparenleft}{\kern0pt}eval{\isacharunderscore}{\kern0pt}func\ X\ A{\isacharparenright}{\kern0pt}\isactrlsup {\isasymsharp}\ {\isacharequal}{\kern0pt}\ id\isactrlsub c{\isacharparenleft}{\kern0pt}X\isactrlbsup A\isactrlesup {\isacharparenright}{\kern0pt}{\isachardoublequoteclose}\isanewline
%
\isadelimproof
\ \ %
\endisadelimproof
%
\isatagproof
\isacommand{by}\isamarkupfalse%
\ {\isacharparenleft}{\kern0pt}metis\ cfunc{\isacharunderscore}{\kern0pt}type{\isacharunderscore}{\kern0pt}def\ eval{\isacharunderscore}{\kern0pt}func{\isacharunderscore}{\kern0pt}type\ id{\isacharunderscore}{\kern0pt}cross{\isacharunderscore}{\kern0pt}prod\ id{\isacharunderscore}{\kern0pt}right{\isacharunderscore}{\kern0pt}unit\ id{\isacharunderscore}{\kern0pt}type\ transpose{\isacharunderscore}{\kern0pt}func{\isacharunderscore}{\kern0pt}unique{\isacharparenright}{\kern0pt}%
\endisatagproof
{\isafoldproof}%
%
\isadelimproof
\isanewline
%
\endisadelimproof
\isanewline
\isacommand{lemma}\isamarkupfalse%
\ eval{\isacharunderscore}{\kern0pt}func{\isacharunderscore}{\kern0pt}X{\isacharunderscore}{\kern0pt}empty{\isacharunderscore}{\kern0pt}injective{\isacharcolon}{\kern0pt}\isanewline
\ \ \isakeyword{assumes}\ {\isachardoublequoteopen}is{\isacharunderscore}{\kern0pt}empty\ Y{\isachardoublequoteclose}\isanewline
\ \ \isakeyword{shows}\ {\isachardoublequoteopen}injective\ {\isacharparenleft}{\kern0pt}eval{\isacharunderscore}{\kern0pt}func\ X\ Y{\isacharparenright}{\kern0pt}{\isachardoublequoteclose}\isanewline
%
\isadelimproof
\ \ %
\endisadelimproof
%
\isatagproof
\isacommand{unfolding}\isamarkupfalse%
\ injective{\isacharunderscore}{\kern0pt}def\isanewline
\ \ \isacommand{by}\isamarkupfalse%
\ {\isacharparenleft}{\kern0pt}typecheck{\isacharunderscore}{\kern0pt}cfuncs{\isacharcomma}{\kern0pt}metis\ assms\ cfunc{\isacharunderscore}{\kern0pt}type{\isacharunderscore}{\kern0pt}def\ comp{\isacharunderscore}{\kern0pt}type\ left{\isacharunderscore}{\kern0pt}cart{\isacharunderscore}{\kern0pt}proj{\isacharunderscore}{\kern0pt}type\ is{\isacharunderscore}{\kern0pt}empty{\isacharunderscore}{\kern0pt}def{\isacharparenright}{\kern0pt}%
\endisatagproof
{\isafoldproof}%
%
\isadelimproof
%
\endisadelimproof
%
\isadelimdocument
%
\endisadelimdocument
%
\isatagdocument
%
\isamarkupsubsection{Lifting Functions%
}
\isamarkuptrue%
%
\endisatagdocument
{\isafolddocument}%
%
\isadelimdocument
%
\endisadelimdocument
%
\begin{isamarkuptext}%
The definition below corresponds to Definition 2.5.1 in Halvorson.%
\end{isamarkuptext}\isamarkuptrue%
\isacommand{definition}\isamarkupfalse%
\ exp{\isacharunderscore}{\kern0pt}func\ {\isacharcolon}{\kern0pt}{\isacharcolon}{\kern0pt}\ {\isachardoublequoteopen}cfunc\ {\isasymRightarrow}\ cset\ {\isasymRightarrow}\ cfunc{\isachardoublequoteclose}\ {\isacharparenleft}{\kern0pt}{\isachardoublequoteopen}{\isacharparenleft}{\kern0pt}{\isacharunderscore}{\kern0pt}{\isacharparenright}{\kern0pt}\isactrlbsup {\isacharunderscore}{\kern0pt}\isactrlesup \isactrlsub f{\isachardoublequoteclose}\ {\isacharbrackleft}{\kern0pt}{\isadigit{1}}{\isadigit{0}}{\isadigit{0}}{\isacharcomma}{\kern0pt}{\isadigit{1}}{\isadigit{0}}{\isadigit{0}}{\isacharbrackright}{\kern0pt}{\isadigit{1}}{\isadigit{0}}{\isadigit{0}}{\isacharparenright}{\kern0pt}\ \isakeyword{where}\isanewline
\ \ {\isachardoublequoteopen}exp{\isacharunderscore}{\kern0pt}func\ g\ A\ {\isacharequal}{\kern0pt}\ {\isacharparenleft}{\kern0pt}g\ {\isasymcirc}\isactrlsub c\ eval{\isacharunderscore}{\kern0pt}func\ {\isacharparenleft}{\kern0pt}domain\ g{\isacharparenright}{\kern0pt}\ A{\isacharparenright}{\kern0pt}\isactrlsup {\isasymsharp}{\isachardoublequoteclose}\isanewline
\isanewline
\isacommand{lemma}\isamarkupfalse%
\ exp{\isacharunderscore}{\kern0pt}func{\isacharunderscore}{\kern0pt}def{\isadigit{2}}{\isacharcolon}{\kern0pt}\isanewline
\ \ \isakeyword{assumes}\ {\isachardoublequoteopen}g\ {\isacharcolon}{\kern0pt}\ X\ {\isasymrightarrow}\ Y{\isachardoublequoteclose}\isanewline
\ \ \isakeyword{shows}\ {\isachardoublequoteopen}exp{\isacharunderscore}{\kern0pt}func\ g\ A\ {\isacharequal}{\kern0pt}\ {\isacharparenleft}{\kern0pt}g\ {\isasymcirc}\isactrlsub c\ eval{\isacharunderscore}{\kern0pt}func\ X\ A{\isacharparenright}{\kern0pt}\isactrlsup {\isasymsharp}{\isachardoublequoteclose}\isanewline
%
\isadelimproof
\ \ %
\endisadelimproof
%
\isatagproof
\isacommand{using}\isamarkupfalse%
\ assms\ cfunc{\isacharunderscore}{\kern0pt}type{\isacharunderscore}{\kern0pt}def\ exp{\isacharunderscore}{\kern0pt}func{\isacharunderscore}{\kern0pt}def\ \isacommand{by}\isamarkupfalse%
\ auto%
\endisatagproof
{\isafoldproof}%
%
\isadelimproof
\isanewline
%
\endisadelimproof
\isanewline
\isacommand{lemma}\isamarkupfalse%
\ exp{\isacharunderscore}{\kern0pt}func{\isacharunderscore}{\kern0pt}type{\isacharbrackleft}{\kern0pt}type{\isacharunderscore}{\kern0pt}rule{\isacharbrackright}{\kern0pt}{\isacharcolon}{\kern0pt}\isanewline
\ \ \isakeyword{assumes}\ {\isachardoublequoteopen}g\ {\isacharcolon}{\kern0pt}\ X\ {\isasymrightarrow}\ Y{\isachardoublequoteclose}\isanewline
\ \ \isakeyword{shows}\ {\isachardoublequoteopen}g\isactrlbsup A\isactrlesup \isactrlsub f\ {\isacharcolon}{\kern0pt}\ X\isactrlbsup A\isactrlesup \ {\isasymrightarrow}\ Y\isactrlbsup A\isactrlesup {\isachardoublequoteclose}\isanewline
%
\isadelimproof
\ \ %
\endisadelimproof
%
\isatagproof
\isacommand{using}\isamarkupfalse%
\ assms\ \isacommand{by}\isamarkupfalse%
\ {\isacharparenleft}{\kern0pt}unfold\ exp{\isacharunderscore}{\kern0pt}func{\isacharunderscore}{\kern0pt}def{\isadigit{2}}{\isacharcomma}{\kern0pt}\ typecheck{\isacharunderscore}{\kern0pt}cfuncs{\isacharparenright}{\kern0pt}%
\endisatagproof
{\isafoldproof}%
%
\isadelimproof
\isanewline
%
\endisadelimproof
\isanewline
\isacommand{lemma}\isamarkupfalse%
\ exp{\isacharunderscore}{\kern0pt}of{\isacharunderscore}{\kern0pt}id{\isacharunderscore}{\kern0pt}is{\isacharunderscore}{\kern0pt}id{\isacharunderscore}{\kern0pt}of{\isacharunderscore}{\kern0pt}exp{\isacharcolon}{\kern0pt}\isanewline
\ \ {\isachardoublequoteopen}id{\isacharparenleft}{\kern0pt}X\isactrlbsup A\isactrlesup {\isacharparenright}{\kern0pt}\ {\isacharequal}{\kern0pt}\ {\isacharparenleft}{\kern0pt}id{\isacharparenleft}{\kern0pt}X{\isacharparenright}{\kern0pt}{\isacharparenright}{\kern0pt}\isactrlbsup A\isactrlesup \isactrlsub f{\isachardoublequoteclose}\isanewline
%
\isadelimproof
\ \ %
\endisadelimproof
%
\isatagproof
\isacommand{by}\isamarkupfalse%
\ {\isacharparenleft}{\kern0pt}metis\ {\isacharparenleft}{\kern0pt}no{\isacharunderscore}{\kern0pt}types{\isacharparenright}{\kern0pt}\ eval{\isacharunderscore}{\kern0pt}func{\isacharunderscore}{\kern0pt}type\ exp{\isacharunderscore}{\kern0pt}func{\isacharunderscore}{\kern0pt}def\ exponential{\isacharunderscore}{\kern0pt}object{\isacharunderscore}{\kern0pt}identity\ id{\isacharunderscore}{\kern0pt}domain\ id{\isacharunderscore}{\kern0pt}left{\isacharunderscore}{\kern0pt}unit{\isadigit{2}}{\isacharparenright}{\kern0pt}%
\endisatagproof
{\isafoldproof}%
%
\isadelimproof
%
\endisadelimproof
%
\begin{isamarkuptext}%
The lemma below corresponds to a note below Definition 2.5.1 in Halvorson.%
\end{isamarkuptext}\isamarkuptrue%
\isacommand{lemma}\isamarkupfalse%
\ exponential{\isacharunderscore}{\kern0pt}square{\isacharunderscore}{\kern0pt}diagram{\isacharcolon}{\kern0pt}\isanewline
\ \ \isakeyword{assumes}\ {\isachardoublequoteopen}g\ {\isacharcolon}{\kern0pt}\ Y\ {\isasymrightarrow}\ Z{\isachardoublequoteclose}\isanewline
\ \ \isakeyword{shows}\ {\isachardoublequoteopen}{\isacharparenleft}{\kern0pt}eval{\isacharunderscore}{\kern0pt}func\ Z\ A{\isacharparenright}{\kern0pt}\ {\isasymcirc}\isactrlsub c\ {\isacharparenleft}{\kern0pt}id\isactrlsub c{\isacharparenleft}{\kern0pt}A{\isacharparenright}{\kern0pt}{\isasymtimes}\isactrlsub f\ g\isactrlbsup A\isactrlesup \isactrlsub f{\isacharparenright}{\kern0pt}\ \ {\isacharequal}{\kern0pt}\ g\ {\isasymcirc}\isactrlsub c\ {\isacharparenleft}{\kern0pt}eval{\isacharunderscore}{\kern0pt}func\ Y\ A{\isacharparenright}{\kern0pt}{\isachardoublequoteclose}\isanewline
%
\isadelimproof
\ \ %
\endisadelimproof
%
\isatagproof
\isacommand{using}\isamarkupfalse%
\ assms\ \isacommand{by}\isamarkupfalse%
\ {\isacharparenleft}{\kern0pt}typecheck{\isacharunderscore}{\kern0pt}cfuncs{\isacharcomma}{\kern0pt}\ simp\ add{\isacharcolon}{\kern0pt}\ exp{\isacharunderscore}{\kern0pt}func{\isacharunderscore}{\kern0pt}def{\isadigit{2}}\ transpose{\isacharunderscore}{\kern0pt}func{\isacharunderscore}{\kern0pt}def{\isacharparenright}{\kern0pt}%
\endisatagproof
{\isafoldproof}%
%
\isadelimproof
%
\endisadelimproof
%
\begin{isamarkuptext}%
The lemma below corresponds to Proposition 2.5.2 in Halvorson.%
\end{isamarkuptext}\isamarkuptrue%
\isacommand{lemma}\isamarkupfalse%
\ transpose{\isacharunderscore}{\kern0pt}of{\isacharunderscore}{\kern0pt}comp{\isacharcolon}{\kern0pt}\isanewline
\ \ \isakeyword{assumes}\ f{\isacharunderscore}{\kern0pt}type{\isacharcolon}{\kern0pt}\ {\isachardoublequoteopen}f{\isacharcolon}{\kern0pt}\ A\ {\isasymtimes}\isactrlsub c\ X\ {\isasymrightarrow}\ Y{\isachardoublequoteclose}\ \isakeyword{and}\ g{\isacharunderscore}{\kern0pt}type{\isacharcolon}{\kern0pt}\ {\isachardoublequoteopen}g{\isacharcolon}{\kern0pt}\ Y\ {\isasymrightarrow}\ Z{\isachardoublequoteclose}\isanewline
\ \ \isakeyword{shows}\ {\isachardoublequoteopen}f{\isacharcolon}{\kern0pt}\ A\ {\isasymtimes}\isactrlsub c\ X\ {\isasymrightarrow}\ Y\ {\isasymand}\ g{\isacharcolon}{\kern0pt}\ Y\ {\isasymrightarrow}\ Z\ \ {\isasymLongrightarrow}\ \ {\isacharparenleft}{\kern0pt}g\ {\isasymcirc}\isactrlsub c\ f{\isacharparenright}{\kern0pt}\isactrlsup {\isasymsharp}\ {\isacharequal}{\kern0pt}\ g\isactrlbsup A\isactrlesup \isactrlsub f\ {\isasymcirc}\isactrlsub c\ f\isactrlsup {\isasymsharp}{\isachardoublequoteclose}\isanewline
%
\isadelimproof
%
\endisadelimproof
%
\isatagproof
\isacommand{proof}\isamarkupfalse%
\ clarify\isanewline
\ \ \isacommand{have}\isamarkupfalse%
\ left{\isacharunderscore}{\kern0pt}eq{\isacharcolon}{\kern0pt}\ {\isachardoublequoteopen}{\isacharparenleft}{\kern0pt}eval{\isacharunderscore}{\kern0pt}func\ Z\ A{\isacharparenright}{\kern0pt}\ {\isasymcirc}\isactrlsub c{\isacharparenleft}{\kern0pt}id{\isacharparenleft}{\kern0pt}A{\isacharparenright}{\kern0pt}\ {\isasymtimes}\isactrlsub f\ {\isacharparenleft}{\kern0pt}g\ {\isasymcirc}\isactrlsub c\ f{\isacharparenright}{\kern0pt}\isactrlsup {\isasymsharp}{\isacharparenright}{\kern0pt}\ {\isacharequal}{\kern0pt}\ g\ {\isasymcirc}\isactrlsub c\ f{\isachardoublequoteclose}\isanewline
\ \ \ \ \isacommand{using}\isamarkupfalse%
\ comp{\isacharunderscore}{\kern0pt}type\ f{\isacharunderscore}{\kern0pt}type\ g{\isacharunderscore}{\kern0pt}type\ transpose{\isacharunderscore}{\kern0pt}func{\isacharunderscore}{\kern0pt}def\ \isacommand{by}\isamarkupfalse%
\ blast\isanewline
\ \ \isacommand{have}\isamarkupfalse%
\ right{\isacharunderscore}{\kern0pt}eq{\isacharcolon}{\kern0pt}\ {\isachardoublequoteopen}{\isacharparenleft}{\kern0pt}eval{\isacharunderscore}{\kern0pt}func\ Z\ A{\isacharparenright}{\kern0pt}\ {\isasymcirc}\isactrlsub c\ {\isacharparenleft}{\kern0pt}id\isactrlsub c\ A\ {\isasymtimes}\isactrlsub f\ {\isacharparenleft}{\kern0pt}g\isactrlbsup A\isactrlesup \isactrlsub f\ {\isasymcirc}\isactrlsub c\ f\isactrlsup {\isasymsharp}{\isacharparenright}{\kern0pt}{\isacharparenright}{\kern0pt}\ {\isacharequal}{\kern0pt}\ g\ {\isasymcirc}\isactrlsub c\ f{\isachardoublequoteclose}\isanewline
\ \ \isacommand{proof}\isamarkupfalse%
\ {\isacharminus}{\kern0pt}\ \isanewline
\ \ \ \ \isacommand{have}\isamarkupfalse%
\ {\isachardoublequoteopen}{\isacharparenleft}{\kern0pt}eval{\isacharunderscore}{\kern0pt}func\ Z\ A{\isacharparenright}{\kern0pt}\ {\isasymcirc}\isactrlsub c\ {\isacharparenleft}{\kern0pt}id\isactrlsub c\ A\ {\isasymtimes}\isactrlsub f\ {\isacharparenleft}{\kern0pt}g\isactrlbsup A\isactrlesup \isactrlsub f\ {\isasymcirc}\isactrlsub c\ f\isactrlsup {\isasymsharp}{\isacharparenright}{\kern0pt}{\isacharparenright}{\kern0pt}\ {\isacharequal}{\kern0pt}\isanewline
\ \ \ \ \ \ \ \ \ \ \ \ \ \ \ \ \ \ \ {\isacharparenleft}{\kern0pt}eval{\isacharunderscore}{\kern0pt}func\ Z\ A{\isacharparenright}{\kern0pt}\ {\isasymcirc}\isactrlsub c\ {\isacharparenleft}{\kern0pt}{\isacharparenleft}{\kern0pt}id\isactrlsub c\ A\ {\isasymtimes}\isactrlsub f\ {\isacharparenleft}{\kern0pt}g\isactrlbsup A\isactrlesup \isactrlsub f{\isacharparenright}{\kern0pt}{\isacharparenright}{\kern0pt}\ {\isasymcirc}\isactrlsub c\ \ {\isacharparenleft}{\kern0pt}id\isactrlsub c\ A\ {\isasymtimes}\isactrlsub f\ f\isactrlsup {\isasymsharp}{\isacharparenright}{\kern0pt}{\isacharparenright}{\kern0pt}{\isachardoublequoteclose}\isanewline
\ \ \ \ \ \ \isacommand{by}\isamarkupfalse%
\ {\isacharparenleft}{\kern0pt}typecheck{\isacharunderscore}{\kern0pt}cfuncs{\isacharcomma}{\kern0pt}\ smt\ identity{\isacharunderscore}{\kern0pt}distributes{\isacharunderscore}{\kern0pt}across{\isacharunderscore}{\kern0pt}composition\ assms{\isacharparenright}{\kern0pt}\isanewline
\ \ \ \ \isacommand{also}\isamarkupfalse%
\ \isacommand{have}\isamarkupfalse%
\ {\isachardoublequoteopen}{\isachardot}{\kern0pt}{\isachardot}{\kern0pt}{\isachardot}{\kern0pt}\ {\isacharequal}{\kern0pt}\ {\isacharparenleft}{\kern0pt}g\ {\isasymcirc}\isactrlsub c\ eval{\isacharunderscore}{\kern0pt}func\ Y\ A{\isacharparenright}{\kern0pt}\ {\isasymcirc}\isactrlsub c\ \ {\isacharparenleft}{\kern0pt}id\isactrlsub c\ A\ {\isasymtimes}\isactrlsub f\ f\isactrlsup {\isasymsharp}{\isacharparenright}{\kern0pt}{\isachardoublequoteclose}\isanewline
\ \ \ \ \ \ \isacommand{by}\isamarkupfalse%
\ {\isacharparenleft}{\kern0pt}typecheck{\isacharunderscore}{\kern0pt}cfuncs{\isacharcomma}{\kern0pt}\ smt\ comp{\isacharunderscore}{\kern0pt}associative{\isadigit{2}}\ exp{\isacharunderscore}{\kern0pt}func{\isacharunderscore}{\kern0pt}def{\isadigit{2}}\ transpose{\isacharunderscore}{\kern0pt}func{\isacharunderscore}{\kern0pt}def\ assms{\isacharparenright}{\kern0pt}\isanewline
\ \ \ \ \isacommand{also}\isamarkupfalse%
\ \isacommand{have}\isamarkupfalse%
\ {\isachardoublequoteopen}{\isachardot}{\kern0pt}{\isachardot}{\kern0pt}{\isachardot}{\kern0pt}\ {\isacharequal}{\kern0pt}\ g\ {\isasymcirc}\isactrlsub c\ f{\isachardoublequoteclose}\isanewline
\ \ \ \ \ \ \isacommand{by}\isamarkupfalse%
\ {\isacharparenleft}{\kern0pt}typecheck{\isacharunderscore}{\kern0pt}cfuncs{\isacharcomma}{\kern0pt}\ smt\ {\isacharparenleft}{\kern0pt}verit{\isacharcomma}{\kern0pt}\ best{\isacharparenright}{\kern0pt}\ comp{\isacharunderscore}{\kern0pt}associative{\isadigit{2}}\ transpose{\isacharunderscore}{\kern0pt}func{\isacharunderscore}{\kern0pt}def\ assms{\isacharparenright}{\kern0pt}\isanewline
\ \ \ \ \isacommand{then}\isamarkupfalse%
\ \isacommand{show}\isamarkupfalse%
\ {\isacharquery}{\kern0pt}thesis\isanewline
\ \ \ \ \ \ \isacommand{by}\isamarkupfalse%
\ {\isacharparenleft}{\kern0pt}simp\ add{\isacharcolon}{\kern0pt}\ calculation{\isacharparenright}{\kern0pt}\isanewline
\ \ \isacommand{qed}\isamarkupfalse%
\isanewline
\ \ \isacommand{show}\isamarkupfalse%
\ {\isachardoublequoteopen}{\isacharparenleft}{\kern0pt}g\ {\isasymcirc}\isactrlsub c\ f{\isacharparenright}{\kern0pt}\isactrlsup {\isasymsharp}\ {\isacharequal}{\kern0pt}\ g\isactrlbsup A\isactrlesup \isactrlsub f\ {\isasymcirc}\isactrlsub c\ f\isactrlsup {\isasymsharp}{\isachardoublequoteclose}\isanewline
\ \ \ \ \isacommand{using}\isamarkupfalse%
\ assms\ \isacommand{by}\isamarkupfalse%
\ {\isacharparenleft}{\kern0pt}typecheck{\isacharunderscore}{\kern0pt}cfuncs{\isacharcomma}{\kern0pt}\ metis\ right{\isacharunderscore}{\kern0pt}eq\ transpose{\isacharunderscore}{\kern0pt}func{\isacharunderscore}{\kern0pt}unique{\isacharparenright}{\kern0pt}\isanewline
\isacommand{qed}\isamarkupfalse%
%
\endisatagproof
{\isafoldproof}%
%
\isadelimproof
\isanewline
%
\endisadelimproof
\isanewline
\isacommand{lemma}\isamarkupfalse%
\ exponential{\isacharunderscore}{\kern0pt}object{\isacharunderscore}{\kern0pt}identity{\isadigit{2}}{\isacharcolon}{\kern0pt}\ \isanewline
\ \ {\isachardoublequoteopen}id{\isacharparenleft}{\kern0pt}X{\isacharparenright}{\kern0pt}\isactrlbsup A\isactrlesup \isactrlsub f\ {\isacharequal}{\kern0pt}\ id\isactrlsub c{\isacharparenleft}{\kern0pt}X\isactrlbsup A\isactrlesup {\isacharparenright}{\kern0pt}{\isachardoublequoteclose}\isanewline
%
\isadelimproof
\ \ %
\endisadelimproof
%
\isatagproof
\isacommand{by}\isamarkupfalse%
\ {\isacharparenleft}{\kern0pt}metis\ eval{\isacharunderscore}{\kern0pt}func{\isacharunderscore}{\kern0pt}type\ exp{\isacharunderscore}{\kern0pt}func{\isacharunderscore}{\kern0pt}def\ exponential{\isacharunderscore}{\kern0pt}object{\isacharunderscore}{\kern0pt}identity\ id{\isacharunderscore}{\kern0pt}domain\ id{\isacharunderscore}{\kern0pt}left{\isacharunderscore}{\kern0pt}unit{\isadigit{2}}{\isacharparenright}{\kern0pt}%
\endisatagproof
{\isafoldproof}%
%
\isadelimproof
%
\endisadelimproof
%
\begin{isamarkuptext}%
The lemma below corresponds to comments below Proposition 2.5.2 and above Definition 2.5.3 in Halvorson.%
\end{isamarkuptext}\isamarkuptrue%
\isacommand{lemma}\isamarkupfalse%
\ eval{\isacharunderscore}{\kern0pt}of{\isacharunderscore}{\kern0pt}id{\isacharunderscore}{\kern0pt}cross{\isacharunderscore}{\kern0pt}id{\isacharunderscore}{\kern0pt}sharp{\isadigit{1}}{\isacharcolon}{\kern0pt}\isanewline
\ \ {\isachardoublequoteopen}{\isacharparenleft}{\kern0pt}eval{\isacharunderscore}{\kern0pt}func\ {\isacharparenleft}{\kern0pt}A\ {\isasymtimes}\isactrlsub c\ X{\isacharparenright}{\kern0pt}\ A{\isacharparenright}{\kern0pt}\ {\isasymcirc}\isactrlsub c\ {\isacharparenleft}{\kern0pt}id{\isacharparenleft}{\kern0pt}A{\isacharparenright}{\kern0pt}\ {\isasymtimes}\isactrlsub f\ {\isacharparenleft}{\kern0pt}id{\isacharparenleft}{\kern0pt}A\ {\isasymtimes}\isactrlsub c\ X{\isacharparenright}{\kern0pt}{\isacharparenright}{\kern0pt}\isactrlsup {\isasymsharp}{\isacharparenright}{\kern0pt}\ \ {\isacharequal}{\kern0pt}\ id{\isacharparenleft}{\kern0pt}A\ {\isasymtimes}\isactrlsub c\ X{\isacharparenright}{\kern0pt}{\isachardoublequoteclose}\isanewline
%
\isadelimproof
\ \ %
\endisadelimproof
%
\isatagproof
\isacommand{using}\isamarkupfalse%
\ id{\isacharunderscore}{\kern0pt}type\ transpose{\isacharunderscore}{\kern0pt}func{\isacharunderscore}{\kern0pt}def\ \isacommand{by}\isamarkupfalse%
\ blast%
\endisatagproof
{\isafoldproof}%
%
\isadelimproof
\isanewline
%
\endisadelimproof
\isacommand{lemma}\isamarkupfalse%
\ eval{\isacharunderscore}{\kern0pt}of{\isacharunderscore}{\kern0pt}id{\isacharunderscore}{\kern0pt}cross{\isacharunderscore}{\kern0pt}id{\isacharunderscore}{\kern0pt}sharp{\isadigit{2}}{\isacharcolon}{\kern0pt}\isanewline
\ \ \isakeyword{assumes}\ {\isachardoublequoteopen}a\ {\isacharcolon}{\kern0pt}\ Z\ {\isasymrightarrow}\ A{\isachardoublequoteclose}\ {\isachardoublequoteopen}x\ {\isacharcolon}{\kern0pt}\ Z\ {\isasymrightarrow}\ X{\isachardoublequoteclose}\isanewline
\ \ \isakeyword{shows}\ {\isachardoublequoteopen}{\isacharparenleft}{\kern0pt}{\isacharparenleft}{\kern0pt}eval{\isacharunderscore}{\kern0pt}func\ {\isacharparenleft}{\kern0pt}A\ {\isasymtimes}\isactrlsub c\ X{\isacharparenright}{\kern0pt}\ A{\isacharparenright}{\kern0pt}\ {\isasymcirc}\isactrlsub c\ {\isacharparenleft}{\kern0pt}id{\isacharparenleft}{\kern0pt}A{\isacharparenright}{\kern0pt}\ {\isasymtimes}\isactrlsub f\ {\isacharparenleft}{\kern0pt}id{\isacharparenleft}{\kern0pt}A\ {\isasymtimes}\isactrlsub c\ X{\isacharparenright}{\kern0pt}{\isacharparenright}{\kern0pt}\isactrlsup {\isasymsharp}{\isacharparenright}{\kern0pt}{\isacharparenright}{\kern0pt}\ {\isasymcirc}\isactrlsub c\ {\isasymlangle}a{\isacharcomma}{\kern0pt}x{\isasymrangle}\ {\isacharequal}{\kern0pt}\ {\isasymlangle}a{\isacharcomma}{\kern0pt}x{\isasymrangle}{\isachardoublequoteclose}\isanewline
%
\isadelimproof
\ \ %
\endisadelimproof
%
\isatagproof
\isacommand{by}\isamarkupfalse%
\ {\isacharparenleft}{\kern0pt}smt\ assms\ cfunc{\isacharunderscore}{\kern0pt}cross{\isacharunderscore}{\kern0pt}prod{\isacharunderscore}{\kern0pt}comp{\isacharunderscore}{\kern0pt}cfunc{\isacharunderscore}{\kern0pt}prod\ eval{\isacharunderscore}{\kern0pt}of{\isacharunderscore}{\kern0pt}id{\isacharunderscore}{\kern0pt}cross{\isacharunderscore}{\kern0pt}id{\isacharunderscore}{\kern0pt}sharp{\isadigit{1}}\ id{\isacharunderscore}{\kern0pt}cross{\isacharunderscore}{\kern0pt}prod\ id{\isacharunderscore}{\kern0pt}left{\isacharunderscore}{\kern0pt}unit{\isadigit{2}}\ id{\isacharunderscore}{\kern0pt}type{\isacharparenright}{\kern0pt}%
\endisatagproof
{\isafoldproof}%
%
\isadelimproof
\isanewline
%
\endisadelimproof
\isanewline
\isacommand{lemma}\isamarkupfalse%
\ transpose{\isacharunderscore}{\kern0pt}factors{\isacharcolon}{\kern0pt}\ \isanewline
\ \ \isakeyword{assumes}\ {\isachardoublequoteopen}f{\isacharcolon}{\kern0pt}\ X\ {\isasymrightarrow}\ Y{\isachardoublequoteclose}\isanewline
\ \ \isakeyword{assumes}\ {\isachardoublequoteopen}g{\isacharcolon}{\kern0pt}\ Y\ {\isasymrightarrow}\ Z{\isachardoublequoteclose}\isanewline
\ \ \isakeyword{shows}\ {\isachardoublequoteopen}{\isacharparenleft}{\kern0pt}g\ {\isasymcirc}\isactrlsub c\ f{\isacharparenright}{\kern0pt}\isactrlbsup A\isactrlesup \isactrlsub f\ {\isacharequal}{\kern0pt}\ {\isacharparenleft}{\kern0pt}g\isactrlbsup A\isactrlesup \isactrlsub f{\isacharparenright}{\kern0pt}\ {\isasymcirc}\isactrlsub c\ {\isacharparenleft}{\kern0pt}f\isactrlbsup A\isactrlesup \isactrlsub f{\isacharparenright}{\kern0pt}{\isachardoublequoteclose}\isanewline
%
\isadelimproof
\ \ %
\endisadelimproof
%
\isatagproof
\isacommand{using}\isamarkupfalse%
\ assms\ \isacommand{by}\isamarkupfalse%
\ {\isacharparenleft}{\kern0pt}typecheck{\isacharunderscore}{\kern0pt}cfuncs{\isacharcomma}{\kern0pt}\ smt\ comp{\isacharunderscore}{\kern0pt}associative{\isadigit{2}}\ comp{\isacharunderscore}{\kern0pt}type\ eval{\isacharunderscore}{\kern0pt}func{\isacharunderscore}{\kern0pt}type\ exp{\isacharunderscore}{\kern0pt}func{\isacharunderscore}{\kern0pt}def{\isadigit{2}}\ transpose{\isacharunderscore}{\kern0pt}of{\isacharunderscore}{\kern0pt}comp{\isacharparenright}{\kern0pt}%
\endisatagproof
{\isafoldproof}%
%
\isadelimproof
%
\endisadelimproof
%
\isadelimdocument
%
\endisadelimdocument
%
\isatagdocument
%
\isamarkupsubsection{Inverse Transpose Function (flat)%
}
\isamarkuptrue%
%
\endisatagdocument
{\isafolddocument}%
%
\isadelimdocument
%
\endisadelimdocument
%
\begin{isamarkuptext}%
The definition below corresponds to Definition 2.5.3 in Halvorson.%
\end{isamarkuptext}\isamarkuptrue%
\isacommand{definition}\isamarkupfalse%
\ inv{\isacharunderscore}{\kern0pt}transpose{\isacharunderscore}{\kern0pt}func\ {\isacharcolon}{\kern0pt}{\isacharcolon}{\kern0pt}\ {\isachardoublequoteopen}cfunc\ {\isasymRightarrow}\ cfunc{\isachardoublequoteclose}\ {\isacharparenleft}{\kern0pt}{\isachardoublequoteopen}{\isacharunderscore}{\kern0pt}\isactrlsup {\isasymflat}{\isachardoublequoteclose}\ {\isacharbrackleft}{\kern0pt}{\isadigit{1}}{\isadigit{0}}{\isadigit{0}}{\isacharbrackright}{\kern0pt}{\isadigit{1}}{\isadigit{0}}{\isadigit{0}}{\isacharparenright}{\kern0pt}\ \isakeyword{where}\isanewline
\ \ {\isachardoublequoteopen}f\isactrlsup {\isasymflat}\ {\isacharequal}{\kern0pt}\ {\isacharparenleft}{\kern0pt}THE\ g{\isachardot}{\kern0pt}\ {\isasymexists}\ Z\ X\ A{\isachardot}{\kern0pt}\ domain\ f\ {\isacharequal}{\kern0pt}\ Z\ {\isasymand}\ codomain\ f\ {\isacharequal}{\kern0pt}\ X\isactrlbsup A\isactrlesup \ {\isasymand}\ g\ {\isacharequal}{\kern0pt}\ {\isacharparenleft}{\kern0pt}eval{\isacharunderscore}{\kern0pt}func\ X\ A{\isacharparenright}{\kern0pt}\ {\isasymcirc}\isactrlsub c\ {\isacharparenleft}{\kern0pt}id\ A\ {\isasymtimes}\isactrlsub f\ f{\isacharparenright}{\kern0pt}{\isacharparenright}{\kern0pt}{\isachardoublequoteclose}\isanewline
\isanewline
\isacommand{lemma}\isamarkupfalse%
\ inv{\isacharunderscore}{\kern0pt}transpose{\isacharunderscore}{\kern0pt}func{\isacharunderscore}{\kern0pt}def{\isadigit{2}}{\isacharcolon}{\kern0pt}\isanewline
\ \ \isakeyword{assumes}\ {\isachardoublequoteopen}f\ {\isacharcolon}{\kern0pt}\ Z\ {\isasymrightarrow}\ X\isactrlbsup A\isactrlesup {\isachardoublequoteclose}\isanewline
\ \ \isakeyword{shows}\ {\isachardoublequoteopen}{\isasymexists}\ Z\ X\ A{\isachardot}{\kern0pt}\ domain\ f\ {\isacharequal}{\kern0pt}\ Z\ {\isasymand}\ codomain\ f\ {\isacharequal}{\kern0pt}\ X\isactrlbsup A\isactrlesup \ {\isasymand}\ f\isactrlsup {\isasymflat}\ {\isacharequal}{\kern0pt}\ {\isacharparenleft}{\kern0pt}eval{\isacharunderscore}{\kern0pt}func\ X\ A{\isacharparenright}{\kern0pt}\ {\isasymcirc}\isactrlsub c\ {\isacharparenleft}{\kern0pt}id\ A\ {\isasymtimes}\isactrlsub f\ f{\isacharparenright}{\kern0pt}{\isachardoublequoteclose}\isanewline
%
\isadelimproof
\ \ %
\endisadelimproof
%
\isatagproof
\isacommand{unfolding}\isamarkupfalse%
\ inv{\isacharunderscore}{\kern0pt}transpose{\isacharunderscore}{\kern0pt}func{\isacharunderscore}{\kern0pt}def\isanewline
\isacommand{proof}\isamarkupfalse%
\ {\isacharparenleft}{\kern0pt}rule\ theI{\isacharparenright}{\kern0pt}\isanewline
\ \ \isacommand{show}\isamarkupfalse%
\ {\isachardoublequoteopen}{\isasymexists}Z\ Y\ B{\isachardot}{\kern0pt}\ domain\ f\ {\isacharequal}{\kern0pt}\ Z\ {\isasymand}\ codomain\ f\ {\isacharequal}{\kern0pt}\ Y\isactrlbsup B\isactrlesup \ {\isasymand}\ eval{\isacharunderscore}{\kern0pt}func\ X\ A\ {\isasymcirc}\isactrlsub c\ id\isactrlsub c\ A\ {\isasymtimes}\isactrlsub f\ f\ {\isacharequal}{\kern0pt}\ eval{\isacharunderscore}{\kern0pt}func\ Y\ B\ {\isasymcirc}\isactrlsub c\ id\isactrlsub c\ B\ {\isasymtimes}\isactrlsub f\ f{\isachardoublequoteclose}\isanewline
\ \ \ \ \isacommand{using}\isamarkupfalse%
\ assms\ cfunc{\isacharunderscore}{\kern0pt}type{\isacharunderscore}{\kern0pt}def\ \isacommand{by}\isamarkupfalse%
\ blast\isanewline
\isacommand{next}\isamarkupfalse%
\isanewline
\ \ \isacommand{fix}\isamarkupfalse%
\ g\isanewline
\ \ \isacommand{assume}\isamarkupfalse%
\ {\isachardoublequoteopen}{\isasymexists}Z\ X\ A{\isachardot}{\kern0pt}\ domain\ f\ {\isacharequal}{\kern0pt}\ Z\ {\isasymand}\ codomain\ f\ {\isacharequal}{\kern0pt}\ X\isactrlbsup A\isactrlesup \ {\isasymand}\ g\ {\isacharequal}{\kern0pt}\ eval{\isacharunderscore}{\kern0pt}func\ X\ A\ {\isasymcirc}\isactrlsub c\ id\isactrlsub c\ A\ {\isasymtimes}\isactrlsub f\ f{\isachardoublequoteclose}\isanewline
\ \ \isacommand{then}\isamarkupfalse%
\ \isacommand{show}\isamarkupfalse%
\ {\isachardoublequoteopen}g\ {\isacharequal}{\kern0pt}\ eval{\isacharunderscore}{\kern0pt}func\ X\ A\ {\isasymcirc}\isactrlsub c\ id\isactrlsub c\ A\ {\isasymtimes}\isactrlsub f\ f{\isachardoublequoteclose}\isanewline
\ \ \ \ \isacommand{by}\isamarkupfalse%
\ {\isacharparenleft}{\kern0pt}metis\ assms\ cfunc{\isacharunderscore}{\kern0pt}type{\isacharunderscore}{\kern0pt}def\ exp{\isacharunderscore}{\kern0pt}set{\isacharunderscore}{\kern0pt}inj{\isacharparenright}{\kern0pt}\isanewline
\isacommand{qed}\isamarkupfalse%
%
\endisatagproof
{\isafoldproof}%
%
\isadelimproof
\isanewline
%
\endisadelimproof
\isanewline
\isacommand{lemma}\isamarkupfalse%
\ inv{\isacharunderscore}{\kern0pt}transpose{\isacharunderscore}{\kern0pt}func{\isacharunderscore}{\kern0pt}def{\isadigit{3}}{\isacharcolon}{\kern0pt}\isanewline
\ \ \isakeyword{assumes}\ f{\isacharunderscore}{\kern0pt}type{\isacharcolon}{\kern0pt}\ {\isachardoublequoteopen}f\ {\isacharcolon}{\kern0pt}\ Z\ {\isasymrightarrow}\ X\isactrlbsup A\isactrlesup {\isachardoublequoteclose}\isanewline
\ \ \isakeyword{shows}\ {\isachardoublequoteopen}f\isactrlsup {\isasymflat}\ {\isacharequal}{\kern0pt}\ {\isacharparenleft}{\kern0pt}eval{\isacharunderscore}{\kern0pt}func\ X\ A{\isacharparenright}{\kern0pt}\ {\isasymcirc}\isactrlsub c\ {\isacharparenleft}{\kern0pt}id\ A\ {\isasymtimes}\isactrlsub f\ f{\isacharparenright}{\kern0pt}{\isachardoublequoteclose}\isanewline
%
\isadelimproof
\ \ %
\endisadelimproof
%
\isatagproof
\isacommand{by}\isamarkupfalse%
\ {\isacharparenleft}{\kern0pt}metis\ cfunc{\isacharunderscore}{\kern0pt}type{\isacharunderscore}{\kern0pt}def\ exp{\isacharunderscore}{\kern0pt}set{\isacharunderscore}{\kern0pt}inj\ f{\isacharunderscore}{\kern0pt}type\ inv{\isacharunderscore}{\kern0pt}transpose{\isacharunderscore}{\kern0pt}func{\isacharunderscore}{\kern0pt}def{\isadigit{2}}{\isacharparenright}{\kern0pt}%
\endisatagproof
{\isafoldproof}%
%
\isadelimproof
\isanewline
%
\endisadelimproof
\isanewline
\isacommand{lemma}\isamarkupfalse%
\ flat{\isacharunderscore}{\kern0pt}type{\isacharbrackleft}{\kern0pt}type{\isacharunderscore}{\kern0pt}rule{\isacharbrackright}{\kern0pt}{\isacharcolon}{\kern0pt}\isanewline
\ \ \isakeyword{assumes}\ f{\isacharunderscore}{\kern0pt}type{\isacharbrackleft}{\kern0pt}type{\isacharunderscore}{\kern0pt}rule{\isacharbrackright}{\kern0pt}{\isacharcolon}{\kern0pt}\ {\isachardoublequoteopen}f\ {\isacharcolon}{\kern0pt}\ Z\ {\isasymrightarrow}\ X\isactrlbsup A\isactrlesup {\isachardoublequoteclose}\isanewline
\ \ \isakeyword{shows}\ {\isachardoublequoteopen}f\isactrlsup {\isasymflat}\ {\isacharcolon}{\kern0pt}\ A\ {\isasymtimes}\isactrlsub c\ Z\ {\isasymrightarrow}\ X{\isachardoublequoteclose}\isanewline
%
\isadelimproof
\ \ %
\endisadelimproof
%
\isatagproof
\isacommand{by}\isamarkupfalse%
\ {\isacharparenleft}{\kern0pt}etcs{\isacharunderscore}{\kern0pt}subst\ inv{\isacharunderscore}{\kern0pt}transpose{\isacharunderscore}{\kern0pt}func{\isacharunderscore}{\kern0pt}def{\isadigit{3}}{\isacharcomma}{\kern0pt}\ typecheck{\isacharunderscore}{\kern0pt}cfuncs{\isacharparenright}{\kern0pt}%
\endisatagproof
{\isafoldproof}%
%
\isadelimproof
%
\endisadelimproof
%
\begin{isamarkuptext}%
The lemma below corresponds to Proposition 2.5.4 in Halvorson.%
\end{isamarkuptext}\isamarkuptrue%
\isacommand{lemma}\isamarkupfalse%
\ inv{\isacharunderscore}{\kern0pt}transpose{\isacharunderscore}{\kern0pt}of{\isacharunderscore}{\kern0pt}composition{\isacharcolon}{\kern0pt}\isanewline
\ \ \isakeyword{assumes}\ {\isachardoublequoteopen}f{\isacharcolon}{\kern0pt}\ X\ {\isasymrightarrow}\ Y{\isachardoublequoteclose}\ {\isachardoublequoteopen}g{\isacharcolon}{\kern0pt}\ Y\ {\isasymrightarrow}\ Z\isactrlbsup A\isactrlesup {\isachardoublequoteclose}\isanewline
\ \ \isakeyword{shows}\ {\isachardoublequoteopen}{\isacharparenleft}{\kern0pt}g\ {\isasymcirc}\isactrlsub c\ f{\isacharparenright}{\kern0pt}\isactrlsup {\isasymflat}\ {\isacharequal}{\kern0pt}\ g\isactrlsup {\isasymflat}\ {\isasymcirc}\isactrlsub c\ {\isacharparenleft}{\kern0pt}id{\isacharparenleft}{\kern0pt}A{\isacharparenright}{\kern0pt}\ {\isasymtimes}\isactrlsub f\ f{\isacharparenright}{\kern0pt}{\isachardoublequoteclose}\isanewline
%
\isadelimproof
\ \ %
\endisadelimproof
%
\isatagproof
\isacommand{using}\isamarkupfalse%
\ assms\ comp{\isacharunderscore}{\kern0pt}associative{\isadigit{2}}\ identity{\isacharunderscore}{\kern0pt}distributes{\isacharunderscore}{\kern0pt}across{\isacharunderscore}{\kern0pt}composition\isanewline
\ \ \isacommand{by}\isamarkupfalse%
\ {\isacharparenleft}{\kern0pt}{\isacharparenleft}{\kern0pt}etcs{\isacharunderscore}{\kern0pt}subst\ inv{\isacharunderscore}{\kern0pt}transpose{\isacharunderscore}{\kern0pt}func{\isacharunderscore}{\kern0pt}def{\isadigit{3}}{\isacharparenright}{\kern0pt}{\isacharplus}{\kern0pt}{\isacharcomma}{\kern0pt}\ typecheck{\isacharunderscore}{\kern0pt}cfuncs{\isacharcomma}{\kern0pt}\ auto{\isacharparenright}{\kern0pt}%
\endisatagproof
{\isafoldproof}%
%
\isadelimproof
%
\endisadelimproof
%
\begin{isamarkuptext}%
The lemma below corresponds to Proposition 2.5.5 in Halvorson.%
\end{isamarkuptext}\isamarkuptrue%
\isacommand{lemma}\isamarkupfalse%
\ flat{\isacharunderscore}{\kern0pt}cancels{\isacharunderscore}{\kern0pt}sharp{\isacharcolon}{\kern0pt}\isanewline
\ \ {\isachardoublequoteopen}f\ {\isacharcolon}{\kern0pt}\ A\ {\isasymtimes}\isactrlsub c\ Z\ {\isasymrightarrow}\ X\ \ {\isasymLongrightarrow}\ {\isacharparenleft}{\kern0pt}f\isactrlsup {\isasymsharp}{\isacharparenright}{\kern0pt}\isactrlsup {\isasymflat}\ {\isacharequal}{\kern0pt}\ f{\isachardoublequoteclose}\isanewline
%
\isadelimproof
\ \ %
\endisadelimproof
%
\isatagproof
\isacommand{using}\isamarkupfalse%
\ inv{\isacharunderscore}{\kern0pt}transpose{\isacharunderscore}{\kern0pt}func{\isacharunderscore}{\kern0pt}def{\isadigit{3}}\ transpose{\isacharunderscore}{\kern0pt}func{\isacharunderscore}{\kern0pt}def\ transpose{\isacharunderscore}{\kern0pt}func{\isacharunderscore}{\kern0pt}type\ \isacommand{by}\isamarkupfalse%
\ fastforce%
\endisatagproof
{\isafoldproof}%
%
\isadelimproof
%
\endisadelimproof
%
\begin{isamarkuptext}%
The lemma below corresponds to Proposition 2.5.6 in Halvorson.%
\end{isamarkuptext}\isamarkuptrue%
\isacommand{lemma}\isamarkupfalse%
\ sharp{\isacharunderscore}{\kern0pt}cancels{\isacharunderscore}{\kern0pt}flat{\isacharcolon}{\kern0pt}\isanewline
\ {\isachardoublequoteopen}f{\isacharcolon}{\kern0pt}\ Z\ {\isasymrightarrow}\ X\isactrlbsup A\isactrlesup \ \ {\isasymLongrightarrow}\ {\isacharparenleft}{\kern0pt}f\isactrlsup {\isasymflat}{\isacharparenright}{\kern0pt}\isactrlsup {\isasymsharp}\ {\isacharequal}{\kern0pt}\ f{\isachardoublequoteclose}\isanewline
%
\isadelimproof
%
\endisadelimproof
%
\isatagproof
\isacommand{proof}\isamarkupfalse%
\ {\isacharminus}{\kern0pt}\ \isanewline
\ \ \isacommand{assume}\isamarkupfalse%
\ f{\isacharunderscore}{\kern0pt}type{\isacharcolon}{\kern0pt}\ {\isachardoublequoteopen}f\ {\isacharcolon}{\kern0pt}\ Z\ {\isasymrightarrow}\ X\isactrlbsup A\isactrlesup {\isachardoublequoteclose}\isanewline
\ \ \isacommand{then}\isamarkupfalse%
\ \isacommand{have}\isamarkupfalse%
\ uniqueness{\isacharcolon}{\kern0pt}\ {\isachardoublequoteopen}{\isasymforall}\ g{\isachardot}{\kern0pt}\ g\ {\isacharcolon}{\kern0pt}\ Z\ {\isasymrightarrow}\ X\isactrlbsup A\isactrlesup \ {\isasymlongrightarrow}\ eval{\isacharunderscore}{\kern0pt}func\ X\ A\ {\isasymcirc}\isactrlsub c\ {\isacharparenleft}{\kern0pt}id\ A\ {\isasymtimes}\isactrlsub f\ g{\isacharparenright}{\kern0pt}\ {\isacharequal}{\kern0pt}\ f\isactrlsup {\isasymflat}\ {\isasymlongrightarrow}\ g\ {\isacharequal}{\kern0pt}\ {\isacharparenleft}{\kern0pt}f\isactrlsup {\isasymflat}{\isacharparenright}{\kern0pt}\isactrlsup {\isasymsharp}{\isachardoublequoteclose}\isanewline
\ \ \ \ \isacommand{by}\isamarkupfalse%
\ {\isacharparenleft}{\kern0pt}typecheck{\isacharunderscore}{\kern0pt}cfuncs{\isacharcomma}{\kern0pt}\ simp\ add{\isacharcolon}{\kern0pt}\ transpose{\isacharunderscore}{\kern0pt}func{\isacharunderscore}{\kern0pt}unique{\isacharparenright}{\kern0pt}\isanewline
\ \ \isacommand{have}\isamarkupfalse%
\ {\isachardoublequoteopen}eval{\isacharunderscore}{\kern0pt}func\ X\ A\ {\isasymcirc}\isactrlsub c\ {\isacharparenleft}{\kern0pt}id\ A\ {\isasymtimes}\isactrlsub f\ f{\isacharparenright}{\kern0pt}\ {\isacharequal}{\kern0pt}\ f\isactrlsup {\isasymflat}{\isachardoublequoteclose}\isanewline
\ \ \ \ \isacommand{by}\isamarkupfalse%
\ {\isacharparenleft}{\kern0pt}metis\ f{\isacharunderscore}{\kern0pt}type\ inv{\isacharunderscore}{\kern0pt}transpose{\isacharunderscore}{\kern0pt}func{\isacharunderscore}{\kern0pt}def{\isadigit{3}}{\isacharparenright}{\kern0pt}\isanewline
\ \ \isacommand{then}\isamarkupfalse%
\ \isacommand{show}\isamarkupfalse%
\ {\isachardoublequoteopen}f\isactrlsup {\isasymflat}\isactrlsup {\isasymsharp}\ {\isacharequal}{\kern0pt}\ f{\isachardoublequoteclose}\isanewline
\ \ \ \ \isacommand{using}\isamarkupfalse%
\ f{\isacharunderscore}{\kern0pt}type\ uniqueness\ \isacommand{by}\isamarkupfalse%
\ auto\isanewline
\isacommand{qed}\isamarkupfalse%
%
\endisatagproof
{\isafoldproof}%
%
\isadelimproof
\isanewline
%
\endisadelimproof
\isanewline
\isacommand{lemma}\isamarkupfalse%
\ same{\isacharunderscore}{\kern0pt}evals{\isacharunderscore}{\kern0pt}equal{\isacharcolon}{\kern0pt}\isanewline
\ \ \isakeyword{assumes}\ {\isachardoublequoteopen}f\ {\isacharcolon}{\kern0pt}\ Z\ {\isasymrightarrow}\ X\isactrlbsup A\isactrlesup {\isachardoublequoteclose}\ {\isachardoublequoteopen}g{\isacharcolon}{\kern0pt}\ Z\ {\isasymrightarrow}\ X\isactrlbsup A\isactrlesup {\isachardoublequoteclose}\isanewline
\ \ \isakeyword{shows}\ {\isachardoublequoteopen}eval{\isacharunderscore}{\kern0pt}func\ X\ A\ {\isasymcirc}\isactrlsub c\ {\isacharparenleft}{\kern0pt}id\ A\ {\isasymtimes}\isactrlsub f\ f{\isacharparenright}{\kern0pt}\ {\isacharequal}{\kern0pt}\ eval{\isacharunderscore}{\kern0pt}func\ X\ A\ {\isasymcirc}\isactrlsub c\ {\isacharparenleft}{\kern0pt}id\ A\ {\isasymtimes}\isactrlsub f\ g{\isacharparenright}{\kern0pt}\ {\isasymLongrightarrow}\ f\ {\isacharequal}{\kern0pt}\ g{\isachardoublequoteclose}\isanewline
%
\isadelimproof
\ \ %
\endisadelimproof
%
\isatagproof
\isacommand{by}\isamarkupfalse%
\ {\isacharparenleft}{\kern0pt}metis\ assms\ inv{\isacharunderscore}{\kern0pt}transpose{\isacharunderscore}{\kern0pt}func{\isacharunderscore}{\kern0pt}def{\isadigit{3}}\ sharp{\isacharunderscore}{\kern0pt}cancels{\isacharunderscore}{\kern0pt}flat{\isacharparenright}{\kern0pt}%
\endisatagproof
{\isafoldproof}%
%
\isadelimproof
\isanewline
%
\endisadelimproof
\isanewline
\isacommand{lemma}\isamarkupfalse%
\ sharp{\isacharunderscore}{\kern0pt}comp{\isacharcolon}{\kern0pt}\isanewline
\ \ \isakeyword{assumes}\ f{\isacharunderscore}{\kern0pt}type{\isacharbrackleft}{\kern0pt}type{\isacharunderscore}{\kern0pt}rule{\isacharbrackright}{\kern0pt}{\isacharcolon}{\kern0pt}\ {\isachardoublequoteopen}f\ {\isacharcolon}{\kern0pt}\ A\ {\isasymtimes}\isactrlsub c\ Z\ {\isasymrightarrow}\ X{\isachardoublequoteclose}\ \isakeyword{and}\ g{\isacharunderscore}{\kern0pt}type{\isacharbrackleft}{\kern0pt}type{\isacharunderscore}{\kern0pt}rule{\isacharbrackright}{\kern0pt}{\isacharcolon}{\kern0pt}\ {\isachardoublequoteopen}g\ {\isacharcolon}{\kern0pt}\ W\ {\isasymrightarrow}\ Z{\isachardoublequoteclose}\isanewline
\ \ \isakeyword{shows}\ {\isachardoublequoteopen}f\isactrlsup {\isasymsharp}\ {\isasymcirc}\isactrlsub c\ g\ {\isacharequal}{\kern0pt}\ {\isacharparenleft}{\kern0pt}f\ {\isasymcirc}\isactrlsub c\ {\isacharparenleft}{\kern0pt}id\ A\ {\isasymtimes}\isactrlsub f\ g{\isacharparenright}{\kern0pt}{\isacharparenright}{\kern0pt}\isactrlsup {\isasymsharp}{\isachardoublequoteclose}\isanewline
%
\isadelimproof
%
\endisadelimproof
%
\isatagproof
\isacommand{proof}\isamarkupfalse%
\ {\isacharparenleft}{\kern0pt}etcs{\isacharunderscore}{\kern0pt}rule\ same{\isacharunderscore}{\kern0pt}evals{\isacharunderscore}{\kern0pt}equal{\isacharbrackleft}{\kern0pt}\isakeyword{where}\ X{\isacharequal}{\kern0pt}X{\isacharcomma}{\kern0pt}\ \isakeyword{where}\ A{\isacharequal}{\kern0pt}A{\isacharbrackright}{\kern0pt}{\isacharparenright}{\kern0pt}\isanewline
\isanewline
\ \ \isacommand{have}\isamarkupfalse%
\ {\isachardoublequoteopen}eval{\isacharunderscore}{\kern0pt}func\ X\ A\ {\isasymcirc}\isactrlsub c\ {\isacharparenleft}{\kern0pt}id\ A\ {\isasymtimes}\isactrlsub f\ {\isacharparenleft}{\kern0pt}f\isactrlsup {\isasymsharp}\ {\isasymcirc}\isactrlsub c\ g{\isacharparenright}{\kern0pt}{\isacharparenright}{\kern0pt}\ {\isacharequal}{\kern0pt}\ eval{\isacharunderscore}{\kern0pt}func\ X\ A\ {\isasymcirc}\isactrlsub c\ {\isacharparenleft}{\kern0pt}id\ A\ {\isasymtimes}\isactrlsub f\ f\isactrlsup {\isasymsharp}{\isacharparenright}{\kern0pt}\ {\isasymcirc}\isactrlsub c\ {\isacharparenleft}{\kern0pt}id\ A\ {\isasymtimes}\isactrlsub f\ g{\isacharparenright}{\kern0pt}{\isachardoublequoteclose}\isanewline
\ \ \ \ \isacommand{using}\isamarkupfalse%
\ assms\ \isacommand{by}\isamarkupfalse%
\ {\isacharparenleft}{\kern0pt}typecheck{\isacharunderscore}{\kern0pt}cfuncs{\isacharcomma}{\kern0pt}\ simp\ add{\isacharcolon}{\kern0pt}\ identity{\isacharunderscore}{\kern0pt}distributes{\isacharunderscore}{\kern0pt}across{\isacharunderscore}{\kern0pt}composition{\isacharparenright}{\kern0pt}\isanewline
\ \ \isacommand{also}\isamarkupfalse%
\ \isacommand{have}\isamarkupfalse%
\ {\isachardoublequoteopen}{\isachardot}{\kern0pt}{\isachardot}{\kern0pt}{\isachardot}{\kern0pt}\ {\isacharequal}{\kern0pt}\ f\ {\isasymcirc}\isactrlsub c\ {\isacharparenleft}{\kern0pt}id\ A\ {\isasymtimes}\isactrlsub f\ g{\isacharparenright}{\kern0pt}{\isachardoublequoteclose}\isanewline
\ \ \ \ \isacommand{using}\isamarkupfalse%
\ assms\ \isacommand{by}\isamarkupfalse%
\ {\isacharparenleft}{\kern0pt}typecheck{\isacharunderscore}{\kern0pt}cfuncs{\isacharcomma}{\kern0pt}\ simp\ add{\isacharcolon}{\kern0pt}\ comp{\isacharunderscore}{\kern0pt}associative{\isadigit{2}}\ transpose{\isacharunderscore}{\kern0pt}func{\isacharunderscore}{\kern0pt}def{\isacharparenright}{\kern0pt}\isanewline
\ \ \isacommand{also}\isamarkupfalse%
\ \isacommand{have}\isamarkupfalse%
\ {\isachardoublequoteopen}{\isachardot}{\kern0pt}{\isachardot}{\kern0pt}{\isachardot}{\kern0pt}\ {\isacharequal}{\kern0pt}\ eval{\isacharunderscore}{\kern0pt}func\ X\ A\ {\isasymcirc}\isactrlsub c\ {\isacharparenleft}{\kern0pt}id\isactrlsub c\ A\ {\isasymtimes}\isactrlsub f\ {\isacharparenleft}{\kern0pt}f\ {\isasymcirc}\isactrlsub c\ {\isacharparenleft}{\kern0pt}id\isactrlsub c\ A\ {\isasymtimes}\isactrlsub f\ g{\isacharparenright}{\kern0pt}{\isacharparenright}{\kern0pt}\isactrlsup {\isasymsharp}{\isacharparenright}{\kern0pt}{\isachardoublequoteclose}\isanewline
\ \ \ \ \isacommand{using}\isamarkupfalse%
\ assms\ \isacommand{by}\isamarkupfalse%
\ {\isacharparenleft}{\kern0pt}typecheck{\isacharunderscore}{\kern0pt}cfuncs{\isacharcomma}{\kern0pt}\ simp\ add{\isacharcolon}{\kern0pt}\ transpose{\isacharunderscore}{\kern0pt}func{\isacharunderscore}{\kern0pt}def{\isacharparenright}{\kern0pt}\isanewline
\ \ \isacommand{then}\isamarkupfalse%
\ \isacommand{show}\isamarkupfalse%
\ {\isachardoublequoteopen}eval{\isacharunderscore}{\kern0pt}func\ X\ A\ {\isasymcirc}\isactrlsub c\ {\isacharparenleft}{\kern0pt}id\ A\ {\isasymtimes}\isactrlsub f\ {\isacharparenleft}{\kern0pt}f\isactrlsup {\isasymsharp}\ {\isasymcirc}\isactrlsub c\ g{\isacharparenright}{\kern0pt}{\isacharparenright}{\kern0pt}\ {\isacharequal}{\kern0pt}\ eval{\isacharunderscore}{\kern0pt}func\ X\ A\ {\isasymcirc}\isactrlsub c\ {\isacharparenleft}{\kern0pt}id\isactrlsub c\ A\ {\isasymtimes}\isactrlsub f\ {\isacharparenleft}{\kern0pt}f\ {\isasymcirc}\isactrlsub c\ {\isacharparenleft}{\kern0pt}id\isactrlsub c\ A\ {\isasymtimes}\isactrlsub f\ g{\isacharparenright}{\kern0pt}{\isacharparenright}{\kern0pt}\isactrlsup {\isasymsharp}{\isacharparenright}{\kern0pt}{\isachardoublequoteclose}\isanewline
\ \ \ \ \isacommand{using}\isamarkupfalse%
\ calculation\ \isacommand{by}\isamarkupfalse%
\ auto\isanewline
\isacommand{qed}\isamarkupfalse%
%
\endisatagproof
{\isafoldproof}%
%
\isadelimproof
\isanewline
%
\endisadelimproof
\isanewline
\isacommand{lemma}\isamarkupfalse%
\ flat{\isacharunderscore}{\kern0pt}pres{\isacharunderscore}{\kern0pt}epi{\isacharcolon}{\kern0pt}\isanewline
\ \ \isakeyword{assumes}\ {\isachardoublequoteopen}nonempty{\isacharparenleft}{\kern0pt}A{\isacharparenright}{\kern0pt}{\isachardoublequoteclose}\isanewline
\ \ \isakeyword{assumes}\ {\isachardoublequoteopen}f\ {\isacharcolon}{\kern0pt}\ Z\ {\isasymrightarrow}\ X\isactrlbsup A\isactrlesup {\isachardoublequoteclose}\isanewline
\ \ \isakeyword{assumes}\ {\isachardoublequoteopen}epimorphism\ f{\isachardoublequoteclose}\isanewline
\ \ \isakeyword{shows}\ {\isachardoublequoteopen}epimorphism{\isacharparenleft}{\kern0pt}f\isactrlsup {\isasymflat}{\isacharparenright}{\kern0pt}{\isachardoublequoteclose}\isanewline
%
\isadelimproof
%
\endisadelimproof
%
\isatagproof
\isacommand{proof}\isamarkupfalse%
\ {\isacharminus}{\kern0pt}\ \isanewline
\ \ \isacommand{have}\isamarkupfalse%
\ equals{\isacharcolon}{\kern0pt}\ {\isachardoublequoteopen}f\isactrlsup {\isasymflat}\ {\isacharequal}{\kern0pt}\ {\isacharparenleft}{\kern0pt}eval{\isacharunderscore}{\kern0pt}func\ X\ A{\isacharparenright}{\kern0pt}\ {\isasymcirc}\isactrlsub c\ {\isacharparenleft}{\kern0pt}id{\isacharparenleft}{\kern0pt}A{\isacharparenright}{\kern0pt}\ {\isasymtimes}\isactrlsub f\ f{\isacharparenright}{\kern0pt}{\isachardoublequoteclose}\isanewline
\ \ \ \ \isacommand{using}\isamarkupfalse%
\ assms{\isacharparenleft}{\kern0pt}{\isadigit{2}}{\isacharparenright}{\kern0pt}\ inv{\isacharunderscore}{\kern0pt}transpose{\isacharunderscore}{\kern0pt}func{\isacharunderscore}{\kern0pt}def{\isadigit{3}}\ \isacommand{by}\isamarkupfalse%
\ auto\isanewline
\ \ \isacommand{have}\isamarkupfalse%
\ idA{\isacharunderscore}{\kern0pt}f{\isacharunderscore}{\kern0pt}epi{\isacharcolon}{\kern0pt}\ {\isachardoublequoteopen}epimorphism{\isacharparenleft}{\kern0pt}{\isacharparenleft}{\kern0pt}id{\isacharparenleft}{\kern0pt}A{\isacharparenright}{\kern0pt}\ {\isasymtimes}\isactrlsub f\ f{\isacharparenright}{\kern0pt}{\isacharparenright}{\kern0pt}{\isachardoublequoteclose}\isanewline
\ \ \ \ \isacommand{using}\isamarkupfalse%
\ assms{\isacharparenleft}{\kern0pt}{\isadigit{2}}{\isacharparenright}{\kern0pt}\ assms{\isacharparenleft}{\kern0pt}{\isadigit{3}}{\isacharparenright}{\kern0pt}\ cfunc{\isacharunderscore}{\kern0pt}cross{\isacharunderscore}{\kern0pt}prod{\isacharunderscore}{\kern0pt}surj\ epi{\isacharunderscore}{\kern0pt}is{\isacharunderscore}{\kern0pt}surj\ id{\isacharunderscore}{\kern0pt}isomorphism\ id{\isacharunderscore}{\kern0pt}type\ iso{\isacharunderscore}{\kern0pt}imp{\isacharunderscore}{\kern0pt}epi{\isacharunderscore}{\kern0pt}and{\isacharunderscore}{\kern0pt}monic\ surjective{\isacharunderscore}{\kern0pt}is{\isacharunderscore}{\kern0pt}epimorphism\ \isacommand{by}\isamarkupfalse%
\ blast\isanewline
\ \ \isacommand{have}\isamarkupfalse%
\ eval{\isacharunderscore}{\kern0pt}epi{\isacharcolon}{\kern0pt}\ {\isachardoublequoteopen}epimorphism{\isacharparenleft}{\kern0pt}{\isacharparenleft}{\kern0pt}eval{\isacharunderscore}{\kern0pt}func\ X\ A{\isacharparenright}{\kern0pt}{\isacharparenright}{\kern0pt}{\isachardoublequoteclose}\isanewline
\ \ \ \ \isacommand{by}\isamarkupfalse%
\ {\isacharparenleft}{\kern0pt}simp\ add{\isacharcolon}{\kern0pt}\ assms{\isacharparenleft}{\kern0pt}{\isadigit{1}}{\isacharparenright}{\kern0pt}\ eval{\isacharunderscore}{\kern0pt}func{\isacharunderscore}{\kern0pt}surj\ surjective{\isacharunderscore}{\kern0pt}is{\isacharunderscore}{\kern0pt}epimorphism{\isacharparenright}{\kern0pt}\isanewline
\ \ \isacommand{have}\isamarkupfalse%
\ {\isachardoublequoteopen}codomain\ {\isacharparenleft}{\kern0pt}{\isacharparenleft}{\kern0pt}id{\isacharparenleft}{\kern0pt}A{\isacharparenright}{\kern0pt}\ {\isasymtimes}\isactrlsub f\ f{\isacharparenright}{\kern0pt}{\isacharparenright}{\kern0pt}\ {\isacharequal}{\kern0pt}\ domain\ {\isacharparenleft}{\kern0pt}{\isacharparenleft}{\kern0pt}eval{\isacharunderscore}{\kern0pt}func\ X\ A{\isacharparenright}{\kern0pt}{\isacharparenright}{\kern0pt}{\isachardoublequoteclose}\isanewline
\ \ \ \ \isacommand{using}\isamarkupfalse%
\ assms{\isacharparenleft}{\kern0pt}{\isadigit{2}}{\isacharparenright}{\kern0pt}\ cfunc{\isacharunderscore}{\kern0pt}type{\isacharunderscore}{\kern0pt}def\ \isacommand{by}\isamarkupfalse%
\ {\isacharparenleft}{\kern0pt}typecheck{\isacharunderscore}{\kern0pt}cfuncs{\isacharcomma}{\kern0pt}\ auto{\isacharparenright}{\kern0pt}\isanewline
\ \ \isacommand{then}\isamarkupfalse%
\ \isacommand{show}\isamarkupfalse%
\ {\isacharquery}{\kern0pt}thesis\isanewline
\ \ \ \ \isacommand{by}\isamarkupfalse%
\ {\isacharparenleft}{\kern0pt}simp\ add{\isacharcolon}{\kern0pt}\ composition{\isacharunderscore}{\kern0pt}of{\isacharunderscore}{\kern0pt}epi{\isacharunderscore}{\kern0pt}pair{\isacharunderscore}{\kern0pt}is{\isacharunderscore}{\kern0pt}epi\ equals\ eval{\isacharunderscore}{\kern0pt}epi\ idA{\isacharunderscore}{\kern0pt}f{\isacharunderscore}{\kern0pt}epi{\isacharparenright}{\kern0pt}\isanewline
\isacommand{qed}\isamarkupfalse%
%
\endisatagproof
{\isafoldproof}%
%
\isadelimproof
\isanewline
%
\endisadelimproof
\isanewline
\isacommand{lemma}\isamarkupfalse%
\ transpose{\isacharunderscore}{\kern0pt}inj{\isacharunderscore}{\kern0pt}is{\isacharunderscore}{\kern0pt}inj{\isacharcolon}{\kern0pt}\isanewline
\ \ \isakeyword{assumes}\ {\isachardoublequoteopen}g{\isacharcolon}{\kern0pt}\ X\ {\isasymrightarrow}\ Y{\isachardoublequoteclose}\isanewline
\ \ \isakeyword{assumes}\ {\isachardoublequoteopen}injective\ g{\isachardoublequoteclose}\isanewline
\ \ \isakeyword{shows}\ {\isachardoublequoteopen}injective{\isacharparenleft}{\kern0pt}g\isactrlbsup A\isactrlesup \isactrlsub f{\isacharparenright}{\kern0pt}{\isachardoublequoteclose}\isanewline
%
\isadelimproof
\ \ %
\endisadelimproof
%
\isatagproof
\isacommand{unfolding}\isamarkupfalse%
\ injective{\isacharunderscore}{\kern0pt}def\isanewline
\isacommand{proof}\isamarkupfalse%
{\isacharparenleft}{\kern0pt}clarify{\isacharparenright}{\kern0pt}\isanewline
\ \ \isacommand{fix}\isamarkupfalse%
\ x\ y\ \isanewline
\ \ \isacommand{assume}\isamarkupfalse%
\ x{\isacharunderscore}{\kern0pt}type{\isacharbrackleft}{\kern0pt}type{\isacharunderscore}{\kern0pt}rule{\isacharbrackright}{\kern0pt}{\isacharcolon}{\kern0pt}\ {\isachardoublequoteopen}x\ {\isasymin}\isactrlsub c\ domain\ {\isacharparenleft}{\kern0pt}g\isactrlbsup A\isactrlesup \isactrlsub f{\isacharparenright}{\kern0pt}{\isachardoublequoteclose}\ \isanewline
\ \ \isacommand{assume}\isamarkupfalse%
\ y{\isacharunderscore}{\kern0pt}type{\isacharbrackleft}{\kern0pt}type{\isacharunderscore}{\kern0pt}rule{\isacharbrackright}{\kern0pt}{\isacharcolon}{\kern0pt}{\isachardoublequoteopen}y\ {\isasymin}\isactrlsub c\ domain\ {\isacharparenleft}{\kern0pt}g\isactrlbsup A\isactrlesup \isactrlsub f{\isacharparenright}{\kern0pt}{\isachardoublequoteclose}\isanewline
\ \ \isacommand{assume}\isamarkupfalse%
\ eqs{\isacharcolon}{\kern0pt}\ {\isachardoublequoteopen}g\isactrlbsup A\isactrlesup \isactrlsub f\ {\isasymcirc}\isactrlsub c\ x\ {\isacharequal}{\kern0pt}\ g\isactrlbsup A\isactrlesup \isactrlsub f\ {\isasymcirc}\isactrlsub c\ y{\isachardoublequoteclose}\isanewline
\ \ \isacommand{have}\isamarkupfalse%
\ mono{\isacharunderscore}{\kern0pt}g{\isacharcolon}{\kern0pt}\ {\isachardoublequoteopen}monomorphism\ g{\isachardoublequoteclose}\isanewline
\ \ \ \ \isacommand{by}\isamarkupfalse%
\ {\isacharparenleft}{\kern0pt}meson\ CollectI\ assms{\isacharparenleft}{\kern0pt}{\isadigit{2}}{\isacharparenright}{\kern0pt}\ injective{\isacharunderscore}{\kern0pt}imp{\isacharunderscore}{\kern0pt}monomorphism{\isacharparenright}{\kern0pt}\ \isanewline
\ \ \isacommand{have}\isamarkupfalse%
\ x{\isacharunderscore}{\kern0pt}type{\isacharprime}{\kern0pt}{\isacharbrackleft}{\kern0pt}type{\isacharunderscore}{\kern0pt}rule{\isacharbrackright}{\kern0pt}{\isacharcolon}{\kern0pt}\ {\isachardoublequoteopen}x\ {\isasymin}\isactrlsub c\ \ X\isactrlbsup A\isactrlesup {\isachardoublequoteclose}\isanewline
\ \ \ \ \isacommand{using}\isamarkupfalse%
\ assms{\isacharparenleft}{\kern0pt}{\isadigit{1}}{\isacharparenright}{\kern0pt}\ cfunc{\isacharunderscore}{\kern0pt}type{\isacharunderscore}{\kern0pt}def\ exp{\isacharunderscore}{\kern0pt}func{\isacharunderscore}{\kern0pt}type\ \isacommand{by}\isamarkupfalse%
\ {\isacharparenleft}{\kern0pt}typecheck{\isacharunderscore}{\kern0pt}cfuncs{\isacharcomma}{\kern0pt}\ force{\isacharparenright}{\kern0pt}\isanewline
\ \ \isacommand{have}\isamarkupfalse%
\ y{\isacharunderscore}{\kern0pt}type{\isacharprime}{\kern0pt}{\isacharbrackleft}{\kern0pt}type{\isacharunderscore}{\kern0pt}rule{\isacharbrackright}{\kern0pt}{\isacharcolon}{\kern0pt}\ {\isachardoublequoteopen}y\ {\isasymin}\isactrlsub c\ \ X\isactrlbsup A\isactrlesup {\isachardoublequoteclose}\isanewline
\ \ \ \ \isacommand{using}\isamarkupfalse%
\ cfunc{\isacharunderscore}{\kern0pt}type{\isacharunderscore}{\kern0pt}def\ x{\isacharunderscore}{\kern0pt}type\ x{\isacharunderscore}{\kern0pt}type{\isacharprime}{\kern0pt}\ y{\isacharunderscore}{\kern0pt}type\ \isacommand{by}\isamarkupfalse%
\ presburger\ \ \isanewline
\ \ \isacommand{have}\isamarkupfalse%
\ {\isachardoublequoteopen}{\isacharparenleft}{\kern0pt}g\ {\isasymcirc}\isactrlsub c\ eval{\isacharunderscore}{\kern0pt}func\ X\ A{\isacharparenright}{\kern0pt}\isactrlsup {\isasymsharp}\ {\isasymcirc}\isactrlsub c\ x\ {\isacharequal}{\kern0pt}\ {\isacharparenleft}{\kern0pt}g\ {\isasymcirc}\isactrlsub c\ eval{\isacharunderscore}{\kern0pt}func\ X\ A{\isacharparenright}{\kern0pt}\isactrlsup {\isasymsharp}\ {\isasymcirc}\isactrlsub c\ y{\isachardoublequoteclose}\isanewline
\ \ \ \ \isacommand{unfolding}\isamarkupfalse%
\ exp{\isacharunderscore}{\kern0pt}func{\isacharunderscore}{\kern0pt}def\ \isacommand{using}\isamarkupfalse%
\ assms\ eqs\ exp{\isacharunderscore}{\kern0pt}func{\isacharunderscore}{\kern0pt}def{\isadigit{2}}\ \isacommand{by}\isamarkupfalse%
\ force\ \isanewline
\ \ \isacommand{then}\isamarkupfalse%
\ \isacommand{have}\isamarkupfalse%
\ {\isachardoublequoteopen}g\ {\isasymcirc}\isactrlsub c\ {\isacharparenleft}{\kern0pt}eval{\isacharunderscore}{\kern0pt}func\ X\ A\ {\isasymcirc}\isactrlsub c{\isacharparenleft}{\kern0pt}id{\isacharparenleft}{\kern0pt}A{\isacharparenright}{\kern0pt}\ {\isasymtimes}\isactrlsub f\ \ x{\isacharparenright}{\kern0pt}{\isacharparenright}{\kern0pt}\ {\isacharequal}{\kern0pt}\ g\ {\isasymcirc}\isactrlsub c\ {\isacharparenleft}{\kern0pt}eval{\isacharunderscore}{\kern0pt}func\ X\ A\ {\isasymcirc}\isactrlsub c\ {\isacharparenleft}{\kern0pt}id{\isacharparenleft}{\kern0pt}A{\isacharparenright}{\kern0pt}\ {\isasymtimes}\isactrlsub f\ \ y{\isacharparenright}{\kern0pt}{\isacharparenright}{\kern0pt}{\isachardoublequoteclose}\isanewline
\ \ \ \ \isacommand{by}\isamarkupfalse%
\ {\isacharparenleft}{\kern0pt}smt\ {\isacharparenleft}{\kern0pt}z{\isadigit{3}}{\isacharparenright}{\kern0pt}\ assms{\isacharparenleft}{\kern0pt}{\isadigit{1}}{\isacharparenright}{\kern0pt}\ comp{\isacharunderscore}{\kern0pt}type\ eqs\ flat{\isacharunderscore}{\kern0pt}cancels{\isacharunderscore}{\kern0pt}sharp\ flat{\isacharunderscore}{\kern0pt}type\ inv{\isacharunderscore}{\kern0pt}transpose{\isacharunderscore}{\kern0pt}func{\isacharunderscore}{\kern0pt}def{\isadigit{3}}\ sharp{\isacharunderscore}{\kern0pt}cancels{\isacharunderscore}{\kern0pt}flat\ transpose{\isacharunderscore}{\kern0pt}of{\isacharunderscore}{\kern0pt}comp\ x{\isacharunderscore}{\kern0pt}type{\isacharprime}{\kern0pt}\ y{\isacharunderscore}{\kern0pt}type{\isacharprime}{\kern0pt}{\isacharparenright}{\kern0pt}\isanewline
\ \ \isacommand{then}\isamarkupfalse%
\ \isacommand{have}\isamarkupfalse%
\ {\isachardoublequoteopen}eval{\isacharunderscore}{\kern0pt}func\ X\ A\ {\isasymcirc}\isactrlsub c{\isacharparenleft}{\kern0pt}id{\isacharparenleft}{\kern0pt}A{\isacharparenright}{\kern0pt}\ {\isasymtimes}\isactrlsub f\ \ x{\isacharparenright}{\kern0pt}\ {\isacharequal}{\kern0pt}\ \ \ eval{\isacharunderscore}{\kern0pt}func\ X\ A\ {\isasymcirc}\isactrlsub c\ {\isacharparenleft}{\kern0pt}id{\isacharparenleft}{\kern0pt}A{\isacharparenright}{\kern0pt}\ {\isasymtimes}\isactrlsub f\ \ y{\isacharparenright}{\kern0pt}{\isachardoublequoteclose}\ \ \isanewline
\ \ \ \ \isacommand{by}\isamarkupfalse%
\ {\isacharparenleft}{\kern0pt}metis\ assms{\isacharparenleft}{\kern0pt}{\isadigit{1}}{\isacharparenright}{\kern0pt}\ mono{\isacharunderscore}{\kern0pt}g\ flat{\isacharunderscore}{\kern0pt}type\ inv{\isacharunderscore}{\kern0pt}transpose{\isacharunderscore}{\kern0pt}func{\isacharunderscore}{\kern0pt}def{\isadigit{3}}\ \ monomorphism{\isacharunderscore}{\kern0pt}def{\isadigit{2}}\ x{\isacharunderscore}{\kern0pt}type{\isacharprime}{\kern0pt}\ y{\isacharunderscore}{\kern0pt}type{\isacharprime}{\kern0pt}{\isacharparenright}{\kern0pt}\isanewline
\ \ \isacommand{then}\isamarkupfalse%
\ \isacommand{show}\isamarkupfalse%
\ {\isachardoublequoteopen}x\ {\isacharequal}{\kern0pt}\ y{\isachardoublequoteclose}\isanewline
\ \ \ \ \isacommand{by}\isamarkupfalse%
\ {\isacharparenleft}{\kern0pt}meson\ same{\isacharunderscore}{\kern0pt}evals{\isacharunderscore}{\kern0pt}equal\ x{\isacharunderscore}{\kern0pt}type{\isacharprime}{\kern0pt}\ y{\isacharunderscore}{\kern0pt}type{\isacharprime}{\kern0pt}{\isacharparenright}{\kern0pt}\isanewline
\isacommand{qed}\isamarkupfalse%
%
\endisatagproof
{\isafoldproof}%
%
\isadelimproof
\isanewline
%
\endisadelimproof
\isanewline
\isacommand{lemma}\isamarkupfalse%
\ eval{\isacharunderscore}{\kern0pt}func{\isacharunderscore}{\kern0pt}X{\isacharunderscore}{\kern0pt}one{\isacharunderscore}{\kern0pt}injective{\isacharcolon}{\kern0pt}\isanewline
\ \ {\isachardoublequoteopen}injective\ {\isacharparenleft}{\kern0pt}eval{\isacharunderscore}{\kern0pt}func\ X\ {\isasymone}{\isacharparenright}{\kern0pt}{\isachardoublequoteclose}\isanewline
%
\isadelimproof
%
\endisadelimproof
%
\isatagproof
\isacommand{proof}\isamarkupfalse%
\ {\isacharparenleft}{\kern0pt}cases\ {\isachardoublequoteopen}{\isasymexists}\ x{\isachardot}{\kern0pt}\ x\ {\isasymin}\isactrlsub c\ X{\isachardoublequoteclose}{\isacharparenright}{\kern0pt}\isanewline
\ \ \isacommand{assume}\isamarkupfalse%
\ {\isachardoublequoteopen}{\isasymexists}x{\isachardot}{\kern0pt}\ x\ {\isasymin}\isactrlsub c\ X{\isachardoublequoteclose}\isanewline
\ \ \isacommand{then}\isamarkupfalse%
\ \isacommand{obtain}\isamarkupfalse%
\ x\ \isakeyword{where}\ x{\isacharunderscore}{\kern0pt}type{\isacharcolon}{\kern0pt}\ {\isachardoublequoteopen}x\ {\isasymin}\isactrlsub c\ X{\isachardoublequoteclose}\isanewline
\ \ \ \ \isacommand{by}\isamarkupfalse%
\ auto\isanewline
\ \ \isacommand{then}\isamarkupfalse%
\ \isacommand{have}\isamarkupfalse%
\ {\isachardoublequoteopen}eval{\isacharunderscore}{\kern0pt}func\ X\ {\isasymone}\ {\isasymcirc}\isactrlsub c\ id\isactrlsub c\ {\isasymone}\ {\isasymtimes}\isactrlsub f\ {\isacharparenleft}{\kern0pt}x\ {\isasymcirc}\isactrlsub c\ {\isasymbeta}\isactrlbsub {\isasymone}\ {\isasymtimes}\isactrlsub c\ {\isasymone}\isactrlesub {\isacharparenright}{\kern0pt}\isactrlsup {\isasymsharp}\ {\isacharequal}{\kern0pt}\ x\ {\isasymcirc}\isactrlsub c\ {\isasymbeta}\isactrlbsub {\isasymone}\ {\isasymtimes}\isactrlsub c\ {\isasymone}\isactrlesub {\isachardoublequoteclose}\isanewline
\ \ \ \ \isacommand{using}\isamarkupfalse%
\ comp{\isacharunderscore}{\kern0pt}type\ terminal{\isacharunderscore}{\kern0pt}func{\isacharunderscore}{\kern0pt}type\ transpose{\isacharunderscore}{\kern0pt}func{\isacharunderscore}{\kern0pt}def\ \isacommand{by}\isamarkupfalse%
\ blast\isanewline
\ \ \isanewline
\ \ \isacommand{show}\isamarkupfalse%
\ {\isachardoublequoteopen}injective\ {\isacharparenleft}{\kern0pt}eval{\isacharunderscore}{\kern0pt}func\ X\ {\isasymone}{\isacharparenright}{\kern0pt}{\isachardoublequoteclose}\isanewline
\ \ \ \ \isacommand{unfolding}\isamarkupfalse%
\ injective{\isacharunderscore}{\kern0pt}def\isanewline
\ \ \isacommand{proof}\isamarkupfalse%
\ clarify\isanewline
\ \ \ \ \isacommand{fix}\isamarkupfalse%
\ a\ b\isanewline
\ \ \ \ \isacommand{assume}\isamarkupfalse%
\ a{\isacharunderscore}{\kern0pt}type{\isacharcolon}{\kern0pt}\ {\isachardoublequoteopen}a\ {\isasymin}\isactrlsub c\ domain\ {\isacharparenleft}{\kern0pt}eval{\isacharunderscore}{\kern0pt}func\ X\ {\isasymone}{\isacharparenright}{\kern0pt}{\isachardoublequoteclose}\isanewline
\ \ \ \ \isacommand{assume}\isamarkupfalse%
\ b{\isacharunderscore}{\kern0pt}type{\isacharcolon}{\kern0pt}\ {\isachardoublequoteopen}b\ {\isasymin}\isactrlsub c\ domain\ {\isacharparenleft}{\kern0pt}eval{\isacharunderscore}{\kern0pt}func\ X\ {\isasymone}{\isacharparenright}{\kern0pt}{\isachardoublequoteclose}\isanewline
\ \ \ \ \isacommand{assume}\isamarkupfalse%
\ evals{\isacharunderscore}{\kern0pt}equal{\isacharcolon}{\kern0pt}\ {\isachardoublequoteopen}eval{\isacharunderscore}{\kern0pt}func\ X\ {\isasymone}\ {\isasymcirc}\isactrlsub c\ a\ {\isacharequal}{\kern0pt}\ eval{\isacharunderscore}{\kern0pt}func\ X\ {\isasymone}\ {\isasymcirc}\isactrlsub c\ b{\isachardoublequoteclose}\isanewline
\isanewline
\ \ \ \ \isacommand{have}\isamarkupfalse%
\ eval{\isacharunderscore}{\kern0pt}dom{\isacharcolon}{\kern0pt}\ {\isachardoublequoteopen}domain{\isacharparenleft}{\kern0pt}eval{\isacharunderscore}{\kern0pt}func\ X\ {\isasymone}{\isacharparenright}{\kern0pt}\ {\isacharequal}{\kern0pt}\ {\isasymone}\ {\isasymtimes}\isactrlsub c\ {\isacharparenleft}{\kern0pt}X\isactrlbsup {\isasymone}\isactrlesup {\isacharparenright}{\kern0pt}{\isachardoublequoteclose}\isanewline
\ \ \ \ \ \ \isacommand{using}\isamarkupfalse%
\ cfunc{\isacharunderscore}{\kern0pt}type{\isacharunderscore}{\kern0pt}def\ eval{\isacharunderscore}{\kern0pt}func{\isacharunderscore}{\kern0pt}type\ \isacommand{by}\isamarkupfalse%
\ auto\isanewline
\isanewline
\ \ \ \ \isacommand{obtain}\isamarkupfalse%
\ A\ \isakeyword{where}\ a{\isacharunderscore}{\kern0pt}def{\isacharcolon}{\kern0pt}\ {\isachardoublequoteopen}A\ {\isasymin}\isactrlsub c\ X\isactrlbsup {\isasymone}\isactrlesup \ {\isasymand}\ a\ {\isacharequal}{\kern0pt}\ {\isasymlangle}id\ {\isasymone}{\isacharcomma}{\kern0pt}\ A{\isasymrangle}{\isachardoublequoteclose}\isanewline
\ \ \ \ \ \ \isacommand{by}\isamarkupfalse%
\ {\isacharparenleft}{\kern0pt}typecheck{\isacharunderscore}{\kern0pt}cfuncs{\isacharcomma}{\kern0pt}\ metis\ a{\isacharunderscore}{\kern0pt}type\ cart{\isacharunderscore}{\kern0pt}prod{\isacharunderscore}{\kern0pt}decomp\ eval{\isacharunderscore}{\kern0pt}dom\ terminal{\isacharunderscore}{\kern0pt}func{\isacharunderscore}{\kern0pt}unique{\isacharparenright}{\kern0pt}\isanewline
\isanewline
\ \ \ \ \isacommand{obtain}\isamarkupfalse%
\ B\ \isakeyword{where}\ b{\isacharunderscore}{\kern0pt}def{\isacharcolon}{\kern0pt}\ {\isachardoublequoteopen}B\ {\isasymin}\isactrlsub c\ X\isactrlbsup {\isasymone}\isactrlesup \ {\isasymand}\ b\ {\isacharequal}{\kern0pt}\ {\isasymlangle}id\ {\isasymone}{\isacharcomma}{\kern0pt}\ B{\isasymrangle}{\isachardoublequoteclose}\isanewline
\ \ \ \ \ \ \isacommand{by}\isamarkupfalse%
\ {\isacharparenleft}{\kern0pt}typecheck{\isacharunderscore}{\kern0pt}cfuncs{\isacharcomma}{\kern0pt}\ metis\ b{\isacharunderscore}{\kern0pt}type\ cart{\isacharunderscore}{\kern0pt}prod{\isacharunderscore}{\kern0pt}decomp\ eval{\isacharunderscore}{\kern0pt}dom\ terminal{\isacharunderscore}{\kern0pt}func{\isacharunderscore}{\kern0pt}unique{\isacharparenright}{\kern0pt}\isanewline
\isanewline
\ \ \ \ \isacommand{have}\isamarkupfalse%
\ {\isachardoublequoteopen}A\isactrlsup {\isasymflat}\ {\isasymcirc}\isactrlsub c\ {\isasymlangle}id\ {\isasymone}{\isacharcomma}{\kern0pt}\ id\ {\isasymone}{\isasymrangle}\ {\isacharequal}{\kern0pt}\ B\isactrlsup {\isasymflat}\ {\isasymcirc}\isactrlsub c\ {\isasymlangle}id\ {\isasymone}{\isacharcomma}{\kern0pt}\ id\ {\isasymone}{\isasymrangle}{\isachardoublequoteclose}\isanewline
\ \ \ \ \isacommand{proof}\isamarkupfalse%
\ {\isacharminus}{\kern0pt}\ \isanewline
\ \ \ \ \ \ \isacommand{have}\isamarkupfalse%
\ {\isachardoublequoteopen}A\isactrlsup {\isasymflat}\ {\isasymcirc}\isactrlsub c\ {\isasymlangle}id\ {\isasymone}\ {\isacharcomma}{\kern0pt}\ id\ {\isasymone}{\isasymrangle}\ {\isacharequal}{\kern0pt}\ {\isacharparenleft}{\kern0pt}eval{\isacharunderscore}{\kern0pt}func\ X\ {\isasymone}{\isacharparenright}{\kern0pt}\ {\isasymcirc}\isactrlsub c\ {\isacharparenleft}{\kern0pt}id\ {\isasymone}\ {\isasymtimes}\isactrlsub f\ {\isacharparenleft}{\kern0pt}A\isactrlsup {\isasymflat}{\isacharparenright}{\kern0pt}\isactrlsup {\isasymsharp}{\isacharparenright}{\kern0pt}\ {\isasymcirc}\isactrlsub c\ {\isasymlangle}id\ {\isasymone}{\isacharcomma}{\kern0pt}\ id\ {\isasymone}{\isasymrangle}{\isachardoublequoteclose}\isanewline
\ \ \ \ \ \ \ \ \isacommand{by}\isamarkupfalse%
\ {\isacharparenleft}{\kern0pt}typecheck{\isacharunderscore}{\kern0pt}cfuncs{\isacharcomma}{\kern0pt}\ smt\ {\isacharparenleft}{\kern0pt}verit{\isacharcomma}{\kern0pt}\ best{\isacharparenright}{\kern0pt}\ a{\isacharunderscore}{\kern0pt}def\ comp{\isacharunderscore}{\kern0pt}associative{\isadigit{2}}\ inv{\isacharunderscore}{\kern0pt}transpose{\isacharunderscore}{\kern0pt}func{\isacharunderscore}{\kern0pt}def{\isadigit{3}}\ sharp{\isacharunderscore}{\kern0pt}cancels{\isacharunderscore}{\kern0pt}flat{\isacharparenright}{\kern0pt}\isanewline
\ \ \ \ \ \ \isacommand{also}\isamarkupfalse%
\ \isacommand{have}\isamarkupfalse%
\ {\isachardoublequoteopen}{\isachardot}{\kern0pt}{\isachardot}{\kern0pt}{\isachardot}{\kern0pt}\ {\isacharequal}{\kern0pt}\ eval{\isacharunderscore}{\kern0pt}func\ X\ {\isasymone}\ {\isasymcirc}\isactrlsub c\ a{\isachardoublequoteclose}\isanewline
\ \ \ \ \ \ \ \ \isacommand{using}\isamarkupfalse%
\ a{\isacharunderscore}{\kern0pt}def\ cfunc{\isacharunderscore}{\kern0pt}cross{\isacharunderscore}{\kern0pt}prod{\isacharunderscore}{\kern0pt}comp{\isacharunderscore}{\kern0pt}cfunc{\isacharunderscore}{\kern0pt}prod\ id{\isacharunderscore}{\kern0pt}right{\isacharunderscore}{\kern0pt}unit{\isadigit{2}}\ sharp{\isacharunderscore}{\kern0pt}cancels{\isacharunderscore}{\kern0pt}flat\ \isacommand{by}\isamarkupfalse%
\ {\isacharparenleft}{\kern0pt}typecheck{\isacharunderscore}{\kern0pt}cfuncs{\isacharcomma}{\kern0pt}\ force{\isacharparenright}{\kern0pt}\isanewline
\ \ \ \ \ \ \isacommand{also}\isamarkupfalse%
\ \isacommand{have}\isamarkupfalse%
\ {\isachardoublequoteopen}{\isachardot}{\kern0pt}{\isachardot}{\kern0pt}{\isachardot}{\kern0pt}\ {\isacharequal}{\kern0pt}\ eval{\isacharunderscore}{\kern0pt}func\ X\ {\isasymone}\ {\isasymcirc}\isactrlsub c\ b{\isachardoublequoteclose}\isanewline
\ \ \ \ \ \ \ \ \isacommand{by}\isamarkupfalse%
\ {\isacharparenleft}{\kern0pt}simp\ add{\isacharcolon}{\kern0pt}\ evals{\isacharunderscore}{\kern0pt}equal{\isacharparenright}{\kern0pt}\isanewline
\ \ \ \ \ \ \isacommand{also}\isamarkupfalse%
\ \isacommand{have}\isamarkupfalse%
\ {\isachardoublequoteopen}{\isachardot}{\kern0pt}{\isachardot}{\kern0pt}{\isachardot}{\kern0pt}\ {\isacharequal}{\kern0pt}\ {\isacharparenleft}{\kern0pt}eval{\isacharunderscore}{\kern0pt}func\ X\ {\isasymone}{\isacharparenright}{\kern0pt}\ {\isasymcirc}\isactrlsub c\ {\isacharparenleft}{\kern0pt}id\ {\isasymone}\ {\isasymtimes}\isactrlsub f\ {\isacharparenleft}{\kern0pt}B\isactrlsup {\isasymflat}{\isacharparenright}{\kern0pt}\isactrlsup {\isasymsharp}{\isacharparenright}{\kern0pt}\ {\isasymcirc}\isactrlsub c\ {\isasymlangle}id\ {\isasymone}{\isacharcomma}{\kern0pt}\ id\ {\isasymone}{\isasymrangle}{\isachardoublequoteclose}\isanewline
\ \ \ \ \ \ \ \ \isacommand{using}\isamarkupfalse%
\ b{\isacharunderscore}{\kern0pt}def\ cfunc{\isacharunderscore}{\kern0pt}cross{\isacharunderscore}{\kern0pt}prod{\isacharunderscore}{\kern0pt}comp{\isacharunderscore}{\kern0pt}cfunc{\isacharunderscore}{\kern0pt}prod\ id{\isacharunderscore}{\kern0pt}right{\isacharunderscore}{\kern0pt}unit{\isadigit{2}}\ sharp{\isacharunderscore}{\kern0pt}cancels{\isacharunderscore}{\kern0pt}flat\ \isacommand{by}\isamarkupfalse%
\ {\isacharparenleft}{\kern0pt}typecheck{\isacharunderscore}{\kern0pt}cfuncs{\isacharcomma}{\kern0pt}\ auto{\isacharparenright}{\kern0pt}\isanewline
\ \ \ \ \ \ \isacommand{also}\isamarkupfalse%
\ \isacommand{have}\isamarkupfalse%
\ {\isachardoublequoteopen}{\isachardot}{\kern0pt}{\isachardot}{\kern0pt}{\isachardot}{\kern0pt}\ {\isacharequal}{\kern0pt}\ B\isactrlsup {\isasymflat}\ {\isasymcirc}\isactrlsub c\ {\isasymlangle}id\ {\isasymone}{\isacharcomma}{\kern0pt}\ id\ {\isasymone}{\isasymrangle}{\isachardoublequoteclose}\isanewline
\ \ \ \ \ \ \ \ \isacommand{by}\isamarkupfalse%
\ {\isacharparenleft}{\kern0pt}typecheck{\isacharunderscore}{\kern0pt}cfuncs{\isacharcomma}{\kern0pt}\ smt\ {\isacharparenleft}{\kern0pt}verit{\isacharparenright}{\kern0pt}\ b{\isacharunderscore}{\kern0pt}def\ comp{\isacharunderscore}{\kern0pt}associative{\isadigit{2}}\ inv{\isacharunderscore}{\kern0pt}transpose{\isacharunderscore}{\kern0pt}func{\isacharunderscore}{\kern0pt}def{\isadigit{3}}\ sharp{\isacharunderscore}{\kern0pt}cancels{\isacharunderscore}{\kern0pt}flat{\isacharparenright}{\kern0pt}\isanewline
\ \ \ \ \ \ \isacommand{then}\isamarkupfalse%
\ \isacommand{show}\isamarkupfalse%
\ {\isachardoublequoteopen}A\isactrlsup {\isasymflat}\ {\isasymcirc}\isactrlsub c\ {\isasymlangle}id\ {\isasymone}{\isacharcomma}{\kern0pt}\ id\ {\isasymone}{\isasymrangle}\ {\isacharequal}{\kern0pt}\ B\isactrlsup {\isasymflat}\ {\isasymcirc}\isactrlsub c\ {\isasymlangle}id\ {\isasymone}{\isacharcomma}{\kern0pt}\ id\ {\isasymone}{\isasymrangle}{\isachardoublequoteclose}\isanewline
\ \ \ \ \ \ \ \ \isacommand{using}\isamarkupfalse%
\ calculation\ \isacommand{by}\isamarkupfalse%
\ auto\isanewline
\ \ \ \ \isacommand{qed}\isamarkupfalse%
\isanewline
\ \ \ \ \isacommand{then}\isamarkupfalse%
\ \isacommand{have}\isamarkupfalse%
\ {\isachardoublequoteopen}A\isactrlsup {\isasymflat}\ {\isacharequal}{\kern0pt}\ B\isactrlsup {\isasymflat}{\isachardoublequoteclose}\isanewline
\ \ \ \ \ \ \isacommand{by}\isamarkupfalse%
\ {\isacharparenleft}{\kern0pt}typecheck{\isacharunderscore}{\kern0pt}cfuncs{\isacharcomma}{\kern0pt}\ smt\ swap{\isacharunderscore}{\kern0pt}def\ a{\isacharunderscore}{\kern0pt}def\ b{\isacharunderscore}{\kern0pt}def\ cfunc{\isacharunderscore}{\kern0pt}prod{\isacharunderscore}{\kern0pt}comp\ comp{\isacharunderscore}{\kern0pt}associative{\isadigit{2}}\ diagonal{\isacharunderscore}{\kern0pt}def\ diagonal{\isacharunderscore}{\kern0pt}type\ id{\isacharunderscore}{\kern0pt}right{\isacharunderscore}{\kern0pt}unit{\isadigit{2}}\ id{\isacharunderscore}{\kern0pt}type\ left{\isacharunderscore}{\kern0pt}cart{\isacharunderscore}{\kern0pt}proj{\isacharunderscore}{\kern0pt}type\ right{\isacharunderscore}{\kern0pt}cart{\isacharunderscore}{\kern0pt}proj{\isacharunderscore}{\kern0pt}type\ swap{\isacharunderscore}{\kern0pt}idempotent\ swap{\isacharunderscore}{\kern0pt}type\ terminal{\isacharunderscore}{\kern0pt}func{\isacharunderscore}{\kern0pt}comp\ terminal{\isacharunderscore}{\kern0pt}func{\isacharunderscore}{\kern0pt}unique{\isacharparenright}{\kern0pt}\isanewline
\ \ \ \ \isacommand{then}\isamarkupfalse%
\ \isacommand{have}\isamarkupfalse%
\ {\isachardoublequoteopen}A\ {\isacharequal}{\kern0pt}\ B{\isachardoublequoteclose}\isanewline
\ \ \ \ \ \ \isacommand{by}\isamarkupfalse%
\ {\isacharparenleft}{\kern0pt}metis\ a{\isacharunderscore}{\kern0pt}def\ b{\isacharunderscore}{\kern0pt}def\ sharp{\isacharunderscore}{\kern0pt}cancels{\isacharunderscore}{\kern0pt}flat{\isacharparenright}{\kern0pt}\isanewline
\ \ \ \ \isacommand{then}\isamarkupfalse%
\ \isacommand{show}\isamarkupfalse%
\ {\isachardoublequoteopen}a\ {\isacharequal}{\kern0pt}\ b{\isachardoublequoteclose}\isanewline
\ \ \ \ \ \ \isacommand{by}\isamarkupfalse%
\ {\isacharparenleft}{\kern0pt}simp\ add{\isacharcolon}{\kern0pt}\ a{\isacharunderscore}{\kern0pt}def\ b{\isacharunderscore}{\kern0pt}def{\isacharparenright}{\kern0pt}\isanewline
\ \ \isacommand{qed}\isamarkupfalse%
\isanewline
\isacommand{next}\isamarkupfalse%
\isanewline
\ \ \isacommand{assume}\isamarkupfalse%
\ {\isachardoublequoteopen}{\isasymnexists}x{\isachardot}{\kern0pt}\ x\ {\isasymin}\isactrlsub c\ X{\isachardoublequoteclose}\isanewline
\ \ \isacommand{then}\isamarkupfalse%
\ \isacommand{show}\isamarkupfalse%
\ {\isachardoublequoteopen}injective\ {\isacharparenleft}{\kern0pt}eval{\isacharunderscore}{\kern0pt}func\ X\ {\isasymone}{\isacharparenright}{\kern0pt}{\isachardoublequoteclose}\isanewline
\ \ \ \ \isacommand{by}\isamarkupfalse%
\ {\isacharparenleft}{\kern0pt}typecheck{\isacharunderscore}{\kern0pt}cfuncs{\isacharcomma}{\kern0pt}\ metis\ \ cfunc{\isacharunderscore}{\kern0pt}type{\isacharunderscore}{\kern0pt}def\ comp{\isacharunderscore}{\kern0pt}type\ injective{\isacharunderscore}{\kern0pt}def{\isacharparenright}{\kern0pt}\isanewline
\isacommand{qed}\isamarkupfalse%
%
\endisatagproof
{\isafoldproof}%
%
\isadelimproof
%
\endisadelimproof
%
\begin{isamarkuptext}%
In the lemma below, the nonempty assumption is required.
      Consider, for example, \isa{X\ {\isacharequal}{\kern0pt}\ {\isasymOmega}} and \isa{A\ {\isacharequal}{\kern0pt}\ {\isasymemptyset}}%
\end{isamarkuptext}\isamarkuptrue%
\isacommand{lemma}\isamarkupfalse%
\ sharp{\isacharunderscore}{\kern0pt}pres{\isacharunderscore}{\kern0pt}mono{\isacharcolon}{\kern0pt}\isanewline
\ \ \isakeyword{assumes}\ {\isachardoublequoteopen}f\ {\isacharcolon}{\kern0pt}\ A\ {\isasymtimes}\isactrlsub c\ Z\ {\isasymrightarrow}\ X{\isachardoublequoteclose}\isanewline
\ \ \isakeyword{assumes}\ {\isachardoublequoteopen}monomorphism{\isacharparenleft}{\kern0pt}f{\isacharparenright}{\kern0pt}{\isachardoublequoteclose}\isanewline
\ \ \isakeyword{assumes}\ {\isachardoublequoteopen}nonempty\ A{\isachardoublequoteclose}\isanewline
\ \ \isakeyword{shows}\ \ \ {\isachardoublequoteopen}monomorphism{\isacharparenleft}{\kern0pt}f\isactrlsup {\isasymsharp}{\isacharparenright}{\kern0pt}{\isachardoublequoteclose}\isanewline
%
\isadelimproof
\ \ %
\endisadelimproof
%
\isatagproof
\isacommand{unfolding}\isamarkupfalse%
\ monomorphism{\isacharunderscore}{\kern0pt}def{\isadigit{2}}\isanewline
\isacommand{proof}\isamarkupfalse%
{\isacharparenleft}{\kern0pt}clarify{\isacharparenright}{\kern0pt}\isanewline
\ \ \isacommand{fix}\isamarkupfalse%
\ g\ h\ U\ Y\ x\isanewline
\ \ \isacommand{assume}\isamarkupfalse%
\ g{\isacharunderscore}{\kern0pt}type{\isacharbrackleft}{\kern0pt}type{\isacharunderscore}{\kern0pt}rule{\isacharbrackright}{\kern0pt}{\isacharcolon}{\kern0pt}\ {\isachardoublequoteopen}g\ {\isacharcolon}{\kern0pt}\ U\ {\isasymrightarrow}\ Y{\isachardoublequoteclose}\isanewline
\ \ \isacommand{assume}\isamarkupfalse%
\ h{\isacharunderscore}{\kern0pt}type{\isacharbrackleft}{\kern0pt}type{\isacharunderscore}{\kern0pt}rule{\isacharbrackright}{\kern0pt}{\isacharcolon}{\kern0pt}\ {\isachardoublequoteopen}h\ {\isacharcolon}{\kern0pt}\ U\ {\isasymrightarrow}\ Y{\isachardoublequoteclose}\isanewline
\ \ \isacommand{assume}\isamarkupfalse%
\ f{\isacharunderscore}{\kern0pt}sharp{\isacharunderscore}{\kern0pt}type{\isacharbrackleft}{\kern0pt}type{\isacharunderscore}{\kern0pt}rule{\isacharbrackright}{\kern0pt}{\isacharcolon}{\kern0pt}\ {\isachardoublequoteopen}f\isactrlsup {\isasymsharp}\ {\isacharcolon}{\kern0pt}\ Y\ {\isasymrightarrow}\ x{\isachardoublequoteclose}\isanewline
\ \ \isacommand{assume}\isamarkupfalse%
\ equals{\isacharcolon}{\kern0pt}\ {\isachardoublequoteopen}f\isactrlsup {\isasymsharp}\ {\isasymcirc}\isactrlsub c\ g\ {\isacharequal}{\kern0pt}\ f\isactrlsup {\isasymsharp}\ {\isasymcirc}\isactrlsub c\ h{\isachardoublequoteclose}\isanewline
\isanewline
\ \ \isacommand{have}\isamarkupfalse%
\ f{\isacharunderscore}{\kern0pt}sharp{\isacharunderscore}{\kern0pt}type{\isadigit{2}}{\isacharcolon}{\kern0pt}\ {\isachardoublequoteopen}f\isactrlsup {\isasymsharp}\ {\isacharcolon}{\kern0pt}\ Z\ {\isasymrightarrow}\ X\isactrlbsup A\isactrlesup {\isachardoublequoteclose}\isanewline
\ \ \ \ \isacommand{by}\isamarkupfalse%
\ {\isacharparenleft}{\kern0pt}simp\ add{\isacharcolon}{\kern0pt}\ assms{\isacharparenleft}{\kern0pt}{\isadigit{1}}{\isacharparenright}{\kern0pt}\ transpose{\isacharunderscore}{\kern0pt}func{\isacharunderscore}{\kern0pt}type{\isacharparenright}{\kern0pt}\isanewline
\ \ \isacommand{have}\isamarkupfalse%
\ Y{\isacharunderscore}{\kern0pt}is{\isacharunderscore}{\kern0pt}Z{\isacharcolon}{\kern0pt}\ {\isachardoublequoteopen}Y\ {\isacharequal}{\kern0pt}\ Z{\isachardoublequoteclose}\isanewline
\ \ \ \ \isacommand{using}\isamarkupfalse%
\ cfunc{\isacharunderscore}{\kern0pt}type{\isacharunderscore}{\kern0pt}def\ f{\isacharunderscore}{\kern0pt}sharp{\isacharunderscore}{\kern0pt}type\ f{\isacharunderscore}{\kern0pt}sharp{\isacharunderscore}{\kern0pt}type{\isadigit{2}}\ \isacommand{by}\isamarkupfalse%
\ auto\isanewline
\ \ \isacommand{have}\isamarkupfalse%
\ x{\isacharunderscore}{\kern0pt}is{\isacharunderscore}{\kern0pt}XA{\isacharcolon}{\kern0pt}\ {\isachardoublequoteopen}x\ {\isacharequal}{\kern0pt}\ X\isactrlbsup A\isactrlesup {\isachardoublequoteclose}\isanewline
\ \ \ \ \isacommand{using}\isamarkupfalse%
\ cfunc{\isacharunderscore}{\kern0pt}type{\isacharunderscore}{\kern0pt}def\ f{\isacharunderscore}{\kern0pt}sharp{\isacharunderscore}{\kern0pt}type\ f{\isacharunderscore}{\kern0pt}sharp{\isacharunderscore}{\kern0pt}type{\isadigit{2}}\ \isacommand{by}\isamarkupfalse%
\ auto\isanewline
\ \ \isacommand{have}\isamarkupfalse%
\ g{\isacharunderscore}{\kern0pt}type{\isadigit{2}}{\isacharcolon}{\kern0pt}\ {\isachardoublequoteopen}g\ {\isacharcolon}{\kern0pt}\ U\ {\isasymrightarrow}\ Z{\isachardoublequoteclose}\isanewline
\ \ \ \ \isacommand{using}\isamarkupfalse%
\ Y{\isacharunderscore}{\kern0pt}is{\isacharunderscore}{\kern0pt}Z\ g{\isacharunderscore}{\kern0pt}type\ \isacommand{by}\isamarkupfalse%
\ blast\isanewline
\ \ \isacommand{have}\isamarkupfalse%
\ h{\isacharunderscore}{\kern0pt}type{\isadigit{2}}{\isacharcolon}{\kern0pt}\ {\isachardoublequoteopen}h\ {\isacharcolon}{\kern0pt}\ U\ {\isasymrightarrow}\ Z{\isachardoublequoteclose}\isanewline
\ \ \ \ \isacommand{using}\isamarkupfalse%
\ Y{\isacharunderscore}{\kern0pt}is{\isacharunderscore}{\kern0pt}Z\ h{\isacharunderscore}{\kern0pt}type\ \isacommand{by}\isamarkupfalse%
\ blast\isanewline
\ \ \isacommand{have}\isamarkupfalse%
\ idg{\isacharunderscore}{\kern0pt}type{\isacharcolon}{\kern0pt}\ {\isachardoublequoteopen}{\isacharparenleft}{\kern0pt}id{\isacharparenleft}{\kern0pt}A{\isacharparenright}{\kern0pt}\ {\isasymtimes}\isactrlsub f\ g{\isacharparenright}{\kern0pt}\ {\isacharcolon}{\kern0pt}\ A\ {\isasymtimes}\isactrlsub c\ U\ {\isasymrightarrow}\ A\ {\isasymtimes}\isactrlsub c\ Z{\isachardoublequoteclose}\isanewline
\ \ \ \ \isacommand{by}\isamarkupfalse%
\ {\isacharparenleft}{\kern0pt}simp\ add{\isacharcolon}{\kern0pt}\ cfunc{\isacharunderscore}{\kern0pt}cross{\isacharunderscore}{\kern0pt}prod{\isacharunderscore}{\kern0pt}type\ g{\isacharunderscore}{\kern0pt}type{\isadigit{2}}\ id{\isacharunderscore}{\kern0pt}type{\isacharparenright}{\kern0pt}\isanewline
\ \ \isacommand{have}\isamarkupfalse%
\ idh{\isacharunderscore}{\kern0pt}type{\isacharcolon}{\kern0pt}\ {\isachardoublequoteopen}{\isacharparenleft}{\kern0pt}id{\isacharparenleft}{\kern0pt}A{\isacharparenright}{\kern0pt}\ {\isasymtimes}\isactrlsub f\ h{\isacharparenright}{\kern0pt}\ {\isacharcolon}{\kern0pt}\ A\ {\isasymtimes}\isactrlsub c\ U\ {\isasymrightarrow}\ A\ {\isasymtimes}\isactrlsub c\ Z{\isachardoublequoteclose}\isanewline
\ \ \ \ \isacommand{by}\isamarkupfalse%
\ {\isacharparenleft}{\kern0pt}simp\ add{\isacharcolon}{\kern0pt}\ cfunc{\isacharunderscore}{\kern0pt}cross{\isacharunderscore}{\kern0pt}prod{\isacharunderscore}{\kern0pt}type\ h{\isacharunderscore}{\kern0pt}type{\isadigit{2}}\ id{\isacharunderscore}{\kern0pt}type{\isacharparenright}{\kern0pt}\isanewline
\isanewline
\ \ \ \isacommand{then}\isamarkupfalse%
\ \isacommand{have}\isamarkupfalse%
\ epic{\isacharcolon}{\kern0pt}\ {\isachardoublequoteopen}epimorphism{\isacharparenleft}{\kern0pt}right{\isacharunderscore}{\kern0pt}cart{\isacharunderscore}{\kern0pt}proj\ A\ U{\isacharparenright}{\kern0pt}{\isachardoublequoteclose}\isanewline
\ \ \ \ \ \isacommand{using}\isamarkupfalse%
\ assms{\isacharparenleft}{\kern0pt}{\isadigit{3}}{\isacharparenright}{\kern0pt}\ nonempty{\isacharunderscore}{\kern0pt}left{\isacharunderscore}{\kern0pt}imp{\isacharunderscore}{\kern0pt}right{\isacharunderscore}{\kern0pt}proj{\isacharunderscore}{\kern0pt}epimorphism\ \isacommand{by}\isamarkupfalse%
\ blast\isanewline
\isanewline
\ \ \ \isacommand{have}\isamarkupfalse%
\ fIdg{\isacharunderscore}{\kern0pt}is{\isacharunderscore}{\kern0pt}fIdh{\isacharcolon}{\kern0pt}\ {\isachardoublequoteopen}f\ {\isasymcirc}\isactrlsub c\ {\isacharparenleft}{\kern0pt}id{\isacharparenleft}{\kern0pt}A{\isacharparenright}{\kern0pt}\ {\isasymtimes}\isactrlsub f\ g{\isacharparenright}{\kern0pt}\ {\isacharequal}{\kern0pt}\ f\ {\isasymcirc}\isactrlsub c\ {\isacharparenleft}{\kern0pt}id{\isacharparenleft}{\kern0pt}A{\isacharparenright}{\kern0pt}\ {\isasymtimes}\isactrlsub f\ h{\isacharparenright}{\kern0pt}{\isachardoublequoteclose}\isanewline
\ \ \ \isacommand{proof}\isamarkupfalse%
\ {\isacharminus}{\kern0pt}\ \isanewline
\ \ \ \ \isacommand{have}\isamarkupfalse%
\ {\isachardoublequoteopen}f\ {\isasymcirc}\isactrlsub c\ {\isacharparenleft}{\kern0pt}id{\isacharparenleft}{\kern0pt}A{\isacharparenright}{\kern0pt}\ {\isasymtimes}\isactrlsub f\ g{\isacharparenright}{\kern0pt}\ {\isacharequal}{\kern0pt}\ {\isacharparenleft}{\kern0pt}eval{\isacharunderscore}{\kern0pt}func\ X\ A\ {\isasymcirc}\isactrlsub c\ {\isacharparenleft}{\kern0pt}id{\isacharparenleft}{\kern0pt}A{\isacharparenright}{\kern0pt}\ {\isasymtimes}\isactrlsub f\ f\isactrlsup {\isasymsharp}{\isacharparenright}{\kern0pt}{\isacharparenright}{\kern0pt}\ {\isasymcirc}\isactrlsub c\ {\isacharparenleft}{\kern0pt}id{\isacharparenleft}{\kern0pt}A{\isacharparenright}{\kern0pt}\ {\isasymtimes}\isactrlsub f\ g{\isacharparenright}{\kern0pt}{\isachardoublequoteclose}\isanewline
\ \ \ \ \ \ \isacommand{using}\isamarkupfalse%
\ assms{\isacharparenleft}{\kern0pt}{\isadigit{1}}{\isacharparenright}{\kern0pt}\ transpose{\isacharunderscore}{\kern0pt}func{\isacharunderscore}{\kern0pt}def\ \isacommand{by}\isamarkupfalse%
\ auto\isanewline
\ \ \ \ \isacommand{also}\isamarkupfalse%
\ \isacommand{have}\isamarkupfalse%
\ {\isachardoublequoteopen}{\isachardot}{\kern0pt}{\isachardot}{\kern0pt}{\isachardot}{\kern0pt}\ {\isacharequal}{\kern0pt}\ eval{\isacharunderscore}{\kern0pt}func\ X\ A\ {\isasymcirc}\isactrlsub c\ {\isacharparenleft}{\kern0pt}{\isacharparenleft}{\kern0pt}id{\isacharparenleft}{\kern0pt}A{\isacharparenright}{\kern0pt}\ {\isasymtimes}\isactrlsub f\ f\isactrlsup {\isasymsharp}{\isacharparenright}{\kern0pt}\ {\isasymcirc}\isactrlsub c\ {\isacharparenleft}{\kern0pt}id{\isacharparenleft}{\kern0pt}A{\isacharparenright}{\kern0pt}\ {\isasymtimes}\isactrlsub f\ g{\isacharparenright}{\kern0pt}{\isacharparenright}{\kern0pt}{\isachardoublequoteclose}\isanewline
\ \ \ \ \ \ \isacommand{using}\isamarkupfalse%
\ comp{\isacharunderscore}{\kern0pt}associative{\isadigit{2}}\ f{\isacharunderscore}{\kern0pt}sharp{\isacharunderscore}{\kern0pt}type{\isadigit{2}}\ idg{\isacharunderscore}{\kern0pt}type\ \isacommand{by}\isamarkupfalse%
\ {\isacharparenleft}{\kern0pt}typecheck{\isacharunderscore}{\kern0pt}cfuncs{\isacharcomma}{\kern0pt}\ fastforce{\isacharparenright}{\kern0pt}\isanewline
\ \ \ \ \isacommand{also}\isamarkupfalse%
\ \isacommand{have}\isamarkupfalse%
\ {\isachardoublequoteopen}{\isachardot}{\kern0pt}{\isachardot}{\kern0pt}{\isachardot}{\kern0pt}\ {\isacharequal}{\kern0pt}\ eval{\isacharunderscore}{\kern0pt}func\ X\ A\ {\isasymcirc}\isactrlsub c\ {\isacharparenleft}{\kern0pt}id{\isacharparenleft}{\kern0pt}A{\isacharparenright}{\kern0pt}\ {\isasymtimes}\isactrlsub f\ {\isacharparenleft}{\kern0pt}f\isactrlsup {\isasymsharp}\ {\isasymcirc}\isactrlsub c\ g{\isacharparenright}{\kern0pt}{\isacharparenright}{\kern0pt}{\isachardoublequoteclose}\isanewline
\ \ \ \ \ \ \isacommand{using}\isamarkupfalse%
\ f{\isacharunderscore}{\kern0pt}sharp{\isacharunderscore}{\kern0pt}type{\isadigit{2}}\ g{\isacharunderscore}{\kern0pt}type{\isadigit{2}}\ identity{\isacharunderscore}{\kern0pt}distributes{\isacharunderscore}{\kern0pt}across{\isacharunderscore}{\kern0pt}composition\ \isacommand{by}\isamarkupfalse%
\ auto\isanewline
\ \ \ \ \isacommand{also}\isamarkupfalse%
\ \isacommand{have}\isamarkupfalse%
\ {\isachardoublequoteopen}{\isachardot}{\kern0pt}{\isachardot}{\kern0pt}{\isachardot}{\kern0pt}\ {\isacharequal}{\kern0pt}\ eval{\isacharunderscore}{\kern0pt}func\ X\ A\ {\isasymcirc}\isactrlsub c\ {\isacharparenleft}{\kern0pt}id{\isacharparenleft}{\kern0pt}A{\isacharparenright}{\kern0pt}\ {\isasymtimes}\isactrlsub f\ {\isacharparenleft}{\kern0pt}f\isactrlsup {\isasymsharp}\ {\isasymcirc}\isactrlsub c\ h{\isacharparenright}{\kern0pt}{\isacharparenright}{\kern0pt}{\isachardoublequoteclose}\isanewline
\ \ \ \ \ \ \isacommand{by}\isamarkupfalse%
\ {\isacharparenleft}{\kern0pt}simp\ add{\isacharcolon}{\kern0pt}\ equals{\isacharparenright}{\kern0pt}\isanewline
\ \ \ \ \isacommand{also}\isamarkupfalse%
\ \isacommand{have}\isamarkupfalse%
\ {\isachardoublequoteopen}{\isachardot}{\kern0pt}{\isachardot}{\kern0pt}{\isachardot}{\kern0pt}\ {\isacharequal}{\kern0pt}\ eval{\isacharunderscore}{\kern0pt}func\ X\ A\ {\isasymcirc}\isactrlsub c\ {\isacharparenleft}{\kern0pt}{\isacharparenleft}{\kern0pt}id{\isacharparenleft}{\kern0pt}A{\isacharparenright}{\kern0pt}\ {\isasymtimes}\isactrlsub f\ f\isactrlsup {\isasymsharp}{\isacharparenright}{\kern0pt}\ {\isasymcirc}\isactrlsub c\ {\isacharparenleft}{\kern0pt}id{\isacharparenleft}{\kern0pt}A{\isacharparenright}{\kern0pt}\ {\isasymtimes}\isactrlsub f\ h{\isacharparenright}{\kern0pt}{\isacharparenright}{\kern0pt}{\isachardoublequoteclose}\isanewline
\ \ \ \ \ \ \isacommand{using}\isamarkupfalse%
\ f{\isacharunderscore}{\kern0pt}sharp{\isacharunderscore}{\kern0pt}type\ h{\isacharunderscore}{\kern0pt}type\ identity{\isacharunderscore}{\kern0pt}distributes{\isacharunderscore}{\kern0pt}across{\isacharunderscore}{\kern0pt}composition\ \isacommand{by}\isamarkupfalse%
\ auto\isanewline
\ \ \ \ \isacommand{also}\isamarkupfalse%
\ \isacommand{have}\isamarkupfalse%
\ {\isachardoublequoteopen}{\isachardot}{\kern0pt}{\isachardot}{\kern0pt}{\isachardot}{\kern0pt}\ {\isacharequal}{\kern0pt}\ {\isacharparenleft}{\kern0pt}eval{\isacharunderscore}{\kern0pt}func\ X\ A\ {\isasymcirc}\isactrlsub c\ {\isacharparenleft}{\kern0pt}id{\isacharparenleft}{\kern0pt}A{\isacharparenright}{\kern0pt}\ {\isasymtimes}\isactrlsub f\ f\isactrlsup {\isasymsharp}{\isacharparenright}{\kern0pt}{\isacharparenright}{\kern0pt}\ {\isasymcirc}\isactrlsub c\ {\isacharparenleft}{\kern0pt}id{\isacharparenleft}{\kern0pt}A{\isacharparenright}{\kern0pt}\ {\isasymtimes}\isactrlsub f\ h{\isacharparenright}{\kern0pt}{\isachardoublequoteclose}\isanewline
\ \ \ \ \ \ \isacommand{by}\isamarkupfalse%
\ {\isacharparenleft}{\kern0pt}metis\ Y{\isacharunderscore}{\kern0pt}is{\isacharunderscore}{\kern0pt}Z\ assms{\isacharparenleft}{\kern0pt}{\isadigit{1}}{\isacharparenright}{\kern0pt}\ calculation\ equals\ f{\isacharunderscore}{\kern0pt}sharp{\isacharunderscore}{\kern0pt}type{\isadigit{2}}\ g{\isacharunderscore}{\kern0pt}type\ h{\isacharunderscore}{\kern0pt}type\ inv{\isacharunderscore}{\kern0pt}transpose{\isacharunderscore}{\kern0pt}func{\isacharunderscore}{\kern0pt}def{\isadigit{3}}\ inv{\isacharunderscore}{\kern0pt}transpose{\isacharunderscore}{\kern0pt}of{\isacharunderscore}{\kern0pt}composition\ transpose{\isacharunderscore}{\kern0pt}func{\isacharunderscore}{\kern0pt}def{\isacharparenright}{\kern0pt}\isanewline
\ \ \ \ \isacommand{also}\isamarkupfalse%
\ \isacommand{have}\isamarkupfalse%
\ {\isachardoublequoteopen}{\isachardot}{\kern0pt}{\isachardot}{\kern0pt}{\isachardot}{\kern0pt}\ {\isacharequal}{\kern0pt}\ f\ {\isasymcirc}\isactrlsub c\ {\isacharparenleft}{\kern0pt}id{\isacharparenleft}{\kern0pt}A{\isacharparenright}{\kern0pt}\ {\isasymtimes}\isactrlsub f\ h{\isacharparenright}{\kern0pt}{\isachardoublequoteclose}\isanewline
\ \ \ \ \ \ \isacommand{using}\isamarkupfalse%
\ assms{\isacharparenleft}{\kern0pt}{\isadigit{1}}{\isacharparenright}{\kern0pt}\ transpose{\isacharunderscore}{\kern0pt}func{\isacharunderscore}{\kern0pt}def\ \isacommand{by}\isamarkupfalse%
\ auto\isanewline
\ \ \ \ \isacommand{then}\isamarkupfalse%
\ \isacommand{show}\isamarkupfalse%
\ {\isacharquery}{\kern0pt}thesis\isanewline
\ \ \ \ \ \ \isacommand{by}\isamarkupfalse%
\ {\isacharparenleft}{\kern0pt}simp\ add{\isacharcolon}{\kern0pt}\ calculation{\isacharparenright}{\kern0pt}\ \ \ \isanewline
\ \ \ \isacommand{qed}\isamarkupfalse%
\isanewline
\ \ \ \isacommand{then}\isamarkupfalse%
\ \isacommand{have}\isamarkupfalse%
\ idg{\isacharunderscore}{\kern0pt}is{\isacharunderscore}{\kern0pt}idh{\isacharcolon}{\kern0pt}\ {\isachardoublequoteopen}{\isacharparenleft}{\kern0pt}id{\isacharparenleft}{\kern0pt}A{\isacharparenright}{\kern0pt}\ {\isasymtimes}\isactrlsub f\ g{\isacharparenright}{\kern0pt}\ {\isacharequal}{\kern0pt}\ {\isacharparenleft}{\kern0pt}id{\isacharparenleft}{\kern0pt}A{\isacharparenright}{\kern0pt}\ {\isasymtimes}\isactrlsub f\ h{\isacharparenright}{\kern0pt}{\isachardoublequoteclose}\isanewline
\ \ \ \ \isacommand{using}\isamarkupfalse%
\ assms\ fIdg{\isacharunderscore}{\kern0pt}is{\isacharunderscore}{\kern0pt}fIdh\ idg{\isacharunderscore}{\kern0pt}type\ idh{\isacharunderscore}{\kern0pt}type\ monomorphism{\isacharunderscore}{\kern0pt}def{\isadigit{3}}\ \isacommand{by}\isamarkupfalse%
\ blast\isanewline
\ \ \ \isacommand{then}\isamarkupfalse%
\ \isacommand{have}\isamarkupfalse%
\ {\isachardoublequoteopen}g\ {\isasymcirc}\isactrlsub c\ {\isacharparenleft}{\kern0pt}right{\isacharunderscore}{\kern0pt}cart{\isacharunderscore}{\kern0pt}proj\ A\ U{\isacharparenright}{\kern0pt}\ {\isacharequal}{\kern0pt}\ h\ {\isasymcirc}\isactrlsub c\ {\isacharparenleft}{\kern0pt}right{\isacharunderscore}{\kern0pt}cart{\isacharunderscore}{\kern0pt}proj\ A\ U{\isacharparenright}{\kern0pt}{\isachardoublequoteclose}\isanewline
\ \ \ \ \isacommand{by}\isamarkupfalse%
\ {\isacharparenleft}{\kern0pt}smt\ g{\isacharunderscore}{\kern0pt}type{\isadigit{2}}\ h{\isacharunderscore}{\kern0pt}type{\isadigit{2}}\ id{\isacharunderscore}{\kern0pt}type\ right{\isacharunderscore}{\kern0pt}cart{\isacharunderscore}{\kern0pt}proj{\isacharunderscore}{\kern0pt}cfunc{\isacharunderscore}{\kern0pt}cross{\isacharunderscore}{\kern0pt}prod{\isacharparenright}{\kern0pt}\isanewline
\ \ \ \isacommand{then}\isamarkupfalse%
\ \isacommand{show}\isamarkupfalse%
\ {\isachardoublequoteopen}g\ {\isacharequal}{\kern0pt}\ h{\isachardoublequoteclose}\isanewline
\ \ \ \ \isacommand{using}\isamarkupfalse%
\ epic\ epimorphism{\isacharunderscore}{\kern0pt}def{\isadigit{2}}\ g{\isacharunderscore}{\kern0pt}type{\isadigit{2}}\ h{\isacharunderscore}{\kern0pt}type{\isadigit{2}}\ right{\isacharunderscore}{\kern0pt}cart{\isacharunderscore}{\kern0pt}proj{\isacharunderscore}{\kern0pt}type\ \isacommand{by}\isamarkupfalse%
\ blast\isanewline
\isacommand{qed}\isamarkupfalse%
%
\endisatagproof
{\isafoldproof}%
%
\isadelimproof
%
\endisadelimproof
%
\isadelimdocument
%
\endisadelimdocument
%
\isatagdocument
%
\isamarkupsubsection{Metafunctions and their Inverses (Cnufatems)%
}
\isamarkuptrue%
%
\isamarkupsubsubsection{Metafunctions%
}
\isamarkuptrue%
%
\endisatagdocument
{\isafolddocument}%
%
\isadelimdocument
%
\endisadelimdocument
\isacommand{definition}\isamarkupfalse%
\ metafunc\ {\isacharcolon}{\kern0pt}{\isacharcolon}{\kern0pt}\ {\isachardoublequoteopen}cfunc\ {\isasymRightarrow}\ cfunc{\isachardoublequoteclose}\ \isakeyword{where}\isanewline
\ \ {\isachardoublequoteopen}metafunc\ f\ {\isasymequiv}\ {\isacharparenleft}{\kern0pt}f\ {\isasymcirc}\isactrlsub c\ {\isacharparenleft}{\kern0pt}left{\isacharunderscore}{\kern0pt}cart{\isacharunderscore}{\kern0pt}proj\ {\isacharparenleft}{\kern0pt}domain\ f{\isacharparenright}{\kern0pt}\ {\isasymone}{\isacharparenright}{\kern0pt}{\isacharparenright}{\kern0pt}\isactrlsup {\isasymsharp}{\isachardoublequoteclose}\isanewline
\isanewline
\isacommand{lemma}\isamarkupfalse%
\ metafunc{\isacharunderscore}{\kern0pt}def{\isadigit{2}}{\isacharcolon}{\kern0pt}\isanewline
\ \ \isakeyword{assumes}\ {\isachardoublequoteopen}f\ {\isacharcolon}{\kern0pt}\ X\ {\isasymrightarrow}\ Y{\isachardoublequoteclose}\isanewline
\ \ \isakeyword{shows}\ {\isachardoublequoteopen}metafunc\ f\ {\isacharequal}{\kern0pt}\ {\isacharparenleft}{\kern0pt}f\ {\isasymcirc}\isactrlsub c\ {\isacharparenleft}{\kern0pt}left{\isacharunderscore}{\kern0pt}cart{\isacharunderscore}{\kern0pt}proj\ X\ {\isasymone}{\isacharparenright}{\kern0pt}{\isacharparenright}{\kern0pt}\isactrlsup {\isasymsharp}{\isachardoublequoteclose}\isanewline
%
\isadelimproof
\ \ %
\endisadelimproof
%
\isatagproof
\isacommand{using}\isamarkupfalse%
\ assms\ \isacommand{unfolding}\isamarkupfalse%
\ metafunc{\isacharunderscore}{\kern0pt}def\ cfunc{\isacharunderscore}{\kern0pt}type{\isacharunderscore}{\kern0pt}def\ \isacommand{by}\isamarkupfalse%
\ auto%
\endisatagproof
{\isafoldproof}%
%
\isadelimproof
\isanewline
%
\endisadelimproof
\isanewline
\isacommand{lemma}\isamarkupfalse%
\ metafunc{\isacharunderscore}{\kern0pt}type{\isacharbrackleft}{\kern0pt}type{\isacharunderscore}{\kern0pt}rule{\isacharbrackright}{\kern0pt}{\isacharcolon}{\kern0pt}\isanewline
\ \ \isakeyword{assumes}\ {\isachardoublequoteopen}f\ {\isacharcolon}{\kern0pt}\ X\ {\isasymrightarrow}\ Y{\isachardoublequoteclose}\isanewline
\ \ \isakeyword{shows}\ {\isachardoublequoteopen}metafunc\ f\ {\isasymin}\isactrlsub c\ Y\isactrlbsup X\isactrlesup {\isachardoublequoteclose}\isanewline
%
\isadelimproof
\ \ %
\endisadelimproof
%
\isatagproof
\isacommand{using}\isamarkupfalse%
\ assms\ \isacommand{by}\isamarkupfalse%
\ {\isacharparenleft}{\kern0pt}unfold\ metafunc{\isacharunderscore}{\kern0pt}def{\isadigit{2}}{\isacharcomma}{\kern0pt}\ typecheck{\isacharunderscore}{\kern0pt}cfuncs{\isacharparenright}{\kern0pt}%
\endisatagproof
{\isafoldproof}%
%
\isadelimproof
\isanewline
%
\endisadelimproof
\isanewline
\isacommand{lemma}\isamarkupfalse%
\ eval{\isacharunderscore}{\kern0pt}lemma{\isacharcolon}{\kern0pt}\isanewline
\ \ \isakeyword{assumes}\ {\isachardoublequoteopen}f\ {\isacharcolon}{\kern0pt}\ X\ {\isasymrightarrow}\ Y{\isachardoublequoteclose}\isanewline
\ \ \isakeyword{assumes}\ {\isachardoublequoteopen}x\ \ {\isasymin}\isactrlsub c\ X{\isachardoublequoteclose}\isanewline
\ \ \isakeyword{shows}\ {\isachardoublequoteopen}eval{\isacharunderscore}{\kern0pt}func\ Y\ X\ {\isasymcirc}\isactrlsub c\ {\isasymlangle}x{\isacharcomma}{\kern0pt}\ metafunc\ f{\isasymrangle}\ {\isacharequal}{\kern0pt}\ f\ {\isasymcirc}\isactrlsub c\ x{\isachardoublequoteclose}\isanewline
%
\isadelimproof
%
\endisadelimproof
%
\isatagproof
\isacommand{proof}\isamarkupfalse%
\ {\isacharminus}{\kern0pt}\ \isanewline
\ \ \isacommand{have}\isamarkupfalse%
\ {\isachardoublequoteopen}eval{\isacharunderscore}{\kern0pt}func\ Y\ X\ {\isasymcirc}\isactrlsub c\ {\isasymlangle}x{\isacharcomma}{\kern0pt}\ metafunc\ f{\isasymrangle}\ {\isacharequal}{\kern0pt}\ eval{\isacharunderscore}{\kern0pt}func\ Y\ X\ {\isasymcirc}\isactrlsub c\ {\isacharparenleft}{\kern0pt}id\ X\ {\isasymtimes}\isactrlsub f\ {\isacharparenleft}{\kern0pt}f\ {\isasymcirc}\isactrlsub c\ {\isacharparenleft}{\kern0pt}left{\isacharunderscore}{\kern0pt}cart{\isacharunderscore}{\kern0pt}proj\ X\ {\isasymone}{\isacharparenright}{\kern0pt}{\isacharparenright}{\kern0pt}\isactrlsup {\isasymsharp}{\isacharparenright}{\kern0pt}\ {\isasymcirc}\isactrlsub c\ {\isasymlangle}x{\isacharcomma}{\kern0pt}\ id\ {\isasymone}{\isasymrangle}{\isachardoublequoteclose}\isanewline
\ \ \ \ \isacommand{using}\isamarkupfalse%
\ assms\ \isacommand{by}\isamarkupfalse%
\ {\isacharparenleft}{\kern0pt}typecheck{\isacharunderscore}{\kern0pt}cfuncs{\isacharcomma}{\kern0pt}\ simp\ add{\isacharcolon}{\kern0pt}\ cfunc{\isacharunderscore}{\kern0pt}cross{\isacharunderscore}{\kern0pt}prod{\isacharunderscore}{\kern0pt}comp{\isacharunderscore}{\kern0pt}cfunc{\isacharunderscore}{\kern0pt}prod\ id{\isacharunderscore}{\kern0pt}left{\isacharunderscore}{\kern0pt}unit{\isadigit{2}}\ id{\isacharunderscore}{\kern0pt}right{\isacharunderscore}{\kern0pt}unit{\isadigit{2}}\ metafunc{\isacharunderscore}{\kern0pt}def{\isadigit{2}}{\isacharparenright}{\kern0pt}\isanewline
\ \ \isacommand{also}\isamarkupfalse%
\ \isacommand{have}\isamarkupfalse%
\ {\isachardoublequoteopen}{\isachardot}{\kern0pt}{\isachardot}{\kern0pt}{\isachardot}{\kern0pt}\ {\isacharequal}{\kern0pt}\ {\isacharparenleft}{\kern0pt}eval{\isacharunderscore}{\kern0pt}func\ Y\ X\ {\isasymcirc}\isactrlsub c\ {\isacharparenleft}{\kern0pt}id\ X\ {\isasymtimes}\isactrlsub f\ {\isacharparenleft}{\kern0pt}f\ {\isasymcirc}\isactrlsub c\ {\isacharparenleft}{\kern0pt}left{\isacharunderscore}{\kern0pt}cart{\isacharunderscore}{\kern0pt}proj\ X\ {\isasymone}{\isacharparenright}{\kern0pt}{\isacharparenright}{\kern0pt}\isactrlsup {\isasymsharp}{\isacharparenright}{\kern0pt}{\isacharparenright}{\kern0pt}\ {\isasymcirc}\isactrlsub c\ {\isasymlangle}x{\isacharcomma}{\kern0pt}\ id\ {\isasymone}{\isasymrangle}{\isachardoublequoteclose}\isanewline
\ \ \ \ \isacommand{using}\isamarkupfalse%
\ assms\ comp{\isacharunderscore}{\kern0pt}associative{\isadigit{2}}\ \isacommand{by}\isamarkupfalse%
\ {\isacharparenleft}{\kern0pt}typecheck{\isacharunderscore}{\kern0pt}cfuncs{\isacharcomma}{\kern0pt}\ blast{\isacharparenright}{\kern0pt}\isanewline
\ \ \isacommand{also}\isamarkupfalse%
\ \isacommand{have}\isamarkupfalse%
\ {\isachardoublequoteopen}{\isachardot}{\kern0pt}{\isachardot}{\kern0pt}{\isachardot}{\kern0pt}\ {\isacharequal}{\kern0pt}\ {\isacharparenleft}{\kern0pt}f\ {\isasymcirc}\isactrlsub c\ {\isacharparenleft}{\kern0pt}left{\isacharunderscore}{\kern0pt}cart{\isacharunderscore}{\kern0pt}proj\ X\ {\isasymone}{\isacharparenright}{\kern0pt}{\isacharparenright}{\kern0pt}\ {\isasymcirc}\isactrlsub c\ {\isasymlangle}x{\isacharcomma}{\kern0pt}\ id\ {\isasymone}{\isasymrangle}{\isachardoublequoteclose}\isanewline
\ \ \ \ \isacommand{using}\isamarkupfalse%
\ assms\ \isacommand{by}\isamarkupfalse%
\ {\isacharparenleft}{\kern0pt}typecheck{\isacharunderscore}{\kern0pt}cfuncs{\isacharcomma}{\kern0pt}\ metis\ transpose{\isacharunderscore}{\kern0pt}func{\isacharunderscore}{\kern0pt}def{\isacharparenright}{\kern0pt}\isanewline
\ \ \isacommand{also}\isamarkupfalse%
\ \isacommand{have}\isamarkupfalse%
\ {\isachardoublequoteopen}{\isachardot}{\kern0pt}{\isachardot}{\kern0pt}{\isachardot}{\kern0pt}\ {\isacharequal}{\kern0pt}\ f\ {\isasymcirc}\isactrlsub c\ x{\isachardoublequoteclose}\isanewline
\ \ \ \ \isacommand{by}\isamarkupfalse%
\ {\isacharparenleft}{\kern0pt}typecheck{\isacharunderscore}{\kern0pt}cfuncs{\isacharcomma}{\kern0pt}\ metis\ assms\ cfunc{\isacharunderscore}{\kern0pt}type{\isacharunderscore}{\kern0pt}def\ comp{\isacharunderscore}{\kern0pt}associative\ left{\isacharunderscore}{\kern0pt}cart{\isacharunderscore}{\kern0pt}proj{\isacharunderscore}{\kern0pt}cfunc{\isacharunderscore}{\kern0pt}prod{\isacharparenright}{\kern0pt}\isanewline
\ \ \isacommand{then}\isamarkupfalse%
\ \isacommand{show}\isamarkupfalse%
\ {\isachardoublequoteopen}eval{\isacharunderscore}{\kern0pt}func\ Y\ X\ {\isasymcirc}\isactrlsub c\ {\isasymlangle}x{\isacharcomma}{\kern0pt}\ metafunc\ f{\isasymrangle}\ {\isacharequal}{\kern0pt}\ f\ {\isasymcirc}\isactrlsub c\ x{\isachardoublequoteclose}\isanewline
\ \ \ \ \isacommand{by}\isamarkupfalse%
\ {\isacharparenleft}{\kern0pt}simp\ add{\isacharcolon}{\kern0pt}\ calculation{\isacharparenright}{\kern0pt}\isanewline
\isacommand{qed}\isamarkupfalse%
%
\endisatagproof
{\isafoldproof}%
%
\isadelimproof
%
\endisadelimproof
%
\isadelimdocument
%
\endisadelimdocument
%
\isatagdocument
%
\isamarkupsubsubsection{Inverse Metafunctions (Cnufatems)%
}
\isamarkuptrue%
%
\endisatagdocument
{\isafolddocument}%
%
\isadelimdocument
%
\endisadelimdocument
\isacommand{definition}\isamarkupfalse%
\ cnufatem\ {\isacharcolon}{\kern0pt}{\isacharcolon}{\kern0pt}\ {\isachardoublequoteopen}cfunc\ {\isasymRightarrow}\ cfunc{\isachardoublequoteclose}\ \isakeyword{where}\isanewline
\ \ {\isachardoublequoteopen}cnufatem\ f\ {\isacharequal}{\kern0pt}\ {\isacharparenleft}{\kern0pt}THE\ g{\isachardot}{\kern0pt}\ {\isasymforall}\ Y\ X{\isachardot}{\kern0pt}\ f\ {\isacharcolon}{\kern0pt}\ {\isasymone}\ {\isasymrightarrow}\ Y\isactrlbsup X\isactrlesup \ {\isasymlongrightarrow}\ g\ {\isacharequal}{\kern0pt}\ eval{\isacharunderscore}{\kern0pt}func\ Y\ X\ {\isasymcirc}\isactrlsub c\ {\isasymlangle}id\ X{\isacharcomma}{\kern0pt}\ f\ {\isasymcirc}\isactrlsub c\ {\isasymbeta}\isactrlbsub X\isactrlesub {\isasymrangle}{\isacharparenright}{\kern0pt}{\isachardoublequoteclose}\isanewline
\isanewline
\isacommand{lemma}\isamarkupfalse%
\ cnufatem{\isacharunderscore}{\kern0pt}def{\isadigit{2}}{\isacharcolon}{\kern0pt}\isanewline
\ \ \isakeyword{assumes}\ {\isachardoublequoteopen}f\ {\isasymin}\isactrlsub c\ Y\isactrlbsup X\isactrlesup {\isachardoublequoteclose}\isanewline
\ \ \isakeyword{shows}\ {\isachardoublequoteopen}cnufatem\ f\ {\isacharequal}{\kern0pt}\ eval{\isacharunderscore}{\kern0pt}func\ Y\ X\ {\isasymcirc}\isactrlsub c\ {\isasymlangle}id\ X{\isacharcomma}{\kern0pt}\ f\ {\isasymcirc}\isactrlsub c\ {\isasymbeta}\isactrlbsub X\isactrlesub {\isasymrangle}{\isachardoublequoteclose}\isanewline
%
\isadelimproof
\ \ %
\endisadelimproof
%
\isatagproof
\isacommand{using}\isamarkupfalse%
\ assms\ \isacommand{unfolding}\isamarkupfalse%
\ cnufatem{\isacharunderscore}{\kern0pt}def\ cfunc{\isacharunderscore}{\kern0pt}type{\isacharunderscore}{\kern0pt}def\isanewline
\ \ \isacommand{by}\isamarkupfalse%
\ {\isacharparenleft}{\kern0pt}smt\ {\isacharparenleft}{\kern0pt}verit{\isacharcomma}{\kern0pt}\ ccfv{\isacharunderscore}{\kern0pt}threshold{\isacharparenright}{\kern0pt}\ exp{\isacharunderscore}{\kern0pt}set{\isacharunderscore}{\kern0pt}inj\ theI{\isacharprime}{\kern0pt}{\isacharparenright}{\kern0pt}%
\endisatagproof
{\isafoldproof}%
%
\isadelimproof
\ \isanewline
%
\endisadelimproof
\isanewline
\isacommand{lemma}\isamarkupfalse%
\ cnufatem{\isacharunderscore}{\kern0pt}type{\isacharbrackleft}{\kern0pt}type{\isacharunderscore}{\kern0pt}rule{\isacharbrackright}{\kern0pt}{\isacharcolon}{\kern0pt}\isanewline
\ \ \isakeyword{assumes}\ {\isachardoublequoteopen}f\ {\isasymin}\isactrlsub c\ Y\isactrlbsup X\isactrlesup {\isachardoublequoteclose}\isanewline
\ \ \isakeyword{shows}\ {\isachardoublequoteopen}cnufatem\ f\ {\isacharcolon}{\kern0pt}\ X\ \ {\isasymrightarrow}\ Y{\isachardoublequoteclose}\isanewline
%
\isadelimproof
\ \ %
\endisadelimproof
%
\isatagproof
\isacommand{using}\isamarkupfalse%
\ assms\ cnufatem{\isacharunderscore}{\kern0pt}def{\isadigit{2}}\ \isanewline
\ \ \isacommand{by}\isamarkupfalse%
\ {\isacharparenleft}{\kern0pt}auto{\isacharcomma}{\kern0pt}\ typecheck{\isacharunderscore}{\kern0pt}cfuncs{\isacharparenright}{\kern0pt}%
\endisatagproof
{\isafoldproof}%
%
\isadelimproof
\isanewline
%
\endisadelimproof
\isanewline
\isacommand{lemma}\isamarkupfalse%
\ cnufatem{\isacharunderscore}{\kern0pt}metafunc{\isacharcolon}{\kern0pt}\isanewline
\ \ \isakeyword{assumes}\ f{\isacharunderscore}{\kern0pt}type{\isacharbrackleft}{\kern0pt}type{\isacharunderscore}{\kern0pt}rule{\isacharbrackright}{\kern0pt}{\isacharcolon}{\kern0pt}\ {\isachardoublequoteopen}f\ {\isacharcolon}{\kern0pt}\ X\ {\isasymrightarrow}\ Y{\isachardoublequoteclose}\isanewline
\ \ \isakeyword{shows}\ {\isachardoublequoteopen}cnufatem\ {\isacharparenleft}{\kern0pt}metafunc\ f{\isacharparenright}{\kern0pt}\ {\isacharequal}{\kern0pt}\ f{\isachardoublequoteclose}\isanewline
%
\isadelimproof
%
\endisadelimproof
%
\isatagproof
\isacommand{proof}\isamarkupfalse%
{\isacharparenleft}{\kern0pt}etcs{\isacharunderscore}{\kern0pt}rule\ one{\isacharunderscore}{\kern0pt}separator{\isacharparenright}{\kern0pt}\isanewline
\ \ \isacommand{fix}\isamarkupfalse%
\ x\isanewline
\ \ \isacommand{assume}\isamarkupfalse%
\ x{\isacharunderscore}{\kern0pt}type{\isacharbrackleft}{\kern0pt}type{\isacharunderscore}{\kern0pt}rule{\isacharbrackright}{\kern0pt}{\isacharcolon}{\kern0pt}\ {\isachardoublequoteopen}x\ {\isasymin}\isactrlsub c\ X{\isachardoublequoteclose}\isanewline
\isanewline
\ \ \isacommand{have}\isamarkupfalse%
\ {\isachardoublequoteopen}cnufatem\ {\isacharparenleft}{\kern0pt}metafunc\ f{\isacharparenright}{\kern0pt}\ {\isasymcirc}\isactrlsub c\ x\ {\isacharequal}{\kern0pt}\ \ eval{\isacharunderscore}{\kern0pt}func\ Y\ X\ {\isasymcirc}\isactrlsub c\ {\isasymlangle}id\ X{\isacharcomma}{\kern0pt}\ {\isacharparenleft}{\kern0pt}metafunc\ f{\isacharparenright}{\kern0pt}\ {\isasymcirc}\isactrlsub c\ {\isasymbeta}\isactrlbsub X\isactrlesub {\isasymrangle}\ {\isasymcirc}\isactrlsub c\ x{\isachardoublequoteclose}\isanewline
\ \ \ \ \isacommand{using}\isamarkupfalse%
\ cnufatem{\isacharunderscore}{\kern0pt}def{\isadigit{2}}\ comp{\isacharunderscore}{\kern0pt}associative{\isadigit{2}}\ \isacommand{by}\isamarkupfalse%
\ {\isacharparenleft}{\kern0pt}typecheck{\isacharunderscore}{\kern0pt}cfuncs{\isacharcomma}{\kern0pt}\ fastforce{\isacharparenright}{\kern0pt}\isanewline
\ \ \isacommand{also}\isamarkupfalse%
\ \isacommand{have}\isamarkupfalse%
\ {\isachardoublequoteopen}{\isachardot}{\kern0pt}{\isachardot}{\kern0pt}{\isachardot}{\kern0pt}\ {\isacharequal}{\kern0pt}\ eval{\isacharunderscore}{\kern0pt}func\ Y\ X\ {\isasymcirc}\isactrlsub c\ {\isasymlangle}x{\isacharcomma}{\kern0pt}\ {\isacharparenleft}{\kern0pt}metafunc\ f{\isacharparenright}{\kern0pt}{\isasymrangle}{\isachardoublequoteclose}\isanewline
\ \ \ \ \isacommand{by}\isamarkupfalse%
\ {\isacharparenleft}{\kern0pt}typecheck{\isacharunderscore}{\kern0pt}cfuncs{\isacharcomma}{\kern0pt}\ metis\ cart{\isacharunderscore}{\kern0pt}prod{\isacharunderscore}{\kern0pt}extract{\isacharunderscore}{\kern0pt}left{\isacharparenright}{\kern0pt}\isanewline
\ \ \isacommand{also}\isamarkupfalse%
\ \isacommand{have}\isamarkupfalse%
\ {\isachardoublequoteopen}{\isachardot}{\kern0pt}{\isachardot}{\kern0pt}{\isachardot}{\kern0pt}\ {\isacharequal}{\kern0pt}\ f\ {\isasymcirc}\isactrlsub c\ x{\isachardoublequoteclose}\isanewline
\ \ \ \ \isacommand{using}\isamarkupfalse%
\ eval{\isacharunderscore}{\kern0pt}lemma\ \isacommand{by}\isamarkupfalse%
\ {\isacharparenleft}{\kern0pt}typecheck{\isacharunderscore}{\kern0pt}cfuncs{\isacharcomma}{\kern0pt}\ presburger{\isacharparenright}{\kern0pt}\isanewline
\ \ \isacommand{then}\isamarkupfalse%
\ \isacommand{show}\isamarkupfalse%
\ {\isachardoublequoteopen}cnufatem\ {\isacharparenleft}{\kern0pt}metafunc\ f{\isacharparenright}{\kern0pt}\ {\isasymcirc}\isactrlsub c\ x\ {\isacharequal}{\kern0pt}\ f\ {\isasymcirc}\isactrlsub c\ x{\isachardoublequoteclose}\isanewline
\ \ \ \ \isacommand{by}\isamarkupfalse%
\ {\isacharparenleft}{\kern0pt}simp\ add{\isacharcolon}{\kern0pt}\ calculation{\isacharparenright}{\kern0pt}\isanewline
\isacommand{qed}\isamarkupfalse%
%
\endisatagproof
{\isafoldproof}%
%
\isadelimproof
\isanewline
%
\endisadelimproof
\isanewline
\isacommand{lemma}\isamarkupfalse%
\ metafunc{\isacharunderscore}{\kern0pt}cnufatem{\isacharcolon}{\kern0pt}\isanewline
\ \ \isakeyword{assumes}\ f{\isacharunderscore}{\kern0pt}type{\isacharbrackleft}{\kern0pt}type{\isacharunderscore}{\kern0pt}rule{\isacharbrackright}{\kern0pt}{\isacharcolon}{\kern0pt}\ {\isachardoublequoteopen}f\ {\isasymin}\isactrlsub c\ Y\isactrlbsup X\isactrlesup {\isachardoublequoteclose}\isanewline
\ \ \isakeyword{shows}\ {\isachardoublequoteopen}metafunc\ {\isacharparenleft}{\kern0pt}cnufatem\ f{\isacharparenright}{\kern0pt}\ {\isacharequal}{\kern0pt}\ f{\isachardoublequoteclose}\isanewline
%
\isadelimproof
%
\endisadelimproof
%
\isatagproof
\isacommand{proof}\isamarkupfalse%
\ {\isacharparenleft}{\kern0pt}etcs{\isacharunderscore}{\kern0pt}rule\ same{\isacharunderscore}{\kern0pt}evals{\isacharunderscore}{\kern0pt}equal{\isacharbrackleft}{\kern0pt}\isakeyword{where}\ X\ {\isacharequal}{\kern0pt}\ Y{\isacharcomma}{\kern0pt}\ \isakeyword{where}\ A\ {\isacharequal}{\kern0pt}\ X{\isacharbrackright}{\kern0pt}{\isacharcomma}{\kern0pt}\ etcs{\isacharunderscore}{\kern0pt}rule\ one{\isacharunderscore}{\kern0pt}separator{\isacharparenright}{\kern0pt}\isanewline
\ \ \isacommand{fix}\isamarkupfalse%
\ x{\isadigit{1}}\isanewline
\ \ \isacommand{assume}\isamarkupfalse%
\ x{\isadigit{1}}{\isacharunderscore}{\kern0pt}type{\isacharbrackleft}{\kern0pt}type{\isacharunderscore}{\kern0pt}rule{\isacharbrackright}{\kern0pt}{\isacharcolon}{\kern0pt}\ {\isachardoublequoteopen}x{\isadigit{1}}\ {\isasymin}\isactrlsub c\ X\ {\isasymtimes}\isactrlsub c\ {\isasymone}{\isachardoublequoteclose}\isanewline
\ \ \isacommand{then}\isamarkupfalse%
\ \isacommand{obtain}\isamarkupfalse%
\ x\ \isakeyword{where}\ x{\isacharunderscore}{\kern0pt}type{\isacharbrackleft}{\kern0pt}type{\isacharunderscore}{\kern0pt}rule{\isacharbrackright}{\kern0pt}{\isacharcolon}{\kern0pt}\ {\isachardoublequoteopen}x\ {\isasymin}\isactrlsub c\ X{\isachardoublequoteclose}\ \isakeyword{and}\ x{\isacharunderscore}{\kern0pt}def{\isacharcolon}{\kern0pt}\ {\isachardoublequoteopen}\ x{\isadigit{1}}\ {\isacharequal}{\kern0pt}\ {\isasymlangle}x{\isacharcomma}{\kern0pt}\ id\ {\isasymone}{\isasymrangle}{\isachardoublequoteclose}\isanewline
\ \ \ \ \isacommand{by}\isamarkupfalse%
\ {\isacharparenleft}{\kern0pt}typecheck{\isacharunderscore}{\kern0pt}cfuncs{\isacharcomma}{\kern0pt}\ metis\ cart{\isacharunderscore}{\kern0pt}prod{\isacharunderscore}{\kern0pt}decomp\ one{\isacharunderscore}{\kern0pt}unique{\isacharunderscore}{\kern0pt}element{\isacharparenright}{\kern0pt}\isanewline
\ \ \isacommand{have}\isamarkupfalse%
\ {\isachardoublequoteopen}{\isacharparenleft}{\kern0pt}eval{\isacharunderscore}{\kern0pt}func\ Y\ X\ {\isasymcirc}\isactrlsub c\ id\isactrlsub c\ X\ {\isasymtimes}\isactrlsub f\ metafunc\ {\isacharparenleft}{\kern0pt}cnufatem\ f{\isacharparenright}{\kern0pt}{\isacharparenright}{\kern0pt}\ {\isasymcirc}\isactrlsub c\ {\isasymlangle}x{\isacharcomma}{\kern0pt}\ id\ {\isasymone}{\isasymrangle}\ {\isacharequal}{\kern0pt}\isanewline
\ \ \ \ \ \ \ \ \ eval{\isacharunderscore}{\kern0pt}func\ Y\ X\ {\isasymcirc}\isactrlsub c\ {\isasymlangle}x\ {\isacharcomma}{\kern0pt}\ metafunc\ {\isacharparenleft}{\kern0pt}cnufatem\ f{\isacharparenright}{\kern0pt}{\isasymrangle}{\isachardoublequoteclose}\isanewline
\ \ \ \ \isacommand{by}\isamarkupfalse%
\ {\isacharparenleft}{\kern0pt}typecheck{\isacharunderscore}{\kern0pt}cfuncs{\isacharcomma}{\kern0pt}\ smt\ {\isacharparenleft}{\kern0pt}z{\isadigit{3}}{\isacharparenright}{\kern0pt}\ cfunc{\isacharunderscore}{\kern0pt}cross{\isacharunderscore}{\kern0pt}prod{\isacharunderscore}{\kern0pt}comp{\isacharunderscore}{\kern0pt}cfunc{\isacharunderscore}{\kern0pt}prod\ comp{\isacharunderscore}{\kern0pt}associative{\isadigit{2}}\ id{\isacharunderscore}{\kern0pt}left{\isacharunderscore}{\kern0pt}unit{\isadigit{2}}\ id{\isacharunderscore}{\kern0pt}right{\isacharunderscore}{\kern0pt}unit{\isadigit{2}}{\isacharparenright}{\kern0pt}\isanewline
\ \ \isacommand{also}\isamarkupfalse%
\ \isacommand{have}\isamarkupfalse%
\ {\isachardoublequoteopen}{\isachardot}{\kern0pt}{\isachardot}{\kern0pt}{\isachardot}{\kern0pt}\ {\isacharequal}{\kern0pt}\ {\isacharparenleft}{\kern0pt}cnufatem\ f{\isacharparenright}{\kern0pt}\ {\isasymcirc}\isactrlsub c\ x{\isachardoublequoteclose}\isanewline
\ \ \ \ \isacommand{using}\isamarkupfalse%
\ eval{\isacharunderscore}{\kern0pt}lemma\ \isacommand{by}\isamarkupfalse%
\ {\isacharparenleft}{\kern0pt}typecheck{\isacharunderscore}{\kern0pt}cfuncs{\isacharcomma}{\kern0pt}\ presburger{\isacharparenright}{\kern0pt}\isanewline
\ \ \isacommand{also}\isamarkupfalse%
\ \isacommand{have}\isamarkupfalse%
\ {\isachardoublequoteopen}{\isachardot}{\kern0pt}{\isachardot}{\kern0pt}{\isachardot}{\kern0pt}\ {\isacharequal}{\kern0pt}\ {\isacharparenleft}{\kern0pt}eval{\isacharunderscore}{\kern0pt}func\ Y\ X\ {\isasymcirc}\isactrlsub c\ {\isasymlangle}id\ X{\isacharcomma}{\kern0pt}\ f\ {\isasymcirc}\isactrlsub c\ {\isasymbeta}\isactrlbsub X\isactrlesub {\isasymrangle}{\isacharparenright}{\kern0pt}\ {\isasymcirc}\isactrlsub c\ x{\isachardoublequoteclose}\isanewline
\ \ \ \ \isacommand{using}\isamarkupfalse%
\ assms\ cnufatem{\isacharunderscore}{\kern0pt}def{\isadigit{2}}\ \isacommand{by}\isamarkupfalse%
\ presburger\isanewline
\ \ \isacommand{also}\isamarkupfalse%
\ \isacommand{have}\isamarkupfalse%
\ {\isachardoublequoteopen}{\isachardot}{\kern0pt}{\isachardot}{\kern0pt}{\isachardot}{\kern0pt}\ {\isacharequal}{\kern0pt}\ eval{\isacharunderscore}{\kern0pt}func\ Y\ X\ {\isasymcirc}\isactrlsub c\ {\isasymlangle}id\ X{\isacharcomma}{\kern0pt}\ f\ {\isasymcirc}\isactrlsub c\ {\isasymbeta}\isactrlbsub X\isactrlesub {\isasymrangle}\ {\isasymcirc}\isactrlsub c\ x{\isachardoublequoteclose}\isanewline
\ \ \ \ \isacommand{by}\isamarkupfalse%
\ {\isacharparenleft}{\kern0pt}typecheck{\isacharunderscore}{\kern0pt}cfuncs{\isacharcomma}{\kern0pt}\ metis\ comp{\isacharunderscore}{\kern0pt}associative{\isadigit{2}}{\isacharparenright}{\kern0pt}\isanewline
\ \ \isacommand{also}\isamarkupfalse%
\ \isacommand{have}\isamarkupfalse%
\ {\isachardoublequoteopen}{\isachardot}{\kern0pt}{\isachardot}{\kern0pt}{\isachardot}{\kern0pt}\ {\isacharequal}{\kern0pt}\ eval{\isacharunderscore}{\kern0pt}func\ Y\ X\ {\isasymcirc}\isactrlsub c\ {\isasymlangle}id\ X\ {\isasymcirc}\isactrlsub c\ x\ {\isacharcomma}{\kern0pt}\ f\ {\isasymcirc}\isactrlsub c\ {\isacharparenleft}{\kern0pt}{\isasymbeta}\isactrlbsub X\isactrlesub \ {\isasymcirc}\isactrlsub c\ x{\isacharparenright}{\kern0pt}{\isasymrangle}{\isachardoublequoteclose}\isanewline
\ \ \ \ \isacommand{by}\isamarkupfalse%
\ {\isacharparenleft}{\kern0pt}typecheck{\isacharunderscore}{\kern0pt}cfuncs{\isacharcomma}{\kern0pt}\ metis\ cart{\isacharunderscore}{\kern0pt}prod{\isacharunderscore}{\kern0pt}extract{\isacharunderscore}{\kern0pt}left\ id{\isacharunderscore}{\kern0pt}left{\isacharunderscore}{\kern0pt}unit{\isadigit{2}}\ id{\isacharunderscore}{\kern0pt}right{\isacharunderscore}{\kern0pt}unit{\isadigit{2}}\ terminal{\isacharunderscore}{\kern0pt}func{\isacharunderscore}{\kern0pt}comp{\isacharunderscore}{\kern0pt}elem{\isacharparenright}{\kern0pt}\isanewline
\ \ \isacommand{also}\isamarkupfalse%
\ \isacommand{have}\isamarkupfalse%
\ {\isachardoublequoteopen}{\isachardot}{\kern0pt}{\isachardot}{\kern0pt}{\isachardot}{\kern0pt}\ {\isacharequal}{\kern0pt}\ eval{\isacharunderscore}{\kern0pt}func\ Y\ X\ {\isasymcirc}\isactrlsub c\ {\isasymlangle}id\ X\ {\isasymcirc}\isactrlsub c\ x\ {\isacharcomma}{\kern0pt}\ f\ {\isasymcirc}\isactrlsub c\ id\ {\isasymone}{\isasymrangle}{\isachardoublequoteclose}\isanewline
\ \ \ \ \isacommand{by}\isamarkupfalse%
\ {\isacharparenleft}{\kern0pt}simp\ add{\isacharcolon}{\kern0pt}\ terminal{\isacharunderscore}{\kern0pt}func{\isacharunderscore}{\kern0pt}comp{\isacharunderscore}{\kern0pt}elem\ x{\isacharunderscore}{\kern0pt}type{\isacharparenright}{\kern0pt}\isanewline
\ \ \isacommand{also}\isamarkupfalse%
\ \isacommand{have}\isamarkupfalse%
\ {\isachardoublequoteopen}{\isachardot}{\kern0pt}{\isachardot}{\kern0pt}{\isachardot}{\kern0pt}\ {\isacharequal}{\kern0pt}\ eval{\isacharunderscore}{\kern0pt}func\ Y\ X\ {\isasymcirc}\isactrlsub c\ {\isacharparenleft}{\kern0pt}id\isactrlsub c\ X\ {\isasymtimes}\isactrlsub f\ f{\isacharparenright}{\kern0pt}\ {\isasymcirc}\isactrlsub c\ {\isasymlangle}x{\isacharcomma}{\kern0pt}\ id\ {\isasymone}{\isasymrangle}{\isachardoublequoteclose}\isanewline
\ \ \ \ \isacommand{using}\isamarkupfalse%
\ cfunc{\isacharunderscore}{\kern0pt}cross{\isacharunderscore}{\kern0pt}prod{\isacharunderscore}{\kern0pt}comp{\isacharunderscore}{\kern0pt}cfunc{\isacharunderscore}{\kern0pt}prod\ \isacommand{by}\isamarkupfalse%
\ {\isacharparenleft}{\kern0pt}typecheck{\isacharunderscore}{\kern0pt}cfuncs{\isacharcomma}{\kern0pt}\ force{\isacharparenright}{\kern0pt}\isanewline
\ \ \isacommand{also}\isamarkupfalse%
\ \isacommand{have}\isamarkupfalse%
\ {\isachardoublequoteopen}{\isachardot}{\kern0pt}{\isachardot}{\kern0pt}{\isachardot}{\kern0pt}\ {\isacharequal}{\kern0pt}\ {\isacharparenleft}{\kern0pt}eval{\isacharunderscore}{\kern0pt}func\ Y\ X\ {\isasymcirc}\isactrlsub c\ id\isactrlsub c\ X\ {\isasymtimes}\isactrlsub f\ f{\isacharparenright}{\kern0pt}\ {\isasymcirc}\isactrlsub c\ x{\isadigit{1}}{\isachardoublequoteclose}\isanewline
\ \ \ \ \isacommand{by}\isamarkupfalse%
\ {\isacharparenleft}{\kern0pt}typecheck{\isacharunderscore}{\kern0pt}cfuncs{\isacharcomma}{\kern0pt}\ metis\ comp{\isacharunderscore}{\kern0pt}associative{\isadigit{2}}\ x{\isacharunderscore}{\kern0pt}def{\isacharparenright}{\kern0pt}\isanewline
\ \ \isacommand{then}\isamarkupfalse%
\ \isacommand{show}\isamarkupfalse%
\ {\isachardoublequoteopen}{\isacharparenleft}{\kern0pt}eval{\isacharunderscore}{\kern0pt}func\ Y\ X\ {\isasymcirc}\isactrlsub c\ id\isactrlsub c\ X\ {\isasymtimes}\isactrlsub f\ metafunc\ {\isacharparenleft}{\kern0pt}cnufatem\ f{\isacharparenright}{\kern0pt}{\isacharparenright}{\kern0pt}\ {\isasymcirc}\isactrlsub c\ x{\isadigit{1}}\ {\isacharequal}{\kern0pt}\ {\isacharparenleft}{\kern0pt}eval{\isacharunderscore}{\kern0pt}func\ Y\ X\ {\isasymcirc}\isactrlsub c\ id\isactrlsub c\ X\ {\isasymtimes}\isactrlsub f\ f{\isacharparenright}{\kern0pt}\ {\isasymcirc}\isactrlsub c\ x{\isadigit{1}}{\isachardoublequoteclose}\isanewline
\ \ \ \ \isacommand{using}\isamarkupfalse%
\ \ calculation\ x{\isacharunderscore}{\kern0pt}def\ \isacommand{by}\isamarkupfalse%
\ presburger\isanewline
\isacommand{qed}\isamarkupfalse%
%
\endisatagproof
{\isafoldproof}%
%
\isadelimproof
%
\endisadelimproof
%
\isadelimdocument
%
\endisadelimdocument
%
\isatagdocument
%
\isamarkupsubsubsection{Metafunction Composition%
}
\isamarkuptrue%
%
\endisatagdocument
{\isafolddocument}%
%
\isadelimdocument
%
\endisadelimdocument
\isacommand{definition}\isamarkupfalse%
\ meta{\isacharunderscore}{\kern0pt}comp\ {\isacharcolon}{\kern0pt}{\isacharcolon}{\kern0pt}\ {\isachardoublequoteopen}cset\ {\isasymRightarrow}\ cset\ {\isasymRightarrow}\ cset\ {\isasymRightarrow}\ cfunc{\isachardoublequoteclose}\ \ \isakeyword{where}\ \isanewline
\ \ {\isachardoublequoteopen}meta{\isacharunderscore}{\kern0pt}comp\ X\ Y\ Z\ \ {\isacharequal}{\kern0pt}\ {\isacharparenleft}{\kern0pt}eval{\isacharunderscore}{\kern0pt}func\ Z\ Y\ {\isasymcirc}\isactrlsub c\ swap\ {\isacharparenleft}{\kern0pt}Z\isactrlbsup Y\isactrlesup {\isacharparenright}{\kern0pt}\ Y\ {\isasymcirc}\isactrlsub c\ {\isacharparenleft}{\kern0pt}id{\isacharparenleft}{\kern0pt}Z\isactrlbsup Y\isactrlesup {\isacharparenright}{\kern0pt}\ {\isasymtimes}\isactrlsub f\ {\isacharparenleft}{\kern0pt}eval{\isacharunderscore}{\kern0pt}func\ Y\ X\ {\isasymcirc}\isactrlsub c\ swap\ {\isacharparenleft}{\kern0pt}Y\isactrlbsup X\isactrlesup {\isacharparenright}{\kern0pt}\ X{\isacharparenright}{\kern0pt}{\isacharparenright}{\kern0pt}\ {\isasymcirc}\isactrlsub c\ {\isacharparenleft}{\kern0pt}associate{\isacharunderscore}{\kern0pt}right\ {\isacharparenleft}{\kern0pt}Z\isactrlbsup Y\isactrlesup {\isacharparenright}{\kern0pt}\ {\isacharparenleft}{\kern0pt}Y\isactrlbsup X\isactrlesup {\isacharparenright}{\kern0pt}\ X{\isacharparenright}{\kern0pt}\ {\isasymcirc}\isactrlsub c\ swap\ X\ {\isacharparenleft}{\kern0pt}{\isacharparenleft}{\kern0pt}Z\isactrlbsup Y\isactrlesup {\isacharparenright}{\kern0pt}\ {\isasymtimes}\isactrlsub c\ {\isacharparenleft}{\kern0pt}Y\isactrlbsup X\isactrlesup {\isacharparenright}{\kern0pt}{\isacharparenright}{\kern0pt}{\isacharparenright}{\kern0pt}\isactrlsup {\isasymsharp}{\isachardoublequoteclose}\isanewline
\isanewline
\isacommand{lemma}\isamarkupfalse%
\ meta{\isacharunderscore}{\kern0pt}comp{\isacharunderscore}{\kern0pt}type{\isacharbrackleft}{\kern0pt}type{\isacharunderscore}{\kern0pt}rule{\isacharbrackright}{\kern0pt}{\isacharcolon}{\kern0pt}\isanewline
\ \ {\isachardoublequoteopen}meta{\isacharunderscore}{\kern0pt}comp\ X\ Y\ Z\ {\isacharcolon}{\kern0pt}\ Z\isactrlbsup Y\isactrlesup \ {\isasymtimes}\isactrlsub c\ Y\isactrlbsup X\isactrlesup \ {\isasymrightarrow}\ Z\isactrlbsup X\isactrlesup {\isachardoublequoteclose}\isanewline
%
\isadelimproof
\ \ %
\endisadelimproof
%
\isatagproof
\isacommand{unfolding}\isamarkupfalse%
\ meta{\isacharunderscore}{\kern0pt}comp{\isacharunderscore}{\kern0pt}def\ \isacommand{by}\isamarkupfalse%
\ typecheck{\isacharunderscore}{\kern0pt}cfuncs%
\endisatagproof
{\isafoldproof}%
%
\isadelimproof
\isanewline
%
\endisadelimproof
\isanewline
\isacommand{definition}\isamarkupfalse%
\ meta{\isacharunderscore}{\kern0pt}comp{\isadigit{2}}\ {\isacharcolon}{\kern0pt}{\isacharcolon}{\kern0pt}\ {\isachardoublequoteopen}cfunc\ {\isasymRightarrow}\ cfunc\ {\isasymRightarrow}\ cfunc{\isachardoublequoteclose}\ {\isacharparenleft}{\kern0pt}\isakeyword{infixr}\ {\isachardoublequoteopen}{\isasymbox}{\isachardoublequoteclose}\ {\isadigit{5}}{\isadigit{5}}{\isacharparenright}{\kern0pt}\isanewline
\ \ \isakeyword{where}\ {\isachardoublequoteopen}meta{\isacharunderscore}{\kern0pt}comp{\isadigit{2}}\ f\ g\ {\isacharequal}{\kern0pt}\ {\isacharparenleft}{\kern0pt}THE\ h{\isachardot}{\kern0pt}\ {\isasymexists}\ W\ X\ Y{\isachardot}{\kern0pt}\ g\ {\isacharcolon}{\kern0pt}\ W\ {\isasymrightarrow}\ Y\isactrlbsup X\isactrlesup \ {\isasymand}\ h\ {\isacharequal}{\kern0pt}\ {\isacharparenleft}{\kern0pt}f\isactrlsup {\isasymflat}\ \ {\isasymcirc}\isactrlsub c\ {\isasymlangle}g\isactrlsup {\isasymflat}{\isacharcomma}{\kern0pt}\ right{\isacharunderscore}{\kern0pt}cart{\isacharunderscore}{\kern0pt}proj\ X\ W{\isasymrangle}{\isacharparenright}{\kern0pt}\isactrlsup {\isasymsharp}{\isacharparenright}{\kern0pt}{\isachardoublequoteclose}\isanewline
\isanewline
\isacommand{lemma}\isamarkupfalse%
\ meta{\isacharunderscore}{\kern0pt}comp{\isadigit{2}}{\isacharunderscore}{\kern0pt}def{\isadigit{2}}{\isacharcolon}{\kern0pt}\ \isanewline
\ \ \isakeyword{assumes}\ {\isachardoublequoteopen}f{\isacharcolon}{\kern0pt}\ W\ {\isasymrightarrow}\ Z\isactrlbsup Y\isactrlesup {\isachardoublequoteclose}\isanewline
\ \ \isakeyword{assumes}\ {\isachardoublequoteopen}g{\isacharcolon}{\kern0pt}\ W\ {\isasymrightarrow}\ Y\isactrlbsup X\isactrlesup {\isachardoublequoteclose}\isanewline
\ \ \isakeyword{shows}\ {\isachardoublequoteopen}f\ {\isasymbox}\ g\ \ {\isacharequal}{\kern0pt}\ {\isacharparenleft}{\kern0pt}f\isactrlsup {\isasymflat}\ \ {\isasymcirc}\isactrlsub c\ {\isasymlangle}g\isactrlsup {\isasymflat}{\isacharcomma}{\kern0pt}\ right{\isacharunderscore}{\kern0pt}cart{\isacharunderscore}{\kern0pt}proj\ X\ W{\isasymrangle}{\isacharparenright}{\kern0pt}\isactrlsup {\isasymsharp}{\isachardoublequoteclose}\isanewline
%
\isadelimproof
\ \ %
\endisadelimproof
%
\isatagproof
\isacommand{using}\isamarkupfalse%
\ assms\ \isacommand{unfolding}\isamarkupfalse%
\ meta{\isacharunderscore}{\kern0pt}comp{\isadigit{2}}{\isacharunderscore}{\kern0pt}def\isanewline
\ \ \isacommand{by}\isamarkupfalse%
\ {\isacharparenleft}{\kern0pt}smt\ {\isacharparenleft}{\kern0pt}z{\isadigit{3}}{\isacharparenright}{\kern0pt}\ cfunc{\isacharunderscore}{\kern0pt}type{\isacharunderscore}{\kern0pt}def\ exp{\isacharunderscore}{\kern0pt}set{\isacharunderscore}{\kern0pt}inj\ the{\isacharunderscore}{\kern0pt}equality{\isacharparenright}{\kern0pt}%
\endisatagproof
{\isafoldproof}%
%
\isadelimproof
\isanewline
%
\endisadelimproof
\isanewline
\isacommand{lemma}\isamarkupfalse%
\ meta{\isacharunderscore}{\kern0pt}comp{\isadigit{2}}{\isacharunderscore}{\kern0pt}type{\isacharbrackleft}{\kern0pt}type{\isacharunderscore}{\kern0pt}rule{\isacharbrackright}{\kern0pt}{\isacharcolon}{\kern0pt}\ \isanewline
\ \ \isakeyword{assumes}\ {\isachardoublequoteopen}f{\isacharcolon}{\kern0pt}\ W\ {\isasymrightarrow}\ Z\isactrlbsup Y\isactrlesup {\isachardoublequoteclose}\isanewline
\ \ \isakeyword{assumes}\ {\isachardoublequoteopen}g{\isacharcolon}{\kern0pt}\ W\ {\isasymrightarrow}\ Y\isactrlbsup X\isactrlesup {\isachardoublequoteclose}\isanewline
\ \ \isakeyword{shows}\ {\isachardoublequoteopen}f\ {\isasymbox}\ g\ {\isacharcolon}{\kern0pt}\ W\ {\isasymrightarrow}\ Z\isactrlbsup X\isactrlesup {\isachardoublequoteclose}\isanewline
%
\isadelimproof
%
\endisadelimproof
%
\isatagproof
\isacommand{proof}\isamarkupfalse%
\ {\isacharminus}{\kern0pt}\ \isanewline
\ \ \isacommand{have}\isamarkupfalse%
\ {\isachardoublequoteopen}{\isacharparenleft}{\kern0pt}f\isactrlsup {\isasymflat}\ \ {\isasymcirc}\isactrlsub c\ {\isasymlangle}g\isactrlsup {\isasymflat}{\isacharcomma}{\kern0pt}\ right{\isacharunderscore}{\kern0pt}cart{\isacharunderscore}{\kern0pt}proj\ X\ W{\isasymrangle}{\isacharparenright}{\kern0pt}\isactrlsup {\isasymsharp}\ {\isacharcolon}{\kern0pt}\ W\ {\isasymrightarrow}\ Z\isactrlbsup X\isactrlesup {\isachardoublequoteclose}\isanewline
\ \ \ \ \isacommand{using}\isamarkupfalse%
\ assms\ \isacommand{by}\isamarkupfalse%
\ typecheck{\isacharunderscore}{\kern0pt}cfuncs\isanewline
\ \ \isacommand{then}\isamarkupfalse%
\ \isacommand{show}\isamarkupfalse%
\ {\isacharquery}{\kern0pt}thesis\ \isanewline
\ \ \ \ \isacommand{using}\isamarkupfalse%
\ assms\ \isacommand{by}\isamarkupfalse%
\ {\isacharparenleft}{\kern0pt}simp\ add{\isacharcolon}{\kern0pt}\ meta{\isacharunderscore}{\kern0pt}comp{\isadigit{2}}{\isacharunderscore}{\kern0pt}def{\isadigit{2}}{\isacharparenright}{\kern0pt}\isanewline
\isacommand{qed}\isamarkupfalse%
%
\endisatagproof
{\isafoldproof}%
%
\isadelimproof
\isanewline
%
\endisadelimproof
\isanewline
\isacommand{lemma}\isamarkupfalse%
\ meta{\isacharunderscore}{\kern0pt}comp{\isadigit{2}}{\isacharunderscore}{\kern0pt}elements{\isacharunderscore}{\kern0pt}aux{\isacharcolon}{\kern0pt}\ \isanewline
\ \ \isakeyword{assumes}\ {\isachardoublequoteopen}f\ {\isasymin}\isactrlsub c\ Z\isactrlbsup Y\isactrlesup {\isachardoublequoteclose}\isanewline
\ \ \isakeyword{assumes}\ {\isachardoublequoteopen}g\ {\isasymin}\isactrlsub c\ Y\isactrlbsup X\isactrlesup {\isachardoublequoteclose}\isanewline
\ \ \isakeyword{assumes}\ {\isachardoublequoteopen}x\ {\isasymin}\isactrlsub c\ X{\isachardoublequoteclose}\isanewline
\ \ \isakeyword{shows}\ {\isachardoublequoteopen}{\isacharparenleft}{\kern0pt}f\isactrlsup {\isasymflat}\ {\isasymcirc}\isactrlsub c\ {\isasymlangle}g\isactrlsup {\isasymflat}{\isacharcomma}{\kern0pt}right{\isacharunderscore}{\kern0pt}cart{\isacharunderscore}{\kern0pt}proj\ X\ {\isasymone}{\isasymrangle}{\isacharparenright}{\kern0pt}\ \ {\isasymcirc}\isactrlsub c\ {\isasymlangle}x{\isacharcomma}{\kern0pt}\ id\isactrlsub c\ {\isasymone}{\isasymrangle}\ {\isacharequal}{\kern0pt}\ eval{\isacharunderscore}{\kern0pt}func\ Z\ Y\ {\isasymcirc}\isactrlsub c\ {\isasymlangle}eval{\isacharunderscore}{\kern0pt}func\ Y\ X\ {\isasymcirc}\isactrlsub c\ {\isasymlangle}x{\isacharcomma}{\kern0pt}g{\isasymrangle}{\isacharcomma}{\kern0pt}f{\isasymrangle}{\isachardoublequoteclose}\isanewline
%
\isadelimproof
%
\endisadelimproof
%
\isatagproof
\isacommand{proof}\isamarkupfalse%
{\isacharminus}{\kern0pt}\isanewline
\ \ \ \ \isacommand{have}\isamarkupfalse%
\ {\isachardoublequoteopen}{\isacharparenleft}{\kern0pt}f\isactrlsup {\isasymflat}\ {\isasymcirc}\isactrlsub c\ {\isasymlangle}g\isactrlsup {\isasymflat}{\isacharcomma}{\kern0pt}right{\isacharunderscore}{\kern0pt}cart{\isacharunderscore}{\kern0pt}proj\ X\ {\isasymone}{\isasymrangle}{\isacharparenright}{\kern0pt}\ \ {\isasymcirc}\isactrlsub c\ {\isasymlangle}x{\isacharcomma}{\kern0pt}\ id\isactrlsub c\ {\isasymone}{\isasymrangle}{\isacharequal}{\kern0pt}\ \ f\isactrlsup {\isasymflat}\ {\isasymcirc}\isactrlsub c\ {\isasymlangle}g\isactrlsup {\isasymflat}{\isacharcomma}{\kern0pt}right{\isacharunderscore}{\kern0pt}cart{\isacharunderscore}{\kern0pt}proj\ X\ {\isasymone}{\isasymrangle}\ \ {\isasymcirc}\isactrlsub c\ {\isasymlangle}x{\isacharcomma}{\kern0pt}\ id\isactrlsub c\ {\isasymone}{\isasymrangle}{\isachardoublequoteclose}\isanewline
\ \ \ \ \ \ \isacommand{using}\isamarkupfalse%
\ assms\ \isacommand{by}\isamarkupfalse%
\ {\isacharparenleft}{\kern0pt}typecheck{\isacharunderscore}{\kern0pt}cfuncs{\isacharcomma}{\kern0pt}\ simp\ add{\isacharcolon}{\kern0pt}\ comp{\isacharunderscore}{\kern0pt}associative{\isadigit{2}}{\isacharparenright}{\kern0pt}\isanewline
\ \ \ \ \isacommand{also}\isamarkupfalse%
\ \isacommand{have}\isamarkupfalse%
\ {\isachardoublequoteopen}{\isachardot}{\kern0pt}{\isachardot}{\kern0pt}{\isachardot}{\kern0pt}\ {\isacharequal}{\kern0pt}\ f\isactrlsup {\isasymflat}\ {\isasymcirc}\isactrlsub c\ {\isasymlangle}g\isactrlsup {\isasymflat}\ {\isasymcirc}\isactrlsub c\ {\isasymlangle}x{\isacharcomma}{\kern0pt}\ id\isactrlsub c\ {\isasymone}{\isasymrangle}{\isacharcomma}{\kern0pt}right{\isacharunderscore}{\kern0pt}cart{\isacharunderscore}{\kern0pt}proj\ X\ {\isasymone}\ {\isasymcirc}\isactrlsub c\ {\isasymlangle}x{\isacharcomma}{\kern0pt}\ id\isactrlsub c\ {\isasymone}{\isasymrangle}\ {\isasymrangle}{\isachardoublequoteclose}\isanewline
\ \ \ \ \ \ \isacommand{using}\isamarkupfalse%
\ assms\ \isacommand{by}\isamarkupfalse%
\ {\isacharparenleft}{\kern0pt}typecheck{\isacharunderscore}{\kern0pt}cfuncs{\isacharcomma}{\kern0pt}\ simp\ add{\isacharcolon}{\kern0pt}\ cfunc{\isacharunderscore}{\kern0pt}prod{\isacharunderscore}{\kern0pt}comp{\isacharparenright}{\kern0pt}\isanewline
\ \ \ \ \isacommand{also}\isamarkupfalse%
\ \isacommand{have}\isamarkupfalse%
\ {\isachardoublequoteopen}{\isachardot}{\kern0pt}{\isachardot}{\kern0pt}{\isachardot}{\kern0pt}\ {\isacharequal}{\kern0pt}\ f\isactrlsup {\isasymflat}\ {\isasymcirc}\isactrlsub c\ {\isasymlangle}g\isactrlsup {\isasymflat}\ {\isasymcirc}\isactrlsub c\ {\isasymlangle}x{\isacharcomma}{\kern0pt}\ id\isactrlsub c\ {\isasymone}{\isasymrangle}{\isacharcomma}{\kern0pt}id\isactrlsub c\ {\isasymone}{\isasymrangle}{\isachardoublequoteclose}\isanewline
\ \ \ \ \ \ \isacommand{using}\isamarkupfalse%
\ assms\ \isacommand{by}\isamarkupfalse%
\ {\isacharparenleft}{\kern0pt}typecheck{\isacharunderscore}{\kern0pt}cfuncs{\isacharcomma}{\kern0pt}\ metis\ one{\isacharunderscore}{\kern0pt}unique{\isacharunderscore}{\kern0pt}element{\isacharparenright}{\kern0pt}\isanewline
\ \ \ \ \isacommand{also}\isamarkupfalse%
\ \isacommand{have}\isamarkupfalse%
\ {\isachardoublequoteopen}{\isachardot}{\kern0pt}{\isachardot}{\kern0pt}{\isachardot}{\kern0pt}\ {\isacharequal}{\kern0pt}\ f\isactrlsup {\isasymflat}\ {\isasymcirc}\isactrlsub c\ {\isasymlangle}{\isacharparenleft}{\kern0pt}eval{\isacharunderscore}{\kern0pt}func\ Y\ X{\isacharparenright}{\kern0pt}\ {\isasymcirc}\isactrlsub c\ {\isacharparenleft}{\kern0pt}id\ X\ {\isasymtimes}\isactrlsub f\ g{\isacharparenright}{\kern0pt}\ {\isasymcirc}\isactrlsub c\ {\isasymlangle}x{\isacharcomma}{\kern0pt}\ id\isactrlsub c\ {\isasymone}{\isasymrangle}{\isacharcomma}{\kern0pt}id\isactrlsub c\ {\isasymone}{\isasymrangle}{\isachardoublequoteclose}\isanewline
\ \ \ \ \ \ \isacommand{using}\isamarkupfalse%
\ assms\ \isacommand{by}\isamarkupfalse%
\ {\isacharparenleft}{\kern0pt}typecheck{\isacharunderscore}{\kern0pt}cfuncs{\isacharcomma}{\kern0pt}\ simp\ add{\isacharcolon}{\kern0pt}\ comp{\isacharunderscore}{\kern0pt}associative{\isadigit{2}}\ inv{\isacharunderscore}{\kern0pt}transpose{\isacharunderscore}{\kern0pt}func{\isacharunderscore}{\kern0pt}def{\isadigit{3}}{\isacharparenright}{\kern0pt}\isanewline
\ \ \ \ \isacommand{also}\isamarkupfalse%
\ \isacommand{have}\isamarkupfalse%
\ {\isachardoublequoteopen}{\isachardot}{\kern0pt}{\isachardot}{\kern0pt}{\isachardot}{\kern0pt}\ {\isacharequal}{\kern0pt}\ f\isactrlsup {\isasymflat}\ {\isasymcirc}\isactrlsub c\ {\isasymlangle}{\isacharparenleft}{\kern0pt}eval{\isacharunderscore}{\kern0pt}func\ Y\ X{\isacharparenright}{\kern0pt}\ {\isasymcirc}\isactrlsub c\ \ {\isasymlangle}x{\isacharcomma}{\kern0pt}\ g{\isasymrangle}{\isacharcomma}{\kern0pt}id\isactrlsub c\ {\isasymone}{\isasymrangle}{\isachardoublequoteclose}\isanewline
\ \ \ \ \ \ \isacommand{using}\isamarkupfalse%
\ assms\ cfunc{\isacharunderscore}{\kern0pt}cross{\isacharunderscore}{\kern0pt}prod{\isacharunderscore}{\kern0pt}comp{\isacharunderscore}{\kern0pt}cfunc{\isacharunderscore}{\kern0pt}prod\ id{\isacharunderscore}{\kern0pt}left{\isacharunderscore}{\kern0pt}unit{\isadigit{2}}\ id{\isacharunderscore}{\kern0pt}right{\isacharunderscore}{\kern0pt}unit{\isadigit{2}}\ \isacommand{by}\isamarkupfalse%
\ {\isacharparenleft}{\kern0pt}typecheck{\isacharunderscore}{\kern0pt}cfuncs{\isacharcomma}{\kern0pt}force{\isacharparenright}{\kern0pt}\isanewline
\ \ \ \ \isacommand{also}\isamarkupfalse%
\ \isacommand{have}\isamarkupfalse%
\ {\isachardoublequoteopen}{\isachardot}{\kern0pt}{\isachardot}{\kern0pt}{\isachardot}{\kern0pt}\ {\isacharequal}{\kern0pt}\ {\isacharparenleft}{\kern0pt}eval{\isacharunderscore}{\kern0pt}func\ Z\ Y{\isacharparenright}{\kern0pt}\ {\isasymcirc}\isactrlsub c\ {\isacharparenleft}{\kern0pt}id\ Y\ {\isasymtimes}\isactrlsub f\ f{\isacharparenright}{\kern0pt}\ {\isasymcirc}\isactrlsub c\ {\isasymlangle}{\isacharparenleft}{\kern0pt}eval{\isacharunderscore}{\kern0pt}func\ Y\ X{\isacharparenright}{\kern0pt}\ {\isasymcirc}\isactrlsub c\ \ {\isasymlangle}x{\isacharcomma}{\kern0pt}\ g{\isasymrangle}{\isacharcomma}{\kern0pt}id\isactrlsub c\ {\isasymone}{\isasymrangle}{\isachardoublequoteclose}\isanewline
\ \ \ \ \ \ \isacommand{using}\isamarkupfalse%
\ assms\ \isacommand{by}\isamarkupfalse%
\ {\isacharparenleft}{\kern0pt}typecheck{\isacharunderscore}{\kern0pt}cfuncs{\isacharcomma}{\kern0pt}\ simp\ add{\isacharcolon}{\kern0pt}\ comp{\isacharunderscore}{\kern0pt}associative{\isadigit{2}}\ inv{\isacharunderscore}{\kern0pt}transpose{\isacharunderscore}{\kern0pt}func{\isacharunderscore}{\kern0pt}def{\isadigit{3}}{\isacharparenright}{\kern0pt}\isanewline
\ \ \ \ \isacommand{also}\isamarkupfalse%
\ \isacommand{have}\isamarkupfalse%
\ {\isachardoublequoteopen}{\isachardot}{\kern0pt}{\isachardot}{\kern0pt}{\isachardot}{\kern0pt}\ {\isacharequal}{\kern0pt}\ {\isacharparenleft}{\kern0pt}eval{\isacharunderscore}{\kern0pt}func\ Z\ Y{\isacharparenright}{\kern0pt}\ {\isasymcirc}\isactrlsub c\ \ {\isasymlangle}{\isacharparenleft}{\kern0pt}eval{\isacharunderscore}{\kern0pt}func\ Y\ X{\isacharparenright}{\kern0pt}\ {\isasymcirc}\isactrlsub c\ \ {\isasymlangle}x{\isacharcomma}{\kern0pt}\ g{\isasymrangle}{\isacharcomma}{\kern0pt}f{\isasymrangle}{\isachardoublequoteclose}\isanewline
\ \ \ \ \ \ \isacommand{using}\isamarkupfalse%
\ assms\ \isacommand{by}\isamarkupfalse%
\ {\isacharparenleft}{\kern0pt}typecheck{\isacharunderscore}{\kern0pt}cfuncs{\isacharcomma}{\kern0pt}\ simp\ add{\isacharcolon}{\kern0pt}\ cfunc{\isacharunderscore}{\kern0pt}cross{\isacharunderscore}{\kern0pt}prod{\isacharunderscore}{\kern0pt}comp{\isacharunderscore}{\kern0pt}cfunc{\isacharunderscore}{\kern0pt}prod\ id{\isacharunderscore}{\kern0pt}left{\isacharunderscore}{\kern0pt}unit{\isadigit{2}}\ id{\isacharunderscore}{\kern0pt}right{\isacharunderscore}{\kern0pt}unit{\isadigit{2}}{\isacharparenright}{\kern0pt}\isanewline
\ \ \ \ \isacommand{then}\isamarkupfalse%
\ \isacommand{show}\isamarkupfalse%
\ {\isachardoublequoteopen}{\isacharparenleft}{\kern0pt}f\isactrlsup {\isasymflat}\ {\isasymcirc}\isactrlsub c\ {\isasymlangle}g\isactrlsup {\isasymflat}{\isacharcomma}{\kern0pt}right{\isacharunderscore}{\kern0pt}cart{\isacharunderscore}{\kern0pt}proj\ X\ {\isasymone}{\isasymrangle}{\isacharparenright}{\kern0pt}\ {\isasymcirc}\isactrlsub c\ {\isasymlangle}x{\isacharcomma}{\kern0pt}id\isactrlsub c\ {\isasymone}{\isasymrangle}\ {\isacharequal}{\kern0pt}\ eval{\isacharunderscore}{\kern0pt}func\ Z\ Y\ {\isasymcirc}\isactrlsub c\ {\isasymlangle}eval{\isacharunderscore}{\kern0pt}func\ Y\ X\ {\isasymcirc}\isactrlsub c\ {\isasymlangle}x{\isacharcomma}{\kern0pt}g{\isasymrangle}{\isacharcomma}{\kern0pt}f{\isasymrangle}{\isachardoublequoteclose}\isanewline
\ \ \ \ \ \ \isacommand{by}\isamarkupfalse%
\ {\isacharparenleft}{\kern0pt}simp\ add{\isacharcolon}{\kern0pt}\ calculation{\isacharparenright}{\kern0pt}\isanewline
\isacommand{qed}\isamarkupfalse%
%
\endisatagproof
{\isafoldproof}%
%
\isadelimproof
\isanewline
%
\endisadelimproof
\isanewline
\isacommand{lemma}\isamarkupfalse%
\ meta{\isacharunderscore}{\kern0pt}comp{\isadigit{2}}{\isacharunderscore}{\kern0pt}def{\isadigit{3}}{\isacharcolon}{\kern0pt}\ \isanewline
\ \ \isakeyword{assumes}\ {\isachardoublequoteopen}f\ {\isasymin}\isactrlsub c\ Z\isactrlbsup Y\isactrlesup {\isachardoublequoteclose}\isanewline
\ \ \isakeyword{assumes}\ {\isachardoublequoteopen}g\ {\isasymin}\isactrlsub c\ Y\isactrlbsup X\isactrlesup {\isachardoublequoteclose}\isanewline
\ \ \isakeyword{shows}\ {\isachardoublequoteopen}f\ {\isasymbox}\ g\ {\isacharequal}{\kern0pt}\ metafunc\ {\isacharparenleft}{\kern0pt}{\isacharparenleft}{\kern0pt}cnufatem\ f{\isacharparenright}{\kern0pt}\ {\isasymcirc}\isactrlsub c\ {\isacharparenleft}{\kern0pt}cnufatem\ g{\isacharparenright}{\kern0pt}{\isacharparenright}{\kern0pt}{\isachardoublequoteclose}\isanewline
%
\isadelimproof
\ \ %
\endisadelimproof
%
\isatagproof
\isacommand{using}\isamarkupfalse%
\ assms\isanewline
\isacommand{proof}\isamarkupfalse%
{\isacharparenleft}{\kern0pt}unfold\ meta{\isacharunderscore}{\kern0pt}comp{\isadigit{2}}{\isacharunderscore}{\kern0pt}def{\isadigit{2}}\ cnufatem{\isacharunderscore}{\kern0pt}def{\isadigit{2}}\ metafunc{\isacharunderscore}{\kern0pt}def\ meta{\isacharunderscore}{\kern0pt}comp{\isacharunderscore}{\kern0pt}def{\isacharparenright}{\kern0pt}\ \ \ \ \ \ \ \ \ \ \isanewline
\ \ \isacommand{have}\isamarkupfalse%
\ {\isachardoublequoteopen}f\isactrlsup {\isasymflat}\ {\isasymcirc}\isactrlsub c\ {\isasymlangle}g\isactrlsup {\isasymflat}{\isacharcomma}{\kern0pt}right{\isacharunderscore}{\kern0pt}cart{\isacharunderscore}{\kern0pt}proj\ X\ {\isasymone}{\isasymrangle}\ {\isacharequal}{\kern0pt}\ {\isacharparenleft}{\kern0pt}{\isacharparenleft}{\kern0pt}eval{\isacharunderscore}{\kern0pt}func\ Z\ Y\ {\isasymcirc}\isactrlsub c\ {\isasymlangle}id\isactrlsub c\ Y{\isacharcomma}{\kern0pt}f\ {\isasymcirc}\isactrlsub c\ {\isasymbeta}\isactrlbsub Y\isactrlesub {\isasymrangle}{\isacharparenright}{\kern0pt}\ {\isasymcirc}\isactrlsub c\ eval{\isacharunderscore}{\kern0pt}func\ Y\ X\ {\isasymcirc}\isactrlsub c\ {\isasymlangle}id\isactrlsub c\ X{\isacharcomma}{\kern0pt}g\ {\isasymcirc}\isactrlsub c\ {\isasymbeta}\isactrlbsub X\isactrlesub {\isasymrangle}{\isacharparenright}{\kern0pt}\ {\isasymcirc}\isactrlsub c\ \ left{\isacharunderscore}{\kern0pt}cart{\isacharunderscore}{\kern0pt}proj\ X\ {\isasymone}{\isachardoublequoteclose}\isanewline
\ \ \isacommand{proof}\isamarkupfalse%
{\isacharparenleft}{\kern0pt}rule\ one{\isacharunderscore}{\kern0pt}separator{\isacharbrackleft}{\kern0pt}\isakeyword{where}\ X\ {\isacharequal}{\kern0pt}\ {\isachardoublequoteopen}X\ {\isasymtimes}\isactrlsub c\ {\isasymone}{\isachardoublequoteclose}{\isacharcomma}{\kern0pt}\ \isakeyword{where}\ Y\ {\isacharequal}{\kern0pt}\ Z{\isacharbrackright}{\kern0pt}{\isacharparenright}{\kern0pt}\isanewline
\ \ \ \ \isacommand{show}\isamarkupfalse%
\ {\isachardoublequoteopen}f\isactrlsup {\isasymflat}\ {\isasymcirc}\isactrlsub c\ {\isasymlangle}g\isactrlsup {\isasymflat}{\isacharcomma}{\kern0pt}right{\isacharunderscore}{\kern0pt}cart{\isacharunderscore}{\kern0pt}proj\ X\ {\isasymone}{\isasymrangle}\ {\isacharcolon}{\kern0pt}\ X\ {\isasymtimes}\isactrlsub c\ {\isasymone}\ {\isasymrightarrow}\ Z{\isachardoublequoteclose}\isanewline
\ \ \ \ \ \ \isacommand{using}\isamarkupfalse%
\ assms\ \isacommand{by}\isamarkupfalse%
\ typecheck{\isacharunderscore}{\kern0pt}cfuncs\isanewline
\ \ \ \ \isacommand{show}\isamarkupfalse%
\ {\isachardoublequoteopen}{\isacharparenleft}{\kern0pt}{\isacharparenleft}{\kern0pt}eval{\isacharunderscore}{\kern0pt}func\ Z\ Y\ {\isasymcirc}\isactrlsub c\ {\isasymlangle}id\isactrlsub c\ Y{\isacharcomma}{\kern0pt}f\ {\isasymcirc}\isactrlsub c\ {\isasymbeta}\isactrlbsub Y\isactrlesub {\isasymrangle}{\isacharparenright}{\kern0pt}\ {\isasymcirc}\isactrlsub c\ eval{\isacharunderscore}{\kern0pt}func\ Y\ X\ {\isasymcirc}\isactrlsub c\ {\isasymlangle}id\isactrlsub c\ X{\isacharcomma}{\kern0pt}g\ {\isasymcirc}\isactrlsub c\ {\isasymbeta}\isactrlbsub X\isactrlesub {\isasymrangle}{\isacharparenright}{\kern0pt}\ {\isasymcirc}\isactrlsub c\ left{\isacharunderscore}{\kern0pt}cart{\isacharunderscore}{\kern0pt}proj\ X\ {\isasymone}\ {\isacharcolon}{\kern0pt}\ X\ {\isasymtimes}\isactrlsub c\ {\isasymone}\ {\isasymrightarrow}\ Z{\isachardoublequoteclose}\isanewline
\ \ \ \ \ \ \isacommand{using}\isamarkupfalse%
\ assms\ \isacommand{by}\isamarkupfalse%
\ typecheck{\isacharunderscore}{\kern0pt}cfuncs\isanewline
\ \ \isacommand{next}\isamarkupfalse%
\isanewline
\ \ \ \ \isacommand{fix}\isamarkupfalse%
\ x{\isadigit{1}}\ \isanewline
\ \ \ \ \isacommand{assume}\isamarkupfalse%
\ x{\isadigit{1}}{\isacharunderscore}{\kern0pt}type{\isacharbrackleft}{\kern0pt}type{\isacharunderscore}{\kern0pt}rule{\isacharbrackright}{\kern0pt}{\isacharcolon}{\kern0pt}\ {\isachardoublequoteopen}x{\isadigit{1}}\ \ {\isasymin}\isactrlsub c\ {\isacharparenleft}{\kern0pt}X\ {\isasymtimes}\isactrlsub c\ {\isasymone}{\isacharparenright}{\kern0pt}{\isachardoublequoteclose}\isanewline
\ \ \ \ \isacommand{then}\isamarkupfalse%
\ \isacommand{obtain}\isamarkupfalse%
\ x\ \isakeyword{where}\ x{\isacharunderscore}{\kern0pt}type{\isacharbrackleft}{\kern0pt}type{\isacharunderscore}{\kern0pt}rule{\isacharbrackright}{\kern0pt}{\isacharcolon}{\kern0pt}\ {\isachardoublequoteopen}x\ {\isasymin}\isactrlsub c\ X{\isachardoublequoteclose}\ \isakeyword{and}\ x{\isacharunderscore}{\kern0pt}def{\isacharcolon}{\kern0pt}\ {\isachardoublequoteopen}x{\isadigit{1}}\ {\isacharequal}{\kern0pt}\ {\isasymlangle}x{\isacharcomma}{\kern0pt}\ id\isactrlsub c\ {\isasymone}{\isasymrangle}{\isachardoublequoteclose}\isanewline
\ \ \ \ \ \ \isacommand{by}\isamarkupfalse%
\ {\isacharparenleft}{\kern0pt}metis\ cart{\isacharunderscore}{\kern0pt}prod{\isacharunderscore}{\kern0pt}decomp\ id{\isacharunderscore}{\kern0pt}type\ terminal{\isacharunderscore}{\kern0pt}func{\isacharunderscore}{\kern0pt}unique{\isacharparenright}{\kern0pt}\isanewline
\ \ \ \ \isacommand{then}\isamarkupfalse%
\ \isacommand{have}\isamarkupfalse%
\ {\isachardoublequoteopen}{\isacharparenleft}{\kern0pt}f\isactrlsup {\isasymflat}\ {\isasymcirc}\isactrlsub c\ {\isasymlangle}g\isactrlsup {\isasymflat}{\isacharcomma}{\kern0pt}right{\isacharunderscore}{\kern0pt}cart{\isacharunderscore}{\kern0pt}proj\ X\ {\isasymone}{\isasymrangle}{\isacharparenright}{\kern0pt}\ {\isasymcirc}\isactrlsub c\ x{\isadigit{1}}\ {\isacharequal}{\kern0pt}\ eval{\isacharunderscore}{\kern0pt}func\ Z\ Y\ {\isasymcirc}\isactrlsub c\ {\isasymlangle}eval{\isacharunderscore}{\kern0pt}func\ Y\ X\ {\isasymcirc}\isactrlsub c\ {\isasymlangle}x{\isacharcomma}{\kern0pt}g{\isasymrangle}{\isacharcomma}{\kern0pt}f{\isasymrangle}{\isachardoublequoteclose}\isanewline
\ \ \ \ \ \ \isacommand{using}\isamarkupfalse%
\ assms\ meta{\isacharunderscore}{\kern0pt}comp{\isadigit{2}}{\isacharunderscore}{\kern0pt}elements{\isacharunderscore}{\kern0pt}aux\ x{\isacharunderscore}{\kern0pt}def\ \isacommand{by}\isamarkupfalse%
\ blast\isanewline
\ \ \ \ \isacommand{also}\isamarkupfalse%
\ \isacommand{have}\isamarkupfalse%
\ {\isachardoublequoteopen}{\isachardot}{\kern0pt}{\isachardot}{\kern0pt}{\isachardot}{\kern0pt}\ {\isacharequal}{\kern0pt}\ eval{\isacharunderscore}{\kern0pt}func\ Z\ Y\ {\isasymcirc}\isactrlsub c\ {\isasymlangle}id\isactrlsub c\ Y{\isacharcomma}{\kern0pt}f\ {\isasymcirc}\isactrlsub c\ {\isasymbeta}\isactrlbsub Y\isactrlesub {\isasymrangle}\ {\isasymcirc}\isactrlsub c\ eval{\isacharunderscore}{\kern0pt}func\ Y\ X\ {\isasymcirc}\isactrlsub c\ {\isasymlangle}id\isactrlsub c\ X{\isacharcomma}{\kern0pt}g\ {\isasymcirc}\isactrlsub c\ {\isasymbeta}\isactrlbsub X\isactrlesub {\isasymrangle}\ {\isasymcirc}\isactrlsub c\ x{\isachardoublequoteclose}\isanewline
\ \ \ \ \ \ \isacommand{using}\isamarkupfalse%
\ assms\ \isacommand{by}\isamarkupfalse%
\ {\isacharparenleft}{\kern0pt}typecheck{\isacharunderscore}{\kern0pt}cfuncs{\isacharcomma}{\kern0pt}\ metis\ cart{\isacharunderscore}{\kern0pt}prod{\isacharunderscore}{\kern0pt}extract{\isacharunderscore}{\kern0pt}left{\isacharparenright}{\kern0pt}\isanewline
\ \ \ \ \isacommand{also}\isamarkupfalse%
\ \isacommand{have}\isamarkupfalse%
\ {\isachardoublequoteopen}{\isachardot}{\kern0pt}{\isachardot}{\kern0pt}{\isachardot}{\kern0pt}\ {\isacharequal}{\kern0pt}\ \ {\isacharparenleft}{\kern0pt}eval{\isacharunderscore}{\kern0pt}func\ Z\ Y\ {\isasymcirc}\isactrlsub c\ {\isasymlangle}id\isactrlsub c\ Y{\isacharcomma}{\kern0pt}f\ {\isasymcirc}\isactrlsub c\ {\isasymbeta}\isactrlbsub Y\isactrlesub {\isasymrangle}{\isacharparenright}{\kern0pt}\ {\isasymcirc}\isactrlsub c\ eval{\isacharunderscore}{\kern0pt}func\ Y\ X\ {\isasymcirc}\isactrlsub c\ {\isasymlangle}id\isactrlsub c\ X{\isacharcomma}{\kern0pt}g\ {\isasymcirc}\isactrlsub c\ {\isasymbeta}\isactrlbsub X\isactrlesub {\isasymrangle}\ {\isasymcirc}\isactrlsub c\ x{\isachardoublequoteclose}\isanewline
\ \ \ \ \ \ \isacommand{using}\isamarkupfalse%
\ assms\ \isacommand{by}\isamarkupfalse%
\ {\isacharparenleft}{\kern0pt}typecheck{\isacharunderscore}{\kern0pt}cfuncs{\isacharcomma}{\kern0pt}\ meson\ comp{\isacharunderscore}{\kern0pt}associative{\isadigit{2}}{\isacharparenright}{\kern0pt}\isanewline
\ \ \ \ \isacommand{also}\isamarkupfalse%
\ \isacommand{have}\isamarkupfalse%
\ {\isachardoublequoteopen}{\isachardot}{\kern0pt}{\isachardot}{\kern0pt}{\isachardot}{\kern0pt}\ {\isacharequal}{\kern0pt}\ {\isacharparenleft}{\kern0pt}{\isacharparenleft}{\kern0pt}eval{\isacharunderscore}{\kern0pt}func\ Z\ Y\ {\isasymcirc}\isactrlsub c\ {\isasymlangle}id\isactrlsub c\ Y{\isacharcomma}{\kern0pt}f\ {\isasymcirc}\isactrlsub c\ {\isasymbeta}\isactrlbsub Y\isactrlesub {\isasymrangle}{\isacharparenright}{\kern0pt}\ {\isasymcirc}\isactrlsub c\ eval{\isacharunderscore}{\kern0pt}func\ Y\ X\ {\isasymcirc}\isactrlsub c\ {\isasymlangle}id\isactrlsub c\ X{\isacharcomma}{\kern0pt}g\ {\isasymcirc}\isactrlsub c\ {\isasymbeta}\isactrlbsub X\isactrlesub {\isasymrangle}{\isacharparenright}{\kern0pt}\ {\isasymcirc}\isactrlsub c\ x{\isachardoublequoteclose}\isanewline
\ \ \ \ \ \ \isacommand{using}\isamarkupfalse%
\ assms\ \isacommand{by}\isamarkupfalse%
\ {\isacharparenleft}{\kern0pt}typecheck{\isacharunderscore}{\kern0pt}cfuncs{\isacharcomma}{\kern0pt}\ simp\ add{\isacharcolon}{\kern0pt}\ comp{\isacharunderscore}{\kern0pt}associative{\isadigit{2}}{\isacharparenright}{\kern0pt}\isanewline
\ \ \ \ \isacommand{also}\isamarkupfalse%
\ \isacommand{have}\isamarkupfalse%
\ {\isachardoublequoteopen}{\isachardot}{\kern0pt}{\isachardot}{\kern0pt}{\isachardot}{\kern0pt}\ {\isacharequal}{\kern0pt}\ {\isacharparenleft}{\kern0pt}{\isacharparenleft}{\kern0pt}eval{\isacharunderscore}{\kern0pt}func\ Z\ Y\ {\isasymcirc}\isactrlsub c\ {\isasymlangle}id\isactrlsub c\ Y{\isacharcomma}{\kern0pt}f\ {\isasymcirc}\isactrlsub c\ {\isasymbeta}\isactrlbsub Y\isactrlesub {\isasymrangle}{\isacharparenright}{\kern0pt}\ {\isasymcirc}\isactrlsub c\ eval{\isacharunderscore}{\kern0pt}func\ Y\ X\ {\isasymcirc}\isactrlsub c\ {\isasymlangle}id\isactrlsub c\ X{\isacharcomma}{\kern0pt}g\ {\isasymcirc}\isactrlsub c\ {\isasymbeta}\isactrlbsub X\isactrlesub {\isasymrangle}{\isacharparenright}{\kern0pt}\ {\isasymcirc}\isactrlsub c\ left{\isacharunderscore}{\kern0pt}cart{\isacharunderscore}{\kern0pt}proj\ X\ {\isasymone}\ {\isasymcirc}\isactrlsub c\ x{\isadigit{1}}{\isachardoublequoteclose}\isanewline
\ \ \ \ \ \ \isacommand{using}\isamarkupfalse%
\ assms\ id{\isacharunderscore}{\kern0pt}type\ left{\isacharunderscore}{\kern0pt}cart{\isacharunderscore}{\kern0pt}proj{\isacharunderscore}{\kern0pt}cfunc{\isacharunderscore}{\kern0pt}prod\ x{\isacharunderscore}{\kern0pt}def\ \isacommand{by}\isamarkupfalse%
\ {\isacharparenleft}{\kern0pt}typecheck{\isacharunderscore}{\kern0pt}cfuncs{\isacharcomma}{\kern0pt}\ auto{\isacharparenright}{\kern0pt}\isanewline
\ \ \ \ \isacommand{also}\isamarkupfalse%
\ \isacommand{have}\isamarkupfalse%
\ {\isachardoublequoteopen}{\isachardot}{\kern0pt}{\isachardot}{\kern0pt}{\isachardot}{\kern0pt}\ {\isacharequal}{\kern0pt}\ {\isacharparenleft}{\kern0pt}{\isacharparenleft}{\kern0pt}{\isacharparenleft}{\kern0pt}eval{\isacharunderscore}{\kern0pt}func\ Z\ Y\ {\isasymcirc}\isactrlsub c\ {\isasymlangle}id\isactrlsub c\ Y{\isacharcomma}{\kern0pt}f\ {\isasymcirc}\isactrlsub c\ {\isasymbeta}\isactrlbsub Y\isactrlesub {\isasymrangle}{\isacharparenright}{\kern0pt}\ {\isasymcirc}\isactrlsub c\ eval{\isacharunderscore}{\kern0pt}func\ Y\ X\ {\isasymcirc}\isactrlsub c\ {\isasymlangle}id\isactrlsub c\ X{\isacharcomma}{\kern0pt}g\ {\isasymcirc}\isactrlsub c\ {\isasymbeta}\isactrlbsub X\isactrlesub {\isasymrangle}{\isacharparenright}{\kern0pt}\ {\isasymcirc}\isactrlsub c\ left{\isacharunderscore}{\kern0pt}cart{\isacharunderscore}{\kern0pt}proj\ X\ {\isasymone}{\isacharparenright}{\kern0pt}\ {\isasymcirc}\isactrlsub c\ x{\isadigit{1}}{\isachardoublequoteclose}\isanewline
\ \ \ \ \ \ \isacommand{using}\isamarkupfalse%
\ assms\ \isacommand{by}\isamarkupfalse%
\ {\isacharparenleft}{\kern0pt}typecheck{\isacharunderscore}{\kern0pt}cfuncs{\isacharcomma}{\kern0pt}\ meson\ comp{\isacharunderscore}{\kern0pt}associative{\isadigit{2}}{\isacharparenright}{\kern0pt}\isanewline
\ \ \ \ \isacommand{then}\isamarkupfalse%
\ \isacommand{show}\isamarkupfalse%
\ {\isachardoublequoteopen}{\isacharparenleft}{\kern0pt}f\isactrlsup {\isasymflat}\ {\isasymcirc}\isactrlsub c\ {\isasymlangle}g\isactrlsup {\isasymflat}{\isacharcomma}{\kern0pt}right{\isacharunderscore}{\kern0pt}cart{\isacharunderscore}{\kern0pt}proj\ X\ {\isasymone}{\isasymrangle}{\isacharparenright}{\kern0pt}\ {\isasymcirc}\isactrlsub c\ x{\isadigit{1}}\ {\isacharequal}{\kern0pt}\ {\isacharparenleft}{\kern0pt}{\isacharparenleft}{\kern0pt}{\isacharparenleft}{\kern0pt}eval{\isacharunderscore}{\kern0pt}func\ Z\ Y\ {\isasymcirc}\isactrlsub c\ {\isasymlangle}id\isactrlsub c\ Y{\isacharcomma}{\kern0pt}f\ {\isasymcirc}\isactrlsub c\ {\isasymbeta}\isactrlbsub Y\isactrlesub {\isasymrangle}{\isacharparenright}{\kern0pt}\ {\isasymcirc}\isactrlsub c\ eval{\isacharunderscore}{\kern0pt}func\ Y\ X\ {\isasymcirc}\isactrlsub c\ {\isasymlangle}id\isactrlsub c\ X{\isacharcomma}{\kern0pt}g\ {\isasymcirc}\isactrlsub c\ {\isasymbeta}\isactrlbsub X\isactrlesub {\isasymrangle}{\isacharparenright}{\kern0pt}\ {\isasymcirc}\isactrlsub c\ left{\isacharunderscore}{\kern0pt}cart{\isacharunderscore}{\kern0pt}proj\ X\ {\isasymone}{\isacharparenright}{\kern0pt}\ {\isasymcirc}\isactrlsub c\ x{\isadigit{1}}{\isachardoublequoteclose}\isanewline
\ \ \ \ \ \ \isacommand{by}\isamarkupfalse%
\ {\isacharparenleft}{\kern0pt}simp\ add{\isacharcolon}{\kern0pt}\ calculation{\isacharparenright}{\kern0pt}\isanewline
\ \ \isacommand{qed}\isamarkupfalse%
\isanewline
\ \ \isacommand{then}\isamarkupfalse%
\ \isacommand{show}\isamarkupfalse%
\ {\isachardoublequoteopen}{\isacharparenleft}{\kern0pt}f\isactrlsup {\isasymflat}\ {\isasymcirc}\isactrlsub c\ {\isasymlangle}g\isactrlsup {\isasymflat}{\isacharcomma}{\kern0pt}right{\isacharunderscore}{\kern0pt}cart{\isacharunderscore}{\kern0pt}proj\ X\ {\isasymone}{\isasymrangle}{\isacharparenright}{\kern0pt}\isactrlsup {\isasymsharp}\ {\isacharequal}{\kern0pt}\ {\isacharparenleft}{\kern0pt}{\isacharparenleft}{\kern0pt}{\isacharparenleft}{\kern0pt}eval{\isacharunderscore}{\kern0pt}func\ Z\ Y\ {\isasymcirc}\isactrlsub c\ {\isasymlangle}id\isactrlsub c\ Y{\isacharcomma}{\kern0pt}f\ {\isasymcirc}\isactrlsub c\ {\isasymbeta}\isactrlbsub Y\isactrlesub {\isasymrangle}{\isacharparenright}{\kern0pt}\ {\isasymcirc}\isactrlsub c\ eval{\isacharunderscore}{\kern0pt}func\ Y\ X\ {\isasymcirc}\isactrlsub c\ {\isasymlangle}id\isactrlsub c\ X{\isacharcomma}{\kern0pt}g\ {\isasymcirc}\isactrlsub c\ {\isasymbeta}\isactrlbsub X\isactrlesub {\isasymrangle}{\isacharparenright}{\kern0pt}\ {\isasymcirc}\isactrlsub c\ left{\isacharunderscore}{\kern0pt}cart{\isacharunderscore}{\kern0pt}proj\ {\isacharparenleft}{\kern0pt}domain\ {\isacharparenleft}{\kern0pt}{\isacharparenleft}{\kern0pt}eval{\isacharunderscore}{\kern0pt}func\ Z\ Y\ {\isasymcirc}\isactrlsub c\ {\isasymlangle}id\isactrlsub c\ Y{\isacharcomma}{\kern0pt}f\ {\isasymcirc}\isactrlsub c\ {\isasymbeta}\isactrlbsub Y\isactrlesub {\isasymrangle}{\isacharparenright}{\kern0pt}\ {\isasymcirc}\isactrlsub c\ eval{\isacharunderscore}{\kern0pt}func\ Y\ X\ {\isasymcirc}\isactrlsub c\ {\isasymlangle}id\isactrlsub c\ X{\isacharcomma}{\kern0pt}g\ {\isasymcirc}\isactrlsub c\ {\isasymbeta}\isactrlbsub X\isactrlesub {\isasymrangle}{\isacharparenright}{\kern0pt}{\isacharparenright}{\kern0pt}\ {\isasymone}{\isacharparenright}{\kern0pt}\isactrlsup {\isasymsharp}{\isachardoublequoteclose}\isanewline
\ \ \ \ \isacommand{using}\isamarkupfalse%
\ assms\ cfunc{\isacharunderscore}{\kern0pt}type{\isacharunderscore}{\kern0pt}def\ cnufatem{\isacharunderscore}{\kern0pt}def{\isadigit{2}}\ cnufatem{\isacharunderscore}{\kern0pt}type\ domain{\isacharunderscore}{\kern0pt}comp\ \isacommand{by}\isamarkupfalse%
\ force\isanewline
\isacommand{qed}\isamarkupfalse%
%
\endisatagproof
{\isafoldproof}%
%
\isadelimproof
\isanewline
%
\endisadelimproof
\isanewline
\isacommand{lemma}\isamarkupfalse%
\ meta{\isacharunderscore}{\kern0pt}comp{\isadigit{2}}{\isacharunderscore}{\kern0pt}def{\isadigit{4}}{\isacharcolon}{\kern0pt}\isanewline
\ \ \isakeyword{assumes}\ f{\isacharunderscore}{\kern0pt}type{\isacharbrackleft}{\kern0pt}type{\isacharunderscore}{\kern0pt}rule{\isacharbrackright}{\kern0pt}{\isacharcolon}{\kern0pt}\ {\isachardoublequoteopen}f\ {\isasymin}\isactrlsub c\ Z\isactrlbsup Y\isactrlesup {\isachardoublequoteclose}\ \isakeyword{and}\ g{\isacharunderscore}{\kern0pt}type{\isacharbrackleft}{\kern0pt}type{\isacharunderscore}{\kern0pt}rule{\isacharbrackright}{\kern0pt}{\isacharcolon}{\kern0pt}\ {\isachardoublequoteopen}g\ {\isasymin}\isactrlsub c\ Y\isactrlbsup X\isactrlesup {\isachardoublequoteclose}\isanewline
\ \ \isakeyword{shows}\ {\isachardoublequoteopen}f\ {\isasymbox}\ g\ \ \ {\isacharequal}{\kern0pt}\ meta{\isacharunderscore}{\kern0pt}comp\ X\ Y\ Z\ {\isasymcirc}\isactrlsub c\ {\isasymlangle}f{\isacharcomma}{\kern0pt}g{\isasymrangle}{\isachardoublequoteclose}\isanewline
%
\isadelimproof
\ \ %
\endisadelimproof
%
\isatagproof
\isacommand{using}\isamarkupfalse%
\ assms\ \isanewline
\isacommand{proof}\isamarkupfalse%
{\isacharparenleft}{\kern0pt}unfold\ meta{\isacharunderscore}{\kern0pt}comp{\isadigit{2}}{\isacharunderscore}{\kern0pt}def{\isadigit{2}}\ cnufatem{\isacharunderscore}{\kern0pt}def{\isadigit{2}}\ metafunc{\isacharunderscore}{\kern0pt}def\ meta{\isacharunderscore}{\kern0pt}comp{\isacharunderscore}{\kern0pt}def{\isacharparenright}{\kern0pt}\ \ \ \ \ \ \ \ \ \ \isanewline
\ \ \isacommand{have}\isamarkupfalse%
\ {\isachardoublequoteopen}{\isacharparenleft}{\kern0pt}{\isacharparenleft}{\kern0pt}{\isacharparenleft}{\kern0pt}eval{\isacharunderscore}{\kern0pt}func\ Z\ Y\ {\isasymcirc}\isactrlsub c\ {\isasymlangle}id\isactrlsub c\ Y{\isacharcomma}{\kern0pt}f\ {\isasymcirc}\isactrlsub c\ {\isasymbeta}\isactrlbsub Y\isactrlesub {\isasymrangle}{\isacharparenright}{\kern0pt}\ {\isasymcirc}\isactrlsub c\ eval{\isacharunderscore}{\kern0pt}func\ Y\ X\ {\isasymcirc}\isactrlsub c\ {\isasymlangle}id\isactrlsub c\ X{\isacharcomma}{\kern0pt}g\ {\isasymcirc}\isactrlsub c\ {\isasymbeta}\isactrlbsub X\isactrlesub {\isasymrangle}{\isacharparenright}{\kern0pt}\ {\isasymcirc}\isactrlsub c\ left{\isacharunderscore}{\kern0pt}cart{\isacharunderscore}{\kern0pt}proj\ X\ {\isasymone}{\isacharparenright}{\kern0pt}\ {\isacharequal}{\kern0pt}\ \ \isanewline
\ \ \ \ \ \ \ \ \ \ {\isacharparenleft}{\kern0pt}eval{\isacharunderscore}{\kern0pt}func\ Z\ Y\ {\isasymcirc}\isactrlsub c\ \ swap\ {\isacharparenleft}{\kern0pt}Z\isactrlbsup Y\isactrlesup {\isacharparenright}{\kern0pt}\ Y\ {\isasymcirc}\isactrlsub c\ \ {\isacharparenleft}{\kern0pt}id\isactrlsub c\ {\isacharparenleft}{\kern0pt}Z\isactrlbsup Y\isactrlesup {\isacharparenright}{\kern0pt}\ {\isasymtimes}\isactrlsub f\ {\isacharparenleft}{\kern0pt}eval{\isacharunderscore}{\kern0pt}func\ Y\ X\ {\isasymcirc}\isactrlsub c\ swap\ {\isacharparenleft}{\kern0pt}Y\isactrlbsup X\isactrlesup {\isacharparenright}{\kern0pt}\ X{\isacharparenright}{\kern0pt}{\isacharparenright}{\kern0pt}\ {\isasymcirc}\isactrlsub c\ associate{\isacharunderscore}{\kern0pt}right\ {\isacharparenleft}{\kern0pt}Z\isactrlbsup Y\isactrlesup {\isacharparenright}{\kern0pt}\ {\isacharparenleft}{\kern0pt}Y\isactrlbsup X\isactrlesup {\isacharparenright}{\kern0pt}\ X\ {\isasymcirc}\isactrlsub c\ swap\ X\ {\isacharparenleft}{\kern0pt}Z\isactrlbsup Y\isactrlesup \ {\isasymtimes}\isactrlsub c\ Y\isactrlbsup X\isactrlesup {\isacharparenright}{\kern0pt}{\isacharparenright}{\kern0pt}\ {\isasymcirc}\isactrlsub c\ {\isacharparenleft}{\kern0pt}id\ {\isacharparenleft}{\kern0pt}X{\isacharparenright}{\kern0pt}\ \ {\isasymtimes}\isactrlsub f\ \ {\isasymlangle}f{\isacharcomma}{\kern0pt}g{\isasymrangle}{\isacharparenright}{\kern0pt}{\isachardoublequoteclose}\isanewline
\ \ \isacommand{proof}\isamarkupfalse%
{\isacharparenleft}{\kern0pt}etcs{\isacharunderscore}{\kern0pt}rule\ one{\isacharunderscore}{\kern0pt}separator{\isacharparenright}{\kern0pt}\isanewline
\ \ \ \ \isacommand{fix}\isamarkupfalse%
\ x{\isadigit{1}}\ \isanewline
\ \ \ \ \isacommand{assume}\isamarkupfalse%
\ x{\isadigit{1}}{\isacharunderscore}{\kern0pt}type{\isacharbrackleft}{\kern0pt}type{\isacharunderscore}{\kern0pt}rule{\isacharbrackright}{\kern0pt}{\isacharcolon}{\kern0pt}\ {\isachardoublequoteopen}x{\isadigit{1}}\ \ {\isasymin}\isactrlsub c\ X\ {\isasymtimes}\isactrlsub c\ {\isasymone}{\isachardoublequoteclose}\isanewline
\ \ \ \ \isacommand{then}\isamarkupfalse%
\ \isacommand{obtain}\isamarkupfalse%
\ x\ \isakeyword{where}\ x{\isacharunderscore}{\kern0pt}type{\isacharbrackleft}{\kern0pt}type{\isacharunderscore}{\kern0pt}rule{\isacharbrackright}{\kern0pt}{\isacharcolon}{\kern0pt}\ {\isachardoublequoteopen}x\ {\isasymin}\isactrlsub c\ X{\isachardoublequoteclose}\ \isakeyword{and}\ x{\isacharunderscore}{\kern0pt}def{\isacharcolon}{\kern0pt}\ {\isachardoublequoteopen}x{\isadigit{1}}\ {\isacharequal}{\kern0pt}\ {\isasymlangle}x{\isacharcomma}{\kern0pt}\ id\isactrlsub c\ {\isasymone}{\isasymrangle}{\isachardoublequoteclose}\isanewline
\ \ \ \ \ \ \isacommand{by}\isamarkupfalse%
\ {\isacharparenleft}{\kern0pt}metis\ cart{\isacharunderscore}{\kern0pt}prod{\isacharunderscore}{\kern0pt}decomp\ id{\isacharunderscore}{\kern0pt}type\ terminal{\isacharunderscore}{\kern0pt}func{\isacharunderscore}{\kern0pt}unique{\isacharparenright}{\kern0pt}\isanewline
\ \ \ \ \isacommand{have}\isamarkupfalse%
\ {\isachardoublequoteopen}{\isacharparenleft}{\kern0pt}{\isacharparenleft}{\kern0pt}{\isacharparenleft}{\kern0pt}eval{\isacharunderscore}{\kern0pt}func\ Z\ Y\ {\isasymcirc}\isactrlsub c\ {\isasymlangle}id\isactrlsub c\ Y{\isacharcomma}{\kern0pt}f\ {\isasymcirc}\isactrlsub c\ {\isasymbeta}\isactrlbsub Y\isactrlesub {\isasymrangle}{\isacharparenright}{\kern0pt}\ {\isasymcirc}\isactrlsub c\ eval{\isacharunderscore}{\kern0pt}func\ Y\ X\ {\isasymcirc}\isactrlsub c\ {\isasymlangle}id\isactrlsub c\ X{\isacharcomma}{\kern0pt}g\ {\isasymcirc}\isactrlsub c\ {\isasymbeta}\isactrlbsub X\isactrlesub {\isasymrangle}{\isacharparenright}{\kern0pt}\ {\isasymcirc}\isactrlsub c\ left{\isacharunderscore}{\kern0pt}cart{\isacharunderscore}{\kern0pt}proj\ X\ {\isasymone}{\isacharparenright}{\kern0pt}\ {\isasymcirc}\isactrlsub c\ x{\isadigit{1}}\ {\isacharequal}{\kern0pt}\ \isanewline
\ \ \ \ \ \ \ \ \ \ \ {\isacharparenleft}{\kern0pt}{\isacharparenleft}{\kern0pt}eval{\isacharunderscore}{\kern0pt}func\ Z\ Y\ {\isasymcirc}\isactrlsub c\ {\isasymlangle}id\isactrlsub c\ Y{\isacharcomma}{\kern0pt}f\ {\isasymcirc}\isactrlsub c\ {\isasymbeta}\isactrlbsub Y\isactrlesub {\isasymrangle}{\isacharparenright}{\kern0pt}\ {\isasymcirc}\isactrlsub c\ eval{\isacharunderscore}{\kern0pt}func\ Y\ X\ {\isasymcirc}\isactrlsub c\ {\isasymlangle}id\isactrlsub c\ X{\isacharcomma}{\kern0pt}g\ {\isasymcirc}\isactrlsub c\ {\isasymbeta}\isactrlbsub X\isactrlesub {\isasymrangle}{\isacharparenright}{\kern0pt}\ {\isasymcirc}\isactrlsub c\ left{\isacharunderscore}{\kern0pt}cart{\isacharunderscore}{\kern0pt}proj\ X\ {\isasymone}\ {\isasymcirc}\isactrlsub c\ x{\isadigit{1}}{\isachardoublequoteclose}\isanewline
\ \ \ \ \ \ \isacommand{by}\isamarkupfalse%
\ {\isacharparenleft}{\kern0pt}typecheck{\isacharunderscore}{\kern0pt}cfuncs{\isacharcomma}{\kern0pt}\ metis\ cfunc{\isacharunderscore}{\kern0pt}type{\isacharunderscore}{\kern0pt}def\ comp{\isacharunderscore}{\kern0pt}associative{\isacharparenright}{\kern0pt}\isanewline
\ \ \ \ \isacommand{also}\isamarkupfalse%
\ \isacommand{have}\isamarkupfalse%
\ {\isachardoublequoteopen}{\isachardot}{\kern0pt}{\isachardot}{\kern0pt}{\isachardot}{\kern0pt}\ {\isacharequal}{\kern0pt}\ {\isacharparenleft}{\kern0pt}{\isacharparenleft}{\kern0pt}eval{\isacharunderscore}{\kern0pt}func\ Z\ Y\ {\isasymcirc}\isactrlsub c\ {\isasymlangle}id\isactrlsub c\ Y{\isacharcomma}{\kern0pt}f\ {\isasymcirc}\isactrlsub c\ {\isasymbeta}\isactrlbsub Y\isactrlesub {\isasymrangle}{\isacharparenright}{\kern0pt}\ {\isasymcirc}\isactrlsub c\ eval{\isacharunderscore}{\kern0pt}func\ Y\ X\ {\isasymcirc}\isactrlsub c\ {\isasymlangle}id\isactrlsub c\ X{\isacharcomma}{\kern0pt}g\ {\isasymcirc}\isactrlsub c\ {\isasymbeta}\isactrlbsub X\isactrlesub {\isasymrangle}{\isacharparenright}{\kern0pt}\ {\isasymcirc}\isactrlsub c\ x{\isachardoublequoteclose}\isanewline
\ \ \ \ \ \ \isacommand{using}\isamarkupfalse%
\ id{\isacharunderscore}{\kern0pt}type\ left{\isacharunderscore}{\kern0pt}cart{\isacharunderscore}{\kern0pt}proj{\isacharunderscore}{\kern0pt}cfunc{\isacharunderscore}{\kern0pt}prod\ x{\isacharunderscore}{\kern0pt}def\ \isacommand{by}\isamarkupfalse%
\ {\isacharparenleft}{\kern0pt}typecheck{\isacharunderscore}{\kern0pt}cfuncs{\isacharcomma}{\kern0pt}\ presburger{\isacharparenright}{\kern0pt}\isanewline
\ \ \ \ \isacommand{also}\isamarkupfalse%
\ \isacommand{have}\isamarkupfalse%
\ {\isachardoublequoteopen}{\isachardot}{\kern0pt}{\isachardot}{\kern0pt}{\isachardot}{\kern0pt}\ {\isacharequal}{\kern0pt}\ \ {\isacharparenleft}{\kern0pt}eval{\isacharunderscore}{\kern0pt}func\ Z\ Y\ {\isasymcirc}\isactrlsub c\ {\isasymlangle}id\isactrlsub c\ Y{\isacharcomma}{\kern0pt}f\ {\isasymcirc}\isactrlsub c\ {\isasymbeta}\isactrlbsub Y\isactrlesub {\isasymrangle}{\isacharparenright}{\kern0pt}\ {\isasymcirc}\isactrlsub c\ eval{\isacharunderscore}{\kern0pt}func\ Y\ X\ {\isasymcirc}\isactrlsub c\ {\isasymlangle}id\isactrlsub c\ X{\isacharcomma}{\kern0pt}g\ {\isasymcirc}\isactrlsub c\ {\isasymbeta}\isactrlbsub X\isactrlesub {\isasymrangle}\ {\isasymcirc}\isactrlsub c\ x{\isachardoublequoteclose}\isanewline
\ \ \ \ \ \ \isacommand{by}\isamarkupfalse%
\ {\isacharparenleft}{\kern0pt}typecheck{\isacharunderscore}{\kern0pt}cfuncs{\isacharcomma}{\kern0pt}\ metis\ cfunc{\isacharunderscore}{\kern0pt}type{\isacharunderscore}{\kern0pt}def\ comp{\isacharunderscore}{\kern0pt}associative{\isacharparenright}{\kern0pt}\isanewline
\ \ \ \ \isacommand{also}\isamarkupfalse%
\ \isacommand{have}\isamarkupfalse%
\ {\isachardoublequoteopen}{\isachardot}{\kern0pt}{\isachardot}{\kern0pt}{\isachardot}{\kern0pt}\ {\isacharequal}{\kern0pt}\ eval{\isacharunderscore}{\kern0pt}func\ Z\ Y\ {\isasymcirc}\isactrlsub c\ {\isasymlangle}id\isactrlsub c\ Y{\isacharcomma}{\kern0pt}f\ {\isasymcirc}\isactrlsub c\ {\isasymbeta}\isactrlbsub Y\isactrlesub {\isasymrangle}\ {\isasymcirc}\isactrlsub c\ eval{\isacharunderscore}{\kern0pt}func\ Y\ X\ {\isasymcirc}\isactrlsub c\ {\isasymlangle}id\isactrlsub c\ X{\isacharcomma}{\kern0pt}g\ {\isasymcirc}\isactrlsub c\ {\isasymbeta}\isactrlbsub X\isactrlesub {\isasymrangle}\ {\isasymcirc}\isactrlsub c\ x{\isachardoublequoteclose}\isanewline
\ \ \ \ \ \ \isacommand{by}\isamarkupfalse%
\ {\isacharparenleft}{\kern0pt}typecheck{\isacharunderscore}{\kern0pt}cfuncs{\isacharcomma}{\kern0pt}\ metis\ cfunc{\isacharunderscore}{\kern0pt}type{\isacharunderscore}{\kern0pt}def\ comp{\isacharunderscore}{\kern0pt}associative{\isacharparenright}{\kern0pt}\isanewline
\ \ \ \ \isacommand{also}\isamarkupfalse%
\ \isacommand{have}\isamarkupfalse%
\ {\isachardoublequoteopen}{\isachardot}{\kern0pt}{\isachardot}{\kern0pt}{\isachardot}{\kern0pt}\ {\isacharequal}{\kern0pt}\ eval{\isacharunderscore}{\kern0pt}func\ Z\ Y\ {\isasymcirc}\isactrlsub c\ {\isasymlangle}id\isactrlsub c\ Y{\isacharcomma}{\kern0pt}f\ {\isasymcirc}\isactrlsub c\ {\isasymbeta}\isactrlbsub Y\isactrlesub {\isasymrangle}\ {\isasymcirc}\isactrlsub c\ eval{\isacharunderscore}{\kern0pt}func\ Y\ X\ {\isasymcirc}\isactrlsub c\ {\isasymlangle}x\ {\isacharcomma}{\kern0pt}g{\isasymrangle}{\isachardoublequoteclose}\isanewline
\ \ \ \ \ \ \isacommand{by}\isamarkupfalse%
\ {\isacharparenleft}{\kern0pt}typecheck{\isacharunderscore}{\kern0pt}cfuncs{\isacharcomma}{\kern0pt}\ metis\ cart{\isacharunderscore}{\kern0pt}prod{\isacharunderscore}{\kern0pt}extract{\isacharunderscore}{\kern0pt}left{\isacharparenright}{\kern0pt}\isanewline
\ \ \ \ \isacommand{also}\isamarkupfalse%
\ \isacommand{have}\isamarkupfalse%
\ {\isachardoublequoteopen}{\isachardot}{\kern0pt}{\isachardot}{\kern0pt}{\isachardot}{\kern0pt}\ {\isacharequal}{\kern0pt}\ eval{\isacharunderscore}{\kern0pt}func\ Z\ Y\ {\isasymcirc}\isactrlsub c\ {\isasymlangle}eval{\isacharunderscore}{\kern0pt}func\ Y\ X\ {\isasymcirc}\isactrlsub c\ {\isasymlangle}x\ {\isacharcomma}{\kern0pt}g{\isasymrangle}\ {\isacharcomma}{\kern0pt}f{\isasymrangle}{\isachardoublequoteclose}\isanewline
\ \ \ \ \ \ \isacommand{by}\isamarkupfalse%
\ {\isacharparenleft}{\kern0pt}typecheck{\isacharunderscore}{\kern0pt}cfuncs{\isacharcomma}{\kern0pt}\ metis\ cart{\isacharunderscore}{\kern0pt}prod{\isacharunderscore}{\kern0pt}extract{\isacharunderscore}{\kern0pt}left{\isacharparenright}{\kern0pt}\isanewline
\ \ \ \ \isacommand{also}\isamarkupfalse%
\ \isacommand{have}\isamarkupfalse%
\ {\isachardoublequoteopen}{\isachardot}{\kern0pt}{\isachardot}{\kern0pt}{\isachardot}{\kern0pt}\ {\isacharequal}{\kern0pt}\ {\isacharparenleft}{\kern0pt}eval{\isacharunderscore}{\kern0pt}func\ Z\ Y\ {\isasymcirc}\isactrlsub c\ swap\ {\isacharparenleft}{\kern0pt}Z\isactrlbsup Y\isactrlesup {\isacharparenright}{\kern0pt}\ Y{\isacharparenright}{\kern0pt}\ {\isasymcirc}\isactrlsub c\ {\isasymlangle}f\ {\isacharcomma}{\kern0pt}\ eval{\isacharunderscore}{\kern0pt}func\ Y\ X\ {\isasymcirc}\isactrlsub c\ \ {\isasymlangle}x{\isacharcomma}{\kern0pt}\ g{\isasymrangle}{\isasymrangle}{\isachardoublequoteclose}\isanewline
\ \ \ \ \ \ \isacommand{by}\isamarkupfalse%
\ {\isacharparenleft}{\kern0pt}typecheck{\isacharunderscore}{\kern0pt}cfuncs{\isacharcomma}{\kern0pt}\ metis\ comp{\isacharunderscore}{\kern0pt}associative{\isadigit{2}}\ swap{\isacharunderscore}{\kern0pt}ap{\isacharparenright}{\kern0pt}\isanewline
\ \ \ \ \isacommand{also}\isamarkupfalse%
\ \isacommand{have}\isamarkupfalse%
\ {\isachardoublequoteopen}{\isachardot}{\kern0pt}{\isachardot}{\kern0pt}{\isachardot}{\kern0pt}\ {\isacharequal}{\kern0pt}\ {\isacharparenleft}{\kern0pt}eval{\isacharunderscore}{\kern0pt}func\ Z\ Y\ {\isasymcirc}\isactrlsub c\ swap\ {\isacharparenleft}{\kern0pt}Z\isactrlbsup Y\isactrlesup {\isacharparenright}{\kern0pt}\ Y{\isacharparenright}{\kern0pt}\ {\isasymcirc}\isactrlsub c\ {\isasymlangle}id\isactrlsub c\ {\isacharparenleft}{\kern0pt}Z\isactrlbsup Y\isactrlesup {\isacharparenright}{\kern0pt}\ \ {\isasymcirc}\isactrlsub c\ \ f\ {\isacharcomma}{\kern0pt}\ {\isacharparenleft}{\kern0pt}eval{\isacharunderscore}{\kern0pt}func\ Y\ X\ {\isasymcirc}\isactrlsub c\ swap\ {\isacharparenleft}{\kern0pt}Y\isactrlbsup X\isactrlesup {\isacharparenright}{\kern0pt}\ X{\isacharparenright}{\kern0pt}\ \ {\isasymcirc}\isactrlsub c\ {\isasymlangle}g{\isacharcomma}{\kern0pt}\ x{\isasymrangle}{\isasymrangle}{\isachardoublequoteclose}\isanewline
\ \ \ \ \ \ \isacommand{by}\isamarkupfalse%
\ {\isacharparenleft}{\kern0pt}typecheck{\isacharunderscore}{\kern0pt}cfuncs{\isacharcomma}{\kern0pt}\ smt\ {\isacharparenleft}{\kern0pt}z{\isadigit{3}}{\isacharparenright}{\kern0pt}\ comp{\isacharunderscore}{\kern0pt}associative{\isadigit{2}}\ id{\isacharunderscore}{\kern0pt}left{\isacharunderscore}{\kern0pt}unit{\isadigit{2}}\ swap{\isacharunderscore}{\kern0pt}ap{\isacharparenright}{\kern0pt}\isanewline
\ \ \ \ \isacommand{also}\isamarkupfalse%
\ \isacommand{have}\isamarkupfalse%
\ {\isachardoublequoteopen}{\isachardot}{\kern0pt}{\isachardot}{\kern0pt}{\isachardot}{\kern0pt}\ {\isacharequal}{\kern0pt}\ {\isacharparenleft}{\kern0pt}eval{\isacharunderscore}{\kern0pt}func\ Z\ Y\ {\isasymcirc}\isactrlsub c\ swap\ {\isacharparenleft}{\kern0pt}Z\isactrlbsup Y\isactrlesup {\isacharparenright}{\kern0pt}\ Y{\isacharparenright}{\kern0pt}\ {\isasymcirc}\isactrlsub c\ {\isacharparenleft}{\kern0pt}id\isactrlsub c\ {\isacharparenleft}{\kern0pt}Z\isactrlbsup Y\isactrlesup {\isacharparenright}{\kern0pt}\ {\isasymtimes}\isactrlsub f\ {\isacharparenleft}{\kern0pt}eval{\isacharunderscore}{\kern0pt}func\ Y\ X\ {\isasymcirc}\isactrlsub c\ swap\ {\isacharparenleft}{\kern0pt}Y\isactrlbsup X\isactrlesup {\isacharparenright}{\kern0pt}\ X{\isacharparenright}{\kern0pt}{\isacharparenright}{\kern0pt}\ {\isasymcirc}\isactrlsub c\ \ \ {\isasymlangle}f{\isacharcomma}{\kern0pt}{\isasymlangle}g{\isacharcomma}{\kern0pt}\ x{\isasymrangle}{\isasymrangle}{\isachardoublequoteclose}\isanewline
\ \ \ \ \ \ \isacommand{using}\isamarkupfalse%
\ assms\ \isacommand{by}\isamarkupfalse%
\ {\isacharparenleft}{\kern0pt}typecheck{\isacharunderscore}{\kern0pt}cfuncs{\isacharcomma}{\kern0pt}\ simp\ add{\isacharcolon}{\kern0pt}\ cfunc{\isacharunderscore}{\kern0pt}cross{\isacharunderscore}{\kern0pt}prod{\isacharunderscore}{\kern0pt}comp{\isacharunderscore}{\kern0pt}cfunc{\isacharunderscore}{\kern0pt}prod{\isacharparenright}{\kern0pt}\isanewline
\ \ \ \ \isacommand{also}\isamarkupfalse%
\ \isacommand{have}\isamarkupfalse%
\ {\isachardoublequoteopen}{\isachardot}{\kern0pt}{\isachardot}{\kern0pt}{\isachardot}{\kern0pt}\ {\isacharequal}{\kern0pt}\ {\isacharparenleft}{\kern0pt}eval{\isacharunderscore}{\kern0pt}func\ Z\ Y\ {\isasymcirc}\isactrlsub c\ swap\ {\isacharparenleft}{\kern0pt}Z\isactrlbsup Y\isactrlesup {\isacharparenright}{\kern0pt}\ Y\ {\isasymcirc}\isactrlsub c\ {\isacharparenleft}{\kern0pt}id\isactrlsub c\ {\isacharparenleft}{\kern0pt}Z\isactrlbsup Y\isactrlesup {\isacharparenright}{\kern0pt}\ {\isasymtimes}\isactrlsub f\ eval{\isacharunderscore}{\kern0pt}func\ Y\ X\ {\isasymcirc}\isactrlsub c\ swap\ {\isacharparenleft}{\kern0pt}Y\isactrlbsup X\isactrlesup {\isacharparenright}{\kern0pt}\ X{\isacharparenright}{\kern0pt}{\isacharparenright}{\kern0pt}\ {\isasymcirc}\isactrlsub c\ \ \ {\isasymlangle}f{\isacharcomma}{\kern0pt}{\isasymlangle}g{\isacharcomma}{\kern0pt}\ x{\isasymrangle}{\isasymrangle}{\isachardoublequoteclose}\isanewline
\ \ \ \ \ \ \isacommand{using}\isamarkupfalse%
\ assms\ comp{\isacharunderscore}{\kern0pt}associative{\isadigit{2}}\ \isacommand{by}\isamarkupfalse%
\ {\isacharparenleft}{\kern0pt}typecheck{\isacharunderscore}{\kern0pt}cfuncs{\isacharcomma}{\kern0pt}\ force{\isacharparenright}{\kern0pt}\isanewline
\ \ \ \ \isacommand{also}\isamarkupfalse%
\ \isacommand{have}\isamarkupfalse%
\ {\isachardoublequoteopen}{\isachardot}{\kern0pt}{\isachardot}{\kern0pt}{\isachardot}{\kern0pt}\ {\isacharequal}{\kern0pt}\ {\isacharparenleft}{\kern0pt}eval{\isacharunderscore}{\kern0pt}func\ Z\ Y\ {\isasymcirc}\isactrlsub c\ swap\ {\isacharparenleft}{\kern0pt}Z\isactrlbsup Y\isactrlesup {\isacharparenright}{\kern0pt}\ Y\ {\isasymcirc}\isactrlsub c\ {\isacharparenleft}{\kern0pt}id\isactrlsub c\ {\isacharparenleft}{\kern0pt}Z\isactrlbsup Y\isactrlesup {\isacharparenright}{\kern0pt}\ {\isasymtimes}\isactrlsub f\ eval{\isacharunderscore}{\kern0pt}func\ Y\ X\ {\isasymcirc}\isactrlsub c\ swap\ {\isacharparenleft}{\kern0pt}Y\isactrlbsup X\isactrlesup {\isacharparenright}{\kern0pt}\ X{\isacharparenright}{\kern0pt}{\isacharparenright}{\kern0pt}\ {\isasymcirc}\isactrlsub c\ associate{\isacharunderscore}{\kern0pt}right\ {\isacharparenleft}{\kern0pt}Z\isactrlbsup Y\isactrlesup {\isacharparenright}{\kern0pt}\ {\isacharparenleft}{\kern0pt}Y\isactrlbsup X\isactrlesup {\isacharparenright}{\kern0pt}\ X\ {\isasymcirc}\isactrlsub c\ \ \ {\isasymlangle}{\isasymlangle}f{\isacharcomma}{\kern0pt}g{\isasymrangle}{\isacharcomma}{\kern0pt}\ x\ {\isasymrangle}{\isachardoublequoteclose}\isanewline
\ \ \ \ \ \ \isacommand{using}\isamarkupfalse%
\ assms\ \isacommand{by}\isamarkupfalse%
\ {\isacharparenleft}{\kern0pt}typecheck{\isacharunderscore}{\kern0pt}cfuncs{\isacharcomma}{\kern0pt}\ simp\ add{\isacharcolon}{\kern0pt}\ associate{\isacharunderscore}{\kern0pt}right{\isacharunderscore}{\kern0pt}ap{\isacharparenright}{\kern0pt}\isanewline
\ \ \ \ \isacommand{also}\isamarkupfalse%
\ \isacommand{have}\isamarkupfalse%
\ {\isachardoublequoteopen}{\isachardot}{\kern0pt}{\isachardot}{\kern0pt}{\isachardot}{\kern0pt}\ {\isacharequal}{\kern0pt}\ {\isacharparenleft}{\kern0pt}eval{\isacharunderscore}{\kern0pt}func\ Z\ Y\ {\isasymcirc}\isactrlsub c\ swap\ {\isacharparenleft}{\kern0pt}Z\isactrlbsup Y\isactrlesup {\isacharparenright}{\kern0pt}\ Y\ {\isasymcirc}\isactrlsub c\ {\isacharparenleft}{\kern0pt}id\isactrlsub c\ {\isacharparenleft}{\kern0pt}Z\isactrlbsup Y\isactrlesup {\isacharparenright}{\kern0pt}\ {\isasymtimes}\isactrlsub f\ eval{\isacharunderscore}{\kern0pt}func\ Y\ X\ {\isasymcirc}\isactrlsub c\ swap\ {\isacharparenleft}{\kern0pt}Y\isactrlbsup X\isactrlesup {\isacharparenright}{\kern0pt}\ X{\isacharparenright}{\kern0pt}\ {\isasymcirc}\isactrlsub c\ associate{\isacharunderscore}{\kern0pt}right\ {\isacharparenleft}{\kern0pt}Z\isactrlbsup Y\isactrlesup {\isacharparenright}{\kern0pt}\ {\isacharparenleft}{\kern0pt}Y\isactrlbsup X\isactrlesup {\isacharparenright}{\kern0pt}\ X{\isacharparenright}{\kern0pt}\ {\isasymcirc}\isactrlsub c\ \ \ {\isasymlangle}{\isasymlangle}f{\isacharcomma}{\kern0pt}g{\isasymrangle}{\isacharcomma}{\kern0pt}\ x\ {\isasymrangle}{\isachardoublequoteclose}\isanewline
\ \ \ \ \ \ \isacommand{using}\isamarkupfalse%
\ assms\ comp{\isacharunderscore}{\kern0pt}associative{\isadigit{2}}\ \isacommand{by}\isamarkupfalse%
\ {\isacharparenleft}{\kern0pt}typecheck{\isacharunderscore}{\kern0pt}cfuncs{\isacharcomma}{\kern0pt}\ force{\isacharparenright}{\kern0pt}\isanewline
\ \ \ \ \isacommand{also}\isamarkupfalse%
\ \isacommand{have}\isamarkupfalse%
\ {\isachardoublequoteopen}{\isachardot}{\kern0pt}{\isachardot}{\kern0pt}{\isachardot}{\kern0pt}\ {\isacharequal}{\kern0pt}\ {\isacharparenleft}{\kern0pt}eval{\isacharunderscore}{\kern0pt}func\ Z\ Y\ {\isasymcirc}\isactrlsub c\ swap\ {\isacharparenleft}{\kern0pt}Z\isactrlbsup Y\isactrlesup {\isacharparenright}{\kern0pt}\ Y\ {\isasymcirc}\isactrlsub c\ {\isacharparenleft}{\kern0pt}id\isactrlsub c\ {\isacharparenleft}{\kern0pt}Z\isactrlbsup Y\isactrlesup {\isacharparenright}{\kern0pt}\ {\isasymtimes}\isactrlsub f\ eval{\isacharunderscore}{\kern0pt}func\ Y\ X\ {\isasymcirc}\isactrlsub c\ swap\ {\isacharparenleft}{\kern0pt}Y\isactrlbsup X\isactrlesup {\isacharparenright}{\kern0pt}\ X{\isacharparenright}{\kern0pt}\ {\isasymcirc}\isactrlsub c\ associate{\isacharunderscore}{\kern0pt}right\ {\isacharparenleft}{\kern0pt}Z\isactrlbsup Y\isactrlesup {\isacharparenright}{\kern0pt}\ {\isacharparenleft}{\kern0pt}Y\isactrlbsup X\isactrlesup {\isacharparenright}{\kern0pt}\ X{\isacharparenright}{\kern0pt}\ {\isasymcirc}\isactrlsub c\ swap\ X\ {\isacharparenleft}{\kern0pt}Z\isactrlbsup Y\isactrlesup \ {\isasymtimes}\isactrlsub c\ Y\isactrlbsup X\isactrlesup {\isacharparenright}{\kern0pt}\ {\isasymcirc}\isactrlsub c\ \ \ {\isasymlangle}x{\isacharcomma}{\kern0pt}\ \ {\isasymlangle}f{\isacharcomma}{\kern0pt}g{\isasymrangle}{\isasymrangle}{\isachardoublequoteclose}\isanewline
\ \ \ \ \ \ \isacommand{using}\isamarkupfalse%
\ assms\ \isacommand{by}\isamarkupfalse%
\ {\isacharparenleft}{\kern0pt}typecheck{\isacharunderscore}{\kern0pt}cfuncs{\isacharcomma}{\kern0pt}\ simp\ add{\isacharcolon}{\kern0pt}\ swap{\isacharunderscore}{\kern0pt}ap{\isacharparenright}{\kern0pt}\isanewline
\ \ \ \ \isacommand{also}\isamarkupfalse%
\ \isacommand{have}\isamarkupfalse%
\ {\isachardoublequoteopen}{\isachardot}{\kern0pt}{\isachardot}{\kern0pt}{\isachardot}{\kern0pt}\ {\isacharequal}{\kern0pt}\ {\isacharparenleft}{\kern0pt}eval{\isacharunderscore}{\kern0pt}func\ Z\ Y\ {\isasymcirc}\isactrlsub c\ swap\ {\isacharparenleft}{\kern0pt}Z\isactrlbsup Y\isactrlesup {\isacharparenright}{\kern0pt}\ Y\ {\isasymcirc}\isactrlsub c\ {\isacharparenleft}{\kern0pt}id\isactrlsub c\ {\isacharparenleft}{\kern0pt}Z\isactrlbsup Y\isactrlesup {\isacharparenright}{\kern0pt}\ {\isasymtimes}\isactrlsub f\ eval{\isacharunderscore}{\kern0pt}func\ Y\ X\ {\isasymcirc}\isactrlsub c\ swap\ {\isacharparenleft}{\kern0pt}Y\isactrlbsup X\isactrlesup {\isacharparenright}{\kern0pt}\ X{\isacharparenright}{\kern0pt}\ {\isasymcirc}\isactrlsub c\ associate{\isacharunderscore}{\kern0pt}right\ {\isacharparenleft}{\kern0pt}Z\isactrlbsup Y\isactrlesup {\isacharparenright}{\kern0pt}\ {\isacharparenleft}{\kern0pt}Y\isactrlbsup X\isactrlesup {\isacharparenright}{\kern0pt}\ X\ {\isasymcirc}\isactrlsub c\ swap\ X\ {\isacharparenleft}{\kern0pt}Z\isactrlbsup Y\isactrlesup \ {\isasymtimes}\isactrlsub c\ Y\isactrlbsup X\isactrlesup {\isacharparenright}{\kern0pt}{\isacharparenright}{\kern0pt}\ {\isasymcirc}\isactrlsub c\ \ \ {\isasymlangle}x{\isacharcomma}{\kern0pt}\ \ {\isasymlangle}f{\isacharcomma}{\kern0pt}g{\isasymrangle}{\isasymrangle}{\isachardoublequoteclose}\isanewline
\ \ \ \ \ \ \isacommand{using}\isamarkupfalse%
\ assms\ comp{\isacharunderscore}{\kern0pt}associative{\isadigit{2}}\ \isacommand{by}\isamarkupfalse%
\ {\isacharparenleft}{\kern0pt}typecheck{\isacharunderscore}{\kern0pt}cfuncs{\isacharcomma}{\kern0pt}\ force{\isacharparenright}{\kern0pt}\isanewline
\ \ \ \ \isacommand{also}\isamarkupfalse%
\ \isacommand{have}\isamarkupfalse%
\ {\isachardoublequoteopen}{\isachardot}{\kern0pt}{\isachardot}{\kern0pt}{\isachardot}{\kern0pt}\ {\isacharequal}{\kern0pt}\ {\isacharparenleft}{\kern0pt}eval{\isacharunderscore}{\kern0pt}func\ Z\ Y\ {\isasymcirc}\isactrlsub c\ swap\ {\isacharparenleft}{\kern0pt}Z\isactrlbsup Y\isactrlesup {\isacharparenright}{\kern0pt}\ Y\ {\isasymcirc}\isactrlsub c\ {\isacharparenleft}{\kern0pt}id\isactrlsub c\ {\isacharparenleft}{\kern0pt}Z\isactrlbsup Y\isactrlesup {\isacharparenright}{\kern0pt}\ {\isasymtimes}\isactrlsub f\ eval{\isacharunderscore}{\kern0pt}func\ Y\ X\ {\isasymcirc}\isactrlsub c\ swap\ {\isacharparenleft}{\kern0pt}Y\isactrlbsup X\isactrlesup {\isacharparenright}{\kern0pt}\ X{\isacharparenright}{\kern0pt}\ {\isasymcirc}\isactrlsub c\ associate{\isacharunderscore}{\kern0pt}right\ {\isacharparenleft}{\kern0pt}Z\isactrlbsup Y\isactrlesup {\isacharparenright}{\kern0pt}\ {\isacharparenleft}{\kern0pt}Y\isactrlbsup X\isactrlesup {\isacharparenright}{\kern0pt}\ X\ {\isasymcirc}\isactrlsub c\ swap\ X\ {\isacharparenleft}{\kern0pt}Z\isactrlbsup Y\isactrlesup \ {\isasymtimes}\isactrlsub c\ Y\isactrlbsup X\isactrlesup {\isacharparenright}{\kern0pt}{\isacharparenright}{\kern0pt}\ {\isasymcirc}\isactrlsub c\ \ \ {\isacharparenleft}{\kern0pt}{\isacharparenleft}{\kern0pt}id\isactrlsub c\ X\ {\isasymtimes}\isactrlsub f\ {\isasymlangle}f{\isacharcomma}{\kern0pt}g{\isasymrangle}{\isacharparenright}{\kern0pt}\ {\isasymcirc}\isactrlsub c\ \ x{\isadigit{1}}{\isacharparenright}{\kern0pt}{\isachardoublequoteclose}\isanewline
\ \ \ \ \ \ \isacommand{using}\isamarkupfalse%
\ assms\ \isacommand{by}\isamarkupfalse%
\ {\isacharparenleft}{\kern0pt}typecheck{\isacharunderscore}{\kern0pt}cfuncs{\isacharcomma}{\kern0pt}\ smt\ {\isacharparenleft}{\kern0pt}z{\isadigit{3}}{\isacharparenright}{\kern0pt}\ cfunc{\isacharunderscore}{\kern0pt}cross{\isacharunderscore}{\kern0pt}prod{\isacharunderscore}{\kern0pt}comp{\isacharunderscore}{\kern0pt}cfunc{\isacharunderscore}{\kern0pt}prod\ id{\isacharunderscore}{\kern0pt}left{\isacharunderscore}{\kern0pt}unit{\isadigit{2}}\ id{\isacharunderscore}{\kern0pt}right{\isacharunderscore}{\kern0pt}unit{\isadigit{2}}\ id{\isacharunderscore}{\kern0pt}type\ x{\isacharunderscore}{\kern0pt}def{\isacharparenright}{\kern0pt}\isanewline
\ \ \ \ \isacommand{also}\isamarkupfalse%
\ \isacommand{have}\isamarkupfalse%
\ {\isachardoublequoteopen}{\isachardot}{\kern0pt}{\isachardot}{\kern0pt}{\isachardot}{\kern0pt}\ {\isacharequal}{\kern0pt}\ {\isacharparenleft}{\kern0pt}{\isacharparenleft}{\kern0pt}eval{\isacharunderscore}{\kern0pt}func\ Z\ Y\ {\isasymcirc}\isactrlsub c\ swap\ {\isacharparenleft}{\kern0pt}Z\isactrlbsup Y\isactrlesup {\isacharparenright}{\kern0pt}\ Y\ {\isasymcirc}\isactrlsub c\ {\isacharparenleft}{\kern0pt}id\isactrlsub c\ {\isacharparenleft}{\kern0pt}Z\isactrlbsup Y\isactrlesup {\isacharparenright}{\kern0pt}\ {\isasymtimes}\isactrlsub f\ eval{\isacharunderscore}{\kern0pt}func\ Y\ X\ {\isasymcirc}\isactrlsub c\ swap\ {\isacharparenleft}{\kern0pt}Y\isactrlbsup X\isactrlesup {\isacharparenright}{\kern0pt}\ X{\isacharparenright}{\kern0pt}\ {\isasymcirc}\isactrlsub c\ associate{\isacharunderscore}{\kern0pt}right\ {\isacharparenleft}{\kern0pt}Z\isactrlbsup Y\isactrlesup {\isacharparenright}{\kern0pt}\ {\isacharparenleft}{\kern0pt}Y\isactrlbsup X\isactrlesup {\isacharparenright}{\kern0pt}\ X\ {\isasymcirc}\isactrlsub c\ swap\ X\ {\isacharparenleft}{\kern0pt}Z\isactrlbsup Y\isactrlesup \ {\isasymtimes}\isactrlsub c\ Y\isactrlbsup X\isactrlesup {\isacharparenright}{\kern0pt}{\isacharparenright}{\kern0pt}\ {\isasymcirc}\isactrlsub c\ id\isactrlsub c\ X\ {\isasymtimes}\isactrlsub f\ {\isasymlangle}f{\isacharcomma}{\kern0pt}g{\isasymrangle}{\isacharparenright}{\kern0pt}\ {\isasymcirc}\isactrlsub c\ x{\isadigit{1}}{\isachardoublequoteclose}\isanewline
\ \ \ \ \ \ \isacommand{by}\isamarkupfalse%
\ {\isacharparenleft}{\kern0pt}typecheck{\isacharunderscore}{\kern0pt}cfuncs{\isacharcomma}{\kern0pt}\ meson\ comp{\isacharunderscore}{\kern0pt}associative{\isadigit{2}}{\isacharparenright}{\kern0pt}\isanewline
\ \ \ \ \isacommand{then}\isamarkupfalse%
\ \isacommand{show}\isamarkupfalse%
\ {\isachardoublequoteopen}{\isacharparenleft}{\kern0pt}{\isacharparenleft}{\kern0pt}{\isacharparenleft}{\kern0pt}eval{\isacharunderscore}{\kern0pt}func\ Z\ Y\ {\isasymcirc}\isactrlsub c\ {\isasymlangle}id\isactrlsub c\ Y{\isacharcomma}{\kern0pt}f\ {\isasymcirc}\isactrlsub c\ {\isasymbeta}\isactrlbsub Y\isactrlesub {\isasymrangle}{\isacharparenright}{\kern0pt}\ {\isasymcirc}\isactrlsub c\ eval{\isacharunderscore}{\kern0pt}func\ Y\ X\ {\isasymcirc}\isactrlsub c\ {\isasymlangle}id\isactrlsub c\ X{\isacharcomma}{\kern0pt}g\ {\isasymcirc}\isactrlsub c\ {\isasymbeta}\isactrlbsub X\isactrlesub {\isasymrangle}{\isacharparenright}{\kern0pt}\ {\isasymcirc}\isactrlsub c\ left{\isacharunderscore}{\kern0pt}cart{\isacharunderscore}{\kern0pt}proj\ X\ {\isasymone}{\isacharparenright}{\kern0pt}\ {\isasymcirc}\isactrlsub c\ x{\isadigit{1}}\ {\isacharequal}{\kern0pt}\isanewline
\ \ \ \ \ \ \ \ \ {\isacharparenleft}{\kern0pt}{\isacharparenleft}{\kern0pt}eval{\isacharunderscore}{\kern0pt}func\ Z\ Y\ {\isasymcirc}\isactrlsub c\ swap\ {\isacharparenleft}{\kern0pt}Z\isactrlbsup Y\isactrlesup {\isacharparenright}{\kern0pt}\ Y\ {\isasymcirc}\isactrlsub c\ {\isacharparenleft}{\kern0pt}id\isactrlsub c\ {\isacharparenleft}{\kern0pt}Z\isactrlbsup Y\isactrlesup {\isacharparenright}{\kern0pt}\ {\isasymtimes}\isactrlsub f\ eval{\isacharunderscore}{\kern0pt}func\ Y\ X\ {\isasymcirc}\isactrlsub c\ swap\ {\isacharparenleft}{\kern0pt}Y\isactrlbsup X\isactrlesup {\isacharparenright}{\kern0pt}\ X{\isacharparenright}{\kern0pt}\ {\isasymcirc}\isactrlsub c\ associate{\isacharunderscore}{\kern0pt}right\ {\isacharparenleft}{\kern0pt}Z\isactrlbsup Y\isactrlesup {\isacharparenright}{\kern0pt}\ {\isacharparenleft}{\kern0pt}Y\isactrlbsup X\isactrlesup {\isacharparenright}{\kern0pt}\ X\ {\isasymcirc}\isactrlsub c\ swap\ X\ {\isacharparenleft}{\kern0pt}Z\isactrlbsup Y\isactrlesup \ {\isasymtimes}\isactrlsub c\ Y\isactrlbsup X\isactrlesup {\isacharparenright}{\kern0pt}{\isacharparenright}{\kern0pt}\ {\isasymcirc}\isactrlsub c\ id\isactrlsub c\ X\ {\isasymtimes}\isactrlsub f\ {\isasymlangle}f{\isacharcomma}{\kern0pt}g{\isasymrangle}{\isacharparenright}{\kern0pt}\ {\isasymcirc}\isactrlsub c\ x{\isadigit{1}}{\isachardoublequoteclose}\isanewline
\ \ \ \ \ \ \isacommand{using}\isamarkupfalse%
\ calculation\ \isacommand{by}\isamarkupfalse%
\ presburger\isanewline
\ \ \isacommand{qed}\isamarkupfalse%
\isanewline
\ \ \isacommand{then}\isamarkupfalse%
\ \isacommand{have}\isamarkupfalse%
\ {\isachardoublequoteopen}{\isacharparenleft}{\kern0pt}{\isacharparenleft}{\kern0pt}{\isacharparenleft}{\kern0pt}eval{\isacharunderscore}{\kern0pt}func\ Z\ Y\ {\isasymcirc}\isactrlsub c\ {\isasymlangle}id\isactrlsub c\ Y{\isacharcomma}{\kern0pt}f\ {\isasymcirc}\isactrlsub c\ {\isasymbeta}\isactrlbsub Y\isactrlesub {\isasymrangle}{\isacharparenright}{\kern0pt}\ {\isasymcirc}\isactrlsub c\ eval{\isacharunderscore}{\kern0pt}func\ Y\ X\ {\isasymcirc}\isactrlsub c\ {\isasymlangle}id\isactrlsub c\ X{\isacharcomma}{\kern0pt}g\ {\isasymcirc}\isactrlsub c\ {\isasymbeta}\isactrlbsub X\isactrlesub {\isasymrangle}{\isacharparenright}{\kern0pt}\ {\isasymcirc}\isactrlsub c\isanewline
\ \ \ \ \ left{\isacharunderscore}{\kern0pt}cart{\isacharunderscore}{\kern0pt}proj\ X\ {\isasymone}{\isacharparenright}{\kern0pt}\isactrlsup {\isasymsharp}\ {\isacharequal}{\kern0pt}\ \ {\isacharparenleft}{\kern0pt}eval{\isacharunderscore}{\kern0pt}func\ Z\ Y\ {\isasymcirc}\isactrlsub c\ \ swap\ {\isacharparenleft}{\kern0pt}Z\isactrlbsup Y\isactrlesup {\isacharparenright}{\kern0pt}\ Y\ {\isasymcirc}\isactrlsub c\ {\isacharparenleft}{\kern0pt}id\isactrlsub c\ {\isacharparenleft}{\kern0pt}Z\isactrlbsup Y\isactrlesup {\isacharparenright}{\kern0pt}\ {\isasymtimes}\isactrlsub f\ {\isacharparenleft}{\kern0pt}eval{\isacharunderscore}{\kern0pt}func\ Y\ X\ {\isasymcirc}\isactrlsub c\ swap\ {\isacharparenleft}{\kern0pt}Y\isactrlbsup X\isactrlesup {\isacharparenright}{\kern0pt}\ X{\isacharparenright}{\kern0pt}{\isacharparenright}{\kern0pt}\isanewline
\ \ \ \ \ \ \ \ \ {\isasymcirc}\isactrlsub c\ associate{\isacharunderscore}{\kern0pt}right\ {\isacharparenleft}{\kern0pt}Z\isactrlbsup Y\isactrlesup {\isacharparenright}{\kern0pt}\ {\isacharparenleft}{\kern0pt}Y\isactrlbsup X\isactrlesup {\isacharparenright}{\kern0pt}\ X\ {\isasymcirc}\isactrlsub c\ swap\ X\ {\isacharparenleft}{\kern0pt}Z\isactrlbsup Y\isactrlesup \ {\isasymtimes}\isactrlsub c\ Y\isactrlbsup X\isactrlesup {\isacharparenright}{\kern0pt}{\isacharparenright}{\kern0pt}\isactrlsup {\isasymsharp}\ {\isasymcirc}\isactrlsub c\ {\isasymlangle}f{\isacharcomma}{\kern0pt}g{\isasymrangle}{\isachardoublequoteclose}\isanewline
\ \ \ \ \isacommand{using}\isamarkupfalse%
\ assms\ \isacommand{by}\isamarkupfalse%
\ {\isacharparenleft}{\kern0pt}typecheck{\isacharunderscore}{\kern0pt}cfuncs{\isacharcomma}{\kern0pt}\ simp\ add{\isacharcolon}{\kern0pt}\ sharp{\isacharunderscore}{\kern0pt}comp{\isacharparenright}{\kern0pt}\ \ \isanewline
\ \ \isacommand{then}\isamarkupfalse%
\ \isacommand{show}\isamarkupfalse%
\ {\isachardoublequoteopen}{\isacharparenleft}{\kern0pt}f\isactrlsup {\isasymflat}\ {\isasymcirc}\isactrlsub c\ {\isasymlangle}g\isactrlsup {\isasymflat}{\isacharcomma}{\kern0pt}right{\isacharunderscore}{\kern0pt}cart{\isacharunderscore}{\kern0pt}proj\ X\ {\isasymone}{\isasymrangle}{\isacharparenright}{\kern0pt}\isactrlsup {\isasymsharp}\ {\isacharequal}{\kern0pt}\isanewline
\ \ \ \ {\isacharparenleft}{\kern0pt}eval{\isacharunderscore}{\kern0pt}func\ Z\ Y\ {\isasymcirc}\isactrlsub c\ swap\ {\isacharparenleft}{\kern0pt}Z\isactrlbsup Y\isactrlesup {\isacharparenright}{\kern0pt}\ Y\ {\isasymcirc}\isactrlsub c\ {\isacharparenleft}{\kern0pt}id\isactrlsub c\ {\isacharparenleft}{\kern0pt}Z\isactrlbsup Y\isactrlesup {\isacharparenright}{\kern0pt}\ {\isasymtimes}\isactrlsub f\ eval{\isacharunderscore}{\kern0pt}func\ Y\ X\ {\isasymcirc}\isactrlsub c\ swap\ {\isacharparenleft}{\kern0pt}Y\isactrlbsup X\isactrlesup {\isacharparenright}{\kern0pt}\ X{\isacharparenright}{\kern0pt}\ {\isasymcirc}\isactrlsub c\ associate{\isacharunderscore}{\kern0pt}right\ {\isacharparenleft}{\kern0pt}Z\isactrlbsup Y\isactrlesup {\isacharparenright}{\kern0pt}\ {\isacharparenleft}{\kern0pt}Y\isactrlbsup X\isactrlesup {\isacharparenright}{\kern0pt}\ X\ {\isasymcirc}\isactrlsub c\ swap\ X\ {\isacharparenleft}{\kern0pt}Z\isactrlbsup Y\isactrlesup \ {\isasymtimes}\isactrlsub c\ Y\isactrlbsup X\isactrlesup {\isacharparenright}{\kern0pt}{\isacharparenright}{\kern0pt}\isactrlsup {\isasymsharp}\ {\isasymcirc}\isactrlsub c\ {\isasymlangle}f{\isacharcomma}{\kern0pt}g{\isasymrangle}{\isachardoublequoteclose}\isanewline
\ \ \ \ \isacommand{using}\isamarkupfalse%
\ assms\ cfunc{\isacharunderscore}{\kern0pt}type{\isacharunderscore}{\kern0pt}def\ cnufatem{\isacharunderscore}{\kern0pt}def{\isadigit{2}}\ cnufatem{\isacharunderscore}{\kern0pt}type\ domain{\isacharunderscore}{\kern0pt}comp\ meta{\isacharunderscore}{\kern0pt}comp{\isadigit{2}}{\isacharunderscore}{\kern0pt}def{\isadigit{2}}\ meta{\isacharunderscore}{\kern0pt}comp{\isadigit{2}}{\isacharunderscore}{\kern0pt}def{\isadigit{3}}\ metafunc{\isacharunderscore}{\kern0pt}def\ \isacommand{by}\isamarkupfalse%
\ force\isanewline
\isacommand{qed}\isamarkupfalse%
%
\endisatagproof
{\isafoldproof}%
%
\isadelimproof
\isanewline
%
\endisadelimproof
\isanewline
\isacommand{lemma}\isamarkupfalse%
\ meta{\isacharunderscore}{\kern0pt}comp{\isacharunderscore}{\kern0pt}on{\isacharunderscore}{\kern0pt}els{\isacharcolon}{\kern0pt}\isanewline
\ \ \isakeyword{assumes}\ {\isachardoublequoteopen}f\ {\isacharcolon}{\kern0pt}\ W\ {\isasymrightarrow}\ Z\isactrlbsup Y\isactrlesup {\isachardoublequoteclose}\isanewline
\ \ \isakeyword{assumes}\ {\isachardoublequoteopen}g\ {\isacharcolon}{\kern0pt}\ W\ {\isasymrightarrow}\ Y\isactrlbsup X\isactrlesup {\isachardoublequoteclose}\isanewline
\ \ \isakeyword{assumes}\ {\isachardoublequoteopen}w\ {\isasymin}\isactrlsub c\ W{\isachardoublequoteclose}\isanewline
\ \ \isakeyword{shows}\ {\isachardoublequoteopen}{\isacharparenleft}{\kern0pt}f\ {\isasymbox}\ g{\isacharparenright}{\kern0pt}\ {\isasymcirc}\isactrlsub c\ w\ {\isacharequal}{\kern0pt}\ {\isacharparenleft}{\kern0pt}f\ {\isasymcirc}\isactrlsub c\ w{\isacharparenright}{\kern0pt}\ {\isasymbox}\ {\isacharparenleft}{\kern0pt}g\ {\isasymcirc}\isactrlsub c\ w{\isacharparenright}{\kern0pt}{\isachardoublequoteclose}\isanewline
%
\isadelimproof
%
\endisadelimproof
%
\isatagproof
\isacommand{proof}\isamarkupfalse%
\ {\isacharminus}{\kern0pt}\ \isanewline
\ \ \isacommand{have}\isamarkupfalse%
\ {\isachardoublequoteopen}{\isacharparenleft}{\kern0pt}f\ {\isasymbox}\ g{\isacharparenright}{\kern0pt}\ {\isasymcirc}\isactrlsub c\ w\ {\isacharequal}{\kern0pt}\ {\isacharparenleft}{\kern0pt}f\isactrlsup {\isasymflat}\ {\isasymcirc}\isactrlsub c\ {\isasymlangle}g\isactrlsup {\isasymflat}{\isacharcomma}{\kern0pt}\ right{\isacharunderscore}{\kern0pt}cart{\isacharunderscore}{\kern0pt}proj\ X\ W{\isasymrangle}{\isacharparenright}{\kern0pt}\isactrlsup {\isasymsharp}\ {\isasymcirc}\isactrlsub c\ w{\isachardoublequoteclose}\isanewline
\ \ \ \ \isacommand{using}\isamarkupfalse%
\ assms\ \isacommand{by}\isamarkupfalse%
\ {\isacharparenleft}{\kern0pt}typecheck{\isacharunderscore}{\kern0pt}cfuncs{\isacharcomma}{\kern0pt}\ simp\ add{\isacharcolon}{\kern0pt}\ meta{\isacharunderscore}{\kern0pt}comp{\isadigit{2}}{\isacharunderscore}{\kern0pt}def{\isadigit{2}}{\isacharparenright}{\kern0pt}\isanewline
\ \ \isacommand{also}\isamarkupfalse%
\ \isacommand{have}\isamarkupfalse%
\ {\isachardoublequoteopen}{\isachardot}{\kern0pt}{\isachardot}{\kern0pt}{\isachardot}{\kern0pt}\ {\isacharequal}{\kern0pt}\ {\isacharparenleft}{\kern0pt}eval{\isacharunderscore}{\kern0pt}func\ Z\ Y\ {\isasymcirc}\isactrlsub c\ {\isacharparenleft}{\kern0pt}id\ Y\ {\isasymtimes}\isactrlsub f\ f{\isacharparenright}{\kern0pt}\ {\isasymcirc}\isactrlsub c\ {\isasymlangle}eval{\isacharunderscore}{\kern0pt}func\ Y\ X\ {\isasymcirc}\isactrlsub c\ {\isacharparenleft}{\kern0pt}id\ X\ {\isasymtimes}\isactrlsub f\ g{\isacharparenright}{\kern0pt}{\isacharcomma}{\kern0pt}\ right{\isacharunderscore}{\kern0pt}cart{\isacharunderscore}{\kern0pt}proj\ X\ W{\isasymrangle}{\isacharparenright}{\kern0pt}\isactrlsup {\isasymsharp}\ {\isasymcirc}\isactrlsub c\ w{\isachardoublequoteclose}\isanewline
\ \ \ \ \isacommand{using}\isamarkupfalse%
\ assms\ comp{\isacharunderscore}{\kern0pt}associative{\isadigit{2}}\ inv{\isacharunderscore}{\kern0pt}transpose{\isacharunderscore}{\kern0pt}func{\isacharunderscore}{\kern0pt}def{\isadigit{3}}\ \isacommand{by}\isamarkupfalse%
\ {\isacharparenleft}{\kern0pt}typecheck{\isacharunderscore}{\kern0pt}cfuncs{\isacharcomma}{\kern0pt}\ force{\isacharparenright}{\kern0pt}\isanewline
\ \ \isacommand{also}\isamarkupfalse%
\ \isacommand{have}\isamarkupfalse%
\ {\isachardoublequoteopen}{\isachardot}{\kern0pt}{\isachardot}{\kern0pt}{\isachardot}{\kern0pt}\ {\isacharequal}{\kern0pt}\ {\isacharparenleft}{\kern0pt}eval{\isacharunderscore}{\kern0pt}func\ Z\ Y\ {\isasymcirc}\isactrlsub c\ {\isasymlangle}eval{\isacharunderscore}{\kern0pt}func\ Y\ X\ {\isasymcirc}\isactrlsub c\ {\isacharparenleft}{\kern0pt}id\ X\ {\isasymtimes}\isactrlsub f\ g{\isacharparenright}{\kern0pt}{\isacharcomma}{\kern0pt}\ f\ {\isasymcirc}\isactrlsub c\ right{\isacharunderscore}{\kern0pt}cart{\isacharunderscore}{\kern0pt}proj\ X\ W{\isasymrangle}{\isacharparenright}{\kern0pt}\isactrlsup {\isasymsharp}\ {\isasymcirc}\isactrlsub c\ w{\isachardoublequoteclose}\isanewline
\ \ \ \ \isacommand{using}\isamarkupfalse%
\ assms\ \isacommand{by}\isamarkupfalse%
\ {\isacharparenleft}{\kern0pt}typecheck{\isacharunderscore}{\kern0pt}cfuncs{\isacharcomma}{\kern0pt}\ simp\ add{\isacharcolon}{\kern0pt}\ cfunc{\isacharunderscore}{\kern0pt}cross{\isacharunderscore}{\kern0pt}prod{\isacharunderscore}{\kern0pt}comp{\isacharunderscore}{\kern0pt}cfunc{\isacharunderscore}{\kern0pt}prod\ id{\isacharunderscore}{\kern0pt}left{\isacharunderscore}{\kern0pt}unit{\isadigit{2}}{\isacharparenright}{\kern0pt}\isanewline
\ \ \isacommand{also}\isamarkupfalse%
\ \isacommand{have}\isamarkupfalse%
\ {\isachardoublequoteopen}{\isachardot}{\kern0pt}{\isachardot}{\kern0pt}{\isachardot}{\kern0pt}\ {\isacharequal}{\kern0pt}\ {\isacharparenleft}{\kern0pt}eval{\isacharunderscore}{\kern0pt}func\ Z\ Y\ {\isasymcirc}\isactrlsub c\ {\isasymlangle}eval{\isacharunderscore}{\kern0pt}func\ Y\ X\ {\isasymcirc}\isactrlsub c\ {\isacharparenleft}{\kern0pt}id\ X\ {\isasymtimes}\isactrlsub f\ {\isacharparenleft}{\kern0pt}g{\isasymcirc}\isactrlsub c\ w{\isacharparenright}{\kern0pt}{\isacharparenright}{\kern0pt}{\isacharcomma}{\kern0pt}\ {\isacharparenleft}{\kern0pt}f\ {\isasymcirc}\isactrlsub c\ w{\isacharparenright}{\kern0pt}\ {\isasymcirc}\isactrlsub c\ right{\isacharunderscore}{\kern0pt}cart{\isacharunderscore}{\kern0pt}proj\ X\ {\isasymone}{\isasymrangle}{\isacharparenright}{\kern0pt}\isactrlsup {\isasymsharp}{\isachardoublequoteclose}\isanewline
\ \ \isacommand{proof}\isamarkupfalse%
\ {\isacharminus}{\kern0pt}\ \isanewline
\ \ \ \ \isacommand{have}\isamarkupfalse%
\ {\isachardoublequoteopen}{\isacharparenleft}{\kern0pt}eval{\isacharunderscore}{\kern0pt}func\ Z\ Y\ {\isasymcirc}\isactrlsub c\ {\isasymlangle}eval{\isacharunderscore}{\kern0pt}func\ Y\ X\ {\isasymcirc}\isactrlsub c\ {\isacharparenleft}{\kern0pt}id\ X\ {\isasymtimes}\isactrlsub f\ g{\isacharparenright}{\kern0pt}{\isacharcomma}{\kern0pt}\ f\ {\isasymcirc}\isactrlsub c\ right{\isacharunderscore}{\kern0pt}cart{\isacharunderscore}{\kern0pt}proj\ X\ W{\isasymrangle}{\isacharparenright}{\kern0pt}\isactrlsup {\isasymsharp}\isactrlsup {\isasymflat}\ {\isasymcirc}\isactrlsub c\ {\isacharparenleft}{\kern0pt}id\ X\ {\isasymtimes}\isactrlsub f\ w{\isacharparenright}{\kern0pt}\ {\isacharequal}{\kern0pt}\ \isanewline
\ \ \ \ \ \ \ \ \ \ eval{\isacharunderscore}{\kern0pt}func\ Z\ Y\ {\isasymcirc}\isactrlsub c\ {\isasymlangle}eval{\isacharunderscore}{\kern0pt}func\ Y\ X\ {\isasymcirc}\isactrlsub c\ {\isacharparenleft}{\kern0pt}id\ X\ {\isasymtimes}\isactrlsub f\ {\isacharparenleft}{\kern0pt}g{\isasymcirc}\isactrlsub c\ w{\isacharparenright}{\kern0pt}{\isacharparenright}{\kern0pt}{\isacharcomma}{\kern0pt}\ f\ {\isasymcirc}\isactrlsub c\ right{\isacharunderscore}{\kern0pt}cart{\isacharunderscore}{\kern0pt}proj\ X\ W\ {\isasymcirc}\isactrlsub c\ {\isacharparenleft}{\kern0pt}id\ X\ {\isasymtimes}\isactrlsub f\ w{\isacharparenright}{\kern0pt}{\isasymrangle}{\isachardoublequoteclose}\isanewline
\ \ \ \ \isacommand{proof}\isamarkupfalse%
\ {\isacharminus}{\kern0pt}\ \isanewline
\ \ \ \ \ \ \isacommand{have}\isamarkupfalse%
\ {\isachardoublequoteopen}eval{\isacharunderscore}{\kern0pt}func\ Z\ Y\ {\isasymcirc}\isactrlsub c\ {\isasymlangle}eval{\isacharunderscore}{\kern0pt}func\ Y\ X\ {\isasymcirc}\isactrlsub c\ {\isacharparenleft}{\kern0pt}id\ X\ {\isasymtimes}\isactrlsub f\ g{\isacharparenright}{\kern0pt}{\isacharcomma}{\kern0pt}\ f\ {\isasymcirc}\isactrlsub c\ right{\isacharunderscore}{\kern0pt}cart{\isacharunderscore}{\kern0pt}proj\ X\ W{\isasymrangle}\ {\isasymcirc}\isactrlsub c\ {\isacharparenleft}{\kern0pt}id\ X\ {\isasymtimes}\isactrlsub f\ w{\isacharparenright}{\kern0pt}\ \isanewline
\ \ \ \ \ \ \ \ \ \ {\isacharequal}{\kern0pt}\ \ eval{\isacharunderscore}{\kern0pt}func\ Z\ Y\ {\isasymcirc}\isactrlsub c\ {\isasymlangle}{\isacharparenleft}{\kern0pt}eval{\isacharunderscore}{\kern0pt}func\ Y\ X\ {\isasymcirc}\isactrlsub c\ {\isacharparenleft}{\kern0pt}id\ X\ {\isasymtimes}\isactrlsub f\ g{\isacharparenright}{\kern0pt}{\isacharparenright}{\kern0pt}\ {\isasymcirc}\isactrlsub c\ {\isacharparenleft}{\kern0pt}id\ X\ {\isasymtimes}\isactrlsub f\ w{\isacharparenright}{\kern0pt}{\isacharcomma}{\kern0pt}\ {\isacharparenleft}{\kern0pt}f\ {\isasymcirc}\isactrlsub c\ right{\isacharunderscore}{\kern0pt}cart{\isacharunderscore}{\kern0pt}proj\ X\ W{\isacharparenright}{\kern0pt}\ {\isasymcirc}\isactrlsub c\ {\isacharparenleft}{\kern0pt}id\ X\ {\isasymtimes}\isactrlsub f\ w{\isacharparenright}{\kern0pt}{\isasymrangle}{\isachardoublequoteclose}\isanewline
\ \ \ \ \ \ \ \ \ \isacommand{using}\isamarkupfalse%
\ assms\ cfunc{\isacharunderscore}{\kern0pt}prod{\isacharunderscore}{\kern0pt}comp\ \isacommand{by}\isamarkupfalse%
\ {\isacharparenleft}{\kern0pt}typecheck{\isacharunderscore}{\kern0pt}cfuncs{\isacharcomma}{\kern0pt}\ force{\isacharparenright}{\kern0pt}\isanewline
\ \ \ \ \ \ \ \isacommand{also}\isamarkupfalse%
\ \isacommand{have}\isamarkupfalse%
\ {\isachardoublequoteopen}{\isachardot}{\kern0pt}{\isachardot}{\kern0pt}{\isachardot}{\kern0pt}\ {\isacharequal}{\kern0pt}\ eval{\isacharunderscore}{\kern0pt}func\ Z\ Y\ {\isasymcirc}\isactrlsub c\ {\isasymlangle}eval{\isacharunderscore}{\kern0pt}func\ Y\ X\ {\isasymcirc}\isactrlsub c\ {\isacharparenleft}{\kern0pt}id\ X\ {\isasymtimes}\isactrlsub f\ g{\isacharparenright}{\kern0pt}\ {\isasymcirc}\isactrlsub c\ {\isacharparenleft}{\kern0pt}id\ X\ {\isasymtimes}\isactrlsub f\ w{\isacharparenright}{\kern0pt}{\isacharcomma}{\kern0pt}\ f\ {\isasymcirc}\isactrlsub c\ right{\isacharunderscore}{\kern0pt}cart{\isacharunderscore}{\kern0pt}proj\ X\ W\ {\isasymcirc}\isactrlsub c\ {\isacharparenleft}{\kern0pt}id\ X\ {\isasymtimes}\isactrlsub f\ w{\isacharparenright}{\kern0pt}{\isasymrangle}{\isachardoublequoteclose}\isanewline
\ \ \ \ \ \ \ \ \ \isacommand{using}\isamarkupfalse%
\ assms\ comp{\isacharunderscore}{\kern0pt}associative{\isadigit{2}}\ \isacommand{by}\isamarkupfalse%
\ {\isacharparenleft}{\kern0pt}typecheck{\isacharunderscore}{\kern0pt}cfuncs{\isacharcomma}{\kern0pt}\ auto{\isacharparenright}{\kern0pt}\isanewline
\ \ \ \ \ \ \ \isacommand{also}\isamarkupfalse%
\ \isacommand{have}\isamarkupfalse%
\ {\isachardoublequoteopen}{\isachardot}{\kern0pt}{\isachardot}{\kern0pt}{\isachardot}{\kern0pt}\ {\isacharequal}{\kern0pt}\ eval{\isacharunderscore}{\kern0pt}func\ Z\ Y\ {\isasymcirc}\isactrlsub c\ {\isasymlangle}eval{\isacharunderscore}{\kern0pt}func\ Y\ X\ {\isasymcirc}\isactrlsub c\ {\isacharparenleft}{\kern0pt}id\ X\ {\isasymtimes}\isactrlsub f\ {\isacharparenleft}{\kern0pt}g{\isasymcirc}\isactrlsub c\ w{\isacharparenright}{\kern0pt}{\isacharparenright}{\kern0pt}{\isacharcomma}{\kern0pt}\ f\ {\isasymcirc}\isactrlsub c\ right{\isacharunderscore}{\kern0pt}cart{\isacharunderscore}{\kern0pt}proj\ X\ W\ {\isasymcirc}\isactrlsub c\ {\isacharparenleft}{\kern0pt}id\ X\ {\isasymtimes}\isactrlsub f\ w{\isacharparenright}{\kern0pt}{\isasymrangle}{\isachardoublequoteclose}\isanewline
\ \ \ \ \ \ \ \ \ \isacommand{using}\isamarkupfalse%
\ assms\ \isacommand{by}\isamarkupfalse%
\ {\isacharparenleft}{\kern0pt}typecheck{\isacharunderscore}{\kern0pt}cfuncs{\isacharcomma}{\kern0pt}\ metis\ identity{\isacharunderscore}{\kern0pt}distributes{\isacharunderscore}{\kern0pt}across{\isacharunderscore}{\kern0pt}composition{\isacharparenright}{\kern0pt}\isanewline
\ \ \ \ \ \ \ \isacommand{then}\isamarkupfalse%
\ \isacommand{show}\isamarkupfalse%
\ {\isacharquery}{\kern0pt}thesis\isanewline
\ \ \ \ \ \ \ \ \ \isacommand{using}\isamarkupfalse%
\ assms\ calculation\ comp{\isacharunderscore}{\kern0pt}associative{\isadigit{2}}\ flat{\isacharunderscore}{\kern0pt}cancels{\isacharunderscore}{\kern0pt}sharp\ \isacommand{by}\isamarkupfalse%
\ {\isacharparenleft}{\kern0pt}typecheck{\isacharunderscore}{\kern0pt}cfuncs{\isacharcomma}{\kern0pt}\ auto{\isacharparenright}{\kern0pt}\isanewline
\ \ \ \ \ \isacommand{qed}\isamarkupfalse%
\isanewline
\ \ \ \ \ \isacommand{then}\isamarkupfalse%
\ \isacommand{show}\isamarkupfalse%
\ {\isacharquery}{\kern0pt}thesis\isanewline
\ \ \ \ \ \ \ \isacommand{using}\isamarkupfalse%
\ assms\ \isacommand{by}\isamarkupfalse%
\ {\isacharparenleft}{\kern0pt}typecheck{\isacharunderscore}{\kern0pt}cfuncs{\isacharcomma}{\kern0pt}\ smt\ {\isacharparenleft}{\kern0pt}z{\isadigit{3}}{\isacharparenright}{\kern0pt}\ comp{\isacharunderscore}{\kern0pt}associative{\isadigit{2}}\ inv{\isacharunderscore}{\kern0pt}transpose{\isacharunderscore}{\kern0pt}func{\isacharunderscore}{\kern0pt}def{\isadigit{3}}\ \isanewline
\ \ \ \ \ \ \ inv{\isacharunderscore}{\kern0pt}transpose{\isacharunderscore}{\kern0pt}of{\isacharunderscore}{\kern0pt}composition\ right{\isacharunderscore}{\kern0pt}cart{\isacharunderscore}{\kern0pt}proj{\isacharunderscore}{\kern0pt}cfunc{\isacharunderscore}{\kern0pt}cross{\isacharunderscore}{\kern0pt}prod\ transpose{\isacharunderscore}{\kern0pt}func{\isacharunderscore}{\kern0pt}unique{\isacharparenright}{\kern0pt}\isanewline
\ \ \isacommand{qed}\isamarkupfalse%
\isanewline
\ \ \isacommand{also}\isamarkupfalse%
\ \isacommand{have}\isamarkupfalse%
\ {\isachardoublequoteopen}{\isachardot}{\kern0pt}{\isachardot}{\kern0pt}{\isachardot}{\kern0pt}\ {\isacharequal}{\kern0pt}\ {\isacharparenleft}{\kern0pt}eval{\isacharunderscore}{\kern0pt}func\ Z\ Y\ {\isasymcirc}\isactrlsub c\ {\isacharparenleft}{\kern0pt}id\isactrlsub c\ Y\ {\isasymtimes}\isactrlsub f\ {\isacharparenleft}{\kern0pt}{\isacharparenleft}{\kern0pt}f\ {\isasymcirc}\isactrlsub c\ w{\isacharparenright}{\kern0pt}\ {\isasymcirc}\isactrlsub c\ right{\isacharunderscore}{\kern0pt}cart{\isacharunderscore}{\kern0pt}proj\ X\ {\isasymone}{\isacharparenright}{\kern0pt}{\isacharparenright}{\kern0pt}\ {\isasymcirc}\isactrlsub c\ {\isasymlangle}eval{\isacharunderscore}{\kern0pt}func\ Y\ X\ {\isasymcirc}\isactrlsub c\ {\isacharparenleft}{\kern0pt}id\ X\ {\isasymtimes}\isactrlsub f\ {\isacharparenleft}{\kern0pt}g{\isasymcirc}\isactrlsub c\ w{\isacharparenright}{\kern0pt}{\isacharparenright}{\kern0pt}{\isacharcomma}{\kern0pt}\ id\ {\isacharparenleft}{\kern0pt}X{\isasymtimes}\isactrlsub c\ {\isasymone}{\isacharparenright}{\kern0pt}{\isasymrangle}{\isacharparenright}{\kern0pt}\isactrlsup {\isasymsharp}{\isachardoublequoteclose}\isanewline
\ \ \ \ \isacommand{using}\isamarkupfalse%
\ assms\ \isacommand{by}\isamarkupfalse%
\ {\isacharparenleft}{\kern0pt}typecheck{\isacharunderscore}{\kern0pt}cfuncs{\isacharcomma}{\kern0pt}\ simp\ add{\isacharcolon}{\kern0pt}\ cfunc{\isacharunderscore}{\kern0pt}cross{\isacharunderscore}{\kern0pt}prod{\isacharunderscore}{\kern0pt}comp{\isacharunderscore}{\kern0pt}cfunc{\isacharunderscore}{\kern0pt}prod\ id{\isacharunderscore}{\kern0pt}left{\isacharunderscore}{\kern0pt}unit{\isadigit{2}}\ id{\isacharunderscore}{\kern0pt}right{\isacharunderscore}{\kern0pt}unit{\isadigit{2}}{\isacharparenright}{\kern0pt}\isanewline
\ \ \isacommand{also}\isamarkupfalse%
\ \isacommand{have}\isamarkupfalse%
\ {\isachardoublequoteopen}{\isachardot}{\kern0pt}{\isachardot}{\kern0pt}{\isachardot}{\kern0pt}\ {\isacharequal}{\kern0pt}\ {\isacharparenleft}{\kern0pt}eval{\isacharunderscore}{\kern0pt}func\ Z\ Y\ {\isasymcirc}\isactrlsub c\ {\isacharparenleft}{\kern0pt}id\isactrlsub c\ Y\ {\isasymtimes}\isactrlsub f\ {\isacharparenleft}{\kern0pt}f\ {\isasymcirc}\isactrlsub c\ w{\isacharparenright}{\kern0pt}{\isacharparenright}{\kern0pt}\ {\isasymcirc}\isactrlsub c\ {\isacharparenleft}{\kern0pt}id\ {\isacharparenleft}{\kern0pt}Y{\isacharparenright}{\kern0pt}\ {\isasymtimes}\isactrlsub f\ right{\isacharunderscore}{\kern0pt}cart{\isacharunderscore}{\kern0pt}proj\ X\ {\isasymone}{\isacharparenright}{\kern0pt}\ {\isasymcirc}\isactrlsub c\ {\isasymlangle}eval{\isacharunderscore}{\kern0pt}func\ Y\ X\ {\isasymcirc}\isactrlsub c\ {\isacharparenleft}{\kern0pt}id\ X\ {\isasymtimes}\isactrlsub f\ {\isacharparenleft}{\kern0pt}g{\isasymcirc}\isactrlsub c\ w{\isacharparenright}{\kern0pt}{\isacharparenright}{\kern0pt}{\isacharcomma}{\kern0pt}\ id\ {\isacharparenleft}{\kern0pt}X{\isasymtimes}\isactrlsub c\ {\isasymone}{\isacharparenright}{\kern0pt}{\isasymrangle}{\isacharparenright}{\kern0pt}\isactrlsup {\isasymsharp}{\isachardoublequoteclose}\isanewline
\ \ \ \ \isacommand{using}\isamarkupfalse%
\ assms\ comp{\isacharunderscore}{\kern0pt}associative{\isadigit{2}}\ identity{\isacharunderscore}{\kern0pt}distributes{\isacharunderscore}{\kern0pt}across{\isacharunderscore}{\kern0pt}composition\ \isacommand{by}\isamarkupfalse%
\ {\isacharparenleft}{\kern0pt}typecheck{\isacharunderscore}{\kern0pt}cfuncs{\isacharcomma}{\kern0pt}\ force{\isacharparenright}{\kern0pt}\isanewline
\ \ \isacommand{also}\isamarkupfalse%
\ \isacommand{have}\isamarkupfalse%
\ {\isachardoublequoteopen}{\isachardot}{\kern0pt}{\isachardot}{\kern0pt}{\isachardot}{\kern0pt}\ {\isacharequal}{\kern0pt}\ {\isacharparenleft}{\kern0pt}{\isacharparenleft}{\kern0pt}f{\isasymcirc}\isactrlsub cw{\isacharparenright}{\kern0pt}\isactrlsup {\isasymflat}\ {\isasymcirc}\isactrlsub c\ {\isacharparenleft}{\kern0pt}id\ {\isacharparenleft}{\kern0pt}Y{\isacharparenright}{\kern0pt}\ {\isasymtimes}\isactrlsub f\ right{\isacharunderscore}{\kern0pt}cart{\isacharunderscore}{\kern0pt}proj\ X\ {\isasymone}{\isacharparenright}{\kern0pt}\ {\isasymcirc}\isactrlsub c\ {\isasymlangle}eval{\isacharunderscore}{\kern0pt}func\ Y\ X\ {\isasymcirc}\isactrlsub c\ {\isacharparenleft}{\kern0pt}id\ X\ {\isasymtimes}\isactrlsub f\ {\isacharparenleft}{\kern0pt}g{\isasymcirc}\isactrlsub c\ w{\isacharparenright}{\kern0pt}{\isacharparenright}{\kern0pt}{\isacharcomma}{\kern0pt}\ id\ {\isacharparenleft}{\kern0pt}X{\isasymtimes}\isactrlsub c\ {\isasymone}{\isacharparenright}{\kern0pt}{\isasymrangle}{\isacharparenright}{\kern0pt}\isactrlsup {\isasymsharp}{\isachardoublequoteclose}\isanewline
\ \ \ \ \isacommand{using}\isamarkupfalse%
\ assms\ \isacommand{by}\isamarkupfalse%
\ {\isacharparenleft}{\kern0pt}typecheck{\isacharunderscore}{\kern0pt}cfuncs{\isacharcomma}{\kern0pt}\ smt\ {\isacharparenleft}{\kern0pt}z{\isadigit{3}}{\isacharparenright}{\kern0pt}\ comp{\isacharunderscore}{\kern0pt}associative{\isadigit{2}}\ inv{\isacharunderscore}{\kern0pt}transpose{\isacharunderscore}{\kern0pt}func{\isacharunderscore}{\kern0pt}def{\isadigit{3}}{\isacharparenright}{\kern0pt}\isanewline
\ \ \isacommand{also}\isamarkupfalse%
\ \isacommand{have}\isamarkupfalse%
\ {\isachardoublequoteopen}{\isachardot}{\kern0pt}{\isachardot}{\kern0pt}{\isachardot}{\kern0pt}\ {\isacharequal}{\kern0pt}\ {\isacharparenleft}{\kern0pt}{\isacharparenleft}{\kern0pt}f{\isasymcirc}\isactrlsub cw{\isacharparenright}{\kern0pt}\isactrlsup {\isasymflat}\ {\isasymcirc}\isactrlsub c\ {\isacharparenleft}{\kern0pt}id\ {\isacharparenleft}{\kern0pt}Y{\isacharparenright}{\kern0pt}\ {\isasymtimes}\isactrlsub f\ right{\isacharunderscore}{\kern0pt}cart{\isacharunderscore}{\kern0pt}proj\ X\ {\isasymone}{\isacharparenright}{\kern0pt}\ {\isasymcirc}\isactrlsub c\ {\isasymlangle}{\isacharparenleft}{\kern0pt}g{\isasymcirc}\isactrlsub c\ w{\isacharparenright}{\kern0pt}\isactrlsup {\isasymflat}{\isacharcomma}{\kern0pt}\ id\ {\isacharparenleft}{\kern0pt}X{\isasymtimes}\isactrlsub c\ {\isasymone}{\isacharparenright}{\kern0pt}{\isasymrangle}{\isacharparenright}{\kern0pt}\isactrlsup {\isasymsharp}{\isachardoublequoteclose}\isanewline
\ \ \ \ \isacommand{using}\isamarkupfalse%
\ assms\ inv{\isacharunderscore}{\kern0pt}transpose{\isacharunderscore}{\kern0pt}func{\isacharunderscore}{\kern0pt}def{\isadigit{3}}\ \isacommand{by}\isamarkupfalse%
\ {\isacharparenleft}{\kern0pt}typecheck{\isacharunderscore}{\kern0pt}cfuncs{\isacharcomma}{\kern0pt}\ force{\isacharparenright}{\kern0pt}\isanewline
\ \ \isacommand{also}\isamarkupfalse%
\ \isacommand{have}\isamarkupfalse%
\ {\isachardoublequoteopen}{\isachardot}{\kern0pt}{\isachardot}{\kern0pt}{\isachardot}{\kern0pt}\ {\isacharequal}{\kern0pt}\ {\isacharparenleft}{\kern0pt}{\isacharparenleft}{\kern0pt}f{\isasymcirc}\isactrlsub c\ w{\isacharparenright}{\kern0pt}\isactrlsup {\isasymflat}\ {\isasymcirc}\isactrlsub c\ {\isasymlangle}{\isacharparenleft}{\kern0pt}g{\isasymcirc}\isactrlsub c\ w{\isacharparenright}{\kern0pt}\isactrlsup {\isasymflat}{\isacharcomma}{\kern0pt}\ right{\isacharunderscore}{\kern0pt}cart{\isacharunderscore}{\kern0pt}proj\ X\ {\isasymone}{\isasymrangle}{\isacharparenright}{\kern0pt}\isactrlsup {\isasymsharp}{\isachardoublequoteclose}\isanewline
\ \ \ \ \isacommand{using}\isamarkupfalse%
\ assms\ \isacommand{by}\isamarkupfalse%
\ {\isacharparenleft}{\kern0pt}typecheck{\isacharunderscore}{\kern0pt}cfuncs{\isacharcomma}{\kern0pt}\ simp\ add{\isacharcolon}{\kern0pt}\ cfunc{\isacharunderscore}{\kern0pt}cross{\isacharunderscore}{\kern0pt}prod{\isacharunderscore}{\kern0pt}comp{\isacharunderscore}{\kern0pt}cfunc{\isacharunderscore}{\kern0pt}prod\ id{\isacharunderscore}{\kern0pt}left{\isacharunderscore}{\kern0pt}unit{\isadigit{2}}\ id{\isacharunderscore}{\kern0pt}right{\isacharunderscore}{\kern0pt}unit{\isadigit{2}}{\isacharparenright}{\kern0pt}\isanewline
\ \ \isacommand{also}\isamarkupfalse%
\ \isacommand{have}\isamarkupfalse%
\ {\isachardoublequoteopen}{\isachardot}{\kern0pt}{\isachardot}{\kern0pt}{\isachardot}{\kern0pt}\ {\isacharequal}{\kern0pt}\ {\isacharparenleft}{\kern0pt}f{\isasymcirc}\isactrlsub c\ w{\isacharparenright}{\kern0pt}\ {\isasymbox}\ {\isacharparenleft}{\kern0pt}g\ {\isasymcirc}\isactrlsub c\ w{\isacharparenright}{\kern0pt}{\isachardoublequoteclose}\isanewline
\ \ \ \ \isacommand{using}\isamarkupfalse%
\ assms\ \isacommand{by}\isamarkupfalse%
\ {\isacharparenleft}{\kern0pt}typecheck{\isacharunderscore}{\kern0pt}cfuncs{\isacharcomma}{\kern0pt}\ simp\ add{\isacharcolon}{\kern0pt}\ meta{\isacharunderscore}{\kern0pt}comp{\isadigit{2}}{\isacharunderscore}{\kern0pt}def{\isadigit{2}}{\isacharparenright}{\kern0pt}\isanewline
\ \ \isacommand{then}\isamarkupfalse%
\ \isacommand{show}\isamarkupfalse%
\ {\isacharquery}{\kern0pt}thesis\isanewline
\ \ \ \ \isacommand{by}\isamarkupfalse%
\ {\isacharparenleft}{\kern0pt}simp\ add{\isacharcolon}{\kern0pt}\ calculation{\isacharparenright}{\kern0pt}\isanewline
\isacommand{qed}\isamarkupfalse%
%
\endisatagproof
{\isafoldproof}%
%
\isadelimproof
\isanewline
%
\endisadelimproof
\isanewline
\isacommand{lemma}\isamarkupfalse%
\ meta{\isacharunderscore}{\kern0pt}comp{\isadigit{2}}{\isacharunderscore}{\kern0pt}def{\isadigit{5}}{\isacharcolon}{\kern0pt}\isanewline
\ \ \isakeyword{assumes}\ {\isachardoublequoteopen}f\ {\isacharcolon}{\kern0pt}\ W\ {\isasymrightarrow}\ Z\isactrlbsup Y\isactrlesup {\isachardoublequoteclose}\isanewline
\ \ \isakeyword{assumes}\ {\isachardoublequoteopen}g\ {\isacharcolon}{\kern0pt}\ W\ {\isasymrightarrow}\ Y\isactrlbsup X\isactrlesup {\isachardoublequoteclose}\isanewline
\ \ \isakeyword{shows}\ {\isachardoublequoteopen}f\ {\isasymbox}\ g\ \ \ {\isacharequal}{\kern0pt}\ meta{\isacharunderscore}{\kern0pt}comp\ X\ Y\ Z\ {\isasymcirc}\isactrlsub c\ {\isasymlangle}f{\isacharcomma}{\kern0pt}g{\isasymrangle}{\isachardoublequoteclose}\isanewline
%
\isadelimproof
%
\endisadelimproof
%
\isatagproof
\isacommand{proof}\isamarkupfalse%
{\isacharparenleft}{\kern0pt}rule\ one{\isacharunderscore}{\kern0pt}separator{\isacharbrackleft}{\kern0pt}\isakeyword{where}\ X\ {\isacharequal}{\kern0pt}\ W{\isacharcomma}{\kern0pt}\ \isakeyword{where}\ Y\ {\isacharequal}{\kern0pt}\ {\isachardoublequoteopen}Z\isactrlbsup X\isactrlesup {\isachardoublequoteclose}{\isacharbrackright}{\kern0pt}{\isacharparenright}{\kern0pt}\isanewline
\ \ \isacommand{show}\isamarkupfalse%
\ {\isachardoublequoteopen}f\ {\isasymbox}\ g\ {\isacharcolon}{\kern0pt}\ W\ {\isasymrightarrow}\ Z\isactrlbsup X\isactrlesup {\isachardoublequoteclose}\isanewline
\ \ \ \ \isacommand{using}\isamarkupfalse%
\ assms\ \isacommand{by}\isamarkupfalse%
\ typecheck{\isacharunderscore}{\kern0pt}cfuncs\isanewline
\ \ \isacommand{show}\isamarkupfalse%
\ {\isachardoublequoteopen}meta{\isacharunderscore}{\kern0pt}comp\ X\ Y\ Z\ {\isasymcirc}\isactrlsub c\ {\isasymlangle}f{\isacharcomma}{\kern0pt}g{\isasymrangle}\ {\isacharcolon}{\kern0pt}\ W\ {\isasymrightarrow}\ Z\isactrlbsup X\isactrlesup {\isachardoublequoteclose}\isanewline
\ \ \ \ \isacommand{using}\isamarkupfalse%
\ assms\ \isacommand{by}\isamarkupfalse%
\ typecheck{\isacharunderscore}{\kern0pt}cfuncs\isanewline
\isacommand{next}\isamarkupfalse%
\isanewline
\ \ \isacommand{fix}\isamarkupfalse%
\ w\ \isanewline
\ \ \isacommand{assume}\isamarkupfalse%
\ w{\isacharunderscore}{\kern0pt}type{\isacharbrackleft}{\kern0pt}type{\isacharunderscore}{\kern0pt}rule{\isacharbrackright}{\kern0pt}{\isacharcolon}{\kern0pt}\ {\isachardoublequoteopen}w\ {\isasymin}\isactrlsub c\ W{\isachardoublequoteclose}\isanewline
\ \ \isacommand{have}\isamarkupfalse%
\ {\isachardoublequoteopen}{\isacharparenleft}{\kern0pt}meta{\isacharunderscore}{\kern0pt}comp\ X\ Y\ Z\ {\isasymcirc}\isactrlsub c\ {\isasymlangle}f{\isacharcomma}{\kern0pt}g{\isasymrangle}{\isacharparenright}{\kern0pt}\ {\isasymcirc}\isactrlsub c\ w\ {\isacharequal}{\kern0pt}\ meta{\isacharunderscore}{\kern0pt}comp\ X\ Y\ Z\ {\isasymcirc}\isactrlsub c\ {\isasymlangle}f{\isacharcomma}{\kern0pt}g{\isasymrangle}\ {\isasymcirc}\isactrlsub c\ w{\isachardoublequoteclose}\isanewline
\ \ \ \ \isacommand{using}\isamarkupfalse%
\ assms\ \isacommand{by}\isamarkupfalse%
\ {\isacharparenleft}{\kern0pt}typecheck{\isacharunderscore}{\kern0pt}cfuncs{\isacharcomma}{\kern0pt}\ simp\ add{\isacharcolon}{\kern0pt}\ comp{\isacharunderscore}{\kern0pt}associative{\isadigit{2}}{\isacharparenright}{\kern0pt}\isanewline
\ \ \isacommand{also}\isamarkupfalse%
\ \isacommand{have}\isamarkupfalse%
\ {\isachardoublequoteopen}{\isachardot}{\kern0pt}{\isachardot}{\kern0pt}{\isachardot}{\kern0pt}\ {\isacharequal}{\kern0pt}\ meta{\isacharunderscore}{\kern0pt}comp\ X\ Y\ Z\ {\isasymcirc}\isactrlsub c\ {\isasymlangle}f\ {\isasymcirc}\isactrlsub c\ w{\isacharcomma}{\kern0pt}\ g\ {\isasymcirc}\isactrlsub c\ w{\isasymrangle}{\isachardoublequoteclose}\isanewline
\ \ \ \ \isacommand{using}\isamarkupfalse%
\ assms\ \isacommand{by}\isamarkupfalse%
\ {\isacharparenleft}{\kern0pt}typecheck{\isacharunderscore}{\kern0pt}cfuncs{\isacharcomma}{\kern0pt}\ simp\ add{\isacharcolon}{\kern0pt}\ cfunc{\isacharunderscore}{\kern0pt}prod{\isacharunderscore}{\kern0pt}comp{\isacharparenright}{\kern0pt}\isanewline
\ \ \isacommand{also}\isamarkupfalse%
\ \isacommand{have}\isamarkupfalse%
\ {\isachardoublequoteopen}{\isachardot}{\kern0pt}{\isachardot}{\kern0pt}{\isachardot}{\kern0pt}\ {\isacharequal}{\kern0pt}\ {\isacharparenleft}{\kern0pt}f{\isasymcirc}\isactrlsub c\ w{\isacharparenright}{\kern0pt}\ {\isasymbox}\ {\isacharparenleft}{\kern0pt}g\ {\isasymcirc}\isactrlsub c\ w{\isacharparenright}{\kern0pt}{\isachardoublequoteclose}\isanewline
\ \ \ \ \isacommand{using}\isamarkupfalse%
\ assms\ \isacommand{by}\isamarkupfalse%
\ {\isacharparenleft}{\kern0pt}typecheck{\isacharunderscore}{\kern0pt}cfuncs{\isacharcomma}{\kern0pt}\ simp\ add{\isacharcolon}{\kern0pt}\ meta{\isacharunderscore}{\kern0pt}comp{\isadigit{2}}{\isacharunderscore}{\kern0pt}def{\isadigit{4}}{\isacharparenright}{\kern0pt}\isanewline
\ \ \isacommand{also}\isamarkupfalse%
\ \isacommand{have}\isamarkupfalse%
\ {\isachardoublequoteopen}{\isachardot}{\kern0pt}{\isachardot}{\kern0pt}{\isachardot}{\kern0pt}\ {\isacharequal}{\kern0pt}\ {\isacharparenleft}{\kern0pt}f\ {\isasymbox}\ g{\isacharparenright}{\kern0pt}\ {\isasymcirc}\isactrlsub c\ w{\isachardoublequoteclose}\isanewline
\ \ \ \ \isacommand{using}\isamarkupfalse%
\ assms\ \isacommand{by}\isamarkupfalse%
\ {\isacharparenleft}{\kern0pt}typecheck{\isacharunderscore}{\kern0pt}cfuncs{\isacharcomma}{\kern0pt}\ simp\ add{\isacharcolon}{\kern0pt}\ meta{\isacharunderscore}{\kern0pt}comp{\isacharunderscore}{\kern0pt}on{\isacharunderscore}{\kern0pt}els{\isacharparenright}{\kern0pt}\isanewline
\ \ \isacommand{then}\isamarkupfalse%
\ \isacommand{show}\isamarkupfalse%
\ {\isachardoublequoteopen}{\isacharparenleft}{\kern0pt}f\ {\isasymbox}\ g{\isacharparenright}{\kern0pt}\ {\isasymcirc}\isactrlsub c\ w\ {\isacharequal}{\kern0pt}\ {\isacharparenleft}{\kern0pt}meta{\isacharunderscore}{\kern0pt}comp\ X\ Y\ Z\ {\isasymcirc}\isactrlsub c\ {\isasymlangle}f{\isacharcomma}{\kern0pt}g{\isasymrangle}{\isacharparenright}{\kern0pt}\ {\isasymcirc}\isactrlsub c\ w{\isachardoublequoteclose}\isanewline
\ \ \ \ \isacommand{by}\isamarkupfalse%
\ {\isacharparenleft}{\kern0pt}simp\ add{\isacharcolon}{\kern0pt}\ calculation{\isacharparenright}{\kern0pt}\isanewline
\isacommand{qed}\isamarkupfalse%
%
\endisatagproof
{\isafoldproof}%
%
\isadelimproof
\isanewline
%
\endisadelimproof
\isanewline
\isacommand{lemma}\isamarkupfalse%
\ meta{\isacharunderscore}{\kern0pt}left{\isacharunderscore}{\kern0pt}identity{\isacharcolon}{\kern0pt}\isanewline
\ \ \isakeyword{assumes}\ {\isachardoublequoteopen}g\ {\isasymin}\isactrlsub c\ X\isactrlbsup X\isactrlesup {\isachardoublequoteclose}\isanewline
\ \ \isakeyword{shows}\ {\isachardoublequoteopen}g\ {\isasymbox}\ metafunc\ {\isacharparenleft}{\kern0pt}id\ X{\isacharparenright}{\kern0pt}\ {\isacharequal}{\kern0pt}\ g{\isachardoublequoteclose}\isanewline
%
\isadelimproof
\ \ %
\endisadelimproof
%
\isatagproof
\isacommand{using}\isamarkupfalse%
\ assms\ \isacommand{by}\isamarkupfalse%
\ {\isacharparenleft}{\kern0pt}typecheck{\isacharunderscore}{\kern0pt}cfuncs{\isacharcomma}{\kern0pt}\ metis\ cfunc{\isacharunderscore}{\kern0pt}type{\isacharunderscore}{\kern0pt}def\ cnufatem{\isacharunderscore}{\kern0pt}metafunc\ cnufatem{\isacharunderscore}{\kern0pt}type\ id{\isacharunderscore}{\kern0pt}right{\isacharunderscore}{\kern0pt}unit\ meta{\isacharunderscore}{\kern0pt}comp{\isadigit{2}}{\isacharunderscore}{\kern0pt}def{\isadigit{3}}\ metafunc{\isacharunderscore}{\kern0pt}cnufatem{\isacharparenright}{\kern0pt}%
\endisatagproof
{\isafoldproof}%
%
\isadelimproof
\isanewline
%
\endisadelimproof
\ \ \isanewline
\isacommand{lemma}\isamarkupfalse%
\ meta{\isacharunderscore}{\kern0pt}right{\isacharunderscore}{\kern0pt}identity{\isacharcolon}{\kern0pt}\isanewline
\ \ \isakeyword{assumes}\ {\isachardoublequoteopen}g\ {\isasymin}\isactrlsub c\ X\isactrlbsup X\isactrlesup {\isachardoublequoteclose}\isanewline
\ \ \isakeyword{shows}\ {\isachardoublequoteopen}metafunc{\isacharparenleft}{\kern0pt}id\ X{\isacharparenright}{\kern0pt}\ {\isasymbox}\ g\ {\isacharequal}{\kern0pt}\ g{\isachardoublequoteclose}\isanewline
%
\isadelimproof
\ \ %
\endisadelimproof
%
\isatagproof
\isacommand{using}\isamarkupfalse%
\ assms\ \isacommand{by}\isamarkupfalse%
\ {\isacharparenleft}{\kern0pt}typecheck{\isacharunderscore}{\kern0pt}cfuncs{\isacharcomma}{\kern0pt}\ smt\ {\isacharparenleft}{\kern0pt}z{\isadigit{3}}{\isacharparenright}{\kern0pt}\ cnufatem{\isacharunderscore}{\kern0pt}metafunc\ cnufatem{\isacharunderscore}{\kern0pt}type\ id{\isacharunderscore}{\kern0pt}left{\isacharunderscore}{\kern0pt}unit{\isadigit{2}}\ meta{\isacharunderscore}{\kern0pt}comp{\isadigit{2}}{\isacharunderscore}{\kern0pt}def{\isadigit{3}}\ metafunc{\isacharunderscore}{\kern0pt}cnufatem{\isacharparenright}{\kern0pt}%
\endisatagproof
{\isafoldproof}%
%
\isadelimproof
\isanewline
%
\endisadelimproof
\isanewline
\isacommand{lemma}\isamarkupfalse%
\ comp{\isacharunderscore}{\kern0pt}as{\isacharunderscore}{\kern0pt}metacomp{\isacharcolon}{\kern0pt}\isanewline
\ \ \isakeyword{assumes}\ {\isachardoublequoteopen}g\ {\isacharcolon}{\kern0pt}\ X\ {\isasymrightarrow}\ Y{\isachardoublequoteclose}\isanewline
\ \ \isakeyword{assumes}\ {\isachardoublequoteopen}f\ {\isacharcolon}{\kern0pt}\ Y\ {\isasymrightarrow}\ Z{\isachardoublequoteclose}\isanewline
\ \ \isakeyword{shows}\ {\isachardoublequoteopen}f\ {\isasymcirc}\isactrlsub c\ g\ {\isacharequal}{\kern0pt}\ cnufatem{\isacharparenleft}{\kern0pt}metafunc\ f\ {\isasymbox}\ metafunc\ g{\isacharparenright}{\kern0pt}{\isachardoublequoteclose}\isanewline
%
\isadelimproof
\ \ %
\endisadelimproof
%
\isatagproof
\isacommand{using}\isamarkupfalse%
\ assms\ \isacommand{by}\isamarkupfalse%
\ {\isacharparenleft}{\kern0pt}typecheck{\isacharunderscore}{\kern0pt}cfuncs{\isacharcomma}{\kern0pt}\ simp\ add{\isacharcolon}{\kern0pt}\ cnufatem{\isacharunderscore}{\kern0pt}metafunc\ meta{\isacharunderscore}{\kern0pt}comp{\isadigit{2}}{\isacharunderscore}{\kern0pt}def{\isadigit{3}}{\isacharparenright}{\kern0pt}%
\endisatagproof
{\isafoldproof}%
%
\isadelimproof
\isanewline
%
\endisadelimproof
\isanewline
\isacommand{lemma}\isamarkupfalse%
\ metacomp{\isacharunderscore}{\kern0pt}as{\isacharunderscore}{\kern0pt}comp{\isacharcolon}{\kern0pt}\isanewline
\ \ \isakeyword{assumes}\ {\isachardoublequoteopen}g\ {\isasymin}\isactrlsub c\ Y\isactrlbsup X\isactrlesup {\isachardoublequoteclose}\isanewline
\ \ \isakeyword{assumes}\ {\isachardoublequoteopen}f\ {\isasymin}\isactrlsub c\ Z\isactrlbsup Y\isactrlesup {\isachardoublequoteclose}\isanewline
\ \ \isakeyword{shows}\ {\isachardoublequoteopen}cnufatem\ f\ {\isasymcirc}\isactrlsub c\ cnufatem\ g\ {\isacharequal}{\kern0pt}\ cnufatem{\isacharparenleft}{\kern0pt}f\ {\isasymbox}\ g{\isacharparenright}{\kern0pt}{\isachardoublequoteclose}\isanewline
%
\isadelimproof
\ \ %
\endisadelimproof
%
\isatagproof
\isacommand{using}\isamarkupfalse%
\ assms\ \isacommand{by}\isamarkupfalse%
\ {\isacharparenleft}{\kern0pt}typecheck{\isacharunderscore}{\kern0pt}cfuncs{\isacharcomma}{\kern0pt}\ simp\ add{\isacharcolon}{\kern0pt}\ comp{\isacharunderscore}{\kern0pt}as{\isacharunderscore}{\kern0pt}metacomp\ metafunc{\isacharunderscore}{\kern0pt}cnufatem{\isacharparenright}{\kern0pt}%
\endisatagproof
{\isafoldproof}%
%
\isadelimproof
\isanewline
%
\endisadelimproof
\isanewline
\isacommand{lemma}\isamarkupfalse%
\ meta{\isacharunderscore}{\kern0pt}comp{\isacharunderscore}{\kern0pt}assoc{\isacharcolon}{\kern0pt}\isanewline
\ \ \isakeyword{assumes}\ {\isachardoublequoteopen}e\ {\isacharcolon}{\kern0pt}\ W\ {\isasymrightarrow}\ A\isactrlbsup Z\isactrlesup {\isachardoublequoteclose}\isanewline
\ \ \isakeyword{assumes}\ {\isachardoublequoteopen}f\ {\isacharcolon}{\kern0pt}\ W\ {\isasymrightarrow}\ Z\isactrlbsup Y\isactrlesup {\isachardoublequoteclose}\isanewline
\ \ \isakeyword{assumes}\ {\isachardoublequoteopen}g\ {\isacharcolon}{\kern0pt}\ W\ {\isasymrightarrow}\ Y\isactrlbsup X\isactrlesup {\isachardoublequoteclose}\isanewline
\ \ \isakeyword{shows}\ {\isachardoublequoteopen}{\isacharparenleft}{\kern0pt}e\ {\isasymbox}\ f{\isacharparenright}{\kern0pt}\ {\isasymbox}\ \ g\ \ {\isacharequal}{\kern0pt}\ e\ {\isasymbox}\ {\isacharparenleft}{\kern0pt}f\ {\isasymbox}\ g{\isacharparenright}{\kern0pt}{\isachardoublequoteclose}\isanewline
%
\isadelimproof
%
\endisadelimproof
%
\isatagproof
\isacommand{proof}\isamarkupfalse%
\ {\isacharminus}{\kern0pt}\isanewline
\ \ \isacommand{have}\isamarkupfalse%
\ {\isachardoublequoteopen}{\isacharparenleft}{\kern0pt}e\ {\isasymbox}\ f{\isacharparenright}{\kern0pt}\ {\isasymbox}\ \ g\ {\isacharequal}{\kern0pt}\ {\isacharparenleft}{\kern0pt}e\isactrlsup {\isasymflat}\ {\isasymcirc}\isactrlsub c\ {\isasymlangle}f\isactrlsup {\isasymflat}{\isacharcomma}{\kern0pt}\ right{\isacharunderscore}{\kern0pt}cart{\isacharunderscore}{\kern0pt}proj\ Y\ W{\isasymrangle}{\isacharparenright}{\kern0pt}\isactrlsup {\isasymsharp}\ {\isasymbox}\ g{\isachardoublequoteclose}\isanewline
\ \ \ \ \isacommand{using}\isamarkupfalse%
\ assms\ \isacommand{by}\isamarkupfalse%
\ {\isacharparenleft}{\kern0pt}simp\ add{\isacharcolon}{\kern0pt}\ meta{\isacharunderscore}{\kern0pt}comp{\isadigit{2}}{\isacharunderscore}{\kern0pt}def{\isadigit{2}}{\isacharparenright}{\kern0pt}\isanewline
\ \ \isacommand{also}\isamarkupfalse%
\ \isacommand{have}\isamarkupfalse%
\ {\isachardoublequoteopen}{\isachardot}{\kern0pt}{\isachardot}{\kern0pt}{\isachardot}{\kern0pt}\ {\isacharequal}{\kern0pt}\ {\isacharparenleft}{\kern0pt}{\isacharparenleft}{\kern0pt}e\isactrlsup {\isasymflat}\ {\isasymcirc}\isactrlsub c\ {\isasymlangle}f\isactrlsup {\isasymflat}{\isacharcomma}{\kern0pt}\ right{\isacharunderscore}{\kern0pt}cart{\isacharunderscore}{\kern0pt}proj\ Y\ W{\isasymrangle}{\isacharparenright}{\kern0pt}\isactrlsup {\isasymsharp}\isactrlsup {\isasymflat}\ {\isasymcirc}\isactrlsub c\ {\isasymlangle}g\isactrlsup {\isasymflat}{\isacharcomma}{\kern0pt}\ right{\isacharunderscore}{\kern0pt}cart{\isacharunderscore}{\kern0pt}proj\ X\ W{\isasymrangle}{\isacharparenright}{\kern0pt}\isactrlsup {\isasymsharp}{\isachardoublequoteclose}\isanewline
\ \ \ \ \isacommand{using}\isamarkupfalse%
\ assms\ \isacommand{by}\isamarkupfalse%
\ {\isacharparenleft}{\kern0pt}typecheck{\isacharunderscore}{\kern0pt}cfuncs{\isacharcomma}{\kern0pt}\ simp\ add{\isacharcolon}{\kern0pt}\ meta{\isacharunderscore}{\kern0pt}comp{\isadigit{2}}{\isacharunderscore}{\kern0pt}def{\isadigit{2}}{\isacharparenright}{\kern0pt}\isanewline
\ \ \isacommand{also}\isamarkupfalse%
\ \isacommand{have}\isamarkupfalse%
\ {\isachardoublequoteopen}{\isachardot}{\kern0pt}{\isachardot}{\kern0pt}{\isachardot}{\kern0pt}\ {\isacharequal}{\kern0pt}\ {\isacharparenleft}{\kern0pt}{\isacharparenleft}{\kern0pt}e\isactrlsup {\isasymflat}\ {\isasymcirc}\isactrlsub c\ {\isasymlangle}f\isactrlsup {\isasymflat}{\isacharcomma}{\kern0pt}\ right{\isacharunderscore}{\kern0pt}cart{\isacharunderscore}{\kern0pt}proj\ Y\ W{\isasymrangle}{\isacharparenright}{\kern0pt}\ {\isasymcirc}\isactrlsub c\ {\isasymlangle}g\isactrlsup {\isasymflat}{\isacharcomma}{\kern0pt}\ right{\isacharunderscore}{\kern0pt}cart{\isacharunderscore}{\kern0pt}proj\ X\ W{\isasymrangle}{\isacharparenright}{\kern0pt}\isactrlsup {\isasymsharp}{\isachardoublequoteclose}\isanewline
\ \ \ \ \isacommand{using}\isamarkupfalse%
\ assms\ \isacommand{by}\isamarkupfalse%
\ {\isacharparenleft}{\kern0pt}typecheck{\isacharunderscore}{\kern0pt}cfuncs{\isacharcomma}{\kern0pt}\ simp\ add{\isacharcolon}{\kern0pt}\ flat{\isacharunderscore}{\kern0pt}cancels{\isacharunderscore}{\kern0pt}sharp{\isacharparenright}{\kern0pt}\ \ \ \ \isanewline
\ \ \isacommand{also}\isamarkupfalse%
\ \isacommand{have}\isamarkupfalse%
\ {\isachardoublequoteopen}{\isachardot}{\kern0pt}{\isachardot}{\kern0pt}{\isachardot}{\kern0pt}\ {\isacharequal}{\kern0pt}\ {\isacharparenleft}{\kern0pt}e\isactrlsup {\isasymflat}\ {\isasymcirc}\isactrlsub c\ {\isasymlangle}f\isactrlsup {\isasymflat}\ {\isasymcirc}\isactrlsub c\ {\isasymlangle}g\isactrlsup {\isasymflat}{\isacharcomma}{\kern0pt}\ right{\isacharunderscore}{\kern0pt}cart{\isacharunderscore}{\kern0pt}proj\ X\ W{\isasymrangle}\ {\isacharcomma}{\kern0pt}right{\isacharunderscore}{\kern0pt}cart{\isacharunderscore}{\kern0pt}proj\ X\ W{\isasymrangle}{\isacharparenright}{\kern0pt}\isactrlsup {\isasymsharp}{\isachardoublequoteclose}\isanewline
\ \ \ \ \isacommand{using}\isamarkupfalse%
\ assms\ \isacommand{by}\isamarkupfalse%
\ {\isacharparenleft}{\kern0pt}typecheck{\isacharunderscore}{\kern0pt}cfuncs{\isacharcomma}{\kern0pt}\ smt\ {\isacharparenleft}{\kern0pt}z{\isadigit{3}}{\isacharparenright}{\kern0pt}\ cfunc{\isacharunderscore}{\kern0pt}prod{\isacharunderscore}{\kern0pt}comp\ comp{\isacharunderscore}{\kern0pt}associative{\isadigit{2}}\ right{\isacharunderscore}{\kern0pt}cart{\isacharunderscore}{\kern0pt}proj{\isacharunderscore}{\kern0pt}cfunc{\isacharunderscore}{\kern0pt}prod{\isacharparenright}{\kern0pt}\isanewline
\ \ \isacommand{also}\isamarkupfalse%
\ \isacommand{have}\isamarkupfalse%
\ {\isachardoublequoteopen}{\isachardot}{\kern0pt}{\isachardot}{\kern0pt}{\isachardot}{\kern0pt}\ {\isacharequal}{\kern0pt}\ {\isacharparenleft}{\kern0pt}e\isactrlsup {\isasymflat}\ {\isasymcirc}\isactrlsub c\ {\isasymlangle}{\isacharparenleft}{\kern0pt}f\isactrlsup {\isasymflat}\ {\isasymcirc}\isactrlsub c\ {\isasymlangle}g\isactrlsup {\isasymflat}{\isacharcomma}{\kern0pt}\ right{\isacharunderscore}{\kern0pt}cart{\isacharunderscore}{\kern0pt}proj\ X\ W{\isasymrangle}{\isacharparenright}{\kern0pt}\isactrlsup {\isasymsharp}\isactrlsup {\isasymflat}\ {\isacharcomma}{\kern0pt}right{\isacharunderscore}{\kern0pt}cart{\isacharunderscore}{\kern0pt}proj\ X\ W{\isasymrangle}{\isacharparenright}{\kern0pt}\isactrlsup {\isasymsharp}{\isachardoublequoteclose}\isanewline
\ \ \ \ \isacommand{using}\isamarkupfalse%
\ assms\ \isacommand{by}\isamarkupfalse%
\ {\isacharparenleft}{\kern0pt}typecheck{\isacharunderscore}{\kern0pt}cfuncs{\isacharcomma}{\kern0pt}\ simp\ add{\isacharcolon}{\kern0pt}\ flat{\isacharunderscore}{\kern0pt}cancels{\isacharunderscore}{\kern0pt}sharp{\isacharparenright}{\kern0pt}\isanewline
\ \ \isacommand{also}\isamarkupfalse%
\ \isacommand{have}\isamarkupfalse%
\ {\isachardoublequoteopen}{\isachardot}{\kern0pt}{\isachardot}{\kern0pt}{\isachardot}{\kern0pt}\ {\isacharequal}{\kern0pt}\ e\ {\isasymbox}\ {\isacharparenleft}{\kern0pt}f\isactrlsup {\isasymflat}\ {\isasymcirc}\isactrlsub c\ {\isasymlangle}g\isactrlsup {\isasymflat}{\isacharcomma}{\kern0pt}\ right{\isacharunderscore}{\kern0pt}cart{\isacharunderscore}{\kern0pt}proj\ X\ W{\isasymrangle}{\isacharparenright}{\kern0pt}\isactrlsup {\isasymsharp}{\isachardoublequoteclose}\isanewline
\ \ \ \ \isacommand{using}\isamarkupfalse%
\ assms\ \isacommand{by}\isamarkupfalse%
\ {\isacharparenleft}{\kern0pt}typecheck{\isacharunderscore}{\kern0pt}cfuncs{\isacharcomma}{\kern0pt}\ simp\ add{\isacharcolon}{\kern0pt}\ meta{\isacharunderscore}{\kern0pt}comp{\isadigit{2}}{\isacharunderscore}{\kern0pt}def{\isadigit{2}}{\isacharparenright}{\kern0pt}\isanewline
\ \ \isacommand{also}\isamarkupfalse%
\ \isacommand{have}\isamarkupfalse%
\ {\isachardoublequoteopen}{\isachardot}{\kern0pt}{\isachardot}{\kern0pt}{\isachardot}{\kern0pt}\ {\isacharequal}{\kern0pt}\ e\ {\isasymbox}\ {\isacharparenleft}{\kern0pt}f\ {\isasymbox}\ g{\isacharparenright}{\kern0pt}{\isachardoublequoteclose}\isanewline
\ \ \ \ \isacommand{using}\isamarkupfalse%
\ assms\ \isacommand{by}\isamarkupfalse%
\ {\isacharparenleft}{\kern0pt}simp\ add{\isacharcolon}{\kern0pt}\ meta{\isacharunderscore}{\kern0pt}comp{\isadigit{2}}{\isacharunderscore}{\kern0pt}def{\isadigit{2}}{\isacharparenright}{\kern0pt}\isanewline
\ \ \isacommand{then}\isamarkupfalse%
\ \isacommand{show}\isamarkupfalse%
\ {\isacharquery}{\kern0pt}thesis\isanewline
\ \ \ \ \isacommand{by}\isamarkupfalse%
\ {\isacharparenleft}{\kern0pt}simp\ add{\isacharcolon}{\kern0pt}\ calculation{\isacharparenright}{\kern0pt}\isanewline
\isacommand{qed}\isamarkupfalse%
%
\endisatagproof
{\isafoldproof}%
%
\isadelimproof
%
\endisadelimproof
%
\isadelimdocument
%
\endisadelimdocument
%
\isatagdocument
%
\isamarkupsubsection{Partially Parameterized Functions on Pairs%
}
\isamarkuptrue%
%
\endisatagdocument
{\isafolddocument}%
%
\isadelimdocument
%
\endisadelimdocument
\isacommand{definition}\isamarkupfalse%
\ left{\isacharunderscore}{\kern0pt}param\ {\isacharcolon}{\kern0pt}{\isacharcolon}{\kern0pt}\ {\isachardoublequoteopen}cfunc\ {\isasymRightarrow}\ cfunc\ {\isasymRightarrow}\ cfunc{\isachardoublequoteclose}\ {\isacharparenleft}{\kern0pt}{\isachardoublequoteopen}{\isacharunderscore}{\kern0pt}\isactrlbsub {\isacharbrackleft}{\kern0pt}{\isacharunderscore}{\kern0pt}{\isacharcomma}{\kern0pt}{\isacharminus}{\kern0pt}{\isacharbrackright}{\kern0pt}\isactrlesub {\isachardoublequoteclose}\ {\isacharbrackleft}{\kern0pt}{\isadigit{1}}{\isadigit{0}}{\isadigit{0}}{\isacharcomma}{\kern0pt}{\isadigit{0}}{\isacharbrackright}{\kern0pt}{\isadigit{1}}{\isadigit{0}}{\isadigit{0}}{\isacharparenright}{\kern0pt}\ \isakeyword{where}\isanewline
\ \ {\isachardoublequoteopen}left{\isacharunderscore}{\kern0pt}param\ k\ p\ {\isasymequiv}\ {\isacharparenleft}{\kern0pt}THE\ f{\isachardot}{\kern0pt}\ \ {\isasymexists}\ P\ Q\ R{\isachardot}{\kern0pt}\ k\ {\isacharcolon}{\kern0pt}\ P\ {\isasymtimes}\isactrlsub c\ Q\ {\isasymrightarrow}\ R\ {\isasymand}\ f\ {\isacharequal}{\kern0pt}\ k\ {\isasymcirc}\isactrlsub c\ {\isasymlangle}p\ {\isasymcirc}\isactrlsub c\ {\isasymbeta}\isactrlbsub Q\isactrlesub {\isacharcomma}{\kern0pt}\ id\ Q{\isasymrangle}{\isacharparenright}{\kern0pt}{\isachardoublequoteclose}\isanewline
\isanewline
\isacommand{lemma}\isamarkupfalse%
\ left{\isacharunderscore}{\kern0pt}param{\isacharunderscore}{\kern0pt}def{\isadigit{2}}{\isacharcolon}{\kern0pt}\isanewline
\ \ \isakeyword{assumes}\ {\isachardoublequoteopen}k\ {\isacharcolon}{\kern0pt}\ P\ {\isasymtimes}\isactrlsub c\ Q\ {\isasymrightarrow}\ R{\isachardoublequoteclose}\isanewline
\ \ \isakeyword{shows}\ {\isachardoublequoteopen}k\isactrlbsub {\isacharbrackleft}{\kern0pt}p{\isacharcomma}{\kern0pt}{\isacharminus}{\kern0pt}{\isacharbrackright}{\kern0pt}\isactrlesub \ {\isasymequiv}\ k\ {\isasymcirc}\isactrlsub c\ {\isasymlangle}p\ {\isasymcirc}\isactrlsub c\ {\isasymbeta}\isactrlbsub Q\isactrlesub {\isacharcomma}{\kern0pt}\ id\ Q{\isasymrangle}{\isachardoublequoteclose}\isanewline
%
\isadelimproof
%
\endisadelimproof
%
\isatagproof
\isacommand{proof}\isamarkupfalse%
\ {\isacharminus}{\kern0pt}\ \isanewline
\ \ \isacommand{have}\isamarkupfalse%
\ {\isachardoublequoteopen}{\isasymexists}\ P\ Q\ R{\isachardot}{\kern0pt}\ k\ {\isacharcolon}{\kern0pt}\ P\ {\isasymtimes}\isactrlsub c\ Q\ {\isasymrightarrow}\ R\ {\isasymand}\ left{\isacharunderscore}{\kern0pt}param\ k\ p\ {\isacharequal}{\kern0pt}\ k\ {\isasymcirc}\isactrlsub c\ {\isasymlangle}p\ {\isasymcirc}\isactrlsub c\ {\isasymbeta}\isactrlbsub Q\isactrlesub {\isacharcomma}{\kern0pt}\ id\ Q{\isasymrangle}{\isachardoublequoteclose}\isanewline
\ \ \ \ \isacommand{unfolding}\isamarkupfalse%
\ left{\isacharunderscore}{\kern0pt}param{\isacharunderscore}{\kern0pt}def\ \isacommand{by}\isamarkupfalse%
\ {\isacharparenleft}{\kern0pt}smt\ {\isacharparenleft}{\kern0pt}z{\isadigit{3}}{\isacharparenright}{\kern0pt}\ cfunc{\isacharunderscore}{\kern0pt}type{\isacharunderscore}{\kern0pt}def\ the{\isadigit{1}}I{\isadigit{2}}\ transpose{\isacharunderscore}{\kern0pt}func{\isacharunderscore}{\kern0pt}type\ assms{\isacharparenright}{\kern0pt}\ \isanewline
\ \ \isacommand{then}\isamarkupfalse%
\ \isacommand{show}\isamarkupfalse%
\ {\isachardoublequoteopen}k\isactrlbsub {\isacharbrackleft}{\kern0pt}p{\isacharcomma}{\kern0pt}{\isacharminus}{\kern0pt}{\isacharbrackright}{\kern0pt}\isactrlesub \ {\isasymequiv}\ k\ {\isasymcirc}\isactrlsub c\ {\isasymlangle}p\ {\isasymcirc}\isactrlsub c\ {\isasymbeta}\isactrlbsub Q\isactrlesub {\isacharcomma}{\kern0pt}\ id\ Q{\isasymrangle}{\isachardoublequoteclose}\isanewline
\ \ \ \ \isacommand{by}\isamarkupfalse%
\ {\isacharparenleft}{\kern0pt}smt\ {\isacharparenleft}{\kern0pt}z{\isadigit{3}}{\isacharparenright}{\kern0pt}\ assms\ cfunc{\isacharunderscore}{\kern0pt}type{\isacharunderscore}{\kern0pt}def\ transpose{\isacharunderscore}{\kern0pt}func{\isacharunderscore}{\kern0pt}type{\isacharparenright}{\kern0pt}\isanewline
\isacommand{qed}\isamarkupfalse%
%
\endisatagproof
{\isafoldproof}%
%
\isadelimproof
\isanewline
%
\endisadelimproof
\isanewline
\isacommand{lemma}\isamarkupfalse%
\ left{\isacharunderscore}{\kern0pt}param{\isacharunderscore}{\kern0pt}type{\isacharbrackleft}{\kern0pt}type{\isacharunderscore}{\kern0pt}rule{\isacharbrackright}{\kern0pt}{\isacharcolon}{\kern0pt}\isanewline
\ \ \isakeyword{assumes}\ {\isachardoublequoteopen}k\ {\isacharcolon}{\kern0pt}\ P\ {\isasymtimes}\isactrlsub c\ Q\ {\isasymrightarrow}\ R{\isachardoublequoteclose}\isanewline
\ \ \isakeyword{assumes}\ {\isachardoublequoteopen}p\ {\isasymin}\isactrlsub c\ P{\isachardoublequoteclose}\isanewline
\ \ \isakeyword{shows}\ {\isachardoublequoteopen}k\isactrlbsub {\isacharbrackleft}{\kern0pt}p{\isacharcomma}{\kern0pt}{\isacharminus}{\kern0pt}{\isacharbrackright}{\kern0pt}\isactrlesub \ {\isacharcolon}{\kern0pt}\ Q\ {\isasymrightarrow}\ R{\isachardoublequoteclose}\isanewline
%
\isadelimproof
\ \ %
\endisadelimproof
%
\isatagproof
\isacommand{using}\isamarkupfalse%
\ assms\ \isacommand{by}\isamarkupfalse%
\ {\isacharparenleft}{\kern0pt}unfold\ left{\isacharunderscore}{\kern0pt}param{\isacharunderscore}{\kern0pt}def{\isadigit{2}}{\isacharcomma}{\kern0pt}\ typecheck{\isacharunderscore}{\kern0pt}cfuncs{\isacharparenright}{\kern0pt}%
\endisatagproof
{\isafoldproof}%
%
\isadelimproof
\isanewline
%
\endisadelimproof
\isanewline
\isacommand{lemma}\isamarkupfalse%
\ left{\isacharunderscore}{\kern0pt}param{\isacharunderscore}{\kern0pt}on{\isacharunderscore}{\kern0pt}el{\isacharcolon}{\kern0pt}\isanewline
\ \ \isakeyword{assumes}\ {\isachardoublequoteopen}k\ {\isacharcolon}{\kern0pt}\ P\ {\isasymtimes}\isactrlsub c\ Q\ {\isasymrightarrow}\ R{\isachardoublequoteclose}\isanewline
\ \ \isakeyword{assumes}\ {\isachardoublequoteopen}p\ {\isasymin}\isactrlsub c\ P{\isachardoublequoteclose}\isanewline
\ \ \isakeyword{assumes}\ {\isachardoublequoteopen}q\ {\isasymin}\isactrlsub c\ Q{\isachardoublequoteclose}\isanewline
\ \ \isakeyword{shows}\ \ {\isachardoublequoteopen}k\isactrlbsub {\isacharbrackleft}{\kern0pt}p{\isacharcomma}{\kern0pt}{\isacharminus}{\kern0pt}{\isacharbrackright}{\kern0pt}\isactrlesub \ {\isasymcirc}\isactrlsub c\ q\ {\isacharequal}{\kern0pt}\ k\ {\isasymcirc}\isactrlsub c\ {\isasymlangle}p{\isacharcomma}{\kern0pt}\ q{\isasymrangle}{\isachardoublequoteclose}\isanewline
%
\isadelimproof
%
\endisadelimproof
%
\isatagproof
\isacommand{proof}\isamarkupfalse%
\ {\isacharminus}{\kern0pt}\ \isanewline
\ \ \isacommand{have}\isamarkupfalse%
\ {\isachardoublequoteopen}k\isactrlbsub {\isacharbrackleft}{\kern0pt}p{\isacharcomma}{\kern0pt}{\isacharminus}{\kern0pt}{\isacharbrackright}{\kern0pt}\isactrlesub \ {\isasymcirc}\isactrlsub c\ q\ {\isacharequal}{\kern0pt}\ k\ {\isasymcirc}\isactrlsub c\ {\isasymlangle}p\ {\isasymcirc}\isactrlsub c\ {\isasymbeta}\isactrlbsub Q\isactrlesub {\isacharcomma}{\kern0pt}\ id\ Q{\isasymrangle}\ \ {\isasymcirc}\isactrlsub c\ q{\isachardoublequoteclose}\isanewline
\ \ \ \ \isacommand{using}\isamarkupfalse%
\ assms\ cfunc{\isacharunderscore}{\kern0pt}type{\isacharunderscore}{\kern0pt}def\ comp{\isacharunderscore}{\kern0pt}associative\ left{\isacharunderscore}{\kern0pt}param{\isacharunderscore}{\kern0pt}def{\isadigit{2}}\ \isacommand{by}\isamarkupfalse%
\ {\isacharparenleft}{\kern0pt}typecheck{\isacharunderscore}{\kern0pt}cfuncs{\isacharcomma}{\kern0pt}\ force{\isacharparenright}{\kern0pt}\isanewline
\ \ \isacommand{also}\isamarkupfalse%
\ \isacommand{have}\isamarkupfalse%
\ {\isachardoublequoteopen}{\isachardot}{\kern0pt}{\isachardot}{\kern0pt}{\isachardot}{\kern0pt}\ {\isacharequal}{\kern0pt}\ k\ {\isasymcirc}\isactrlsub c\ {\isasymlangle}p{\isacharcomma}{\kern0pt}\ q{\isasymrangle}{\isachardoublequoteclose}\isanewline
\ \ \ \ \isacommand{using}\isamarkupfalse%
\ \ assms{\isacharparenleft}{\kern0pt}{\isadigit{2}}{\isacharcomma}{\kern0pt}{\isadigit{3}}{\isacharparenright}{\kern0pt}\ cart{\isacharunderscore}{\kern0pt}prod{\isacharunderscore}{\kern0pt}extract{\isacharunderscore}{\kern0pt}right\ \isacommand{by}\isamarkupfalse%
\ force\isanewline
\ \ \isacommand{then}\isamarkupfalse%
\ \isacommand{show}\isamarkupfalse%
\ {\isacharquery}{\kern0pt}thesis\isanewline
\ \ \ \ \isacommand{by}\isamarkupfalse%
\ {\isacharparenleft}{\kern0pt}simp\ add{\isacharcolon}{\kern0pt}\ calculation{\isacharparenright}{\kern0pt}\isanewline
\isacommand{qed}\isamarkupfalse%
%
\endisatagproof
{\isafoldproof}%
%
\isadelimproof
\isanewline
%
\endisadelimproof
\isanewline
\isacommand{definition}\isamarkupfalse%
\ right{\isacharunderscore}{\kern0pt}param\ {\isacharcolon}{\kern0pt}{\isacharcolon}{\kern0pt}\ {\isachardoublequoteopen}cfunc\ {\isasymRightarrow}\ cfunc\ {\isasymRightarrow}\ cfunc{\isachardoublequoteclose}\ {\isacharparenleft}{\kern0pt}{\isachardoublequoteopen}{\isacharunderscore}{\kern0pt}\isactrlbsub {\isacharbrackleft}{\kern0pt}{\isacharminus}{\kern0pt}{\isacharcomma}{\kern0pt}{\isacharunderscore}{\kern0pt}{\isacharbrackright}{\kern0pt}\isactrlesub {\isachardoublequoteclose}\ {\isacharbrackleft}{\kern0pt}{\isadigit{1}}{\isadigit{0}}{\isadigit{0}}{\isacharcomma}{\kern0pt}{\isadigit{0}}{\isacharbrackright}{\kern0pt}{\isadigit{1}}{\isadigit{0}}{\isadigit{0}}{\isacharparenright}{\kern0pt}\ \isakeyword{where}\isanewline
\ \ {\isachardoublequoteopen}right{\isacharunderscore}{\kern0pt}param\ k\ q\ {\isasymequiv}\ {\isacharparenleft}{\kern0pt}THE\ f{\isachardot}{\kern0pt}\ \ {\isasymexists}\ P\ Q\ R{\isachardot}{\kern0pt}\ k\ {\isacharcolon}{\kern0pt}\ P\ {\isasymtimes}\isactrlsub c\ Q\ {\isasymrightarrow}\ R\ {\isasymand}\ f\ {\isacharequal}{\kern0pt}\ k\ {\isasymcirc}\isactrlsub c\ {\isasymlangle}id\ P{\isacharcomma}{\kern0pt}\ q\ {\isasymcirc}\isactrlsub c\ {\isasymbeta}\isactrlbsub P\isactrlesub {\isasymrangle}{\isacharparenright}{\kern0pt}{\isachardoublequoteclose}\isanewline
\isanewline
\isacommand{lemma}\isamarkupfalse%
\ right{\isacharunderscore}{\kern0pt}param{\isacharunderscore}{\kern0pt}def{\isadigit{2}}{\isacharcolon}{\kern0pt}\isanewline
\ \ \isakeyword{assumes}\ {\isachardoublequoteopen}k\ {\isacharcolon}{\kern0pt}\ P\ {\isasymtimes}\isactrlsub c\ Q\ {\isasymrightarrow}\ R{\isachardoublequoteclose}\isanewline
\ \ \isakeyword{shows}\ {\isachardoublequoteopen}k\isactrlbsub {\isacharbrackleft}{\kern0pt}{\isacharminus}{\kern0pt}{\isacharcomma}{\kern0pt}q{\isacharbrackright}{\kern0pt}\isactrlesub \ {\isasymequiv}\ k\ {\isasymcirc}\isactrlsub c\ {\isasymlangle}id\ P{\isacharcomma}{\kern0pt}\ q\ {\isasymcirc}\isactrlsub c\ {\isasymbeta}\isactrlbsub P\isactrlesub {\isasymrangle}{\isachardoublequoteclose}\isanewline
%
\isadelimproof
%
\endisadelimproof
%
\isatagproof
\isacommand{proof}\isamarkupfalse%
\ {\isacharminus}{\kern0pt}\ \isanewline
\ \ \isacommand{have}\isamarkupfalse%
\ {\isachardoublequoteopen}{\isasymexists}\ P\ Q\ R{\isachardot}{\kern0pt}\ k\ {\isacharcolon}{\kern0pt}\ P\ {\isasymtimes}\isactrlsub c\ Q\ {\isasymrightarrow}\ R\ {\isasymand}\ right{\isacharunderscore}{\kern0pt}param\ k\ q\ {\isacharequal}{\kern0pt}\ k\ {\isasymcirc}\isactrlsub c\ {\isasymlangle}id\ P{\isacharcomma}{\kern0pt}\ q\ {\isasymcirc}\isactrlsub c\ {\isasymbeta}\isactrlbsub P\isactrlesub {\isasymrangle}{\isachardoublequoteclose}\isanewline
\ \ \ \ \isacommand{unfolding}\isamarkupfalse%
\ right{\isacharunderscore}{\kern0pt}param{\isacharunderscore}{\kern0pt}def\ \isacommand{by}\isamarkupfalse%
\ {\isacharparenleft}{\kern0pt}rule\ theI{\isacharprime}{\kern0pt}{\isacharcomma}{\kern0pt}\ insert\ assms{\isacharcomma}{\kern0pt}\ auto{\isacharcomma}{\kern0pt}\ metis\ cfunc{\isacharunderscore}{\kern0pt}type{\isacharunderscore}{\kern0pt}def\ exp{\isacharunderscore}{\kern0pt}set{\isacharunderscore}{\kern0pt}inj\ transpose{\isacharunderscore}{\kern0pt}func{\isacharunderscore}{\kern0pt}type{\isacharparenright}{\kern0pt}\ \isanewline
\ \ \isacommand{then}\isamarkupfalse%
\ \isacommand{show}\isamarkupfalse%
\ {\isachardoublequoteopen}k\isactrlbsub {\isacharbrackleft}{\kern0pt}{\isacharminus}{\kern0pt}{\isacharcomma}{\kern0pt}q{\isacharbrackright}{\kern0pt}\isactrlesub \ {\isasymequiv}\ k\ {\isasymcirc}\isactrlsub c\ {\isasymlangle}id\isactrlsub c\ P{\isacharcomma}{\kern0pt}q\ {\isasymcirc}\isactrlsub c\ {\isasymbeta}\isactrlbsub P\isactrlesub {\isasymrangle}{\isachardoublequoteclose}\isanewline
\ \ \ \ \isacommand{by}\isamarkupfalse%
\ {\isacharparenleft}{\kern0pt}smt\ {\isacharparenleft}{\kern0pt}z{\isadigit{3}}{\isacharparenright}{\kern0pt}\ assms\ cfunc{\isacharunderscore}{\kern0pt}type{\isacharunderscore}{\kern0pt}def\ exp{\isacharunderscore}{\kern0pt}set{\isacharunderscore}{\kern0pt}inj\ transpose{\isacharunderscore}{\kern0pt}func{\isacharunderscore}{\kern0pt}type{\isacharparenright}{\kern0pt}\isanewline
\isacommand{qed}\isamarkupfalse%
%
\endisatagproof
{\isafoldproof}%
%
\isadelimproof
\isanewline
%
\endisadelimproof
\isanewline
\isacommand{lemma}\isamarkupfalse%
\ right{\isacharunderscore}{\kern0pt}param{\isacharunderscore}{\kern0pt}type{\isacharbrackleft}{\kern0pt}type{\isacharunderscore}{\kern0pt}rule{\isacharbrackright}{\kern0pt}{\isacharcolon}{\kern0pt}\isanewline
\ \ \isakeyword{assumes}\ {\isachardoublequoteopen}k\ {\isacharcolon}{\kern0pt}\ P\ {\isasymtimes}\isactrlsub c\ Q\ {\isasymrightarrow}\ R{\isachardoublequoteclose}\isanewline
\ \ \isakeyword{assumes}\ {\isachardoublequoteopen}q\ {\isasymin}\isactrlsub c\ Q{\isachardoublequoteclose}\isanewline
\ \ \isakeyword{shows}\ {\isachardoublequoteopen}k\isactrlbsub {\isacharbrackleft}{\kern0pt}{\isacharminus}{\kern0pt}{\isacharcomma}{\kern0pt}q{\isacharbrackright}{\kern0pt}\isactrlesub \ {\isacharcolon}{\kern0pt}\ P\ {\isasymrightarrow}\ R{\isachardoublequoteclose}\isanewline
%
\isadelimproof
\ \ %
\endisadelimproof
%
\isatagproof
\isacommand{using}\isamarkupfalse%
\ assms\ \isacommand{by}\isamarkupfalse%
\ {\isacharparenleft}{\kern0pt}unfold\ right{\isacharunderscore}{\kern0pt}param{\isacharunderscore}{\kern0pt}def{\isadigit{2}}{\isacharcomma}{\kern0pt}\ typecheck{\isacharunderscore}{\kern0pt}cfuncs{\isacharparenright}{\kern0pt}%
\endisatagproof
{\isafoldproof}%
%
\isadelimproof
\isanewline
%
\endisadelimproof
\isanewline
\isacommand{lemma}\isamarkupfalse%
\ right{\isacharunderscore}{\kern0pt}param{\isacharunderscore}{\kern0pt}on{\isacharunderscore}{\kern0pt}el{\isacharcolon}{\kern0pt}\isanewline
\ \ \isakeyword{assumes}\ {\isachardoublequoteopen}k\ {\isacharcolon}{\kern0pt}\ P\ {\isasymtimes}\isactrlsub c\ Q\ {\isasymrightarrow}\ R{\isachardoublequoteclose}\isanewline
\ \ \isakeyword{assumes}\ {\isachardoublequoteopen}p\ {\isasymin}\isactrlsub c\ P{\isachardoublequoteclose}\isanewline
\ \ \isakeyword{assumes}\ {\isachardoublequoteopen}q\ {\isasymin}\isactrlsub c\ Q{\isachardoublequoteclose}\isanewline
\ \ \isakeyword{shows}\ \ {\isachardoublequoteopen}k\isactrlbsub {\isacharbrackleft}{\kern0pt}{\isacharminus}{\kern0pt}{\isacharcomma}{\kern0pt}q{\isacharbrackright}{\kern0pt}\isactrlesub \ {\isasymcirc}\isactrlsub c\ p\ {\isacharequal}{\kern0pt}\ k\ {\isasymcirc}\isactrlsub c\ {\isasymlangle}p{\isacharcomma}{\kern0pt}\ q{\isasymrangle}{\isachardoublequoteclose}\isanewline
%
\isadelimproof
%
\endisadelimproof
%
\isatagproof
\isacommand{proof}\isamarkupfalse%
\ {\isacharminus}{\kern0pt}\ \isanewline
\ \ \isacommand{have}\isamarkupfalse%
\ {\isachardoublequoteopen}k\isactrlbsub {\isacharbrackleft}{\kern0pt}{\isacharminus}{\kern0pt}{\isacharcomma}{\kern0pt}q{\isacharbrackright}{\kern0pt}\isactrlesub \ {\isasymcirc}\isactrlsub c\ p\ {\isacharequal}{\kern0pt}\ k\ {\isasymcirc}\isactrlsub c\ \ {\isasymlangle}id\ P{\isacharcomma}{\kern0pt}\ q\ {\isasymcirc}\isactrlsub c\ {\isasymbeta}\isactrlbsub P\isactrlesub {\isasymrangle}\ \ {\isasymcirc}\isactrlsub c\ p{\isachardoublequoteclose}\isanewline
\ \ \ \ \isacommand{using}\isamarkupfalse%
\ assms\ cfunc{\isacharunderscore}{\kern0pt}type{\isacharunderscore}{\kern0pt}def\ comp{\isacharunderscore}{\kern0pt}associative\ right{\isacharunderscore}{\kern0pt}param{\isacharunderscore}{\kern0pt}def{\isadigit{2}}\ \isacommand{by}\isamarkupfalse%
\ {\isacharparenleft}{\kern0pt}typecheck{\isacharunderscore}{\kern0pt}cfuncs{\isacharcomma}{\kern0pt}\ force{\isacharparenright}{\kern0pt}\isanewline
\ \ \isacommand{also}\isamarkupfalse%
\ \isacommand{have}\isamarkupfalse%
\ {\isachardoublequoteopen}{\isachardot}{\kern0pt}{\isachardot}{\kern0pt}{\isachardot}{\kern0pt}\ {\isacharequal}{\kern0pt}\ k\ {\isasymcirc}\isactrlsub c\ {\isasymlangle}p{\isacharcomma}{\kern0pt}\ q{\isasymrangle}{\isachardoublequoteclose}\isanewline
\ \ \ \ \isacommand{using}\isamarkupfalse%
\ assms{\isacharparenleft}{\kern0pt}{\isadigit{2}}{\isacharparenright}{\kern0pt}\ assms{\isacharparenleft}{\kern0pt}{\isadigit{3}}{\isacharparenright}{\kern0pt}\ cart{\isacharunderscore}{\kern0pt}prod{\isacharunderscore}{\kern0pt}extract{\isacharunderscore}{\kern0pt}left\ \isacommand{by}\isamarkupfalse%
\ force\isanewline
\ \ \isacommand{then}\isamarkupfalse%
\ \isacommand{show}\isamarkupfalse%
\ {\isacharquery}{\kern0pt}thesis\isanewline
\ \ \ \ \isacommand{by}\isamarkupfalse%
\ {\isacharparenleft}{\kern0pt}simp\ add{\isacharcolon}{\kern0pt}\ calculation{\isacharparenright}{\kern0pt}\isanewline
\isacommand{qed}\isamarkupfalse%
%
\endisatagproof
{\isafoldproof}%
%
\isadelimproof
%
\endisadelimproof
%
\isadelimdocument
%
\endisadelimdocument
%
\isatagdocument
%
\isamarkupsubsection{Exponential Set Facts%
}
\isamarkuptrue%
%
\endisatagdocument
{\isafolddocument}%
%
\isadelimdocument
%
\endisadelimdocument
%
\begin{isamarkuptext}%
The lemma below corresponds to Proposition 2.5.7 in Halvorson.%
\end{isamarkuptext}\isamarkuptrue%
\isacommand{lemma}\isamarkupfalse%
\ exp{\isacharunderscore}{\kern0pt}one{\isacharcolon}{\kern0pt}\isanewline
\ \ {\isachardoublequoteopen}X\isactrlbsup {\isasymone}\isactrlesup \ {\isasymcong}\ X{\isachardoublequoteclose}\isanewline
%
\isadelimproof
%
\endisadelimproof
%
\isatagproof
\isacommand{proof}\isamarkupfalse%
\ {\isacharminus}{\kern0pt}\isanewline
\ \ \isacommand{obtain}\isamarkupfalse%
\ e\ \isakeyword{where}\ e{\isacharunderscore}{\kern0pt}defn{\isacharcolon}{\kern0pt}\ {\isachardoublequoteopen}e\ {\isacharequal}{\kern0pt}\ eval{\isacharunderscore}{\kern0pt}func\ X\ {\isasymone}{\isachardoublequoteclose}\ \isakeyword{and}\ e{\isacharunderscore}{\kern0pt}type{\isacharcolon}{\kern0pt}\ {\isachardoublequoteopen}e\ {\isacharcolon}{\kern0pt}\ {\isasymone}\ {\isasymtimes}\isactrlsub c\ X\isactrlbsup {\isasymone}\isactrlesup \ {\isasymrightarrow}\ X{\isachardoublequoteclose}\isanewline
\ \ \ \ \isacommand{using}\isamarkupfalse%
\ eval{\isacharunderscore}{\kern0pt}func{\isacharunderscore}{\kern0pt}type\ \isacommand{by}\isamarkupfalse%
\ auto\isanewline
\ \ \isacommand{obtain}\isamarkupfalse%
\ i\ \isakeyword{where}\ i{\isacharunderscore}{\kern0pt}type{\isacharcolon}{\kern0pt}\ {\isachardoublequoteopen}i{\isacharcolon}{\kern0pt}\ {\isasymone}\ {\isasymtimes}\isactrlsub c\ {\isasymone}\ {\isasymrightarrow}\ {\isasymone}{\isachardoublequoteclose}\isanewline
\ \ \ \ \isacommand{using}\isamarkupfalse%
\ terminal{\isacharunderscore}{\kern0pt}func{\isacharunderscore}{\kern0pt}type\ \isacommand{by}\isamarkupfalse%
\ blast\isanewline
\ \ \isacommand{obtain}\isamarkupfalse%
\ i{\isacharunderscore}{\kern0pt}inv\ \isakeyword{where}\ i{\isacharunderscore}{\kern0pt}iso{\isacharcolon}{\kern0pt}\ {\isachardoublequoteopen}i{\isacharunderscore}{\kern0pt}inv{\isacharcolon}{\kern0pt}\ {\isasymone}{\isasymrightarrow}\ \ {\isasymone}\ {\isasymtimes}\isactrlsub c\ {\isasymone}\ {\isasymand}\ \isanewline
\ \ \ \ \ \ \ \ \ \ \ \ \ \ \ \ \ \ \ \ \ \ \ \ \ \ \ \ \ i\ {\isasymcirc}\isactrlsub c\ i{\isacharunderscore}{\kern0pt}inv\ {\isacharequal}{\kern0pt}\ id{\isacharparenleft}{\kern0pt}{\isasymone}{\isacharparenright}{\kern0pt}\ {\isasymand}\ \ \isanewline
\ \ \ \ \ \ \ \ \ \ \ \ \ \ \ \ \ \ \ \ \ \ \ \ \ \ \ \ \ i{\isacharunderscore}{\kern0pt}inv\ {\isasymcirc}\isactrlsub c\ i\ {\isacharequal}{\kern0pt}\ id{\isacharparenleft}{\kern0pt}{\isasymone}\ {\isasymtimes}\isactrlsub c\ {\isasymone}{\isacharparenright}{\kern0pt}{\isachardoublequoteclose}\isanewline
\ \ \ \ \isacommand{by}\isamarkupfalse%
\ {\isacharparenleft}{\kern0pt}smt\ cfunc{\isacharunderscore}{\kern0pt}cross{\isacharunderscore}{\kern0pt}prod{\isacharunderscore}{\kern0pt}comp{\isacharunderscore}{\kern0pt}cfunc{\isacharunderscore}{\kern0pt}prod\ cfunc{\isacharunderscore}{\kern0pt}cross{\isacharunderscore}{\kern0pt}prod{\isacharunderscore}{\kern0pt}comp{\isacharunderscore}{\kern0pt}diagonal\ cfunc{\isacharunderscore}{\kern0pt}cross{\isacharunderscore}{\kern0pt}prod{\isacharunderscore}{\kern0pt}def\ cfunc{\isacharunderscore}{\kern0pt}prod{\isacharunderscore}{\kern0pt}type\ cfunc{\isacharunderscore}{\kern0pt}type{\isacharunderscore}{\kern0pt}def\ diagonal{\isacharunderscore}{\kern0pt}def\ i{\isacharunderscore}{\kern0pt}type\ id{\isacharunderscore}{\kern0pt}cross{\isacharunderscore}{\kern0pt}prod\ id{\isacharunderscore}{\kern0pt}left{\isacharunderscore}{\kern0pt}unit\ id{\isacharunderscore}{\kern0pt}type\ left{\isacharunderscore}{\kern0pt}cart{\isacharunderscore}{\kern0pt}proj{\isacharunderscore}{\kern0pt}type\ right{\isacharunderscore}{\kern0pt}cart{\isacharunderscore}{\kern0pt}proj{\isacharunderscore}{\kern0pt}cfunc{\isacharunderscore}{\kern0pt}prod\ right{\isacharunderscore}{\kern0pt}cart{\isacharunderscore}{\kern0pt}proj{\isacharunderscore}{\kern0pt}type\ terminal{\isacharunderscore}{\kern0pt}func{\isacharunderscore}{\kern0pt}unique{\isacharparenright}{\kern0pt}\isanewline
\ \ \isacommand{then}\isamarkupfalse%
\ \isacommand{have}\isamarkupfalse%
\ i{\isacharunderscore}{\kern0pt}inv{\isacharunderscore}{\kern0pt}type{\isacharcolon}{\kern0pt}\ {\isachardoublequoteopen}i{\isacharunderscore}{\kern0pt}inv{\isacharcolon}{\kern0pt}\ {\isasymone}{\isasymrightarrow}\ \ {\isasymone}\ {\isasymtimes}\isactrlsub c\ {\isasymone}{\isachardoublequoteclose}\isanewline
\ \ \ \ \isacommand{by}\isamarkupfalse%
\ auto\isanewline
\isanewline
\ \ \isacommand{have}\isamarkupfalse%
\ inj{\isacharcolon}{\kern0pt}\ {\isachardoublequoteopen}injective{\isacharparenleft}{\kern0pt}e{\isacharparenright}{\kern0pt}{\isachardoublequoteclose}\isanewline
\ \ \ \ \isacommand{by}\isamarkupfalse%
\ {\isacharparenleft}{\kern0pt}simp\ add{\isacharcolon}{\kern0pt}\ e{\isacharunderscore}{\kern0pt}defn\ eval{\isacharunderscore}{\kern0pt}func{\isacharunderscore}{\kern0pt}X{\isacharunderscore}{\kern0pt}one{\isacharunderscore}{\kern0pt}injective{\isacharparenright}{\kern0pt}\isanewline
\isanewline
\ \ \isacommand{have}\isamarkupfalse%
\ surj{\isacharcolon}{\kern0pt}\ {\isachardoublequoteopen}surjective{\isacharparenleft}{\kern0pt}e{\isacharparenright}{\kern0pt}{\isachardoublequoteclose}\isanewline
\ \ \ \ \ \isacommand{unfolding}\isamarkupfalse%
\ surjective{\isacharunderscore}{\kern0pt}def\isanewline
\ \ \ \isacommand{proof}\isamarkupfalse%
\ clarify\isanewline
\ \ \ \ \isacommand{fix}\isamarkupfalse%
\ y\ \isanewline
\ \ \ \ \isacommand{assume}\isamarkupfalse%
\ {\isachardoublequoteopen}y\ {\isasymin}\isactrlsub c\ codomain\ e{\isachardoublequoteclose}\isanewline
\ \ \ \ \isacommand{then}\isamarkupfalse%
\ \isacommand{have}\isamarkupfalse%
\ y{\isacharunderscore}{\kern0pt}type{\isacharcolon}{\kern0pt}\ {\isachardoublequoteopen}y\ {\isasymin}\isactrlsub c\ X{\isachardoublequoteclose}\isanewline
\ \ \ \ \ \ \isacommand{using}\isamarkupfalse%
\ cfunc{\isacharunderscore}{\kern0pt}type{\isacharunderscore}{\kern0pt}def\ e{\isacharunderscore}{\kern0pt}type\ \isacommand{by}\isamarkupfalse%
\ auto\isanewline
\isanewline
\ \ \ \ \isacommand{have}\isamarkupfalse%
\ witness{\isacharunderscore}{\kern0pt}type{\isacharcolon}{\kern0pt}\ {\isachardoublequoteopen}{\isacharparenleft}{\kern0pt}id\isactrlsub c\ {\isasymone}\ {\isasymtimes}\isactrlsub f\ {\isacharparenleft}{\kern0pt}y\ {\isasymcirc}\isactrlsub c\ i{\isacharparenright}{\kern0pt}\isactrlsup {\isasymsharp}{\isacharparenright}{\kern0pt}\ {\isasymcirc}\isactrlsub c\ i{\isacharunderscore}{\kern0pt}inv\ {\isasymin}\isactrlsub c\ {\isasymone}\ {\isasymtimes}\isactrlsub c\ X\isactrlbsup {\isasymone}\isactrlesup {\isachardoublequoteclose}\isanewline
\ \ \ \ \ \ \isacommand{using}\isamarkupfalse%
\ y{\isacharunderscore}{\kern0pt}type\ i{\isacharunderscore}{\kern0pt}type\ i{\isacharunderscore}{\kern0pt}inv{\isacharunderscore}{\kern0pt}type\ \isacommand{by}\isamarkupfalse%
\ typecheck{\isacharunderscore}{\kern0pt}cfuncs\isanewline
\isanewline
\ \ \ \ \isacommand{have}\isamarkupfalse%
\ square{\isacharcolon}{\kern0pt}\ {\isachardoublequoteopen}e\ {\isasymcirc}\isactrlsub c\ {\isacharparenleft}{\kern0pt}id{\isacharparenleft}{\kern0pt}{\isasymone}{\isacharparenright}{\kern0pt}\ {\isasymtimes}\isactrlsub f\ {\isacharparenleft}{\kern0pt}y\ {\isasymcirc}\isactrlsub c\ i{\isacharparenright}{\kern0pt}\isactrlsup {\isasymsharp}{\isacharparenright}{\kern0pt}\ {\isacharequal}{\kern0pt}\ y\ {\isasymcirc}\isactrlsub c\ i{\isachardoublequoteclose}\isanewline
\ \ \ \ \ \ \isacommand{using}\isamarkupfalse%
\ comp{\isacharunderscore}{\kern0pt}type\ e{\isacharunderscore}{\kern0pt}defn\ i{\isacharunderscore}{\kern0pt}type\ transpose{\isacharunderscore}{\kern0pt}func{\isacharunderscore}{\kern0pt}def\ y{\isacharunderscore}{\kern0pt}type\ \isacommand{by}\isamarkupfalse%
\ blast\isanewline
\ \ \ \ \isacommand{then}\isamarkupfalse%
\ \isacommand{show}\isamarkupfalse%
\ {\isachardoublequoteopen}{\isasymexists}x{\isachardot}{\kern0pt}\ x\ {\isasymin}\isactrlsub c\ domain\ e\ {\isasymand}\ e\ {\isasymcirc}\isactrlsub c\ x\ {\isacharequal}{\kern0pt}\ y{\isachardoublequoteclose}\ \isanewline
\ \ \ \ \ \ \isacommand{unfolding}\isamarkupfalse%
\ cfunc{\isacharunderscore}{\kern0pt}type{\isacharunderscore}{\kern0pt}def\ \isacommand{using}\isamarkupfalse%
\ y{\isacharunderscore}{\kern0pt}type\ i{\isacharunderscore}{\kern0pt}type\ i{\isacharunderscore}{\kern0pt}inv{\isacharunderscore}{\kern0pt}type\ e{\isacharunderscore}{\kern0pt}type\ \isanewline
\ \ \ \ \ \ \isacommand{by}\isamarkupfalse%
\ {\isacharparenleft}{\kern0pt}rule{\isacharunderscore}{\kern0pt}tac\ x{\isacharequal}{\kern0pt}{\isachardoublequoteopen}{\isacharparenleft}{\kern0pt}id{\isacharparenleft}{\kern0pt}{\isasymone}{\isacharparenright}{\kern0pt}\ {\isasymtimes}\isactrlsub f\ {\isacharparenleft}{\kern0pt}y\ {\isasymcirc}\isactrlsub c\ i{\isacharparenright}{\kern0pt}\isactrlsup {\isasymsharp}{\isacharparenright}{\kern0pt}\ {\isasymcirc}\isactrlsub c\ i{\isacharunderscore}{\kern0pt}inv{\isachardoublequoteclose}\ \isakeyword{in}\ exI{\isacharcomma}{\kern0pt}\ typecheck{\isacharunderscore}{\kern0pt}cfuncs{\isacharcomma}{\kern0pt}\ metis\ cfunc{\isacharunderscore}{\kern0pt}type{\isacharunderscore}{\kern0pt}def\ comp{\isacharunderscore}{\kern0pt}associative\ i{\isacharunderscore}{\kern0pt}iso\ id{\isacharunderscore}{\kern0pt}right{\isacharunderscore}{\kern0pt}unit{\isadigit{2}}{\isacharparenright}{\kern0pt}\isanewline
\ \ \isacommand{qed}\isamarkupfalse%
\isanewline
\isanewline
\ \ \isacommand{have}\isamarkupfalse%
\ {\isachardoublequoteopen}isomorphism\ e{\isachardoublequoteclose}\isanewline
\ \ \ \ \isacommand{using}\isamarkupfalse%
\ epi{\isacharunderscore}{\kern0pt}mon{\isacharunderscore}{\kern0pt}is{\isacharunderscore}{\kern0pt}iso\ inj\ injective{\isacharunderscore}{\kern0pt}imp{\isacharunderscore}{\kern0pt}monomorphism\ surj\ surjective{\isacharunderscore}{\kern0pt}is{\isacharunderscore}{\kern0pt}epimorphism\ \isacommand{by}\isamarkupfalse%
\ fastforce\isanewline
\ \ \isacommand{then}\isamarkupfalse%
\ \isacommand{show}\isamarkupfalse%
\ {\isachardoublequoteopen}X\isactrlbsup {\isasymone}\isactrlesup \ {\isasymcong}\ X{\isachardoublequoteclose}\isanewline
\ \ \ \ \isacommand{using}\isamarkupfalse%
\ e{\isacharunderscore}{\kern0pt}type\ is{\isacharunderscore}{\kern0pt}isomorphic{\isacharunderscore}{\kern0pt}def\ isomorphic{\isacharunderscore}{\kern0pt}is{\isacharunderscore}{\kern0pt}symmetric\ isomorphic{\isacharunderscore}{\kern0pt}is{\isacharunderscore}{\kern0pt}transitive\ one{\isacharunderscore}{\kern0pt}x{\isacharunderscore}{\kern0pt}A{\isacharunderscore}{\kern0pt}iso{\isacharunderscore}{\kern0pt}A\ \isacommand{by}\isamarkupfalse%
\ blast\isanewline
\isacommand{qed}\isamarkupfalse%
%
\endisatagproof
{\isafoldproof}%
%
\isadelimproof
%
\endisadelimproof
%
\begin{isamarkuptext}%
The lemma below corresponds to Proposition 2.5.8 in Halvorson.%
\end{isamarkuptext}\isamarkuptrue%
\isacommand{lemma}\isamarkupfalse%
\ exp{\isacharunderscore}{\kern0pt}empty{\isacharcolon}{\kern0pt}\isanewline
\ \ {\isachardoublequoteopen}X\isactrlbsup {\isasymemptyset}\isactrlesup \ {\isasymcong}\ {\isasymone}{\isachardoublequoteclose}\isanewline
%
\isadelimproof
%
\endisadelimproof
%
\isatagproof
\isacommand{proof}\isamarkupfalse%
\ {\isacharminus}{\kern0pt}\ \isanewline
\ \ \isacommand{obtain}\isamarkupfalse%
\ f\ \isakeyword{where}\ f{\isacharunderscore}{\kern0pt}type{\isacharcolon}{\kern0pt}\ {\isachardoublequoteopen}f\ {\isacharequal}{\kern0pt}\ {\isasymalpha}\isactrlbsub X\isactrlesub {\isasymcirc}\isactrlsub c\ {\isacharparenleft}{\kern0pt}left{\isacharunderscore}{\kern0pt}cart{\isacharunderscore}{\kern0pt}proj\ {\isasymemptyset}\ {\isasymone}{\isacharparenright}{\kern0pt}{\isachardoublequoteclose}\ \isakeyword{and}\ fsharp{\isacharunderscore}{\kern0pt}type{\isacharbrackleft}{\kern0pt}type{\isacharunderscore}{\kern0pt}rule{\isacharbrackright}{\kern0pt}{\isacharcolon}{\kern0pt}\ {\isachardoublequoteopen}f\isactrlsup {\isasymsharp}\ {\isasymin}\isactrlsub c\ X\isactrlbsup {\isasymemptyset}\isactrlesup {\isachardoublequoteclose}\isanewline
\ \ \ \ \isacommand{using}\isamarkupfalse%
\ transpose{\isacharunderscore}{\kern0pt}func{\isacharunderscore}{\kern0pt}type\ \isacommand{by}\isamarkupfalse%
\ {\isacharparenleft}{\kern0pt}typecheck{\isacharunderscore}{\kern0pt}cfuncs{\isacharcomma}{\kern0pt}\ force{\isacharparenright}{\kern0pt}\isanewline
\ \ \isacommand{have}\isamarkupfalse%
\ uniqueness{\isacharcolon}{\kern0pt}\ {\isachardoublequoteopen}{\isasymforall}z{\isachardot}{\kern0pt}\ z\ {\isasymin}\isactrlsub c\ X\isactrlbsup {\isasymemptyset}\isactrlesup \ {\isasymlongrightarrow}\ z{\isacharequal}{\kern0pt}f\isactrlsup {\isasymsharp}{\isachardoublequoteclose}\isanewline
\ \ \isacommand{proof}\isamarkupfalse%
\ clarify\isanewline
\ \ \ \ \isacommand{fix}\isamarkupfalse%
\ z\isanewline
\ \ \ \ \isacommand{assume}\isamarkupfalse%
\ z{\isacharunderscore}{\kern0pt}type{\isacharbrackleft}{\kern0pt}type{\isacharunderscore}{\kern0pt}rule{\isacharbrackright}{\kern0pt}{\isacharcolon}{\kern0pt}\ {\isachardoublequoteopen}z\ {\isasymin}\isactrlsub c\ X\isactrlbsup {\isasymemptyset}\isactrlesup {\isachardoublequoteclose}\isanewline
\ \ \ \ \isacommand{obtain}\isamarkupfalse%
\ j\ \isakeyword{where}\ j{\isacharunderscore}{\kern0pt}iso{\isacharcolon}{\kern0pt}{\isachardoublequoteopen}j{\isacharcolon}{\kern0pt}{\isasymemptyset}\ {\isasymrightarrow}\ {\isasymemptyset}\ {\isasymtimes}\isactrlsub c\ {\isasymone}\ {\isasymand}\ isomorphism{\isacharparenleft}{\kern0pt}j{\isacharparenright}{\kern0pt}{\isachardoublequoteclose}\isanewline
\ \ \ \ \ \ \isacommand{using}\isamarkupfalse%
\ is{\isacharunderscore}{\kern0pt}isomorphic{\isacharunderscore}{\kern0pt}def\ isomorphic{\isacharunderscore}{\kern0pt}is{\isacharunderscore}{\kern0pt}symmetric\ empty{\isacharunderscore}{\kern0pt}prod{\isacharunderscore}{\kern0pt}X\ \isacommand{by}\isamarkupfalse%
\ presburger\isanewline
\ \ \ \ \isacommand{obtain}\isamarkupfalse%
\ {\isasympsi}\ \isakeyword{where}\ psi{\isacharunderscore}{\kern0pt}type{\isacharcolon}{\kern0pt}\ {\isachardoublequoteopen}{\isasympsi}\ {\isacharcolon}{\kern0pt}\ {\isasymemptyset}\ {\isasymtimes}\isactrlsub c\ {\isasymone}\ {\isasymrightarrow}\ {\isasymemptyset}\ {\isasymand}\isanewline
\ \ \ \ \ \ \ \ \ \ \ \ \ \ \ \ \ \ \ \ \ j\ {\isasymcirc}\isactrlsub c\ {\isasympsi}\ {\isacharequal}{\kern0pt}\ id{\isacharparenleft}{\kern0pt}{\isasymemptyset}\ {\isasymtimes}\isactrlsub c\ {\isasymone}{\isacharparenright}{\kern0pt}\ {\isasymand}\ {\isasympsi}\ {\isasymcirc}\isactrlsub c\ j\ {\isacharequal}{\kern0pt}\ id{\isacharparenleft}{\kern0pt}{\isasymemptyset}{\isacharparenright}{\kern0pt}{\isachardoublequoteclose}\isanewline
\ \ \ \ \ \ \isacommand{using}\isamarkupfalse%
\ cfunc{\isacharunderscore}{\kern0pt}type{\isacharunderscore}{\kern0pt}def\ isomorphism{\isacharunderscore}{\kern0pt}def\ j{\isacharunderscore}{\kern0pt}iso\ \isacommand{by}\isamarkupfalse%
\ fastforce\ \isanewline
\ \ \ \ \isacommand{then}\isamarkupfalse%
\ \isacommand{have}\isamarkupfalse%
\ f{\isacharunderscore}{\kern0pt}sharp\ {\isacharcolon}{\kern0pt}\ {\isachardoublequoteopen}id{\isacharparenleft}{\kern0pt}{\isasymemptyset}{\isacharparenright}{\kern0pt}{\isasymtimes}\isactrlsub f\ z\ {\isacharequal}{\kern0pt}\ id{\isacharparenleft}{\kern0pt}{\isasymemptyset}{\isacharparenright}{\kern0pt}{\isasymtimes}\isactrlsub f\ f\isactrlsup {\isasymsharp}{\isachardoublequoteclose}\isanewline
\ \ \ \ \ \ \isacommand{by}\isamarkupfalse%
\ {\isacharparenleft}{\kern0pt}typecheck{\isacharunderscore}{\kern0pt}cfuncs{\isacharcomma}{\kern0pt}\ meson\ comp{\isacharunderscore}{\kern0pt}type\ emptyset{\isacharunderscore}{\kern0pt}is{\isacharunderscore}{\kern0pt}empty\ one{\isacharunderscore}{\kern0pt}separator{\isacharparenright}{\kern0pt}\isanewline
\ \ \ \ \isacommand{then}\isamarkupfalse%
\ \isacommand{show}\isamarkupfalse%
\ {\isachardoublequoteopen}z\ {\isacharequal}{\kern0pt}\ f\isactrlsup {\isasymsharp}{\isachardoublequoteclose}\isanewline
\ \ \ \ \ \ \isacommand{using}\isamarkupfalse%
\ \ fsharp{\isacharunderscore}{\kern0pt}type\ same{\isacharunderscore}{\kern0pt}evals{\isacharunderscore}{\kern0pt}equal\ z{\isacharunderscore}{\kern0pt}type\ \isacommand{by}\isamarkupfalse%
\ force\isanewline
\ \ \isacommand{qed}\isamarkupfalse%
\isanewline
\ \ \isacommand{then}\isamarkupfalse%
\ \isacommand{have}\isamarkupfalse%
\ {\isachardoublequoteopen}{\isasymexists}{\isacharbang}{\kern0pt}\ x{\isachardot}{\kern0pt}\ x\ {\isasymin}\isactrlsub c\ X\isactrlbsup {\isasymemptyset}\isactrlesup {\isachardoublequoteclose}\isanewline
\ \ \ \ \isacommand{by}\isamarkupfalse%
\ {\isacharparenleft}{\kern0pt}rule{\isacharunderscore}{\kern0pt}tac\ a{\isacharequal}{\kern0pt}{\isachardoublequoteopen}f\isactrlsup {\isasymsharp}{\isachardoublequoteclose}\ \isakeyword{in}\ ex{\isadigit{1}}I{\isacharcomma}{\kern0pt}\ simp{\isacharunderscore}{\kern0pt}all\ add{\isacharcolon}{\kern0pt}\ fsharp{\isacharunderscore}{\kern0pt}type{\isacharparenright}{\kern0pt}\isanewline
\ \ \isacommand{then}\isamarkupfalse%
\ \isacommand{show}\isamarkupfalse%
\ {\isachardoublequoteopen}X\isactrlbsup {\isasymemptyset}\isactrlesup \ {\isasymcong}\ {\isasymone}{\isachardoublequoteclose}\isanewline
\ \ \ \ \isacommand{using}\isamarkupfalse%
\ single{\isacharunderscore}{\kern0pt}elem{\isacharunderscore}{\kern0pt}iso{\isacharunderscore}{\kern0pt}one\ \isacommand{by}\isamarkupfalse%
\ auto\isanewline
\isacommand{qed}\isamarkupfalse%
%
\endisatagproof
{\isafoldproof}%
%
\isadelimproof
\isanewline
%
\endisadelimproof
\isanewline
\isacommand{lemma}\isamarkupfalse%
\ one{\isacharunderscore}{\kern0pt}exp{\isacharcolon}{\kern0pt}\isanewline
\ \ {\isachardoublequoteopen}{\isasymone}\isactrlbsup X\isactrlesup \ {\isasymcong}\ {\isasymone}{\isachardoublequoteclose}\isanewline
%
\isadelimproof
%
\endisadelimproof
%
\isatagproof
\isacommand{proof}\isamarkupfalse%
\ {\isacharminus}{\kern0pt}\ \isanewline
\ \ \isacommand{have}\isamarkupfalse%
\ nonempty{\isacharcolon}{\kern0pt}\ {\isachardoublequoteopen}nonempty{\isacharparenleft}{\kern0pt}{\isasymone}\isactrlbsup X\isactrlesup {\isacharparenright}{\kern0pt}{\isachardoublequoteclose}\isanewline
\ \ \ \ \isacommand{using}\isamarkupfalse%
\ nonempty{\isacharunderscore}{\kern0pt}def\ right{\isacharunderscore}{\kern0pt}cart{\isacharunderscore}{\kern0pt}proj{\isacharunderscore}{\kern0pt}type\ transpose{\isacharunderscore}{\kern0pt}func{\isacharunderscore}{\kern0pt}type\ \isacommand{by}\isamarkupfalse%
\ blast\isanewline
\ \ \isacommand{obtain}\isamarkupfalse%
\ e\ \isakeyword{where}\ e{\isacharunderscore}{\kern0pt}defn{\isacharcolon}{\kern0pt}\ {\isachardoublequoteopen}e\ {\isacharequal}{\kern0pt}\ eval{\isacharunderscore}{\kern0pt}func\ {\isasymone}\ X{\isachardoublequoteclose}\ \isakeyword{and}\ e{\isacharunderscore}{\kern0pt}type{\isacharcolon}{\kern0pt}\ {\isachardoublequoteopen}e\ {\isacharcolon}{\kern0pt}\ X\ {\isasymtimes}\isactrlsub c\ {\isasymone}\isactrlbsup X\isactrlesup \ {\isasymrightarrow}\ {\isasymone}{\isachardoublequoteclose}\isanewline
\ \ \ \ \isacommand{by}\isamarkupfalse%
\ {\isacharparenleft}{\kern0pt}simp\ add{\isacharcolon}{\kern0pt}\ eval{\isacharunderscore}{\kern0pt}func{\isacharunderscore}{\kern0pt}type{\isacharparenright}{\kern0pt}\isanewline
\ \ \isacommand{have}\isamarkupfalse%
\ uniqueness{\isacharcolon}{\kern0pt}\ {\isachardoublequoteopen}{\isasymforall}y{\isachardot}{\kern0pt}\ {\isacharparenleft}{\kern0pt}y{\isasymin}\isactrlsub c\ {\isasymone}\isactrlbsup X\isactrlesup \ {\isasymlongrightarrow}\ e\ {\isasymcirc}\isactrlsub c\ {\isacharparenleft}{\kern0pt}id{\isacharparenleft}{\kern0pt}X{\isacharparenright}{\kern0pt}\ {\isasymtimes}\isactrlsub f\ y{\isacharparenright}{\kern0pt}\ {\isacharcolon}{\kern0pt}\ X\ {\isasymtimes}\isactrlsub c\ {\isasymone}\ \ {\isasymrightarrow}\ {\isasymone}{\isacharparenright}{\kern0pt}{\isachardoublequoteclose}\isanewline
\ \ \ \ \isacommand{by}\isamarkupfalse%
\ {\isacharparenleft}{\kern0pt}meson\ cfunc{\isacharunderscore}{\kern0pt}cross{\isacharunderscore}{\kern0pt}prod{\isacharunderscore}{\kern0pt}type\ comp{\isacharunderscore}{\kern0pt}type\ e{\isacharunderscore}{\kern0pt}type\ id{\isacharunderscore}{\kern0pt}type{\isacharparenright}{\kern0pt}\isanewline
\ \ \isacommand{have}\isamarkupfalse%
\ uniquess{\isacharunderscore}{\kern0pt}form{\isacharcolon}{\kern0pt}\ {\isachardoublequoteopen}{\isasymforall}y{\isachardot}{\kern0pt}\ {\isacharparenleft}{\kern0pt}y{\isasymin}\isactrlsub c\ {\isasymone}\isactrlbsup X\isactrlesup \ {\isasymlongrightarrow}\ e\ {\isasymcirc}\isactrlsub c\ {\isacharparenleft}{\kern0pt}id{\isacharparenleft}{\kern0pt}X{\isacharparenright}{\kern0pt}\ {\isasymtimes}\isactrlsub f\ y{\isacharparenright}{\kern0pt}\ {\isacharequal}{\kern0pt}\ {\isasymbeta}\isactrlbsub X\ {\isasymtimes}\isactrlsub c\ {\isasymone}\isactrlesub {\isacharparenright}{\kern0pt}{\isachardoublequoteclose}\isanewline
\ \ \ \ \isacommand{using}\isamarkupfalse%
\ terminal{\isacharunderscore}{\kern0pt}func{\isacharunderscore}{\kern0pt}unique\ uniqueness\ \isacommand{by}\isamarkupfalse%
\ blast\isanewline
\ \ \isacommand{then}\isamarkupfalse%
\ \isacommand{have}\isamarkupfalse%
\ ex{\isadigit{1}}{\isacharcolon}{\kern0pt}\ {\isachardoublequoteopen}{\isacharparenleft}{\kern0pt}{\isasymexists}{\isacharbang}{\kern0pt}\ x{\isachardot}{\kern0pt}\ x\ {\isasymin}\isactrlsub c\ {\isasymone}\isactrlbsup X\isactrlesup {\isacharparenright}{\kern0pt}{\isachardoublequoteclose}\isanewline
\ \ \ \ \isacommand{by}\isamarkupfalse%
\ {\isacharparenleft}{\kern0pt}metis\ e{\isacharunderscore}{\kern0pt}defn\ nonempty\ nonempty{\isacharunderscore}{\kern0pt}def\ transpose{\isacharunderscore}{\kern0pt}func{\isacharunderscore}{\kern0pt}unique\ uniqueness{\isacharparenright}{\kern0pt}\isanewline
\ \ \isacommand{show}\isamarkupfalse%
\ {\isachardoublequoteopen}{\isasymone}\isactrlbsup X\isactrlesup \ {\isasymcong}\ {\isasymone}{\isachardoublequoteclose}\isanewline
\ \ \ \ \isacommand{using}\isamarkupfalse%
\ ex{\isadigit{1}}\ single{\isacharunderscore}{\kern0pt}elem{\isacharunderscore}{\kern0pt}iso{\isacharunderscore}{\kern0pt}one\ \isacommand{by}\isamarkupfalse%
\ auto\isanewline
\isacommand{qed}\isamarkupfalse%
%
\endisatagproof
{\isafoldproof}%
%
\isadelimproof
%
\endisadelimproof
%
\begin{isamarkuptext}%
The lemma below corresponds to Proposition 2.5.9 in Halvorson.%
\end{isamarkuptext}\isamarkuptrue%
\isacommand{lemma}\isamarkupfalse%
\ power{\isacharunderscore}{\kern0pt}rule{\isacharcolon}{\kern0pt}\isanewline
\ \ {\isachardoublequoteopen}{\isacharparenleft}{\kern0pt}X\ {\isasymtimes}\isactrlsub c\ Y{\isacharparenright}{\kern0pt}\isactrlbsup A\isactrlesup \ {\isasymcong}\ X\isactrlbsup A\isactrlesup \ {\isasymtimes}\isactrlsub c\ Y\isactrlbsup A\isactrlesup {\isachardoublequoteclose}\isanewline
%
\isadelimproof
%
\endisadelimproof
%
\isatagproof
\isacommand{proof}\isamarkupfalse%
\ {\isacharminus}{\kern0pt}\ \isanewline
\ \ \isacommand{have}\isamarkupfalse%
\ {\isachardoublequoteopen}is{\isacharunderscore}{\kern0pt}cart{\isacharunderscore}{\kern0pt}prod\ {\isacharparenleft}{\kern0pt}{\isacharparenleft}{\kern0pt}X\ {\isasymtimes}\isactrlsub c\ Y{\isacharparenright}{\kern0pt}\isactrlbsup A\isactrlesup {\isacharparenright}{\kern0pt}\ {\isacharparenleft}{\kern0pt}{\isacharparenleft}{\kern0pt}left{\isacharunderscore}{\kern0pt}cart{\isacharunderscore}{\kern0pt}proj\ X\ Y{\isacharparenright}{\kern0pt}\isactrlbsup A\isactrlesup \isactrlsub f{\isacharparenright}{\kern0pt}\ {\isacharparenleft}{\kern0pt}right{\isacharunderscore}{\kern0pt}cart{\isacharunderscore}{\kern0pt}proj\ X\ Y\isactrlbsup A\isactrlesup \isactrlsub f{\isacharparenright}{\kern0pt}\ {\isacharparenleft}{\kern0pt}X\isactrlbsup A\isactrlesup {\isacharparenright}{\kern0pt}\ {\isacharparenleft}{\kern0pt}Y\isactrlbsup A\isactrlesup {\isacharparenright}{\kern0pt}{\isachardoublequoteclose}\isanewline
\ \ \isacommand{proof}\isamarkupfalse%
\ {\isacharparenleft}{\kern0pt}etcs{\isacharunderscore}{\kern0pt}subst\ is{\isacharunderscore}{\kern0pt}cart{\isacharunderscore}{\kern0pt}prod{\isacharunderscore}{\kern0pt}def{\isadigit{2}}{\isacharcomma}{\kern0pt}\ clarify{\isacharparenright}{\kern0pt}\isanewline
\ \ \ \ \isacommand{fix}\isamarkupfalse%
\ f\ g\ Z\ \isanewline
\ \ \ \ \isacommand{assume}\isamarkupfalse%
\ f{\isacharunderscore}{\kern0pt}type{\isacharbrackleft}{\kern0pt}type{\isacharunderscore}{\kern0pt}rule{\isacharbrackright}{\kern0pt}{\isacharcolon}{\kern0pt}\ {\isachardoublequoteopen}f\ {\isacharcolon}{\kern0pt}\ Z\ {\isasymrightarrow}\ X\isactrlbsup A\isactrlesup {\isachardoublequoteclose}\isanewline
\ \ \ \ \isacommand{assume}\isamarkupfalse%
\ g{\isacharunderscore}{\kern0pt}type{\isacharbrackleft}{\kern0pt}type{\isacharunderscore}{\kern0pt}rule{\isacharbrackright}{\kern0pt}{\isacharcolon}{\kern0pt}\ {\isachardoublequoteopen}g\ {\isacharcolon}{\kern0pt}\ Z\ {\isasymrightarrow}\ Y\isactrlbsup A\isactrlesup {\isachardoublequoteclose}\isanewline
\isanewline
\ \ \ \ \isacommand{show}\isamarkupfalse%
\ {\isachardoublequoteopen}{\isasymexists}h{\isachardot}{\kern0pt}\ h\ {\isacharcolon}{\kern0pt}\ Z\ {\isasymrightarrow}\ {\isacharparenleft}{\kern0pt}X\ {\isasymtimes}\isactrlsub c\ Y{\isacharparenright}{\kern0pt}\isactrlbsup A\isactrlesup \ {\isasymand}\isanewline
\ \ \ \ \ \ \ \ \ \ \ left{\isacharunderscore}{\kern0pt}cart{\isacharunderscore}{\kern0pt}proj\ X\ Y\isactrlbsup A\isactrlesup \isactrlsub f\ {\isasymcirc}\isactrlsub c\ h\ {\isacharequal}{\kern0pt}\ f\ {\isasymand}\isanewline
\ \ \ \ \ \ \ \ \ \ \ right{\isacharunderscore}{\kern0pt}cart{\isacharunderscore}{\kern0pt}proj\ X\ Y\isactrlbsup A\isactrlesup \isactrlsub f\ {\isasymcirc}\isactrlsub c\ h\ {\isacharequal}{\kern0pt}\ g\ {\isasymand}\isanewline
\ \ \ \ \ \ \ \ \ \ \ {\isacharparenleft}{\kern0pt}{\isasymforall}h{\isadigit{2}}{\isachardot}{\kern0pt}\ h{\isadigit{2}}\ {\isacharcolon}{\kern0pt}\ Z\ {\isasymrightarrow}\ {\isacharparenleft}{\kern0pt}X\ {\isasymtimes}\isactrlsub c\ Y{\isacharparenright}{\kern0pt}\isactrlbsup A\isactrlesup \ {\isasymand}\ left{\isacharunderscore}{\kern0pt}cart{\isacharunderscore}{\kern0pt}proj\ X\ Y\isactrlbsup A\isactrlesup \isactrlsub f\ {\isasymcirc}\isactrlsub c\ h{\isadigit{2}}\ {\isacharequal}{\kern0pt}\ f\ {\isasymand}\ right{\isacharunderscore}{\kern0pt}cart{\isacharunderscore}{\kern0pt}proj\ X\ Y\isactrlbsup A\isactrlesup \isactrlsub f\ {\isasymcirc}\isactrlsub c\ h{\isadigit{2}}\ {\isacharequal}{\kern0pt}\ g\ {\isasymlongrightarrow}\isanewline
\ \ \ \ \ \ \ \ \ \ \ \ \ \ \ \ \ h{\isadigit{2}}\ {\isacharequal}{\kern0pt}\ h{\isacharparenright}{\kern0pt}{\isachardoublequoteclose}\isanewline
\ \ \ \ \isacommand{proof}\isamarkupfalse%
\ {\isacharparenleft}{\kern0pt}rule{\isacharunderscore}{\kern0pt}tac\ x{\isacharequal}{\kern0pt}{\isachardoublequoteopen}{\isasymlangle}f\isactrlsup {\isasymflat}\ {\isacharcomma}{\kern0pt}g\isactrlsup {\isasymflat}{\isasymrangle}\isactrlsup {\isasymsharp}{\isachardoublequoteclose}\ \isakeyword{in}\ exI{\isacharcomma}{\kern0pt}\ safe{\isacharcomma}{\kern0pt}\ typecheck{\isacharunderscore}{\kern0pt}cfuncs{\isacharparenright}{\kern0pt}\isanewline
\ \ \ \ \ \ \isacommand{have}\isamarkupfalse%
\ {\isachardoublequoteopen}{\isacharparenleft}{\kern0pt}{\isacharparenleft}{\kern0pt}left{\isacharunderscore}{\kern0pt}cart{\isacharunderscore}{\kern0pt}proj\ X\ Y{\isacharparenright}{\kern0pt}\isactrlbsup A\isactrlesup \isactrlsub f{\isacharparenright}{\kern0pt}\ {\isasymcirc}\isactrlsub c\ {\isasymlangle}f\isactrlsup {\isasymflat}\ {\isacharcomma}{\kern0pt}g\isactrlsup {\isasymflat}{\isasymrangle}\isactrlsup {\isasymsharp}\ {\isacharequal}{\kern0pt}\ {\isacharparenleft}{\kern0pt}{\isacharparenleft}{\kern0pt}left{\isacharunderscore}{\kern0pt}cart{\isacharunderscore}{\kern0pt}proj\ X\ Y{\isacharparenright}{\kern0pt}\ {\isasymcirc}\isactrlsub c\ {\isasymlangle}f\isactrlsup {\isasymflat}\ {\isacharcomma}{\kern0pt}g\isactrlsup {\isasymflat}{\isasymrangle}{\isacharparenright}{\kern0pt}\isactrlsup {\isasymsharp}{\isachardoublequoteclose}\isanewline
\ \ \ \ \ \ \ \ \isacommand{by}\isamarkupfalse%
\ {\isacharparenleft}{\kern0pt}typecheck{\isacharunderscore}{\kern0pt}cfuncs{\isacharcomma}{\kern0pt}\ metis\ transpose{\isacharunderscore}{\kern0pt}of{\isacharunderscore}{\kern0pt}comp{\isacharparenright}{\kern0pt}\isanewline
\ \ \ \ \ \ \isacommand{also}\isamarkupfalse%
\ \isacommand{have}\isamarkupfalse%
\ {\isachardoublequoteopen}{\isachardot}{\kern0pt}{\isachardot}{\kern0pt}{\isachardot}{\kern0pt}\ {\isacharequal}{\kern0pt}\ f\isactrlsup {\isasymflat}\isactrlsup {\isasymsharp}{\isachardoublequoteclose}\isanewline
\ \ \ \ \ \ \ \ \isacommand{by}\isamarkupfalse%
\ {\isacharparenleft}{\kern0pt}typecheck{\isacharunderscore}{\kern0pt}cfuncs{\isacharcomma}{\kern0pt}\ simp\ add{\isacharcolon}{\kern0pt}\ left{\isacharunderscore}{\kern0pt}cart{\isacharunderscore}{\kern0pt}proj{\isacharunderscore}{\kern0pt}cfunc{\isacharunderscore}{\kern0pt}prod{\isacharparenright}{\kern0pt}\isanewline
\ \ \ \ \ \ \isacommand{also}\isamarkupfalse%
\ \isacommand{have}\isamarkupfalse%
\ {\isachardoublequoteopen}{\isachardot}{\kern0pt}{\isachardot}{\kern0pt}{\isachardot}{\kern0pt}\ {\isacharequal}{\kern0pt}\ f{\isachardoublequoteclose}\isanewline
\ \ \ \ \ \ \ \ \isacommand{by}\isamarkupfalse%
\ {\isacharparenleft}{\kern0pt}typecheck{\isacharunderscore}{\kern0pt}cfuncs{\isacharcomma}{\kern0pt}\ simp\ add{\isacharcolon}{\kern0pt}\ sharp{\isacharunderscore}{\kern0pt}cancels{\isacharunderscore}{\kern0pt}flat{\isacharparenright}{\kern0pt}\isanewline
\ \ \ \ \ \ \isacommand{then}\isamarkupfalse%
\ \isacommand{show}\isamarkupfalse%
\ projection{\isacharunderscore}{\kern0pt}property{\isadigit{1}}{\isacharcolon}{\kern0pt}\ {\isachardoublequoteopen}{\isacharparenleft}{\kern0pt}{\isacharparenleft}{\kern0pt}left{\isacharunderscore}{\kern0pt}cart{\isacharunderscore}{\kern0pt}proj\ X\ Y{\isacharparenright}{\kern0pt}\isactrlbsup A\isactrlesup \isactrlsub f{\isacharparenright}{\kern0pt}\ {\isasymcirc}\isactrlsub c\ {\isasymlangle}f\isactrlsup {\isasymflat}\ {\isacharcomma}{\kern0pt}g\isactrlsup {\isasymflat}{\isasymrangle}\isactrlsup {\isasymsharp}\ {\isacharequal}{\kern0pt}\ f{\isachardoublequoteclose}\isanewline
\ \ \ \ \ \ \ \ \isacommand{by}\isamarkupfalse%
\ {\isacharparenleft}{\kern0pt}simp\ add{\isacharcolon}{\kern0pt}\ calculation{\isacharparenright}{\kern0pt}\isanewline
\ \ \ \ \ \ \isacommand{show}\isamarkupfalse%
\ projection{\isacharunderscore}{\kern0pt}property{\isadigit{2}}{\isacharcolon}{\kern0pt}\ {\isachardoublequoteopen}{\isacharparenleft}{\kern0pt}{\isacharparenleft}{\kern0pt}right{\isacharunderscore}{\kern0pt}cart{\isacharunderscore}{\kern0pt}proj\ X\ Y{\isacharparenright}{\kern0pt}\isactrlbsup A\isactrlesup \isactrlsub f{\isacharparenright}{\kern0pt}\ {\isasymcirc}\isactrlsub c\ {\isasymlangle}f\isactrlsup {\isasymflat}\ {\isacharcomma}{\kern0pt}g\isactrlsup {\isasymflat}{\isasymrangle}\isactrlsup {\isasymsharp}\ {\isacharequal}{\kern0pt}\ g{\isachardoublequoteclose}\isanewline
\ \ \ \ \ \ \ \ \isacommand{by}\isamarkupfalse%
\ {\isacharparenleft}{\kern0pt}typecheck{\isacharunderscore}{\kern0pt}cfuncs{\isacharcomma}{\kern0pt}\ metis\ right{\isacharunderscore}{\kern0pt}cart{\isacharunderscore}{\kern0pt}proj{\isacharunderscore}{\kern0pt}cfunc{\isacharunderscore}{\kern0pt}prod\ sharp{\isacharunderscore}{\kern0pt}cancels{\isacharunderscore}{\kern0pt}flat\ transpose{\isacharunderscore}{\kern0pt}of{\isacharunderscore}{\kern0pt}comp{\isacharparenright}{\kern0pt}\isanewline
\ \ \ \ \ \ \isacommand{show}\isamarkupfalse%
\ {\isachardoublequoteopen}{\isasymAnd}h{\isadigit{2}}{\isachardot}{\kern0pt}\ h{\isadigit{2}}\ {\isacharcolon}{\kern0pt}\ Z\ {\isasymrightarrow}\ {\isacharparenleft}{\kern0pt}X\ {\isasymtimes}\isactrlsub c\ Y{\isacharparenright}{\kern0pt}\isactrlbsup A\isactrlesup \ {\isasymLongrightarrow}\isanewline
\ \ \ \ \ \ \ \ \ \ f\ {\isacharequal}{\kern0pt}\ left{\isacharunderscore}{\kern0pt}cart{\isacharunderscore}{\kern0pt}proj\ X\ Y\isactrlbsup A\isactrlesup \isactrlsub f\ {\isasymcirc}\isactrlsub c\ h{\isadigit{2}}\ {\isasymLongrightarrow}\isanewline
\ \ \ \ \ \ \ \ \ \ g\ {\isacharequal}{\kern0pt}\ right{\isacharunderscore}{\kern0pt}cart{\isacharunderscore}{\kern0pt}proj\ X\ Y\isactrlbsup A\isactrlesup \isactrlsub f\ {\isasymcirc}\isactrlsub c\ h{\isadigit{2}}\ {\isasymLongrightarrow}\isanewline
\ \ \ \ \ \ \ \ \ \ h{\isadigit{2}}\ {\isacharequal}{\kern0pt}\ {\isasymlangle}{\isacharparenleft}{\kern0pt}left{\isacharunderscore}{\kern0pt}cart{\isacharunderscore}{\kern0pt}proj\ X\ Y\isactrlbsup A\isactrlesup \isactrlsub f\ {\isasymcirc}\isactrlsub c\ h{\isadigit{2}}{\isacharparenright}{\kern0pt}\isactrlsup {\isasymflat}{\isacharcomma}{\kern0pt}{\isacharparenleft}{\kern0pt}right{\isacharunderscore}{\kern0pt}cart{\isacharunderscore}{\kern0pt}proj\ X\ Y\isactrlbsup A\isactrlesup \isactrlsub f\ {\isasymcirc}\isactrlsub c\ h{\isadigit{2}}{\isacharparenright}{\kern0pt}\isactrlsup {\isasymflat}{\isasymrangle}\isactrlsup {\isasymsharp}{\isachardoublequoteclose}\isanewline
\ \ \ \ \ \ \isacommand{proof}\isamarkupfalse%
\ {\isacharminus}{\kern0pt}\isanewline
\ \ \ \ \ \ \ \ \isacommand{fix}\isamarkupfalse%
\ h\isanewline
\ \ \ \ \ \ \ \ \isacommand{assume}\isamarkupfalse%
\ h{\isacharunderscore}{\kern0pt}type{\isacharbrackleft}{\kern0pt}type{\isacharunderscore}{\kern0pt}rule{\isacharbrackright}{\kern0pt}{\isacharcolon}{\kern0pt}\ {\isachardoublequoteopen}h\ {\isacharcolon}{\kern0pt}\ Z\ {\isasymrightarrow}\ {\isacharparenleft}{\kern0pt}X\ {\isasymtimes}\isactrlsub c\ Y{\isacharparenright}{\kern0pt}\isactrlbsup A\isactrlesup {\isachardoublequoteclose}\isanewline
\ \ \ \ \ \ \ \ \isacommand{assume}\isamarkupfalse%
\ h{\isacharunderscore}{\kern0pt}property{\isadigit{1}}{\isacharcolon}{\kern0pt}\ \ {\isachardoublequoteopen}f\ {\isacharequal}{\kern0pt}\ {\isacharparenleft}{\kern0pt}{\isacharparenleft}{\kern0pt}left{\isacharunderscore}{\kern0pt}cart{\isacharunderscore}{\kern0pt}proj\ X\ Y{\isacharparenright}{\kern0pt}\isactrlbsup A\isactrlesup \isactrlsub f{\isacharparenright}{\kern0pt}\ {\isasymcirc}\isactrlsub c\ h{\isachardoublequoteclose}\isanewline
\ \ \ \ \ \ \ \ \isacommand{assume}\isamarkupfalse%
\ h{\isacharunderscore}{\kern0pt}property{\isadigit{2}}{\isacharcolon}{\kern0pt}\ \ {\isachardoublequoteopen}g\ {\isacharequal}{\kern0pt}\ {\isacharparenleft}{\kern0pt}{\isacharparenleft}{\kern0pt}right{\isacharunderscore}{\kern0pt}cart{\isacharunderscore}{\kern0pt}proj\ X\ Y{\isacharparenright}{\kern0pt}\isactrlbsup A\isactrlesup \isactrlsub f{\isacharparenright}{\kern0pt}\ {\isasymcirc}\isactrlsub c\ h{\isachardoublequoteclose}\isanewline
\ \ \ \ \isanewline
\ \ \ \ \ \ \ \ \isacommand{have}\isamarkupfalse%
\ {\isachardoublequoteopen}f\ {\isacharequal}{\kern0pt}\ {\isacharparenleft}{\kern0pt}left{\isacharunderscore}{\kern0pt}cart{\isacharunderscore}{\kern0pt}proj\ X\ Y{\isacharparenright}{\kern0pt}\isactrlbsup A\isactrlesup \isactrlsub f\ {\isasymcirc}\isactrlsub c\ h\isactrlsup {\isasymflat}\isactrlsup {\isasymsharp}{\isachardoublequoteclose}\isanewline
\ \ \ \ \ \ \ \ \ \ \isacommand{by}\isamarkupfalse%
\ {\isacharparenleft}{\kern0pt}metis\ \ h{\isacharunderscore}{\kern0pt}property{\isadigit{1}}\ h{\isacharunderscore}{\kern0pt}type\ sharp{\isacharunderscore}{\kern0pt}cancels{\isacharunderscore}{\kern0pt}flat{\isacharparenright}{\kern0pt}\isanewline
\ \ \ \ \ \ \ \ \isacommand{also}\isamarkupfalse%
\ \isacommand{have}\isamarkupfalse%
\ {\isachardoublequoteopen}{\isachardot}{\kern0pt}{\isachardot}{\kern0pt}{\isachardot}{\kern0pt}\ {\isacharequal}{\kern0pt}\ {\isacharparenleft}{\kern0pt}{\isacharparenleft}{\kern0pt}left{\isacharunderscore}{\kern0pt}cart{\isacharunderscore}{\kern0pt}proj\ X\ Y{\isacharparenright}{\kern0pt}\ {\isasymcirc}\isactrlsub c\ h\isactrlsup {\isasymflat}{\isacharparenright}{\kern0pt}\isactrlsup {\isasymsharp}{\isachardoublequoteclose}\isanewline
\ \ \ \ \ \ \ \ \ \ \isacommand{by}\isamarkupfalse%
\ {\isacharparenleft}{\kern0pt}typecheck{\isacharunderscore}{\kern0pt}cfuncs{\isacharcomma}{\kern0pt}\ simp\ add{\isacharcolon}{\kern0pt}\ transpose{\isacharunderscore}{\kern0pt}of{\isacharunderscore}{\kern0pt}comp{\isacharparenright}{\kern0pt}\isanewline
\ \ \ \ \ \ \ \ \isacommand{have}\isamarkupfalse%
\ computation{\isadigit{1}}{\isacharcolon}{\kern0pt}\ {\isachardoublequoteopen}f\ {\isacharequal}{\kern0pt}\ {\isacharparenleft}{\kern0pt}{\isacharparenleft}{\kern0pt}left{\isacharunderscore}{\kern0pt}cart{\isacharunderscore}{\kern0pt}proj\ X\ Y{\isacharparenright}{\kern0pt}\ {\isasymcirc}\isactrlsub c\ h\isactrlsup {\isasymflat}{\isacharparenright}{\kern0pt}\isactrlsup {\isasymsharp}{\isachardoublequoteclose}\isanewline
\ \ \ \ \ \ \ \ \ \ \isacommand{by}\isamarkupfalse%
\ {\isacharparenleft}{\kern0pt}simp\ add{\isacharcolon}{\kern0pt}\ {\isacartoucheopen}left{\isacharunderscore}{\kern0pt}cart{\isacharunderscore}{\kern0pt}proj\ X\ Y\isactrlbsup A\isactrlesup \isactrlsub f\ {\isasymcirc}\isactrlsub c\ h\isactrlsup {\isasymflat}\isactrlsup {\isasymsharp}\ {\isacharequal}{\kern0pt}\ {\isacharparenleft}{\kern0pt}left{\isacharunderscore}{\kern0pt}cart{\isacharunderscore}{\kern0pt}proj\ X\ Y\ {\isasymcirc}\isactrlsub c\ h\isactrlsup {\isasymflat}{\isacharparenright}{\kern0pt}\isactrlsup {\isasymsharp}{\isacartoucheclose}\ calculation{\isacharparenright}{\kern0pt}\isanewline
\ \ \ \ \ \ \ \ \isacommand{then}\isamarkupfalse%
\ \isacommand{have}\isamarkupfalse%
\ unqiueness{\isadigit{1}}{\isacharcolon}{\kern0pt}\ {\isachardoublequoteopen}{\isacharparenleft}{\kern0pt}left{\isacharunderscore}{\kern0pt}cart{\isacharunderscore}{\kern0pt}proj\ X\ Y{\isacharparenright}{\kern0pt}\ {\isasymcirc}\isactrlsub c\ \ h\isactrlsup {\isasymflat}\ {\isacharequal}{\kern0pt}\ \ f\isactrlsup {\isasymflat}{\isachardoublequoteclose}\isanewline
\ \ \ \ \ \ \ \ \ \ \isacommand{using}\isamarkupfalse%
\ h{\isacharunderscore}{\kern0pt}type\ f{\isacharunderscore}{\kern0pt}type\ \isacommand{by}\isamarkupfalse%
\ {\isacharparenleft}{\kern0pt}typecheck{\isacharunderscore}{\kern0pt}cfuncs{\isacharcomma}{\kern0pt}\ simp\ add{\isacharcolon}{\kern0pt}\ computation{\isadigit{1}}\ flat{\isacharunderscore}{\kern0pt}cancels{\isacharunderscore}{\kern0pt}sharp{\isacharparenright}{\kern0pt}\isanewline
\ \ \ \ \ \ \ \ \isacommand{have}\isamarkupfalse%
\ \ \ {\isachardoublequoteopen}g\ {\isacharequal}{\kern0pt}\ {\isacharparenleft}{\kern0pt}{\isacharparenleft}{\kern0pt}right{\isacharunderscore}{\kern0pt}cart{\isacharunderscore}{\kern0pt}proj\ X\ Y{\isacharparenright}{\kern0pt}\isactrlbsup A\isactrlesup \isactrlsub f{\isacharparenright}{\kern0pt}\ {\isasymcirc}\isactrlsub c\ {\isacharparenleft}{\kern0pt}h\isactrlsup {\isasymflat}{\isacharparenright}{\kern0pt}\isactrlsup {\isasymsharp}{\isachardoublequoteclose}\isanewline
\ \ \ \ \ \ \ \ \ \ \isacommand{by}\isamarkupfalse%
\ {\isacharparenleft}{\kern0pt}metis\ \ h{\isacharunderscore}{\kern0pt}property{\isadigit{2}}\ h{\isacharunderscore}{\kern0pt}type\ sharp{\isacharunderscore}{\kern0pt}cancels{\isacharunderscore}{\kern0pt}flat{\isacharparenright}{\kern0pt}\isanewline
\ \ \ \ \ \ \ \ \isacommand{have}\isamarkupfalse%
\ {\isachardoublequoteopen}{\isachardot}{\kern0pt}{\isachardot}{\kern0pt}{\isachardot}{\kern0pt}\ {\isacharequal}{\kern0pt}\ {\isacharparenleft}{\kern0pt}{\isacharparenleft}{\kern0pt}right{\isacharunderscore}{\kern0pt}cart{\isacharunderscore}{\kern0pt}proj\ X\ Y{\isacharparenright}{\kern0pt}\ {\isasymcirc}\isactrlsub c\ h\isactrlsup {\isasymflat}{\isacharparenright}{\kern0pt}\isactrlsup {\isasymsharp}{\isachardoublequoteclose}\isanewline
\ \ \ \ \ \ \ \ \ \ \isacommand{by}\isamarkupfalse%
\ {\isacharparenleft}{\kern0pt}typecheck{\isacharunderscore}{\kern0pt}cfuncs{\isacharcomma}{\kern0pt}\ metis\ transpose{\isacharunderscore}{\kern0pt}of{\isacharunderscore}{\kern0pt}comp{\isacharparenright}{\kern0pt}\isanewline
\ \ \ \ \ \ \ \ \isacommand{have}\isamarkupfalse%
\ computation{\isadigit{2}}{\isacharcolon}{\kern0pt}\ {\isachardoublequoteopen}g\ {\isacharequal}{\kern0pt}\ {\isacharparenleft}{\kern0pt}{\isacharparenleft}{\kern0pt}right{\isacharunderscore}{\kern0pt}cart{\isacharunderscore}{\kern0pt}proj\ X\ Y{\isacharparenright}{\kern0pt}\ {\isasymcirc}\isactrlsub c\ h\isactrlsup {\isasymflat}{\isacharparenright}{\kern0pt}\isactrlsup {\isasymsharp}{\isachardoublequoteclose}\isanewline
\ \ \ \ \ \ \ \ \ \ \ \isacommand{by}\isamarkupfalse%
\ {\isacharparenleft}{\kern0pt}simp\ add{\isacharcolon}{\kern0pt}\ {\isacartoucheopen}g\ {\isacharequal}{\kern0pt}\ right{\isacharunderscore}{\kern0pt}cart{\isacharunderscore}{\kern0pt}proj\ X\ Y\isactrlbsup A\isactrlesup \isactrlsub f\ {\isasymcirc}\isactrlsub c\ h\isactrlsup {\isasymflat}\isactrlsup {\isasymsharp}{\isacartoucheclose}\ {\isacartoucheopen}right{\isacharunderscore}{\kern0pt}cart{\isacharunderscore}{\kern0pt}proj\ X\ Y\isactrlbsup A\isactrlesup \isactrlsub f\ {\isasymcirc}\isactrlsub c\ h\isactrlsup {\isasymflat}\isactrlsup {\isasymsharp}\ {\isacharequal}{\kern0pt}\ {\isacharparenleft}{\kern0pt}right{\isacharunderscore}{\kern0pt}cart{\isacharunderscore}{\kern0pt}proj\ X\ Y\ {\isasymcirc}\isactrlsub c\ h\isactrlsup {\isasymflat}{\isacharparenright}{\kern0pt}\isactrlsup {\isasymsharp}{\isacartoucheclose}{\isacharparenright}{\kern0pt}\isanewline
\ \ \ \ \ \ \ \ \isacommand{then}\isamarkupfalse%
\ \isacommand{have}\isamarkupfalse%
\ unqiueness{\isadigit{2}}{\isacharcolon}{\kern0pt}\ {\isachardoublequoteopen}{\isacharparenleft}{\kern0pt}right{\isacharunderscore}{\kern0pt}cart{\isacharunderscore}{\kern0pt}proj\ X\ Y{\isacharparenright}{\kern0pt}\ {\isasymcirc}\isactrlsub c\ \ h\isactrlsup {\isasymflat}\ {\isacharequal}{\kern0pt}\ \ g\isactrlsup {\isasymflat}{\isachardoublequoteclose}\isanewline
\ \ \ \ \ \ \ \ \ \ \isacommand{using}\isamarkupfalse%
\ h{\isacharunderscore}{\kern0pt}type\ g{\isacharunderscore}{\kern0pt}type\ \isacommand{by}\isamarkupfalse%
\ {\isacharparenleft}{\kern0pt}typecheck{\isacharunderscore}{\kern0pt}cfuncs{\isacharcomma}{\kern0pt}\ simp\ add{\isacharcolon}{\kern0pt}\ computation{\isadigit{2}}\ flat{\isacharunderscore}{\kern0pt}cancels{\isacharunderscore}{\kern0pt}sharp{\isacharparenright}{\kern0pt}\isanewline
\ \ \ \ \ \ \ \ \isacommand{then}\isamarkupfalse%
\ \isacommand{have}\isamarkupfalse%
\ h{\isacharunderscore}{\kern0pt}flat{\isacharcolon}{\kern0pt}\ {\isachardoublequoteopen}h\isactrlsup {\isasymflat}\ {\isacharequal}{\kern0pt}\ {\isasymlangle}f\isactrlsup {\isasymflat}{\isacharcomma}{\kern0pt}\ g\isactrlsup {\isasymflat}{\isasymrangle}{\isachardoublequoteclose}\isanewline
\ \ \ \ \ \ \ \ \ \ \isacommand{by}\isamarkupfalse%
\ {\isacharparenleft}{\kern0pt}typecheck{\isacharunderscore}{\kern0pt}cfuncs{\isacharcomma}{\kern0pt}\ simp\ add{\isacharcolon}{\kern0pt}\ cfunc{\isacharunderscore}{\kern0pt}prod{\isacharunderscore}{\kern0pt}unique\ unqiueness{\isadigit{1}}\ unqiueness{\isadigit{2}}{\isacharparenright}{\kern0pt}\isanewline
\ \ \ \ \ \ \ \ \isacommand{then}\isamarkupfalse%
\ \isacommand{have}\isamarkupfalse%
\ h{\isacharunderscore}{\kern0pt}is{\isacharunderscore}{\kern0pt}sharp{\isacharunderscore}{\kern0pt}prod{\isacharunderscore}{\kern0pt}fflat{\isacharunderscore}{\kern0pt}gflat{\isacharcolon}{\kern0pt}\ {\isachardoublequoteopen}h\ {\isacharequal}{\kern0pt}\ {\isasymlangle}f\isactrlsup {\isasymflat}{\isacharcomma}{\kern0pt}\ g\isactrlsup {\isasymflat}{\isasymrangle}\isactrlsup {\isasymsharp}{\isachardoublequoteclose}\isanewline
\ \ \ \ \ \ \ \ \ \ \isacommand{by}\isamarkupfalse%
\ {\isacharparenleft}{\kern0pt}metis\ \ h{\isacharunderscore}{\kern0pt}type\ sharp{\isacharunderscore}{\kern0pt}cancels{\isacharunderscore}{\kern0pt}flat{\isacharparenright}{\kern0pt}\isanewline
\ \ \ \ \ \ \ \ \isacommand{then}\isamarkupfalse%
\ \isacommand{show}\isamarkupfalse%
\ {\isachardoublequoteopen}h\ {\isacharequal}{\kern0pt}\ {\isasymlangle}{\isacharparenleft}{\kern0pt}left{\isacharunderscore}{\kern0pt}cart{\isacharunderscore}{\kern0pt}proj\ X\ Y\isactrlbsup A\isactrlesup \isactrlsub f\ {\isasymcirc}\isactrlsub c\ h{\isacharparenright}{\kern0pt}\isactrlsup {\isasymflat}{\isacharcomma}{\kern0pt}{\isacharparenleft}{\kern0pt}right{\isacharunderscore}{\kern0pt}cart{\isacharunderscore}{\kern0pt}proj\ X\ Y\isactrlbsup A\isactrlesup \isactrlsub f\ {\isasymcirc}\isactrlsub c\ h{\isacharparenright}{\kern0pt}\isactrlsup {\isasymflat}{\isasymrangle}\isactrlsup {\isasymsharp}{\isachardoublequoteclose}\isanewline
\ \ \ \ \ \ \ \ \ \ \isacommand{using}\isamarkupfalse%
\ h{\isacharunderscore}{\kern0pt}property{\isadigit{1}}\ h{\isacharunderscore}{\kern0pt}property{\isadigit{2}}\ \isacommand{by}\isamarkupfalse%
\ force\isanewline
\ \ \ \ \ \ \isacommand{qed}\isamarkupfalse%
\isanewline
\ \ \ \ \isacommand{qed}\isamarkupfalse%
\isanewline
\ \ \isacommand{qed}\isamarkupfalse%
\isanewline
\ \ \isacommand{then}\isamarkupfalse%
\ \isacommand{show}\isamarkupfalse%
\ {\isachardoublequoteopen}{\isacharparenleft}{\kern0pt}X\ {\isasymtimes}\isactrlsub c\ Y{\isacharparenright}{\kern0pt}\isactrlbsup A\isactrlesup \ {\isasymcong}\ X\isactrlbsup A\isactrlesup \ {\isasymtimes}\isactrlsub c\ Y\isactrlbsup A\isactrlesup {\isachardoublequoteclose}\isanewline
\ \ \ \ \isacommand{using}\isamarkupfalse%
\ canonical{\isacharunderscore}{\kern0pt}cart{\isacharunderscore}{\kern0pt}prod{\isacharunderscore}{\kern0pt}is{\isacharunderscore}{\kern0pt}cart{\isacharunderscore}{\kern0pt}prod\ cart{\isacharunderscore}{\kern0pt}prods{\isacharunderscore}{\kern0pt}isomorphic\ fst{\isacharunderscore}{\kern0pt}conv\ is{\isacharunderscore}{\kern0pt}isomorphic{\isacharunderscore}{\kern0pt}def\ \isacommand{by}\isamarkupfalse%
\ fastforce\isanewline
\isacommand{qed}\isamarkupfalse%
%
\endisatagproof
{\isafoldproof}%
%
\isadelimproof
\isanewline
%
\endisadelimproof
\isanewline
\isacommand{lemma}\isamarkupfalse%
\ exponential{\isacharunderscore}{\kern0pt}coprod{\isacharunderscore}{\kern0pt}distribution{\isacharcolon}{\kern0pt}\isanewline
\ \ {\isachardoublequoteopen}Z\isactrlbsup {\isacharparenleft}{\kern0pt}X\ {\isasymCoprod}\ Y{\isacharparenright}{\kern0pt}\isactrlesup \ {\isasymcong}\ {\isacharparenleft}{\kern0pt}Z\isactrlbsup X\isactrlesup {\isacharparenright}{\kern0pt}\ {\isasymtimes}\isactrlsub c\ {\isacharparenleft}{\kern0pt}Z\isactrlbsup Y\isactrlesup {\isacharparenright}{\kern0pt}{\isachardoublequoteclose}\isanewline
%
\isadelimproof
%
\endisadelimproof
%
\isatagproof
\isacommand{proof}\isamarkupfalse%
\ {\isacharminus}{\kern0pt}\ \isanewline
\ \ \isacommand{have}\isamarkupfalse%
\ {\isachardoublequoteopen}is{\isacharunderscore}{\kern0pt}cart{\isacharunderscore}{\kern0pt}prod\ {\isacharparenleft}{\kern0pt}Z\isactrlbsup {\isacharparenleft}{\kern0pt}X\ {\isasymCoprod}\ Y{\isacharparenright}{\kern0pt}\isactrlesup {\isacharparenright}{\kern0pt}\ {\isacharparenleft}{\kern0pt}{\isacharparenleft}{\kern0pt}eval{\isacharunderscore}{\kern0pt}func\ Z\ {\isacharparenleft}{\kern0pt}X\ {\isasymCoprod}\ Y{\isacharparenright}{\kern0pt}\ {\isasymcirc}\isactrlsub c\ {\isacharparenleft}{\kern0pt}left{\isacharunderscore}{\kern0pt}coproj\ X\ Y{\isacharparenright}{\kern0pt}\ {\isasymtimes}\isactrlsub f\ {\isacharparenleft}{\kern0pt}id{\isacharparenleft}{\kern0pt}Z\isactrlbsup {\isacharparenleft}{\kern0pt}X\ {\isasymCoprod}\ Y{\isacharparenright}{\kern0pt}\isactrlesup {\isacharparenright}{\kern0pt}{\isacharparenright}{\kern0pt}\ {\isacharparenright}{\kern0pt}\isactrlsup {\isasymsharp}{\isacharparenright}{\kern0pt}\ {\isacharparenleft}{\kern0pt}{\isacharparenleft}{\kern0pt}eval{\isacharunderscore}{\kern0pt}func\ Z\ {\isacharparenleft}{\kern0pt}X\ {\isasymCoprod}\ Y{\isacharparenright}{\kern0pt}\ {\isasymcirc}\isactrlsub c\ {\isacharparenleft}{\kern0pt}right{\isacharunderscore}{\kern0pt}coproj\ X\ Y{\isacharparenright}{\kern0pt}\ {\isasymtimes}\isactrlsub f\ {\isacharparenleft}{\kern0pt}id{\isacharparenleft}{\kern0pt}Z\isactrlbsup {\isacharparenleft}{\kern0pt}X\ {\isasymCoprod}\ Y{\isacharparenright}{\kern0pt}\isactrlesup {\isacharparenright}{\kern0pt}{\isacharparenright}{\kern0pt}\ {\isacharparenright}{\kern0pt}\isactrlsup {\isasymsharp}{\isacharparenright}{\kern0pt}\ {\isacharparenleft}{\kern0pt}Z\isactrlbsup X\isactrlesup {\isacharparenright}{\kern0pt}\ {\isacharparenleft}{\kern0pt}Z\isactrlbsup Y\isactrlesup {\isacharparenright}{\kern0pt}{\isachardoublequoteclose}\isanewline
\ \ \isacommand{proof}\isamarkupfalse%
\ {\isacharparenleft}{\kern0pt}etcs{\isacharunderscore}{\kern0pt}subst\ is{\isacharunderscore}{\kern0pt}cart{\isacharunderscore}{\kern0pt}prod{\isacharunderscore}{\kern0pt}def{\isadigit{2}}{\isacharcomma}{\kern0pt}\ clarify{\isacharparenright}{\kern0pt}\isanewline
\ \ \ \ \isacommand{fix}\isamarkupfalse%
\ f\ g\ H\isanewline
\ \ \ \ \isacommand{assume}\isamarkupfalse%
\ f{\isacharunderscore}{\kern0pt}type{\isacharbrackleft}{\kern0pt}type{\isacharunderscore}{\kern0pt}rule{\isacharbrackright}{\kern0pt}{\isacharcolon}{\kern0pt}\ {\isachardoublequoteopen}f\ {\isacharcolon}{\kern0pt}\ H\ {\isasymrightarrow}\ Z\isactrlbsup X\isactrlesup {\isachardoublequoteclose}\isanewline
\ \ \ \ \isacommand{assume}\isamarkupfalse%
\ g{\isacharunderscore}{\kern0pt}type{\isacharbrackleft}{\kern0pt}type{\isacharunderscore}{\kern0pt}rule{\isacharbrackright}{\kern0pt}{\isacharcolon}{\kern0pt}\ {\isachardoublequoteopen}g\ {\isacharcolon}{\kern0pt}\ H\ {\isasymrightarrow}\ Z\isactrlbsup Y\isactrlesup {\isachardoublequoteclose}\isanewline
\ \ \ \ \isacommand{show}\isamarkupfalse%
\ {\isachardoublequoteopen}{\isasymexists}h{\isachardot}{\kern0pt}\ h\ {\isacharcolon}{\kern0pt}\ H\ {\isasymrightarrow}\ Z\isactrlbsup {\isacharparenleft}{\kern0pt}X\ {\isasymCoprod}\ Y{\isacharparenright}{\kern0pt}\isactrlesup \ {\isasymand}\isanewline
\ \ \ \ \ \ \ \ \ \ \ {\isacharparenleft}{\kern0pt}eval{\isacharunderscore}{\kern0pt}func\ Z\ {\isacharparenleft}{\kern0pt}X\ {\isasymCoprod}\ Y{\isacharparenright}{\kern0pt}\ {\isasymcirc}\isactrlsub c\ left{\isacharunderscore}{\kern0pt}coproj\ X\ Y\ {\isasymtimes}\isactrlsub f\ id\isactrlsub c\ {\isacharparenleft}{\kern0pt}Z\isactrlbsup {\isacharparenleft}{\kern0pt}X\ {\isasymCoprod}\ Y{\isacharparenright}{\kern0pt}\isactrlesup {\isacharparenright}{\kern0pt}{\isacharparenright}{\kern0pt}\isactrlsup {\isasymsharp}\ {\isasymcirc}\isactrlsub c\ h\ {\isacharequal}{\kern0pt}\ f\ {\isasymand}\isanewline
\ \ \ \ \ \ \ \ \ \ \ {\isacharparenleft}{\kern0pt}eval{\isacharunderscore}{\kern0pt}func\ Z\ {\isacharparenleft}{\kern0pt}X\ {\isasymCoprod}\ Y{\isacharparenright}{\kern0pt}\ {\isasymcirc}\isactrlsub c\ right{\isacharunderscore}{\kern0pt}coproj\ X\ Y\ {\isasymtimes}\isactrlsub f\ id\isactrlsub c\ {\isacharparenleft}{\kern0pt}Z\isactrlbsup {\isacharparenleft}{\kern0pt}X\ {\isasymCoprod}\ Y{\isacharparenright}{\kern0pt}\isactrlesup {\isacharparenright}{\kern0pt}{\isacharparenright}{\kern0pt}\isactrlsup {\isasymsharp}\ {\isasymcirc}\isactrlsub c\ h\ {\isacharequal}{\kern0pt}\ g\ {\isasymand}\isanewline
\ \ \ \ \ \ \ \ \ \ \ {\isacharparenleft}{\kern0pt}{\isasymforall}h{\isadigit{2}}{\isachardot}{\kern0pt}\ h{\isadigit{2}}\ {\isacharcolon}{\kern0pt}\ H\ {\isasymrightarrow}\ Z\isactrlbsup {\isacharparenleft}{\kern0pt}X\ {\isasymCoprod}\ Y{\isacharparenright}{\kern0pt}\isactrlesup \ {\isasymand}\isanewline
\ \ \ \ \ \ \ \ \ \ \ \ \ \ \ \ \ {\isacharparenleft}{\kern0pt}eval{\isacharunderscore}{\kern0pt}func\ Z\ {\isacharparenleft}{\kern0pt}X\ {\isasymCoprod}\ Y{\isacharparenright}{\kern0pt}\ {\isasymcirc}\isactrlsub c\ left{\isacharunderscore}{\kern0pt}coproj\ X\ Y\ {\isasymtimes}\isactrlsub f\ id\isactrlsub c\ {\isacharparenleft}{\kern0pt}Z\isactrlbsup {\isacharparenleft}{\kern0pt}X\ {\isasymCoprod}\ Y{\isacharparenright}{\kern0pt}\isactrlesup {\isacharparenright}{\kern0pt}{\isacharparenright}{\kern0pt}\isactrlsup {\isasymsharp}\ {\isasymcirc}\isactrlsub c\ h{\isadigit{2}}\ {\isacharequal}{\kern0pt}\ f\ {\isasymand}\isanewline
\ \ \ \ \ \ \ \ \ \ \ \ \ \ \ \ \ {\isacharparenleft}{\kern0pt}eval{\isacharunderscore}{\kern0pt}func\ Z\ {\isacharparenleft}{\kern0pt}X\ {\isasymCoprod}\ Y{\isacharparenright}{\kern0pt}\ {\isasymcirc}\isactrlsub c\ right{\isacharunderscore}{\kern0pt}coproj\ X\ Y\ {\isasymtimes}\isactrlsub f\ id\isactrlsub c\ {\isacharparenleft}{\kern0pt}Z\isactrlbsup {\isacharparenleft}{\kern0pt}X\ {\isasymCoprod}\ Y{\isacharparenright}{\kern0pt}\isactrlesup {\isacharparenright}{\kern0pt}{\isacharparenright}{\kern0pt}\isactrlsup {\isasymsharp}\ {\isasymcirc}\isactrlsub c\ h{\isadigit{2}}\ {\isacharequal}{\kern0pt}\ g\ {\isasymlongrightarrow}\isanewline
\ \ \ \ \ \ \ \ \ \ \ \ \ \ \ \ \ h{\isadigit{2}}\ {\isacharequal}{\kern0pt}\ h{\isacharparenright}{\kern0pt}{\isachardoublequoteclose}\isanewline
\ \ \ \ \isacommand{proof}\isamarkupfalse%
\ {\isacharparenleft}{\kern0pt}rule{\isacharunderscore}{\kern0pt}tac\ x{\isacharequal}{\kern0pt}{\isachardoublequoteopen}{\isacharparenleft}{\kern0pt}f\isactrlsup {\isasymflat}\ {\isasymamalg}\ g\isactrlsup {\isasymflat}\ {\isasymcirc}\isactrlsub c\ dist{\isacharunderscore}{\kern0pt}prod{\isacharunderscore}{\kern0pt}coprod{\isacharunderscore}{\kern0pt}right\ X\ Y\ H{\isacharparenright}{\kern0pt}\isactrlsup {\isasymsharp}{\isachardoublequoteclose}\ \isakeyword{in}\ exI{\isacharcomma}{\kern0pt}\ safe{\isacharcomma}{\kern0pt}\ typecheck{\isacharunderscore}{\kern0pt}cfuncs{\isacharparenright}{\kern0pt}\isanewline
\ \ \ \ \ \ \isacommand{have}\isamarkupfalse%
\ {\isachardoublequoteopen}{\isacharparenleft}{\kern0pt}eval{\isacharunderscore}{\kern0pt}func\ Z\ {\isacharparenleft}{\kern0pt}X\ {\isasymCoprod}\ Y{\isacharparenright}{\kern0pt}\ {\isasymcirc}\isactrlsub c\ left{\isacharunderscore}{\kern0pt}coproj\ X\ Y\ {\isasymtimes}\isactrlsub f\ id\isactrlsub c\ {\isacharparenleft}{\kern0pt}Z\isactrlbsup {\isacharparenleft}{\kern0pt}X\ {\isasymCoprod}\ Y{\isacharparenright}{\kern0pt}\isactrlesup {\isacharparenright}{\kern0pt}{\isacharparenright}{\kern0pt}\isactrlsup {\isasymsharp}\ {\isasymcirc}\isactrlsub c\ {\isacharparenleft}{\kern0pt}f\isactrlsup {\isasymflat}\ {\isasymamalg}\ g\isactrlsup {\isasymflat}\ {\isasymcirc}\isactrlsub c\ dist{\isacharunderscore}{\kern0pt}prod{\isacharunderscore}{\kern0pt}coprod{\isacharunderscore}{\kern0pt}right\ X\ Y\ H{\isacharparenright}{\kern0pt}\isactrlsup {\isasymsharp}\ {\isacharequal}{\kern0pt}\ \isanewline
\ \ \ \ \ \ \ \ \ \ \ \ {\isacharparenleft}{\kern0pt}{\isacharparenleft}{\kern0pt}eval{\isacharunderscore}{\kern0pt}func\ Z\ {\isacharparenleft}{\kern0pt}X\ {\isasymCoprod}\ Y{\isacharparenright}{\kern0pt}\ {\isasymcirc}\isactrlsub c\ left{\isacharunderscore}{\kern0pt}coproj\ X\ Y\ {\isasymtimes}\isactrlsub f\ id\isactrlsub c\ {\isacharparenleft}{\kern0pt}Z\isactrlbsup {\isacharparenleft}{\kern0pt}X\ {\isasymCoprod}\ Y{\isacharparenright}{\kern0pt}\isactrlesup {\isacharparenright}{\kern0pt}{\isacharparenright}{\kern0pt}\ {\isasymcirc}\isactrlsub c\ {\isacharparenleft}{\kern0pt}id\ X\ {\isasymtimes}\isactrlsub f\ {\isacharparenleft}{\kern0pt}f\isactrlsup {\isasymflat}\ {\isasymamalg}\ g\isactrlsup {\isasymflat}\ {\isasymcirc}\isactrlsub c\ dist{\isacharunderscore}{\kern0pt}prod{\isacharunderscore}{\kern0pt}coprod{\isacharunderscore}{\kern0pt}right\ X\ Y\ H{\isacharparenright}{\kern0pt}\isactrlsup {\isasymsharp}{\isacharparenright}{\kern0pt}{\isacharparenright}{\kern0pt}\isactrlsup {\isasymsharp}{\isachardoublequoteclose}\isanewline
\ \ \ \ \ \ \ \ \isacommand{using}\isamarkupfalse%
\ sharp{\isacharunderscore}{\kern0pt}comp\ \isacommand{by}\isamarkupfalse%
\ {\isacharparenleft}{\kern0pt}typecheck{\isacharunderscore}{\kern0pt}cfuncs{\isacharcomma}{\kern0pt}\ blast{\isacharparenright}{\kern0pt}\isanewline
\ \ \ \ \ \ \isacommand{also}\isamarkupfalse%
\ \isacommand{have}\isamarkupfalse%
\ {\isachardoublequoteopen}{\isachardot}{\kern0pt}{\isachardot}{\kern0pt}{\isachardot}{\kern0pt}\ {\isacharequal}{\kern0pt}\ {\isacharparenleft}{\kern0pt}eval{\isacharunderscore}{\kern0pt}func\ Z\ {\isacharparenleft}{\kern0pt}X\ {\isasymCoprod}\ Y{\isacharparenright}{\kern0pt}\ {\isasymcirc}\isactrlsub c\ \ {\isacharparenleft}{\kern0pt}left{\isacharunderscore}{\kern0pt}coproj\ X\ Y\ {\isasymtimes}\isactrlsub f\ {\isacharparenleft}{\kern0pt}f\isactrlsup {\isasymflat}\ {\isasymamalg}\ g\isactrlsup {\isasymflat}\ {\isasymcirc}\isactrlsub c\ dist{\isacharunderscore}{\kern0pt}prod{\isacharunderscore}{\kern0pt}coprod{\isacharunderscore}{\kern0pt}right\ X\ Y\ H{\isacharparenright}{\kern0pt}\isactrlsup {\isasymsharp}{\isacharparenright}{\kern0pt}{\isacharparenright}{\kern0pt}\isactrlsup {\isasymsharp}{\isachardoublequoteclose}\isanewline
\ \ \ \ \ \ \ \ \isacommand{by}\isamarkupfalse%
\ {\isacharparenleft}{\kern0pt}typecheck{\isacharunderscore}{\kern0pt}cfuncs{\isacharcomma}{\kern0pt}\ smt\ {\isacharparenleft}{\kern0pt}z{\isadigit{3}}{\isacharparenright}{\kern0pt}\ cfunc{\isacharunderscore}{\kern0pt}cross{\isacharunderscore}{\kern0pt}prod{\isacharunderscore}{\kern0pt}comp{\isacharunderscore}{\kern0pt}cfunc{\isacharunderscore}{\kern0pt}cross{\isacharunderscore}{\kern0pt}prod\ comp{\isacharunderscore}{\kern0pt}associative{\isadigit{2}}\ id{\isacharunderscore}{\kern0pt}left{\isacharunderscore}{\kern0pt}unit{\isadigit{2}}\ id{\isacharunderscore}{\kern0pt}right{\isacharunderscore}{\kern0pt}unit{\isadigit{2}}{\isacharparenright}{\kern0pt}\isanewline
\ \ \ \ \ \ \isacommand{also}\isamarkupfalse%
\ \isacommand{have}\isamarkupfalse%
\ {\isachardoublequoteopen}{\isachardot}{\kern0pt}{\isachardot}{\kern0pt}{\isachardot}{\kern0pt}\ {\isacharequal}{\kern0pt}\ {\isacharparenleft}{\kern0pt}eval{\isacharunderscore}{\kern0pt}func\ Z\ {\isacharparenleft}{\kern0pt}X\ {\isasymCoprod}\ Y{\isacharparenright}{\kern0pt}\ {\isasymcirc}\isactrlsub c\ \ {\isacharparenleft}{\kern0pt}id\ {\isacharparenleft}{\kern0pt}X\ {\isasymCoprod}\ Y{\isacharparenright}{\kern0pt}\ {\isasymtimes}\isactrlsub f\ {\isacharparenleft}{\kern0pt}f\isactrlsup {\isasymflat}\ {\isasymamalg}\ g\isactrlsup {\isasymflat}\ {\isasymcirc}\isactrlsub c\ dist{\isacharunderscore}{\kern0pt}prod{\isacharunderscore}{\kern0pt}coprod{\isacharunderscore}{\kern0pt}right\ X\ Y\ H{\isacharparenright}{\kern0pt}\isactrlsup {\isasymsharp}{\isacharparenright}{\kern0pt}\ {\isasymcirc}\isactrlsub c\ {\isacharparenleft}{\kern0pt}left{\isacharunderscore}{\kern0pt}coproj\ X\ Y\ {\isasymtimes}\isactrlsub f\ id\ H{\isacharparenright}{\kern0pt}{\isacharparenright}{\kern0pt}\isactrlsup {\isasymsharp}{\isachardoublequoteclose}\isanewline
\ \ \ \ \ \ \ \ \isacommand{by}\isamarkupfalse%
\ {\isacharparenleft}{\kern0pt}typecheck{\isacharunderscore}{\kern0pt}cfuncs{\isacharcomma}{\kern0pt}\ simp\ add{\isacharcolon}{\kern0pt}\ cfunc{\isacharunderscore}{\kern0pt}cross{\isacharunderscore}{\kern0pt}prod{\isacharunderscore}{\kern0pt}comp{\isacharunderscore}{\kern0pt}cfunc{\isacharunderscore}{\kern0pt}cross{\isacharunderscore}{\kern0pt}prod\ id{\isacharunderscore}{\kern0pt}left{\isacharunderscore}{\kern0pt}unit{\isadigit{2}}\ id{\isacharunderscore}{\kern0pt}right{\isacharunderscore}{\kern0pt}unit{\isadigit{2}}{\isacharparenright}{\kern0pt}\isanewline
\ \ \ \ \ \ \isacommand{also}\isamarkupfalse%
\ \isacommand{have}\isamarkupfalse%
\ {\isachardoublequoteopen}{\isachardot}{\kern0pt}{\isachardot}{\kern0pt}{\isachardot}{\kern0pt}\ {\isacharequal}{\kern0pt}\ {\isacharparenleft}{\kern0pt}f\isactrlsup {\isasymflat}\ {\isasymamalg}\ g\isactrlsup {\isasymflat}\ {\isasymcirc}\isactrlsub c\ {\isacharparenleft}{\kern0pt}dist{\isacharunderscore}{\kern0pt}prod{\isacharunderscore}{\kern0pt}coprod{\isacharunderscore}{\kern0pt}right\ X\ Y\ H\ {\isasymcirc}\isactrlsub c\ left{\isacharunderscore}{\kern0pt}coproj\ X\ Y\ {\isasymtimes}\isactrlsub f\ id\ H{\isacharparenright}{\kern0pt}{\isacharparenright}{\kern0pt}\isactrlsup {\isasymsharp}{\isachardoublequoteclose}\isanewline
\ \ \ \ \ \ \ \ \isacommand{using}\isamarkupfalse%
\ comp{\isacharunderscore}{\kern0pt}associative{\isadigit{2}}\ transpose{\isacharunderscore}{\kern0pt}func{\isacharunderscore}{\kern0pt}def\ \isacommand{by}\isamarkupfalse%
\ {\isacharparenleft}{\kern0pt}typecheck{\isacharunderscore}{\kern0pt}cfuncs{\isacharcomma}{\kern0pt}\ force{\isacharparenright}{\kern0pt}\isanewline
\ \ \ \ \ \ \isacommand{also}\isamarkupfalse%
\ \isacommand{have}\isamarkupfalse%
\ {\isachardoublequoteopen}{\isachardot}{\kern0pt}{\isachardot}{\kern0pt}{\isachardot}{\kern0pt}\ {\isacharequal}{\kern0pt}\ {\isacharparenleft}{\kern0pt}f\isactrlsup {\isasymflat}\ {\isasymamalg}\ g\isactrlsup {\isasymflat}\ {\isasymcirc}\isactrlsub c\ left{\isacharunderscore}{\kern0pt}coproj\ {\isacharparenleft}{\kern0pt}X\ {\isasymtimes}\isactrlsub c\ H{\isacharparenright}{\kern0pt}\ {\isacharparenleft}{\kern0pt}Y\ {\isasymtimes}\isactrlsub c\ H{\isacharparenright}{\kern0pt}{\isacharparenright}{\kern0pt}\isactrlsup {\isasymsharp}{\isachardoublequoteclose}\isanewline
\ \ \ \ \ \ \ \ \isacommand{by}\isamarkupfalse%
\ {\isacharparenleft}{\kern0pt}simp\ add{\isacharcolon}{\kern0pt}\ dist{\isacharunderscore}{\kern0pt}prod{\isacharunderscore}{\kern0pt}coprod{\isacharunderscore}{\kern0pt}right{\isacharunderscore}{\kern0pt}left{\isacharunderscore}{\kern0pt}coproj{\isacharparenright}{\kern0pt}\isanewline
\ \ \ \ \ \ \isacommand{also}\isamarkupfalse%
\ \isacommand{have}\isamarkupfalse%
\ {\isachardoublequoteopen}{\isachardot}{\kern0pt}{\isachardot}{\kern0pt}{\isachardot}{\kern0pt}\ {\isacharequal}{\kern0pt}\ f{\isachardoublequoteclose}\isanewline
\ \ \ \ \ \ \ \ \isacommand{by}\isamarkupfalse%
\ {\isacharparenleft}{\kern0pt}typecheck{\isacharunderscore}{\kern0pt}cfuncs{\isacharcomma}{\kern0pt}\ simp\ add{\isacharcolon}{\kern0pt}\ left{\isacharunderscore}{\kern0pt}coproj{\isacharunderscore}{\kern0pt}cfunc{\isacharunderscore}{\kern0pt}coprod\ sharp{\isacharunderscore}{\kern0pt}cancels{\isacharunderscore}{\kern0pt}flat{\isacharparenright}{\kern0pt}\isanewline
\ \ \ \ \ \ \isacommand{then}\isamarkupfalse%
\ \isacommand{show}\isamarkupfalse%
\ {\isachardoublequoteopen}{\isacharparenleft}{\kern0pt}eval{\isacharunderscore}{\kern0pt}func\ Z\ {\isacharparenleft}{\kern0pt}X\ {\isasymCoprod}\ Y{\isacharparenright}{\kern0pt}\ {\isasymcirc}\isactrlsub c\ left{\isacharunderscore}{\kern0pt}coproj\ X\ Y\ {\isasymtimes}\isactrlsub f\ id\isactrlsub c\ {\isacharparenleft}{\kern0pt}Z\isactrlbsup {\isacharparenleft}{\kern0pt}X\ {\isasymCoprod}\ Y{\isacharparenright}{\kern0pt}\isactrlesup {\isacharparenright}{\kern0pt}{\isacharparenright}{\kern0pt}\isactrlsup {\isasymsharp}\ {\isasymcirc}\isactrlsub c\ {\isacharparenleft}{\kern0pt}f\isactrlsup {\isasymflat}\ {\isasymamalg}\ g\isactrlsup {\isasymflat}\ {\isasymcirc}\isactrlsub c\ dist{\isacharunderscore}{\kern0pt}prod{\isacharunderscore}{\kern0pt}coprod{\isacharunderscore}{\kern0pt}right\ X\ Y\ H{\isacharparenright}{\kern0pt}\isactrlsup {\isasymsharp}\ {\isacharequal}{\kern0pt}\ f{\isachardoublequoteclose}\isanewline
\ \ \ \ \ \ \ \ \isacommand{by}\isamarkupfalse%
\ {\isacharparenleft}{\kern0pt}simp\ add{\isacharcolon}{\kern0pt}\ calculation{\isacharparenright}{\kern0pt}\isanewline
\ \ \ \ \isacommand{next}\isamarkupfalse%
\isanewline
\ \ \ \ \ \ \isacommand{have}\isamarkupfalse%
\ {\isachardoublequoteopen}{\isacharparenleft}{\kern0pt}eval{\isacharunderscore}{\kern0pt}func\ Z\ {\isacharparenleft}{\kern0pt}X\ {\isasymCoprod}\ Y{\isacharparenright}{\kern0pt}\ {\isasymcirc}\isactrlsub c\ right{\isacharunderscore}{\kern0pt}coproj\ X\ Y\ {\isasymtimes}\isactrlsub f\ id\isactrlsub c\ {\isacharparenleft}{\kern0pt}Z\isactrlbsup {\isacharparenleft}{\kern0pt}X\ {\isasymCoprod}\ Y{\isacharparenright}{\kern0pt}\isactrlesup {\isacharparenright}{\kern0pt}{\isacharparenright}{\kern0pt}\isactrlsup {\isasymsharp}\ {\isasymcirc}\isactrlsub c\ {\isacharparenleft}{\kern0pt}f\isactrlsup {\isasymflat}\ {\isasymamalg}\ g\isactrlsup {\isasymflat}\ {\isasymcirc}\isactrlsub c\ dist{\isacharunderscore}{\kern0pt}prod{\isacharunderscore}{\kern0pt}coprod{\isacharunderscore}{\kern0pt}right\ X\ Y\ H{\isacharparenright}{\kern0pt}\isactrlsup {\isasymsharp}\ {\isacharequal}{\kern0pt}\ \isanewline
\ \ \ \ \ \ \ \ \ \ \ \ {\isacharparenleft}{\kern0pt}{\isacharparenleft}{\kern0pt}eval{\isacharunderscore}{\kern0pt}func\ Z\ {\isacharparenleft}{\kern0pt}X\ {\isasymCoprod}\ Y{\isacharparenright}{\kern0pt}\ {\isasymcirc}\isactrlsub c\ right{\isacharunderscore}{\kern0pt}coproj\ X\ Y\ {\isasymtimes}\isactrlsub f\ id\isactrlsub c\ {\isacharparenleft}{\kern0pt}Z\isactrlbsup {\isacharparenleft}{\kern0pt}X\ {\isasymCoprod}\ Y{\isacharparenright}{\kern0pt}\isactrlesup {\isacharparenright}{\kern0pt}{\isacharparenright}{\kern0pt}\ {\isasymcirc}\isactrlsub c\ {\isacharparenleft}{\kern0pt}id\ Y\ {\isasymtimes}\isactrlsub f\ {\isacharparenleft}{\kern0pt}f\isactrlsup {\isasymflat}\ {\isasymamalg}\ g\isactrlsup {\isasymflat}\ {\isasymcirc}\isactrlsub c\ dist{\isacharunderscore}{\kern0pt}prod{\isacharunderscore}{\kern0pt}coprod{\isacharunderscore}{\kern0pt}right\ X\ Y\ H{\isacharparenright}{\kern0pt}\isactrlsup {\isasymsharp}{\isacharparenright}{\kern0pt}{\isacharparenright}{\kern0pt}\isactrlsup {\isasymsharp}{\isachardoublequoteclose}\isanewline
\ \ \ \ \ \ \ \ \isacommand{using}\isamarkupfalse%
\ sharp{\isacharunderscore}{\kern0pt}comp\ \isacommand{by}\isamarkupfalse%
\ {\isacharparenleft}{\kern0pt}typecheck{\isacharunderscore}{\kern0pt}cfuncs{\isacharcomma}{\kern0pt}\ blast{\isacharparenright}{\kern0pt}\isanewline
\ \ \ \ \ \ \isacommand{also}\isamarkupfalse%
\ \isacommand{have}\isamarkupfalse%
\ {\isachardoublequoteopen}{\isachardot}{\kern0pt}{\isachardot}{\kern0pt}{\isachardot}{\kern0pt}\ {\isacharequal}{\kern0pt}\ {\isacharparenleft}{\kern0pt}eval{\isacharunderscore}{\kern0pt}func\ Z\ {\isacharparenleft}{\kern0pt}X\ {\isasymCoprod}\ Y{\isacharparenright}{\kern0pt}\ {\isasymcirc}\isactrlsub c\ \ {\isacharparenleft}{\kern0pt}right{\isacharunderscore}{\kern0pt}coproj\ X\ Y\ {\isasymtimes}\isactrlsub f\ {\isacharparenleft}{\kern0pt}f\isactrlsup {\isasymflat}\ {\isasymamalg}\ g\isactrlsup {\isasymflat}\ {\isasymcirc}\isactrlsub c\ dist{\isacharunderscore}{\kern0pt}prod{\isacharunderscore}{\kern0pt}coprod{\isacharunderscore}{\kern0pt}right\ X\ Y\ H{\isacharparenright}{\kern0pt}\isactrlsup {\isasymsharp}{\isacharparenright}{\kern0pt}{\isacharparenright}{\kern0pt}\isactrlsup {\isasymsharp}{\isachardoublequoteclose}\isanewline
\ \ \ \ \ \ \ \ \isacommand{by}\isamarkupfalse%
\ {\isacharparenleft}{\kern0pt}typecheck{\isacharunderscore}{\kern0pt}cfuncs{\isacharcomma}{\kern0pt}\ smt\ {\isacharparenleft}{\kern0pt}z{\isadigit{3}}{\isacharparenright}{\kern0pt}\ cfunc{\isacharunderscore}{\kern0pt}cross{\isacharunderscore}{\kern0pt}prod{\isacharunderscore}{\kern0pt}comp{\isacharunderscore}{\kern0pt}cfunc{\isacharunderscore}{\kern0pt}cross{\isacharunderscore}{\kern0pt}prod\ comp{\isacharunderscore}{\kern0pt}associative{\isadigit{2}}\ id{\isacharunderscore}{\kern0pt}left{\isacharunderscore}{\kern0pt}unit{\isadigit{2}}\ id{\isacharunderscore}{\kern0pt}right{\isacharunderscore}{\kern0pt}unit{\isadigit{2}}{\isacharparenright}{\kern0pt}\isanewline
\ \ \ \ \ \ \isacommand{also}\isamarkupfalse%
\ \isacommand{have}\isamarkupfalse%
\ {\isachardoublequoteopen}{\isachardot}{\kern0pt}{\isachardot}{\kern0pt}{\isachardot}{\kern0pt}\ {\isacharequal}{\kern0pt}\ {\isacharparenleft}{\kern0pt}eval{\isacharunderscore}{\kern0pt}func\ Z\ {\isacharparenleft}{\kern0pt}X\ {\isasymCoprod}\ Y{\isacharparenright}{\kern0pt}\ {\isasymcirc}\isactrlsub c\ \ {\isacharparenleft}{\kern0pt}id\ {\isacharparenleft}{\kern0pt}X\ {\isasymCoprod}\ Y{\isacharparenright}{\kern0pt}\ {\isasymtimes}\isactrlsub f\ {\isacharparenleft}{\kern0pt}f\isactrlsup {\isasymflat}\ {\isasymamalg}\ g\isactrlsup {\isasymflat}\ {\isasymcirc}\isactrlsub c\ dist{\isacharunderscore}{\kern0pt}prod{\isacharunderscore}{\kern0pt}coprod{\isacharunderscore}{\kern0pt}right\ X\ Y\ H{\isacharparenright}{\kern0pt}\isactrlsup {\isasymsharp}{\isacharparenright}{\kern0pt}\ {\isasymcirc}\isactrlsub c\ {\isacharparenleft}{\kern0pt}right{\isacharunderscore}{\kern0pt}coproj\ X\ Y\ {\isasymtimes}\isactrlsub f\ id\ H{\isacharparenright}{\kern0pt}{\isacharparenright}{\kern0pt}\isactrlsup {\isasymsharp}{\isachardoublequoteclose}\isanewline
\ \ \ \ \ \ \ \ \isacommand{by}\isamarkupfalse%
\ {\isacharparenleft}{\kern0pt}typecheck{\isacharunderscore}{\kern0pt}cfuncs{\isacharcomma}{\kern0pt}\ simp\ add{\isacharcolon}{\kern0pt}\ cfunc{\isacharunderscore}{\kern0pt}cross{\isacharunderscore}{\kern0pt}prod{\isacharunderscore}{\kern0pt}comp{\isacharunderscore}{\kern0pt}cfunc{\isacharunderscore}{\kern0pt}cross{\isacharunderscore}{\kern0pt}prod\ id{\isacharunderscore}{\kern0pt}left{\isacharunderscore}{\kern0pt}unit{\isadigit{2}}\ id{\isacharunderscore}{\kern0pt}right{\isacharunderscore}{\kern0pt}unit{\isadigit{2}}{\isacharparenright}{\kern0pt}\isanewline
\ \ \ \ \ \ \isacommand{also}\isamarkupfalse%
\ \isacommand{have}\isamarkupfalse%
\ {\isachardoublequoteopen}{\isachardot}{\kern0pt}{\isachardot}{\kern0pt}{\isachardot}{\kern0pt}\ {\isacharequal}{\kern0pt}\ {\isacharparenleft}{\kern0pt}f\isactrlsup {\isasymflat}\ {\isasymamalg}\ g\isactrlsup {\isasymflat}\ {\isasymcirc}\isactrlsub c\ {\isacharparenleft}{\kern0pt}dist{\isacharunderscore}{\kern0pt}prod{\isacharunderscore}{\kern0pt}coprod{\isacharunderscore}{\kern0pt}right\ X\ Y\ H\ {\isasymcirc}\isactrlsub c\ right{\isacharunderscore}{\kern0pt}coproj\ X\ Y\ {\isasymtimes}\isactrlsub f\ id\ H{\isacharparenright}{\kern0pt}{\isacharparenright}{\kern0pt}\isactrlsup {\isasymsharp}{\isachardoublequoteclose}\isanewline
\ \ \ \ \ \ \ \ \isacommand{using}\isamarkupfalse%
\ comp{\isacharunderscore}{\kern0pt}associative{\isadigit{2}}\ transpose{\isacharunderscore}{\kern0pt}func{\isacharunderscore}{\kern0pt}def\ \isacommand{by}\isamarkupfalse%
\ {\isacharparenleft}{\kern0pt}typecheck{\isacharunderscore}{\kern0pt}cfuncs{\isacharcomma}{\kern0pt}\ force{\isacharparenright}{\kern0pt}\isanewline
\ \ \ \ \ \ \isacommand{also}\isamarkupfalse%
\ \isacommand{have}\isamarkupfalse%
\ {\isachardoublequoteopen}{\isachardot}{\kern0pt}{\isachardot}{\kern0pt}{\isachardot}{\kern0pt}\ {\isacharequal}{\kern0pt}\ {\isacharparenleft}{\kern0pt}f\isactrlsup {\isasymflat}\ {\isasymamalg}\ g\isactrlsup {\isasymflat}\ {\isasymcirc}\isactrlsub c\ right{\isacharunderscore}{\kern0pt}coproj\ {\isacharparenleft}{\kern0pt}X\ {\isasymtimes}\isactrlsub c\ H{\isacharparenright}{\kern0pt}\ {\isacharparenleft}{\kern0pt}Y\ {\isasymtimes}\isactrlsub c\ H{\isacharparenright}{\kern0pt}{\isacharparenright}{\kern0pt}\isactrlsup {\isasymsharp}{\isachardoublequoteclose}\isanewline
\ \ \ \ \ \ \ \ \isacommand{by}\isamarkupfalse%
\ {\isacharparenleft}{\kern0pt}simp\ add{\isacharcolon}{\kern0pt}\ dist{\isacharunderscore}{\kern0pt}prod{\isacharunderscore}{\kern0pt}coprod{\isacharunderscore}{\kern0pt}right{\isacharunderscore}{\kern0pt}right{\isacharunderscore}{\kern0pt}coproj{\isacharparenright}{\kern0pt}\isanewline
\ \ \ \ \ \ \isacommand{also}\isamarkupfalse%
\ \isacommand{have}\isamarkupfalse%
\ {\isachardoublequoteopen}{\isachardot}{\kern0pt}{\isachardot}{\kern0pt}{\isachardot}{\kern0pt}\ {\isacharequal}{\kern0pt}\ g{\isachardoublequoteclose}\isanewline
\ \ \ \ \ \ \ \ \isacommand{by}\isamarkupfalse%
\ {\isacharparenleft}{\kern0pt}typecheck{\isacharunderscore}{\kern0pt}cfuncs{\isacharcomma}{\kern0pt}\ simp\ add{\isacharcolon}{\kern0pt}\ right{\isacharunderscore}{\kern0pt}coproj{\isacharunderscore}{\kern0pt}cfunc{\isacharunderscore}{\kern0pt}coprod\ sharp{\isacharunderscore}{\kern0pt}cancels{\isacharunderscore}{\kern0pt}flat{\isacharparenright}{\kern0pt}\isanewline
\ \ \ \ \ \ \isacommand{then}\isamarkupfalse%
\ \isacommand{show}\isamarkupfalse%
\ {\isachardoublequoteopen}{\isacharparenleft}{\kern0pt}eval{\isacharunderscore}{\kern0pt}func\ Z\ {\isacharparenleft}{\kern0pt}X\ {\isasymCoprod}\ Y{\isacharparenright}{\kern0pt}\ {\isasymcirc}\isactrlsub c\ right{\isacharunderscore}{\kern0pt}coproj\ X\ Y\ {\isasymtimes}\isactrlsub f\ id\isactrlsub c\ {\isacharparenleft}{\kern0pt}Z\isactrlbsup {\isacharparenleft}{\kern0pt}X\ {\isasymCoprod}\ Y{\isacharparenright}{\kern0pt}\isactrlesup {\isacharparenright}{\kern0pt}{\isacharparenright}{\kern0pt}\isactrlsup {\isasymsharp}\ {\isasymcirc}\isactrlsub c\ {\isacharparenleft}{\kern0pt}f\isactrlsup {\isasymflat}\ {\isasymamalg}\ g\isactrlsup {\isasymflat}\ {\isasymcirc}\isactrlsub c\ dist{\isacharunderscore}{\kern0pt}prod{\isacharunderscore}{\kern0pt}coprod{\isacharunderscore}{\kern0pt}right\ X\ Y\ H{\isacharparenright}{\kern0pt}\isactrlsup {\isasymsharp}\ {\isacharequal}{\kern0pt}\ g{\isachardoublequoteclose}\isanewline
\ \ \ \ \ \ \ \ \isacommand{by}\isamarkupfalse%
\ {\isacharparenleft}{\kern0pt}simp\ add{\isacharcolon}{\kern0pt}\ calculation{\isacharparenright}{\kern0pt}\isanewline
\ \ \ \ \isacommand{next}\isamarkupfalse%
\isanewline
\ \ \ \ \ \ \isacommand{fix}\isamarkupfalse%
\ h\ \isanewline
\ \ \ \ \ \ \isacommand{assume}\isamarkupfalse%
\ h{\isacharunderscore}{\kern0pt}type{\isacharbrackleft}{\kern0pt}type{\isacharunderscore}{\kern0pt}rule{\isacharbrackright}{\kern0pt}{\isacharcolon}{\kern0pt}\ {\isachardoublequoteopen}h\ {\isacharcolon}{\kern0pt}\ H\ {\isasymrightarrow}\ Z\isactrlbsup {\isacharparenleft}{\kern0pt}X\ {\isasymCoprod}\ Y{\isacharparenright}{\kern0pt}\isactrlesup {\isachardoublequoteclose}\isanewline
\ \ \ \ \ \ \isacommand{assume}\isamarkupfalse%
\ f{\isacharunderscore}{\kern0pt}eqs{\isacharcolon}{\kern0pt}\ {\isachardoublequoteopen}f\ {\isacharequal}{\kern0pt}\ {\isacharparenleft}{\kern0pt}eval{\isacharunderscore}{\kern0pt}func\ Z\ {\isacharparenleft}{\kern0pt}X\ {\isasymCoprod}\ Y{\isacharparenright}{\kern0pt}\ {\isasymcirc}\isactrlsub c\ left{\isacharunderscore}{\kern0pt}coproj\ X\ Y\ {\isasymtimes}\isactrlsub f\ id\isactrlsub c\ {\isacharparenleft}{\kern0pt}Z\isactrlbsup {\isacharparenleft}{\kern0pt}X\ {\isasymCoprod}\ Y{\isacharparenright}{\kern0pt}\isactrlesup {\isacharparenright}{\kern0pt}{\isacharparenright}{\kern0pt}\isactrlsup {\isasymsharp}\ {\isasymcirc}\isactrlsub c\ \ h{\isachardoublequoteclose}\isanewline
\ \ \ \ \ \ \isacommand{assume}\isamarkupfalse%
\ g{\isacharunderscore}{\kern0pt}eqs{\isacharcolon}{\kern0pt}\ {\isachardoublequoteopen}g\ {\isacharequal}{\kern0pt}\ {\isacharparenleft}{\kern0pt}eval{\isacharunderscore}{\kern0pt}func\ Z\ {\isacharparenleft}{\kern0pt}X\ {\isasymCoprod}\ Y{\isacharparenright}{\kern0pt}\ {\isasymcirc}\isactrlsub c\ right{\isacharunderscore}{\kern0pt}coproj\ X\ Y\ {\isasymtimes}\isactrlsub f\ id\isactrlsub c\ {\isacharparenleft}{\kern0pt}Z\isactrlbsup {\isacharparenleft}{\kern0pt}X\ {\isasymCoprod}\ Y{\isacharparenright}{\kern0pt}\isactrlesup {\isacharparenright}{\kern0pt}{\isacharparenright}{\kern0pt}\isactrlsup {\isasymsharp}\ {\isasymcirc}\isactrlsub c\ h{\isachardoublequoteclose}\isanewline
\ \ \ \ \ \ \isacommand{have}\isamarkupfalse%
\ {\isachardoublequoteopen}{\isacharparenleft}{\kern0pt}f\isactrlsup {\isasymflat}\ {\isasymamalg}\ g\isactrlsup {\isasymflat}\ {\isasymcirc}\isactrlsub c\ dist{\isacharunderscore}{\kern0pt}prod{\isacharunderscore}{\kern0pt}coprod{\isacharunderscore}{\kern0pt}right\ X\ Y\ H{\isacharparenright}{\kern0pt}\ {\isacharequal}{\kern0pt}\ h\isactrlsup {\isasymflat}{\isachardoublequoteclose}\isanewline
\ \ \ \ \ \ \isacommand{proof}\isamarkupfalse%
{\isacharparenleft}{\kern0pt}etcs{\isacharunderscore}{\kern0pt}rule\ one{\isacharunderscore}{\kern0pt}separator{\isacharbrackleft}{\kern0pt}\isakeyword{where}\ X\ {\isacharequal}{\kern0pt}\ {\isachardoublequoteopen}{\isacharparenleft}{\kern0pt}X\ {\isasymCoprod}\ Y{\isacharparenright}{\kern0pt}\ {\isasymtimes}\isactrlsub c\ H{\isachardoublequoteclose}{\isacharcomma}{\kern0pt}\ \isakeyword{where}\ Y\ {\isacharequal}{\kern0pt}\ Z{\isacharbrackright}{\kern0pt}{\isacharparenright}{\kern0pt}\isanewline
\ \ \ \ \ \ \ \ \isacommand{show}\isamarkupfalse%
\ {\isachardoublequoteopen}{\isasymAnd}xyh{\isachardot}{\kern0pt}\ xyh\ {\isasymin}\isactrlsub c\ {\isacharparenleft}{\kern0pt}X\ {\isasymCoprod}\ Y{\isacharparenright}{\kern0pt}\ {\isasymtimes}\isactrlsub c\ H\ {\isasymLongrightarrow}\ {\isacharparenleft}{\kern0pt}f\isactrlsup {\isasymflat}\ {\isasymamalg}\ g\isactrlsup {\isasymflat}\ {\isasymcirc}\isactrlsub c\ dist{\isacharunderscore}{\kern0pt}prod{\isacharunderscore}{\kern0pt}coprod{\isacharunderscore}{\kern0pt}right\ X\ Y\ H{\isacharparenright}{\kern0pt}\ {\isasymcirc}\isactrlsub c\ xyh\ {\isacharequal}{\kern0pt}\ h\isactrlsup {\isasymflat}\ {\isasymcirc}\isactrlsub c\ xyh{\isachardoublequoteclose}\isanewline
\ \ \ \ \ \ \ \ \isacommand{proof}\isamarkupfalse%
{\isacharminus}{\kern0pt}\isanewline
\ \ \ \ \ \ \ \ \ \ \isacommand{fix}\isamarkupfalse%
\ xyh\isanewline
\ \ \ \ \ \ \ \ \ \ \isacommand{assume}\isamarkupfalse%
\ l{\isacharunderscore}{\kern0pt}type{\isacharbrackleft}{\kern0pt}type{\isacharunderscore}{\kern0pt}rule{\isacharbrackright}{\kern0pt}{\isacharcolon}{\kern0pt}\ {\isachardoublequoteopen}xyh\ {\isasymin}\isactrlsub c\ {\isacharparenleft}{\kern0pt}X\ {\isasymCoprod}\ Y{\isacharparenright}{\kern0pt}\ {\isasymtimes}\isactrlsub c\ H{\isachardoublequoteclose}\isanewline
\ \ \ \ \ \ \ \ \ \ \isacommand{then}\isamarkupfalse%
\ \isacommand{obtain}\isamarkupfalse%
\ xy\ \isakeyword{and}\ z\ \isakeyword{where}\ xy{\isacharunderscore}{\kern0pt}type{\isacharbrackleft}{\kern0pt}type{\isacharunderscore}{\kern0pt}rule{\isacharbrackright}{\kern0pt}{\isacharcolon}{\kern0pt}\ {\isachardoublequoteopen}xy\ {\isasymin}\isactrlsub c\ X\ {\isasymCoprod}\ Y{\isachardoublequoteclose}\ \isakeyword{and}\ z{\isacharunderscore}{\kern0pt}type{\isacharbrackleft}{\kern0pt}type{\isacharunderscore}{\kern0pt}rule{\isacharbrackright}{\kern0pt}{\isacharcolon}{\kern0pt}\ {\isachardoublequoteopen}z\ {\isasymin}\isactrlsub c\ H{\isachardoublequoteclose}\isanewline
\ \ \ \ \ \ \ \ \ \ \ \ \isakeyword{and}\ xyh{\isacharunderscore}{\kern0pt}def{\isacharcolon}{\kern0pt}\ {\isachardoublequoteopen}xyh\ {\isacharequal}{\kern0pt}\ {\isasymlangle}xy{\isacharcomma}{\kern0pt}z{\isasymrangle}{\isachardoublequoteclose}\isanewline
\ \ \ \ \ \ \ \ \ \ \ \ \isacommand{using}\isamarkupfalse%
\ cart{\isacharunderscore}{\kern0pt}prod{\isacharunderscore}{\kern0pt}decomp\ \isacommand{by}\isamarkupfalse%
\ blast\isanewline
\ \ \ \ \ \ \ \ \ \ \isacommand{show}\isamarkupfalse%
\ {\isachardoublequoteopen}{\isacharparenleft}{\kern0pt}f\isactrlsup {\isasymflat}\ {\isasymamalg}\ g\isactrlsup {\isasymflat}\ {\isasymcirc}\isactrlsub c\ dist{\isacharunderscore}{\kern0pt}prod{\isacharunderscore}{\kern0pt}coprod{\isacharunderscore}{\kern0pt}right\ X\ Y\ H{\isacharparenright}{\kern0pt}\ {\isasymcirc}\isactrlsub c\ xyh\ {\isacharequal}{\kern0pt}\ h\isactrlsup {\isasymflat}\ {\isasymcirc}\isactrlsub c\ xyh{\isachardoublequoteclose}\isanewline
\ \ \ \ \ \ \ \ \ \ \isacommand{proof}\isamarkupfalse%
{\isacharparenleft}{\kern0pt}cases\ {\isachardoublequoteopen}{\isasymexists}x{\isachardot}{\kern0pt}\ x\ {\isasymin}\isactrlsub c\ X\ {\isasymand}\ xy\ {\isacharequal}{\kern0pt}\ \ left{\isacharunderscore}{\kern0pt}coproj\ X\ Y\ {\isasymcirc}\isactrlsub c\ x{\isachardoublequoteclose}{\isacharparenright}{\kern0pt}\isanewline
\ \ \ \ \ \ \ \ \ \ \ \ \isacommand{assume}\isamarkupfalse%
\ {\isachardoublequoteopen}{\isasymexists}x{\isachardot}{\kern0pt}\ x\ {\isasymin}\isactrlsub c\ X\ {\isasymand}\ xy\ {\isacharequal}{\kern0pt}\ left{\isacharunderscore}{\kern0pt}coproj\ X\ Y\ {\isasymcirc}\isactrlsub c\ x{\isachardoublequoteclose}\isanewline
\ \ \ \ \ \ \ \ \ \ \ \ \isacommand{then}\isamarkupfalse%
\ \isacommand{obtain}\isamarkupfalse%
\ x\ \isakeyword{where}\ x{\isacharunderscore}{\kern0pt}type{\isacharbrackleft}{\kern0pt}type{\isacharunderscore}{\kern0pt}rule{\isacharbrackright}{\kern0pt}{\isacharcolon}{\kern0pt}\ {\isachardoublequoteopen}x\ {\isasymin}\isactrlsub c\ X{\isachardoublequoteclose}\ \isakeyword{and}\ xy{\isacharunderscore}{\kern0pt}def{\isacharcolon}{\kern0pt}\ {\isachardoublequoteopen}xy\ {\isacharequal}{\kern0pt}\ \ left{\isacharunderscore}{\kern0pt}coproj\ X\ Y\ {\isasymcirc}\isactrlsub c\ x{\isachardoublequoteclose}\isanewline
\ \ \ \ \ \ \ \ \ \ \ \ \ \ \isacommand{by}\isamarkupfalse%
\ blast\isanewline
\ \ \ \ \ \ \ \ \ \ \ \ \isacommand{have}\isamarkupfalse%
\ {\isachardoublequoteopen}{\isacharparenleft}{\kern0pt}f\isactrlsup {\isasymflat}\ {\isasymamalg}\ g\isactrlsup {\isasymflat}\ {\isasymcirc}\isactrlsub c\ dist{\isacharunderscore}{\kern0pt}prod{\isacharunderscore}{\kern0pt}coprod{\isacharunderscore}{\kern0pt}right\ X\ Y\ H{\isacharparenright}{\kern0pt}\ {\isasymcirc}\isactrlsub c\ xyh\ {\isacharequal}{\kern0pt}\ {\isacharparenleft}{\kern0pt}f\isactrlsup {\isasymflat}\ {\isasymamalg}\ g\isactrlsup {\isasymflat}{\isacharparenright}{\kern0pt}\ {\isasymcirc}\isactrlsub c\ {\isacharparenleft}{\kern0pt}dist{\isacharunderscore}{\kern0pt}prod{\isacharunderscore}{\kern0pt}coprod{\isacharunderscore}{\kern0pt}right\ X\ Y\ H\ \ {\isasymcirc}\isactrlsub c\ {\isasymlangle}left{\isacharunderscore}{\kern0pt}coproj\ X\ Y\ {\isasymcirc}\isactrlsub c\ x{\isacharcomma}{\kern0pt}z{\isasymrangle}{\isacharparenright}{\kern0pt}{\isachardoublequoteclose}\isanewline
\ \ \ \ \ \ \ \ \ \ \ \ \ \ \isacommand{by}\isamarkupfalse%
\ {\isacharparenleft}{\kern0pt}typecheck{\isacharunderscore}{\kern0pt}cfuncs{\isacharcomma}{\kern0pt}\ simp\ add{\isacharcolon}{\kern0pt}\ comp{\isacharunderscore}{\kern0pt}associative{\isadigit{2}}\ xy{\isacharunderscore}{\kern0pt}def\ xyh{\isacharunderscore}{\kern0pt}def{\isacharparenright}{\kern0pt}\isanewline
\ \ \ \ \ \ \ \ \ \ \ \ \isacommand{also}\isamarkupfalse%
\ \isacommand{have}\isamarkupfalse%
\ {\isachardoublequoteopen}{\isachardot}{\kern0pt}{\isachardot}{\kern0pt}{\isachardot}{\kern0pt}\ {\isacharequal}{\kern0pt}\ {\isacharparenleft}{\kern0pt}f\isactrlsup {\isasymflat}\ {\isasymamalg}\ g\isactrlsup {\isasymflat}{\isacharparenright}{\kern0pt}\ {\isasymcirc}\isactrlsub c\ {\isacharparenleft}{\kern0pt}{\isacharparenleft}{\kern0pt}dist{\isacharunderscore}{\kern0pt}prod{\isacharunderscore}{\kern0pt}coprod{\isacharunderscore}{\kern0pt}right\ X\ Y\ H\ \ {\isasymcirc}\isactrlsub c\ {\isacharparenleft}{\kern0pt}left{\isacharunderscore}{\kern0pt}coproj\ X\ Y\ {\isasymtimes}\isactrlsub f\ id\ H{\isacharparenright}{\kern0pt}{\isacharparenright}{\kern0pt}\ {\isasymcirc}\isactrlsub c\ {\isasymlangle}x{\isacharcomma}{\kern0pt}z{\isasymrangle}{\isacharparenright}{\kern0pt}{\isachardoublequoteclose}\isanewline
\ \ \ \ \ \ \ \ \ \ \ \ \ \ \isacommand{using}\isamarkupfalse%
\ dist{\isacharunderscore}{\kern0pt}prod{\isacharunderscore}{\kern0pt}coprod{\isacharunderscore}{\kern0pt}right{\isacharunderscore}{\kern0pt}ap{\isacharunderscore}{\kern0pt}left\ dist{\isacharunderscore}{\kern0pt}prod{\isacharunderscore}{\kern0pt}coprod{\isacharunderscore}{\kern0pt}right{\isacharunderscore}{\kern0pt}left{\isacharunderscore}{\kern0pt}coproj\ \isacommand{by}\isamarkupfalse%
\ {\isacharparenleft}{\kern0pt}typecheck{\isacharunderscore}{\kern0pt}cfuncs{\isacharcomma}{\kern0pt}\ presburger{\isacharparenright}{\kern0pt}\isanewline
\ \ \ \ \ \ \ \ \ \ \ \ \isacommand{also}\isamarkupfalse%
\ \isacommand{have}\isamarkupfalse%
\ {\isachardoublequoteopen}{\isachardot}{\kern0pt}{\isachardot}{\kern0pt}{\isachardot}{\kern0pt}\ {\isacharequal}{\kern0pt}\ {\isacharparenleft}{\kern0pt}f\isactrlsup {\isasymflat}\ {\isasymamalg}\ g\isactrlsup {\isasymflat}{\isacharparenright}{\kern0pt}\ {\isasymcirc}\isactrlsub c\ {\isacharparenleft}{\kern0pt}left{\isacharunderscore}{\kern0pt}coproj\ {\isacharparenleft}{\kern0pt}X\ {\isasymtimes}\isactrlsub c\ H{\isacharparenright}{\kern0pt}\ {\isacharparenleft}{\kern0pt}Y\ {\isasymtimes}\isactrlsub c\ H{\isacharparenright}{\kern0pt}\ \ {\isasymcirc}\isactrlsub c\ {\isasymlangle}x{\isacharcomma}{\kern0pt}z{\isasymrangle}{\isacharparenright}{\kern0pt}{\isachardoublequoteclose}\isanewline
\ \ \ \ \ \ \ \ \ \ \ \ \ \ \isacommand{using}\isamarkupfalse%
\ dist{\isacharunderscore}{\kern0pt}prod{\isacharunderscore}{\kern0pt}coprod{\isacharunderscore}{\kern0pt}right{\isacharunderscore}{\kern0pt}left{\isacharunderscore}{\kern0pt}coproj\ \isacommand{by}\isamarkupfalse%
\ presburger\isanewline
\ \ \ \ \ \ \ \ \ \ \ \ \isacommand{also}\isamarkupfalse%
\ \isacommand{have}\isamarkupfalse%
\ {\isachardoublequoteopen}{\isachardot}{\kern0pt}{\isachardot}{\kern0pt}{\isachardot}{\kern0pt}\ {\isacharequal}{\kern0pt}\ f\isactrlsup {\isasymflat}\ {\isasymcirc}\isactrlsub c\ {\isasymlangle}x{\isacharcomma}{\kern0pt}z{\isasymrangle}{\isachardoublequoteclose}\isanewline
\ \ \ \ \ \ \ \ \ \ \ \ \ \ \isacommand{by}\isamarkupfalse%
\ {\isacharparenleft}{\kern0pt}typecheck{\isacharunderscore}{\kern0pt}cfuncs{\isacharcomma}{\kern0pt}\ \ simp\ add{\isacharcolon}{\kern0pt}\ comp{\isacharunderscore}{\kern0pt}associative{\isadigit{2}}\ left{\isacharunderscore}{\kern0pt}coproj{\isacharunderscore}{\kern0pt}cfunc{\isacharunderscore}{\kern0pt}coprod{\isacharparenright}{\kern0pt}\isanewline
\ \ \ \ \ \ \ \ \ \ \ \ \isacommand{also}\isamarkupfalse%
\ \isacommand{have}\isamarkupfalse%
\ {\isachardoublequoteopen}{\isachardot}{\kern0pt}{\isachardot}{\kern0pt}{\isachardot}{\kern0pt}\ {\isacharequal}{\kern0pt}\ {\isacharparenleft}{\kern0pt}{\isacharparenleft}{\kern0pt}eval{\isacharunderscore}{\kern0pt}func\ Z\ {\isacharparenleft}{\kern0pt}X\ {\isasymCoprod}\ Y{\isacharparenright}{\kern0pt}\ {\isasymcirc}\isactrlsub c\ left{\isacharunderscore}{\kern0pt}coproj\ X\ Y\ {\isasymtimes}\isactrlsub f\ id\isactrlsub c\ {\isacharparenleft}{\kern0pt}Z\isactrlbsup {\isacharparenleft}{\kern0pt}X\ {\isasymCoprod}\ Y{\isacharparenright}{\kern0pt}\isactrlesup {\isacharparenright}{\kern0pt}{\isacharparenright}{\kern0pt}\isactrlsup {\isasymsharp}\ {\isasymcirc}\isactrlsub c\ \ h{\isacharparenright}{\kern0pt}\isactrlsup {\isasymflat}\ \ {\isasymcirc}\isactrlsub c\ {\isasymlangle}x{\isacharcomma}{\kern0pt}z{\isasymrangle}{\isachardoublequoteclose}\isanewline
\ \ \ \ \ \ \ \ \ \ \ \ \ \ \isacommand{using}\isamarkupfalse%
\ f{\isacharunderscore}{\kern0pt}eqs\ \isacommand{by}\isamarkupfalse%
\ fastforce\isanewline
\ \ \ \ \ \ \ \ \ \ \ \ \isacommand{also}\isamarkupfalse%
\ \isacommand{have}\isamarkupfalse%
\ {\isachardoublequoteopen}{\isachardot}{\kern0pt}{\isachardot}{\kern0pt}{\isachardot}{\kern0pt}\ {\isacharequal}{\kern0pt}\ {\isacharparenleft}{\kern0pt}{\isacharparenleft}{\kern0pt}{\isacharparenleft}{\kern0pt}eval{\isacharunderscore}{\kern0pt}func\ Z\ {\isacharparenleft}{\kern0pt}X\ {\isasymCoprod}\ Y{\isacharparenright}{\kern0pt}\ {\isasymcirc}\isactrlsub c\ left{\isacharunderscore}{\kern0pt}coproj\ X\ Y\ {\isasymtimes}\isactrlsub f\ id\isactrlsub c\ {\isacharparenleft}{\kern0pt}Z\isactrlbsup {\isacharparenleft}{\kern0pt}X\ {\isasymCoprod}\ Y{\isacharparenright}{\kern0pt}\isactrlesup {\isacharparenright}{\kern0pt}{\isacharparenright}{\kern0pt}\isactrlsup {\isasymsharp}\isactrlsup {\isasymflat}{\isacharparenright}{\kern0pt}\ {\isasymcirc}\isactrlsub c\ \ {\isacharparenleft}{\kern0pt}id\ X\ {\isasymtimes}\isactrlsub f\ h{\isacharparenright}{\kern0pt}{\isacharparenright}{\kern0pt}\ {\isasymcirc}\isactrlsub c\ {\isasymlangle}x{\isacharcomma}{\kern0pt}z{\isasymrangle}{\isachardoublequoteclose}\isanewline
\ \ \ \ \ \ \ \ \ \ \ \ \ \ \isacommand{using}\isamarkupfalse%
\ inv{\isacharunderscore}{\kern0pt}transpose{\isacharunderscore}{\kern0pt}of{\isacharunderscore}{\kern0pt}composition\ \isacommand{by}\isamarkupfalse%
\ {\isacharparenleft}{\kern0pt}typecheck{\isacharunderscore}{\kern0pt}cfuncs{\isacharcomma}{\kern0pt}\ presburger{\isacharparenright}{\kern0pt}\isanewline
\ \ \ \ \ \ \ \ \ \ \ \ \isacommand{also}\isamarkupfalse%
\ \isacommand{have}\isamarkupfalse%
\ {\isachardoublequoteopen}{\isachardot}{\kern0pt}{\isachardot}{\kern0pt}{\isachardot}{\kern0pt}\ {\isacharequal}{\kern0pt}\ {\isacharparenleft}{\kern0pt}{\isacharparenleft}{\kern0pt}eval{\isacharunderscore}{\kern0pt}func\ Z\ {\isacharparenleft}{\kern0pt}X\ {\isasymCoprod}\ Y{\isacharparenright}{\kern0pt}\ {\isasymcirc}\isactrlsub c\ left{\isacharunderscore}{\kern0pt}coproj\ X\ Y\ {\isasymtimes}\isactrlsub f\ id\isactrlsub c\ {\isacharparenleft}{\kern0pt}Z\isactrlbsup {\isacharparenleft}{\kern0pt}X\ {\isasymCoprod}\ Y{\isacharparenright}{\kern0pt}\isactrlesup {\isacharparenright}{\kern0pt}{\isacharparenright}{\kern0pt}\ {\isasymcirc}\isactrlsub c\ \ {\isacharparenleft}{\kern0pt}id\ X\ {\isasymtimes}\isactrlsub f\ h{\isacharparenright}{\kern0pt}{\isacharparenright}{\kern0pt}\ {\isasymcirc}\isactrlsub c\ {\isasymlangle}x{\isacharcomma}{\kern0pt}z{\isasymrangle}{\isachardoublequoteclose}\isanewline
\ \ \ \ \ \ \ \ \ \ \ \ \ \ \isacommand{by}\isamarkupfalse%
\ {\isacharparenleft}{\kern0pt}typecheck{\isacharunderscore}{\kern0pt}cfuncs{\isacharcomma}{\kern0pt}\ simp\ add{\isacharcolon}{\kern0pt}\ flat{\isacharunderscore}{\kern0pt}cancels{\isacharunderscore}{\kern0pt}sharp{\isacharparenright}{\kern0pt}\isanewline
\ \ \ \ \ \ \ \ \ \ \ \ \isacommand{also}\isamarkupfalse%
\ \isacommand{have}\isamarkupfalse%
\ {\isachardoublequoteopen}{\isachardot}{\kern0pt}{\isachardot}{\kern0pt}{\isachardot}{\kern0pt}\ {\isacharequal}{\kern0pt}\ {\isacharparenleft}{\kern0pt}eval{\isacharunderscore}{\kern0pt}func\ Z\ {\isacharparenleft}{\kern0pt}X\ {\isasymCoprod}\ Y{\isacharparenright}{\kern0pt}\ {\isasymcirc}\isactrlsub c\ left{\isacharunderscore}{\kern0pt}coproj\ X\ Y\ {\isasymtimes}\isactrlsub f\ h{\isacharparenright}{\kern0pt}\ {\isasymcirc}\isactrlsub c\ {\isasymlangle}x{\isacharcomma}{\kern0pt}z{\isasymrangle}{\isachardoublequoteclose}\isanewline
\ \ \ \ \ \ \ \ \ \ \ \ \ \ \isacommand{by}\isamarkupfalse%
\ {\isacharparenleft}{\kern0pt}typecheck{\isacharunderscore}{\kern0pt}cfuncs{\isacharcomma}{\kern0pt}\ smt\ {\isacharparenleft}{\kern0pt}z{\isadigit{3}}{\isacharparenright}{\kern0pt}\ cfunc{\isacharunderscore}{\kern0pt}cross{\isacharunderscore}{\kern0pt}prod{\isacharunderscore}{\kern0pt}comp{\isacharunderscore}{\kern0pt}cfunc{\isacharunderscore}{\kern0pt}cross{\isacharunderscore}{\kern0pt}prod\ comp{\isacharunderscore}{\kern0pt}associative{\isadigit{2}}\ id{\isacharunderscore}{\kern0pt}left{\isacharunderscore}{\kern0pt}unit{\isadigit{2}}\ id{\isacharunderscore}{\kern0pt}right{\isacharunderscore}{\kern0pt}unit{\isadigit{2}}{\isacharparenright}{\kern0pt}\isanewline
\ \ \ \ \ \ \ \ \ \ \ \ \isacommand{also}\isamarkupfalse%
\ \isacommand{have}\isamarkupfalse%
\ {\isachardoublequoteopen}{\isachardot}{\kern0pt}{\isachardot}{\kern0pt}{\isachardot}{\kern0pt}\ {\isacharequal}{\kern0pt}\ eval{\isacharunderscore}{\kern0pt}func\ Z\ {\isacharparenleft}{\kern0pt}X\ {\isasymCoprod}\ Y{\isacharparenright}{\kern0pt}\ {\isasymcirc}\isactrlsub c\ \ {\isasymlangle}left{\isacharunderscore}{\kern0pt}coproj\ X\ Y\ {\isasymcirc}\isactrlsub c\ x{\isacharcomma}{\kern0pt}\ h\ {\isasymcirc}\isactrlsub c\ z{\isasymrangle}{\isachardoublequoteclose}\isanewline
\ \ \ \ \ \ \ \ \ \ \ \ \ \ \isacommand{by}\isamarkupfalse%
\ {\isacharparenleft}{\kern0pt}typecheck{\isacharunderscore}{\kern0pt}cfuncs{\isacharcomma}{\kern0pt}\ smt\ {\isacharparenleft}{\kern0pt}z{\isadigit{3}}{\isacharparenright}{\kern0pt}\ cfunc{\isacharunderscore}{\kern0pt}cross{\isacharunderscore}{\kern0pt}prod{\isacharunderscore}{\kern0pt}comp{\isacharunderscore}{\kern0pt}cfunc{\isacharunderscore}{\kern0pt}prod\ comp{\isacharunderscore}{\kern0pt}associative{\isadigit{2}}{\isacharparenright}{\kern0pt}\isanewline
\ \ \ \ \ \ \ \ \ \ \ \ \isacommand{also}\isamarkupfalse%
\ \isacommand{have}\isamarkupfalse%
\ {\isachardoublequoteopen}{\isachardot}{\kern0pt}{\isachardot}{\kern0pt}{\isachardot}{\kern0pt}\ {\isacharequal}{\kern0pt}\ eval{\isacharunderscore}{\kern0pt}func\ Z\ {\isacharparenleft}{\kern0pt}X\ {\isasymCoprod}\ Y{\isacharparenright}{\kern0pt}\ {\isasymcirc}\isactrlsub c\ \ {\isacharparenleft}{\kern0pt}{\isacharparenleft}{\kern0pt}id{\isacharparenleft}{\kern0pt}X\ {\isasymCoprod}\ Y{\isacharparenright}{\kern0pt}\ {\isasymtimes}\isactrlsub f\ h{\isacharparenright}{\kern0pt}\ {\isasymcirc}\isactrlsub c\ {\isasymlangle}xy{\isacharcomma}{\kern0pt}z{\isasymrangle}{\isacharparenright}{\kern0pt}{\isachardoublequoteclose}\isanewline
\ \ \ \ \ \ \ \ \ \ \ \ \ \ \isacommand{by}\isamarkupfalse%
\ {\isacharparenleft}{\kern0pt}typecheck{\isacharunderscore}{\kern0pt}cfuncs{\isacharcomma}{\kern0pt}\ simp\ add{\isacharcolon}{\kern0pt}\ cfunc{\isacharunderscore}{\kern0pt}cross{\isacharunderscore}{\kern0pt}prod{\isacharunderscore}{\kern0pt}comp{\isacharunderscore}{\kern0pt}cfunc{\isacharunderscore}{\kern0pt}prod\ id{\isacharunderscore}{\kern0pt}left{\isacharunderscore}{\kern0pt}unit{\isadigit{2}}\ xy{\isacharunderscore}{\kern0pt}def{\isacharparenright}{\kern0pt}\isanewline
\ \ \ \ \ \ \ \ \ \ \ \ \isacommand{also}\isamarkupfalse%
\ \isacommand{have}\isamarkupfalse%
\ {\isachardoublequoteopen}{\isachardot}{\kern0pt}{\isachardot}{\kern0pt}{\isachardot}{\kern0pt}\ {\isacharequal}{\kern0pt}\ h\isactrlsup {\isasymflat}\ {\isasymcirc}\isactrlsub c\ xyh{\isachardoublequoteclose}\isanewline
\ \ \ \ \ \ \ \ \ \ \ \ \ \ \isacommand{by}\isamarkupfalse%
\ {\isacharparenleft}{\kern0pt}typecheck{\isacharunderscore}{\kern0pt}cfuncs{\isacharcomma}{\kern0pt}\ simp\ add{\isacharcolon}{\kern0pt}\ comp{\isacharunderscore}{\kern0pt}associative{\isadigit{2}}\ inv{\isacharunderscore}{\kern0pt}transpose{\isacharunderscore}{\kern0pt}func{\isacharunderscore}{\kern0pt}def{\isadigit{3}}\ xyh{\isacharunderscore}{\kern0pt}def{\isacharparenright}{\kern0pt}\isanewline
\ \ \ \ \ \ \ \ \ \ \ \ \isacommand{then}\isamarkupfalse%
\ \isacommand{show}\isamarkupfalse%
\ {\isacharquery}{\kern0pt}thesis\isanewline
\ \ \ \ \ \ \ \ \ \ \ \ \ \ \isacommand{by}\isamarkupfalse%
\ {\isacharparenleft}{\kern0pt}simp\ add{\isacharcolon}{\kern0pt}\ calculation{\isacharparenright}{\kern0pt}\isanewline
\ \ \ \ \ \ \ \ \ \ \isacommand{next}\isamarkupfalse%
\isanewline
\ \ \ \ \ \ \ \ \ \ \ \ \isacommand{assume}\isamarkupfalse%
\ {\isachardoublequoteopen}{\isasymnexists}x{\isachardot}{\kern0pt}\ x\ {\isasymin}\isactrlsub c\ X\ {\isasymand}\ xy\ {\isacharequal}{\kern0pt}\ left{\isacharunderscore}{\kern0pt}coproj\ X\ Y\ {\isasymcirc}\isactrlsub c\ x{\isachardoublequoteclose}\isanewline
\ \ \ \ \ \ \ \ \ \ \ \ \isacommand{then}\isamarkupfalse%
\ \isacommand{obtain}\isamarkupfalse%
\ y\ \isakeyword{where}\ y{\isacharunderscore}{\kern0pt}type{\isacharbrackleft}{\kern0pt}type{\isacharunderscore}{\kern0pt}rule{\isacharbrackright}{\kern0pt}{\isacharcolon}{\kern0pt}\ {\isachardoublequoteopen}y\ {\isasymin}\isactrlsub c\ Y{\isachardoublequoteclose}\ \isakeyword{and}\ xy{\isacharunderscore}{\kern0pt}def{\isacharcolon}{\kern0pt}\ {\isachardoublequoteopen}xy\ {\isacharequal}{\kern0pt}\ \ right{\isacharunderscore}{\kern0pt}coproj\ X\ Y\ {\isasymcirc}\isactrlsub c\ y{\isachardoublequoteclose}\isanewline
\ \ \ \ \ \ \ \ \ \ \ \ \ \ \isacommand{using}\isamarkupfalse%
\ \ coprojs{\isacharunderscore}{\kern0pt}jointly{\isacharunderscore}{\kern0pt}surj\ \isacommand{by}\isamarkupfalse%
\ {\isacharparenleft}{\kern0pt}typecheck{\isacharunderscore}{\kern0pt}cfuncs{\isacharcomma}{\kern0pt}\ blast{\isacharparenright}{\kern0pt}\isanewline
\ \ \ \ \ \ \ \ \ \ \ \ \isacommand{have}\isamarkupfalse%
\ {\isachardoublequoteopen}{\isacharparenleft}{\kern0pt}f\isactrlsup {\isasymflat}\ {\isasymamalg}\ g\isactrlsup {\isasymflat}\ {\isasymcirc}\isactrlsub c\ dist{\isacharunderscore}{\kern0pt}prod{\isacharunderscore}{\kern0pt}coprod{\isacharunderscore}{\kern0pt}right\ X\ Y\ H{\isacharparenright}{\kern0pt}\ {\isasymcirc}\isactrlsub c\ xyh\ {\isacharequal}{\kern0pt}\ {\isacharparenleft}{\kern0pt}f\isactrlsup {\isasymflat}\ {\isasymamalg}\ g\isactrlsup {\isasymflat}{\isacharparenright}{\kern0pt}\ {\isasymcirc}\isactrlsub c\ {\isacharparenleft}{\kern0pt}dist{\isacharunderscore}{\kern0pt}prod{\isacharunderscore}{\kern0pt}coprod{\isacharunderscore}{\kern0pt}right\ X\ Y\ H\ \ {\isasymcirc}\isactrlsub c\ {\isasymlangle}right{\isacharunderscore}{\kern0pt}coproj\ X\ Y\ {\isasymcirc}\isactrlsub c\ y{\isacharcomma}{\kern0pt}z{\isasymrangle}{\isacharparenright}{\kern0pt}{\isachardoublequoteclose}\isanewline
\ \ \ \ \ \ \ \ \ \ \ \ \ \ \isacommand{by}\isamarkupfalse%
\ {\isacharparenleft}{\kern0pt}typecheck{\isacharunderscore}{\kern0pt}cfuncs{\isacharcomma}{\kern0pt}\ simp\ add{\isacharcolon}{\kern0pt}\ comp{\isacharunderscore}{\kern0pt}associative{\isadigit{2}}\ xy{\isacharunderscore}{\kern0pt}def\ xyh{\isacharunderscore}{\kern0pt}def{\isacharparenright}{\kern0pt}\isanewline
\ \ \ \ \ \ \ \ \ \ \ \ \isacommand{also}\isamarkupfalse%
\ \isacommand{have}\isamarkupfalse%
\ {\isachardoublequoteopen}{\isachardot}{\kern0pt}{\isachardot}{\kern0pt}{\isachardot}{\kern0pt}\ {\isacharequal}{\kern0pt}\ {\isacharparenleft}{\kern0pt}f\isactrlsup {\isasymflat}\ {\isasymamalg}\ g\isactrlsup {\isasymflat}{\isacharparenright}{\kern0pt}\ {\isasymcirc}\isactrlsub c\ {\isacharparenleft}{\kern0pt}{\isacharparenleft}{\kern0pt}dist{\isacharunderscore}{\kern0pt}prod{\isacharunderscore}{\kern0pt}coprod{\isacharunderscore}{\kern0pt}right\ X\ Y\ H\ \ {\isasymcirc}\isactrlsub c\ {\isacharparenleft}{\kern0pt}right{\isacharunderscore}{\kern0pt}coproj\ X\ Y\ {\isasymtimes}\isactrlsub f\ id\ H{\isacharparenright}{\kern0pt}{\isacharparenright}{\kern0pt}\ {\isasymcirc}\isactrlsub c\ {\isasymlangle}y{\isacharcomma}{\kern0pt}z{\isasymrangle}{\isacharparenright}{\kern0pt}{\isachardoublequoteclose}\isanewline
\ \ \ \ \ \ \ \ \ \ \ \ \ \ \isacommand{using}\isamarkupfalse%
\ dist{\isacharunderscore}{\kern0pt}prod{\isacharunderscore}{\kern0pt}coprod{\isacharunderscore}{\kern0pt}right{\isacharunderscore}{\kern0pt}ap{\isacharunderscore}{\kern0pt}right\ dist{\isacharunderscore}{\kern0pt}prod{\isacharunderscore}{\kern0pt}coprod{\isacharunderscore}{\kern0pt}right{\isacharunderscore}{\kern0pt}right{\isacharunderscore}{\kern0pt}coproj\ \isacommand{by}\isamarkupfalse%
\ {\isacharparenleft}{\kern0pt}typecheck{\isacharunderscore}{\kern0pt}cfuncs{\isacharcomma}{\kern0pt}\ presburger{\isacharparenright}{\kern0pt}\isanewline
\ \ \ \ \ \ \ \ \ \ \ \ \isacommand{also}\isamarkupfalse%
\ \isacommand{have}\isamarkupfalse%
\ {\isachardoublequoteopen}{\isachardot}{\kern0pt}{\isachardot}{\kern0pt}{\isachardot}{\kern0pt}\ {\isacharequal}{\kern0pt}\ {\isacharparenleft}{\kern0pt}f\isactrlsup {\isasymflat}\ {\isasymamalg}\ g\isactrlsup {\isasymflat}{\isacharparenright}{\kern0pt}\ {\isasymcirc}\isactrlsub c\ {\isacharparenleft}{\kern0pt}right{\isacharunderscore}{\kern0pt}coproj\ {\isacharparenleft}{\kern0pt}X\ {\isasymtimes}\isactrlsub c\ H{\isacharparenright}{\kern0pt}\ {\isacharparenleft}{\kern0pt}Y\ {\isasymtimes}\isactrlsub c\ H{\isacharparenright}{\kern0pt}\ \ {\isasymcirc}\isactrlsub c\ {\isasymlangle}y{\isacharcomma}{\kern0pt}z{\isasymrangle}{\isacharparenright}{\kern0pt}{\isachardoublequoteclose}\isanewline
\ \ \ \ \ \ \ \ \ \ \ \ \ \ \isacommand{using}\isamarkupfalse%
\ dist{\isacharunderscore}{\kern0pt}prod{\isacharunderscore}{\kern0pt}coprod{\isacharunderscore}{\kern0pt}right{\isacharunderscore}{\kern0pt}right{\isacharunderscore}{\kern0pt}coproj\ \isacommand{by}\isamarkupfalse%
\ presburger\isanewline
\ \ \ \ \ \ \ \ \ \ \ \ \isacommand{also}\isamarkupfalse%
\ \isacommand{have}\isamarkupfalse%
\ {\isachardoublequoteopen}{\isachardot}{\kern0pt}{\isachardot}{\kern0pt}{\isachardot}{\kern0pt}\ {\isacharequal}{\kern0pt}\ g\isactrlsup {\isasymflat}\ {\isasymcirc}\isactrlsub c\ {\isasymlangle}y{\isacharcomma}{\kern0pt}z{\isasymrangle}{\isachardoublequoteclose}\isanewline
\ \ \ \ \ \ \ \ \ \ \ \ \ \ \isacommand{by}\isamarkupfalse%
\ {\isacharparenleft}{\kern0pt}typecheck{\isacharunderscore}{\kern0pt}cfuncs{\isacharcomma}{\kern0pt}\ \ simp\ add{\isacharcolon}{\kern0pt}\ comp{\isacharunderscore}{\kern0pt}associative{\isadigit{2}}\ right{\isacharunderscore}{\kern0pt}coproj{\isacharunderscore}{\kern0pt}cfunc{\isacharunderscore}{\kern0pt}coprod{\isacharparenright}{\kern0pt}\isanewline
\ \ \ \ \ \ \ \ \ \ \ \ \isacommand{also}\isamarkupfalse%
\ \isacommand{have}\isamarkupfalse%
\ {\isachardoublequoteopen}{\isachardot}{\kern0pt}{\isachardot}{\kern0pt}{\isachardot}{\kern0pt}\ {\isacharequal}{\kern0pt}\ {\isacharparenleft}{\kern0pt}{\isacharparenleft}{\kern0pt}eval{\isacharunderscore}{\kern0pt}func\ Z\ {\isacharparenleft}{\kern0pt}X\ {\isasymCoprod}\ Y{\isacharparenright}{\kern0pt}\ {\isasymcirc}\isactrlsub c\ right{\isacharunderscore}{\kern0pt}coproj\ X\ Y\ {\isasymtimes}\isactrlsub f\ id\isactrlsub c\ {\isacharparenleft}{\kern0pt}Z\isactrlbsup {\isacharparenleft}{\kern0pt}X\ {\isasymCoprod}\ Y{\isacharparenright}{\kern0pt}\isactrlesup {\isacharparenright}{\kern0pt}{\isacharparenright}{\kern0pt}\isactrlsup {\isasymsharp}\ {\isasymcirc}\isactrlsub c\ \ h{\isacharparenright}{\kern0pt}\isactrlsup {\isasymflat}\ \ {\isasymcirc}\isactrlsub c\ {\isasymlangle}y{\isacharcomma}{\kern0pt}z{\isasymrangle}{\isachardoublequoteclose}\isanewline
\ \ \ \ \ \ \ \ \ \ \ \ \ \ \isacommand{using}\isamarkupfalse%
\ g{\isacharunderscore}{\kern0pt}eqs\ \isacommand{by}\isamarkupfalse%
\ fastforce\isanewline
\ \ \ \ \ \ \ \ \ \ \ \ \isacommand{also}\isamarkupfalse%
\ \isacommand{have}\isamarkupfalse%
\ {\isachardoublequoteopen}{\isachardot}{\kern0pt}{\isachardot}{\kern0pt}{\isachardot}{\kern0pt}\ {\isacharequal}{\kern0pt}\ {\isacharparenleft}{\kern0pt}{\isacharparenleft}{\kern0pt}{\isacharparenleft}{\kern0pt}eval{\isacharunderscore}{\kern0pt}func\ Z\ {\isacharparenleft}{\kern0pt}X\ {\isasymCoprod}\ Y{\isacharparenright}{\kern0pt}\ {\isasymcirc}\isactrlsub c\ right{\isacharunderscore}{\kern0pt}coproj\ X\ Y\ {\isasymtimes}\isactrlsub f\ id\isactrlsub c\ {\isacharparenleft}{\kern0pt}Z\isactrlbsup {\isacharparenleft}{\kern0pt}X\ {\isasymCoprod}\ Y{\isacharparenright}{\kern0pt}\isactrlesup {\isacharparenright}{\kern0pt}{\isacharparenright}{\kern0pt}\isactrlsup {\isasymsharp}\isactrlsup {\isasymflat}{\isacharparenright}{\kern0pt}\ {\isasymcirc}\isactrlsub c\ \ {\isacharparenleft}{\kern0pt}id\ Y\ {\isasymtimes}\isactrlsub f\ h{\isacharparenright}{\kern0pt}{\isacharparenright}{\kern0pt}\ {\isasymcirc}\isactrlsub c\ {\isasymlangle}y{\isacharcomma}{\kern0pt}z{\isasymrangle}{\isachardoublequoteclose}\isanewline
\ \ \ \ \ \ \ \ \ \ \ \ \ \ \isacommand{using}\isamarkupfalse%
\ inv{\isacharunderscore}{\kern0pt}transpose{\isacharunderscore}{\kern0pt}of{\isacharunderscore}{\kern0pt}composition\ \isacommand{by}\isamarkupfalse%
\ {\isacharparenleft}{\kern0pt}typecheck{\isacharunderscore}{\kern0pt}cfuncs{\isacharcomma}{\kern0pt}\ presburger{\isacharparenright}{\kern0pt}\isanewline
\ \ \ \ \ \ \ \ \ \ \ \ \isacommand{also}\isamarkupfalse%
\ \isacommand{have}\isamarkupfalse%
\ {\isachardoublequoteopen}{\isachardot}{\kern0pt}{\isachardot}{\kern0pt}{\isachardot}{\kern0pt}\ {\isacharequal}{\kern0pt}\ {\isacharparenleft}{\kern0pt}{\isacharparenleft}{\kern0pt}eval{\isacharunderscore}{\kern0pt}func\ Z\ {\isacharparenleft}{\kern0pt}X\ {\isasymCoprod}\ Y{\isacharparenright}{\kern0pt}\ {\isasymcirc}\isactrlsub c\ right{\isacharunderscore}{\kern0pt}coproj\ X\ Y\ {\isasymtimes}\isactrlsub f\ id\isactrlsub c\ {\isacharparenleft}{\kern0pt}Z\isactrlbsup {\isacharparenleft}{\kern0pt}X\ {\isasymCoprod}\ Y{\isacharparenright}{\kern0pt}\isactrlesup {\isacharparenright}{\kern0pt}{\isacharparenright}{\kern0pt}\ {\isasymcirc}\isactrlsub c\ \ {\isacharparenleft}{\kern0pt}id\ Y\ {\isasymtimes}\isactrlsub f\ h{\isacharparenright}{\kern0pt}{\isacharparenright}{\kern0pt}\ {\isasymcirc}\isactrlsub c\ {\isasymlangle}y{\isacharcomma}{\kern0pt}z{\isasymrangle}{\isachardoublequoteclose}\isanewline
\ \ \ \ \ \ \ \ \ \ \ \ \ \ \isacommand{by}\isamarkupfalse%
\ {\isacharparenleft}{\kern0pt}typecheck{\isacharunderscore}{\kern0pt}cfuncs{\isacharcomma}{\kern0pt}\ simp\ add{\isacharcolon}{\kern0pt}\ flat{\isacharunderscore}{\kern0pt}cancels{\isacharunderscore}{\kern0pt}sharp{\isacharparenright}{\kern0pt}\isanewline
\ \ \ \ \ \ \ \ \ \ \ \ \isacommand{also}\isamarkupfalse%
\ \isacommand{have}\isamarkupfalse%
\ {\isachardoublequoteopen}{\isachardot}{\kern0pt}{\isachardot}{\kern0pt}{\isachardot}{\kern0pt}\ {\isacharequal}{\kern0pt}\ {\isacharparenleft}{\kern0pt}eval{\isacharunderscore}{\kern0pt}func\ Z\ {\isacharparenleft}{\kern0pt}X\ {\isasymCoprod}\ Y{\isacharparenright}{\kern0pt}\ {\isasymcirc}\isactrlsub c\ right{\isacharunderscore}{\kern0pt}coproj\ X\ Y\ {\isasymtimes}\isactrlsub f\ h{\isacharparenright}{\kern0pt}\ {\isasymcirc}\isactrlsub c\ {\isasymlangle}y{\isacharcomma}{\kern0pt}z{\isasymrangle}{\isachardoublequoteclose}\isanewline
\ \ \ \ \ \ \ \ \ \ \ \ \ \ \isacommand{by}\isamarkupfalse%
\ {\isacharparenleft}{\kern0pt}typecheck{\isacharunderscore}{\kern0pt}cfuncs{\isacharcomma}{\kern0pt}\ smt\ {\isacharparenleft}{\kern0pt}z{\isadigit{3}}{\isacharparenright}{\kern0pt}\ cfunc{\isacharunderscore}{\kern0pt}cross{\isacharunderscore}{\kern0pt}prod{\isacharunderscore}{\kern0pt}comp{\isacharunderscore}{\kern0pt}cfunc{\isacharunderscore}{\kern0pt}cross{\isacharunderscore}{\kern0pt}prod\ comp{\isacharunderscore}{\kern0pt}associative{\isadigit{2}}\ id{\isacharunderscore}{\kern0pt}left{\isacharunderscore}{\kern0pt}unit{\isadigit{2}}\ id{\isacharunderscore}{\kern0pt}right{\isacharunderscore}{\kern0pt}unit{\isadigit{2}}{\isacharparenright}{\kern0pt}\isanewline
\ \ \ \ \ \ \ \ \ \ \ \ \isacommand{also}\isamarkupfalse%
\ \isacommand{have}\isamarkupfalse%
\ {\isachardoublequoteopen}{\isachardot}{\kern0pt}{\isachardot}{\kern0pt}{\isachardot}{\kern0pt}\ {\isacharequal}{\kern0pt}\ eval{\isacharunderscore}{\kern0pt}func\ Z\ {\isacharparenleft}{\kern0pt}X\ {\isasymCoprod}\ Y{\isacharparenright}{\kern0pt}\ {\isasymcirc}\isactrlsub c\ \ {\isasymlangle}right{\isacharunderscore}{\kern0pt}coproj\ X\ Y\ {\isasymcirc}\isactrlsub c\ y{\isacharcomma}{\kern0pt}\ h\ {\isasymcirc}\isactrlsub c\ z{\isasymrangle}{\isachardoublequoteclose}\isanewline
\ \ \ \ \ \ \ \ \ \ \ \ \ \ \isacommand{by}\isamarkupfalse%
\ {\isacharparenleft}{\kern0pt}typecheck{\isacharunderscore}{\kern0pt}cfuncs{\isacharcomma}{\kern0pt}\ smt\ {\isacharparenleft}{\kern0pt}z{\isadigit{3}}{\isacharparenright}{\kern0pt}\ cfunc{\isacharunderscore}{\kern0pt}cross{\isacharunderscore}{\kern0pt}prod{\isacharunderscore}{\kern0pt}comp{\isacharunderscore}{\kern0pt}cfunc{\isacharunderscore}{\kern0pt}prod\ comp{\isacharunderscore}{\kern0pt}associative{\isadigit{2}}{\isacharparenright}{\kern0pt}\isanewline
\ \ \ \ \ \ \ \ \ \ \ \ \isacommand{also}\isamarkupfalse%
\ \isacommand{have}\isamarkupfalse%
\ {\isachardoublequoteopen}{\isachardot}{\kern0pt}{\isachardot}{\kern0pt}{\isachardot}{\kern0pt}\ {\isacharequal}{\kern0pt}\ eval{\isacharunderscore}{\kern0pt}func\ Z\ {\isacharparenleft}{\kern0pt}X\ {\isasymCoprod}\ Y{\isacharparenright}{\kern0pt}\ {\isasymcirc}\isactrlsub c\ \ {\isacharparenleft}{\kern0pt}{\isacharparenleft}{\kern0pt}id{\isacharparenleft}{\kern0pt}X\ {\isasymCoprod}\ Y{\isacharparenright}{\kern0pt}\ {\isasymtimes}\isactrlsub f\ h{\isacharparenright}{\kern0pt}\ {\isasymcirc}\isactrlsub c\ {\isasymlangle}xy{\isacharcomma}{\kern0pt}z{\isasymrangle}{\isacharparenright}{\kern0pt}{\isachardoublequoteclose}\isanewline
\ \ \ \ \ \ \ \ \ \ \ \ \ \ \isacommand{by}\isamarkupfalse%
\ {\isacharparenleft}{\kern0pt}typecheck{\isacharunderscore}{\kern0pt}cfuncs{\isacharcomma}{\kern0pt}\ simp\ add{\isacharcolon}{\kern0pt}\ cfunc{\isacharunderscore}{\kern0pt}cross{\isacharunderscore}{\kern0pt}prod{\isacharunderscore}{\kern0pt}comp{\isacharunderscore}{\kern0pt}cfunc{\isacharunderscore}{\kern0pt}prod\ id{\isacharunderscore}{\kern0pt}left{\isacharunderscore}{\kern0pt}unit{\isadigit{2}}\ xy{\isacharunderscore}{\kern0pt}def{\isacharparenright}{\kern0pt}\isanewline
\ \ \ \ \ \ \ \ \ \ \ \ \isacommand{also}\isamarkupfalse%
\ \isacommand{have}\isamarkupfalse%
\ {\isachardoublequoteopen}{\isachardot}{\kern0pt}{\isachardot}{\kern0pt}{\isachardot}{\kern0pt}\ {\isacharequal}{\kern0pt}\ h\isactrlsup {\isasymflat}\ {\isasymcirc}\isactrlsub c\ xyh{\isachardoublequoteclose}\isanewline
\ \ \ \ \ \ \ \ \ \ \ \ \ \ \isacommand{by}\isamarkupfalse%
\ {\isacharparenleft}{\kern0pt}typecheck{\isacharunderscore}{\kern0pt}cfuncs{\isacharcomma}{\kern0pt}\ simp\ add{\isacharcolon}{\kern0pt}\ comp{\isacharunderscore}{\kern0pt}associative{\isadigit{2}}\ inv{\isacharunderscore}{\kern0pt}transpose{\isacharunderscore}{\kern0pt}func{\isacharunderscore}{\kern0pt}def{\isadigit{3}}\ xyh{\isacharunderscore}{\kern0pt}def{\isacharparenright}{\kern0pt}\isanewline
\ \ \ \ \ \ \ \ \ \ \ \ \isacommand{then}\isamarkupfalse%
\ \isacommand{show}\isamarkupfalse%
\ {\isacharquery}{\kern0pt}thesis\isanewline
\ \ \ \ \ \ \ \ \ \ \ \ \ \ \isacommand{by}\isamarkupfalse%
\ {\isacharparenleft}{\kern0pt}simp\ add{\isacharcolon}{\kern0pt}\ calculation{\isacharparenright}{\kern0pt}\isanewline
\ \ \ \ \ \ \ \ \ \ \isacommand{qed}\isamarkupfalse%
\isanewline
\ \ \ \ \ \ \ \ \isacommand{qed}\isamarkupfalse%
\isanewline
\ \ \ \ \ \ \isacommand{qed}\isamarkupfalse%
\isanewline
\ \ \ \ \ \ \isacommand{then}\isamarkupfalse%
\ \isacommand{show}\isamarkupfalse%
\ {\isachardoublequoteopen}h\ {\isacharequal}{\kern0pt}\ {\isacharparenleft}{\kern0pt}{\isacharparenleft}{\kern0pt}{\isacharparenleft}{\kern0pt}eval{\isacharunderscore}{\kern0pt}func\ Z\ {\isacharparenleft}{\kern0pt}X\ {\isasymCoprod}\ Y{\isacharparenright}{\kern0pt}\ {\isasymcirc}\isactrlsub c\ left{\isacharunderscore}{\kern0pt}coproj\ X\ Y\ {\isasymtimes}\isactrlsub f\ id\isactrlsub c\ {\isacharparenleft}{\kern0pt}Z\isactrlbsup {\isacharparenleft}{\kern0pt}X\ {\isasymCoprod}\ Y{\isacharparenright}{\kern0pt}\isactrlesup {\isacharparenright}{\kern0pt}{\isacharparenright}{\kern0pt}\isactrlsup {\isasymsharp}\ {\isasymcirc}\isactrlsub c\ h{\isacharparenright}{\kern0pt}\isactrlsup {\isasymflat}\ {\isasymamalg}\isanewline
\ \ \ \ \ \ \ \ \ \ \ \ \ \ \ \ \ \ \ \ \ {\isacharparenleft}{\kern0pt}{\isacharparenleft}{\kern0pt}eval{\isacharunderscore}{\kern0pt}func\ Z\ {\isacharparenleft}{\kern0pt}X\ {\isasymCoprod}\ Y{\isacharparenright}{\kern0pt}\ {\isasymcirc}\isactrlsub c\ right{\isacharunderscore}{\kern0pt}coproj\ X\ Y\ {\isasymtimes}\isactrlsub f\ id\isactrlsub c\ {\isacharparenleft}{\kern0pt}Z\isactrlbsup {\isacharparenleft}{\kern0pt}X\ {\isasymCoprod}\ Y{\isacharparenright}{\kern0pt}\isactrlesup {\isacharparenright}{\kern0pt}{\isacharparenright}{\kern0pt}\isactrlsup {\isasymsharp}\ {\isasymcirc}\isactrlsub c\ h{\isacharparenright}{\kern0pt}\isactrlsup {\isasymflat}\ {\isasymcirc}\isactrlsub c\isanewline
\ \ \ \ \ \ \ \ \ \ \ \ \ \ \ \ \ \ \ \ \ \ \ \ \ \ \ \ \ \ \ \ \ \ \ \ \ \ \ \ \ \ \ \ \ \ \ \ \ \ \ \ \ \ \ \ \ \ \ \ \ \ \ \ \ \ \ \ \ \ dist{\isacharunderscore}{\kern0pt}prod{\isacharunderscore}{\kern0pt}coprod{\isacharunderscore}{\kern0pt}right\ X\ Y\ H{\isacharparenright}{\kern0pt}\isactrlsup {\isasymsharp}{\isachardoublequoteclose}\isanewline
\ \ \ \ \ \ \ \ \isacommand{using}\isamarkupfalse%
\ f{\isacharunderscore}{\kern0pt}eqs\ g{\isacharunderscore}{\kern0pt}eqs\ h{\isacharunderscore}{\kern0pt}type\ sharp{\isacharunderscore}{\kern0pt}cancels{\isacharunderscore}{\kern0pt}flat\ \isacommand{by}\isamarkupfalse%
\ force\isanewline
\ \ \ \ \isacommand{qed}\isamarkupfalse%
\isanewline
\ \ \isacommand{qed}\isamarkupfalse%
\isanewline
\ \ \isacommand{then}\isamarkupfalse%
\ \isacommand{show}\isamarkupfalse%
\ {\isacharquery}{\kern0pt}thesis\isanewline
\ \ \ \ \isacommand{by}\isamarkupfalse%
\ {\isacharparenleft}{\kern0pt}metis\ canonical{\isacharunderscore}{\kern0pt}cart{\isacharunderscore}{\kern0pt}prod{\isacharunderscore}{\kern0pt}is{\isacharunderscore}{\kern0pt}cart{\isacharunderscore}{\kern0pt}prod\ cart{\isacharunderscore}{\kern0pt}prods{\isacharunderscore}{\kern0pt}isomorphic\ is{\isacharunderscore}{\kern0pt}isomorphic{\isacharunderscore}{\kern0pt}def\ prod{\isachardot}{\kern0pt}sel{\isacharparenleft}{\kern0pt}{\isadigit{1}}{\isacharcomma}{\kern0pt}{\isadigit{2}}{\isacharparenright}{\kern0pt}{\isacharparenright}{\kern0pt}\isanewline
\isacommand{qed}\isamarkupfalse%
%
\endisatagproof
{\isafoldproof}%
%
\isadelimproof
\isanewline
%
\endisadelimproof
\isanewline
\isacommand{lemma}\isamarkupfalse%
\ empty{\isacharunderscore}{\kern0pt}exp{\isacharunderscore}{\kern0pt}nonempty{\isacharcolon}{\kern0pt}\isanewline
\ \ \isakeyword{assumes}\ {\isachardoublequoteopen}nonempty\ X{\isachardoublequoteclose}\isanewline
\ \ \isakeyword{shows}\ {\isachardoublequoteopen}{\isasymemptyset}\isactrlbsup X\isactrlesup \ {\isasymcong}\ {\isasymemptyset}{\isachardoublequoteclose}\isanewline
%
\isadelimproof
%
\endisadelimproof
%
\isatagproof
\isacommand{proof}\isamarkupfalse%
{\isacharminus}{\kern0pt}\isanewline
\ \ \isacommand{obtain}\isamarkupfalse%
\ j\ \isakeyword{where}\ j{\isacharunderscore}{\kern0pt}type{\isacharbrackleft}{\kern0pt}type{\isacharunderscore}{\kern0pt}rule{\isacharbrackright}{\kern0pt}{\isacharcolon}{\kern0pt}\ {\isachardoublequoteopen}j{\isacharcolon}{\kern0pt}\ {\isasymemptyset}\isactrlbsup X\isactrlesup \ {\isasymrightarrow}\ {\isasymone}{\isasymtimes}\isactrlsub c\ {\isasymemptyset}\isactrlbsup X\isactrlesup {\isachardoublequoteclose}\ \isakeyword{and}\ j{\isacharunderscore}{\kern0pt}def{\isacharcolon}{\kern0pt}\ {\isachardoublequoteopen}isomorphism{\isacharparenleft}{\kern0pt}j{\isacharparenright}{\kern0pt}{\isachardoublequoteclose}\isanewline
\ \ \ \ \isacommand{using}\isamarkupfalse%
\ is{\isacharunderscore}{\kern0pt}isomorphic{\isacharunderscore}{\kern0pt}def\ isomorphic{\isacharunderscore}{\kern0pt}is{\isacharunderscore}{\kern0pt}symmetric\ one{\isacharunderscore}{\kern0pt}x{\isacharunderscore}{\kern0pt}A{\isacharunderscore}{\kern0pt}iso{\isacharunderscore}{\kern0pt}A\ \isacommand{by}\isamarkupfalse%
\ blast\isanewline
\ \ \isacommand{obtain}\isamarkupfalse%
\ y\ \isakeyword{where}\ y{\isacharunderscore}{\kern0pt}type{\isacharbrackleft}{\kern0pt}type{\isacharunderscore}{\kern0pt}rule{\isacharbrackright}{\kern0pt}{\isacharcolon}{\kern0pt}\ {\isachardoublequoteopen}y\ {\isasymin}\isactrlsub c\ X{\isachardoublequoteclose}\isanewline
\ \ \ \ \isacommand{using}\isamarkupfalse%
\ assms\ nonempty{\isacharunderscore}{\kern0pt}def\ \isacommand{by}\isamarkupfalse%
\ blast\isanewline
\ \ \isacommand{obtain}\isamarkupfalse%
\ e\ \isakeyword{where}\ e{\isacharunderscore}{\kern0pt}type{\isacharbrackleft}{\kern0pt}type{\isacharunderscore}{\kern0pt}rule{\isacharbrackright}{\kern0pt}{\isacharcolon}{\kern0pt}\ {\isachardoublequoteopen}e{\isacharcolon}{\kern0pt}\ X{\isasymtimes}\isactrlsub c\ {\isasymemptyset}\isactrlbsup X\isactrlesup \ {\isasymrightarrow}\ {\isasymemptyset}{\isachardoublequoteclose}\isanewline
\ \ \ \ \isacommand{using}\isamarkupfalse%
\ eval{\isacharunderscore}{\kern0pt}func{\isacharunderscore}{\kern0pt}type\ \isacommand{by}\isamarkupfalse%
\ blast\isanewline
\ \ \isacommand{have}\isamarkupfalse%
\ iso{\isacharunderscore}{\kern0pt}type{\isacharbrackleft}{\kern0pt}type{\isacharunderscore}{\kern0pt}rule{\isacharbrackright}{\kern0pt}{\isacharcolon}{\kern0pt}\ {\isachardoublequoteopen}{\isacharparenleft}{\kern0pt}e\ {\isasymcirc}\isactrlsub c\ y\ {\isasymtimes}\isactrlsub f\ id{\isacharparenleft}{\kern0pt}{\isasymemptyset}\isactrlbsup X\isactrlesup {\isacharparenright}{\kern0pt}{\isacharparenright}{\kern0pt}\ {\isasymcirc}\isactrlsub c\ j\ {\isacharcolon}{\kern0pt}\ \ {\isasymemptyset}\isactrlbsup X\isactrlesup \ {\isasymrightarrow}\ {\isasymemptyset}{\isachardoublequoteclose}\isanewline
\ \ \ \ \isacommand{by}\isamarkupfalse%
\ typecheck{\isacharunderscore}{\kern0pt}cfuncs\isanewline
\ \ \isacommand{show}\isamarkupfalse%
\ {\isachardoublequoteopen}{\isasymemptyset}\isactrlbsup X\isactrlesup \ {\isasymcong}\ {\isasymemptyset}{\isachardoublequoteclose}\isanewline
\ \ \ \ \isacommand{using}\isamarkupfalse%
\ function{\isacharunderscore}{\kern0pt}to{\isacharunderscore}{\kern0pt}empty{\isacharunderscore}{\kern0pt}is{\isacharunderscore}{\kern0pt}iso\ is{\isacharunderscore}{\kern0pt}isomorphic{\isacharunderscore}{\kern0pt}def\ iso{\isacharunderscore}{\kern0pt}type\ \isacommand{by}\isamarkupfalse%
\ blast\isanewline
\isacommand{qed}\isamarkupfalse%
%
\endisatagproof
{\isafoldproof}%
%
\isadelimproof
\isanewline
%
\endisadelimproof
\isanewline
\isacommand{lemma}\isamarkupfalse%
\ exp{\isacharunderscore}{\kern0pt}pres{\isacharunderscore}{\kern0pt}iso{\isacharunderscore}{\kern0pt}left{\isacharcolon}{\kern0pt}\isanewline
\ \ \isakeyword{assumes}\ {\isachardoublequoteopen}A\ {\isasymcong}\ X{\isachardoublequoteclose}\ \isanewline
\ \ \isakeyword{shows}\ {\isachardoublequoteopen}A\isactrlbsup Y\isactrlesup \ {\isasymcong}\ \ X\isactrlbsup Y\isactrlesup {\isachardoublequoteclose}\isanewline
%
\isadelimproof
%
\endisadelimproof
%
\isatagproof
\isacommand{proof}\isamarkupfalse%
\ {\isacharminus}{\kern0pt}\ \isanewline
\ \ \isacommand{obtain}\isamarkupfalse%
\ {\isasymphi}\ \isakeyword{where}\ {\isasymphi}{\isacharunderscore}{\kern0pt}def{\isacharcolon}{\kern0pt}\ {\isachardoublequoteopen}{\isasymphi}{\isacharcolon}{\kern0pt}\ X\ {\isasymrightarrow}\ A\ {\isasymand}\ isomorphism{\isacharparenleft}{\kern0pt}{\isasymphi}{\isacharparenright}{\kern0pt}{\isachardoublequoteclose}\isanewline
\ \ \ \ \isacommand{using}\isamarkupfalse%
\ assms\ is{\isacharunderscore}{\kern0pt}isomorphic{\isacharunderscore}{\kern0pt}def\ isomorphic{\isacharunderscore}{\kern0pt}is{\isacharunderscore}{\kern0pt}symmetric\ \isacommand{by}\isamarkupfalse%
\ blast\isanewline
\ \ \isacommand{obtain}\isamarkupfalse%
\ {\isasympsi}\ \isakeyword{where}\ {\isasympsi}{\isacharunderscore}{\kern0pt}def{\isacharcolon}{\kern0pt}\ {\isachardoublequoteopen}{\isasympsi}{\isacharcolon}{\kern0pt}\ A\ {\isasymrightarrow}\ X\ {\isasymand}\ isomorphism{\isacharparenleft}{\kern0pt}{\isasympsi}{\isacharparenright}{\kern0pt}\ {\isasymand}\ {\isacharparenleft}{\kern0pt}{\isasympsi}\ {\isasymcirc}\isactrlsub c\ {\isasymphi}\ {\isacharequal}{\kern0pt}\ id{\isacharparenleft}{\kern0pt}X{\isacharparenright}{\kern0pt}{\isacharparenright}{\kern0pt}{\isachardoublequoteclose}\isanewline
\ \ \ \ \isacommand{using}\isamarkupfalse%
\ {\isasymphi}{\isacharunderscore}{\kern0pt}def\ cfunc{\isacharunderscore}{\kern0pt}type{\isacharunderscore}{\kern0pt}def\ isomorphism{\isacharunderscore}{\kern0pt}def\ \isacommand{by}\isamarkupfalse%
\ fastforce\isanewline
\ \ \isacommand{have}\isamarkupfalse%
\ idA{\isacharcolon}{\kern0pt}\ {\isachardoublequoteopen}{\isasymphi}\ {\isasymcirc}\isactrlsub c\ {\isasympsi}\ {\isacharequal}{\kern0pt}\ id{\isacharparenleft}{\kern0pt}A{\isacharparenright}{\kern0pt}{\isachardoublequoteclose}\isanewline
\ \ \ \ \isacommand{by}\isamarkupfalse%
\ {\isacharparenleft}{\kern0pt}metis\ {\isasymphi}{\isacharunderscore}{\kern0pt}def\ {\isasympsi}{\isacharunderscore}{\kern0pt}def\ cfunc{\isacharunderscore}{\kern0pt}type{\isacharunderscore}{\kern0pt}def\ comp{\isacharunderscore}{\kern0pt}associative\ id{\isacharunderscore}{\kern0pt}left{\isacharunderscore}{\kern0pt}unit{\isadigit{2}}\ isomorphism{\isacharunderscore}{\kern0pt}def{\isacharparenright}{\kern0pt}\isanewline
\ \ \isacommand{have}\isamarkupfalse%
\ phi{\isacharunderscore}{\kern0pt}eval{\isacharunderscore}{\kern0pt}type{\isacharcolon}{\kern0pt}\ {\isachardoublequoteopen}{\isacharparenleft}{\kern0pt}{\isasymphi}\ {\isasymcirc}\isactrlsub c\ eval{\isacharunderscore}{\kern0pt}func\ X\ Y{\isacharparenright}{\kern0pt}\isactrlsup {\isasymsharp}{\isacharcolon}{\kern0pt}\ X\isactrlbsup Y\isactrlesup \ {\isasymrightarrow}\ A\isactrlbsup Y\isactrlesup {\isachardoublequoteclose}\isanewline
\ \ \ \ \isacommand{using}\isamarkupfalse%
\ {\isasymphi}{\isacharunderscore}{\kern0pt}def\ \isacommand{by}\isamarkupfalse%
\ {\isacharparenleft}{\kern0pt}typecheck{\isacharunderscore}{\kern0pt}cfuncs{\isacharcomma}{\kern0pt}\ blast{\isacharparenright}{\kern0pt}\isanewline
\ \ \isacommand{have}\isamarkupfalse%
\ psi{\isacharunderscore}{\kern0pt}eval{\isacharunderscore}{\kern0pt}type{\isacharcolon}{\kern0pt}\ {\isachardoublequoteopen}{\isacharparenleft}{\kern0pt}{\isasympsi}\ {\isasymcirc}\isactrlsub c\ eval{\isacharunderscore}{\kern0pt}func\ A\ Y{\isacharparenright}{\kern0pt}\isactrlsup {\isasymsharp}{\isacharcolon}{\kern0pt}\ A\isactrlbsup Y\isactrlesup \ {\isasymrightarrow}\ X\isactrlbsup Y\isactrlesup {\isachardoublequoteclose}\isanewline
\ \ \ \ \isacommand{using}\isamarkupfalse%
\ {\isasympsi}{\isacharunderscore}{\kern0pt}def\ \isacommand{by}\isamarkupfalse%
\ {\isacharparenleft}{\kern0pt}typecheck{\isacharunderscore}{\kern0pt}cfuncs{\isacharcomma}{\kern0pt}\ blast{\isacharparenright}{\kern0pt}\isanewline
\isanewline
\ \ \isacommand{have}\isamarkupfalse%
\ idXY{\isacharcolon}{\kern0pt}\ {\isachardoublequoteopen}{\isacharparenleft}{\kern0pt}{\isasympsi}\ {\isasymcirc}\isactrlsub c\ eval{\isacharunderscore}{\kern0pt}func\ A\ Y{\isacharparenright}{\kern0pt}\isactrlsup {\isasymsharp}\ {\isasymcirc}\isactrlsub c\ \ {\isacharparenleft}{\kern0pt}{\isasymphi}\ {\isasymcirc}\isactrlsub c\ eval{\isacharunderscore}{\kern0pt}func\ X\ Y{\isacharparenright}{\kern0pt}\isactrlsup {\isasymsharp}\ {\isacharequal}{\kern0pt}\ id{\isacharparenleft}{\kern0pt}X\isactrlbsup Y\isactrlesup {\isacharparenright}{\kern0pt}{\isachardoublequoteclose}\isanewline
\ \ \isacommand{proof}\isamarkupfalse%
\ {\isacharminus}{\kern0pt}\ \isanewline
\ \ \ \ \isacommand{have}\isamarkupfalse%
\ {\isachardoublequoteopen}{\isacharparenleft}{\kern0pt}{\isasympsi}\ {\isasymcirc}\isactrlsub c\ eval{\isacharunderscore}{\kern0pt}func\ A\ Y{\isacharparenright}{\kern0pt}\isactrlsup {\isasymsharp}\ {\isasymcirc}\isactrlsub c\ \ {\isacharparenleft}{\kern0pt}{\isasymphi}\ {\isasymcirc}\isactrlsub c\ eval{\isacharunderscore}{\kern0pt}func\ X\ Y{\isacharparenright}{\kern0pt}\isactrlsup {\isasymsharp}\ {\isacharequal}{\kern0pt}\ \isanewline
\ \ \ \ \ \ \ \ \ \ {\isacharparenleft}{\kern0pt}{\isasympsi}\isactrlbsup Y\isactrlesup \isactrlsub f\ {\isasymcirc}\isactrlsub c\ {\isacharparenleft}{\kern0pt}eval{\isacharunderscore}{\kern0pt}func\ A\ Y{\isacharparenright}{\kern0pt}\isactrlsup {\isasymsharp}{\isacharparenright}{\kern0pt}\ {\isasymcirc}\isactrlsub c\ \ {\isacharparenleft}{\kern0pt}{\isasymphi}\isactrlbsup Y\isactrlesup \isactrlsub f\ {\isasymcirc}\isactrlsub c\ {\isacharparenleft}{\kern0pt}eval{\isacharunderscore}{\kern0pt}func\ X\ Y{\isacharparenright}{\kern0pt}\isactrlsup {\isasymsharp}{\isacharparenright}{\kern0pt}{\isachardoublequoteclose}\isanewline
\ \ \ \ \ \ \isacommand{using}\isamarkupfalse%
\ {\isasymphi}{\isacharunderscore}{\kern0pt}def\ {\isasympsi}{\isacharunderscore}{\kern0pt}def\ exp{\isacharunderscore}{\kern0pt}func{\isacharunderscore}{\kern0pt}def{\isadigit{2}}\ exponential{\isacharunderscore}{\kern0pt}object{\isacharunderscore}{\kern0pt}identity\ id{\isacharunderscore}{\kern0pt}right{\isacharunderscore}{\kern0pt}unit{\isadigit{2}}\ phi{\isacharunderscore}{\kern0pt}eval{\isacharunderscore}{\kern0pt}type\ psi{\isacharunderscore}{\kern0pt}eval{\isacharunderscore}{\kern0pt}type\ \isacommand{by}\isamarkupfalse%
\ auto\isanewline
\ \ \ \ \isacommand{also}\isamarkupfalse%
\ \isacommand{have}\isamarkupfalse%
\ {\isachardoublequoteopen}{\isachardot}{\kern0pt}{\isachardot}{\kern0pt}{\isachardot}{\kern0pt}\ {\isacharequal}{\kern0pt}\ {\isacharparenleft}{\kern0pt}{\isasympsi}\isactrlbsup Y\isactrlesup \isactrlsub f\ {\isasymcirc}\isactrlsub c\ id{\isacharparenleft}{\kern0pt}A\isactrlbsup Y\isactrlesup {\isacharparenright}{\kern0pt}{\isacharparenright}{\kern0pt}\ {\isasymcirc}\isactrlsub c\ \ {\isacharparenleft}{\kern0pt}{\isasymphi}\isactrlbsup Y\isactrlesup \isactrlsub f\ {\isasymcirc}\isactrlsub c\ id{\isacharparenleft}{\kern0pt}X\isactrlbsup Y\isactrlesup {\isacharparenright}{\kern0pt}{\isacharparenright}{\kern0pt}{\isachardoublequoteclose}\isanewline
\ \ \ \ \ \ \isacommand{by}\isamarkupfalse%
\ {\isacharparenleft}{\kern0pt}simp\ add{\isacharcolon}{\kern0pt}\ exponential{\isacharunderscore}{\kern0pt}object{\isacharunderscore}{\kern0pt}identity{\isacharparenright}{\kern0pt}\isanewline
\ \ \ \ \isacommand{also}\isamarkupfalse%
\ \isacommand{have}\isamarkupfalse%
\ {\isachardoublequoteopen}{\isachardot}{\kern0pt}{\isachardot}{\kern0pt}{\isachardot}{\kern0pt}\ {\isacharequal}{\kern0pt}\ {\isasympsi}\isactrlbsup Y\isactrlesup \isactrlsub f\ {\isasymcirc}\isactrlsub c\ {\isacharparenleft}{\kern0pt}id{\isacharparenleft}{\kern0pt}A\isactrlbsup Y\isactrlesup {\isacharparenright}{\kern0pt}\ {\isasymcirc}\isactrlsub c\ \ {\isacharparenleft}{\kern0pt}{\isasymphi}\isactrlbsup Y\isactrlesup \isactrlsub f\ {\isasymcirc}\isactrlsub c\ id{\isacharparenleft}{\kern0pt}X\isactrlbsup Y\isactrlesup {\isacharparenright}{\kern0pt}{\isacharparenright}{\kern0pt}{\isacharparenright}{\kern0pt}{\isachardoublequoteclose}\ \isanewline
\ \ \ \ \ \ \isacommand{by}\isamarkupfalse%
\ {\isacharparenleft}{\kern0pt}typecheck{\isacharunderscore}{\kern0pt}cfuncs{\isacharcomma}{\kern0pt}\ metis\ {\isasymphi}{\isacharunderscore}{\kern0pt}def\ {\isasympsi}{\isacharunderscore}{\kern0pt}def\ comp{\isacharunderscore}{\kern0pt}associative{\isadigit{2}}{\isacharparenright}{\kern0pt}\isanewline
\ \ \ \ \isacommand{also}\isamarkupfalse%
\ \isacommand{have}\isamarkupfalse%
\ {\isachardoublequoteopen}{\isachardot}{\kern0pt}{\isachardot}{\kern0pt}{\isachardot}{\kern0pt}\ {\isacharequal}{\kern0pt}\ {\isasympsi}\isactrlbsup Y\isactrlesup \isactrlsub f\ {\isasymcirc}\isactrlsub c\ {\isacharparenleft}{\kern0pt}id{\isacharparenleft}{\kern0pt}A\isactrlbsup Y\isactrlesup {\isacharparenright}{\kern0pt}\ {\isasymcirc}\isactrlsub c\ \ {\isasymphi}\isactrlbsup Y\isactrlesup \isactrlsub f\ {\isacharparenright}{\kern0pt}{\isachardoublequoteclose}\isanewline
\ \ \ \ \ \ \isacommand{using}\isamarkupfalse%
\ {\isasymphi}{\isacharunderscore}{\kern0pt}def\ exp{\isacharunderscore}{\kern0pt}func{\isacharunderscore}{\kern0pt}def{\isadigit{2}}\ id{\isacharunderscore}{\kern0pt}right{\isacharunderscore}{\kern0pt}unit{\isadigit{2}}\ phi{\isacharunderscore}{\kern0pt}eval{\isacharunderscore}{\kern0pt}type\ \isacommand{by}\isamarkupfalse%
\ auto\isanewline
\ \ \ \ \isacommand{also}\isamarkupfalse%
\ \isacommand{have}\isamarkupfalse%
\ {\isachardoublequoteopen}{\isachardot}{\kern0pt}{\isachardot}{\kern0pt}{\isachardot}{\kern0pt}\ {\isacharequal}{\kern0pt}\ {\isasympsi}\isactrlbsup Y\isactrlesup \isactrlsub f\ {\isasymcirc}\isactrlsub c\ {\isasymphi}\isactrlbsup Y\isactrlesup \isactrlsub f{\isachardoublequoteclose}\isanewline
\ \ \ \ \ \ \isacommand{using}\isamarkupfalse%
\ {\isasymphi}{\isacharunderscore}{\kern0pt}def\ {\isasympsi}{\isacharunderscore}{\kern0pt}def\ calculation\ exp{\isacharunderscore}{\kern0pt}func{\isacharunderscore}{\kern0pt}def{\isadigit{2}}\ \isacommand{by}\isamarkupfalse%
\ auto\isanewline
\ \ \ \ \isacommand{also}\isamarkupfalse%
\ \isacommand{have}\isamarkupfalse%
\ {\isachardoublequoteopen}{\isachardot}{\kern0pt}{\isachardot}{\kern0pt}{\isachardot}{\kern0pt}\ {\isacharequal}{\kern0pt}\ {\isacharparenleft}{\kern0pt}{\isasympsi}\ {\isasymcirc}\isactrlsub c\ {\isasymphi}{\isacharparenright}{\kern0pt}\isactrlbsup Y\isactrlesup \isactrlsub f{\isachardoublequoteclose}\isanewline
\ \ \ \ \ \ \isacommand{by}\isamarkupfalse%
\ {\isacharparenleft}{\kern0pt}metis\ {\isasymphi}{\isacharunderscore}{\kern0pt}def\ {\isasympsi}{\isacharunderscore}{\kern0pt}def\ transpose{\isacharunderscore}{\kern0pt}factors{\isacharparenright}{\kern0pt}\isanewline
\ \ \ \ \isacommand{also}\isamarkupfalse%
\ \isacommand{have}\isamarkupfalse%
\ {\isachardoublequoteopen}{\isachardot}{\kern0pt}{\isachardot}{\kern0pt}{\isachardot}{\kern0pt}\ {\isacharequal}{\kern0pt}\ {\isacharparenleft}{\kern0pt}id\ X{\isacharparenright}{\kern0pt}\isactrlbsup Y\isactrlesup \isactrlsub f{\isachardoublequoteclose}\isanewline
\ \ \ \ \ \ \isacommand{by}\isamarkupfalse%
\ {\isacharparenleft}{\kern0pt}simp\ add{\isacharcolon}{\kern0pt}\ {\isasympsi}{\isacharunderscore}{\kern0pt}def{\isacharparenright}{\kern0pt}\isanewline
\ \ \ \ \isacommand{also}\isamarkupfalse%
\ \isacommand{have}\isamarkupfalse%
\ {\isachardoublequoteopen}{\isachardot}{\kern0pt}{\isachardot}{\kern0pt}{\isachardot}{\kern0pt}\ \ {\isacharequal}{\kern0pt}\ id{\isacharparenleft}{\kern0pt}X\isactrlbsup Y\isactrlesup {\isacharparenright}{\kern0pt}{\isachardoublequoteclose}\isanewline
\ \ \ \ \ \ \isacommand{by}\isamarkupfalse%
\ {\isacharparenleft}{\kern0pt}simp\ add{\isacharcolon}{\kern0pt}\ exponential{\isacharunderscore}{\kern0pt}object{\isacharunderscore}{\kern0pt}identity{\isadigit{2}}{\isacharparenright}{\kern0pt}\isanewline
\ \ \ \ \isacommand{then}\isamarkupfalse%
\ \isacommand{show}\isamarkupfalse%
\ {\isachardoublequoteopen}{\isacharparenleft}{\kern0pt}{\isasympsi}\ {\isasymcirc}\isactrlsub c\ eval{\isacharunderscore}{\kern0pt}func\ A\ Y{\isacharparenright}{\kern0pt}\isactrlsup {\isasymsharp}\ {\isasymcirc}\isactrlsub c\ \ {\isacharparenleft}{\kern0pt}{\isasymphi}\ {\isasymcirc}\isactrlsub c\ eval{\isacharunderscore}{\kern0pt}func\ X\ Y{\isacharparenright}{\kern0pt}\isactrlsup {\isasymsharp}\ {\isacharequal}{\kern0pt}\ id{\isacharparenleft}{\kern0pt}X\isactrlbsup Y\isactrlesup {\isacharparenright}{\kern0pt}{\isachardoublequoteclose}\isanewline
\ \ \ \ \ \ \isacommand{by}\isamarkupfalse%
\ {\isacharparenleft}{\kern0pt}simp\ add{\isacharcolon}{\kern0pt}\ calculation{\isacharparenright}{\kern0pt}\isanewline
\ \ \isacommand{qed}\isamarkupfalse%
\isanewline
\ \ \isacommand{have}\isamarkupfalse%
\ idAY{\isacharcolon}{\kern0pt}\ {\isachardoublequoteopen}{\isacharparenleft}{\kern0pt}{\isasymphi}\ {\isasymcirc}\isactrlsub c\ eval{\isacharunderscore}{\kern0pt}func\ X\ Y{\isacharparenright}{\kern0pt}\isactrlsup {\isasymsharp}\ {\isasymcirc}\isactrlsub c\ {\isacharparenleft}{\kern0pt}{\isasympsi}\ {\isasymcirc}\isactrlsub c\ eval{\isacharunderscore}{\kern0pt}func\ A\ Y{\isacharparenright}{\kern0pt}\isactrlsup {\isasymsharp}\ \ {\isacharequal}{\kern0pt}\ id{\isacharparenleft}{\kern0pt}A\isactrlbsup Y\isactrlesup {\isacharparenright}{\kern0pt}{\isachardoublequoteclose}\isanewline
\ \ \isacommand{proof}\isamarkupfalse%
\ {\isacharminus}{\kern0pt}\ \isanewline
\ \ \ \ \isacommand{have}\isamarkupfalse%
\ {\isachardoublequoteopen}{\isacharparenleft}{\kern0pt}{\isasymphi}\ {\isasymcirc}\isactrlsub c\ eval{\isacharunderscore}{\kern0pt}func\ X\ Y{\isacharparenright}{\kern0pt}\isactrlsup {\isasymsharp}\ {\isasymcirc}\isactrlsub c\ {\isacharparenleft}{\kern0pt}{\isasympsi}\ {\isasymcirc}\isactrlsub c\ eval{\isacharunderscore}{\kern0pt}func\ A\ Y{\isacharparenright}{\kern0pt}\isactrlsup {\isasymsharp}\ {\isacharequal}{\kern0pt}\ \isanewline
\ \ \ \ \ \ \ \ \ \ {\isacharparenleft}{\kern0pt}{\isasymphi}\isactrlbsup Y\isactrlesup \isactrlsub f\ {\isasymcirc}\isactrlsub c\ {\isacharparenleft}{\kern0pt}eval{\isacharunderscore}{\kern0pt}func\ X\ Y{\isacharparenright}{\kern0pt}\isactrlsup {\isasymsharp}{\isacharparenright}{\kern0pt}\ {\isasymcirc}\isactrlsub c\ {\isacharparenleft}{\kern0pt}{\isasympsi}\isactrlbsup Y\isactrlesup \isactrlsub f\ {\isasymcirc}\isactrlsub c\ {\isacharparenleft}{\kern0pt}eval{\isacharunderscore}{\kern0pt}func\ A\ Y{\isacharparenright}{\kern0pt}\isactrlsup {\isasymsharp}{\isacharparenright}{\kern0pt}{\isachardoublequoteclose}\isanewline
\ \ \ \ \ \ \isacommand{using}\isamarkupfalse%
\ {\isasymphi}{\isacharunderscore}{\kern0pt}def\ {\isasympsi}{\isacharunderscore}{\kern0pt}def\ exp{\isacharunderscore}{\kern0pt}func{\isacharunderscore}{\kern0pt}def{\isadigit{2}}\ exponential{\isacharunderscore}{\kern0pt}object{\isacharunderscore}{\kern0pt}identity\ id{\isacharunderscore}{\kern0pt}right{\isacharunderscore}{\kern0pt}unit{\isadigit{2}}\ phi{\isacharunderscore}{\kern0pt}eval{\isacharunderscore}{\kern0pt}type\ psi{\isacharunderscore}{\kern0pt}eval{\isacharunderscore}{\kern0pt}type\ \isacommand{by}\isamarkupfalse%
\ auto\isanewline
\ \ \ \ \isacommand{also}\isamarkupfalse%
\ \isacommand{have}\isamarkupfalse%
\ {\isachardoublequoteopen}{\isachardot}{\kern0pt}{\isachardot}{\kern0pt}{\isachardot}{\kern0pt}\ {\isacharequal}{\kern0pt}\ {\isacharparenleft}{\kern0pt}{\isasymphi}\isactrlbsup Y\isactrlesup \isactrlsub f\ {\isasymcirc}\isactrlsub c\ id{\isacharparenleft}{\kern0pt}X\isactrlbsup Y\isactrlesup {\isacharparenright}{\kern0pt}{\isacharparenright}{\kern0pt}\ {\isasymcirc}\isactrlsub c\ {\isacharparenleft}{\kern0pt}{\isasympsi}\isactrlbsup Y\isactrlesup \isactrlsub f\ {\isasymcirc}\isactrlsub c\ id{\isacharparenleft}{\kern0pt}A\isactrlbsup Y\isactrlesup {\isacharparenright}{\kern0pt}{\isacharparenright}{\kern0pt}{\isachardoublequoteclose}\isanewline
\ \ \ \ \ \ \isacommand{by}\isamarkupfalse%
\ {\isacharparenleft}{\kern0pt}simp\ add{\isacharcolon}{\kern0pt}\ exponential{\isacharunderscore}{\kern0pt}object{\isacharunderscore}{\kern0pt}identity{\isacharparenright}{\kern0pt}\isanewline
\ \ \ \ \isacommand{also}\isamarkupfalse%
\ \isacommand{have}\isamarkupfalse%
\ {\isachardoublequoteopen}{\isachardot}{\kern0pt}{\isachardot}{\kern0pt}{\isachardot}{\kern0pt}\ {\isacharequal}{\kern0pt}\ {\isasymphi}\isactrlbsup Y\isactrlesup \isactrlsub f\ {\isasymcirc}\isactrlsub c\ {\isacharparenleft}{\kern0pt}id{\isacharparenleft}{\kern0pt}X\isactrlbsup Y\isactrlesup {\isacharparenright}{\kern0pt}\ {\isasymcirc}\isactrlsub c\ {\isacharparenleft}{\kern0pt}{\isasympsi}\isactrlbsup Y\isactrlesup \isactrlsub f\ {\isasymcirc}\isactrlsub c\ id{\isacharparenleft}{\kern0pt}A\isactrlbsup Y\isactrlesup {\isacharparenright}{\kern0pt}{\isacharparenright}{\kern0pt}{\isacharparenright}{\kern0pt}{\isachardoublequoteclose}\ \isanewline
\ \ \ \ \ \ \isacommand{by}\isamarkupfalse%
\ {\isacharparenleft}{\kern0pt}typecheck{\isacharunderscore}{\kern0pt}cfuncs{\isacharcomma}{\kern0pt}\ metis\ {\isasymphi}{\isacharunderscore}{\kern0pt}def\ {\isasympsi}{\isacharunderscore}{\kern0pt}def\ comp{\isacharunderscore}{\kern0pt}associative{\isadigit{2}}{\isacharparenright}{\kern0pt}\isanewline
\ \ \ \ \isacommand{also}\isamarkupfalse%
\ \isacommand{have}\isamarkupfalse%
\ {\isachardoublequoteopen}{\isachardot}{\kern0pt}{\isachardot}{\kern0pt}{\isachardot}{\kern0pt}\ {\isacharequal}{\kern0pt}\ {\isasymphi}\isactrlbsup Y\isactrlesup \isactrlsub f\ {\isasymcirc}\isactrlsub c\ {\isacharparenleft}{\kern0pt}id{\isacharparenleft}{\kern0pt}X\isactrlbsup Y\isactrlesup {\isacharparenright}{\kern0pt}\ {\isasymcirc}\isactrlsub c\ {\isasympsi}\isactrlbsup Y\isactrlesup \isactrlsub f\ {\isacharparenright}{\kern0pt}{\isachardoublequoteclose}\isanewline
\ \ \ \ \ \ \isacommand{using}\isamarkupfalse%
\ {\isasympsi}{\isacharunderscore}{\kern0pt}def\ exp{\isacharunderscore}{\kern0pt}func{\isacharunderscore}{\kern0pt}def{\isadigit{2}}\ id{\isacharunderscore}{\kern0pt}right{\isacharunderscore}{\kern0pt}unit{\isadigit{2}}\ psi{\isacharunderscore}{\kern0pt}eval{\isacharunderscore}{\kern0pt}type\ \isacommand{by}\isamarkupfalse%
\ auto\isanewline
\ \ \ \ \isacommand{also}\isamarkupfalse%
\ \isacommand{have}\isamarkupfalse%
\ {\isachardoublequoteopen}{\isachardot}{\kern0pt}{\isachardot}{\kern0pt}{\isachardot}{\kern0pt}\ {\isacharequal}{\kern0pt}\ {\isasymphi}\isactrlbsup Y\isactrlesup \isactrlsub f\ {\isasymcirc}\isactrlsub c\ {\isasympsi}\isactrlbsup Y\isactrlesup \isactrlsub f{\isachardoublequoteclose}\isanewline
\ \ \ \ \ \ \isacommand{using}\isamarkupfalse%
\ {\isasymphi}{\isacharunderscore}{\kern0pt}def\ {\isasympsi}{\isacharunderscore}{\kern0pt}def\ calculation\ exp{\isacharunderscore}{\kern0pt}func{\isacharunderscore}{\kern0pt}def{\isadigit{2}}\ \isacommand{by}\isamarkupfalse%
\ auto\isanewline
\ \ \ \ \isacommand{also}\isamarkupfalse%
\ \isacommand{have}\isamarkupfalse%
\ {\isachardoublequoteopen}{\isachardot}{\kern0pt}{\isachardot}{\kern0pt}{\isachardot}{\kern0pt}\ {\isacharequal}{\kern0pt}\ {\isacharparenleft}{\kern0pt}{\isasymphi}\ {\isasymcirc}\isactrlsub c\ {\isasympsi}{\isacharparenright}{\kern0pt}\isactrlbsup Y\isactrlesup \isactrlsub f{\isachardoublequoteclose}\isanewline
\ \ \ \ \ \ \isacommand{by}\isamarkupfalse%
\ {\isacharparenleft}{\kern0pt}metis\ {\isasymphi}{\isacharunderscore}{\kern0pt}def\ {\isasympsi}{\isacharunderscore}{\kern0pt}def\ transpose{\isacharunderscore}{\kern0pt}factors{\isacharparenright}{\kern0pt}\isanewline
\ \ \ \ \isacommand{also}\isamarkupfalse%
\ \isacommand{have}\isamarkupfalse%
\ {\isachardoublequoteopen}{\isachardot}{\kern0pt}{\isachardot}{\kern0pt}{\isachardot}{\kern0pt}\ {\isacharequal}{\kern0pt}\ {\isacharparenleft}{\kern0pt}id\ A{\isacharparenright}{\kern0pt}\isactrlbsup Y\isactrlesup \isactrlsub f{\isachardoublequoteclose}\isanewline
\ \ \ \ \ \ \isacommand{by}\isamarkupfalse%
\ {\isacharparenleft}{\kern0pt}simp\ add{\isacharcolon}{\kern0pt}\ idA{\isacharparenright}{\kern0pt}\isanewline
\ \ \ \ \isacommand{also}\isamarkupfalse%
\ \isacommand{have}\isamarkupfalse%
\ {\isachardoublequoteopen}{\isachardot}{\kern0pt}{\isachardot}{\kern0pt}{\isachardot}{\kern0pt}\ \ {\isacharequal}{\kern0pt}\ id{\isacharparenleft}{\kern0pt}A\isactrlbsup Y\isactrlesup {\isacharparenright}{\kern0pt}{\isachardoublequoteclose}\isanewline
\ \ \ \ \ \ \isacommand{by}\isamarkupfalse%
\ {\isacharparenleft}{\kern0pt}simp\ add{\isacharcolon}{\kern0pt}\ exponential{\isacharunderscore}{\kern0pt}object{\isacharunderscore}{\kern0pt}identity{\isadigit{2}}{\isacharparenright}{\kern0pt}\isanewline
\ \ \ \ \isacommand{then}\isamarkupfalse%
\ \isacommand{show}\isamarkupfalse%
\ {\isachardoublequoteopen}{\isacharparenleft}{\kern0pt}{\isasymphi}\ {\isasymcirc}\isactrlsub c\ eval{\isacharunderscore}{\kern0pt}func\ X\ Y{\isacharparenright}{\kern0pt}\isactrlsup {\isasymsharp}\ {\isasymcirc}\isactrlsub c\ {\isacharparenleft}{\kern0pt}{\isasympsi}\ {\isasymcirc}\isactrlsub c\ eval{\isacharunderscore}{\kern0pt}func\ A\ Y{\isacharparenright}{\kern0pt}\isactrlsup {\isasymsharp}\ \ {\isacharequal}{\kern0pt}\ id{\isacharparenleft}{\kern0pt}A\isactrlbsup Y\isactrlesup {\isacharparenright}{\kern0pt}{\isachardoublequoteclose}\isanewline
\ \ \ \ \ \ \isacommand{by}\isamarkupfalse%
\ {\isacharparenleft}{\kern0pt}simp\ add{\isacharcolon}{\kern0pt}\ calculation{\isacharparenright}{\kern0pt}\isanewline
\ \ \isacommand{qed}\isamarkupfalse%
\isanewline
\ \ \isacommand{show}\isamarkupfalse%
\ \ {\isachardoublequoteopen}A\isactrlbsup Y\isactrlesup \ {\isasymcong}\ \ X\isactrlbsup Y\isactrlesup {\isachardoublequoteclose}\isanewline
\ \ \ \ \isacommand{by}\isamarkupfalse%
\ {\isacharparenleft}{\kern0pt}metis\ cfunc{\isacharunderscore}{\kern0pt}type{\isacharunderscore}{\kern0pt}def\ comp{\isacharunderscore}{\kern0pt}epi{\isacharunderscore}{\kern0pt}imp{\isacharunderscore}{\kern0pt}epi\ comp{\isacharunderscore}{\kern0pt}monic{\isacharunderscore}{\kern0pt}imp{\isacharunderscore}{\kern0pt}monic\ epi{\isacharunderscore}{\kern0pt}mon{\isacharunderscore}{\kern0pt}is{\isacharunderscore}{\kern0pt}iso\ idAY\ idXY\ id{\isacharunderscore}{\kern0pt}isomorphism\ is{\isacharunderscore}{\kern0pt}isomorphic{\isacharunderscore}{\kern0pt}def\ iso{\isacharunderscore}{\kern0pt}imp{\isacharunderscore}{\kern0pt}epi{\isacharunderscore}{\kern0pt}and{\isacharunderscore}{\kern0pt}monic\ phi{\isacharunderscore}{\kern0pt}eval{\isacharunderscore}{\kern0pt}type\ psi{\isacharunderscore}{\kern0pt}eval{\isacharunderscore}{\kern0pt}type{\isacharparenright}{\kern0pt}\isanewline
\isacommand{qed}\isamarkupfalse%
%
\endisatagproof
{\isafoldproof}%
%
\isadelimproof
\isanewline
%
\endisadelimproof
\isanewline
\isacommand{lemma}\isamarkupfalse%
\ expset{\isacharunderscore}{\kern0pt}power{\isacharunderscore}{\kern0pt}tower{\isacharcolon}{\kern0pt}\isanewline
\ \ {\isachardoublequoteopen}{\isacharparenleft}{\kern0pt}A\isactrlbsup B\isactrlesup {\isacharparenright}{\kern0pt}\isactrlbsup C\isactrlesup \ {\isasymcong}\ A\isactrlbsup {\isacharparenleft}{\kern0pt}B{\isasymtimes}\isactrlsub c\ C{\isacharparenright}{\kern0pt}\isactrlesup {\isachardoublequoteclose}\isanewline
%
\isadelimproof
%
\endisadelimproof
%
\isatagproof
\isacommand{proof}\isamarkupfalse%
\ {\isacharminus}{\kern0pt}\ \isanewline
\ \ \isacommand{obtain}\isamarkupfalse%
\ {\isasymphi}\ \isakeyword{where}\ {\isasymphi}{\isacharunderscore}{\kern0pt}def{\isacharcolon}{\kern0pt}\ {\isachardoublequoteopen}{\isasymphi}\ {\isacharequal}{\kern0pt}\ {\isacharparenleft}{\kern0pt}{\isacharparenleft}{\kern0pt}eval{\isacharunderscore}{\kern0pt}func\ A\ {\isacharparenleft}{\kern0pt}B{\isasymtimes}\isactrlsub c\ C{\isacharparenright}{\kern0pt}{\isacharparenright}{\kern0pt}\ {\isasymcirc}\isactrlsub c\ {\isacharparenleft}{\kern0pt}associate{\isacharunderscore}{\kern0pt}left\ B\ C\ {\isacharparenleft}{\kern0pt}A\isactrlbsup {\isacharparenleft}{\kern0pt}B{\isasymtimes}\isactrlsub c\ C{\isacharparenright}{\kern0pt}\isactrlesup {\isacharparenright}{\kern0pt}{\isacharparenright}{\kern0pt}{\isacharparenright}{\kern0pt}{\isachardoublequoteclose}\ \isakeyword{and}\isanewline
\ \ \ \ \ \ \ \ \ \ \ \ \ \ \ \ \ {\isasymphi}{\isacharunderscore}{\kern0pt}type{\isacharbrackleft}{\kern0pt}type{\isacharunderscore}{\kern0pt}rule{\isacharbrackright}{\kern0pt}{\isacharcolon}{\kern0pt}\ {\isachardoublequoteopen}{\isasymphi}{\isacharcolon}{\kern0pt}\ B\ {\isasymtimes}\isactrlsub c\ {\isacharparenleft}{\kern0pt}C{\isasymtimes}\isactrlsub c\ {\isacharparenleft}{\kern0pt}A\isactrlbsup {\isacharparenleft}{\kern0pt}B{\isasymtimes}\isactrlsub c\ C{\isacharparenright}{\kern0pt}\isactrlesup {\isacharparenright}{\kern0pt}{\isacharparenright}{\kern0pt}\ {\isasymrightarrow}\ A{\isachardoublequoteclose}\ \isakeyword{and}\ \isanewline
\ \ \ \ \ \ \ \ \ \ \ \ \ \ \ \ \ {\isasymphi}dbsharp{\isacharunderscore}{\kern0pt}type{\isacharbrackleft}{\kern0pt}type{\isacharunderscore}{\kern0pt}rule{\isacharbrackright}{\kern0pt}{\isacharcolon}{\kern0pt}\ {\isachardoublequoteopen}{\isacharparenleft}{\kern0pt}{\isasymphi}\isactrlsup {\isasymsharp}{\isacharparenright}{\kern0pt}\isactrlsup {\isasymsharp}\ {\isacharcolon}{\kern0pt}\ {\isacharparenleft}{\kern0pt}A\isactrlbsup {\isacharparenleft}{\kern0pt}B{\isasymtimes}\isactrlsub c\ C{\isacharparenright}{\kern0pt}\isactrlesup {\isacharparenright}{\kern0pt}\ {\isasymrightarrow}\ {\isacharparenleft}{\kern0pt}{\isacharparenleft}{\kern0pt}A\isactrlbsup B\isactrlesup {\isacharparenright}{\kern0pt}\isactrlbsup C\isactrlesup {\isacharparenright}{\kern0pt}{\isachardoublequoteclose}\isanewline
\ \ \ \ \isacommand{using}\isamarkupfalse%
\ transpose{\isacharunderscore}{\kern0pt}func{\isacharunderscore}{\kern0pt}type\ \isacommand{by}\isamarkupfalse%
\ {\isacharparenleft}{\kern0pt}typecheck{\isacharunderscore}{\kern0pt}cfuncs{\isacharcomma}{\kern0pt}\ fastforce{\isacharparenright}{\kern0pt}\isanewline
\isanewline
\ \ \isacommand{obtain}\isamarkupfalse%
\ {\isasympsi}\ \isakeyword{where}\ {\isasympsi}{\isacharunderscore}{\kern0pt}def{\isacharcolon}{\kern0pt}\ {\isachardoublequoteopen}{\isasympsi}\ {\isacharequal}{\kern0pt}\ {\isacharparenleft}{\kern0pt}eval{\isacharunderscore}{\kern0pt}func\ A\ B{\isacharparenright}{\kern0pt}\ {\isasymcirc}\isactrlsub c\ {\isacharparenleft}{\kern0pt}id{\isacharparenleft}{\kern0pt}B{\isacharparenright}{\kern0pt}{\isasymtimes}\isactrlsub f\ eval{\isacharunderscore}{\kern0pt}func\ {\isacharparenleft}{\kern0pt}A\isactrlbsup B\isactrlesup {\isacharparenright}{\kern0pt}\ C{\isacharparenright}{\kern0pt}\ {\isasymcirc}\isactrlsub c\ {\isacharparenleft}{\kern0pt}associate{\isacharunderscore}{\kern0pt}right\ B\ C\ {\isacharparenleft}{\kern0pt}{\isacharparenleft}{\kern0pt}A\isactrlbsup B\isactrlesup {\isacharparenright}{\kern0pt}\isactrlbsup C\isactrlesup {\isacharparenright}{\kern0pt}{\isacharparenright}{\kern0pt}{\isachardoublequoteclose}\ \isakeyword{and}\isanewline
\ \ \ \ \ \ \ \ \ \ \ \ \ \ \ \ \ {\isasympsi}{\isacharunderscore}{\kern0pt}type{\isacharbrackleft}{\kern0pt}type{\isacharunderscore}{\kern0pt}rule{\isacharbrackright}{\kern0pt}{\isacharcolon}{\kern0pt}\ {\isachardoublequoteopen}{\isasympsi}\ {\isacharcolon}{\kern0pt}\ \ {\isacharparenleft}{\kern0pt}B\ {\isasymtimes}\isactrlsub c\ C{\isacharparenright}{\kern0pt}\ {\isasymtimes}\isactrlsub c\ {\isacharparenleft}{\kern0pt}{\isacharparenleft}{\kern0pt}A\isactrlbsup B\isactrlesup {\isacharparenright}{\kern0pt}\isactrlbsup C\isactrlesup {\isacharparenright}{\kern0pt}\ {\isasymrightarrow}\ A{\isachardoublequoteclose}\ \isakeyword{and}\isanewline
\ \ \ \ \ \ \ \ \ \ \ \ \ \ \ \ \ {\isasympsi}sharp{\isacharunderscore}{\kern0pt}type{\isacharbrackleft}{\kern0pt}type{\isacharunderscore}{\kern0pt}rule{\isacharbrackright}{\kern0pt}{\isacharcolon}{\kern0pt}\ {\isachardoublequoteopen}{\isasympsi}\isactrlsup {\isasymsharp}{\isacharcolon}{\kern0pt}\ {\isacharparenleft}{\kern0pt}A\isactrlbsup B\isactrlesup {\isacharparenright}{\kern0pt}\isactrlbsup C\isactrlesup \ {\isasymrightarrow}\ {\isacharparenleft}{\kern0pt}A\isactrlbsup {\isacharparenleft}{\kern0pt}B{\isasymtimes}\isactrlsub c\ C{\isacharparenright}{\kern0pt}\isactrlesup {\isacharparenright}{\kern0pt}{\isachardoublequoteclose}\isanewline
\ \ \ \ \isacommand{using}\isamarkupfalse%
\ transpose{\isacharunderscore}{\kern0pt}func{\isacharunderscore}{\kern0pt}type\ \isacommand{by}\isamarkupfalse%
\ {\isacharparenleft}{\kern0pt}typecheck{\isacharunderscore}{\kern0pt}cfuncs{\isacharcomma}{\kern0pt}\ blast{\isacharparenright}{\kern0pt}\isanewline
\isanewline
\ \ \isacommand{have}\isamarkupfalse%
\ {\isachardoublequoteopen}{\isasymphi}\isactrlsup {\isasymsharp}\isactrlsup {\isasymsharp}\ {\isasymcirc}\isactrlsub c\ {\isasympsi}\isactrlsup {\isasymsharp}\ {\isacharequal}{\kern0pt}\ id{\isacharparenleft}{\kern0pt}{\isacharparenleft}{\kern0pt}A\isactrlbsup B\isactrlesup {\isacharparenright}{\kern0pt}\isactrlbsup C\isactrlesup {\isacharparenright}{\kern0pt}{\isachardoublequoteclose}\isanewline
\ \ \isacommand{proof}\isamarkupfalse%
{\isacharparenleft}{\kern0pt}etcs{\isacharunderscore}{\kern0pt}rule\ same{\isacharunderscore}{\kern0pt}evals{\isacharunderscore}{\kern0pt}equal{\isacharbrackleft}{\kern0pt}\isakeyword{where}\ X\ {\isacharequal}{\kern0pt}\ {\isachardoublequoteopen}{\isacharparenleft}{\kern0pt}A\isactrlbsup B\isactrlesup {\isacharparenright}{\kern0pt}{\isachardoublequoteclose}{\isacharcomma}{\kern0pt}\ \isakeyword{where}\ A\ {\isacharequal}{\kern0pt}\ {\isachardoublequoteopen}C{\isachardoublequoteclose}{\isacharbrackright}{\kern0pt}{\isacharparenright}{\kern0pt}\isanewline
\ \ \ \ \isacommand{show}\isamarkupfalse%
\ {\isachardoublequoteopen}eval{\isacharunderscore}{\kern0pt}func\ {\isacharparenleft}{\kern0pt}A\isactrlbsup B\isactrlesup {\isacharparenright}{\kern0pt}\ C\ {\isasymcirc}\isactrlsub c\ id\isactrlsub c\ C\ {\isasymtimes}\isactrlsub f\ {\isasymphi}\isactrlsup {\isasymsharp}\isactrlsup {\isasymsharp}\ {\isasymcirc}\isactrlsub c\ {\isasympsi}\isactrlsup {\isasymsharp}\ {\isacharequal}{\kern0pt}\isanewline
\ \ \ \ \ \ \ \ \ \ eval{\isacharunderscore}{\kern0pt}func\ {\isacharparenleft}{\kern0pt}A\isactrlbsup B\isactrlesup {\isacharparenright}{\kern0pt}\ C\ {\isasymcirc}\isactrlsub c\ id\isactrlsub c\ C\ {\isasymtimes}\isactrlsub f\ id\isactrlsub c\ {\isacharparenleft}{\kern0pt}A\isactrlbsup B\isactrlesup \isactrlbsup C\isactrlesup {\isacharparenright}{\kern0pt}{\isachardoublequoteclose}\isanewline
\ \ \ \ \isacommand{proof}\isamarkupfalse%
{\isacharparenleft}{\kern0pt}etcs{\isacharunderscore}{\kern0pt}rule\ same{\isacharunderscore}{\kern0pt}evals{\isacharunderscore}{\kern0pt}equal{\isacharbrackleft}{\kern0pt}\isakeyword{where}\ X\ {\isacharequal}{\kern0pt}\ {\isachardoublequoteopen}A{\isachardoublequoteclose}{\isacharcomma}{\kern0pt}\ \isakeyword{where}\ A\ {\isacharequal}{\kern0pt}\ {\isachardoublequoteopen}B{\isachardoublequoteclose}{\isacharbrackright}{\kern0pt}{\isacharparenright}{\kern0pt}\isanewline
\ \ \ \ \ \ \isacommand{show}\isamarkupfalse%
\ {\isachardoublequoteopen}eval{\isacharunderscore}{\kern0pt}func\ A\ B\ {\isasymcirc}\isactrlsub c\ id\isactrlsub c\ B\ {\isasymtimes}\isactrlsub f\ {\isacharparenleft}{\kern0pt}eval{\isacharunderscore}{\kern0pt}func\ {\isacharparenleft}{\kern0pt}A\isactrlbsup B\isactrlesup {\isacharparenright}{\kern0pt}\ C\ {\isasymcirc}\isactrlsub c\ {\isacharparenleft}{\kern0pt}id\isactrlsub c\ C\ {\isasymtimes}\isactrlsub f\ {\isasymphi}\isactrlsup {\isasymsharp}\isactrlsup {\isasymsharp}\ {\isasymcirc}\isactrlsub c\ {\isasympsi}\isactrlsup {\isasymsharp}{\isacharparenright}{\kern0pt}{\isacharparenright}{\kern0pt}\ {\isacharequal}{\kern0pt}\isanewline
\ \ \ \ \ \ \ \ \ \ \ \ eval{\isacharunderscore}{\kern0pt}func\ A\ B\ {\isasymcirc}\isactrlsub c\ id\isactrlsub c\ B\ {\isasymtimes}\isactrlsub f\ eval{\isacharunderscore}{\kern0pt}func\ {\isacharparenleft}{\kern0pt}A\isactrlbsup B\isactrlesup {\isacharparenright}{\kern0pt}\ C\ {\isasymcirc}\isactrlsub c\ id\isactrlsub c\ C\ {\isasymtimes}\isactrlsub f\ id\isactrlsub c\ {\isacharparenleft}{\kern0pt}A\isactrlbsup B\isactrlesup \isactrlbsup C\isactrlesup {\isacharparenright}{\kern0pt}{\isachardoublequoteclose}\isanewline
\ \ \ \ \ \ \isacommand{proof}\isamarkupfalse%
\ {\isacharminus}{\kern0pt}\ \isanewline
\ \ \ \ \ \ \ \ \isacommand{have}\isamarkupfalse%
\ {\isachardoublequoteopen}eval{\isacharunderscore}{\kern0pt}func\ A\ B\ {\isasymcirc}\isactrlsub c\ id\isactrlsub c\ B\ {\isasymtimes}\isactrlsub f\ {\isacharparenleft}{\kern0pt}eval{\isacharunderscore}{\kern0pt}func\ {\isacharparenleft}{\kern0pt}A\isactrlbsup B\isactrlesup {\isacharparenright}{\kern0pt}\ C\ {\isasymcirc}\isactrlsub c\ {\isacharparenleft}{\kern0pt}id\isactrlsub c\ C\ {\isasymtimes}\isactrlsub f\ {\isasymphi}\isactrlsup {\isasymsharp}\isactrlsup {\isasymsharp}\ {\isasymcirc}\isactrlsub c\ {\isasympsi}\isactrlsup {\isasymsharp}{\isacharparenright}{\kern0pt}{\isacharparenright}{\kern0pt}\ {\isacharequal}{\kern0pt}\isanewline
\ \ \ \ \ \ \ \ \ \ \ \ \ \ eval{\isacharunderscore}{\kern0pt}func\ A\ B\ {\isasymcirc}\isactrlsub c\ id\isactrlsub c\ B\ {\isasymtimes}\isactrlsub f\ {\isacharparenleft}{\kern0pt}eval{\isacharunderscore}{\kern0pt}func\ {\isacharparenleft}{\kern0pt}A\isactrlbsup B\isactrlesup {\isacharparenright}{\kern0pt}\ C\ {\isasymcirc}\isactrlsub c\ {\isacharparenleft}{\kern0pt}id\isactrlsub c\ C\ {\isasymtimes}\isactrlsub f\ {\isasymphi}\isactrlsup {\isasymsharp}\isactrlsup {\isasymsharp}{\isacharparenright}{\kern0pt}\ {\isasymcirc}\isactrlsub c\ {\isacharparenleft}{\kern0pt}id\isactrlsub c\ C\ {\isasymtimes}\isactrlsub f\ {\isasympsi}\isactrlsup {\isasymsharp}{\isacharparenright}{\kern0pt}{\isacharparenright}{\kern0pt}{\isachardoublequoteclose}\isanewline
\ \ \ \ \ \ \ \ \ \ \isacommand{by}\isamarkupfalse%
\ {\isacharparenleft}{\kern0pt}typecheck{\isacharunderscore}{\kern0pt}cfuncs{\isacharcomma}{\kern0pt}\ metis\ identity{\isacharunderscore}{\kern0pt}distributes{\isacharunderscore}{\kern0pt}across{\isacharunderscore}{\kern0pt}composition{\isacharparenright}{\kern0pt}\isanewline
\ \ \ \ \ \ \ \ \isacommand{also}\isamarkupfalse%
\ \isacommand{have}\isamarkupfalse%
\ {\isachardoublequoteopen}{\isachardot}{\kern0pt}{\isachardot}{\kern0pt}{\isachardot}{\kern0pt}\ {\isacharequal}{\kern0pt}\ eval{\isacharunderscore}{\kern0pt}func\ A\ B\ {\isasymcirc}\isactrlsub c\ id\isactrlsub c\ B\ {\isasymtimes}\isactrlsub f\ {\isacharparenleft}{\kern0pt}{\isacharparenleft}{\kern0pt}eval{\isacharunderscore}{\kern0pt}func\ {\isacharparenleft}{\kern0pt}A\isactrlbsup B\isactrlesup {\isacharparenright}{\kern0pt}\ C\ {\isasymcirc}\isactrlsub c\ {\isacharparenleft}{\kern0pt}id\isactrlsub c\ C\ {\isasymtimes}\isactrlsub f\ {\isasymphi}\isactrlsup {\isasymsharp}\isactrlsup {\isasymsharp}{\isacharparenright}{\kern0pt}{\isacharparenright}{\kern0pt}\ {\isasymcirc}\isactrlsub c\ {\isacharparenleft}{\kern0pt}id\isactrlsub c\ C\ {\isasymtimes}\isactrlsub f\ {\isasympsi}\isactrlsup {\isasymsharp}{\isacharparenright}{\kern0pt}{\isacharparenright}{\kern0pt}{\isachardoublequoteclose}\isanewline
\ \ \ \ \ \ \ \ \ \ \isacommand{by}\isamarkupfalse%
\ {\isacharparenleft}{\kern0pt}typecheck{\isacharunderscore}{\kern0pt}cfuncs{\isacharcomma}{\kern0pt}\ simp\ add{\isacharcolon}{\kern0pt}\ comp{\isacharunderscore}{\kern0pt}associative{\isadigit{2}}{\isacharparenright}{\kern0pt}\isanewline
\ \ \ \ \ \ \ \ \isacommand{also}\isamarkupfalse%
\ \isacommand{have}\isamarkupfalse%
\ {\isachardoublequoteopen}{\isachardot}{\kern0pt}{\isachardot}{\kern0pt}{\isachardot}{\kern0pt}\ {\isacharequal}{\kern0pt}\ eval{\isacharunderscore}{\kern0pt}func\ A\ B\ {\isasymcirc}\isactrlsub c\ id\isactrlsub c\ B\ {\isasymtimes}\isactrlsub f\ {\isacharparenleft}{\kern0pt}{\isasymphi}\isactrlsup {\isasymsharp}\ {\isasymcirc}\isactrlsub c\ {\isacharparenleft}{\kern0pt}id\isactrlsub c\ C\ {\isasymtimes}\isactrlsub f\ {\isasympsi}\isactrlsup {\isasymsharp}{\isacharparenright}{\kern0pt}{\isacharparenright}{\kern0pt}{\isachardoublequoteclose}\isanewline
\ \ \ \ \ \ \ \ \ \ \isacommand{by}\isamarkupfalse%
\ {\isacharparenleft}{\kern0pt}typecheck{\isacharunderscore}{\kern0pt}cfuncs{\isacharcomma}{\kern0pt}\ simp\ add{\isacharcolon}{\kern0pt}\ transpose{\isacharunderscore}{\kern0pt}func{\isacharunderscore}{\kern0pt}def{\isacharparenright}{\kern0pt}\ \ \ \ \ \ \ \ \isanewline
\ \ \ \ \ \ \ \ \isacommand{also}\isamarkupfalse%
\ \isacommand{have}\isamarkupfalse%
\ {\isachardoublequoteopen}{\isachardot}{\kern0pt}{\isachardot}{\kern0pt}{\isachardot}{\kern0pt}\ {\isacharequal}{\kern0pt}\ eval{\isacharunderscore}{\kern0pt}func\ A\ B\ {\isasymcirc}\isactrlsub c\ {\isacharparenleft}{\kern0pt}{\isacharparenleft}{\kern0pt}id\isactrlsub c\ B\ {\isasymtimes}\isactrlsub f\ {\isasymphi}\isactrlsup {\isasymsharp}{\isacharparenright}{\kern0pt}\ \ {\isasymcirc}\isactrlsub c\ {\isacharparenleft}{\kern0pt}id\isactrlsub c\ B\ {\isasymtimes}\isactrlsub f\ {\isacharparenleft}{\kern0pt}id\isactrlsub c\ C\ {\isasymtimes}\isactrlsub f\ {\isasympsi}\isactrlsup {\isasymsharp}{\isacharparenright}{\kern0pt}{\isacharparenright}{\kern0pt}{\isacharparenright}{\kern0pt}{\isachardoublequoteclose}\isanewline
\ \ \ \ \ \ \ \ \ \ \isacommand{using}\isamarkupfalse%
\ identity{\isacharunderscore}{\kern0pt}distributes{\isacharunderscore}{\kern0pt}across{\isacharunderscore}{\kern0pt}composition\ \isacommand{by}\isamarkupfalse%
\ {\isacharparenleft}{\kern0pt}typecheck{\isacharunderscore}{\kern0pt}cfuncs{\isacharcomma}{\kern0pt}\ auto{\isacharparenright}{\kern0pt}\isanewline
\ \ \ \ \ \ \ \ \isacommand{also}\isamarkupfalse%
\ \isacommand{have}\isamarkupfalse%
\ {\isachardoublequoteopen}{\isachardot}{\kern0pt}{\isachardot}{\kern0pt}{\isachardot}{\kern0pt}\ {\isacharequal}{\kern0pt}\ {\isacharparenleft}{\kern0pt}eval{\isacharunderscore}{\kern0pt}func\ A\ B\ {\isasymcirc}\isactrlsub c\ {\isacharparenleft}{\kern0pt}{\isacharparenleft}{\kern0pt}id\isactrlsub c\ B\ {\isasymtimes}\isactrlsub f\ {\isasymphi}\isactrlsup {\isasymsharp}{\isacharparenright}{\kern0pt}{\isacharparenright}{\kern0pt}{\isacharparenright}{\kern0pt}\ \ {\isasymcirc}\isactrlsub c\ {\isacharparenleft}{\kern0pt}id\isactrlsub c\ B\ {\isasymtimes}\isactrlsub f\ {\isacharparenleft}{\kern0pt}id\isactrlsub c\ C\ {\isasymtimes}\isactrlsub f\ {\isasympsi}\isactrlsup {\isasymsharp}{\isacharparenright}{\kern0pt}{\isacharparenright}{\kern0pt}{\isachardoublequoteclose}\isanewline
\ \ \ \ \ \ \ \ \ \ \isacommand{using}\isamarkupfalse%
\ comp{\isacharunderscore}{\kern0pt}associative{\isadigit{2}}\ \isacommand{by}\isamarkupfalse%
\ {\isacharparenleft}{\kern0pt}typecheck{\isacharunderscore}{\kern0pt}cfuncs{\isacharcomma}{\kern0pt}blast{\isacharparenright}{\kern0pt}\isanewline
\ \ \ \ \ \ \ \ \isacommand{also}\isamarkupfalse%
\ \isacommand{have}\isamarkupfalse%
\ {\isachardoublequoteopen}{\isachardot}{\kern0pt}{\isachardot}{\kern0pt}{\isachardot}{\kern0pt}\ {\isacharequal}{\kern0pt}\ {\isasymphi}\ \ {\isasymcirc}\isactrlsub c\ {\isacharparenleft}{\kern0pt}id\isactrlsub c\ B\ {\isasymtimes}\isactrlsub f\ {\isacharparenleft}{\kern0pt}id\isactrlsub c\ C\ {\isasymtimes}\isactrlsub f\ {\isasympsi}\isactrlsup {\isasymsharp}{\isacharparenright}{\kern0pt}{\isacharparenright}{\kern0pt}{\isachardoublequoteclose}\isanewline
\ \ \ \ \ \ \ \ \ \ \isacommand{by}\isamarkupfalse%
\ {\isacharparenleft}{\kern0pt}typecheck{\isacharunderscore}{\kern0pt}cfuncs{\isacharcomma}{\kern0pt}\ simp\ add{\isacharcolon}{\kern0pt}\ transpose{\isacharunderscore}{\kern0pt}func{\isacharunderscore}{\kern0pt}def{\isacharparenright}{\kern0pt}\isanewline
\ \ \ \ \ \ \ \ \isacommand{also}\isamarkupfalse%
\ \isacommand{have}\isamarkupfalse%
\ {\isachardoublequoteopen}{\isachardot}{\kern0pt}{\isachardot}{\kern0pt}{\isachardot}{\kern0pt}\ {\isacharequal}{\kern0pt}\ {\isacharparenleft}{\kern0pt}{\isacharparenleft}{\kern0pt}eval{\isacharunderscore}{\kern0pt}func\ A\ {\isacharparenleft}{\kern0pt}B{\isasymtimes}\isactrlsub c\ C{\isacharparenright}{\kern0pt}{\isacharparenright}{\kern0pt}\ {\isasymcirc}\isactrlsub c\ {\isacharparenleft}{\kern0pt}associate{\isacharunderscore}{\kern0pt}left\ B\ C\ {\isacharparenleft}{\kern0pt}A\isactrlbsup {\isacharparenleft}{\kern0pt}B{\isasymtimes}\isactrlsub c\ C{\isacharparenright}{\kern0pt}\isactrlesup {\isacharparenright}{\kern0pt}{\isacharparenright}{\kern0pt}{\isacharparenright}{\kern0pt}\ {\isasymcirc}\isactrlsub c\ {\isacharparenleft}{\kern0pt}id\isactrlsub c\ B\ {\isasymtimes}\isactrlsub f\ {\isacharparenleft}{\kern0pt}id\isactrlsub c\ C\ {\isasymtimes}\isactrlsub f\ {\isasympsi}\isactrlsup {\isasymsharp}{\isacharparenright}{\kern0pt}{\isacharparenright}{\kern0pt}{\isachardoublequoteclose}\isanewline
\ \ \ \ \ \ \ \ \ \ \isacommand{by}\isamarkupfalse%
\ {\isacharparenleft}{\kern0pt}simp\ add{\isacharcolon}{\kern0pt}\ {\isasymphi}{\isacharunderscore}{\kern0pt}def{\isacharparenright}{\kern0pt}\isanewline
\ \ \ \ \ \ \ \ \isacommand{also}\isamarkupfalse%
\ \isacommand{have}\isamarkupfalse%
\ {\isachardoublequoteopen}{\isachardot}{\kern0pt}{\isachardot}{\kern0pt}{\isachardot}{\kern0pt}\ {\isacharequal}{\kern0pt}\ {\isacharparenleft}{\kern0pt}eval{\isacharunderscore}{\kern0pt}func\ A\ {\isacharparenleft}{\kern0pt}B{\isasymtimes}\isactrlsub c\ C{\isacharparenright}{\kern0pt}{\isacharparenright}{\kern0pt}\ {\isasymcirc}\isactrlsub c\ {\isacharparenleft}{\kern0pt}associate{\isacharunderscore}{\kern0pt}left\ B\ C\ {\isacharparenleft}{\kern0pt}A\isactrlbsup {\isacharparenleft}{\kern0pt}B{\isasymtimes}\isactrlsub c\ C{\isacharparenright}{\kern0pt}\isactrlesup {\isacharparenright}{\kern0pt}{\isacharparenright}{\kern0pt}\ {\isasymcirc}\isactrlsub c\ {\isacharparenleft}{\kern0pt}id\isactrlsub c\ B\ {\isasymtimes}\isactrlsub f\ {\isacharparenleft}{\kern0pt}id\isactrlsub c\ C\ {\isasymtimes}\isactrlsub f\ {\isasympsi}\isactrlsup {\isasymsharp}{\isacharparenright}{\kern0pt}{\isacharparenright}{\kern0pt}{\isachardoublequoteclose}\isanewline
\ \ \ \ \ \ \ \ \ \ \isacommand{using}\isamarkupfalse%
\ comp{\isacharunderscore}{\kern0pt}associative{\isadigit{2}}\ \isacommand{by}\isamarkupfalse%
\ {\isacharparenleft}{\kern0pt}typecheck{\isacharunderscore}{\kern0pt}cfuncs{\isacharcomma}{\kern0pt}\ auto{\isacharparenright}{\kern0pt}\isanewline
\ \ \ \ \ \ \ \ \isacommand{also}\isamarkupfalse%
\ \isacommand{have}\isamarkupfalse%
\ {\isachardoublequoteopen}{\isachardot}{\kern0pt}{\isachardot}{\kern0pt}{\isachardot}{\kern0pt}\ {\isacharequal}{\kern0pt}\ {\isacharparenleft}{\kern0pt}eval{\isacharunderscore}{\kern0pt}func\ A\ {\isacharparenleft}{\kern0pt}B{\isasymtimes}\isactrlsub c\ C{\isacharparenright}{\kern0pt}{\isacharparenright}{\kern0pt}\ {\isasymcirc}\isactrlsub c\ {\isacharparenleft}{\kern0pt}{\isacharparenleft}{\kern0pt}id\isactrlsub c\ B\ {\isasymtimes}\isactrlsub f\ id\isactrlsub c\ C{\isacharparenright}{\kern0pt}\ {\isasymtimes}\isactrlsub f\ {\isasympsi}\isactrlsup {\isasymsharp}{\isacharparenright}{\kern0pt}\ {\isasymcirc}\isactrlsub c\ associate{\isacharunderscore}{\kern0pt}left\ B\ C\ {\isacharparenleft}{\kern0pt}{\isacharparenleft}{\kern0pt}A\isactrlbsup B\isactrlesup {\isacharparenright}{\kern0pt}\isactrlbsup C\isactrlesup {\isacharparenright}{\kern0pt}{\isachardoublequoteclose}\isanewline
\ \ \ \ \ \ \ \ \ \ \isacommand{by}\isamarkupfalse%
\ {\isacharparenleft}{\kern0pt}typecheck{\isacharunderscore}{\kern0pt}cfuncs{\isacharcomma}{\kern0pt}\ simp\ add{\isacharcolon}{\kern0pt}\ associate{\isacharunderscore}{\kern0pt}left{\isacharunderscore}{\kern0pt}crossprod{\isacharunderscore}{\kern0pt}ap{\isacharparenright}{\kern0pt}\isanewline
\ \ \ \ \ \ \ \ \isacommand{also}\isamarkupfalse%
\ \isacommand{have}\isamarkupfalse%
\ {\isachardoublequoteopen}{\isachardot}{\kern0pt}{\isachardot}{\kern0pt}{\isachardot}{\kern0pt}\ {\isacharequal}{\kern0pt}\ {\isacharparenleft}{\kern0pt}eval{\isacharunderscore}{\kern0pt}func\ A\ {\isacharparenleft}{\kern0pt}B{\isasymtimes}\isactrlsub c\ C{\isacharparenright}{\kern0pt}{\isacharparenright}{\kern0pt}\ {\isasymcirc}\isactrlsub c\ {\isacharparenleft}{\kern0pt}{\isacharparenleft}{\kern0pt}id\isactrlsub c\ {\isacharparenleft}{\kern0pt}B\ {\isasymtimes}\isactrlsub c\ C{\isacharparenright}{\kern0pt}{\isacharparenright}{\kern0pt}\ {\isasymtimes}\isactrlsub f\ {\isasympsi}\isactrlsup {\isasymsharp}{\isacharparenright}{\kern0pt}\ {\isasymcirc}\isactrlsub c\ associate{\isacharunderscore}{\kern0pt}left\ B\ C\ {\isacharparenleft}{\kern0pt}{\isacharparenleft}{\kern0pt}A\isactrlbsup B\isactrlesup {\isacharparenright}{\kern0pt}\isactrlbsup C\isactrlesup {\isacharparenright}{\kern0pt}{\isachardoublequoteclose}\isanewline
\ \ \ \ \ \ \ \ \ \ \isacommand{by}\isamarkupfalse%
\ {\isacharparenleft}{\kern0pt}simp\ add{\isacharcolon}{\kern0pt}\ id{\isacharunderscore}{\kern0pt}cross{\isacharunderscore}{\kern0pt}prod{\isacharparenright}{\kern0pt}\isanewline
\ \ \ \ \ \ \ \ \isacommand{also}\isamarkupfalse%
\ \isacommand{have}\isamarkupfalse%
\ {\isachardoublequoteopen}{\isachardot}{\kern0pt}{\isachardot}{\kern0pt}{\isachardot}{\kern0pt}\ {\isacharequal}{\kern0pt}\ {\isasympsi}\ {\isasymcirc}\isactrlsub c\ associate{\isacharunderscore}{\kern0pt}left\ B\ C\ {\isacharparenleft}{\kern0pt}{\isacharparenleft}{\kern0pt}A\isactrlbsup B\isactrlesup {\isacharparenright}{\kern0pt}\isactrlbsup C\isactrlesup {\isacharparenright}{\kern0pt}{\isachardoublequoteclose}\isanewline
\ \ \ \ \ \ \ \ \ \ \isacommand{by}\isamarkupfalse%
\ {\isacharparenleft}{\kern0pt}typecheck{\isacharunderscore}{\kern0pt}cfuncs{\isacharcomma}{\kern0pt}\ simp\ add{\isacharcolon}{\kern0pt}\ comp{\isacharunderscore}{\kern0pt}associative{\isadigit{2}}\ transpose{\isacharunderscore}{\kern0pt}func{\isacharunderscore}{\kern0pt}def{\isacharparenright}{\kern0pt}\isanewline
\ \ \ \ \ \ \ \ \isacommand{also}\isamarkupfalse%
\ \isacommand{have}\isamarkupfalse%
\ {\isachardoublequoteopen}{\isachardot}{\kern0pt}{\isachardot}{\kern0pt}{\isachardot}{\kern0pt}\ {\isacharequal}{\kern0pt}\ {\isacharparenleft}{\kern0pt}{\isacharparenleft}{\kern0pt}eval{\isacharunderscore}{\kern0pt}func\ A\ B{\isacharparenright}{\kern0pt}\ {\isasymcirc}\isactrlsub c\ {\isacharparenleft}{\kern0pt}id{\isacharparenleft}{\kern0pt}B{\isacharparenright}{\kern0pt}{\isasymtimes}\isactrlsub f\ eval{\isacharunderscore}{\kern0pt}func\ {\isacharparenleft}{\kern0pt}A\isactrlbsup B\isactrlesup {\isacharparenright}{\kern0pt}\ C{\isacharparenright}{\kern0pt}{\isacharparenright}{\kern0pt}\ {\isasymcirc}\isactrlsub c\ {\isacharparenleft}{\kern0pt}{\isacharparenleft}{\kern0pt}associate{\isacharunderscore}{\kern0pt}right\ B\ C\ {\isacharparenleft}{\kern0pt}{\isacharparenleft}{\kern0pt}A\isactrlbsup B\isactrlesup {\isacharparenright}{\kern0pt}\isactrlbsup C\isactrlesup {\isacharparenright}{\kern0pt}{\isacharparenright}{\kern0pt}{\isasymcirc}\isactrlsub c\ \ associate{\isacharunderscore}{\kern0pt}left\ B\ C\ {\isacharparenleft}{\kern0pt}{\isacharparenleft}{\kern0pt}A\isactrlbsup B\isactrlesup {\isacharparenright}{\kern0pt}\isactrlbsup C\isactrlesup {\isacharparenright}{\kern0pt}{\isacharparenright}{\kern0pt}{\isachardoublequoteclose}\isanewline
\ \ \ \ \ \ \ \ \ \ \isacommand{by}\isamarkupfalse%
\ {\isacharparenleft}{\kern0pt}typecheck{\isacharunderscore}{\kern0pt}cfuncs{\isacharcomma}{\kern0pt}\ simp\ add{\isacharcolon}{\kern0pt}\ {\isasympsi}{\isacharunderscore}{\kern0pt}def\ cfunc{\isacharunderscore}{\kern0pt}type{\isacharunderscore}{\kern0pt}def\ comp{\isacharunderscore}{\kern0pt}associative{\isacharparenright}{\kern0pt}\isanewline
\ \ \ \ \ \ \ \ \isacommand{also}\isamarkupfalse%
\ \isacommand{have}\isamarkupfalse%
\ {\isachardoublequoteopen}{\isachardot}{\kern0pt}{\isachardot}{\kern0pt}{\isachardot}{\kern0pt}\ {\isacharequal}{\kern0pt}\ {\isacharparenleft}{\kern0pt}{\isacharparenleft}{\kern0pt}eval{\isacharunderscore}{\kern0pt}func\ A\ B{\isacharparenright}{\kern0pt}\ {\isasymcirc}\isactrlsub c\ {\isacharparenleft}{\kern0pt}id{\isacharparenleft}{\kern0pt}B{\isacharparenright}{\kern0pt}{\isasymtimes}\isactrlsub f\ eval{\isacharunderscore}{\kern0pt}func\ {\isacharparenleft}{\kern0pt}A\isactrlbsup B\isactrlesup {\isacharparenright}{\kern0pt}\ C{\isacharparenright}{\kern0pt}{\isacharparenright}{\kern0pt}\ {\isasymcirc}\isactrlsub c\ id{\isacharparenleft}{\kern0pt}B\ {\isasymtimes}\isactrlsub c\ {\isacharparenleft}{\kern0pt}C\ {\isasymtimes}\isactrlsub c\ {\isacharparenleft}{\kern0pt}{\isacharparenleft}{\kern0pt}A\isactrlbsup B\isactrlesup {\isacharparenright}{\kern0pt}\isactrlbsup C\isactrlesup {\isacharparenright}{\kern0pt}{\isacharparenright}{\kern0pt}{\isacharparenright}{\kern0pt}{\isachardoublequoteclose}\isanewline
\ \ \ \ \ \ \ \ \ \ \isacommand{by}\isamarkupfalse%
\ {\isacharparenleft}{\kern0pt}simp\ add{\isacharcolon}{\kern0pt}\ right{\isacharunderscore}{\kern0pt}left{\isacharparenright}{\kern0pt}\isanewline
\ \ \ \ \ \ \ \ \isacommand{also}\isamarkupfalse%
\ \isacommand{have}\isamarkupfalse%
\ {\isachardoublequoteopen}{\isachardot}{\kern0pt}{\isachardot}{\kern0pt}{\isachardot}{\kern0pt}\ {\isacharequal}{\kern0pt}\ {\isacharparenleft}{\kern0pt}eval{\isacharunderscore}{\kern0pt}func\ A\ B{\isacharparenright}{\kern0pt}\ {\isasymcirc}\isactrlsub c\ {\isacharparenleft}{\kern0pt}id{\isacharparenleft}{\kern0pt}B{\isacharparenright}{\kern0pt}{\isasymtimes}\isactrlsub f\ eval{\isacharunderscore}{\kern0pt}func\ {\isacharparenleft}{\kern0pt}A\isactrlbsup B\isactrlesup {\isacharparenright}{\kern0pt}\ C{\isacharparenright}{\kern0pt}{\isachardoublequoteclose}\isanewline
\ \ \ \ \ \ \ \ \ \ \isacommand{by}\isamarkupfalse%
\ {\isacharparenleft}{\kern0pt}typecheck{\isacharunderscore}{\kern0pt}cfuncs{\isacharcomma}{\kern0pt}\ meson\ id{\isacharunderscore}{\kern0pt}right{\isacharunderscore}{\kern0pt}unit{\isadigit{2}}{\isacharparenright}{\kern0pt}\isanewline
\ \ \ \ \ \ \ \ \isacommand{also}\isamarkupfalse%
\ \isacommand{have}\isamarkupfalse%
\ {\isachardoublequoteopen}{\isachardot}{\kern0pt}{\isachardot}{\kern0pt}{\isachardot}{\kern0pt}\ {\isacharequal}{\kern0pt}\ eval{\isacharunderscore}{\kern0pt}func\ A\ B\ {\isasymcirc}\isactrlsub c\ id\isactrlsub c\ B\ {\isasymtimes}\isactrlsub f\ eval{\isacharunderscore}{\kern0pt}func\ {\isacharparenleft}{\kern0pt}A\isactrlbsup B\isactrlesup {\isacharparenright}{\kern0pt}\ C\ {\isasymcirc}\isactrlsub c\ id\isactrlsub c\ C\ {\isasymtimes}\isactrlsub f\ id\isactrlsub c\ {\isacharparenleft}{\kern0pt}A\isactrlbsup B\isactrlesup \isactrlbsup C\isactrlesup {\isacharparenright}{\kern0pt}{\isachardoublequoteclose}\isanewline
\ \ \ \ \ \ \ \ \ \ \isacommand{by}\isamarkupfalse%
\ {\isacharparenleft}{\kern0pt}typecheck{\isacharunderscore}{\kern0pt}cfuncs{\isacharcomma}{\kern0pt}\ simp\ add{\isacharcolon}{\kern0pt}\ id{\isacharunderscore}{\kern0pt}cross{\isacharunderscore}{\kern0pt}prod\ id{\isacharunderscore}{\kern0pt}right{\isacharunderscore}{\kern0pt}unit{\isadigit{2}}{\isacharparenright}{\kern0pt}\isanewline
\ \ \ \ \ \ \ \ \isacommand{then}\isamarkupfalse%
\ \isacommand{show}\isamarkupfalse%
\ {\isacharquery}{\kern0pt}thesis\ \isacommand{using}\isamarkupfalse%
\ calculation\ \isacommand{by}\isamarkupfalse%
\ auto\isanewline
\ \ \ \ \ \ \isacommand{qed}\isamarkupfalse%
\isanewline
\ \ \ \ \isacommand{qed}\isamarkupfalse%
\isanewline
\ \ \isacommand{qed}\isamarkupfalse%
\isanewline
\ \ \isacommand{have}\isamarkupfalse%
\ {\isachardoublequoteopen}{\isasympsi}\isactrlsup {\isasymsharp}\ {\isasymcirc}\isactrlsub c\ {\isasymphi}\isactrlsup {\isasymsharp}\isactrlsup {\isasymsharp}\ {\isacharequal}{\kern0pt}\ id{\isacharparenleft}{\kern0pt}A\isactrlbsup {\isacharparenleft}{\kern0pt}B\ {\isasymtimes}\isactrlsub c\ C{\isacharparenright}{\kern0pt}\isactrlesup {\isacharparenright}{\kern0pt}{\isachardoublequoteclose}\isanewline
\ \ \isacommand{proof}\isamarkupfalse%
{\isacharparenleft}{\kern0pt}etcs{\isacharunderscore}{\kern0pt}rule\ same{\isacharunderscore}{\kern0pt}evals{\isacharunderscore}{\kern0pt}equal{\isacharbrackleft}{\kern0pt}\isakeyword{where}\ X\ {\isacharequal}{\kern0pt}\ {\isachardoublequoteopen}A{\isachardoublequoteclose}{\isacharcomma}{\kern0pt}\ \isakeyword{where}\ A\ {\isacharequal}{\kern0pt}\ {\isachardoublequoteopen}{\isacharparenleft}{\kern0pt}B\ {\isasymtimes}\isactrlsub c\ C{\isacharparenright}{\kern0pt}{\isachardoublequoteclose}{\isacharbrackright}{\kern0pt}{\isacharparenright}{\kern0pt}\isanewline
\ \ \ \ \isacommand{show}\isamarkupfalse%
\ {\isachardoublequoteopen}eval{\isacharunderscore}{\kern0pt}func\ A\ {\isacharparenleft}{\kern0pt}B\ {\isasymtimes}\isactrlsub c\ C{\isacharparenright}{\kern0pt}\ {\isasymcirc}\isactrlsub c\ {\isacharparenleft}{\kern0pt}id\isactrlsub c\ {\isacharparenleft}{\kern0pt}B\ {\isasymtimes}\isactrlsub c\ C{\isacharparenright}{\kern0pt}\ {\isasymtimes}\isactrlsub f\ {\isacharparenleft}{\kern0pt}{\isasympsi}\isactrlsup {\isasymsharp}\ {\isasymcirc}\isactrlsub c\ {\isasymphi}\isactrlsup {\isasymsharp}\isactrlsup {\isasymsharp}{\isacharparenright}{\kern0pt}{\isacharparenright}{\kern0pt}\ {\isacharequal}{\kern0pt}\ \isanewline
\ \ \ \ \ \ \ \ \ \ eval{\isacharunderscore}{\kern0pt}func\ A\ {\isacharparenleft}{\kern0pt}B\ {\isasymtimes}\isactrlsub c\ C{\isacharparenright}{\kern0pt}\ {\isasymcirc}\isactrlsub c\ id\isactrlsub c\ {\isacharparenleft}{\kern0pt}B\ {\isasymtimes}\isactrlsub c\ C{\isacharparenright}{\kern0pt}\ {\isasymtimes}\isactrlsub f\ id\isactrlsub c\ {\isacharparenleft}{\kern0pt}A\isactrlbsup {\isacharparenleft}{\kern0pt}B\ {\isasymtimes}\isactrlsub c\ C{\isacharparenright}{\kern0pt}\isactrlesup {\isacharparenright}{\kern0pt}{\isachardoublequoteclose}\isanewline
\ \ \ \ \isacommand{proof}\isamarkupfalse%
\ {\isacharminus}{\kern0pt}\isanewline
\ \ \ \ \ \ \isacommand{have}\isamarkupfalse%
\ {\isachardoublequoteopen}eval{\isacharunderscore}{\kern0pt}func\ A\ {\isacharparenleft}{\kern0pt}B\ {\isasymtimes}\isactrlsub c\ C{\isacharparenright}{\kern0pt}\ {\isasymcirc}\isactrlsub c\ {\isacharparenleft}{\kern0pt}id\isactrlsub c\ {\isacharparenleft}{\kern0pt}B\ {\isasymtimes}\isactrlsub c\ C{\isacharparenright}{\kern0pt}\ {\isasymtimes}\isactrlsub f\ {\isacharparenleft}{\kern0pt}{\isasympsi}\isactrlsup {\isasymsharp}\ {\isasymcirc}\isactrlsub c\ {\isasymphi}\isactrlsup {\isasymsharp}\isactrlsup {\isasymsharp}{\isacharparenright}{\kern0pt}{\isacharparenright}{\kern0pt}\ {\isacharequal}{\kern0pt}\isanewline
\ \ \ \ \ \ \ \ \ \ \ \ eval{\isacharunderscore}{\kern0pt}func\ A\ {\isacharparenleft}{\kern0pt}B\ {\isasymtimes}\isactrlsub c\ C{\isacharparenright}{\kern0pt}\ {\isasymcirc}\isactrlsub c\ {\isacharparenleft}{\kern0pt}{\isacharparenleft}{\kern0pt}id\isactrlsub c\ {\isacharparenleft}{\kern0pt}B\ {\isasymtimes}\isactrlsub c\ C{\isacharparenright}{\kern0pt}\ {\isasymtimes}\isactrlsub f\ {\isacharparenleft}{\kern0pt}{\isasympsi}\isactrlsup {\isasymsharp}{\isacharparenright}{\kern0pt}{\isacharparenright}{\kern0pt}\ {\isasymcirc}\isactrlsub c\ {\isacharparenleft}{\kern0pt}id\isactrlsub c\ {\isacharparenleft}{\kern0pt}B\ {\isasymtimes}\isactrlsub c\ C{\isacharparenright}{\kern0pt}\ {\isasymtimes}\isactrlsub f\ {\isasymphi}\isactrlsup {\isasymsharp}\isactrlsup {\isasymsharp}{\isacharparenright}{\kern0pt}{\isacharparenright}{\kern0pt}{\isachardoublequoteclose}\isanewline
\ \ \ \ \ \ \ \ \isacommand{by}\isamarkupfalse%
\ {\isacharparenleft}{\kern0pt}typecheck{\isacharunderscore}{\kern0pt}cfuncs{\isacharcomma}{\kern0pt}\ simp\ add{\isacharcolon}{\kern0pt}\ identity{\isacharunderscore}{\kern0pt}distributes{\isacharunderscore}{\kern0pt}across{\isacharunderscore}{\kern0pt}composition{\isacharparenright}{\kern0pt}\isanewline
\ \ \ \ \ \ \isacommand{also}\isamarkupfalse%
\ \isacommand{have}\isamarkupfalse%
\ {\isachardoublequoteopen}{\isachardot}{\kern0pt}{\isachardot}{\kern0pt}{\isachardot}{\kern0pt}\ {\isacharequal}{\kern0pt}\ {\isacharparenleft}{\kern0pt}\ eval{\isacharunderscore}{\kern0pt}func\ A\ {\isacharparenleft}{\kern0pt}B\ {\isasymtimes}\isactrlsub c\ C{\isacharparenright}{\kern0pt}\ {\isasymcirc}\isactrlsub c\ {\isacharparenleft}{\kern0pt}id\isactrlsub c\ {\isacharparenleft}{\kern0pt}B\ {\isasymtimes}\isactrlsub c\ C{\isacharparenright}{\kern0pt}\ {\isasymtimes}\isactrlsub f\ {\isacharparenleft}{\kern0pt}{\isasympsi}\isactrlsup {\isasymsharp}{\isacharparenright}{\kern0pt}{\isacharparenright}{\kern0pt}{\isacharparenright}{\kern0pt}\ {\isasymcirc}\isactrlsub c\ {\isacharparenleft}{\kern0pt}id\isactrlsub c\ {\isacharparenleft}{\kern0pt}B\ {\isasymtimes}\isactrlsub c\ C{\isacharparenright}{\kern0pt}\ {\isasymtimes}\isactrlsub f\ {\isasymphi}\isactrlsup {\isasymsharp}\isactrlsup {\isasymsharp}{\isacharparenright}{\kern0pt}{\isachardoublequoteclose}\isanewline
\ \ \ \ \ \ \ \ \isacommand{using}\isamarkupfalse%
\ comp{\isacharunderscore}{\kern0pt}associative{\isadigit{2}}\ \isacommand{by}\isamarkupfalse%
\ {\isacharparenleft}{\kern0pt}typecheck{\isacharunderscore}{\kern0pt}cfuncs{\isacharcomma}{\kern0pt}\ blast{\isacharparenright}{\kern0pt}\isanewline
\ \ \ \ \ \ \isacommand{also}\isamarkupfalse%
\ \isacommand{have}\isamarkupfalse%
\ {\isachardoublequoteopen}{\isachardot}{\kern0pt}{\isachardot}{\kern0pt}{\isachardot}{\kern0pt}\ {\isacharequal}{\kern0pt}\ {\isasympsi}\ {\isasymcirc}\isactrlsub c\ {\isacharparenleft}{\kern0pt}id\isactrlsub c\ {\isacharparenleft}{\kern0pt}B\ {\isasymtimes}\isactrlsub c\ C{\isacharparenright}{\kern0pt}\ {\isasymtimes}\isactrlsub f\ {\isasymphi}\isactrlsup {\isasymsharp}\isactrlsup {\isasymsharp}{\isacharparenright}{\kern0pt}{\isachardoublequoteclose}\isanewline
\ \ \ \ \ \ \ \ \isacommand{by}\isamarkupfalse%
\ {\isacharparenleft}{\kern0pt}typecheck{\isacharunderscore}{\kern0pt}cfuncs{\isacharcomma}{\kern0pt}\ simp\ add{\isacharcolon}{\kern0pt}\ transpose{\isacharunderscore}{\kern0pt}func{\isacharunderscore}{\kern0pt}def{\isacharparenright}{\kern0pt}\isanewline
\ \ \ \ \ \ \isacommand{also}\isamarkupfalse%
\ \isacommand{have}\isamarkupfalse%
\ {\isachardoublequoteopen}{\isachardot}{\kern0pt}{\isachardot}{\kern0pt}{\isachardot}{\kern0pt}\ {\isacharequal}{\kern0pt}{\isacharparenleft}{\kern0pt}eval{\isacharunderscore}{\kern0pt}func\ A\ B{\isacharparenright}{\kern0pt}\ {\isasymcirc}\isactrlsub c\ {\isacharparenleft}{\kern0pt}id{\isacharparenleft}{\kern0pt}B{\isacharparenright}{\kern0pt}{\isasymtimes}\isactrlsub f\ eval{\isacharunderscore}{\kern0pt}func\ {\isacharparenleft}{\kern0pt}A\isactrlbsup B\isactrlesup {\isacharparenright}{\kern0pt}\ C{\isacharparenright}{\kern0pt}\ {\isasymcirc}\isactrlsub c\ {\isacharparenleft}{\kern0pt}associate{\isacharunderscore}{\kern0pt}right\ B\ C\ {\isacharparenleft}{\kern0pt}{\isacharparenleft}{\kern0pt}A\isactrlbsup B\isactrlesup {\isacharparenright}{\kern0pt}\isactrlbsup C\isactrlesup {\isacharparenright}{\kern0pt}{\isacharparenright}{\kern0pt}\ {\isasymcirc}\isactrlsub c\ {\isacharparenleft}{\kern0pt}id\isactrlsub c\ {\isacharparenleft}{\kern0pt}B\ {\isasymtimes}\isactrlsub c\ C{\isacharparenright}{\kern0pt}\ {\isasymtimes}\isactrlsub f\ {\isasymphi}\isactrlsup {\isasymsharp}\isactrlsup {\isasymsharp}{\isacharparenright}{\kern0pt}{\isachardoublequoteclose}\isanewline
\ \ \ \ \ \ \ \ \isacommand{by}\isamarkupfalse%
\ {\isacharparenleft}{\kern0pt}typecheck{\isacharunderscore}{\kern0pt}cfuncs{\isacharcomma}{\kern0pt}\ smt\ {\isasympsi}{\isacharunderscore}{\kern0pt}def\ cfunc{\isacharunderscore}{\kern0pt}type{\isacharunderscore}{\kern0pt}def\ comp{\isacharunderscore}{\kern0pt}associative\ domain{\isacharunderscore}{\kern0pt}comp{\isacharparenright}{\kern0pt}\isanewline
\ \ \ \ \ \ \isacommand{also}\isamarkupfalse%
\ \isacommand{have}\isamarkupfalse%
\ {\isachardoublequoteopen}{\isachardot}{\kern0pt}{\isachardot}{\kern0pt}{\isachardot}{\kern0pt}\ {\isacharequal}{\kern0pt}{\isacharparenleft}{\kern0pt}eval{\isacharunderscore}{\kern0pt}func\ A\ B{\isacharparenright}{\kern0pt}\ {\isasymcirc}\isactrlsub c\ {\isacharparenleft}{\kern0pt}id{\isacharparenleft}{\kern0pt}B{\isacharparenright}{\kern0pt}{\isasymtimes}\isactrlsub f\ eval{\isacharunderscore}{\kern0pt}func\ {\isacharparenleft}{\kern0pt}A\isactrlbsup B\isactrlesup {\isacharparenright}{\kern0pt}\ C{\isacharparenright}{\kern0pt}\ {\isasymcirc}\isactrlsub c\ {\isacharparenleft}{\kern0pt}associate{\isacharunderscore}{\kern0pt}right\ B\ C\ {\isacharparenleft}{\kern0pt}{\isacharparenleft}{\kern0pt}A\isactrlbsup B\isactrlesup {\isacharparenright}{\kern0pt}\isactrlbsup C\isactrlesup {\isacharparenright}{\kern0pt}{\isacharparenright}{\kern0pt}\ {\isasymcirc}\isactrlsub c\ {\isacharparenleft}{\kern0pt}{\isacharparenleft}{\kern0pt}id\isactrlsub c\ {\isacharparenleft}{\kern0pt}B{\isacharparenright}{\kern0pt}\ {\isasymtimes}\isactrlsub f\ id{\isacharparenleft}{\kern0pt}\ C{\isacharparenright}{\kern0pt}{\isacharparenright}{\kern0pt}\ {\isasymtimes}\isactrlsub f\ {\isasymphi}\isactrlsup {\isasymsharp}\isactrlsup {\isasymsharp}{\isacharparenright}{\kern0pt}{\isachardoublequoteclose}\isanewline
\ \ \ \ \ \ \ \ \isacommand{by}\isamarkupfalse%
\ {\isacharparenleft}{\kern0pt}typecheck{\isacharunderscore}{\kern0pt}cfuncs{\isacharcomma}{\kern0pt}\ simp\ add{\isacharcolon}{\kern0pt}\ id{\isacharunderscore}{\kern0pt}cross{\isacharunderscore}{\kern0pt}prod{\isacharparenright}{\kern0pt}\isanewline
\ \ \ \ \ \ \isacommand{also}\isamarkupfalse%
\ \isacommand{have}\isamarkupfalse%
\ {\isachardoublequoteopen}{\isachardot}{\kern0pt}{\isachardot}{\kern0pt}{\isachardot}{\kern0pt}\ {\isacharequal}{\kern0pt}{\isacharparenleft}{\kern0pt}eval{\isacharunderscore}{\kern0pt}func\ A\ B{\isacharparenright}{\kern0pt}\ {\isasymcirc}\isactrlsub c\ {\isacharparenleft}{\kern0pt}{\isacharparenleft}{\kern0pt}id{\isacharparenleft}{\kern0pt}B{\isacharparenright}{\kern0pt}{\isasymtimes}\isactrlsub f\ eval{\isacharunderscore}{\kern0pt}func\ {\isacharparenleft}{\kern0pt}A\isactrlbsup B\isactrlesup {\isacharparenright}{\kern0pt}\ C{\isacharparenright}{\kern0pt}\ {\isasymcirc}\isactrlsub c\ {\isacharparenleft}{\kern0pt}{\isacharparenleft}{\kern0pt}id\isactrlsub c\ {\isacharparenleft}{\kern0pt}B{\isacharparenright}{\kern0pt}\ {\isasymtimes}\isactrlsub f\ {\isacharparenleft}{\kern0pt}id{\isacharparenleft}{\kern0pt}C{\isacharparenright}{\kern0pt}\ {\isasymtimes}\isactrlsub f\ {\isasymphi}\isactrlsup {\isasymsharp}\isactrlsup {\isasymsharp}{\isacharparenright}{\kern0pt}{\isacharparenright}{\kern0pt}\ {\isasymcirc}\isactrlsub c\ {\isacharparenleft}{\kern0pt}associate{\isacharunderscore}{\kern0pt}right\ B\ C\ {\isacharparenleft}{\kern0pt}A\isactrlbsup {\isacharparenleft}{\kern0pt}B\ {\isasymtimes}\isactrlsub c\ C{\isacharparenright}{\kern0pt}\isactrlesup {\isacharparenright}{\kern0pt}{\isacharparenright}{\kern0pt}{\isacharparenright}{\kern0pt}{\isacharparenright}{\kern0pt}{\isachardoublequoteclose}\isanewline
\ \ \ \ \ \ \ \ \isacommand{using}\isamarkupfalse%
\ associate{\isacharunderscore}{\kern0pt}right{\isacharunderscore}{\kern0pt}crossprod{\isacharunderscore}{\kern0pt}ap\ \isacommand{by}\isamarkupfalse%
\ {\isacharparenleft}{\kern0pt}typecheck{\isacharunderscore}{\kern0pt}cfuncs{\isacharcomma}{\kern0pt}\ auto{\isacharparenright}{\kern0pt}\isanewline
\ \ \ \ \ \ \isacommand{also}\isamarkupfalse%
\ \isacommand{have}\isamarkupfalse%
\ {\isachardoublequoteopen}{\isachardot}{\kern0pt}{\isachardot}{\kern0pt}{\isachardot}{\kern0pt}\ {\isacharequal}{\kern0pt}{\isacharparenleft}{\kern0pt}eval{\isacharunderscore}{\kern0pt}func\ A\ B{\isacharparenright}{\kern0pt}\ {\isasymcirc}\isactrlsub c\ {\isacharparenleft}{\kern0pt}{\isacharparenleft}{\kern0pt}id{\isacharparenleft}{\kern0pt}B{\isacharparenright}{\kern0pt}{\isasymtimes}\isactrlsub f\ eval{\isacharunderscore}{\kern0pt}func\ {\isacharparenleft}{\kern0pt}A\isactrlbsup B\isactrlesup {\isacharparenright}{\kern0pt}\ C{\isacharparenright}{\kern0pt}\ {\isasymcirc}\isactrlsub c\ {\isacharparenleft}{\kern0pt}id\isactrlsub c\ {\isacharparenleft}{\kern0pt}B{\isacharparenright}{\kern0pt}\ {\isasymtimes}\isactrlsub f\ {\isacharparenleft}{\kern0pt}id{\isacharparenleft}{\kern0pt}C{\isacharparenright}{\kern0pt}\ {\isasymtimes}\isactrlsub f\ {\isasymphi}\isactrlsup {\isasymsharp}\isactrlsup {\isasymsharp}{\isacharparenright}{\kern0pt}{\isacharparenright}{\kern0pt}{\isacharparenright}{\kern0pt}\ {\isasymcirc}\isactrlsub c\ {\isacharparenleft}{\kern0pt}associate{\isacharunderscore}{\kern0pt}right\ B\ C\ {\isacharparenleft}{\kern0pt}A\isactrlbsup {\isacharparenleft}{\kern0pt}B\ {\isasymtimes}\isactrlsub c\ C{\isacharparenright}{\kern0pt}\isactrlesup {\isacharparenright}{\kern0pt}{\isacharparenright}{\kern0pt}{\isachardoublequoteclose}\isanewline
\ \ \ \ \ \ \ \ \isacommand{by}\isamarkupfalse%
\ {\isacharparenleft}{\kern0pt}typecheck{\isacharunderscore}{\kern0pt}cfuncs{\isacharcomma}{\kern0pt}\ simp\ add{\isacharcolon}{\kern0pt}\ comp{\isacharunderscore}{\kern0pt}associative{\isadigit{2}}{\isacharparenright}{\kern0pt}\isanewline
\ \ \ \ \ \ \isacommand{also}\isamarkupfalse%
\ \isacommand{have}\isamarkupfalse%
\ {\isachardoublequoteopen}{\isachardot}{\kern0pt}{\isachardot}{\kern0pt}{\isachardot}{\kern0pt}\ {\isacharequal}{\kern0pt}{\isacharparenleft}{\kern0pt}eval{\isacharunderscore}{\kern0pt}func\ A\ B{\isacharparenright}{\kern0pt}\ {\isasymcirc}\isactrlsub c\ {\isacharparenleft}{\kern0pt}id{\isacharparenleft}{\kern0pt}B{\isacharparenright}{\kern0pt}{\isasymtimes}\isactrlsub f\ {\isacharparenleft}{\kern0pt}{\isacharparenleft}{\kern0pt}eval{\isacharunderscore}{\kern0pt}func\ {\isacharparenleft}{\kern0pt}A\isactrlbsup B\isactrlesup {\isacharparenright}{\kern0pt}\ C{\isacharparenright}{\kern0pt}{\isasymcirc}\isactrlsub c\ {\isacharparenleft}{\kern0pt}id{\isacharparenleft}{\kern0pt}C{\isacharparenright}{\kern0pt}\ {\isasymtimes}\isactrlsub f\ {\isasymphi}\isactrlsup {\isasymsharp}\isactrlsup {\isasymsharp}{\isacharparenright}{\kern0pt}{\isacharparenright}{\kern0pt}{\isacharparenright}{\kern0pt}\ {\isasymcirc}\isactrlsub c\ {\isacharparenleft}{\kern0pt}associate{\isacharunderscore}{\kern0pt}right\ B\ C\ {\isacharparenleft}{\kern0pt}A\isactrlbsup {\isacharparenleft}{\kern0pt}B\ {\isasymtimes}\isactrlsub c\ C{\isacharparenright}{\kern0pt}\isactrlesup {\isacharparenright}{\kern0pt}{\isacharparenright}{\kern0pt}{\isachardoublequoteclose}\isanewline
\ \ \ \ \ \ \ \ \isacommand{using}\isamarkupfalse%
\ identity{\isacharunderscore}{\kern0pt}distributes{\isacharunderscore}{\kern0pt}across{\isacharunderscore}{\kern0pt}composition\ \isacommand{by}\isamarkupfalse%
\ {\isacharparenleft}{\kern0pt}typecheck{\isacharunderscore}{\kern0pt}cfuncs{\isacharcomma}{\kern0pt}\ auto{\isacharparenright}{\kern0pt}\isanewline
\ \ \ \ \ \ \isacommand{also}\isamarkupfalse%
\ \isacommand{have}\isamarkupfalse%
\ {\isachardoublequoteopen}{\isachardot}{\kern0pt}{\isachardot}{\kern0pt}{\isachardot}{\kern0pt}\ {\isacharequal}{\kern0pt}{\isacharparenleft}{\kern0pt}eval{\isacharunderscore}{\kern0pt}func\ A\ B{\isacharparenright}{\kern0pt}\ {\isasymcirc}\isactrlsub c\ {\isacharparenleft}{\kern0pt}id{\isacharparenleft}{\kern0pt}B{\isacharparenright}{\kern0pt}{\isasymtimes}\isactrlsub f\ {\isasymphi}\isactrlsup {\isasymsharp}{\isacharparenright}{\kern0pt}\ {\isasymcirc}\isactrlsub c\ {\isacharparenleft}{\kern0pt}associate{\isacharunderscore}{\kern0pt}right\ B\ C\ {\isacharparenleft}{\kern0pt}A\isactrlbsup {\isacharparenleft}{\kern0pt}B\ {\isasymtimes}\isactrlsub c\ C{\isacharparenright}{\kern0pt}\isactrlesup {\isacharparenright}{\kern0pt}{\isacharparenright}{\kern0pt}{\isachardoublequoteclose}\isanewline
\ \ \ \ \ \ \ \ \isacommand{by}\isamarkupfalse%
\ {\isacharparenleft}{\kern0pt}typecheck{\isacharunderscore}{\kern0pt}cfuncs{\isacharcomma}{\kern0pt}\ simp\ add{\isacharcolon}{\kern0pt}\ transpose{\isacharunderscore}{\kern0pt}func{\isacharunderscore}{\kern0pt}def{\isacharparenright}{\kern0pt}\isanewline
\ \ \ \ \ \ \isacommand{also}\isamarkupfalse%
\ \isacommand{have}\isamarkupfalse%
\ {\isachardoublequoteopen}{\isachardot}{\kern0pt}{\isachardot}{\kern0pt}{\isachardot}{\kern0pt}\ {\isacharequal}{\kern0pt}{\isacharparenleft}{\kern0pt}{\isacharparenleft}{\kern0pt}eval{\isacharunderscore}{\kern0pt}func\ A\ B{\isacharparenright}{\kern0pt}\ {\isasymcirc}\isactrlsub c\ {\isacharparenleft}{\kern0pt}id{\isacharparenleft}{\kern0pt}B{\isacharparenright}{\kern0pt}{\isasymtimes}\isactrlsub f\ {\isasymphi}\isactrlsup {\isasymsharp}{\isacharparenright}{\kern0pt}{\isacharparenright}{\kern0pt}\ {\isasymcirc}\isactrlsub c\ {\isacharparenleft}{\kern0pt}associate{\isacharunderscore}{\kern0pt}right\ B\ C\ {\isacharparenleft}{\kern0pt}A\isactrlbsup {\isacharparenleft}{\kern0pt}B\ {\isasymtimes}\isactrlsub c\ C{\isacharparenright}{\kern0pt}\isactrlesup {\isacharparenright}{\kern0pt}{\isacharparenright}{\kern0pt}{\isachardoublequoteclose}\isanewline
\ \ \ \ \ \ \ \ \isacommand{using}\isamarkupfalse%
\ comp{\isacharunderscore}{\kern0pt}associative{\isadigit{2}}\ \isacommand{by}\isamarkupfalse%
\ {\isacharparenleft}{\kern0pt}typecheck{\isacharunderscore}{\kern0pt}cfuncs{\isacharcomma}{\kern0pt}\ blast{\isacharparenright}{\kern0pt}\isanewline
\ \ \ \ \ \ \isacommand{also}\isamarkupfalse%
\ \isacommand{have}\isamarkupfalse%
\ {\isachardoublequoteopen}{\isachardot}{\kern0pt}{\isachardot}{\kern0pt}{\isachardot}{\kern0pt}\ {\isacharequal}{\kern0pt}\ {\isasymphi}\ {\isasymcirc}\isactrlsub c\ {\isacharparenleft}{\kern0pt}associate{\isacharunderscore}{\kern0pt}right\ B\ C\ {\isacharparenleft}{\kern0pt}A\isactrlbsup {\isacharparenleft}{\kern0pt}B\ {\isasymtimes}\isactrlsub c\ C{\isacharparenright}{\kern0pt}\isactrlesup {\isacharparenright}{\kern0pt}{\isacharparenright}{\kern0pt}{\isachardoublequoteclose}\isanewline
\ \ \ \ \ \ \ \ \isacommand{by}\isamarkupfalse%
\ {\isacharparenleft}{\kern0pt}typecheck{\isacharunderscore}{\kern0pt}cfuncs{\isacharcomma}{\kern0pt}\ simp\ add{\isacharcolon}{\kern0pt}\ transpose{\isacharunderscore}{\kern0pt}func{\isacharunderscore}{\kern0pt}def{\isacharparenright}{\kern0pt}\isanewline
\ \ \ \ \ \ \isacommand{also}\isamarkupfalse%
\ \isacommand{have}\isamarkupfalse%
\ {\isachardoublequoteopen}{\isachardot}{\kern0pt}{\isachardot}{\kern0pt}{\isachardot}{\kern0pt}\ {\isacharequal}{\kern0pt}\ {\isacharparenleft}{\kern0pt}eval{\isacharunderscore}{\kern0pt}func\ A\ {\isacharparenleft}{\kern0pt}B{\isasymtimes}\isactrlsub c\ C{\isacharparenright}{\kern0pt}{\isacharparenright}{\kern0pt}\ {\isasymcirc}\isactrlsub c\ {\isacharparenleft}{\kern0pt}{\isacharparenleft}{\kern0pt}associate{\isacharunderscore}{\kern0pt}left\ B\ C\ {\isacharparenleft}{\kern0pt}A\isactrlbsup {\isacharparenleft}{\kern0pt}B{\isasymtimes}\isactrlsub c\ C{\isacharparenright}{\kern0pt}\isactrlesup {\isacharparenright}{\kern0pt}{\isacharparenright}{\kern0pt}\ {\isasymcirc}\isactrlsub c\ {\isacharparenleft}{\kern0pt}associate{\isacharunderscore}{\kern0pt}right\ B\ C\ {\isacharparenleft}{\kern0pt}A\isactrlbsup {\isacharparenleft}{\kern0pt}B\ {\isasymtimes}\isactrlsub c\ C{\isacharparenright}{\kern0pt}\isactrlesup {\isacharparenright}{\kern0pt}{\isacharparenright}{\kern0pt}{\isacharparenright}{\kern0pt}{\isachardoublequoteclose}\isanewline
\ \ \ \ \ \ \ \ \isacommand{by}\isamarkupfalse%
\ {\isacharparenleft}{\kern0pt}typecheck{\isacharunderscore}{\kern0pt}cfuncs{\isacharcomma}{\kern0pt}\ simp\ add{\isacharcolon}{\kern0pt}\ {\isasymphi}{\isacharunderscore}{\kern0pt}def\ comp{\isacharunderscore}{\kern0pt}associative{\isadigit{2}}{\isacharparenright}{\kern0pt}\ \ \isanewline
\ \ \ \ \ \ \isacommand{also}\isamarkupfalse%
\ \isacommand{have}\isamarkupfalse%
\ {\isachardoublequoteopen}{\isachardot}{\kern0pt}{\isachardot}{\kern0pt}{\isachardot}{\kern0pt}\ {\isacharequal}{\kern0pt}\ eval{\isacharunderscore}{\kern0pt}func\ A\ {\isacharparenleft}{\kern0pt}B{\isasymtimes}\isactrlsub c\ C{\isacharparenright}{\kern0pt}\ {\isasymcirc}\isactrlsub c\ id\ {\isacharparenleft}{\kern0pt}{\isacharparenleft}{\kern0pt}B\ {\isasymtimes}\isactrlsub c\ C{\isacharparenright}{\kern0pt}\ {\isasymtimes}\isactrlsub c\ {\isacharparenleft}{\kern0pt}A\isactrlbsup {\isacharparenleft}{\kern0pt}B{\isasymtimes}\isactrlsub c\ C{\isacharparenright}{\kern0pt}\isactrlesup {\isacharparenright}{\kern0pt}{\isacharparenright}{\kern0pt}{\isachardoublequoteclose}\isanewline
\ \ \ \ \ \ \ \ \isacommand{by}\isamarkupfalse%
\ {\isacharparenleft}{\kern0pt}typecheck{\isacharunderscore}{\kern0pt}cfuncs{\isacharcomma}{\kern0pt}\ simp\ add{\isacharcolon}{\kern0pt}\ left{\isacharunderscore}{\kern0pt}right{\isacharparenright}{\kern0pt}\isanewline
\ \ \ \ \ \ \isacommand{also}\isamarkupfalse%
\ \isacommand{have}\isamarkupfalse%
\ {\isachardoublequoteopen}{\isachardot}{\kern0pt}{\isachardot}{\kern0pt}{\isachardot}{\kern0pt}\ {\isacharequal}{\kern0pt}\ eval{\isacharunderscore}{\kern0pt}func\ A\ {\isacharparenleft}{\kern0pt}B\ {\isasymtimes}\isactrlsub c\ C{\isacharparenright}{\kern0pt}\ {\isasymcirc}\isactrlsub c\ id\isactrlsub c\ {\isacharparenleft}{\kern0pt}B\ {\isasymtimes}\isactrlsub c\ C{\isacharparenright}{\kern0pt}\ {\isasymtimes}\isactrlsub f\ id\isactrlsub c\ {\isacharparenleft}{\kern0pt}A\isactrlbsup {\isacharparenleft}{\kern0pt}B\ {\isasymtimes}\isactrlsub c\ C{\isacharparenright}{\kern0pt}\isactrlesup {\isacharparenright}{\kern0pt}{\isachardoublequoteclose}\isanewline
\ \ \ \ \ \ \ \ \isacommand{by}\isamarkupfalse%
\ {\isacharparenleft}{\kern0pt}typecheck{\isacharunderscore}{\kern0pt}cfuncs{\isacharcomma}{\kern0pt}\ simp\ add{\isacharcolon}{\kern0pt}\ id{\isacharunderscore}{\kern0pt}cross{\isacharunderscore}{\kern0pt}prod{\isacharparenright}{\kern0pt}\isanewline
\ \ \ \ \ \ \isacommand{then}\isamarkupfalse%
\ \isacommand{show}\isamarkupfalse%
\ {\isacharquery}{\kern0pt}thesis\ \isacommand{using}\isamarkupfalse%
\ calculation\ \isacommand{by}\isamarkupfalse%
\ auto\isanewline
\ \ \ \ \isacommand{qed}\isamarkupfalse%
\isanewline
\ \ \isacommand{qed}\isamarkupfalse%
\isanewline
\ \ \isacommand{show}\isamarkupfalse%
\ {\isacharquery}{\kern0pt}thesis\isanewline
\ \ \ \ \isacommand{by}\isamarkupfalse%
\ {\isacharparenleft}{\kern0pt}metis\ {\isacartoucheopen}{\isasymphi}\isactrlsup {\isasymsharp}\isactrlsup {\isasymsharp}\ {\isasymcirc}\isactrlsub c\ {\isasympsi}\isactrlsup {\isasymsharp}\ {\isacharequal}{\kern0pt}\ id\isactrlsub c\ {\isacharparenleft}{\kern0pt}A\isactrlbsup B\isactrlesup \isactrlbsup C\isactrlesup {\isacharparenright}{\kern0pt}{\isacartoucheclose}\ {\isacartoucheopen}{\isasympsi}\isactrlsup {\isasymsharp}\ {\isasymcirc}\isactrlsub c\ {\isasymphi}\isactrlsup {\isasymsharp}\isactrlsup {\isasymsharp}\ {\isacharequal}{\kern0pt}\ id\isactrlsub c\ {\isacharparenleft}{\kern0pt}A\isactrlbsup {\isacharparenleft}{\kern0pt}B\ {\isasymtimes}\isactrlsub c\ C{\isacharparenright}{\kern0pt}\isactrlesup {\isacharparenright}{\kern0pt}{\isacartoucheclose}\ {\isasymphi}dbsharp{\isacharunderscore}{\kern0pt}type\ {\isasympsi}sharp{\isacharunderscore}{\kern0pt}type\ cfunc{\isacharunderscore}{\kern0pt}type{\isacharunderscore}{\kern0pt}def\ is{\isacharunderscore}{\kern0pt}isomorphic{\isacharunderscore}{\kern0pt}def\ isomorphism{\isacharunderscore}{\kern0pt}def{\isacharparenright}{\kern0pt}\isanewline
\isacommand{qed}\isamarkupfalse%
%
\endisatagproof
{\isafoldproof}%
%
\isadelimproof
\isanewline
%
\endisadelimproof
\isanewline
\isacommand{lemma}\isamarkupfalse%
\ exp{\isacharunderscore}{\kern0pt}pres{\isacharunderscore}{\kern0pt}iso{\isacharunderscore}{\kern0pt}right{\isacharcolon}{\kern0pt}\isanewline
\ \ \isakeyword{assumes}\ {\isachardoublequoteopen}A\ {\isasymcong}\ X{\isachardoublequoteclose}\ \isanewline
\ \ \isakeyword{shows}\ {\isachardoublequoteopen}Y\isactrlbsup A\isactrlesup \ {\isasymcong}\ \ Y\isactrlbsup X\isactrlesup {\isachardoublequoteclose}\isanewline
%
\isadelimproof
%
\endisadelimproof
%
\isatagproof
\isacommand{proof}\isamarkupfalse%
\ {\isacharminus}{\kern0pt}\ \isanewline
\ \ \isacommand{obtain}\isamarkupfalse%
\ {\isasymphi}\ \isakeyword{where}\ {\isasymphi}{\isacharunderscore}{\kern0pt}def{\isacharcolon}{\kern0pt}\ {\isachardoublequoteopen}{\isasymphi}{\isacharcolon}{\kern0pt}\ X\ {\isasymrightarrow}\ A\ {\isasymand}\ isomorphism{\isacharparenleft}{\kern0pt}{\isasymphi}{\isacharparenright}{\kern0pt}{\isachardoublequoteclose}\isanewline
\ \ \ \ \isacommand{using}\isamarkupfalse%
\ assms\ is{\isacharunderscore}{\kern0pt}isomorphic{\isacharunderscore}{\kern0pt}def\ isomorphic{\isacharunderscore}{\kern0pt}is{\isacharunderscore}{\kern0pt}symmetric\ \isacommand{by}\isamarkupfalse%
\ blast\isanewline
\ \ \isacommand{obtain}\isamarkupfalse%
\ {\isasympsi}\ \isakeyword{where}\ {\isasympsi}{\isacharunderscore}{\kern0pt}def{\isacharcolon}{\kern0pt}\ {\isachardoublequoteopen}{\isasympsi}{\isacharcolon}{\kern0pt}\ A\ {\isasymrightarrow}\ X\ {\isasymand}\ isomorphism{\isacharparenleft}{\kern0pt}{\isasympsi}{\isacharparenright}{\kern0pt}\ {\isasymand}\ {\isacharparenleft}{\kern0pt}{\isasympsi}\ {\isasymcirc}\isactrlsub c\ {\isasymphi}\ {\isacharequal}{\kern0pt}\ id{\isacharparenleft}{\kern0pt}X{\isacharparenright}{\kern0pt}{\isacharparenright}{\kern0pt}{\isachardoublequoteclose}\isanewline
\ \ \ \ \isacommand{using}\isamarkupfalse%
\ {\isasymphi}{\isacharunderscore}{\kern0pt}def\ cfunc{\isacharunderscore}{\kern0pt}type{\isacharunderscore}{\kern0pt}def\ isomorphism{\isacharunderscore}{\kern0pt}def\ \isacommand{by}\isamarkupfalse%
\ fastforce\isanewline
\ \ \isacommand{have}\isamarkupfalse%
\ idA{\isacharcolon}{\kern0pt}\ {\isachardoublequoteopen}{\isasymphi}\ {\isasymcirc}\isactrlsub c\ {\isasympsi}\ {\isacharequal}{\kern0pt}\ id{\isacharparenleft}{\kern0pt}A{\isacharparenright}{\kern0pt}{\isachardoublequoteclose}\isanewline
\ \ \ \ \isacommand{by}\isamarkupfalse%
\ {\isacharparenleft}{\kern0pt}metis\ {\isasymphi}{\isacharunderscore}{\kern0pt}def\ {\isasympsi}{\isacharunderscore}{\kern0pt}def\ cfunc{\isacharunderscore}{\kern0pt}type{\isacharunderscore}{\kern0pt}def\ comp{\isacharunderscore}{\kern0pt}associative\ id{\isacharunderscore}{\kern0pt}left{\isacharunderscore}{\kern0pt}unit{\isadigit{2}}\ isomorphism{\isacharunderscore}{\kern0pt}def{\isacharparenright}{\kern0pt}\isanewline
\ \ \isacommand{obtain}\isamarkupfalse%
\ f\ \isakeyword{where}\ f{\isacharunderscore}{\kern0pt}def{\isacharcolon}{\kern0pt}\ {\isachardoublequoteopen}f\ {\isacharequal}{\kern0pt}\ {\isacharparenleft}{\kern0pt}eval{\isacharunderscore}{\kern0pt}func\ Y\ X{\isacharparenright}{\kern0pt}\ {\isasymcirc}\isactrlsub c\ {\isacharparenleft}{\kern0pt}{\isasympsi}\ {\isasymtimes}\isactrlsub f\ id{\isacharparenleft}{\kern0pt}Y\isactrlbsup X\isactrlesup {\isacharparenright}{\kern0pt}{\isacharparenright}{\kern0pt}{\isachardoublequoteclose}\ \isakeyword{and}\ f{\isacharunderscore}{\kern0pt}type{\isacharbrackleft}{\kern0pt}type{\isacharunderscore}{\kern0pt}rule{\isacharbrackright}{\kern0pt}{\isacharcolon}{\kern0pt}\ {\isachardoublequoteopen}f{\isacharcolon}{\kern0pt}\ A{\isasymtimes}\isactrlsub c\ {\isacharparenleft}{\kern0pt}Y\isactrlbsup X\isactrlesup {\isacharparenright}{\kern0pt}\ {\isasymrightarrow}\ Y{\isachardoublequoteclose}\ \isakeyword{and}\ fsharp{\isacharunderscore}{\kern0pt}type{\isacharbrackleft}{\kern0pt}type{\isacharunderscore}{\kern0pt}rule{\isacharbrackright}{\kern0pt}{\isacharcolon}{\kern0pt}\ {\isachardoublequoteopen}f\isactrlsup {\isasymsharp}\ {\isacharcolon}{\kern0pt}\ Y\isactrlbsup X\isactrlesup \ {\isasymrightarrow}\ Y\isactrlbsup A\isactrlesup {\isachardoublequoteclose}\isanewline
\ \ \ \ \isacommand{using}\isamarkupfalse%
\ {\isasympsi}{\isacharunderscore}{\kern0pt}def\ transpose{\isacharunderscore}{\kern0pt}func{\isacharunderscore}{\kern0pt}type\ \isacommand{by}\isamarkupfalse%
\ {\isacharparenleft}{\kern0pt}typecheck{\isacharunderscore}{\kern0pt}cfuncs{\isacharcomma}{\kern0pt}\ presburger{\isacharparenright}{\kern0pt}\isanewline
\ \ \isacommand{obtain}\isamarkupfalse%
\ g\ \isakeyword{where}\ g{\isacharunderscore}{\kern0pt}def{\isacharcolon}{\kern0pt}\ {\isachardoublequoteopen}g\ {\isacharequal}{\kern0pt}\ {\isacharparenleft}{\kern0pt}eval{\isacharunderscore}{\kern0pt}func\ Y\ A{\isacharparenright}{\kern0pt}\ {\isasymcirc}\isactrlsub c\ {\isacharparenleft}{\kern0pt}{\isasymphi}\ {\isasymtimes}\isactrlsub f\ id{\isacharparenleft}{\kern0pt}Y\isactrlbsup A\isactrlesup {\isacharparenright}{\kern0pt}{\isacharparenright}{\kern0pt}{\isachardoublequoteclose}\ \isakeyword{and}\ \ g{\isacharunderscore}{\kern0pt}type{\isacharbrackleft}{\kern0pt}type{\isacharunderscore}{\kern0pt}rule{\isacharbrackright}{\kern0pt}{\isacharcolon}{\kern0pt}\ {\isachardoublequoteopen}g{\isacharcolon}{\kern0pt}\ X{\isasymtimes}\isactrlsub c\ {\isacharparenleft}{\kern0pt}Y\isactrlbsup A\isactrlesup {\isacharparenright}{\kern0pt}\ {\isasymrightarrow}\ Y{\isachardoublequoteclose}\ \isakeyword{and}\ gsharp{\isacharunderscore}{\kern0pt}type{\isacharbrackleft}{\kern0pt}type{\isacharunderscore}{\kern0pt}rule{\isacharbrackright}{\kern0pt}{\isacharcolon}{\kern0pt}\ {\isachardoublequoteopen}g\isactrlsup {\isasymsharp}\ {\isacharcolon}{\kern0pt}\ Y\isactrlbsup A\isactrlesup \ {\isasymrightarrow}\ Y\isactrlbsup X\isactrlesup {\isachardoublequoteclose}\isanewline
\ \ \ \ \isacommand{using}\isamarkupfalse%
\ {\isasymphi}{\isacharunderscore}{\kern0pt}def\ transpose{\isacharunderscore}{\kern0pt}func{\isacharunderscore}{\kern0pt}type\ \isacommand{by}\isamarkupfalse%
\ {\isacharparenleft}{\kern0pt}typecheck{\isacharunderscore}{\kern0pt}cfuncs{\isacharcomma}{\kern0pt}\ presburger{\isacharparenright}{\kern0pt}\isanewline
\isanewline
\ \ \isacommand{have}\isamarkupfalse%
\ fsharp{\isacharunderscore}{\kern0pt}gsharp{\isacharunderscore}{\kern0pt}id{\isacharcolon}{\kern0pt}\ {\isachardoublequoteopen}f\isactrlsup {\isasymsharp}\ {\isasymcirc}\isactrlsub c\ g\isactrlsup {\isasymsharp}\ {\isacharequal}{\kern0pt}\ id{\isacharparenleft}{\kern0pt}Y\isactrlbsup A\isactrlesup {\isacharparenright}{\kern0pt}{\isachardoublequoteclose}\isanewline
\ \ \isacommand{proof}\isamarkupfalse%
{\isacharparenleft}{\kern0pt}etcs{\isacharunderscore}{\kern0pt}rule\ same{\isacharunderscore}{\kern0pt}evals{\isacharunderscore}{\kern0pt}equal{\isacharbrackleft}{\kern0pt}\isakeyword{where}\ X\ {\isacharequal}{\kern0pt}\ Y{\isacharcomma}{\kern0pt}\ \isakeyword{where}\ A\ {\isacharequal}{\kern0pt}\ A{\isacharbrackright}{\kern0pt}{\isacharparenright}{\kern0pt}\isanewline
\ \ \ \ \isacommand{have}\isamarkupfalse%
\ {\isachardoublequoteopen}eval{\isacharunderscore}{\kern0pt}func\ Y\ A\ {\isasymcirc}\isactrlsub c\ id\isactrlsub c\ A\ {\isasymtimes}\isactrlsub f\ f\isactrlsup {\isasymsharp}\ {\isasymcirc}\isactrlsub c\ g\isactrlsup {\isasymsharp}\ {\isacharequal}{\kern0pt}\ eval{\isacharunderscore}{\kern0pt}func\ Y\ A\ {\isasymcirc}\isactrlsub c\ {\isacharparenleft}{\kern0pt}id\isactrlsub c\ A\ {\isasymtimes}\isactrlsub f\ f\isactrlsup {\isasymsharp}{\isacharparenright}{\kern0pt}\ {\isasymcirc}\isactrlsub c\ {\isacharparenleft}{\kern0pt}id\isactrlsub c\ A\ {\isasymtimes}\isactrlsub f\ g\isactrlsup {\isasymsharp}{\isacharparenright}{\kern0pt}{\isachardoublequoteclose}\isanewline
\ \ \ \ \ \ \isacommand{using}\isamarkupfalse%
\ fsharp{\isacharunderscore}{\kern0pt}type\ gsharp{\isacharunderscore}{\kern0pt}type\ identity{\isacharunderscore}{\kern0pt}distributes{\isacharunderscore}{\kern0pt}across{\isacharunderscore}{\kern0pt}composition\ \isacommand{by}\isamarkupfalse%
\ auto\isanewline
\ \ \ \ \isacommand{also}\isamarkupfalse%
\ \isacommand{have}\isamarkupfalse%
\ {\isachardoublequoteopen}{\isachardot}{\kern0pt}{\isachardot}{\kern0pt}{\isachardot}{\kern0pt}\ {\isacharequal}{\kern0pt}\ eval{\isacharunderscore}{\kern0pt}func\ Y\ X\ {\isasymcirc}\isactrlsub c\ {\isacharparenleft}{\kern0pt}{\isasympsi}\ {\isasymtimes}\isactrlsub f\ id{\isacharparenleft}{\kern0pt}Y\isactrlbsup X\isactrlesup {\isacharparenright}{\kern0pt}{\isacharparenright}{\kern0pt}\ {\isasymcirc}\isactrlsub c\ {\isacharparenleft}{\kern0pt}id\isactrlsub c\ A\ {\isasymtimes}\isactrlsub f\ g\isactrlsup {\isasymsharp}{\isacharparenright}{\kern0pt}{\isachardoublequoteclose}\isanewline
\ \ \ \ \ \ \isacommand{using}\isamarkupfalse%
\ {\isasympsi}{\isacharunderscore}{\kern0pt}def\ cfunc{\isacharunderscore}{\kern0pt}type{\isacharunderscore}{\kern0pt}def\ comp{\isacharunderscore}{\kern0pt}associative\ f{\isacharunderscore}{\kern0pt}def\ f{\isacharunderscore}{\kern0pt}type\ gsharp{\isacharunderscore}{\kern0pt}type\ transpose{\isacharunderscore}{\kern0pt}func{\isacharunderscore}{\kern0pt}def\ \isacommand{by}\isamarkupfalse%
\ {\isacharparenleft}{\kern0pt}typecheck{\isacharunderscore}{\kern0pt}cfuncs{\isacharcomma}{\kern0pt}\ smt{\isacharparenright}{\kern0pt}\isanewline
\ \ \ \ \isacommand{also}\isamarkupfalse%
\ \isacommand{have}\isamarkupfalse%
\ {\isachardoublequoteopen}{\isachardot}{\kern0pt}{\isachardot}{\kern0pt}{\isachardot}{\kern0pt}\ {\isacharequal}{\kern0pt}\ eval{\isacharunderscore}{\kern0pt}func\ Y\ X\ {\isasymcirc}\isactrlsub c\ {\isacharparenleft}{\kern0pt}{\isasympsi}\ {\isasymtimes}\isactrlsub f\ g\isactrlsup {\isasymsharp}{\isacharparenright}{\kern0pt}{\isachardoublequoteclose}\isanewline
\ \ \ \ \ \ \isacommand{by}\isamarkupfalse%
\ {\isacharparenleft}{\kern0pt}smt\ {\isasympsi}{\isacharunderscore}{\kern0pt}def\ cfunc{\isacharunderscore}{\kern0pt}cross{\isacharunderscore}{\kern0pt}prod{\isacharunderscore}{\kern0pt}comp{\isacharunderscore}{\kern0pt}cfunc{\isacharunderscore}{\kern0pt}cross{\isacharunderscore}{\kern0pt}prod\ gsharp{\isacharunderscore}{\kern0pt}type\ id{\isacharunderscore}{\kern0pt}left{\isacharunderscore}{\kern0pt}unit{\isadigit{2}}\ id{\isacharunderscore}{\kern0pt}right{\isacharunderscore}{\kern0pt}unit{\isadigit{2}}\ id{\isacharunderscore}{\kern0pt}type{\isacharparenright}{\kern0pt}\isanewline
\ \ \ \ \isacommand{also}\isamarkupfalse%
\ \isacommand{have}\isamarkupfalse%
\ {\isachardoublequoteopen}{\isachardot}{\kern0pt}{\isachardot}{\kern0pt}{\isachardot}{\kern0pt}\ {\isacharequal}{\kern0pt}\ eval{\isacharunderscore}{\kern0pt}func\ Y\ X\ {\isasymcirc}\isactrlsub c\ {\isacharparenleft}{\kern0pt}id\ X\ {\isasymtimes}\isactrlsub f\ g\isactrlsup {\isasymsharp}{\isacharparenright}{\kern0pt}\ {\isasymcirc}\isactrlsub c\ {\isacharparenleft}{\kern0pt}{\isasympsi}\ {\isasymtimes}\isactrlsub f\ id{\isacharparenleft}{\kern0pt}Y\isactrlbsup A\isactrlesup {\isacharparenright}{\kern0pt}{\isacharparenright}{\kern0pt}{\isachardoublequoteclose}\isanewline
\ \ \ \ \ \ \isacommand{by}\isamarkupfalse%
\ {\isacharparenleft}{\kern0pt}smt\ {\isasympsi}{\isacharunderscore}{\kern0pt}def\ cfunc{\isacharunderscore}{\kern0pt}cross{\isacharunderscore}{\kern0pt}prod{\isacharunderscore}{\kern0pt}comp{\isacharunderscore}{\kern0pt}cfunc{\isacharunderscore}{\kern0pt}cross{\isacharunderscore}{\kern0pt}prod\ gsharp{\isacharunderscore}{\kern0pt}type\ id{\isacharunderscore}{\kern0pt}left{\isacharunderscore}{\kern0pt}unit{\isadigit{2}}\ id{\isacharunderscore}{\kern0pt}right{\isacharunderscore}{\kern0pt}unit{\isadigit{2}}\ id{\isacharunderscore}{\kern0pt}type{\isacharparenright}{\kern0pt}\isanewline
\ \ \ \ \isacommand{also}\isamarkupfalse%
\ \isacommand{have}\isamarkupfalse%
\ {\isachardoublequoteopen}{\isachardot}{\kern0pt}{\isachardot}{\kern0pt}{\isachardot}{\kern0pt}\ {\isacharequal}{\kern0pt}\ eval{\isacharunderscore}{\kern0pt}func\ Y\ A\ {\isasymcirc}\isactrlsub c\ {\isacharparenleft}{\kern0pt}{\isasymphi}\ {\isasymtimes}\isactrlsub f\ id{\isacharparenleft}{\kern0pt}Y\isactrlbsup A\isactrlesup {\isacharparenright}{\kern0pt}{\isacharparenright}{\kern0pt}\ {\isasymcirc}\isactrlsub c\ {\isacharparenleft}{\kern0pt}{\isasympsi}\ {\isasymtimes}\isactrlsub f\ id{\isacharparenleft}{\kern0pt}Y\isactrlbsup A\isactrlesup {\isacharparenright}{\kern0pt}{\isacharparenright}{\kern0pt}{\isachardoublequoteclose}\isanewline
\ \ \ \ \ \ \isacommand{by}\isamarkupfalse%
\ {\isacharparenleft}{\kern0pt}typecheck{\isacharunderscore}{\kern0pt}cfuncs{\isacharcomma}{\kern0pt}\ smt\ {\isasymphi}{\isacharunderscore}{\kern0pt}def\ {\isasympsi}{\isacharunderscore}{\kern0pt}def\ comp{\isacharunderscore}{\kern0pt}associative{\isadigit{2}}\ flat{\isacharunderscore}{\kern0pt}cancels{\isacharunderscore}{\kern0pt}sharp\ g{\isacharunderscore}{\kern0pt}def\ g{\isacharunderscore}{\kern0pt}type\ inv{\isacharunderscore}{\kern0pt}transpose{\isacharunderscore}{\kern0pt}func{\isacharunderscore}{\kern0pt}def{\isadigit{3}}{\isacharparenright}{\kern0pt}\isanewline
\ \ \ \ \isacommand{also}\isamarkupfalse%
\ \isacommand{have}\isamarkupfalse%
\ {\isachardoublequoteopen}{\isachardot}{\kern0pt}{\isachardot}{\kern0pt}{\isachardot}{\kern0pt}\ {\isacharequal}{\kern0pt}\ eval{\isacharunderscore}{\kern0pt}func\ Y\ A\ {\isasymcirc}\isactrlsub c\ {\isacharparenleft}{\kern0pt}{\isacharparenleft}{\kern0pt}{\isasymphi}\ {\isasymcirc}\isactrlsub c\ {\isasympsi}{\isacharparenright}{\kern0pt}\ {\isasymtimes}\isactrlsub f\ {\isacharparenleft}{\kern0pt}id{\isacharparenleft}{\kern0pt}Y\isactrlbsup A\isactrlesup {\isacharparenright}{\kern0pt}\ {\isasymcirc}\isactrlsub c\ id{\isacharparenleft}{\kern0pt}Y\isactrlbsup A\isactrlesup {\isacharparenright}{\kern0pt}{\isacharparenright}{\kern0pt}{\isacharparenright}{\kern0pt}{\isachardoublequoteclose}\isanewline
\ \ \ \ \ \ \isacommand{using}\isamarkupfalse%
\ {\isasymphi}{\isacharunderscore}{\kern0pt}def\ {\isasympsi}{\isacharunderscore}{\kern0pt}def\ cfunc{\isacharunderscore}{\kern0pt}cross{\isacharunderscore}{\kern0pt}prod{\isacharunderscore}{\kern0pt}comp{\isacharunderscore}{\kern0pt}cfunc{\isacharunderscore}{\kern0pt}cross{\isacharunderscore}{\kern0pt}prod\ \isacommand{by}\isamarkupfalse%
\ {\isacharparenleft}{\kern0pt}typecheck{\isacharunderscore}{\kern0pt}cfuncs{\isacharcomma}{\kern0pt}\ auto{\isacharparenright}{\kern0pt}\isanewline
\ \ \ \ \isacommand{also}\isamarkupfalse%
\ \isacommand{have}\isamarkupfalse%
\ {\isachardoublequoteopen}{\isachardot}{\kern0pt}{\isachardot}{\kern0pt}{\isachardot}{\kern0pt}\ {\isacharequal}{\kern0pt}\ eval{\isacharunderscore}{\kern0pt}func\ Y\ A\ {\isasymcirc}\isactrlsub c\ id{\isacharparenleft}{\kern0pt}A{\isacharparenright}{\kern0pt}\ {\isasymtimes}\isactrlsub f\ id{\isacharparenleft}{\kern0pt}Y\isactrlbsup A\isactrlesup {\isacharparenright}{\kern0pt}{\isachardoublequoteclose}\isanewline
\ \ \ \ \ \ \isacommand{using}\isamarkupfalse%
\ idA\ id{\isacharunderscore}{\kern0pt}right{\isacharunderscore}{\kern0pt}unit{\isadigit{2}}\ \isacommand{by}\isamarkupfalse%
\ {\isacharparenleft}{\kern0pt}typecheck{\isacharunderscore}{\kern0pt}cfuncs{\isacharcomma}{\kern0pt}\ auto{\isacharparenright}{\kern0pt}\isanewline
\ \ \ \ \isacommand{then}\isamarkupfalse%
\ \isacommand{show}\isamarkupfalse%
\ {\isachardoublequoteopen}eval{\isacharunderscore}{\kern0pt}func\ Y\ A\ {\isasymcirc}\isactrlsub c\ id\isactrlsub c\ A\ {\isasymtimes}\isactrlsub f\ f\isactrlsup {\isasymsharp}\ {\isasymcirc}\isactrlsub c\ g\isactrlsup {\isasymsharp}\ {\isacharequal}{\kern0pt}\ eval{\isacharunderscore}{\kern0pt}func\ Y\ A\ {\isasymcirc}\isactrlsub c\ id\isactrlsub c\ A\ {\isasymtimes}\isactrlsub f\ id\isactrlsub c\ {\isacharparenleft}{\kern0pt}Y\isactrlbsup A\isactrlesup {\isacharparenright}{\kern0pt}{\isachardoublequoteclose}\isanewline
\ \ \ \ \ \ \isacommand{by}\isamarkupfalse%
\ {\isacharparenleft}{\kern0pt}simp\ add{\isacharcolon}{\kern0pt}\ calculation{\isacharparenright}{\kern0pt}\isanewline
\ \ \isacommand{qed}\isamarkupfalse%
\isanewline
\isanewline
\ \ \isacommand{have}\isamarkupfalse%
\ gsharp{\isacharunderscore}{\kern0pt}fsharp{\isacharunderscore}{\kern0pt}id{\isacharcolon}{\kern0pt}\ {\isachardoublequoteopen}g\isactrlsup {\isasymsharp}\ {\isasymcirc}\isactrlsub c\ f\isactrlsup {\isasymsharp}\ {\isacharequal}{\kern0pt}\ id{\isacharparenleft}{\kern0pt}Y\isactrlbsup X\isactrlesup {\isacharparenright}{\kern0pt}{\isachardoublequoteclose}\isanewline
\ \ \isacommand{proof}\isamarkupfalse%
{\isacharparenleft}{\kern0pt}etcs{\isacharunderscore}{\kern0pt}rule\ same{\isacharunderscore}{\kern0pt}evals{\isacharunderscore}{\kern0pt}equal{\isacharbrackleft}{\kern0pt}\isakeyword{where}\ X\ {\isacharequal}{\kern0pt}\ Y{\isacharcomma}{\kern0pt}\ \isakeyword{where}\ A\ {\isacharequal}{\kern0pt}\ X{\isacharbrackright}{\kern0pt}{\isacharparenright}{\kern0pt}\isanewline
\ \ \ \ \isacommand{have}\isamarkupfalse%
\ {\isachardoublequoteopen}eval{\isacharunderscore}{\kern0pt}func\ Y\ X\ {\isasymcirc}\isactrlsub c\ id\isactrlsub c\ X\ {\isasymtimes}\isactrlsub f\ g\isactrlsup {\isasymsharp}\ {\isasymcirc}\isactrlsub c\ f\isactrlsup {\isasymsharp}\ {\isacharequal}{\kern0pt}\ eval{\isacharunderscore}{\kern0pt}func\ Y\ X\ {\isasymcirc}\isactrlsub c\ {\isacharparenleft}{\kern0pt}id\isactrlsub c\ X\ {\isasymtimes}\isactrlsub f\ g\isactrlsup {\isasymsharp}{\isacharparenright}{\kern0pt}\ {\isasymcirc}\isactrlsub c\ {\isacharparenleft}{\kern0pt}id\isactrlsub c\ X\ {\isasymtimes}\isactrlsub f\ f\isactrlsup {\isasymsharp}{\isacharparenright}{\kern0pt}{\isachardoublequoteclose}\isanewline
\ \ \ \ \ \ \isacommand{using}\isamarkupfalse%
\ fsharp{\isacharunderscore}{\kern0pt}type\ gsharp{\isacharunderscore}{\kern0pt}type\ identity{\isacharunderscore}{\kern0pt}distributes{\isacharunderscore}{\kern0pt}across{\isacharunderscore}{\kern0pt}composition\ \isacommand{by}\isamarkupfalse%
\ auto\isanewline
\ \ \ \ \isacommand{also}\isamarkupfalse%
\ \isacommand{have}\isamarkupfalse%
\ {\isachardoublequoteopen}{\isachardot}{\kern0pt}{\isachardot}{\kern0pt}{\isachardot}{\kern0pt}\ {\isacharequal}{\kern0pt}\ eval{\isacharunderscore}{\kern0pt}func\ Y\ A\ {\isasymcirc}\isactrlsub c\ {\isacharparenleft}{\kern0pt}{\isasymphi}\ {\isasymtimes}\isactrlsub f\ id\isactrlsub c\ {\isacharparenleft}{\kern0pt}Y\isactrlbsup A\isactrlesup {\isacharparenright}{\kern0pt}{\isacharparenright}{\kern0pt}\ {\isasymcirc}\isactrlsub c\ {\isacharparenleft}{\kern0pt}id\isactrlsub c\ X\ {\isasymtimes}\isactrlsub f\ f\isactrlsup {\isasymsharp}{\isacharparenright}{\kern0pt}{\isachardoublequoteclose}\isanewline
\ \ \ \ \ \ \isacommand{using}\isamarkupfalse%
\ {\isasymphi}{\isacharunderscore}{\kern0pt}def\ cfunc{\isacharunderscore}{\kern0pt}type{\isacharunderscore}{\kern0pt}def\ comp{\isacharunderscore}{\kern0pt}associative\ fsharp{\isacharunderscore}{\kern0pt}type\ g{\isacharunderscore}{\kern0pt}def\ g{\isacharunderscore}{\kern0pt}type\ transpose{\isacharunderscore}{\kern0pt}func{\isacharunderscore}{\kern0pt}def\ \isacommand{by}\isamarkupfalse%
\ {\isacharparenleft}{\kern0pt}typecheck{\isacharunderscore}{\kern0pt}cfuncs{\isacharcomma}{\kern0pt}\ smt{\isacharparenright}{\kern0pt}\isanewline
\ \ \ \ \isacommand{also}\isamarkupfalse%
\ \isacommand{have}\isamarkupfalse%
\ {\isachardoublequoteopen}{\isachardot}{\kern0pt}{\isachardot}{\kern0pt}{\isachardot}{\kern0pt}\ {\isacharequal}{\kern0pt}\ eval{\isacharunderscore}{\kern0pt}func\ Y\ A\ {\isasymcirc}\isactrlsub c\ {\isacharparenleft}{\kern0pt}{\isasymphi}\ {\isasymtimes}\isactrlsub f\ f\isactrlsup {\isasymsharp}{\isacharparenright}{\kern0pt}{\isachardoublequoteclose}\isanewline
\ \ \ \ \ \ \isacommand{by}\isamarkupfalse%
\ {\isacharparenleft}{\kern0pt}smt\ {\isasymphi}{\isacharunderscore}{\kern0pt}def\ cfunc{\isacharunderscore}{\kern0pt}cross{\isacharunderscore}{\kern0pt}prod{\isacharunderscore}{\kern0pt}comp{\isacharunderscore}{\kern0pt}cfunc{\isacharunderscore}{\kern0pt}cross{\isacharunderscore}{\kern0pt}prod\ fsharp{\isacharunderscore}{\kern0pt}type\ id{\isacharunderscore}{\kern0pt}left{\isacharunderscore}{\kern0pt}unit{\isadigit{2}}\ id{\isacharunderscore}{\kern0pt}right{\isacharunderscore}{\kern0pt}unit{\isadigit{2}}\ id{\isacharunderscore}{\kern0pt}type{\isacharparenright}{\kern0pt}\isanewline
\ \ \ \ \isacommand{also}\isamarkupfalse%
\ \isacommand{have}\isamarkupfalse%
\ {\isachardoublequoteopen}{\isachardot}{\kern0pt}{\isachardot}{\kern0pt}{\isachardot}{\kern0pt}\ {\isacharequal}{\kern0pt}\ eval{\isacharunderscore}{\kern0pt}func\ Y\ A\ {\isasymcirc}\isactrlsub c\ {\isacharparenleft}{\kern0pt}id{\isacharparenleft}{\kern0pt}A{\isacharparenright}{\kern0pt}\ {\isasymtimes}\isactrlsub f\ f\isactrlsup {\isasymsharp}{\isacharparenright}{\kern0pt}\ {\isasymcirc}\isactrlsub c\ {\isacharparenleft}{\kern0pt}{\isasymphi}\ {\isasymtimes}\isactrlsub f\ id\isactrlsub c\ {\isacharparenleft}{\kern0pt}Y\isactrlbsup X\isactrlesup {\isacharparenright}{\kern0pt}{\isacharparenright}{\kern0pt}{\isachardoublequoteclose}\isanewline
\ \ \ \ \ \ \isacommand{by}\isamarkupfalse%
\ {\isacharparenleft}{\kern0pt}smt\ {\isasymphi}{\isacharunderscore}{\kern0pt}def\ cfunc{\isacharunderscore}{\kern0pt}cross{\isacharunderscore}{\kern0pt}prod{\isacharunderscore}{\kern0pt}comp{\isacharunderscore}{\kern0pt}cfunc{\isacharunderscore}{\kern0pt}cross{\isacharunderscore}{\kern0pt}prod\ fsharp{\isacharunderscore}{\kern0pt}type\ id{\isacharunderscore}{\kern0pt}left{\isacharunderscore}{\kern0pt}unit{\isadigit{2}}\ id{\isacharunderscore}{\kern0pt}right{\isacharunderscore}{\kern0pt}unit{\isadigit{2}}\ id{\isacharunderscore}{\kern0pt}type{\isacharparenright}{\kern0pt}\isanewline
\ \ \ \ \isacommand{also}\isamarkupfalse%
\ \isacommand{have}\isamarkupfalse%
\ {\isachardoublequoteopen}{\isachardot}{\kern0pt}{\isachardot}{\kern0pt}{\isachardot}{\kern0pt}\ {\isacharequal}{\kern0pt}\ eval{\isacharunderscore}{\kern0pt}func\ Y\ X\ {\isasymcirc}\isactrlsub c\ {\isacharparenleft}{\kern0pt}{\isasympsi}\ {\isasymtimes}\isactrlsub f\ id\isactrlsub c\ {\isacharparenleft}{\kern0pt}Y\isactrlbsup X\isactrlesup {\isacharparenright}{\kern0pt}{\isacharparenright}{\kern0pt}\ {\isasymcirc}\isactrlsub c\ {\isacharparenleft}{\kern0pt}{\isasymphi}\ {\isasymtimes}\isactrlsub f\ id\isactrlsub c\ {\isacharparenleft}{\kern0pt}Y\isactrlbsup X\isactrlesup {\isacharparenright}{\kern0pt}{\isacharparenright}{\kern0pt}{\isachardoublequoteclose}\isanewline
\ \ \ \ \ \ \isacommand{by}\isamarkupfalse%
\ {\isacharparenleft}{\kern0pt}typecheck{\isacharunderscore}{\kern0pt}cfuncs{\isacharcomma}{\kern0pt}\ smt\ {\isasymphi}{\isacharunderscore}{\kern0pt}def\ {\isasympsi}{\isacharunderscore}{\kern0pt}def\ comp{\isacharunderscore}{\kern0pt}associative{\isadigit{2}}\ f{\isacharunderscore}{\kern0pt}def\ f{\isacharunderscore}{\kern0pt}type\ flat{\isacharunderscore}{\kern0pt}cancels{\isacharunderscore}{\kern0pt}sharp\ inv{\isacharunderscore}{\kern0pt}transpose{\isacharunderscore}{\kern0pt}func{\isacharunderscore}{\kern0pt}def{\isadigit{3}}{\isacharparenright}{\kern0pt}\isanewline
\ \ \ \ \isacommand{also}\isamarkupfalse%
\ \isacommand{have}\isamarkupfalse%
\ {\isachardoublequoteopen}{\isachardot}{\kern0pt}{\isachardot}{\kern0pt}{\isachardot}{\kern0pt}\ {\isacharequal}{\kern0pt}\ eval{\isacharunderscore}{\kern0pt}func\ Y\ X\ {\isasymcirc}\isactrlsub c\ {\isacharparenleft}{\kern0pt}{\isacharparenleft}{\kern0pt}{\isasympsi}\ {\isasymcirc}\isactrlsub c\ {\isasymphi}{\isacharparenright}{\kern0pt}\ {\isasymtimes}\isactrlsub f\ {\isacharparenleft}{\kern0pt}id{\isacharparenleft}{\kern0pt}Y\isactrlbsup X\isactrlesup {\isacharparenright}{\kern0pt}\ {\isasymcirc}\isactrlsub c\ id{\isacharparenleft}{\kern0pt}Y\isactrlbsup X\isactrlesup {\isacharparenright}{\kern0pt}{\isacharparenright}{\kern0pt}{\isacharparenright}{\kern0pt}{\isachardoublequoteclose}\isanewline
\ \ \ \ \ \ \isacommand{using}\isamarkupfalse%
\ {\isasymphi}{\isacharunderscore}{\kern0pt}def\ {\isasympsi}{\isacharunderscore}{\kern0pt}def\ cfunc{\isacharunderscore}{\kern0pt}cross{\isacharunderscore}{\kern0pt}prod{\isacharunderscore}{\kern0pt}comp{\isacharunderscore}{\kern0pt}cfunc{\isacharunderscore}{\kern0pt}cross{\isacharunderscore}{\kern0pt}prod\ \isacommand{by}\isamarkupfalse%
\ {\isacharparenleft}{\kern0pt}typecheck{\isacharunderscore}{\kern0pt}cfuncs{\isacharcomma}{\kern0pt}\ auto{\isacharparenright}{\kern0pt}\isanewline
\ \ \ \ \isacommand{also}\isamarkupfalse%
\ \isacommand{have}\isamarkupfalse%
\ {\isachardoublequoteopen}{\isachardot}{\kern0pt}{\isachardot}{\kern0pt}{\isachardot}{\kern0pt}\ {\isacharequal}{\kern0pt}\ eval{\isacharunderscore}{\kern0pt}func\ Y\ X\ {\isasymcirc}\isactrlsub c\ id{\isacharparenleft}{\kern0pt}X{\isacharparenright}{\kern0pt}\ {\isasymtimes}\isactrlsub f\ id{\isacharparenleft}{\kern0pt}Y\isactrlbsup X\isactrlesup {\isacharparenright}{\kern0pt}{\isachardoublequoteclose}\isanewline
\ \ \ \ \ \ \isacommand{using}\isamarkupfalse%
\ {\isasympsi}{\isacharunderscore}{\kern0pt}def\ id{\isacharunderscore}{\kern0pt}left{\isacharunderscore}{\kern0pt}unit{\isadigit{2}}\ \isacommand{by}\isamarkupfalse%
\ {\isacharparenleft}{\kern0pt}typecheck{\isacharunderscore}{\kern0pt}cfuncs{\isacharcomma}{\kern0pt}\ auto{\isacharparenright}{\kern0pt}\isanewline
\ \ \ \ \isacommand{then}\isamarkupfalse%
\ \isacommand{show}\isamarkupfalse%
\ {\isachardoublequoteopen}eval{\isacharunderscore}{\kern0pt}func\ Y\ X\ {\isasymcirc}\isactrlsub c\ id\isactrlsub c\ X\ {\isasymtimes}\isactrlsub f\ g\isactrlsup {\isasymsharp}\ {\isasymcirc}\isactrlsub c\ f\isactrlsup {\isasymsharp}\ {\isacharequal}{\kern0pt}\ eval{\isacharunderscore}{\kern0pt}func\ Y\ X\ {\isasymcirc}\isactrlsub c\ id\isactrlsub c\ X\ {\isasymtimes}\isactrlsub f\ id\isactrlsub c\ {\isacharparenleft}{\kern0pt}Y\isactrlbsup X\isactrlesup {\isacharparenright}{\kern0pt}{\isachardoublequoteclose}\isanewline
\ \ \ \ \ \ \isacommand{by}\isamarkupfalse%
\ {\isacharparenleft}{\kern0pt}simp\ add{\isacharcolon}{\kern0pt}\ calculation{\isacharparenright}{\kern0pt}\isanewline
\ \ \isacommand{qed}\isamarkupfalse%
\isanewline
\ \ \isacommand{show}\isamarkupfalse%
\ {\isacharquery}{\kern0pt}thesis\isanewline
\ \ \ \ \isacommand{by}\isamarkupfalse%
\ {\isacharparenleft}{\kern0pt}metis\ cfunc{\isacharunderscore}{\kern0pt}type{\isacharunderscore}{\kern0pt}def\ comp{\isacharunderscore}{\kern0pt}epi{\isacharunderscore}{\kern0pt}imp{\isacharunderscore}{\kern0pt}epi\ comp{\isacharunderscore}{\kern0pt}monic{\isacharunderscore}{\kern0pt}imp{\isacharunderscore}{\kern0pt}monic\ epi{\isacharunderscore}{\kern0pt}mon{\isacharunderscore}{\kern0pt}is{\isacharunderscore}{\kern0pt}iso\ fsharp{\isacharunderscore}{\kern0pt}gsharp{\isacharunderscore}{\kern0pt}id\ fsharp{\isacharunderscore}{\kern0pt}type\ gsharp{\isacharunderscore}{\kern0pt}fsharp{\isacharunderscore}{\kern0pt}id\ gsharp{\isacharunderscore}{\kern0pt}type\ id{\isacharunderscore}{\kern0pt}isomorphism\ is{\isacharunderscore}{\kern0pt}isomorphic{\isacharunderscore}{\kern0pt}def\ iso{\isacharunderscore}{\kern0pt}imp{\isacharunderscore}{\kern0pt}epi{\isacharunderscore}{\kern0pt}and{\isacharunderscore}{\kern0pt}monic{\isacharparenright}{\kern0pt}\isanewline
\isacommand{qed}\isamarkupfalse%
%
\endisatagproof
{\isafoldproof}%
%
\isadelimproof
\isanewline
%
\endisadelimproof
\isanewline
\isacommand{lemma}\isamarkupfalse%
\ exp{\isacharunderscore}{\kern0pt}pres{\isacharunderscore}{\kern0pt}iso{\isacharcolon}{\kern0pt}\isanewline
\ \ \isakeyword{assumes}\ {\isachardoublequoteopen}A\ {\isasymcong}\ X{\isachardoublequoteclose}\ {\isachardoublequoteopen}B\ {\isasymcong}\ Y{\isachardoublequoteclose}\ \isanewline
\ \ \isakeyword{shows}\ {\isachardoublequoteopen}A\isactrlbsup B\isactrlesup \ {\isasymcong}\ \ X\isactrlbsup Y\isactrlesup {\isachardoublequoteclose}\isanewline
%
\isadelimproof
\ \ %
\endisadelimproof
%
\isatagproof
\isacommand{by}\isamarkupfalse%
\ {\isacharparenleft}{\kern0pt}meson\ assms\ exp{\isacharunderscore}{\kern0pt}pres{\isacharunderscore}{\kern0pt}iso{\isacharunderscore}{\kern0pt}left\ exp{\isacharunderscore}{\kern0pt}pres{\isacharunderscore}{\kern0pt}iso{\isacharunderscore}{\kern0pt}right\ isomorphic{\isacharunderscore}{\kern0pt}is{\isacharunderscore}{\kern0pt}transitive{\isacharparenright}{\kern0pt}%
\endisatagproof
{\isafoldproof}%
%
\isadelimproof
\isanewline
%
\endisadelimproof
\isanewline
\isacommand{lemma}\isamarkupfalse%
\ empty{\isacharunderscore}{\kern0pt}to{\isacharunderscore}{\kern0pt}nonempty{\isacharcolon}{\kern0pt}\isanewline
\ \ \isakeyword{assumes}\ {\isachardoublequoteopen}nonempty\ X{\isachardoublequoteclose}\ {\isachardoublequoteopen}is{\isacharunderscore}{\kern0pt}empty\ Y{\isachardoublequoteclose}\ \isanewline
\ \ \isakeyword{shows}\ {\isachardoublequoteopen}Y\isactrlbsup X\isactrlesup \ {\isasymcong}\ {\isasymemptyset}{\isachardoublequoteclose}\isanewline
%
\isadelimproof
\ \ %
\endisadelimproof
%
\isatagproof
\isacommand{by}\isamarkupfalse%
\ {\isacharparenleft}{\kern0pt}meson\ assms\ exp{\isacharunderscore}{\kern0pt}pres{\isacharunderscore}{\kern0pt}iso{\isacharunderscore}{\kern0pt}left\ isomorphic{\isacharunderscore}{\kern0pt}is{\isacharunderscore}{\kern0pt}transitive\ no{\isacharunderscore}{\kern0pt}el{\isacharunderscore}{\kern0pt}iff{\isacharunderscore}{\kern0pt}iso{\isacharunderscore}{\kern0pt}empty\ empty{\isacharunderscore}{\kern0pt}exp{\isacharunderscore}{\kern0pt}nonempty{\isacharparenright}{\kern0pt}%
\endisatagproof
{\isafoldproof}%
%
\isadelimproof
\isanewline
%
\endisadelimproof
\isanewline
\isacommand{lemma}\isamarkupfalse%
\ exp{\isacharunderscore}{\kern0pt}is{\isacharunderscore}{\kern0pt}empty{\isacharcolon}{\kern0pt}\isanewline
\ \ \isakeyword{assumes}\ {\isachardoublequoteopen}is{\isacharunderscore}{\kern0pt}empty\ X{\isachardoublequoteclose}\ \isanewline
\ \ \isakeyword{shows}\ {\isachardoublequoteopen}Y\isactrlbsup X\isactrlesup \ {\isasymcong}\ {\isasymone}{\isachardoublequoteclose}\isanewline
%
\isadelimproof
\ \ %
\endisadelimproof
%
\isatagproof
\isacommand{using}\isamarkupfalse%
\ assms\ exp{\isacharunderscore}{\kern0pt}pres{\isacharunderscore}{\kern0pt}iso{\isacharunderscore}{\kern0pt}right\ isomorphic{\isacharunderscore}{\kern0pt}is{\isacharunderscore}{\kern0pt}transitive\ no{\isacharunderscore}{\kern0pt}el{\isacharunderscore}{\kern0pt}iff{\isacharunderscore}{\kern0pt}iso{\isacharunderscore}{\kern0pt}empty\ exp{\isacharunderscore}{\kern0pt}empty\ \isacommand{by}\isamarkupfalse%
\ blast%
\endisatagproof
{\isafoldproof}%
%
\isadelimproof
\isanewline
%
\endisadelimproof
\isanewline
\isacommand{lemma}\isamarkupfalse%
\ nonempty{\isacharunderscore}{\kern0pt}to{\isacharunderscore}{\kern0pt}nonempty{\isacharcolon}{\kern0pt}\isanewline
\ \ \isakeyword{assumes}\ {\isachardoublequoteopen}nonempty\ X{\isachardoublequoteclose}\ {\isachardoublequoteopen}nonempty\ Y{\isachardoublequoteclose}\isanewline
\ \ \isakeyword{shows}\ {\isachardoublequoteopen}nonempty{\isacharparenleft}{\kern0pt}Y\isactrlbsup X\isactrlesup {\isacharparenright}{\kern0pt}{\isachardoublequoteclose}\isanewline
%
\isadelimproof
\ \ %
\endisadelimproof
%
\isatagproof
\isacommand{by}\isamarkupfalse%
\ {\isacharparenleft}{\kern0pt}meson\ assms{\isacharparenleft}{\kern0pt}{\isadigit{2}}{\isacharparenright}{\kern0pt}\ comp{\isacharunderscore}{\kern0pt}type\ nonempty{\isacharunderscore}{\kern0pt}def\ terminal{\isacharunderscore}{\kern0pt}func{\isacharunderscore}{\kern0pt}type\ transpose{\isacharunderscore}{\kern0pt}func{\isacharunderscore}{\kern0pt}type{\isacharparenright}{\kern0pt}%
\endisatagproof
{\isafoldproof}%
%
\isadelimproof
\isanewline
%
\endisadelimproof
\isanewline
\isacommand{lemma}\isamarkupfalse%
\ empty{\isacharunderscore}{\kern0pt}to{\isacharunderscore}{\kern0pt}nonempty{\isacharunderscore}{\kern0pt}converse{\isacharcolon}{\kern0pt}\isanewline
\ \ \isakeyword{assumes}\ {\isachardoublequoteopen}Y\isactrlbsup X\isactrlesup \ {\isasymcong}\ {\isasymemptyset}{\isachardoublequoteclose}\isanewline
\ \ \isakeyword{shows}\ {\isachardoublequoteopen}is{\isacharunderscore}{\kern0pt}empty\ Y\ {\isasymand}\ nonempty\ X{\isachardoublequoteclose}\isanewline
%
\isadelimproof
\ \ %
\endisadelimproof
%
\isatagproof
\isacommand{by}\isamarkupfalse%
\ {\isacharparenleft}{\kern0pt}metis\ is{\isacharunderscore}{\kern0pt}empty{\isacharunderscore}{\kern0pt}def\ exp{\isacharunderscore}{\kern0pt}is{\isacharunderscore}{\kern0pt}empty\ assms\ no{\isacharunderscore}{\kern0pt}el{\isacharunderscore}{\kern0pt}iff{\isacharunderscore}{\kern0pt}iso{\isacharunderscore}{\kern0pt}empty\ nonempty{\isacharunderscore}{\kern0pt}def\ nonempty{\isacharunderscore}{\kern0pt}to{\isacharunderscore}{\kern0pt}nonempty\ single{\isacharunderscore}{\kern0pt}elem{\isacharunderscore}{\kern0pt}iso{\isacharunderscore}{\kern0pt}one{\isacharparenright}{\kern0pt}%
\endisatagproof
{\isafoldproof}%
%
\isadelimproof
%
\endisadelimproof
%
\begin{isamarkuptext}%
The definition below corresponds to Definition 2.5.11 in Halvorson.%
\end{isamarkuptext}\isamarkuptrue%
\isacommand{definition}\isamarkupfalse%
\ powerset\ {\isacharcolon}{\kern0pt}{\isacharcolon}{\kern0pt}\ {\isachardoublequoteopen}cset\ {\isasymRightarrow}\ cset{\isachardoublequoteclose}\ {\isacharparenleft}{\kern0pt}{\isachardoublequoteopen}{\isasymP}{\isacharunderscore}{\kern0pt}{\isachardoublequoteclose}\ {\isacharbrackleft}{\kern0pt}{\isadigit{1}}{\isadigit{0}}{\isadigit{1}}{\isacharbrackright}{\kern0pt}{\isadigit{1}}{\isadigit{0}}{\isadigit{0}}{\isacharparenright}{\kern0pt}\ \isakeyword{where}\isanewline
\ \ {\isachardoublequoteopen}{\isasymP}\ X\ {\isacharequal}{\kern0pt}\ {\isasymOmega}\isactrlbsup X\isactrlesup {\isachardoublequoteclose}\isanewline
\isanewline
\isacommand{lemma}\isamarkupfalse%
\ sets{\isacharunderscore}{\kern0pt}squared{\isacharcolon}{\kern0pt}\isanewline
\ \ {\isachardoublequoteopen}A\isactrlbsup {\isasymOmega}\isactrlesup \ {\isasymcong}\ A\ {\isasymtimes}\isactrlsub c\ A{\isachardoublequoteclose}\isanewline
%
\isadelimproof
%
\endisadelimproof
%
\isatagproof
\isacommand{proof}\isamarkupfalse%
\ {\isacharminus}{\kern0pt}\ \isanewline
\ \ \isacommand{obtain}\isamarkupfalse%
\ {\isasymphi}\ \isakeyword{where}\ {\isasymphi}{\isacharunderscore}{\kern0pt}def{\isacharcolon}{\kern0pt}\ {\isachardoublequoteopen}{\isasymphi}\ {\isacharequal}{\kern0pt}\ {\isasymlangle}eval{\isacharunderscore}{\kern0pt}func\ A\ {\isasymOmega}\ {\isasymcirc}\isactrlsub c\ {\isasymlangle}{\isasymt}\ {\isasymcirc}\isactrlsub c\ {\isasymbeta}\isactrlbsub A\isactrlbsup {\isasymOmega}\isactrlesup \isactrlesub {\isacharcomma}{\kern0pt}\ id{\isacharparenleft}{\kern0pt}A\isactrlbsup {\isasymOmega}\isactrlesup {\isacharparenright}{\kern0pt}{\isasymrangle}{\isacharcomma}{\kern0pt}\isanewline
\ \ \ \ \ \ \ \ \ \ \ \ \ \ \ \ \ \ \ \ \ \ \ \ \ \ \ \ \ \ eval{\isacharunderscore}{\kern0pt}func\ A\ {\isasymOmega}\ {\isasymcirc}\isactrlsub c\ {\isasymlangle}{\isasymf}\ {\isasymcirc}\isactrlsub c\ {\isasymbeta}\isactrlbsub A\isactrlbsup {\isasymOmega}\isactrlesup \isactrlesub {\isacharcomma}{\kern0pt}\ id{\isacharparenleft}{\kern0pt}A\isactrlbsup {\isasymOmega}\isactrlesup {\isacharparenright}{\kern0pt}{\isasymrangle}{\isasymrangle}{\isachardoublequoteclose}\ \isakeyword{and}\isanewline
\ \ \ \ \ \ \ \ \ \ \ \ \ \ \ \ \ {\isasymphi}{\isacharunderscore}{\kern0pt}type{\isacharbrackleft}{\kern0pt}type{\isacharunderscore}{\kern0pt}rule{\isacharbrackright}{\kern0pt}{\isacharcolon}{\kern0pt}\ {\isachardoublequoteopen}{\isasymphi}\ {\isacharcolon}{\kern0pt}\ A\isactrlbsup {\isasymOmega}\isactrlesup \ {\isasymrightarrow}\ A\ {\isasymtimes}\isactrlsub c\ A{\isachardoublequoteclose}\isanewline
\ \ \ \ \ \ \ \ \ \ \ \ \ \ \ \ \ \ \isacommand{by}\isamarkupfalse%
\ {\isacharparenleft}{\kern0pt}typecheck{\isacharunderscore}{\kern0pt}cfuncs{\isacharcomma}{\kern0pt}\ simp{\isacharparenright}{\kern0pt}\isanewline
\ \ \isacommand{have}\isamarkupfalse%
\ {\isachardoublequoteopen}injective\ {\isasymphi}{\isachardoublequoteclose}\isanewline
\ \ \isacommand{proof}\isamarkupfalse%
{\isacharparenleft}{\kern0pt}unfold\ injective{\isacharunderscore}{\kern0pt}def{\isacharcomma}{\kern0pt}\ clarify{\isacharparenright}{\kern0pt}\isanewline
\ \ \ \ \isacommand{fix}\isamarkupfalse%
\ f\ g\ \isanewline
\ \ \ \ \isacommand{assume}\isamarkupfalse%
\ {\isachardoublequoteopen}f\ {\isasymin}\isactrlsub c\ domain\ {\isasymphi}{\isachardoublequoteclose}\ \isacommand{then}\isamarkupfalse%
\ \isacommand{have}\isamarkupfalse%
\ f{\isacharunderscore}{\kern0pt}type{\isacharbrackleft}{\kern0pt}type{\isacharunderscore}{\kern0pt}rule{\isacharbrackright}{\kern0pt}{\isacharcolon}{\kern0pt}\ {\isachardoublequoteopen}f\ {\isasymin}\isactrlsub c\ A\isactrlbsup {\isasymOmega}\isactrlesup {\isachardoublequoteclose}\ \isanewline
\ \ \ \ \ \ \isacommand{using}\isamarkupfalse%
\ {\isasymphi}{\isacharunderscore}{\kern0pt}type\ cfunc{\isacharunderscore}{\kern0pt}type{\isacharunderscore}{\kern0pt}def\ \isacommand{by}\isamarkupfalse%
\ {\isacharparenleft}{\kern0pt}typecheck{\isacharunderscore}{\kern0pt}cfuncs{\isacharcomma}{\kern0pt}\ auto{\isacharparenright}{\kern0pt}\isanewline
\ \ \ \ \isacommand{assume}\isamarkupfalse%
\ {\isachardoublequoteopen}g\ {\isasymin}\isactrlsub c\ domain\ {\isasymphi}{\isachardoublequoteclose}\ \isacommand{then}\isamarkupfalse%
\ \isacommand{have}\isamarkupfalse%
\ g{\isacharunderscore}{\kern0pt}type{\isacharbrackleft}{\kern0pt}type{\isacharunderscore}{\kern0pt}rule{\isacharbrackright}{\kern0pt}{\isacharcolon}{\kern0pt}\ {\isachardoublequoteopen}g\ {\isasymin}\isactrlsub c\ A\isactrlbsup {\isasymOmega}\isactrlesup {\isachardoublequoteclose}\ \isanewline
\ \ \ \ \ \ \isacommand{using}\isamarkupfalse%
\ {\isasymphi}{\isacharunderscore}{\kern0pt}type\ cfunc{\isacharunderscore}{\kern0pt}type{\isacharunderscore}{\kern0pt}def\ \isacommand{by}\isamarkupfalse%
\ {\isacharparenleft}{\kern0pt}typecheck{\isacharunderscore}{\kern0pt}cfuncs{\isacharcomma}{\kern0pt}\ auto{\isacharparenright}{\kern0pt}\isanewline
\ \ \ \ \isacommand{assume}\isamarkupfalse%
\ eqs{\isacharcolon}{\kern0pt}\ {\isachardoublequoteopen}{\isasymphi}\ {\isasymcirc}\isactrlsub c\ f\ {\isacharequal}{\kern0pt}\ {\isasymphi}\ {\isasymcirc}\isactrlsub c\ g{\isachardoublequoteclose}\isanewline
\ \ \ \ \isacommand{show}\isamarkupfalse%
\ {\isachardoublequoteopen}f\ {\isacharequal}{\kern0pt}\ g{\isachardoublequoteclose}\isanewline
\ \ \ \ \isacommand{proof}\isamarkupfalse%
{\isacharparenleft}{\kern0pt}etcs{\isacharunderscore}{\kern0pt}rule\ one{\isacharunderscore}{\kern0pt}separator{\isacharparenright}{\kern0pt}\isanewline
\ \ \ \ \ \ \isacommand{show}\isamarkupfalse%
\ {\isachardoublequoteopen}{\isasymAnd}id{\isacharunderscore}{\kern0pt}{\isadigit{1}}{\isachardot}{\kern0pt}\ id{\isacharunderscore}{\kern0pt}{\isadigit{1}}\ {\isasymin}\isactrlsub c\ {\isasymone}\ {\isasymLongrightarrow}\ f\ {\isasymcirc}\isactrlsub c\ id{\isacharunderscore}{\kern0pt}{\isadigit{1}}\ {\isacharequal}{\kern0pt}\ g\ {\isasymcirc}\isactrlsub c\ id{\isacharunderscore}{\kern0pt}{\isadigit{1}}{\isachardoublequoteclose}\isanewline
\ \ \ \ \ \ \isacommand{proof}\isamarkupfalse%
{\isacharparenleft}{\kern0pt}etcs{\isacharunderscore}{\kern0pt}rule\ same{\isacharunderscore}{\kern0pt}evals{\isacharunderscore}{\kern0pt}equal{\isacharbrackleft}{\kern0pt}\isakeyword{where}\ X\ {\isacharequal}{\kern0pt}\ A{\isacharcomma}{\kern0pt}\ \isakeyword{where}\ A\ {\isacharequal}{\kern0pt}\ {\isasymOmega}{\isacharbrackright}{\kern0pt}{\isacharparenright}{\kern0pt}\isanewline
\ \ \ \ \ \ \ \ \isacommand{fix}\isamarkupfalse%
\ id{\isacharunderscore}{\kern0pt}{\isadigit{1}}\isanewline
\ \ \ \ \ \ \ \ \isacommand{assume}\isamarkupfalse%
\ id{\isadigit{1}}{\isacharunderscore}{\kern0pt}is{\isacharcolon}{\kern0pt}\ {\isachardoublequoteopen}id{\isacharunderscore}{\kern0pt}{\isadigit{1}}\ {\isasymin}\isactrlsub c\ {\isasymone}{\isachardoublequoteclose}\isanewline
\ \ \ \ \ \ \ \ \isacommand{then}\isamarkupfalse%
\ \isacommand{have}\isamarkupfalse%
\ id{\isadigit{1}}{\isacharunderscore}{\kern0pt}eq{\isacharcolon}{\kern0pt}\ {\isachardoublequoteopen}id{\isacharunderscore}{\kern0pt}{\isadigit{1}}\ {\isacharequal}{\kern0pt}\ id{\isacharparenleft}{\kern0pt}{\isasymone}{\isacharparenright}{\kern0pt}{\isachardoublequoteclose}\isanewline
\ \ \ \ \ \ \ \ \ \ \isacommand{using}\isamarkupfalse%
\ id{\isacharunderscore}{\kern0pt}type\ one{\isacharunderscore}{\kern0pt}unique{\isacharunderscore}{\kern0pt}element\ \isacommand{by}\isamarkupfalse%
\ auto\isanewline
\isanewline
\ \ \ \ \ \ \ \ \isacommand{obtain}\isamarkupfalse%
\ a{\isadigit{1}}\ a{\isadigit{2}}\ \isakeyword{where}\ phi{\isacharunderscore}{\kern0pt}f{\isacharunderscore}{\kern0pt}def{\isacharcolon}{\kern0pt}\ {\isachardoublequoteopen}{\isasymphi}\ {\isasymcirc}\isactrlsub c\ f\ {\isacharequal}{\kern0pt}\ {\isasymlangle}a{\isadigit{1}}{\isacharcomma}{\kern0pt}a{\isadigit{2}}{\isasymrangle}\ {\isasymand}\ a{\isadigit{1}}\ {\isasymin}\isactrlsub c\ A\ {\isasymand}\ a{\isadigit{2}}\ {\isasymin}\isactrlsub c\ A{\isachardoublequoteclose}\isanewline
\ \ \ \ \ \ \ \ \ \ \isacommand{using}\isamarkupfalse%
\ {\isasymphi}{\isacharunderscore}{\kern0pt}type\ cart{\isacharunderscore}{\kern0pt}prod{\isacharunderscore}{\kern0pt}decomp\ comp{\isacharunderscore}{\kern0pt}type\ f{\isacharunderscore}{\kern0pt}type\ \isacommand{by}\isamarkupfalse%
\ blast\isanewline
\ \ \ \ \ \ \ \ \isacommand{have}\isamarkupfalse%
\ equation{\isadigit{1}}{\isacharcolon}{\kern0pt}\ {\isachardoublequoteopen}{\isasymlangle}a{\isadigit{1}}{\isacharcomma}{\kern0pt}a{\isadigit{2}}{\isasymrangle}\ {\isacharequal}{\kern0pt}\ \ {\isasymlangle}eval{\isacharunderscore}{\kern0pt}func\ A\ {\isasymOmega}\ {\isasymcirc}\isactrlsub c\ {\isasymlangle}{\isasymt}{\isacharcomma}{\kern0pt}\ f{\isasymrangle}{\isacharcomma}{\kern0pt}\isanewline
\ \ \ \ \ \ \ \ \ \ \ \ \ \ \ \ \ \ \ \ \ \ \ \ \ \ \ \ eval{\isacharunderscore}{\kern0pt}func\ A\ {\isasymOmega}\ {\isasymcirc}\isactrlsub c\ {\isasymlangle}{\isasymf}{\isacharcomma}{\kern0pt}\ f{\isasymrangle}{\isasymrangle}{\isachardoublequoteclose}\isanewline
\ \ \ \ \ \ \ \ \isacommand{proof}\isamarkupfalse%
\ {\isacharminus}{\kern0pt}\ \isanewline
\ \ \ \ \ \ \ \ \ \ \isacommand{have}\isamarkupfalse%
\ {\isachardoublequoteopen}{\isasymlangle}a{\isadigit{1}}{\isacharcomma}{\kern0pt}a{\isadigit{2}}{\isasymrangle}\ {\isacharequal}{\kern0pt}\ {\isasymlangle}eval{\isacharunderscore}{\kern0pt}func\ A\ {\isasymOmega}\ {\isasymcirc}\isactrlsub c\ {\isasymlangle}{\isasymt}\ {\isasymcirc}\isactrlsub c\ {\isasymbeta}\isactrlbsub A\isactrlbsup {\isasymOmega}\isactrlesup \isactrlesub {\isacharcomma}{\kern0pt}\ id{\isacharparenleft}{\kern0pt}A\isactrlbsup {\isasymOmega}\isactrlesup {\isacharparenright}{\kern0pt}{\isasymrangle}{\isacharcomma}{\kern0pt}\isanewline
\ \ \ \ \ \ \ \ \ \ \ \ \ \ \ \ \ \ \ \ \ \ \ \ \ \ \ \ \ \ eval{\isacharunderscore}{\kern0pt}func\ A\ {\isasymOmega}\ {\isasymcirc}\isactrlsub c\ {\isasymlangle}{\isasymf}\ {\isasymcirc}\isactrlsub c\ {\isasymbeta}\isactrlbsub A\isactrlbsup {\isasymOmega}\isactrlesup \isactrlesub {\isacharcomma}{\kern0pt}\ id{\isacharparenleft}{\kern0pt}A\isactrlbsup {\isasymOmega}\isactrlesup {\isacharparenright}{\kern0pt}{\isasymrangle}{\isasymrangle}\ {\isasymcirc}\isactrlsub c\ f{\isachardoublequoteclose}\isanewline
\ \ \ \ \ \ \ \ \ \ \ \ \isacommand{using}\isamarkupfalse%
\ {\isasymphi}{\isacharunderscore}{\kern0pt}def\ phi{\isacharunderscore}{\kern0pt}f{\isacharunderscore}{\kern0pt}def\ \isacommand{by}\isamarkupfalse%
\ auto\isanewline
\ \ \ \ \ \ \ \ \ \ \isacommand{also}\isamarkupfalse%
\ \isacommand{have}\isamarkupfalse%
\ {\isachardoublequoteopen}{\isachardot}{\kern0pt}{\isachardot}{\kern0pt}{\isachardot}{\kern0pt}\ {\isacharequal}{\kern0pt}\ {\isasymlangle}eval{\isacharunderscore}{\kern0pt}func\ A\ {\isasymOmega}\ {\isasymcirc}\isactrlsub c\ {\isasymlangle}{\isasymt}\ {\isasymcirc}\isactrlsub c\ {\isasymbeta}\isactrlbsub A\isactrlbsup {\isasymOmega}\isactrlesup \isactrlesub {\isacharcomma}{\kern0pt}\ id{\isacharparenleft}{\kern0pt}A\isactrlbsup {\isasymOmega}\isactrlesup {\isacharparenright}{\kern0pt}{\isasymrangle}\ {\isasymcirc}\isactrlsub c\ f{\isacharcomma}{\kern0pt}\isanewline
\ \ \ \ \ \ \ \ \ \ \ \ \ \ \ \ \ \ \ \ \ \ \ \ \ \ \ \ \ \ eval{\isacharunderscore}{\kern0pt}func\ A\ {\isasymOmega}\ {\isasymcirc}\isactrlsub c\ {\isasymlangle}{\isasymf}\ {\isasymcirc}\isactrlsub c\ {\isasymbeta}\isactrlbsub A\isactrlbsup {\isasymOmega}\isactrlesup \isactrlesub {\isacharcomma}{\kern0pt}\ id{\isacharparenleft}{\kern0pt}A\isactrlbsup {\isasymOmega}\isactrlesup {\isacharparenright}{\kern0pt}{\isasymrangle}\ {\isasymcirc}\isactrlsub c\ f{\isasymrangle}{\isachardoublequoteclose}\isanewline
\ \ \ \ \ \ \ \ \ \ \ \ \isacommand{by}\isamarkupfalse%
\ {\isacharparenleft}{\kern0pt}typecheck{\isacharunderscore}{\kern0pt}cfuncs{\isacharcomma}{\kern0pt}smt\ cfunc{\isacharunderscore}{\kern0pt}prod{\isacharunderscore}{\kern0pt}comp\ comp{\isacharunderscore}{\kern0pt}associative{\isadigit{2}}{\isacharparenright}{\kern0pt}\isanewline
\ \ \ \ \ \ \ \ \ \ \isacommand{also}\isamarkupfalse%
\ \isacommand{have}\isamarkupfalse%
\ {\isachardoublequoteopen}{\isachardot}{\kern0pt}{\isachardot}{\kern0pt}{\isachardot}{\kern0pt}\ {\isacharequal}{\kern0pt}\ {\isasymlangle}eval{\isacharunderscore}{\kern0pt}func\ A\ {\isasymOmega}\ {\isasymcirc}\isactrlsub c\ {\isasymlangle}{\isasymt}\ {\isasymcirc}\isactrlsub c\ {\isasymbeta}\isactrlbsub A\isactrlbsup {\isasymOmega}\isactrlesup \isactrlesub \ {\isasymcirc}\isactrlsub c\ f{\isacharcomma}{\kern0pt}\ id{\isacharparenleft}{\kern0pt}A\isactrlbsup {\isasymOmega}\isactrlesup {\isacharparenright}{\kern0pt}\ {\isasymcirc}\isactrlsub c\ f{\isasymrangle}{\isacharcomma}{\kern0pt}\isanewline
\ \ \ \ \ \ \ \ \ \ \ \ \ \ \ \ \ \ \ \ \ \ \ \ \ \ \ \ \ \ eval{\isacharunderscore}{\kern0pt}func\ A\ {\isasymOmega}\ {\isasymcirc}\isactrlsub c\ {\isasymlangle}{\isasymf}\ {\isasymcirc}\isactrlsub c\ {\isasymbeta}\isactrlbsub A\isactrlbsup {\isasymOmega}\isactrlesup \isactrlesub \ {\isasymcirc}\isactrlsub c\ f{\isacharcomma}{\kern0pt}\ id{\isacharparenleft}{\kern0pt}A\isactrlbsup {\isasymOmega}\isactrlesup {\isacharparenright}{\kern0pt}{\isasymcirc}\isactrlsub c\ f{\isasymrangle}{\isasymrangle}{\isachardoublequoteclose}\isanewline
\ \ \ \ \ \ \ \ \ \ \ \ \isacommand{by}\isamarkupfalse%
\ {\isacharparenleft}{\kern0pt}typecheck{\isacharunderscore}{\kern0pt}cfuncs{\isacharcomma}{\kern0pt}\ simp\ add{\isacharcolon}{\kern0pt}\ cfunc{\isacharunderscore}{\kern0pt}prod{\isacharunderscore}{\kern0pt}comp\ comp{\isacharunderscore}{\kern0pt}associative{\isadigit{2}}{\isacharparenright}{\kern0pt}\isanewline
\ \ \ \ \ \ \ \ \ \ \isacommand{also}\isamarkupfalse%
\ \isacommand{have}\isamarkupfalse%
\ {\isachardoublequoteopen}{\isachardot}{\kern0pt}{\isachardot}{\kern0pt}{\isachardot}{\kern0pt}\ {\isacharequal}{\kern0pt}\ {\isasymlangle}eval{\isacharunderscore}{\kern0pt}func\ A\ {\isasymOmega}\ {\isasymcirc}\isactrlsub c\ {\isasymlangle}{\isasymt}{\isacharcomma}{\kern0pt}\ f{\isasymrangle}{\isacharcomma}{\kern0pt}\isanewline
\ \ \ \ \ \ \ \ \ \ \ \ \ \ \ \ \ \ \ \ \ \ \ \ \ \ \ \ \ \ eval{\isacharunderscore}{\kern0pt}func\ A\ {\isasymOmega}\ {\isasymcirc}\isactrlsub c\ {\isasymlangle}{\isasymf}{\isacharcomma}{\kern0pt}\ f{\isasymrangle}{\isasymrangle}{\isachardoublequoteclose}\ \ \ \ \isanewline
\ \ \ \ \ \ \ \ \ \ \ \ \isacommand{by}\isamarkupfalse%
\ {\isacharparenleft}{\kern0pt}typecheck{\isacharunderscore}{\kern0pt}cfuncs{\isacharcomma}{\kern0pt}\ metis\ id{\isadigit{1}}{\isacharunderscore}{\kern0pt}eq\ id{\isadigit{1}}{\isacharunderscore}{\kern0pt}is\ id{\isacharunderscore}{\kern0pt}left{\isacharunderscore}{\kern0pt}unit{\isadigit{2}}\ id{\isacharunderscore}{\kern0pt}right{\isacharunderscore}{\kern0pt}unit{\isadigit{2}}\ terminal{\isacharunderscore}{\kern0pt}func{\isacharunderscore}{\kern0pt}unique{\isacharparenright}{\kern0pt}\isanewline
\ \ \ \ \ \ \ \ \ \ \isacommand{then}\isamarkupfalse%
\ \isacommand{show}\isamarkupfalse%
\ {\isacharquery}{\kern0pt}thesis\ \isacommand{using}\isamarkupfalse%
\ calculation\ \isacommand{by}\isamarkupfalse%
\ auto\isanewline
\ \ \ \ \ \ \ \ \isacommand{qed}\isamarkupfalse%
\isanewline
\ \ \ \ \ \ \ \ \isacommand{have}\isamarkupfalse%
\ equation{\isadigit{2}}{\isacharcolon}{\kern0pt}\ {\isachardoublequoteopen}{\isasymlangle}a{\isadigit{1}}{\isacharcomma}{\kern0pt}a{\isadigit{2}}{\isasymrangle}\ {\isacharequal}{\kern0pt}\ \ {\isasymlangle}eval{\isacharunderscore}{\kern0pt}func\ A\ {\isasymOmega}\ {\isasymcirc}\isactrlsub c\ {\isasymlangle}{\isasymt}{\isacharcomma}{\kern0pt}\ g{\isasymrangle}{\isacharcomma}{\kern0pt}\isanewline
\ \ \ \ \ \ \ \ \ \ \ \ \ \ \ \ \ \ \ \ \ \ \ \ \ \ \ \ \ \ \ \ \ \ \ \ eval{\isacharunderscore}{\kern0pt}func\ A\ {\isasymOmega}\ {\isasymcirc}\isactrlsub c\ {\isasymlangle}{\isasymf}{\isacharcomma}{\kern0pt}\ g{\isasymrangle}{\isasymrangle}{\isachardoublequoteclose}\isanewline
\ \ \ \ \ \ \ \ \isacommand{proof}\isamarkupfalse%
\ {\isacharminus}{\kern0pt}\ \isanewline
\ \ \ \ \ \ \ \ \ \ \isacommand{have}\isamarkupfalse%
\ {\isachardoublequoteopen}{\isasymlangle}a{\isadigit{1}}{\isacharcomma}{\kern0pt}a{\isadigit{2}}{\isasymrangle}\ {\isacharequal}{\kern0pt}\ {\isasymlangle}eval{\isacharunderscore}{\kern0pt}func\ A\ {\isasymOmega}\ {\isasymcirc}\isactrlsub c\ {\isasymlangle}{\isasymt}\ {\isasymcirc}\isactrlsub c\ {\isasymbeta}\isactrlbsub A\isactrlbsup {\isasymOmega}\isactrlesup \isactrlesub {\isacharcomma}{\kern0pt}\ id{\isacharparenleft}{\kern0pt}A\isactrlbsup {\isasymOmega}\isactrlesup {\isacharparenright}{\kern0pt}{\isasymrangle}{\isacharcomma}{\kern0pt}\isanewline
\ \ \ \ \ \ \ \ \ \ \ \ \ \ \ \ \ \ \ \ \ \ \ \ \ \ eval{\isacharunderscore}{\kern0pt}func\ A\ {\isasymOmega}\ {\isasymcirc}\isactrlsub c\ {\isasymlangle}{\isasymf}\ {\isasymcirc}\isactrlsub c\ {\isasymbeta}\isactrlbsub A\isactrlbsup {\isasymOmega}\isactrlesup \isactrlesub {\isacharcomma}{\kern0pt}\ id{\isacharparenleft}{\kern0pt}A\isactrlbsup {\isasymOmega}\isactrlesup {\isacharparenright}{\kern0pt}{\isasymrangle}{\isasymrangle}\ {\isasymcirc}\isactrlsub c\ g{\isachardoublequoteclose}\isanewline
\ \ \ \ \ \ \ \ \ \ \ \ \isacommand{using}\isamarkupfalse%
\ {\isasymphi}{\isacharunderscore}{\kern0pt}def\ eqs\ phi{\isacharunderscore}{\kern0pt}f{\isacharunderscore}{\kern0pt}def\ \isacommand{by}\isamarkupfalse%
\ auto\isanewline
\ \ \ \ \ \ \ \ \ \ \isacommand{also}\isamarkupfalse%
\ \isacommand{have}\isamarkupfalse%
\ {\isachardoublequoteopen}{\isachardot}{\kern0pt}{\isachardot}{\kern0pt}{\isachardot}{\kern0pt}\ {\isacharequal}{\kern0pt}\ {\isasymlangle}eval{\isacharunderscore}{\kern0pt}func\ A\ {\isasymOmega}\ {\isasymcirc}\isactrlsub c\ {\isasymlangle}{\isasymt}\ {\isasymcirc}\isactrlsub c\ {\isasymbeta}\isactrlbsub A\isactrlbsup {\isasymOmega}\isactrlesup \isactrlesub {\isacharcomma}{\kern0pt}\ id{\isacharparenleft}{\kern0pt}A\isactrlbsup {\isasymOmega}\isactrlesup {\isacharparenright}{\kern0pt}{\isasymrangle}\ {\isasymcirc}\isactrlsub c\ g\ {\isacharcomma}{\kern0pt}\isanewline
\ \ \ \ \ \ \ \ \ \ \ \ \ \ \ \ \ \ \ \ \ \ \ \ \ \ \ \ eval{\isacharunderscore}{\kern0pt}func\ A\ {\isasymOmega}\ {\isasymcirc}\isactrlsub c\ {\isasymlangle}{\isasymf}\ {\isasymcirc}\isactrlsub c\ {\isasymbeta}\isactrlbsub A\isactrlbsup {\isasymOmega}\isactrlesup \isactrlesub {\isacharcomma}{\kern0pt}\ id{\isacharparenleft}{\kern0pt}A\isactrlbsup {\isasymOmega}\isactrlesup {\isacharparenright}{\kern0pt}{\isasymrangle}\ {\isasymcirc}\isactrlsub c\ g{\isasymrangle}{\isachardoublequoteclose}\isanewline
\ \ \ \ \ \ \ \ \ \ \ \ \isacommand{by}\isamarkupfalse%
\ {\isacharparenleft}{\kern0pt}typecheck{\isacharunderscore}{\kern0pt}cfuncs{\isacharcomma}{\kern0pt}smt\ cfunc{\isacharunderscore}{\kern0pt}prod{\isacharunderscore}{\kern0pt}comp\ comp{\isacharunderscore}{\kern0pt}associative{\isadigit{2}}{\isacharparenright}{\kern0pt}\isanewline
\ \ \ \ \ \ \ \ \ \ \isacommand{also}\isamarkupfalse%
\ \isacommand{have}\isamarkupfalse%
\ {\isachardoublequoteopen}{\isachardot}{\kern0pt}{\isachardot}{\kern0pt}{\isachardot}{\kern0pt}\ {\isacharequal}{\kern0pt}\ {\isasymlangle}eval{\isacharunderscore}{\kern0pt}func\ A\ {\isasymOmega}\ {\isasymcirc}\isactrlsub c\ {\isasymlangle}{\isasymt}\ {\isasymcirc}\isactrlsub c\ {\isasymbeta}\isactrlbsub A\isactrlbsup {\isasymOmega}\isactrlesup \isactrlesub \ {\isasymcirc}\isactrlsub c\ g{\isacharcomma}{\kern0pt}\ id{\isacharparenleft}{\kern0pt}A\isactrlbsup {\isasymOmega}\isactrlesup {\isacharparenright}{\kern0pt}\ {\isasymcirc}\isactrlsub c\ g{\isasymrangle}{\isacharcomma}{\kern0pt}\isanewline
\ \ \ \ \ \ \ \ \ \ \ \ \ \ \ \ \ \ \ \ \ \ \ \ \ \ \ \ eval{\isacharunderscore}{\kern0pt}func\ A\ {\isasymOmega}\ {\isasymcirc}\isactrlsub c\ {\isasymlangle}{\isasymf}\ {\isasymcirc}\isactrlsub c\ {\isasymbeta}\isactrlbsub A\isactrlbsup {\isasymOmega}\isactrlesup \isactrlesub \ {\isasymcirc}\isactrlsub c\ g{\isacharcomma}{\kern0pt}\ id{\isacharparenleft}{\kern0pt}A\isactrlbsup {\isasymOmega}\isactrlesup {\isacharparenright}{\kern0pt}{\isasymcirc}\isactrlsub c\ g\ {\isasymrangle}{\isasymrangle}{\isachardoublequoteclose}\isanewline
\ \ \ \ \ \ \ \ \ \ \ \ \isacommand{by}\isamarkupfalse%
\ {\isacharparenleft}{\kern0pt}typecheck{\isacharunderscore}{\kern0pt}cfuncs{\isacharcomma}{\kern0pt}\ simp\ add{\isacharcolon}{\kern0pt}\ cfunc{\isacharunderscore}{\kern0pt}prod{\isacharunderscore}{\kern0pt}comp\ comp{\isacharunderscore}{\kern0pt}associative{\isadigit{2}}{\isacharparenright}{\kern0pt}\isanewline
\ \ \ \ \ \ \ \ \ \ \isacommand{also}\isamarkupfalse%
\ \isacommand{have}\isamarkupfalse%
\ {\isachardoublequoteopen}{\isachardot}{\kern0pt}{\isachardot}{\kern0pt}{\isachardot}{\kern0pt}\ {\isacharequal}{\kern0pt}\ {\isasymlangle}eval{\isacharunderscore}{\kern0pt}func\ A\ {\isasymOmega}\ {\isasymcirc}\isactrlsub c\ {\isasymlangle}{\isasymt}{\isacharcomma}{\kern0pt}\ g{\isasymrangle}{\isacharcomma}{\kern0pt}\isanewline
\ \ \ \ \ \ \ \ \ \ \ \ \ \ \ \ \ \ \ \ \ \ \ \ \ \ \ \ eval{\isacharunderscore}{\kern0pt}func\ A\ {\isasymOmega}\ {\isasymcirc}\isactrlsub c\ {\isasymlangle}{\isasymf}{\isacharcomma}{\kern0pt}\ g{\isasymrangle}{\isasymrangle}{\isachardoublequoteclose}\ \ \ \ \isanewline
\ \ \ \ \ \ \ \ \ \ \ \ \isacommand{by}\isamarkupfalse%
\ {\isacharparenleft}{\kern0pt}typecheck{\isacharunderscore}{\kern0pt}cfuncs{\isacharcomma}{\kern0pt}\ metis\ id{\isadigit{1}}{\isacharunderscore}{\kern0pt}eq\ id{\isadigit{1}}{\isacharunderscore}{\kern0pt}is\ id{\isacharunderscore}{\kern0pt}left{\isacharunderscore}{\kern0pt}unit{\isadigit{2}}\ id{\isacharunderscore}{\kern0pt}right{\isacharunderscore}{\kern0pt}unit{\isadigit{2}}\ terminal{\isacharunderscore}{\kern0pt}func{\isacharunderscore}{\kern0pt}unique{\isacharparenright}{\kern0pt}\isanewline
\ \ \ \ \ \ \ \ \ \ \isacommand{then}\isamarkupfalse%
\ \isacommand{show}\isamarkupfalse%
\ {\isacharquery}{\kern0pt}thesis\ \isacommand{using}\isamarkupfalse%
\ calculation\ \isacommand{by}\isamarkupfalse%
\ auto\isanewline
\ \ \ \ \ \ \ \ \isacommand{qed}\isamarkupfalse%
\isanewline
\ \ \ \ \ \ \ \ \isacommand{have}\isamarkupfalse%
\ {\isachardoublequoteopen}{\isasymlangle}eval{\isacharunderscore}{\kern0pt}func\ A\ {\isasymOmega}\ {\isasymcirc}\isactrlsub c\ {\isasymlangle}{\isasymt}{\isacharcomma}{\kern0pt}\ f{\isasymrangle}{\isacharcomma}{\kern0pt}\ eval{\isacharunderscore}{\kern0pt}func\ A\ {\isasymOmega}\ {\isasymcirc}\isactrlsub c\ {\isasymlangle}{\isasymf}{\isacharcomma}{\kern0pt}\ f{\isasymrangle}{\isasymrangle}\ {\isacharequal}{\kern0pt}\ \isanewline
\ \ \ \ \ \ \ \ \ \ \ \ \ \ {\isasymlangle}eval{\isacharunderscore}{\kern0pt}func\ A\ {\isasymOmega}\ {\isasymcirc}\isactrlsub c\ {\isasymlangle}{\isasymt}{\isacharcomma}{\kern0pt}\ g{\isasymrangle}{\isacharcomma}{\kern0pt}\ eval{\isacharunderscore}{\kern0pt}func\ A\ {\isasymOmega}\ {\isasymcirc}\isactrlsub c\ {\isasymlangle}{\isasymf}{\isacharcomma}{\kern0pt}\ g{\isasymrangle}{\isasymrangle}{\isachardoublequoteclose}\isanewline
\ \ \ \ \ \ \ \ \ \ \isacommand{using}\isamarkupfalse%
\ equation{\isadigit{1}}\ equation{\isadigit{2}}\ \isacommand{by}\isamarkupfalse%
\ auto\isanewline
\ \ \ \ \ \ \ \ \isacommand{then}\isamarkupfalse%
\ \isacommand{have}\isamarkupfalse%
\ equation{\isadigit{3}}{\isacharcolon}{\kern0pt}\ {\isachardoublequoteopen}{\isacharparenleft}{\kern0pt}eval{\isacharunderscore}{\kern0pt}func\ A\ {\isasymOmega}\ {\isasymcirc}\isactrlsub c\ {\isasymlangle}{\isasymt}{\isacharcomma}{\kern0pt}\ f{\isasymrangle}\ {\isacharequal}{\kern0pt}\ eval{\isacharunderscore}{\kern0pt}func\ A\ {\isasymOmega}\ {\isasymcirc}\isactrlsub c\ {\isasymlangle}{\isasymt}{\isacharcomma}{\kern0pt}\ g{\isasymrangle}{\isacharparenright}{\kern0pt}\ {\isasymand}\ \isanewline
\ \ \ \ \ \ \ \ \ \ \ \ \ \ \ \ \ \ \ \ \ \ \ \ \ \ \ \ \ \ {\isacharparenleft}{\kern0pt}eval{\isacharunderscore}{\kern0pt}func\ A\ {\isasymOmega}\ {\isasymcirc}\isactrlsub c\ {\isasymlangle}{\isasymf}{\isacharcomma}{\kern0pt}\ f{\isasymrangle}\ {\isacharequal}{\kern0pt}\ eval{\isacharunderscore}{\kern0pt}func\ A\ {\isasymOmega}\ {\isasymcirc}\isactrlsub c\ {\isasymlangle}{\isasymf}{\isacharcomma}{\kern0pt}\ g{\isasymrangle}{\isacharparenright}{\kern0pt}{\isachardoublequoteclose}\isanewline
\ \ \ \ \ \ \ \ \ \ \isacommand{using}\isamarkupfalse%
\ \ cart{\isacharunderscore}{\kern0pt}prod{\isacharunderscore}{\kern0pt}eq{\isadigit{2}}\ \isacommand{by}\isamarkupfalse%
\ {\isacharparenleft}{\kern0pt}typecheck{\isacharunderscore}{\kern0pt}cfuncs{\isacharcomma}{\kern0pt}\ auto{\isacharparenright}{\kern0pt}\isanewline
\ \ \ \ \ \ \ \ \isacommand{have}\isamarkupfalse%
\ {\isachardoublequoteopen}eval{\isacharunderscore}{\kern0pt}func\ A\ {\isasymOmega}\ {\isasymcirc}\isactrlsub c\ id\isactrlsub c\ {\isasymOmega}\ {\isasymtimes}\isactrlsub f\ f\ \ {\isacharequal}{\kern0pt}\ eval{\isacharunderscore}{\kern0pt}func\ A\ {\isasymOmega}\ {\isasymcirc}\isactrlsub c\ id\isactrlsub c\ {\isasymOmega}\ {\isasymtimes}\isactrlsub f\ g{\isachardoublequoteclose}\isanewline
\ \ \ \ \ \ \ \ \isacommand{proof}\isamarkupfalse%
{\isacharparenleft}{\kern0pt}etcs{\isacharunderscore}{\kern0pt}rule\ one{\isacharunderscore}{\kern0pt}separator{\isacharparenright}{\kern0pt}\isanewline
\ \ \ \ \ \ \ \ \ \ \isacommand{fix}\isamarkupfalse%
\ x\isanewline
\ \ \ \ \ \ \ \ \ \ \isacommand{assume}\isamarkupfalse%
\ x{\isacharunderscore}{\kern0pt}type{\isacharbrackleft}{\kern0pt}type{\isacharunderscore}{\kern0pt}rule{\isacharbrackright}{\kern0pt}{\isacharcolon}{\kern0pt}\ {\isachardoublequoteopen}x\ {\isasymin}\isactrlsub c\ {\isasymOmega}\ {\isasymtimes}\isactrlsub c\ {\isasymone}{\isachardoublequoteclose}\isanewline
\ \ \ \ \ \ \ \ \ \ \isacommand{then}\isamarkupfalse%
\ \isacommand{obtain}\isamarkupfalse%
\ w\ i\ \isakeyword{where}\ \ x{\isacharunderscore}{\kern0pt}def{\isacharcolon}{\kern0pt}\ {\isachardoublequoteopen}{\isacharparenleft}{\kern0pt}w\ {\isasymin}\isactrlsub c\ {\isasymOmega}{\isacharparenright}{\kern0pt}\ {\isasymand}\ {\isacharparenleft}{\kern0pt}i\ {\isasymin}\isactrlsub c\ {\isasymone}{\isacharparenright}{\kern0pt}\ {\isasymand}\ {\isacharparenleft}{\kern0pt}x\ {\isacharequal}{\kern0pt}\ {\isasymlangle}w{\isacharcomma}{\kern0pt}i{\isasymrangle}{\isacharparenright}{\kern0pt}{\isachardoublequoteclose}\isanewline
\ \ \ \ \ \ \ \ \ \ \ \ \isacommand{using}\isamarkupfalse%
\ cart{\isacharunderscore}{\kern0pt}prod{\isacharunderscore}{\kern0pt}decomp\ \isacommand{by}\isamarkupfalse%
\ blast\isanewline
\ \ \ \ \ \ \ \ \ \ \isacommand{then}\isamarkupfalse%
\ \isacommand{have}\isamarkupfalse%
\ i{\isacharunderscore}{\kern0pt}def{\isacharcolon}{\kern0pt}\ {\isachardoublequoteopen}i\ {\isacharequal}{\kern0pt}\ id{\isacharparenleft}{\kern0pt}{\isasymone}{\isacharparenright}{\kern0pt}{\isachardoublequoteclose}\isanewline
\ \ \ \ \ \ \ \ \ \ \ \ \isacommand{using}\isamarkupfalse%
\ id{\isadigit{1}}{\isacharunderscore}{\kern0pt}eq\ id{\isadigit{1}}{\isacharunderscore}{\kern0pt}is\ one{\isacharunderscore}{\kern0pt}unique{\isacharunderscore}{\kern0pt}element\ \isacommand{by}\isamarkupfalse%
\ auto\isanewline
\ \ \ \ \ \ \ \ \ \ \isacommand{have}\isamarkupfalse%
\ w{\isacharunderscore}{\kern0pt}def{\isacharcolon}{\kern0pt}\ {\isachardoublequoteopen}{\isacharparenleft}{\kern0pt}w\ {\isacharequal}{\kern0pt}\ {\isasymf}{\isacharparenright}{\kern0pt}\ {\isasymor}\ {\isacharparenleft}{\kern0pt}w\ {\isacharequal}{\kern0pt}\ {\isasymt}{\isacharparenright}{\kern0pt}{\isachardoublequoteclose}\isanewline
\ \ \ \ \ \ \ \ \ \ \ \ \isacommand{by}\isamarkupfalse%
\ {\isacharparenleft}{\kern0pt}simp\ add{\isacharcolon}{\kern0pt}\ true{\isacharunderscore}{\kern0pt}false{\isacharunderscore}{\kern0pt}only{\isacharunderscore}{\kern0pt}truth{\isacharunderscore}{\kern0pt}values\ x{\isacharunderscore}{\kern0pt}def{\isacharparenright}{\kern0pt}\isanewline
\ \ \ \ \ \ \ \ \ \ \isacommand{then}\isamarkupfalse%
\ \isacommand{have}\isamarkupfalse%
\ x{\isacharunderscore}{\kern0pt}def{\isadigit{2}}{\isacharcolon}{\kern0pt}\ {\isachardoublequoteopen}{\isacharparenleft}{\kern0pt}x\ {\isacharequal}{\kern0pt}\ {\isasymlangle}{\isasymf}{\isacharcomma}{\kern0pt}i{\isasymrangle}{\isacharparenright}{\kern0pt}\ {\isasymor}\ {\isacharparenleft}{\kern0pt}x\ {\isacharequal}{\kern0pt}\ {\isasymlangle}{\isasymt}{\isacharcomma}{\kern0pt}i{\isasymrangle}{\isacharparenright}{\kern0pt}{\isachardoublequoteclose}\isanewline
\ \ \ \ \ \ \ \ \ \ \ \ \isacommand{using}\isamarkupfalse%
\ x{\isacharunderscore}{\kern0pt}def\ \isacommand{by}\isamarkupfalse%
\ auto\isanewline
\ \ \ \ \ \ \ \ \ \ \isacommand{show}\isamarkupfalse%
\ {\isachardoublequoteopen}{\isacharparenleft}{\kern0pt}eval{\isacharunderscore}{\kern0pt}func\ A\ {\isasymOmega}\ {\isasymcirc}\isactrlsub c\ id\isactrlsub c\ {\isasymOmega}\ {\isasymtimes}\isactrlsub f\ f{\isacharparenright}{\kern0pt}\ {\isasymcirc}\isactrlsub c\ x\ {\isacharequal}{\kern0pt}\ {\isacharparenleft}{\kern0pt}eval{\isacharunderscore}{\kern0pt}func\ A\ {\isasymOmega}\ {\isasymcirc}\isactrlsub c\ id\isactrlsub c\ {\isasymOmega}\ {\isasymtimes}\isactrlsub f\ g{\isacharparenright}{\kern0pt}\ {\isasymcirc}\isactrlsub c\ x{\isachardoublequoteclose}\isanewline
\ \ \ \ \ \ \ \ \ \ \isacommand{proof}\isamarkupfalse%
{\isacharparenleft}{\kern0pt}cases\ {\isachardoublequoteopen}{\isacharparenleft}{\kern0pt}x\ {\isacharequal}{\kern0pt}\ {\isasymlangle}{\isasymf}{\isacharcomma}{\kern0pt}i{\isasymrangle}{\isacharparenright}{\kern0pt}{\isachardoublequoteclose}{\isacharcomma}{\kern0pt}clarify{\isacharparenright}{\kern0pt}\isanewline
\ \ \ \ \ \ \ \ \ \ \ \ \isacommand{assume}\isamarkupfalse%
\ case{\isadigit{1}}{\isacharcolon}{\kern0pt}\ {\isachardoublequoteopen}x\ {\isacharequal}{\kern0pt}\ {\isasymlangle}{\isasymf}{\isacharcomma}{\kern0pt}i{\isasymrangle}{\isachardoublequoteclose}\isanewline
\ \ \ \ \ \ \ \ \ \ \ \ \isacommand{have}\isamarkupfalse%
\ {\isachardoublequoteopen}{\isacharparenleft}{\kern0pt}eval{\isacharunderscore}{\kern0pt}func\ A\ {\isasymOmega}\ {\isasymcirc}\isactrlsub c\ {\isacharparenleft}{\kern0pt}id\isactrlsub c\ {\isasymOmega}\ {\isasymtimes}\isactrlsub f\ f{\isacharparenright}{\kern0pt}{\isacharparenright}{\kern0pt}\ {\isasymcirc}\isactrlsub c\ {\isasymlangle}{\isasymf}{\isacharcomma}{\kern0pt}i{\isasymrangle}\ {\isacharequal}{\kern0pt}\ eval{\isacharunderscore}{\kern0pt}func\ A\ {\isasymOmega}\ {\isasymcirc}\isactrlsub c\ {\isacharparenleft}{\kern0pt}{\isacharparenleft}{\kern0pt}id\isactrlsub c\ {\isasymOmega}\ {\isasymtimes}\isactrlsub f\ f{\isacharparenright}{\kern0pt}\ {\isasymcirc}\isactrlsub c\ {\isasymlangle}{\isasymf}{\isacharcomma}{\kern0pt}i{\isasymrangle}{\isacharparenright}{\kern0pt}{\isachardoublequoteclose}\isanewline
\ \ \ \ \ \ \ \ \ \ \ \ \ \ \isacommand{using}\isamarkupfalse%
\ case{\isadigit{1}}\ comp{\isacharunderscore}{\kern0pt}associative{\isadigit{2}}\ x{\isacharunderscore}{\kern0pt}type\ \isacommand{by}\isamarkupfalse%
\ {\isacharparenleft}{\kern0pt}typecheck{\isacharunderscore}{\kern0pt}cfuncs{\isacharcomma}{\kern0pt}\ auto{\isacharparenright}{\kern0pt}\isanewline
\ \ \ \ \ \ \ \ \ \ \ \ \isacommand{also}\isamarkupfalse%
\ \isacommand{have}\isamarkupfalse%
\ {\isachardoublequoteopen}{\isachardot}{\kern0pt}{\isachardot}{\kern0pt}{\isachardot}{\kern0pt}\ {\isacharequal}{\kern0pt}\ eval{\isacharunderscore}{\kern0pt}func\ A\ {\isasymOmega}\ {\isasymcirc}\isactrlsub c\ {\isasymlangle}id\isactrlsub c\ {\isasymOmega}\ {\isasymcirc}\isactrlsub c\ \ {\isasymf}{\isacharcomma}{\kern0pt}f\ {\isasymcirc}\isactrlsub c\ i{\isasymrangle}{\isachardoublequoteclose}\isanewline
\ \ \ \ \ \ \ \ \ \ \ \ \ \ \isacommand{using}\isamarkupfalse%
\ cfunc{\isacharunderscore}{\kern0pt}cross{\isacharunderscore}{\kern0pt}prod{\isacharunderscore}{\kern0pt}comp{\isacharunderscore}{\kern0pt}cfunc{\isacharunderscore}{\kern0pt}prod\ i{\isacharunderscore}{\kern0pt}def\ id{\isadigit{1}}{\isacharunderscore}{\kern0pt}eq\ id{\isadigit{1}}{\isacharunderscore}{\kern0pt}is\ \isacommand{by}\isamarkupfalse%
\ {\isacharparenleft}{\kern0pt}typecheck{\isacharunderscore}{\kern0pt}cfuncs{\isacharcomma}{\kern0pt}\ auto{\isacharparenright}{\kern0pt}\isanewline
\ \ \ \ \ \ \ \ \ \ \ \ \isacommand{also}\isamarkupfalse%
\ \isacommand{have}\isamarkupfalse%
\ {\isachardoublequoteopen}{\isachardot}{\kern0pt}{\isachardot}{\kern0pt}{\isachardot}{\kern0pt}\ {\isacharequal}{\kern0pt}\ eval{\isacharunderscore}{\kern0pt}func\ A\ {\isasymOmega}\ {\isasymcirc}\isactrlsub c\ {\isasymlangle}{\isasymf}{\isacharcomma}{\kern0pt}\ f\ {\isasymrangle}{\isachardoublequoteclose}\isanewline
\ \ \ \ \ \ \ \ \ \ \ \ \ \ \isacommand{using}\isamarkupfalse%
\ f{\isacharunderscore}{\kern0pt}type\ false{\isacharunderscore}{\kern0pt}func{\isacharunderscore}{\kern0pt}type\ i{\isacharunderscore}{\kern0pt}def\ id{\isacharunderscore}{\kern0pt}left{\isacharunderscore}{\kern0pt}unit{\isadigit{2}}\ id{\isacharunderscore}{\kern0pt}right{\isacharunderscore}{\kern0pt}unit{\isadigit{2}}\ \isacommand{by}\isamarkupfalse%
\ auto\isanewline
\ \ \ \ \ \ \ \ \ \ \ \ \isacommand{also}\isamarkupfalse%
\ \isacommand{have}\isamarkupfalse%
\ {\isachardoublequoteopen}{\isachardot}{\kern0pt}{\isachardot}{\kern0pt}{\isachardot}{\kern0pt}\ {\isacharequal}{\kern0pt}\ eval{\isacharunderscore}{\kern0pt}func\ A\ {\isasymOmega}\ {\isasymcirc}\isactrlsub c\ {\isasymlangle}{\isasymf}{\isacharcomma}{\kern0pt}\ g{\isasymrangle}{\isachardoublequoteclose}\isanewline
\ \ \ \ \ \ \ \ \ \ \ \ \ \ \isacommand{using}\isamarkupfalse%
\ equation{\isadigit{3}}\ \isacommand{by}\isamarkupfalse%
\ blast\isanewline
\ \ \ \ \ \ \ \ \ \ \ \ \isacommand{also}\isamarkupfalse%
\ \isacommand{have}\isamarkupfalse%
\ {\isachardoublequoteopen}{\isachardot}{\kern0pt}{\isachardot}{\kern0pt}{\isachardot}{\kern0pt}\ {\isacharequal}{\kern0pt}\ eval{\isacharunderscore}{\kern0pt}func\ A\ {\isasymOmega}\ {\isasymcirc}\isactrlsub c\ {\isasymlangle}id\isactrlsub c\ {\isasymOmega}\ {\isasymcirc}\isactrlsub c\ \ {\isasymf}{\isacharcomma}{\kern0pt}g\ {\isasymcirc}\isactrlsub c\ i{\isasymrangle}{\isachardoublequoteclose}\isanewline
\ \ \ \ \ \ \ \ \ \ \ \ \ \ \isacommand{by}\isamarkupfalse%
\ {\isacharparenleft}{\kern0pt}typecheck{\isacharunderscore}{\kern0pt}cfuncs{\isacharcomma}{\kern0pt}\ simp\ add{\isacharcolon}{\kern0pt}\ i{\isacharunderscore}{\kern0pt}def\ id{\isacharunderscore}{\kern0pt}left{\isacharunderscore}{\kern0pt}unit{\isadigit{2}}\ id{\isacharunderscore}{\kern0pt}right{\isacharunderscore}{\kern0pt}unit{\isadigit{2}}{\isacharparenright}{\kern0pt}\isanewline
\ \ \ \ \ \ \ \ \ \ \ \ \isacommand{also}\isamarkupfalse%
\ \isacommand{have}\isamarkupfalse%
\ {\isachardoublequoteopen}{\isachardot}{\kern0pt}{\isachardot}{\kern0pt}{\isachardot}{\kern0pt}\ {\isacharequal}{\kern0pt}\ eval{\isacharunderscore}{\kern0pt}func\ A\ {\isasymOmega}\ {\isasymcirc}\isactrlsub c\ {\isacharparenleft}{\kern0pt}{\isacharparenleft}{\kern0pt}id\isactrlsub c\ {\isasymOmega}\ {\isasymtimes}\isactrlsub f\ g{\isacharparenright}{\kern0pt}\ {\isasymcirc}\isactrlsub c\ {\isasymlangle}{\isasymf}{\isacharcomma}{\kern0pt}i{\isasymrangle}{\isacharparenright}{\kern0pt}{\isachardoublequoteclose}\isanewline
\ \ \ \ \ \ \ \ \ \ \ \ \ \ \isacommand{using}\isamarkupfalse%
\ cfunc{\isacharunderscore}{\kern0pt}cross{\isacharunderscore}{\kern0pt}prod{\isacharunderscore}{\kern0pt}comp{\isacharunderscore}{\kern0pt}cfunc{\isacharunderscore}{\kern0pt}prod\ i{\isacharunderscore}{\kern0pt}def\ id{\isadigit{1}}{\isacharunderscore}{\kern0pt}eq\ id{\isadigit{1}}{\isacharunderscore}{\kern0pt}is\ \isacommand{by}\isamarkupfalse%
\ {\isacharparenleft}{\kern0pt}typecheck{\isacharunderscore}{\kern0pt}cfuncs{\isacharcomma}{\kern0pt}\ auto{\isacharparenright}{\kern0pt}\isanewline
\ \ \ \ \ \ \ \ \ \ \ \ \isacommand{also}\isamarkupfalse%
\ \isacommand{have}\isamarkupfalse%
\ {\isachardoublequoteopen}{\isachardot}{\kern0pt}{\isachardot}{\kern0pt}{\isachardot}{\kern0pt}\ {\isacharequal}{\kern0pt}\ {\isacharparenleft}{\kern0pt}eval{\isacharunderscore}{\kern0pt}func\ A\ {\isasymOmega}\ {\isasymcirc}\isactrlsub c\ {\isacharparenleft}{\kern0pt}id\isactrlsub c\ {\isasymOmega}\ {\isasymtimes}\isactrlsub f\ g{\isacharparenright}{\kern0pt}{\isacharparenright}{\kern0pt}\ {\isasymcirc}\isactrlsub c\ {\isasymlangle}{\isasymf}{\isacharcomma}{\kern0pt}i{\isasymrangle}{\isachardoublequoteclose}\isanewline
\ \ \ \ \ \ \ \ \ \ \ \ \ \ \isacommand{using}\isamarkupfalse%
\ case{\isadigit{1}}\ comp{\isacharunderscore}{\kern0pt}associative{\isadigit{2}}\ x{\isacharunderscore}{\kern0pt}type\ \isacommand{by}\isamarkupfalse%
\ {\isacharparenleft}{\kern0pt}typecheck{\isacharunderscore}{\kern0pt}cfuncs{\isacharcomma}{\kern0pt}\ auto{\isacharparenright}{\kern0pt}\isanewline
\ \ \ \ \ \ \ \ \ \ \ \ \isacommand{then}\isamarkupfalse%
\ \isacommand{show}\isamarkupfalse%
\ {\isachardoublequoteopen}{\isacharparenleft}{\kern0pt}eval{\isacharunderscore}{\kern0pt}func\ A\ {\isasymOmega}\ {\isasymcirc}\isactrlsub c\ id\isactrlsub c\ {\isasymOmega}\ {\isasymtimes}\isactrlsub f\ f{\isacharparenright}{\kern0pt}\ {\isasymcirc}\isactrlsub c\ {\isasymlangle}{\isasymf}{\isacharcomma}{\kern0pt}i{\isasymrangle}\ {\isacharequal}{\kern0pt}\ {\isacharparenleft}{\kern0pt}eval{\isacharunderscore}{\kern0pt}func\ A\ {\isasymOmega}\ {\isasymcirc}\isactrlsub c\ id\isactrlsub c\ {\isasymOmega}\ {\isasymtimes}\isactrlsub f\ g{\isacharparenright}{\kern0pt}\ {\isasymcirc}\isactrlsub c\ {\isasymlangle}{\isasymf}{\isacharcomma}{\kern0pt}i{\isasymrangle}{\isachardoublequoteclose}\isanewline
\ \ \ \ \ \ \ \ \ \ \ \ \ \ \isacommand{by}\isamarkupfalse%
\ {\isacharparenleft}{\kern0pt}simp\ add{\isacharcolon}{\kern0pt}\ calculation{\isacharparenright}{\kern0pt}\isanewline
\ \ \ \ \ \ \ \ \ \ \isacommand{next}\isamarkupfalse%
\isanewline
\ \ \ \ \ \ \ \ \ \ \ \ \isacommand{assume}\isamarkupfalse%
\ case{\isadigit{2}}{\isacharcolon}{\kern0pt}\ {\isachardoublequoteopen}x\ {\isasymnoteq}\ {\isasymlangle}{\isasymf}{\isacharcomma}{\kern0pt}i{\isasymrangle}{\isachardoublequoteclose}\isanewline
\ \ \ \ \ \ \ \ \ \ \ \ \isacommand{then}\isamarkupfalse%
\ \isacommand{have}\isamarkupfalse%
\ x{\isacharunderscore}{\kern0pt}eq{\isacharcolon}{\kern0pt}\ {\isachardoublequoteopen}x\ {\isacharequal}{\kern0pt}\ {\isasymlangle}{\isasymt}{\isacharcomma}{\kern0pt}i{\isasymrangle}{\isachardoublequoteclose}\isanewline
\ \ \ \ \ \ \ \ \ \ \ \ \ \ \isacommand{using}\isamarkupfalse%
\ x{\isacharunderscore}{\kern0pt}def{\isadigit{2}}\ \isacommand{by}\isamarkupfalse%
\ blast\isanewline
\ \ \ \ \ \ \ \ \ \ \ \ \isacommand{have}\isamarkupfalse%
\ {\isachardoublequoteopen}{\isacharparenleft}{\kern0pt}eval{\isacharunderscore}{\kern0pt}func\ A\ {\isasymOmega}\ {\isasymcirc}\isactrlsub c\ {\isacharparenleft}{\kern0pt}id\isactrlsub c\ {\isasymOmega}\ {\isasymtimes}\isactrlsub f\ f{\isacharparenright}{\kern0pt}{\isacharparenright}{\kern0pt}\ {\isasymcirc}\isactrlsub c\ {\isasymlangle}{\isasymt}{\isacharcomma}{\kern0pt}i{\isasymrangle}\ {\isacharequal}{\kern0pt}\ eval{\isacharunderscore}{\kern0pt}func\ A\ {\isasymOmega}\ {\isasymcirc}\isactrlsub c\ {\isacharparenleft}{\kern0pt}{\isacharparenleft}{\kern0pt}id\isactrlsub c\ {\isasymOmega}\ {\isasymtimes}\isactrlsub f\ f{\isacharparenright}{\kern0pt}\ {\isasymcirc}\isactrlsub c\ {\isasymlangle}{\isasymt}{\isacharcomma}{\kern0pt}i{\isasymrangle}{\isacharparenright}{\kern0pt}{\isachardoublequoteclose}\isanewline
\ \ \ \ \ \ \ \ \ \ \ \ \ \ \ \ \isacommand{using}\isamarkupfalse%
\ case{\isadigit{2}}\ x{\isacharunderscore}{\kern0pt}eq\ comp{\isacharunderscore}{\kern0pt}associative{\isadigit{2}}\ x{\isacharunderscore}{\kern0pt}type\ \isacommand{by}\isamarkupfalse%
\ {\isacharparenleft}{\kern0pt}typecheck{\isacharunderscore}{\kern0pt}cfuncs{\isacharcomma}{\kern0pt}\ auto{\isacharparenright}{\kern0pt}\isanewline
\ \ \ \ \ \ \ \ \ \ \ \ \isacommand{also}\isamarkupfalse%
\ \isacommand{have}\isamarkupfalse%
\ {\isachardoublequoteopen}{\isachardot}{\kern0pt}{\isachardot}{\kern0pt}{\isachardot}{\kern0pt}\ {\isacharequal}{\kern0pt}\ eval{\isacharunderscore}{\kern0pt}func\ A\ {\isasymOmega}\ {\isasymcirc}\isactrlsub c\ {\isasymlangle}id\isactrlsub c\ {\isasymOmega}\ {\isasymcirc}\isactrlsub c\ \ {\isasymt}{\isacharcomma}{\kern0pt}f\ {\isasymcirc}\isactrlsub c\ i{\isasymrangle}{\isachardoublequoteclose}\isanewline
\ \ \ \ \ \ \ \ \ \ \ \ \ \ \ \ \isacommand{using}\isamarkupfalse%
\ cfunc{\isacharunderscore}{\kern0pt}cross{\isacharunderscore}{\kern0pt}prod{\isacharunderscore}{\kern0pt}comp{\isacharunderscore}{\kern0pt}cfunc{\isacharunderscore}{\kern0pt}prod\ i{\isacharunderscore}{\kern0pt}def\ id{\isadigit{1}}{\isacharunderscore}{\kern0pt}eq\ id{\isadigit{1}}{\isacharunderscore}{\kern0pt}is\ \isacommand{by}\isamarkupfalse%
\ {\isacharparenleft}{\kern0pt}typecheck{\isacharunderscore}{\kern0pt}cfuncs{\isacharcomma}{\kern0pt}\ auto{\isacharparenright}{\kern0pt}\isanewline
\ \ \ \ \ \ \ \ \ \ \ \ \isacommand{also}\isamarkupfalse%
\ \isacommand{have}\isamarkupfalse%
\ {\isachardoublequoteopen}{\isachardot}{\kern0pt}{\isachardot}{\kern0pt}{\isachardot}{\kern0pt}\ {\isacharequal}{\kern0pt}\ eval{\isacharunderscore}{\kern0pt}func\ A\ {\isasymOmega}\ {\isasymcirc}\isactrlsub c\ {\isasymlangle}{\isasymt}{\isacharcomma}{\kern0pt}\ f\ {\isasymrangle}{\isachardoublequoteclose}\isanewline
\ \ \ \ \ \ \ \ \ \ \ \ \ \ \isacommand{using}\isamarkupfalse%
\ f{\isacharunderscore}{\kern0pt}type\ i{\isacharunderscore}{\kern0pt}def\ id{\isacharunderscore}{\kern0pt}left{\isacharunderscore}{\kern0pt}unit{\isadigit{2}}\ id{\isacharunderscore}{\kern0pt}right{\isacharunderscore}{\kern0pt}unit{\isadigit{2}}\ true{\isacharunderscore}{\kern0pt}func{\isacharunderscore}{\kern0pt}type\ \isacommand{by}\isamarkupfalse%
\ auto\isanewline
\ \ \ \ \ \ \ \ \ \ \ \ \isacommand{also}\isamarkupfalse%
\ \isacommand{have}\isamarkupfalse%
\ {\isachardoublequoteopen}{\isachardot}{\kern0pt}{\isachardot}{\kern0pt}{\isachardot}{\kern0pt}\ {\isacharequal}{\kern0pt}\ eval{\isacharunderscore}{\kern0pt}func\ A\ {\isasymOmega}\ {\isasymcirc}\isactrlsub c\ {\isasymlangle}{\isasymt}{\isacharcomma}{\kern0pt}\ g{\isasymrangle}{\isachardoublequoteclose}\isanewline
\ \ \ \ \ \ \ \ \ \ \ \ \ \ \isacommand{using}\isamarkupfalse%
\ equation{\isadigit{3}}\ \isacommand{by}\isamarkupfalse%
\ blast\isanewline
\ \ \ \ \ \ \ \ \ \ \ \ \isacommand{also}\isamarkupfalse%
\ \isacommand{have}\isamarkupfalse%
\ {\isachardoublequoteopen}{\isachardot}{\kern0pt}{\isachardot}{\kern0pt}{\isachardot}{\kern0pt}\ {\isacharequal}{\kern0pt}\ eval{\isacharunderscore}{\kern0pt}func\ A\ {\isasymOmega}\ {\isasymcirc}\isactrlsub c\ {\isasymlangle}id\isactrlsub c\ {\isasymOmega}\ {\isasymcirc}\isactrlsub c\ \ {\isasymt}{\isacharcomma}{\kern0pt}g\ {\isasymcirc}\isactrlsub c\ i{\isasymrangle}{\isachardoublequoteclose}\isanewline
\ \ \ \ \ \ \ \ \ \ \ \ \ \ \ \ \isacommand{by}\isamarkupfalse%
\ {\isacharparenleft}{\kern0pt}typecheck{\isacharunderscore}{\kern0pt}cfuncs{\isacharcomma}{\kern0pt}\ simp\ add{\isacharcolon}{\kern0pt}\ i{\isacharunderscore}{\kern0pt}def\ id{\isacharunderscore}{\kern0pt}left{\isacharunderscore}{\kern0pt}unit{\isadigit{2}}\ id{\isacharunderscore}{\kern0pt}right{\isacharunderscore}{\kern0pt}unit{\isadigit{2}}{\isacharparenright}{\kern0pt}\isanewline
\ \ \ \ \ \ \ \ \ \ \ \ \isacommand{also}\isamarkupfalse%
\ \isacommand{have}\isamarkupfalse%
\ {\isachardoublequoteopen}{\isachardot}{\kern0pt}{\isachardot}{\kern0pt}{\isachardot}{\kern0pt}\ {\isacharequal}{\kern0pt}\ eval{\isacharunderscore}{\kern0pt}func\ A\ {\isasymOmega}\ {\isasymcirc}\isactrlsub c\ {\isacharparenleft}{\kern0pt}{\isacharparenleft}{\kern0pt}id\isactrlsub c\ {\isasymOmega}\ {\isasymtimes}\isactrlsub f\ g{\isacharparenright}{\kern0pt}\ {\isasymcirc}\isactrlsub c\ {\isasymlangle}{\isasymt}{\isacharcomma}{\kern0pt}i{\isasymrangle}{\isacharparenright}{\kern0pt}{\isachardoublequoteclose}\isanewline
\ \ \ \ \ \ \ \ \ \ \ \ \ \ \ \ \isacommand{using}\isamarkupfalse%
\ cfunc{\isacharunderscore}{\kern0pt}cross{\isacharunderscore}{\kern0pt}prod{\isacharunderscore}{\kern0pt}comp{\isacharunderscore}{\kern0pt}cfunc{\isacharunderscore}{\kern0pt}prod\ i{\isacharunderscore}{\kern0pt}def\ id{\isadigit{1}}{\isacharunderscore}{\kern0pt}eq\ id{\isadigit{1}}{\isacharunderscore}{\kern0pt}is\ \isacommand{by}\isamarkupfalse%
\ {\isacharparenleft}{\kern0pt}typecheck{\isacharunderscore}{\kern0pt}cfuncs{\isacharcomma}{\kern0pt}\ auto{\isacharparenright}{\kern0pt}\isanewline
\ \ \ \ \ \ \ \ \ \ \ \ \isacommand{also}\isamarkupfalse%
\ \isacommand{have}\isamarkupfalse%
\ {\isachardoublequoteopen}{\isachardot}{\kern0pt}{\isachardot}{\kern0pt}{\isachardot}{\kern0pt}\ {\isacharequal}{\kern0pt}\ {\isacharparenleft}{\kern0pt}eval{\isacharunderscore}{\kern0pt}func\ A\ {\isasymOmega}\ {\isasymcirc}\isactrlsub c\ {\isacharparenleft}{\kern0pt}id\isactrlsub c\ {\isasymOmega}\ {\isasymtimes}\isactrlsub f\ g{\isacharparenright}{\kern0pt}{\isacharparenright}{\kern0pt}\ {\isasymcirc}\isactrlsub c\ {\isasymlangle}{\isasymt}{\isacharcomma}{\kern0pt}i{\isasymrangle}{\isachardoublequoteclose}\isanewline
\ \ \ \ \ \ \ \ \ \ \ \ \ \ \isacommand{using}\isamarkupfalse%
\ comp{\isacharunderscore}{\kern0pt}associative{\isadigit{2}}\ x{\isacharunderscore}{\kern0pt}eq\ x{\isacharunderscore}{\kern0pt}type\ \isacommand{by}\isamarkupfalse%
\ {\isacharparenleft}{\kern0pt}typecheck{\isacharunderscore}{\kern0pt}cfuncs{\isacharcomma}{\kern0pt}\ blast{\isacharparenright}{\kern0pt}\isanewline
\ \ \ \ \ \ \ \ \ \ \ \ \isacommand{then}\isamarkupfalse%
\ \isacommand{show}\isamarkupfalse%
\ {\isachardoublequoteopen}{\isacharparenleft}{\kern0pt}eval{\isacharunderscore}{\kern0pt}func\ A\ {\isasymOmega}\ {\isasymcirc}\isactrlsub c\ id\isactrlsub c\ {\isasymOmega}\ {\isasymtimes}\isactrlsub f\ f{\isacharparenright}{\kern0pt}\ {\isasymcirc}\isactrlsub c\ x\ {\isacharequal}{\kern0pt}\ {\isacharparenleft}{\kern0pt}eval{\isacharunderscore}{\kern0pt}func\ A\ {\isasymOmega}\ {\isasymcirc}\isactrlsub c\ id\isactrlsub c\ {\isasymOmega}\ {\isasymtimes}\isactrlsub f\ g{\isacharparenright}{\kern0pt}\ {\isasymcirc}\isactrlsub c\ x{\isachardoublequoteclose}\isanewline
\ \ \ \ \ \ \ \ \ \ \ \ \ \ \isacommand{by}\isamarkupfalse%
\ {\isacharparenleft}{\kern0pt}simp\ add{\isacharcolon}{\kern0pt}\ calculation\ x{\isacharunderscore}{\kern0pt}eq{\isacharparenright}{\kern0pt}\isanewline
\ \ \ \ \ \ \ \ \ \ \isacommand{qed}\isamarkupfalse%
\isanewline
\ \ \ \ \ \ \ \ \isacommand{qed}\isamarkupfalse%
\isanewline
\ \ \ \ \ \ \ \ \isacommand{then}\isamarkupfalse%
\ \isacommand{show}\isamarkupfalse%
\ {\isachardoublequoteopen}eval{\isacharunderscore}{\kern0pt}func\ A\ {\isasymOmega}\ {\isasymcirc}\isactrlsub c\ id\isactrlsub c\ {\isasymOmega}\ {\isasymtimes}\isactrlsub f\ f\ {\isasymcirc}\isactrlsub c\ id{\isacharunderscore}{\kern0pt}{\isadigit{1}}\ {\isacharequal}{\kern0pt}\ eval{\isacharunderscore}{\kern0pt}func\ A\ {\isasymOmega}\ {\isasymcirc}\isactrlsub c\ id\isactrlsub c\ {\isasymOmega}\ {\isasymtimes}\isactrlsub f\ g\ {\isasymcirc}\isactrlsub c\ id{\isacharunderscore}{\kern0pt}{\isadigit{1}}{\isachardoublequoteclose}\isanewline
\ \ \ \ \ \ \ \ \ \ \isacommand{using}\isamarkupfalse%
\ \ f{\isacharunderscore}{\kern0pt}type\ g{\isacharunderscore}{\kern0pt}type\ same{\isacharunderscore}{\kern0pt}evals{\isacharunderscore}{\kern0pt}equal\ \isacommand{by}\isamarkupfalse%
\ blast\isanewline
\ \ \ \ \ \ \ \ \isacommand{qed}\isamarkupfalse%
\isanewline
\ \ \ \ \ \ \isacommand{qed}\isamarkupfalse%
\isanewline
\ \ \ \ \isacommand{qed}\isamarkupfalse%
\isanewline
\ \ \ \ \isacommand{then}\isamarkupfalse%
\ \isacommand{have}\isamarkupfalse%
\ {\isachardoublequoteopen}monomorphism{\isacharparenleft}{\kern0pt}{\isasymphi}{\isacharparenright}{\kern0pt}{\isachardoublequoteclose}\isanewline
\ \ \ \ \ \ \isacommand{using}\isamarkupfalse%
\ injective{\isacharunderscore}{\kern0pt}imp{\isacharunderscore}{\kern0pt}monomorphism\ \isacommand{by}\isamarkupfalse%
\ auto\isanewline
\ \ \ \ \isacommand{have}\isamarkupfalse%
\ {\isachardoublequoteopen}surjective{\isacharparenleft}{\kern0pt}{\isasymphi}{\isacharparenright}{\kern0pt}{\isachardoublequoteclose}\isanewline
\ \ \ \ \ \ \isacommand{unfolding}\isamarkupfalse%
\ surjective{\isacharunderscore}{\kern0pt}def\isanewline
\ \ \ \ \isacommand{proof}\isamarkupfalse%
{\isacharparenleft}{\kern0pt}clarify{\isacharparenright}{\kern0pt}\isanewline
\ \ \ \ \ \ \isacommand{fix}\isamarkupfalse%
\ y\ \isanewline
\ \ \ \ \ \ \isacommand{assume}\isamarkupfalse%
\ {\isachardoublequoteopen}y\ {\isasymin}\isactrlsub c\ codomain\ {\isasymphi}{\isachardoublequoteclose}\ \isacommand{then}\isamarkupfalse%
\ \isacommand{have}\isamarkupfalse%
\ y{\isacharunderscore}{\kern0pt}type{\isacharbrackleft}{\kern0pt}type{\isacharunderscore}{\kern0pt}rule{\isacharbrackright}{\kern0pt}{\isacharcolon}{\kern0pt}\ {\isachardoublequoteopen}y\ {\isasymin}\isactrlsub c\ A\ {\isasymtimes}\isactrlsub c\ A{\isachardoublequoteclose}\isanewline
\ \ \ \ \ \ \ \ \isacommand{using}\isamarkupfalse%
\ {\isasymphi}{\isacharunderscore}{\kern0pt}type\ cfunc{\isacharunderscore}{\kern0pt}type{\isacharunderscore}{\kern0pt}def\ \isacommand{by}\isamarkupfalse%
\ auto\isanewline
\ \ \ \ \ \ \isacommand{then}\isamarkupfalse%
\ \isacommand{obtain}\isamarkupfalse%
\ a{\isadigit{1}}\ a{\isadigit{2}}\ \isakeyword{where}\ y{\isacharunderscore}{\kern0pt}def{\isacharbrackleft}{\kern0pt}type{\isacharunderscore}{\kern0pt}rule{\isacharbrackright}{\kern0pt}{\isacharcolon}{\kern0pt}\ {\isachardoublequoteopen}y\ {\isacharequal}{\kern0pt}\ {\isasymlangle}a{\isadigit{1}}{\isacharcomma}{\kern0pt}a{\isadigit{2}}{\isasymrangle}\ {\isasymand}\ a{\isadigit{1}}\ {\isasymin}\isactrlsub c\ A\ {\isasymand}\ a{\isadigit{2}}\ {\isasymin}\isactrlsub c\ A{\isachardoublequoteclose}\isanewline
\ \ \ \ \ \ \ \ \isacommand{using}\isamarkupfalse%
\ cart{\isacharunderscore}{\kern0pt}prod{\isacharunderscore}{\kern0pt}decomp\ \isacommand{by}\isamarkupfalse%
\ blast\isanewline
\ \ \ \ \ \ \isacommand{then}\isamarkupfalse%
\ \isacommand{have}\isamarkupfalse%
\ aua{\isacharcolon}{\kern0pt}\ {\isachardoublequoteopen}{\isacharparenleft}{\kern0pt}a{\isadigit{1}}\ {\isasymamalg}\ a{\isadigit{2}}{\isacharparenright}{\kern0pt}{\isacharcolon}{\kern0pt}\ {\isasymone}\ {\isasymCoprod}\ {\isasymone}\ {\isasymrightarrow}\ A{\isachardoublequoteclose}\isanewline
\ \ \ \ \ \ \ \ \isacommand{by}\isamarkupfalse%
\ {\isacharparenleft}{\kern0pt}typecheck{\isacharunderscore}{\kern0pt}cfuncs{\isacharcomma}{\kern0pt}\ simp\ add{\isacharcolon}{\kern0pt}\ y{\isacharunderscore}{\kern0pt}def{\isacharparenright}{\kern0pt}\ \ \ \ \ \isanewline
\ \ \ \ \isanewline
\ \ \ \ \ \ \isacommand{obtain}\isamarkupfalse%
\ f\ \isakeyword{where}\ f{\isacharunderscore}{\kern0pt}def{\isacharcolon}{\kern0pt}\ {\isachardoublequoteopen}f\ {\isacharequal}{\kern0pt}\ {\isacharparenleft}{\kern0pt}{\isacharparenleft}{\kern0pt}a{\isadigit{1}}\ {\isasymamalg}\ a{\isadigit{2}}{\isacharparenright}{\kern0pt}\ {\isasymcirc}\isactrlsub c\ case{\isacharunderscore}{\kern0pt}bool\ \ {\isasymcirc}\isactrlsub c\ left{\isacharunderscore}{\kern0pt}cart{\isacharunderscore}{\kern0pt}proj\ {\isasymOmega}\ {\isasymone}{\isacharparenright}{\kern0pt}\isactrlsup {\isasymsharp}{\isachardoublequoteclose}\ \isakeyword{and}\isanewline
\ \ \ \ \ \ \ \ \ \ \ \ \ \ \ \ \ \ \ \ \ f{\isacharunderscore}{\kern0pt}type{\isacharbrackleft}{\kern0pt}type{\isacharunderscore}{\kern0pt}rule{\isacharbrackright}{\kern0pt}{\isacharcolon}{\kern0pt}\ {\isachardoublequoteopen}f\ {\isasymin}\isactrlsub c\ A\isactrlbsup {\isasymOmega}\isactrlesup {\isachardoublequoteclose}\isanewline
\ \ \ \ \ \ \ \ \isacommand{by}\isamarkupfalse%
\ {\isacharparenleft}{\kern0pt}meson\ aua\ case{\isacharunderscore}{\kern0pt}bool{\isacharunderscore}{\kern0pt}type\ comp{\isacharunderscore}{\kern0pt}type\ left{\isacharunderscore}{\kern0pt}cart{\isacharunderscore}{\kern0pt}proj{\isacharunderscore}{\kern0pt}type\ transpose{\isacharunderscore}{\kern0pt}func{\isacharunderscore}{\kern0pt}type{\isacharparenright}{\kern0pt}\isanewline
\ \ \ \ \ \isacommand{have}\isamarkupfalse%
\ a{\isadigit{1}}{\isacharunderscore}{\kern0pt}is{\isacharcolon}{\kern0pt}\ {\isachardoublequoteopen}{\isacharparenleft}{\kern0pt}eval{\isacharunderscore}{\kern0pt}func\ A\ {\isasymOmega}\ {\isasymcirc}\isactrlsub c\ {\isasymlangle}{\isasymt}\ {\isasymcirc}\isactrlsub c\ {\isasymbeta}\isactrlbsub A\isactrlbsup {\isasymOmega}\isactrlesup \isactrlesub {\isacharcomma}{\kern0pt}\ id{\isacharparenleft}{\kern0pt}A\isactrlbsup {\isasymOmega}\isactrlesup {\isacharparenright}{\kern0pt}{\isasymrangle}{\isacharparenright}{\kern0pt}\ {\isasymcirc}\isactrlsub c\ f\ {\isacharequal}{\kern0pt}\ a{\isadigit{1}}{\isachardoublequoteclose}\isanewline
\ \ \ \ \ \isacommand{proof}\isamarkupfalse%
{\isacharminus}{\kern0pt}\isanewline
\ \ \ \ \ \ \ \isacommand{have}\isamarkupfalse%
\ {\isachardoublequoteopen}{\isacharparenleft}{\kern0pt}eval{\isacharunderscore}{\kern0pt}func\ A\ {\isasymOmega}\ {\isasymcirc}\isactrlsub c\ {\isasymlangle}{\isasymt}\ {\isasymcirc}\isactrlsub c\ {\isasymbeta}\isactrlbsub A\isactrlbsup {\isasymOmega}\isactrlesup \isactrlesub {\isacharcomma}{\kern0pt}\ id{\isacharparenleft}{\kern0pt}A\isactrlbsup {\isasymOmega}\isactrlesup {\isacharparenright}{\kern0pt}{\isasymrangle}{\isacharparenright}{\kern0pt}\ {\isasymcirc}\isactrlsub c\ f\ {\isacharequal}{\kern0pt}\ eval{\isacharunderscore}{\kern0pt}func\ A\ {\isasymOmega}\ {\isasymcirc}\isactrlsub c\ {\isasymlangle}{\isasymt}\ {\isasymcirc}\isactrlsub c\ {\isasymbeta}\isactrlbsub A\isactrlbsup {\isasymOmega}\isactrlesup \isactrlesub {\isacharcomma}{\kern0pt}\ id{\isacharparenleft}{\kern0pt}A\isactrlbsup {\isasymOmega}\isactrlesup {\isacharparenright}{\kern0pt}{\isasymrangle}\ {\isasymcirc}\isactrlsub c\ f{\isachardoublequoteclose}\isanewline
\ \ \ \ \ \ \ \ \ \isacommand{by}\isamarkupfalse%
\ {\isacharparenleft}{\kern0pt}typecheck{\isacharunderscore}{\kern0pt}cfuncs{\isacharcomma}{\kern0pt}\ simp\ add{\isacharcolon}{\kern0pt}\ comp{\isacharunderscore}{\kern0pt}associative{\isadigit{2}}{\isacharparenright}{\kern0pt}\isanewline
\ \ \ \ \ \ \ \isacommand{also}\isamarkupfalse%
\ \isacommand{have}\isamarkupfalse%
\ {\isachardoublequoteopen}{\isachardot}{\kern0pt}{\isachardot}{\kern0pt}{\isachardot}{\kern0pt}\ {\isacharequal}{\kern0pt}\ eval{\isacharunderscore}{\kern0pt}func\ A\ {\isasymOmega}\ {\isasymcirc}\isactrlsub c\ {\isasymlangle}{\isasymt}\ {\isasymcirc}\isactrlsub c\ {\isasymbeta}\isactrlbsub A\isactrlbsup {\isasymOmega}\isactrlesup \isactrlesub \ {\isasymcirc}\isactrlsub c\ f{\isacharcomma}{\kern0pt}\ id{\isacharparenleft}{\kern0pt}A\isactrlbsup {\isasymOmega}\isactrlesup {\isacharparenright}{\kern0pt}\ {\isasymcirc}\isactrlsub c\ f{\isasymrangle}{\isachardoublequoteclose}\isanewline
\ \ \ \ \ \ \ \ \ \isacommand{by}\isamarkupfalse%
\ {\isacharparenleft}{\kern0pt}typecheck{\isacharunderscore}{\kern0pt}cfuncs{\isacharcomma}{\kern0pt}\ simp\ add{\isacharcolon}{\kern0pt}\ cfunc{\isacharunderscore}{\kern0pt}prod{\isacharunderscore}{\kern0pt}comp\ comp{\isacharunderscore}{\kern0pt}associative{\isadigit{2}}{\isacharparenright}{\kern0pt}\isanewline
\ \ \ \ \ \ \ \isacommand{also}\isamarkupfalse%
\ \isacommand{have}\isamarkupfalse%
\ {\isachardoublequoteopen}{\isachardot}{\kern0pt}{\isachardot}{\kern0pt}{\isachardot}{\kern0pt}\ {\isacharequal}{\kern0pt}\ eval{\isacharunderscore}{\kern0pt}func\ A\ {\isasymOmega}\ {\isasymcirc}\isactrlsub c\ {\isasymlangle}{\isasymt}{\isacharcomma}{\kern0pt}\ f{\isasymrangle}{\isachardoublequoteclose}\isanewline
\ \ \ \ \ \ \ \ \ \isacommand{by}\isamarkupfalse%
\ {\isacharparenleft}{\kern0pt}metis\ cfunc{\isacharunderscore}{\kern0pt}type{\isacharunderscore}{\kern0pt}def\ f{\isacharunderscore}{\kern0pt}type\ id{\isacharunderscore}{\kern0pt}left{\isacharunderscore}{\kern0pt}unit\ id{\isacharunderscore}{\kern0pt}right{\isacharunderscore}{\kern0pt}unit\ id{\isacharunderscore}{\kern0pt}type\ one{\isacharunderscore}{\kern0pt}unique{\isacharunderscore}{\kern0pt}element\ terminal{\isacharunderscore}{\kern0pt}func{\isacharunderscore}{\kern0pt}comp\ terminal{\isacharunderscore}{\kern0pt}func{\isacharunderscore}{\kern0pt}type\ true{\isacharunderscore}{\kern0pt}func{\isacharunderscore}{\kern0pt}type{\isacharparenright}{\kern0pt}\isanewline
\ \ \ \ \ \ \ \isacommand{also}\isamarkupfalse%
\ \isacommand{have}\isamarkupfalse%
\ {\isachardoublequoteopen}{\isachardot}{\kern0pt}{\isachardot}{\kern0pt}{\isachardot}{\kern0pt}\ {\isacharequal}{\kern0pt}\ eval{\isacharunderscore}{\kern0pt}func\ A\ {\isasymOmega}\ {\isasymcirc}\isactrlsub c\ {\isasymlangle}id{\isacharparenleft}{\kern0pt}{\isasymOmega}{\isacharparenright}{\kern0pt}\ {\isasymcirc}\isactrlsub c\ {\isasymt}{\isacharcomma}{\kern0pt}\ f\ {\isasymcirc}\isactrlsub c\ id{\isacharparenleft}{\kern0pt}{\isasymone}{\isacharparenright}{\kern0pt}{\isasymrangle}{\isachardoublequoteclose}\isanewline
\ \ \ \ \ \ \ \ \ \isacommand{by}\isamarkupfalse%
\ {\isacharparenleft}{\kern0pt}typecheck{\isacharunderscore}{\kern0pt}cfuncs{\isacharcomma}{\kern0pt}\ simp\ add{\isacharcolon}{\kern0pt}\ id{\isacharunderscore}{\kern0pt}left{\isacharunderscore}{\kern0pt}unit{\isadigit{2}}\ id{\isacharunderscore}{\kern0pt}right{\isacharunderscore}{\kern0pt}unit{\isadigit{2}}{\isacharparenright}{\kern0pt}\isanewline
\ \ \ \ \ \ \ \isacommand{also}\isamarkupfalse%
\ \isacommand{have}\isamarkupfalse%
\ {\isachardoublequoteopen}{\isachardot}{\kern0pt}{\isachardot}{\kern0pt}{\isachardot}{\kern0pt}\ {\isacharequal}{\kern0pt}\ eval{\isacharunderscore}{\kern0pt}func\ A\ {\isasymOmega}\ {\isasymcirc}\isactrlsub c\ {\isacharparenleft}{\kern0pt}id{\isacharparenleft}{\kern0pt}{\isasymOmega}{\isacharparenright}{\kern0pt}\ {\isasymtimes}\isactrlsub f\ f{\isacharparenright}{\kern0pt}\ {\isasymcirc}\isactrlsub c\ {\isasymlangle}{\isasymt}{\isacharcomma}{\kern0pt}\ id{\isacharparenleft}{\kern0pt}{\isasymone}{\isacharparenright}{\kern0pt}{\isasymrangle}{\isachardoublequoteclose}\isanewline
\ \ \ \ \ \ \ \ \ \isacommand{by}\isamarkupfalse%
\ {\isacharparenleft}{\kern0pt}typecheck{\isacharunderscore}{\kern0pt}cfuncs{\isacharcomma}{\kern0pt}\ simp\ add{\isacharcolon}{\kern0pt}\ cfunc{\isacharunderscore}{\kern0pt}cross{\isacharunderscore}{\kern0pt}prod{\isacharunderscore}{\kern0pt}comp{\isacharunderscore}{\kern0pt}cfunc{\isacharunderscore}{\kern0pt}prod{\isacharparenright}{\kern0pt}\isanewline
\ \ \ \ \ \ \ \isacommand{also}\isamarkupfalse%
\ \isacommand{have}\isamarkupfalse%
\ {\isachardoublequoteopen}{\isachardot}{\kern0pt}{\isachardot}{\kern0pt}{\isachardot}{\kern0pt}\ {\isacharequal}{\kern0pt}\ {\isacharparenleft}{\kern0pt}eval{\isacharunderscore}{\kern0pt}func\ A\ {\isasymOmega}\ {\isasymcirc}\isactrlsub c\ {\isacharparenleft}{\kern0pt}id{\isacharparenleft}{\kern0pt}{\isasymOmega}{\isacharparenright}{\kern0pt}\ {\isasymtimes}\isactrlsub f\ f{\isacharparenright}{\kern0pt}{\isacharparenright}{\kern0pt}\ {\isasymcirc}\isactrlsub c\ {\isasymlangle}{\isasymt}{\isacharcomma}{\kern0pt}\ id{\isacharparenleft}{\kern0pt}{\isasymone}{\isacharparenright}{\kern0pt}{\isasymrangle}{\isachardoublequoteclose}\isanewline
\ \ \ \ \ \ \ \ \ \isacommand{using}\isamarkupfalse%
\ comp{\isacharunderscore}{\kern0pt}associative{\isadigit{2}}\ \isacommand{by}\isamarkupfalse%
\ {\isacharparenleft}{\kern0pt}typecheck{\isacharunderscore}{\kern0pt}cfuncs{\isacharcomma}{\kern0pt}\ blast{\isacharparenright}{\kern0pt}\isanewline
\ \ \ \ \ \ \ \isacommand{also}\isamarkupfalse%
\ \isacommand{have}\isamarkupfalse%
\ {\isachardoublequoteopen}{\isachardot}{\kern0pt}{\isachardot}{\kern0pt}{\isachardot}{\kern0pt}\ {\isacharequal}{\kern0pt}\ {\isacharparenleft}{\kern0pt}{\isacharparenleft}{\kern0pt}a{\isadigit{1}}\ {\isasymamalg}\ a{\isadigit{2}}{\isacharparenright}{\kern0pt}\ {\isasymcirc}\isactrlsub c\ case{\isacharunderscore}{\kern0pt}bool\ \ {\isasymcirc}\isactrlsub c\ left{\isacharunderscore}{\kern0pt}cart{\isacharunderscore}{\kern0pt}proj\ {\isasymOmega}\ {\isasymone}{\isacharparenright}{\kern0pt}\ {\isasymcirc}\isactrlsub c\ {\isasymlangle}{\isasymt}{\isacharcomma}{\kern0pt}\ id{\isacharparenleft}{\kern0pt}{\isasymone}{\isacharparenright}{\kern0pt}{\isasymrangle}{\isachardoublequoteclose}\isanewline
\ \ \ \ \ \ \ \ \ \isacommand{by}\isamarkupfalse%
\ {\isacharparenleft}{\kern0pt}typecheck{\isacharunderscore}{\kern0pt}cfuncs{\isacharcomma}{\kern0pt}\ metis\ \ aua\ f{\isacharunderscore}{\kern0pt}def\ flat{\isacharunderscore}{\kern0pt}cancels{\isacharunderscore}{\kern0pt}sharp\ inv{\isacharunderscore}{\kern0pt}transpose{\isacharunderscore}{\kern0pt}func{\isacharunderscore}{\kern0pt}def{\isadigit{3}}{\isacharparenright}{\kern0pt}\isanewline
\ \ \ \ \ \ \ \isacommand{also}\isamarkupfalse%
\ \isacommand{have}\isamarkupfalse%
\ {\isachardoublequoteopen}{\isachardot}{\kern0pt}{\isachardot}{\kern0pt}{\isachardot}{\kern0pt}\ {\isacharequal}{\kern0pt}\ {\isacharparenleft}{\kern0pt}a{\isadigit{1}}\ {\isasymamalg}\ a{\isadigit{2}}{\isacharparenright}{\kern0pt}\ {\isasymcirc}\isactrlsub c\ case{\isacharunderscore}{\kern0pt}bool\ \ {\isasymcirc}\isactrlsub c\ {\isasymt}{\isachardoublequoteclose}\isanewline
\ \ \ \ \ \ \ \ \ \isacommand{by}\isamarkupfalse%
\ {\isacharparenleft}{\kern0pt}typecheck{\isacharunderscore}{\kern0pt}cfuncs{\isacharcomma}{\kern0pt}\ smt\ case{\isacharunderscore}{\kern0pt}bool{\isacharunderscore}{\kern0pt}type\ aua\ comp{\isacharunderscore}{\kern0pt}associative{\isadigit{2}}\ left{\isacharunderscore}{\kern0pt}cart{\isacharunderscore}{\kern0pt}proj{\isacharunderscore}{\kern0pt}cfunc{\isacharunderscore}{\kern0pt}prod{\isacharparenright}{\kern0pt}\isanewline
\ \ \ \ \ \ \ \isacommand{also}\isamarkupfalse%
\ \isacommand{have}\isamarkupfalse%
\ {\isachardoublequoteopen}{\isachardot}{\kern0pt}{\isachardot}{\kern0pt}{\isachardot}{\kern0pt}\ {\isacharequal}{\kern0pt}\ {\isacharparenleft}{\kern0pt}a{\isadigit{1}}\ {\isasymamalg}\ a{\isadigit{2}}{\isacharparenright}{\kern0pt}\ {\isasymcirc}\isactrlsub c\ left{\isacharunderscore}{\kern0pt}coproj\ {\isasymone}\ {\isasymone}{\isachardoublequoteclose}\isanewline
\ \ \ \ \ \ \ \ \ \isacommand{by}\isamarkupfalse%
\ {\isacharparenleft}{\kern0pt}simp\ add{\isacharcolon}{\kern0pt}\ case{\isacharunderscore}{\kern0pt}bool{\isacharunderscore}{\kern0pt}true{\isacharparenright}{\kern0pt}\isanewline
\ \ \ \ \ \ \ \isacommand{also}\isamarkupfalse%
\ \isacommand{have}\isamarkupfalse%
\ {\isachardoublequoteopen}{\isachardot}{\kern0pt}{\isachardot}{\kern0pt}{\isachardot}{\kern0pt}\ {\isacharequal}{\kern0pt}\ a{\isadigit{1}}{\isachardoublequoteclose}\isanewline
\ \ \ \ \ \ \ \ \ \isacommand{using}\isamarkupfalse%
\ left{\isacharunderscore}{\kern0pt}coproj{\isacharunderscore}{\kern0pt}cfunc{\isacharunderscore}{\kern0pt}coprod\ y{\isacharunderscore}{\kern0pt}def\ \isacommand{by}\isamarkupfalse%
\ blast\isanewline
\ \ \ \ \ \ \ \isacommand{then}\isamarkupfalse%
\ \isacommand{show}\isamarkupfalse%
\ {\isacharquery}{\kern0pt}thesis\ \isacommand{using}\isamarkupfalse%
\ calculation\ \isacommand{by}\isamarkupfalse%
\ auto\isanewline
\ \ \ \ \ \isacommand{qed}\isamarkupfalse%
\isanewline
\ \ \ \ \ \isacommand{have}\isamarkupfalse%
\ a{\isadigit{2}}{\isacharunderscore}{\kern0pt}is{\isacharcolon}{\kern0pt}\ {\isachardoublequoteopen}{\isacharparenleft}{\kern0pt}eval{\isacharunderscore}{\kern0pt}func\ A\ {\isasymOmega}\ {\isasymcirc}\isactrlsub c\ {\isasymlangle}{\isasymf}\ {\isasymcirc}\isactrlsub c\ {\isasymbeta}\isactrlbsub A\isactrlbsup {\isasymOmega}\isactrlesup \isactrlesub {\isacharcomma}{\kern0pt}\ id{\isacharparenleft}{\kern0pt}A\isactrlbsup {\isasymOmega}\isactrlesup {\isacharparenright}{\kern0pt}{\isasymrangle}{\isacharparenright}{\kern0pt}\ {\isasymcirc}\isactrlsub c\ f\ {\isacharequal}{\kern0pt}\ a{\isadigit{2}}{\isachardoublequoteclose}\isanewline
\ \ \ \ \ \isacommand{proof}\isamarkupfalse%
{\isacharminus}{\kern0pt}\isanewline
\ \ \ \ \ \ \ \isacommand{have}\isamarkupfalse%
\ {\isachardoublequoteopen}{\isacharparenleft}{\kern0pt}eval{\isacharunderscore}{\kern0pt}func\ A\ {\isasymOmega}\ {\isasymcirc}\isactrlsub c\ {\isasymlangle}{\isasymf}\ {\isasymcirc}\isactrlsub c\ {\isasymbeta}\isactrlbsub A\isactrlbsup {\isasymOmega}\isactrlesup \isactrlesub {\isacharcomma}{\kern0pt}\ id{\isacharparenleft}{\kern0pt}A\isactrlbsup {\isasymOmega}\isactrlesup {\isacharparenright}{\kern0pt}{\isasymrangle}{\isacharparenright}{\kern0pt}\ {\isasymcirc}\isactrlsub c\ f\ {\isacharequal}{\kern0pt}\ eval{\isacharunderscore}{\kern0pt}func\ A\ {\isasymOmega}\ {\isasymcirc}\isactrlsub c\ {\isasymlangle}{\isasymf}\ {\isasymcirc}\isactrlsub c\ {\isasymbeta}\isactrlbsub A\isactrlbsup {\isasymOmega}\isactrlesup \isactrlesub {\isacharcomma}{\kern0pt}\ id{\isacharparenleft}{\kern0pt}A\isactrlbsup {\isasymOmega}\isactrlesup {\isacharparenright}{\kern0pt}{\isasymrangle}\ {\isasymcirc}\isactrlsub c\ f{\isachardoublequoteclose}\isanewline
\ \ \ \ \ \ \ \ \ \isacommand{by}\isamarkupfalse%
\ {\isacharparenleft}{\kern0pt}typecheck{\isacharunderscore}{\kern0pt}cfuncs{\isacharcomma}{\kern0pt}\ simp\ add{\isacharcolon}{\kern0pt}\ comp{\isacharunderscore}{\kern0pt}associative{\isadigit{2}}{\isacharparenright}{\kern0pt}\isanewline
\ \ \ \ \ \ \ \isacommand{also}\isamarkupfalse%
\ \isacommand{have}\isamarkupfalse%
\ {\isachardoublequoteopen}{\isachardot}{\kern0pt}{\isachardot}{\kern0pt}{\isachardot}{\kern0pt}\ {\isacharequal}{\kern0pt}\ eval{\isacharunderscore}{\kern0pt}func\ A\ {\isasymOmega}\ {\isasymcirc}\isactrlsub c\ {\isasymlangle}{\isasymf}\ {\isasymcirc}\isactrlsub c\ {\isasymbeta}\isactrlbsub A\isactrlbsup {\isasymOmega}\isactrlesup \isactrlesub \ {\isasymcirc}\isactrlsub c\ f{\isacharcomma}{\kern0pt}\ id{\isacharparenleft}{\kern0pt}A\isactrlbsup {\isasymOmega}\isactrlesup {\isacharparenright}{\kern0pt}\ {\isasymcirc}\isactrlsub c\ f{\isasymrangle}{\isachardoublequoteclose}\isanewline
\ \ \ \ \ \ \ \ \ \isacommand{by}\isamarkupfalse%
\ {\isacharparenleft}{\kern0pt}typecheck{\isacharunderscore}{\kern0pt}cfuncs{\isacharcomma}{\kern0pt}\ simp\ add{\isacharcolon}{\kern0pt}\ cfunc{\isacharunderscore}{\kern0pt}prod{\isacharunderscore}{\kern0pt}comp\ comp{\isacharunderscore}{\kern0pt}associative{\isadigit{2}}{\isacharparenright}{\kern0pt}\isanewline
\ \ \ \ \ \ \ \isacommand{also}\isamarkupfalse%
\ \isacommand{have}\isamarkupfalse%
\ {\isachardoublequoteopen}{\isachardot}{\kern0pt}{\isachardot}{\kern0pt}{\isachardot}{\kern0pt}\ {\isacharequal}{\kern0pt}\ eval{\isacharunderscore}{\kern0pt}func\ A\ {\isasymOmega}\ {\isasymcirc}\isactrlsub c\ {\isasymlangle}{\isasymf}{\isacharcomma}{\kern0pt}\ f{\isasymrangle}{\isachardoublequoteclose}\isanewline
\ \ \ \ \ \ \ \ \ \isacommand{by}\isamarkupfalse%
\ {\isacharparenleft}{\kern0pt}metis\ cfunc{\isacharunderscore}{\kern0pt}type{\isacharunderscore}{\kern0pt}def\ f{\isacharunderscore}{\kern0pt}type\ id{\isacharunderscore}{\kern0pt}left{\isacharunderscore}{\kern0pt}unit\ id{\isacharunderscore}{\kern0pt}right{\isacharunderscore}{\kern0pt}unit\ id{\isacharunderscore}{\kern0pt}type\ one{\isacharunderscore}{\kern0pt}unique{\isacharunderscore}{\kern0pt}element\ terminal{\isacharunderscore}{\kern0pt}func{\isacharunderscore}{\kern0pt}comp\ terminal{\isacharunderscore}{\kern0pt}func{\isacharunderscore}{\kern0pt}type\ false{\isacharunderscore}{\kern0pt}func{\isacharunderscore}{\kern0pt}type{\isacharparenright}{\kern0pt}\isanewline
\ \ \ \ \ \ \ \isacommand{also}\isamarkupfalse%
\ \isacommand{have}\isamarkupfalse%
\ {\isachardoublequoteopen}{\isachardot}{\kern0pt}{\isachardot}{\kern0pt}{\isachardot}{\kern0pt}\ {\isacharequal}{\kern0pt}\ eval{\isacharunderscore}{\kern0pt}func\ A\ {\isasymOmega}\ {\isasymcirc}\isactrlsub c\ {\isasymlangle}id{\isacharparenleft}{\kern0pt}{\isasymOmega}{\isacharparenright}{\kern0pt}\ {\isasymcirc}\isactrlsub c\ {\isasymf}{\isacharcomma}{\kern0pt}\ f\ {\isasymcirc}\isactrlsub c\ id{\isacharparenleft}{\kern0pt}{\isasymone}{\isacharparenright}{\kern0pt}{\isasymrangle}{\isachardoublequoteclose}\isanewline
\ \ \ \ \ \ \ \ \ \isacommand{by}\isamarkupfalse%
\ {\isacharparenleft}{\kern0pt}typecheck{\isacharunderscore}{\kern0pt}cfuncs{\isacharcomma}{\kern0pt}\ simp\ add{\isacharcolon}{\kern0pt}\ id{\isacharunderscore}{\kern0pt}left{\isacharunderscore}{\kern0pt}unit{\isadigit{2}}\ id{\isacharunderscore}{\kern0pt}right{\isacharunderscore}{\kern0pt}unit{\isadigit{2}}{\isacharparenright}{\kern0pt}\isanewline
\ \ \ \ \ \ \ \isacommand{also}\isamarkupfalse%
\ \isacommand{have}\isamarkupfalse%
\ {\isachardoublequoteopen}{\isachardot}{\kern0pt}{\isachardot}{\kern0pt}{\isachardot}{\kern0pt}\ {\isacharequal}{\kern0pt}\ eval{\isacharunderscore}{\kern0pt}func\ A\ {\isasymOmega}\ {\isasymcirc}\isactrlsub c\ {\isacharparenleft}{\kern0pt}id{\isacharparenleft}{\kern0pt}{\isasymOmega}{\isacharparenright}{\kern0pt}\ {\isasymtimes}\isactrlsub f\ f{\isacharparenright}{\kern0pt}\ {\isasymcirc}\isactrlsub c\ {\isasymlangle}{\isasymf}{\isacharcomma}{\kern0pt}\ id{\isacharparenleft}{\kern0pt}{\isasymone}{\isacharparenright}{\kern0pt}{\isasymrangle}{\isachardoublequoteclose}\isanewline
\ \ \ \ \ \ \ \ \ \isacommand{by}\isamarkupfalse%
\ {\isacharparenleft}{\kern0pt}typecheck{\isacharunderscore}{\kern0pt}cfuncs{\isacharcomma}{\kern0pt}\ simp\ add{\isacharcolon}{\kern0pt}\ cfunc{\isacharunderscore}{\kern0pt}cross{\isacharunderscore}{\kern0pt}prod{\isacharunderscore}{\kern0pt}comp{\isacharunderscore}{\kern0pt}cfunc{\isacharunderscore}{\kern0pt}prod{\isacharparenright}{\kern0pt}\isanewline
\ \ \ \ \ \ \ \isacommand{also}\isamarkupfalse%
\ \isacommand{have}\isamarkupfalse%
\ {\isachardoublequoteopen}{\isachardot}{\kern0pt}{\isachardot}{\kern0pt}{\isachardot}{\kern0pt}\ {\isacharequal}{\kern0pt}\ {\isacharparenleft}{\kern0pt}eval{\isacharunderscore}{\kern0pt}func\ A\ {\isasymOmega}\ {\isasymcirc}\isactrlsub c\ {\isacharparenleft}{\kern0pt}id{\isacharparenleft}{\kern0pt}{\isasymOmega}{\isacharparenright}{\kern0pt}\ {\isasymtimes}\isactrlsub f\ f{\isacharparenright}{\kern0pt}{\isacharparenright}{\kern0pt}\ {\isasymcirc}\isactrlsub c\ {\isasymlangle}{\isasymf}{\isacharcomma}{\kern0pt}\ id{\isacharparenleft}{\kern0pt}{\isasymone}{\isacharparenright}{\kern0pt}{\isasymrangle}{\isachardoublequoteclose}\isanewline
\ \ \ \ \ \ \ \ \ \isacommand{using}\isamarkupfalse%
\ comp{\isacharunderscore}{\kern0pt}associative{\isadigit{2}}\ \isacommand{by}\isamarkupfalse%
\ {\isacharparenleft}{\kern0pt}typecheck{\isacharunderscore}{\kern0pt}cfuncs{\isacharcomma}{\kern0pt}\ blast{\isacharparenright}{\kern0pt}\isanewline
\ \ \ \ \ \ \ \isacommand{also}\isamarkupfalse%
\ \isacommand{have}\isamarkupfalse%
\ {\isachardoublequoteopen}{\isachardot}{\kern0pt}{\isachardot}{\kern0pt}{\isachardot}{\kern0pt}\ {\isacharequal}{\kern0pt}\ {\isacharparenleft}{\kern0pt}{\isacharparenleft}{\kern0pt}a{\isadigit{1}}\ {\isasymamalg}\ a{\isadigit{2}}{\isacharparenright}{\kern0pt}\ {\isasymcirc}\isactrlsub c\ case{\isacharunderscore}{\kern0pt}bool\ \ {\isasymcirc}\isactrlsub c\ left{\isacharunderscore}{\kern0pt}cart{\isacharunderscore}{\kern0pt}proj\ {\isasymOmega}\ {\isasymone}{\isacharparenright}{\kern0pt}\ {\isasymcirc}\isactrlsub c\ {\isasymlangle}{\isasymf}{\isacharcomma}{\kern0pt}\ id{\isacharparenleft}{\kern0pt}{\isasymone}{\isacharparenright}{\kern0pt}{\isasymrangle}{\isachardoublequoteclose}\isanewline
\ \ \ \ \ \ \ \ \ \isacommand{by}\isamarkupfalse%
\ {\isacharparenleft}{\kern0pt}typecheck{\isacharunderscore}{\kern0pt}cfuncs{\isacharcomma}{\kern0pt}\ metis\ \ aua\ f{\isacharunderscore}{\kern0pt}def\ flat{\isacharunderscore}{\kern0pt}cancels{\isacharunderscore}{\kern0pt}sharp\ inv{\isacharunderscore}{\kern0pt}transpose{\isacharunderscore}{\kern0pt}func{\isacharunderscore}{\kern0pt}def{\isadigit{3}}{\isacharparenright}{\kern0pt}\isanewline
\ \ \ \ \ \ \ \isacommand{also}\isamarkupfalse%
\ \isacommand{have}\isamarkupfalse%
\ {\isachardoublequoteopen}{\isachardot}{\kern0pt}{\isachardot}{\kern0pt}{\isachardot}{\kern0pt}\ {\isacharequal}{\kern0pt}\ {\isacharparenleft}{\kern0pt}a{\isadigit{1}}\ {\isasymamalg}\ a{\isadigit{2}}{\isacharparenright}{\kern0pt}\ {\isasymcirc}\isactrlsub c\ case{\isacharunderscore}{\kern0pt}bool\ \ {\isasymcirc}\isactrlsub c\ {\isasymf}{\isachardoublequoteclose}\isanewline
\ \ \ \ \ \ \ \ \ \isacommand{by}\isamarkupfalse%
\ {\isacharparenleft}{\kern0pt}typecheck{\isacharunderscore}{\kern0pt}cfuncs{\isacharcomma}{\kern0pt}\ smt\ aua\ comp{\isacharunderscore}{\kern0pt}associative{\isadigit{2}}\ left{\isacharunderscore}{\kern0pt}cart{\isacharunderscore}{\kern0pt}proj{\isacharunderscore}{\kern0pt}cfunc{\isacharunderscore}{\kern0pt}prod{\isacharparenright}{\kern0pt}\isanewline
\ \ \ \ \ \ \ \isacommand{also}\isamarkupfalse%
\ \isacommand{have}\isamarkupfalse%
\ {\isachardoublequoteopen}{\isachardot}{\kern0pt}{\isachardot}{\kern0pt}{\isachardot}{\kern0pt}\ {\isacharequal}{\kern0pt}\ {\isacharparenleft}{\kern0pt}a{\isadigit{1}}\ {\isasymamalg}\ a{\isadigit{2}}{\isacharparenright}{\kern0pt}\ {\isasymcirc}\isactrlsub c\ right{\isacharunderscore}{\kern0pt}coproj\ {\isasymone}\ {\isasymone}{\isachardoublequoteclose}\isanewline
\ \ \ \ \ \ \ \ \ \isacommand{by}\isamarkupfalse%
\ {\isacharparenleft}{\kern0pt}simp\ add{\isacharcolon}{\kern0pt}\ case{\isacharunderscore}{\kern0pt}bool{\isacharunderscore}{\kern0pt}false{\isacharparenright}{\kern0pt}\isanewline
\ \ \ \ \ \ \ \isacommand{also}\isamarkupfalse%
\ \isacommand{have}\isamarkupfalse%
\ {\isachardoublequoteopen}{\isachardot}{\kern0pt}{\isachardot}{\kern0pt}{\isachardot}{\kern0pt}\ {\isacharequal}{\kern0pt}\ a{\isadigit{2}}{\isachardoublequoteclose}\isanewline
\ \ \ \ \ \ \ \ \ \isacommand{using}\isamarkupfalse%
\ right{\isacharunderscore}{\kern0pt}coproj{\isacharunderscore}{\kern0pt}cfunc{\isacharunderscore}{\kern0pt}coprod\ y{\isacharunderscore}{\kern0pt}def\ \isacommand{by}\isamarkupfalse%
\ blast\isanewline
\ \ \ \ \ \ \ \isacommand{then}\isamarkupfalse%
\ \isacommand{show}\isamarkupfalse%
\ {\isacharquery}{\kern0pt}thesis\ \isacommand{using}\isamarkupfalse%
\ calculation\ \isacommand{by}\isamarkupfalse%
\ auto\isanewline
\ \ \ \ \ \isacommand{qed}\isamarkupfalse%
\isanewline
\ \ \ \ \ \isacommand{have}\isamarkupfalse%
\ {\isachardoublequoteopen}{\isasymphi}\ {\isasymcirc}\isactrlsub c\ f\ \ {\isacharequal}{\kern0pt}\ {\isasymlangle}a{\isadigit{1}}{\isacharcomma}{\kern0pt}a{\isadigit{2}}{\isasymrangle}{\isachardoublequoteclose}\isanewline
\ \ \ \ \ \ \ \isacommand{unfolding}\isamarkupfalse%
\ {\isasymphi}{\isacharunderscore}{\kern0pt}def\ \isacommand{by}\isamarkupfalse%
\ {\isacharparenleft}{\kern0pt}typecheck{\isacharunderscore}{\kern0pt}cfuncs{\isacharcomma}{\kern0pt}\ simp\ add{\isacharcolon}{\kern0pt}\ a{\isadigit{1}}{\isacharunderscore}{\kern0pt}is\ a{\isadigit{2}}{\isacharunderscore}{\kern0pt}is\ cfunc{\isacharunderscore}{\kern0pt}prod{\isacharunderscore}{\kern0pt}comp{\isacharparenright}{\kern0pt}\isanewline
\ \ \ \ \ \isacommand{then}\isamarkupfalse%
\ \isacommand{show}\isamarkupfalse%
\ {\isachardoublequoteopen}{\isasymexists}x{\isachardot}{\kern0pt}\ x\ {\isasymin}\isactrlsub c\ domain\ {\isasymphi}\ {\isasymand}\ {\isasymphi}\ {\isasymcirc}\isactrlsub c\ x\ {\isacharequal}{\kern0pt}\ y{\isachardoublequoteclose}\isanewline
\ \ \ \ \ \ \ \isacommand{using}\isamarkupfalse%
\ {\isasymphi}{\isacharunderscore}{\kern0pt}type\ cfunc{\isacharunderscore}{\kern0pt}type{\isacharunderscore}{\kern0pt}def\ f{\isacharunderscore}{\kern0pt}type\ y{\isacharunderscore}{\kern0pt}def\ \isacommand{by}\isamarkupfalse%
\ auto\isanewline
\ \ \ \isacommand{qed}\isamarkupfalse%
\isanewline
\ \ \ \isacommand{then}\isamarkupfalse%
\ \isacommand{have}\isamarkupfalse%
\ {\isachardoublequoteopen}epimorphism{\isacharparenleft}{\kern0pt}{\isasymphi}{\isacharparenright}{\kern0pt}{\isachardoublequoteclose}\isanewline
\ \ \ \ \ \isacommand{by}\isamarkupfalse%
\ {\isacharparenleft}{\kern0pt}simp\ add{\isacharcolon}{\kern0pt}\ surjective{\isacharunderscore}{\kern0pt}is{\isacharunderscore}{\kern0pt}epimorphism{\isacharparenright}{\kern0pt}\isanewline
\ \ \ \isacommand{then}\isamarkupfalse%
\ \isacommand{have}\isamarkupfalse%
\ {\isachardoublequoteopen}isomorphism{\isacharparenleft}{\kern0pt}{\isasymphi}{\isacharparenright}{\kern0pt}{\isachardoublequoteclose}\isanewline
\ \ \ \ \ \isacommand{by}\isamarkupfalse%
\ {\isacharparenleft}{\kern0pt}simp\ add{\isacharcolon}{\kern0pt}\ {\isacartoucheopen}monomorphism\ {\isasymphi}{\isacartoucheclose}\ epi{\isacharunderscore}{\kern0pt}mon{\isacharunderscore}{\kern0pt}is{\isacharunderscore}{\kern0pt}iso{\isacharparenright}{\kern0pt}\isanewline
\ \ \ \isacommand{then}\isamarkupfalse%
\ \isacommand{show}\isamarkupfalse%
\ {\isacharquery}{\kern0pt}thesis\isanewline
\ \ \ \ \ \isacommand{using}\isamarkupfalse%
\ {\isasymphi}{\isacharunderscore}{\kern0pt}type\ is{\isacharunderscore}{\kern0pt}isomorphic{\isacharunderscore}{\kern0pt}def\ \isacommand{by}\isamarkupfalse%
\ blast\isanewline
\isacommand{qed}\isamarkupfalse%
%
\endisatagproof
{\isafoldproof}%
%
\isadelimproof
\isanewline
%
\endisadelimproof
%
\isadelimtheory
\isanewline
%
\endisadelimtheory
%
\isatagtheory
\isacommand{end}\isamarkupfalse%
%
\endisatagtheory
{\isafoldtheory}%
%
\isadelimtheory
%
\endisadelimtheory
%
\end{isabellebody}%
\endinput
%:%file=~/ETCS/HOL-ETCS/Exponential_Objects.thy%:%
%:%11=1%:%
%:%27=3%:%
%:%28=3%:%
%:%29=4%:%
%:%30=5%:%
%:%39=7%:%
%:%41=8%:%
%:%42=8%:%
%:%43=9%:%
%:%44=10%:%
%:%45=11%:%
%:%46=12%:%
%:%47=13%:%
%:%48=14%:%
%:%49=15%:%
%:%50=16%:%
%:%51=17%:%
%:%52=18%:%
%:%53=19%:%
%:%54=20%:%
%:%55=20%:%
%:%56=21%:%
%:%57=22%:%
%:%60=23%:%
%:%64=23%:%
%:%65=23%:%
%:%66=24%:%
%:%67=24%:%
%:%68=25%:%
%:%69=25%:%
%:%70=26%:%
%:%71=26%:%
%:%72=27%:%
%:%73=27%:%
%:%74=27%:%
%:%75=28%:%
%:%76=28%:%
%:%77=28%:%
%:%78=29%:%
%:%79=29%:%
%:%80=30%:%
%:%81=30%:%
%:%82=30%:%
%:%83=31%:%
%:%84=31%:%
%:%85=32%:%
%:%86=32%:%
%:%87=32%:%
%:%88=33%:%
%:%89=33%:%
%:%90=34%:%
%:%91=35%:%
%:%92=35%:%
%:%93=36%:%
%:%94=36%:%
%:%95=36%:%
%:%96=37%:%
%:%97=37%:%
%:%98=38%:%
%:%99=38%:%
%:%100=38%:%
%:%101=39%:%
%:%102=39%:%
%:%103=40%:%
%:%104=40%:%
%:%105=40%:%
%:%106=41%:%
%:%107=41%:%
%:%108=41%:%
%:%109=42%:%
%:%110=42%:%
%:%111=42%:%
%:%112=43%:%
%:%113=43%:%
%:%114=43%:%
%:%115=44%:%
%:%116=44%:%
%:%117=44%:%
%:%118=45%:%
%:%119=45%:%
%:%120=45%:%
%:%121=46%:%
%:%131=48%:%
%:%133=49%:%
%:%134=49%:%
%:%135=50%:%
%:%138=51%:%
%:%142=51%:%
%:%143=51%:%
%:%148=51%:%
%:%151=52%:%
%:%152=53%:%
%:%153=53%:%
%:%154=54%:%
%:%155=55%:%
%:%158=56%:%
%:%162=56%:%
%:%163=56%:%
%:%164=57%:%
%:%165=57%:%
%:%179=59%:%
%:%191=61%:%
%:%193=62%:%
%:%194=62%:%
%:%195=63%:%
%:%196=64%:%
%:%197=65%:%
%:%198=65%:%
%:%199=66%:%
%:%200=67%:%
%:%203=68%:%
%:%207=68%:%
%:%208=68%:%
%:%209=68%:%
%:%214=68%:%
%:%217=69%:%
%:%218=70%:%
%:%219=70%:%
%:%220=71%:%
%:%221=72%:%
%:%224=73%:%
%:%228=73%:%
%:%229=73%:%
%:%230=73%:%
%:%235=73%:%
%:%238=74%:%
%:%239=75%:%
%:%240=75%:%
%:%241=76%:%
%:%244=77%:%
%:%248=77%:%
%:%249=77%:%
%:%258=79%:%
%:%260=80%:%
%:%261=80%:%
%:%262=81%:%
%:%263=82%:%
%:%266=83%:%
%:%270=83%:%
%:%271=83%:%
%:%272=83%:%
%:%281=85%:%
%:%283=86%:%
%:%284=86%:%
%:%285=87%:%
%:%286=88%:%
%:%293=89%:%
%:%294=89%:%
%:%295=90%:%
%:%296=90%:%
%:%297=91%:%
%:%298=91%:%
%:%299=91%:%
%:%300=92%:%
%:%301=92%:%
%:%302=93%:%
%:%303=93%:%
%:%304=94%:%
%:%305=94%:%
%:%306=95%:%
%:%307=96%:%
%:%308=96%:%
%:%309=97%:%
%:%310=97%:%
%:%311=97%:%
%:%312=98%:%
%:%313=98%:%
%:%314=99%:%
%:%315=99%:%
%:%316=99%:%
%:%317=100%:%
%:%318=100%:%
%:%319=101%:%
%:%320=101%:%
%:%321=101%:%
%:%322=102%:%
%:%323=102%:%
%:%324=103%:%
%:%325=103%:%
%:%326=104%:%
%:%327=104%:%
%:%328=105%:%
%:%329=105%:%
%:%330=105%:%
%:%331=106%:%
%:%337=106%:%
%:%340=107%:%
%:%341=108%:%
%:%342=108%:%
%:%343=109%:%
%:%346=110%:%
%:%350=110%:%
%:%351=110%:%
%:%360=112%:%
%:%362=113%:%
%:%363=113%:%
%:%364=114%:%
%:%367=115%:%
%:%371=115%:%
%:%372=115%:%
%:%373=115%:%
%:%378=115%:%
%:%381=116%:%
%:%382=116%:%
%:%383=117%:%
%:%384=118%:%
%:%387=119%:%
%:%391=119%:%
%:%392=119%:%
%:%397=119%:%
%:%400=120%:%
%:%401=121%:%
%:%402=121%:%
%:%403=122%:%
%:%404=123%:%
%:%405=124%:%
%:%408=125%:%
%:%412=125%:%
%:%413=125%:%
%:%414=125%:%
%:%428=127%:%
%:%440=129%:%
%:%442=130%:%
%:%443=130%:%
%:%444=131%:%
%:%445=132%:%
%:%446=133%:%
%:%447=133%:%
%:%448=134%:%
%:%449=135%:%
%:%452=136%:%
%:%456=136%:%
%:%457=136%:%
%:%458=137%:%
%:%459=137%:%
%:%460=138%:%
%:%461=138%:%
%:%462=139%:%
%:%463=139%:%
%:%464=139%:%
%:%465=140%:%
%:%466=140%:%
%:%467=141%:%
%:%468=141%:%
%:%469=142%:%
%:%470=142%:%
%:%471=143%:%
%:%472=143%:%
%:%473=143%:%
%:%474=144%:%
%:%475=144%:%
%:%476=145%:%
%:%482=145%:%
%:%485=146%:%
%:%486=147%:%
%:%487=147%:%
%:%488=148%:%
%:%489=149%:%
%:%492=150%:%
%:%496=150%:%
%:%497=150%:%
%:%502=150%:%
%:%505=151%:%
%:%506=152%:%
%:%507=152%:%
%:%508=153%:%
%:%509=154%:%
%:%512=155%:%
%:%516=155%:%
%:%517=155%:%
%:%526=157%:%
%:%528=158%:%
%:%529=158%:%
%:%530=159%:%
%:%531=160%:%
%:%534=161%:%
%:%538=161%:%
%:%539=161%:%
%:%540=162%:%
%:%541=162%:%
%:%550=164%:%
%:%552=165%:%
%:%553=165%:%
%:%554=166%:%
%:%557=167%:%
%:%561=167%:%
%:%562=167%:%
%:%563=167%:%
%:%572=169%:%
%:%574=170%:%
%:%575=170%:%
%:%576=171%:%
%:%583=172%:%
%:%584=172%:%
%:%585=173%:%
%:%586=173%:%
%:%587=174%:%
%:%588=174%:%
%:%589=174%:%
%:%590=175%:%
%:%591=175%:%
%:%592=176%:%
%:%593=176%:%
%:%594=177%:%
%:%595=177%:%
%:%596=178%:%
%:%597=178%:%
%:%598=178%:%
%:%599=179%:%
%:%600=179%:%
%:%601=179%:%
%:%602=180%:%
%:%608=180%:%
%:%611=181%:%
%:%612=182%:%
%:%613=182%:%
%:%614=183%:%
%:%615=184%:%
%:%618=185%:%
%:%622=185%:%
%:%623=185%:%
%:%628=185%:%
%:%631=186%:%
%:%632=187%:%
%:%633=187%:%
%:%634=188%:%
%:%635=189%:%
%:%642=190%:%
%:%643=190%:%
%:%644=191%:%
%:%645=192%:%
%:%646=192%:%
%:%647=193%:%
%:%648=193%:%
%:%649=193%:%
%:%650=194%:%
%:%651=194%:%
%:%652=194%:%
%:%653=195%:%
%:%654=195%:%
%:%655=195%:%
%:%656=196%:%
%:%657=196%:%
%:%658=196%:%
%:%659=197%:%
%:%660=197%:%
%:%661=197%:%
%:%662=198%:%
%:%663=198%:%
%:%664=198%:%
%:%665=199%:%
%:%666=199%:%
%:%667=199%:%
%:%668=200%:%
%:%674=200%:%
%:%677=201%:%
%:%678=202%:%
%:%679=202%:%
%:%680=203%:%
%:%681=204%:%
%:%682=205%:%
%:%683=206%:%
%:%690=207%:%
%:%691=207%:%
%:%692=208%:%
%:%693=208%:%
%:%694=209%:%
%:%695=209%:%
%:%696=209%:%
%:%697=210%:%
%:%698=210%:%
%:%699=211%:%
%:%700=211%:%
%:%701=211%:%
%:%702=212%:%
%:%703=212%:%
%:%704=213%:%
%:%705=213%:%
%:%706=214%:%
%:%707=214%:%
%:%708=215%:%
%:%709=215%:%
%:%710=215%:%
%:%711=216%:%
%:%712=216%:%
%:%713=216%:%
%:%714=217%:%
%:%715=217%:%
%:%716=218%:%
%:%722=218%:%
%:%725=219%:%
%:%726=220%:%
%:%727=220%:%
%:%728=221%:%
%:%729=222%:%
%:%730=223%:%
%:%733=224%:%
%:%737=224%:%
%:%738=224%:%
%:%739=225%:%
%:%740=225%:%
%:%741=226%:%
%:%742=226%:%
%:%743=227%:%
%:%744=227%:%
%:%745=228%:%
%:%746=228%:%
%:%747=229%:%
%:%748=229%:%
%:%749=230%:%
%:%750=230%:%
%:%751=231%:%
%:%752=231%:%
%:%753=232%:%
%:%754=232%:%
%:%755=233%:%
%:%756=233%:%
%:%757=233%:%
%:%758=234%:%
%:%759=234%:%
%:%760=235%:%
%:%761=235%:%
%:%762=235%:%
%:%763=236%:%
%:%764=236%:%
%:%765=237%:%
%:%766=237%:%
%:%767=237%:%
%:%768=237%:%
%:%769=238%:%
%:%770=238%:%
%:%771=238%:%
%:%772=239%:%
%:%773=239%:%
%:%774=240%:%
%:%775=240%:%
%:%776=240%:%
%:%777=241%:%
%:%778=241%:%
%:%779=242%:%
%:%780=242%:%
%:%781=242%:%
%:%782=243%:%
%:%783=243%:%
%:%784=244%:%
%:%790=244%:%
%:%793=245%:%
%:%794=246%:%
%:%795=246%:%
%:%796=247%:%
%:%803=248%:%
%:%804=248%:%
%:%805=249%:%
%:%806=249%:%
%:%807=250%:%
%:%808=250%:%
%:%809=250%:%
%:%810=251%:%
%:%811=251%:%
%:%812=252%:%
%:%813=252%:%
%:%814=252%:%
%:%815=253%:%
%:%816=253%:%
%:%817=253%:%
%:%818=254%:%
%:%819=255%:%
%:%820=255%:%
%:%821=256%:%
%:%822=256%:%
%:%823=257%:%
%:%824=257%:%
%:%825=258%:%
%:%826=258%:%
%:%827=259%:%
%:%828=259%:%
%:%829=260%:%
%:%830=260%:%
%:%831=261%:%
%:%832=261%:%
%:%833=262%:%
%:%834=263%:%
%:%835=263%:%
%:%836=264%:%
%:%837=264%:%
%:%838=264%:%
%:%839=265%:%
%:%840=266%:%
%:%841=266%:%
%:%842=267%:%
%:%843=267%:%
%:%844=268%:%
%:%845=269%:%
%:%846=269%:%
%:%847=270%:%
%:%848=270%:%
%:%849=271%:%
%:%850=272%:%
%:%851=272%:%
%:%852=273%:%
%:%853=273%:%
%:%854=274%:%
%:%855=274%:%
%:%856=275%:%
%:%857=275%:%
%:%858=276%:%
%:%859=276%:%
%:%860=276%:%
%:%861=277%:%
%:%862=277%:%
%:%863=277%:%
%:%864=278%:%
%:%865=278%:%
%:%866=278%:%
%:%867=279%:%
%:%868=279%:%
%:%869=280%:%
%:%870=280%:%
%:%871=280%:%
%:%872=281%:%
%:%873=281%:%
%:%874=281%:%
%:%875=282%:%
%:%876=282%:%
%:%877=282%:%
%:%878=283%:%
%:%879=283%:%
%:%880=284%:%
%:%881=284%:%
%:%882=284%:%
%:%883=285%:%
%:%884=285%:%
%:%885=285%:%
%:%886=286%:%
%:%887=286%:%
%:%888=287%:%
%:%889=287%:%
%:%890=287%:%
%:%891=288%:%
%:%892=288%:%
%:%893=289%:%
%:%894=289%:%
%:%895=289%:%
%:%896=290%:%
%:%897=290%:%
%:%898=291%:%
%:%899=291%:%
%:%900=291%:%
%:%901=292%:%
%:%902=292%:%
%:%903=293%:%
%:%904=293%:%
%:%905=294%:%
%:%906=294%:%
%:%907=295%:%
%:%908=295%:%
%:%909=296%:%
%:%910=296%:%
%:%911=296%:%
%:%912=297%:%
%:%913=297%:%
%:%914=298%:%
%:%924=300%:%
%:%925=301%:%
%:%927=302%:%
%:%928=302%:%
%:%929=303%:%
%:%930=304%:%
%:%931=305%:%
%:%932=306%:%
%:%935=307%:%
%:%939=307%:%
%:%940=307%:%
%:%941=308%:%
%:%942=308%:%
%:%943=309%:%
%:%944=309%:%
%:%945=310%:%
%:%946=310%:%
%:%947=311%:%
%:%948=311%:%
%:%949=312%:%
%:%950=312%:%
%:%951=313%:%
%:%952=313%:%
%:%953=314%:%
%:%954=315%:%
%:%955=315%:%
%:%956=316%:%
%:%957=316%:%
%:%958=317%:%
%:%959=317%:%
%:%960=318%:%
%:%961=318%:%
%:%962=318%:%
%:%963=319%:%
%:%964=319%:%
%:%965=320%:%
%:%966=320%:%
%:%967=320%:%
%:%968=321%:%
%:%969=321%:%
%:%970=322%:%
%:%971=322%:%
%:%972=322%:%
%:%973=323%:%
%:%974=323%:%
%:%975=324%:%
%:%976=324%:%
%:%977=324%:%
%:%978=325%:%
%:%979=325%:%
%:%980=326%:%
%:%981=326%:%
%:%982=327%:%
%:%983=327%:%
%:%984=328%:%
%:%985=328%:%
%:%986=329%:%
%:%987=330%:%
%:%988=330%:%
%:%989=330%:%
%:%990=331%:%
%:%991=331%:%
%:%992=331%:%
%:%993=332%:%
%:%994=333%:%
%:%995=333%:%
%:%996=334%:%
%:%997=334%:%
%:%998=335%:%
%:%999=335%:%
%:%1000=336%:%
%:%1001=336%:%
%:%1002=336%:%
%:%1003=337%:%
%:%1004=337%:%
%:%1005=337%:%
%:%1006=338%:%
%:%1007=338%:%
%:%1008=338%:%
%:%1009=339%:%
%:%1010=339%:%
%:%1011=339%:%
%:%1012=340%:%
%:%1013=340%:%
%:%1014=340%:%
%:%1015=341%:%
%:%1016=341%:%
%:%1017=341%:%
%:%1018=342%:%
%:%1019=342%:%
%:%1020=343%:%
%:%1021=343%:%
%:%1022=343%:%
%:%1023=344%:%
%:%1024=344%:%
%:%1025=344%:%
%:%1026=345%:%
%:%1027=345%:%
%:%1028=345%:%
%:%1029=346%:%
%:%1030=346%:%
%:%1031=347%:%
%:%1032=347%:%
%:%1033=347%:%
%:%1034=348%:%
%:%1035=348%:%
%:%1036=348%:%
%:%1037=349%:%
%:%1038=349%:%
%:%1039=349%:%
%:%1040=350%:%
%:%1041=350%:%
%:%1042=351%:%
%:%1043=351%:%
%:%1044=352%:%
%:%1045=352%:%
%:%1046=352%:%
%:%1047=353%:%
%:%1048=353%:%
%:%1049=353%:%
%:%1050=354%:%
%:%1051=354%:%
%:%1052=354%:%
%:%1053=355%:%
%:%1054=355%:%
%:%1055=356%:%
%:%1056=356%:%
%:%1057=356%:%
%:%1058=357%:%
%:%1059=357%:%
%:%1060=357%:%
%:%1061=358%:%
%:%1076=360%:%
%:%1080=362%:%
%:%1090=364%:%
%:%1091=364%:%
%:%1092=365%:%
%:%1093=366%:%
%:%1094=367%:%
%:%1095=367%:%
%:%1096=368%:%
%:%1097=369%:%
%:%1100=370%:%
%:%1104=370%:%
%:%1105=370%:%
%:%1106=370%:%
%:%1107=370%:%
%:%1112=370%:%
%:%1115=371%:%
%:%1116=372%:%
%:%1117=372%:%
%:%1118=373%:%
%:%1119=374%:%
%:%1122=375%:%
%:%1126=375%:%
%:%1127=375%:%
%:%1128=375%:%
%:%1133=375%:%
%:%1136=376%:%
%:%1137=377%:%
%:%1138=377%:%
%:%1139=378%:%
%:%1140=379%:%
%:%1141=380%:%
%:%1148=381%:%
%:%1149=381%:%
%:%1150=382%:%
%:%1151=382%:%
%:%1152=383%:%
%:%1153=383%:%
%:%1154=383%:%
%:%1155=384%:%
%:%1156=384%:%
%:%1157=384%:%
%:%1158=385%:%
%:%1159=385%:%
%:%1160=385%:%
%:%1161=386%:%
%:%1162=386%:%
%:%1163=386%:%
%:%1164=387%:%
%:%1165=387%:%
%:%1166=387%:%
%:%1167=388%:%
%:%1168=388%:%
%:%1169=388%:%
%:%1170=389%:%
%:%1171=389%:%
%:%1172=390%:%
%:%1173=390%:%
%:%1174=390%:%
%:%1175=391%:%
%:%1176=391%:%
%:%1177=392%:%
%:%1192=394%:%
%:%1202=396%:%
%:%1203=396%:%
%:%1204=397%:%
%:%1205=398%:%
%:%1206=399%:%
%:%1207=399%:%
%:%1208=400%:%
%:%1209=401%:%
%:%1212=402%:%
%:%1216=402%:%
%:%1217=402%:%
%:%1218=402%:%
%:%1219=403%:%
%:%1220=403%:%
%:%1225=403%:%
%:%1228=404%:%
%:%1229=405%:%
%:%1230=405%:%
%:%1231=406%:%
%:%1232=407%:%
%:%1235=408%:%
%:%1239=408%:%
%:%1240=408%:%
%:%1241=409%:%
%:%1242=409%:%
%:%1247=409%:%
%:%1250=410%:%
%:%1251=411%:%
%:%1252=411%:%
%:%1253=412%:%
%:%1254=413%:%
%:%1261=414%:%
%:%1262=414%:%
%:%1263=415%:%
%:%1264=415%:%
%:%1265=416%:%
%:%1266=416%:%
%:%1267=417%:%
%:%1268=418%:%
%:%1269=418%:%
%:%1270=419%:%
%:%1271=419%:%
%:%1272=419%:%
%:%1273=420%:%
%:%1274=420%:%
%:%1275=420%:%
%:%1276=421%:%
%:%1277=421%:%
%:%1278=422%:%
%:%1279=422%:%
%:%1280=422%:%
%:%1281=423%:%
%:%1282=423%:%
%:%1283=423%:%
%:%1284=424%:%
%:%1285=424%:%
%:%1286=424%:%
%:%1287=425%:%
%:%1288=425%:%
%:%1289=426%:%
%:%1295=426%:%
%:%1298=427%:%
%:%1299=428%:%
%:%1300=428%:%
%:%1301=429%:%
%:%1302=430%:%
%:%1309=431%:%
%:%1310=431%:%
%:%1311=432%:%
%:%1312=432%:%
%:%1313=433%:%
%:%1314=433%:%
%:%1315=434%:%
%:%1316=434%:%
%:%1317=434%:%
%:%1318=435%:%
%:%1319=435%:%
%:%1320=436%:%
%:%1321=436%:%
%:%1322=437%:%
%:%1323=438%:%
%:%1324=438%:%
%:%1325=439%:%
%:%1326=439%:%
%:%1327=439%:%
%:%1328=440%:%
%:%1329=440%:%
%:%1330=440%:%
%:%1331=441%:%
%:%1332=441%:%
%:%1333=441%:%
%:%1334=442%:%
%:%1335=442%:%
%:%1336=442%:%
%:%1337=443%:%
%:%1338=443%:%
%:%1339=443%:%
%:%1340=444%:%
%:%1341=444%:%
%:%1342=445%:%
%:%1343=445%:%
%:%1344=445%:%
%:%1345=446%:%
%:%1346=446%:%
%:%1347=447%:%
%:%1348=447%:%
%:%1349=447%:%
%:%1350=448%:%
%:%1351=448%:%
%:%1352=449%:%
%:%1353=449%:%
%:%1354=449%:%
%:%1355=450%:%
%:%1356=450%:%
%:%1357=450%:%
%:%1358=451%:%
%:%1359=451%:%
%:%1360=451%:%
%:%1361=452%:%
%:%1362=452%:%
%:%1363=453%:%
%:%1364=453%:%
%:%1365=453%:%
%:%1366=454%:%
%:%1367=454%:%
%:%1368=454%:%
%:%1369=455%:%
%:%1384=457%:%
%:%1394=459%:%
%:%1395=459%:%
%:%1396=460%:%
%:%1397=461%:%
%:%1398=462%:%
%:%1399=462%:%
%:%1400=463%:%
%:%1403=464%:%
%:%1407=464%:%
%:%1408=464%:%
%:%1409=464%:%
%:%1414=464%:%
%:%1417=465%:%
%:%1418=466%:%
%:%1419=466%:%
%:%1420=467%:%
%:%1421=468%:%
%:%1422=469%:%
%:%1423=469%:%
%:%1424=470%:%
%:%1425=471%:%
%:%1426=472%:%
%:%1429=473%:%
%:%1433=473%:%
%:%1434=473%:%
%:%1435=473%:%
%:%1436=474%:%
%:%1437=474%:%
%:%1442=474%:%
%:%1445=475%:%
%:%1446=476%:%
%:%1447=476%:%
%:%1448=477%:%
%:%1449=478%:%
%:%1450=479%:%
%:%1457=480%:%
%:%1458=480%:%
%:%1459=481%:%
%:%1460=481%:%
%:%1461=482%:%
%:%1462=482%:%
%:%1463=482%:%
%:%1464=483%:%
%:%1465=483%:%
%:%1466=483%:%
%:%1467=484%:%
%:%1468=484%:%
%:%1469=484%:%
%:%1470=485%:%
%:%1476=485%:%
%:%1479=486%:%
%:%1480=487%:%
%:%1481=487%:%
%:%1482=488%:%
%:%1483=489%:%
%:%1484=490%:%
%:%1485=491%:%
%:%1492=492%:%
%:%1493=492%:%
%:%1494=493%:%
%:%1495=493%:%
%:%1496=494%:%
%:%1497=494%:%
%:%1498=494%:%
%:%1499=495%:%
%:%1500=495%:%
%:%1501=495%:%
%:%1502=496%:%
%:%1503=496%:%
%:%1504=496%:%
%:%1505=497%:%
%:%1506=497%:%
%:%1507=497%:%
%:%1508=498%:%
%:%1509=498%:%
%:%1510=498%:%
%:%1511=499%:%
%:%1512=499%:%
%:%1513=499%:%
%:%1514=500%:%
%:%1515=500%:%
%:%1516=500%:%
%:%1517=501%:%
%:%1518=501%:%
%:%1519=501%:%
%:%1520=502%:%
%:%1521=502%:%
%:%1522=502%:%
%:%1523=503%:%
%:%1524=503%:%
%:%1525=503%:%
%:%1526=504%:%
%:%1527=504%:%
%:%1528=504%:%
%:%1529=505%:%
%:%1530=505%:%
%:%1531=505%:%
%:%1532=506%:%
%:%1533=506%:%
%:%1534=506%:%
%:%1535=507%:%
%:%1536=507%:%
%:%1537=507%:%
%:%1538=508%:%
%:%1539=508%:%
%:%1540=509%:%
%:%1546=509%:%
%:%1549=510%:%
%:%1550=511%:%
%:%1551=511%:%
%:%1552=512%:%
%:%1553=513%:%
%:%1554=514%:%
%:%1557=515%:%
%:%1561=515%:%
%:%1562=515%:%
%:%1563=516%:%
%:%1564=516%:%
%:%1565=517%:%
%:%1566=517%:%
%:%1567=518%:%
%:%1568=518%:%
%:%1569=519%:%
%:%1570=519%:%
%:%1571=520%:%
%:%1572=520%:%
%:%1573=520%:%
%:%1574=521%:%
%:%1575=521%:%
%:%1576=522%:%
%:%1577=522%:%
%:%1578=522%:%
%:%1579=523%:%
%:%1580=523%:%
%:%1581=524%:%
%:%1582=524%:%
%:%1583=525%:%
%:%1584=525%:%
%:%1585=526%:%
%:%1586=526%:%
%:%1587=526%:%
%:%1588=527%:%
%:%1589=527%:%
%:%1590=528%:%
%:%1591=528%:%
%:%1592=528%:%
%:%1593=529%:%
%:%1594=529%:%
%:%1595=529%:%
%:%1596=530%:%
%:%1597=530%:%
%:%1598=530%:%
%:%1599=531%:%
%:%1600=531%:%
%:%1601=531%:%
%:%1602=532%:%
%:%1603=532%:%
%:%1604=532%:%
%:%1605=533%:%
%:%1606=533%:%
%:%1607=533%:%
%:%1608=534%:%
%:%1609=534%:%
%:%1610=534%:%
%:%1611=535%:%
%:%1612=535%:%
%:%1613=535%:%
%:%1614=536%:%
%:%1615=536%:%
%:%1616=536%:%
%:%1617=537%:%
%:%1618=537%:%
%:%1619=537%:%
%:%1620=538%:%
%:%1621=538%:%
%:%1622=538%:%
%:%1623=539%:%
%:%1624=539%:%
%:%1625=539%:%
%:%1626=540%:%
%:%1627=540%:%
%:%1628=540%:%
%:%1629=541%:%
%:%1630=541%:%
%:%1631=542%:%
%:%1632=542%:%
%:%1633=543%:%
%:%1634=543%:%
%:%1635=543%:%
%:%1636=544%:%
%:%1637=544%:%
%:%1638=544%:%
%:%1639=545%:%
%:%1645=545%:%
%:%1648=546%:%
%:%1649=547%:%
%:%1650=547%:%
%:%1651=548%:%
%:%1652=549%:%
%:%1655=550%:%
%:%1659=550%:%
%:%1660=550%:%
%:%1661=551%:%
%:%1662=551%:%
%:%1663=552%:%
%:%1664=552%:%
%:%1665=553%:%
%:%1666=554%:%
%:%1667=554%:%
%:%1668=555%:%
%:%1669=555%:%
%:%1670=556%:%
%:%1671=556%:%
%:%1672=557%:%
%:%1673=557%:%
%:%1674=557%:%
%:%1675=558%:%
%:%1676=558%:%
%:%1677=559%:%
%:%1678=559%:%
%:%1679=560%:%
%:%1680=561%:%
%:%1681=561%:%
%:%1682=562%:%
%:%1683=562%:%
%:%1684=562%:%
%:%1685=563%:%
%:%1686=563%:%
%:%1687=563%:%
%:%1688=564%:%
%:%1689=564%:%
%:%1690=564%:%
%:%1691=565%:%
%:%1692=565%:%
%:%1693=566%:%
%:%1694=566%:%
%:%1695=566%:%
%:%1696=567%:%
%:%1697=567%:%
%:%1698=568%:%
%:%1699=568%:%
%:%1700=568%:%
%:%1701=569%:%
%:%1702=569%:%
%:%1703=570%:%
%:%1704=570%:%
%:%1705=570%:%
%:%1706=571%:%
%:%1707=571%:%
%:%1708=572%:%
%:%1709=572%:%
%:%1710=572%:%
%:%1711=573%:%
%:%1712=573%:%
%:%1713=574%:%
%:%1714=574%:%
%:%1715=574%:%
%:%1716=575%:%
%:%1717=575%:%
%:%1718=576%:%
%:%1719=576%:%
%:%1720=576%:%
%:%1721=577%:%
%:%1722=577%:%
%:%1723=577%:%
%:%1724=578%:%
%:%1725=578%:%
%:%1726=578%:%
%:%1727=579%:%
%:%1728=579%:%
%:%1729=579%:%
%:%1730=580%:%
%:%1731=580%:%
%:%1732=580%:%
%:%1733=581%:%
%:%1734=581%:%
%:%1735=581%:%
%:%1736=582%:%
%:%1737=582%:%
%:%1738=582%:%
%:%1739=583%:%
%:%1740=583%:%
%:%1741=583%:%
%:%1742=584%:%
%:%1743=584%:%
%:%1744=584%:%
%:%1745=585%:%
%:%1746=585%:%
%:%1747=585%:%
%:%1748=586%:%
%:%1749=586%:%
%:%1750=586%:%
%:%1751=587%:%
%:%1752=587%:%
%:%1753=587%:%
%:%1754=588%:%
%:%1755=588%:%
%:%1756=588%:%
%:%1757=589%:%
%:%1758=589%:%
%:%1759=589%:%
%:%1760=590%:%
%:%1761=590%:%
%:%1762=590%:%
%:%1763=591%:%
%:%1764=591%:%
%:%1765=592%:%
%:%1766=592%:%
%:%1767=592%:%
%:%1768=593%:%
%:%1769=594%:%
%:%1770=594%:%
%:%1771=594%:%
%:%1772=595%:%
%:%1773=595%:%
%:%1774=596%:%
%:%1775=596%:%
%:%1776=596%:%
%:%1778=598%:%
%:%1779=599%:%
%:%1780=599%:%
%:%1781=599%:%
%:%1782=600%:%
%:%1783=600%:%
%:%1784=600%:%
%:%1785=601%:%
%:%1786=602%:%
%:%1787=602%:%
%:%1788=602%:%
%:%1789=603%:%
%:%1795=603%:%
%:%1798=604%:%
%:%1799=605%:%
%:%1800=605%:%
%:%1801=606%:%
%:%1802=607%:%
%:%1803=608%:%
%:%1804=609%:%
%:%1811=610%:%
%:%1812=610%:%
%:%1813=611%:%
%:%1814=611%:%
%:%1815=612%:%
%:%1816=612%:%
%:%1817=612%:%
%:%1818=613%:%
%:%1819=613%:%
%:%1820=613%:%
%:%1821=614%:%
%:%1822=614%:%
%:%1823=614%:%
%:%1824=615%:%
%:%1825=615%:%
%:%1826=615%:%
%:%1827=616%:%
%:%1828=616%:%
%:%1829=616%:%
%:%1830=617%:%
%:%1831=617%:%
%:%1832=617%:%
%:%1833=618%:%
%:%1834=618%:%
%:%1835=619%:%
%:%1836=619%:%
%:%1837=620%:%
%:%1838=621%:%
%:%1839=621%:%
%:%1840=622%:%
%:%1841=622%:%
%:%1842=623%:%
%:%1843=624%:%
%:%1844=624%:%
%:%1845=624%:%
%:%1846=625%:%
%:%1847=625%:%
%:%1848=625%:%
%:%1849=626%:%
%:%1850=626%:%
%:%1851=626%:%
%:%1852=627%:%
%:%1853=627%:%
%:%1854=627%:%
%:%1855=628%:%
%:%1856=628%:%
%:%1857=628%:%
%:%1858=629%:%
%:%1859=629%:%
%:%1860=629%:%
%:%1861=630%:%
%:%1862=630%:%
%:%1863=630%:%
%:%1864=631%:%
%:%1865=631%:%
%:%1866=632%:%
%:%1867=632%:%
%:%1868=632%:%
%:%1869=633%:%
%:%1870=633%:%
%:%1871=633%:%
%:%1872=634%:%
%:%1873=635%:%
%:%1874=635%:%
%:%1875=636%:%
%:%1876=636%:%
%:%1877=636%:%
%:%1878=637%:%
%:%1879=637%:%
%:%1880=637%:%
%:%1881=638%:%
%:%1882=638%:%
%:%1883=638%:%
%:%1884=639%:%
%:%1885=639%:%
%:%1886=639%:%
%:%1887=640%:%
%:%1888=640%:%
%:%1889=640%:%
%:%1890=641%:%
%:%1891=641%:%
%:%1892=641%:%
%:%1893=642%:%
%:%1894=642%:%
%:%1895=642%:%
%:%1896=643%:%
%:%1897=643%:%
%:%1898=643%:%
%:%1899=644%:%
%:%1900=644%:%
%:%1901=644%:%
%:%1902=645%:%
%:%1903=645%:%
%:%1904=645%:%
%:%1905=646%:%
%:%1906=646%:%
%:%1907=646%:%
%:%1908=647%:%
%:%1909=647%:%
%:%1910=647%:%
%:%1911=648%:%
%:%1912=648%:%
%:%1913=648%:%
%:%1914=649%:%
%:%1915=649%:%
%:%1916=650%:%
%:%1922=650%:%
%:%1925=651%:%
%:%1926=652%:%
%:%1927=652%:%
%:%1928=653%:%
%:%1929=654%:%
%:%1930=655%:%
%:%1937=656%:%
%:%1938=656%:%
%:%1939=657%:%
%:%1940=657%:%
%:%1941=658%:%
%:%1942=658%:%
%:%1943=658%:%
%:%1944=659%:%
%:%1945=659%:%
%:%1946=660%:%
%:%1947=660%:%
%:%1948=660%:%
%:%1949=661%:%
%:%1950=661%:%
%:%1951=662%:%
%:%1952=662%:%
%:%1953=663%:%
%:%1954=663%:%
%:%1955=664%:%
%:%1956=664%:%
%:%1957=665%:%
%:%1958=665%:%
%:%1959=665%:%
%:%1960=666%:%
%:%1961=666%:%
%:%1962=666%:%
%:%1963=667%:%
%:%1964=667%:%
%:%1965=667%:%
%:%1966=668%:%
%:%1967=668%:%
%:%1968=668%:%
%:%1969=669%:%
%:%1970=669%:%
%:%1971=669%:%
%:%1972=670%:%
%:%1973=670%:%
%:%1974=670%:%
%:%1975=671%:%
%:%1976=671%:%
%:%1977=671%:%
%:%1978=672%:%
%:%1979=672%:%
%:%1980=672%:%
%:%1981=673%:%
%:%1982=673%:%
%:%1983=674%:%
%:%1989=674%:%
%:%1992=675%:%
%:%1993=676%:%
%:%1994=676%:%
%:%1995=677%:%
%:%1996=678%:%
%:%1999=679%:%
%:%2003=679%:%
%:%2004=679%:%
%:%2005=679%:%
%:%2010=679%:%
%:%2013=680%:%
%:%2014=681%:%
%:%2015=681%:%
%:%2016=682%:%
%:%2017=683%:%
%:%2020=684%:%
%:%2024=684%:%
%:%2025=684%:%
%:%2026=684%:%
%:%2031=684%:%
%:%2034=685%:%
%:%2035=686%:%
%:%2036=686%:%
%:%2037=687%:%
%:%2038=688%:%
%:%2039=689%:%
%:%2042=690%:%
%:%2046=690%:%
%:%2047=690%:%
%:%2048=690%:%
%:%2053=690%:%
%:%2056=691%:%
%:%2057=692%:%
%:%2058=692%:%
%:%2059=693%:%
%:%2060=694%:%
%:%2061=695%:%
%:%2064=696%:%
%:%2068=696%:%
%:%2069=696%:%
%:%2070=696%:%
%:%2075=696%:%
%:%2078=697%:%
%:%2079=698%:%
%:%2080=698%:%
%:%2081=699%:%
%:%2082=700%:%
%:%2083=701%:%
%:%2084=702%:%
%:%2091=703%:%
%:%2092=703%:%
%:%2093=704%:%
%:%2094=704%:%
%:%2095=705%:%
%:%2096=705%:%
%:%2097=705%:%
%:%2098=706%:%
%:%2099=706%:%
%:%2100=706%:%
%:%2101=707%:%
%:%2102=707%:%
%:%2103=707%:%
%:%2104=708%:%
%:%2105=708%:%
%:%2106=708%:%
%:%2107=709%:%
%:%2108=709%:%
%:%2109=709%:%
%:%2110=710%:%
%:%2111=710%:%
%:%2112=710%:%
%:%2113=711%:%
%:%2114=711%:%
%:%2115=711%:%
%:%2116=712%:%
%:%2117=712%:%
%:%2118=712%:%
%:%2119=713%:%
%:%2120=713%:%
%:%2121=713%:%
%:%2122=714%:%
%:%2123=714%:%
%:%2124=714%:%
%:%2125=715%:%
%:%2126=715%:%
%:%2127=715%:%
%:%2128=716%:%
%:%2129=716%:%
%:%2130=716%:%
%:%2131=717%:%
%:%2132=717%:%
%:%2133=717%:%
%:%2134=718%:%
%:%2135=718%:%
%:%2136=718%:%
%:%2137=719%:%
%:%2138=719%:%
%:%2139=720%:%
%:%2154=722%:%
%:%2164=724%:%
%:%2165=724%:%
%:%2166=725%:%
%:%2167=726%:%
%:%2168=727%:%
%:%2169=727%:%
%:%2170=728%:%
%:%2171=729%:%
%:%2178=730%:%
%:%2179=730%:%
%:%2180=731%:%
%:%2181=731%:%
%:%2182=732%:%
%:%2183=732%:%
%:%2184=732%:%
%:%2185=733%:%
%:%2186=733%:%
%:%2187=733%:%
%:%2188=734%:%
%:%2189=734%:%
%:%2190=735%:%
%:%2196=735%:%
%:%2199=736%:%
%:%2200=737%:%
%:%2201=737%:%
%:%2202=738%:%
%:%2203=739%:%
%:%2204=740%:%
%:%2207=741%:%
%:%2211=741%:%
%:%2212=741%:%
%:%2213=741%:%
%:%2218=741%:%
%:%2221=742%:%
%:%2222=743%:%
%:%2223=743%:%
%:%2224=744%:%
%:%2225=745%:%
%:%2226=746%:%
%:%2227=747%:%
%:%2234=748%:%
%:%2235=748%:%
%:%2236=749%:%
%:%2237=749%:%
%:%2238=750%:%
%:%2239=750%:%
%:%2240=750%:%
%:%2241=751%:%
%:%2242=751%:%
%:%2243=751%:%
%:%2244=752%:%
%:%2245=752%:%
%:%2246=752%:%
%:%2247=753%:%
%:%2248=753%:%
%:%2249=753%:%
%:%2250=754%:%
%:%2251=754%:%
%:%2252=755%:%
%:%2258=755%:%
%:%2261=756%:%
%:%2262=757%:%
%:%2263=757%:%
%:%2264=758%:%
%:%2265=759%:%
%:%2266=760%:%
%:%2267=760%:%
%:%2268=761%:%
%:%2269=762%:%
%:%2276=763%:%
%:%2277=763%:%
%:%2278=764%:%
%:%2279=764%:%
%:%2280=765%:%
%:%2281=765%:%
%:%2282=765%:%
%:%2283=766%:%
%:%2284=766%:%
%:%2285=766%:%
%:%2286=767%:%
%:%2287=767%:%
%:%2288=768%:%
%:%2294=768%:%
%:%2297=769%:%
%:%2298=770%:%
%:%2299=770%:%
%:%2300=771%:%
%:%2301=772%:%
%:%2302=773%:%
%:%2305=774%:%
%:%2309=774%:%
%:%2310=774%:%
%:%2311=774%:%
%:%2316=774%:%
%:%2319=775%:%
%:%2320=776%:%
%:%2321=776%:%
%:%2322=777%:%
%:%2323=778%:%
%:%2324=779%:%
%:%2325=780%:%
%:%2332=781%:%
%:%2333=781%:%
%:%2334=782%:%
%:%2335=782%:%
%:%2336=783%:%
%:%2337=783%:%
%:%2338=783%:%
%:%2339=784%:%
%:%2340=784%:%
%:%2341=784%:%
%:%2342=785%:%
%:%2343=785%:%
%:%2344=785%:%
%:%2345=786%:%
%:%2346=786%:%
%:%2347=786%:%
%:%2348=787%:%
%:%2349=787%:%
%:%2350=788%:%
%:%2365=790%:%
%:%2377=792%:%
%:%2379=793%:%
%:%2380=793%:%
%:%2381=794%:%
%:%2388=795%:%
%:%2389=795%:%
%:%2390=796%:%
%:%2391=796%:%
%:%2392=797%:%
%:%2393=797%:%
%:%2394=797%:%
%:%2395=798%:%
%:%2396=798%:%
%:%2397=799%:%
%:%2398=799%:%
%:%2399=799%:%
%:%2400=800%:%
%:%2401=800%:%
%:%2403=802%:%
%:%2404=803%:%
%:%2405=803%:%
%:%2406=804%:%
%:%2407=804%:%
%:%2408=804%:%
%:%2409=805%:%
%:%2410=805%:%
%:%2411=806%:%
%:%2412=807%:%
%:%2413=807%:%
%:%2414=808%:%
%:%2415=808%:%
%:%2416=809%:%
%:%2417=810%:%
%:%2418=810%:%
%:%2419=811%:%
%:%2420=811%:%
%:%2421=812%:%
%:%2422=812%:%
%:%2423=813%:%
%:%2424=813%:%
%:%2425=814%:%
%:%2426=814%:%
%:%2427=815%:%
%:%2428=815%:%
%:%2429=815%:%
%:%2430=816%:%
%:%2431=816%:%
%:%2432=816%:%
%:%2433=817%:%
%:%2434=818%:%
%:%2435=818%:%
%:%2436=819%:%
%:%2437=819%:%
%:%2438=819%:%
%:%2439=820%:%
%:%2440=821%:%
%:%2441=821%:%
%:%2442=822%:%
%:%2443=822%:%
%:%2444=822%:%
%:%2445=823%:%
%:%2446=823%:%
%:%2447=823%:%
%:%2448=824%:%
%:%2449=824%:%
%:%2450=824%:%
%:%2451=825%:%
%:%2452=825%:%
%:%2453=826%:%
%:%2454=826%:%
%:%2455=827%:%
%:%2456=828%:%
%:%2457=828%:%
%:%2458=829%:%
%:%2459=829%:%
%:%2460=829%:%
%:%2461=830%:%
%:%2462=830%:%
%:%2463=830%:%
%:%2464=831%:%
%:%2465=831%:%
%:%2466=831%:%
%:%2467=832%:%
%:%2477=834%:%
%:%2479=835%:%
%:%2480=835%:%
%:%2481=836%:%
%:%2488=837%:%
%:%2489=837%:%
%:%2490=838%:%
%:%2491=838%:%
%:%2492=839%:%
%:%2493=839%:%
%:%2494=839%:%
%:%2495=840%:%
%:%2496=840%:%
%:%2497=841%:%
%:%2498=841%:%
%:%2499=842%:%
%:%2500=842%:%
%:%2501=843%:%
%:%2502=843%:%
%:%2503=844%:%
%:%2504=844%:%
%:%2505=845%:%
%:%2506=845%:%
%:%2507=845%:%
%:%2508=846%:%
%:%2509=846%:%
%:%2510=847%:%
%:%2511=848%:%
%:%2512=848%:%
%:%2513=848%:%
%:%2514=849%:%
%:%2515=849%:%
%:%2516=849%:%
%:%2517=850%:%
%:%2518=850%:%
%:%2519=851%:%
%:%2520=851%:%
%:%2521=851%:%
%:%2522=852%:%
%:%2523=852%:%
%:%2524=852%:%
%:%2525=853%:%
%:%2526=853%:%
%:%2527=854%:%
%:%2528=854%:%
%:%2529=854%:%
%:%2530=855%:%
%:%2531=855%:%
%:%2532=856%:%
%:%2533=856%:%
%:%2534=856%:%
%:%2535=857%:%
%:%2536=857%:%
%:%2537=857%:%
%:%2538=858%:%
%:%2544=858%:%
%:%2547=859%:%
%:%2548=860%:%
%:%2549=860%:%
%:%2550=861%:%
%:%2557=862%:%
%:%2558=862%:%
%:%2559=863%:%
%:%2560=863%:%
%:%2561=864%:%
%:%2562=864%:%
%:%2563=864%:%
%:%2564=865%:%
%:%2565=865%:%
%:%2566=866%:%
%:%2567=866%:%
%:%2568=867%:%
%:%2569=867%:%
%:%2570=868%:%
%:%2571=868%:%
%:%2572=869%:%
%:%2573=869%:%
%:%2574=870%:%
%:%2575=870%:%
%:%2576=870%:%
%:%2577=871%:%
%:%2578=871%:%
%:%2579=871%:%
%:%2580=872%:%
%:%2581=872%:%
%:%2582=873%:%
%:%2583=873%:%
%:%2584=874%:%
%:%2585=874%:%
%:%2586=874%:%
%:%2587=875%:%
%:%2597=877%:%
%:%2599=878%:%
%:%2600=878%:%
%:%2601=879%:%
%:%2608=880%:%
%:%2609=880%:%
%:%2610=881%:%
%:%2611=881%:%
%:%2612=882%:%
%:%2613=882%:%
%:%2614=883%:%
%:%2615=883%:%
%:%2616=884%:%
%:%2617=884%:%
%:%2618=885%:%
%:%2619=885%:%
%:%2620=886%:%
%:%2621=887%:%
%:%2622=887%:%
%:%2626=891%:%
%:%2627=892%:%
%:%2628=892%:%
%:%2629=893%:%
%:%2630=893%:%
%:%2631=894%:%
%:%2632=894%:%
%:%2633=895%:%
%:%2634=895%:%
%:%2635=895%:%
%:%2636=896%:%
%:%2637=896%:%
%:%2638=897%:%
%:%2639=897%:%
%:%2640=897%:%
%:%2641=898%:%
%:%2642=898%:%
%:%2643=899%:%
%:%2644=899%:%
%:%2645=899%:%
%:%2646=900%:%
%:%2647=900%:%
%:%2648=901%:%
%:%2649=901%:%
%:%2650=902%:%
%:%2651=902%:%
%:%2652=903%:%
%:%2653=903%:%
%:%2656=906%:%
%:%2657=907%:%
%:%2658=907%:%
%:%2659=908%:%
%:%2660=908%:%
%:%2661=909%:%
%:%2662=909%:%
%:%2663=910%:%
%:%2664=910%:%
%:%2665=911%:%
%:%2666=911%:%
%:%2667=912%:%
%:%2668=913%:%
%:%2669=913%:%
%:%2670=914%:%
%:%2671=914%:%
%:%2672=915%:%
%:%2673=915%:%
%:%2674=915%:%
%:%2675=916%:%
%:%2676=916%:%
%:%2677=917%:%
%:%2678=917%:%
%:%2679=918%:%
%:%2680=918%:%
%:%2681=919%:%
%:%2682=919%:%
%:%2683=919%:%
%:%2684=920%:%
%:%2685=920%:%
%:%2686=920%:%
%:%2687=921%:%
%:%2688=921%:%
%:%2689=922%:%
%:%2690=922%:%
%:%2691=923%:%
%:%2692=923%:%
%:%2693=924%:%
%:%2694=924%:%
%:%2695=925%:%
%:%2696=925%:%
%:%2697=926%:%
%:%2698=926%:%
%:%2699=927%:%
%:%2700=927%:%
%:%2701=927%:%
%:%2702=928%:%
%:%2703=928%:%
%:%2704=928%:%
%:%2705=929%:%
%:%2706=929%:%
%:%2707=929%:%
%:%2708=930%:%
%:%2709=930%:%
%:%2710=931%:%
%:%2711=931%:%
%:%2712=931%:%
%:%2713=932%:%
%:%2714=932%:%
%:%2715=933%:%
%:%2716=933%:%
%:%2717=933%:%
%:%2718=934%:%
%:%2719=934%:%
%:%2720=934%:%
%:%2721=935%:%
%:%2722=935%:%
%:%2723=936%:%
%:%2724=936%:%
%:%2725=937%:%
%:%2726=937%:%
%:%2727=938%:%
%:%2728=938%:%
%:%2729=938%:%
%:%2730=939%:%
%:%2731=939%:%
%:%2732=939%:%
%:%2733=940%:%
%:%2739=940%:%
%:%2742=941%:%
%:%2743=942%:%
%:%2744=942%:%
%:%2745=943%:%
%:%2752=944%:%
%:%2753=944%:%
%:%2754=945%:%
%:%2755=945%:%
%:%2756=946%:%
%:%2757=946%:%
%:%2758=947%:%
%:%2759=947%:%
%:%2760=948%:%
%:%2761=948%:%
%:%2762=949%:%
%:%2763=949%:%
%:%2764=950%:%
%:%2765=950%:%
%:%2771=956%:%
%:%2772=957%:%
%:%2773=957%:%
%:%2774=958%:%
%:%2775=958%:%
%:%2776=959%:%
%:%2777=960%:%
%:%2778=960%:%
%:%2779=960%:%
%:%2780=961%:%
%:%2781=961%:%
%:%2782=961%:%
%:%2783=962%:%
%:%2784=962%:%
%:%2785=963%:%
%:%2786=963%:%
%:%2787=963%:%
%:%2788=964%:%
%:%2789=964%:%
%:%2790=965%:%
%:%2791=965%:%
%:%2792=965%:%
%:%2793=966%:%
%:%2794=966%:%
%:%2795=966%:%
%:%2796=967%:%
%:%2797=967%:%
%:%2798=967%:%
%:%2799=968%:%
%:%2800=968%:%
%:%2801=969%:%
%:%2802=969%:%
%:%2803=969%:%
%:%2804=970%:%
%:%2805=970%:%
%:%2806=971%:%
%:%2807=971%:%
%:%2808=971%:%
%:%2809=972%:%
%:%2810=972%:%
%:%2811=973%:%
%:%2812=973%:%
%:%2813=974%:%
%:%2814=974%:%
%:%2815=975%:%
%:%2816=976%:%
%:%2817=976%:%
%:%2818=976%:%
%:%2819=977%:%
%:%2820=977%:%
%:%2821=977%:%
%:%2822=978%:%
%:%2823=978%:%
%:%2824=979%:%
%:%2825=979%:%
%:%2826=979%:%
%:%2827=980%:%
%:%2828=980%:%
%:%2829=981%:%
%:%2830=981%:%
%:%2831=981%:%
%:%2832=982%:%
%:%2833=982%:%
%:%2834=982%:%
%:%2835=983%:%
%:%2836=983%:%
%:%2837=983%:%
%:%2838=984%:%
%:%2839=984%:%
%:%2840=985%:%
%:%2841=985%:%
%:%2842=985%:%
%:%2843=986%:%
%:%2844=986%:%
%:%2845=987%:%
%:%2846=987%:%
%:%2847=987%:%
%:%2848=988%:%
%:%2849=988%:%
%:%2850=989%:%
%:%2851=989%:%
%:%2852=990%:%
%:%2853=990%:%
%:%2854=991%:%
%:%2855=991%:%
%:%2856=992%:%
%:%2857=992%:%
%:%2858=993%:%
%:%2859=993%:%
%:%2860=994%:%
%:%2861=994%:%
%:%2862=995%:%
%:%2863=995%:%
%:%2864=996%:%
%:%2865=996%:%
%:%2866=997%:%
%:%2867=997%:%
%:%2868=998%:%
%:%2869=998%:%
%:%2870=999%:%
%:%2871=999%:%
%:%2872=1000%:%
%:%2873=1000%:%
%:%2874=1000%:%
%:%2875=1001%:%
%:%2876=1002%:%
%:%2877=1002%:%
%:%2878=1002%:%
%:%2879=1003%:%
%:%2880=1003%:%
%:%2881=1004%:%
%:%2882=1004%:%
%:%2883=1005%:%
%:%2884=1005%:%
%:%2885=1006%:%
%:%2886=1006%:%
%:%2887=1006%:%
%:%2888=1007%:%
%:%2889=1007%:%
%:%2890=1008%:%
%:%2891=1008%:%
%:%2892=1009%:%
%:%2893=1009%:%
%:%2894=1010%:%
%:%2895=1010%:%
%:%2896=1010%:%
%:%2897=1011%:%
%:%2898=1011%:%
%:%2899=1011%:%
%:%2900=1012%:%
%:%2901=1012%:%
%:%2902=1012%:%
%:%2903=1013%:%
%:%2904=1013%:%
%:%2905=1013%:%
%:%2906=1014%:%
%:%2907=1014%:%
%:%2908=1014%:%
%:%2909=1015%:%
%:%2910=1015%:%
%:%2911=1016%:%
%:%2912=1016%:%
%:%2913=1016%:%
%:%2914=1017%:%
%:%2915=1017%:%
%:%2916=1017%:%
%:%2917=1018%:%
%:%2918=1018%:%
%:%2919=1018%:%
%:%2920=1019%:%
%:%2921=1019%:%
%:%2922=1019%:%
%:%2923=1020%:%
%:%2924=1020%:%
%:%2925=1020%:%
%:%2926=1021%:%
%:%2927=1021%:%
%:%2928=1022%:%
%:%2929=1022%:%
%:%2930=1022%:%
%:%2931=1023%:%
%:%2932=1023%:%
%:%2933=1024%:%
%:%2934=1024%:%
%:%2935=1024%:%
%:%2936=1025%:%
%:%2937=1025%:%
%:%2938=1026%:%
%:%2939=1026%:%
%:%2940=1026%:%
%:%2941=1027%:%
%:%2942=1027%:%
%:%2943=1028%:%
%:%2944=1028%:%
%:%2945=1028%:%
%:%2946=1029%:%
%:%2947=1029%:%
%:%2948=1030%:%
%:%2949=1030%:%
%:%2950=1030%:%
%:%2951=1031%:%
%:%2952=1031%:%
%:%2953=1032%:%
%:%2954=1032%:%
%:%2955=1033%:%
%:%2956=1033%:%
%:%2957=1034%:%
%:%2958=1034%:%
%:%2959=1034%:%
%:%2960=1035%:%
%:%2961=1035%:%
%:%2962=1035%:%
%:%2963=1036%:%
%:%2964=1036%:%
%:%2965=1037%:%
%:%2966=1037%:%
%:%2967=1038%:%
%:%2968=1038%:%
%:%2969=1038%:%
%:%2970=1039%:%
%:%2971=1039%:%
%:%2972=1039%:%
%:%2973=1040%:%
%:%2974=1040%:%
%:%2975=1040%:%
%:%2976=1041%:%
%:%2977=1041%:%
%:%2978=1041%:%
%:%2979=1042%:%
%:%2980=1042%:%
%:%2981=1042%:%
%:%2982=1043%:%
%:%2983=1043%:%
%:%2984=1044%:%
%:%2985=1044%:%
%:%2986=1044%:%
%:%2987=1045%:%
%:%2988=1045%:%
%:%2989=1045%:%
%:%2990=1046%:%
%:%2991=1046%:%
%:%2992=1046%:%
%:%2993=1047%:%
%:%2994=1047%:%
%:%2995=1047%:%
%:%2996=1048%:%
%:%2997=1048%:%
%:%2998=1048%:%
%:%2999=1049%:%
%:%3000=1049%:%
%:%3001=1050%:%
%:%3002=1050%:%
%:%3003=1050%:%
%:%3004=1051%:%
%:%3005=1051%:%
%:%3006=1052%:%
%:%3007=1052%:%
%:%3008=1052%:%
%:%3009=1053%:%
%:%3010=1053%:%
%:%3011=1054%:%
%:%3012=1054%:%
%:%3013=1054%:%
%:%3014=1055%:%
%:%3015=1055%:%
%:%3016=1056%:%
%:%3017=1056%:%
%:%3018=1056%:%
%:%3019=1057%:%
%:%3020=1057%:%
%:%3021=1058%:%
%:%3022=1058%:%
%:%3023=1058%:%
%:%3024=1059%:%
%:%3025=1059%:%
%:%3026=1060%:%
%:%3027=1060%:%
%:%3028=1061%:%
%:%3029=1061%:%
%:%3030=1062%:%
%:%3031=1062%:%
%:%3032=1063%:%
%:%3033=1063%:%
%:%3034=1063%:%
%:%3036=1065%:%
%:%3037=1066%:%
%:%3038=1066%:%
%:%3039=1066%:%
%:%3040=1067%:%
%:%3041=1067%:%
%:%3042=1068%:%
%:%3043=1068%:%
%:%3044=1069%:%
%:%3045=1069%:%
%:%3046=1069%:%
%:%3047=1070%:%
%:%3048=1070%:%
%:%3049=1071%:%
%:%3055=1071%:%
%:%3058=1072%:%
%:%3059=1073%:%
%:%3060=1073%:%
%:%3061=1074%:%
%:%3062=1075%:%
%:%3069=1076%:%
%:%3070=1076%:%
%:%3071=1077%:%
%:%3072=1077%:%
%:%3073=1078%:%
%:%3074=1078%:%
%:%3075=1078%:%
%:%3076=1079%:%
%:%3077=1079%:%
%:%3078=1080%:%
%:%3079=1080%:%
%:%3080=1080%:%
%:%3081=1081%:%
%:%3082=1081%:%
%:%3083=1082%:%
%:%3084=1082%:%
%:%3085=1082%:%
%:%3086=1083%:%
%:%3087=1083%:%
%:%3088=1084%:%
%:%3089=1084%:%
%:%3090=1085%:%
%:%3091=1085%:%
%:%3092=1086%:%
%:%3093=1086%:%
%:%3094=1086%:%
%:%3095=1087%:%
%:%3101=1087%:%
%:%3104=1088%:%
%:%3105=1089%:%
%:%3106=1089%:%
%:%3107=1090%:%
%:%3108=1091%:%
%:%3115=1092%:%
%:%3116=1092%:%
%:%3117=1093%:%
%:%3118=1093%:%
%:%3119=1094%:%
%:%3120=1094%:%
%:%3121=1094%:%
%:%3122=1095%:%
%:%3123=1095%:%
%:%3124=1096%:%
%:%3125=1096%:%
%:%3126=1096%:%
%:%3127=1097%:%
%:%3128=1097%:%
%:%3129=1098%:%
%:%3130=1098%:%
%:%3131=1099%:%
%:%3132=1099%:%
%:%3133=1100%:%
%:%3134=1100%:%
%:%3135=1100%:%
%:%3136=1101%:%
%:%3137=1101%:%
%:%3138=1102%:%
%:%3139=1102%:%
%:%3140=1102%:%
%:%3141=1103%:%
%:%3142=1104%:%
%:%3143=1104%:%
%:%3144=1105%:%
%:%3145=1105%:%
%:%3146=1106%:%
%:%3147=1106%:%
%:%3148=1107%:%
%:%3149=1108%:%
%:%3150=1108%:%
%:%3151=1108%:%
%:%3152=1109%:%
%:%3153=1109%:%
%:%3154=1109%:%
%:%3155=1110%:%
%:%3156=1110%:%
%:%3157=1111%:%
%:%3158=1111%:%
%:%3159=1111%:%
%:%3160=1112%:%
%:%3161=1112%:%
%:%3162=1113%:%
%:%3163=1113%:%
%:%3164=1113%:%
%:%3165=1114%:%
%:%3166=1114%:%
%:%3167=1114%:%
%:%3168=1115%:%
%:%3169=1115%:%
%:%3170=1115%:%
%:%3171=1116%:%
%:%3172=1116%:%
%:%3173=1116%:%
%:%3174=1117%:%
%:%3175=1117%:%
%:%3176=1117%:%
%:%3177=1118%:%
%:%3178=1118%:%
%:%3179=1119%:%
%:%3180=1119%:%
%:%3181=1119%:%
%:%3182=1120%:%
%:%3183=1120%:%
%:%3184=1121%:%
%:%3185=1121%:%
%:%3186=1121%:%
%:%3187=1122%:%
%:%3188=1122%:%
%:%3189=1123%:%
%:%3190=1123%:%
%:%3191=1123%:%
%:%3192=1124%:%
%:%3193=1124%:%
%:%3194=1125%:%
%:%3195=1125%:%
%:%3196=1126%:%
%:%3197=1126%:%
%:%3198=1127%:%
%:%3199=1127%:%
%:%3200=1128%:%
%:%3201=1128%:%
%:%3202=1129%:%
%:%3203=1130%:%
%:%3204=1130%:%
%:%3205=1130%:%
%:%3206=1131%:%
%:%3207=1131%:%
%:%3208=1131%:%
%:%3209=1132%:%
%:%3210=1132%:%
%:%3211=1133%:%
%:%3212=1133%:%
%:%3213=1133%:%
%:%3214=1134%:%
%:%3215=1134%:%
%:%3216=1135%:%
%:%3217=1135%:%
%:%3218=1135%:%
%:%3219=1136%:%
%:%3220=1136%:%
%:%3221=1136%:%
%:%3222=1137%:%
%:%3223=1137%:%
%:%3224=1137%:%
%:%3225=1138%:%
%:%3226=1138%:%
%:%3227=1138%:%
%:%3228=1139%:%
%:%3229=1139%:%
%:%3230=1139%:%
%:%3231=1140%:%
%:%3232=1140%:%
%:%3233=1141%:%
%:%3234=1141%:%
%:%3235=1141%:%
%:%3236=1142%:%
%:%3237=1142%:%
%:%3238=1143%:%
%:%3239=1143%:%
%:%3240=1143%:%
%:%3241=1144%:%
%:%3242=1144%:%
%:%3243=1145%:%
%:%3244=1145%:%
%:%3245=1145%:%
%:%3246=1146%:%
%:%3247=1146%:%
%:%3248=1147%:%
%:%3249=1147%:%
%:%3250=1148%:%
%:%3251=1148%:%
%:%3252=1149%:%
%:%3253=1149%:%
%:%3254=1150%:%
%:%3260=1150%:%
%:%3263=1151%:%
%:%3264=1152%:%
%:%3265=1152%:%
%:%3266=1153%:%
%:%3273=1154%:%
%:%3274=1154%:%
%:%3275=1155%:%
%:%3276=1155%:%
%:%3277=1156%:%
%:%3278=1157%:%
%:%3279=1158%:%
%:%3280=1158%:%
%:%3281=1158%:%
%:%3282=1159%:%
%:%3283=1160%:%
%:%3284=1160%:%
%:%3285=1161%:%
%:%3286=1162%:%
%:%3287=1163%:%
%:%3288=1163%:%
%:%3289=1163%:%
%:%3290=1164%:%
%:%3291=1165%:%
%:%3292=1165%:%
%:%3293=1166%:%
%:%3294=1166%:%
%:%3295=1167%:%
%:%3296=1167%:%
%:%3297=1168%:%
%:%3298=1169%:%
%:%3299=1169%:%
%:%3300=1170%:%
%:%3301=1170%:%
%:%3302=1171%:%
%:%3303=1172%:%
%:%3304=1172%:%
%:%3305=1173%:%
%:%3306=1173%:%
%:%3307=1174%:%
%:%3308=1175%:%
%:%3309=1175%:%
%:%3310=1176%:%
%:%3311=1176%:%
%:%3312=1176%:%
%:%3313=1177%:%
%:%3314=1177%:%
%:%3315=1178%:%
%:%3316=1178%:%
%:%3317=1178%:%
%:%3318=1179%:%
%:%3319=1179%:%
%:%3320=1180%:%
%:%3321=1180%:%
%:%3322=1180%:%
%:%3323=1181%:%
%:%3324=1181%:%
%:%3325=1181%:%
%:%3326=1182%:%
%:%3327=1182%:%
%:%3328=1182%:%
%:%3329=1183%:%
%:%3330=1183%:%
%:%3331=1183%:%
%:%3332=1184%:%
%:%3333=1184%:%
%:%3334=1184%:%
%:%3335=1185%:%
%:%3336=1185%:%
%:%3337=1186%:%
%:%3338=1186%:%
%:%3339=1186%:%
%:%3340=1187%:%
%:%3341=1187%:%
%:%3342=1188%:%
%:%3343=1188%:%
%:%3344=1188%:%
%:%3345=1189%:%
%:%3346=1189%:%
%:%3347=1189%:%
%:%3348=1190%:%
%:%3349=1190%:%
%:%3350=1190%:%
%:%3351=1191%:%
%:%3352=1191%:%
%:%3353=1192%:%
%:%3354=1192%:%
%:%3355=1192%:%
%:%3356=1193%:%
%:%3357=1193%:%
%:%3358=1194%:%
%:%3359=1194%:%
%:%3360=1194%:%
%:%3361=1195%:%
%:%3362=1195%:%
%:%3363=1196%:%
%:%3364=1196%:%
%:%3365=1196%:%
%:%3366=1197%:%
%:%3367=1197%:%
%:%3368=1198%:%
%:%3369=1198%:%
%:%3370=1198%:%
%:%3371=1199%:%
%:%3372=1199%:%
%:%3373=1200%:%
%:%3374=1200%:%
%:%3375=1200%:%
%:%3376=1201%:%
%:%3377=1201%:%
%:%3378=1202%:%
%:%3379=1202%:%
%:%3380=1202%:%
%:%3381=1203%:%
%:%3382=1203%:%
%:%3383=1204%:%
%:%3384=1204%:%
%:%3385=1204%:%
%:%3386=1204%:%
%:%3387=1204%:%
%:%3388=1205%:%
%:%3389=1205%:%
%:%3390=1206%:%
%:%3391=1206%:%
%:%3392=1207%:%
%:%3393=1207%:%
%:%3394=1208%:%
%:%3395=1208%:%
%:%3396=1209%:%
%:%3397=1209%:%
%:%3398=1210%:%
%:%3399=1210%:%
%:%3400=1211%:%
%:%3401=1212%:%
%:%3402=1212%:%
%:%3403=1213%:%
%:%3404=1213%:%
%:%3405=1214%:%
%:%3406=1215%:%
%:%3407=1215%:%
%:%3408=1216%:%
%:%3409=1216%:%
%:%3410=1216%:%
%:%3411=1217%:%
%:%3412=1217%:%
%:%3413=1217%:%
%:%3414=1218%:%
%:%3415=1218%:%
%:%3416=1218%:%
%:%3417=1219%:%
%:%3418=1219%:%
%:%3419=1220%:%
%:%3420=1220%:%
%:%3421=1220%:%
%:%3422=1221%:%
%:%3423=1221%:%
%:%3424=1222%:%
%:%3425=1222%:%
%:%3426=1222%:%
%:%3427=1223%:%
%:%3428=1223%:%
%:%3429=1224%:%
%:%3430=1224%:%
%:%3431=1224%:%
%:%3432=1225%:%
%:%3433=1225%:%
%:%3434=1225%:%
%:%3435=1226%:%
%:%3436=1226%:%
%:%3437=1226%:%
%:%3438=1227%:%
%:%3439=1227%:%
%:%3440=1228%:%
%:%3441=1228%:%
%:%3442=1228%:%
%:%3443=1229%:%
%:%3444=1229%:%
%:%3445=1229%:%
%:%3446=1230%:%
%:%3447=1230%:%
%:%3448=1230%:%
%:%3449=1231%:%
%:%3450=1231%:%
%:%3451=1232%:%
%:%3452=1232%:%
%:%3453=1232%:%
%:%3454=1233%:%
%:%3455=1233%:%
%:%3456=1233%:%
%:%3457=1234%:%
%:%3458=1234%:%
%:%3459=1234%:%
%:%3460=1235%:%
%:%3461=1235%:%
%:%3462=1236%:%
%:%3463=1236%:%
%:%3464=1236%:%
%:%3465=1237%:%
%:%3466=1237%:%
%:%3467=1238%:%
%:%3468=1238%:%
%:%3469=1238%:%
%:%3470=1239%:%
%:%3471=1239%:%
%:%3472=1240%:%
%:%3473=1240%:%
%:%3474=1240%:%
%:%3475=1241%:%
%:%3476=1241%:%
%:%3477=1242%:%
%:%3478=1242%:%
%:%3479=1242%:%
%:%3480=1242%:%
%:%3481=1242%:%
%:%3482=1243%:%
%:%3483=1243%:%
%:%3484=1244%:%
%:%3485=1244%:%
%:%3486=1245%:%
%:%3487=1245%:%
%:%3488=1246%:%
%:%3489=1246%:%
%:%3490=1247%:%
%:%3496=1247%:%
%:%3499=1248%:%
%:%3500=1249%:%
%:%3501=1249%:%
%:%3502=1250%:%
%:%3503=1251%:%
%:%3510=1252%:%
%:%3511=1252%:%
%:%3512=1253%:%
%:%3513=1253%:%
%:%3514=1254%:%
%:%3515=1254%:%
%:%3516=1254%:%
%:%3517=1255%:%
%:%3518=1255%:%
%:%3519=1256%:%
%:%3520=1256%:%
%:%3521=1256%:%
%:%3522=1257%:%
%:%3523=1257%:%
%:%3524=1258%:%
%:%3525=1258%:%
%:%3526=1259%:%
%:%3527=1259%:%
%:%3528=1260%:%
%:%3529=1260%:%
%:%3530=1260%:%
%:%3531=1261%:%
%:%3532=1261%:%
%:%3533=1262%:%
%:%3534=1262%:%
%:%3535=1262%:%
%:%3536=1263%:%
%:%3537=1264%:%
%:%3538=1264%:%
%:%3539=1265%:%
%:%3540=1265%:%
%:%3541=1266%:%
%:%3542=1266%:%
%:%3543=1267%:%
%:%3544=1267%:%
%:%3545=1267%:%
%:%3546=1268%:%
%:%3547=1268%:%
%:%3548=1268%:%
%:%3549=1269%:%
%:%3550=1269%:%
%:%3551=1269%:%
%:%3552=1270%:%
%:%3553=1270%:%
%:%3554=1270%:%
%:%3555=1271%:%
%:%3556=1271%:%
%:%3557=1272%:%
%:%3558=1272%:%
%:%3559=1272%:%
%:%3560=1273%:%
%:%3561=1273%:%
%:%3562=1274%:%
%:%3563=1274%:%
%:%3564=1274%:%
%:%3565=1275%:%
%:%3566=1275%:%
%:%3567=1276%:%
%:%3568=1276%:%
%:%3569=1276%:%
%:%3570=1277%:%
%:%3571=1277%:%
%:%3572=1277%:%
%:%3573=1278%:%
%:%3574=1278%:%
%:%3575=1278%:%
%:%3576=1279%:%
%:%3577=1279%:%
%:%3578=1279%:%
%:%3579=1280%:%
%:%3580=1280%:%
%:%3581=1280%:%
%:%3582=1281%:%
%:%3583=1281%:%
%:%3584=1282%:%
%:%3585=1282%:%
%:%3586=1283%:%
%:%3587=1284%:%
%:%3588=1284%:%
%:%3589=1285%:%
%:%3590=1285%:%
%:%3591=1286%:%
%:%3592=1286%:%
%:%3593=1287%:%
%:%3594=1287%:%
%:%3595=1287%:%
%:%3596=1288%:%
%:%3597=1288%:%
%:%3598=1288%:%
%:%3599=1289%:%
%:%3600=1289%:%
%:%3601=1289%:%
%:%3602=1290%:%
%:%3603=1290%:%
%:%3604=1290%:%
%:%3605=1291%:%
%:%3606=1291%:%
%:%3607=1292%:%
%:%3608=1292%:%
%:%3609=1292%:%
%:%3610=1293%:%
%:%3611=1293%:%
%:%3612=1294%:%
%:%3613=1294%:%
%:%3614=1294%:%
%:%3615=1295%:%
%:%3616=1295%:%
%:%3617=1296%:%
%:%3618=1296%:%
%:%3619=1296%:%
%:%3620=1297%:%
%:%3621=1297%:%
%:%3622=1297%:%
%:%3623=1298%:%
%:%3624=1298%:%
%:%3625=1298%:%
%:%3626=1299%:%
%:%3627=1299%:%
%:%3628=1299%:%
%:%3629=1300%:%
%:%3630=1300%:%
%:%3631=1300%:%
%:%3632=1301%:%
%:%3633=1301%:%
%:%3634=1302%:%
%:%3635=1302%:%
%:%3636=1303%:%
%:%3637=1303%:%
%:%3638=1304%:%
%:%3639=1304%:%
%:%3640=1305%:%
%:%3646=1305%:%
%:%3649=1306%:%
%:%3650=1307%:%
%:%3651=1307%:%
%:%3652=1308%:%
%:%3653=1309%:%
%:%3656=1310%:%
%:%3660=1310%:%
%:%3661=1310%:%
%:%3666=1310%:%
%:%3669=1311%:%
%:%3670=1312%:%
%:%3671=1312%:%
%:%3672=1313%:%
%:%3673=1314%:%
%:%3676=1315%:%
%:%3680=1315%:%
%:%3681=1315%:%
%:%3686=1315%:%
%:%3689=1316%:%
%:%3690=1317%:%
%:%3691=1317%:%
%:%3692=1318%:%
%:%3693=1319%:%
%:%3696=1320%:%
%:%3700=1320%:%
%:%3701=1320%:%
%:%3702=1320%:%
%:%3707=1320%:%
%:%3710=1321%:%
%:%3711=1322%:%
%:%3712=1322%:%
%:%3713=1323%:%
%:%3714=1324%:%
%:%3717=1325%:%
%:%3721=1325%:%
%:%3722=1325%:%
%:%3727=1325%:%
%:%3730=1326%:%
%:%3731=1327%:%
%:%3732=1327%:%
%:%3733=1328%:%
%:%3734=1329%:%
%:%3737=1330%:%
%:%3741=1330%:%
%:%3742=1330%:%
%:%3751=1332%:%
%:%3753=1333%:%
%:%3754=1333%:%
%:%3755=1334%:%
%:%3756=1335%:%
%:%3757=1336%:%
%:%3758=1336%:%
%:%3759=1337%:%
%:%3766=1338%:%
%:%3767=1338%:%
%:%3768=1339%:%
%:%3769=1339%:%
%:%3770=1340%:%
%:%3771=1341%:%
%:%3772=1342%:%
%:%3773=1342%:%
%:%3774=1343%:%
%:%3775=1343%:%
%:%3776=1344%:%
%:%3777=1344%:%
%:%3778=1345%:%
%:%3779=1345%:%
%:%3780=1346%:%
%:%3781=1346%:%
%:%3782=1346%:%
%:%3783=1346%:%
%:%3784=1347%:%
%:%3785=1347%:%
%:%3786=1347%:%
%:%3787=1348%:%
%:%3788=1348%:%
%:%3789=1348%:%
%:%3790=1348%:%
%:%3791=1349%:%
%:%3792=1349%:%
%:%3793=1349%:%
%:%3794=1350%:%
%:%3795=1350%:%
%:%3796=1351%:%
%:%3797=1351%:%
%:%3798=1352%:%
%:%3799=1352%:%
%:%3800=1353%:%
%:%3801=1353%:%
%:%3802=1354%:%
%:%3803=1354%:%
%:%3804=1355%:%
%:%3805=1355%:%
%:%3806=1356%:%
%:%3807=1356%:%
%:%3808=1357%:%
%:%3809=1357%:%
%:%3810=1357%:%
%:%3811=1358%:%
%:%3812=1358%:%
%:%3813=1358%:%
%:%3814=1359%:%
%:%3815=1360%:%
%:%3816=1360%:%
%:%3817=1361%:%
%:%3818=1361%:%
%:%3819=1361%:%
%:%3820=1362%:%
%:%3821=1362%:%
%:%3822=1363%:%
%:%3823=1364%:%
%:%3824=1364%:%
%:%3825=1365%:%
%:%3826=1365%:%
%:%3827=1366%:%
%:%3828=1367%:%
%:%3829=1367%:%
%:%3830=1367%:%
%:%3831=1368%:%
%:%3832=1368%:%
%:%3833=1368%:%
%:%3834=1369%:%
%:%3835=1370%:%
%:%3836=1370%:%
%:%3837=1371%:%
%:%3838=1371%:%
%:%3839=1371%:%
%:%3840=1372%:%
%:%3841=1373%:%
%:%3842=1373%:%
%:%3843=1374%:%
%:%3844=1374%:%
%:%3845=1374%:%
%:%3846=1375%:%
%:%3847=1376%:%
%:%3848=1376%:%
%:%3849=1377%:%
%:%3850=1377%:%
%:%3851=1377%:%
%:%3852=1377%:%
%:%3853=1377%:%
%:%3854=1378%:%
%:%3855=1378%:%
%:%3856=1379%:%
%:%3857=1379%:%
%:%3858=1380%:%
%:%3859=1381%:%
%:%3860=1381%:%
%:%3861=1382%:%
%:%3862=1382%:%
%:%3863=1383%:%
%:%3864=1384%:%
%:%3865=1384%:%
%:%3866=1384%:%
%:%3867=1385%:%
%:%3868=1385%:%
%:%3869=1385%:%
%:%3870=1386%:%
%:%3871=1387%:%
%:%3872=1387%:%
%:%3873=1388%:%
%:%3874=1388%:%
%:%3875=1388%:%
%:%3876=1389%:%
%:%3877=1390%:%
%:%3878=1390%:%
%:%3879=1391%:%
%:%3880=1391%:%
%:%3881=1391%:%
%:%3882=1392%:%
%:%3883=1393%:%
%:%3884=1393%:%
%:%3885=1394%:%
%:%3886=1394%:%
%:%3887=1394%:%
%:%3888=1394%:%
%:%3889=1394%:%
%:%3890=1395%:%
%:%3891=1395%:%
%:%3892=1396%:%
%:%3893=1396%:%
%:%3894=1397%:%
%:%3895=1398%:%
%:%3896=1398%:%
%:%3897=1398%:%
%:%3898=1399%:%
%:%3899=1399%:%
%:%3900=1399%:%
%:%3901=1400%:%
%:%3902=1401%:%
%:%3903=1401%:%
%:%3904=1401%:%
%:%3905=1402%:%
%:%3906=1402%:%
%:%3907=1403%:%
%:%3908=1403%:%
%:%3909=1404%:%
%:%3910=1404%:%
%:%3911=1405%:%
%:%3912=1405%:%
%:%3913=1406%:%
%:%3914=1406%:%
%:%3915=1406%:%
%:%3916=1407%:%
%:%3917=1407%:%
%:%3918=1407%:%
%:%3919=1408%:%
%:%3920=1408%:%
%:%3921=1408%:%
%:%3922=1409%:%
%:%3923=1409%:%
%:%3924=1409%:%
%:%3925=1410%:%
%:%3926=1410%:%
%:%3927=1411%:%
%:%3928=1411%:%
%:%3929=1412%:%
%:%3930=1412%:%
%:%3931=1412%:%
%:%3932=1413%:%
%:%3933=1413%:%
%:%3934=1413%:%
%:%3935=1414%:%
%:%3936=1414%:%
%:%3937=1415%:%
%:%3938=1415%:%
%:%3939=1416%:%
%:%3940=1416%:%
%:%3941=1417%:%
%:%3942=1417%:%
%:%3943=1418%:%
%:%3944=1418%:%
%:%3945=1418%:%
%:%3946=1419%:%
%:%3947=1419%:%
%:%3948=1419%:%
%:%3949=1420%:%
%:%3950=1420%:%
%:%3951=1420%:%
%:%3952=1421%:%
%:%3953=1421%:%
%:%3954=1421%:%
%:%3955=1422%:%
%:%3956=1422%:%
%:%3957=1422%:%
%:%3958=1423%:%
%:%3959=1423%:%
%:%3960=1423%:%
%:%3961=1424%:%
%:%3962=1424%:%
%:%3963=1424%:%
%:%3964=1425%:%
%:%3965=1425%:%
%:%3966=1425%:%
%:%3967=1426%:%
%:%3968=1426%:%
%:%3969=1427%:%
%:%3970=1427%:%
%:%3971=1427%:%
%:%3972=1428%:%
%:%3973=1428%:%
%:%3974=1428%:%
%:%3975=1429%:%
%:%3976=1429%:%
%:%3977=1429%:%
%:%3978=1430%:%
%:%3979=1430%:%
%:%3980=1430%:%
%:%3981=1431%:%
%:%3982=1431%:%
%:%3983=1431%:%
%:%3984=1432%:%
%:%3985=1432%:%
%:%3986=1433%:%
%:%3987=1433%:%
%:%3988=1434%:%
%:%3989=1434%:%
%:%3990=1435%:%
%:%3991=1435%:%
%:%3992=1435%:%
%:%3993=1436%:%
%:%3994=1436%:%
%:%3995=1436%:%
%:%3996=1437%:%
%:%3997=1437%:%
%:%3998=1438%:%
%:%3999=1438%:%
%:%4000=1438%:%
%:%4001=1439%:%
%:%4002=1439%:%
%:%4003=1439%:%
%:%4004=1440%:%
%:%4005=1440%:%
%:%4006=1440%:%
%:%4007=1441%:%
%:%4008=1441%:%
%:%4009=1441%:%
%:%4010=1442%:%
%:%4011=1442%:%
%:%4012=1442%:%
%:%4013=1443%:%
%:%4014=1443%:%
%:%4015=1443%:%
%:%4016=1444%:%
%:%4017=1444%:%
%:%4018=1444%:%
%:%4019=1445%:%
%:%4020=1445%:%
%:%4021=1445%:%
%:%4022=1446%:%
%:%4023=1446%:%
%:%4024=1447%:%
%:%4025=1447%:%
%:%4026=1447%:%
%:%4027=1448%:%
%:%4028=1448%:%
%:%4029=1448%:%
%:%4030=1449%:%
%:%4031=1449%:%
%:%4032=1449%:%
%:%4033=1450%:%
%:%4034=1450%:%
%:%4035=1450%:%
%:%4036=1451%:%
%:%4037=1451%:%
%:%4038=1451%:%
%:%4039=1452%:%
%:%4040=1452%:%
%:%4041=1453%:%
%:%4042=1453%:%
%:%4043=1454%:%
%:%4044=1454%:%
%:%4045=1455%:%
%:%4046=1455%:%
%:%4047=1455%:%
%:%4048=1456%:%
%:%4049=1456%:%
%:%4050=1456%:%
%:%4051=1457%:%
%:%4052=1457%:%
%:%4053=1458%:%
%:%4054=1458%:%
%:%4055=1459%:%
%:%4056=1459%:%
%:%4057=1460%:%
%:%4058=1460%:%
%:%4059=1460%:%
%:%4060=1461%:%
%:%4061=1461%:%
%:%4062=1461%:%
%:%4063=1462%:%
%:%4064=1462%:%
%:%4065=1463%:%
%:%4066=1463%:%
%:%4067=1464%:%
%:%4068=1464%:%
%:%4069=1465%:%
%:%4070=1465%:%
%:%4071=1466%:%
%:%4072=1466%:%
%:%4073=1466%:%
%:%4074=1466%:%
%:%4075=1467%:%
%:%4076=1467%:%
%:%4077=1467%:%
%:%4078=1468%:%
%:%4079=1468%:%
%:%4080=1468%:%
%:%4081=1469%:%
%:%4082=1469%:%
%:%4083=1469%:%
%:%4084=1470%:%
%:%4085=1470%:%
%:%4086=1470%:%
%:%4087=1471%:%
%:%4088=1471%:%
%:%4089=1472%:%
%:%4090=1473%:%
%:%4091=1473%:%
%:%4092=1474%:%
%:%4093=1475%:%
%:%4094=1475%:%
%:%4095=1476%:%
%:%4096=1476%:%
%:%4097=1477%:%
%:%4098=1477%:%
%:%4099=1478%:%
%:%4100=1478%:%
%:%4101=1479%:%
%:%4102=1479%:%
%:%4103=1480%:%
%:%4104=1480%:%
%:%4105=1480%:%
%:%4106=1481%:%
%:%4107=1481%:%
%:%4108=1482%:%
%:%4109=1482%:%
%:%4110=1482%:%
%:%4111=1483%:%
%:%4112=1483%:%
%:%4113=1484%:%
%:%4114=1484%:%
%:%4115=1484%:%
%:%4116=1485%:%
%:%4117=1485%:%
%:%4118=1486%:%
%:%4119=1486%:%
%:%4120=1486%:%
%:%4121=1487%:%
%:%4122=1487%:%
%:%4123=1488%:%
%:%4124=1488%:%
%:%4125=1488%:%
%:%4126=1489%:%
%:%4127=1489%:%
%:%4128=1489%:%
%:%4129=1490%:%
%:%4130=1490%:%
%:%4131=1490%:%
%:%4132=1491%:%
%:%4133=1491%:%
%:%4134=1492%:%
%:%4135=1492%:%
%:%4136=1492%:%
%:%4137=1493%:%
%:%4138=1493%:%
%:%4139=1494%:%
%:%4140=1494%:%
%:%4141=1494%:%
%:%4142=1495%:%
%:%4143=1495%:%
%:%4144=1496%:%
%:%4145=1496%:%
%:%4146=1496%:%
%:%4147=1497%:%
%:%4148=1497%:%
%:%4149=1497%:%
%:%4150=1498%:%
%:%4151=1498%:%
%:%4152=1498%:%
%:%4153=1498%:%
%:%4154=1498%:%
%:%4155=1499%:%
%:%4156=1499%:%
%:%4157=1500%:%
%:%4158=1500%:%
%:%4159=1501%:%
%:%4160=1501%:%
%:%4161=1502%:%
%:%4162=1502%:%
%:%4163=1503%:%
%:%4164=1503%:%
%:%4165=1504%:%
%:%4166=1504%:%
%:%4167=1504%:%
%:%4168=1505%:%
%:%4169=1505%:%
%:%4170=1506%:%
%:%4171=1506%:%
%:%4172=1506%:%
%:%4173=1507%:%
%:%4174=1507%:%
%:%4175=1508%:%
%:%4176=1508%:%
%:%4177=1508%:%
%:%4178=1509%:%
%:%4179=1509%:%
%:%4180=1510%:%
%:%4181=1510%:%
%:%4182=1510%:%
%:%4183=1511%:%
%:%4184=1511%:%
%:%4185=1512%:%
%:%4186=1512%:%
%:%4187=1512%:%
%:%4188=1513%:%
%:%4189=1513%:%
%:%4190=1513%:%
%:%4191=1514%:%
%:%4192=1514%:%
%:%4193=1514%:%
%:%4194=1515%:%
%:%4195=1515%:%
%:%4196=1516%:%
%:%4197=1516%:%
%:%4198=1516%:%
%:%4199=1517%:%
%:%4200=1517%:%
%:%4201=1518%:%
%:%4202=1518%:%
%:%4203=1518%:%
%:%4204=1519%:%
%:%4205=1519%:%
%:%4206=1520%:%
%:%4207=1520%:%
%:%4208=1520%:%
%:%4209=1521%:%
%:%4210=1521%:%
%:%4211=1521%:%
%:%4212=1522%:%
%:%4213=1522%:%
%:%4214=1522%:%
%:%4215=1522%:%
%:%4216=1522%:%
%:%4217=1523%:%
%:%4218=1523%:%
%:%4219=1524%:%
%:%4220=1524%:%
%:%4221=1525%:%
%:%4222=1525%:%
%:%4223=1525%:%
%:%4224=1526%:%
%:%4225=1526%:%
%:%4226=1526%:%
%:%4227=1527%:%
%:%4228=1527%:%
%:%4229=1527%:%
%:%4230=1528%:%
%:%4231=1528%:%
%:%4232=1529%:%
%:%4233=1529%:%
%:%4234=1529%:%
%:%4235=1530%:%
%:%4236=1530%:%
%:%4237=1531%:%
%:%4238=1531%:%
%:%4239=1531%:%
%:%4240=1532%:%
%:%4241=1532%:%
%:%4242=1533%:%
%:%4243=1533%:%
%:%4244=1533%:%
%:%4245=1534%:%
%:%4246=1534%:%
%:%4247=1534%:%
%:%4248=1535%:%
%:%4254=1535%:%
%:%4259=1536%:%
%:%4264=1537%:%

%
\begin{isabellebody}%
\setisabellecontext{Nats}%
%
\isadelimdocument
%
\endisadelimdocument
%
\isatagdocument
%
\isamarkupsection{Natural Number Object%
}
\isamarkuptrue%
%
\endisatagdocument
{\isafolddocument}%
%
\isadelimdocument
%
\endisadelimdocument
%
\isadelimtheory
%
\endisadelimtheory
%
\isatagtheory
\isacommand{theory}\isamarkupfalse%
\ Nats\isanewline
\ \ \isakeyword{imports}\ Exponential{\isacharunderscore}{\kern0pt}Objects\isanewline
\isakeyword{begin}%
\endisatagtheory
{\isafoldtheory}%
%
\isadelimtheory
%
\endisadelimtheory
%
\begin{isamarkuptext}%
The axiomatization below corresponds to Axiom 10 (Natural Number Object) in Halvorson.%
\end{isamarkuptext}\isamarkuptrue%
\isacommand{axiomatization}\isamarkupfalse%
\isanewline
\ \ natural{\isacharunderscore}{\kern0pt}numbers\ {\isacharcolon}{\kern0pt}{\isacharcolon}{\kern0pt}\ {\isachardoublequoteopen}cset{\isachardoublequoteclose}\ {\isacharparenleft}{\kern0pt}{\isachardoublequoteopen}{\isasymnat}\isactrlsub c{\isachardoublequoteclose}{\isacharparenright}{\kern0pt}\ \isakeyword{and}\isanewline
\ \ zero\ {\isacharcolon}{\kern0pt}{\isacharcolon}{\kern0pt}\ {\isachardoublequoteopen}cfunc{\isachardoublequoteclose}\ \isakeyword{and}\isanewline
\ \ successor\ {\isacharcolon}{\kern0pt}{\isacharcolon}{\kern0pt}\ {\isachardoublequoteopen}cfunc{\isachardoublequoteclose}\isanewline
\ \ \isakeyword{where}\isanewline
\ \ zero{\isacharunderscore}{\kern0pt}type{\isacharbrackleft}{\kern0pt}type{\isacharunderscore}{\kern0pt}rule{\isacharbrackright}{\kern0pt}{\isacharcolon}{\kern0pt}\ {\isachardoublequoteopen}zero\ {\isasymin}\isactrlsub c\ {\isasymnat}\isactrlsub c{\isachardoublequoteclose}\ \isakeyword{and}\ \isanewline
\ \ successor{\isacharunderscore}{\kern0pt}type{\isacharbrackleft}{\kern0pt}type{\isacharunderscore}{\kern0pt}rule{\isacharbrackright}{\kern0pt}{\isacharcolon}{\kern0pt}\ {\isachardoublequoteopen}successor{\isacharcolon}{\kern0pt}\ {\isasymnat}\isactrlsub c\ {\isasymrightarrow}\ {\isasymnat}\isactrlsub c{\isachardoublequoteclose}\ \isakeyword{and}\ \isanewline
\ \ natural{\isacharunderscore}{\kern0pt}number{\isacharunderscore}{\kern0pt}object{\isacharunderscore}{\kern0pt}property{\isacharcolon}{\kern0pt}\ \isanewline
\ \ {\isachardoublequoteopen}q\ {\isacharcolon}{\kern0pt}\ {\isasymone}\ {\isasymrightarrow}\ X\ {\isasymLongrightarrow}\ f{\isacharcolon}{\kern0pt}\ X\ {\isasymrightarrow}\ X\ {\isasymLongrightarrow}\isanewline
\ \ \ {\isacharparenleft}{\kern0pt}{\isasymexists}{\isacharbang}{\kern0pt}u{\isachardot}{\kern0pt}\ u{\isacharcolon}{\kern0pt}\ {\isasymnat}\isactrlsub c\ {\isasymrightarrow}\ X\ {\isasymand}\isanewline
\ \ \ q\ {\isacharequal}{\kern0pt}\ u\ {\isasymcirc}\isactrlsub c\ zero\ {\isasymand}\isanewline
\ \ \ f\ {\isasymcirc}\isactrlsub c\ u\ {\isacharequal}{\kern0pt}\ u\ {\isasymcirc}\isactrlsub c\ successor{\isacharparenright}{\kern0pt}{\isachardoublequoteclose}\isanewline
\isanewline
\isacommand{lemma}\isamarkupfalse%
\ beta{\isacharunderscore}{\kern0pt}N{\isacharunderscore}{\kern0pt}succ{\isacharunderscore}{\kern0pt}nEqs{\isacharunderscore}{\kern0pt}Id{\isadigit{1}}{\isacharcolon}{\kern0pt}\isanewline
\ \ \isakeyword{assumes}\ n{\isacharunderscore}{\kern0pt}type{\isacharbrackleft}{\kern0pt}type{\isacharunderscore}{\kern0pt}rule{\isacharbrackright}{\kern0pt}{\isacharcolon}{\kern0pt}\ {\isachardoublequoteopen}n\ {\isasymin}\isactrlsub c\ {\isasymnat}\isactrlsub c{\isachardoublequoteclose}\isanewline
\ \ \isakeyword{shows}\ {\isachardoublequoteopen}{\isasymbeta}\isactrlbsub {\isasymnat}\isactrlsub c\isactrlesub \ {\isasymcirc}\isactrlsub c\ successor\ {\isasymcirc}\isactrlsub c\ n\ {\isacharequal}{\kern0pt}\ id\ {\isasymone}{\isachardoublequoteclose}\isanewline
%
\isadelimproof
\ \ %
\endisadelimproof
%
\isatagproof
\isacommand{by}\isamarkupfalse%
\ {\isacharparenleft}{\kern0pt}typecheck{\isacharunderscore}{\kern0pt}cfuncs{\isacharcomma}{\kern0pt}\ simp\ add{\isacharcolon}{\kern0pt}\ terminal{\isacharunderscore}{\kern0pt}func{\isacharunderscore}{\kern0pt}comp{\isacharunderscore}{\kern0pt}elem{\isacharparenright}{\kern0pt}%
\endisatagproof
{\isafoldproof}%
%
\isadelimproof
\isanewline
%
\endisadelimproof
\isanewline
\isacommand{lemma}\isamarkupfalse%
\ natural{\isacharunderscore}{\kern0pt}number{\isacharunderscore}{\kern0pt}object{\isacharunderscore}{\kern0pt}property{\isadigit{2}}{\isacharcolon}{\kern0pt}\isanewline
\ \ \isakeyword{assumes}\ {\isachardoublequoteopen}q\ {\isacharcolon}{\kern0pt}\ {\isasymone}\ {\isasymrightarrow}\ X{\isachardoublequoteclose}\ {\isachardoublequoteopen}f{\isacharcolon}{\kern0pt}\ X\ {\isasymrightarrow}\ X{\isachardoublequoteclose}\isanewline
\ \ \isakeyword{shows}\ {\isachardoublequoteopen}{\isasymexists}{\isacharbang}{\kern0pt}u{\isachardot}{\kern0pt}\ u{\isacharcolon}{\kern0pt}\ {\isasymnat}\isactrlsub c\ {\isasymrightarrow}\ X\ {\isasymand}\ u\ {\isasymcirc}\isactrlsub c\ zero\ {\isacharequal}{\kern0pt}\ q\ {\isasymand}\ f\ {\isasymcirc}\isactrlsub c\ u\ {\isacharequal}{\kern0pt}\ u\ {\isasymcirc}\isactrlsub c\ successor{\isachardoublequoteclose}\isanewline
%
\isadelimproof
\ \ %
\endisadelimproof
%
\isatagproof
\isacommand{using}\isamarkupfalse%
\ assms\ natural{\isacharunderscore}{\kern0pt}number{\isacharunderscore}{\kern0pt}object{\isacharunderscore}{\kern0pt}property{\isacharbrackleft}{\kern0pt}\isakeyword{where}\ q{\isacharequal}{\kern0pt}q{\isacharcomma}{\kern0pt}\ \isakeyword{where}\ f{\isacharequal}{\kern0pt}f{\isacharcomma}{\kern0pt}\ \isakeyword{where}\ X{\isacharequal}{\kern0pt}X{\isacharbrackright}{\kern0pt}\isanewline
\ \ \isacommand{by}\isamarkupfalse%
\ metis%
\endisatagproof
{\isafoldproof}%
%
\isadelimproof
\isanewline
%
\endisadelimproof
\isanewline
\isacommand{lemma}\isamarkupfalse%
\ natural{\isacharunderscore}{\kern0pt}number{\isacharunderscore}{\kern0pt}object{\isacharunderscore}{\kern0pt}func{\isacharunderscore}{\kern0pt}unique{\isacharcolon}{\kern0pt}\isanewline
\ \ \isakeyword{assumes}\ u{\isacharunderscore}{\kern0pt}type{\isacharcolon}{\kern0pt}\ {\isachardoublequoteopen}u\ {\isacharcolon}{\kern0pt}\ {\isasymnat}\isactrlsub c\ {\isasymrightarrow}\ X{\isachardoublequoteclose}\ \isakeyword{and}\ v{\isacharunderscore}{\kern0pt}type{\isacharcolon}{\kern0pt}\ {\isachardoublequoteopen}v\ {\isacharcolon}{\kern0pt}\ {\isasymnat}\isactrlsub c\ {\isasymrightarrow}\ X{\isachardoublequoteclose}\ \isakeyword{and}\ f{\isacharunderscore}{\kern0pt}type{\isacharcolon}{\kern0pt}\ {\isachardoublequoteopen}f{\isacharcolon}{\kern0pt}\ X\ {\isasymrightarrow}\ X{\isachardoublequoteclose}\isanewline
\ \ \isakeyword{assumes}\ zeros{\isacharunderscore}{\kern0pt}eq{\isacharcolon}{\kern0pt}\ {\isachardoublequoteopen}u\ {\isasymcirc}\isactrlsub c\ zero\ {\isacharequal}{\kern0pt}\ v\ {\isasymcirc}\isactrlsub c\ zero{\isachardoublequoteclose}\isanewline
\ \ \isakeyword{assumes}\ u{\isacharunderscore}{\kern0pt}successor{\isacharunderscore}{\kern0pt}eq{\isacharcolon}{\kern0pt}\ {\isachardoublequoteopen}u\ {\isasymcirc}\isactrlsub c\ successor\ {\isacharequal}{\kern0pt}\ f\ {\isasymcirc}\isactrlsub c\ u{\isachardoublequoteclose}\isanewline
\ \ \isakeyword{assumes}\ v{\isacharunderscore}{\kern0pt}successor{\isacharunderscore}{\kern0pt}eq{\isacharcolon}{\kern0pt}\ {\isachardoublequoteopen}v\ {\isasymcirc}\isactrlsub c\ successor\ {\isacharequal}{\kern0pt}\ f\ {\isasymcirc}\isactrlsub c\ v{\isachardoublequoteclose}\isanewline
\ \ \isakeyword{shows}\ {\isachardoublequoteopen}u\ {\isacharequal}{\kern0pt}\ v{\isachardoublequoteclose}\isanewline
%
\isadelimproof
\ \ %
\endisadelimproof
%
\isatagproof
\isacommand{by}\isamarkupfalse%
\ {\isacharparenleft}{\kern0pt}smt\ {\isacharparenleft}{\kern0pt}verit{\isacharcomma}{\kern0pt}\ best{\isacharparenright}{\kern0pt}\ comp{\isacharunderscore}{\kern0pt}type\ f{\isacharunderscore}{\kern0pt}type\ natural{\isacharunderscore}{\kern0pt}number{\isacharunderscore}{\kern0pt}object{\isacharunderscore}{\kern0pt}property{\isadigit{2}}\ u{\isacharunderscore}{\kern0pt}successor{\isacharunderscore}{\kern0pt}eq\ u{\isacharunderscore}{\kern0pt}type\ v{\isacharunderscore}{\kern0pt}successor{\isacharunderscore}{\kern0pt}eq\ v{\isacharunderscore}{\kern0pt}type\ zero{\isacharunderscore}{\kern0pt}type\ zeros{\isacharunderscore}{\kern0pt}eq{\isacharparenright}{\kern0pt}%
\endisatagproof
{\isafoldproof}%
%
\isadelimproof
\isanewline
%
\endisadelimproof
\isanewline
\isacommand{definition}\isamarkupfalse%
\ is{\isacharunderscore}{\kern0pt}NNO\ {\isacharcolon}{\kern0pt}{\isacharcolon}{\kern0pt}\ {\isachardoublequoteopen}cset\ {\isasymRightarrow}\ cfunc\ {\isasymRightarrow}\ cfunc\ {\isasymRightarrow}\ bool{\isachardoublequoteclose}\ \ \isakeyword{where}\isanewline
\ \ \ {\isachardoublequoteopen}is{\isacharunderscore}{\kern0pt}NNO\ Y\ z\ s\ {\isasymlongleftrightarrow}{\isacharparenleft}{\kern0pt}z{\isacharcolon}{\kern0pt}\ {\isasymone}\ {\isasymrightarrow}\ Y\ {\isasymand}\ s{\isacharcolon}{\kern0pt}\ Y\ {\isasymrightarrow}\ Y\ \ {\isasymand}\ {\isacharparenleft}{\kern0pt}{\isasymforall}\ X\ f\ q{\isachardot}{\kern0pt}\ {\isacharparenleft}{\kern0pt}{\isacharparenleft}{\kern0pt}q\ {\isacharcolon}{\kern0pt}\ {\isasymone}\ {\isasymrightarrow}\ X{\isacharparenright}{\kern0pt}\ {\isasymand}\ {\isacharparenleft}{\kern0pt}f{\isacharcolon}{\kern0pt}\ X\ {\isasymrightarrow}\ X{\isacharparenright}{\kern0pt}{\isacharparenright}{\kern0pt}{\isasymlongrightarrow}\isanewline
\ \ \ {\isacharparenleft}{\kern0pt}{\isasymexists}{\isacharbang}{\kern0pt}u{\isachardot}{\kern0pt}\ u{\isacharcolon}{\kern0pt}\ Y\ {\isasymrightarrow}\ X\ {\isasymand}\isanewline
\ \ \ q\ {\isacharequal}{\kern0pt}\ u\ {\isasymcirc}\isactrlsub c\ z\ {\isasymand}\isanewline
\ \ \ f\ {\isasymcirc}\isactrlsub c\ u\ {\isacharequal}{\kern0pt}\ u\ {\isasymcirc}\isactrlsub c\ s{\isacharparenright}{\kern0pt}{\isacharparenright}{\kern0pt}{\isacharparenright}{\kern0pt}{\isachardoublequoteclose}\isanewline
\isanewline
\isacommand{lemma}\isamarkupfalse%
\ N{\isacharunderscore}{\kern0pt}is{\isacharunderscore}{\kern0pt}a{\isacharunderscore}{\kern0pt}NNO{\isacharcolon}{\kern0pt}\isanewline
\ \ \ \ {\isachardoublequoteopen}is{\isacharunderscore}{\kern0pt}NNO\ {\isasymnat}\isactrlsub c\ zero\ successor{\isachardoublequoteclose}\isanewline
%
\isadelimproof
%
\endisadelimproof
%
\isatagproof
\isacommand{by}\isamarkupfalse%
\ {\isacharparenleft}{\kern0pt}simp\ add{\isacharcolon}{\kern0pt}\ is{\isacharunderscore}{\kern0pt}NNO{\isacharunderscore}{\kern0pt}def\ natural{\isacharunderscore}{\kern0pt}number{\isacharunderscore}{\kern0pt}object{\isacharunderscore}{\kern0pt}property\ successor{\isacharunderscore}{\kern0pt}type\ zero{\isacharunderscore}{\kern0pt}type{\isacharparenright}{\kern0pt}%
\endisatagproof
{\isafoldproof}%
%
\isadelimproof
%
\endisadelimproof
%
\begin{isamarkuptext}%
The lemma below corresponds to Exercise 2.6.5 in Halvorson.%
\end{isamarkuptext}\isamarkuptrue%
\isacommand{lemma}\isamarkupfalse%
\ NNOs{\isacharunderscore}{\kern0pt}are{\isacharunderscore}{\kern0pt}iso{\isacharunderscore}{\kern0pt}N{\isacharcolon}{\kern0pt}\isanewline
\ \ \isakeyword{assumes}\ {\isachardoublequoteopen}is{\isacharunderscore}{\kern0pt}NNO\ N\ z\ s{\isachardoublequoteclose}\isanewline
\ \ \isakeyword{shows}\ {\isachardoublequoteopen}N\ {\isasymcong}\ {\isasymnat}\isactrlsub c{\isachardoublequoteclose}\isanewline
%
\isadelimproof
%
\endisadelimproof
%
\isatagproof
\isacommand{proof}\isamarkupfalse%
{\isacharminus}{\kern0pt}\isanewline
\ \ \isacommand{have}\isamarkupfalse%
\ z{\isacharunderscore}{\kern0pt}type{\isacharbrackleft}{\kern0pt}type{\isacharunderscore}{\kern0pt}rule{\isacharbrackright}{\kern0pt}{\isacharcolon}{\kern0pt}\ {\isachardoublequoteopen}{\isacharparenleft}{\kern0pt}z\ {\isacharcolon}{\kern0pt}\ {\isasymone}\ {\isasymrightarrow}\ \ N{\isacharparenright}{\kern0pt}{\isachardoublequoteclose}\ \isanewline
\ \ \ \ \isacommand{using}\isamarkupfalse%
\ assms\ is{\isacharunderscore}{\kern0pt}NNO{\isacharunderscore}{\kern0pt}def\ \isacommand{by}\isamarkupfalse%
\ blast\isanewline
\ \ \isacommand{have}\isamarkupfalse%
\ s{\isacharunderscore}{\kern0pt}type{\isacharbrackleft}{\kern0pt}type{\isacharunderscore}{\kern0pt}rule{\isacharbrackright}{\kern0pt}{\isacharcolon}{\kern0pt}\ {\isachardoublequoteopen}{\isacharparenleft}{\kern0pt}s\ {\isacharcolon}{\kern0pt}\ N\ {\isasymrightarrow}\ \ N{\isacharparenright}{\kern0pt}{\isachardoublequoteclose}\isanewline
\ \ \ \ \isacommand{using}\isamarkupfalse%
\ assms\ is{\isacharunderscore}{\kern0pt}NNO{\isacharunderscore}{\kern0pt}def\ \isacommand{by}\isamarkupfalse%
\ blast\ \isanewline
\ \ \isacommand{then}\isamarkupfalse%
\ \isacommand{obtain}\isamarkupfalse%
\ u\ \isakeyword{where}\ u{\isacharunderscore}{\kern0pt}type{\isacharbrackleft}{\kern0pt}type{\isacharunderscore}{\kern0pt}rule{\isacharbrackright}{\kern0pt}{\isacharcolon}{\kern0pt}\ {\isachardoublequoteopen}u{\isacharcolon}{\kern0pt}\ {\isasymnat}\isactrlsub c\ {\isasymrightarrow}\ N{\isachardoublequoteclose}\ \isanewline
\ \ \ \ \ \ \ \ \ \ \ \ \ \ \ \ \ \isakeyword{and}\ \ u{\isacharunderscore}{\kern0pt}triangle{\isacharcolon}{\kern0pt}\ {\isachardoublequoteopen}u\ {\isasymcirc}\isactrlsub c\ zero\ {\isacharequal}{\kern0pt}\ z{\isachardoublequoteclose}\ \isanewline
\ \ \ \ \ \ \ \ \ \ \ \ \ \ \ \ \ \isakeyword{and}\ \ u{\isacharunderscore}{\kern0pt}square{\isacharcolon}{\kern0pt}\ {\isachardoublequoteopen}s\ {\isasymcirc}\isactrlsub c\ u\ {\isacharequal}{\kern0pt}\ u\ {\isasymcirc}\isactrlsub c\ successor{\isachardoublequoteclose}\isanewline
\ \ \ \ \isacommand{using}\isamarkupfalse%
\ natural{\isacharunderscore}{\kern0pt}number{\isacharunderscore}{\kern0pt}object{\isacharunderscore}{\kern0pt}property\ z{\isacharunderscore}{\kern0pt}type\ \isacommand{by}\isamarkupfalse%
\ blast\isanewline
\ \ \isacommand{obtain}\isamarkupfalse%
\ v\ \isakeyword{where}\ v{\isacharunderscore}{\kern0pt}type{\isacharbrackleft}{\kern0pt}type{\isacharunderscore}{\kern0pt}rule{\isacharbrackright}{\kern0pt}{\isacharcolon}{\kern0pt}\ {\isachardoublequoteopen}v{\isacharcolon}{\kern0pt}\ N\ {\isasymrightarrow}\ {\isasymnat}\isactrlsub c{\isachardoublequoteclose}\ \isanewline
\ \ \ \ \ \ \ \ \ \ \ \ \ \ \ \ \ \isakeyword{and}\ \ v{\isacharunderscore}{\kern0pt}triangle{\isacharcolon}{\kern0pt}\ {\isachardoublequoteopen}v\ {\isasymcirc}\isactrlsub c\ z\ {\isacharequal}{\kern0pt}\ zero{\isachardoublequoteclose}\ \isanewline
\ \ \ \ \ \ \ \ \ \ \ \ \ \ \ \ \ \isakeyword{and}\ \ v{\isacharunderscore}{\kern0pt}square{\isacharcolon}{\kern0pt}\ {\isachardoublequoteopen}successor\ {\isasymcirc}\isactrlsub c\ v\ {\isacharequal}{\kern0pt}\ v\ {\isasymcirc}\isactrlsub c\ s{\isachardoublequoteclose}\isanewline
\ \ \ \ \isacommand{by}\isamarkupfalse%
\ {\isacharparenleft}{\kern0pt}metis\ assms\ is{\isacharunderscore}{\kern0pt}NNO{\isacharunderscore}{\kern0pt}def\ successor{\isacharunderscore}{\kern0pt}type\ zero{\isacharunderscore}{\kern0pt}type{\isacharparenright}{\kern0pt}\isanewline
\ \ \isacommand{then}\isamarkupfalse%
\ \isacommand{have}\isamarkupfalse%
\ vuzeroEqzero{\isacharcolon}{\kern0pt}\ {\isachardoublequoteopen}v\ {\isasymcirc}\isactrlsub c\ {\isacharparenleft}{\kern0pt}u\ {\isasymcirc}\isactrlsub c\ zero{\isacharparenright}{\kern0pt}\ {\isacharequal}{\kern0pt}\ zero{\isachardoublequoteclose}\isanewline
\ \ \ \ \isacommand{by}\isamarkupfalse%
\ {\isacharparenleft}{\kern0pt}simp\ add{\isacharcolon}{\kern0pt}\ u{\isacharunderscore}{\kern0pt}triangle\ v{\isacharunderscore}{\kern0pt}triangle{\isacharparenright}{\kern0pt}\isanewline
\ \ \isacommand{have}\isamarkupfalse%
\ id{\isacharunderscore}{\kern0pt}facts{\isadigit{1}}{\isacharcolon}{\kern0pt}\ {\isachardoublequoteopen}id{\isacharparenleft}{\kern0pt}{\isasymnat}\isactrlsub c{\isacharparenright}{\kern0pt}{\isacharcolon}{\kern0pt}\ {\isasymnat}\isactrlsub c\ {\isasymrightarrow}\ {\isasymnat}\isactrlsub c{\isasymand}\ id{\isacharparenleft}{\kern0pt}{\isasymnat}\isactrlsub c{\isacharparenright}{\kern0pt}\ {\isasymcirc}\isactrlsub c\ zero\ {\isacharequal}{\kern0pt}\ zero\ {\isasymand}\isanewline
\ \ \ \ \ \ \ \ \ \ {\isacharparenleft}{\kern0pt}successor\ {\isasymcirc}\isactrlsub c\ id{\isacharparenleft}{\kern0pt}{\isasymnat}\isactrlsub c{\isacharparenright}{\kern0pt}\ {\isacharequal}{\kern0pt}\ id{\isacharparenleft}{\kern0pt}{\isasymnat}\isactrlsub c{\isacharparenright}{\kern0pt}\ {\isasymcirc}\isactrlsub c\ successor{\isacharparenright}{\kern0pt}{\isachardoublequoteclose}\isanewline
\ \ \ \ \isacommand{by}\isamarkupfalse%
\ {\isacharparenleft}{\kern0pt}typecheck{\isacharunderscore}{\kern0pt}cfuncs{\isacharcomma}{\kern0pt}\ simp\ add{\isacharcolon}{\kern0pt}\ id{\isacharunderscore}{\kern0pt}left{\isacharunderscore}{\kern0pt}unit{\isadigit{2}}\ id{\isacharunderscore}{\kern0pt}right{\isacharunderscore}{\kern0pt}unit{\isadigit{2}}{\isacharparenright}{\kern0pt}\isanewline
\ \ \isacommand{then}\isamarkupfalse%
\ \isacommand{have}\isamarkupfalse%
\ vu{\isacharunderscore}{\kern0pt}facts{\isacharcolon}{\kern0pt}\ {\isachardoublequoteopen}v\ {\isasymcirc}\isactrlsub c\ u{\isacharcolon}{\kern0pt}\ {\isasymnat}\isactrlsub c\ {\isasymrightarrow}\ {\isasymnat}\isactrlsub c{\isasymand}\ {\isacharparenleft}{\kern0pt}v\ {\isasymcirc}\isactrlsub c\ u{\isacharparenright}{\kern0pt}\ {\isasymcirc}\isactrlsub c\ zero\ {\isacharequal}{\kern0pt}\ zero\ {\isasymand}\ \isanewline
\ \ \ \ \ \ \ \ \ \ successor\ {\isasymcirc}\isactrlsub c\ {\isacharparenleft}{\kern0pt}v\ {\isasymcirc}\isactrlsub c\ u{\isacharparenright}{\kern0pt}\ {\isacharequal}{\kern0pt}\ {\isacharparenleft}{\kern0pt}v\ {\isasymcirc}\isactrlsub c\ u{\isacharparenright}{\kern0pt}\ {\isasymcirc}\isactrlsub c\ successor{\isachardoublequoteclose}\isanewline
\ \ \ \ \isacommand{by}\isamarkupfalse%
\ {\isacharparenleft}{\kern0pt}typecheck{\isacharunderscore}{\kern0pt}cfuncs{\isacharcomma}{\kern0pt}\ smt\ {\isacharparenleft}{\kern0pt}verit{\isacharcomma}{\kern0pt}\ best{\isacharparenright}{\kern0pt}\ comp{\isacharunderscore}{\kern0pt}associative{\isadigit{2}}\ s{\isacharunderscore}{\kern0pt}type\ u{\isacharunderscore}{\kern0pt}square\ v{\isacharunderscore}{\kern0pt}square\ vuzeroEqzero{\isacharparenright}{\kern0pt}\isanewline
\ \ \isacommand{then}\isamarkupfalse%
\ \isacommand{have}\isamarkupfalse%
\ half{\isacharunderscore}{\kern0pt}isomorphism{\isacharcolon}{\kern0pt}\ {\isachardoublequoteopen}{\isacharparenleft}{\kern0pt}v\ {\isasymcirc}\isactrlsub c\ u{\isacharparenright}{\kern0pt}\ {\isacharequal}{\kern0pt}\ id{\isacharparenleft}{\kern0pt}{\isasymnat}\isactrlsub c{\isacharparenright}{\kern0pt}{\isachardoublequoteclose}\isanewline
\ \ \ \ \isacommand{by}\isamarkupfalse%
\ {\isacharparenleft}{\kern0pt}metis\ id{\isacharunderscore}{\kern0pt}facts{\isadigit{1}}\ natural{\isacharunderscore}{\kern0pt}number{\isacharunderscore}{\kern0pt}object{\isacharunderscore}{\kern0pt}property\ successor{\isacharunderscore}{\kern0pt}type\ vu{\isacharunderscore}{\kern0pt}facts\ zero{\isacharunderscore}{\kern0pt}type{\isacharparenright}{\kern0pt}\isanewline
\ \ \isacommand{have}\isamarkupfalse%
\ uvzEqz{\isacharcolon}{\kern0pt}\ {\isachardoublequoteopen}u\ {\isasymcirc}\isactrlsub c\ {\isacharparenleft}{\kern0pt}v\ {\isasymcirc}\isactrlsub c\ z{\isacharparenright}{\kern0pt}\ {\isacharequal}{\kern0pt}\ z{\isachardoublequoteclose}\isanewline
\ \ \ \ \isacommand{by}\isamarkupfalse%
\ {\isacharparenleft}{\kern0pt}simp\ add{\isacharcolon}{\kern0pt}\ u{\isacharunderscore}{\kern0pt}triangle\ v{\isacharunderscore}{\kern0pt}triangle{\isacharparenright}{\kern0pt}\isanewline
\ \ \isacommand{have}\isamarkupfalse%
\ id{\isacharunderscore}{\kern0pt}facts{\isadigit{2}}{\isacharcolon}{\kern0pt}\ {\isachardoublequoteopen}id{\isacharparenleft}{\kern0pt}N{\isacharparenright}{\kern0pt}{\isacharcolon}{\kern0pt}\ N\ {\isasymrightarrow}\ N\ {\isasymand}\ id{\isacharparenleft}{\kern0pt}N{\isacharparenright}{\kern0pt}\ {\isasymcirc}\isactrlsub c\ z\ {\isacharequal}{\kern0pt}\ z\ {\isasymand}\ s\ {\isasymcirc}\isactrlsub c\ id{\isacharparenleft}{\kern0pt}N{\isacharparenright}{\kern0pt}\ {\isacharequal}{\kern0pt}\ id{\isacharparenleft}{\kern0pt}N{\isacharparenright}{\kern0pt}\ {\isasymcirc}\isactrlsub c\ \ s{\isachardoublequoteclose}\isanewline
\ \ \ \ \isacommand{by}\isamarkupfalse%
\ {\isacharparenleft}{\kern0pt}typecheck{\isacharunderscore}{\kern0pt}cfuncs{\isacharcomma}{\kern0pt}\ simp\ add{\isacharcolon}{\kern0pt}\ id{\isacharunderscore}{\kern0pt}left{\isacharunderscore}{\kern0pt}unit{\isadigit{2}}\ id{\isacharunderscore}{\kern0pt}right{\isacharunderscore}{\kern0pt}unit{\isadigit{2}}{\isacharparenright}{\kern0pt}\isanewline
\ \ \isacommand{then}\isamarkupfalse%
\ \isacommand{have}\isamarkupfalse%
\ uv{\isacharunderscore}{\kern0pt}facts{\isacharcolon}{\kern0pt}\ {\isachardoublequoteopen}u\ {\isasymcirc}\isactrlsub c\ v{\isacharcolon}{\kern0pt}\ N\ {\isasymrightarrow}\ N\ {\isasymand}\isanewline
\ \ \ \ \ \ \ \ \ \ {\isacharparenleft}{\kern0pt}u\ {\isasymcirc}\isactrlsub c\ v{\isacharparenright}{\kern0pt}\ {\isasymcirc}\isactrlsub c\ \ z\ {\isacharequal}{\kern0pt}\ z\ {\isasymand}\ \ s\ {\isasymcirc}\isactrlsub c\ {\isacharparenleft}{\kern0pt}u\ {\isasymcirc}\isactrlsub c\ v{\isacharparenright}{\kern0pt}\ {\isacharequal}{\kern0pt}\ \ {\isacharparenleft}{\kern0pt}u\ {\isasymcirc}\isactrlsub c\ v{\isacharparenright}{\kern0pt}\ {\isasymcirc}\isactrlsub c\ s{\isachardoublequoteclose}\isanewline
\ \ \ \ \isacommand{by}\isamarkupfalse%
\ {\isacharparenleft}{\kern0pt}typecheck{\isacharunderscore}{\kern0pt}cfuncs{\isacharcomma}{\kern0pt}\ smt\ {\isacharparenleft}{\kern0pt}verit{\isacharcomma}{\kern0pt}\ best{\isacharparenright}{\kern0pt}\ comp{\isacharunderscore}{\kern0pt}associative{\isadigit{2}}\ successor{\isacharunderscore}{\kern0pt}type\ u{\isacharunderscore}{\kern0pt}square\ uvzEqz\ v{\isacharunderscore}{\kern0pt}square{\isacharparenright}{\kern0pt}\isanewline
\ \isacommand{then}\isamarkupfalse%
\ \isacommand{have}\isamarkupfalse%
\ half{\isacharunderscore}{\kern0pt}isomorphism{\isadigit{2}}{\isacharcolon}{\kern0pt}\ {\isachardoublequoteopen}{\isacharparenleft}{\kern0pt}u\ {\isasymcirc}\isactrlsub c\ v{\isacharparenright}{\kern0pt}\ {\isacharequal}{\kern0pt}\ id{\isacharparenleft}{\kern0pt}N{\isacharparenright}{\kern0pt}{\isachardoublequoteclose}\isanewline
\ \ \ \isacommand{by}\isamarkupfalse%
\ {\isacharparenleft}{\kern0pt}smt\ {\isacharparenleft}{\kern0pt}verit{\isacharcomma}{\kern0pt}\ ccfv{\isacharunderscore}{\kern0pt}threshold{\isacharparenright}{\kern0pt}\ assms\ id{\isacharunderscore}{\kern0pt}facts{\isadigit{2}}\ is{\isacharunderscore}{\kern0pt}NNO{\isacharunderscore}{\kern0pt}def{\isacharparenright}{\kern0pt}\isanewline
\ \ \isacommand{then}\isamarkupfalse%
\ \isacommand{show}\isamarkupfalse%
\ {\isachardoublequoteopen}N\ {\isasymcong}\ {\isasymnat}\isactrlsub c{\isachardoublequoteclose}\isanewline
\ \ \ \ \isacommand{using}\isamarkupfalse%
\ cfunc{\isacharunderscore}{\kern0pt}type{\isacharunderscore}{\kern0pt}def\ half{\isacharunderscore}{\kern0pt}isomorphism\ is{\isacharunderscore}{\kern0pt}isomorphic{\isacharunderscore}{\kern0pt}def\ isomorphism{\isacharunderscore}{\kern0pt}def\ u{\isacharunderscore}{\kern0pt}type\ v{\isacharunderscore}{\kern0pt}type\ \isacommand{by}\isamarkupfalse%
\ fastforce\isanewline
\isacommand{qed}\isamarkupfalse%
%
\endisatagproof
{\isafoldproof}%
%
\isadelimproof
%
\endisadelimproof
%
\begin{isamarkuptext}%
The lemma below is the converse to Exercise 2.6.5 in Halvorson.%
\end{isamarkuptext}\isamarkuptrue%
\isacommand{lemma}\isamarkupfalse%
\ Iso{\isacharunderscore}{\kern0pt}to{\isacharunderscore}{\kern0pt}N{\isacharunderscore}{\kern0pt}is{\isacharunderscore}{\kern0pt}NNO{\isacharcolon}{\kern0pt}\isanewline
\ \ \isakeyword{assumes}\ {\isachardoublequoteopen}N\ {\isasymcong}\ {\isasymnat}\isactrlsub c{\isachardoublequoteclose}\isanewline
\ \ \isakeyword{shows}\ {\isachardoublequoteopen}{\isasymexists}\ z\ s{\isachardot}{\kern0pt}\ is{\isacharunderscore}{\kern0pt}NNO\ N\ z\ s{\isachardoublequoteclose}\isanewline
%
\isadelimproof
%
\endisadelimproof
%
\isatagproof
\isacommand{proof}\isamarkupfalse%
\ {\isacharminus}{\kern0pt}\ \isanewline
\ \ \isacommand{obtain}\isamarkupfalse%
\ i\ \isakeyword{where}\ i{\isacharunderscore}{\kern0pt}type{\isacharbrackleft}{\kern0pt}type{\isacharunderscore}{\kern0pt}rule{\isacharbrackright}{\kern0pt}{\isacharcolon}{\kern0pt}\ {\isachardoublequoteopen}i{\isacharcolon}{\kern0pt}\ {\isasymnat}\isactrlsub c\ {\isasymrightarrow}\ N{\isachardoublequoteclose}\ \isakeyword{and}\ i{\isacharunderscore}{\kern0pt}iso{\isacharcolon}{\kern0pt}\ {\isachardoublequoteopen}isomorphism{\isacharparenleft}{\kern0pt}i{\isacharparenright}{\kern0pt}{\isachardoublequoteclose}\isanewline
\ \ \ \ \isacommand{using}\isamarkupfalse%
\ assms\ isomorphic{\isacharunderscore}{\kern0pt}is{\isacharunderscore}{\kern0pt}symmetric\ is{\isacharunderscore}{\kern0pt}isomorphic{\isacharunderscore}{\kern0pt}def\ \isacommand{by}\isamarkupfalse%
\ blast\ \isanewline
\ \ \isacommand{obtain}\isamarkupfalse%
\ z\ \isakeyword{where}\ z{\isacharunderscore}{\kern0pt}type{\isacharbrackleft}{\kern0pt}type{\isacharunderscore}{\kern0pt}rule{\isacharbrackright}{\kern0pt}{\isacharcolon}{\kern0pt}\ {\isachardoublequoteopen}z\ {\isasymin}\isactrlsub c\ N{\isachardoublequoteclose}\ \isakeyword{and}\ z{\isacharunderscore}{\kern0pt}def{\isacharcolon}{\kern0pt}\ {\isachardoublequoteopen}z\ {\isacharequal}{\kern0pt}\ i\ {\isasymcirc}\isactrlsub c\ zero{\isachardoublequoteclose}\isanewline
\ \ \ \ \isacommand{by}\isamarkupfalse%
\ {\isacharparenleft}{\kern0pt}typecheck{\isacharunderscore}{\kern0pt}cfuncs{\isacharcomma}{\kern0pt}\ simp{\isacharparenright}{\kern0pt}\isanewline
\ \ \isacommand{obtain}\isamarkupfalse%
\ s\ \isakeyword{where}\ s{\isacharunderscore}{\kern0pt}type{\isacharbrackleft}{\kern0pt}type{\isacharunderscore}{\kern0pt}rule{\isacharbrackright}{\kern0pt}{\isacharcolon}{\kern0pt}\ {\isachardoublequoteopen}s{\isacharcolon}{\kern0pt}\ N\ {\isasymrightarrow}\ N{\isachardoublequoteclose}\ \isakeyword{and}\ s{\isacharunderscore}{\kern0pt}def{\isacharcolon}{\kern0pt}\ {\isachardoublequoteopen}s\ {\isacharequal}{\kern0pt}\ {\isacharparenleft}{\kern0pt}i\ {\isasymcirc}\isactrlsub c\ successor{\isacharparenright}{\kern0pt}\ {\isasymcirc}\isactrlsub c\ i\isactrlbold {\isasyminverse}{\isachardoublequoteclose}\isanewline
\ \ \ \ \isacommand{using}\isamarkupfalse%
\ i{\isacharunderscore}{\kern0pt}iso\ \isacommand{by}\isamarkupfalse%
\ {\isacharparenleft}{\kern0pt}typecheck{\isacharunderscore}{\kern0pt}cfuncs{\isacharcomma}{\kern0pt}\ simp{\isacharparenright}{\kern0pt}\isanewline
\ \ \isacommand{have}\isamarkupfalse%
\ {\isachardoublequoteopen}is{\isacharunderscore}{\kern0pt}NNO\ N\ z\ s{\isachardoublequoteclose}\isanewline
\ \ \isacommand{proof}\isamarkupfalse%
{\isacharparenleft}{\kern0pt}unfold\ is{\isacharunderscore}{\kern0pt}NNO{\isacharunderscore}{\kern0pt}def{\isacharcomma}{\kern0pt}\ typecheck{\isacharunderscore}{\kern0pt}cfuncs{\isacharparenright}{\kern0pt}\isanewline
\ \ \ \ \isacommand{fix}\isamarkupfalse%
\ X\ q\ f\ \isanewline
\ \ \ \ \isacommand{assume}\isamarkupfalse%
\ q{\isacharunderscore}{\kern0pt}type{\isacharbrackleft}{\kern0pt}type{\isacharunderscore}{\kern0pt}rule{\isacharbrackright}{\kern0pt}{\isacharcolon}{\kern0pt}\ {\isachardoublequoteopen}q{\isacharcolon}{\kern0pt}\ {\isasymone}\ {\isasymrightarrow}\ X{\isachardoublequoteclose}\isanewline
\ \ \ \ \isacommand{assume}\isamarkupfalse%
\ f{\isacharunderscore}{\kern0pt}type{\isacharbrackleft}{\kern0pt}type{\isacharunderscore}{\kern0pt}rule{\isacharbrackright}{\kern0pt}{\isacharcolon}{\kern0pt}\ {\isachardoublequoteopen}f{\isacharcolon}{\kern0pt}\ \ \ X\ {\isasymrightarrow}\ X{\isachardoublequoteclose}\isanewline
\isanewline
\ \ \ \ \isacommand{obtain}\isamarkupfalse%
\ u\ \isakeyword{where}\ u{\isacharunderscore}{\kern0pt}type{\isacharbrackleft}{\kern0pt}type{\isacharunderscore}{\kern0pt}rule{\isacharbrackright}{\kern0pt}{\isacharcolon}{\kern0pt}\ {\isachardoublequoteopen}u{\isacharcolon}{\kern0pt}\ {\isasymnat}\isactrlsub c\ {\isasymrightarrow}\ X{\isachardoublequoteclose}\ \isakeyword{and}\ u{\isacharunderscore}{\kern0pt}def{\isacharcolon}{\kern0pt}\ \ {\isachardoublequoteopen}u\ {\isasymcirc}\isactrlsub c\ zero\ {\isacharequal}{\kern0pt}\ \ q\ {\isasymand}\ f\ {\isasymcirc}\isactrlsub c\ u\ {\isacharequal}{\kern0pt}\ u\ {\isasymcirc}\isactrlsub c\ successor{\isachardoublequoteclose}\isanewline
\ \ \ \ \ \ \isacommand{using}\isamarkupfalse%
\ natural{\isacharunderscore}{\kern0pt}number{\isacharunderscore}{\kern0pt}object{\isacharunderscore}{\kern0pt}property{\isadigit{2}}\ \isacommand{by}\isamarkupfalse%
\ {\isacharparenleft}{\kern0pt}typecheck{\isacharunderscore}{\kern0pt}cfuncs{\isacharcomma}{\kern0pt}\ blast{\isacharparenright}{\kern0pt}\isanewline
\ \ \ \ \isacommand{obtain}\isamarkupfalse%
\ v\ \isakeyword{where}\ v{\isacharunderscore}{\kern0pt}type{\isacharbrackleft}{\kern0pt}type{\isacharunderscore}{\kern0pt}rule{\isacharbrackright}{\kern0pt}{\isacharcolon}{\kern0pt}\ {\isachardoublequoteopen}v{\isacharcolon}{\kern0pt}\ N\ {\isasymrightarrow}\ X{\isachardoublequoteclose}\ \isakeyword{and}\ v{\isacharunderscore}{\kern0pt}def{\isacharcolon}{\kern0pt}\ {\isachardoublequoteopen}v\ {\isacharequal}{\kern0pt}\ u\ {\isasymcirc}\isactrlsub c\ i\isactrlbold {\isasyminverse}{\isachardoublequoteclose}\isanewline
\ \ \ \ \ \ \isacommand{using}\isamarkupfalse%
\ i{\isacharunderscore}{\kern0pt}iso\ \isacommand{by}\isamarkupfalse%
\ {\isacharparenleft}{\kern0pt}typecheck{\isacharunderscore}{\kern0pt}cfuncs{\isacharcomma}{\kern0pt}\ simp{\isacharparenright}{\kern0pt}\isanewline
\ \ \ \ \isacommand{then}\isamarkupfalse%
\ \isacommand{have}\isamarkupfalse%
\ bottom{\isacharunderscore}{\kern0pt}triangle{\isacharcolon}{\kern0pt}\ {\isachardoublequoteopen}v\ {\isasymcirc}\isactrlsub c\ z\ {\isacharequal}{\kern0pt}\ q{\isachardoublequoteclose}\isanewline
\ \ \ \ \ \ \isacommand{unfolding}\isamarkupfalse%
\ v{\isacharunderscore}{\kern0pt}def\ u{\isacharunderscore}{\kern0pt}def\ z{\isacharunderscore}{\kern0pt}def\ \isacommand{using}\isamarkupfalse%
\ i{\isacharunderscore}{\kern0pt}iso\isanewline
\ \ \ \ \ \ \isacommand{by}\isamarkupfalse%
\ {\isacharparenleft}{\kern0pt}typecheck{\isacharunderscore}{\kern0pt}cfuncs{\isacharcomma}{\kern0pt}\ metis\ cfunc{\isacharunderscore}{\kern0pt}type{\isacharunderscore}{\kern0pt}def\ comp{\isacharunderscore}{\kern0pt}associative\ id{\isacharunderscore}{\kern0pt}right{\isacharunderscore}{\kern0pt}unit{\isadigit{2}}\ inv{\isacharunderscore}{\kern0pt}left\ u{\isacharunderscore}{\kern0pt}def{\isacharparenright}{\kern0pt}\isanewline
\ \ \ \ \isacommand{have}\isamarkupfalse%
\ bottom{\isacharunderscore}{\kern0pt}square{\isacharcolon}{\kern0pt}\ {\isachardoublequoteopen}v\ {\isasymcirc}\isactrlsub c\ s\ {\isacharequal}{\kern0pt}\ f\ {\isasymcirc}\isactrlsub c\ v{\isachardoublequoteclose}\isanewline
\ \ \ \ \ \ \isacommand{unfolding}\isamarkupfalse%
\ v{\isacharunderscore}{\kern0pt}def\ u{\isacharunderscore}{\kern0pt}def\ s{\isacharunderscore}{\kern0pt}def\ \isacommand{using}\isamarkupfalse%
\ i{\isacharunderscore}{\kern0pt}iso\isanewline
\ \ \ \ \ \ \isacommand{by}\isamarkupfalse%
\ {\isacharparenleft}{\kern0pt}typecheck{\isacharunderscore}{\kern0pt}cfuncs{\isacharcomma}{\kern0pt}\ smt\ {\isacharparenleft}{\kern0pt}verit{\isacharcomma}{\kern0pt}\ ccfv{\isacharunderscore}{\kern0pt}SIG{\isacharparenright}{\kern0pt}\ comp{\isacharunderscore}{\kern0pt}associative{\isadigit{2}}\ id{\isacharunderscore}{\kern0pt}right{\isacharunderscore}{\kern0pt}unit{\isadigit{2}}\ inv{\isacharunderscore}{\kern0pt}left\ u{\isacharunderscore}{\kern0pt}def{\isacharparenright}{\kern0pt}\isanewline
\ \ \ \ \isacommand{show}\isamarkupfalse%
\ {\isachardoublequoteopen}{\isasymexists}{\isacharbang}{\kern0pt}u{\isachardot}{\kern0pt}\ u\ {\isacharcolon}{\kern0pt}\ N\ {\isasymrightarrow}\ X\ {\isasymand}\ q\ {\isacharequal}{\kern0pt}\ u\ {\isasymcirc}\isactrlsub c\ z\ {\isasymand}\ f\ {\isasymcirc}\isactrlsub c\ u\ {\isacharequal}{\kern0pt}\ u\ {\isasymcirc}\isactrlsub c\ s{\isachardoublequoteclose}\isanewline
\ \ \ \ \isacommand{proof}\isamarkupfalse%
\ safe\isanewline
\ \ \ \ \ \ \isacommand{show}\isamarkupfalse%
\ {\isachardoublequoteopen}{\isasymexists}u{\isachardot}{\kern0pt}\ u\ {\isacharcolon}{\kern0pt}\ N\ {\isasymrightarrow}\ X\ {\isasymand}\ q\ {\isacharequal}{\kern0pt}\ u\ {\isasymcirc}\isactrlsub c\ z\ {\isasymand}\ f\ {\isasymcirc}\isactrlsub c\ u\ {\isacharequal}{\kern0pt}\ u\ {\isasymcirc}\isactrlsub c\ s{\isachardoublequoteclose}\isanewline
\ \ \ \ \ \ \ \ \isacommand{by}\isamarkupfalse%
\ {\isacharparenleft}{\kern0pt}rule{\isacharunderscore}{\kern0pt}tac\ x{\isacharequal}{\kern0pt}v\ \isakeyword{in}\ exI{\isacharcomma}{\kern0pt}\ auto\ simp\ add{\isacharcolon}{\kern0pt}\ bottom{\isacharunderscore}{\kern0pt}triangle\ bottom{\isacharunderscore}{\kern0pt}square\ v{\isacharunderscore}{\kern0pt}type{\isacharparenright}{\kern0pt}\isanewline
\ \ \ \ \isacommand{next}\isamarkupfalse%
\isanewline
\ \ \ \ \ \ \isacommand{fix}\isamarkupfalse%
\ w\ y\isanewline
\ \ \ \ \ \ \isacommand{assume}\isamarkupfalse%
\ w{\isacharunderscore}{\kern0pt}type{\isacharbrackleft}{\kern0pt}type{\isacharunderscore}{\kern0pt}rule{\isacharbrackright}{\kern0pt}{\isacharcolon}{\kern0pt}\ {\isachardoublequoteopen}w\ {\isacharcolon}{\kern0pt}\ N\ {\isasymrightarrow}\ X{\isachardoublequoteclose}\isanewline
\ \ \ \ \ \ \isacommand{assume}\isamarkupfalse%
\ y{\isacharunderscore}{\kern0pt}type{\isacharbrackleft}{\kern0pt}type{\isacharunderscore}{\kern0pt}rule{\isacharbrackright}{\kern0pt}{\isacharcolon}{\kern0pt}\ {\isachardoublequoteopen}y\ {\isacharcolon}{\kern0pt}\ N\ {\isasymrightarrow}\ X{\isachardoublequoteclose}\isanewline
\ \ \ \ \ \ \isacommand{assume}\isamarkupfalse%
\ f{\isacharunderscore}{\kern0pt}w{\isacharcolon}{\kern0pt}\ {\isachardoublequoteopen}f\ {\isasymcirc}\isactrlsub c\ w\ {\isacharequal}{\kern0pt}\ w\ {\isasymcirc}\isactrlsub c\ s{\isachardoublequoteclose}\isanewline
\ \ \ \ \ \ \isacommand{assume}\isamarkupfalse%
\ f{\isacharunderscore}{\kern0pt}y{\isacharcolon}{\kern0pt}\ {\isachardoublequoteopen}f\ {\isasymcirc}\isactrlsub c\ y\ {\isacharequal}{\kern0pt}\ y\ {\isasymcirc}\isactrlsub c\ s{\isachardoublequoteclose}\isanewline
\ \ \ \ \ \ \isacommand{assume}\isamarkupfalse%
\ w{\isacharunderscore}{\kern0pt}y{\isacharunderscore}{\kern0pt}z{\isacharcolon}{\kern0pt}\ {\isachardoublequoteopen}w\ {\isasymcirc}\isactrlsub c\ z\ {\isacharequal}{\kern0pt}\ y\ {\isasymcirc}\isactrlsub c\ z{\isachardoublequoteclose}\isanewline
\ \ \ \ \ \ \isacommand{assume}\isamarkupfalse%
\ q{\isacharunderscore}{\kern0pt}def{\isacharcolon}{\kern0pt}\ {\isachardoublequoteopen}q\ {\isacharequal}{\kern0pt}\ w\ {\isasymcirc}\isactrlsub c\ z{\isachardoublequoteclose}\isanewline
\ \ \ \ \ \ \isanewline
\isanewline
\isanewline
\ \ \ \ \ \ \isacommand{have}\isamarkupfalse%
\ {\isachardoublequoteopen}w\ {\isasymcirc}\isactrlsub c\ i\ {\isacharequal}{\kern0pt}\ u{\isachardoublequoteclose}\isanewline
\ \ \ \ \ \ \isacommand{proof}\isamarkupfalse%
\ {\isacharparenleft}{\kern0pt}etcs{\isacharunderscore}{\kern0pt}rule\ natural{\isacharunderscore}{\kern0pt}number{\isacharunderscore}{\kern0pt}object{\isacharunderscore}{\kern0pt}func{\isacharunderscore}{\kern0pt}unique{\isacharbrackleft}{\kern0pt}\isakeyword{where}\ f{\isacharequal}{\kern0pt}f{\isacharbrackright}{\kern0pt}{\isacharparenright}{\kern0pt}\isanewline
\ \ \ \ \ \ \ \ \isacommand{show}\isamarkupfalse%
\ {\isachardoublequoteopen}{\isacharparenleft}{\kern0pt}w\ {\isasymcirc}\isactrlsub c\ i{\isacharparenright}{\kern0pt}\ {\isasymcirc}\isactrlsub c\ zero\ {\isacharequal}{\kern0pt}\ u\ {\isasymcirc}\isactrlsub c\ zero{\isachardoublequoteclose}\isanewline
\ \ \ \ \ \ \ \ \ \ \isacommand{using}\isamarkupfalse%
\ q{\isacharunderscore}{\kern0pt}def\ u{\isacharunderscore}{\kern0pt}def\ w{\isacharunderscore}{\kern0pt}y{\isacharunderscore}{\kern0pt}z\ z{\isacharunderscore}{\kern0pt}def\ \isacommand{by}\isamarkupfalse%
\ {\isacharparenleft}{\kern0pt}etcs{\isacharunderscore}{\kern0pt}assocr{\isacharcomma}{\kern0pt}\ argo{\isacharparenright}{\kern0pt}\isanewline
\ \ \ \ \ \ \ \ \isacommand{show}\isamarkupfalse%
\ {\isachardoublequoteopen}{\isacharparenleft}{\kern0pt}w\ {\isasymcirc}\isactrlsub c\ i{\isacharparenright}{\kern0pt}\ {\isasymcirc}\isactrlsub c\ successor\ {\isacharequal}{\kern0pt}\ f\ {\isasymcirc}\isactrlsub c\ w\ {\isasymcirc}\isactrlsub c\ i{\isachardoublequoteclose}\isanewline
\ \ \ \ \ \ \ \ \ \ \isacommand{using}\isamarkupfalse%
\ i{\isacharunderscore}{\kern0pt}iso\ \isacommand{by}\isamarkupfalse%
\ {\isacharparenleft}{\kern0pt}typecheck{\isacharunderscore}{\kern0pt}cfuncs{\isacharcomma}{\kern0pt}\ smt\ {\isacharparenleft}{\kern0pt}verit{\isacharcomma}{\kern0pt}\ best{\isacharparenright}{\kern0pt}\ comp{\isacharunderscore}{\kern0pt}associative{\isadigit{2}}\ comp{\isacharunderscore}{\kern0pt}type\ f{\isacharunderscore}{\kern0pt}w\ id{\isacharunderscore}{\kern0pt}right{\isacharunderscore}{\kern0pt}unit{\isadigit{2}}\ inv{\isacharunderscore}{\kern0pt}left\ inverse{\isacharunderscore}{\kern0pt}type\ s{\isacharunderscore}{\kern0pt}def{\isacharparenright}{\kern0pt}\isanewline
\ \ \ \ \ \ \ \ \isacommand{show}\isamarkupfalse%
\ {\isachardoublequoteopen}u\ {\isasymcirc}\isactrlsub c\ successor\ {\isacharequal}{\kern0pt}\ f\ {\isasymcirc}\isactrlsub c\ u{\isachardoublequoteclose}\isanewline
\ \ \ \ \ \ \ \ \ \ \isacommand{by}\isamarkupfalse%
\ {\isacharparenleft}{\kern0pt}simp\ add{\isacharcolon}{\kern0pt}\ u{\isacharunderscore}{\kern0pt}def{\isacharparenright}{\kern0pt}\isanewline
\ \ \ \ \ \ \isacommand{qed}\isamarkupfalse%
\isanewline
\ \ \ \ \ \ \isacommand{then}\isamarkupfalse%
\ \isacommand{have}\isamarkupfalse%
\ w{\isacharunderscore}{\kern0pt}eq{\isacharunderscore}{\kern0pt}v{\isacharcolon}{\kern0pt}\ {\isachardoublequoteopen}w\ {\isacharequal}{\kern0pt}\ v{\isachardoublequoteclose}\isanewline
\ \ \ \ \ \ \ \ \isacommand{unfolding}\isamarkupfalse%
\ v{\isacharunderscore}{\kern0pt}def\ \isacommand{using}\isamarkupfalse%
\ i{\isacharunderscore}{\kern0pt}iso\isanewline
\ \ \ \ \ \ \ \ \isacommand{by}\isamarkupfalse%
\ {\isacharparenleft}{\kern0pt}typecheck{\isacharunderscore}{\kern0pt}cfuncs{\isacharcomma}{\kern0pt}\ smt\ {\isacharparenleft}{\kern0pt}verit{\isacharcomma}{\kern0pt}\ best{\isacharparenright}{\kern0pt}\ comp{\isacharunderscore}{\kern0pt}associative{\isadigit{2}}\ id{\isacharunderscore}{\kern0pt}right{\isacharunderscore}{\kern0pt}unit{\isadigit{2}}\ inv{\isacharunderscore}{\kern0pt}right{\isacharparenright}{\kern0pt}\isanewline
\isanewline
\ \ \ \ \ \ \isacommand{have}\isamarkupfalse%
\ {\isachardoublequoteopen}y\ {\isasymcirc}\isactrlsub c\ i\ {\isacharequal}{\kern0pt}\ u{\isachardoublequoteclose}\isanewline
\ \ \ \ \ \ \isacommand{proof}\isamarkupfalse%
\ {\isacharparenleft}{\kern0pt}etcs{\isacharunderscore}{\kern0pt}rule\ natural{\isacharunderscore}{\kern0pt}number{\isacharunderscore}{\kern0pt}object{\isacharunderscore}{\kern0pt}func{\isacharunderscore}{\kern0pt}unique{\isacharbrackleft}{\kern0pt}\isakeyword{where}\ f{\isacharequal}{\kern0pt}f{\isacharbrackright}{\kern0pt}{\isacharparenright}{\kern0pt}\isanewline
\ \ \ \ \ \ \ \ \isacommand{show}\isamarkupfalse%
\ {\isachardoublequoteopen}{\isacharparenleft}{\kern0pt}y\ {\isasymcirc}\isactrlsub c\ i{\isacharparenright}{\kern0pt}\ {\isasymcirc}\isactrlsub c\ zero\ {\isacharequal}{\kern0pt}\ u\ {\isasymcirc}\isactrlsub c\ zero{\isachardoublequoteclose}\isanewline
\ \ \ \ \ \ \ \ \ \ \isacommand{using}\isamarkupfalse%
\ q{\isacharunderscore}{\kern0pt}def\ u{\isacharunderscore}{\kern0pt}def\ w{\isacharunderscore}{\kern0pt}y{\isacharunderscore}{\kern0pt}z\ z{\isacharunderscore}{\kern0pt}def\ \isacommand{by}\isamarkupfalse%
\ {\isacharparenleft}{\kern0pt}etcs{\isacharunderscore}{\kern0pt}assocr{\isacharcomma}{\kern0pt}\ argo{\isacharparenright}{\kern0pt}\isanewline
\ \ \ \ \ \ \ \ \isacommand{show}\isamarkupfalse%
\ {\isachardoublequoteopen}{\isacharparenleft}{\kern0pt}y\ {\isasymcirc}\isactrlsub c\ i{\isacharparenright}{\kern0pt}\ {\isasymcirc}\isactrlsub c\ successor\ {\isacharequal}{\kern0pt}\ f\ {\isasymcirc}\isactrlsub c\ y\ {\isasymcirc}\isactrlsub c\ i{\isachardoublequoteclose}\isanewline
\ \ \ \ \ \ \ \ \ \ \isacommand{using}\isamarkupfalse%
\ i{\isacharunderscore}{\kern0pt}iso\ \isacommand{by}\isamarkupfalse%
\ {\isacharparenleft}{\kern0pt}typecheck{\isacharunderscore}{\kern0pt}cfuncs{\isacharcomma}{\kern0pt}\ smt\ {\isacharparenleft}{\kern0pt}verit{\isacharcomma}{\kern0pt}\ best{\isacharparenright}{\kern0pt}\ comp{\isacharunderscore}{\kern0pt}associative{\isadigit{2}}\ comp{\isacharunderscore}{\kern0pt}type\ f{\isacharunderscore}{\kern0pt}y\ id{\isacharunderscore}{\kern0pt}right{\isacharunderscore}{\kern0pt}unit{\isadigit{2}}\ inv{\isacharunderscore}{\kern0pt}left\ inverse{\isacharunderscore}{\kern0pt}type\ s{\isacharunderscore}{\kern0pt}def{\isacharparenright}{\kern0pt}\isanewline
\ \ \ \ \ \ \ \ \isacommand{show}\isamarkupfalse%
\ {\isachardoublequoteopen}u\ {\isasymcirc}\isactrlsub c\ successor\ {\isacharequal}{\kern0pt}\ f\ {\isasymcirc}\isactrlsub c\ u{\isachardoublequoteclose}\isanewline
\ \ \ \ \ \ \ \ \ \ \isacommand{by}\isamarkupfalse%
\ {\isacharparenleft}{\kern0pt}simp\ add{\isacharcolon}{\kern0pt}\ u{\isacharunderscore}{\kern0pt}def{\isacharparenright}{\kern0pt}\isanewline
\ \ \ \ \ \ \isacommand{qed}\isamarkupfalse%
\isanewline
\ \ \ \ \ \ \isacommand{then}\isamarkupfalse%
\ \isacommand{have}\isamarkupfalse%
\ y{\isacharunderscore}{\kern0pt}eq{\isacharunderscore}{\kern0pt}v{\isacharcolon}{\kern0pt}\ {\isachardoublequoteopen}y\ {\isacharequal}{\kern0pt}\ v{\isachardoublequoteclose}\isanewline
\ \ \ \ \ \ \ \ \isacommand{unfolding}\isamarkupfalse%
\ v{\isacharunderscore}{\kern0pt}def\ \isacommand{using}\isamarkupfalse%
\ i{\isacharunderscore}{\kern0pt}iso\isanewline
\ \ \ \ \ \ \ \ \isacommand{by}\isamarkupfalse%
\ {\isacharparenleft}{\kern0pt}typecheck{\isacharunderscore}{\kern0pt}cfuncs{\isacharcomma}{\kern0pt}\ smt\ {\isacharparenleft}{\kern0pt}verit{\isacharcomma}{\kern0pt}\ best{\isacharparenright}{\kern0pt}\ comp{\isacharunderscore}{\kern0pt}associative{\isadigit{2}}\ id{\isacharunderscore}{\kern0pt}right{\isacharunderscore}{\kern0pt}unit{\isadigit{2}}\ inv{\isacharunderscore}{\kern0pt}right{\isacharparenright}{\kern0pt}\isanewline
\ \ \ \ \ \ \isacommand{show}\isamarkupfalse%
\ {\isachardoublequoteopen}w\ {\isacharequal}{\kern0pt}\ y{\isachardoublequoteclose}\isanewline
\ \ \ \ \ \ \ \ \isacommand{using}\isamarkupfalse%
\ w{\isacharunderscore}{\kern0pt}eq{\isacharunderscore}{\kern0pt}v\ y{\isacharunderscore}{\kern0pt}eq{\isacharunderscore}{\kern0pt}v\ \isacommand{by}\isamarkupfalse%
\ auto\isanewline
\ \ \ \ \isacommand{qed}\isamarkupfalse%
\isanewline
\ \ \isacommand{qed}\isamarkupfalse%
\isanewline
\ \ \isacommand{then}\isamarkupfalse%
\ \isacommand{show}\isamarkupfalse%
\ {\isacharquery}{\kern0pt}thesis\isanewline
\ \ \ \ \isacommand{by}\isamarkupfalse%
\ auto\isanewline
\isacommand{qed}\isamarkupfalse%
%
\endisatagproof
{\isafoldproof}%
%
\isadelimproof
%
\endisadelimproof
%
\isadelimdocument
%
\endisadelimdocument
%
\isatagdocument
%
\isamarkupsubsection{Zero and Successor%
}
\isamarkuptrue%
%
\endisatagdocument
{\isafolddocument}%
%
\isadelimdocument
%
\endisadelimdocument
\isacommand{lemma}\isamarkupfalse%
\ zero{\isacharunderscore}{\kern0pt}is{\isacharunderscore}{\kern0pt}not{\isacharunderscore}{\kern0pt}successor{\isacharcolon}{\kern0pt}\isanewline
\ \ \isakeyword{assumes}\ {\isachardoublequoteopen}n\ {\isasymin}\isactrlsub c\ {\isasymnat}\isactrlsub c{\isachardoublequoteclose}\isanewline
\ \ \isakeyword{shows}\ {\isachardoublequoteopen}zero\ {\isasymnoteq}\ successor\ {\isasymcirc}\isactrlsub c\ n{\isachardoublequoteclose}\isanewline
%
\isadelimproof
%
\endisadelimproof
%
\isatagproof
\isacommand{proof}\isamarkupfalse%
\ {\isacharparenleft}{\kern0pt}rule\ ccontr{\isacharcomma}{\kern0pt}\ clarify{\isacharparenright}{\kern0pt}\isanewline
\ \ \isacommand{assume}\isamarkupfalse%
\ for{\isacharunderscore}{\kern0pt}contradiction{\isacharcolon}{\kern0pt}\ {\isachardoublequoteopen}zero\ {\isacharequal}{\kern0pt}\ successor\ {\isasymcirc}\isactrlsub c\ n{\isachardoublequoteclose}\isanewline
\ \ \isacommand{have}\isamarkupfalse%
\ {\isachardoublequoteopen}{\isasymexists}{\isacharbang}{\kern0pt}u{\isachardot}{\kern0pt}\ u{\isacharcolon}{\kern0pt}\ {\isasymnat}\isactrlsub c\ {\isasymrightarrow}\ {\isasymOmega}\ {\isasymand}\ u\ {\isasymcirc}\isactrlsub c\ zero\ {\isacharequal}{\kern0pt}\ {\isasymt}\ {\isasymand}\ {\isacharparenleft}{\kern0pt}{\isasymf}\ {\isasymcirc}\isactrlsub c\ {\isasymbeta}\isactrlbsub {\isasymOmega}\isactrlesub {\isacharparenright}{\kern0pt}\ {\isasymcirc}\isactrlsub c\ u\ {\isacharequal}{\kern0pt}\ u\ {\isasymcirc}\isactrlsub c\ successor{\isachardoublequoteclose}\isanewline
\ \ \ \ \isacommand{by}\isamarkupfalse%
\ {\isacharparenleft}{\kern0pt}typecheck{\isacharunderscore}{\kern0pt}cfuncs{\isacharcomma}{\kern0pt}\ rule\ natural{\isacharunderscore}{\kern0pt}number{\isacharunderscore}{\kern0pt}object{\isacharunderscore}{\kern0pt}property{\isadigit{2}}{\isacharparenright}{\kern0pt}\isanewline
\ \ \isacommand{then}\isamarkupfalse%
\ \isacommand{obtain}\isamarkupfalse%
\ u\ \isakeyword{where}\ \ u{\isacharunderscore}{\kern0pt}type{\isacharcolon}{\kern0pt}\ \ {\isachardoublequoteopen}u{\isacharcolon}{\kern0pt}\ {\isasymnat}\isactrlsub c\ {\isasymrightarrow}\ {\isasymOmega}{\isachardoublequoteclose}\ \isakeyword{and}\ \isanewline
\ \ \ \ \ \ \ \ \ \ \ \ \ \ \ \ \ \ \ \ \ \ \ u{\isacharunderscore}{\kern0pt}triangle{\isacharcolon}{\kern0pt}\ {\isachardoublequoteopen}u\ {\isasymcirc}\isactrlsub c\ zero\ {\isacharequal}{\kern0pt}\ {\isasymt}{\isachardoublequoteclose}\ \isakeyword{and}\ \ \isanewline
\ \ \ \ \ \ \ \ \ \ \ \ \ \ \ \ \ \ \ \ \ \ \ u{\isacharunderscore}{\kern0pt}square{\isacharcolon}{\kern0pt}\ {\isachardoublequoteopen}{\isacharparenleft}{\kern0pt}{\isasymf}\ {\isasymcirc}\isactrlsub c\ {\isasymbeta}\isactrlbsub {\isasymOmega}\isactrlesub {\isacharparenright}{\kern0pt}\ {\isasymcirc}\isactrlsub c\ u\ {\isacharequal}{\kern0pt}\ u\ {\isasymcirc}\isactrlsub c\ successor{\isachardoublequoteclose}\isanewline
\ \ \ \ \isacommand{by}\isamarkupfalse%
\ auto\isanewline
\ \ \isacommand{have}\isamarkupfalse%
\ {\isachardoublequoteopen}{\isasymt}\ {\isacharequal}{\kern0pt}\ {\isasymf}{\isachardoublequoteclose}\ \isanewline
\ \ \isacommand{proof}\isamarkupfalse%
\ {\isacharminus}{\kern0pt}\isanewline
\ \ \ \ \isacommand{have}\isamarkupfalse%
\ {\isachardoublequoteopen}{\isasymt}\ {\isacharequal}{\kern0pt}\ u\ \ {\isasymcirc}\isactrlsub c\ zero{\isachardoublequoteclose}\isanewline
\ \ \ \ \ \ \isacommand{by}\isamarkupfalse%
\ {\isacharparenleft}{\kern0pt}simp\ add{\isacharcolon}{\kern0pt}\ u{\isacharunderscore}{\kern0pt}triangle{\isacharparenright}{\kern0pt}\isanewline
\ \ \ \ \isacommand{also}\isamarkupfalse%
\ \isacommand{have}\isamarkupfalse%
\ {\isachardoublequoteopen}{\isachardot}{\kern0pt}{\isachardot}{\kern0pt}{\isachardot}{\kern0pt}\ {\isacharequal}{\kern0pt}\ u\ {\isasymcirc}\isactrlsub c\ successor\ {\isasymcirc}\isactrlsub c\ n{\isachardoublequoteclose}\isanewline
\ \ \ \ \ \ \isacommand{by}\isamarkupfalse%
\ {\isacharparenleft}{\kern0pt}simp\ add{\isacharcolon}{\kern0pt}\ for{\isacharunderscore}{\kern0pt}contradiction{\isacharparenright}{\kern0pt}\isanewline
\ \ \ \ \isacommand{also}\isamarkupfalse%
\ \isacommand{have}\isamarkupfalse%
\ {\isachardoublequoteopen}{\isachardot}{\kern0pt}{\isachardot}{\kern0pt}{\isachardot}{\kern0pt}\ {\isacharequal}{\kern0pt}\ {\isacharparenleft}{\kern0pt}{\isasymf}\ {\isasymcirc}\isactrlsub c\ {\isasymbeta}\isactrlbsub {\isasymOmega}\isactrlesub {\isacharparenright}{\kern0pt}\ {\isasymcirc}\isactrlsub c\ u\ {\isasymcirc}\isactrlsub c\ n{\isachardoublequoteclose}\isanewline
\ \ \ \ \ \ \isacommand{using}\isamarkupfalse%
\ assms\ u{\isacharunderscore}{\kern0pt}type\ \isacommand{by}\isamarkupfalse%
\ {\isacharparenleft}{\kern0pt}typecheck{\isacharunderscore}{\kern0pt}cfuncs{\isacharcomma}{\kern0pt}\ simp\ add{\isacharcolon}{\kern0pt}\ \ comp{\isacharunderscore}{\kern0pt}associative{\isadigit{2}}\ u{\isacharunderscore}{\kern0pt}square{\isacharparenright}{\kern0pt}\isanewline
\ \ \ \ \isacommand{also}\isamarkupfalse%
\ \isacommand{have}\isamarkupfalse%
\ {\isachardoublequoteopen}{\isachardot}{\kern0pt}{\isachardot}{\kern0pt}{\isachardot}{\kern0pt}\ {\isacharequal}{\kern0pt}\ {\isasymf}{\isachardoublequoteclose}\isanewline
\ \ \ \ \ \ \isacommand{using}\isamarkupfalse%
\ assms\ u{\isacharunderscore}{\kern0pt}type\ \isacommand{by}\isamarkupfalse%
\ {\isacharparenleft}{\kern0pt}etcs{\isacharunderscore}{\kern0pt}assocr{\isacharcomma}{\kern0pt}typecheck{\isacharunderscore}{\kern0pt}cfuncs{\isacharcomma}{\kern0pt}\ simp\ add{\isacharcolon}{\kern0pt}\ id{\isacharunderscore}{\kern0pt}right{\isacharunderscore}{\kern0pt}unit{\isadigit{2}}\ terminal{\isacharunderscore}{\kern0pt}func{\isacharunderscore}{\kern0pt}comp{\isacharunderscore}{\kern0pt}elem{\isacharparenright}{\kern0pt}\isanewline
\ \ \ \ \isacommand{then}\isamarkupfalse%
\ \isacommand{show}\isamarkupfalse%
\ {\isacharquery}{\kern0pt}thesis\ \isacommand{using}\isamarkupfalse%
\ calculation\ \isacommand{by}\isamarkupfalse%
\ auto\isanewline
\ \ \isacommand{qed}\isamarkupfalse%
\isanewline
\ \ \isacommand{then}\isamarkupfalse%
\ \isacommand{show}\isamarkupfalse%
\ False\isanewline
\ \ \ \ \isacommand{using}\isamarkupfalse%
\ true{\isacharunderscore}{\kern0pt}false{\isacharunderscore}{\kern0pt}distinct\ \isacommand{by}\isamarkupfalse%
\ blast\isanewline
\isacommand{qed}\isamarkupfalse%
%
\endisatagproof
{\isafoldproof}%
%
\isadelimproof
%
\endisadelimproof
%
\begin{isamarkuptext}%
The lemma below corresponds to Proposition 2.6.6 in Halvorson.%
\end{isamarkuptext}\isamarkuptrue%
\isacommand{lemma}\isamarkupfalse%
\ oneUN{\isacharunderscore}{\kern0pt}iso{\isacharunderscore}{\kern0pt}N{\isacharunderscore}{\kern0pt}isomorphism{\isacharcolon}{\kern0pt}\isanewline
\ {\isachardoublequoteopen}isomorphism{\isacharparenleft}{\kern0pt}zero\ {\isasymamalg}\ successor{\isacharparenright}{\kern0pt}{\isachardoublequoteclose}\ \isanewline
%
\isadelimproof
%
\endisadelimproof
%
\isatagproof
\isacommand{proof}\isamarkupfalse%
\ {\isacharminus}{\kern0pt}\ \isanewline
\ \ \isacommand{obtain}\isamarkupfalse%
\ i{\isadigit{0}}\ \isakeyword{where}\ i{\isadigit{0}}{\isacharunderscore}{\kern0pt}type{\isacharbrackleft}{\kern0pt}type{\isacharunderscore}{\kern0pt}rule{\isacharbrackright}{\kern0pt}{\isacharcolon}{\kern0pt}\ \ {\isachardoublequoteopen}i{\isadigit{0}}{\isacharcolon}{\kern0pt}\ {\isasymone}\ {\isasymrightarrow}\ {\isacharparenleft}{\kern0pt}{\isasymone}\ {\isasymCoprod}\ {\isasymnat}\isactrlsub c{\isacharparenright}{\kern0pt}{\isachardoublequoteclose}\ \isakeyword{and}\ i{\isadigit{0}}{\isacharunderscore}{\kern0pt}def{\isacharcolon}{\kern0pt}\ {\isachardoublequoteopen}i{\isadigit{0}}\ {\isacharequal}{\kern0pt}\ left{\isacharunderscore}{\kern0pt}coproj\ {\isasymone}\ {\isasymnat}\isactrlsub c{\isachardoublequoteclose}\isanewline
\ \ \ \ \isacommand{by}\isamarkupfalse%
\ {\isacharparenleft}{\kern0pt}typecheck{\isacharunderscore}{\kern0pt}cfuncs{\isacharcomma}{\kern0pt}\ simp{\isacharparenright}{\kern0pt}\isanewline
\ \ \isacommand{obtain}\isamarkupfalse%
\ i{\isadigit{1}}\ \isakeyword{where}\ i{\isadigit{1}}{\isacharunderscore}{\kern0pt}type{\isacharbrackleft}{\kern0pt}type{\isacharunderscore}{\kern0pt}rule{\isacharbrackright}{\kern0pt}{\isacharcolon}{\kern0pt}\ \ {\isachardoublequoteopen}i{\isadigit{1}}{\isacharcolon}{\kern0pt}\ {\isasymnat}\isactrlsub c\ {\isasymrightarrow}\ {\isacharparenleft}{\kern0pt}{\isasymone}\ {\isasymCoprod}\ {\isasymnat}\isactrlsub c{\isacharparenright}{\kern0pt}{\isachardoublequoteclose}\ \isakeyword{and}\ i{\isadigit{1}}{\isacharunderscore}{\kern0pt}def{\isacharcolon}{\kern0pt}\ {\isachardoublequoteopen}i{\isadigit{1}}\ {\isacharequal}{\kern0pt}\ right{\isacharunderscore}{\kern0pt}coproj\ {\isasymone}\ {\isasymnat}\isactrlsub c{\isachardoublequoteclose}\isanewline
\ \ \ \ \isacommand{by}\isamarkupfalse%
\ {\isacharparenleft}{\kern0pt}typecheck{\isacharunderscore}{\kern0pt}cfuncs{\isacharcomma}{\kern0pt}\ simp{\isacharparenright}{\kern0pt}\isanewline
\ \ \isacommand{obtain}\isamarkupfalse%
\ g\ \isakeyword{where}\ g{\isacharunderscore}{\kern0pt}type{\isacharbrackleft}{\kern0pt}type{\isacharunderscore}{\kern0pt}rule{\isacharbrackright}{\kern0pt}{\isacharcolon}{\kern0pt}\ {\isachardoublequoteopen}g{\isacharcolon}{\kern0pt}\ {\isasymnat}\isactrlsub c\ {\isasymrightarrow}\ {\isacharparenleft}{\kern0pt}{\isasymone}\ {\isasymCoprod}\ {\isasymnat}\isactrlsub c{\isacharparenright}{\kern0pt}{\isachardoublequoteclose}\ \isakeyword{and}\isanewline
\ \ \ g{\isacharunderscore}{\kern0pt}triangle{\isacharcolon}{\kern0pt}\ {\isachardoublequoteopen}\ g\ {\isasymcirc}\isactrlsub c\ zero\ {\isacharequal}{\kern0pt}\ i{\isadigit{0}}{\isachardoublequoteclose}\ \isakeyword{and}\isanewline
\ \ \ g{\isacharunderscore}{\kern0pt}square{\isacharcolon}{\kern0pt}\ {\isachardoublequoteopen}g\ {\isasymcirc}\isactrlsub c\ successor\ {\isacharequal}{\kern0pt}\ {\isacharparenleft}{\kern0pt}{\isacharparenleft}{\kern0pt}i{\isadigit{1}}\ {\isasymcirc}\isactrlsub c\ zero{\isacharparenright}{\kern0pt}\ {\isasymamalg}\ {\isacharparenleft}{\kern0pt}i{\isadigit{1}}\ {\isasymcirc}\isactrlsub c\ successor{\isacharparenright}{\kern0pt}{\isacharparenright}{\kern0pt}\ {\isasymcirc}\isactrlsub c\ g{\isachardoublequoteclose}\isanewline
\ \ \ \ \isacommand{by}\isamarkupfalse%
\ {\isacharparenleft}{\kern0pt}typecheck{\isacharunderscore}{\kern0pt}cfuncs{\isacharcomma}{\kern0pt}\ metis\ natural{\isacharunderscore}{\kern0pt}number{\isacharunderscore}{\kern0pt}object{\isacharunderscore}{\kern0pt}property{\isacharparenright}{\kern0pt}\isanewline
\ \ \isacommand{then}\isamarkupfalse%
\ \isacommand{have}\isamarkupfalse%
\ second{\isacharunderscore}{\kern0pt}diagram{\isadigit{3}}{\isacharcolon}{\kern0pt}\ {\isachardoublequoteopen}g\ {\isasymcirc}\isactrlsub c\ {\isacharparenleft}{\kern0pt}successor\ {\isasymcirc}\isactrlsub c\ zero{\isacharparenright}{\kern0pt}\ \ {\isacharequal}{\kern0pt}\ {\isacharparenleft}{\kern0pt}i{\isadigit{1}}\ {\isasymcirc}\isactrlsub c\ zero{\isacharparenright}{\kern0pt}{\isachardoublequoteclose}\isanewline
\ \ \ \ \isacommand{by}\isamarkupfalse%
\ {\isacharparenleft}{\kern0pt}typecheck{\isacharunderscore}{\kern0pt}cfuncs{\isacharcomma}{\kern0pt}\ smt\ {\isacharparenleft}{\kern0pt}verit{\isacharcomma}{\kern0pt}\ best{\isacharparenright}{\kern0pt}\ cfunc{\isacharunderscore}{\kern0pt}coprod{\isacharunderscore}{\kern0pt}type\ comp{\isacharunderscore}{\kern0pt}associative{\isadigit{2}}\ comp{\isacharunderscore}{\kern0pt}type\ i{\isadigit{0}}{\isacharunderscore}{\kern0pt}def\ left{\isacharunderscore}{\kern0pt}coproj{\isacharunderscore}{\kern0pt}cfunc{\isacharunderscore}{\kern0pt}coprod{\isacharparenright}{\kern0pt}\isanewline
\ \ \isacommand{then}\isamarkupfalse%
\ \isacommand{have}\isamarkupfalse%
\ g{\isacharunderscore}{\kern0pt}s{\isacharunderscore}{\kern0pt}s{\isacharunderscore}{\kern0pt}Eqs{\isacharunderscore}{\kern0pt}i{\isadigit{1}}zUi{\isadigit{1}}s{\isacharunderscore}{\kern0pt}g{\isacharunderscore}{\kern0pt}s{\isacharcolon}{\kern0pt}\isanewline
\ \ \ \ {\isachardoublequoteopen}{\isacharparenleft}{\kern0pt}g\ {\isasymcirc}\isactrlsub c\ successor{\isacharparenright}{\kern0pt}\ {\isasymcirc}\isactrlsub c\ successor\ {\isacharequal}{\kern0pt}\ {\isacharparenleft}{\kern0pt}{\isacharparenleft}{\kern0pt}i{\isadigit{1}}\ {\isasymcirc}\isactrlsub c\ zero{\isacharparenright}{\kern0pt}\ {\isasymamalg}\ {\isacharparenleft}{\kern0pt}i{\isadigit{1}}\ {\isasymcirc}\isactrlsub c\ successor{\isacharparenright}{\kern0pt}{\isacharparenright}{\kern0pt}\ {\isasymcirc}\isactrlsub c\ {\isacharparenleft}{\kern0pt}g\ {\isasymcirc}\isactrlsub c\ successor{\isacharparenright}{\kern0pt}{\isachardoublequoteclose}\isanewline
\ \ \ \ \isacommand{by}\isamarkupfalse%
\ {\isacharparenleft}{\kern0pt}typecheck{\isacharunderscore}{\kern0pt}cfuncs{\isacharcomma}{\kern0pt}\ smt\ {\isacharparenleft}{\kern0pt}verit{\isacharcomma}{\kern0pt}\ del{\isacharunderscore}{\kern0pt}insts{\isacharparenright}{\kern0pt}\ comp{\isacharunderscore}{\kern0pt}associative{\isadigit{2}}\ g{\isacharunderscore}{\kern0pt}square{\isacharparenright}{\kern0pt}\isanewline
\ \ \isacommand{then}\isamarkupfalse%
\ \isacommand{have}\isamarkupfalse%
\ g{\isacharunderscore}{\kern0pt}s{\isacharunderscore}{\kern0pt}s{\isacharunderscore}{\kern0pt}zEqs{\isacharunderscore}{\kern0pt}i{\isadigit{1}}zUi{\isadigit{1}}s{\isacharunderscore}{\kern0pt}i{\isadigit{1}}z{\isacharcolon}{\kern0pt}\ {\isachardoublequoteopen}{\isacharparenleft}{\kern0pt}{\isacharparenleft}{\kern0pt}g\ {\isasymcirc}\isactrlsub c\ successor{\isacharparenright}{\kern0pt}\ {\isasymcirc}\isactrlsub c\ successor{\isacharparenright}{\kern0pt}{\isasymcirc}\isactrlsub c\ zero\ {\isacharequal}{\kern0pt}\isanewline
\ \ \ \ {\isacharparenleft}{\kern0pt}{\isacharparenleft}{\kern0pt}i{\isadigit{1}}\ {\isasymcirc}\isactrlsub c\ zero{\isacharparenright}{\kern0pt}\ {\isasymamalg}\ {\isacharparenleft}{\kern0pt}i{\isadigit{1}}\ {\isasymcirc}\isactrlsub c\ successor{\isacharparenright}{\kern0pt}{\isacharparenright}{\kern0pt}\ {\isasymcirc}\isactrlsub c\ {\isacharparenleft}{\kern0pt}i{\isadigit{1}}\ {\isasymcirc}\isactrlsub c\ zero{\isacharparenright}{\kern0pt}{\isachardoublequoteclose}\isanewline
\ \ \ \ \isacommand{by}\isamarkupfalse%
\ {\isacharparenleft}{\kern0pt}typecheck{\isacharunderscore}{\kern0pt}cfuncs{\isacharcomma}{\kern0pt}\ smt\ {\isacharparenleft}{\kern0pt}verit{\isacharcomma}{\kern0pt}\ ccfv{\isacharunderscore}{\kern0pt}SIG{\isacharparenright}{\kern0pt}\ comp{\isacharunderscore}{\kern0pt}associative{\isadigit{2}}\ g{\isacharunderscore}{\kern0pt}square\ second{\isacharunderscore}{\kern0pt}diagram{\isadigit{3}}{\isacharparenright}{\kern0pt}\isanewline
\ \ \isacommand{then}\isamarkupfalse%
\ \isacommand{have}\isamarkupfalse%
\ i{\isadigit{1}}{\isacharunderscore}{\kern0pt}sEqs{\isacharunderscore}{\kern0pt}i{\isadigit{1}}zUi{\isadigit{1}}s{\isacharunderscore}{\kern0pt}i{\isadigit{1}}{\isacharcolon}{\kern0pt}\ {\isachardoublequoteopen}i{\isadigit{1}}\ {\isasymcirc}\isactrlsub c\ successor\ {\isacharequal}{\kern0pt}\ {\isacharparenleft}{\kern0pt}{\isacharparenleft}{\kern0pt}i{\isadigit{1}}\ {\isasymcirc}\isactrlsub c\ zero{\isacharparenright}{\kern0pt}\ {\isasymamalg}\ {\isacharparenleft}{\kern0pt}i{\isadigit{1}}\ {\isasymcirc}\isactrlsub c\ successor{\isacharparenright}{\kern0pt}{\isacharparenright}{\kern0pt}\ {\isasymcirc}\isactrlsub c\ i{\isadigit{1}}{\isachardoublequoteclose}\isanewline
\ \ \ \ \isacommand{by}\isamarkupfalse%
\ {\isacharparenleft}{\kern0pt}typecheck{\isacharunderscore}{\kern0pt}cfuncs{\isacharcomma}{\kern0pt}\ simp\ add{\isacharcolon}{\kern0pt}\ i{\isadigit{1}}{\isacharunderscore}{\kern0pt}def\ right{\isacharunderscore}{\kern0pt}coproj{\isacharunderscore}{\kern0pt}cfunc{\isacharunderscore}{\kern0pt}coprod{\isacharparenright}{\kern0pt}\ \ \ \isanewline
\ \ \isacommand{then}\isamarkupfalse%
\ \isacommand{obtain}\isamarkupfalse%
\ u\ \isakeyword{where}\ u{\isacharunderscore}{\kern0pt}type{\isacharbrackleft}{\kern0pt}type{\isacharunderscore}{\kern0pt}rule{\isacharbrackright}{\kern0pt}{\isacharcolon}{\kern0pt}\ {\isachardoublequoteopen}{\isacharparenleft}{\kern0pt}u{\isacharcolon}{\kern0pt}\ {\isasymnat}\isactrlsub c\ {\isasymrightarrow}\ {\isacharparenleft}{\kern0pt}{\isasymone}\ {\isasymCoprod}\ {\isasymnat}\isactrlsub c{\isacharparenright}{\kern0pt}{\isacharparenright}{\kern0pt}{\isachardoublequoteclose}\ \isakeyword{and}\isanewline
\ \ \ \ \ \ u{\isacharunderscore}{\kern0pt}triangle{\isacharcolon}{\kern0pt}\ {\isachardoublequoteopen}u\ {\isasymcirc}\isactrlsub c\ zero\ {\isacharequal}{\kern0pt}\ i{\isadigit{1}}\ {\isasymcirc}\isactrlsub c\ zero{\isachardoublequoteclose}\ \isakeyword{and}\isanewline
\ \ \ \ \ \ u{\isacharunderscore}{\kern0pt}square{\isacharcolon}{\kern0pt}\ {\isachardoublequoteopen}u\ {\isasymcirc}\isactrlsub c\ successor\ {\isacharequal}{\kern0pt}\ \ {\isacharparenleft}{\kern0pt}{\isacharparenleft}{\kern0pt}i{\isadigit{1}}\ {\isasymcirc}\isactrlsub c\ zero{\isacharparenright}{\kern0pt}\ {\isasymamalg}\ {\isacharparenleft}{\kern0pt}i{\isadigit{1}}\ {\isasymcirc}\isactrlsub c\ successor{\isacharparenright}{\kern0pt}{\isacharparenright}{\kern0pt}\ {\isasymcirc}\isactrlsub c\ u\ {\isachardoublequoteclose}\isanewline
\ \ \ \ \isacommand{using}\isamarkupfalse%
\ i{\isadigit{1}}{\isacharunderscore}{\kern0pt}sEqs{\isacharunderscore}{\kern0pt}i{\isadigit{1}}zUi{\isadigit{1}}s{\isacharunderscore}{\kern0pt}i{\isadigit{1}}\ \isacommand{by}\isamarkupfalse%
\ {\isacharparenleft}{\kern0pt}typecheck{\isacharunderscore}{\kern0pt}cfuncs{\isacharcomma}{\kern0pt}\ blast{\isacharparenright}{\kern0pt}\ \ \ \ \isanewline
\ \ \isacommand{then}\isamarkupfalse%
\ \isacommand{have}\isamarkupfalse%
\ u{\isacharunderscore}{\kern0pt}Eqs{\isacharunderscore}{\kern0pt}i{\isadigit{1}}{\isacharcolon}{\kern0pt}\ {\isachardoublequoteopen}u{\isacharequal}{\kern0pt}i{\isadigit{1}}{\isachardoublequoteclose}\isanewline
\ \ \ \ \isacommand{by}\isamarkupfalse%
\ {\isacharparenleft}{\kern0pt}typecheck{\isacharunderscore}{\kern0pt}cfuncs{\isacharcomma}{\kern0pt}\ meson\ cfunc{\isacharunderscore}{\kern0pt}coprod{\isacharunderscore}{\kern0pt}type\ comp{\isacharunderscore}{\kern0pt}type\ i{\isadigit{1}}{\isacharunderscore}{\kern0pt}sEqs{\isacharunderscore}{\kern0pt}i{\isadigit{1}}zUi{\isadigit{1}}s{\isacharunderscore}{\kern0pt}i{\isadigit{1}}\ natural{\isacharunderscore}{\kern0pt}number{\isacharunderscore}{\kern0pt}object{\isacharunderscore}{\kern0pt}func{\isacharunderscore}{\kern0pt}unique\ successor{\isacharunderscore}{\kern0pt}type\ zero{\isacharunderscore}{\kern0pt}type{\isacharparenright}{\kern0pt}\isanewline
\ \ \isacommand{have}\isamarkupfalse%
\ g{\isacharunderscore}{\kern0pt}s{\isacharunderscore}{\kern0pt}type{\isacharbrackleft}{\kern0pt}type{\isacharunderscore}{\kern0pt}rule{\isacharbrackright}{\kern0pt}{\isacharcolon}{\kern0pt}\ {\isachardoublequoteopen}g\ {\isasymcirc}\isactrlsub c\ successor{\isacharcolon}{\kern0pt}\ {\isasymnat}\isactrlsub c\ {\isasymrightarrow}\ {\isacharparenleft}{\kern0pt}{\isasymone}\ {\isasymCoprod}\ {\isasymnat}\isactrlsub c{\isacharparenright}{\kern0pt}{\isachardoublequoteclose}\isanewline
\ \ \ \ \isacommand{by}\isamarkupfalse%
\ typecheck{\isacharunderscore}{\kern0pt}cfuncs\isanewline
\ \ \isacommand{have}\isamarkupfalse%
\ g{\isacharunderscore}{\kern0pt}s{\isacharunderscore}{\kern0pt}triangle{\isacharcolon}{\kern0pt}\ {\isachardoublequoteopen}{\isacharparenleft}{\kern0pt}g{\isasymcirc}\isactrlsub c\ successor{\isacharparenright}{\kern0pt}\ {\isasymcirc}\isactrlsub c\ zero\ {\isacharequal}{\kern0pt}\ i{\isadigit{1}}\ {\isasymcirc}\isactrlsub c\ zero{\isachardoublequoteclose}\isanewline
\ \ \ \ \isacommand{using}\isamarkupfalse%
\ comp{\isacharunderscore}{\kern0pt}associative{\isadigit{2}}\ second{\isacharunderscore}{\kern0pt}diagram{\isadigit{3}}\ \isacommand{by}\isamarkupfalse%
\ {\isacharparenleft}{\kern0pt}typecheck{\isacharunderscore}{\kern0pt}cfuncs{\isacharcomma}{\kern0pt}\ force{\isacharparenright}{\kern0pt}\isanewline
\ \ \isacommand{then}\isamarkupfalse%
\ \isacommand{have}\isamarkupfalse%
\ u{\isacharunderscore}{\kern0pt}Eqs{\isacharunderscore}{\kern0pt}g{\isacharunderscore}{\kern0pt}s{\isacharcolon}{\kern0pt}\ {\isachardoublequoteopen}u{\isacharequal}{\kern0pt}\ g{\isasymcirc}\isactrlsub c\ successor{\isachardoublequoteclose}\isanewline
\ \ \ \ \isacommand{by}\isamarkupfalse%
\ {\isacharparenleft}{\kern0pt}typecheck{\isacharunderscore}{\kern0pt}cfuncs{\isacharcomma}{\kern0pt}\ smt\ {\isacharparenleft}{\kern0pt}verit{\isacharcomma}{\kern0pt}\ ccfv{\isacharunderscore}{\kern0pt}SIG{\isacharparenright}{\kern0pt}\ cfunc{\isacharunderscore}{\kern0pt}coprod{\isacharunderscore}{\kern0pt}type\ comp{\isacharunderscore}{\kern0pt}type\ g{\isacharunderscore}{\kern0pt}s{\isacharunderscore}{\kern0pt}s{\isacharunderscore}{\kern0pt}Eqs{\isacharunderscore}{\kern0pt}i{\isadigit{1}}zUi{\isadigit{1}}s{\isacharunderscore}{\kern0pt}g{\isacharunderscore}{\kern0pt}s\ g{\isacharunderscore}{\kern0pt}s{\isacharunderscore}{\kern0pt}triangle\ i{\isadigit{1}}{\isacharunderscore}{\kern0pt}sEqs{\isacharunderscore}{\kern0pt}i{\isadigit{1}}zUi{\isadigit{1}}s{\isacharunderscore}{\kern0pt}i{\isadigit{1}}\ natural{\isacharunderscore}{\kern0pt}number{\isacharunderscore}{\kern0pt}object{\isacharunderscore}{\kern0pt}func{\isacharunderscore}{\kern0pt}unique\ u{\isacharunderscore}{\kern0pt}Eqs{\isacharunderscore}{\kern0pt}i{\isadigit{1}}\ zero{\isacharunderscore}{\kern0pt}type{\isacharparenright}{\kern0pt}\isanewline
\ \ \isacommand{then}\isamarkupfalse%
\ \isacommand{have}\isamarkupfalse%
\ g{\isacharunderscore}{\kern0pt}sEqs{\isacharunderscore}{\kern0pt}i{\isadigit{1}}{\isacharcolon}{\kern0pt}\ {\isachardoublequoteopen}g{\isasymcirc}\isactrlsub c\ successor\ {\isacharequal}{\kern0pt}\ i{\isadigit{1}}{\isachardoublequoteclose}\isanewline
\ \ \ \ \isacommand{using}\isamarkupfalse%
\ u{\isacharunderscore}{\kern0pt}Eqs{\isacharunderscore}{\kern0pt}i{\isadigit{1}}\ \isacommand{by}\isamarkupfalse%
\ blast\isanewline
\ \ \isacommand{have}\isamarkupfalse%
\ eq{\isadigit{1}}{\isacharcolon}{\kern0pt}\ {\isachardoublequoteopen}{\isacharparenleft}{\kern0pt}zero\ {\isasymamalg}\ successor{\isacharparenright}{\kern0pt}\ {\isasymcirc}\isactrlsub c\ g\ {\isacharequal}{\kern0pt}\ id{\isacharparenleft}{\kern0pt}{\isasymnat}\isactrlsub c{\isacharparenright}{\kern0pt}{\isachardoublequoteclose}\isanewline
\ \ \ \ \isacommand{by}\isamarkupfalse%
\ {\isacharparenleft}{\kern0pt}typecheck{\isacharunderscore}{\kern0pt}cfuncs{\isacharcomma}{\kern0pt}\ smt\ {\isacharparenleft}{\kern0pt}verit{\isacharcomma}{\kern0pt}\ best{\isacharparenright}{\kern0pt}\ cfunc{\isacharunderscore}{\kern0pt}coprod{\isacharunderscore}{\kern0pt}comp\ comp{\isacharunderscore}{\kern0pt}associative{\isadigit{2}}\ g{\isacharunderscore}{\kern0pt}square\ g{\isacharunderscore}{\kern0pt}triangle\ i{\isadigit{0}}{\isacharunderscore}{\kern0pt}def\ i{\isadigit{1}}{\isacharunderscore}{\kern0pt}def\ i{\isadigit{1}}{\isacharunderscore}{\kern0pt}type\ id{\isacharunderscore}{\kern0pt}left{\isacharunderscore}{\kern0pt}unit{\isadigit{2}}\ id{\isacharunderscore}{\kern0pt}right{\isacharunderscore}{\kern0pt}unit{\isadigit{2}}\ left{\isacharunderscore}{\kern0pt}coproj{\isacharunderscore}{\kern0pt}cfunc{\isacharunderscore}{\kern0pt}coprod\ natural{\isacharunderscore}{\kern0pt}number{\isacharunderscore}{\kern0pt}object{\isacharunderscore}{\kern0pt}func{\isacharunderscore}{\kern0pt}unique\ right{\isacharunderscore}{\kern0pt}coproj{\isacharunderscore}{\kern0pt}cfunc{\isacharunderscore}{\kern0pt}coprod{\isacharparenright}{\kern0pt}\isanewline
\ \ \isacommand{then}\isamarkupfalse%
\ \isacommand{have}\isamarkupfalse%
\ eq{\isadigit{2}}{\isacharcolon}{\kern0pt}\ {\isachardoublequoteopen}g\ {\isasymcirc}\isactrlsub c\ {\isacharparenleft}{\kern0pt}zero\ {\isasymamalg}\ successor{\isacharparenright}{\kern0pt}\ {\isacharequal}{\kern0pt}\ id{\isacharparenleft}{\kern0pt}{\isasymone}\ {\isasymCoprod}\ {\isasymnat}\isactrlsub c{\isacharparenright}{\kern0pt}{\isachardoublequoteclose}\isanewline
\ \ \ \ \isacommand{by}\isamarkupfalse%
\ {\isacharparenleft}{\kern0pt}typecheck{\isacharunderscore}{\kern0pt}cfuncs{\isacharcomma}{\kern0pt}\ metis\ cfunc{\isacharunderscore}{\kern0pt}coprod{\isacharunderscore}{\kern0pt}comp\ g{\isacharunderscore}{\kern0pt}sEqs{\isacharunderscore}{\kern0pt}i{\isadigit{1}}\ g{\isacharunderscore}{\kern0pt}triangle\ i{\isadigit{0}}{\isacharunderscore}{\kern0pt}def\ i{\isadigit{1}}{\isacharunderscore}{\kern0pt}def\ id{\isacharunderscore}{\kern0pt}coprod{\isacharparenright}{\kern0pt}\isanewline
\ \ \isacommand{show}\isamarkupfalse%
\ {\isachardoublequoteopen}isomorphism{\isacharparenleft}{\kern0pt}zero\ {\isasymamalg}\ successor{\isacharparenright}{\kern0pt}{\isachardoublequoteclose}\isanewline
\ \ \ \ \isacommand{using}\isamarkupfalse%
\ cfunc{\isacharunderscore}{\kern0pt}coprod{\isacharunderscore}{\kern0pt}type\ eq{\isadigit{1}}\ eq{\isadigit{2}}\ g{\isacharunderscore}{\kern0pt}type\ isomorphism{\isacharunderscore}{\kern0pt}def{\isadigit{3}}\ successor{\isacharunderscore}{\kern0pt}type\ zero{\isacharunderscore}{\kern0pt}type\ \isacommand{by}\isamarkupfalse%
\ blast\isanewline
\isacommand{qed}\isamarkupfalse%
%
\endisatagproof
{\isafoldproof}%
%
\isadelimproof
\isanewline
%
\endisadelimproof
\isanewline
\isacommand{lemma}\isamarkupfalse%
\ zUs{\isacharunderscore}{\kern0pt}epic{\isacharcolon}{\kern0pt}\isanewline
\ {\isachardoublequoteopen}epimorphism{\isacharparenleft}{\kern0pt}zero\ {\isasymamalg}\ successor{\isacharparenright}{\kern0pt}{\isachardoublequoteclose}\isanewline
%
\isadelimproof
\ \ %
\endisadelimproof
%
\isatagproof
\isacommand{by}\isamarkupfalse%
\ {\isacharparenleft}{\kern0pt}simp\ add{\isacharcolon}{\kern0pt}\ iso{\isacharunderscore}{\kern0pt}imp{\isacharunderscore}{\kern0pt}epi{\isacharunderscore}{\kern0pt}and{\isacharunderscore}{\kern0pt}monic\ oneUN{\isacharunderscore}{\kern0pt}iso{\isacharunderscore}{\kern0pt}N{\isacharunderscore}{\kern0pt}isomorphism{\isacharparenright}{\kern0pt}%
\endisatagproof
{\isafoldproof}%
%
\isadelimproof
\isanewline
%
\endisadelimproof
\isanewline
\isacommand{lemma}\isamarkupfalse%
\ zUs{\isacharunderscore}{\kern0pt}surj{\isacharcolon}{\kern0pt}\isanewline
\ {\isachardoublequoteopen}surjective{\isacharparenleft}{\kern0pt}zero\ {\isasymamalg}\ successor{\isacharparenright}{\kern0pt}{\isachardoublequoteclose}\isanewline
%
\isadelimproof
\ \ %
\endisadelimproof
%
\isatagproof
\isacommand{by}\isamarkupfalse%
\ {\isacharparenleft}{\kern0pt}simp\ add{\isacharcolon}{\kern0pt}\ cfunc{\isacharunderscore}{\kern0pt}type{\isacharunderscore}{\kern0pt}def\ epi{\isacharunderscore}{\kern0pt}is{\isacharunderscore}{\kern0pt}surj\ zUs{\isacharunderscore}{\kern0pt}epic{\isacharparenright}{\kern0pt}%
\endisatagproof
{\isafoldproof}%
%
\isadelimproof
\isanewline
%
\endisadelimproof
\isanewline
\isacommand{lemma}\isamarkupfalse%
\ nonzero{\isacharunderscore}{\kern0pt}is{\isacharunderscore}{\kern0pt}succ{\isacharunderscore}{\kern0pt}aux{\isacharcolon}{\kern0pt}\isanewline
\ \ \isakeyword{assumes}\ {\isachardoublequoteopen}x\ {\isasymin}\isactrlsub c\ {\isacharparenleft}{\kern0pt}{\isasymone}\ {\isasymCoprod}\ {\isasymnat}\isactrlsub c{\isacharparenright}{\kern0pt}{\isachardoublequoteclose}\isanewline
\ \ \isakeyword{shows}\ {\isachardoublequoteopen}{\isacharparenleft}{\kern0pt}x\ {\isacharequal}{\kern0pt}\ {\isacharparenleft}{\kern0pt}left{\isacharunderscore}{\kern0pt}coproj\ {\isasymone}\ {\isasymnat}\isactrlsub c{\isacharparenright}{\kern0pt}\ {\isasymcirc}\isactrlsub c\ id\ {\isasymone}{\isacharparenright}{\kern0pt}\ {\isasymor}\isanewline
\ \ \ \ \ \ \ \ \ {\isacharparenleft}{\kern0pt}{\isasymexists}n{\isachardot}{\kern0pt}\ {\isacharparenleft}{\kern0pt}n\ {\isasymin}\isactrlsub c\ {\isasymnat}\isactrlsub c{\isacharparenright}{\kern0pt}\ {\isasymand}\ {\isacharparenleft}{\kern0pt}x\ {\isacharequal}{\kern0pt}\ {\isacharparenleft}{\kern0pt}right{\isacharunderscore}{\kern0pt}coproj\ {\isasymone}\ {\isasymnat}\isactrlsub c{\isacharparenright}{\kern0pt}\ {\isasymcirc}\isactrlsub c\ n{\isacharparenright}{\kern0pt}{\isacharparenright}{\kern0pt}{\isachardoublequoteclose}\isanewline
%
\isadelimproof
\ \ %
\endisadelimproof
%
\isatagproof
\isacommand{by}\isamarkupfalse%
{\isacharparenleft}{\kern0pt}clarify{\isacharcomma}{\kern0pt}\ metis\ assms\ coprojs{\isacharunderscore}{\kern0pt}jointly{\isacharunderscore}{\kern0pt}surj\ id{\isacharunderscore}{\kern0pt}type\ one{\isacharunderscore}{\kern0pt}unique{\isacharunderscore}{\kern0pt}element{\isacharparenright}{\kern0pt}%
\endisatagproof
{\isafoldproof}%
%
\isadelimproof
\isanewline
%
\endisadelimproof
\isanewline
\isacommand{lemma}\isamarkupfalse%
\ nonzero{\isacharunderscore}{\kern0pt}is{\isacharunderscore}{\kern0pt}succ{\isacharcolon}{\kern0pt}\isanewline
\ \ \isakeyword{assumes}\ {\isachardoublequoteopen}k\ {\isasymin}\isactrlsub c\ {\isasymnat}\isactrlsub c{\isachardoublequoteclose}\isanewline
\ \ \isakeyword{assumes}\ {\isachardoublequoteopen}k\ {\isasymnoteq}\ zero{\isachardoublequoteclose}\isanewline
\ \ \isakeyword{shows}\ {\isachardoublequoteopen}{\isasymexists}n{\isachardot}{\kern0pt}{\isacharparenleft}{\kern0pt}n{\isasymin}\isactrlsub c\ {\isasymnat}\isactrlsub c\ {\isasymand}\ k\ {\isacharequal}{\kern0pt}\ successor\ {\isasymcirc}\isactrlsub c\ n{\isacharparenright}{\kern0pt}{\isachardoublequoteclose}\isanewline
%
\isadelimproof
%
\endisadelimproof
%
\isatagproof
\isacommand{proof}\isamarkupfalse%
\ {\isacharminus}{\kern0pt}\ \isanewline
\ \ \isacommand{have}\isamarkupfalse%
\ x{\isacharunderscore}{\kern0pt}exists{\isacharcolon}{\kern0pt}\ {\isachardoublequoteopen}{\isasymexists}x{\isachardot}{\kern0pt}\ {\isacharparenleft}{\kern0pt}{\isacharparenleft}{\kern0pt}x\ {\isasymin}\isactrlsub c\ {\isasymone}\ {\isasymCoprod}\ {\isasymnat}\isactrlsub c{\isacharparenright}{\kern0pt}\ {\isasymand}\ {\isacharparenleft}{\kern0pt}zero\ {\isasymamalg}\ successor\ {\isasymcirc}\isactrlsub c\ x\ {\isacharequal}{\kern0pt}\ k{\isacharparenright}{\kern0pt}{\isacharparenright}{\kern0pt}{\isachardoublequoteclose}\isanewline
\ \ \ \ \isacommand{using}\isamarkupfalse%
\ assms\ cfunc{\isacharunderscore}{\kern0pt}type{\isacharunderscore}{\kern0pt}def\ surjective{\isacharunderscore}{\kern0pt}def\ zUs{\isacharunderscore}{\kern0pt}surj\ \isacommand{by}\isamarkupfalse%
\ {\isacharparenleft}{\kern0pt}typecheck{\isacharunderscore}{\kern0pt}cfuncs{\isacharcomma}{\kern0pt}\ auto{\isacharparenright}{\kern0pt}\isanewline
\ \ \isacommand{obtain}\isamarkupfalse%
\ x\ \isakeyword{where}\ x{\isacharunderscore}{\kern0pt}def{\isacharcolon}{\kern0pt}\ {\isachardoublequoteopen}{\isacharparenleft}{\kern0pt}{\isacharparenleft}{\kern0pt}x\ {\isasymin}\isactrlsub c\ {\isasymone}\ {\isasymCoprod}\ {\isasymnat}\isactrlsub c{\isacharparenright}{\kern0pt}\ {\isasymand}\ {\isacharparenleft}{\kern0pt}zero\ {\isasymamalg}\ successor\ {\isasymcirc}\isactrlsub c\ x\ {\isacharequal}{\kern0pt}\ k{\isacharparenright}{\kern0pt}{\isacharparenright}{\kern0pt}{\isachardoublequoteclose}\isanewline
\ \ \ \ \isacommand{using}\isamarkupfalse%
\ x{\isacharunderscore}{\kern0pt}exists\ \isacommand{by}\isamarkupfalse%
\ blast\isanewline
\ \ \isacommand{have}\isamarkupfalse%
\ cases{\isacharcolon}{\kern0pt}\ {\isachardoublequoteopen}{\isacharparenleft}{\kern0pt}x\ {\isacharequal}{\kern0pt}\ {\isacharparenleft}{\kern0pt}left{\isacharunderscore}{\kern0pt}coproj\ {\isasymone}\ {\isasymnat}\isactrlsub c{\isacharparenright}{\kern0pt}\ {\isasymcirc}\isactrlsub c\ id\ {\isasymone}{\isacharparenright}{\kern0pt}\ {\isasymor}\ \isanewline
\ \ \ \ \ \ \ \ \ \ \ \ \ \ \ \ {\isacharparenleft}{\kern0pt}{\isasymexists}n{\isachardot}{\kern0pt}\ {\isacharparenleft}{\kern0pt}n\ {\isasymin}\isactrlsub c\ {\isasymnat}\isactrlsub c\ {\isasymand}\ x\ {\isacharequal}{\kern0pt}\ {\isacharparenleft}{\kern0pt}right{\isacharunderscore}{\kern0pt}coproj\ {\isasymone}\ {\isasymnat}\isactrlsub c{\isacharparenright}{\kern0pt}\ {\isasymcirc}\isactrlsub c\ n{\isacharparenright}{\kern0pt}{\isacharparenright}{\kern0pt}{\isachardoublequoteclose}\isanewline
\ \ \ \ \isacommand{by}\isamarkupfalse%
\ {\isacharparenleft}{\kern0pt}simp\ add{\isacharcolon}{\kern0pt}\ nonzero{\isacharunderscore}{\kern0pt}is{\isacharunderscore}{\kern0pt}succ{\isacharunderscore}{\kern0pt}aux\ x{\isacharunderscore}{\kern0pt}def{\isacharparenright}{\kern0pt}\isanewline
\ \ \isacommand{have}\isamarkupfalse%
\ not{\isacharunderscore}{\kern0pt}case{\isacharunderscore}{\kern0pt}{\isadigit{1}}{\isacharcolon}{\kern0pt}\ {\isachardoublequoteopen}x\ {\isasymnoteq}\ {\isacharparenleft}{\kern0pt}left{\isacharunderscore}{\kern0pt}coproj\ {\isasymone}\ {\isasymnat}\isactrlsub c{\isacharparenright}{\kern0pt}\ {\isasymcirc}\isactrlsub c\ id\ {\isasymone}{\isachardoublequoteclose}\isanewline
\ \ \isacommand{proof}\isamarkupfalse%
{\isacharparenleft}{\kern0pt}rule\ ccontr{\isacharcomma}{\kern0pt}clarify{\isacharparenright}{\kern0pt}\isanewline
\ \ \ \ \isacommand{assume}\isamarkupfalse%
\ bwoc{\isacharcolon}{\kern0pt}\ {\isachardoublequoteopen}x\ {\isacharequal}{\kern0pt}\ left{\isacharunderscore}{\kern0pt}coproj\ {\isasymone}\ {\isasymnat}\isactrlsub c\ {\isasymcirc}\isactrlsub c\ id\isactrlsub c\ {\isasymone}{\isachardoublequoteclose}\isanewline
\ \ \ \ \isacommand{have}\isamarkupfalse%
\ contradiction{\isacharcolon}{\kern0pt}\ {\isachardoublequoteopen}k\ {\isacharequal}{\kern0pt}\ zero{\isachardoublequoteclose}\isanewline
\ \ \ \ \ \ \isacommand{by}\isamarkupfalse%
\ {\isacharparenleft}{\kern0pt}metis\ bwoc\ id{\isacharunderscore}{\kern0pt}right{\isacharunderscore}{\kern0pt}unit{\isadigit{2}}\ left{\isacharunderscore}{\kern0pt}coproj{\isacharunderscore}{\kern0pt}cfunc{\isacharunderscore}{\kern0pt}coprod\ left{\isacharunderscore}{\kern0pt}proj{\isacharunderscore}{\kern0pt}type\ successor{\isacharunderscore}{\kern0pt}type\ x{\isacharunderscore}{\kern0pt}def\ zero{\isacharunderscore}{\kern0pt}type{\isacharparenright}{\kern0pt}\isanewline
\ \ \ \ \isacommand{show}\isamarkupfalse%
\ False\isanewline
\ \ \ \ \ \ \isacommand{using}\isamarkupfalse%
\ contradiction\ assms{\isacharparenleft}{\kern0pt}{\isadigit{2}}{\isacharparenright}{\kern0pt}\ \isacommand{by}\isamarkupfalse%
\ force\isanewline
\ \ \isacommand{qed}\isamarkupfalse%
\isanewline
\ \ \isacommand{then}\isamarkupfalse%
\ \isacommand{obtain}\isamarkupfalse%
\ n\ \isakeyword{where}\ n{\isacharunderscore}{\kern0pt}def{\isacharcolon}{\kern0pt}\ {\isachardoublequoteopen}n\ {\isasymin}\isactrlsub c\ {\isasymnat}\isactrlsub c\ {\isasymand}\ x\ {\isacharequal}{\kern0pt}\ {\isacharparenleft}{\kern0pt}right{\isacharunderscore}{\kern0pt}coproj\ {\isasymone}\ {\isasymnat}\isactrlsub c{\isacharparenright}{\kern0pt}\ {\isasymcirc}\isactrlsub c\ n{\isachardoublequoteclose}\isanewline
\ \ \ \ \isacommand{using}\isamarkupfalse%
\ cases\ \isacommand{by}\isamarkupfalse%
\ blast\isanewline
\ \ \isacommand{then}\isamarkupfalse%
\ \isacommand{have}\isamarkupfalse%
\ {\isachardoublequoteopen}k\ {\isacharequal}{\kern0pt}\ zero\ {\isasymamalg}\ successor\ {\isasymcirc}\isactrlsub c\ x{\isachardoublequoteclose}\isanewline
\ \ \ \ \isacommand{using}\isamarkupfalse%
\ x{\isacharunderscore}{\kern0pt}def\ \isacommand{by}\isamarkupfalse%
\ blast\isanewline
\ \ \isacommand{also}\isamarkupfalse%
\ \isacommand{have}\isamarkupfalse%
\ {\isachardoublequoteopen}{\isachardot}{\kern0pt}{\isachardot}{\kern0pt}{\isachardot}{\kern0pt}\ {\isacharequal}{\kern0pt}\ zero\ {\isasymamalg}\ successor\ {\isasymcirc}\isactrlsub c\ \ right{\isacharunderscore}{\kern0pt}coproj\ {\isasymone}\ {\isasymnat}\isactrlsub c\ {\isasymcirc}\isactrlsub c\ n{\isachardoublequoteclose}\isanewline
\ \ \ \ \isacommand{by}\isamarkupfalse%
\ {\isacharparenleft}{\kern0pt}simp\ add{\isacharcolon}{\kern0pt}\ n{\isacharunderscore}{\kern0pt}def{\isacharparenright}{\kern0pt}\isanewline
\ \ \isacommand{also}\isamarkupfalse%
\ \isacommand{have}\isamarkupfalse%
\ {\isachardoublequoteopen}{\isachardot}{\kern0pt}{\isachardot}{\kern0pt}{\isachardot}{\kern0pt}\ {\isacharequal}{\kern0pt}\ \ {\isacharparenleft}{\kern0pt}zero\ {\isasymamalg}\ successor\ {\isasymcirc}\isactrlsub c\ \ right{\isacharunderscore}{\kern0pt}coproj\ {\isasymone}\ {\isasymnat}\isactrlsub c{\isacharparenright}{\kern0pt}\ {\isasymcirc}\isactrlsub c\ n{\isachardoublequoteclose}\isanewline
\ \ \ \ \isacommand{using}\isamarkupfalse%
\ cfunc{\isacharunderscore}{\kern0pt}coprod{\isacharunderscore}{\kern0pt}type\ cfunc{\isacharunderscore}{\kern0pt}type{\isacharunderscore}{\kern0pt}def\ comp{\isacharunderscore}{\kern0pt}associative\ n{\isacharunderscore}{\kern0pt}def\ right{\isacharunderscore}{\kern0pt}proj{\isacharunderscore}{\kern0pt}type\ successor{\isacharunderscore}{\kern0pt}type\ zero{\isacharunderscore}{\kern0pt}type\ \isacommand{by}\isamarkupfalse%
\ auto\isanewline
\ \ \isacommand{also}\isamarkupfalse%
\ \isacommand{have}\isamarkupfalse%
\ {\isachardoublequoteopen}{\isachardot}{\kern0pt}{\isachardot}{\kern0pt}{\isachardot}{\kern0pt}\ {\isacharequal}{\kern0pt}\ successor\ {\isasymcirc}\isactrlsub c\ n{\isachardoublequoteclose}\isanewline
\ \ \ \ \isacommand{using}\isamarkupfalse%
\ right{\isacharunderscore}{\kern0pt}coproj{\isacharunderscore}{\kern0pt}cfunc{\isacharunderscore}{\kern0pt}coprod\ successor{\isacharunderscore}{\kern0pt}type\ zero{\isacharunderscore}{\kern0pt}type\ \isacommand{by}\isamarkupfalse%
\ auto\isanewline
\ \ \isacommand{then}\isamarkupfalse%
\ \isacommand{show}\isamarkupfalse%
\ {\isacharquery}{\kern0pt}thesis\isanewline
\ \ \ \ \isacommand{using}\isamarkupfalse%
\ \ \ calculation\ n{\isacharunderscore}{\kern0pt}def\ \isacommand{by}\isamarkupfalse%
\ auto\isanewline
\isacommand{qed}\isamarkupfalse%
%
\endisatagproof
{\isafoldproof}%
%
\isadelimproof
%
\endisadelimproof
%
\isadelimdocument
%
\endisadelimdocument
%
\isatagdocument
%
\isamarkupsubsection{Predecessor%
}
\isamarkuptrue%
%
\endisatagdocument
{\isafolddocument}%
%
\isadelimdocument
%
\endisadelimdocument
\isacommand{definition}\isamarkupfalse%
\ predecessor\ {\isacharcolon}{\kern0pt}{\isacharcolon}{\kern0pt}\ {\isachardoublequoteopen}cfunc{\isachardoublequoteclose}\ \isakeyword{where}\isanewline
\ \ {\isachardoublequoteopen}predecessor\ {\isacharequal}{\kern0pt}\ {\isacharparenleft}{\kern0pt}THE\ f{\isachardot}{\kern0pt}\ f\ {\isacharcolon}{\kern0pt}\ {\isasymnat}\isactrlsub c\ {\isasymrightarrow}\ {\isasymone}\ {\isasymCoprod}\ {\isasymnat}\isactrlsub c\ \isanewline
\ \ \ \ {\isasymand}\ f\ {\isasymcirc}\isactrlsub c\ {\isacharparenleft}{\kern0pt}zero\ {\isasymamalg}\ successor{\isacharparenright}{\kern0pt}\ {\isacharequal}{\kern0pt}\ id\ {\isacharparenleft}{\kern0pt}{\isasymone}\ {\isasymCoprod}\ {\isasymnat}\isactrlsub c{\isacharparenright}{\kern0pt}\ {\isasymand}\ \ {\isacharparenleft}{\kern0pt}zero\ {\isasymamalg}\ successor{\isacharparenright}{\kern0pt}\ {\isasymcirc}\isactrlsub c\ f\ {\isacharequal}{\kern0pt}\ id\ {\isasymnat}\isactrlsub c{\isacharparenright}{\kern0pt}{\isachardoublequoteclose}\isanewline
\isanewline
\isacommand{lemma}\isamarkupfalse%
\ predecessor{\isacharunderscore}{\kern0pt}def{\isadigit{2}}{\isacharcolon}{\kern0pt}\isanewline
\ \ {\isachardoublequoteopen}predecessor\ {\isacharcolon}{\kern0pt}\ {\isasymnat}\isactrlsub c\ {\isasymrightarrow}\ {\isasymone}\ {\isasymCoprod}\ {\isasymnat}\isactrlsub c\ {\isasymand}\ predecessor\ {\isasymcirc}\isactrlsub c\ {\isacharparenleft}{\kern0pt}zero\ {\isasymamalg}\ successor{\isacharparenright}{\kern0pt}\ {\isacharequal}{\kern0pt}\ id\ {\isacharparenleft}{\kern0pt}{\isasymone}\ {\isasymCoprod}\ {\isasymnat}\isactrlsub c{\isacharparenright}{\kern0pt}\isanewline
\ \ \ \ {\isasymand}\ {\isacharparenleft}{\kern0pt}zero\ {\isasymamalg}\ successor{\isacharparenright}{\kern0pt}\ {\isasymcirc}\isactrlsub c\ predecessor\ {\isacharequal}{\kern0pt}\ id\ {\isasymnat}\isactrlsub c{\isachardoublequoteclose}\isanewline
%
\isadelimproof
%
\endisadelimproof
%
\isatagproof
\isacommand{proof}\isamarkupfalse%
\ {\isacharparenleft}{\kern0pt}unfold\ predecessor{\isacharunderscore}{\kern0pt}def{\isacharcomma}{\kern0pt}\ rule\ theI{\isacharprime}{\kern0pt}{\isacharcomma}{\kern0pt}\ safe{\isacharparenright}{\kern0pt}\isanewline
\ \ \isacommand{show}\isamarkupfalse%
\ {\isachardoublequoteopen}{\isasymexists}x{\isachardot}{\kern0pt}\ x\ {\isacharcolon}{\kern0pt}\ {\isasymnat}\isactrlsub c\ {\isasymrightarrow}\ {\isasymone}\ {\isasymCoprod}\ {\isasymnat}\isactrlsub c\ {\isasymand}\isanewline
\ \ \ \ \ \ \ \ x\ {\isasymcirc}\isactrlsub c\ zero\ {\isasymamalg}\ successor\ {\isacharequal}{\kern0pt}\ id\isactrlsub c\ {\isacharparenleft}{\kern0pt}{\isasymone}\ {\isasymCoprod}\ {\isasymnat}\isactrlsub c{\isacharparenright}{\kern0pt}\ {\isasymand}\ zero\ {\isasymamalg}\ successor\ {\isasymcirc}\isactrlsub c\ x\ {\isacharequal}{\kern0pt}\ id\isactrlsub c\ {\isasymnat}\isactrlsub c{\isachardoublequoteclose}\isanewline
\ \ \ \ \isacommand{using}\isamarkupfalse%
\ oneUN{\isacharunderscore}{\kern0pt}iso{\isacharunderscore}{\kern0pt}N{\isacharunderscore}{\kern0pt}isomorphism\ \isacommand{by}\isamarkupfalse%
\ {\isacharparenleft}{\kern0pt}typecheck{\isacharunderscore}{\kern0pt}cfuncs{\isacharcomma}{\kern0pt}\ unfold\ isomorphism{\isacharunderscore}{\kern0pt}def\ cfunc{\isacharunderscore}{\kern0pt}type{\isacharunderscore}{\kern0pt}def{\isacharcomma}{\kern0pt}\ auto{\isacharparenright}{\kern0pt}\isanewline
\isacommand{next}\isamarkupfalse%
\isanewline
\ \ \isacommand{fix}\isamarkupfalse%
\ x\ y\isanewline
\ \ \isacommand{assume}\isamarkupfalse%
\ x{\isacharunderscore}{\kern0pt}type{\isacharbrackleft}{\kern0pt}type{\isacharunderscore}{\kern0pt}rule{\isacharbrackright}{\kern0pt}{\isacharcolon}{\kern0pt}\ {\isachardoublequoteopen}x\ {\isacharcolon}{\kern0pt}\ {\isasymnat}\isactrlsub c\ {\isasymrightarrow}\ {\isasymone}\ {\isasymCoprod}\ {\isasymnat}\isactrlsub c{\isachardoublequoteclose}\ \isakeyword{and}\ y{\isacharunderscore}{\kern0pt}type{\isacharbrackleft}{\kern0pt}type{\isacharunderscore}{\kern0pt}rule{\isacharbrackright}{\kern0pt}{\isacharcolon}{\kern0pt}\ {\isachardoublequoteopen}y\ {\isacharcolon}{\kern0pt}\ {\isasymnat}\isactrlsub c\ {\isasymrightarrow}\ {\isasymone}\ {\isasymCoprod}\ {\isasymnat}\isactrlsub c{\isachardoublequoteclose}\isanewline
\ \ \isacommand{assume}\isamarkupfalse%
\ x{\isacharunderscore}{\kern0pt}left{\isacharunderscore}{\kern0pt}inv{\isacharcolon}{\kern0pt}\ {\isachardoublequoteopen}zero\ {\isasymamalg}\ successor\ {\isasymcirc}\isactrlsub c\ x\ {\isacharequal}{\kern0pt}\ id\isactrlsub c\ {\isasymnat}\isactrlsub c{\isachardoublequoteclose}\isanewline
\ \ \isacommand{assume}\isamarkupfalse%
\ {\isachardoublequoteopen}x\ {\isasymcirc}\isactrlsub c\ zero\ {\isasymamalg}\ successor\ {\isacharequal}{\kern0pt}\ id\isactrlsub c\ {\isacharparenleft}{\kern0pt}{\isasymone}\ {\isasymCoprod}\ {\isasymnat}\isactrlsub c{\isacharparenright}{\kern0pt}{\isachardoublequoteclose}\ {\isachardoublequoteopen}y\ {\isasymcirc}\isactrlsub c\ zero\ {\isasymamalg}\ successor\ {\isacharequal}{\kern0pt}\ id\isactrlsub c\ {\isacharparenleft}{\kern0pt}{\isasymone}\ {\isasymCoprod}\ {\isasymnat}\isactrlsub c{\isacharparenright}{\kern0pt}{\isachardoublequoteclose}\isanewline
\ \ \isacommand{then}\isamarkupfalse%
\ \isacommand{have}\isamarkupfalse%
\ {\isachardoublequoteopen}x\ {\isasymcirc}\isactrlsub c\ zero\ {\isasymamalg}\ successor\ {\isacharequal}{\kern0pt}\ y\ {\isasymcirc}\isactrlsub c\ zero\ {\isasymamalg}\ successor{\isachardoublequoteclose}\isanewline
\ \ \ \ \isacommand{by}\isamarkupfalse%
\ auto\isanewline
\ \ \isacommand{then}\isamarkupfalse%
\ \isacommand{have}\isamarkupfalse%
\ {\isachardoublequoteopen}x\ {\isasymcirc}\isactrlsub c\ zero\ {\isasymamalg}\ successor\ {\isasymcirc}\isactrlsub c\ x\ {\isacharequal}{\kern0pt}\ y\ {\isasymcirc}\isactrlsub c\ zero\ {\isasymamalg}\ successor\ {\isasymcirc}\isactrlsub c\ x{\isachardoublequoteclose}\isanewline
\ \ \ \ \isacommand{by}\isamarkupfalse%
\ {\isacharparenleft}{\kern0pt}typecheck{\isacharunderscore}{\kern0pt}cfuncs{\isacharcomma}{\kern0pt}\ auto\ simp\ add{\isacharcolon}{\kern0pt}\ comp{\isacharunderscore}{\kern0pt}associative{\isadigit{2}}{\isacharparenright}{\kern0pt}\isanewline
\ \ \isacommand{then}\isamarkupfalse%
\ \isacommand{show}\isamarkupfalse%
\ {\isachardoublequoteopen}x\ {\isacharequal}{\kern0pt}\ y{\isachardoublequoteclose}\isanewline
\ \ \ \ \isacommand{using}\isamarkupfalse%
\ id{\isacharunderscore}{\kern0pt}right{\isacharunderscore}{\kern0pt}unit{\isadigit{2}}\ x{\isacharunderscore}{\kern0pt}left{\isacharunderscore}{\kern0pt}inv\ x{\isacharunderscore}{\kern0pt}type\ y{\isacharunderscore}{\kern0pt}type\ \isacommand{by}\isamarkupfalse%
\ auto\isanewline
\isacommand{qed}\isamarkupfalse%
%
\endisatagproof
{\isafoldproof}%
%
\isadelimproof
\isanewline
%
\endisadelimproof
\isanewline
\isacommand{lemma}\isamarkupfalse%
\ predecessor{\isacharunderscore}{\kern0pt}type{\isacharbrackleft}{\kern0pt}type{\isacharunderscore}{\kern0pt}rule{\isacharbrackright}{\kern0pt}{\isacharcolon}{\kern0pt}\isanewline
\ \ {\isachardoublequoteopen}predecessor\ {\isacharcolon}{\kern0pt}\ {\isasymnat}\isactrlsub c\ {\isasymrightarrow}\ {\isasymone}\ {\isasymCoprod}\ {\isasymnat}\isactrlsub c{\isachardoublequoteclose}\isanewline
%
\isadelimproof
\ \ %
\endisadelimproof
%
\isatagproof
\isacommand{by}\isamarkupfalse%
\ {\isacharparenleft}{\kern0pt}simp\ add{\isacharcolon}{\kern0pt}\ predecessor{\isacharunderscore}{\kern0pt}def{\isadigit{2}}{\isacharparenright}{\kern0pt}%
\endisatagproof
{\isafoldproof}%
%
\isadelimproof
\isanewline
%
\endisadelimproof
\isanewline
\isacommand{lemma}\isamarkupfalse%
\ predecessor{\isacharunderscore}{\kern0pt}left{\isacharunderscore}{\kern0pt}inv{\isacharcolon}{\kern0pt}\isanewline
\ \ {\isachardoublequoteopen}{\isacharparenleft}{\kern0pt}zero\ {\isasymamalg}\ successor{\isacharparenright}{\kern0pt}\ {\isasymcirc}\isactrlsub c\ predecessor\ {\isacharequal}{\kern0pt}\ id\ {\isasymnat}\isactrlsub c{\isachardoublequoteclose}\isanewline
%
\isadelimproof
\ \ %
\endisadelimproof
%
\isatagproof
\isacommand{by}\isamarkupfalse%
\ {\isacharparenleft}{\kern0pt}simp\ add{\isacharcolon}{\kern0pt}\ predecessor{\isacharunderscore}{\kern0pt}def{\isadigit{2}}{\isacharparenright}{\kern0pt}%
\endisatagproof
{\isafoldproof}%
%
\isadelimproof
\isanewline
%
\endisadelimproof
\isanewline
\isacommand{lemma}\isamarkupfalse%
\ predecessor{\isacharunderscore}{\kern0pt}right{\isacharunderscore}{\kern0pt}inv{\isacharcolon}{\kern0pt}\isanewline
\ \ {\isachardoublequoteopen}predecessor\ {\isasymcirc}\isactrlsub c\ {\isacharparenleft}{\kern0pt}zero\ {\isasymamalg}\ successor{\isacharparenright}{\kern0pt}\ {\isacharequal}{\kern0pt}\ id\ {\isacharparenleft}{\kern0pt}{\isasymone}\ {\isasymCoprod}\ {\isasymnat}\isactrlsub c{\isacharparenright}{\kern0pt}{\isachardoublequoteclose}\isanewline
%
\isadelimproof
\ \ %
\endisadelimproof
%
\isatagproof
\isacommand{by}\isamarkupfalse%
\ {\isacharparenleft}{\kern0pt}simp\ add{\isacharcolon}{\kern0pt}\ predecessor{\isacharunderscore}{\kern0pt}def{\isadigit{2}}{\isacharparenright}{\kern0pt}%
\endisatagproof
{\isafoldproof}%
%
\isadelimproof
\isanewline
%
\endisadelimproof
\isanewline
\isacommand{lemma}\isamarkupfalse%
\ predecessor{\isacharunderscore}{\kern0pt}successor{\isacharcolon}{\kern0pt}\isanewline
\ \ {\isachardoublequoteopen}predecessor\ {\isasymcirc}\isactrlsub c\ successor\ {\isacharequal}{\kern0pt}\ right{\isacharunderscore}{\kern0pt}coproj\ {\isasymone}\ {\isasymnat}\isactrlsub c{\isachardoublequoteclose}\isanewline
%
\isadelimproof
%
\endisadelimproof
%
\isatagproof
\isacommand{proof}\isamarkupfalse%
\ {\isacharminus}{\kern0pt}\isanewline
\ \ \isacommand{have}\isamarkupfalse%
\ {\isachardoublequoteopen}predecessor\ {\isasymcirc}\isactrlsub c\ successor\ {\isacharequal}{\kern0pt}\ predecessor\ {\isasymcirc}\isactrlsub c\ {\isacharparenleft}{\kern0pt}zero\ {\isasymamalg}\ successor{\isacharparenright}{\kern0pt}\ {\isasymcirc}\isactrlsub c\ right{\isacharunderscore}{\kern0pt}coproj\ {\isasymone}\ {\isasymnat}\isactrlsub c{\isachardoublequoteclose}\isanewline
\ \ \ \ \isacommand{using}\isamarkupfalse%
\ right{\isacharunderscore}{\kern0pt}coproj{\isacharunderscore}{\kern0pt}cfunc{\isacharunderscore}{\kern0pt}coprod\ \isacommand{by}\isamarkupfalse%
\ {\isacharparenleft}{\kern0pt}typecheck{\isacharunderscore}{\kern0pt}cfuncs{\isacharcomma}{\kern0pt}\ auto{\isacharparenright}{\kern0pt}\isanewline
\ \ \isacommand{also}\isamarkupfalse%
\ \isacommand{have}\isamarkupfalse%
\ {\isachardoublequoteopen}{\isachardot}{\kern0pt}{\isachardot}{\kern0pt}{\isachardot}{\kern0pt}\ {\isacharequal}{\kern0pt}\ {\isacharparenleft}{\kern0pt}predecessor\ {\isasymcirc}\isactrlsub c\ {\isacharparenleft}{\kern0pt}zero\ {\isasymamalg}\ successor{\isacharparenright}{\kern0pt}{\isacharparenright}{\kern0pt}\ {\isasymcirc}\isactrlsub c\ right{\isacharunderscore}{\kern0pt}coproj\ {\isasymone}\ {\isasymnat}\isactrlsub c{\isachardoublequoteclose}\isanewline
\ \ \ \ \isacommand{by}\isamarkupfalse%
\ {\isacharparenleft}{\kern0pt}typecheck{\isacharunderscore}{\kern0pt}cfuncs{\isacharcomma}{\kern0pt}\ auto\ simp\ add{\isacharcolon}{\kern0pt}\ comp{\isacharunderscore}{\kern0pt}associative{\isadigit{2}}{\isacharparenright}{\kern0pt}\isanewline
\ \ \isacommand{also}\isamarkupfalse%
\ \isacommand{have}\isamarkupfalse%
\ {\isachardoublequoteopen}{\isachardot}{\kern0pt}{\isachardot}{\kern0pt}{\isachardot}{\kern0pt}\ {\isacharequal}{\kern0pt}\ right{\isacharunderscore}{\kern0pt}coproj\ {\isasymone}\ {\isasymnat}\isactrlsub c{\isachardoublequoteclose}\isanewline
\ \ \ \ \isacommand{by}\isamarkupfalse%
\ {\isacharparenleft}{\kern0pt}typecheck{\isacharunderscore}{\kern0pt}cfuncs{\isacharcomma}{\kern0pt}\ simp\ add{\isacharcolon}{\kern0pt}\ id{\isacharunderscore}{\kern0pt}left{\isacharunderscore}{\kern0pt}unit{\isadigit{2}}\ predecessor{\isacharunderscore}{\kern0pt}def{\isadigit{2}}{\isacharparenright}{\kern0pt}\isanewline
\ \ \isacommand{then}\isamarkupfalse%
\ \isacommand{show}\isamarkupfalse%
\ {\isacharquery}{\kern0pt}thesis\isanewline
\ \ \ \ \isacommand{using}\isamarkupfalse%
\ calculation\ \isacommand{by}\isamarkupfalse%
\ auto\isanewline
\isacommand{qed}\isamarkupfalse%
%
\endisatagproof
{\isafoldproof}%
%
\isadelimproof
\isanewline
%
\endisadelimproof
\isanewline
\isacommand{lemma}\isamarkupfalse%
\ predecessor{\isacharunderscore}{\kern0pt}zero{\isacharcolon}{\kern0pt}\isanewline
\ \ {\isachardoublequoteopen}predecessor\ {\isasymcirc}\isactrlsub c\ zero\ {\isacharequal}{\kern0pt}\ left{\isacharunderscore}{\kern0pt}coproj\ {\isasymone}\ {\isasymnat}\isactrlsub c{\isachardoublequoteclose}\isanewline
%
\isadelimproof
%
\endisadelimproof
%
\isatagproof
\isacommand{proof}\isamarkupfalse%
\ {\isacharminus}{\kern0pt}\isanewline
\ \ \isacommand{have}\isamarkupfalse%
\ {\isachardoublequoteopen}predecessor\ {\isasymcirc}\isactrlsub c\ zero\ {\isacharequal}{\kern0pt}\ predecessor\ {\isasymcirc}\isactrlsub c\ {\isacharparenleft}{\kern0pt}zero\ {\isasymamalg}\ successor{\isacharparenright}{\kern0pt}\ {\isasymcirc}\isactrlsub c\ left{\isacharunderscore}{\kern0pt}coproj\ {\isasymone}\ {\isasymnat}\isactrlsub c{\isachardoublequoteclose}\isanewline
\ \ \ \ \isacommand{using}\isamarkupfalse%
\ left{\isacharunderscore}{\kern0pt}coproj{\isacharunderscore}{\kern0pt}cfunc{\isacharunderscore}{\kern0pt}coprod\ \isacommand{by}\isamarkupfalse%
\ {\isacharparenleft}{\kern0pt}typecheck{\isacharunderscore}{\kern0pt}cfuncs{\isacharcomma}{\kern0pt}\ auto{\isacharparenright}{\kern0pt}\isanewline
\ \ \isacommand{also}\isamarkupfalse%
\ \isacommand{have}\isamarkupfalse%
\ {\isachardoublequoteopen}{\isachardot}{\kern0pt}{\isachardot}{\kern0pt}{\isachardot}{\kern0pt}\ {\isacharequal}{\kern0pt}\ {\isacharparenleft}{\kern0pt}predecessor\ {\isasymcirc}\isactrlsub c\ {\isacharparenleft}{\kern0pt}zero\ {\isasymamalg}\ successor{\isacharparenright}{\kern0pt}{\isacharparenright}{\kern0pt}\ {\isasymcirc}\isactrlsub c\ left{\isacharunderscore}{\kern0pt}coproj\ {\isasymone}\ {\isasymnat}\isactrlsub c{\isachardoublequoteclose}\isanewline
\ \ \ \ \isacommand{by}\isamarkupfalse%
\ {\isacharparenleft}{\kern0pt}typecheck{\isacharunderscore}{\kern0pt}cfuncs{\isacharcomma}{\kern0pt}\ auto\ simp\ add{\isacharcolon}{\kern0pt}\ comp{\isacharunderscore}{\kern0pt}associative{\isadigit{2}}{\isacharparenright}{\kern0pt}\isanewline
\ \ \isacommand{also}\isamarkupfalse%
\ \isacommand{have}\isamarkupfalse%
\ {\isachardoublequoteopen}{\isachardot}{\kern0pt}{\isachardot}{\kern0pt}{\isachardot}{\kern0pt}\ {\isacharequal}{\kern0pt}\ left{\isacharunderscore}{\kern0pt}coproj\ {\isasymone}\ {\isasymnat}\isactrlsub c{\isachardoublequoteclose}\isanewline
\ \ \ \ \isacommand{by}\isamarkupfalse%
\ {\isacharparenleft}{\kern0pt}typecheck{\isacharunderscore}{\kern0pt}cfuncs{\isacharcomma}{\kern0pt}\ simp\ add{\isacharcolon}{\kern0pt}\ id{\isacharunderscore}{\kern0pt}left{\isacharunderscore}{\kern0pt}unit{\isadigit{2}}\ predecessor{\isacharunderscore}{\kern0pt}def{\isadigit{2}}{\isacharparenright}{\kern0pt}\isanewline
\ \ \isacommand{then}\isamarkupfalse%
\ \isacommand{show}\isamarkupfalse%
\ {\isacharquery}{\kern0pt}thesis\isanewline
\ \ \ \ \isacommand{using}\isamarkupfalse%
\ calculation\ \isacommand{by}\isamarkupfalse%
\ auto\isanewline
\isacommand{qed}\isamarkupfalse%
%
\endisatagproof
{\isafoldproof}%
%
\isadelimproof
%
\endisadelimproof
%
\isadelimdocument
%
\endisadelimdocument
%
\isatagdocument
%
\isamarkupsubsection{Peano's Axioms and Induction%
}
\isamarkuptrue%
%
\endisatagdocument
{\isafolddocument}%
%
\isadelimdocument
%
\endisadelimdocument
%
\begin{isamarkuptext}%
The lemma below corresponds to Proposition 2.6.7 in Halvorson.%
\end{isamarkuptext}\isamarkuptrue%
\isacommand{lemma}\isamarkupfalse%
\ Peano{\isacharprime}{\kern0pt}s{\isacharunderscore}{\kern0pt}Axioms{\isacharcolon}{\kern0pt}\isanewline
\ {\isachardoublequoteopen}injective\ successor\ \ {\isasymand}\ {\isasymnot}\ surjective\ successor{\isachardoublequoteclose}\isanewline
%
\isadelimproof
%
\endisadelimproof
%
\isatagproof
\isacommand{proof}\isamarkupfalse%
\ {\isacharminus}{\kern0pt}\ \isanewline
\ \ \isacommand{have}\isamarkupfalse%
\ i{\isadigit{1}}{\isacharunderscore}{\kern0pt}mono{\isacharcolon}{\kern0pt}\ {\isachardoublequoteopen}monomorphism{\isacharparenleft}{\kern0pt}right{\isacharunderscore}{\kern0pt}coproj\ {\isasymone}\ {\isasymnat}\isactrlsub c{\isacharparenright}{\kern0pt}{\isachardoublequoteclose}\isanewline
\ \ \ \ \isacommand{by}\isamarkupfalse%
\ {\isacharparenleft}{\kern0pt}simp\ add{\isacharcolon}{\kern0pt}\ right{\isacharunderscore}{\kern0pt}coproj{\isacharunderscore}{\kern0pt}are{\isacharunderscore}{\kern0pt}monomorphisms{\isacharparenright}{\kern0pt}\isanewline
\ \ \isacommand{have}\isamarkupfalse%
\ zUs{\isacharunderscore}{\kern0pt}iso{\isacharcolon}{\kern0pt}\ {\isachardoublequoteopen}isomorphism{\isacharparenleft}{\kern0pt}zero\ {\isasymamalg}\ successor{\isacharparenright}{\kern0pt}{\isachardoublequoteclose}\isanewline
\ \ \ \ \isacommand{using}\isamarkupfalse%
\ oneUN{\isacharunderscore}{\kern0pt}iso{\isacharunderscore}{\kern0pt}N{\isacharunderscore}{\kern0pt}isomorphism\ \isacommand{by}\isamarkupfalse%
\ blast\isanewline
\ \ \isacommand{have}\isamarkupfalse%
\ zUsi{\isadigit{1}}EqsS{\isacharcolon}{\kern0pt}\ {\isachardoublequoteopen}{\isacharparenleft}{\kern0pt}zero\ {\isasymamalg}\ successor{\isacharparenright}{\kern0pt}\ {\isasymcirc}\isactrlsub c\ {\isacharparenleft}{\kern0pt}right{\isacharunderscore}{\kern0pt}coproj\ {\isasymone}\ {\isasymnat}\isactrlsub c{\isacharparenright}{\kern0pt}\ {\isacharequal}{\kern0pt}\ successor{\isachardoublequoteclose}\isanewline
\ \ \ \ \isacommand{using}\isamarkupfalse%
\ right{\isacharunderscore}{\kern0pt}coproj{\isacharunderscore}{\kern0pt}cfunc{\isacharunderscore}{\kern0pt}coprod\ successor{\isacharunderscore}{\kern0pt}type\ zero{\isacharunderscore}{\kern0pt}type\ \isacommand{by}\isamarkupfalse%
\ auto\isanewline
\ \ \isacommand{then}\isamarkupfalse%
\ \isacommand{have}\isamarkupfalse%
\ succ{\isacharunderscore}{\kern0pt}mono{\isacharcolon}{\kern0pt}\ {\isachardoublequoteopen}monomorphism{\isacharparenleft}{\kern0pt}successor{\isacharparenright}{\kern0pt}{\isachardoublequoteclose}\isanewline
\ \ \ \ \isacommand{by}\isamarkupfalse%
\ {\isacharparenleft}{\kern0pt}metis\ cfunc{\isacharunderscore}{\kern0pt}coprod{\isacharunderscore}{\kern0pt}type\ cfunc{\isacharunderscore}{\kern0pt}type{\isacharunderscore}{\kern0pt}def\ composition{\isacharunderscore}{\kern0pt}of{\isacharunderscore}{\kern0pt}monic{\isacharunderscore}{\kern0pt}pair{\isacharunderscore}{\kern0pt}is{\isacharunderscore}{\kern0pt}monic\ i{\isadigit{1}}{\isacharunderscore}{\kern0pt}mono\ iso{\isacharunderscore}{\kern0pt}imp{\isacharunderscore}{\kern0pt}epi{\isacharunderscore}{\kern0pt}and{\isacharunderscore}{\kern0pt}monic\ oneUN{\isacharunderscore}{\kern0pt}iso{\isacharunderscore}{\kern0pt}N{\isacharunderscore}{\kern0pt}isomorphism\ right{\isacharunderscore}{\kern0pt}proj{\isacharunderscore}{\kern0pt}type\ successor{\isacharunderscore}{\kern0pt}type\ zero{\isacharunderscore}{\kern0pt}type{\isacharparenright}{\kern0pt}\isanewline
\ \ \isacommand{obtain}\isamarkupfalse%
\ u\ \isakeyword{where}\ u{\isacharunderscore}{\kern0pt}type{\isacharcolon}{\kern0pt}\ {\isachardoublequoteopen}u{\isacharcolon}{\kern0pt}\ \ {\isasymnat}\isactrlsub c\ \ {\isasymrightarrow}\ {\isasymOmega}{\isachardoublequoteclose}\ \isakeyword{and}\ u{\isacharunderscore}{\kern0pt}def{\isacharcolon}{\kern0pt}\ {\isachardoublequoteopen}u\ {\isasymcirc}\isactrlsub c\ zero\ {\isacharequal}{\kern0pt}\ {\isasymt}\ \ {\isasymand}\ {\isacharparenleft}{\kern0pt}{\isasymf}{\isasymcirc}\isactrlsub c{\isasymbeta}\isactrlbsub {\isasymOmega}\isactrlesub {\isacharparenright}{\kern0pt}\ {\isasymcirc}\isactrlsub c\ u\ {\isacharequal}{\kern0pt}\ u\ {\isasymcirc}\isactrlsub c\ \ successor{\isachardoublequoteclose}\isanewline
\ \ \ \ \isacommand{by}\isamarkupfalse%
\ {\isacharparenleft}{\kern0pt}typecheck{\isacharunderscore}{\kern0pt}cfuncs{\isacharcomma}{\kern0pt}\ metis\ natural{\isacharunderscore}{\kern0pt}number{\isacharunderscore}{\kern0pt}object{\isacharunderscore}{\kern0pt}property{\isacharparenright}{\kern0pt}\ \ \ \ \isanewline
\ \ \isacommand{have}\isamarkupfalse%
\ s{\isacharunderscore}{\kern0pt}not{\isacharunderscore}{\kern0pt}surj{\isacharcolon}{\kern0pt}\ {\isachardoublequoteopen}{\isasymnot}\ surjective\ successor{\isachardoublequoteclose}\isanewline
\ \ \ \ \isacommand{proof}\isamarkupfalse%
\ {\isacharparenleft}{\kern0pt}rule\ ccontr{\isacharcomma}{\kern0pt}\ clarify{\isacharparenright}{\kern0pt}\isanewline
\ \ \ \ \ \ \isacommand{assume}\isamarkupfalse%
\ BWOC\ {\isacharcolon}{\kern0pt}\ {\isachardoublequoteopen}surjective\ successor{\isachardoublequoteclose}\isanewline
\ \ \ \ \ \ \isacommand{obtain}\isamarkupfalse%
\ n\ \isakeyword{where}\ n{\isacharunderscore}{\kern0pt}type{\isacharcolon}{\kern0pt}\ {\isachardoublequoteopen}n\ {\isacharcolon}{\kern0pt}\ {\isasymone}\ {\isasymrightarrow}\ {\isasymnat}\isactrlsub c{\isachardoublequoteclose}\ \isakeyword{and}\ snEqz{\isacharcolon}{\kern0pt}\ {\isachardoublequoteopen}successor\ {\isasymcirc}\isactrlsub c\ n\ {\isacharequal}{\kern0pt}\ zero{\isachardoublequoteclose}\isanewline
\ \ \ \ \ \ \ \ \isacommand{using}\isamarkupfalse%
\ BWOC\ cfunc{\isacharunderscore}{\kern0pt}type{\isacharunderscore}{\kern0pt}def\ successor{\isacharunderscore}{\kern0pt}type\ surjective{\isacharunderscore}{\kern0pt}def\ zero{\isacharunderscore}{\kern0pt}type\ \isacommand{by}\isamarkupfalse%
\ auto\isanewline
\ \ \ \ \ \ \isacommand{then}\isamarkupfalse%
\ \isacommand{show}\isamarkupfalse%
\ False\isanewline
\ \ \ \ \ \ \ \ \isacommand{by}\isamarkupfalse%
\ {\isacharparenleft}{\kern0pt}metis\ zero{\isacharunderscore}{\kern0pt}is{\isacharunderscore}{\kern0pt}not{\isacharunderscore}{\kern0pt}successor{\isacharparenright}{\kern0pt}\isanewline
\ \ \ \ \isacommand{qed}\isamarkupfalse%
\isanewline
\ \ \isacommand{then}\isamarkupfalse%
\ \isacommand{show}\isamarkupfalse%
\ {\isachardoublequoteopen}injective\ successor\ {\isasymand}\ {\isasymnot}\ surjective\ successor{\isachardoublequoteclose}\isanewline
\ \ \ \ \isacommand{using}\isamarkupfalse%
\ monomorphism{\isacharunderscore}{\kern0pt}imp{\isacharunderscore}{\kern0pt}injective\ succ{\isacharunderscore}{\kern0pt}mono\ \isacommand{by}\isamarkupfalse%
\ blast\isanewline
\isacommand{qed}\isamarkupfalse%
%
\endisatagproof
{\isafoldproof}%
%
\isadelimproof
\isanewline
%
\endisadelimproof
\isanewline
\isacommand{lemma}\isamarkupfalse%
\ succ{\isacharunderscore}{\kern0pt}inject{\isacharcolon}{\kern0pt}\isanewline
\ \ \isakeyword{assumes}\ {\isachardoublequoteopen}n\ {\isasymin}\isactrlsub c\ {\isasymnat}\isactrlsub c{\isachardoublequoteclose}\ {\isachardoublequoteopen}m\ {\isasymin}\isactrlsub c\ {\isasymnat}\isactrlsub c{\isachardoublequoteclose}\isanewline
\ \ \isakeyword{shows}\ {\isachardoublequoteopen}successor\ {\isasymcirc}\isactrlsub c\ n\ {\isacharequal}{\kern0pt}\ successor\ {\isasymcirc}\isactrlsub c\ m\ {\isasymLongrightarrow}\ n\ {\isacharequal}{\kern0pt}\ m{\isachardoublequoteclose}\isanewline
%
\isadelimproof
\ \ %
\endisadelimproof
%
\isatagproof
\isacommand{by}\isamarkupfalse%
\ {\isacharparenleft}{\kern0pt}metis\ Peano{\isacharprime}{\kern0pt}s{\isacharunderscore}{\kern0pt}Axioms\ assms\ cfunc{\isacharunderscore}{\kern0pt}type{\isacharunderscore}{\kern0pt}def\ injective{\isacharunderscore}{\kern0pt}def\ successor{\isacharunderscore}{\kern0pt}type{\isacharparenright}{\kern0pt}%
\endisatagproof
{\isafoldproof}%
%
\isadelimproof
\ \isanewline
%
\endisadelimproof
\isanewline
\isacommand{theorem}\isamarkupfalse%
\ nat{\isacharunderscore}{\kern0pt}induction{\isacharcolon}{\kern0pt}\isanewline
\ \ \isakeyword{assumes}\ p{\isacharunderscore}{\kern0pt}type{\isacharbrackleft}{\kern0pt}type{\isacharunderscore}{\kern0pt}rule{\isacharbrackright}{\kern0pt}{\isacharcolon}{\kern0pt}\ {\isachardoublequoteopen}p\ {\isacharcolon}{\kern0pt}\ {\isasymnat}\isactrlsub c\ {\isasymrightarrow}\ {\isasymOmega}{\isachardoublequoteclose}\ \isakeyword{and}\ n{\isacharunderscore}{\kern0pt}type{\isacharbrackleft}{\kern0pt}type{\isacharunderscore}{\kern0pt}rule{\isacharbrackright}{\kern0pt}{\isacharcolon}{\kern0pt}\ {\isachardoublequoteopen}n\ {\isasymin}\isactrlsub c\ {\isasymnat}\isactrlsub c{\isachardoublequoteclose}\isanewline
\ \ \isakeyword{assumes}\ base{\isacharunderscore}{\kern0pt}case{\isacharcolon}{\kern0pt}\ {\isachardoublequoteopen}p\ {\isasymcirc}\isactrlsub c\ zero\ {\isacharequal}{\kern0pt}\ {\isasymt}{\isachardoublequoteclose}\isanewline
\ \ \isakeyword{assumes}\ induction{\isacharunderscore}{\kern0pt}case{\isacharcolon}{\kern0pt}\ {\isachardoublequoteopen}{\isasymAnd}n{\isachardot}{\kern0pt}\ n\ {\isasymin}\isactrlsub c\ {\isasymnat}\isactrlsub c\ {\isasymLongrightarrow}\ p\ {\isasymcirc}\isactrlsub c\ n\ {\isacharequal}{\kern0pt}\ {\isasymt}\ {\isasymLongrightarrow}\ p\ {\isasymcirc}\isactrlsub c\ successor\ {\isasymcirc}\isactrlsub c\ n\ {\isacharequal}{\kern0pt}\ {\isasymt}{\isachardoublequoteclose}\isanewline
\ \ \isakeyword{shows}\ {\isachardoublequoteopen}p\ {\isasymcirc}\isactrlsub c\ n\ {\isacharequal}{\kern0pt}\ {\isasymt}{\isachardoublequoteclose}\isanewline
%
\isadelimproof
%
\endisadelimproof
%
\isatagproof
\isacommand{proof}\isamarkupfalse%
\ {\isacharminus}{\kern0pt}\isanewline
\ \ \isacommand{obtain}\isamarkupfalse%
\ p{\isacharprime}{\kern0pt}\ P\ \isakeyword{where}\isanewline
\ \ \ \ p{\isacharprime}{\kern0pt}{\isacharunderscore}{\kern0pt}type{\isacharbrackleft}{\kern0pt}type{\isacharunderscore}{\kern0pt}rule{\isacharbrackright}{\kern0pt}{\isacharcolon}{\kern0pt}\ {\isachardoublequoteopen}p{\isacharprime}{\kern0pt}\ {\isacharcolon}{\kern0pt}\ P\ {\isasymrightarrow}\ {\isasymnat}\isactrlsub c{\isachardoublequoteclose}\ \isakeyword{and}\isanewline
\ \ \ \ p{\isacharprime}{\kern0pt}{\isacharunderscore}{\kern0pt}equalizer{\isacharcolon}{\kern0pt}\ {\isachardoublequoteopen}p\ {\isasymcirc}\isactrlsub c\ p{\isacharprime}{\kern0pt}\ {\isacharequal}{\kern0pt}\ {\isacharparenleft}{\kern0pt}{\isasymt}\ {\isasymcirc}\isactrlsub c\ {\isasymbeta}\isactrlbsub {\isasymnat}\isactrlsub c\isactrlesub {\isacharparenright}{\kern0pt}\ {\isasymcirc}\isactrlsub c\ p{\isacharprime}{\kern0pt}{\isachardoublequoteclose}\ \isakeyword{and}\isanewline
\ \ \ \ p{\isacharprime}{\kern0pt}{\isacharunderscore}{\kern0pt}uni{\isacharunderscore}{\kern0pt}prop{\isacharcolon}{\kern0pt}\ {\isachardoublequoteopen}{\isasymforall}\ h\ F{\isachardot}{\kern0pt}\ {\isacharparenleft}{\kern0pt}h\ {\isacharcolon}{\kern0pt}\ F\ {\isasymrightarrow}\ {\isasymnat}\isactrlsub c\ {\isasymand}\ p\ {\isasymcirc}\isactrlsub c\ h\ {\isacharequal}{\kern0pt}\ {\isacharparenleft}{\kern0pt}{\isasymt}\ {\isasymcirc}\isactrlsub c\ {\isasymbeta}\isactrlbsub {\isasymnat}\isactrlsub c\isactrlesub {\isacharparenright}{\kern0pt}\ {\isasymcirc}\isactrlsub c\ h{\isacharparenright}{\kern0pt}\ {\isasymlongrightarrow}\ {\isacharparenleft}{\kern0pt}{\isasymexists}{\isacharbang}{\kern0pt}\ k{\isachardot}{\kern0pt}\ k\ {\isacharcolon}{\kern0pt}\ F\ {\isasymrightarrow}\ P\ {\isasymand}\ p{\isacharprime}{\kern0pt}\ {\isasymcirc}\isactrlsub c\ k\ {\isacharequal}{\kern0pt}\ h{\isacharparenright}{\kern0pt}{\isachardoublequoteclose}\isanewline
\ \ \ \ \isacommand{using}\isamarkupfalse%
\ equalizer{\isacharunderscore}{\kern0pt}exists{\isadigit{2}}\ \isacommand{by}\isamarkupfalse%
\ {\isacharparenleft}{\kern0pt}typecheck{\isacharunderscore}{\kern0pt}cfuncs{\isacharcomma}{\kern0pt}\ blast{\isacharparenright}{\kern0pt}\isanewline
\isanewline
\ \ \isacommand{from}\isamarkupfalse%
\ base{\isacharunderscore}{\kern0pt}case\ \isacommand{have}\isamarkupfalse%
\ {\isachardoublequoteopen}p\ {\isasymcirc}\isactrlsub c\ zero\ {\isacharequal}{\kern0pt}\ {\isacharparenleft}{\kern0pt}{\isasymt}\ {\isasymcirc}\isactrlsub c\ {\isasymbeta}\isactrlbsub {\isasymnat}\isactrlsub c\isactrlesub {\isacharparenright}{\kern0pt}\ {\isasymcirc}\isactrlsub c\ zero{\isachardoublequoteclose}\isanewline
\ \ \ \ \isacommand{by}\isamarkupfalse%
\ {\isacharparenleft}{\kern0pt}etcs{\isacharunderscore}{\kern0pt}assocr{\isacharcomma}{\kern0pt}\ etcs{\isacharunderscore}{\kern0pt}subst\ terminal{\isacharunderscore}{\kern0pt}func{\isacharunderscore}{\kern0pt}comp{\isacharunderscore}{\kern0pt}elem\ id{\isacharunderscore}{\kern0pt}right{\isacharunderscore}{\kern0pt}unit{\isadigit{2}}{\isacharcomma}{\kern0pt}\ {\isacharminus}{\kern0pt}{\isacharparenright}{\kern0pt}\isanewline
\ \ \isacommand{then}\isamarkupfalse%
\ \isacommand{obtain}\isamarkupfalse%
\ z{\isacharprime}{\kern0pt}\ \isakeyword{where}\isanewline
\ \ \ \ z{\isacharprime}{\kern0pt}{\isacharunderscore}{\kern0pt}type{\isacharbrackleft}{\kern0pt}type{\isacharunderscore}{\kern0pt}rule{\isacharbrackright}{\kern0pt}{\isacharcolon}{\kern0pt}\ {\isachardoublequoteopen}z{\isacharprime}{\kern0pt}\ {\isasymin}\isactrlsub c\ P{\isachardoublequoteclose}\ \isakeyword{and}\isanewline
\ \ \ \ z{\isacharprime}{\kern0pt}{\isacharunderscore}{\kern0pt}def{\isacharcolon}{\kern0pt}\ {\isachardoublequoteopen}zero\ {\isacharequal}{\kern0pt}\ p{\isacharprime}{\kern0pt}\ {\isasymcirc}\isactrlsub c\ z{\isacharprime}{\kern0pt}{\isachardoublequoteclose}\isanewline
\ \ \ \ \isacommand{using}\isamarkupfalse%
\ p{\isacharprime}{\kern0pt}{\isacharunderscore}{\kern0pt}uni{\isacharunderscore}{\kern0pt}prop\ \isacommand{by}\isamarkupfalse%
\ {\isacharparenleft}{\kern0pt}typecheck{\isacharunderscore}{\kern0pt}cfuncs{\isacharcomma}{\kern0pt}\ metis{\isacharparenright}{\kern0pt}\isanewline
\isanewline
\ \ \isacommand{have}\isamarkupfalse%
\ {\isachardoublequoteopen}p\ {\isasymcirc}\isactrlsub c\ successor\ {\isasymcirc}\isactrlsub c\ p{\isacharprime}{\kern0pt}\ {\isacharequal}{\kern0pt}\ {\isacharparenleft}{\kern0pt}{\isasymt}\ {\isasymcirc}\isactrlsub c\ {\isasymbeta}\isactrlbsub {\isasymnat}\isactrlsub c\isactrlesub {\isacharparenright}{\kern0pt}\ {\isasymcirc}\isactrlsub c\ successor\ {\isasymcirc}\isactrlsub c\ p{\isacharprime}{\kern0pt}{\isachardoublequoteclose}\isanewline
\ \ \isacommand{proof}\isamarkupfalse%
\ {\isacharparenleft}{\kern0pt}etcs{\isacharunderscore}{\kern0pt}rule\ one{\isacharunderscore}{\kern0pt}separator{\isacharparenright}{\kern0pt}\isanewline
\ \ \ \ \isacommand{fix}\isamarkupfalse%
\ m\isanewline
\ \ \ \ \isacommand{assume}\isamarkupfalse%
\ m{\isacharunderscore}{\kern0pt}type{\isacharbrackleft}{\kern0pt}type{\isacharunderscore}{\kern0pt}rule{\isacharbrackright}{\kern0pt}{\isacharcolon}{\kern0pt}\ {\isachardoublequoteopen}m\ {\isasymin}\isactrlsub c\ P{\isachardoublequoteclose}\isanewline
\isanewline
\ \ \ \ \isacommand{have}\isamarkupfalse%
\ {\isachardoublequoteopen}p\ \ {\isasymcirc}\isactrlsub c\ p{\isacharprime}{\kern0pt}\ {\isasymcirc}\isactrlsub c\ m\ {\isacharequal}{\kern0pt}\ {\isasymt}\ {\isasymcirc}\isactrlsub c\ {\isasymbeta}\isactrlbsub {\isasymnat}\isactrlsub c\isactrlesub \ {\isasymcirc}\isactrlsub c\ p{\isacharprime}{\kern0pt}\ {\isasymcirc}\isactrlsub c\ m{\isachardoublequoteclose}\isanewline
\ \ \ \ \ \ \isacommand{by}\isamarkupfalse%
\ {\isacharparenleft}{\kern0pt}etcs{\isacharunderscore}{\kern0pt}assocl{\isacharcomma}{\kern0pt}\ simp\ add{\isacharcolon}{\kern0pt}\ p{\isacharprime}{\kern0pt}{\isacharunderscore}{\kern0pt}equalizer{\isacharparenright}{\kern0pt}\isanewline
\ \ \ \ \isacommand{then}\isamarkupfalse%
\ \isacommand{have}\isamarkupfalse%
\ {\isachardoublequoteopen}p\ {\isasymcirc}\isactrlsub c\ p{\isacharprime}{\kern0pt}\ {\isasymcirc}\isactrlsub c\ m\ {\isacharequal}{\kern0pt}\ {\isasymt}{\isachardoublequoteclose}\isanewline
\ \ \ \ \ \ \isacommand{by}\isamarkupfalse%
\ {\isacharparenleft}{\kern0pt}{\isacharminus}{\kern0pt}{\isacharcomma}{\kern0pt}\ etcs{\isacharunderscore}{\kern0pt}subst{\isacharunderscore}{\kern0pt}asm\ terminal{\isacharunderscore}{\kern0pt}func{\isacharunderscore}{\kern0pt}comp{\isacharunderscore}{\kern0pt}elem\ id{\isacharunderscore}{\kern0pt}right{\isacharunderscore}{\kern0pt}unit{\isadigit{2}}{\isacharcomma}{\kern0pt}\ simp{\isacharparenright}{\kern0pt}\isanewline
\ \ \ \ \isacommand{then}\isamarkupfalse%
\ \isacommand{have}\isamarkupfalse%
\ {\isachardoublequoteopen}p\ {\isasymcirc}\isactrlsub c\ successor\ {\isasymcirc}\isactrlsub c\ p{\isacharprime}{\kern0pt}\ {\isasymcirc}\isactrlsub c\ m\ {\isacharequal}{\kern0pt}\ {\isasymt}{\isachardoublequoteclose}\isanewline
\ \ \ \ \ \ \isacommand{using}\isamarkupfalse%
\ induction{\isacharunderscore}{\kern0pt}case\ \isacommand{by}\isamarkupfalse%
\ {\isacharparenleft}{\kern0pt}typecheck{\isacharunderscore}{\kern0pt}cfuncs{\isacharcomma}{\kern0pt}\ simp{\isacharparenright}{\kern0pt}\isanewline
\ \ \ \ \isacommand{then}\isamarkupfalse%
\ \isacommand{show}\isamarkupfalse%
\ {\isachardoublequoteopen}{\isacharparenleft}{\kern0pt}p\ {\isasymcirc}\isactrlsub c\ successor\ {\isasymcirc}\isactrlsub c\ p{\isacharprime}{\kern0pt}{\isacharparenright}{\kern0pt}\ {\isasymcirc}\isactrlsub c\ m\ {\isacharequal}{\kern0pt}\ {\isacharparenleft}{\kern0pt}{\isacharparenleft}{\kern0pt}{\isasymt}\ {\isasymcirc}\isactrlsub c\ {\isasymbeta}\isactrlbsub {\isasymnat}\isactrlsub c\isactrlesub {\isacharparenright}{\kern0pt}\ {\isasymcirc}\isactrlsub c\ successor\ {\isasymcirc}\isactrlsub c\ p{\isacharprime}{\kern0pt}{\isacharparenright}{\kern0pt}\ {\isasymcirc}\isactrlsub c\ m{\isachardoublequoteclose}\isanewline
\ \ \ \ \ \ \isacommand{by}\isamarkupfalse%
\ {\isacharparenleft}{\kern0pt}etcs{\isacharunderscore}{\kern0pt}assocr{\isacharcomma}{\kern0pt}\ etcs{\isacharunderscore}{\kern0pt}subst\ terminal{\isacharunderscore}{\kern0pt}func{\isacharunderscore}{\kern0pt}comp{\isacharunderscore}{\kern0pt}elem\ id{\isacharunderscore}{\kern0pt}right{\isacharunderscore}{\kern0pt}unit{\isadigit{2}}{\isacharcomma}{\kern0pt}\ {\isacharminus}{\kern0pt}{\isacharparenright}{\kern0pt}\isanewline
\ \ \isacommand{qed}\isamarkupfalse%
\isanewline
\ \ \isacommand{then}\isamarkupfalse%
\ \isacommand{obtain}\isamarkupfalse%
\ s{\isacharprime}{\kern0pt}\ \isakeyword{where}\isanewline
\ \ \ \ s{\isacharprime}{\kern0pt}{\isacharunderscore}{\kern0pt}type{\isacharbrackleft}{\kern0pt}type{\isacharunderscore}{\kern0pt}rule{\isacharbrackright}{\kern0pt}{\isacharcolon}{\kern0pt}\ {\isachardoublequoteopen}s{\isacharprime}{\kern0pt}\ {\isacharcolon}{\kern0pt}\ P\ {\isasymrightarrow}\ P{\isachardoublequoteclose}\ \isakeyword{and}\isanewline
\ \ \ \ s{\isacharprime}{\kern0pt}{\isacharunderscore}{\kern0pt}def{\isacharcolon}{\kern0pt}\ {\isachardoublequoteopen}p{\isacharprime}{\kern0pt}\ {\isasymcirc}\isactrlsub c\ s{\isacharprime}{\kern0pt}\ {\isacharequal}{\kern0pt}\ successor\ {\isasymcirc}\isactrlsub c\ p{\isacharprime}{\kern0pt}{\isachardoublequoteclose}\isanewline
\ \ \ \ \isacommand{using}\isamarkupfalse%
\ p{\isacharprime}{\kern0pt}{\isacharunderscore}{\kern0pt}uni{\isacharunderscore}{\kern0pt}prop\ \isacommand{by}\isamarkupfalse%
\ {\isacharparenleft}{\kern0pt}typecheck{\isacharunderscore}{\kern0pt}cfuncs{\isacharcomma}{\kern0pt}\ metis{\isacharparenright}{\kern0pt}\isanewline
\isanewline
\ \ \isacommand{obtain}\isamarkupfalse%
\ u\ \isakeyword{where}\isanewline
\ \ \ \ u{\isacharunderscore}{\kern0pt}type{\isacharbrackleft}{\kern0pt}type{\isacharunderscore}{\kern0pt}rule{\isacharbrackright}{\kern0pt}{\isacharcolon}{\kern0pt}\ {\isachardoublequoteopen}u\ {\isacharcolon}{\kern0pt}\ {\isasymnat}\isactrlsub c\ {\isasymrightarrow}\ P{\isachardoublequoteclose}\ \isakeyword{and}\isanewline
\ \ \ \ u{\isacharunderscore}{\kern0pt}zero{\isacharcolon}{\kern0pt}\ {\isachardoublequoteopen}u\ {\isasymcirc}\isactrlsub c\ zero\ {\isacharequal}{\kern0pt}\ z{\isacharprime}{\kern0pt}{\isachardoublequoteclose}\ \isakeyword{and}\isanewline
\ \ \ \ u{\isacharunderscore}{\kern0pt}succ{\isacharcolon}{\kern0pt}\ {\isachardoublequoteopen}u\ {\isasymcirc}\isactrlsub c\ successor\ {\isacharequal}{\kern0pt}\ s{\isacharprime}{\kern0pt}\ {\isasymcirc}\isactrlsub c\ u{\isachardoublequoteclose}\isanewline
\ \ \ \ \isacommand{using}\isamarkupfalse%
\ natural{\isacharunderscore}{\kern0pt}number{\isacharunderscore}{\kern0pt}object{\isacharunderscore}{\kern0pt}property{\isadigit{2}}\ \isacommand{by}\isamarkupfalse%
\ {\isacharparenleft}{\kern0pt}typecheck{\isacharunderscore}{\kern0pt}cfuncs{\isacharcomma}{\kern0pt}\ metis\ s{\isacharprime}{\kern0pt}{\isacharunderscore}{\kern0pt}type{\isacharparenright}{\kern0pt}\isanewline
\isanewline
\ \ \isacommand{have}\isamarkupfalse%
\ p{\isacharprime}{\kern0pt}{\isacharunderscore}{\kern0pt}u{\isacharunderscore}{\kern0pt}is{\isacharunderscore}{\kern0pt}id{\isacharcolon}{\kern0pt}\ {\isachardoublequoteopen}p{\isacharprime}{\kern0pt}\ {\isasymcirc}\isactrlsub c\ u\ {\isacharequal}{\kern0pt}\ id\ {\isasymnat}\isactrlsub c{\isachardoublequoteclose}\isanewline
\ \ \isacommand{proof}\isamarkupfalse%
\ {\isacharparenleft}{\kern0pt}etcs{\isacharunderscore}{\kern0pt}rule\ natural{\isacharunderscore}{\kern0pt}number{\isacharunderscore}{\kern0pt}object{\isacharunderscore}{\kern0pt}func{\isacharunderscore}{\kern0pt}unique{\isacharbrackleft}{\kern0pt}\isakeyword{where}\ f{\isacharequal}{\kern0pt}successor{\isacharbrackright}{\kern0pt}{\isacharparenright}{\kern0pt}\isanewline
\ \ \ \ \isacommand{show}\isamarkupfalse%
\ {\isachardoublequoteopen}{\isacharparenleft}{\kern0pt}p{\isacharprime}{\kern0pt}\ {\isasymcirc}\isactrlsub c\ u{\isacharparenright}{\kern0pt}\ {\isasymcirc}\isactrlsub c\ zero\ {\isacharequal}{\kern0pt}\ id\isactrlsub c\ {\isasymnat}\isactrlsub c\ {\isasymcirc}\isactrlsub c\ zero{\isachardoublequoteclose}\isanewline
\ \ \ \ \ \ \isacommand{by}\isamarkupfalse%
\ {\isacharparenleft}{\kern0pt}etcs{\isacharunderscore}{\kern0pt}subst\ id{\isacharunderscore}{\kern0pt}left{\isacharunderscore}{\kern0pt}unit{\isadigit{2}}{\isacharcomma}{\kern0pt}\ etcs{\isacharunderscore}{\kern0pt}assocr{\isacharcomma}{\kern0pt}\ etcs{\isacharunderscore}{\kern0pt}subst\ u{\isacharunderscore}{\kern0pt}zero\ z{\isacharprime}{\kern0pt}{\isacharunderscore}{\kern0pt}def{\isacharcomma}{\kern0pt}\ simp{\isacharparenright}{\kern0pt}\isanewline
\ \ \ \ \isacommand{show}\isamarkupfalse%
\ {\isachardoublequoteopen}{\isacharparenleft}{\kern0pt}p{\isacharprime}{\kern0pt}\ {\isasymcirc}\isactrlsub c\ u{\isacharparenright}{\kern0pt}\ {\isasymcirc}\isactrlsub c\ successor\ {\isacharequal}{\kern0pt}\ successor\ {\isasymcirc}\isactrlsub c\ p{\isacharprime}{\kern0pt}\ {\isasymcirc}\isactrlsub c\ u{\isachardoublequoteclose}\isanewline
\ \ \ \ \ \ \isacommand{by}\isamarkupfalse%
\ {\isacharparenleft}{\kern0pt}etcs{\isacharunderscore}{\kern0pt}assocr{\isacharcomma}{\kern0pt}\ etcs{\isacharunderscore}{\kern0pt}subst\ u{\isacharunderscore}{\kern0pt}succ{\isacharcomma}{\kern0pt}\ etcs{\isacharunderscore}{\kern0pt}assocl{\isacharcomma}{\kern0pt}\ etcs{\isacharunderscore}{\kern0pt}subst\ s{\isacharprime}{\kern0pt}{\isacharunderscore}{\kern0pt}def{\isacharcomma}{\kern0pt}\ simp{\isacharparenright}{\kern0pt}\isanewline
\ \ \ \ \isacommand{show}\isamarkupfalse%
\ {\isachardoublequoteopen}id\isactrlsub c\ {\isasymnat}\isactrlsub c\ {\isasymcirc}\isactrlsub c\ successor\ {\isacharequal}{\kern0pt}\ successor\ {\isasymcirc}\isactrlsub c\ id\isactrlsub c\ {\isasymnat}\isactrlsub c{\isachardoublequoteclose}\isanewline
\ \ \ \ \ \ \isacommand{by}\isamarkupfalse%
\ {\isacharparenleft}{\kern0pt}etcs{\isacharunderscore}{\kern0pt}subst\ id{\isacharunderscore}{\kern0pt}right{\isacharunderscore}{\kern0pt}unit{\isadigit{2}}\ id{\isacharunderscore}{\kern0pt}left{\isacharunderscore}{\kern0pt}unit{\isadigit{2}}{\isacharcomma}{\kern0pt}\ simp{\isacharparenright}{\kern0pt}\isanewline
\ \ \isacommand{qed}\isamarkupfalse%
\isanewline
\isanewline
\ \ \isacommand{have}\isamarkupfalse%
\ {\isachardoublequoteopen}p\ {\isasymcirc}\isactrlsub c\ p{\isacharprime}{\kern0pt}\ {\isasymcirc}\isactrlsub c\ u\ {\isasymcirc}\isactrlsub c\ n\ {\isacharequal}{\kern0pt}\ {\isacharparenleft}{\kern0pt}{\isasymt}\ {\isasymcirc}\isactrlsub c\ {\isasymbeta}\isactrlbsub {\isasymnat}\isactrlsub c\isactrlesub {\isacharparenright}{\kern0pt}\ {\isasymcirc}\isactrlsub c\ p{\isacharprime}{\kern0pt}\ {\isasymcirc}\isactrlsub c\ u\ {\isasymcirc}\isactrlsub c\ n{\isachardoublequoteclose}\isanewline
\ \ \ \ \isacommand{by}\isamarkupfalse%
\ {\isacharparenleft}{\kern0pt}typecheck{\isacharunderscore}{\kern0pt}cfuncs{\isacharcomma}{\kern0pt}\ smt\ comp{\isacharunderscore}{\kern0pt}associative{\isadigit{2}}\ p{\isacharprime}{\kern0pt}{\isacharunderscore}{\kern0pt}equalizer{\isacharparenright}{\kern0pt}\isanewline
\ \ \isacommand{then}\isamarkupfalse%
\ \isacommand{show}\isamarkupfalse%
\ {\isachardoublequoteopen}p\ {\isasymcirc}\isactrlsub c\ n\ {\isacharequal}{\kern0pt}\ {\isasymt}{\isachardoublequoteclose}\isanewline
\ \ \ \ \isacommand{by}\isamarkupfalse%
\ {\isacharparenleft}{\kern0pt}typecheck{\isacharunderscore}{\kern0pt}cfuncs{\isacharcomma}{\kern0pt}\ smt\ {\isacharparenleft}{\kern0pt}z{\isadigit{3}}{\isacharparenright}{\kern0pt}\ comp{\isacharunderscore}{\kern0pt}associative{\isadigit{2}}\ id{\isacharunderscore}{\kern0pt}left{\isacharunderscore}{\kern0pt}unit{\isadigit{2}}\ id{\isacharunderscore}{\kern0pt}right{\isacharunderscore}{\kern0pt}unit{\isadigit{2}}\ p{\isacharprime}{\kern0pt}{\isacharunderscore}{\kern0pt}type\ p{\isacharprime}{\kern0pt}{\isacharunderscore}{\kern0pt}u{\isacharunderscore}{\kern0pt}is{\isacharunderscore}{\kern0pt}id\ terminal{\isacharunderscore}{\kern0pt}func{\isacharunderscore}{\kern0pt}comp{\isacharunderscore}{\kern0pt}elem\ terminal{\isacharunderscore}{\kern0pt}func{\isacharunderscore}{\kern0pt}type\ u{\isacharunderscore}{\kern0pt}type{\isacharparenright}{\kern0pt}\isanewline
\isacommand{qed}\isamarkupfalse%
%
\endisatagproof
{\isafoldproof}%
%
\isadelimproof
%
\endisadelimproof
%
\isadelimdocument
%
\endisadelimdocument
%
\isatagdocument
%
\isamarkupsubsection{Function Iteration%
}
\isamarkuptrue%
%
\endisatagdocument
{\isafolddocument}%
%
\isadelimdocument
%
\endisadelimdocument
\isacommand{definition}\isamarkupfalse%
\ ITER{\isacharunderscore}{\kern0pt}curried\ {\isacharcolon}{\kern0pt}{\isacharcolon}{\kern0pt}\ {\isachardoublequoteopen}cset\ {\isasymRightarrow}\ cfunc{\isachardoublequoteclose}\ \isakeyword{where}\ \isanewline
\ \ {\isachardoublequoteopen}ITER{\isacharunderscore}{\kern0pt}curried\ U\ {\isacharequal}{\kern0pt}\ {\isacharparenleft}{\kern0pt}THE\ u\ {\isachardot}{\kern0pt}\ u\ {\isacharcolon}{\kern0pt}\ {\isasymnat}\isactrlsub c\ {\isasymrightarrow}\ {\isacharparenleft}{\kern0pt}U\isactrlbsup U\isactrlesup {\isacharparenright}{\kern0pt}\isactrlbsup U\isactrlbsup U\isactrlesup \isactrlesup \ {\isasymand}\ \ u\ {\isasymcirc}\isactrlsub c\ zero\ {\isacharequal}{\kern0pt}\ {\isacharparenleft}{\kern0pt}metafunc\ {\isacharparenleft}{\kern0pt}id\ U{\isacharparenright}{\kern0pt}\ {\isasymcirc}\isactrlsub c\ {\isacharparenleft}{\kern0pt}right{\isacharunderscore}{\kern0pt}cart{\isacharunderscore}{\kern0pt}proj\ {\isacharparenleft}{\kern0pt}U\isactrlbsup U\isactrlesup {\isacharparenright}{\kern0pt}\ {\isasymone}{\isacharparenright}{\kern0pt}{\isacharparenright}{\kern0pt}\isactrlsup {\isasymsharp}\ {\isasymand}\isanewline
\ \ \ \ {\isacharparenleft}{\kern0pt}{\isacharparenleft}{\kern0pt}meta{\isacharunderscore}{\kern0pt}comp\ U\ U\ U{\isacharparenright}{\kern0pt}\ {\isasymcirc}\isactrlsub c\ {\isacharparenleft}{\kern0pt}id\ {\isacharparenleft}{\kern0pt}U\isactrlbsup U\isactrlesup {\isacharparenright}{\kern0pt}\ {\isasymtimes}\isactrlsub f\ eval{\isacharunderscore}{\kern0pt}func\ {\isacharparenleft}{\kern0pt}U\isactrlbsup U\isactrlesup {\isacharparenright}{\kern0pt}\ {\isacharparenleft}{\kern0pt}U\isactrlbsup U\isactrlesup {\isacharparenright}{\kern0pt}{\isacharparenright}{\kern0pt}\ {\isasymcirc}\isactrlsub c\ {\isacharparenleft}{\kern0pt}associate{\isacharunderscore}{\kern0pt}right\ {\isacharparenleft}{\kern0pt}U\isactrlbsup U\isactrlesup {\isacharparenright}{\kern0pt}\ {\isacharparenleft}{\kern0pt}U\isactrlbsup U\isactrlesup {\isacharparenright}{\kern0pt}\ {\isacharparenleft}{\kern0pt}{\isacharparenleft}{\kern0pt}U\isactrlbsup U\isactrlesup {\isacharparenright}{\kern0pt}\isactrlbsup U\isactrlbsup U\isactrlesup \isactrlesup {\isacharparenright}{\kern0pt}{\isacharparenright}{\kern0pt}\ {\isasymcirc}\isactrlsub c\ {\isacharparenleft}{\kern0pt}diagonal{\isacharparenleft}{\kern0pt}U\isactrlbsup U\isactrlesup {\isacharparenright}{\kern0pt}{\isasymtimes}\isactrlsub f\ id\ {\isacharparenleft}{\kern0pt}{\isacharparenleft}{\kern0pt}U\isactrlbsup U\isactrlesup {\isacharparenright}{\kern0pt}\isactrlbsup U\isactrlbsup U\isactrlesup \isactrlesup {\isacharparenright}{\kern0pt}{\isacharparenright}{\kern0pt}{\isacharparenright}{\kern0pt}\isactrlsup {\isasymsharp}\ \ \ \ {\isasymcirc}\isactrlsub c\ u\ {\isacharequal}{\kern0pt}\ u\ {\isasymcirc}\isactrlsub c\ successor{\isacharparenright}{\kern0pt}{\isachardoublequoteclose}\isanewline
\isanewline
\isacommand{lemma}\isamarkupfalse%
\ ITER{\isacharunderscore}{\kern0pt}curried{\isacharunderscore}{\kern0pt}def{\isadigit{2}}{\isacharcolon}{\kern0pt}\ \isanewline
{\isachardoublequoteopen}ITER{\isacharunderscore}{\kern0pt}curried\ U\ {\isacharcolon}{\kern0pt}\ {\isasymnat}\isactrlsub c\ {\isasymrightarrow}\ {\isacharparenleft}{\kern0pt}U\isactrlbsup U\isactrlesup {\isacharparenright}{\kern0pt}\isactrlbsup U\isactrlbsup U\isactrlesup \isactrlesup \ {\isasymand}\ \ ITER{\isacharunderscore}{\kern0pt}curried\ U\ {\isasymcirc}\isactrlsub c\ zero\ {\isacharequal}{\kern0pt}\ {\isacharparenleft}{\kern0pt}metafunc\ {\isacharparenleft}{\kern0pt}id\ U{\isacharparenright}{\kern0pt}\ {\isasymcirc}\isactrlsub c\ {\isacharparenleft}{\kern0pt}right{\isacharunderscore}{\kern0pt}cart{\isacharunderscore}{\kern0pt}proj\ {\isacharparenleft}{\kern0pt}U\isactrlbsup U\isactrlesup {\isacharparenright}{\kern0pt}\ {\isasymone}{\isacharparenright}{\kern0pt}{\isacharparenright}{\kern0pt}\isactrlsup {\isasymsharp}\ {\isasymand}\isanewline
\ \ {\isacharparenleft}{\kern0pt}{\isacharparenleft}{\kern0pt}meta{\isacharunderscore}{\kern0pt}comp\ U\ U\ U{\isacharparenright}{\kern0pt}\ {\isasymcirc}\isactrlsub c\ {\isacharparenleft}{\kern0pt}id\ {\isacharparenleft}{\kern0pt}U\isactrlbsup U\isactrlesup {\isacharparenright}{\kern0pt}\ {\isasymtimes}\isactrlsub f\ eval{\isacharunderscore}{\kern0pt}func\ {\isacharparenleft}{\kern0pt}U\isactrlbsup U\isactrlesup {\isacharparenright}{\kern0pt}\ {\isacharparenleft}{\kern0pt}U\isactrlbsup U\isactrlesup {\isacharparenright}{\kern0pt}{\isacharparenright}{\kern0pt}\ {\isasymcirc}\isactrlsub c\ {\isacharparenleft}{\kern0pt}associate{\isacharunderscore}{\kern0pt}right\ {\isacharparenleft}{\kern0pt}U\isactrlbsup U\isactrlesup {\isacharparenright}{\kern0pt}\ {\isacharparenleft}{\kern0pt}U\isactrlbsup U\isactrlesup {\isacharparenright}{\kern0pt}\ {\isacharparenleft}{\kern0pt}{\isacharparenleft}{\kern0pt}U\isactrlbsup U\isactrlesup {\isacharparenright}{\kern0pt}\isactrlbsup U\isactrlbsup U\isactrlesup \isactrlesup {\isacharparenright}{\kern0pt}{\isacharparenright}{\kern0pt}\ {\isasymcirc}\isactrlsub c\ {\isacharparenleft}{\kern0pt}diagonal{\isacharparenleft}{\kern0pt}U\isactrlbsup U\isactrlesup {\isacharparenright}{\kern0pt}{\isasymtimes}\isactrlsub f\ id\ {\isacharparenleft}{\kern0pt}{\isacharparenleft}{\kern0pt}U\isactrlbsup U\isactrlesup {\isacharparenright}{\kern0pt}\isactrlbsup U\isactrlbsup U\isactrlesup \isactrlesup {\isacharparenright}{\kern0pt}{\isacharparenright}{\kern0pt}{\isacharparenright}{\kern0pt}\isactrlsup {\isasymsharp}\ \ \ \ {\isasymcirc}\isactrlsub c\ ITER{\isacharunderscore}{\kern0pt}curried\ U\ {\isacharequal}{\kern0pt}\ ITER{\isacharunderscore}{\kern0pt}curried\ U\ \ {\isasymcirc}\isactrlsub c\ successor{\isachardoublequoteclose}\isanewline
%
\isadelimproof
\ \ %
\endisadelimproof
%
\isatagproof
\isacommand{unfolding}\isamarkupfalse%
\ ITER{\isacharunderscore}{\kern0pt}curried{\isacharunderscore}{\kern0pt}def\isanewline
\ \ \isacommand{by}\isamarkupfalse%
{\isacharparenleft}{\kern0pt}rule\ theI{\isacharprime}{\kern0pt}{\isacharcomma}{\kern0pt}\ etcs{\isacharunderscore}{\kern0pt}rule\ natural{\isacharunderscore}{\kern0pt}number{\isacharunderscore}{\kern0pt}object{\isacharunderscore}{\kern0pt}property{\isadigit{2}}{\isacharparenright}{\kern0pt}%
\endisatagproof
{\isafoldproof}%
%
\isadelimproof
\isanewline
%
\endisadelimproof
\ \ \isanewline
\isacommand{lemma}\isamarkupfalse%
\ ITER{\isacharunderscore}{\kern0pt}curried{\isacharunderscore}{\kern0pt}type{\isacharbrackleft}{\kern0pt}type{\isacharunderscore}{\kern0pt}rule{\isacharbrackright}{\kern0pt}{\isacharcolon}{\kern0pt}\isanewline
\ \ {\isachardoublequoteopen}ITER{\isacharunderscore}{\kern0pt}curried\ U\ {\isacharcolon}{\kern0pt}\ {\isasymnat}\isactrlsub c\ {\isasymrightarrow}\ {\isacharparenleft}{\kern0pt}U\isactrlbsup U\isactrlesup {\isacharparenright}{\kern0pt}\isactrlbsup U\isactrlbsup U\isactrlesup \isactrlesup {\isachardoublequoteclose}\isanewline
%
\isadelimproof
\ \ %
\endisadelimproof
%
\isatagproof
\isacommand{by}\isamarkupfalse%
\ {\isacharparenleft}{\kern0pt}simp\ add{\isacharcolon}{\kern0pt}\ ITER{\isacharunderscore}{\kern0pt}curried{\isacharunderscore}{\kern0pt}def{\isadigit{2}}{\isacharparenright}{\kern0pt}%
\endisatagproof
{\isafoldproof}%
%
\isadelimproof
\isanewline
%
\endisadelimproof
\isanewline
\isacommand{lemma}\isamarkupfalse%
\ ITER{\isacharunderscore}{\kern0pt}curried{\isacharunderscore}{\kern0pt}zero{\isacharcolon}{\kern0pt}\ \isanewline
\ \ {\isachardoublequoteopen}ITER{\isacharunderscore}{\kern0pt}curried\ U\ {\isasymcirc}\isactrlsub c\ zero\ {\isacharequal}{\kern0pt}\ {\isacharparenleft}{\kern0pt}metafunc\ {\isacharparenleft}{\kern0pt}id\ U{\isacharparenright}{\kern0pt}\ {\isasymcirc}\isactrlsub c\ {\isacharparenleft}{\kern0pt}right{\isacharunderscore}{\kern0pt}cart{\isacharunderscore}{\kern0pt}proj\ {\isacharparenleft}{\kern0pt}U\isactrlbsup U\isactrlesup {\isacharparenright}{\kern0pt}\ {\isasymone}{\isacharparenright}{\kern0pt}{\isacharparenright}{\kern0pt}\isactrlsup {\isasymsharp}{\isachardoublequoteclose}\isanewline
%
\isadelimproof
\ \ %
\endisadelimproof
%
\isatagproof
\isacommand{by}\isamarkupfalse%
\ {\isacharparenleft}{\kern0pt}simp\ add{\isacharcolon}{\kern0pt}\ ITER{\isacharunderscore}{\kern0pt}curried{\isacharunderscore}{\kern0pt}def{\isadigit{2}}{\isacharparenright}{\kern0pt}%
\endisatagproof
{\isafoldproof}%
%
\isadelimproof
\isanewline
%
\endisadelimproof
\isanewline
\isacommand{lemma}\isamarkupfalse%
\ ITER{\isacharunderscore}{\kern0pt}curried{\isacharunderscore}{\kern0pt}successor{\isacharcolon}{\kern0pt}\isanewline
{\isachardoublequoteopen}ITER{\isacharunderscore}{\kern0pt}curried\ U\ {\isasymcirc}\isactrlsub c\ successor\ {\isacharequal}{\kern0pt}\ {\isacharparenleft}{\kern0pt}meta{\isacharunderscore}{\kern0pt}comp\ U\ U\ U\ {\isasymcirc}\isactrlsub c\ {\isacharparenleft}{\kern0pt}id\ {\isacharparenleft}{\kern0pt}U\isactrlbsup U\isactrlesup {\isacharparenright}{\kern0pt}\ {\isasymtimes}\isactrlsub f\ eval{\isacharunderscore}{\kern0pt}func\ {\isacharparenleft}{\kern0pt}U\isactrlbsup U\isactrlesup {\isacharparenright}{\kern0pt}\ {\isacharparenleft}{\kern0pt}U\isactrlbsup U\isactrlesup {\isacharparenright}{\kern0pt}{\isacharparenright}{\kern0pt}\ {\isasymcirc}\isactrlsub c\ {\isacharparenleft}{\kern0pt}associate{\isacharunderscore}{\kern0pt}right\ {\isacharparenleft}{\kern0pt}U\isactrlbsup U\isactrlesup {\isacharparenright}{\kern0pt}\ {\isacharparenleft}{\kern0pt}U\isactrlbsup U\isactrlesup {\isacharparenright}{\kern0pt}\ {\isacharparenleft}{\kern0pt}{\isacharparenleft}{\kern0pt}U\isactrlbsup U\isactrlesup {\isacharparenright}{\kern0pt}\isactrlbsup U\isactrlbsup U\isactrlesup \isactrlesup {\isacharparenright}{\kern0pt}{\isacharparenright}{\kern0pt}\ {\isasymcirc}\isactrlsub c\ {\isacharparenleft}{\kern0pt}diagonal{\isacharparenleft}{\kern0pt}U\isactrlbsup U\isactrlesup {\isacharparenright}{\kern0pt}{\isasymtimes}\isactrlsub f\ id\ {\isacharparenleft}{\kern0pt}{\isacharparenleft}{\kern0pt}U\isactrlbsup U\isactrlesup {\isacharparenright}{\kern0pt}\isactrlbsup U\isactrlbsup U\isactrlesup \isactrlesup {\isacharparenright}{\kern0pt}{\isacharparenright}{\kern0pt}{\isacharparenright}{\kern0pt}\isactrlsup {\isasymsharp}\ \ \ \ {\isasymcirc}\isactrlsub c\ ITER{\isacharunderscore}{\kern0pt}curried\ U{\isachardoublequoteclose}\isanewline
%
\isadelimproof
\ \ %
\endisadelimproof
%
\isatagproof
\isacommand{using}\isamarkupfalse%
\ ITER{\isacharunderscore}{\kern0pt}curried{\isacharunderscore}{\kern0pt}def{\isadigit{2}}\ \isacommand{by}\isamarkupfalse%
\ simp%
\endisatagproof
{\isafoldproof}%
%
\isadelimproof
\ \isanewline
%
\endisadelimproof
\isanewline
\isacommand{definition}\isamarkupfalse%
\ ITER\ {\isacharcolon}{\kern0pt}{\isacharcolon}{\kern0pt}\ {\isachardoublequoteopen}cset\ {\isasymRightarrow}\ cfunc{\isachardoublequoteclose}\ \isakeyword{where}\ \isanewline
\ \ {\isachardoublequoteopen}ITER\ U\ {\isacharequal}{\kern0pt}\ {\isacharparenleft}{\kern0pt}ITER{\isacharunderscore}{\kern0pt}curried\ U{\isacharparenright}{\kern0pt}\isactrlsup {\isasymflat}{\isachardoublequoteclose}\isanewline
\isanewline
\isacommand{lemma}\isamarkupfalse%
\ ITER{\isacharunderscore}{\kern0pt}type{\isacharbrackleft}{\kern0pt}type{\isacharunderscore}{\kern0pt}rule{\isacharbrackright}{\kern0pt}{\isacharcolon}{\kern0pt}\isanewline
\ \ {\isachardoublequoteopen}ITER\ U\ {\isacharcolon}{\kern0pt}\ {\isacharparenleft}{\kern0pt}{\isacharparenleft}{\kern0pt}U\isactrlbsup U\isactrlesup {\isacharparenright}{\kern0pt}\ {\isasymtimes}\isactrlsub c\ {\isasymnat}\isactrlsub c{\isacharparenright}{\kern0pt}\ {\isasymrightarrow}\ {\isacharparenleft}{\kern0pt}U\isactrlbsup U\isactrlesup {\isacharparenright}{\kern0pt}{\isachardoublequoteclose}\isanewline
%
\isadelimproof
\ \ %
\endisadelimproof
%
\isatagproof
\isacommand{unfolding}\isamarkupfalse%
\ ITER{\isacharunderscore}{\kern0pt}def\ \isacommand{by}\isamarkupfalse%
\ typecheck{\isacharunderscore}{\kern0pt}cfuncs%
\endisatagproof
{\isafoldproof}%
%
\isadelimproof
\isanewline
%
\endisadelimproof
\isanewline
\isacommand{lemma}\isamarkupfalse%
\ ITER{\isacharunderscore}{\kern0pt}zero{\isacharcolon}{\kern0pt}\ \isanewline
\ \ \isakeyword{assumes}\ f{\isacharunderscore}{\kern0pt}type{\isacharbrackleft}{\kern0pt}type{\isacharunderscore}{\kern0pt}rule{\isacharbrackright}{\kern0pt}{\isacharcolon}{\kern0pt}\ {\isachardoublequoteopen}f\ {\isacharcolon}{\kern0pt}\ Z\ {\isasymrightarrow}\ {\isacharparenleft}{\kern0pt}U\isactrlbsup U\isactrlesup {\isacharparenright}{\kern0pt}{\isachardoublequoteclose}\isanewline
\ \ \isakeyword{shows}\ {\isachardoublequoteopen}ITER\ U\ {\isasymcirc}\isactrlsub c\ {\isasymlangle}f{\isacharcomma}{\kern0pt}\ zero\ {\isasymcirc}\isactrlsub c\ {\isasymbeta}\isactrlbsub Z\isactrlesub {\isasymrangle}\ {\isacharequal}{\kern0pt}\ metafunc\ {\isacharparenleft}{\kern0pt}id\ U{\isacharparenright}{\kern0pt}\ {\isasymcirc}\isactrlsub c\ {\isasymbeta}\isactrlbsub Z\isactrlesub {\isachardoublequoteclose}\isanewline
%
\isadelimproof
%
\endisadelimproof
%
\isatagproof
\isacommand{proof}\isamarkupfalse%
{\isacharparenleft}{\kern0pt}etcs{\isacharunderscore}{\kern0pt}rule\ one{\isacharunderscore}{\kern0pt}separator{\isacharparenright}{\kern0pt}\isanewline
\ \ \isacommand{fix}\isamarkupfalse%
\ z\ \isanewline
\ \ \isacommand{assume}\isamarkupfalse%
\ z{\isacharunderscore}{\kern0pt}type{\isacharbrackleft}{\kern0pt}type{\isacharunderscore}{\kern0pt}rule{\isacharbrackright}{\kern0pt}{\isacharcolon}{\kern0pt}\ {\isachardoublequoteopen}z\ {\isasymin}\isactrlsub c\ Z{\isachardoublequoteclose}\isanewline
\ \ \isacommand{have}\isamarkupfalse%
\ {\isachardoublequoteopen}{\isacharparenleft}{\kern0pt}ITER\ U\ {\isasymcirc}\isactrlsub c\ {\isasymlangle}f{\isacharcomma}{\kern0pt}zero\ {\isasymcirc}\isactrlsub c\ {\isasymbeta}\isactrlbsub Z\isactrlesub {\isasymrangle}{\isacharparenright}{\kern0pt}\ {\isasymcirc}\isactrlsub c\ z\ {\isacharequal}{\kern0pt}\ ITER\ U\ {\isasymcirc}\isactrlsub c\ {\isasymlangle}f{\isacharcomma}{\kern0pt}zero\ {\isasymcirc}\isactrlsub c\ {\isasymbeta}\isactrlbsub Z\isactrlesub {\isasymrangle}\ {\isasymcirc}\isactrlsub c\ z{\isachardoublequoteclose}\isanewline
\ \ \ \ \isacommand{using}\isamarkupfalse%
\ assms\ \isacommand{by}\isamarkupfalse%
\ {\isacharparenleft}{\kern0pt}typecheck{\isacharunderscore}{\kern0pt}cfuncs{\isacharcomma}{\kern0pt}\ simp\ add{\isacharcolon}{\kern0pt}\ comp{\isacharunderscore}{\kern0pt}associative{\isadigit{2}}{\isacharparenright}{\kern0pt}\isanewline
\ \ \isacommand{also}\isamarkupfalse%
\ \isacommand{have}\isamarkupfalse%
\ {\isachardoublequoteopen}{\isachardot}{\kern0pt}{\isachardot}{\kern0pt}{\isachardot}{\kern0pt}\ {\isacharequal}{\kern0pt}\ ITER\ U\ {\isasymcirc}\isactrlsub c\ {\isasymlangle}f\ {\isasymcirc}\isactrlsub c\ z{\isacharcomma}{\kern0pt}zero{\isasymrangle}{\isachardoublequoteclose}\isanewline
\ \ \ \ \isacommand{using}\isamarkupfalse%
\ assms\ \isacommand{by}\isamarkupfalse%
\ {\isacharparenleft}{\kern0pt}typecheck{\isacharunderscore}{\kern0pt}cfuncs{\isacharcomma}{\kern0pt}\ smt\ {\isacharparenleft}{\kern0pt}z{\isadigit{3}}{\isacharparenright}{\kern0pt}\ cfunc{\isacharunderscore}{\kern0pt}prod{\isacharunderscore}{\kern0pt}comp\ comp{\isacharunderscore}{\kern0pt}associative{\isadigit{2}}\ id{\isacharunderscore}{\kern0pt}right{\isacharunderscore}{\kern0pt}unit{\isadigit{2}}\ terminal{\isacharunderscore}{\kern0pt}func{\isacharunderscore}{\kern0pt}comp{\isacharunderscore}{\kern0pt}elem{\isacharparenright}{\kern0pt}\isanewline
\ \ \isacommand{also}\isamarkupfalse%
\ \isacommand{have}\isamarkupfalse%
\ {\isachardoublequoteopen}{\isachardot}{\kern0pt}{\isachardot}{\kern0pt}{\isachardot}{\kern0pt}\ {\isacharequal}{\kern0pt}\ {\isacharparenleft}{\kern0pt}eval{\isacharunderscore}{\kern0pt}func\ {\isacharparenleft}{\kern0pt}U\isactrlbsup U\isactrlesup {\isacharparenright}{\kern0pt}\ {\isacharparenleft}{\kern0pt}U\isactrlbsup U\isactrlesup {\isacharparenright}{\kern0pt}{\isacharparenright}{\kern0pt}\ {\isasymcirc}\isactrlsub c\ {\isacharparenleft}{\kern0pt}id\isactrlsub c\ {\isacharparenleft}{\kern0pt}U\isactrlbsup U\isactrlesup {\isacharparenright}{\kern0pt}\ {\isasymtimes}\isactrlsub f\ ITER{\isacharunderscore}{\kern0pt}curried\ U{\isacharparenright}{\kern0pt}\ {\isasymcirc}\isactrlsub c\ {\isasymlangle}f\ {\isasymcirc}\isactrlsub c\ z{\isacharcomma}{\kern0pt}zero{\isasymrangle}{\isachardoublequoteclose}\isanewline
\ \ \ \ \isacommand{using}\isamarkupfalse%
\ assms\ ITER{\isacharunderscore}{\kern0pt}def\ comp{\isacharunderscore}{\kern0pt}associative{\isadigit{2}}\ inv{\isacharunderscore}{\kern0pt}transpose{\isacharunderscore}{\kern0pt}func{\isacharunderscore}{\kern0pt}def{\isadigit{3}}\ \isacommand{by}\isamarkupfalse%
\ {\isacharparenleft}{\kern0pt}typecheck{\isacharunderscore}{\kern0pt}cfuncs{\isacharcomma}{\kern0pt}\ auto{\isacharparenright}{\kern0pt}\isanewline
\ \ \isacommand{also}\isamarkupfalse%
\ \isacommand{have}\isamarkupfalse%
\ {\isachardoublequoteopen}{\isachardot}{\kern0pt}{\isachardot}{\kern0pt}{\isachardot}{\kern0pt}\ {\isacharequal}{\kern0pt}\ {\isacharparenleft}{\kern0pt}eval{\isacharunderscore}{\kern0pt}func\ {\isacharparenleft}{\kern0pt}U\isactrlbsup U\isactrlesup {\isacharparenright}{\kern0pt}\ {\isacharparenleft}{\kern0pt}U\isactrlbsup U\isactrlesup {\isacharparenright}{\kern0pt}{\isacharparenright}{\kern0pt}\ {\isasymcirc}\isactrlsub c\ {\isasymlangle}f\ {\isasymcirc}\isactrlsub c\ z{\isacharcomma}{\kern0pt}ITER{\isacharunderscore}{\kern0pt}curried\ U\ {\isasymcirc}\isactrlsub c\ zero{\isasymrangle}{\isachardoublequoteclose}\isanewline
\ \ \ \ \isacommand{using}\isamarkupfalse%
\ assms\ \isacommand{by}\isamarkupfalse%
\ {\isacharparenleft}{\kern0pt}typecheck{\isacharunderscore}{\kern0pt}cfuncs{\isacharcomma}{\kern0pt}\ simp\ add{\isacharcolon}{\kern0pt}\ cfunc{\isacharunderscore}{\kern0pt}cross{\isacharunderscore}{\kern0pt}prod{\isacharunderscore}{\kern0pt}comp{\isacharunderscore}{\kern0pt}cfunc{\isacharunderscore}{\kern0pt}prod\ id{\isacharunderscore}{\kern0pt}left{\isacharunderscore}{\kern0pt}unit{\isadigit{2}}{\isacharparenright}{\kern0pt}\isanewline
\ \ \isacommand{also}\isamarkupfalse%
\ \isacommand{have}\isamarkupfalse%
\ {\isachardoublequoteopen}{\isachardot}{\kern0pt}{\isachardot}{\kern0pt}{\isachardot}{\kern0pt}\ {\isacharequal}{\kern0pt}\ {\isacharparenleft}{\kern0pt}eval{\isacharunderscore}{\kern0pt}func\ {\isacharparenleft}{\kern0pt}U\isactrlbsup U\isactrlesup {\isacharparenright}{\kern0pt}\ {\isacharparenleft}{\kern0pt}U\isactrlbsup U\isactrlesup {\isacharparenright}{\kern0pt}{\isacharparenright}{\kern0pt}\ {\isasymcirc}\isactrlsub c\ {\isasymlangle}f\ {\isasymcirc}\isactrlsub c\ z{\isacharcomma}{\kern0pt}{\isacharparenleft}{\kern0pt}metafunc\ {\isacharparenleft}{\kern0pt}id\ U{\isacharparenright}{\kern0pt}\ {\isasymcirc}\isactrlsub c\ {\isacharparenleft}{\kern0pt}right{\isacharunderscore}{\kern0pt}cart{\isacharunderscore}{\kern0pt}proj\ {\isacharparenleft}{\kern0pt}U\isactrlbsup U\isactrlesup {\isacharparenright}{\kern0pt}\ {\isasymone}{\isacharparenright}{\kern0pt}{\isacharparenright}{\kern0pt}\isactrlsup {\isasymsharp}{\isasymrangle}{\isachardoublequoteclose}\isanewline
\ \ \ \ \isacommand{using}\isamarkupfalse%
\ assms\ \isacommand{by}\isamarkupfalse%
\ {\isacharparenleft}{\kern0pt}simp\ add{\isacharcolon}{\kern0pt}\ ITER{\isacharunderscore}{\kern0pt}curried{\isacharunderscore}{\kern0pt}def{\isadigit{2}}{\isacharparenright}{\kern0pt}\ \ \ \isanewline
\ \ \isacommand{also}\isamarkupfalse%
\ \isacommand{have}\isamarkupfalse%
\ {\isachardoublequoteopen}{\isachardot}{\kern0pt}{\isachardot}{\kern0pt}{\isachardot}{\kern0pt}\ {\isacharequal}{\kern0pt}\ {\isacharparenleft}{\kern0pt}eval{\isacharunderscore}{\kern0pt}func\ {\isacharparenleft}{\kern0pt}U\isactrlbsup U\isactrlesup {\isacharparenright}{\kern0pt}\ {\isacharparenleft}{\kern0pt}U\isactrlbsup U\isactrlesup {\isacharparenright}{\kern0pt}{\isacharparenright}{\kern0pt}\ {\isasymcirc}\isactrlsub c\ {\isasymlangle}f\ {\isasymcirc}\isactrlsub c\ z{\isacharcomma}{\kern0pt}{\isacharparenleft}{\kern0pt}{\isacharparenleft}{\kern0pt}left{\isacharunderscore}{\kern0pt}cart{\isacharunderscore}{\kern0pt}proj\ {\isacharparenleft}{\kern0pt}U{\isacharparenright}{\kern0pt}\ {\isasymone}{\isacharparenright}{\kern0pt}\isactrlsup {\isasymsharp}\ {\isasymcirc}\isactrlsub c\ {\isacharparenleft}{\kern0pt}right{\isacharunderscore}{\kern0pt}cart{\isacharunderscore}{\kern0pt}proj\ {\isacharparenleft}{\kern0pt}U\isactrlbsup U\isactrlesup {\isacharparenright}{\kern0pt}\ {\isasymone}{\isacharparenright}{\kern0pt}{\isacharparenright}{\kern0pt}\isactrlsup {\isasymsharp}{\isasymrangle}{\isachardoublequoteclose}\isanewline
\ \ \ \ \isacommand{using}\isamarkupfalse%
\ assms\ \isacommand{by}\isamarkupfalse%
\ {\isacharparenleft}{\kern0pt}typecheck{\isacharunderscore}{\kern0pt}cfuncs{\isacharcomma}{\kern0pt}\ simp\ add{\isacharcolon}{\kern0pt}\ id{\isacharunderscore}{\kern0pt}left{\isacharunderscore}{\kern0pt}unit{\isadigit{2}}\ metafunc{\isacharunderscore}{\kern0pt}def{\isadigit{2}}{\isacharparenright}{\kern0pt}\isanewline
\ \ \isacommand{also}\isamarkupfalse%
\ \isacommand{have}\isamarkupfalse%
\ {\isachardoublequoteopen}{\isachardot}{\kern0pt}{\isachardot}{\kern0pt}{\isachardot}{\kern0pt}\ {\isacharequal}{\kern0pt}\ {\isacharparenleft}{\kern0pt}eval{\isacharunderscore}{\kern0pt}func\ {\isacharparenleft}{\kern0pt}U\isactrlbsup U\isactrlesup {\isacharparenright}{\kern0pt}\ {\isacharparenleft}{\kern0pt}U\isactrlbsup U\isactrlesup {\isacharparenright}{\kern0pt}{\isacharparenright}{\kern0pt}\ {\isasymcirc}\isactrlsub c\ {\isacharparenleft}{\kern0pt}id\isactrlsub c\ {\isacharparenleft}{\kern0pt}U\isactrlbsup U\isactrlesup {\isacharparenright}{\kern0pt}\ {\isasymtimes}\isactrlsub f\ \ {\isacharparenleft}{\kern0pt}{\isacharparenleft}{\kern0pt}left{\isacharunderscore}{\kern0pt}cart{\isacharunderscore}{\kern0pt}proj\ {\isacharparenleft}{\kern0pt}U{\isacharparenright}{\kern0pt}\ {\isasymone}{\isacharparenright}{\kern0pt}\isactrlsup {\isasymsharp}\ {\isasymcirc}\isactrlsub c\ {\isacharparenleft}{\kern0pt}right{\isacharunderscore}{\kern0pt}cart{\isacharunderscore}{\kern0pt}proj\ {\isacharparenleft}{\kern0pt}U\isactrlbsup U\isactrlesup {\isacharparenright}{\kern0pt}\ {\isasymone}{\isacharparenright}{\kern0pt}{\isacharparenright}{\kern0pt}\isactrlsup {\isasymsharp}{\isacharparenright}{\kern0pt}\ {\isasymcirc}\isactrlsub c\ {\isasymlangle}f\ \ {\isasymcirc}\isactrlsub c\ z{\isacharcomma}{\kern0pt}id\isactrlsub c\ {\isasymone}{\isasymrangle}{\isachardoublequoteclose}\isanewline
\ \ \ \ \isacommand{using}\isamarkupfalse%
\ assms\ \isacommand{by}\isamarkupfalse%
\ {\isacharparenleft}{\kern0pt}typecheck{\isacharunderscore}{\kern0pt}cfuncs{\isacharcomma}{\kern0pt}\ simp\ add{\isacharcolon}{\kern0pt}\ cfunc{\isacharunderscore}{\kern0pt}cross{\isacharunderscore}{\kern0pt}prod{\isacharunderscore}{\kern0pt}comp{\isacharunderscore}{\kern0pt}cfunc{\isacharunderscore}{\kern0pt}prod\ id{\isacharunderscore}{\kern0pt}left{\isacharunderscore}{\kern0pt}unit{\isadigit{2}}\ id{\isacharunderscore}{\kern0pt}right{\isacharunderscore}{\kern0pt}unit{\isadigit{2}}{\isacharparenright}{\kern0pt}\isanewline
\ \ \isacommand{also}\isamarkupfalse%
\ \isacommand{have}\isamarkupfalse%
\ {\isachardoublequoteopen}{\isachardot}{\kern0pt}{\isachardot}{\kern0pt}{\isachardot}{\kern0pt}\ {\isacharequal}{\kern0pt}\ {\isacharparenleft}{\kern0pt}left{\isacharunderscore}{\kern0pt}cart{\isacharunderscore}{\kern0pt}proj\ {\isacharparenleft}{\kern0pt}U{\isacharparenright}{\kern0pt}\ {\isasymone}{\isacharparenright}{\kern0pt}\isactrlsup {\isasymsharp}\ {\isasymcirc}\isactrlsub c\ {\isacharparenleft}{\kern0pt}right{\isacharunderscore}{\kern0pt}cart{\isacharunderscore}{\kern0pt}proj\ {\isacharparenleft}{\kern0pt}U\isactrlbsup U\isactrlesup {\isacharparenright}{\kern0pt}\ {\isasymone}{\isacharparenright}{\kern0pt}\ \ {\isasymcirc}\isactrlsub c\ {\isasymlangle}f\ \ {\isasymcirc}\isactrlsub c\ z{\isacharcomma}{\kern0pt}id\isactrlsub c\ {\isasymone}{\isasymrangle}{\isachardoublequoteclose}\isanewline
\ \ \ \ \isacommand{using}\isamarkupfalse%
\ assms\ \isacommand{by}\isamarkupfalse%
\ {\isacharparenleft}{\kern0pt}typecheck{\isacharunderscore}{\kern0pt}cfuncs{\isacharcomma}{\kern0pt}simp\ add{\isacharcolon}{\kern0pt}\ cfunc{\isacharunderscore}{\kern0pt}type{\isacharunderscore}{\kern0pt}def\ comp{\isacharunderscore}{\kern0pt}associative\ transpose{\isacharunderscore}{\kern0pt}func{\isacharunderscore}{\kern0pt}def{\isacharparenright}{\kern0pt}\isanewline
\ \ \isacommand{also}\isamarkupfalse%
\ \isacommand{have}\isamarkupfalse%
\ {\isachardoublequoteopen}{\isachardot}{\kern0pt}{\isachardot}{\kern0pt}{\isachardot}{\kern0pt}\ {\isacharequal}{\kern0pt}\ {\isacharparenleft}{\kern0pt}left{\isacharunderscore}{\kern0pt}cart{\isacharunderscore}{\kern0pt}proj\ {\isacharparenleft}{\kern0pt}U{\isacharparenright}{\kern0pt}\ {\isasymone}{\isacharparenright}{\kern0pt}\isactrlsup {\isasymsharp}{\isachardoublequoteclose}\isanewline
\ \ \ \ \isacommand{using}\isamarkupfalse%
\ assms\ \isacommand{by}\isamarkupfalse%
\ {\isacharparenleft}{\kern0pt}typecheck{\isacharunderscore}{\kern0pt}cfuncs{\isacharcomma}{\kern0pt}\ simp\ add{\isacharcolon}{\kern0pt}\ id{\isacharunderscore}{\kern0pt}right{\isacharunderscore}{\kern0pt}unit{\isadigit{2}}\ right{\isacharunderscore}{\kern0pt}cart{\isacharunderscore}{\kern0pt}proj{\isacharunderscore}{\kern0pt}cfunc{\isacharunderscore}{\kern0pt}prod{\isacharparenright}{\kern0pt}\isanewline
\ \ \isacommand{also}\isamarkupfalse%
\ \isacommand{have}\isamarkupfalse%
\ {\isachardoublequoteopen}{\isachardot}{\kern0pt}{\isachardot}{\kern0pt}{\isachardot}{\kern0pt}\ {\isacharequal}{\kern0pt}\ {\isacharparenleft}{\kern0pt}metafunc\ {\isacharparenleft}{\kern0pt}id\isactrlsub c\ U{\isacharparenright}{\kern0pt}{\isacharparenright}{\kern0pt}{\isachardoublequoteclose}\isanewline
\ \ \ \ \isacommand{using}\isamarkupfalse%
\ assms\ \isacommand{by}\isamarkupfalse%
\ {\isacharparenleft}{\kern0pt}typecheck{\isacharunderscore}{\kern0pt}cfuncs{\isacharcomma}{\kern0pt}\ simp\ add{\isacharcolon}{\kern0pt}\ id{\isacharunderscore}{\kern0pt}left{\isacharunderscore}{\kern0pt}unit{\isadigit{2}}\ metafunc{\isacharunderscore}{\kern0pt}def{\isadigit{2}}{\isacharparenright}{\kern0pt}\isanewline
\ \ \isacommand{also}\isamarkupfalse%
\ \isacommand{have}\isamarkupfalse%
\ {\isachardoublequoteopen}{\isachardot}{\kern0pt}{\isachardot}{\kern0pt}{\isachardot}{\kern0pt}\ {\isacharequal}{\kern0pt}\ {\isacharparenleft}{\kern0pt}metafunc\ {\isacharparenleft}{\kern0pt}id\isactrlsub c\ U{\isacharparenright}{\kern0pt}\ {\isasymcirc}\isactrlsub c\ {\isasymbeta}\isactrlbsub Z\isactrlesub {\isacharparenright}{\kern0pt}\ {\isasymcirc}\isactrlsub c\ z{\isachardoublequoteclose}\isanewline
\ \ \ \ \isacommand{using}\isamarkupfalse%
\ assms\ \isacommand{by}\isamarkupfalse%
\ {\isacharparenleft}{\kern0pt}typecheck{\isacharunderscore}{\kern0pt}cfuncs{\isacharcomma}{\kern0pt}\ metis\ cfunc{\isacharunderscore}{\kern0pt}type{\isacharunderscore}{\kern0pt}def\ comp{\isacharunderscore}{\kern0pt}associative\ id{\isacharunderscore}{\kern0pt}right{\isacharunderscore}{\kern0pt}unit{\isadigit{2}}\ terminal{\isacharunderscore}{\kern0pt}func{\isacharunderscore}{\kern0pt}comp{\isacharunderscore}{\kern0pt}elem{\isacharparenright}{\kern0pt}\isanewline
\ \ \isacommand{then}\isamarkupfalse%
\ \isacommand{show}\isamarkupfalse%
\ {\isachardoublequoteopen}{\isacharparenleft}{\kern0pt}ITER\ U\ {\isasymcirc}\isactrlsub c\ {\isasymlangle}f{\isacharcomma}{\kern0pt}zero\ {\isasymcirc}\isactrlsub c\ {\isasymbeta}\isactrlbsub Z\isactrlesub {\isasymrangle}{\isacharparenright}{\kern0pt}\ {\isasymcirc}\isactrlsub c\ z\ {\isacharequal}{\kern0pt}\ {\isacharparenleft}{\kern0pt}metafunc\ {\isacharparenleft}{\kern0pt}id\isactrlsub c\ U{\isacharparenright}{\kern0pt}\ {\isasymcirc}\isactrlsub c\ {\isasymbeta}\isactrlbsub Z\isactrlesub {\isacharparenright}{\kern0pt}\ {\isasymcirc}\isactrlsub c\ z{\isachardoublequoteclose}\isanewline
\ \ \ \ \isacommand{using}\isamarkupfalse%
\ calculation\ \isacommand{by}\isamarkupfalse%
\ auto\isanewline
\isacommand{qed}\isamarkupfalse%
%
\endisatagproof
{\isafoldproof}%
%
\isadelimproof
\isanewline
%
\endisadelimproof
\isanewline
\isacommand{lemma}\isamarkupfalse%
\ ITER{\isacharunderscore}{\kern0pt}zero{\isacharprime}{\kern0pt}{\isacharcolon}{\kern0pt}\ \isanewline
\ \ \isakeyword{assumes}\ {\isachardoublequoteopen}f\ {\isasymin}\isactrlsub c\ {\isacharparenleft}{\kern0pt}U\isactrlbsup U\isactrlesup {\isacharparenright}{\kern0pt}{\isachardoublequoteclose}\isanewline
\ \ \isakeyword{shows}\ {\isachardoublequoteopen}ITER\ U\ {\isasymcirc}\isactrlsub c\ {\isasymlangle}f{\isacharcomma}{\kern0pt}\ zero{\isasymrangle}\ {\isacharequal}{\kern0pt}\ metafunc\ {\isacharparenleft}{\kern0pt}id\ U{\isacharparenright}{\kern0pt}{\isachardoublequoteclose}\isanewline
%
\isadelimproof
\ \ %
\endisadelimproof
%
\isatagproof
\isacommand{by}\isamarkupfalse%
\ {\isacharparenleft}{\kern0pt}typecheck{\isacharunderscore}{\kern0pt}cfuncs{\isacharcomma}{\kern0pt}\ metis\ ITER{\isacharunderscore}{\kern0pt}zero\ assms\ id{\isacharunderscore}{\kern0pt}right{\isacharunderscore}{\kern0pt}unit{\isadigit{2}}\ id{\isacharunderscore}{\kern0pt}type\ one{\isacharunderscore}{\kern0pt}unique{\isacharunderscore}{\kern0pt}element\ terminal{\isacharunderscore}{\kern0pt}func{\isacharunderscore}{\kern0pt}type{\isacharparenright}{\kern0pt}%
\endisatagproof
{\isafoldproof}%
%
\isadelimproof
\isanewline
%
\endisadelimproof
\isanewline
\isacommand{lemma}\isamarkupfalse%
\ ITER{\isacharunderscore}{\kern0pt}succ{\isacharcolon}{\kern0pt}\isanewline
\ \isakeyword{assumes}\ f{\isacharunderscore}{\kern0pt}type{\isacharbrackleft}{\kern0pt}type{\isacharunderscore}{\kern0pt}rule{\isacharbrackright}{\kern0pt}{\isacharcolon}{\kern0pt}\ {\isachardoublequoteopen}f\ {\isacharcolon}{\kern0pt}\ Z\ {\isasymrightarrow}\ {\isacharparenleft}{\kern0pt}U\isactrlbsup U\isactrlesup {\isacharparenright}{\kern0pt}{\isachardoublequoteclose}\ \isakeyword{and}\ n{\isacharunderscore}{\kern0pt}type{\isacharbrackleft}{\kern0pt}type{\isacharunderscore}{\kern0pt}rule{\isacharbrackright}{\kern0pt}{\isacharcolon}{\kern0pt}\ {\isachardoublequoteopen}n\ {\isacharcolon}{\kern0pt}\ Z\ {\isasymrightarrow}\ {\isasymnat}\isactrlsub c{\isachardoublequoteclose}\isanewline
\ \isakeyword{shows}\ {\isachardoublequoteopen}ITER\ U\ {\isasymcirc}\isactrlsub c\ {\isasymlangle}f{\isacharcomma}{\kern0pt}\ successor\ {\isasymcirc}\isactrlsub c\ n{\isasymrangle}\ {\isacharequal}{\kern0pt}\ f\ {\isasymbox}\ {\isacharparenleft}{\kern0pt}ITER\ U\ {\isasymcirc}\isactrlsub c\ {\isasymlangle}f{\isacharcomma}{\kern0pt}\ n\ {\isasymrangle}{\isacharparenright}{\kern0pt}{\isachardoublequoteclose}\isanewline
%
\isadelimproof
%
\endisadelimproof
%
\isatagproof
\isacommand{proof}\isamarkupfalse%
{\isacharparenleft}{\kern0pt}etcs{\isacharunderscore}{\kern0pt}rule\ one{\isacharunderscore}{\kern0pt}separator{\isacharparenright}{\kern0pt}\isanewline
\ \ \isacommand{fix}\isamarkupfalse%
\ z\ \isanewline
\ \ \isacommand{assume}\isamarkupfalse%
\ z{\isacharunderscore}{\kern0pt}type{\isacharbrackleft}{\kern0pt}type{\isacharunderscore}{\kern0pt}rule{\isacharbrackright}{\kern0pt}{\isacharcolon}{\kern0pt}\ {\isachardoublequoteopen}z\ {\isasymin}\isactrlsub c\ Z{\isachardoublequoteclose}\isanewline
\ \ \isacommand{have}\isamarkupfalse%
\ {\isachardoublequoteopen}{\isacharparenleft}{\kern0pt}ITER\ U\ {\isasymcirc}\isactrlsub c\ {\isasymlangle}f{\isacharcomma}{\kern0pt}successor\ {\isasymcirc}\isactrlsub c\ n{\isasymrangle}{\isacharparenright}{\kern0pt}\ {\isasymcirc}\isactrlsub c\ z\ \ {\isacharequal}{\kern0pt}\ ITER\ U\ {\isasymcirc}\isactrlsub c\ {\isasymlangle}f{\isacharcomma}{\kern0pt}successor\ {\isasymcirc}\isactrlsub c\ n{\isasymrangle}\ {\isasymcirc}\isactrlsub c\ z{\isachardoublequoteclose}\isanewline
\ \ \ \ \isacommand{using}\isamarkupfalse%
\ assms\ \isacommand{by}\isamarkupfalse%
\ {\isacharparenleft}{\kern0pt}typecheck{\isacharunderscore}{\kern0pt}cfuncs{\isacharcomma}{\kern0pt}\ simp\ add{\isacharcolon}{\kern0pt}\ comp{\isacharunderscore}{\kern0pt}associative{\isadigit{2}}{\isacharparenright}{\kern0pt}\isanewline
\ \ \isacommand{also}\isamarkupfalse%
\ \isacommand{have}\isamarkupfalse%
\ {\isachardoublequoteopen}{\isachardot}{\kern0pt}{\isachardot}{\kern0pt}{\isachardot}{\kern0pt}\ {\isacharequal}{\kern0pt}\ ITER\ U\ {\isasymcirc}\isactrlsub c\ {\isasymlangle}f\ {\isasymcirc}\isactrlsub c\ z{\isacharcomma}{\kern0pt}\ successor\ {\isasymcirc}\isactrlsub c\ {\isacharparenleft}{\kern0pt}n\ \ {\isasymcirc}\isactrlsub c\ z{\isacharparenright}{\kern0pt}{\isasymrangle}{\isachardoublequoteclose}\isanewline
\ \ \ \ \isacommand{using}\isamarkupfalse%
\ assms\ \isacommand{by}\isamarkupfalse%
\ {\isacharparenleft}{\kern0pt}typecheck{\isacharunderscore}{\kern0pt}cfuncs{\isacharcomma}{\kern0pt}\ simp\ add{\isacharcolon}{\kern0pt}\ cfunc{\isacharunderscore}{\kern0pt}prod{\isacharunderscore}{\kern0pt}comp\ comp{\isacharunderscore}{\kern0pt}associative{\isadigit{2}}{\isacharparenright}{\kern0pt}\isanewline
\ \ \isacommand{also}\isamarkupfalse%
\ \isacommand{have}\isamarkupfalse%
\ {\isachardoublequoteopen}{\isachardot}{\kern0pt}{\isachardot}{\kern0pt}{\isachardot}{\kern0pt}\ {\isacharequal}{\kern0pt}\ {\isacharparenleft}{\kern0pt}eval{\isacharunderscore}{\kern0pt}func\ {\isacharparenleft}{\kern0pt}U\isactrlbsup U\isactrlesup {\isacharparenright}{\kern0pt}\ {\isacharparenleft}{\kern0pt}U\isactrlbsup U\isactrlesup {\isacharparenright}{\kern0pt}{\isacharparenright}{\kern0pt}\ {\isasymcirc}\isactrlsub c\ {\isacharparenleft}{\kern0pt}id\isactrlsub c\ {\isacharparenleft}{\kern0pt}U\isactrlbsup U\isactrlesup {\isacharparenright}{\kern0pt}\ {\isasymtimes}\isactrlsub f\ ITER{\isacharunderscore}{\kern0pt}curried\ U{\isacharparenright}{\kern0pt}\ {\isasymcirc}\isactrlsub c\ {\isasymlangle}f\ {\isasymcirc}\isactrlsub c\ z{\isacharcomma}{\kern0pt}\ successor\ {\isasymcirc}\isactrlsub c\ {\isacharparenleft}{\kern0pt}n\ \ {\isasymcirc}\isactrlsub c\ z{\isacharparenright}{\kern0pt}{\isasymrangle}{\isachardoublequoteclose}\isanewline
\ \ \ \ \isacommand{using}\isamarkupfalse%
\ assms\ \isacommand{by}\isamarkupfalse%
\ {\isacharparenleft}{\kern0pt}typecheck{\isacharunderscore}{\kern0pt}cfuncs{\isacharcomma}{\kern0pt}\ simp\ add{\isacharcolon}{\kern0pt}\ ITER{\isacharunderscore}{\kern0pt}def\ comp{\isacharunderscore}{\kern0pt}associative{\isadigit{2}}\ inv{\isacharunderscore}{\kern0pt}transpose{\isacharunderscore}{\kern0pt}func{\isacharunderscore}{\kern0pt}def{\isadigit{3}}{\isacharparenright}{\kern0pt}\isanewline
\ \ \isacommand{also}\isamarkupfalse%
\ \isacommand{have}\isamarkupfalse%
\ {\isachardoublequoteopen}{\isachardot}{\kern0pt}{\isachardot}{\kern0pt}{\isachardot}{\kern0pt}\ {\isacharequal}{\kern0pt}\ {\isacharparenleft}{\kern0pt}eval{\isacharunderscore}{\kern0pt}func\ {\isacharparenleft}{\kern0pt}U\isactrlbsup U\isactrlesup {\isacharparenright}{\kern0pt}\ {\isacharparenleft}{\kern0pt}U\isactrlbsup U\isactrlesup {\isacharparenright}{\kern0pt}{\isacharparenright}{\kern0pt}\ {\isasymcirc}\isactrlsub c\ {\isasymlangle}f\ {\isasymcirc}\isactrlsub c\ z{\isacharcomma}{\kern0pt}\ ITER{\isacharunderscore}{\kern0pt}curried\ U\ {\isasymcirc}\isactrlsub c\ {\isacharparenleft}{\kern0pt}successor\ {\isasymcirc}\isactrlsub c\ {\isacharparenleft}{\kern0pt}n\ \ {\isasymcirc}\isactrlsub c\ z{\isacharparenright}{\kern0pt}{\isacharparenright}{\kern0pt}{\isasymrangle}{\isachardoublequoteclose}\isanewline
\ \ \ \ \isacommand{using}\isamarkupfalse%
\ assms\ cfunc{\isacharunderscore}{\kern0pt}cross{\isacharunderscore}{\kern0pt}prod{\isacharunderscore}{\kern0pt}comp{\isacharunderscore}{\kern0pt}cfunc{\isacharunderscore}{\kern0pt}prod\ id{\isacharunderscore}{\kern0pt}left{\isacharunderscore}{\kern0pt}unit{\isadigit{2}}\ \isacommand{by}\isamarkupfalse%
\ {\isacharparenleft}{\kern0pt}typecheck{\isacharunderscore}{\kern0pt}cfuncs{\isacharcomma}{\kern0pt}\ force{\isacharparenright}{\kern0pt}\isanewline
\ \ \isacommand{also}\isamarkupfalse%
\ \isacommand{have}\isamarkupfalse%
\ {\isachardoublequoteopen}{\isachardot}{\kern0pt}{\isachardot}{\kern0pt}{\isachardot}{\kern0pt}\ {\isacharequal}{\kern0pt}\ {\isacharparenleft}{\kern0pt}eval{\isacharunderscore}{\kern0pt}func\ {\isacharparenleft}{\kern0pt}U\isactrlbsup U\isactrlesup {\isacharparenright}{\kern0pt}\ {\isacharparenleft}{\kern0pt}U\isactrlbsup U\isactrlesup {\isacharparenright}{\kern0pt}{\isacharparenright}{\kern0pt}\ {\isasymcirc}\isactrlsub c\ {\isasymlangle}f\ {\isasymcirc}\isactrlsub c\ z{\isacharcomma}{\kern0pt}\ {\isacharparenleft}{\kern0pt}ITER{\isacharunderscore}{\kern0pt}curried\ U\ {\isasymcirc}\isactrlsub c\ successor{\isacharparenright}{\kern0pt}\ {\isasymcirc}\isactrlsub c\ {\isacharparenleft}{\kern0pt}n\ \ {\isasymcirc}\isactrlsub c\ z{\isacharparenright}{\kern0pt}{\isasymrangle}{\isachardoublequoteclose}\isanewline
\ \ \ \ \isacommand{using}\isamarkupfalse%
\ assms\ \isacommand{by}\isamarkupfalse%
{\isacharparenleft}{\kern0pt}typecheck{\isacharunderscore}{\kern0pt}cfuncs{\isacharcomma}{\kern0pt}\ metis\ comp{\isacharunderscore}{\kern0pt}associative{\isadigit{2}}{\isacharparenright}{\kern0pt}\isanewline
\ \ \isacommand{also}\isamarkupfalse%
\ \isacommand{have}\isamarkupfalse%
\ {\isachardoublequoteopen}{\isachardot}{\kern0pt}{\isachardot}{\kern0pt}{\isachardot}{\kern0pt}\ {\isacharequal}{\kern0pt}\ {\isacharparenleft}{\kern0pt}eval{\isacharunderscore}{\kern0pt}func\ {\isacharparenleft}{\kern0pt}U\isactrlbsup U\isactrlesup {\isacharparenright}{\kern0pt}\ {\isacharparenleft}{\kern0pt}U\isactrlbsup U\isactrlesup {\isacharparenright}{\kern0pt}{\isacharparenright}{\kern0pt}\ {\isasymcirc}\isactrlsub c\ {\isasymlangle}f\ {\isasymcirc}\isactrlsub c\ z{\isacharcomma}{\kern0pt}\ {\isacharparenleft}{\kern0pt}{\isacharparenleft}{\kern0pt}meta{\isacharunderscore}{\kern0pt}comp\ U\ U\ U\ {\isasymcirc}\isactrlsub c\ {\isacharparenleft}{\kern0pt}id\ {\isacharparenleft}{\kern0pt}U\isactrlbsup U\isactrlesup {\isacharparenright}{\kern0pt}\ {\isasymtimes}\isactrlsub f\ eval{\isacharunderscore}{\kern0pt}func\ {\isacharparenleft}{\kern0pt}U\isactrlbsup U\isactrlesup {\isacharparenright}{\kern0pt}\ {\isacharparenleft}{\kern0pt}U\isactrlbsup U\isactrlesup {\isacharparenright}{\kern0pt}{\isacharparenright}{\kern0pt}\ {\isasymcirc}\isactrlsub c\ {\isacharparenleft}{\kern0pt}associate{\isacharunderscore}{\kern0pt}right\ {\isacharparenleft}{\kern0pt}U\isactrlbsup U\isactrlesup {\isacharparenright}{\kern0pt}\ {\isacharparenleft}{\kern0pt}U\isactrlbsup U\isactrlesup {\isacharparenright}{\kern0pt}\ {\isacharparenleft}{\kern0pt}{\isacharparenleft}{\kern0pt}U\isactrlbsup U\isactrlesup {\isacharparenright}{\kern0pt}\isactrlbsup U\isactrlbsup U\isactrlesup \isactrlesup {\isacharparenright}{\kern0pt}{\isacharparenright}{\kern0pt}\ {\isasymcirc}\isactrlsub c\ {\isacharparenleft}{\kern0pt}diagonal{\isacharparenleft}{\kern0pt}U\isactrlbsup U\isactrlesup {\isacharparenright}{\kern0pt}{\isasymtimes}\isactrlsub f\ id\ {\isacharparenleft}{\kern0pt}{\isacharparenleft}{\kern0pt}U\isactrlbsup U\isactrlesup {\isacharparenright}{\kern0pt}\isactrlbsup U\isactrlbsup U\isactrlesup \isactrlesup {\isacharparenright}{\kern0pt}{\isacharparenright}{\kern0pt}{\isacharparenright}{\kern0pt}\isactrlsup {\isasymsharp}\ {\isasymcirc}\isactrlsub c\ ITER{\isacharunderscore}{\kern0pt}curried\ U{\isacharparenright}{\kern0pt}\ {\isasymcirc}\isactrlsub c\ {\isacharparenleft}{\kern0pt}n\ \ {\isasymcirc}\isactrlsub c\ z{\isacharparenright}{\kern0pt}{\isasymrangle}{\isachardoublequoteclose}\isanewline
\ \ \ \ \isacommand{using}\isamarkupfalse%
\ assms\ ITER{\isacharunderscore}{\kern0pt}curried{\isacharunderscore}{\kern0pt}successor\ \isacommand{by}\isamarkupfalse%
\ presburger\isanewline
\ \ \isacommand{also}\isamarkupfalse%
\ \isacommand{have}\isamarkupfalse%
\ {\isachardoublequoteopen}{\isachardot}{\kern0pt}{\isachardot}{\kern0pt}{\isachardot}{\kern0pt}\ {\isacharequal}{\kern0pt}\ {\isacharparenleft}{\kern0pt}eval{\isacharunderscore}{\kern0pt}func\ {\isacharparenleft}{\kern0pt}U\isactrlbsup U\isactrlesup {\isacharparenright}{\kern0pt}\ {\isacharparenleft}{\kern0pt}U\isactrlbsup U\isactrlesup {\isacharparenright}{\kern0pt}{\isacharparenright}{\kern0pt}\ {\isasymcirc}\isactrlsub c\ {\isacharparenleft}{\kern0pt}id\ {\isacharparenleft}{\kern0pt}U\isactrlbsup U\isactrlesup {\isacharparenright}{\kern0pt}\ {\isasymtimes}\isactrlsub f\ {\isacharparenleft}{\kern0pt}{\isacharparenleft}{\kern0pt}meta{\isacharunderscore}{\kern0pt}comp\ U\ U\ U\ {\isasymcirc}\isactrlsub c\ {\isacharparenleft}{\kern0pt}id\ {\isacharparenleft}{\kern0pt}U\isactrlbsup U\isactrlesup {\isacharparenright}{\kern0pt}\ {\isasymtimes}\isactrlsub f\ eval{\isacharunderscore}{\kern0pt}func\ {\isacharparenleft}{\kern0pt}U\isactrlbsup U\isactrlesup {\isacharparenright}{\kern0pt}\ {\isacharparenleft}{\kern0pt}U\isactrlbsup U\isactrlesup {\isacharparenright}{\kern0pt}{\isacharparenright}{\kern0pt}\ {\isasymcirc}\isactrlsub c\ {\isacharparenleft}{\kern0pt}associate{\isacharunderscore}{\kern0pt}right\ {\isacharparenleft}{\kern0pt}U\isactrlbsup U\isactrlesup {\isacharparenright}{\kern0pt}\ {\isacharparenleft}{\kern0pt}U\isactrlbsup U\isactrlesup {\isacharparenright}{\kern0pt}\ {\isacharparenleft}{\kern0pt}{\isacharparenleft}{\kern0pt}U\isactrlbsup U\isactrlesup {\isacharparenright}{\kern0pt}\isactrlbsup U\isactrlbsup U\isactrlesup \isactrlesup {\isacharparenright}{\kern0pt}{\isacharparenright}{\kern0pt}\ {\isasymcirc}\isactrlsub c\ {\isacharparenleft}{\kern0pt}diagonal{\isacharparenleft}{\kern0pt}U\isactrlbsup U\isactrlesup {\isacharparenright}{\kern0pt}{\isasymtimes}\isactrlsub f\ id\ {\isacharparenleft}{\kern0pt}{\isacharparenleft}{\kern0pt}U\isactrlbsup U\isactrlesup {\isacharparenright}{\kern0pt}\isactrlbsup U\isactrlbsup U\isactrlesup \isactrlesup {\isacharparenright}{\kern0pt}{\isacharparenright}{\kern0pt}{\isacharparenright}{\kern0pt}\isactrlsup {\isasymsharp}\ {\isasymcirc}\isactrlsub c\ ITER{\isacharunderscore}{\kern0pt}curried\ U{\isacharparenright}{\kern0pt}\ {\isasymcirc}\isactrlsub c\ {\isacharparenleft}{\kern0pt}n\ \ {\isasymcirc}\isactrlsub c\ z{\isacharparenright}{\kern0pt}{\isacharparenright}{\kern0pt}{\isasymcirc}\isactrlsub c\ {\isasymlangle}f\ {\isasymcirc}\isactrlsub c\ z{\isacharcomma}{\kern0pt}\ id\ {\isasymone}{\isasymrangle}{\isachardoublequoteclose}\isanewline
\ \ \ \ \isacommand{using}\isamarkupfalse%
\ assms\ \isacommand{by}\isamarkupfalse%
\ {\isacharparenleft}{\kern0pt}typecheck{\isacharunderscore}{\kern0pt}cfuncs{\isacharcomma}{\kern0pt}\ simp\ add{\isacharcolon}{\kern0pt}\ cfunc{\isacharunderscore}{\kern0pt}cross{\isacharunderscore}{\kern0pt}prod{\isacharunderscore}{\kern0pt}comp{\isacharunderscore}{\kern0pt}cfunc{\isacharunderscore}{\kern0pt}prod\ id{\isacharunderscore}{\kern0pt}left{\isacharunderscore}{\kern0pt}unit{\isadigit{2}}\ id{\isacharunderscore}{\kern0pt}right{\isacharunderscore}{\kern0pt}unit{\isadigit{2}}{\isacharparenright}{\kern0pt}\isanewline
\ \ \isacommand{also}\isamarkupfalse%
\ \isacommand{have}\isamarkupfalse%
\ {\isachardoublequoteopen}{\isachardot}{\kern0pt}{\isachardot}{\kern0pt}{\isachardot}{\kern0pt}\ {\isacharequal}{\kern0pt}\ {\isacharparenleft}{\kern0pt}eval{\isacharunderscore}{\kern0pt}func\ {\isacharparenleft}{\kern0pt}U\isactrlbsup U\isactrlesup {\isacharparenright}{\kern0pt}\ {\isacharparenleft}{\kern0pt}U\isactrlbsup U\isactrlesup {\isacharparenright}{\kern0pt}{\isacharparenright}{\kern0pt}\ {\isasymcirc}\isactrlsub c\ {\isacharparenleft}{\kern0pt}id\ {\isacharparenleft}{\kern0pt}U\isactrlbsup U\isactrlesup {\isacharparenright}{\kern0pt}\ {\isasymtimes}\isactrlsub f\ {\isacharparenleft}{\kern0pt}{\isacharparenleft}{\kern0pt}meta{\isacharunderscore}{\kern0pt}comp\ U\ U\ U\ {\isasymcirc}\isactrlsub c\ {\isacharparenleft}{\kern0pt}id\ {\isacharparenleft}{\kern0pt}U\isactrlbsup U\isactrlesup {\isacharparenright}{\kern0pt}\ {\isasymtimes}\isactrlsub f\ eval{\isacharunderscore}{\kern0pt}func\ {\isacharparenleft}{\kern0pt}U\isactrlbsup U\isactrlesup {\isacharparenright}{\kern0pt}\ {\isacharparenleft}{\kern0pt}U\isactrlbsup U\isactrlesup {\isacharparenright}{\kern0pt}{\isacharparenright}{\kern0pt}\ {\isasymcirc}\isactrlsub c\ {\isacharparenleft}{\kern0pt}associate{\isacharunderscore}{\kern0pt}right\ {\isacharparenleft}{\kern0pt}U\isactrlbsup U\isactrlesup {\isacharparenright}{\kern0pt}\ {\isacharparenleft}{\kern0pt}U\isactrlbsup U\isactrlesup {\isacharparenright}{\kern0pt}\ {\isacharparenleft}{\kern0pt}{\isacharparenleft}{\kern0pt}U\isactrlbsup U\isactrlesup {\isacharparenright}{\kern0pt}\isactrlbsup U\isactrlbsup U\isactrlesup \isactrlesup {\isacharparenright}{\kern0pt}{\isacharparenright}{\kern0pt}\ {\isasymcirc}\isactrlsub c\ {\isacharparenleft}{\kern0pt}diagonal{\isacharparenleft}{\kern0pt}U\isactrlbsup U\isactrlesup {\isacharparenright}{\kern0pt}{\isasymtimes}\isactrlsub f\ id\ {\isacharparenleft}{\kern0pt}{\isacharparenleft}{\kern0pt}U\isactrlbsup U\isactrlesup {\isacharparenright}{\kern0pt}\isactrlbsup U\isactrlbsup U\isactrlesup \isactrlesup {\isacharparenright}{\kern0pt}{\isacharparenright}{\kern0pt}{\isacharparenright}{\kern0pt}\isactrlsup {\isasymsharp}\ {\isacharparenright}{\kern0pt}{\isacharparenright}{\kern0pt}{\isasymcirc}\isactrlsub c\ {\isasymlangle}f\ {\isasymcirc}\isactrlsub c\ z{\isacharcomma}{\kern0pt}\ ITER{\isacharunderscore}{\kern0pt}curried\ U\ {\isasymcirc}\isactrlsub c\ {\isacharparenleft}{\kern0pt}n\ \ {\isasymcirc}\isactrlsub c\ z{\isacharparenright}{\kern0pt}{\isasymrangle}{\isachardoublequoteclose}\isanewline
\ \ \ \ \isacommand{using}\isamarkupfalse%
\ assms\ \isacommand{by}\isamarkupfalse%
\ {\isacharparenleft}{\kern0pt}typecheck{\isacharunderscore}{\kern0pt}cfuncs{\isacharcomma}{\kern0pt}\ smt\ {\isacharparenleft}{\kern0pt}z{\isadigit{3}}{\isacharparenright}{\kern0pt}\ cfunc{\isacharunderscore}{\kern0pt}cross{\isacharunderscore}{\kern0pt}prod{\isacharunderscore}{\kern0pt}comp{\isacharunderscore}{\kern0pt}cfunc{\isacharunderscore}{\kern0pt}prod\ comp{\isacharunderscore}{\kern0pt}associative{\isadigit{2}}\ id{\isacharunderscore}{\kern0pt}right{\isacharunderscore}{\kern0pt}unit{\isadigit{2}}{\isacharparenright}{\kern0pt}\isanewline
\ \ \isacommand{also}\isamarkupfalse%
\ \isacommand{have}\isamarkupfalse%
\ {\isachardoublequoteopen}{\isachardot}{\kern0pt}{\isachardot}{\kern0pt}{\isachardot}{\kern0pt}\ {\isacharequal}{\kern0pt}\ {\isacharparenleft}{\kern0pt}meta{\isacharunderscore}{\kern0pt}comp\ U\ U\ U\ {\isasymcirc}\isactrlsub c\ {\isacharparenleft}{\kern0pt}id\ {\isacharparenleft}{\kern0pt}U\isactrlbsup U\isactrlesup {\isacharparenright}{\kern0pt}\ {\isasymtimes}\isactrlsub f\ eval{\isacharunderscore}{\kern0pt}func\ {\isacharparenleft}{\kern0pt}U\isactrlbsup U\isactrlesup {\isacharparenright}{\kern0pt}\ {\isacharparenleft}{\kern0pt}U\isactrlbsup U\isactrlesup {\isacharparenright}{\kern0pt}{\isacharparenright}{\kern0pt}\ {\isasymcirc}\isactrlsub c\ {\isacharparenleft}{\kern0pt}associate{\isacharunderscore}{\kern0pt}right\ {\isacharparenleft}{\kern0pt}U\isactrlbsup U\isactrlesup {\isacharparenright}{\kern0pt}\ {\isacharparenleft}{\kern0pt}U\isactrlbsup U\isactrlesup {\isacharparenright}{\kern0pt}\ {\isacharparenleft}{\kern0pt}{\isacharparenleft}{\kern0pt}U\isactrlbsup U\isactrlesup {\isacharparenright}{\kern0pt}\isactrlbsup U\isactrlbsup U\isactrlesup \isactrlesup {\isacharparenright}{\kern0pt}{\isacharparenright}{\kern0pt}\ {\isasymcirc}\isactrlsub c\ {\isacharparenleft}{\kern0pt}diagonal{\isacharparenleft}{\kern0pt}U\isactrlbsup U\isactrlesup {\isacharparenright}{\kern0pt}{\isasymtimes}\isactrlsub f\ id\ {\isacharparenleft}{\kern0pt}{\isacharparenleft}{\kern0pt}U\isactrlbsup U\isactrlesup {\isacharparenright}{\kern0pt}\isactrlbsup U\isactrlbsup U\isactrlesup \isactrlesup {\isacharparenright}{\kern0pt}{\isacharparenright}{\kern0pt}{\isacharparenright}{\kern0pt}{\isasymcirc}\isactrlsub c\ {\isasymlangle}f\ {\isasymcirc}\isactrlsub c\ z{\isacharcomma}{\kern0pt}\ ITER{\isacharunderscore}{\kern0pt}curried\ U\ {\isasymcirc}\isactrlsub c\ {\isacharparenleft}{\kern0pt}n\ \ {\isasymcirc}\isactrlsub c\ z{\isacharparenright}{\kern0pt}{\isasymrangle}{\isachardoublequoteclose}\isanewline
\ \ \ \ \isacommand{using}\isamarkupfalse%
\ assms\ \isacommand{by}\isamarkupfalse%
\ {\isacharparenleft}{\kern0pt}typecheck{\isacharunderscore}{\kern0pt}cfuncs{\isacharcomma}{\kern0pt}\ metis\ cfunc{\isacharunderscore}{\kern0pt}type{\isacharunderscore}{\kern0pt}def\ comp{\isacharunderscore}{\kern0pt}associative\ transpose{\isacharunderscore}{\kern0pt}func{\isacharunderscore}{\kern0pt}def{\isacharparenright}{\kern0pt}\isanewline
\ \ \isacommand{also}\isamarkupfalse%
\ \isacommand{have}\isamarkupfalse%
\ {\isachardoublequoteopen}{\isachardot}{\kern0pt}{\isachardot}{\kern0pt}{\isachardot}{\kern0pt}\ {\isacharequal}{\kern0pt}\ {\isacharparenleft}{\kern0pt}meta{\isacharunderscore}{\kern0pt}comp\ U\ U\ U\ {\isasymcirc}\isactrlsub c\ {\isacharparenleft}{\kern0pt}id\ {\isacharparenleft}{\kern0pt}U\isactrlbsup U\isactrlesup {\isacharparenright}{\kern0pt}\ {\isasymtimes}\isactrlsub f\ eval{\isacharunderscore}{\kern0pt}func\ {\isacharparenleft}{\kern0pt}U\isactrlbsup U\isactrlesup {\isacharparenright}{\kern0pt}\ {\isacharparenleft}{\kern0pt}U\isactrlbsup U\isactrlesup {\isacharparenright}{\kern0pt}{\isacharparenright}{\kern0pt}\ {\isasymcirc}\isactrlsub c\ {\isacharparenleft}{\kern0pt}associate{\isacharunderscore}{\kern0pt}right\ {\isacharparenleft}{\kern0pt}U\isactrlbsup U\isactrlesup {\isacharparenright}{\kern0pt}\ {\isacharparenleft}{\kern0pt}U\isactrlbsup U\isactrlesup {\isacharparenright}{\kern0pt}\ {\isacharparenleft}{\kern0pt}{\isacharparenleft}{\kern0pt}U\isactrlbsup U\isactrlesup {\isacharparenright}{\kern0pt}\isactrlbsup U\isactrlbsup U\isactrlesup \isactrlesup {\isacharparenright}{\kern0pt}{\isacharparenright}{\kern0pt}{\isacharparenright}{\kern0pt}{\isasymcirc}\isactrlsub c\ {\isasymlangle}{\isasymlangle}f\ {\isasymcirc}\isactrlsub c\ z{\isacharcomma}{\kern0pt}f\ {\isasymcirc}\isactrlsub c\ z{\isasymrangle}{\isacharcomma}{\kern0pt}\ ITER{\isacharunderscore}{\kern0pt}curried\ U\ {\isasymcirc}\isactrlsub c\ {\isacharparenleft}{\kern0pt}n\ \ {\isasymcirc}\isactrlsub c\ z{\isacharparenright}{\kern0pt}{\isasymrangle}{\isachardoublequoteclose}\isanewline
\ \ \ \ \isacommand{using}\isamarkupfalse%
\ assms\ \isacommand{by}\isamarkupfalse%
\ {\isacharparenleft}{\kern0pt}etcs{\isacharunderscore}{\kern0pt}assocr{\isacharcomma}{\kern0pt}\ typecheck{\isacharunderscore}{\kern0pt}cfuncs{\isacharcomma}{\kern0pt}\ smt\ {\isacharparenleft}{\kern0pt}z{\isadigit{3}}{\isacharparenright}{\kern0pt}\ cfunc{\isacharunderscore}{\kern0pt}cross{\isacharunderscore}{\kern0pt}prod{\isacharunderscore}{\kern0pt}comp{\isacharunderscore}{\kern0pt}cfunc{\isacharunderscore}{\kern0pt}prod\ diag{\isacharunderscore}{\kern0pt}on{\isacharunderscore}{\kern0pt}elements\ id{\isacharunderscore}{\kern0pt}left{\isacharunderscore}{\kern0pt}unit{\isadigit{2}}{\isacharparenright}{\kern0pt}\isanewline
\ \ \isacommand{also}\isamarkupfalse%
\ \isacommand{have}\isamarkupfalse%
\ {\isachardoublequoteopen}{\isachardot}{\kern0pt}{\isachardot}{\kern0pt}{\isachardot}{\kern0pt}\ {\isacharequal}{\kern0pt}\ meta{\isacharunderscore}{\kern0pt}comp\ U\ U\ U\ {\isasymcirc}\isactrlsub c\ {\isacharparenleft}{\kern0pt}id\ {\isacharparenleft}{\kern0pt}U\isactrlbsup U\isactrlesup {\isacharparenright}{\kern0pt}\ {\isasymtimes}\isactrlsub f\ eval{\isacharunderscore}{\kern0pt}func\ {\isacharparenleft}{\kern0pt}U\isactrlbsup U\isactrlesup {\isacharparenright}{\kern0pt}\ {\isacharparenleft}{\kern0pt}U\isactrlbsup U\isactrlesup {\isacharparenright}{\kern0pt}{\isacharparenright}{\kern0pt}\ {\isasymcirc}\isactrlsub c\ {\isasymlangle}f\ {\isasymcirc}\isactrlsub c\ z{\isacharcomma}{\kern0pt}\ {\isasymlangle}f\ {\isasymcirc}\isactrlsub c\ z{\isacharcomma}{\kern0pt}\ ITER{\isacharunderscore}{\kern0pt}curried\ U\ {\isasymcirc}\isactrlsub c\ {\isacharparenleft}{\kern0pt}n\ \ {\isasymcirc}\isactrlsub c\ z{\isacharparenright}{\kern0pt}{\isasymrangle}{\isasymrangle}{\isachardoublequoteclose}\isanewline
\ \ \ \ \isacommand{using}\isamarkupfalse%
\ assms\ \isacommand{by}\isamarkupfalse%
\ {\isacharparenleft}{\kern0pt}typecheck{\isacharunderscore}{\kern0pt}cfuncs{\isacharcomma}{\kern0pt}\ smt\ {\isacharparenleft}{\kern0pt}z{\isadigit{3}}{\isacharparenright}{\kern0pt}\ associate{\isacharunderscore}{\kern0pt}right{\isacharunderscore}{\kern0pt}ap\ comp{\isacharunderscore}{\kern0pt}associative{\isadigit{2}}{\isacharparenright}{\kern0pt}\isanewline
\ \ \isacommand{also}\isamarkupfalse%
\ \isacommand{have}\isamarkupfalse%
\ {\isachardoublequoteopen}{\isachardot}{\kern0pt}{\isachardot}{\kern0pt}{\isachardot}{\kern0pt}\ {\isacharequal}{\kern0pt}\ meta{\isacharunderscore}{\kern0pt}comp\ U\ U\ U\ {\isasymcirc}\isactrlsub c\ {\isasymlangle}f\ {\isasymcirc}\isactrlsub c\ z{\isacharcomma}{\kern0pt}\ eval{\isacharunderscore}{\kern0pt}func\ {\isacharparenleft}{\kern0pt}U\isactrlbsup U\isactrlesup {\isacharparenright}{\kern0pt}\ {\isacharparenleft}{\kern0pt}U\isactrlbsup U\isactrlesup {\isacharparenright}{\kern0pt}\ {\isasymcirc}\isactrlsub c\ {\isasymlangle}f\ {\isasymcirc}\isactrlsub c\ z{\isacharcomma}{\kern0pt}\ ITER{\isacharunderscore}{\kern0pt}curried\ U\ {\isasymcirc}\isactrlsub c\ {\isacharparenleft}{\kern0pt}n\ \ {\isasymcirc}\isactrlsub c\ z{\isacharparenright}{\kern0pt}{\isasymrangle}{\isasymrangle}{\isachardoublequoteclose}\isanewline
\ \ \ \ \isacommand{using}\isamarkupfalse%
\ assms\ \isacommand{by}\isamarkupfalse%
\ {\isacharparenleft}{\kern0pt}typecheck{\isacharunderscore}{\kern0pt}cfuncs{\isacharcomma}{\kern0pt}\ smt\ {\isacharparenleft}{\kern0pt}z{\isadigit{3}}{\isacharparenright}{\kern0pt}\ cfunc{\isacharunderscore}{\kern0pt}cross{\isacharunderscore}{\kern0pt}prod{\isacharunderscore}{\kern0pt}comp{\isacharunderscore}{\kern0pt}cfunc{\isacharunderscore}{\kern0pt}prod\ id{\isacharunderscore}{\kern0pt}left{\isacharunderscore}{\kern0pt}unit{\isadigit{2}}{\isacharparenright}{\kern0pt}\isanewline
\ \ \isacommand{also}\isamarkupfalse%
\ \isacommand{have}\isamarkupfalse%
\ {\isachardoublequoteopen}{\isachardot}{\kern0pt}{\isachardot}{\kern0pt}{\isachardot}{\kern0pt}\ {\isacharequal}{\kern0pt}\ meta{\isacharunderscore}{\kern0pt}comp\ U\ U\ U\ {\isasymcirc}\isactrlsub c\ {\isasymlangle}f\ {\isasymcirc}\isactrlsub c\ z{\isacharcomma}{\kern0pt}\ eval{\isacharunderscore}{\kern0pt}func\ {\isacharparenleft}{\kern0pt}U\isactrlbsup U\isactrlesup {\isacharparenright}{\kern0pt}\ {\isacharparenleft}{\kern0pt}U\isactrlbsup U\isactrlesup {\isacharparenright}{\kern0pt}\ {\isasymcirc}\isactrlsub c\ {\isacharparenleft}{\kern0pt}id{\isacharparenleft}{\kern0pt}U\isactrlbsup U\isactrlesup {\isacharparenright}{\kern0pt}\ {\isasymtimes}\isactrlsub f\ ITER{\isacharunderscore}{\kern0pt}curried\ U{\isacharparenright}{\kern0pt}{\isasymcirc}\isactrlsub c\ {\isasymlangle}f\ {\isasymcirc}\isactrlsub c\ z{\isacharcomma}{\kern0pt}\ n\ {\isasymcirc}\isactrlsub c\ z{\isasymrangle}{\isasymrangle}{\isachardoublequoteclose}\isanewline
\ \ \ \ \isacommand{using}\isamarkupfalse%
\ assms\ \isacommand{by}\isamarkupfalse%
\ {\isacharparenleft}{\kern0pt}typecheck{\isacharunderscore}{\kern0pt}cfuncs{\isacharcomma}{\kern0pt}\ smt\ {\isacharparenleft}{\kern0pt}z{\isadigit{3}}{\isacharparenright}{\kern0pt}\ cfunc{\isacharunderscore}{\kern0pt}cross{\isacharunderscore}{\kern0pt}prod{\isacharunderscore}{\kern0pt}comp{\isacharunderscore}{\kern0pt}cfunc{\isacharunderscore}{\kern0pt}prod\ id{\isacharunderscore}{\kern0pt}left{\isacharunderscore}{\kern0pt}unit{\isadigit{2}}{\isacharparenright}{\kern0pt}\isanewline
\ \ \isacommand{also}\isamarkupfalse%
\ \isacommand{have}\isamarkupfalse%
\ {\isachardoublequoteopen}{\isachardot}{\kern0pt}{\isachardot}{\kern0pt}{\isachardot}{\kern0pt}\ {\isacharequal}{\kern0pt}\ meta{\isacharunderscore}{\kern0pt}comp\ U\ U\ U\ {\isasymcirc}\isactrlsub c\ {\isasymlangle}f\ {\isasymcirc}\isactrlsub c\ z{\isacharcomma}{\kern0pt}\ ITER\ U\ {\isasymcirc}\isactrlsub c\ {\isasymlangle}f\ {\isasymcirc}\isactrlsub c\ z{\isacharcomma}{\kern0pt}\ n\ {\isasymcirc}\isactrlsub c\ z{\isasymrangle}{\isasymrangle}{\isachardoublequoteclose}\isanewline
\ \ \ \ \isacommand{using}\isamarkupfalse%
\ assms\ \isacommand{by}\isamarkupfalse%
\ {\isacharparenleft}{\kern0pt}typecheck{\isacharunderscore}{\kern0pt}cfuncs{\isacharcomma}{\kern0pt}\ smt\ {\isacharparenleft}{\kern0pt}z{\isadigit{3}}{\isacharparenright}{\kern0pt}\ ITER{\isacharunderscore}{\kern0pt}def\ comp{\isacharunderscore}{\kern0pt}associative{\isadigit{2}}\ inv{\isacharunderscore}{\kern0pt}transpose{\isacharunderscore}{\kern0pt}func{\isacharunderscore}{\kern0pt}def{\isadigit{3}}{\isacharparenright}{\kern0pt}\isanewline
\ \ \isacommand{also}\isamarkupfalse%
\ \isacommand{have}\isamarkupfalse%
\ {\isachardoublequoteopen}{\isachardot}{\kern0pt}{\isachardot}{\kern0pt}{\isachardot}{\kern0pt}\ {\isacharequal}{\kern0pt}\ meta{\isacharunderscore}{\kern0pt}comp\ U\ U\ U\ {\isasymcirc}\isactrlsub c\ {\isasymlangle}f{\isacharcomma}{\kern0pt}\ ITER\ U\ {\isasymcirc}\isactrlsub c\ {\isasymlangle}f\ {\isacharcomma}{\kern0pt}\ n{\isasymrangle}{\isasymrangle}\ {\isasymcirc}\isactrlsub c\ z{\isachardoublequoteclose}\isanewline
\ \ \ \ \isacommand{using}\isamarkupfalse%
\ assms\ \isacommand{by}\isamarkupfalse%
\ {\isacharparenleft}{\kern0pt}typecheck{\isacharunderscore}{\kern0pt}cfuncs{\isacharcomma}{\kern0pt}\ smt\ {\isacharparenleft}{\kern0pt}z{\isadigit{3}}{\isacharparenright}{\kern0pt}\ cfunc{\isacharunderscore}{\kern0pt}prod{\isacharunderscore}{\kern0pt}comp\ comp{\isacharunderscore}{\kern0pt}associative{\isadigit{2}}{\isacharparenright}{\kern0pt}\isanewline
\ \ \isacommand{also}\isamarkupfalse%
\ \isacommand{have}\isamarkupfalse%
\ {\isachardoublequoteopen}{\isachardot}{\kern0pt}{\isachardot}{\kern0pt}{\isachardot}{\kern0pt}\ {\isacharequal}{\kern0pt}\ {\isacharparenleft}{\kern0pt}meta{\isacharunderscore}{\kern0pt}comp\ U\ U\ U\ {\isasymcirc}\isactrlsub c\ {\isasymlangle}f{\isacharcomma}{\kern0pt}\ ITER\ U\ {\isasymcirc}\isactrlsub c\ {\isasymlangle}f\ {\isacharcomma}{\kern0pt}\ n{\isasymrangle}{\isasymrangle}{\isacharparenright}{\kern0pt}\ {\isasymcirc}\isactrlsub c\ z{\isachardoublequoteclose}\isanewline
\ \ \ \ \isacommand{using}\isamarkupfalse%
\ assms\ \isacommand{by}\isamarkupfalse%
\ {\isacharparenleft}{\kern0pt}typecheck{\isacharunderscore}{\kern0pt}cfuncs{\isacharcomma}{\kern0pt}\ meson\ comp{\isacharunderscore}{\kern0pt}associative{\isadigit{2}}{\isacharparenright}{\kern0pt}\isanewline
\ \ \isacommand{also}\isamarkupfalse%
\ \isacommand{have}\isamarkupfalse%
\ {\isachardoublequoteopen}{\isachardot}{\kern0pt}{\isachardot}{\kern0pt}{\isachardot}{\kern0pt}\ {\isacharequal}{\kern0pt}\ {\isacharparenleft}{\kern0pt}f\ {\isasymbox}\ {\isacharparenleft}{\kern0pt}ITER\ U\ {\isasymcirc}\isactrlsub c\ {\isasymlangle}f{\isacharcomma}{\kern0pt}n{\isasymrangle}{\isacharparenright}{\kern0pt}{\isacharparenright}{\kern0pt}\ {\isasymcirc}\isactrlsub c\ z{\isachardoublequoteclose}\isanewline
\ \ \ \ \isacommand{using}\isamarkupfalse%
\ assms\ \isacommand{by}\isamarkupfalse%
\ {\isacharparenleft}{\kern0pt}typecheck{\isacharunderscore}{\kern0pt}cfuncs{\isacharcomma}{\kern0pt}\ simp\ add{\isacharcolon}{\kern0pt}\ meta{\isacharunderscore}{\kern0pt}comp{\isadigit{2}}{\isacharunderscore}{\kern0pt}def{\isadigit{5}}{\isacharparenright}{\kern0pt}\isanewline
\ \ \isacommand{then}\isamarkupfalse%
\ \isacommand{show}\isamarkupfalse%
\ {\isachardoublequoteopen}{\isacharparenleft}{\kern0pt}ITER\ U\ {\isasymcirc}\isactrlsub c\ {\isasymlangle}f{\isacharcomma}{\kern0pt}successor\ {\isasymcirc}\isactrlsub c\ n{\isasymrangle}{\isacharparenright}{\kern0pt}\ {\isasymcirc}\isactrlsub c\ z\ {\isacharequal}{\kern0pt}\ {\isacharparenleft}{\kern0pt}f\ {\isasymbox}\ ITER\ U\ {\isasymcirc}\isactrlsub c\ {\isasymlangle}f{\isacharcomma}{\kern0pt}n{\isasymrangle}{\isacharparenright}{\kern0pt}\ {\isasymcirc}\isactrlsub c\ z{\isachardoublequoteclose}\isanewline
\ \ \ \ \isacommand{by}\isamarkupfalse%
\ {\isacharparenleft}{\kern0pt}simp\ add{\isacharcolon}{\kern0pt}\ calculation{\isacharparenright}{\kern0pt}\isanewline
\isacommand{qed}\isamarkupfalse%
%
\endisatagproof
{\isafoldproof}%
%
\isadelimproof
\isanewline
%
\endisadelimproof
\isanewline
\isacommand{lemma}\isamarkupfalse%
\ ITER{\isacharunderscore}{\kern0pt}one{\isacharcolon}{\kern0pt}\isanewline
\ \isakeyword{assumes}\ {\isachardoublequoteopen}f\ {\isasymin}\isactrlsub c\ {\isacharparenleft}{\kern0pt}U\isactrlbsup U\isactrlesup {\isacharparenright}{\kern0pt}{\isachardoublequoteclose}\isanewline
\ \isakeyword{shows}\ {\isachardoublequoteopen}ITER\ U\ {\isasymcirc}\isactrlsub c\ {\isasymlangle}f{\isacharcomma}{\kern0pt}\ successor\ {\isasymcirc}\isactrlsub c\ zero{\isasymrangle}\ {\isacharequal}{\kern0pt}\ f\ {\isasymbox}\ {\isacharparenleft}{\kern0pt}metafunc\ {\isacharparenleft}{\kern0pt}id\ U{\isacharparenright}{\kern0pt}{\isacharparenright}{\kern0pt}{\isachardoublequoteclose}\isanewline
%
\isadelimproof
\ \ %
\endisadelimproof
%
\isatagproof
\isacommand{using}\isamarkupfalse%
\ ITER{\isacharunderscore}{\kern0pt}succ\ ITER{\isacharunderscore}{\kern0pt}zero{\isacharprime}{\kern0pt}\ assms\ zero{\isacharunderscore}{\kern0pt}type\ \isacommand{by}\isamarkupfalse%
\ presburger%
\endisatagproof
{\isafoldproof}%
%
\isadelimproof
\isanewline
%
\endisadelimproof
\isanewline
\isacommand{definition}\isamarkupfalse%
\ iter{\isacharunderscore}{\kern0pt}comp\ {\isacharcolon}{\kern0pt}{\isacharcolon}{\kern0pt}\ {\isachardoublequoteopen}cfunc\ {\isasymRightarrow}\ cfunc\ {\isasymRightarrow}\ cfunc{\isachardoublequoteclose}\ {\isacharparenleft}{\kern0pt}{\isachardoublequoteopen}{\isacharunderscore}{\kern0pt}\isactrlbsup {\isasymcirc}{\isacharunderscore}{\kern0pt}\isactrlesup {\isachardoublequoteclose}{\isacharbrackleft}{\kern0pt}{\isadigit{5}}{\isadigit{5}}{\isacharcomma}{\kern0pt}{\isadigit{0}}{\isacharbrackright}{\kern0pt}{\isadigit{5}}{\isadigit{5}}{\isacharparenright}{\kern0pt}\ \isakeyword{where}\isanewline
\ \ {\isachardoublequoteopen}iter{\isacharunderscore}{\kern0pt}comp\ g\ n\ \ {\isasymequiv}\ cnufatem\ {\isacharparenleft}{\kern0pt}ITER\ {\isacharparenleft}{\kern0pt}domain\ g{\isacharparenright}{\kern0pt}\ {\isasymcirc}\isactrlsub c\ {\isasymlangle}metafunc\ g{\isacharcomma}{\kern0pt}n{\isasymrangle}{\isacharparenright}{\kern0pt}{\isachardoublequoteclose}\isanewline
\isanewline
\isacommand{lemma}\isamarkupfalse%
\ iter{\isacharunderscore}{\kern0pt}comp{\isacharunderscore}{\kern0pt}def{\isadigit{2}}{\isacharcolon}{\kern0pt}\ \isanewline
\ \ {\isachardoublequoteopen}g\isactrlbsup {\isasymcirc}n\isactrlesup \ \ {\isasymequiv}\ cnufatem{\isacharparenleft}{\kern0pt}ITER\ {\isacharparenleft}{\kern0pt}domain\ g{\isacharparenright}{\kern0pt}\ {\isasymcirc}\isactrlsub c\ {\isasymlangle}metafunc\ g{\isacharcomma}{\kern0pt}n{\isasymrangle}{\isacharparenright}{\kern0pt}{\isachardoublequoteclose}\isanewline
%
\isadelimproof
\ \ %
\endisadelimproof
%
\isatagproof
\isacommand{by}\isamarkupfalse%
\ {\isacharparenleft}{\kern0pt}simp\ add{\isacharcolon}{\kern0pt}\ iter{\isacharunderscore}{\kern0pt}comp{\isacharunderscore}{\kern0pt}def{\isacharparenright}{\kern0pt}%
\endisatagproof
{\isafoldproof}%
%
\isadelimproof
\isanewline
%
\endisadelimproof
\isanewline
\isacommand{lemma}\isamarkupfalse%
\ iter{\isacharunderscore}{\kern0pt}comp{\isacharunderscore}{\kern0pt}type{\isacharbrackleft}{\kern0pt}type{\isacharunderscore}{\kern0pt}rule{\isacharbrackright}{\kern0pt}{\isacharcolon}{\kern0pt}\isanewline
\ \ \isakeyword{assumes}\ {\isachardoublequoteopen}g\ {\isacharcolon}{\kern0pt}\ X\ {\isasymrightarrow}\ X{\isachardoublequoteclose}\isanewline
\ \ \isakeyword{assumes}\ {\isachardoublequoteopen}n\ {\isasymin}\isactrlsub c\ {\isasymnat}\isactrlsub c{\isachardoublequoteclose}\isanewline
\ \ \isakeyword{shows}\ {\isachardoublequoteopen}g\isactrlbsup {\isasymcirc}n\isactrlesup {\isacharcolon}{\kern0pt}\ X\ {\isasymrightarrow}\ X{\isachardoublequoteclose}\isanewline
%
\isadelimproof
\ \ %
\endisadelimproof
%
\isatagproof
\isacommand{unfolding}\isamarkupfalse%
\ iter{\isacharunderscore}{\kern0pt}comp{\isacharunderscore}{\kern0pt}def{\isadigit{2}}\isanewline
\ \ \isacommand{by}\isamarkupfalse%
\ {\isacharparenleft}{\kern0pt}smt\ {\isacharparenleft}{\kern0pt}verit{\isacharcomma}{\kern0pt}\ ccfv{\isacharunderscore}{\kern0pt}SIG{\isacharparenright}{\kern0pt}\ ITER{\isacharunderscore}{\kern0pt}type\ assms\ cfunc{\isacharunderscore}{\kern0pt}type{\isacharunderscore}{\kern0pt}def\ cnufatem{\isacharunderscore}{\kern0pt}type\ comp{\isacharunderscore}{\kern0pt}type\ metafunc{\isacharunderscore}{\kern0pt}type\ right{\isacharunderscore}{\kern0pt}param{\isacharunderscore}{\kern0pt}on{\isacharunderscore}{\kern0pt}el\ right{\isacharunderscore}{\kern0pt}param{\isacharunderscore}{\kern0pt}type{\isacharparenright}{\kern0pt}%
\endisatagproof
{\isafoldproof}%
%
\isadelimproof
\ \isanewline
%
\endisadelimproof
\isanewline
\isacommand{lemma}\isamarkupfalse%
\ iter{\isacharunderscore}{\kern0pt}comp{\isacharunderscore}{\kern0pt}def{\isadigit{3}}{\isacharcolon}{\kern0pt}\ \isanewline
\ \ \isakeyword{assumes}\ {\isachardoublequoteopen}g\ {\isacharcolon}{\kern0pt}\ X\ {\isasymrightarrow}\ X{\isachardoublequoteclose}\isanewline
\ \ \isakeyword{assumes}\ {\isachardoublequoteopen}n\ {\isasymin}\isactrlsub c\ {\isasymnat}\isactrlsub c{\isachardoublequoteclose}\isanewline
\ \ \isakeyword{shows}\ {\isachardoublequoteopen}g\isactrlbsup {\isasymcirc}n\isactrlesup \ \ {\isacharequal}{\kern0pt}\ cnufatem\ {\isacharparenleft}{\kern0pt}ITER\ X\ {\isasymcirc}\isactrlsub c\ {\isasymlangle}metafunc\ g{\isacharcomma}{\kern0pt}n{\isasymrangle}{\isacharparenright}{\kern0pt}{\isachardoublequoteclose}\isanewline
%
\isadelimproof
\ \ %
\endisadelimproof
%
\isatagproof
\isacommand{using}\isamarkupfalse%
\ assms\ cfunc{\isacharunderscore}{\kern0pt}type{\isacharunderscore}{\kern0pt}def\ iter{\isacharunderscore}{\kern0pt}comp{\isacharunderscore}{\kern0pt}def{\isadigit{2}}\ \isacommand{by}\isamarkupfalse%
\ auto%
\endisatagproof
{\isafoldproof}%
%
\isadelimproof
\isanewline
%
\endisadelimproof
\isanewline
\isacommand{lemma}\isamarkupfalse%
\ zero{\isacharunderscore}{\kern0pt}iters{\isacharcolon}{\kern0pt}\isanewline
\ \ \isakeyword{assumes}\ g{\isacharunderscore}{\kern0pt}type{\isacharbrackleft}{\kern0pt}type{\isacharunderscore}{\kern0pt}rule{\isacharbrackright}{\kern0pt}{\isacharcolon}{\kern0pt}\ {\isachardoublequoteopen}g\ {\isacharcolon}{\kern0pt}\ X\ {\isasymrightarrow}\ X{\isachardoublequoteclose}\isanewline
\ \ \isakeyword{shows}\ {\isachardoublequoteopen}g\isactrlbsup {\isasymcirc}zero\isactrlesup \ {\isacharequal}{\kern0pt}\ id\isactrlsub c\ X{\isachardoublequoteclose}\isanewline
%
\isadelimproof
%
\endisadelimproof
%
\isatagproof
\isacommand{proof}\isamarkupfalse%
{\isacharparenleft}{\kern0pt}etcs{\isacharunderscore}{\kern0pt}rule\ one{\isacharunderscore}{\kern0pt}separator{\isacharparenright}{\kern0pt}\isanewline
\ \ \isacommand{fix}\isamarkupfalse%
\ x\ \isanewline
\ \ \isacommand{assume}\isamarkupfalse%
\ x{\isacharunderscore}{\kern0pt}type{\isacharbrackleft}{\kern0pt}type{\isacharunderscore}{\kern0pt}rule{\isacharbrackright}{\kern0pt}{\isacharcolon}{\kern0pt}\ {\isachardoublequoteopen}x\ {\isasymin}\isactrlsub c\ X{\isachardoublequoteclose}\isanewline
\ \ \isacommand{have}\isamarkupfalse%
\ {\isachardoublequoteopen}{\isacharparenleft}{\kern0pt}g\isactrlbsup {\isasymcirc}zero\isactrlesup {\isacharparenright}{\kern0pt}\ {\isasymcirc}\isactrlsub c\ x\ {\isacharequal}{\kern0pt}\ {\isacharparenleft}{\kern0pt}cnufatem\ {\isacharparenleft}{\kern0pt}ITER\ X\ {\isasymcirc}\isactrlsub c\ {\isasymlangle}metafunc\ g{\isacharcomma}{\kern0pt}zero{\isasymrangle}{\isacharparenright}{\kern0pt}{\isacharparenright}{\kern0pt}\ {\isasymcirc}\isactrlsub c\ x{\isachardoublequoteclose}\isanewline
\ \ \ \ \isacommand{using}\isamarkupfalse%
\ assms\ iter{\isacharunderscore}{\kern0pt}comp{\isacharunderscore}{\kern0pt}def{\isadigit{3}}\ \isacommand{by}\isamarkupfalse%
\ {\isacharparenleft}{\kern0pt}typecheck{\isacharunderscore}{\kern0pt}cfuncs{\isacharcomma}{\kern0pt}\ auto{\isacharparenright}{\kern0pt}\isanewline
\ \ \isacommand{also}\isamarkupfalse%
\ \isacommand{have}\isamarkupfalse%
\ {\isachardoublequoteopen}{\isachardot}{\kern0pt}{\isachardot}{\kern0pt}{\isachardot}{\kern0pt}\ {\isacharequal}{\kern0pt}\ cnufatem\ {\isacharparenleft}{\kern0pt}metafunc\ {\isacharparenleft}{\kern0pt}id\ X{\isacharparenright}{\kern0pt}{\isacharparenright}{\kern0pt}\ {\isasymcirc}\isactrlsub c\ x{\isachardoublequoteclose}\isanewline
\ \ \ \ \isacommand{by}\isamarkupfalse%
\ {\isacharparenleft}{\kern0pt}simp\ add{\isacharcolon}{\kern0pt}\ ITER{\isacharunderscore}{\kern0pt}zero{\isacharprime}{\kern0pt}\ assms\ metafunc{\isacharunderscore}{\kern0pt}type{\isacharparenright}{\kern0pt}\isanewline
\ \ \isacommand{also}\isamarkupfalse%
\ \isacommand{have}\isamarkupfalse%
\ {\isachardoublequoteopen}{\isachardot}{\kern0pt}{\isachardot}{\kern0pt}{\isachardot}{\kern0pt}\ {\isacharequal}{\kern0pt}\ id\isactrlsub c\ X\ {\isasymcirc}\isactrlsub c\ x{\isachardoublequoteclose}\isanewline
\ \ \ \ \isacommand{by}\isamarkupfalse%
\ {\isacharparenleft}{\kern0pt}metis\ cnufatem{\isacharunderscore}{\kern0pt}metafunc\ id{\isacharunderscore}{\kern0pt}type{\isacharparenright}{\kern0pt}\isanewline
\ \ \isacommand{also}\isamarkupfalse%
\ \isacommand{have}\isamarkupfalse%
\ {\isachardoublequoteopen}{\isachardot}{\kern0pt}{\isachardot}{\kern0pt}{\isachardot}{\kern0pt}\ {\isacharequal}{\kern0pt}\ x{\isachardoublequoteclose}\isanewline
\ \ \ \ \isacommand{by}\isamarkupfalse%
\ {\isacharparenleft}{\kern0pt}typecheck{\isacharunderscore}{\kern0pt}cfuncs{\isacharcomma}{\kern0pt}\ simp\ add{\isacharcolon}{\kern0pt}\ id{\isacharunderscore}{\kern0pt}left{\isacharunderscore}{\kern0pt}unit{\isadigit{2}}{\isacharparenright}{\kern0pt}\isanewline
\ \ \isacommand{then}\isamarkupfalse%
\ \isacommand{show}\isamarkupfalse%
\ {\isachardoublequoteopen}{\isacharparenleft}{\kern0pt}g\isactrlbsup {\isasymcirc}zero\isactrlesup {\isacharparenright}{\kern0pt}\ {\isasymcirc}\isactrlsub c\ x\ {\isacharequal}{\kern0pt}\ id\isactrlsub c\ X\ {\isasymcirc}\isactrlsub c\ x{\isachardoublequoteclose}\isanewline
\ \ \ \ \isacommand{by}\isamarkupfalse%
\ {\isacharparenleft}{\kern0pt}simp\ add{\isacharcolon}{\kern0pt}\ calculation{\isacharparenright}{\kern0pt}\isanewline
\isacommand{qed}\isamarkupfalse%
%
\endisatagproof
{\isafoldproof}%
%
\isadelimproof
\isanewline
%
\endisadelimproof
\isanewline
\isacommand{lemma}\isamarkupfalse%
\ succ{\isacharunderscore}{\kern0pt}iters{\isacharcolon}{\kern0pt}\isanewline
\ \ \isakeyword{assumes}\ {\isachardoublequoteopen}g\ {\isacharcolon}{\kern0pt}\ X\ {\isasymrightarrow}\ X{\isachardoublequoteclose}\isanewline
\ \ \isakeyword{assumes}\ {\isachardoublequoteopen}n\ {\isasymin}\isactrlsub c\ {\isasymnat}\isactrlsub c{\isachardoublequoteclose}\isanewline
\ \ \isakeyword{shows}\ {\isachardoublequoteopen}g\isactrlbsup {\isasymcirc}{\isacharparenleft}{\kern0pt}successor\ {\isasymcirc}\isactrlsub c\ n{\isacharparenright}{\kern0pt}\isactrlesup \ {\isacharequal}{\kern0pt}\ g\ {\isasymcirc}\isactrlsub c\ {\isacharparenleft}{\kern0pt}g\isactrlbsup {\isasymcirc}n\isactrlesup {\isacharparenright}{\kern0pt}{\isachardoublequoteclose}\ \ \ \ \isanewline
%
\isadelimproof
%
\endisadelimproof
%
\isatagproof
\isacommand{proof}\isamarkupfalse%
\ {\isacharminus}{\kern0pt}\ \isanewline
\ \ \isacommand{have}\isamarkupfalse%
\ {\isachardoublequoteopen}g\isactrlbsup {\isasymcirc}successor\ {\isasymcirc}\isactrlsub c\ n\isactrlesup \ \ \ {\isacharequal}{\kern0pt}\ cnufatem{\isacharparenleft}{\kern0pt}ITER\ X\ {\isasymcirc}\isactrlsub c\ {\isasymlangle}metafunc\ g{\isacharcomma}{\kern0pt}successor\ {\isasymcirc}\isactrlsub c\ n\ {\isasymrangle}{\isacharparenright}{\kern0pt}{\isachardoublequoteclose}\isanewline
\ \ \ \ \isacommand{using}\isamarkupfalse%
\ assms\ \isacommand{by}\isamarkupfalse%
\ {\isacharparenleft}{\kern0pt}typecheck{\isacharunderscore}{\kern0pt}cfuncs{\isacharcomma}{\kern0pt}\ simp\ add{\isacharcolon}{\kern0pt}\ iter{\isacharunderscore}{\kern0pt}comp{\isacharunderscore}{\kern0pt}def{\isadigit{3}}{\isacharparenright}{\kern0pt}\isanewline
\ \ \isacommand{also}\isamarkupfalse%
\ \isacommand{have}\isamarkupfalse%
\ {\isachardoublequoteopen}{\isachardot}{\kern0pt}{\isachardot}{\kern0pt}{\isachardot}{\kern0pt}\ {\isacharequal}{\kern0pt}\ cnufatem{\isacharparenleft}{\kern0pt}metafunc\ g\ {\isasymbox}\ ITER\ X\ {\isasymcirc}\isactrlsub c\ {\isasymlangle}metafunc\ g{\isacharcomma}{\kern0pt}\ n\ {\isasymrangle}{\isacharparenright}{\kern0pt}{\isachardoublequoteclose}\isanewline
\ \ \ \ \isacommand{using}\isamarkupfalse%
\ assms\ \isacommand{by}\isamarkupfalse%
\ {\isacharparenleft}{\kern0pt}typecheck{\isacharunderscore}{\kern0pt}cfuncs{\isacharcomma}{\kern0pt}\ simp\ add{\isacharcolon}{\kern0pt}\ ITER{\isacharunderscore}{\kern0pt}succ{\isacharparenright}{\kern0pt}\isanewline
\ \ \isacommand{also}\isamarkupfalse%
\ \isacommand{have}\isamarkupfalse%
\ {\isachardoublequoteopen}{\isachardot}{\kern0pt}{\isachardot}{\kern0pt}{\isachardot}{\kern0pt}\ {\isacharequal}{\kern0pt}\ cnufatem{\isacharparenleft}{\kern0pt}metafunc\ g\ {\isasymbox}\ metafunc\ {\isacharparenleft}{\kern0pt}g\isactrlbsup {\isasymcirc}n\isactrlesup {\isacharparenright}{\kern0pt}{\isacharparenright}{\kern0pt}{\isachardoublequoteclose}\isanewline
\ \ \ \ \isacommand{using}\isamarkupfalse%
\ assms\ \isacommand{by}\isamarkupfalse%
\ {\isacharparenleft}{\kern0pt}typecheck{\isacharunderscore}{\kern0pt}cfuncs{\isacharcomma}{\kern0pt}\ metis\ iter{\isacharunderscore}{\kern0pt}comp{\isacharunderscore}{\kern0pt}def{\isadigit{3}}\ metafunc{\isacharunderscore}{\kern0pt}cnufatem{\isacharparenright}{\kern0pt}\isanewline
\ \ \isacommand{also}\isamarkupfalse%
\ \isacommand{have}\isamarkupfalse%
\ {\isachardoublequoteopen}{\isachardot}{\kern0pt}{\isachardot}{\kern0pt}{\isachardot}{\kern0pt}\ {\isacharequal}{\kern0pt}\ g\ {\isasymcirc}\isactrlsub c\ {\isacharparenleft}{\kern0pt}g\isactrlbsup {\isasymcirc}n\isactrlesup {\isacharparenright}{\kern0pt}{\isachardoublequoteclose}\isanewline
\ \ \ \ \isacommand{using}\isamarkupfalse%
\ assms\ \isacommand{by}\isamarkupfalse%
\ {\isacharparenleft}{\kern0pt}typecheck{\isacharunderscore}{\kern0pt}cfuncs{\isacharcomma}{\kern0pt}\ simp\ add{\isacharcolon}{\kern0pt}\ comp{\isacharunderscore}{\kern0pt}as{\isacharunderscore}{\kern0pt}metacomp{\isacharparenright}{\kern0pt}\isanewline
\ \ \isacommand{then}\isamarkupfalse%
\ \isacommand{show}\isamarkupfalse%
\ {\isacharquery}{\kern0pt}thesis\isanewline
\ \ \ \ \isacommand{using}\isamarkupfalse%
\ calculation\ \isacommand{by}\isamarkupfalse%
\ auto\isanewline
\isacommand{qed}\isamarkupfalse%
%
\endisatagproof
{\isafoldproof}%
%
\isadelimproof
\isanewline
%
\endisadelimproof
\isanewline
\isacommand{corollary}\isamarkupfalse%
\ one{\isacharunderscore}{\kern0pt}iter{\isacharcolon}{\kern0pt}\isanewline
\ \ \isakeyword{assumes}\ {\isachardoublequoteopen}g\ {\isacharcolon}{\kern0pt}\ X\ {\isasymrightarrow}\ X{\isachardoublequoteclose}\isanewline
\ \ \isakeyword{shows}\ {\isachardoublequoteopen}g\isactrlbsup {\isasymcirc}{\isacharparenleft}{\kern0pt}successor\ {\isasymcirc}\isactrlsub c\ zero{\isacharparenright}{\kern0pt}\isactrlesup \ {\isacharequal}{\kern0pt}\ g{\isachardoublequoteclose}\isanewline
%
\isadelimproof
\ \ %
\endisadelimproof
%
\isatagproof
\isacommand{using}\isamarkupfalse%
\ assms\ id{\isacharunderscore}{\kern0pt}right{\isacharunderscore}{\kern0pt}unit{\isadigit{2}}\ succ{\isacharunderscore}{\kern0pt}iters\ zero{\isacharunderscore}{\kern0pt}iters\ zero{\isacharunderscore}{\kern0pt}type\ \isacommand{by}\isamarkupfalse%
\ force%
\endisatagproof
{\isafoldproof}%
%
\isadelimproof
\isanewline
%
\endisadelimproof
\isanewline
\isacommand{lemma}\isamarkupfalse%
\ eval{\isacharunderscore}{\kern0pt}lemma{\isacharunderscore}{\kern0pt}for{\isacharunderscore}{\kern0pt}ITER{\isacharcolon}{\kern0pt}\isanewline
\ \ \isakeyword{assumes}\ {\isachardoublequoteopen}f\ {\isacharcolon}{\kern0pt}\ X\ {\isasymrightarrow}\ X{\isachardoublequoteclose}\isanewline
\ \ \isakeyword{assumes}\ {\isachardoublequoteopen}x\ {\isasymin}\isactrlsub c\ X{\isachardoublequoteclose}\isanewline
\ \ \isakeyword{assumes}\ {\isachardoublequoteopen}m\ {\isasymin}\isactrlsub c\ {\isasymnat}\isactrlsub c{\isachardoublequoteclose}\isanewline
\ \ \isakeyword{shows}\ {\isachardoublequoteopen}{\isacharparenleft}{\kern0pt}f\isactrlbsup {\isasymcirc}m\isactrlesup {\isacharparenright}{\kern0pt}\ {\isasymcirc}\isactrlsub c\ x\ {\isacharequal}{\kern0pt}\ eval{\isacharunderscore}{\kern0pt}func\ X\ X\ {\isasymcirc}\isactrlsub c\ {\isasymlangle}x\ {\isacharcomma}{\kern0pt}ITER\ X\ {\isasymcirc}\isactrlsub c\ {\isasymlangle}metafunc\ f\ {\isacharcomma}{\kern0pt}m{\isasymrangle}{\isasymrangle}{\isachardoublequoteclose}\isanewline
%
\isadelimproof
\ \ %
\endisadelimproof
%
\isatagproof
\isacommand{using}\isamarkupfalse%
\ assms\ \isacommand{by}\isamarkupfalse%
\ {\isacharparenleft}{\kern0pt}typecheck{\isacharunderscore}{\kern0pt}cfuncs{\isacharcomma}{\kern0pt}\ metis\ eval{\isacharunderscore}{\kern0pt}lemma\ iter{\isacharunderscore}{\kern0pt}comp{\isacharunderscore}{\kern0pt}def{\isadigit{3}}\ metafunc{\isacharunderscore}{\kern0pt}cnufatem{\isacharparenright}{\kern0pt}%
\endisatagproof
{\isafoldproof}%
%
\isadelimproof
\isanewline
%
\endisadelimproof
\isanewline
\isacommand{lemma}\isamarkupfalse%
\ n{\isacharunderscore}{\kern0pt}accessible{\isacharunderscore}{\kern0pt}by{\isacharunderscore}{\kern0pt}succ{\isacharunderscore}{\kern0pt}iter{\isacharunderscore}{\kern0pt}aux{\isacharcolon}{\kern0pt}\isanewline
\ \ \ {\isachardoublequoteopen}eval{\isacharunderscore}{\kern0pt}func\ {\isasymnat}\isactrlsub c\ {\isasymnat}\isactrlsub c\ {\isasymcirc}\isactrlsub c\ {\isasymlangle}zero\ {\isasymcirc}\isactrlsub c\ {\isasymbeta}\isactrlbsub {\isasymnat}\isactrlsub c\isactrlesub {\isacharcomma}{\kern0pt}\ \ ITER\ {\isasymnat}\isactrlsub c\ {\isasymcirc}\isactrlsub c\ {\isasymlangle}{\isacharparenleft}{\kern0pt}metafunc\ successor{\isacharparenright}{\kern0pt}\ {\isasymcirc}\isactrlsub c\ {\isasymbeta}\isactrlbsub {\isasymnat}\isactrlsub c\isactrlesub \ {\isacharcomma}{\kern0pt}id\ {\isasymnat}\isactrlsub c{\isasymrangle}{\isasymrangle}\ {\isacharequal}{\kern0pt}\ id\ {\isasymnat}\isactrlsub c{\isachardoublequoteclose}\isanewline
%
\isadelimproof
%
\endisadelimproof
%
\isatagproof
\isacommand{proof}\isamarkupfalse%
{\isacharparenleft}{\kern0pt}rule\ natural{\isacharunderscore}{\kern0pt}number{\isacharunderscore}{\kern0pt}object{\isacharunderscore}{\kern0pt}func{\isacharunderscore}{\kern0pt}unique{\isacharbrackleft}{\kern0pt}\isakeyword{where}\ X{\isacharequal}{\kern0pt}{\isachardoublequoteopen}{\isasymnat}\isactrlsub c{\isachardoublequoteclose}{\isacharcomma}{\kern0pt}\ \isakeyword{where}\ f\ {\isacharequal}{\kern0pt}\ successor{\isacharbrackright}{\kern0pt}{\isacharparenright}{\kern0pt}\isanewline
\ \ \isacommand{show}\isamarkupfalse%
\ {\isachardoublequoteopen}eval{\isacharunderscore}{\kern0pt}func\ {\isasymnat}\isactrlsub c\ {\isasymnat}\isactrlsub c\ {\isasymcirc}\isactrlsub c\ {\isasymlangle}zero\ {\isasymcirc}\isactrlsub c\ {\isasymbeta}\isactrlbsub {\isasymnat}\isactrlsub c\isactrlesub {\isacharcomma}{\kern0pt}ITER\ {\isasymnat}\isactrlsub c\ {\isasymcirc}\isactrlsub c\ {\isasymlangle}metafunc\ successor\ {\isasymcirc}\isactrlsub c\ {\isasymbeta}\isactrlbsub {\isasymnat}\isactrlsub c\isactrlesub {\isacharcomma}{\kern0pt}id\isactrlsub c\ {\isasymnat}\isactrlsub c{\isasymrangle}{\isasymrangle}\ {\isacharcolon}{\kern0pt}\ {\isasymnat}\isactrlsub c\ {\isasymrightarrow}\ {\isasymnat}\isactrlsub c{\isachardoublequoteclose}\isanewline
\ \ \ \ \isacommand{by}\isamarkupfalse%
\ typecheck{\isacharunderscore}{\kern0pt}cfuncs\isanewline
\ \ \isacommand{show}\isamarkupfalse%
\ {\isachardoublequoteopen}id\isactrlsub c\ {\isasymnat}\isactrlsub c\ {\isacharcolon}{\kern0pt}\ {\isasymnat}\isactrlsub c\ {\isasymrightarrow}\ {\isasymnat}\isactrlsub c{\isachardoublequoteclose}\isanewline
\ \ \ \ \isacommand{by}\isamarkupfalse%
\ typecheck{\isacharunderscore}{\kern0pt}cfuncs\isanewline
\ \ \isacommand{show}\isamarkupfalse%
\ {\isachardoublequoteopen}successor\ {\isacharcolon}{\kern0pt}\ {\isasymnat}\isactrlsub c\ {\isasymrightarrow}\ {\isasymnat}\isactrlsub c{\isachardoublequoteclose}\isanewline
\ \ \ \ \isacommand{by}\isamarkupfalse%
\ typecheck{\isacharunderscore}{\kern0pt}cfuncs\isanewline
\isacommand{next}\isamarkupfalse%
\isanewline
\ \ \isacommand{have}\isamarkupfalse%
\ {\isachardoublequoteopen}{\isacharparenleft}{\kern0pt}eval{\isacharunderscore}{\kern0pt}func\ {\isasymnat}\isactrlsub c\ {\isasymnat}\isactrlsub c\ {\isasymcirc}\isactrlsub c\ {\isasymlangle}zero\ {\isasymcirc}\isactrlsub c\ {\isasymbeta}\isactrlbsub {\isasymnat}\isactrlsub c\isactrlesub {\isacharcomma}{\kern0pt}ITER\ {\isasymnat}\isactrlsub c\ {\isasymcirc}\isactrlsub c\ {\isasymlangle}metafunc\ successor\ {\isasymcirc}\isactrlsub c\ {\isasymbeta}\isactrlbsub {\isasymnat}\isactrlsub c\isactrlesub {\isacharcomma}{\kern0pt}id\isactrlsub c\ {\isasymnat}\isactrlsub c{\isasymrangle}{\isasymrangle}{\isacharparenright}{\kern0pt}\ {\isasymcirc}\isactrlsub c\ zero\ {\isacharequal}{\kern0pt}\isanewline
\ \ \ \ \ \ \ \ \ eval{\isacharunderscore}{\kern0pt}func\ {\isasymnat}\isactrlsub c\ {\isasymnat}\isactrlsub c\ {\isasymcirc}\isactrlsub c\ {\isasymlangle}zero\ {\isasymcirc}\isactrlsub c\ {\isasymbeta}\isactrlbsub {\isasymnat}\isactrlsub c\isactrlesub \ {\isasymcirc}\isactrlsub c\ zero{\isacharcomma}{\kern0pt}ITER\ {\isasymnat}\isactrlsub c\ {\isasymcirc}\isactrlsub c\ {\isasymlangle}metafunc\ successor\ {\isasymcirc}\isactrlsub c\ {\isasymbeta}\isactrlbsub {\isasymnat}\isactrlsub c\isactrlesub \ {\isasymcirc}\isactrlsub c\ zero{\isacharcomma}{\kern0pt}id\isactrlsub c\ {\isasymnat}\isactrlsub c\ {\isasymcirc}\isactrlsub c\ zero{\isasymrangle}{\isasymrangle}{\isachardoublequoteclose}\isanewline
\ \ \ \ \isacommand{by}\isamarkupfalse%
\ {\isacharparenleft}{\kern0pt}typecheck{\isacharunderscore}{\kern0pt}cfuncs{\isacharcomma}{\kern0pt}\ smt\ {\isacharparenleft}{\kern0pt}z{\isadigit{3}}{\isacharparenright}{\kern0pt}\ cfunc{\isacharunderscore}{\kern0pt}prod{\isacharunderscore}{\kern0pt}comp\ comp{\isacharunderscore}{\kern0pt}associative{\isadigit{2}}{\isacharparenright}{\kern0pt}\isanewline
\ \ \isacommand{also}\isamarkupfalse%
\ \isacommand{have}\isamarkupfalse%
\ {\isachardoublequoteopen}{\isachardot}{\kern0pt}{\isachardot}{\kern0pt}{\isachardot}{\kern0pt}\ {\isacharequal}{\kern0pt}\ \ eval{\isacharunderscore}{\kern0pt}func\ {\isasymnat}\isactrlsub c\ {\isasymnat}\isactrlsub c\ {\isasymcirc}\isactrlsub c\ {\isasymlangle}zero{\isacharcomma}{\kern0pt}ITER\ {\isasymnat}\isactrlsub c\ {\isasymcirc}\isactrlsub c\ {\isasymlangle}metafunc\ successor{\isacharcomma}{\kern0pt}zero{\isasymrangle}{\isasymrangle}{\isachardoublequoteclose}\isanewline
\ \ \ \ \isacommand{by}\isamarkupfalse%
\ {\isacharparenleft}{\kern0pt}typecheck{\isacharunderscore}{\kern0pt}cfuncs{\isacharcomma}{\kern0pt}\ simp\ add{\isacharcolon}{\kern0pt}\ id{\isacharunderscore}{\kern0pt}left{\isacharunderscore}{\kern0pt}unit{\isadigit{2}}\ id{\isacharunderscore}{\kern0pt}right{\isacharunderscore}{\kern0pt}unit{\isadigit{2}}\ terminal{\isacharunderscore}{\kern0pt}func{\isacharunderscore}{\kern0pt}comp{\isacharunderscore}{\kern0pt}elem{\isacharparenright}{\kern0pt}\isanewline
\ \ \isacommand{also}\isamarkupfalse%
\ \isacommand{have}\isamarkupfalse%
\ {\isachardoublequoteopen}{\isachardot}{\kern0pt}{\isachardot}{\kern0pt}{\isachardot}{\kern0pt}\ {\isacharequal}{\kern0pt}\ \ eval{\isacharunderscore}{\kern0pt}func\ {\isasymnat}\isactrlsub c\ {\isasymnat}\isactrlsub c\ {\isasymcirc}\isactrlsub c\ {\isasymlangle}zero{\isacharcomma}{\kern0pt}metafunc\ {\isacharparenleft}{\kern0pt}id\ {\isasymnat}\isactrlsub c{\isacharparenright}{\kern0pt}\ {\isasymrangle}{\isachardoublequoteclose}\isanewline
\ \ \ \ \isacommand{by}\isamarkupfalse%
\ {\isacharparenleft}{\kern0pt}typecheck{\isacharunderscore}{\kern0pt}cfuncs{\isacharcomma}{\kern0pt}\ simp\ add{\isacharcolon}{\kern0pt}\ ITER{\isacharunderscore}{\kern0pt}zero{\isacharprime}{\kern0pt}{\isacharparenright}{\kern0pt}\isanewline
\ \ \isacommand{also}\isamarkupfalse%
\ \isacommand{have}\isamarkupfalse%
\ {\isachardoublequoteopen}{\isachardot}{\kern0pt}{\isachardot}{\kern0pt}{\isachardot}{\kern0pt}\ {\isacharequal}{\kern0pt}\ id\isactrlsub c\ {\isasymnat}\isactrlsub c\ {\isasymcirc}\isactrlsub c\ zero{\isachardoublequoteclose}\isanewline
\ \ \ \ \isacommand{using}\isamarkupfalse%
\ eval{\isacharunderscore}{\kern0pt}lemma\ \isacommand{by}\isamarkupfalse%
\ {\isacharparenleft}{\kern0pt}typecheck{\isacharunderscore}{\kern0pt}cfuncs{\isacharcomma}{\kern0pt}\ blast{\isacharparenright}{\kern0pt}\isanewline
\ \ \isacommand{then}\isamarkupfalse%
\ \isacommand{show}\isamarkupfalse%
\ {\isachardoublequoteopen}{\isacharparenleft}{\kern0pt}eval{\isacharunderscore}{\kern0pt}func\ {\isasymnat}\isactrlsub c\ {\isasymnat}\isactrlsub c\ {\isasymcirc}\isactrlsub c\ {\isasymlangle}zero\ {\isasymcirc}\isactrlsub c\ {\isasymbeta}\isactrlbsub {\isasymnat}\isactrlsub c\isactrlesub {\isacharcomma}{\kern0pt}ITER\ {\isasymnat}\isactrlsub c\ {\isasymcirc}\isactrlsub c\ {\isasymlangle}metafunc\ successor\ {\isasymcirc}\isactrlsub c\ {\isasymbeta}\isactrlbsub {\isasymnat}\isactrlsub c\isactrlesub {\isacharcomma}{\kern0pt}id\isactrlsub c\ {\isasymnat}\isactrlsub c{\isasymrangle}{\isasymrangle}{\isacharparenright}{\kern0pt}\ {\isasymcirc}\isactrlsub c\ zero\ {\isacharequal}{\kern0pt}\ id\isactrlsub c\ {\isasymnat}\isactrlsub c\ {\isasymcirc}\isactrlsub c\ zero{\isachardoublequoteclose}\isanewline
\ \ \ \ \isacommand{using}\isamarkupfalse%
\ calculation\ \isacommand{by}\isamarkupfalse%
\ auto\isanewline
\ \ \isacommand{show}\isamarkupfalse%
\ {\isachardoublequoteopen}{\isacharparenleft}{\kern0pt}eval{\isacharunderscore}{\kern0pt}func\ {\isasymnat}\isactrlsub c\ {\isasymnat}\isactrlsub c\ {\isasymcirc}\isactrlsub c\ {\isasymlangle}zero\ {\isasymcirc}\isactrlsub c\ {\isasymbeta}\isactrlbsub {\isasymnat}\isactrlsub c\isactrlesub {\isacharcomma}{\kern0pt}ITER\ {\isasymnat}\isactrlsub c\ {\isasymcirc}\isactrlsub c\ {\isasymlangle}metafunc\ successor\ {\isasymcirc}\isactrlsub c\ {\isasymbeta}\isactrlbsub {\isasymnat}\isactrlsub c\isactrlesub {\isacharcomma}{\kern0pt}id\isactrlsub c\ {\isasymnat}\isactrlsub c{\isasymrangle}{\isasymrangle}{\isacharparenright}{\kern0pt}\ {\isasymcirc}\isactrlsub c\ successor\ {\isacharequal}{\kern0pt}\isanewline
\ \ \ \ successor\ {\isasymcirc}\isactrlsub c\ eval{\isacharunderscore}{\kern0pt}func\ {\isasymnat}\isactrlsub c\ {\isasymnat}\isactrlsub c\ {\isasymcirc}\isactrlsub c\ {\isasymlangle}zero\ {\isasymcirc}\isactrlsub c\ {\isasymbeta}\isactrlbsub {\isasymnat}\isactrlsub c\isactrlesub {\isacharcomma}{\kern0pt}ITER\ {\isasymnat}\isactrlsub c\ {\isasymcirc}\isactrlsub c\ {\isasymlangle}metafunc\ successor\ {\isasymcirc}\isactrlsub c\ {\isasymbeta}\isactrlbsub {\isasymnat}\isactrlsub c\isactrlesub {\isacharcomma}{\kern0pt}id\isactrlsub c\ {\isasymnat}\isactrlsub c{\isasymrangle}{\isasymrangle}{\isachardoublequoteclose}\isanewline
\ \ \isacommand{proof}\isamarkupfalse%
{\isacharparenleft}{\kern0pt}etcs{\isacharunderscore}{\kern0pt}rule\ one{\isacharunderscore}{\kern0pt}separator{\isacharparenright}{\kern0pt}\isanewline
\ \ \ \ \isacommand{fix}\isamarkupfalse%
\ m\isanewline
\ \ \ \ \isacommand{assume}\isamarkupfalse%
\ m{\isacharunderscore}{\kern0pt}type{\isacharbrackleft}{\kern0pt}type{\isacharunderscore}{\kern0pt}rule{\isacharbrackright}{\kern0pt}{\isacharcolon}{\kern0pt}\ {\isachardoublequoteopen}m\ {\isasymin}\isactrlsub c\ {\isasymnat}\isactrlsub c{\isachardoublequoteclose}\isanewline
\ \ \ \ \isacommand{have}\isamarkupfalse%
\ {\isachardoublequoteopen}{\isacharparenleft}{\kern0pt}successor\ {\isasymcirc}\isactrlsub c\ eval{\isacharunderscore}{\kern0pt}func\ {\isasymnat}\isactrlsub c\ {\isasymnat}\isactrlsub c\ {\isasymcirc}\isactrlsub c\ {\isasymlangle}zero\ {\isasymcirc}\isactrlsub c\ {\isasymbeta}\isactrlbsub {\isasymnat}\isactrlsub c\isactrlesub {\isacharcomma}{\kern0pt}ITER\ {\isasymnat}\isactrlsub c\ {\isasymcirc}\isactrlsub c\ {\isasymlangle}metafunc\ successor\ {\isasymcirc}\isactrlsub c\ {\isasymbeta}\isactrlbsub {\isasymnat}\isactrlsub c\isactrlesub {\isacharcomma}{\kern0pt}id\isactrlsub c\ {\isasymnat}\isactrlsub c{\isasymrangle}{\isasymrangle}{\isacharparenright}{\kern0pt}\ {\isasymcirc}\isactrlsub c\ m\ {\isacharequal}{\kern0pt}\ \isanewline
\ \ \ \ \ \ \ \ \ successor\ {\isasymcirc}\isactrlsub c\ eval{\isacharunderscore}{\kern0pt}func\ {\isasymnat}\isactrlsub c\ {\isasymnat}\isactrlsub c\ {\isasymcirc}\isactrlsub c\ {\isasymlangle}zero\ {\isasymcirc}\isactrlsub c\ {\isasymbeta}\isactrlbsub {\isasymnat}\isactrlsub c\isactrlesub \ {\isasymcirc}\isactrlsub c\ m{\isacharcomma}{\kern0pt}ITER\ {\isasymnat}\isactrlsub c\ {\isasymcirc}\isactrlsub c\ {\isasymlangle}metafunc\ successor\ {\isasymcirc}\isactrlsub c\ {\isasymbeta}\isactrlbsub {\isasymnat}\isactrlsub c\isactrlesub \ {\isasymcirc}\isactrlsub c\ m{\isacharcomma}{\kern0pt}id\isactrlsub c\ {\isasymnat}\isactrlsub c\ {\isasymcirc}\isactrlsub c\ m{\isasymrangle}{\isasymrangle}{\isachardoublequoteclose}\isanewline
\ \ \ \ \ \ \isacommand{by}\isamarkupfalse%
\ {\isacharparenleft}{\kern0pt}typecheck{\isacharunderscore}{\kern0pt}cfuncs{\isacharcomma}{\kern0pt}\ smt\ {\isacharparenleft}{\kern0pt}z{\isadigit{3}}{\isacharparenright}{\kern0pt}\ cfunc{\isacharunderscore}{\kern0pt}prod{\isacharunderscore}{\kern0pt}comp\ comp{\isacharunderscore}{\kern0pt}associative{\isadigit{2}}{\isacharparenright}{\kern0pt}\isanewline
\ \ \ \ \isacommand{also}\isamarkupfalse%
\ \isacommand{have}\isamarkupfalse%
\ {\isachardoublequoteopen}{\isachardot}{\kern0pt}{\isachardot}{\kern0pt}{\isachardot}{\kern0pt}\ {\isacharequal}{\kern0pt}\ successor\ {\isasymcirc}\isactrlsub c\ eval{\isacharunderscore}{\kern0pt}func\ {\isasymnat}\isactrlsub c\ {\isasymnat}\isactrlsub c\ {\isasymcirc}\isactrlsub c\ {\isasymlangle}zero\ {\isacharcomma}{\kern0pt}ITER\ {\isasymnat}\isactrlsub c\ {\isasymcirc}\isactrlsub c\ {\isasymlangle}metafunc\ successor\ {\isacharcomma}{\kern0pt}m{\isasymrangle}{\isasymrangle}{\isachardoublequoteclose}\isanewline
\ \ \ \ \ \ \isacommand{by}\isamarkupfalse%
\ {\isacharparenleft}{\kern0pt}typecheck{\isacharunderscore}{\kern0pt}cfuncs{\isacharcomma}{\kern0pt}\ simp\ add{\isacharcolon}{\kern0pt}\ id{\isacharunderscore}{\kern0pt}left{\isacharunderscore}{\kern0pt}unit{\isadigit{2}}\ id{\isacharunderscore}{\kern0pt}right{\isacharunderscore}{\kern0pt}unit{\isadigit{2}}\ terminal{\isacharunderscore}{\kern0pt}func{\isacharunderscore}{\kern0pt}comp{\isacharunderscore}{\kern0pt}elem{\isacharparenright}{\kern0pt}\isanewline
\ \ \ \ \isacommand{also}\isamarkupfalse%
\ \isacommand{have}\isamarkupfalse%
\ {\isachardoublequoteopen}{\isachardot}{\kern0pt}{\isachardot}{\kern0pt}{\isachardot}{\kern0pt}\ {\isacharequal}{\kern0pt}\ successor\ {\isasymcirc}\isactrlsub c\ {\isacharparenleft}{\kern0pt}successor\isactrlbsup {\isasymcirc}m\isactrlesup {\isacharparenright}{\kern0pt}\ {\isasymcirc}\isactrlsub c\ zero{\isachardoublequoteclose}\isanewline
\ \ \ \ \ \ \isacommand{by}\isamarkupfalse%
\ {\isacharparenleft}{\kern0pt}typecheck{\isacharunderscore}{\kern0pt}cfuncs{\isacharcomma}{\kern0pt}\ simp\ add{\isacharcolon}{\kern0pt}\ eval{\isacharunderscore}{\kern0pt}lemma{\isacharunderscore}{\kern0pt}for{\isacharunderscore}{\kern0pt}ITER{\isacharparenright}{\kern0pt}\isanewline
\ \ \ \ \isacommand{also}\isamarkupfalse%
\ \isacommand{have}\isamarkupfalse%
\ {\isachardoublequoteopen}{\isachardot}{\kern0pt}{\isachardot}{\kern0pt}{\isachardot}{\kern0pt}\ {\isacharequal}{\kern0pt}\ {\isacharparenleft}{\kern0pt}successor\isactrlbsup {\isasymcirc}successor\ {\isasymcirc}\isactrlsub c\ m\isactrlesup {\isacharparenright}{\kern0pt}\ {\isasymcirc}\isactrlsub c\ zero{\isachardoublequoteclose}\isanewline
\ \ \ \ \ \ \isacommand{by}\isamarkupfalse%
\ {\isacharparenleft}{\kern0pt}typecheck{\isacharunderscore}{\kern0pt}cfuncs{\isacharcomma}{\kern0pt}\ simp\ add{\isacharcolon}{\kern0pt}\ comp{\isacharunderscore}{\kern0pt}associative{\isadigit{2}}\ succ{\isacharunderscore}{\kern0pt}iters{\isacharparenright}{\kern0pt}\isanewline
\ \ \ \ \isacommand{also}\isamarkupfalse%
\ \isacommand{have}\isamarkupfalse%
\ {\isachardoublequoteopen}{\isachardot}{\kern0pt}{\isachardot}{\kern0pt}{\isachardot}{\kern0pt}\ {\isacharequal}{\kern0pt}\ eval{\isacharunderscore}{\kern0pt}func\ {\isasymnat}\isactrlsub c\ {\isasymnat}\isactrlsub c\ {\isasymcirc}\isactrlsub c\ {\isasymlangle}zero\ {\isacharcomma}{\kern0pt}ITER\ {\isasymnat}\isactrlsub c\ {\isasymcirc}\isactrlsub c\ {\isasymlangle}metafunc\ successor\ {\isacharcomma}{\kern0pt}successor\ {\isasymcirc}\isactrlsub c\ m{\isasymrangle}{\isasymrangle}{\isachardoublequoteclose}\isanewline
\ \ \ \ \ \ \isacommand{by}\isamarkupfalse%
\ {\isacharparenleft}{\kern0pt}typecheck{\isacharunderscore}{\kern0pt}cfuncs{\isacharcomma}{\kern0pt}\ simp\ add{\isacharcolon}{\kern0pt}\ eval{\isacharunderscore}{\kern0pt}lemma{\isacharunderscore}{\kern0pt}for{\isacharunderscore}{\kern0pt}ITER{\isacharparenright}{\kern0pt}\isanewline
\ \ \ \ \isacommand{also}\isamarkupfalse%
\ \isacommand{have}\isamarkupfalse%
\ {\isachardoublequoteopen}{\isachardot}{\kern0pt}{\isachardot}{\kern0pt}{\isachardot}{\kern0pt}\ {\isacharequal}{\kern0pt}\ eval{\isacharunderscore}{\kern0pt}func\ {\isasymnat}\isactrlsub c\ {\isasymnat}\isactrlsub c\ {\isasymcirc}\isactrlsub c\ {\isasymlangle}zero\ {\isasymcirc}\isactrlsub c\ {\isasymbeta}\isactrlbsub {\isasymnat}\isactrlsub c\isactrlesub \ {\isasymcirc}\isactrlsub c\ {\isacharparenleft}{\kern0pt}successor\ {\isasymcirc}\isactrlsub c\ m{\isacharparenright}{\kern0pt}{\isacharcomma}{\kern0pt}ITER\ {\isasymnat}\isactrlsub c\ {\isasymcirc}\isactrlsub c\ {\isasymlangle}metafunc\ successor\ {\isasymcirc}\isactrlsub c\ {\isasymbeta}\isactrlbsub {\isasymnat}\isactrlsub c\isactrlesub {\isasymcirc}\isactrlsub c\ {\isacharparenleft}{\kern0pt}successor\ {\isasymcirc}\isactrlsub c\ m{\isacharparenright}{\kern0pt}{\isacharcomma}{\kern0pt}id\isactrlsub c\ {\isasymnat}\isactrlsub c\ {\isasymcirc}\isactrlsub c\ {\isacharparenleft}{\kern0pt}successor\ {\isasymcirc}\isactrlsub c\ m{\isacharparenright}{\kern0pt}{\isasymrangle}{\isasymrangle}{\isachardoublequoteclose}\isanewline
\ \ \ \ \ \ \isacommand{by}\isamarkupfalse%
\ {\isacharparenleft}{\kern0pt}typecheck{\isacharunderscore}{\kern0pt}cfuncs{\isacharcomma}{\kern0pt}simp\ add{\isacharcolon}{\kern0pt}\ id{\isacharunderscore}{\kern0pt}left{\isacharunderscore}{\kern0pt}unit{\isadigit{2}}\ id{\isacharunderscore}{\kern0pt}right{\isacharunderscore}{\kern0pt}unit{\isadigit{2}}\ terminal{\isacharunderscore}{\kern0pt}func{\isacharunderscore}{\kern0pt}comp{\isacharunderscore}{\kern0pt}elem{\isacharparenright}{\kern0pt}\isanewline
\ \ \ \ \isacommand{also}\isamarkupfalse%
\ \isacommand{have}\isamarkupfalse%
\ {\isachardoublequoteopen}{\isachardot}{\kern0pt}{\isachardot}{\kern0pt}{\isachardot}{\kern0pt}\ {\isacharequal}{\kern0pt}\ {\isacharparenleft}{\kern0pt}{\isacharparenleft}{\kern0pt}eval{\isacharunderscore}{\kern0pt}func\ {\isasymnat}\isactrlsub c\ {\isasymnat}\isactrlsub c\ {\isasymcirc}\isactrlsub c\ {\isasymlangle}zero\ {\isasymcirc}\isactrlsub c\ {\isasymbeta}\isactrlbsub {\isasymnat}\isactrlsub c\isactrlesub {\isacharcomma}{\kern0pt}ITER\ {\isasymnat}\isactrlsub c\ {\isasymcirc}\isactrlsub c\ {\isasymlangle}metafunc\ successor\ {\isasymcirc}\isactrlsub c\ {\isasymbeta}\isactrlbsub {\isasymnat}\isactrlsub c\isactrlesub {\isacharcomma}{\kern0pt}id\isactrlsub c\ {\isasymnat}\isactrlsub c{\isasymrangle}{\isasymrangle}{\isacharparenright}{\kern0pt}\ {\isasymcirc}\isactrlsub c\ successor{\isacharparenright}{\kern0pt}\ {\isasymcirc}\isactrlsub c\ m{\isachardoublequoteclose}\isanewline
\ \ \ \ \ \ \isacommand{by}\isamarkupfalse%
\ {\isacharparenleft}{\kern0pt}typecheck{\isacharunderscore}{\kern0pt}cfuncs{\isacharcomma}{\kern0pt}\ smt\ {\isacharparenleft}{\kern0pt}z{\isadigit{3}}{\isacharparenright}{\kern0pt}\ cfunc{\isacharunderscore}{\kern0pt}prod{\isacharunderscore}{\kern0pt}comp\ comp{\isacharunderscore}{\kern0pt}associative{\isadigit{2}}{\isacharparenright}{\kern0pt}\isanewline
\ \ \ \ \isacommand{then}\isamarkupfalse%
\ \isacommand{show}\isamarkupfalse%
\ {\isachardoublequoteopen}{\isacharparenleft}{\kern0pt}{\isacharparenleft}{\kern0pt}eval{\isacharunderscore}{\kern0pt}func\ {\isasymnat}\isactrlsub c\ {\isasymnat}\isactrlsub c\ {\isasymcirc}\isactrlsub c\ {\isasymlangle}zero\ {\isasymcirc}\isactrlsub c\ {\isasymbeta}\isactrlbsub {\isasymnat}\isactrlsub c\isactrlesub {\isacharcomma}{\kern0pt}ITER\ {\isasymnat}\isactrlsub c\ {\isasymcirc}\isactrlsub c\ {\isasymlangle}metafunc\ successor\ {\isasymcirc}\isactrlsub c\ {\isasymbeta}\isactrlbsub {\isasymnat}\isactrlsub c\isactrlesub {\isacharcomma}{\kern0pt}id\isactrlsub c\ {\isasymnat}\isactrlsub c{\isasymrangle}{\isasymrangle}{\isacharparenright}{\kern0pt}\ {\isasymcirc}\isactrlsub c\ successor{\isacharparenright}{\kern0pt}\ {\isasymcirc}\isactrlsub c\ m\ {\isacharequal}{\kern0pt}\isanewline
\ \ \ \ \ \ \ \ \ {\isacharparenleft}{\kern0pt}successor\ {\isasymcirc}\isactrlsub c\ eval{\isacharunderscore}{\kern0pt}func\ {\isasymnat}\isactrlsub c\ {\isasymnat}\isactrlsub c\ {\isasymcirc}\isactrlsub c\ {\isasymlangle}zero\ {\isasymcirc}\isactrlsub c\ {\isasymbeta}\isactrlbsub {\isasymnat}\isactrlsub c\isactrlesub {\isacharcomma}{\kern0pt}ITER\ {\isasymnat}\isactrlsub c\ {\isasymcirc}\isactrlsub c\ {\isasymlangle}metafunc\ successor\ {\isasymcirc}\isactrlsub c\ {\isasymbeta}\isactrlbsub {\isasymnat}\isactrlsub c\isactrlesub {\isacharcomma}{\kern0pt}id\isactrlsub c\ {\isasymnat}\isactrlsub c{\isasymrangle}{\isasymrangle}{\isacharparenright}{\kern0pt}\ {\isasymcirc}\isactrlsub c\ m{\isachardoublequoteclose}\isanewline
\ \ \ \ \ \ \isacommand{using}\isamarkupfalse%
\ calculation\ \isacommand{by}\isamarkupfalse%
\ presburger\isanewline
\ \ \isacommand{qed}\isamarkupfalse%
\isanewline
\ \ \isacommand{show}\isamarkupfalse%
\ {\isachardoublequoteopen}id\isactrlsub c\ {\isasymnat}\isactrlsub c\ {\isasymcirc}\isactrlsub c\ successor\ {\isacharequal}{\kern0pt}\ successor\ {\isasymcirc}\isactrlsub c\ id\isactrlsub c\ {\isasymnat}\isactrlsub c{\isachardoublequoteclose}\isanewline
\ \ \ \ \isacommand{by}\isamarkupfalse%
\ {\isacharparenleft}{\kern0pt}typecheck{\isacharunderscore}{\kern0pt}cfuncs{\isacharcomma}{\kern0pt}\ simp\ add{\isacharcolon}{\kern0pt}\ id{\isacharunderscore}{\kern0pt}left{\isacharunderscore}{\kern0pt}unit{\isadigit{2}}\ id{\isacharunderscore}{\kern0pt}right{\isacharunderscore}{\kern0pt}unit{\isadigit{2}}{\isacharparenright}{\kern0pt}\isanewline
\isacommand{qed}\isamarkupfalse%
%
\endisatagproof
{\isafoldproof}%
%
\isadelimproof
\isanewline
%
\endisadelimproof
\isanewline
\isacommand{lemma}\isamarkupfalse%
\ n{\isacharunderscore}{\kern0pt}accessible{\isacharunderscore}{\kern0pt}by{\isacharunderscore}{\kern0pt}succ{\isacharunderscore}{\kern0pt}iter{\isacharcolon}{\kern0pt}\isanewline
\ \ \isakeyword{assumes}\ {\isachardoublequoteopen}n\ {\isasymin}\isactrlsub c\ {\isasymnat}\isactrlsub c{\isachardoublequoteclose}\isanewline
\ \ \isakeyword{shows}\ {\isachardoublequoteopen}{\isacharparenleft}{\kern0pt}successor\isactrlbsup {\isasymcirc}n\isactrlesup {\isacharparenright}{\kern0pt}\ {\isasymcirc}\isactrlsub c\ zero\ {\isacharequal}{\kern0pt}\ n{\isachardoublequoteclose}\isanewline
%
\isadelimproof
%
\endisadelimproof
%
\isatagproof
\isacommand{proof}\isamarkupfalse%
\ {\isacharminus}{\kern0pt}\ \isanewline
\ \ \isacommand{have}\isamarkupfalse%
\ {\isachardoublequoteopen}n\ {\isacharequal}{\kern0pt}\ eval{\isacharunderscore}{\kern0pt}func\ {\isasymnat}\isactrlsub c\ {\isasymnat}\isactrlsub c\ {\isasymcirc}\isactrlsub c\ {\isasymlangle}zero\ {\isasymcirc}\isactrlsub c\ {\isasymbeta}\isactrlbsub {\isasymnat}\isactrlsub c\isactrlesub {\isacharcomma}{\kern0pt}\ ITER\ {\isasymnat}\isactrlsub c\ {\isasymcirc}\isactrlsub c\ {\isasymlangle}metafunc\ successor\ {\isasymcirc}\isactrlsub c\ {\isasymbeta}\isactrlbsub {\isasymnat}\isactrlsub c\isactrlesub {\isacharcomma}{\kern0pt}\ id\ {\isasymnat}\isactrlsub c{\isasymrangle}{\isasymrangle}\ {\isasymcirc}\isactrlsub c\ n{\isachardoublequoteclose}\isanewline
\ \ \ \ \isacommand{using}\isamarkupfalse%
\ assms\ \isacommand{by}\isamarkupfalse%
\ {\isacharparenleft}{\kern0pt}typecheck{\isacharunderscore}{\kern0pt}cfuncs{\isacharcomma}{\kern0pt}\ simp\ add{\isacharcolon}{\kern0pt}\ comp{\isacharunderscore}{\kern0pt}associative{\isadigit{2}}\ id{\isacharunderscore}{\kern0pt}left{\isacharunderscore}{\kern0pt}unit{\isadigit{2}}\ n{\isacharunderscore}{\kern0pt}accessible{\isacharunderscore}{\kern0pt}by{\isacharunderscore}{\kern0pt}succ{\isacharunderscore}{\kern0pt}iter{\isacharunderscore}{\kern0pt}aux{\isacharparenright}{\kern0pt}\isanewline
\ \ \isacommand{also}\isamarkupfalse%
\ \isacommand{have}\isamarkupfalse%
\ {\isachardoublequoteopen}{\isachardot}{\kern0pt}{\isachardot}{\kern0pt}{\isachardot}{\kern0pt}\ {\isacharequal}{\kern0pt}\ eval{\isacharunderscore}{\kern0pt}func\ {\isasymnat}\isactrlsub c\ {\isasymnat}\isactrlsub c\ {\isasymcirc}\isactrlsub c\ {\isasymlangle}zero\ {\isasymcirc}\isactrlsub c\ {\isasymbeta}\isactrlbsub {\isasymnat}\isactrlsub c\isactrlesub \ {\isasymcirc}\isactrlsub c\ n\ {\isacharcomma}{\kern0pt}\ ITER\ {\isasymnat}\isactrlsub c\ {\isasymcirc}\isactrlsub c\ {\isasymlangle}metafunc\ successor\ {\isasymcirc}\isactrlsub c\ {\isasymbeta}\isactrlbsub {\isasymnat}\isactrlsub c\isactrlesub \ {\isasymcirc}\isactrlsub c\ n{\isacharcomma}{\kern0pt}\ id\ {\isasymnat}\isactrlsub c\ {\isasymcirc}\isactrlsub c\ n{\isasymrangle}{\isasymrangle}{\isachardoublequoteclose}\isanewline
\ \ \ \ \isacommand{using}\isamarkupfalse%
\ assms\ \isacommand{by}\isamarkupfalse%
\ {\isacharparenleft}{\kern0pt}typecheck{\isacharunderscore}{\kern0pt}cfuncs{\isacharcomma}{\kern0pt}\ smt\ {\isacharparenleft}{\kern0pt}z{\isadigit{3}}{\isacharparenright}{\kern0pt}\ cfunc{\isacharunderscore}{\kern0pt}prod{\isacharunderscore}{\kern0pt}comp\ comp{\isacharunderscore}{\kern0pt}associative{\isadigit{2}}{\isacharparenright}{\kern0pt}\isanewline
\ \ \isacommand{also}\isamarkupfalse%
\ \isacommand{have}\isamarkupfalse%
\ {\isachardoublequoteopen}{\isachardot}{\kern0pt}{\isachardot}{\kern0pt}{\isachardot}{\kern0pt}\ {\isacharequal}{\kern0pt}\ eval{\isacharunderscore}{\kern0pt}func\ {\isasymnat}\isactrlsub c\ {\isasymnat}\isactrlsub c\ {\isasymcirc}\isactrlsub c\ {\isasymlangle}zero{\isacharcomma}{\kern0pt}\ \ ITER\ {\isasymnat}\isactrlsub c\ {\isasymcirc}\isactrlsub c\ {\isasymlangle}metafunc\ successor{\isacharcomma}{\kern0pt}\ n{\isasymrangle}{\isasymrangle}{\isachardoublequoteclose}\isanewline
\ \ \ \ \isacommand{using}\isamarkupfalse%
\ assms\ \isacommand{by}\isamarkupfalse%
\ {\isacharparenleft}{\kern0pt}typecheck{\isacharunderscore}{\kern0pt}cfuncs{\isacharcomma}{\kern0pt}\ simp\ add{\isacharcolon}{\kern0pt}\ id{\isacharunderscore}{\kern0pt}left{\isacharunderscore}{\kern0pt}unit{\isadigit{2}}\ id{\isacharunderscore}{\kern0pt}right{\isacharunderscore}{\kern0pt}unit{\isadigit{2}}\ terminal{\isacharunderscore}{\kern0pt}func{\isacharunderscore}{\kern0pt}comp{\isacharunderscore}{\kern0pt}elem{\isacharparenright}{\kern0pt}\isanewline
\ \ \isacommand{also}\isamarkupfalse%
\ \isacommand{have}\isamarkupfalse%
\ {\isachardoublequoteopen}{\isachardot}{\kern0pt}{\isachardot}{\kern0pt}{\isachardot}{\kern0pt}\ {\isacharequal}{\kern0pt}\ {\isacharparenleft}{\kern0pt}successor\isactrlbsup {\isasymcirc}n\isactrlesup {\isacharparenright}{\kern0pt}\ {\isasymcirc}\isactrlsub c\ zero{\isachardoublequoteclose}\isanewline
\ \ \ \ \isacommand{using}\isamarkupfalse%
\ assms\ \isacommand{by}\isamarkupfalse%
\ {\isacharparenleft}{\kern0pt}typecheck{\isacharunderscore}{\kern0pt}cfuncs{\isacharcomma}{\kern0pt}\ metis\ eval{\isacharunderscore}{\kern0pt}lemma\ iter{\isacharunderscore}{\kern0pt}comp{\isacharunderscore}{\kern0pt}def{\isadigit{3}}\ metafunc{\isacharunderscore}{\kern0pt}cnufatem{\isacharparenright}{\kern0pt}\isanewline
\ \ \isacommand{then}\isamarkupfalse%
\ \isacommand{show}\isamarkupfalse%
\ {\isacharquery}{\kern0pt}thesis\isanewline
\ \ \ \ \isacommand{using}\isamarkupfalse%
\ calculation\ \isacommand{by}\isamarkupfalse%
\ auto\isanewline
\isacommand{qed}\isamarkupfalse%
%
\endisatagproof
{\isafoldproof}%
%
\isadelimproof
%
\endisadelimproof
%
\isadelimdocument
%
\endisadelimdocument
%
\isatagdocument
%
\isamarkupsubsection{Relation of Nat to Other Sets%
}
\isamarkuptrue%
%
\endisatagdocument
{\isafolddocument}%
%
\isadelimdocument
%
\endisadelimdocument
\isacommand{lemma}\isamarkupfalse%
\ oneUN{\isacharunderscore}{\kern0pt}iso{\isacharunderscore}{\kern0pt}N{\isacharcolon}{\kern0pt}\isanewline
\ \ {\isachardoublequoteopen}{\isasymone}\ {\isasymCoprod}\ {\isasymnat}\isactrlsub c\ {\isasymcong}\ {\isasymnat}\isactrlsub c{\isachardoublequoteclose}\isanewline
%
\isadelimproof
\ \ %
\endisadelimproof
%
\isatagproof
\isacommand{using}\isamarkupfalse%
\ cfunc{\isacharunderscore}{\kern0pt}coprod{\isacharunderscore}{\kern0pt}type\ is{\isacharunderscore}{\kern0pt}isomorphic{\isacharunderscore}{\kern0pt}def\ oneUN{\isacharunderscore}{\kern0pt}iso{\isacharunderscore}{\kern0pt}N{\isacharunderscore}{\kern0pt}isomorphism\ successor{\isacharunderscore}{\kern0pt}type\ zero{\isacharunderscore}{\kern0pt}type\ \isacommand{by}\isamarkupfalse%
\ blast%
\endisatagproof
{\isafoldproof}%
%
\isadelimproof
\isanewline
%
\endisadelimproof
\isanewline
\isacommand{lemma}\isamarkupfalse%
\ NUone{\isacharunderscore}{\kern0pt}iso{\isacharunderscore}{\kern0pt}N{\isacharcolon}{\kern0pt}\isanewline
\ \ {\isachardoublequoteopen}{\isasymnat}\isactrlsub c\ {\isasymCoprod}\ {\isasymone}\ {\isasymcong}\ {\isasymnat}\isactrlsub c{\isachardoublequoteclose}\isanewline
%
\isadelimproof
\ \ %
\endisadelimproof
%
\isatagproof
\isacommand{using}\isamarkupfalse%
\ coproduct{\isacharunderscore}{\kern0pt}commutes\ isomorphic{\isacharunderscore}{\kern0pt}is{\isacharunderscore}{\kern0pt}transitive\ oneUN{\isacharunderscore}{\kern0pt}iso{\isacharunderscore}{\kern0pt}N\ \isacommand{by}\isamarkupfalse%
\ blast%
\endisatagproof
{\isafoldproof}%
%
\isadelimproof
\isanewline
%
\endisadelimproof
%
\isadelimtheory
\ \ \isanewline
%
\endisadelimtheory
%
\isatagtheory
\isacommand{end}\isamarkupfalse%
%
\endisatagtheory
{\isafoldtheory}%
%
\isadelimtheory
%
\endisadelimtheory
%
\end{isabellebody}%
\endinput
%:%file=~/ETCS/HOL-ETCS/Nats.thy%:%
%:%11=1%:%
%:%27=3%:%
%:%28=3%:%
%:%29=4%:%
%:%30=5%:%
%:%39=7%:%
%:%41=8%:%
%:%42=8%:%
%:%43=9%:%
%:%44=10%:%
%:%45=11%:%
%:%46=12%:%
%:%47=13%:%
%:%48=14%:%
%:%49=15%:%
%:%50=16%:%
%:%53=19%:%
%:%54=20%:%
%:%55=21%:%
%:%56=21%:%
%:%57=22%:%
%:%58=23%:%
%:%61=24%:%
%:%65=24%:%
%:%66=24%:%
%:%71=24%:%
%:%74=25%:%
%:%75=26%:%
%:%76=26%:%
%:%77=27%:%
%:%78=28%:%
%:%81=29%:%
%:%85=29%:%
%:%86=29%:%
%:%87=30%:%
%:%88=30%:%
%:%93=30%:%
%:%96=31%:%
%:%97=32%:%
%:%98=32%:%
%:%99=33%:%
%:%100=34%:%
%:%101=35%:%
%:%102=36%:%
%:%103=37%:%
%:%106=38%:%
%:%110=38%:%
%:%111=38%:%
%:%116=38%:%
%:%119=39%:%
%:%120=40%:%
%:%121=40%:%
%:%122=41%:%
%:%125=44%:%
%:%126=45%:%
%:%127=46%:%
%:%128=46%:%
%:%129=47%:%
%:%136=48%:%
%:%137=48%:%
%:%146=50%:%
%:%148=51%:%
%:%149=51%:%
%:%150=52%:%
%:%151=53%:%
%:%158=54%:%
%:%159=54%:%
%:%160=55%:%
%:%161=55%:%
%:%162=56%:%
%:%163=56%:%
%:%164=56%:%
%:%165=57%:%
%:%166=57%:%
%:%167=58%:%
%:%168=58%:%
%:%169=58%:%
%:%170=59%:%
%:%171=59%:%
%:%172=59%:%
%:%173=60%:%
%:%174=61%:%
%:%175=62%:%
%:%176=62%:%
%:%177=62%:%
%:%178=63%:%
%:%179=63%:%
%:%180=64%:%
%:%181=65%:%
%:%182=66%:%
%:%183=66%:%
%:%184=67%:%
%:%185=67%:%
%:%186=67%:%
%:%187=68%:%
%:%188=68%:%
%:%189=69%:%
%:%190=69%:%
%:%191=70%:%
%:%192=71%:%
%:%193=71%:%
%:%194=72%:%
%:%195=72%:%
%:%196=72%:%
%:%197=73%:%
%:%198=74%:%
%:%199=74%:%
%:%200=75%:%
%:%201=75%:%
%:%202=75%:%
%:%203=76%:%
%:%204=76%:%
%:%205=77%:%
%:%206=77%:%
%:%207=78%:%
%:%208=78%:%
%:%209=79%:%
%:%210=79%:%
%:%211=80%:%
%:%212=80%:%
%:%213=81%:%
%:%214=81%:%
%:%215=81%:%
%:%216=82%:%
%:%217=83%:%
%:%218=83%:%
%:%219=84%:%
%:%220=84%:%
%:%221=84%:%
%:%222=85%:%
%:%223=85%:%
%:%224=86%:%
%:%225=86%:%
%:%226=86%:%
%:%227=87%:%
%:%228=87%:%
%:%229=87%:%
%:%230=88%:%
%:%240=90%:%
%:%242=91%:%
%:%243=91%:%
%:%244=92%:%
%:%245=93%:%
%:%252=94%:%
%:%253=94%:%
%:%254=95%:%
%:%255=95%:%
%:%256=96%:%
%:%257=96%:%
%:%258=96%:%
%:%259=97%:%
%:%260=97%:%
%:%261=98%:%
%:%262=98%:%
%:%263=99%:%
%:%264=99%:%
%:%265=100%:%
%:%266=100%:%
%:%267=100%:%
%:%268=101%:%
%:%269=101%:%
%:%270=102%:%
%:%271=102%:%
%:%272=103%:%
%:%273=103%:%
%:%274=104%:%
%:%275=104%:%
%:%276=105%:%
%:%277=105%:%
%:%278=106%:%
%:%279=107%:%
%:%280=107%:%
%:%281=108%:%
%:%282=108%:%
%:%283=108%:%
%:%284=109%:%
%:%285=109%:%
%:%286=110%:%
%:%287=110%:%
%:%288=110%:%
%:%289=111%:%
%:%290=111%:%
%:%291=111%:%
%:%292=112%:%
%:%293=112%:%
%:%294=112%:%
%:%295=113%:%
%:%296=113%:%
%:%297=114%:%
%:%298=114%:%
%:%299=115%:%
%:%300=115%:%
%:%301=115%:%
%:%302=116%:%
%:%303=116%:%
%:%304=117%:%
%:%305=117%:%
%:%306=118%:%
%:%307=118%:%
%:%308=119%:%
%:%309=119%:%
%:%310=120%:%
%:%311=120%:%
%:%312=121%:%
%:%313=121%:%
%:%314=122%:%
%:%315=122%:%
%:%316=123%:%
%:%317=123%:%
%:%318=124%:%
%:%319=124%:%
%:%320=125%:%
%:%321=125%:%
%:%322=126%:%
%:%323=126%:%
%:%324=127%:%
%:%325=127%:%
%:%326=128%:%
%:%327=128%:%
%:%328=129%:%
%:%329=130%:%
%:%330=131%:%
%:%331=132%:%
%:%332=132%:%
%:%333=133%:%
%:%334=133%:%
%:%335=134%:%
%:%336=134%:%
%:%337=135%:%
%:%338=135%:%
%:%339=135%:%
%:%340=136%:%
%:%341=136%:%
%:%342=137%:%
%:%343=137%:%
%:%344=137%:%
%:%345=138%:%
%:%346=138%:%
%:%347=139%:%
%:%348=139%:%
%:%349=140%:%
%:%350=140%:%
%:%351=141%:%
%:%352=141%:%
%:%353=141%:%
%:%354=142%:%
%:%355=142%:%
%:%356=142%:%
%:%357=143%:%
%:%358=143%:%
%:%359=144%:%
%:%360=145%:%
%:%361=145%:%
%:%362=146%:%
%:%363=146%:%
%:%364=147%:%
%:%365=147%:%
%:%366=148%:%
%:%367=148%:%
%:%368=148%:%
%:%369=149%:%
%:%370=149%:%
%:%371=150%:%
%:%372=150%:%
%:%373=150%:%
%:%374=151%:%
%:%375=151%:%
%:%376=152%:%
%:%377=152%:%
%:%378=153%:%
%:%379=153%:%
%:%380=154%:%
%:%381=154%:%
%:%382=154%:%
%:%383=155%:%
%:%384=155%:%
%:%385=155%:%
%:%386=156%:%
%:%387=156%:%
%:%388=157%:%
%:%389=157%:%
%:%390=158%:%
%:%391=158%:%
%:%392=158%:%
%:%393=159%:%
%:%394=159%:%
%:%395=160%:%
%:%396=160%:%
%:%397=161%:%
%:%398=161%:%
%:%399=161%:%
%:%400=162%:%
%:%401=162%:%
%:%402=163%:%
%:%417=165%:%
%:%427=167%:%
%:%428=167%:%
%:%429=168%:%
%:%430=169%:%
%:%437=170%:%
%:%438=170%:%
%:%439=171%:%
%:%440=171%:%
%:%441=172%:%
%:%442=172%:%
%:%443=173%:%
%:%444=173%:%
%:%445=174%:%
%:%446=174%:%
%:%447=174%:%
%:%448=175%:%
%:%449=176%:%
%:%450=177%:%
%:%451=177%:%
%:%452=178%:%
%:%453=178%:%
%:%454=179%:%
%:%455=179%:%
%:%456=180%:%
%:%457=180%:%
%:%458=181%:%
%:%459=181%:%
%:%460=182%:%
%:%461=182%:%
%:%462=182%:%
%:%463=183%:%
%:%464=183%:%
%:%465=184%:%
%:%466=184%:%
%:%467=184%:%
%:%468=185%:%
%:%469=185%:%
%:%470=185%:%
%:%471=186%:%
%:%472=186%:%
%:%473=186%:%
%:%474=187%:%
%:%475=187%:%
%:%476=187%:%
%:%477=188%:%
%:%478=188%:%
%:%479=188%:%
%:%480=188%:%
%:%481=188%:%
%:%482=189%:%
%:%483=189%:%
%:%484=190%:%
%:%485=190%:%
%:%486=190%:%
%:%487=191%:%
%:%488=191%:%
%:%489=191%:%
%:%490=192%:%
%:%500=194%:%
%:%502=195%:%
%:%503=195%:%
%:%504=196%:%
%:%511=197%:%
%:%512=197%:%
%:%513=198%:%
%:%514=198%:%
%:%515=199%:%
%:%516=199%:%
%:%517=200%:%
%:%518=200%:%
%:%519=201%:%
%:%520=201%:%
%:%521=202%:%
%:%522=202%:%
%:%523=203%:%
%:%524=204%:%
%:%525=205%:%
%:%526=205%:%
%:%527=206%:%
%:%528=206%:%
%:%529=206%:%
%:%530=207%:%
%:%531=207%:%
%:%532=208%:%
%:%533=208%:%
%:%534=208%:%
%:%535=209%:%
%:%536=210%:%
%:%537=210%:%
%:%538=211%:%
%:%539=211%:%
%:%540=211%:%
%:%541=212%:%
%:%542=213%:%
%:%543=213%:%
%:%544=214%:%
%:%545=214%:%
%:%546=214%:%
%:%547=215%:%
%:%548=215%:%
%:%549=216%:%
%:%550=216%:%
%:%551=216%:%
%:%552=217%:%
%:%553=218%:%
%:%554=219%:%
%:%555=219%:%
%:%556=219%:%
%:%557=220%:%
%:%558=220%:%
%:%559=220%:%
%:%560=221%:%
%:%561=221%:%
%:%562=222%:%
%:%563=222%:%
%:%564=223%:%
%:%565=223%:%
%:%566=224%:%
%:%567=224%:%
%:%568=225%:%
%:%569=225%:%
%:%570=225%:%
%:%571=226%:%
%:%572=226%:%
%:%573=226%:%
%:%574=227%:%
%:%575=227%:%
%:%576=228%:%
%:%577=228%:%
%:%578=228%:%
%:%579=229%:%
%:%580=229%:%
%:%581=229%:%
%:%582=230%:%
%:%583=230%:%
%:%584=231%:%
%:%585=231%:%
%:%586=232%:%
%:%587=232%:%
%:%588=232%:%
%:%589=233%:%
%:%590=233%:%
%:%591=234%:%
%:%592=234%:%
%:%593=235%:%
%:%594=235%:%
%:%595=235%:%
%:%596=236%:%
%:%602=236%:%
%:%605=237%:%
%:%606=238%:%
%:%607=238%:%
%:%608=239%:%
%:%611=240%:%
%:%615=240%:%
%:%616=240%:%
%:%621=240%:%
%:%624=241%:%
%:%625=242%:%
%:%626=242%:%
%:%627=243%:%
%:%630=244%:%
%:%634=244%:%
%:%635=244%:%
%:%640=244%:%
%:%643=245%:%
%:%644=246%:%
%:%645=246%:%
%:%646=247%:%
%:%647=248%:%
%:%648=249%:%
%:%651=250%:%
%:%655=250%:%
%:%656=250%:%
%:%661=250%:%
%:%664=251%:%
%:%665=252%:%
%:%666=252%:%
%:%667=253%:%
%:%668=254%:%
%:%669=255%:%
%:%676=256%:%
%:%677=256%:%
%:%678=257%:%
%:%679=257%:%
%:%680=258%:%
%:%681=258%:%
%:%682=258%:%
%:%683=259%:%
%:%684=259%:%
%:%685=260%:%
%:%686=260%:%
%:%687=260%:%
%:%688=261%:%
%:%689=261%:%
%:%690=262%:%
%:%691=263%:%
%:%692=263%:%
%:%693=264%:%
%:%694=264%:%
%:%695=265%:%
%:%696=265%:%
%:%697=266%:%
%:%698=266%:%
%:%699=267%:%
%:%700=267%:%
%:%701=268%:%
%:%702=268%:%
%:%703=269%:%
%:%704=269%:%
%:%705=270%:%
%:%706=270%:%
%:%707=270%:%
%:%708=271%:%
%:%709=271%:%
%:%710=272%:%
%:%711=272%:%
%:%712=272%:%
%:%713=273%:%
%:%714=273%:%
%:%715=273%:%
%:%716=274%:%
%:%717=274%:%
%:%718=274%:%
%:%719=275%:%
%:%720=275%:%
%:%721=275%:%
%:%722=276%:%
%:%723=276%:%
%:%724=276%:%
%:%725=277%:%
%:%726=277%:%
%:%727=278%:%
%:%728=278%:%
%:%729=278%:%
%:%730=279%:%
%:%731=279%:%
%:%732=279%:%
%:%733=280%:%
%:%734=280%:%
%:%735=280%:%
%:%736=281%:%
%:%737=281%:%
%:%738=281%:%
%:%739=282%:%
%:%740=282%:%
%:%741=282%:%
%:%742=283%:%
%:%743=283%:%
%:%744=283%:%
%:%745=284%:%
%:%760=286%:%
%:%770=288%:%
%:%771=288%:%
%:%772=289%:%
%:%773=290%:%
%:%774=291%:%
%:%775=292%:%
%:%776=292%:%
%:%777=293%:%
%:%778=294%:%
%:%785=295%:%
%:%786=295%:%
%:%787=296%:%
%:%788=296%:%
%:%789=297%:%
%:%790=298%:%
%:%791=298%:%
%:%792=298%:%
%:%793=299%:%
%:%794=299%:%
%:%795=300%:%
%:%796=300%:%
%:%797=301%:%
%:%798=301%:%
%:%799=302%:%
%:%800=302%:%
%:%801=303%:%
%:%802=303%:%
%:%803=304%:%
%:%804=304%:%
%:%805=304%:%
%:%806=305%:%
%:%807=305%:%
%:%808=306%:%
%:%809=306%:%
%:%810=306%:%
%:%811=307%:%
%:%812=307%:%
%:%813=308%:%
%:%814=308%:%
%:%815=308%:%
%:%816=309%:%
%:%817=309%:%
%:%818=309%:%
%:%819=310%:%
%:%825=310%:%
%:%828=311%:%
%:%829=312%:%
%:%830=312%:%
%:%831=313%:%
%:%834=314%:%
%:%838=314%:%
%:%839=314%:%
%:%844=314%:%
%:%847=315%:%
%:%848=316%:%
%:%849=316%:%
%:%850=317%:%
%:%853=318%:%
%:%857=318%:%
%:%858=318%:%
%:%863=318%:%
%:%866=319%:%
%:%867=320%:%
%:%868=320%:%
%:%869=321%:%
%:%872=322%:%
%:%876=322%:%
%:%877=322%:%
%:%882=322%:%
%:%885=323%:%
%:%886=324%:%
%:%887=324%:%
%:%888=325%:%
%:%895=326%:%
%:%896=326%:%
%:%897=327%:%
%:%898=327%:%
%:%899=328%:%
%:%900=328%:%
%:%901=328%:%
%:%902=329%:%
%:%903=329%:%
%:%904=329%:%
%:%905=330%:%
%:%906=330%:%
%:%907=331%:%
%:%908=331%:%
%:%909=331%:%
%:%910=332%:%
%:%911=332%:%
%:%912=333%:%
%:%913=333%:%
%:%914=333%:%
%:%915=334%:%
%:%916=334%:%
%:%917=334%:%
%:%918=335%:%
%:%924=335%:%
%:%927=336%:%
%:%928=337%:%
%:%929=337%:%
%:%930=338%:%
%:%937=339%:%
%:%938=339%:%
%:%939=340%:%
%:%940=340%:%
%:%941=341%:%
%:%942=341%:%
%:%943=341%:%
%:%944=342%:%
%:%945=342%:%
%:%946=342%:%
%:%947=343%:%
%:%948=343%:%
%:%949=344%:%
%:%950=344%:%
%:%951=344%:%
%:%952=345%:%
%:%953=345%:%
%:%954=346%:%
%:%955=346%:%
%:%956=346%:%
%:%957=347%:%
%:%958=347%:%
%:%959=347%:%
%:%960=348%:%
%:%975=350%:%
%:%987=352%:%
%:%989=353%:%
%:%990=353%:%
%:%991=354%:%
%:%998=355%:%
%:%999=355%:%
%:%1000=356%:%
%:%1001=356%:%
%:%1002=357%:%
%:%1003=357%:%
%:%1004=358%:%
%:%1005=358%:%
%:%1006=359%:%
%:%1007=359%:%
%:%1008=359%:%
%:%1009=360%:%
%:%1010=360%:%
%:%1011=361%:%
%:%1012=361%:%
%:%1013=361%:%
%:%1014=362%:%
%:%1015=362%:%
%:%1016=362%:%
%:%1017=363%:%
%:%1018=363%:%
%:%1019=364%:%
%:%1020=364%:%
%:%1021=365%:%
%:%1022=365%:%
%:%1023=366%:%
%:%1024=366%:%
%:%1025=367%:%
%:%1026=367%:%
%:%1027=368%:%
%:%1028=368%:%
%:%1029=369%:%
%:%1030=369%:%
%:%1031=370%:%
%:%1032=370%:%
%:%1033=370%:%
%:%1034=371%:%
%:%1035=371%:%
%:%1036=371%:%
%:%1037=372%:%
%:%1038=372%:%
%:%1039=373%:%
%:%1040=373%:%
%:%1041=374%:%
%:%1042=374%:%
%:%1043=374%:%
%:%1044=375%:%
%:%1045=375%:%
%:%1046=375%:%
%:%1047=376%:%
%:%1053=376%:%
%:%1056=377%:%
%:%1057=378%:%
%:%1058=378%:%
%:%1059=379%:%
%:%1060=380%:%
%:%1063=381%:%
%:%1067=381%:%
%:%1068=381%:%
%:%1073=381%:%
%:%1076=382%:%
%:%1077=383%:%
%:%1078=383%:%
%:%1079=384%:%
%:%1080=385%:%
%:%1081=386%:%
%:%1082=387%:%
%:%1089=388%:%
%:%1090=388%:%
%:%1091=389%:%
%:%1092=389%:%
%:%1093=390%:%
%:%1094=391%:%
%:%1095=392%:%
%:%1096=393%:%
%:%1097=393%:%
%:%1098=393%:%
%:%1099=394%:%
%:%1100=395%:%
%:%1101=395%:%
%:%1102=395%:%
%:%1103=396%:%
%:%1104=396%:%
%:%1105=397%:%
%:%1106=397%:%
%:%1107=397%:%
%:%1108=398%:%
%:%1109=399%:%
%:%1110=400%:%
%:%1111=400%:%
%:%1112=400%:%
%:%1113=401%:%
%:%1114=402%:%
%:%1115=402%:%
%:%1116=403%:%
%:%1117=403%:%
%:%1118=404%:%
%:%1119=404%:%
%:%1120=405%:%
%:%1121=405%:%
%:%1122=406%:%
%:%1123=407%:%
%:%1124=407%:%
%:%1125=408%:%
%:%1126=408%:%
%:%1127=409%:%
%:%1128=409%:%
%:%1129=409%:%
%:%1130=410%:%
%:%1131=410%:%
%:%1132=411%:%
%:%1133=411%:%
%:%1134=411%:%
%:%1135=412%:%
%:%1136=412%:%
%:%1137=412%:%
%:%1138=413%:%
%:%1139=413%:%
%:%1140=413%:%
%:%1141=414%:%
%:%1142=414%:%
%:%1143=415%:%
%:%1144=415%:%
%:%1145=416%:%
%:%1146=416%:%
%:%1147=416%:%
%:%1148=417%:%
%:%1149=418%:%
%:%1150=419%:%
%:%1151=419%:%
%:%1152=419%:%
%:%1153=420%:%
%:%1154=421%:%
%:%1155=421%:%
%:%1156=422%:%
%:%1157=423%:%
%:%1158=424%:%
%:%1159=425%:%
%:%1160=425%:%
%:%1161=425%:%
%:%1162=426%:%
%:%1163=427%:%
%:%1164=427%:%
%:%1165=428%:%
%:%1166=428%:%
%:%1167=429%:%
%:%1168=429%:%
%:%1169=430%:%
%:%1170=430%:%
%:%1171=431%:%
%:%1172=431%:%
%:%1173=432%:%
%:%1174=432%:%
%:%1175=433%:%
%:%1176=433%:%
%:%1177=434%:%
%:%1178=434%:%
%:%1179=435%:%
%:%1180=435%:%
%:%1181=436%:%
%:%1182=437%:%
%:%1183=437%:%
%:%1184=438%:%
%:%1185=438%:%
%:%1186=439%:%
%:%1187=439%:%
%:%1188=439%:%
%:%1189=440%:%
%:%1190=440%:%
%:%1191=441%:%
%:%1206=443%:%
%:%1216=445%:%
%:%1217=445%:%
%:%1218=446%:%
%:%1219=447%:%
%:%1220=448%:%
%:%1221=449%:%
%:%1222=449%:%
%:%1223=450%:%
%:%1224=451%:%
%:%1227=452%:%
%:%1231=452%:%
%:%1232=452%:%
%:%1233=453%:%
%:%1234=453%:%
%:%1239=453%:%
%:%1242=454%:%
%:%1243=455%:%
%:%1244=455%:%
%:%1245=456%:%
%:%1248=457%:%
%:%1252=457%:%
%:%1253=457%:%
%:%1258=457%:%
%:%1261=458%:%
%:%1262=459%:%
%:%1263=459%:%
%:%1264=460%:%
%:%1267=461%:%
%:%1271=461%:%
%:%1272=461%:%
%:%1277=461%:%
%:%1280=462%:%
%:%1281=463%:%
%:%1282=463%:%
%:%1283=464%:%
%:%1286=465%:%
%:%1290=465%:%
%:%1291=465%:%
%:%1292=465%:%
%:%1297=465%:%
%:%1300=466%:%
%:%1301=467%:%
%:%1302=467%:%
%:%1303=468%:%
%:%1304=469%:%
%:%1305=470%:%
%:%1306=470%:%
%:%1307=471%:%
%:%1310=472%:%
%:%1314=472%:%
%:%1315=472%:%
%:%1316=472%:%
%:%1321=472%:%
%:%1324=473%:%
%:%1325=474%:%
%:%1326=474%:%
%:%1327=475%:%
%:%1328=476%:%
%:%1335=477%:%
%:%1336=477%:%
%:%1337=478%:%
%:%1338=478%:%
%:%1339=479%:%
%:%1340=479%:%
%:%1341=480%:%
%:%1342=480%:%
%:%1343=481%:%
%:%1344=481%:%
%:%1345=481%:%
%:%1346=482%:%
%:%1347=482%:%
%:%1348=482%:%
%:%1349=483%:%
%:%1350=483%:%
%:%1351=483%:%
%:%1352=484%:%
%:%1353=484%:%
%:%1354=484%:%
%:%1355=485%:%
%:%1356=485%:%
%:%1357=485%:%
%:%1358=486%:%
%:%1359=486%:%
%:%1360=486%:%
%:%1361=487%:%
%:%1362=487%:%
%:%1363=487%:%
%:%1364=488%:%
%:%1365=488%:%
%:%1366=488%:%
%:%1367=489%:%
%:%1368=489%:%
%:%1369=489%:%
%:%1370=490%:%
%:%1371=490%:%
%:%1372=490%:%
%:%1373=491%:%
%:%1374=491%:%
%:%1375=491%:%
%:%1376=492%:%
%:%1377=492%:%
%:%1378=492%:%
%:%1379=493%:%
%:%1380=493%:%
%:%1381=493%:%
%:%1382=494%:%
%:%1383=494%:%
%:%1384=494%:%
%:%1385=495%:%
%:%1386=495%:%
%:%1387=495%:%
%:%1388=496%:%
%:%1389=496%:%
%:%1390=496%:%
%:%1391=497%:%
%:%1392=497%:%
%:%1393=497%:%
%:%1394=498%:%
%:%1395=498%:%
%:%1396=498%:%
%:%1397=499%:%
%:%1398=499%:%
%:%1399=499%:%
%:%1400=500%:%
%:%1401=500%:%
%:%1402=500%:%
%:%1403=501%:%
%:%1404=501%:%
%:%1405=501%:%
%:%1406=502%:%
%:%1407=502%:%
%:%1408=502%:%
%:%1409=503%:%
%:%1410=503%:%
%:%1411=503%:%
%:%1412=504%:%
%:%1418=504%:%
%:%1421=505%:%
%:%1422=506%:%
%:%1423=506%:%
%:%1424=507%:%
%:%1425=508%:%
%:%1428=509%:%
%:%1432=509%:%
%:%1433=509%:%
%:%1438=509%:%
%:%1441=510%:%
%:%1442=511%:%
%:%1443=511%:%
%:%1444=512%:%
%:%1445=513%:%
%:%1452=514%:%
%:%1453=514%:%
%:%1454=515%:%
%:%1455=515%:%
%:%1456=516%:%
%:%1457=516%:%
%:%1458=517%:%
%:%1459=517%:%
%:%1460=518%:%
%:%1461=518%:%
%:%1462=518%:%
%:%1463=519%:%
%:%1464=519%:%
%:%1465=519%:%
%:%1466=520%:%
%:%1467=520%:%
%:%1468=520%:%
%:%1469=521%:%
%:%1470=521%:%
%:%1471=521%:%
%:%1472=522%:%
%:%1473=522%:%
%:%1474=522%:%
%:%1475=523%:%
%:%1476=523%:%
%:%1477=523%:%
%:%1478=524%:%
%:%1479=524%:%
%:%1480=524%:%
%:%1481=525%:%
%:%1482=525%:%
%:%1483=525%:%
%:%1484=526%:%
%:%1485=526%:%
%:%1486=526%:%
%:%1487=527%:%
%:%1488=527%:%
%:%1489=527%:%
%:%1490=528%:%
%:%1491=528%:%
%:%1492=528%:%
%:%1493=529%:%
%:%1494=529%:%
%:%1495=529%:%
%:%1496=530%:%
%:%1497=530%:%
%:%1498=530%:%
%:%1499=531%:%
%:%1500=531%:%
%:%1501=531%:%
%:%1502=532%:%
%:%1503=532%:%
%:%1504=532%:%
%:%1505=533%:%
%:%1506=533%:%
%:%1507=533%:%
%:%1508=534%:%
%:%1509=534%:%
%:%1510=534%:%
%:%1511=535%:%
%:%1512=535%:%
%:%1513=535%:%
%:%1514=536%:%
%:%1515=536%:%
%:%1516=536%:%
%:%1517=537%:%
%:%1518=537%:%
%:%1519=537%:%
%:%1520=538%:%
%:%1521=538%:%
%:%1522=538%:%
%:%1523=539%:%
%:%1524=539%:%
%:%1525=539%:%
%:%1526=540%:%
%:%1527=540%:%
%:%1528=540%:%
%:%1529=541%:%
%:%1530=541%:%
%:%1531=541%:%
%:%1532=542%:%
%:%1533=542%:%
%:%1534=542%:%
%:%1535=543%:%
%:%1536=543%:%
%:%1537=543%:%
%:%1538=544%:%
%:%1539=544%:%
%:%1540=544%:%
%:%1541=545%:%
%:%1542=545%:%
%:%1543=545%:%
%:%1544=546%:%
%:%1545=546%:%
%:%1546=546%:%
%:%1547=547%:%
%:%1548=547%:%
%:%1549=547%:%
%:%1550=548%:%
%:%1551=548%:%
%:%1552=548%:%
%:%1553=549%:%
%:%1554=549%:%
%:%1555=549%:%
%:%1556=550%:%
%:%1557=550%:%
%:%1558=550%:%
%:%1559=551%:%
%:%1560=551%:%
%:%1561=551%:%
%:%1562=552%:%
%:%1563=552%:%
%:%1564=553%:%
%:%1570=553%:%
%:%1573=554%:%
%:%1574=555%:%
%:%1575=555%:%
%:%1576=556%:%
%:%1577=557%:%
%:%1580=558%:%
%:%1584=558%:%
%:%1585=558%:%
%:%1586=558%:%
%:%1591=558%:%
%:%1594=559%:%
%:%1595=560%:%
%:%1596=560%:%
%:%1597=561%:%
%:%1598=562%:%
%:%1599=563%:%
%:%1600=563%:%
%:%1601=564%:%
%:%1604=565%:%
%:%1608=565%:%
%:%1609=565%:%
%:%1614=565%:%
%:%1617=566%:%
%:%1618=567%:%
%:%1619=567%:%
%:%1620=568%:%
%:%1621=569%:%
%:%1622=570%:%
%:%1625=571%:%
%:%1629=571%:%
%:%1630=571%:%
%:%1631=572%:%
%:%1632=572%:%
%:%1637=572%:%
%:%1640=573%:%
%:%1641=574%:%
%:%1642=574%:%
%:%1643=575%:%
%:%1644=576%:%
%:%1645=577%:%
%:%1648=578%:%
%:%1652=578%:%
%:%1653=578%:%
%:%1654=578%:%
%:%1659=578%:%
%:%1662=579%:%
%:%1663=580%:%
%:%1664=580%:%
%:%1665=581%:%
%:%1666=582%:%
%:%1673=583%:%
%:%1674=583%:%
%:%1675=584%:%
%:%1676=584%:%
%:%1677=585%:%
%:%1678=585%:%
%:%1679=586%:%
%:%1680=586%:%
%:%1681=587%:%
%:%1682=587%:%
%:%1683=587%:%
%:%1684=588%:%
%:%1685=588%:%
%:%1686=588%:%
%:%1687=589%:%
%:%1688=589%:%
%:%1689=590%:%
%:%1690=590%:%
%:%1691=590%:%
%:%1692=591%:%
%:%1693=591%:%
%:%1694=592%:%
%:%1695=592%:%
%:%1696=592%:%
%:%1697=593%:%
%:%1698=593%:%
%:%1699=594%:%
%:%1700=594%:%
%:%1701=594%:%
%:%1702=595%:%
%:%1703=595%:%
%:%1704=596%:%
%:%1710=596%:%
%:%1713=597%:%
%:%1714=598%:%
%:%1715=598%:%
%:%1716=599%:%
%:%1717=600%:%
%:%1718=601%:%
%:%1725=602%:%
%:%1726=602%:%
%:%1727=603%:%
%:%1728=603%:%
%:%1729=604%:%
%:%1730=604%:%
%:%1731=604%:%
%:%1732=605%:%
%:%1733=605%:%
%:%1734=605%:%
%:%1735=606%:%
%:%1736=606%:%
%:%1737=606%:%
%:%1738=607%:%
%:%1739=607%:%
%:%1740=607%:%
%:%1741=608%:%
%:%1742=608%:%
%:%1743=608%:%
%:%1744=609%:%
%:%1745=609%:%
%:%1746=609%:%
%:%1747=610%:%
%:%1748=610%:%
%:%1749=610%:%
%:%1750=611%:%
%:%1751=611%:%
%:%1752=611%:%
%:%1753=612%:%
%:%1754=612%:%
%:%1755=612%:%
%:%1756=613%:%
%:%1762=613%:%
%:%1765=614%:%
%:%1766=615%:%
%:%1767=615%:%
%:%1768=616%:%
%:%1769=617%:%
%:%1772=618%:%
%:%1776=618%:%
%:%1777=618%:%
%:%1778=618%:%
%:%1783=618%:%
%:%1786=619%:%
%:%1787=620%:%
%:%1788=620%:%
%:%1789=621%:%
%:%1790=622%:%
%:%1791=623%:%
%:%1792=624%:%
%:%1795=625%:%
%:%1799=625%:%
%:%1800=625%:%
%:%1801=625%:%
%:%1806=625%:%
%:%1809=626%:%
%:%1810=627%:%
%:%1811=627%:%
%:%1812=628%:%
%:%1819=629%:%
%:%1820=629%:%
%:%1821=630%:%
%:%1822=630%:%
%:%1823=631%:%
%:%1824=631%:%
%:%1825=632%:%
%:%1826=632%:%
%:%1827=633%:%
%:%1828=633%:%
%:%1829=634%:%
%:%1830=634%:%
%:%1831=635%:%
%:%1832=635%:%
%:%1833=636%:%
%:%1834=636%:%
%:%1835=637%:%
%:%1836=637%:%
%:%1837=638%:%
%:%1838=639%:%
%:%1839=639%:%
%:%1840=640%:%
%:%1841=640%:%
%:%1842=640%:%
%:%1843=641%:%
%:%1844=641%:%
%:%1845=642%:%
%:%1846=642%:%
%:%1847=642%:%
%:%1848=643%:%
%:%1849=643%:%
%:%1850=644%:%
%:%1851=644%:%
%:%1852=644%:%
%:%1853=645%:%
%:%1854=645%:%
%:%1855=645%:%
%:%1856=646%:%
%:%1857=646%:%
%:%1858=646%:%
%:%1859=647%:%
%:%1860=647%:%
%:%1861=647%:%
%:%1862=648%:%
%:%1863=648%:%
%:%1864=649%:%
%:%1865=650%:%
%:%1866=650%:%
%:%1867=651%:%
%:%1868=651%:%
%:%1869=652%:%
%:%1870=652%:%
%:%1871=653%:%
%:%1872=653%:%
%:%1873=654%:%
%:%1874=655%:%
%:%1875=655%:%
%:%1876=656%:%
%:%1877=656%:%
%:%1878=656%:%
%:%1879=657%:%
%:%1880=657%:%
%:%1881=658%:%
%:%1882=658%:%
%:%1883=658%:%
%:%1884=659%:%
%:%1885=659%:%
%:%1886=660%:%
%:%1887=660%:%
%:%1888=660%:%
%:%1889=661%:%
%:%1890=661%:%
%:%1891=662%:%
%:%1892=662%:%
%:%1893=662%:%
%:%1894=663%:%
%:%1895=663%:%
%:%1896=664%:%
%:%1897=664%:%
%:%1898=664%:%
%:%1899=665%:%
%:%1900=665%:%
%:%1901=666%:%
%:%1902=666%:%
%:%1903=666%:%
%:%1904=667%:%
%:%1905=667%:%
%:%1906=668%:%
%:%1907=668%:%
%:%1908=668%:%
%:%1909=669%:%
%:%1910=670%:%
%:%1911=670%:%
%:%1912=670%:%
%:%1913=671%:%
%:%1914=671%:%
%:%1915=672%:%
%:%1916=672%:%
%:%1917=673%:%
%:%1918=673%:%
%:%1919=674%:%
%:%1925=674%:%
%:%1928=675%:%
%:%1929=676%:%
%:%1930=676%:%
%:%1931=677%:%
%:%1932=678%:%
%:%1939=679%:%
%:%1940=679%:%
%:%1941=680%:%
%:%1942=680%:%
%:%1943=681%:%
%:%1944=681%:%
%:%1945=681%:%
%:%1946=682%:%
%:%1947=682%:%
%:%1948=682%:%
%:%1949=683%:%
%:%1950=683%:%
%:%1951=683%:%
%:%1952=684%:%
%:%1953=684%:%
%:%1954=684%:%
%:%1955=685%:%
%:%1956=685%:%
%:%1957=685%:%
%:%1958=686%:%
%:%1959=686%:%
%:%1960=686%:%
%:%1961=687%:%
%:%1962=687%:%
%:%1963=687%:%
%:%1964=688%:%
%:%1965=688%:%
%:%1966=688%:%
%:%1967=689%:%
%:%1968=689%:%
%:%1969=689%:%
%:%1970=690%:%
%:%1985=692%:%
%:%1995=694%:%
%:%1996=694%:%
%:%1997=695%:%
%:%2000=696%:%
%:%2004=696%:%
%:%2005=696%:%
%:%2006=696%:%
%:%2011=696%:%
%:%2014=697%:%
%:%2015=698%:%
%:%2016=698%:%
%:%2017=699%:%
%:%2020=700%:%
%:%2024=700%:%
%:%2025=700%:%
%:%2026=700%:%
%:%2031=700%:%
%:%2036=701%:%
%:%2041=702%:%

%
\begin{isabellebody}%
\setisabellecontext{Pred{\isacharunderscore}{\kern0pt}Logic}%
%
\isadelimdocument
%
\endisadelimdocument
%
\isatagdocument
%
\isamarkupsection{Predicate Logic Functions%
}
\isamarkuptrue%
%
\endisatagdocument
{\isafolddocument}%
%
\isadelimdocument
%
\endisadelimdocument
%
\isadelimtheory
%
\endisadelimtheory
%
\isatagtheory
\isacommand{theory}\isamarkupfalse%
\ Pred{\isacharunderscore}{\kern0pt}Logic\isanewline
\ \ \isakeyword{imports}\ Coproduct\isanewline
\isakeyword{begin}%
\endisatagtheory
{\isafoldtheory}%
%
\isadelimtheory
%
\endisadelimtheory
%
\isadelimdocument
%
\endisadelimdocument
%
\isatagdocument
%
\isamarkupsubsection{NOT%
}
\isamarkuptrue%
%
\endisatagdocument
{\isafolddocument}%
%
\isadelimdocument
%
\endisadelimdocument
\isacommand{definition}\isamarkupfalse%
\ NOT\ {\isacharcolon}{\kern0pt}{\isacharcolon}{\kern0pt}\ {\isachardoublequoteopen}cfunc{\isachardoublequoteclose}\ \isakeyword{where}\isanewline
\ \ {\isachardoublequoteopen}NOT\ {\isacharequal}{\kern0pt}\ {\isacharparenleft}{\kern0pt}THE\ {\isasymchi}{\isachardot}{\kern0pt}\ is{\isacharunderscore}{\kern0pt}pullback\ {\isasymone}\ {\isasymone}\ {\isasymOmega}\ {\isasymOmega}\ {\isacharparenleft}{\kern0pt}{\isasymbeta}\isactrlbsub {\isasymone}\isactrlesub {\isacharparenright}{\kern0pt}\ {\isasymt}\ {\isasymf}\ {\isasymchi}{\isacharparenright}{\kern0pt}{\isachardoublequoteclose}\isanewline
\isanewline
\isacommand{lemma}\isamarkupfalse%
\ NOT{\isacharunderscore}{\kern0pt}is{\isacharunderscore}{\kern0pt}pullback{\isacharcolon}{\kern0pt}\isanewline
\ \ {\isachardoublequoteopen}is{\isacharunderscore}{\kern0pt}pullback\ {\isasymone}\ {\isasymone}\ {\isasymOmega}\ {\isasymOmega}\ {\isacharparenleft}{\kern0pt}{\isasymbeta}\isactrlbsub {\isasymone}\isactrlesub {\isacharparenright}{\kern0pt}\ {\isasymt}\ {\isasymf}\ NOT{\isachardoublequoteclose}\isanewline
%
\isadelimproof
\ \ %
\endisadelimproof
%
\isatagproof
\isacommand{unfolding}\isamarkupfalse%
\ NOT{\isacharunderscore}{\kern0pt}def\isanewline
\ \ \isacommand{using}\isamarkupfalse%
\ characteristic{\isacharunderscore}{\kern0pt}function{\isacharunderscore}{\kern0pt}exists\ false{\isacharunderscore}{\kern0pt}func{\isacharunderscore}{\kern0pt}type\ element{\isacharunderscore}{\kern0pt}monomorphism\isanewline
\ \ \isacommand{by}\isamarkupfalse%
\ {\isacharparenleft}{\kern0pt}subst\ the{\isadigit{1}}I{\isadigit{2}}{\isacharcomma}{\kern0pt}\ auto{\isacharparenright}{\kern0pt}%
\endisatagproof
{\isafoldproof}%
%
\isadelimproof
\isanewline
%
\endisadelimproof
\isanewline
\isacommand{lemma}\isamarkupfalse%
\ NOT{\isacharunderscore}{\kern0pt}type{\isacharbrackleft}{\kern0pt}type{\isacharunderscore}{\kern0pt}rule{\isacharbrackright}{\kern0pt}{\isacharcolon}{\kern0pt}\isanewline
\ \ {\isachardoublequoteopen}NOT\ {\isacharcolon}{\kern0pt}\ {\isasymOmega}\ {\isasymrightarrow}\ {\isasymOmega}{\isachardoublequoteclose}\isanewline
%
\isadelimproof
\ \ %
\endisadelimproof
%
\isatagproof
\isacommand{using}\isamarkupfalse%
\ NOT{\isacharunderscore}{\kern0pt}is{\isacharunderscore}{\kern0pt}pullback\ \isacommand{unfolding}\isamarkupfalse%
\ is{\isacharunderscore}{\kern0pt}pullback{\isacharunderscore}{\kern0pt}def\ \ \isacommand{by}\isamarkupfalse%
\ auto%
\endisatagproof
{\isafoldproof}%
%
\isadelimproof
\isanewline
%
\endisadelimproof
\isanewline
\isacommand{lemma}\isamarkupfalse%
\ NOT{\isacharunderscore}{\kern0pt}false{\isacharunderscore}{\kern0pt}is{\isacharunderscore}{\kern0pt}true{\isacharcolon}{\kern0pt}\isanewline
\ \ {\isachardoublequoteopen}NOT\ {\isasymcirc}\isactrlsub c\ {\isasymf}\ {\isacharequal}{\kern0pt}\ {\isasymt}{\isachardoublequoteclose}\isanewline
%
\isadelimproof
\ \ %
\endisadelimproof
%
\isatagproof
\isacommand{using}\isamarkupfalse%
\ NOT{\isacharunderscore}{\kern0pt}is{\isacharunderscore}{\kern0pt}pullback\ \isacommand{unfolding}\isamarkupfalse%
\ is{\isacharunderscore}{\kern0pt}pullback{\isacharunderscore}{\kern0pt}def\ \isanewline
\ \ \isacommand{by}\isamarkupfalse%
\ {\isacharparenleft}{\kern0pt}metis\ cfunc{\isacharunderscore}{\kern0pt}type{\isacharunderscore}{\kern0pt}def\ id{\isacharunderscore}{\kern0pt}right{\isacharunderscore}{\kern0pt}unit\ id{\isacharunderscore}{\kern0pt}type\ one{\isacharunderscore}{\kern0pt}unique{\isacharunderscore}{\kern0pt}element{\isacharparenright}{\kern0pt}%
\endisatagproof
{\isafoldproof}%
%
\isadelimproof
\isanewline
%
\endisadelimproof
\isanewline
\isacommand{lemma}\isamarkupfalse%
\ NOT{\isacharunderscore}{\kern0pt}true{\isacharunderscore}{\kern0pt}is{\isacharunderscore}{\kern0pt}false{\isacharcolon}{\kern0pt}\isanewline
\ \ {\isachardoublequoteopen}NOT\ {\isasymcirc}\isactrlsub c\ {\isasymt}\ {\isacharequal}{\kern0pt}\ {\isasymf}{\isachardoublequoteclose}\isanewline
%
\isadelimproof
%
\endisadelimproof
%
\isatagproof
\isacommand{proof}\isamarkupfalse%
{\isacharparenleft}{\kern0pt}rule\ ccontr{\isacharparenright}{\kern0pt}\isanewline
\ \ \isacommand{assume}\isamarkupfalse%
\ {\isachardoublequoteopen}NOT\ {\isasymcirc}\isactrlsub c\ {\isasymt}\ {\isasymnoteq}\ {\isasymf}{\isachardoublequoteclose}\isanewline
\ \ \isacommand{then}\isamarkupfalse%
\ \isacommand{have}\isamarkupfalse%
\ {\isachardoublequoteopen}NOT\ {\isasymcirc}\isactrlsub c\ {\isasymt}\ {\isacharequal}{\kern0pt}\ {\isasymt}{\isachardoublequoteclose}\isanewline
\ \ \ \ \isacommand{using}\isamarkupfalse%
\ true{\isacharunderscore}{\kern0pt}false{\isacharunderscore}{\kern0pt}only{\isacharunderscore}{\kern0pt}truth{\isacharunderscore}{\kern0pt}values\ \isacommand{by}\isamarkupfalse%
\ {\isacharparenleft}{\kern0pt}typecheck{\isacharunderscore}{\kern0pt}cfuncs{\isacharcomma}{\kern0pt}\ blast{\isacharparenright}{\kern0pt}\isanewline
\ \ \isacommand{then}\isamarkupfalse%
\ \isacommand{have}\isamarkupfalse%
\ {\isachardoublequoteopen}{\isasymt}\ {\isasymcirc}\isactrlsub c\ id\isactrlsub c\ {\isasymone}\ {\isacharequal}{\kern0pt}\ NOT\ {\isasymcirc}\isactrlsub c\ {\isasymt}{\isachardoublequoteclose}\isanewline
\ \ \ \ \isacommand{using}\isamarkupfalse%
\ id{\isacharunderscore}{\kern0pt}right{\isacharunderscore}{\kern0pt}unit{\isadigit{2}}\ true{\isacharunderscore}{\kern0pt}func{\isacharunderscore}{\kern0pt}type\ \isacommand{by}\isamarkupfalse%
\ auto\isanewline
\ \ \isacommand{then}\isamarkupfalse%
\ \isacommand{obtain}\isamarkupfalse%
\ j\ \isakeyword{where}\ j{\isacharunderscore}{\kern0pt}type{\isacharcolon}{\kern0pt}\ {\isachardoublequoteopen}j\ {\isasymin}\isactrlsub c\ {\isasymone}{\isachardoublequoteclose}\ \isakeyword{and}\ j{\isacharunderscore}{\kern0pt}id{\isacharcolon}{\kern0pt}\ {\isachardoublequoteopen}{\isasymbeta}\isactrlbsub {\isasymone}\isactrlesub \ {\isasymcirc}\isactrlsub c\ j\ {\isacharequal}{\kern0pt}\ id\isactrlsub c\ {\isasymone}{\isachardoublequoteclose}\ \isakeyword{and}\ f{\isacharunderscore}{\kern0pt}j{\isacharunderscore}{\kern0pt}eq{\isacharunderscore}{\kern0pt}t{\isacharcolon}{\kern0pt}\ {\isachardoublequoteopen}{\isasymf}\ {\isasymcirc}\isactrlsub c\ j\ {\isacharequal}{\kern0pt}\ {\isasymt}{\isachardoublequoteclose}\isanewline
\ \ \ \ \isacommand{using}\isamarkupfalse%
\ NOT{\isacharunderscore}{\kern0pt}is{\isacharunderscore}{\kern0pt}pullback\ \isacommand{unfolding}\isamarkupfalse%
\ is{\isacharunderscore}{\kern0pt}pullback{\isacharunderscore}{\kern0pt}def\ \isacommand{by}\isamarkupfalse%
\ {\isacharparenleft}{\kern0pt}typecheck{\isacharunderscore}{\kern0pt}cfuncs{\isacharcomma}{\kern0pt}\ blast{\isacharparenright}{\kern0pt}\isanewline
\ \ \isacommand{then}\isamarkupfalse%
\ \isacommand{have}\isamarkupfalse%
\ {\isachardoublequoteopen}j\ {\isacharequal}{\kern0pt}\ id\isactrlsub c\ {\isasymone}{\isachardoublequoteclose}\isanewline
\ \ \ \ \isacommand{using}\isamarkupfalse%
\ id{\isacharunderscore}{\kern0pt}type\ one{\isacharunderscore}{\kern0pt}unique{\isacharunderscore}{\kern0pt}element\ \isacommand{by}\isamarkupfalse%
\ blast\isanewline
\ \ \isacommand{then}\isamarkupfalse%
\ \isacommand{have}\isamarkupfalse%
\ {\isachardoublequoteopen}{\isasymf}\ {\isacharequal}{\kern0pt}\ {\isasymt}{\isachardoublequoteclose}\isanewline
\ \ \ \ \isacommand{using}\isamarkupfalse%
\ f{\isacharunderscore}{\kern0pt}j{\isacharunderscore}{\kern0pt}eq{\isacharunderscore}{\kern0pt}t\ false{\isacharunderscore}{\kern0pt}func{\isacharunderscore}{\kern0pt}type\ id{\isacharunderscore}{\kern0pt}right{\isacharunderscore}{\kern0pt}unit{\isadigit{2}}\ \isacommand{by}\isamarkupfalse%
\ auto\isanewline
\ \ \isacommand{then}\isamarkupfalse%
\ \isacommand{show}\isamarkupfalse%
\ False\isanewline
\ \ \ \ \isacommand{using}\isamarkupfalse%
\ true{\isacharunderscore}{\kern0pt}false{\isacharunderscore}{\kern0pt}distinct\ \isacommand{by}\isamarkupfalse%
\ auto\isanewline
\isacommand{qed}\isamarkupfalse%
%
\endisatagproof
{\isafoldproof}%
%
\isadelimproof
\isanewline
%
\endisadelimproof
\ \ \isanewline
\isacommand{lemma}\isamarkupfalse%
\ NOT{\isacharunderscore}{\kern0pt}is{\isacharunderscore}{\kern0pt}true{\isacharunderscore}{\kern0pt}implies{\isacharunderscore}{\kern0pt}false{\isacharcolon}{\kern0pt}\isanewline
\ \ \isakeyword{assumes}\ {\isachardoublequoteopen}p\ {\isasymin}\isactrlsub c\ {\isasymOmega}{\isachardoublequoteclose}\isanewline
\ \ \isakeyword{shows}\ {\isachardoublequoteopen}NOT\ {\isasymcirc}\isactrlsub c\ p\ {\isacharequal}{\kern0pt}\ {\isasymt}\ {\isasymLongrightarrow}\ p\ {\isacharequal}{\kern0pt}\ {\isasymf}{\isachardoublequoteclose}\isanewline
%
\isadelimproof
\ \ %
\endisadelimproof
%
\isatagproof
\isacommand{using}\isamarkupfalse%
\ NOT{\isacharunderscore}{\kern0pt}true{\isacharunderscore}{\kern0pt}is{\isacharunderscore}{\kern0pt}false\ assms\ true{\isacharunderscore}{\kern0pt}false{\isacharunderscore}{\kern0pt}only{\isacharunderscore}{\kern0pt}truth{\isacharunderscore}{\kern0pt}values\ \isacommand{by}\isamarkupfalse%
\ fastforce%
\endisatagproof
{\isafoldproof}%
%
\isadelimproof
\isanewline
%
\endisadelimproof
\isanewline
\isacommand{lemma}\isamarkupfalse%
\ NOT{\isacharunderscore}{\kern0pt}is{\isacharunderscore}{\kern0pt}false{\isacharunderscore}{\kern0pt}implies{\isacharunderscore}{\kern0pt}true{\isacharcolon}{\kern0pt}\isanewline
\ \ \isakeyword{assumes}\ {\isachardoublequoteopen}p\ {\isasymin}\isactrlsub c\ {\isasymOmega}{\isachardoublequoteclose}\isanewline
\ \ \isakeyword{shows}\ {\isachardoublequoteopen}NOT\ {\isasymcirc}\isactrlsub c\ p\ {\isacharequal}{\kern0pt}\ {\isasymf}\ {\isasymLongrightarrow}\ p\ {\isacharequal}{\kern0pt}\ {\isasymt}{\isachardoublequoteclose}\isanewline
%
\isadelimproof
\ \ %
\endisadelimproof
%
\isatagproof
\isacommand{using}\isamarkupfalse%
\ NOT{\isacharunderscore}{\kern0pt}false{\isacharunderscore}{\kern0pt}is{\isacharunderscore}{\kern0pt}true\ assms\ true{\isacharunderscore}{\kern0pt}false{\isacharunderscore}{\kern0pt}only{\isacharunderscore}{\kern0pt}truth{\isacharunderscore}{\kern0pt}values\ \isacommand{by}\isamarkupfalse%
\ fastforce%
\endisatagproof
{\isafoldproof}%
%
\isadelimproof
\isanewline
%
\endisadelimproof
\isanewline
\isacommand{lemma}\isamarkupfalse%
\ double{\isacharunderscore}{\kern0pt}negation{\isacharcolon}{\kern0pt}\isanewline
\ \ {\isachardoublequoteopen}NOT\ {\isasymcirc}\isactrlsub c\ NOT\ {\isacharequal}{\kern0pt}\ \ id\ {\isasymOmega}{\isachardoublequoteclose}\isanewline
%
\isadelimproof
\ \ %
\endisadelimproof
%
\isatagproof
\isacommand{by}\isamarkupfalse%
\ {\isacharparenleft}{\kern0pt}typecheck{\isacharunderscore}{\kern0pt}cfuncs{\isacharcomma}{\kern0pt}\ smt\ {\isacharparenleft}{\kern0pt}verit{\isacharcomma}{\kern0pt}\ del{\isacharunderscore}{\kern0pt}insts{\isacharparenright}{\kern0pt}\ \isanewline
\ \ NOT{\isacharunderscore}{\kern0pt}false{\isacharunderscore}{\kern0pt}is{\isacharunderscore}{\kern0pt}true\ NOT{\isacharunderscore}{\kern0pt}true{\isacharunderscore}{\kern0pt}is{\isacharunderscore}{\kern0pt}false\ cfunc{\isacharunderscore}{\kern0pt}type{\isacharunderscore}{\kern0pt}def\ comp{\isacharunderscore}{\kern0pt}associative\ id{\isacharunderscore}{\kern0pt}left{\isacharunderscore}{\kern0pt}unit{\isadigit{2}}\ one{\isacharunderscore}{\kern0pt}separator\isanewline
\ \ true{\isacharunderscore}{\kern0pt}false{\isacharunderscore}{\kern0pt}only{\isacharunderscore}{\kern0pt}truth{\isacharunderscore}{\kern0pt}values{\isacharparenright}{\kern0pt}%
\endisatagproof
{\isafoldproof}%
%
\isadelimproof
%
\endisadelimproof
%
\isadelimdocument
%
\endisadelimdocument
%
\isatagdocument
%
\isamarkupsubsection{AND%
}
\isamarkuptrue%
%
\endisatagdocument
{\isafolddocument}%
%
\isadelimdocument
%
\endisadelimdocument
\isacommand{definition}\isamarkupfalse%
\ AND\ {\isacharcolon}{\kern0pt}{\isacharcolon}{\kern0pt}\ {\isachardoublequoteopen}cfunc{\isachardoublequoteclose}\ \isakeyword{where}\isanewline
\ \ {\isachardoublequoteopen}AND\ {\isacharequal}{\kern0pt}\ {\isacharparenleft}{\kern0pt}THE\ {\isasymchi}{\isachardot}{\kern0pt}\ is{\isacharunderscore}{\kern0pt}pullback\ {\isasymone}\ {\isasymone}\ {\isacharparenleft}{\kern0pt}{\isasymOmega}\ {\isasymtimes}\isactrlsub c\ {\isasymOmega}{\isacharparenright}{\kern0pt}\ {\isasymOmega}\ {\isacharparenleft}{\kern0pt}{\isasymbeta}\isactrlbsub {\isasymone}\isactrlesub {\isacharparenright}{\kern0pt}\ {\isasymt}\ {\isasymlangle}{\isasymt}{\isacharcomma}{\kern0pt}{\isasymt}{\isasymrangle}\ {\isasymchi}{\isacharparenright}{\kern0pt}{\isachardoublequoteclose}\isanewline
\isanewline
\isacommand{lemma}\isamarkupfalse%
\ AND{\isacharunderscore}{\kern0pt}is{\isacharunderscore}{\kern0pt}pullback{\isacharcolon}{\kern0pt}\isanewline
\ \ {\isachardoublequoteopen}is{\isacharunderscore}{\kern0pt}pullback\ {\isasymone}\ {\isasymone}\ {\isacharparenleft}{\kern0pt}{\isasymOmega}\ {\isasymtimes}\isactrlsub c\ {\isasymOmega}{\isacharparenright}{\kern0pt}\ {\isasymOmega}\ {\isacharparenleft}{\kern0pt}{\isasymbeta}\isactrlbsub {\isasymone}\isactrlesub {\isacharparenright}{\kern0pt}\ {\isasymt}\ {\isasymlangle}{\isasymt}{\isacharcomma}{\kern0pt}{\isasymt}{\isasymrangle}\ AND{\isachardoublequoteclose}\isanewline
%
\isadelimproof
\ \ %
\endisadelimproof
%
\isatagproof
\isacommand{unfolding}\isamarkupfalse%
\ AND{\isacharunderscore}{\kern0pt}def\isanewline
\ \ \isacommand{using}\isamarkupfalse%
\ element{\isacharunderscore}{\kern0pt}monomorphism\ characteristic{\isacharunderscore}{\kern0pt}function{\isacharunderscore}{\kern0pt}exists\isanewline
\ \ \isacommand{by}\isamarkupfalse%
\ {\isacharparenleft}{\kern0pt}typecheck{\isacharunderscore}{\kern0pt}cfuncs{\isacharcomma}{\kern0pt}\ subst\ the{\isadigit{1}}I{\isadigit{2}}{\isacharcomma}{\kern0pt}\ auto{\isacharparenright}{\kern0pt}%
\endisatagproof
{\isafoldproof}%
%
\isadelimproof
\isanewline
%
\endisadelimproof
\isanewline
\isacommand{lemma}\isamarkupfalse%
\ AND{\isacharunderscore}{\kern0pt}type{\isacharbrackleft}{\kern0pt}type{\isacharunderscore}{\kern0pt}rule{\isacharbrackright}{\kern0pt}{\isacharcolon}{\kern0pt}\isanewline
\ \ {\isachardoublequoteopen}AND\ {\isacharcolon}{\kern0pt}\ {\isasymOmega}\ {\isasymtimes}\isactrlsub c\ {\isasymOmega}\ {\isasymrightarrow}\ {\isasymOmega}{\isachardoublequoteclose}\isanewline
%
\isadelimproof
\ \ %
\endisadelimproof
%
\isatagproof
\isacommand{using}\isamarkupfalse%
\ AND{\isacharunderscore}{\kern0pt}is{\isacharunderscore}{\kern0pt}pullback\ \isacommand{unfolding}\isamarkupfalse%
\ is{\isacharunderscore}{\kern0pt}pullback{\isacharunderscore}{\kern0pt}def\ \isacommand{by}\isamarkupfalse%
\ auto%
\endisatagproof
{\isafoldproof}%
%
\isadelimproof
\isanewline
%
\endisadelimproof
\isanewline
\isacommand{lemma}\isamarkupfalse%
\ AND{\isacharunderscore}{\kern0pt}true{\isacharunderscore}{\kern0pt}true{\isacharunderscore}{\kern0pt}is{\isacharunderscore}{\kern0pt}true{\isacharcolon}{\kern0pt}\isanewline
\ \ {\isachardoublequoteopen}AND\ {\isasymcirc}\isactrlsub c\ {\isasymlangle}{\isasymt}{\isacharcomma}{\kern0pt}{\isasymt}{\isasymrangle}\ {\isacharequal}{\kern0pt}\ {\isasymt}{\isachardoublequoteclose}\isanewline
%
\isadelimproof
\ \ %
\endisadelimproof
%
\isatagproof
\isacommand{using}\isamarkupfalse%
\ AND{\isacharunderscore}{\kern0pt}is{\isacharunderscore}{\kern0pt}pullback\ \isacommand{unfolding}\isamarkupfalse%
\ is{\isacharunderscore}{\kern0pt}pullback{\isacharunderscore}{\kern0pt}def\isanewline
\ \ \isacommand{by}\isamarkupfalse%
\ {\isacharparenleft}{\kern0pt}metis\ cfunc{\isacharunderscore}{\kern0pt}type{\isacharunderscore}{\kern0pt}def\ id{\isacharunderscore}{\kern0pt}right{\isacharunderscore}{\kern0pt}unit\ id{\isacharunderscore}{\kern0pt}type\ one{\isacharunderscore}{\kern0pt}unique{\isacharunderscore}{\kern0pt}element{\isacharparenright}{\kern0pt}%
\endisatagproof
{\isafoldproof}%
%
\isadelimproof
\isanewline
%
\endisadelimproof
\isanewline
\isacommand{lemma}\isamarkupfalse%
\ AND{\isacharunderscore}{\kern0pt}false{\isacharunderscore}{\kern0pt}left{\isacharunderscore}{\kern0pt}is{\isacharunderscore}{\kern0pt}false{\isacharcolon}{\kern0pt}\isanewline
\ \ \isakeyword{assumes}\ {\isachardoublequoteopen}p\ {\isasymin}\isactrlsub c\ {\isasymOmega}{\isachardoublequoteclose}\isanewline
\ \ \isakeyword{shows}\ {\isachardoublequoteopen}AND\ {\isasymcirc}\isactrlsub c\ {\isasymlangle}{\isasymf}{\isacharcomma}{\kern0pt}p{\isasymrangle}\ {\isacharequal}{\kern0pt}\ {\isasymf}{\isachardoublequoteclose}\isanewline
%
\isadelimproof
%
\endisadelimproof
%
\isatagproof
\isacommand{proof}\isamarkupfalse%
\ {\isacharparenleft}{\kern0pt}rule\ ccontr{\isacharparenright}{\kern0pt}\isanewline
\ \ \isacommand{assume}\isamarkupfalse%
\ {\isachardoublequoteopen}AND\ {\isasymcirc}\isactrlsub c\ {\isasymlangle}{\isasymf}{\isacharcomma}{\kern0pt}p{\isasymrangle}\ {\isasymnoteq}\ {\isasymf}{\isachardoublequoteclose}\isanewline
\ \ \isacommand{then}\isamarkupfalse%
\ \isacommand{have}\isamarkupfalse%
\ {\isachardoublequoteopen}AND\ {\isasymcirc}\isactrlsub c\ {\isasymlangle}{\isasymf}{\isacharcomma}{\kern0pt}p{\isasymrangle}\ {\isacharequal}{\kern0pt}\ {\isasymt}{\isachardoublequoteclose}\isanewline
\ \ \ \ \isacommand{using}\isamarkupfalse%
\ assms\ true{\isacharunderscore}{\kern0pt}false{\isacharunderscore}{\kern0pt}only{\isacharunderscore}{\kern0pt}truth{\isacharunderscore}{\kern0pt}values\ \isacommand{by}\isamarkupfalse%
\ {\isacharparenleft}{\kern0pt}typecheck{\isacharunderscore}{\kern0pt}cfuncs{\isacharcomma}{\kern0pt}\ blast{\isacharparenright}{\kern0pt}\isanewline
\ \ \isacommand{then}\isamarkupfalse%
\ \isacommand{have}\isamarkupfalse%
\ {\isachardoublequoteopen}{\isasymt}\ {\isasymcirc}\isactrlsub c\ id\ {\isasymone}\ {\isacharequal}{\kern0pt}\ AND\ {\isasymcirc}\isactrlsub c\ {\isasymlangle}{\isasymf}{\isacharcomma}{\kern0pt}p{\isasymrangle}{\isachardoublequoteclose}\isanewline
\ \ \ \ \isacommand{using}\isamarkupfalse%
\ assms\ \isacommand{by}\isamarkupfalse%
\ {\isacharparenleft}{\kern0pt}typecheck{\isacharunderscore}{\kern0pt}cfuncs{\isacharcomma}{\kern0pt}\ simp\ add{\isacharcolon}{\kern0pt}\ id{\isacharunderscore}{\kern0pt}right{\isacharunderscore}{\kern0pt}unit{\isadigit{2}}{\isacharparenright}{\kern0pt}\isanewline
\ \ \isacommand{then}\isamarkupfalse%
\ \isacommand{obtain}\isamarkupfalse%
\ j\ \isakeyword{where}\ j{\isacharunderscore}{\kern0pt}type{\isacharcolon}{\kern0pt}\ {\isachardoublequoteopen}j\ {\isasymin}\isactrlsub c\ {\isasymone}{\isachardoublequoteclose}\ \isakeyword{and}\ j{\isacharunderscore}{\kern0pt}id{\isacharcolon}{\kern0pt}\ {\isachardoublequoteopen}{\isasymbeta}\isactrlbsub {\isasymone}\isactrlesub \ {\isasymcirc}\isactrlsub c\ j\ {\isacharequal}{\kern0pt}\ id\isactrlsub c\ {\isasymone}{\isachardoublequoteclose}\ \isakeyword{and}\ tt{\isacharunderscore}{\kern0pt}j{\isacharunderscore}{\kern0pt}eq{\isacharunderscore}{\kern0pt}fp{\isacharcolon}{\kern0pt}\ {\isachardoublequoteopen}{\isasymlangle}{\isasymt}{\isacharcomma}{\kern0pt}{\isasymt}{\isasymrangle}\ {\isasymcirc}\isactrlsub c\ j\ {\isacharequal}{\kern0pt}\ {\isasymlangle}{\isasymf}{\isacharcomma}{\kern0pt}p{\isasymrangle}{\isachardoublequoteclose}\isanewline
\ \ \ \ \isacommand{using}\isamarkupfalse%
\ AND{\isacharunderscore}{\kern0pt}is{\isacharunderscore}{\kern0pt}pullback\ assms\ \isacommand{unfolding}\isamarkupfalse%
\ is{\isacharunderscore}{\kern0pt}pullback{\isacharunderscore}{\kern0pt}def\ \isacommand{by}\isamarkupfalse%
\ {\isacharparenleft}{\kern0pt}typecheck{\isacharunderscore}{\kern0pt}cfuncs{\isacharcomma}{\kern0pt}\ blast{\isacharparenright}{\kern0pt}\isanewline
\ \ \isacommand{then}\isamarkupfalse%
\ \isacommand{have}\isamarkupfalse%
\ {\isachardoublequoteopen}j\ {\isacharequal}{\kern0pt}\ id\isactrlsub c\ {\isasymone}{\isachardoublequoteclose}\isanewline
\ \ \ \ \isacommand{using}\isamarkupfalse%
\ id{\isacharunderscore}{\kern0pt}type\ one{\isacharunderscore}{\kern0pt}unique{\isacharunderscore}{\kern0pt}element\ \isacommand{by}\isamarkupfalse%
\ auto\isanewline
\ \ \isacommand{then}\isamarkupfalse%
\ \isacommand{have}\isamarkupfalse%
\ {\isachardoublequoteopen}{\isasymlangle}{\isasymt}{\isacharcomma}{\kern0pt}{\isasymt}{\isasymrangle}\ {\isacharequal}{\kern0pt}\ {\isasymlangle}{\isasymf}{\isacharcomma}{\kern0pt}p{\isasymrangle}{\isachardoublequoteclose}\isanewline
\ \ \ \ \isacommand{by}\isamarkupfalse%
\ {\isacharparenleft}{\kern0pt}typecheck{\isacharunderscore}{\kern0pt}cfuncs{\isacharcomma}{\kern0pt}\ metis\ tt{\isacharunderscore}{\kern0pt}j{\isacharunderscore}{\kern0pt}eq{\isacharunderscore}{\kern0pt}fp\ id{\isacharunderscore}{\kern0pt}right{\isacharunderscore}{\kern0pt}unit{\isadigit{2}}{\isacharparenright}{\kern0pt}\isanewline
\ \ \isacommand{then}\isamarkupfalse%
\ \isacommand{have}\isamarkupfalse%
\ {\isachardoublequoteopen}{\isasymt}\ {\isacharequal}{\kern0pt}\ {\isasymf}{\isachardoublequoteclose}\isanewline
\ \ \ \ \isacommand{using}\isamarkupfalse%
\ assms\ cart{\isacharunderscore}{\kern0pt}prod{\isacharunderscore}{\kern0pt}eq{\isadigit{2}}\ \isacommand{by}\isamarkupfalse%
\ {\isacharparenleft}{\kern0pt}typecheck{\isacharunderscore}{\kern0pt}cfuncs{\isacharcomma}{\kern0pt}\ auto{\isacharparenright}{\kern0pt}\isanewline
\ \ \isacommand{then}\isamarkupfalse%
\ \isacommand{show}\isamarkupfalse%
\ False\isanewline
\ \ \ \ \isacommand{using}\isamarkupfalse%
\ true{\isacharunderscore}{\kern0pt}false{\isacharunderscore}{\kern0pt}distinct\ \isacommand{by}\isamarkupfalse%
\ auto\isanewline
\isacommand{qed}\isamarkupfalse%
%
\endisatagproof
{\isafoldproof}%
%
\isadelimproof
\isanewline
%
\endisadelimproof
\isanewline
\isacommand{lemma}\isamarkupfalse%
\ AND{\isacharunderscore}{\kern0pt}false{\isacharunderscore}{\kern0pt}right{\isacharunderscore}{\kern0pt}is{\isacharunderscore}{\kern0pt}false{\isacharcolon}{\kern0pt}\isanewline
\ \ \isakeyword{assumes}\ {\isachardoublequoteopen}p\ {\isasymin}\isactrlsub c\ {\isasymOmega}{\isachardoublequoteclose}\isanewline
\ \ \isakeyword{shows}\ {\isachardoublequoteopen}AND\ {\isasymcirc}\isactrlsub c\ {\isasymlangle}p{\isacharcomma}{\kern0pt}{\isasymf}{\isasymrangle}\ {\isacharequal}{\kern0pt}\ {\isasymf}{\isachardoublequoteclose}\isanewline
%
\isadelimproof
%
\endisadelimproof
%
\isatagproof
\isacommand{proof}\isamarkupfalse%
{\isacharparenleft}{\kern0pt}rule\ ccontr{\isacharparenright}{\kern0pt}\isanewline
\ \ \isacommand{assume}\isamarkupfalse%
\ {\isachardoublequoteopen}AND\ {\isasymcirc}\isactrlsub c\ {\isasymlangle}p{\isacharcomma}{\kern0pt}{\isasymf}{\isasymrangle}\ {\isasymnoteq}\ {\isasymf}{\isachardoublequoteclose}\isanewline
\ \ \isacommand{then}\isamarkupfalse%
\ \isacommand{have}\isamarkupfalse%
\ {\isachardoublequoteopen}AND\ {\isasymcirc}\isactrlsub c\ {\isasymlangle}p{\isacharcomma}{\kern0pt}{\isasymf}{\isasymrangle}\ {\isacharequal}{\kern0pt}\ {\isasymt}{\isachardoublequoteclose}\isanewline
\ \ \ \ \isacommand{using}\isamarkupfalse%
\ assms\ true{\isacharunderscore}{\kern0pt}false{\isacharunderscore}{\kern0pt}only{\isacharunderscore}{\kern0pt}truth{\isacharunderscore}{\kern0pt}values\ \isacommand{by}\isamarkupfalse%
\ {\isacharparenleft}{\kern0pt}typecheck{\isacharunderscore}{\kern0pt}cfuncs{\isacharcomma}{\kern0pt}\ blast{\isacharparenright}{\kern0pt}\isanewline
\ \ \isacommand{then}\isamarkupfalse%
\ \isacommand{have}\isamarkupfalse%
\ {\isachardoublequoteopen}{\isasymt}\ {\isasymcirc}\isactrlsub c\ id\ {\isasymone}\ {\isacharequal}{\kern0pt}\ AND\ {\isasymcirc}\isactrlsub c\ {\isasymlangle}p{\isacharcomma}{\kern0pt}{\isasymf}{\isasymrangle}{\isachardoublequoteclose}\isanewline
\ \ \ \ \isacommand{using}\isamarkupfalse%
\ assms\ \isacommand{by}\isamarkupfalse%
\ {\isacharparenleft}{\kern0pt}typecheck{\isacharunderscore}{\kern0pt}cfuncs{\isacharcomma}{\kern0pt}\ simp\ add{\isacharcolon}{\kern0pt}\ id{\isacharunderscore}{\kern0pt}right{\isacharunderscore}{\kern0pt}unit{\isadigit{2}}{\isacharparenright}{\kern0pt}\isanewline
\ \ \isacommand{then}\isamarkupfalse%
\ \isacommand{obtain}\isamarkupfalse%
\ j\ \isakeyword{where}\ j{\isacharunderscore}{\kern0pt}type{\isacharcolon}{\kern0pt}\ {\isachardoublequoteopen}j\ {\isasymin}\isactrlsub c\ {\isasymone}{\isachardoublequoteclose}\ \isakeyword{and}\ j{\isacharunderscore}{\kern0pt}id{\isacharcolon}{\kern0pt}\ {\isachardoublequoteopen}{\isasymbeta}\isactrlbsub {\isasymone}\isactrlesub \ {\isasymcirc}\isactrlsub c\ j\ {\isacharequal}{\kern0pt}\ id\isactrlsub c\ {\isasymone}{\isachardoublequoteclose}\ \isakeyword{and}\ tt{\isacharunderscore}{\kern0pt}j{\isacharunderscore}{\kern0pt}eq{\isacharunderscore}{\kern0pt}fp{\isacharcolon}{\kern0pt}\ {\isachardoublequoteopen}{\isasymlangle}{\isasymt}{\isacharcomma}{\kern0pt}{\isasymt}{\isasymrangle}\ {\isasymcirc}\isactrlsub c\ j\ {\isacharequal}{\kern0pt}\ {\isasymlangle}p{\isacharcomma}{\kern0pt}{\isasymf}{\isasymrangle}{\isachardoublequoteclose}\isanewline
\ \ \ \ \isacommand{using}\isamarkupfalse%
\ AND{\isacharunderscore}{\kern0pt}is{\isacharunderscore}{\kern0pt}pullback\ assms\ \isacommand{unfolding}\isamarkupfalse%
\ is{\isacharunderscore}{\kern0pt}pullback{\isacharunderscore}{\kern0pt}def\ \isacommand{by}\isamarkupfalse%
\ {\isacharparenleft}{\kern0pt}typecheck{\isacharunderscore}{\kern0pt}cfuncs{\isacharcomma}{\kern0pt}\ blast{\isacharparenright}{\kern0pt}\isanewline
\ \ \isacommand{then}\isamarkupfalse%
\ \isacommand{have}\isamarkupfalse%
\ {\isachardoublequoteopen}j\ {\isacharequal}{\kern0pt}\ id\isactrlsub c\ {\isasymone}{\isachardoublequoteclose}\isanewline
\ \ \ \ \isacommand{using}\isamarkupfalse%
\ id{\isacharunderscore}{\kern0pt}type\ one{\isacharunderscore}{\kern0pt}unique{\isacharunderscore}{\kern0pt}element\ \isacommand{by}\isamarkupfalse%
\ auto\isanewline
\ \ \isacommand{then}\isamarkupfalse%
\ \isacommand{have}\isamarkupfalse%
\ {\isachardoublequoteopen}{\isasymlangle}{\isasymt}{\isacharcomma}{\kern0pt}{\isasymt}{\isasymrangle}\ {\isacharequal}{\kern0pt}\ {\isasymlangle}p{\isacharcomma}{\kern0pt}{\isasymf}{\isasymrangle}{\isachardoublequoteclose}\isanewline
\ \ \ \ \isacommand{by}\isamarkupfalse%
\ {\isacharparenleft}{\kern0pt}typecheck{\isacharunderscore}{\kern0pt}cfuncs{\isacharcomma}{\kern0pt}\ metis\ tt{\isacharunderscore}{\kern0pt}j{\isacharunderscore}{\kern0pt}eq{\isacharunderscore}{\kern0pt}fp\ id{\isacharunderscore}{\kern0pt}right{\isacharunderscore}{\kern0pt}unit{\isadigit{2}}{\isacharparenright}{\kern0pt}\isanewline
\ \ \isacommand{then}\isamarkupfalse%
\ \isacommand{have}\isamarkupfalse%
\ {\isachardoublequoteopen}{\isasymt}\ {\isacharequal}{\kern0pt}\ {\isasymf}{\isachardoublequoteclose}\isanewline
\ \ \ \ \isacommand{using}\isamarkupfalse%
\ assms\ cart{\isacharunderscore}{\kern0pt}prod{\isacharunderscore}{\kern0pt}eq{\isadigit{2}}\ \isacommand{by}\isamarkupfalse%
\ {\isacharparenleft}{\kern0pt}typecheck{\isacharunderscore}{\kern0pt}cfuncs{\isacharcomma}{\kern0pt}\ auto{\isacharparenright}{\kern0pt}\isanewline
\ \ \isacommand{then}\isamarkupfalse%
\ \isacommand{show}\isamarkupfalse%
\ {\isachardoublequoteopen}False{\isachardoublequoteclose}\isanewline
\ \ \ \ \isacommand{using}\isamarkupfalse%
\ true{\isacharunderscore}{\kern0pt}false{\isacharunderscore}{\kern0pt}distinct\ \isacommand{by}\isamarkupfalse%
\ auto\isanewline
\isacommand{qed}\isamarkupfalse%
%
\endisatagproof
{\isafoldproof}%
%
\isadelimproof
\isanewline
%
\endisadelimproof
\isanewline
\isacommand{lemma}\isamarkupfalse%
\ AND{\isacharunderscore}{\kern0pt}commutative{\isacharcolon}{\kern0pt}\isanewline
\ \ \isakeyword{assumes}\ {\isachardoublequoteopen}p\ {\isasymin}\isactrlsub c\ {\isasymOmega}{\isachardoublequoteclose}\isanewline
\ \ \isakeyword{assumes}\ {\isachardoublequoteopen}q\ {\isasymin}\isactrlsub c\ {\isasymOmega}{\isachardoublequoteclose}\isanewline
\ \ \isakeyword{shows}\ {\isachardoublequoteopen}AND\ {\isasymcirc}\isactrlsub c\ {\isasymlangle}p{\isacharcomma}{\kern0pt}q{\isasymrangle}\ {\isacharequal}{\kern0pt}\ AND\ {\isasymcirc}\isactrlsub c\ {\isasymlangle}q{\isacharcomma}{\kern0pt}p{\isasymrangle}{\isachardoublequoteclose}\isanewline
%
\isadelimproof
\ \ %
\endisadelimproof
%
\isatagproof
\isacommand{by}\isamarkupfalse%
\ {\isacharparenleft}{\kern0pt}metis\ AND{\isacharunderscore}{\kern0pt}false{\isacharunderscore}{\kern0pt}left{\isacharunderscore}{\kern0pt}is{\isacharunderscore}{\kern0pt}false\ AND{\isacharunderscore}{\kern0pt}false{\isacharunderscore}{\kern0pt}right{\isacharunderscore}{\kern0pt}is{\isacharunderscore}{\kern0pt}false\ assms\ true{\isacharunderscore}{\kern0pt}false{\isacharunderscore}{\kern0pt}only{\isacharunderscore}{\kern0pt}truth{\isacharunderscore}{\kern0pt}values{\isacharparenright}{\kern0pt}%
\endisatagproof
{\isafoldproof}%
%
\isadelimproof
\isanewline
%
\endisadelimproof
\isanewline
\isacommand{lemma}\isamarkupfalse%
\ AND{\isacharunderscore}{\kern0pt}idempotent{\isacharcolon}{\kern0pt}\isanewline
\ \ \isakeyword{assumes}\ {\isachardoublequoteopen}p\ {\isasymin}\isactrlsub c\ {\isasymOmega}{\isachardoublequoteclose}\isanewline
\ \ \isakeyword{shows}\ {\isachardoublequoteopen}AND\ {\isasymcirc}\isactrlsub c\ {\isasymlangle}p{\isacharcomma}{\kern0pt}p{\isasymrangle}\ {\isacharequal}{\kern0pt}\ p{\isachardoublequoteclose}\isanewline
%
\isadelimproof
\ \ %
\endisadelimproof
%
\isatagproof
\isacommand{using}\isamarkupfalse%
\ AND{\isacharunderscore}{\kern0pt}false{\isacharunderscore}{\kern0pt}right{\isacharunderscore}{\kern0pt}is{\isacharunderscore}{\kern0pt}false\ AND{\isacharunderscore}{\kern0pt}true{\isacharunderscore}{\kern0pt}true{\isacharunderscore}{\kern0pt}is{\isacharunderscore}{\kern0pt}true\ assms\ true{\isacharunderscore}{\kern0pt}false{\isacharunderscore}{\kern0pt}only{\isacharunderscore}{\kern0pt}truth{\isacharunderscore}{\kern0pt}values\ \isacommand{by}\isamarkupfalse%
\ blast%
\endisatagproof
{\isafoldproof}%
%
\isadelimproof
\isanewline
%
\endisadelimproof
\isanewline
\isacommand{lemma}\isamarkupfalse%
\ AND{\isacharunderscore}{\kern0pt}associative{\isacharcolon}{\kern0pt}\isanewline
\ \ \isakeyword{assumes}\ {\isachardoublequoteopen}p\ {\isasymin}\isactrlsub c\ {\isasymOmega}{\isachardoublequoteclose}\isanewline
\ \ \isakeyword{assumes}\ {\isachardoublequoteopen}q\ {\isasymin}\isactrlsub c\ {\isasymOmega}{\isachardoublequoteclose}\isanewline
\ \ \isakeyword{assumes}\ {\isachardoublequoteopen}r\ {\isasymin}\isactrlsub c\ {\isasymOmega}{\isachardoublequoteclose}\isanewline
\ \ \isakeyword{shows}\ {\isachardoublequoteopen}AND\ {\isasymcirc}\isactrlsub c\ {\isasymlangle}AND\ {\isasymcirc}\isactrlsub c\ {\isasymlangle}p{\isacharcomma}{\kern0pt}q{\isasymrangle}{\isacharcomma}{\kern0pt}\ r{\isasymrangle}\ {\isacharequal}{\kern0pt}\ AND\ {\isasymcirc}\isactrlsub c\ {\isasymlangle}p{\isacharcomma}{\kern0pt}\ AND\ {\isasymcirc}\isactrlsub c\ {\isasymlangle}q{\isacharcomma}{\kern0pt}r{\isasymrangle}{\isasymrangle}{\isachardoublequoteclose}\isanewline
%
\isadelimproof
\ \ %
\endisadelimproof
%
\isatagproof
\isacommand{by}\isamarkupfalse%
\ {\isacharparenleft}{\kern0pt}metis\ AND{\isacharunderscore}{\kern0pt}commutative\ AND{\isacharunderscore}{\kern0pt}false{\isacharunderscore}{\kern0pt}left{\isacharunderscore}{\kern0pt}is{\isacharunderscore}{\kern0pt}false\ AND{\isacharunderscore}{\kern0pt}true{\isacharunderscore}{\kern0pt}true{\isacharunderscore}{\kern0pt}is{\isacharunderscore}{\kern0pt}true\ assms\ true{\isacharunderscore}{\kern0pt}false{\isacharunderscore}{\kern0pt}only{\isacharunderscore}{\kern0pt}truth{\isacharunderscore}{\kern0pt}values{\isacharparenright}{\kern0pt}%
\endisatagproof
{\isafoldproof}%
%
\isadelimproof
\isanewline
%
\endisadelimproof
\isanewline
\isacommand{lemma}\isamarkupfalse%
\ AND{\isacharunderscore}{\kern0pt}complementary{\isacharcolon}{\kern0pt}\isanewline
\ \ \isakeyword{assumes}\ {\isachardoublequoteopen}p\ {\isasymin}\isactrlsub c\ {\isasymOmega}{\isachardoublequoteclose}\isanewline
\ \ \isakeyword{shows}\ {\isachardoublequoteopen}AND\ {\isasymcirc}\isactrlsub c\ {\isasymlangle}p{\isacharcomma}{\kern0pt}\ NOT\ {\isasymcirc}\isactrlsub c\ p{\isasymrangle}\ {\isacharequal}{\kern0pt}\ \ {\isasymf}{\isachardoublequoteclose}\isanewline
%
\isadelimproof
\ \ %
\endisadelimproof
%
\isatagproof
\isacommand{by}\isamarkupfalse%
\ {\isacharparenleft}{\kern0pt}metis\ AND{\isacharunderscore}{\kern0pt}false{\isacharunderscore}{\kern0pt}left{\isacharunderscore}{\kern0pt}is{\isacharunderscore}{\kern0pt}false\ AND{\isacharunderscore}{\kern0pt}false{\isacharunderscore}{\kern0pt}right{\isacharunderscore}{\kern0pt}is{\isacharunderscore}{\kern0pt}false\ NOT{\isacharunderscore}{\kern0pt}false{\isacharunderscore}{\kern0pt}is{\isacharunderscore}{\kern0pt}true\ NOT{\isacharunderscore}{\kern0pt}true{\isacharunderscore}{\kern0pt}is{\isacharunderscore}{\kern0pt}false\ assms\ true{\isacharunderscore}{\kern0pt}false{\isacharunderscore}{\kern0pt}only{\isacharunderscore}{\kern0pt}truth{\isacharunderscore}{\kern0pt}values\ true{\isacharunderscore}{\kern0pt}func{\isacharunderscore}{\kern0pt}type{\isacharparenright}{\kern0pt}%
\endisatagproof
{\isafoldproof}%
%
\isadelimproof
%
\endisadelimproof
%
\isadelimdocument
%
\endisadelimdocument
%
\isatagdocument
%
\isamarkupsubsection{NOR%
}
\isamarkuptrue%
%
\endisatagdocument
{\isafolddocument}%
%
\isadelimdocument
%
\endisadelimdocument
\isacommand{definition}\isamarkupfalse%
\ NOR\ {\isacharcolon}{\kern0pt}{\isacharcolon}{\kern0pt}\ {\isachardoublequoteopen}cfunc{\isachardoublequoteclose}\ \isakeyword{where}\isanewline
\ \ {\isachardoublequoteopen}NOR\ {\isacharequal}{\kern0pt}\ {\isacharparenleft}{\kern0pt}THE\ {\isasymchi}{\isachardot}{\kern0pt}\ is{\isacharunderscore}{\kern0pt}pullback\ \ {\isasymone}\ {\isasymone}\ {\isacharparenleft}{\kern0pt}{\isasymOmega}\ {\isasymtimes}\isactrlsub c\ {\isasymOmega}{\isacharparenright}{\kern0pt}\ {\isasymOmega}\ {\isacharparenleft}{\kern0pt}{\isasymbeta}\isactrlbsub {\isasymone}\isactrlesub {\isacharparenright}{\kern0pt}\ {\isasymt}\ {\isasymlangle}{\isasymf}{\isacharcomma}{\kern0pt}\ {\isasymf}{\isasymrangle}\ {\isasymchi}{\isacharparenright}{\kern0pt}{\isachardoublequoteclose}\isanewline
\isanewline
\isacommand{lemma}\isamarkupfalse%
\ NOR{\isacharunderscore}{\kern0pt}is{\isacharunderscore}{\kern0pt}pullback{\isacharcolon}{\kern0pt}\isanewline
\ \ {\isachardoublequoteopen}is{\isacharunderscore}{\kern0pt}pullback\ \ {\isasymone}\ {\isasymone}\ {\isacharparenleft}{\kern0pt}{\isasymOmega}\ {\isasymtimes}\isactrlsub c\ {\isasymOmega}{\isacharparenright}{\kern0pt}\ {\isasymOmega}\ {\isacharparenleft}{\kern0pt}{\isasymbeta}\isactrlbsub {\isasymone}\isactrlesub {\isacharparenright}{\kern0pt}\ {\isasymt}\ {\isasymlangle}{\isasymf}{\isacharcomma}{\kern0pt}\ {\isasymf}{\isasymrangle}\ NOR{\isachardoublequoteclose}\isanewline
%
\isadelimproof
\ \ %
\endisadelimproof
%
\isatagproof
\isacommand{unfolding}\isamarkupfalse%
\ NOR{\isacharunderscore}{\kern0pt}def\isanewline
\ \ \isacommand{using}\isamarkupfalse%
\ characteristic{\isacharunderscore}{\kern0pt}function{\isacharunderscore}{\kern0pt}exists\ element{\isacharunderscore}{\kern0pt}monomorphism\isanewline
\ \ \isacommand{by}\isamarkupfalse%
\ {\isacharparenleft}{\kern0pt}typecheck{\isacharunderscore}{\kern0pt}cfuncs{\isacharcomma}{\kern0pt}\ simp\ add{\isacharcolon}{\kern0pt}\ the{\isadigit{1}}I{\isadigit{2}}{\isacharparenright}{\kern0pt}%
\endisatagproof
{\isafoldproof}%
%
\isadelimproof
\isanewline
%
\endisadelimproof
\isanewline
\isacommand{lemma}\isamarkupfalse%
\ NOR{\isacharunderscore}{\kern0pt}type{\isacharbrackleft}{\kern0pt}type{\isacharunderscore}{\kern0pt}rule{\isacharbrackright}{\kern0pt}{\isacharcolon}{\kern0pt}\isanewline
\ \ {\isachardoublequoteopen}NOR\ {\isacharcolon}{\kern0pt}\ {\isasymOmega}\ {\isasymtimes}\isactrlsub c\ {\isasymOmega}\ {\isasymrightarrow}\ {\isasymOmega}{\isachardoublequoteclose}\isanewline
%
\isadelimproof
\ \ %
\endisadelimproof
%
\isatagproof
\isacommand{using}\isamarkupfalse%
\ NOR{\isacharunderscore}{\kern0pt}is{\isacharunderscore}{\kern0pt}pullback\ \isacommand{unfolding}\isamarkupfalse%
\ is{\isacharunderscore}{\kern0pt}pullback{\isacharunderscore}{\kern0pt}def\ \isacommand{by}\isamarkupfalse%
\ auto%
\endisatagproof
{\isafoldproof}%
%
\isadelimproof
\isanewline
%
\endisadelimproof
\isanewline
\isacommand{lemma}\isamarkupfalse%
\ NOR{\isacharunderscore}{\kern0pt}false{\isacharunderscore}{\kern0pt}false{\isacharunderscore}{\kern0pt}is{\isacharunderscore}{\kern0pt}true{\isacharcolon}{\kern0pt}\isanewline
\ \ {\isachardoublequoteopen}NOR\ {\isasymcirc}\isactrlsub c\ {\isasymlangle}{\isasymf}{\isacharcomma}{\kern0pt}{\isasymf}{\isasymrangle}\ {\isacharequal}{\kern0pt}\ {\isasymt}{\isachardoublequoteclose}\isanewline
%
\isadelimproof
\ \ %
\endisadelimproof
%
\isatagproof
\isacommand{using}\isamarkupfalse%
\ NOR{\isacharunderscore}{\kern0pt}is{\isacharunderscore}{\kern0pt}pullback\ \isacommand{unfolding}\isamarkupfalse%
\ is{\isacharunderscore}{\kern0pt}pullback{\isacharunderscore}{\kern0pt}def\ \isanewline
\ \ \isacommand{by}\isamarkupfalse%
\ {\isacharparenleft}{\kern0pt}auto{\isacharcomma}{\kern0pt}\ metis\ cfunc{\isacharunderscore}{\kern0pt}type{\isacharunderscore}{\kern0pt}def\ id{\isacharunderscore}{\kern0pt}right{\isacharunderscore}{\kern0pt}unit\ id{\isacharunderscore}{\kern0pt}type\ one{\isacharunderscore}{\kern0pt}unique{\isacharunderscore}{\kern0pt}element{\isacharparenright}{\kern0pt}%
\endisatagproof
{\isafoldproof}%
%
\isadelimproof
\isanewline
%
\endisadelimproof
\isanewline
\isacommand{lemma}\isamarkupfalse%
\ NOR{\isacharunderscore}{\kern0pt}left{\isacharunderscore}{\kern0pt}true{\isacharunderscore}{\kern0pt}is{\isacharunderscore}{\kern0pt}false{\isacharcolon}{\kern0pt}\isanewline
\ \ \isakeyword{assumes}\ {\isachardoublequoteopen}p\ {\isasymin}\isactrlsub c\ {\isasymOmega}{\isachardoublequoteclose}\isanewline
\ \ \isakeyword{shows}\ {\isachardoublequoteopen}NOR\ {\isasymcirc}\isactrlsub c\ {\isasymlangle}{\isasymt}{\isacharcomma}{\kern0pt}p{\isasymrangle}\ {\isacharequal}{\kern0pt}\ {\isasymf}{\isachardoublequoteclose}\isanewline
%
\isadelimproof
%
\endisadelimproof
%
\isatagproof
\isacommand{proof}\isamarkupfalse%
\ {\isacharparenleft}{\kern0pt}rule\ ccontr{\isacharparenright}{\kern0pt}\isanewline
\ \ \isacommand{assume}\isamarkupfalse%
\ {\isachardoublequoteopen}NOR\ {\isasymcirc}\isactrlsub c\ {\isasymlangle}{\isasymt}{\isacharcomma}{\kern0pt}p{\isasymrangle}\ {\isasymnoteq}\ {\isasymf}{\isachardoublequoteclose}\isanewline
\ \ \isacommand{then}\isamarkupfalse%
\ \isacommand{have}\isamarkupfalse%
\ {\isachardoublequoteopen}NOR\ {\isasymcirc}\isactrlsub c\ {\isasymlangle}{\isasymt}{\isacharcomma}{\kern0pt}p{\isasymrangle}\ {\isacharequal}{\kern0pt}\ {\isasymt}{\isachardoublequoteclose}\isanewline
\ \ \ \ \isacommand{using}\isamarkupfalse%
\ assms\ true{\isacharunderscore}{\kern0pt}false{\isacharunderscore}{\kern0pt}only{\isacharunderscore}{\kern0pt}truth{\isacharunderscore}{\kern0pt}values\ \isacommand{by}\isamarkupfalse%
\ {\isacharparenleft}{\kern0pt}typecheck{\isacharunderscore}{\kern0pt}cfuncs{\isacharcomma}{\kern0pt}\ blast{\isacharparenright}{\kern0pt}\isanewline
\ \ \isacommand{then}\isamarkupfalse%
\ \isacommand{have}\isamarkupfalse%
\ {\isachardoublequoteopen}NOR\ {\isasymcirc}\isactrlsub c\ {\isasymlangle}{\isasymt}{\isacharcomma}{\kern0pt}p{\isasymrangle}\ {\isacharequal}{\kern0pt}\ {\isasymt}\ {\isasymcirc}\isactrlsub c\ id\ {\isasymone}{\isachardoublequoteclose}\isanewline
\ \ \ \ \isacommand{using}\isamarkupfalse%
\ id{\isacharunderscore}{\kern0pt}right{\isacharunderscore}{\kern0pt}unit{\isadigit{2}}\ true{\isacharunderscore}{\kern0pt}func{\isacharunderscore}{\kern0pt}type\ \isacommand{by}\isamarkupfalse%
\ auto\isanewline
\ \ \isacommand{then}\isamarkupfalse%
\ \isacommand{obtain}\isamarkupfalse%
\ j\ \isakeyword{where}\ j{\isacharunderscore}{\kern0pt}type{\isacharcolon}{\kern0pt}\ {\isachardoublequoteopen}j\ {\isasymin}\isactrlsub c\ {\isasymone}{\isachardoublequoteclose}\ \isakeyword{and}\ j{\isacharunderscore}{\kern0pt}id{\isacharcolon}{\kern0pt}\ {\isachardoublequoteopen}{\isasymbeta}\isactrlbsub {\isasymone}\isactrlesub \ {\isasymcirc}\isactrlsub c\ j\ {\isacharequal}{\kern0pt}\ id\ {\isasymone}{\isachardoublequoteclose}\ \isakeyword{and}\ ff{\isacharunderscore}{\kern0pt}j{\isacharunderscore}{\kern0pt}eq{\isacharunderscore}{\kern0pt}tp{\isacharcolon}{\kern0pt}\ {\isachardoublequoteopen}{\isasymlangle}{\isasymf}{\isacharcomma}{\kern0pt}{\isasymf}{\isasymrangle}\ {\isasymcirc}\isactrlsub c\ j\ {\isacharequal}{\kern0pt}\ {\isasymlangle}{\isasymt}{\isacharcomma}{\kern0pt}p{\isasymrangle}{\isachardoublequoteclose}\isanewline
\ \ \ \ \isacommand{using}\isamarkupfalse%
\ NOR{\isacharunderscore}{\kern0pt}is{\isacharunderscore}{\kern0pt}pullback\ assms\ \isacommand{unfolding}\isamarkupfalse%
\ is{\isacharunderscore}{\kern0pt}pullback{\isacharunderscore}{\kern0pt}def\ \isacommand{by}\isamarkupfalse%
\ {\isacharparenleft}{\kern0pt}typecheck{\isacharunderscore}{\kern0pt}cfuncs{\isacharcomma}{\kern0pt}\ metis{\isacharparenright}{\kern0pt}\isanewline
\ \ \isacommand{then}\isamarkupfalse%
\ \isacommand{have}\isamarkupfalse%
\ {\isachardoublequoteopen}j\ {\isacharequal}{\kern0pt}\ id\ {\isasymone}{\isachardoublequoteclose}\isanewline
\ \ \ \ \isacommand{using}\isamarkupfalse%
\ id{\isacharunderscore}{\kern0pt}type\ one{\isacharunderscore}{\kern0pt}unique{\isacharunderscore}{\kern0pt}element\ \isacommand{by}\isamarkupfalse%
\ blast\isanewline
\ \ \isacommand{then}\isamarkupfalse%
\ \isacommand{have}\isamarkupfalse%
\ {\isachardoublequoteopen}{\isasymlangle}{\isasymf}{\isacharcomma}{\kern0pt}{\isasymf}{\isasymrangle}\ {\isacharequal}{\kern0pt}\ {\isasymlangle}{\isasymt}{\isacharcomma}{\kern0pt}p{\isasymrangle}{\isachardoublequoteclose}\isanewline
\ \ \ \ \isacommand{using}\isamarkupfalse%
\ cfunc{\isacharunderscore}{\kern0pt}prod{\isacharunderscore}{\kern0pt}comp\ false{\isacharunderscore}{\kern0pt}func{\isacharunderscore}{\kern0pt}type\ ff{\isacharunderscore}{\kern0pt}j{\isacharunderscore}{\kern0pt}eq{\isacharunderscore}{\kern0pt}tp\ id{\isacharunderscore}{\kern0pt}right{\isacharunderscore}{\kern0pt}unit{\isadigit{2}}\ j{\isacharunderscore}{\kern0pt}type\ \isacommand{by}\isamarkupfalse%
\ auto\isanewline
\ \ \isacommand{then}\isamarkupfalse%
\ \isacommand{have}\isamarkupfalse%
\ {\isachardoublequoteopen}{\isasymf}\ {\isacharequal}{\kern0pt}\ {\isasymt}{\isachardoublequoteclose}\isanewline
\ \ \ \ \isacommand{using}\isamarkupfalse%
\ assms\ cart{\isacharunderscore}{\kern0pt}prod{\isacharunderscore}{\kern0pt}eq{\isadigit{2}}\ false{\isacharunderscore}{\kern0pt}func{\isacharunderscore}{\kern0pt}type\ true{\isacharunderscore}{\kern0pt}func{\isacharunderscore}{\kern0pt}type\ \isacommand{by}\isamarkupfalse%
\ auto\isanewline
\ \ \isacommand{then}\isamarkupfalse%
\ \isacommand{show}\isamarkupfalse%
\ False\isanewline
\ \ \ \ \isacommand{using}\isamarkupfalse%
\ true{\isacharunderscore}{\kern0pt}false{\isacharunderscore}{\kern0pt}distinct\ \isacommand{by}\isamarkupfalse%
\ auto\isanewline
\isacommand{qed}\isamarkupfalse%
%
\endisatagproof
{\isafoldproof}%
%
\isadelimproof
\isanewline
%
\endisadelimproof
\isanewline
\isacommand{lemma}\isamarkupfalse%
\ NOR{\isacharunderscore}{\kern0pt}right{\isacharunderscore}{\kern0pt}true{\isacharunderscore}{\kern0pt}is{\isacharunderscore}{\kern0pt}false{\isacharcolon}{\kern0pt}\isanewline
\ \ \isakeyword{assumes}\ {\isachardoublequoteopen}p\ {\isasymin}\isactrlsub c\ {\isasymOmega}{\isachardoublequoteclose}\isanewline
\ \ \isakeyword{shows}\ {\isachardoublequoteopen}NOR\ {\isasymcirc}\isactrlsub c\ {\isasymlangle}p{\isacharcomma}{\kern0pt}{\isasymt}{\isasymrangle}\ {\isacharequal}{\kern0pt}\ {\isasymf}{\isachardoublequoteclose}\isanewline
%
\isadelimproof
%
\endisadelimproof
%
\isatagproof
\isacommand{proof}\isamarkupfalse%
\ {\isacharparenleft}{\kern0pt}rule\ ccontr{\isacharparenright}{\kern0pt}\isanewline
\ \ \isacommand{assume}\isamarkupfalse%
\ {\isachardoublequoteopen}NOR\ {\isasymcirc}\isactrlsub c\ {\isasymlangle}p{\isacharcomma}{\kern0pt}{\isasymt}{\isasymrangle}\ {\isasymnoteq}\ {\isasymf}{\isachardoublequoteclose}\isanewline
\ \ \isacommand{then}\isamarkupfalse%
\ \isacommand{have}\isamarkupfalse%
\ {\isachardoublequoteopen}NOR\ {\isasymcirc}\isactrlsub c\ {\isasymlangle}p{\isacharcomma}{\kern0pt}{\isasymt}{\isasymrangle}\ {\isacharequal}{\kern0pt}\ {\isasymt}{\isachardoublequoteclose}\isanewline
\ \ \ \ \isacommand{using}\isamarkupfalse%
\ assms\ true{\isacharunderscore}{\kern0pt}false{\isacharunderscore}{\kern0pt}only{\isacharunderscore}{\kern0pt}truth{\isacharunderscore}{\kern0pt}values\ \isacommand{by}\isamarkupfalse%
\ {\isacharparenleft}{\kern0pt}typecheck{\isacharunderscore}{\kern0pt}cfuncs{\isacharcomma}{\kern0pt}\ blast{\isacharparenright}{\kern0pt}\isanewline
\ \ \isacommand{then}\isamarkupfalse%
\ \isacommand{have}\isamarkupfalse%
\ {\isachardoublequoteopen}NOR\ {\isasymcirc}\isactrlsub c\ {\isasymlangle}p{\isacharcomma}{\kern0pt}{\isasymt}{\isasymrangle}\ {\isacharequal}{\kern0pt}\ {\isasymt}\ {\isasymcirc}\isactrlsub c\ id\ {\isasymone}{\isachardoublequoteclose}\isanewline
\ \ \ \ \isacommand{using}\isamarkupfalse%
\ id{\isacharunderscore}{\kern0pt}right{\isacharunderscore}{\kern0pt}unit{\isadigit{2}}\ true{\isacharunderscore}{\kern0pt}func{\isacharunderscore}{\kern0pt}type\ \isacommand{by}\isamarkupfalse%
\ auto\isanewline
\ \ \isacommand{then}\isamarkupfalse%
\ \isacommand{obtain}\isamarkupfalse%
\ j\ \isakeyword{where}\ j{\isacharunderscore}{\kern0pt}type{\isacharcolon}{\kern0pt}\ {\isachardoublequoteopen}j\ {\isasymin}\isactrlsub c\ {\isasymone}{\isachardoublequoteclose}\ \isakeyword{and}\ j{\isacharunderscore}{\kern0pt}id{\isacharcolon}{\kern0pt}\ {\isachardoublequoteopen}{\isasymbeta}\isactrlbsub {\isasymone}\isactrlesub \ {\isasymcirc}\isactrlsub c\ j\ {\isacharequal}{\kern0pt}\ id\ {\isasymone}{\isachardoublequoteclose}\ \isakeyword{and}\ ff{\isacharunderscore}{\kern0pt}j{\isacharunderscore}{\kern0pt}eq{\isacharunderscore}{\kern0pt}tp{\isacharcolon}{\kern0pt}\ {\isachardoublequoteopen}{\isasymlangle}{\isasymf}{\isacharcomma}{\kern0pt}{\isasymf}{\isasymrangle}\ {\isasymcirc}\isactrlsub c\ j\ {\isacharequal}{\kern0pt}\ {\isasymlangle}p{\isacharcomma}{\kern0pt}{\isasymt}{\isasymrangle}{\isachardoublequoteclose}\isanewline
\ \ \ \ \isacommand{using}\isamarkupfalse%
\ NOR{\isacharunderscore}{\kern0pt}is{\isacharunderscore}{\kern0pt}pullback\ assms\ \isacommand{unfolding}\isamarkupfalse%
\ is{\isacharunderscore}{\kern0pt}pullback{\isacharunderscore}{\kern0pt}def\ \isacommand{by}\isamarkupfalse%
\ {\isacharparenleft}{\kern0pt}typecheck{\isacharunderscore}{\kern0pt}cfuncs{\isacharcomma}{\kern0pt}\ metis{\isacharparenright}{\kern0pt}\isanewline
\ \ \isacommand{then}\isamarkupfalse%
\ \isacommand{have}\isamarkupfalse%
\ {\isachardoublequoteopen}j\ {\isacharequal}{\kern0pt}\ id\ {\isasymone}{\isachardoublequoteclose}\isanewline
\ \ \ \ \isacommand{using}\isamarkupfalse%
\ id{\isacharunderscore}{\kern0pt}type\ one{\isacharunderscore}{\kern0pt}unique{\isacharunderscore}{\kern0pt}element\ \isacommand{by}\isamarkupfalse%
\ blast\isanewline
\ \ \isacommand{then}\isamarkupfalse%
\ \isacommand{have}\isamarkupfalse%
\ {\isachardoublequoteopen}{\isasymlangle}{\isasymf}{\isacharcomma}{\kern0pt}{\isasymf}{\isasymrangle}\ {\isacharequal}{\kern0pt}\ {\isasymlangle}p{\isacharcomma}{\kern0pt}{\isasymt}{\isasymrangle}{\isachardoublequoteclose}\isanewline
\ \ \ \ \isacommand{using}\isamarkupfalse%
\ cfunc{\isacharunderscore}{\kern0pt}prod{\isacharunderscore}{\kern0pt}comp\ false{\isacharunderscore}{\kern0pt}func{\isacharunderscore}{\kern0pt}type\ ff{\isacharunderscore}{\kern0pt}j{\isacharunderscore}{\kern0pt}eq{\isacharunderscore}{\kern0pt}tp\ id{\isacharunderscore}{\kern0pt}right{\isacharunderscore}{\kern0pt}unit{\isadigit{2}}\ j{\isacharunderscore}{\kern0pt}type\ \isacommand{by}\isamarkupfalse%
\ auto\isanewline
\ \ \isacommand{then}\isamarkupfalse%
\ \isacommand{have}\isamarkupfalse%
\ {\isachardoublequoteopen}{\isasymf}\ {\isacharequal}{\kern0pt}\ {\isasymt}{\isachardoublequoteclose}\isanewline
\ \ \ \ \isacommand{using}\isamarkupfalse%
\ assms\ cart{\isacharunderscore}{\kern0pt}prod{\isacharunderscore}{\kern0pt}eq{\isadigit{2}}\ false{\isacharunderscore}{\kern0pt}func{\isacharunderscore}{\kern0pt}type\ true{\isacharunderscore}{\kern0pt}func{\isacharunderscore}{\kern0pt}type\ \isacommand{by}\isamarkupfalse%
\ auto\isanewline
\ \ \isacommand{then}\isamarkupfalse%
\ \isacommand{show}\isamarkupfalse%
\ False\isanewline
\ \ \ \ \isacommand{using}\isamarkupfalse%
\ true{\isacharunderscore}{\kern0pt}false{\isacharunderscore}{\kern0pt}distinct\ \isacommand{by}\isamarkupfalse%
\ auto\isanewline
\isacommand{qed}\isamarkupfalse%
%
\endisatagproof
{\isafoldproof}%
%
\isadelimproof
\isanewline
%
\endisadelimproof
\isanewline
\isacommand{lemma}\isamarkupfalse%
\ NOR{\isacharunderscore}{\kern0pt}true{\isacharunderscore}{\kern0pt}implies{\isacharunderscore}{\kern0pt}both{\isacharunderscore}{\kern0pt}false{\isacharcolon}{\kern0pt}\isanewline
\ \ \isakeyword{assumes}\ X{\isacharunderscore}{\kern0pt}nonempty{\isacharcolon}{\kern0pt}\ {\isachardoublequoteopen}nonempty\ X{\isachardoublequoteclose}\ \isakeyword{and}\ Y{\isacharunderscore}{\kern0pt}nonempty{\isacharcolon}{\kern0pt}\ {\isachardoublequoteopen}nonempty\ Y{\isachardoublequoteclose}\isanewline
\ \ \isakeyword{assumes}\ P{\isacharunderscore}{\kern0pt}Q{\isacharunderscore}{\kern0pt}types{\isacharbrackleft}{\kern0pt}type{\isacharunderscore}{\kern0pt}rule{\isacharbrackright}{\kern0pt}{\isacharcolon}{\kern0pt}\ {\isachardoublequoteopen}P\ {\isacharcolon}{\kern0pt}\ X\ {\isasymrightarrow}\ {\isasymOmega}{\isachardoublequoteclose}\ {\isachardoublequoteopen}Q\ {\isacharcolon}{\kern0pt}\ Y\ {\isasymrightarrow}\ {\isasymOmega}{\isachardoublequoteclose}\isanewline
\ \ \isakeyword{assumes}\ NOR{\isacharunderscore}{\kern0pt}true{\isacharcolon}{\kern0pt}\ {\isachardoublequoteopen}NOR\ {\isasymcirc}\isactrlsub c\ {\isacharparenleft}{\kern0pt}P\ {\isasymtimes}\isactrlsub f\ Q{\isacharparenright}{\kern0pt}\ {\isacharequal}{\kern0pt}\ {\isasymt}\ {\isasymcirc}\isactrlsub c\ {\isasymbeta}\isactrlbsub X\ {\isasymtimes}\isactrlsub c\ Y\isactrlesub {\isachardoublequoteclose}\isanewline
\ \ \isakeyword{shows}\ {\isachardoublequoteopen}P\ {\isacharequal}{\kern0pt}\ {\isasymf}\ {\isasymcirc}\isactrlsub c\ {\isasymbeta}\isactrlbsub X\isactrlesub \ {\isasymand}\ Q\ {\isacharequal}{\kern0pt}\ {\isasymf}\ {\isasymcirc}\isactrlsub c\ {\isasymbeta}\isactrlbsub Y\isactrlesub {\isachardoublequoteclose}\isanewline
%
\isadelimproof
%
\endisadelimproof
%
\isatagproof
\isacommand{proof}\isamarkupfalse%
\ {\isacharminus}{\kern0pt}\isanewline
\ \ \isacommand{obtain}\isamarkupfalse%
\ z\ \isakeyword{where}\ z{\isacharunderscore}{\kern0pt}type{\isacharbrackleft}{\kern0pt}type{\isacharunderscore}{\kern0pt}rule{\isacharbrackright}{\kern0pt}{\isacharcolon}{\kern0pt}\ {\isachardoublequoteopen}z\ {\isacharcolon}{\kern0pt}\ X\ {\isasymtimes}\isactrlsub c\ Y\ {\isasymrightarrow}\ {\isasymone}{\isachardoublequoteclose}\ \isakeyword{and}\ {\isachardoublequoteopen}P\ {\isasymtimes}\isactrlsub f\ Q\ {\isacharequal}{\kern0pt}\ {\isasymlangle}{\isasymf}{\isacharcomma}{\kern0pt}{\isasymf}{\isasymrangle}\ {\isasymcirc}\isactrlsub c\ z{\isachardoublequoteclose}\isanewline
\ \ \ \ \isacommand{using}\isamarkupfalse%
\ NOR{\isacharunderscore}{\kern0pt}is{\isacharunderscore}{\kern0pt}pullback\ NOR{\isacharunderscore}{\kern0pt}true\ \isacommand{unfolding}\isamarkupfalse%
\ is{\isacharunderscore}{\kern0pt}pullback{\isacharunderscore}{\kern0pt}def\isanewline
\ \ \ \ \isacommand{by}\isamarkupfalse%
\ {\isacharparenleft}{\kern0pt}metis\ P{\isacharunderscore}{\kern0pt}Q{\isacharunderscore}{\kern0pt}types\ cfunc{\isacharunderscore}{\kern0pt}cross{\isacharunderscore}{\kern0pt}prod{\isacharunderscore}{\kern0pt}type\ terminal{\isacharunderscore}{\kern0pt}func{\isacharunderscore}{\kern0pt}type{\isacharparenright}{\kern0pt}\ \isanewline
\ \ \isacommand{then}\isamarkupfalse%
\ \isacommand{have}\isamarkupfalse%
\ {\isachardoublequoteopen}P\ {\isasymtimes}\isactrlsub f\ Q\ {\isacharequal}{\kern0pt}\ {\isasymlangle}{\isasymf}{\isacharcomma}{\kern0pt}{\isasymf}{\isasymrangle}\ {\isasymcirc}\isactrlsub c\ {\isasymbeta}\isactrlbsub X\ {\isasymtimes}\isactrlsub c\ Y\isactrlesub {\isachardoublequoteclose}\isanewline
\ \ \ \ \isacommand{using}\isamarkupfalse%
\ terminal{\isacharunderscore}{\kern0pt}func{\isacharunderscore}{\kern0pt}unique\ \isacommand{by}\isamarkupfalse%
\ auto\isanewline
\ \ \isacommand{then}\isamarkupfalse%
\ \isacommand{have}\isamarkupfalse%
\ {\isachardoublequoteopen}P\ {\isasymtimes}\isactrlsub f\ Q\ {\isacharequal}{\kern0pt}\ {\isasymlangle}{\isasymf}\ {\isasymcirc}\isactrlsub c\ {\isasymbeta}\isactrlbsub X\ {\isasymtimes}\isactrlsub c\ Y\isactrlesub {\isacharcomma}{\kern0pt}\ {\isasymf}\ {\isasymcirc}\isactrlsub c\ {\isasymbeta}\isactrlbsub X\ {\isasymtimes}\isactrlsub c\ Y\isactrlesub {\isasymrangle}{\isachardoublequoteclose}\isanewline
\ \ \ \ \isacommand{by}\isamarkupfalse%
\ {\isacharparenleft}{\kern0pt}typecheck{\isacharunderscore}{\kern0pt}cfuncs{\isacharcomma}{\kern0pt}\ simp\ add{\isacharcolon}{\kern0pt}\ cfunc{\isacharunderscore}{\kern0pt}prod{\isacharunderscore}{\kern0pt}comp{\isacharparenright}{\kern0pt}\isanewline
\ \ \isacommand{then}\isamarkupfalse%
\ \isacommand{have}\isamarkupfalse%
\ {\isachardoublequoteopen}P\ {\isasymtimes}\isactrlsub f\ Q\ {\isacharequal}{\kern0pt}\ {\isasymlangle}{\isasymf}\ {\isasymcirc}\isactrlsub c\ {\isasymbeta}\isactrlbsub X\isactrlesub \ {\isasymcirc}\isactrlsub c\ left{\isacharunderscore}{\kern0pt}cart{\isacharunderscore}{\kern0pt}proj\ X\ Y{\isacharcomma}{\kern0pt}\ {\isasymf}\ {\isasymcirc}\isactrlsub c\ {\isasymbeta}\isactrlbsub Y\isactrlesub \ {\isasymcirc}\isactrlsub c\ right{\isacharunderscore}{\kern0pt}cart{\isacharunderscore}{\kern0pt}proj\ X\ Y{\isasymrangle}{\isachardoublequoteclose}\isanewline
\ \ \ \ \isacommand{by}\isamarkupfalse%
\ {\isacharparenleft}{\kern0pt}typecheck{\isacharunderscore}{\kern0pt}cfuncs{\isacharunderscore}{\kern0pt}prems{\isacharcomma}{\kern0pt}\ metis\ left{\isacharunderscore}{\kern0pt}cart{\isacharunderscore}{\kern0pt}proj{\isacharunderscore}{\kern0pt}type\ right{\isacharunderscore}{\kern0pt}cart{\isacharunderscore}{\kern0pt}proj{\isacharunderscore}{\kern0pt}type\ terminal{\isacharunderscore}{\kern0pt}func{\isacharunderscore}{\kern0pt}comp{\isacharparenright}{\kern0pt}\isanewline
\ \ \isacommand{then}\isamarkupfalse%
\ \isacommand{have}\isamarkupfalse%
\ {\isachardoublequoteopen}{\isasymlangle}P\ {\isasymcirc}\isactrlsub c\ left{\isacharunderscore}{\kern0pt}cart{\isacharunderscore}{\kern0pt}proj\ X\ Y{\isacharcomma}{\kern0pt}\ Q\ {\isasymcirc}\isactrlsub c\ right{\isacharunderscore}{\kern0pt}cart{\isacharunderscore}{\kern0pt}proj\ X\ Y{\isasymrangle}\isanewline
\ \ \ \ \ \ {\isacharequal}{\kern0pt}\ {\isasymlangle}{\isasymf}\ {\isasymcirc}\isactrlsub c\ {\isasymbeta}\isactrlbsub X\isactrlesub \ {\isasymcirc}\isactrlsub c\ left{\isacharunderscore}{\kern0pt}cart{\isacharunderscore}{\kern0pt}proj\ X\ Y{\isacharcomma}{\kern0pt}\ {\isasymf}\ {\isasymcirc}\isactrlsub c\ {\isasymbeta}\isactrlbsub Y\isactrlesub \ {\isasymcirc}\isactrlsub c\ right{\isacharunderscore}{\kern0pt}cart{\isacharunderscore}{\kern0pt}proj\ X\ Y{\isasymrangle}{\isachardoublequoteclose}\isanewline
\ \ \ \ \isacommand{by}\isamarkupfalse%
\ {\isacharparenleft}{\kern0pt}typecheck{\isacharunderscore}{\kern0pt}cfuncs{\isacharcomma}{\kern0pt}\ unfold\ cfunc{\isacharunderscore}{\kern0pt}cross{\isacharunderscore}{\kern0pt}prod{\isacharunderscore}{\kern0pt}def{\isadigit{2}}{\isacharcomma}{\kern0pt}\ auto{\isacharparenright}{\kern0pt}\isanewline
\ \ \isacommand{then}\isamarkupfalse%
\ \isacommand{have}\isamarkupfalse%
\ {\isachardoublequoteopen}P\ {\isasymcirc}\isactrlsub c\ left{\isacharunderscore}{\kern0pt}cart{\isacharunderscore}{\kern0pt}proj\ X\ Y\ {\isacharequal}{\kern0pt}\ {\isacharparenleft}{\kern0pt}{\isasymf}\ {\isasymcirc}\isactrlsub c\ {\isasymbeta}\isactrlbsub X\isactrlesub {\isacharparenright}{\kern0pt}\ {\isasymcirc}\isactrlsub c\ left{\isacharunderscore}{\kern0pt}cart{\isacharunderscore}{\kern0pt}proj\ X\ Y\isanewline
\ \ \ \ \ \ {\isasymand}\ Q\ {\isasymcirc}\isactrlsub c\ right{\isacharunderscore}{\kern0pt}cart{\isacharunderscore}{\kern0pt}proj\ X\ Y\ {\isacharequal}{\kern0pt}\ {\isacharparenleft}{\kern0pt}{\isasymf}\ {\isasymcirc}\isactrlsub c\ {\isasymbeta}\isactrlbsub Y\isactrlesub {\isacharparenright}{\kern0pt}\ {\isasymcirc}\isactrlsub c\ right{\isacharunderscore}{\kern0pt}cart{\isacharunderscore}{\kern0pt}proj\ X\ Y{\isachardoublequoteclose}\isanewline
\ \ \ \ \isacommand{using}\isamarkupfalse%
\ \ cart{\isacharunderscore}{\kern0pt}prod{\isacharunderscore}{\kern0pt}eq{\isadigit{2}}\ \isacommand{by}\isamarkupfalse%
\ {\isacharparenleft}{\kern0pt}typecheck{\isacharunderscore}{\kern0pt}cfuncs{\isacharcomma}{\kern0pt}\ auto\ simp\ add{\isacharcolon}{\kern0pt}\ comp{\isacharunderscore}{\kern0pt}associative{\isadigit{2}}{\isacharparenright}{\kern0pt}\isanewline
\ \ \isacommand{then}\isamarkupfalse%
\ \isacommand{have}\isamarkupfalse%
\ eqs{\isacharcolon}{\kern0pt}\ {\isachardoublequoteopen}P\ {\isacharequal}{\kern0pt}\ {\isasymf}\ {\isasymcirc}\isactrlsub c\ {\isasymbeta}\isactrlbsub X\isactrlesub \ {\isasymand}\ Q\ {\isacharequal}{\kern0pt}\ {\isasymf}\ {\isasymcirc}\isactrlsub c\ {\isasymbeta}\isactrlbsub Y\isactrlesub {\isachardoublequoteclose}\isanewline
\ \ \ \ \isacommand{using}\isamarkupfalse%
\ assms\ epimorphism{\isacharunderscore}{\kern0pt}def{\isadigit{3}}\ nonempty{\isacharunderscore}{\kern0pt}left{\isacharunderscore}{\kern0pt}imp{\isacharunderscore}{\kern0pt}right{\isacharunderscore}{\kern0pt}proj{\isacharunderscore}{\kern0pt}epimorphism\ nonempty{\isacharunderscore}{\kern0pt}right{\isacharunderscore}{\kern0pt}imp{\isacharunderscore}{\kern0pt}left{\isacharunderscore}{\kern0pt}proj{\isacharunderscore}{\kern0pt}epimorphism\isanewline
\ \ \ \ \isacommand{by}\isamarkupfalse%
\ {\isacharparenleft}{\kern0pt}typecheck{\isacharunderscore}{\kern0pt}cfuncs{\isacharunderscore}{\kern0pt}prems{\isacharcomma}{\kern0pt}\ blast{\isacharparenright}{\kern0pt}\isanewline
\ \ \isacommand{then}\isamarkupfalse%
\ \isacommand{have}\isamarkupfalse%
\ {\isachardoublequoteopen}P\ {\isasymnoteq}\ {\isasymt}\ {\isasymcirc}\isactrlsub c\ {\isasymbeta}\isactrlbsub X\isactrlesub \ {\isasymand}\ Q\ {\isasymnoteq}\ {\isasymt}\ {\isasymcirc}\isactrlsub c\ {\isasymbeta}\isactrlbsub Y\isactrlesub {\isachardoublequoteclose}\isanewline
\ \ \isacommand{proof}\isamarkupfalse%
\ safe\isanewline
\ \ \ \ \isacommand{show}\isamarkupfalse%
\ {\isachardoublequoteopen}{\isasymf}\ {\isasymcirc}\isactrlsub c\ {\isasymbeta}\isactrlbsub X\isactrlesub \ {\isacharequal}{\kern0pt}\ {\isasymt}\ {\isasymcirc}\isactrlsub c\ {\isasymbeta}\isactrlbsub X\isactrlesub \ {\isasymLongrightarrow}\ False{\isachardoublequoteclose}\isanewline
\ \ \ \ \ \ \isacommand{by}\isamarkupfalse%
\ {\isacharparenleft}{\kern0pt}typecheck{\isacharunderscore}{\kern0pt}cfuncs{\isacharunderscore}{\kern0pt}prems{\isacharcomma}{\kern0pt}\ smt\ X{\isacharunderscore}{\kern0pt}nonempty\ comp{\isacharunderscore}{\kern0pt}associative{\isadigit{2}}\ nonempty{\isacharunderscore}{\kern0pt}def\ one{\isacharunderscore}{\kern0pt}separator{\isacharunderscore}{\kern0pt}contrapos\ terminal{\isacharunderscore}{\kern0pt}func{\isacharunderscore}{\kern0pt}comp\ terminal{\isacharunderscore}{\kern0pt}func{\isacharunderscore}{\kern0pt}unique\ true{\isacharunderscore}{\kern0pt}false{\isacharunderscore}{\kern0pt}distinct{\isacharparenright}{\kern0pt}\isanewline
\ \ \ \ \isacommand{show}\isamarkupfalse%
\ {\isachardoublequoteopen}{\isasymf}\ {\isasymcirc}\isactrlsub c\ {\isasymbeta}\isactrlbsub Y\isactrlesub \ {\isacharequal}{\kern0pt}\ {\isasymt}\ {\isasymcirc}\isactrlsub c\ {\isasymbeta}\isactrlbsub Y\isactrlesub \ {\isasymLongrightarrow}\ False{\isachardoublequoteclose}\isanewline
\ \ \ \ \ \ \isacommand{by}\isamarkupfalse%
\ {\isacharparenleft}{\kern0pt}typecheck{\isacharunderscore}{\kern0pt}cfuncs{\isacharunderscore}{\kern0pt}prems{\isacharcomma}{\kern0pt}\ smt\ Y{\isacharunderscore}{\kern0pt}nonempty\ comp{\isacharunderscore}{\kern0pt}associative{\isadigit{2}}\ nonempty{\isacharunderscore}{\kern0pt}def\ one{\isacharunderscore}{\kern0pt}separator{\isacharunderscore}{\kern0pt}contrapos\ terminal{\isacharunderscore}{\kern0pt}func{\isacharunderscore}{\kern0pt}comp\ terminal{\isacharunderscore}{\kern0pt}func{\isacharunderscore}{\kern0pt}unique\ true{\isacharunderscore}{\kern0pt}false{\isacharunderscore}{\kern0pt}distinct{\isacharparenright}{\kern0pt}\isanewline
\ \ \isacommand{qed}\isamarkupfalse%
\isanewline
\ \ \isacommand{then}\isamarkupfalse%
\ \isacommand{show}\isamarkupfalse%
\ {\isacharquery}{\kern0pt}thesis\isanewline
\ \ \ \ \isacommand{using}\isamarkupfalse%
\ eqs\ \isacommand{by}\isamarkupfalse%
\ linarith\isanewline
\isacommand{qed}\isamarkupfalse%
%
\endisatagproof
{\isafoldproof}%
%
\isadelimproof
\isanewline
%
\endisadelimproof
\isanewline
\isacommand{lemma}\isamarkupfalse%
\ NOR{\isacharunderscore}{\kern0pt}true{\isacharunderscore}{\kern0pt}implies{\isacharunderscore}{\kern0pt}neither{\isacharunderscore}{\kern0pt}true{\isacharcolon}{\kern0pt}\isanewline
\ \ \isakeyword{assumes}\ X{\isacharunderscore}{\kern0pt}nonempty{\isacharcolon}{\kern0pt}\ {\isachardoublequoteopen}nonempty\ X{\isachardoublequoteclose}\ \isakeyword{and}\ Y{\isacharunderscore}{\kern0pt}nonempty{\isacharcolon}{\kern0pt}\ {\isachardoublequoteopen}nonempty\ Y{\isachardoublequoteclose}\isanewline
\ \ \isakeyword{assumes}\ P{\isacharunderscore}{\kern0pt}Q{\isacharunderscore}{\kern0pt}types{\isacharbrackleft}{\kern0pt}type{\isacharunderscore}{\kern0pt}rule{\isacharbrackright}{\kern0pt}{\isacharcolon}{\kern0pt}\ {\isachardoublequoteopen}P\ {\isacharcolon}{\kern0pt}\ X\ {\isasymrightarrow}\ {\isasymOmega}{\isachardoublequoteclose}\ {\isachardoublequoteopen}Q\ {\isacharcolon}{\kern0pt}\ Y\ {\isasymrightarrow}\ {\isasymOmega}{\isachardoublequoteclose}\isanewline
\ \ \isakeyword{assumes}\ NOR{\isacharunderscore}{\kern0pt}true{\isacharcolon}{\kern0pt}\ {\isachardoublequoteopen}NOR\ {\isasymcirc}\isactrlsub c\ {\isacharparenleft}{\kern0pt}P\ {\isasymtimes}\isactrlsub f\ Q{\isacharparenright}{\kern0pt}\ {\isacharequal}{\kern0pt}\ {\isasymt}\ {\isasymcirc}\isactrlsub c\ {\isasymbeta}\isactrlbsub X\ {\isasymtimes}\isactrlsub c\ Y\isactrlesub {\isachardoublequoteclose}\isanewline
\ \ \isakeyword{shows}\ {\isachardoublequoteopen}{\isasymnot}\ {\isacharparenleft}{\kern0pt}P\ {\isacharequal}{\kern0pt}\ {\isasymt}\ {\isasymcirc}\isactrlsub c\ {\isasymbeta}\isactrlbsub X\isactrlesub \ {\isasymor}\ Q\ {\isacharequal}{\kern0pt}\ {\isasymt}\ {\isasymcirc}\isactrlsub c\ {\isasymbeta}\isactrlbsub Y\isactrlesub {\isacharparenright}{\kern0pt}{\isachardoublequoteclose}\isanewline
%
\isadelimproof
\ \ %
\endisadelimproof
%
\isatagproof
\isacommand{by}\isamarkupfalse%
\ {\isacharparenleft}{\kern0pt}smt\ {\isacharparenleft}{\kern0pt}verit{\isacharcomma}{\kern0pt}\ ccfv{\isacharunderscore}{\kern0pt}SIG{\isacharparenright}{\kern0pt}\ NOR{\isacharunderscore}{\kern0pt}true\ NOT{\isacharunderscore}{\kern0pt}false{\isacharunderscore}{\kern0pt}is{\isacharunderscore}{\kern0pt}true\ NOT{\isacharunderscore}{\kern0pt}true{\isacharunderscore}{\kern0pt}is{\isacharunderscore}{\kern0pt}false\ NOT{\isacharunderscore}{\kern0pt}type\ X{\isacharunderscore}{\kern0pt}nonempty\ Y{\isacharunderscore}{\kern0pt}nonempty\ assms{\isacharparenleft}{\kern0pt}{\isadigit{3}}{\isacharcomma}{\kern0pt}{\isadigit{4}}{\isacharparenright}{\kern0pt}\ comp{\isacharunderscore}{\kern0pt}associative{\isadigit{2}}\ comp{\isacharunderscore}{\kern0pt}type\ nonempty{\isacharunderscore}{\kern0pt}def\ terminal{\isacharunderscore}{\kern0pt}func{\isacharunderscore}{\kern0pt}type\ true{\isacharunderscore}{\kern0pt}false{\isacharunderscore}{\kern0pt}distinct\ true{\isacharunderscore}{\kern0pt}false{\isacharunderscore}{\kern0pt}only{\isacharunderscore}{\kern0pt}truth{\isacharunderscore}{\kern0pt}values\ NOR{\isacharunderscore}{\kern0pt}true{\isacharunderscore}{\kern0pt}implies{\isacharunderscore}{\kern0pt}both{\isacharunderscore}{\kern0pt}false{\isacharparenright}{\kern0pt}%
\endisatagproof
{\isafoldproof}%
%
\isadelimproof
%
\endisadelimproof
%
\isadelimdocument
%
\endisadelimdocument
%
\isatagdocument
%
\isamarkupsubsection{OR%
}
\isamarkuptrue%
%
\endisatagdocument
{\isafolddocument}%
%
\isadelimdocument
%
\endisadelimdocument
\isacommand{definition}\isamarkupfalse%
\ OR\ {\isacharcolon}{\kern0pt}{\isacharcolon}{\kern0pt}\ {\isachardoublequoteopen}cfunc{\isachardoublequoteclose}\ \isakeyword{where}\isanewline
\ \ {\isachardoublequoteopen}OR\ {\isacharequal}{\kern0pt}\ {\isacharparenleft}{\kern0pt}THE\ {\isasymchi}{\isachardot}{\kern0pt}\ is{\isacharunderscore}{\kern0pt}pullback\ {\isacharparenleft}{\kern0pt}{\isasymone}{\isasymCoprod}{\isacharparenleft}{\kern0pt}{\isasymone}{\isasymCoprod}{\isasymone}{\isacharparenright}{\kern0pt}{\isacharparenright}{\kern0pt}\ {\isasymone}\ {\isacharparenleft}{\kern0pt}{\isasymOmega}{\isasymtimes}\isactrlsub c{\isasymOmega}{\isacharparenright}{\kern0pt}\ {\isasymOmega}\ {\isacharparenleft}{\kern0pt}{\isasymbeta}\isactrlbsub {\isacharparenleft}{\kern0pt}{\isasymone}{\isasymCoprod}{\isacharparenleft}{\kern0pt}{\isasymone}{\isasymCoprod}{\isasymone}{\isacharparenright}{\kern0pt}{\isacharparenright}{\kern0pt}\isactrlesub {\isacharparenright}{\kern0pt}\ {\isasymt}\ {\isacharparenleft}{\kern0pt}{\isasymlangle}{\isasymt}{\isacharcomma}{\kern0pt}\ {\isasymt}{\isasymrangle}{\isasymamalg}\ {\isacharparenleft}{\kern0pt}{\isasymlangle}{\isasymt}{\isacharcomma}{\kern0pt}\ {\isasymf}{\isasymrangle}\ {\isasymamalg}{\isasymlangle}{\isasymf}{\isacharcomma}{\kern0pt}\ {\isasymt}{\isasymrangle}{\isacharparenright}{\kern0pt}{\isacharparenright}{\kern0pt}\ {\isasymchi}{\isacharparenright}{\kern0pt}{\isachardoublequoteclose}\isanewline
\isanewline
\isacommand{lemma}\isamarkupfalse%
\ pre{\isacharunderscore}{\kern0pt}OR{\isacharunderscore}{\kern0pt}type{\isacharbrackleft}{\kern0pt}type{\isacharunderscore}{\kern0pt}rule{\isacharbrackright}{\kern0pt}{\isacharcolon}{\kern0pt}\ \isanewline
\ \ {\isachardoublequoteopen}{\isasymlangle}{\isasymt}{\isacharcomma}{\kern0pt}\ {\isasymt}{\isasymrangle}{\isasymamalg}\ {\isacharparenleft}{\kern0pt}{\isasymlangle}{\isasymt}{\isacharcomma}{\kern0pt}\ {\isasymf}{\isasymrangle}\ {\isasymamalg}{\isasymlangle}{\isasymf}{\isacharcomma}{\kern0pt}\ {\isasymt}{\isasymrangle}{\isacharparenright}{\kern0pt}\ {\isacharcolon}{\kern0pt}\ {\isasymone}{\isasymCoprod}{\isacharparenleft}{\kern0pt}{\isasymone}{\isasymCoprod}{\isasymone}{\isacharparenright}{\kern0pt}\ {\isasymrightarrow}\ {\isasymOmega}\ {\isasymtimes}\isactrlsub c\ {\isasymOmega}{\isachardoublequoteclose}\isanewline
%
\isadelimproof
\ \ %
\endisadelimproof
%
\isatagproof
\isacommand{by}\isamarkupfalse%
\ typecheck{\isacharunderscore}{\kern0pt}cfuncs%
\endisatagproof
{\isafoldproof}%
%
\isadelimproof
\isanewline
%
\endisadelimproof
\isanewline
\isacommand{lemma}\isamarkupfalse%
\ set{\isacharunderscore}{\kern0pt}three{\isacharcolon}{\kern0pt}\ \isanewline
\ \ {\isachardoublequoteopen}{\isacharbraceleft}{\kern0pt}x{\isachardot}{\kern0pt}\ x\ {\isasymin}\isactrlsub c\ {\isacharparenleft}{\kern0pt}{\isasymone}{\isasymCoprod}{\isacharparenleft}{\kern0pt}{\isasymone}{\isasymCoprod}{\isasymone}{\isacharparenright}{\kern0pt}{\isacharparenright}{\kern0pt}{\isacharbraceright}{\kern0pt}\ {\isacharequal}{\kern0pt}\ {\isacharbraceleft}{\kern0pt}\isanewline
\ {\isacharparenleft}{\kern0pt}left{\isacharunderscore}{\kern0pt}coproj\ {\isasymone}\ {\isacharparenleft}{\kern0pt}{\isasymone}{\isasymCoprod}{\isasymone}{\isacharparenright}{\kern0pt}{\isacharparenright}{\kern0pt}\ {\isacharcomma}{\kern0pt}\ \isanewline
\ {\isacharparenleft}{\kern0pt}right{\isacharunderscore}{\kern0pt}coproj\ {\isasymone}\ {\isacharparenleft}{\kern0pt}{\isasymone}{\isasymCoprod}{\isasymone}{\isacharparenright}{\kern0pt}\ {\isasymcirc}\isactrlsub c\ left{\isacharunderscore}{\kern0pt}coproj\ {\isasymone}\ {\isasymone}{\isacharparenright}{\kern0pt}{\isacharcomma}{\kern0pt}\ \isanewline
\ \ right{\isacharunderscore}{\kern0pt}coproj\ {\isasymone}\ {\isacharparenleft}{\kern0pt}{\isasymone}{\isasymCoprod}{\isasymone}{\isacharparenright}{\kern0pt}\ {\isasymcirc}\isactrlsub c{\isacharparenleft}{\kern0pt}right{\isacharunderscore}{\kern0pt}coproj\ {\isasymone}\ {\isasymone}{\isacharparenright}{\kern0pt}{\isacharbraceright}{\kern0pt}{\isachardoublequoteclose}\isanewline
%
\isadelimproof
\ \ %
\endisadelimproof
%
\isatagproof
\isacommand{by}\isamarkupfalse%
{\isacharparenleft}{\kern0pt}typecheck{\isacharunderscore}{\kern0pt}cfuncs{\isacharcomma}{\kern0pt}\ safe{\isacharcomma}{\kern0pt}\ typecheck{\isacharunderscore}{\kern0pt}cfuncs{\isacharcomma}{\kern0pt}\ smt\ {\isacharparenleft}{\kern0pt}z{\isadigit{3}}{\isacharparenright}{\kern0pt}\ comp{\isacharunderscore}{\kern0pt}associative{\isadigit{2}}\ coprojs{\isacharunderscore}{\kern0pt}jointly{\isacharunderscore}{\kern0pt}surj\ one{\isacharunderscore}{\kern0pt}unique{\isacharunderscore}{\kern0pt}element{\isacharparenright}{\kern0pt}%
\endisatagproof
{\isafoldproof}%
%
\isadelimproof
\isanewline
%
\endisadelimproof
\isanewline
\isacommand{lemma}\isamarkupfalse%
\ set{\isacharunderscore}{\kern0pt}three{\isacharunderscore}{\kern0pt}card{\isacharcolon}{\kern0pt}\ \isanewline
\ {\isachardoublequoteopen}card\ {\isacharbraceleft}{\kern0pt}x{\isachardot}{\kern0pt}\ x\ {\isasymin}\isactrlsub c\ {\isacharparenleft}{\kern0pt}{\isasymone}{\isasymCoprod}{\isacharparenleft}{\kern0pt}{\isasymone}{\isasymCoprod}{\isasymone}{\isacharparenright}{\kern0pt}{\isacharparenright}{\kern0pt}{\isacharbraceright}{\kern0pt}\ {\isacharequal}{\kern0pt}\ {\isadigit{3}}{\isachardoublequoteclose}\isanewline
%
\isadelimproof
%
\endisadelimproof
%
\isatagproof
\isacommand{proof}\isamarkupfalse%
\ {\isacharminus}{\kern0pt}\ \isanewline
\ \ \isacommand{have}\isamarkupfalse%
\ f{\isadigit{1}}{\isacharcolon}{\kern0pt}\ {\isachardoublequoteopen}left{\isacharunderscore}{\kern0pt}coproj\ {\isasymone}\ {\isacharparenleft}{\kern0pt}{\isasymone}\ {\isasymCoprod}\ {\isasymone}{\isacharparenright}{\kern0pt}\ {\isasymnoteq}\ right{\isacharunderscore}{\kern0pt}coproj\ {\isasymone}\ {\isacharparenleft}{\kern0pt}{\isasymone}\ {\isasymCoprod}\ {\isasymone}{\isacharparenright}{\kern0pt}\ {\isasymcirc}\isactrlsub c\ left{\isacharunderscore}{\kern0pt}coproj\ {\isasymone}\ {\isasymone}{\isachardoublequoteclose}\isanewline
\ \ \ \ \isacommand{by}\isamarkupfalse%
\ {\isacharparenleft}{\kern0pt}typecheck{\isacharunderscore}{\kern0pt}cfuncs{\isacharcomma}{\kern0pt}\ metis\ cfunc{\isacharunderscore}{\kern0pt}type{\isacharunderscore}{\kern0pt}def\ coproducts{\isacharunderscore}{\kern0pt}disjoint\ id{\isacharunderscore}{\kern0pt}right{\isacharunderscore}{\kern0pt}unit\ id{\isacharunderscore}{\kern0pt}type{\isacharparenright}{\kern0pt}\isanewline
\ \ \isacommand{have}\isamarkupfalse%
\ f{\isadigit{2}}{\isacharcolon}{\kern0pt}\ {\isachardoublequoteopen}left{\isacharunderscore}{\kern0pt}coproj\ {\isasymone}\ {\isacharparenleft}{\kern0pt}{\isasymone}\ {\isasymCoprod}\ {\isasymone}{\isacharparenright}{\kern0pt}\ {\isasymnoteq}\ right{\isacharunderscore}{\kern0pt}coproj\ {\isasymone}\ {\isacharparenleft}{\kern0pt}{\isasymone}\ {\isasymCoprod}\ {\isasymone}{\isacharparenright}{\kern0pt}\ {\isasymcirc}\isactrlsub c\ right{\isacharunderscore}{\kern0pt}coproj\ {\isasymone}\ {\isasymone}{\isachardoublequoteclose}\isanewline
\ \ \ \ \isacommand{by}\isamarkupfalse%
\ {\isacharparenleft}{\kern0pt}typecheck{\isacharunderscore}{\kern0pt}cfuncs{\isacharcomma}{\kern0pt}\ metis\ cfunc{\isacharunderscore}{\kern0pt}type{\isacharunderscore}{\kern0pt}def\ coproducts{\isacharunderscore}{\kern0pt}disjoint\ id{\isacharunderscore}{\kern0pt}right{\isacharunderscore}{\kern0pt}unit\ id{\isacharunderscore}{\kern0pt}type{\isacharparenright}{\kern0pt}\isanewline
\ \ \isacommand{have}\isamarkupfalse%
\ f{\isadigit{3}}{\isacharcolon}{\kern0pt}\ {\isachardoublequoteopen}right{\isacharunderscore}{\kern0pt}coproj\ {\isasymone}\ {\isacharparenleft}{\kern0pt}{\isasymone}\ {\isasymCoprod}\ {\isasymone}{\isacharparenright}{\kern0pt}\ {\isasymcirc}\isactrlsub c\ left{\isacharunderscore}{\kern0pt}coproj\ {\isasymone}\ {\isasymone}\ {\isasymnoteq}\ right{\isacharunderscore}{\kern0pt}coproj\ {\isasymone}\ {\isacharparenleft}{\kern0pt}{\isasymone}\ {\isasymCoprod}\ {\isasymone}{\isacharparenright}{\kern0pt}\ {\isasymcirc}\isactrlsub c\ right{\isacharunderscore}{\kern0pt}coproj\ {\isasymone}\ {\isasymone}{\isachardoublequoteclose}\isanewline
\ \ \ \ \isacommand{by}\isamarkupfalse%
\ {\isacharparenleft}{\kern0pt}typecheck{\isacharunderscore}{\kern0pt}cfuncs{\isacharcomma}{\kern0pt}\ metis\ cfunc{\isacharunderscore}{\kern0pt}type{\isacharunderscore}{\kern0pt}def\ coproducts{\isacharunderscore}{\kern0pt}disjoint\ monomorphism{\isacharunderscore}{\kern0pt}def\ one{\isacharunderscore}{\kern0pt}unique{\isacharunderscore}{\kern0pt}element\ right{\isacharunderscore}{\kern0pt}coproj{\isacharunderscore}{\kern0pt}are{\isacharunderscore}{\kern0pt}monomorphisms{\isacharparenright}{\kern0pt}\isanewline
\ \ \isacommand{show}\isamarkupfalse%
\ {\isacharquery}{\kern0pt}thesis\isanewline
\ \ \ \ \isacommand{by}\isamarkupfalse%
\ {\isacharparenleft}{\kern0pt}simp\ add{\isacharcolon}{\kern0pt}\ f{\isadigit{1}}\ f{\isadigit{2}}\ f{\isadigit{3}}\ set{\isacharunderscore}{\kern0pt}three{\isacharparenright}{\kern0pt}\isanewline
\isacommand{qed}\isamarkupfalse%
%
\endisatagproof
{\isafoldproof}%
%
\isadelimproof
\isanewline
%
\endisadelimproof
\isanewline
\isacommand{lemma}\isamarkupfalse%
\ pre{\isacharunderscore}{\kern0pt}OR{\isacharunderscore}{\kern0pt}injective{\isacharcolon}{\kern0pt}\isanewline
\ \ {\isachardoublequoteopen}injective{\isacharparenleft}{\kern0pt}{\isasymlangle}{\isasymt}{\isacharcomma}{\kern0pt}\ {\isasymt}{\isasymrangle}{\isasymamalg}\ {\isacharparenleft}{\kern0pt}{\isasymlangle}{\isasymt}{\isacharcomma}{\kern0pt}\ {\isasymf}{\isasymrangle}\ {\isasymamalg}{\isasymlangle}{\isasymf}{\isacharcomma}{\kern0pt}\ {\isasymt}{\isasymrangle}{\isacharparenright}{\kern0pt}{\isacharparenright}{\kern0pt}{\isachardoublequoteclose}\isanewline
%
\isadelimproof
\ \ %
\endisadelimproof
%
\isatagproof
\isacommand{unfolding}\isamarkupfalse%
\ injective{\isacharunderscore}{\kern0pt}def\isanewline
\isacommand{proof}\isamarkupfalse%
{\isacharparenleft}{\kern0pt}clarify{\isacharparenright}{\kern0pt}\isanewline
\ \ \isacommand{fix}\isamarkupfalse%
\ x\ y\ \isanewline
\ \ \isacommand{assume}\isamarkupfalse%
\ {\isachardoublequoteopen}x\ {\isasymin}\isactrlsub c\ domain\ {\isacharparenleft}{\kern0pt}{\isasymlangle}{\isasymt}{\isacharcomma}{\kern0pt}{\isasymt}{\isasymrangle}\ {\isasymamalg}\ {\isasymlangle}{\isasymt}{\isacharcomma}{\kern0pt}{\isasymf}{\isasymrangle}\ {\isasymamalg}\ {\isasymlangle}{\isasymf}{\isacharcomma}{\kern0pt}{\isasymt}{\isasymrangle}{\isacharparenright}{\kern0pt}{\isachardoublequoteclose}\ \isanewline
\ \ \isacommand{then}\isamarkupfalse%
\ \isacommand{have}\isamarkupfalse%
\ x{\isacharunderscore}{\kern0pt}type{\isacharcolon}{\kern0pt}\ {\isachardoublequoteopen}x\ {\isasymin}\isactrlsub c\ {\isacharparenleft}{\kern0pt}{\isasymone}{\isasymCoprod}{\isacharparenleft}{\kern0pt}{\isasymone}{\isasymCoprod}{\isasymone}{\isacharparenright}{\kern0pt}{\isacharparenright}{\kern0pt}{\isachardoublequoteclose}\ \ \isanewline
\ \ \ \ \isacommand{using}\isamarkupfalse%
\ cfunc{\isacharunderscore}{\kern0pt}type{\isacharunderscore}{\kern0pt}def\ pre{\isacharunderscore}{\kern0pt}OR{\isacharunderscore}{\kern0pt}type\ \isacommand{by}\isamarkupfalse%
\ force\isanewline
\ \ \isacommand{then}\isamarkupfalse%
\ \isacommand{have}\isamarkupfalse%
\ x{\isacharunderscore}{\kern0pt}form{\isacharcolon}{\kern0pt}\ {\isachardoublequoteopen}{\isacharparenleft}{\kern0pt}{\isasymexists}\ w{\isachardot}{\kern0pt}\ {\isacharparenleft}{\kern0pt}w\ {\isasymin}\isactrlsub c\ {\isasymone}\ {\isasymand}\ x\ {\isacharequal}{\kern0pt}\ {\isacharparenleft}{\kern0pt}left{\isacharunderscore}{\kern0pt}coproj\ {\isasymone}\ {\isacharparenleft}{\kern0pt}{\isasymone}{\isasymCoprod}{\isasymone}{\isacharparenright}{\kern0pt}{\isacharparenright}{\kern0pt}\ {\isasymcirc}\isactrlsub c\ w{\isacharparenright}{\kern0pt}{\isacharparenright}{\kern0pt}\isanewline
\ \ \ \ \ \ {\isasymor}\ \ {\isacharparenleft}{\kern0pt}{\isasymexists}\ w{\isachardot}{\kern0pt}\ {\isacharparenleft}{\kern0pt}w\ {\isasymin}\isactrlsub c\ {\isacharparenleft}{\kern0pt}{\isasymone}{\isasymCoprod}{\isasymone}{\isacharparenright}{\kern0pt}\ {\isasymand}\ x\ {\isacharequal}{\kern0pt}\ {\isacharparenleft}{\kern0pt}right{\isacharunderscore}{\kern0pt}coproj\ {\isasymone}\ {\isacharparenleft}{\kern0pt}{\isasymone}{\isasymCoprod}{\isasymone}{\isacharparenright}{\kern0pt}{\isacharparenright}{\kern0pt}\ {\isasymcirc}\isactrlsub c\ w{\isacharparenright}{\kern0pt}{\isacharparenright}{\kern0pt}{\isachardoublequoteclose}\isanewline
\ \ \ \ \isacommand{using}\isamarkupfalse%
\ coprojs{\isacharunderscore}{\kern0pt}jointly{\isacharunderscore}{\kern0pt}surj\ \isacommand{by}\isamarkupfalse%
\ auto\isanewline
\isanewline
\ \ \isacommand{assume}\isamarkupfalse%
\ {\isachardoublequoteopen}y\ {\isasymin}\isactrlsub c\ domain\ {\isacharparenleft}{\kern0pt}{\isasymlangle}{\isasymt}{\isacharcomma}{\kern0pt}{\isasymt}{\isasymrangle}\ {\isasymamalg}\ {\isasymlangle}{\isasymt}{\isacharcomma}{\kern0pt}{\isasymf}{\isasymrangle}\ {\isasymamalg}\ {\isasymlangle}{\isasymf}{\isacharcomma}{\kern0pt}{\isasymt}{\isasymrangle}{\isacharparenright}{\kern0pt}{\isachardoublequoteclose}\ \isanewline
\ \ \isacommand{then}\isamarkupfalse%
\ \isacommand{have}\isamarkupfalse%
\ y{\isacharunderscore}{\kern0pt}type{\isacharcolon}{\kern0pt}\ {\isachardoublequoteopen}y\ {\isasymin}\isactrlsub c\ {\isacharparenleft}{\kern0pt}{\isasymone}{\isasymCoprod}{\isacharparenleft}{\kern0pt}{\isasymone}{\isasymCoprod}{\isasymone}{\isacharparenright}{\kern0pt}{\isacharparenright}{\kern0pt}{\isachardoublequoteclose}\ \ \isanewline
\ \ \ \ \isacommand{using}\isamarkupfalse%
\ cfunc{\isacharunderscore}{\kern0pt}type{\isacharunderscore}{\kern0pt}def\ pre{\isacharunderscore}{\kern0pt}OR{\isacharunderscore}{\kern0pt}type\ \isacommand{by}\isamarkupfalse%
\ force\isanewline
\ \ \isacommand{then}\isamarkupfalse%
\ \isacommand{have}\isamarkupfalse%
\ y{\isacharunderscore}{\kern0pt}form{\isacharcolon}{\kern0pt}\ {\isachardoublequoteopen}{\isacharparenleft}{\kern0pt}{\isasymexists}\ w{\isachardot}{\kern0pt}\ {\isacharparenleft}{\kern0pt}w\ {\isasymin}\isactrlsub c\ {\isasymone}\ {\isasymand}\ y\ {\isacharequal}{\kern0pt}\ {\isacharparenleft}{\kern0pt}left{\isacharunderscore}{\kern0pt}coproj\ {\isasymone}\ {\isacharparenleft}{\kern0pt}{\isasymone}{\isasymCoprod}{\isasymone}{\isacharparenright}{\kern0pt}{\isacharparenright}{\kern0pt}\ {\isasymcirc}\isactrlsub c\ w{\isacharparenright}{\kern0pt}{\isacharparenright}{\kern0pt}\isanewline
\ \ \ \ \ \ {\isasymor}\ \ {\isacharparenleft}{\kern0pt}{\isasymexists}\ w{\isachardot}{\kern0pt}\ {\isacharparenleft}{\kern0pt}w\ {\isasymin}\isactrlsub c\ {\isacharparenleft}{\kern0pt}{\isasymone}{\isasymCoprod}{\isasymone}{\isacharparenright}{\kern0pt}\ {\isasymand}\ y\ {\isacharequal}{\kern0pt}\ {\isacharparenleft}{\kern0pt}right{\isacharunderscore}{\kern0pt}coproj\ {\isasymone}\ {\isacharparenleft}{\kern0pt}{\isasymone}{\isasymCoprod}{\isasymone}{\isacharparenright}{\kern0pt}{\isacharparenright}{\kern0pt}\ {\isasymcirc}\isactrlsub c\ w{\isacharparenright}{\kern0pt}{\isacharparenright}{\kern0pt}{\isachardoublequoteclose}\isanewline
\ \ \ \ \isacommand{using}\isamarkupfalse%
\ coprojs{\isacharunderscore}{\kern0pt}jointly{\isacharunderscore}{\kern0pt}surj\ \isacommand{by}\isamarkupfalse%
\ auto\isanewline
\isanewline
\ \ \isacommand{assume}\isamarkupfalse%
\ mx{\isacharunderscore}{\kern0pt}eqs{\isacharunderscore}{\kern0pt}my{\isacharcolon}{\kern0pt}\ {\isachardoublequoteopen}{\isasymlangle}{\isasymt}{\isacharcomma}{\kern0pt}{\isasymt}{\isasymrangle}\ {\isasymamalg}\ {\isasymlangle}{\isasymt}{\isacharcomma}{\kern0pt}{\isasymf}{\isasymrangle}\ {\isasymamalg}\ {\isasymlangle}{\isasymf}{\isacharcomma}{\kern0pt}{\isasymt}{\isasymrangle}\ {\isasymcirc}\isactrlsub c\ x\ {\isacharequal}{\kern0pt}\ {\isasymlangle}{\isasymt}{\isacharcomma}{\kern0pt}{\isasymt}{\isasymrangle}\ {\isasymamalg}\ {\isasymlangle}{\isasymt}{\isacharcomma}{\kern0pt}{\isasymf}{\isasymrangle}\ {\isasymamalg}\ {\isasymlangle}{\isasymf}{\isacharcomma}{\kern0pt}{\isasymt}{\isasymrangle}\ {\isasymcirc}\isactrlsub c\ y{\isachardoublequoteclose}\isanewline
\isanewline
\ \ \isacommand{have}\isamarkupfalse%
\ f{\isadigit{1}}{\isacharcolon}{\kern0pt}\ {\isachardoublequoteopen}{\isasymlangle}{\isasymt}{\isacharcomma}{\kern0pt}{\isasymt}{\isasymrangle}\ {\isasymamalg}\ {\isasymlangle}{\isasymt}{\isacharcomma}{\kern0pt}{\isasymf}{\isasymrangle}\ {\isasymamalg}\ {\isasymlangle}{\isasymf}{\isacharcomma}{\kern0pt}{\isasymt}{\isasymrangle}\ {\isasymcirc}\isactrlsub c\ left{\isacharunderscore}{\kern0pt}coproj\ {\isasymone}\ {\isacharparenleft}{\kern0pt}{\isasymone}\ {\isasymCoprod}\ {\isasymone}{\isacharparenright}{\kern0pt}\ {\isacharequal}{\kern0pt}\ {\isasymlangle}{\isasymt}{\isacharcomma}{\kern0pt}{\isasymt}{\isasymrangle}{\isachardoublequoteclose}\isanewline
\ \ \ \ \isacommand{by}\isamarkupfalse%
\ {\isacharparenleft}{\kern0pt}typecheck{\isacharunderscore}{\kern0pt}cfuncs{\isacharcomma}{\kern0pt}\ simp\ add{\isacharcolon}{\kern0pt}\ left{\isacharunderscore}{\kern0pt}coproj{\isacharunderscore}{\kern0pt}cfunc{\isacharunderscore}{\kern0pt}coprod{\isacharparenright}{\kern0pt}\isanewline
\ \ \isacommand{have}\isamarkupfalse%
\ f{\isadigit{2}}{\isacharcolon}{\kern0pt}\ {\isachardoublequoteopen}{\isasymlangle}{\isasymt}{\isacharcomma}{\kern0pt}{\isasymt}{\isasymrangle}\ {\isasymamalg}\ {\isasymlangle}{\isasymt}{\isacharcomma}{\kern0pt}{\isasymf}{\isasymrangle}\ {\isasymamalg}\ {\isasymlangle}{\isasymf}{\isacharcomma}{\kern0pt}{\isasymt}{\isasymrangle}\ {\isasymcirc}\isactrlsub c\ {\isacharparenleft}{\kern0pt}right{\isacharunderscore}{\kern0pt}coproj\ {\isasymone}\ {\isacharparenleft}{\kern0pt}{\isasymone}{\isasymCoprod}{\isasymone}{\isacharparenright}{\kern0pt}{\isasymcirc}\isactrlsub c\ left{\isacharunderscore}{\kern0pt}coproj\ {\isasymone}\ {\isasymone}{\isacharparenright}{\kern0pt}\ {\isacharequal}{\kern0pt}\ {\isasymlangle}{\isasymt}{\isacharcomma}{\kern0pt}{\isasymf}{\isasymrangle}{\isachardoublequoteclose}\isanewline
\ \ \isacommand{proof}\isamarkupfalse%
{\isacharminus}{\kern0pt}\ \isanewline
\ \ \ \ \isacommand{have}\isamarkupfalse%
\ {\isachardoublequoteopen}{\isasymlangle}{\isasymt}{\isacharcomma}{\kern0pt}{\isasymt}{\isasymrangle}\ {\isasymamalg}\ {\isasymlangle}{\isasymt}{\isacharcomma}{\kern0pt}{\isasymf}{\isasymrangle}\ {\isasymamalg}\ {\isasymlangle}{\isasymf}{\isacharcomma}{\kern0pt}{\isasymt}{\isasymrangle}\ {\isasymcirc}\isactrlsub c\ {\isacharparenleft}{\kern0pt}right{\isacharunderscore}{\kern0pt}coproj\ {\isasymone}\ {\isacharparenleft}{\kern0pt}{\isasymone}{\isasymCoprod}{\isasymone}{\isacharparenright}{\kern0pt}{\isasymcirc}\isactrlsub c\ left{\isacharunderscore}{\kern0pt}coproj\ {\isasymone}\ {\isasymone}{\isacharparenright}{\kern0pt}\ {\isacharequal}{\kern0pt}\ \isanewline
\ \ \ \ \ \ \ \ \ \ {\isacharparenleft}{\kern0pt}{\isasymlangle}{\isasymt}{\isacharcomma}{\kern0pt}{\isasymt}{\isasymrangle}\ {\isasymamalg}\ {\isasymlangle}{\isasymt}{\isacharcomma}{\kern0pt}{\isasymf}{\isasymrangle}\ {\isasymamalg}\ {\isasymlangle}{\isasymf}{\isacharcomma}{\kern0pt}{\isasymt}{\isasymrangle}\ {\isasymcirc}\isactrlsub c\ right{\isacharunderscore}{\kern0pt}coproj\ {\isasymone}\ {\isacharparenleft}{\kern0pt}{\isasymone}{\isasymCoprod}{\isasymone}{\isacharparenright}{\kern0pt}\ {\isacharparenright}{\kern0pt}{\isasymcirc}\isactrlsub c\ left{\isacharunderscore}{\kern0pt}coproj\ {\isasymone}\ {\isasymone}{\isachardoublequoteclose}\isanewline
\ \ \ \ \ \ \isacommand{by}\isamarkupfalse%
\ {\isacharparenleft}{\kern0pt}typecheck{\isacharunderscore}{\kern0pt}cfuncs{\isacharcomma}{\kern0pt}\ simp\ add{\isacharcolon}{\kern0pt}\ comp{\isacharunderscore}{\kern0pt}associative{\isadigit{2}}{\isacharparenright}{\kern0pt}\isanewline
\ \ \ \ \isacommand{also}\isamarkupfalse%
\ \isacommand{have}\isamarkupfalse%
\ {\isachardoublequoteopen}{\isachardot}{\kern0pt}{\isachardot}{\kern0pt}{\isachardot}{\kern0pt}\ {\isacharequal}{\kern0pt}\ {\isasymlangle}{\isasymt}{\isacharcomma}{\kern0pt}{\isasymf}{\isasymrangle}\ {\isasymamalg}\ {\isasymlangle}{\isasymf}{\isacharcomma}{\kern0pt}{\isasymt}{\isasymrangle}\ {\isasymcirc}\isactrlsub c\ left{\isacharunderscore}{\kern0pt}coproj\ {\isasymone}\ {\isasymone}{\isachardoublequoteclose}\isanewline
\ \ \ \ \ \ \isacommand{using}\isamarkupfalse%
\ right{\isacharunderscore}{\kern0pt}coproj{\isacharunderscore}{\kern0pt}cfunc{\isacharunderscore}{\kern0pt}coprod\ \isacommand{by}\isamarkupfalse%
\ {\isacharparenleft}{\kern0pt}typecheck{\isacharunderscore}{\kern0pt}cfuncs{\isacharcomma}{\kern0pt}\ smt{\isacharparenright}{\kern0pt}\isanewline
\ \ \ \ \isacommand{also}\isamarkupfalse%
\ \isacommand{have}\isamarkupfalse%
\ {\isachardoublequoteopen}{\isachardot}{\kern0pt}{\isachardot}{\kern0pt}{\isachardot}{\kern0pt}\ {\isacharequal}{\kern0pt}\ {\isasymlangle}{\isasymt}{\isacharcomma}{\kern0pt}{\isasymf}{\isasymrangle}{\isachardoublequoteclose}\isanewline
\ \ \ \ \ \ \isacommand{by}\isamarkupfalse%
\ {\isacharparenleft}{\kern0pt}typecheck{\isacharunderscore}{\kern0pt}cfuncs{\isacharcomma}{\kern0pt}\ simp\ add{\isacharcolon}{\kern0pt}\ left{\isacharunderscore}{\kern0pt}coproj{\isacharunderscore}{\kern0pt}cfunc{\isacharunderscore}{\kern0pt}coprod{\isacharparenright}{\kern0pt}\isanewline
\ \ \ \ \isacommand{then}\isamarkupfalse%
\ \isacommand{show}\isamarkupfalse%
\ {\isacharquery}{\kern0pt}thesis\isanewline
\ \ \ \ \ \ \isacommand{by}\isamarkupfalse%
\ {\isacharparenleft}{\kern0pt}simp\ add{\isacharcolon}{\kern0pt}\ calculation{\isacharparenright}{\kern0pt}\isanewline
\ \ \isacommand{qed}\isamarkupfalse%
\isanewline
\ \ \isacommand{have}\isamarkupfalse%
\ f{\isadigit{3}}{\isacharcolon}{\kern0pt}\ {\isachardoublequoteopen}{\isasymlangle}{\isasymt}{\isacharcomma}{\kern0pt}{\isasymt}{\isasymrangle}\ {\isasymamalg}\ {\isasymlangle}{\isasymt}{\isacharcomma}{\kern0pt}{\isasymf}{\isasymrangle}\ {\isasymamalg}\ {\isasymlangle}{\isasymf}{\isacharcomma}{\kern0pt}{\isasymt}{\isasymrangle}\ {\isasymcirc}\isactrlsub c\ {\isacharparenleft}{\kern0pt}right{\isacharunderscore}{\kern0pt}coproj\ {\isasymone}\ {\isacharparenleft}{\kern0pt}{\isasymone}{\isasymCoprod}{\isasymone}{\isacharparenright}{\kern0pt}{\isasymcirc}\isactrlsub c\ right{\isacharunderscore}{\kern0pt}coproj\ {\isasymone}\ {\isasymone}{\isacharparenright}{\kern0pt}\ {\isacharequal}{\kern0pt}\ {\isasymlangle}{\isasymf}{\isacharcomma}{\kern0pt}{\isasymt}{\isasymrangle}{\isachardoublequoteclose}\isanewline
\ \ \isacommand{proof}\isamarkupfalse%
{\isacharminus}{\kern0pt}\ \isanewline
\ \ \ \ \isacommand{have}\isamarkupfalse%
\ {\isachardoublequoteopen}{\isasymlangle}{\isasymt}{\isacharcomma}{\kern0pt}{\isasymt}{\isasymrangle}\ {\isasymamalg}\ {\isasymlangle}{\isasymt}{\isacharcomma}{\kern0pt}{\isasymf}{\isasymrangle}\ {\isasymamalg}\ {\isasymlangle}{\isasymf}{\isacharcomma}{\kern0pt}{\isasymt}{\isasymrangle}\ {\isasymcirc}\isactrlsub c\ {\isacharparenleft}{\kern0pt}right{\isacharunderscore}{\kern0pt}coproj\ {\isasymone}\ {\isacharparenleft}{\kern0pt}{\isasymone}{\isasymCoprod}{\isasymone}{\isacharparenright}{\kern0pt}{\isasymcirc}\isactrlsub c\ right{\isacharunderscore}{\kern0pt}coproj\ {\isasymone}\ {\isasymone}{\isacharparenright}{\kern0pt}\ {\isacharequal}{\kern0pt}\ \isanewline
\ \ \ \ \ \ \ \ \ \ {\isacharparenleft}{\kern0pt}{\isasymlangle}{\isasymt}{\isacharcomma}{\kern0pt}{\isasymt}{\isasymrangle}\ {\isasymamalg}\ {\isasymlangle}{\isasymt}{\isacharcomma}{\kern0pt}{\isasymf}{\isasymrangle}\ {\isasymamalg}\ {\isasymlangle}{\isasymf}{\isacharcomma}{\kern0pt}{\isasymt}{\isasymrangle}\ {\isasymcirc}\isactrlsub c\ right{\isacharunderscore}{\kern0pt}coproj\ {\isasymone}\ {\isacharparenleft}{\kern0pt}{\isasymone}{\isasymCoprod}{\isasymone}{\isacharparenright}{\kern0pt}\ {\isacharparenright}{\kern0pt}{\isasymcirc}\isactrlsub c\ right{\isacharunderscore}{\kern0pt}coproj\ {\isasymone}\ {\isasymone}{\isachardoublequoteclose}\isanewline
\ \ \ \ \ \ \isacommand{by}\isamarkupfalse%
\ {\isacharparenleft}{\kern0pt}typecheck{\isacharunderscore}{\kern0pt}cfuncs{\isacharcomma}{\kern0pt}\ simp\ add{\isacharcolon}{\kern0pt}\ comp{\isacharunderscore}{\kern0pt}associative{\isadigit{2}}{\isacharparenright}{\kern0pt}\isanewline
\ \ \ \ \isacommand{also}\isamarkupfalse%
\ \isacommand{have}\isamarkupfalse%
\ {\isachardoublequoteopen}{\isachardot}{\kern0pt}{\isachardot}{\kern0pt}{\isachardot}{\kern0pt}\ {\isacharequal}{\kern0pt}\ {\isasymlangle}{\isasymt}{\isacharcomma}{\kern0pt}{\isasymf}{\isasymrangle}\ {\isasymamalg}\ {\isasymlangle}{\isasymf}{\isacharcomma}{\kern0pt}{\isasymt}{\isasymrangle}\ {\isasymcirc}\isactrlsub c\ right{\isacharunderscore}{\kern0pt}coproj\ {\isasymone}\ {\isasymone}{\isachardoublequoteclose}\isanewline
\ \ \ \ \ \ \isacommand{using}\isamarkupfalse%
\ right{\isacharunderscore}{\kern0pt}coproj{\isacharunderscore}{\kern0pt}cfunc{\isacharunderscore}{\kern0pt}coprod\ \isacommand{by}\isamarkupfalse%
\ {\isacharparenleft}{\kern0pt}typecheck{\isacharunderscore}{\kern0pt}cfuncs{\isacharcomma}{\kern0pt}\ smt{\isacharparenright}{\kern0pt}\isanewline
\ \ \ \ \isacommand{also}\isamarkupfalse%
\ \isacommand{have}\isamarkupfalse%
\ {\isachardoublequoteopen}{\isachardot}{\kern0pt}{\isachardot}{\kern0pt}{\isachardot}{\kern0pt}\ {\isacharequal}{\kern0pt}\ {\isasymlangle}{\isasymf}{\isacharcomma}{\kern0pt}{\isasymt}{\isasymrangle}{\isachardoublequoteclose}\isanewline
\ \ \ \ \ \ \isacommand{by}\isamarkupfalse%
\ {\isacharparenleft}{\kern0pt}typecheck{\isacharunderscore}{\kern0pt}cfuncs{\isacharcomma}{\kern0pt}\ simp\ add{\isacharcolon}{\kern0pt}\ right{\isacharunderscore}{\kern0pt}coproj{\isacharunderscore}{\kern0pt}cfunc{\isacharunderscore}{\kern0pt}coprod{\isacharparenright}{\kern0pt}\isanewline
\ \ \ \ \isacommand{then}\isamarkupfalse%
\ \isacommand{show}\isamarkupfalse%
\ {\isacharquery}{\kern0pt}thesis\isanewline
\ \ \ \ \ \ \isacommand{by}\isamarkupfalse%
\ {\isacharparenleft}{\kern0pt}simp\ add{\isacharcolon}{\kern0pt}\ calculation{\isacharparenright}{\kern0pt}\isanewline
\ \ \isacommand{qed}\isamarkupfalse%
\isanewline
\ \ \isacommand{show}\isamarkupfalse%
\ {\isachardoublequoteopen}x\ {\isacharequal}{\kern0pt}\ y{\isachardoublequoteclose}\isanewline
\ \ \isacommand{proof}\isamarkupfalse%
{\isacharparenleft}{\kern0pt}cases\ {\isachardoublequoteopen}x\ {\isacharequal}{\kern0pt}\ left{\isacharunderscore}{\kern0pt}coproj\ {\isasymone}\ {\isacharparenleft}{\kern0pt}{\isasymone}\ {\isasymCoprod}\ {\isasymone}{\isacharparenright}{\kern0pt}{\isachardoublequoteclose}{\isacharparenright}{\kern0pt}\isanewline
\ \ \ \ \isacommand{assume}\isamarkupfalse%
\ case{\isadigit{1}}{\isacharcolon}{\kern0pt}\ {\isachardoublequoteopen}x\ {\isacharequal}{\kern0pt}\ left{\isacharunderscore}{\kern0pt}coproj\ {\isasymone}\ {\isacharparenleft}{\kern0pt}{\isasymone}\ {\isasymCoprod}\ {\isasymone}{\isacharparenright}{\kern0pt}{\isachardoublequoteclose}\isanewline
\ \ \ \ \isacommand{then}\isamarkupfalse%
\ \isacommand{show}\isamarkupfalse%
\ {\isachardoublequoteopen}x\ {\isacharequal}{\kern0pt}\ y{\isachardoublequoteclose}\isanewline
\ \ \ \ \ \ \isacommand{by}\isamarkupfalse%
\ {\isacharparenleft}{\kern0pt}typecheck{\isacharunderscore}{\kern0pt}cfuncs{\isacharcomma}{\kern0pt}\ smt\ {\isacharparenleft}{\kern0pt}z{\isadigit{3}}{\isacharparenright}{\kern0pt}\ mx{\isacharunderscore}{\kern0pt}eqs{\isacharunderscore}{\kern0pt}my\ element{\isacharunderscore}{\kern0pt}pair{\isacharunderscore}{\kern0pt}eq\ f{\isadigit{1}}\ f{\isadigit{2}}\ f{\isadigit{3}}\ false{\isacharunderscore}{\kern0pt}func{\isacharunderscore}{\kern0pt}type\ maps{\isacharunderscore}{\kern0pt}into{\isacharunderscore}{\kern0pt}{\isadigit{1}}u{\isadigit{1}}\ terminal{\isacharunderscore}{\kern0pt}func{\isacharunderscore}{\kern0pt}unique\ true{\isacharunderscore}{\kern0pt}false{\isacharunderscore}{\kern0pt}distinct\ true{\isacharunderscore}{\kern0pt}func{\isacharunderscore}{\kern0pt}type\ x{\isacharunderscore}{\kern0pt}form\ y{\isacharunderscore}{\kern0pt}form{\isacharparenright}{\kern0pt}\isanewline
\ \ \isacommand{next}\isamarkupfalse%
\isanewline
\ \ \ \ \isacommand{assume}\isamarkupfalse%
\ not{\isacharunderscore}{\kern0pt}case{\isadigit{1}}{\isacharcolon}{\kern0pt}\ {\isachardoublequoteopen}x\ {\isasymnoteq}\ left{\isacharunderscore}{\kern0pt}coproj\ {\isasymone}\ {\isacharparenleft}{\kern0pt}{\isasymone}\ {\isasymCoprod}\ {\isasymone}{\isacharparenright}{\kern0pt}{\isachardoublequoteclose}\isanewline
\ \ \ \ \isacommand{then}\isamarkupfalse%
\ \isacommand{have}\isamarkupfalse%
\ case{\isadigit{2}}{\isacharunderscore}{\kern0pt}or{\isacharunderscore}{\kern0pt}{\isadigit{3}}{\isacharcolon}{\kern0pt}\ {\isachardoublequoteopen}x\ {\isacharequal}{\kern0pt}\ {\isacharparenleft}{\kern0pt}right{\isacharunderscore}{\kern0pt}coproj\ {\isasymone}\ {\isacharparenleft}{\kern0pt}{\isasymone}{\isasymCoprod}{\isasymone}{\isacharparenright}{\kern0pt}{\isasymcirc}\isactrlsub c\ left{\isacharunderscore}{\kern0pt}coproj\ {\isasymone}\ {\isasymone}{\isacharparenright}{\kern0pt}{\isasymor}\ \isanewline
\ \ \ \ \ \ \ \ \ \ \ \ \ \ \ x\ {\isacharequal}{\kern0pt}\ right{\isacharunderscore}{\kern0pt}coproj\ {\isasymone}\ {\isacharparenleft}{\kern0pt}{\isasymone}{\isasymCoprod}{\isasymone}{\isacharparenright}{\kern0pt}\ {\isasymcirc}\isactrlsub c{\isacharparenleft}{\kern0pt}right{\isacharunderscore}{\kern0pt}coproj\ {\isasymone}\ {\isasymone}{\isacharparenright}{\kern0pt}{\isachardoublequoteclose}\isanewline
\ \ \ \ \ \ \isacommand{by}\isamarkupfalse%
\ {\isacharparenleft}{\kern0pt}metis\ id{\isacharunderscore}{\kern0pt}right{\isacharunderscore}{\kern0pt}unit{\isadigit{2}}\ id{\isacharunderscore}{\kern0pt}type\ left{\isacharunderscore}{\kern0pt}proj{\isacharunderscore}{\kern0pt}type\ maps{\isacharunderscore}{\kern0pt}into{\isacharunderscore}{\kern0pt}{\isadigit{1}}u{\isadigit{1}}\ terminal{\isacharunderscore}{\kern0pt}func{\isacharunderscore}{\kern0pt}unique\ x{\isacharunderscore}{\kern0pt}form{\isacharparenright}{\kern0pt}\isanewline
\ \ \ \ \isacommand{show}\isamarkupfalse%
\ {\isachardoublequoteopen}x\ {\isacharequal}{\kern0pt}\ y{\isachardoublequoteclose}\isanewline
\ \ \ \ \isacommand{proof}\isamarkupfalse%
{\isacharparenleft}{\kern0pt}cases\ {\isachardoublequoteopen}x\ {\isacharequal}{\kern0pt}\ {\isacharparenleft}{\kern0pt}right{\isacharunderscore}{\kern0pt}coproj\ {\isasymone}\ {\isacharparenleft}{\kern0pt}{\isasymone}{\isasymCoprod}{\isasymone}{\isacharparenright}{\kern0pt}{\isasymcirc}\isactrlsub c\ left{\isacharunderscore}{\kern0pt}coproj\ {\isasymone}\ {\isasymone}{\isacharparenright}{\kern0pt}{\isachardoublequoteclose}{\isacharparenright}{\kern0pt}\isanewline
\ \ \ \ \ \ \isacommand{assume}\isamarkupfalse%
\ case{\isadigit{2}}{\isacharcolon}{\kern0pt}\ {\isachardoublequoteopen}x\ {\isacharequal}{\kern0pt}\ right{\isacharunderscore}{\kern0pt}coproj\ {\isasymone}\ {\isacharparenleft}{\kern0pt}{\isasymone}\ {\isasymCoprod}\ {\isasymone}{\isacharparenright}{\kern0pt}\ {\isasymcirc}\isactrlsub c\ left{\isacharunderscore}{\kern0pt}coproj\ {\isasymone}\ {\isasymone}{\isachardoublequoteclose}\isanewline
\ \ \ \ \ \ \isacommand{then}\isamarkupfalse%
\ \isacommand{show}\isamarkupfalse%
\ {\isachardoublequoteopen}x\ {\isacharequal}{\kern0pt}\ y{\isachardoublequoteclose}\isanewline
\ \ \ \ \ \ \ \ \isacommand{by}\isamarkupfalse%
\ {\isacharparenleft}{\kern0pt}typecheck{\isacharunderscore}{\kern0pt}cfuncs{\isacharcomma}{\kern0pt}\ smt\ {\isacharparenleft}{\kern0pt}z{\isadigit{3}}{\isacharparenright}{\kern0pt}\ cart{\isacharunderscore}{\kern0pt}prod{\isacharunderscore}{\kern0pt}eq{\isadigit{2}}\ case{\isadigit{2}}\ f{\isadigit{1}}\ f{\isadigit{2}}\ f{\isadigit{3}}\ false{\isacharunderscore}{\kern0pt}func{\isacharunderscore}{\kern0pt}type\ id{\isacharunderscore}{\kern0pt}right{\isacharunderscore}{\kern0pt}unit{\isadigit{2}}\ left{\isacharunderscore}{\kern0pt}proj{\isacharunderscore}{\kern0pt}type\ maps{\isacharunderscore}{\kern0pt}into{\isacharunderscore}{\kern0pt}{\isadigit{1}}u{\isadigit{1}}\ mx{\isacharunderscore}{\kern0pt}eqs{\isacharunderscore}{\kern0pt}my\ terminal{\isacharunderscore}{\kern0pt}func{\isacharunderscore}{\kern0pt}comp\ terminal{\isacharunderscore}{\kern0pt}func{\isacharunderscore}{\kern0pt}comp{\isacharunderscore}{\kern0pt}elem\ terminal{\isacharunderscore}{\kern0pt}func{\isacharunderscore}{\kern0pt}unique\ true{\isacharunderscore}{\kern0pt}false{\isacharunderscore}{\kern0pt}distinct\ true{\isacharunderscore}{\kern0pt}func{\isacharunderscore}{\kern0pt}type\ y{\isacharunderscore}{\kern0pt}form{\isacharparenright}{\kern0pt}\ \ \ \ \ \ \ \ \isanewline
\ \ \ \ \isacommand{next}\isamarkupfalse%
\isanewline
\ \ \ \ \ \ \isacommand{assume}\isamarkupfalse%
\ not{\isacharunderscore}{\kern0pt}case{\isadigit{2}}{\isacharcolon}{\kern0pt}\ {\isachardoublequoteopen}x\ {\isasymnoteq}\ right{\isacharunderscore}{\kern0pt}coproj\ {\isasymone}\ {\isacharparenleft}{\kern0pt}{\isasymone}\ {\isasymCoprod}\ {\isasymone}{\isacharparenright}{\kern0pt}\ {\isasymcirc}\isactrlsub c\ left{\isacharunderscore}{\kern0pt}coproj\ {\isasymone}\ {\isasymone}{\isachardoublequoteclose}\isanewline
\ \ \ \ \ \ \isacommand{then}\isamarkupfalse%
\ \isacommand{have}\isamarkupfalse%
\ case{\isadigit{3}}{\isacharcolon}{\kern0pt}\ {\isachardoublequoteopen}x\ {\isacharequal}{\kern0pt}\ right{\isacharunderscore}{\kern0pt}coproj\ {\isasymone}\ {\isacharparenleft}{\kern0pt}{\isasymone}{\isasymCoprod}{\isasymone}{\isacharparenright}{\kern0pt}\ {\isasymcirc}\isactrlsub c{\isacharparenleft}{\kern0pt}right{\isacharunderscore}{\kern0pt}coproj\ {\isasymone}\ {\isasymone}{\isacharparenright}{\kern0pt}{\isachardoublequoteclose}\isanewline
\ \ \ \ \ \ \ \ \isacommand{using}\isamarkupfalse%
\ case{\isadigit{2}}{\isacharunderscore}{\kern0pt}or{\isacharunderscore}{\kern0pt}{\isadigit{3}}\ \isacommand{by}\isamarkupfalse%
\ blast\isanewline
\ \ \ \ \ \ \isacommand{then}\isamarkupfalse%
\ \isacommand{show}\isamarkupfalse%
\ {\isachardoublequoteopen}x\ {\isacharequal}{\kern0pt}\ y{\isachardoublequoteclose}\isanewline
\ \ \ \ \ \ \ \ \isacommand{by}\isamarkupfalse%
\ {\isacharparenleft}{\kern0pt}smt\ {\isacharparenleft}{\kern0pt}verit{\isacharcomma}{\kern0pt}\ best{\isacharparenright}{\kern0pt}\ f{\isadigit{1}}\ f{\isadigit{2}}\ f{\isadigit{3}}\ \ NOR{\isacharunderscore}{\kern0pt}false{\isacharunderscore}{\kern0pt}false{\isacharunderscore}{\kern0pt}is{\isacharunderscore}{\kern0pt}true\ NOR{\isacharunderscore}{\kern0pt}is{\isacharunderscore}{\kern0pt}pullback\ case{\isadigit{3}}\ cfunc{\isacharunderscore}{\kern0pt}prod{\isacharunderscore}{\kern0pt}comp\ comp{\isacharunderscore}{\kern0pt}associative{\isadigit{2}}\ element{\isacharunderscore}{\kern0pt}pair{\isacharunderscore}{\kern0pt}eq\ false{\isacharunderscore}{\kern0pt}func{\isacharunderscore}{\kern0pt}type\ is{\isacharunderscore}{\kern0pt}pullback{\isacharunderscore}{\kern0pt}def\ left{\isacharunderscore}{\kern0pt}proj{\isacharunderscore}{\kern0pt}type\ maps{\isacharunderscore}{\kern0pt}into{\isacharunderscore}{\kern0pt}{\isadigit{1}}u{\isadigit{1}}\ mx{\isacharunderscore}{\kern0pt}eqs{\isacharunderscore}{\kern0pt}my\ pre{\isacharunderscore}{\kern0pt}OR{\isacharunderscore}{\kern0pt}type\ terminal{\isacharunderscore}{\kern0pt}func{\isacharunderscore}{\kern0pt}unique\ true{\isacharunderscore}{\kern0pt}false{\isacharunderscore}{\kern0pt}distinct\ true{\isacharunderscore}{\kern0pt}func{\isacharunderscore}{\kern0pt}type\ y{\isacharunderscore}{\kern0pt}form{\isacharparenright}{\kern0pt}\isanewline
\ \ \ \ \isacommand{qed}\isamarkupfalse%
\isanewline
\ \ \isacommand{qed}\isamarkupfalse%
\isanewline
\isacommand{qed}\isamarkupfalse%
%
\endisatagproof
{\isafoldproof}%
%
\isadelimproof
\isanewline
%
\endisadelimproof
\isanewline
\isacommand{lemma}\isamarkupfalse%
\ OR{\isacharunderscore}{\kern0pt}is{\isacharunderscore}{\kern0pt}pullback{\isacharcolon}{\kern0pt}\isanewline
\ \ {\isachardoublequoteopen}is{\isacharunderscore}{\kern0pt}pullback\ {\isacharparenleft}{\kern0pt}{\isasymone}{\isasymCoprod}{\isacharparenleft}{\kern0pt}{\isasymone}{\isasymCoprod}{\isasymone}{\isacharparenright}{\kern0pt}{\isacharparenright}{\kern0pt}\ {\isasymone}\ {\isacharparenleft}{\kern0pt}{\isasymOmega}{\isasymtimes}\isactrlsub c{\isasymOmega}{\isacharparenright}{\kern0pt}\ {\isasymOmega}\ {\isacharparenleft}{\kern0pt}{\isasymbeta}\isactrlbsub {\isacharparenleft}{\kern0pt}{\isasymone}{\isasymCoprod}{\isacharparenleft}{\kern0pt}{\isasymone}{\isasymCoprod}{\isasymone}{\isacharparenright}{\kern0pt}{\isacharparenright}{\kern0pt}\isactrlesub {\isacharparenright}{\kern0pt}\ {\isasymt}\ {\isacharparenleft}{\kern0pt}{\isasymlangle}{\isasymt}{\isacharcomma}{\kern0pt}\ {\isasymt}{\isasymrangle}{\isasymamalg}\ {\isacharparenleft}{\kern0pt}{\isasymlangle}{\isasymt}{\isacharcomma}{\kern0pt}\ {\isasymf}{\isasymrangle}\ {\isasymamalg}{\isasymlangle}{\isasymf}{\isacharcomma}{\kern0pt}\ {\isasymt}{\isasymrangle}{\isacharparenright}{\kern0pt}{\isacharparenright}{\kern0pt}\ OR{\isachardoublequoteclose}\isanewline
%
\isadelimproof
\ \ %
\endisadelimproof
%
\isatagproof
\isacommand{unfolding}\isamarkupfalse%
\ OR{\isacharunderscore}{\kern0pt}def\isanewline
\ \ \isacommand{using}\isamarkupfalse%
\ element{\isacharunderscore}{\kern0pt}monomorphism\ characteristic{\isacharunderscore}{\kern0pt}function{\isacharunderscore}{\kern0pt}exists\isanewline
\ \ \isacommand{by}\isamarkupfalse%
\ {\isacharparenleft}{\kern0pt}typecheck{\isacharunderscore}{\kern0pt}cfuncs{\isacharcomma}{\kern0pt}\ simp\ add{\isacharcolon}{\kern0pt}\ the{\isadigit{1}}I{\isadigit{2}}\ injective{\isacharunderscore}{\kern0pt}imp{\isacharunderscore}{\kern0pt}monomorphism\ pre{\isacharunderscore}{\kern0pt}OR{\isacharunderscore}{\kern0pt}injective{\isacharparenright}{\kern0pt}%
\endisatagproof
{\isafoldproof}%
%
\isadelimproof
\isanewline
%
\endisadelimproof
\ \ \ \ \ \ \isanewline
\isacommand{lemma}\isamarkupfalse%
\ OR{\isacharunderscore}{\kern0pt}type{\isacharbrackleft}{\kern0pt}type{\isacharunderscore}{\kern0pt}rule{\isacharbrackright}{\kern0pt}{\isacharcolon}{\kern0pt}\isanewline
\ \ {\isachardoublequoteopen}OR\ {\isacharcolon}{\kern0pt}\ {\isasymOmega}\ {\isasymtimes}\isactrlsub c\ {\isasymOmega}\ {\isasymrightarrow}\ {\isasymOmega}{\isachardoublequoteclose}\isanewline
%
\isadelimproof
\ \ %
\endisadelimproof
%
\isatagproof
\isacommand{unfolding}\isamarkupfalse%
\ OR{\isacharunderscore}{\kern0pt}def\isanewline
\ \ \isacommand{by}\isamarkupfalse%
\ {\isacharparenleft}{\kern0pt}metis\ OR{\isacharunderscore}{\kern0pt}def\ OR{\isacharunderscore}{\kern0pt}is{\isacharunderscore}{\kern0pt}pullback\ is{\isacharunderscore}{\kern0pt}pullback{\isacharunderscore}{\kern0pt}def{\isacharparenright}{\kern0pt}%
\endisatagproof
{\isafoldproof}%
%
\isadelimproof
\ \isanewline
%
\endisadelimproof
\isanewline
\isacommand{lemma}\isamarkupfalse%
\ OR{\isacharunderscore}{\kern0pt}true{\isacharunderscore}{\kern0pt}left{\isacharunderscore}{\kern0pt}is{\isacharunderscore}{\kern0pt}true{\isacharcolon}{\kern0pt}\isanewline
\ \ \isakeyword{assumes}\ {\isachardoublequoteopen}p\ {\isasymin}\isactrlsub c\ {\isasymOmega}{\isachardoublequoteclose}\isanewline
\ \ \isakeyword{shows}\ {\isachardoublequoteopen}OR\ {\isasymcirc}\isactrlsub c\ {\isasymlangle}{\isasymt}{\isacharcomma}{\kern0pt}p{\isasymrangle}\ {\isacharequal}{\kern0pt}\ {\isasymt}{\isachardoublequoteclose}\isanewline
%
\isadelimproof
%
\endisadelimproof
%
\isatagproof
\isacommand{proof}\isamarkupfalse%
\ {\isacharminus}{\kern0pt}\ \isanewline
\ \ \isacommand{have}\isamarkupfalse%
\ {\isachardoublequoteopen}{\isasymexists}\ j{\isachardot}{\kern0pt}\ j\ {\isasymin}\isactrlsub c\ {\isasymone}{\isasymCoprod}{\isacharparenleft}{\kern0pt}{\isasymone}{\isasymCoprod}{\isasymone}{\isacharparenright}{\kern0pt}\ {\isasymand}\ {\isacharparenleft}{\kern0pt}{\isasymlangle}{\isasymt}{\isacharcomma}{\kern0pt}\ {\isasymt}{\isasymrangle}{\isasymamalg}\ {\isacharparenleft}{\kern0pt}{\isasymlangle}{\isasymt}{\isacharcomma}{\kern0pt}\ {\isasymf}{\isasymrangle}\ {\isasymamalg}{\isasymlangle}{\isasymf}{\isacharcomma}{\kern0pt}\ {\isasymt}{\isasymrangle}{\isacharparenright}{\kern0pt}{\isacharparenright}{\kern0pt}\ {\isasymcirc}\isactrlsub c\ j\ \ {\isacharequal}{\kern0pt}\ {\isasymlangle}{\isasymt}{\isacharcomma}{\kern0pt}p{\isasymrangle}{\isachardoublequoteclose}\isanewline
\ \ \ \ \isacommand{by}\isamarkupfalse%
\ {\isacharparenleft}{\kern0pt}typecheck{\isacharunderscore}{\kern0pt}cfuncs{\isacharcomma}{\kern0pt}\ smt\ {\isacharparenleft}{\kern0pt}z{\isadigit{3}}{\isacharparenright}{\kern0pt}\ assms\ comp{\isacharunderscore}{\kern0pt}associative{\isadigit{2}}\ comp{\isacharunderscore}{\kern0pt}type\ left{\isacharunderscore}{\kern0pt}coproj{\isacharunderscore}{\kern0pt}cfunc{\isacharunderscore}{\kern0pt}coprod\isanewline
\ \ \ \ \ \ \ \ left{\isacharunderscore}{\kern0pt}proj{\isacharunderscore}{\kern0pt}type\ right{\isacharunderscore}{\kern0pt}coproj{\isacharunderscore}{\kern0pt}cfunc{\isacharunderscore}{\kern0pt}coprod\ right{\isacharunderscore}{\kern0pt}proj{\isacharunderscore}{\kern0pt}type\ true{\isacharunderscore}{\kern0pt}false{\isacharunderscore}{\kern0pt}only{\isacharunderscore}{\kern0pt}truth{\isacharunderscore}{\kern0pt}values{\isacharparenright}{\kern0pt}\isanewline
\ \ \isacommand{then}\isamarkupfalse%
\ \isacommand{show}\isamarkupfalse%
\ {\isacharquery}{\kern0pt}thesis\ \isanewline
\ \ \ \ \isacommand{by}\isamarkupfalse%
\ {\isacharparenleft}{\kern0pt}typecheck{\isacharunderscore}{\kern0pt}cfuncs{\isacharcomma}{\kern0pt}\ smt\ {\isacharparenleft}{\kern0pt}verit{\isacharcomma}{\kern0pt}\ ccfv{\isacharunderscore}{\kern0pt}SIG{\isacharparenright}{\kern0pt}\ \ NOT{\isacharunderscore}{\kern0pt}false{\isacharunderscore}{\kern0pt}is{\isacharunderscore}{\kern0pt}true\ NOT{\isacharunderscore}{\kern0pt}is{\isacharunderscore}{\kern0pt}pullback\ OR{\isacharunderscore}{\kern0pt}is{\isacharunderscore}{\kern0pt}pullback\isanewline
\ \ \ \ \ \ \ \ comp{\isacharunderscore}{\kern0pt}associative{\isadigit{2}}\ is{\isacharunderscore}{\kern0pt}pullback{\isacharunderscore}{\kern0pt}def\ terminal{\isacharunderscore}{\kern0pt}func{\isacharunderscore}{\kern0pt}comp{\isacharparenright}{\kern0pt}\isanewline
\isacommand{qed}\isamarkupfalse%
%
\endisatagproof
{\isafoldproof}%
%
\isadelimproof
\isanewline
%
\endisadelimproof
\isanewline
\isacommand{lemma}\isamarkupfalse%
\ OR{\isacharunderscore}{\kern0pt}true{\isacharunderscore}{\kern0pt}right{\isacharunderscore}{\kern0pt}is{\isacharunderscore}{\kern0pt}true{\isacharcolon}{\kern0pt}\isanewline
\ \ \isakeyword{assumes}\ {\isachardoublequoteopen}p\ {\isasymin}\isactrlsub c\ {\isasymOmega}{\isachardoublequoteclose}\isanewline
\ \ \isakeyword{shows}\ {\isachardoublequoteopen}OR\ {\isasymcirc}\isactrlsub c\ {\isasymlangle}p{\isacharcomma}{\kern0pt}{\isasymt}{\isasymrangle}\ {\isacharequal}{\kern0pt}\ {\isasymt}{\isachardoublequoteclose}\isanewline
%
\isadelimproof
%
\endisadelimproof
%
\isatagproof
\isacommand{proof}\isamarkupfalse%
\ {\isacharminus}{\kern0pt}\ \isanewline
\ \ \isacommand{have}\isamarkupfalse%
\ {\isachardoublequoteopen}{\isasymexists}\ j{\isachardot}{\kern0pt}\ j\ {\isasymin}\isactrlsub c\ {\isasymone}{\isasymCoprod}{\isacharparenleft}{\kern0pt}{\isasymone}{\isasymCoprod}{\isasymone}{\isacharparenright}{\kern0pt}\ {\isasymand}\ {\isacharparenleft}{\kern0pt}{\isasymlangle}{\isasymt}{\isacharcomma}{\kern0pt}\ {\isasymt}{\isasymrangle}{\isasymamalg}\ {\isacharparenleft}{\kern0pt}{\isasymlangle}{\isasymt}{\isacharcomma}{\kern0pt}\ {\isasymf}{\isasymrangle}\ {\isasymamalg}{\isasymlangle}{\isasymf}{\isacharcomma}{\kern0pt}\ {\isasymt}{\isasymrangle}{\isacharparenright}{\kern0pt}{\isacharparenright}{\kern0pt}\ {\isasymcirc}\isactrlsub c\ j\ {\isacharequal}{\kern0pt}\ {\isasymlangle}p{\isacharcomma}{\kern0pt}{\isasymt}{\isasymrangle}{\isachardoublequoteclose}\isanewline
\ \ \ \ \isacommand{by}\isamarkupfalse%
\ {\isacharparenleft}{\kern0pt}typecheck{\isacharunderscore}{\kern0pt}cfuncs{\isacharcomma}{\kern0pt}\ smt\ {\isacharparenleft}{\kern0pt}z{\isadigit{3}}{\isacharparenright}{\kern0pt}\ assms\ comp{\isacharunderscore}{\kern0pt}associative{\isadigit{2}}\ comp{\isacharunderscore}{\kern0pt}type\ left{\isacharunderscore}{\kern0pt}coproj{\isacharunderscore}{\kern0pt}cfunc{\isacharunderscore}{\kern0pt}coprod\isanewline
\ \ \ \ \ \ \ \ left{\isacharunderscore}{\kern0pt}proj{\isacharunderscore}{\kern0pt}type\ right{\isacharunderscore}{\kern0pt}coproj{\isacharunderscore}{\kern0pt}cfunc{\isacharunderscore}{\kern0pt}coprod\ right{\isacharunderscore}{\kern0pt}proj{\isacharunderscore}{\kern0pt}type\ true{\isacharunderscore}{\kern0pt}false{\isacharunderscore}{\kern0pt}only{\isacharunderscore}{\kern0pt}truth{\isacharunderscore}{\kern0pt}values{\isacharparenright}{\kern0pt}\isanewline
\ \ \isacommand{then}\isamarkupfalse%
\ \isacommand{show}\isamarkupfalse%
\ {\isacharquery}{\kern0pt}thesis\ \isanewline
\ \ \ \ \isacommand{by}\isamarkupfalse%
\ {\isacharparenleft}{\kern0pt}typecheck{\isacharunderscore}{\kern0pt}cfuncs{\isacharcomma}{\kern0pt}\ smt\ {\isacharparenleft}{\kern0pt}verit{\isacharcomma}{\kern0pt}\ ccfv{\isacharunderscore}{\kern0pt}SIG{\isacharparenright}{\kern0pt}\ NOT{\isacharunderscore}{\kern0pt}false{\isacharunderscore}{\kern0pt}is{\isacharunderscore}{\kern0pt}true\ NOT{\isacharunderscore}{\kern0pt}is{\isacharunderscore}{\kern0pt}pullback\ OR{\isacharunderscore}{\kern0pt}is{\isacharunderscore}{\kern0pt}pullback\isanewline
\ \ \ \ \ \ \ \ comp{\isacharunderscore}{\kern0pt}associative{\isadigit{2}}\ is{\isacharunderscore}{\kern0pt}pullback{\isacharunderscore}{\kern0pt}def\ \ terminal{\isacharunderscore}{\kern0pt}func{\isacharunderscore}{\kern0pt}comp{\isacharparenright}{\kern0pt}\isanewline
\isacommand{qed}\isamarkupfalse%
%
\endisatagproof
{\isafoldproof}%
%
\isadelimproof
\isanewline
%
\endisadelimproof
\isanewline
\isacommand{lemma}\isamarkupfalse%
\ OR{\isacharunderscore}{\kern0pt}false{\isacharunderscore}{\kern0pt}false{\isacharunderscore}{\kern0pt}is{\isacharunderscore}{\kern0pt}false{\isacharcolon}{\kern0pt}\isanewline
\ \ {\isachardoublequoteopen}OR\ {\isasymcirc}\isactrlsub c\ {\isasymlangle}{\isasymf}{\isacharcomma}{\kern0pt}{\isasymf}{\isasymrangle}\ {\isacharequal}{\kern0pt}\ {\isasymf}{\isachardoublequoteclose}\isanewline
%
\isadelimproof
%
\endisadelimproof
%
\isatagproof
\isacommand{proof}\isamarkupfalse%
{\isacharparenleft}{\kern0pt}rule\ ccontr{\isacharparenright}{\kern0pt}\isanewline
\ \ \isacommand{assume}\isamarkupfalse%
\ {\isachardoublequoteopen}OR\ {\isasymcirc}\isactrlsub c\ {\isasymlangle}{\isasymf}{\isacharcomma}{\kern0pt}{\isasymf}{\isasymrangle}\ {\isasymnoteq}\ {\isasymf}{\isachardoublequoteclose}\isanewline
\ \ \isacommand{then}\isamarkupfalse%
\ \isacommand{have}\isamarkupfalse%
\ {\isachardoublequoteopen}OR\ {\isasymcirc}\isactrlsub c\ {\isasymlangle}{\isasymf}{\isacharcomma}{\kern0pt}{\isasymf}{\isasymrangle}\ {\isacharequal}{\kern0pt}\ {\isasymt}{\isachardoublequoteclose}\isanewline
\ \ \ \ \isacommand{using}\isamarkupfalse%
\ \ true{\isacharunderscore}{\kern0pt}false{\isacharunderscore}{\kern0pt}only{\isacharunderscore}{\kern0pt}truth{\isacharunderscore}{\kern0pt}values\ \isacommand{by}\isamarkupfalse%
\ {\isacharparenleft}{\kern0pt}typecheck{\isacharunderscore}{\kern0pt}cfuncs{\isacharcomma}{\kern0pt}\ blast{\isacharparenright}{\kern0pt}\isanewline
\ \ \isacommand{then}\isamarkupfalse%
\ \isacommand{obtain}\isamarkupfalse%
\ j\ \isakeyword{where}\ \ j{\isacharunderscore}{\kern0pt}type{\isacharbrackleft}{\kern0pt}type{\isacharunderscore}{\kern0pt}rule{\isacharbrackright}{\kern0pt}{\isacharcolon}{\kern0pt}\ {\isachardoublequoteopen}j\ {\isasymin}\isactrlsub c\ {\isasymone}{\isasymCoprod}{\isacharparenleft}{\kern0pt}{\isasymone}{\isasymCoprod}{\isasymone}{\isacharparenright}{\kern0pt}{\isachardoublequoteclose}\ \isakeyword{and}\ j{\isacharunderscore}{\kern0pt}def{\isacharcolon}{\kern0pt}\ {\isachardoublequoteopen}{\isacharparenleft}{\kern0pt}{\isasymlangle}{\isasymt}{\isacharcomma}{\kern0pt}\ {\isasymt}{\isasymrangle}{\isasymamalg}\ {\isacharparenleft}{\kern0pt}{\isasymlangle}{\isasymt}{\isacharcomma}{\kern0pt}\ {\isasymf}{\isasymrangle}\ {\isasymamalg}{\isasymlangle}{\isasymf}{\isacharcomma}{\kern0pt}\ {\isasymt}{\isasymrangle}{\isacharparenright}{\kern0pt}{\isacharparenright}{\kern0pt}\ {\isasymcirc}\isactrlsub c\ j\ \ {\isacharequal}{\kern0pt}\ {\isasymlangle}{\isasymf}{\isacharcomma}{\kern0pt}{\isasymf}{\isasymrangle}{\isachardoublequoteclose}\isanewline
\ \ \ \ \isacommand{using}\isamarkupfalse%
\ \ OR{\isacharunderscore}{\kern0pt}is{\isacharunderscore}{\kern0pt}pullback\ \isacommand{unfolding}\isamarkupfalse%
\ is{\isacharunderscore}{\kern0pt}pullback{\isacharunderscore}{\kern0pt}def\ \isanewline
\ \ \ \ \isacommand{by}\isamarkupfalse%
\ {\isacharparenleft}{\kern0pt}typecheck{\isacharunderscore}{\kern0pt}cfuncs{\isacharcomma}{\kern0pt}\ metis\ id{\isacharunderscore}{\kern0pt}right{\isacharunderscore}{\kern0pt}unit{\isadigit{2}}\ id{\isacharunderscore}{\kern0pt}type{\isacharparenright}{\kern0pt}\isanewline
\ \ \isacommand{have}\isamarkupfalse%
\ trichotomy{\isacharcolon}{\kern0pt}\ {\isachardoublequoteopen}{\isacharparenleft}{\kern0pt}{\isasymlangle}{\isasymt}{\isacharcomma}{\kern0pt}\ {\isasymt}{\isasymrangle}\ {\isacharequal}{\kern0pt}\ {\isasymlangle}{\isasymf}{\isacharcomma}{\kern0pt}{\isasymf}{\isasymrangle}{\isacharparenright}{\kern0pt}\ {\isasymor}\ {\isacharparenleft}{\kern0pt}{\isacharparenleft}{\kern0pt}{\isasymlangle}{\isasymt}{\isacharcomma}{\kern0pt}\ {\isasymf}{\isasymrangle}\ {\isacharequal}{\kern0pt}\ {\isasymlangle}{\isasymf}{\isacharcomma}{\kern0pt}{\isasymf}{\isasymrangle}{\isacharparenright}{\kern0pt}\ {\isasymor}\ {\isacharparenleft}{\kern0pt}{\isasymlangle}{\isasymf}{\isacharcomma}{\kern0pt}\ {\isasymt}{\isasymrangle}\ {\isacharequal}{\kern0pt}\ {\isasymlangle}{\isasymf}{\isacharcomma}{\kern0pt}{\isasymf}{\isasymrangle}{\isacharparenright}{\kern0pt}{\isacharparenright}{\kern0pt}{\isachardoublequoteclose}\isanewline
\ \ \isacommand{proof}\isamarkupfalse%
{\isacharparenleft}{\kern0pt}cases\ {\isachardoublequoteopen}j\ {\isacharequal}{\kern0pt}\ left{\isacharunderscore}{\kern0pt}coproj\ {\isasymone}\ {\isacharparenleft}{\kern0pt}{\isasymone}\ {\isasymCoprod}\ {\isasymone}{\isacharparenright}{\kern0pt}{\isachardoublequoteclose}{\isacharparenright}{\kern0pt}\isanewline
\ \ \ \ \isacommand{assume}\isamarkupfalse%
\ case{\isadigit{1}}{\isacharcolon}{\kern0pt}\ {\isachardoublequoteopen}j\ {\isacharequal}{\kern0pt}\ left{\isacharunderscore}{\kern0pt}coproj\ {\isasymone}\ {\isacharparenleft}{\kern0pt}{\isasymone}\ {\isasymCoprod}\ {\isasymone}{\isacharparenright}{\kern0pt}{\isachardoublequoteclose}\isanewline
\ \ \ \ \isacommand{then}\isamarkupfalse%
\ \isacommand{show}\isamarkupfalse%
\ {\isacharquery}{\kern0pt}thesis\isanewline
\ \ \ \ \ \ \isacommand{using}\isamarkupfalse%
\ case{\isadigit{1}}\ cfunc{\isacharunderscore}{\kern0pt}coprod{\isacharunderscore}{\kern0pt}type\ j{\isacharunderscore}{\kern0pt}def\ left{\isacharunderscore}{\kern0pt}coproj{\isacharunderscore}{\kern0pt}cfunc{\isacharunderscore}{\kern0pt}coprod\ \isacommand{by}\isamarkupfalse%
\ {\isacharparenleft}{\kern0pt}typecheck{\isacharunderscore}{\kern0pt}cfuncs{\isacharcomma}{\kern0pt}\ force{\isacharparenright}{\kern0pt}\isanewline
\ \ \isacommand{next}\isamarkupfalse%
\isanewline
\ \ \ \ \isacommand{assume}\isamarkupfalse%
\ not{\isacharunderscore}{\kern0pt}case{\isadigit{1}}{\isacharcolon}{\kern0pt}\ {\isachardoublequoteopen}j\ {\isasymnoteq}\ left{\isacharunderscore}{\kern0pt}coproj\ {\isasymone}\ {\isacharparenleft}{\kern0pt}{\isasymone}\ {\isasymCoprod}\ {\isasymone}{\isacharparenright}{\kern0pt}{\isachardoublequoteclose}\isanewline
\ \ \ \ \isacommand{then}\isamarkupfalse%
\ \isacommand{have}\isamarkupfalse%
\ case{\isadigit{2}}{\isacharunderscore}{\kern0pt}or{\isacharunderscore}{\kern0pt}{\isadigit{3}}{\isacharcolon}{\kern0pt}\ {\isachardoublequoteopen}j\ {\isacharequal}{\kern0pt}\ right{\isacharunderscore}{\kern0pt}coproj\ {\isasymone}\ {\isacharparenleft}{\kern0pt}{\isasymone}{\isasymCoprod}{\isasymone}{\isacharparenright}{\kern0pt}\ {\isasymcirc}\isactrlsub c\ left{\isacharunderscore}{\kern0pt}coproj\ {\isasymone}\ {\isasymone}\ \ \ {\isasymor}\ \isanewline
\ \ \ \ \ \ \ \ \ \ \ \ \ \ \ \ \ \ \ \ \ \ \ \ \ \ \ j\ {\isacharequal}{\kern0pt}\ right{\isacharunderscore}{\kern0pt}coproj\ {\isasymone}\ {\isacharparenleft}{\kern0pt}{\isasymone}{\isasymCoprod}{\isasymone}{\isacharparenright}{\kern0pt}\ {\isasymcirc}\isactrlsub c\ right{\isacharunderscore}{\kern0pt}coproj\ {\isasymone}\ {\isasymone}{\isachardoublequoteclose}\isanewline
\ \ \ \ \ \ \isacommand{using}\isamarkupfalse%
\ not{\isacharunderscore}{\kern0pt}case{\isadigit{1}}\ set{\isacharunderscore}{\kern0pt}three\ \isacommand{by}\isamarkupfalse%
\ {\isacharparenleft}{\kern0pt}typecheck{\isacharunderscore}{\kern0pt}cfuncs{\isacharcomma}{\kern0pt}\ auto{\isacharparenright}{\kern0pt}\isanewline
\ \ \ \ \isacommand{show}\isamarkupfalse%
\ {\isacharquery}{\kern0pt}thesis\isanewline
\ \ \ \ \isacommand{proof}\isamarkupfalse%
{\isacharparenleft}{\kern0pt}cases\ {\isachardoublequoteopen}j\ {\isacharequal}{\kern0pt}\ {\isacharparenleft}{\kern0pt}right{\isacharunderscore}{\kern0pt}coproj\ {\isasymone}\ {\isacharparenleft}{\kern0pt}{\isasymone}{\isasymCoprod}{\isasymone}{\isacharparenright}{\kern0pt}{\isasymcirc}\isactrlsub c\ left{\isacharunderscore}{\kern0pt}coproj\ {\isasymone}\ {\isasymone}{\isacharparenright}{\kern0pt}{\isachardoublequoteclose}{\isacharparenright}{\kern0pt}\isanewline
\ \ \ \ \ \ \isacommand{assume}\isamarkupfalse%
\ case{\isadigit{2}}{\isacharcolon}{\kern0pt}\ {\isachardoublequoteopen}j\ {\isacharequal}{\kern0pt}\ right{\isacharunderscore}{\kern0pt}coproj\ {\isasymone}\ {\isacharparenleft}{\kern0pt}{\isasymone}\ {\isasymCoprod}\ {\isasymone}{\isacharparenright}{\kern0pt}\ {\isasymcirc}\isactrlsub c\ left{\isacharunderscore}{\kern0pt}coproj\ {\isasymone}\ {\isasymone}{\isachardoublequoteclose}\isanewline
\ \ \ \ \ \ \isacommand{have}\isamarkupfalse%
\ {\isachardoublequoteopen}{\isasymlangle}{\isasymt}{\isacharcomma}{\kern0pt}\ {\isasymf}{\isasymrangle}\ {\isacharequal}{\kern0pt}\ {\isasymlangle}{\isasymf}{\isacharcomma}{\kern0pt}{\isasymf}{\isasymrangle}{\isachardoublequoteclose}\isanewline
\ \ \ \ \ \ \isacommand{proof}\isamarkupfalse%
\ {\isacharminus}{\kern0pt}\ \isanewline
\ \ \ \ \ \ \ \ \isacommand{have}\isamarkupfalse%
\ {\isachardoublequoteopen}{\isacharparenleft}{\kern0pt}{\isasymlangle}{\isasymt}{\isacharcomma}{\kern0pt}\ {\isasymt}{\isasymrangle}{\isasymamalg}\ {\isacharparenleft}{\kern0pt}{\isasymlangle}{\isasymt}{\isacharcomma}{\kern0pt}\ {\isasymf}{\isasymrangle}\ {\isasymamalg}{\isasymlangle}{\isasymf}{\isacharcomma}{\kern0pt}\ {\isasymt}{\isasymrangle}{\isacharparenright}{\kern0pt}{\isacharparenright}{\kern0pt}\ {\isasymcirc}\isactrlsub c\ j\ {\isacharequal}{\kern0pt}\ {\isacharparenleft}{\kern0pt}{\isacharparenleft}{\kern0pt}{\isasymlangle}{\isasymt}{\isacharcomma}{\kern0pt}\ {\isasymt}{\isasymrangle}{\isasymamalg}\ {\isacharparenleft}{\kern0pt}{\isasymlangle}{\isasymt}{\isacharcomma}{\kern0pt}\ {\isasymf}{\isasymrangle}\ {\isasymamalg}{\isasymlangle}{\isasymf}{\isacharcomma}{\kern0pt}\ {\isasymt}{\isasymrangle}{\isacharparenright}{\kern0pt}{\isacharparenright}{\kern0pt}\ {\isasymcirc}\isactrlsub c\ right{\isacharunderscore}{\kern0pt}coproj\ {\isasymone}\ {\isacharparenleft}{\kern0pt}{\isasymone}\ {\isasymCoprod}\ {\isasymone}{\isacharparenright}{\kern0pt}{\isacharparenright}{\kern0pt}\ {\isasymcirc}\isactrlsub c\ left{\isacharunderscore}{\kern0pt}coproj\ {\isasymone}\ {\isasymone}{\isachardoublequoteclose}\isanewline
\ \ \ \ \ \ \ \ \ \ \isacommand{by}\isamarkupfalse%
\ {\isacharparenleft}{\kern0pt}typecheck{\isacharunderscore}{\kern0pt}cfuncs{\isacharcomma}{\kern0pt}\ simp\ add{\isacharcolon}{\kern0pt}\ case{\isadigit{2}}\ comp{\isacharunderscore}{\kern0pt}associative{\isadigit{2}}{\isacharparenright}{\kern0pt}\isanewline
\ \ \ \ \ \ \ \ \isacommand{also}\isamarkupfalse%
\ \isacommand{have}\isamarkupfalse%
\ {\isachardoublequoteopen}{\isachardot}{\kern0pt}{\isachardot}{\kern0pt}{\isachardot}{\kern0pt}\ {\isacharequal}{\kern0pt}\ {\isacharparenleft}{\kern0pt}{\isasymlangle}{\isasymt}{\isacharcomma}{\kern0pt}\ {\isasymf}{\isasymrangle}\ {\isasymamalg}{\isasymlangle}{\isasymf}{\isacharcomma}{\kern0pt}\ {\isasymt}{\isasymrangle}{\isacharparenright}{\kern0pt}\ {\isasymcirc}\isactrlsub c\ left{\isacharunderscore}{\kern0pt}coproj\ {\isasymone}\ {\isasymone}{\isachardoublequoteclose}\isanewline
\ \ \ \ \ \ \ \ \ \ \isacommand{using}\isamarkupfalse%
\ right{\isacharunderscore}{\kern0pt}coproj{\isacharunderscore}{\kern0pt}cfunc{\isacharunderscore}{\kern0pt}coprod\ \isacommand{by}\isamarkupfalse%
\ {\isacharparenleft}{\kern0pt}typecheck{\isacharunderscore}{\kern0pt}cfuncs{\isacharcomma}{\kern0pt}\ presburger{\isacharparenright}{\kern0pt}\isanewline
\ \ \ \ \ \ \ \ \isacommand{also}\isamarkupfalse%
\ \isacommand{have}\isamarkupfalse%
\ {\isachardoublequoteopen}{\isachardot}{\kern0pt}{\isachardot}{\kern0pt}{\isachardot}{\kern0pt}\ {\isacharequal}{\kern0pt}\ {\isasymlangle}{\isasymt}{\isacharcomma}{\kern0pt}\ {\isasymf}{\isasymrangle}{\isachardoublequoteclose}\isanewline
\ \ \ \ \ \ \ \ \ \ \isacommand{by}\isamarkupfalse%
\ {\isacharparenleft}{\kern0pt}typecheck{\isacharunderscore}{\kern0pt}cfuncs{\isacharcomma}{\kern0pt}\ simp\ add{\isacharcolon}{\kern0pt}\ left{\isacharunderscore}{\kern0pt}coproj{\isacharunderscore}{\kern0pt}cfunc{\isacharunderscore}{\kern0pt}coprod{\isacharparenright}{\kern0pt}\isanewline
\ \ \ \ \ \ \ \ \isacommand{then}\isamarkupfalse%
\ \isacommand{show}\isamarkupfalse%
\ {\isacharquery}{\kern0pt}thesis\isanewline
\ \ \ \ \ \ \ \ \ \ \isacommand{using}\isamarkupfalse%
\ calculation\ j{\isacharunderscore}{\kern0pt}def\ \isacommand{by}\isamarkupfalse%
\ presburger\isanewline
\ \ \ \ \ \ \isacommand{qed}\isamarkupfalse%
\isanewline
\ \ \ \ \ \ \isacommand{then}\isamarkupfalse%
\ \isacommand{show}\isamarkupfalse%
\ {\isacharquery}{\kern0pt}thesis\isanewline
\ \ \ \ \ \ \ \ \isacommand{by}\isamarkupfalse%
\ blast\isanewline
\ \ \ \ \isacommand{next}\isamarkupfalse%
\isanewline
\ \ \ \ \ \ \isacommand{assume}\isamarkupfalse%
\ not{\isacharunderscore}{\kern0pt}case{\isadigit{2}}{\isacharcolon}{\kern0pt}\ {\isachardoublequoteopen}j\ {\isasymnoteq}\ right{\isacharunderscore}{\kern0pt}coproj\ {\isasymone}\ {\isacharparenleft}{\kern0pt}{\isasymone}\ {\isasymCoprod}\ {\isasymone}{\isacharparenright}{\kern0pt}\ {\isasymcirc}\isactrlsub c\ left{\isacharunderscore}{\kern0pt}coproj\ {\isasymone}\ {\isasymone}{\isachardoublequoteclose}\isanewline
\ \ \ \ \ \ \isacommand{then}\isamarkupfalse%
\ \isacommand{have}\isamarkupfalse%
\ case{\isadigit{3}}{\isacharcolon}{\kern0pt}\ {\isachardoublequoteopen}j\ {\isacharequal}{\kern0pt}\ right{\isacharunderscore}{\kern0pt}coproj\ {\isasymone}\ {\isacharparenleft}{\kern0pt}{\isasymone}{\isasymCoprod}{\isasymone}{\isacharparenright}{\kern0pt}\ {\isasymcirc}\isactrlsub c\ right{\isacharunderscore}{\kern0pt}coproj\ {\isasymone}\ {\isasymone}{\isachardoublequoteclose}\isanewline
\ \ \ \ \ \ \ \ \isacommand{using}\isamarkupfalse%
\ case{\isadigit{2}}{\isacharunderscore}{\kern0pt}or{\isacharunderscore}{\kern0pt}{\isadigit{3}}\ \isacommand{by}\isamarkupfalse%
\ blast\isanewline
\ \ \ \ \ \ \isacommand{have}\isamarkupfalse%
\ {\isachardoublequoteopen}{\isasymlangle}{\isasymf}{\isacharcomma}{\kern0pt}\ {\isasymt}{\isasymrangle}\ {\isacharequal}{\kern0pt}\ {\isasymlangle}{\isasymf}{\isacharcomma}{\kern0pt}{\isasymf}{\isasymrangle}{\isachardoublequoteclose}\isanewline
\ \ \ \ \ \ \isacommand{proof}\isamarkupfalse%
\ {\isacharminus}{\kern0pt}\ \isanewline
\ \ \ \ \ \ \ \ \isacommand{have}\isamarkupfalse%
\ {\isachardoublequoteopen}{\isacharparenleft}{\kern0pt}{\isasymlangle}{\isasymt}{\isacharcomma}{\kern0pt}\ {\isasymt}{\isasymrangle}{\isasymamalg}\ {\isacharparenleft}{\kern0pt}{\isasymlangle}{\isasymt}{\isacharcomma}{\kern0pt}\ {\isasymf}{\isasymrangle}\ {\isasymamalg}{\isasymlangle}{\isasymf}{\isacharcomma}{\kern0pt}\ {\isasymt}{\isasymrangle}{\isacharparenright}{\kern0pt}{\isacharparenright}{\kern0pt}\ {\isasymcirc}\isactrlsub c\ j\ {\isacharequal}{\kern0pt}\ {\isacharparenleft}{\kern0pt}{\isacharparenleft}{\kern0pt}{\isasymlangle}{\isasymt}{\isacharcomma}{\kern0pt}\ {\isasymt}{\isasymrangle}{\isasymamalg}\ {\isacharparenleft}{\kern0pt}{\isasymlangle}{\isasymt}{\isacharcomma}{\kern0pt}\ {\isasymf}{\isasymrangle}\ {\isasymamalg}{\isasymlangle}{\isasymf}{\isacharcomma}{\kern0pt}\ {\isasymt}{\isasymrangle}{\isacharparenright}{\kern0pt}{\isacharparenright}{\kern0pt}\ {\isasymcirc}\isactrlsub c\ right{\isacharunderscore}{\kern0pt}coproj\ {\isasymone}\ {\isacharparenleft}{\kern0pt}{\isasymone}\ {\isasymCoprod}\ {\isasymone}{\isacharparenright}{\kern0pt}{\isacharparenright}{\kern0pt}\ {\isasymcirc}\isactrlsub c\ right{\isacharunderscore}{\kern0pt}coproj\ {\isasymone}\ {\isasymone}{\isachardoublequoteclose}\isanewline
\ \ \ \ \ \ \ \ \ \ \isacommand{by}\isamarkupfalse%
\ {\isacharparenleft}{\kern0pt}typecheck{\isacharunderscore}{\kern0pt}cfuncs{\isacharcomma}{\kern0pt}\ simp\ add{\isacharcolon}{\kern0pt}\ case{\isadigit{3}}\ comp{\isacharunderscore}{\kern0pt}associative{\isadigit{2}}{\isacharparenright}{\kern0pt}\isanewline
\ \ \ \ \ \ \ \ \isacommand{also}\isamarkupfalse%
\ \isacommand{have}\isamarkupfalse%
\ {\isachardoublequoteopen}{\isachardot}{\kern0pt}{\isachardot}{\kern0pt}{\isachardot}{\kern0pt}\ {\isacharequal}{\kern0pt}\ {\isacharparenleft}{\kern0pt}{\isasymlangle}{\isasymt}{\isacharcomma}{\kern0pt}\ {\isasymf}{\isasymrangle}\ {\isasymamalg}{\isasymlangle}{\isasymf}{\isacharcomma}{\kern0pt}\ {\isasymt}{\isasymrangle}{\isacharparenright}{\kern0pt}\ {\isasymcirc}\isactrlsub c\ right{\isacharunderscore}{\kern0pt}coproj\ {\isasymone}\ {\isasymone}{\isachardoublequoteclose}\isanewline
\ \ \ \ \ \ \ \ \ \ \isacommand{using}\isamarkupfalse%
\ right{\isacharunderscore}{\kern0pt}coproj{\isacharunderscore}{\kern0pt}cfunc{\isacharunderscore}{\kern0pt}coprod\ \isacommand{by}\isamarkupfalse%
\ {\isacharparenleft}{\kern0pt}typecheck{\isacharunderscore}{\kern0pt}cfuncs{\isacharcomma}{\kern0pt}\ presburger{\isacharparenright}{\kern0pt}\isanewline
\ \ \ \ \ \ \ \ \isacommand{also}\isamarkupfalse%
\ \isacommand{have}\isamarkupfalse%
\ {\isachardoublequoteopen}{\isachardot}{\kern0pt}{\isachardot}{\kern0pt}{\isachardot}{\kern0pt}\ {\isacharequal}{\kern0pt}\ {\isasymlangle}{\isasymf}{\isacharcomma}{\kern0pt}\ {\isasymt}{\isasymrangle}{\isachardoublequoteclose}\isanewline
\ \ \ \ \ \ \ \ \ \ \isacommand{by}\isamarkupfalse%
\ {\isacharparenleft}{\kern0pt}typecheck{\isacharunderscore}{\kern0pt}cfuncs{\isacharcomma}{\kern0pt}\ simp\ add{\isacharcolon}{\kern0pt}\ right{\isacharunderscore}{\kern0pt}coproj{\isacharunderscore}{\kern0pt}cfunc{\isacharunderscore}{\kern0pt}coprod{\isacharparenright}{\kern0pt}\isanewline
\ \ \ \ \ \ \ \ \isacommand{then}\isamarkupfalse%
\ \isacommand{show}\isamarkupfalse%
\ {\isacharquery}{\kern0pt}thesis\isanewline
\ \ \ \ \ \ \ \ \ \ \isacommand{using}\isamarkupfalse%
\ calculation\ j{\isacharunderscore}{\kern0pt}def\ \isacommand{by}\isamarkupfalse%
\ presburger\isanewline
\ \ \ \ \ \ \isacommand{qed}\isamarkupfalse%
\isanewline
\ \ \ \ \ \ \isacommand{then}\isamarkupfalse%
\ \isacommand{show}\isamarkupfalse%
\ {\isacharquery}{\kern0pt}thesis\isanewline
\ \ \ \ \ \ \ \ \isacommand{by}\isamarkupfalse%
\ blast\isanewline
\ \ \ \ \isacommand{qed}\isamarkupfalse%
\isanewline
\ \ \isacommand{qed}\isamarkupfalse%
\isanewline
\ \ \ \ \isacommand{then}\isamarkupfalse%
\ \isacommand{have}\isamarkupfalse%
\ {\isachardoublequoteopen}{\isasymt}\ {\isacharequal}{\kern0pt}\ {\isasymf}{\isachardoublequoteclose}\isanewline
\ \ \ \ \ \ \isacommand{using}\isamarkupfalse%
\ trichotomy\ cart{\isacharunderscore}{\kern0pt}prod{\isacharunderscore}{\kern0pt}eq{\isadigit{2}}\ \isacommand{by}\isamarkupfalse%
\ {\isacharparenleft}{\kern0pt}typecheck{\isacharunderscore}{\kern0pt}cfuncs{\isacharcomma}{\kern0pt}\ force{\isacharparenright}{\kern0pt}\isanewline
\ \ \ \ \isacommand{then}\isamarkupfalse%
\ \isacommand{show}\isamarkupfalse%
\ False\isanewline
\ \ \ \ \ \ \isacommand{using}\isamarkupfalse%
\ true{\isacharunderscore}{\kern0pt}false{\isacharunderscore}{\kern0pt}distinct\ \isacommand{by}\isamarkupfalse%
\ smt\isanewline
\isacommand{qed}\isamarkupfalse%
%
\endisatagproof
{\isafoldproof}%
%
\isadelimproof
\isanewline
%
\endisadelimproof
\isanewline
\isacommand{lemma}\isamarkupfalse%
\ OR{\isacharunderscore}{\kern0pt}true{\isacharunderscore}{\kern0pt}implies{\isacharunderscore}{\kern0pt}one{\isacharunderscore}{\kern0pt}is{\isacharunderscore}{\kern0pt}true{\isacharcolon}{\kern0pt}\isanewline
\ \ \isakeyword{assumes}\ {\isachardoublequoteopen}p\ {\isasymin}\isactrlsub c\ {\isasymOmega}{\isachardoublequoteclose}\ \isanewline
\ \ \isakeyword{assumes}\ {\isachardoublequoteopen}q\ {\isasymin}\isactrlsub c\ {\isasymOmega}{\isachardoublequoteclose}\isanewline
\ \ \isakeyword{assumes}\ {\isachardoublequoteopen}OR\ {\isasymcirc}\isactrlsub c\ {\isasymlangle}p{\isacharcomma}{\kern0pt}q{\isasymrangle}\ {\isacharequal}{\kern0pt}\ {\isasymt}{\isachardoublequoteclose}\isanewline
\ \ \isakeyword{shows}\ {\isachardoublequoteopen}{\isacharparenleft}{\kern0pt}p\ {\isacharequal}{\kern0pt}\ {\isasymt}{\isacharparenright}{\kern0pt}\ {\isasymor}\ {\isacharparenleft}{\kern0pt}q\ {\isacharequal}{\kern0pt}\ {\isasymt}{\isacharparenright}{\kern0pt}{\isachardoublequoteclose}\isanewline
%
\isadelimproof
\ \ %
\endisadelimproof
%
\isatagproof
\isacommand{by}\isamarkupfalse%
\ {\isacharparenleft}{\kern0pt}metis\ OR{\isacharunderscore}{\kern0pt}false{\isacharunderscore}{\kern0pt}false{\isacharunderscore}{\kern0pt}is{\isacharunderscore}{\kern0pt}false\ assms\ true{\isacharunderscore}{\kern0pt}false{\isacharunderscore}{\kern0pt}only{\isacharunderscore}{\kern0pt}truth{\isacharunderscore}{\kern0pt}values{\isacharparenright}{\kern0pt}%
\endisatagproof
{\isafoldproof}%
%
\isadelimproof
\isanewline
%
\endisadelimproof
\isanewline
\isacommand{lemma}\isamarkupfalse%
\ NOT{\isacharunderscore}{\kern0pt}NOR{\isacharunderscore}{\kern0pt}is{\isacharunderscore}{\kern0pt}OR{\isacharcolon}{\kern0pt}\isanewline
\ {\isachardoublequoteopen}OR\ {\isacharequal}{\kern0pt}\ NOT\ {\isasymcirc}\isactrlsub c\ NOR{\isachardoublequoteclose}\isanewline
%
\isadelimproof
%
\endisadelimproof
%
\isatagproof
\isacommand{proof}\isamarkupfalse%
{\isacharparenleft}{\kern0pt}etcs{\isacharunderscore}{\kern0pt}rule\ one{\isacharunderscore}{\kern0pt}separator{\isacharparenright}{\kern0pt}\isanewline
\ \ \isacommand{fix}\isamarkupfalse%
\ x\ \isanewline
\ \ \isacommand{assume}\isamarkupfalse%
\ x{\isacharunderscore}{\kern0pt}type{\isacharbrackleft}{\kern0pt}type{\isacharunderscore}{\kern0pt}rule{\isacharbrackright}{\kern0pt}{\isacharcolon}{\kern0pt}\ {\isachardoublequoteopen}x\ {\isasymin}\isactrlsub c\ {\isasymOmega}\ {\isasymtimes}\isactrlsub c\ {\isasymOmega}{\isachardoublequoteclose}\isanewline
\ \ \isacommand{then}\isamarkupfalse%
\ \isacommand{obtain}\isamarkupfalse%
\ p\ q\ \isakeyword{where}\ p{\isacharunderscore}{\kern0pt}type{\isacharbrackleft}{\kern0pt}type{\isacharunderscore}{\kern0pt}rule{\isacharbrackright}{\kern0pt}{\isacharcolon}{\kern0pt}\ {\isachardoublequoteopen}p\ {\isasymin}\isactrlsub c\ {\isasymOmega}{\isachardoublequoteclose}\ \isakeyword{and}\ q{\isacharunderscore}{\kern0pt}type{\isacharbrackleft}{\kern0pt}type{\isacharunderscore}{\kern0pt}rule{\isacharbrackright}{\kern0pt}{\isacharcolon}{\kern0pt}\ \ {\isachardoublequoteopen}q\ {\isasymin}\isactrlsub c\ {\isasymOmega}{\isachardoublequoteclose}\ \isakeyword{and}\ x{\isacharunderscore}{\kern0pt}def{\isacharcolon}{\kern0pt}\ {\isachardoublequoteopen}x\ {\isacharequal}{\kern0pt}\ {\isasymlangle}p{\isacharcomma}{\kern0pt}q{\isasymrangle}{\isachardoublequoteclose}\isanewline
\ \ \ \ \isacommand{by}\isamarkupfalse%
\ {\isacharparenleft}{\kern0pt}meson\ cart{\isacharunderscore}{\kern0pt}prod{\isacharunderscore}{\kern0pt}decomp{\isacharparenright}{\kern0pt}\isanewline
\ \ \isacommand{show}\isamarkupfalse%
\ {\isachardoublequoteopen}OR\ {\isasymcirc}\isactrlsub c\ x\ {\isacharequal}{\kern0pt}\ {\isacharparenleft}{\kern0pt}NOT\ {\isasymcirc}\isactrlsub c\ NOR{\isacharparenright}{\kern0pt}\ {\isasymcirc}\isactrlsub c\ x{\isachardoublequoteclose}\isanewline
\ \ \isacommand{proof}\isamarkupfalse%
{\isacharparenleft}{\kern0pt}cases\ {\isachardoublequoteopen}p\ {\isacharequal}{\kern0pt}\ {\isasymt}{\isachardoublequoteclose}{\isacharparenright}{\kern0pt}\isanewline
\ \ \ \ \isacommand{show}\isamarkupfalse%
\ {\isachardoublequoteopen}p\ {\isacharequal}{\kern0pt}\ {\isasymt}\ {\isasymLongrightarrow}\ OR\ {\isasymcirc}\isactrlsub c\ x\ {\isacharequal}{\kern0pt}\ {\isacharparenleft}{\kern0pt}NOT\ {\isasymcirc}\isactrlsub c\ NOR{\isacharparenright}{\kern0pt}\ {\isasymcirc}\isactrlsub c\ x{\isachardoublequoteclose}\isanewline
\ \ \ \ \ \ \isacommand{by}\isamarkupfalse%
\ {\isacharparenleft}{\kern0pt}typecheck{\isacharunderscore}{\kern0pt}cfuncs{\isacharcomma}{\kern0pt}\ metis\ NOR{\isacharunderscore}{\kern0pt}left{\isacharunderscore}{\kern0pt}true{\isacharunderscore}{\kern0pt}is{\isacharunderscore}{\kern0pt}false\ NOT{\isacharunderscore}{\kern0pt}false{\isacharunderscore}{\kern0pt}is{\isacharunderscore}{\kern0pt}true\ OR{\isacharunderscore}{\kern0pt}true{\isacharunderscore}{\kern0pt}left{\isacharunderscore}{\kern0pt}is{\isacharunderscore}{\kern0pt}true\ comp{\isacharunderscore}{\kern0pt}associative{\isadigit{2}}\ q{\isacharunderscore}{\kern0pt}type\ x{\isacharunderscore}{\kern0pt}def{\isacharparenright}{\kern0pt}\isanewline
\ \ \isacommand{next}\isamarkupfalse%
\isanewline
\ \ \ \ \isacommand{assume}\isamarkupfalse%
\ {\isachardoublequoteopen}p\ {\isasymnoteq}\ {\isasymt}{\isachardoublequoteclose}\isanewline
\ \ \ \ \isacommand{then}\isamarkupfalse%
\ \isacommand{have}\isamarkupfalse%
\ {\isachardoublequoteopen}p\ {\isacharequal}{\kern0pt}\ {\isasymf}{\isachardoublequoteclose}\isanewline
\ \ \ \ \ \ \isacommand{using}\isamarkupfalse%
\ p{\isacharunderscore}{\kern0pt}type\ true{\isacharunderscore}{\kern0pt}false{\isacharunderscore}{\kern0pt}only{\isacharunderscore}{\kern0pt}truth{\isacharunderscore}{\kern0pt}values\ \isacommand{by}\isamarkupfalse%
\ blast\isanewline
\ \ \ \ \isacommand{show}\isamarkupfalse%
\ {\isachardoublequoteopen}OR\ {\isasymcirc}\isactrlsub c\ x\ {\isacharequal}{\kern0pt}\ {\isacharparenleft}{\kern0pt}NOT\ {\isasymcirc}\isactrlsub c\ NOR{\isacharparenright}{\kern0pt}\ {\isasymcirc}\isactrlsub c\ x{\isachardoublequoteclose}\isanewline
\ \ \ \ \isacommand{proof}\isamarkupfalse%
{\isacharparenleft}{\kern0pt}cases\ {\isachardoublequoteopen}q\ {\isacharequal}{\kern0pt}\ {\isasymt}{\isachardoublequoteclose}{\isacharparenright}{\kern0pt}\isanewline
\ \ \ \ \ \ \isacommand{show}\isamarkupfalse%
\ {\isachardoublequoteopen}q\ {\isacharequal}{\kern0pt}\ {\isasymt}\ {\isasymLongrightarrow}\ OR\ {\isasymcirc}\isactrlsub c\ x\ {\isacharequal}{\kern0pt}\ {\isacharparenleft}{\kern0pt}NOT\ {\isasymcirc}\isactrlsub c\ NOR{\isacharparenright}{\kern0pt}\ {\isasymcirc}\isactrlsub c\ x{\isachardoublequoteclose}\isanewline
\ \ \ \ \ \ \ \ \isacommand{by}\isamarkupfalse%
\ {\isacharparenleft}{\kern0pt}typecheck{\isacharunderscore}{\kern0pt}cfuncs{\isacharcomma}{\kern0pt}\ metis\ NOR{\isacharunderscore}{\kern0pt}right{\isacharunderscore}{\kern0pt}true{\isacharunderscore}{\kern0pt}is{\isacharunderscore}{\kern0pt}false\ NOT{\isacharunderscore}{\kern0pt}false{\isacharunderscore}{\kern0pt}is{\isacharunderscore}{\kern0pt}true\ OR{\isacharunderscore}{\kern0pt}true{\isacharunderscore}{\kern0pt}right{\isacharunderscore}{\kern0pt}is{\isacharunderscore}{\kern0pt}true\ \isanewline
\ \ \ \ \ \ \ \ \ \ \ \ cfunc{\isacharunderscore}{\kern0pt}type{\isacharunderscore}{\kern0pt}def\ comp{\isacharunderscore}{\kern0pt}associative\ p{\isacharunderscore}{\kern0pt}type\ x{\isacharunderscore}{\kern0pt}def{\isacharparenright}{\kern0pt}\isanewline
\ \ \ \ \isacommand{next}\isamarkupfalse%
\isanewline
\ \ \ \ \ \ \isacommand{assume}\isamarkupfalse%
\ {\isachardoublequoteopen}q\ {\isasymnoteq}\ {\isasymt}{\isachardoublequoteclose}\isanewline
\ \ \ \ \ \ \isacommand{then}\isamarkupfalse%
\ \isacommand{show}\isamarkupfalse%
\ {\isacharquery}{\kern0pt}thesis\isanewline
\ \ \ \ \ \ \ \ \isacommand{by}\isamarkupfalse%
\ {\isacharparenleft}{\kern0pt}typecheck{\isacharunderscore}{\kern0pt}cfuncs{\isacharcomma}{\kern0pt}metis\ NOR{\isacharunderscore}{\kern0pt}false{\isacharunderscore}{\kern0pt}false{\isacharunderscore}{\kern0pt}is{\isacharunderscore}{\kern0pt}true\ NOT{\isacharunderscore}{\kern0pt}is{\isacharunderscore}{\kern0pt}true{\isacharunderscore}{\kern0pt}implies{\isacharunderscore}{\kern0pt}false\ OR{\isacharunderscore}{\kern0pt}false{\isacharunderscore}{\kern0pt}false{\isacharunderscore}{\kern0pt}is{\isacharunderscore}{\kern0pt}false\isanewline
\ \ \ \ \ \ \ \ \ \ \ \ {\isacartoucheopen}p\ {\isacharequal}{\kern0pt}\ {\isasymf}{\isacartoucheclose}\ \ comp{\isacharunderscore}{\kern0pt}associative{\isadigit{2}}\ q{\isacharunderscore}{\kern0pt}type\ true{\isacharunderscore}{\kern0pt}false{\isacharunderscore}{\kern0pt}only{\isacharunderscore}{\kern0pt}truth{\isacharunderscore}{\kern0pt}values\ x{\isacharunderscore}{\kern0pt}def{\isacharparenright}{\kern0pt}\isanewline
\ \ \ \ \isacommand{qed}\isamarkupfalse%
\isanewline
\ \ \isacommand{qed}\isamarkupfalse%
\isanewline
\isacommand{qed}\isamarkupfalse%
%
\endisatagproof
{\isafoldproof}%
%
\isadelimproof
\isanewline
%
\endisadelimproof
\isanewline
\isacommand{lemma}\isamarkupfalse%
\ OR{\isacharunderscore}{\kern0pt}commutative{\isacharcolon}{\kern0pt}\isanewline
\ \ \isakeyword{assumes}\ {\isachardoublequoteopen}p\ {\isasymin}\isactrlsub c\ {\isasymOmega}{\isachardoublequoteclose}\isanewline
\ \ \isakeyword{assumes}\ {\isachardoublequoteopen}q\ {\isasymin}\isactrlsub c\ {\isasymOmega}{\isachardoublequoteclose}\isanewline
\ \ \isakeyword{shows}\ {\isachardoublequoteopen}OR\ {\isasymcirc}\isactrlsub c\ {\isasymlangle}p{\isacharcomma}{\kern0pt}q{\isasymrangle}\ {\isacharequal}{\kern0pt}\ OR\ {\isasymcirc}\isactrlsub c\ {\isasymlangle}q{\isacharcomma}{\kern0pt}p{\isasymrangle}{\isachardoublequoteclose}\isanewline
%
\isadelimproof
\ \ %
\endisadelimproof
%
\isatagproof
\isacommand{by}\isamarkupfalse%
\ {\isacharparenleft}{\kern0pt}metis\ OR{\isacharunderscore}{\kern0pt}true{\isacharunderscore}{\kern0pt}left{\isacharunderscore}{\kern0pt}is{\isacharunderscore}{\kern0pt}true\ OR{\isacharunderscore}{\kern0pt}true{\isacharunderscore}{\kern0pt}right{\isacharunderscore}{\kern0pt}is{\isacharunderscore}{\kern0pt}true\ assms\ true{\isacharunderscore}{\kern0pt}false{\isacharunderscore}{\kern0pt}only{\isacharunderscore}{\kern0pt}truth{\isacharunderscore}{\kern0pt}values{\isacharparenright}{\kern0pt}%
\endisatagproof
{\isafoldproof}%
%
\isadelimproof
\isanewline
%
\endisadelimproof
\isanewline
\isacommand{lemma}\isamarkupfalse%
\ OR{\isacharunderscore}{\kern0pt}idempotent{\isacharcolon}{\kern0pt}\isanewline
\ \ \isakeyword{assumes}\ {\isachardoublequoteopen}p\ {\isasymin}\isactrlsub c\ {\isasymOmega}{\isachardoublequoteclose}\isanewline
\ \ \isakeyword{shows}\ {\isachardoublequoteopen}OR\ {\isasymcirc}\isactrlsub c\ {\isasymlangle}p{\isacharcomma}{\kern0pt}p{\isasymrangle}\ {\isacharequal}{\kern0pt}\ p{\isachardoublequoteclose}\isanewline
%
\isadelimproof
\ \ %
\endisadelimproof
%
\isatagproof
\isacommand{using}\isamarkupfalse%
\ OR{\isacharunderscore}{\kern0pt}false{\isacharunderscore}{\kern0pt}false{\isacharunderscore}{\kern0pt}is{\isacharunderscore}{\kern0pt}false\ OR{\isacharunderscore}{\kern0pt}true{\isacharunderscore}{\kern0pt}left{\isacharunderscore}{\kern0pt}is{\isacharunderscore}{\kern0pt}true\ assms\ true{\isacharunderscore}{\kern0pt}false{\isacharunderscore}{\kern0pt}only{\isacharunderscore}{\kern0pt}truth{\isacharunderscore}{\kern0pt}values\ \isacommand{by}\isamarkupfalse%
\ blast%
\endisatagproof
{\isafoldproof}%
%
\isadelimproof
\isanewline
%
\endisadelimproof
\isanewline
\isacommand{lemma}\isamarkupfalse%
\ OR{\isacharunderscore}{\kern0pt}associative{\isacharcolon}{\kern0pt}\isanewline
\ \ \isakeyword{assumes}\ {\isachardoublequoteopen}p\ {\isasymin}\isactrlsub c\ {\isasymOmega}{\isachardoublequoteclose}\isanewline
\ \ \isakeyword{assumes}\ {\isachardoublequoteopen}q\ {\isasymin}\isactrlsub c\ {\isasymOmega}{\isachardoublequoteclose}\isanewline
\ \ \isakeyword{assumes}\ {\isachardoublequoteopen}r\ {\isasymin}\isactrlsub c\ {\isasymOmega}{\isachardoublequoteclose}\isanewline
\ \ \isakeyword{shows}\ {\isachardoublequoteopen}OR\ {\isasymcirc}\isactrlsub c\ {\isasymlangle}OR\ {\isasymcirc}\isactrlsub c\ {\isasymlangle}p{\isacharcomma}{\kern0pt}q{\isasymrangle}{\isacharcomma}{\kern0pt}\ r{\isasymrangle}\ {\isacharequal}{\kern0pt}\ OR\ {\isasymcirc}\isactrlsub c\ {\isasymlangle}p{\isacharcomma}{\kern0pt}\ OR\ {\isasymcirc}\isactrlsub c\ {\isasymlangle}q{\isacharcomma}{\kern0pt}r{\isasymrangle}{\isasymrangle}{\isachardoublequoteclose}\isanewline
%
\isadelimproof
\ \ %
\endisadelimproof
%
\isatagproof
\isacommand{by}\isamarkupfalse%
\ {\isacharparenleft}{\kern0pt}metis\ OR{\isacharunderscore}{\kern0pt}commutative\ OR{\isacharunderscore}{\kern0pt}false{\isacharunderscore}{\kern0pt}false{\isacharunderscore}{\kern0pt}is{\isacharunderscore}{\kern0pt}false\ OR{\isacharunderscore}{\kern0pt}true{\isacharunderscore}{\kern0pt}right{\isacharunderscore}{\kern0pt}is{\isacharunderscore}{\kern0pt}true\ assms\ true{\isacharunderscore}{\kern0pt}false{\isacharunderscore}{\kern0pt}only{\isacharunderscore}{\kern0pt}truth{\isacharunderscore}{\kern0pt}values{\isacharparenright}{\kern0pt}%
\endisatagproof
{\isafoldproof}%
%
\isadelimproof
\isanewline
%
\endisadelimproof
\isanewline
\isacommand{lemma}\isamarkupfalse%
\ OR{\isacharunderscore}{\kern0pt}complementary{\isacharcolon}{\kern0pt}\isanewline
\ \ \isakeyword{assumes}\ {\isachardoublequoteopen}p\ {\isasymin}\isactrlsub c\ {\isasymOmega}{\isachardoublequoteclose}\isanewline
\ \ \isakeyword{shows}\ {\isachardoublequoteopen}OR\ {\isasymcirc}\isactrlsub c\ {\isasymlangle}p{\isacharcomma}{\kern0pt}\ NOT\ {\isasymcirc}\isactrlsub c\ p{\isasymrangle}\ {\isacharequal}{\kern0pt}\ \ {\isasymt}{\isachardoublequoteclose}\isanewline
%
\isadelimproof
\ \ %
\endisadelimproof
%
\isatagproof
\isacommand{by}\isamarkupfalse%
\ {\isacharparenleft}{\kern0pt}metis\ NOT{\isacharunderscore}{\kern0pt}false{\isacharunderscore}{\kern0pt}is{\isacharunderscore}{\kern0pt}true\ NOT{\isacharunderscore}{\kern0pt}true{\isacharunderscore}{\kern0pt}is{\isacharunderscore}{\kern0pt}false\ OR{\isacharunderscore}{\kern0pt}true{\isacharunderscore}{\kern0pt}left{\isacharunderscore}{\kern0pt}is{\isacharunderscore}{\kern0pt}true\ OR{\isacharunderscore}{\kern0pt}true{\isacharunderscore}{\kern0pt}right{\isacharunderscore}{\kern0pt}is{\isacharunderscore}{\kern0pt}true\ assms\ false{\isacharunderscore}{\kern0pt}func{\isacharunderscore}{\kern0pt}type\ true{\isacharunderscore}{\kern0pt}false{\isacharunderscore}{\kern0pt}only{\isacharunderscore}{\kern0pt}truth{\isacharunderscore}{\kern0pt}values{\isacharparenright}{\kern0pt}%
\endisatagproof
{\isafoldproof}%
%
\isadelimproof
%
\endisadelimproof
%
\isadelimdocument
%
\endisadelimdocument
%
\isatagdocument
%
\isamarkupsubsection{XOR%
}
\isamarkuptrue%
%
\endisatagdocument
{\isafolddocument}%
%
\isadelimdocument
%
\endisadelimdocument
\isacommand{definition}\isamarkupfalse%
\ XOR\ {\isacharcolon}{\kern0pt}{\isacharcolon}{\kern0pt}\ {\isachardoublequoteopen}cfunc{\isachardoublequoteclose}\ \isakeyword{where}\isanewline
\ \ {\isachardoublequoteopen}XOR\ {\isacharequal}{\kern0pt}\ {\isacharparenleft}{\kern0pt}THE\ {\isasymchi}{\isachardot}{\kern0pt}\ is{\isacharunderscore}{\kern0pt}pullback\ {\isacharparenleft}{\kern0pt}{\isasymone}{\isasymCoprod}{\isasymone}{\isacharparenright}{\kern0pt}\ {\isasymone}\ {\isacharparenleft}{\kern0pt}{\isasymOmega}{\isasymtimes}\isactrlsub c{\isasymOmega}{\isacharparenright}{\kern0pt}\ {\isasymOmega}\ {\isacharparenleft}{\kern0pt}{\isasymbeta}\isactrlbsub {\isacharparenleft}{\kern0pt}{\isasymone}{\isasymCoprod}{\isasymone}{\isacharparenright}{\kern0pt}\isactrlesub {\isacharparenright}{\kern0pt}\ {\isasymt}\ {\isacharparenleft}{\kern0pt}{\isasymlangle}{\isasymt}{\isacharcomma}{\kern0pt}\ {\isasymf}{\isasymrangle}\ {\isasymamalg}{\isasymlangle}{\isasymf}{\isacharcomma}{\kern0pt}\ {\isasymt}{\isasymrangle}{\isacharparenright}{\kern0pt}\ {\isasymchi}{\isacharparenright}{\kern0pt}{\isachardoublequoteclose}\isanewline
\isanewline
\isacommand{lemma}\isamarkupfalse%
\ pre{\isacharunderscore}{\kern0pt}XOR{\isacharunderscore}{\kern0pt}type{\isacharbrackleft}{\kern0pt}type{\isacharunderscore}{\kern0pt}rule{\isacharbrackright}{\kern0pt}{\isacharcolon}{\kern0pt}\ \isanewline
\ \ {\isachardoublequoteopen}{\isasymlangle}{\isasymt}{\isacharcomma}{\kern0pt}\ {\isasymf}{\isasymrangle}\ {\isasymamalg}\ {\isasymlangle}{\isasymf}{\isacharcomma}{\kern0pt}\ {\isasymt}{\isasymrangle}\ {\isacharcolon}{\kern0pt}\ {\isasymone}{\isasymCoprod}{\isasymone}\ {\isasymrightarrow}\ {\isasymOmega}\ {\isasymtimes}\isactrlsub c\ {\isasymOmega}{\isachardoublequoteclose}\isanewline
%
\isadelimproof
\ \ %
\endisadelimproof
%
\isatagproof
\isacommand{by}\isamarkupfalse%
\ typecheck{\isacharunderscore}{\kern0pt}cfuncs%
\endisatagproof
{\isafoldproof}%
%
\isadelimproof
\isanewline
%
\endisadelimproof
\isanewline
\isacommand{lemma}\isamarkupfalse%
\ pre{\isacharunderscore}{\kern0pt}XOR{\isacharunderscore}{\kern0pt}injective{\isacharcolon}{\kern0pt}\isanewline
\ {\isachardoublequoteopen}injective{\isacharparenleft}{\kern0pt}{\isasymlangle}{\isasymt}{\isacharcomma}{\kern0pt}\ {\isasymf}{\isasymrangle}\ {\isasymamalg}{\isasymlangle}{\isasymf}{\isacharcomma}{\kern0pt}\ {\isasymt}{\isasymrangle}{\isacharparenright}{\kern0pt}{\isachardoublequoteclose}\isanewline
%
\isadelimproof
\ %
\endisadelimproof
%
\isatagproof
\isacommand{unfolding}\isamarkupfalse%
\ injective{\isacharunderscore}{\kern0pt}def\isanewline
\isacommand{proof}\isamarkupfalse%
{\isacharparenleft}{\kern0pt}clarify{\isacharparenright}{\kern0pt}\isanewline
\ \ \isacommand{fix}\isamarkupfalse%
\ x\ y\ \isanewline
\ \ \isacommand{assume}\isamarkupfalse%
\ {\isachardoublequoteopen}x\ {\isasymin}\isactrlsub c\ domain\ {\isacharparenleft}{\kern0pt}{\isasymlangle}{\isasymt}{\isacharcomma}{\kern0pt}{\isasymf}{\isasymrangle}\ {\isasymamalg}\ {\isasymlangle}{\isasymf}{\isacharcomma}{\kern0pt}{\isasymt}{\isasymrangle}{\isacharparenright}{\kern0pt}{\isachardoublequoteclose}\ \isanewline
\ \ \isacommand{then}\isamarkupfalse%
\ \isacommand{have}\isamarkupfalse%
\ x{\isacharunderscore}{\kern0pt}type{\isacharcolon}{\kern0pt}\ {\isachardoublequoteopen}x\ {\isasymin}\isactrlsub c\ {\isasymone}{\isasymCoprod}{\isasymone}{\isachardoublequoteclose}\ \ \isanewline
\ \ \ \ \isacommand{using}\isamarkupfalse%
\ cfunc{\isacharunderscore}{\kern0pt}type{\isacharunderscore}{\kern0pt}def\ pre{\isacharunderscore}{\kern0pt}XOR{\isacharunderscore}{\kern0pt}type\ \isacommand{by}\isamarkupfalse%
\ force\isanewline
\ \ \isacommand{then}\isamarkupfalse%
\ \isacommand{have}\isamarkupfalse%
\ x{\isacharunderscore}{\kern0pt}form{\isacharcolon}{\kern0pt}\ {\isachardoublequoteopen}{\isacharparenleft}{\kern0pt}{\isasymexists}\ w{\isachardot}{\kern0pt}\ w\ {\isasymin}\isactrlsub c\ {\isasymone}\ {\isasymand}\ x\ {\isacharequal}{\kern0pt}\ left{\isacharunderscore}{\kern0pt}coproj\ {\isasymone}\ {\isasymone}\ {\isasymcirc}\isactrlsub c\ w{\isacharparenright}{\kern0pt}\isanewline
\ \ \ \ \ \ \ \ \ \ \ \ \ \ \ \ \ \ {\isasymor}\ \ {\isacharparenleft}{\kern0pt}{\isasymexists}\ w{\isachardot}{\kern0pt}\ w\ {\isasymin}\isactrlsub c\ {\isasymone}\ {\isasymand}\ x\ {\isacharequal}{\kern0pt}\ right{\isacharunderscore}{\kern0pt}coproj\ {\isasymone}\ {\isasymone}\ {\isasymcirc}\isactrlsub c\ w{\isacharparenright}{\kern0pt}{\isachardoublequoteclose}\isanewline
\ \ \ \ \isacommand{using}\isamarkupfalse%
\ coprojs{\isacharunderscore}{\kern0pt}jointly{\isacharunderscore}{\kern0pt}surj\ \isacommand{by}\isamarkupfalse%
\ auto\isanewline
\isanewline
\ \ \isacommand{assume}\isamarkupfalse%
\ {\isachardoublequoteopen}y\ {\isasymin}\isactrlsub c\ domain\ {\isacharparenleft}{\kern0pt}{\isasymlangle}{\isasymt}{\isacharcomma}{\kern0pt}{\isasymf}{\isasymrangle}\ {\isasymamalg}\ {\isasymlangle}{\isasymf}{\isacharcomma}{\kern0pt}{\isasymt}{\isasymrangle}{\isacharparenright}{\kern0pt}{\isachardoublequoteclose}\ \isanewline
\ \ \isacommand{then}\isamarkupfalse%
\ \isacommand{have}\isamarkupfalse%
\ y{\isacharunderscore}{\kern0pt}type{\isacharcolon}{\kern0pt}\ {\isachardoublequoteopen}y\ {\isasymin}\isactrlsub c\ {\isasymone}{\isasymCoprod}{\isasymone}{\isachardoublequoteclose}\ \ \isanewline
\ \ \ \ \isacommand{using}\isamarkupfalse%
\ cfunc{\isacharunderscore}{\kern0pt}type{\isacharunderscore}{\kern0pt}def\ pre{\isacharunderscore}{\kern0pt}XOR{\isacharunderscore}{\kern0pt}type\ \isacommand{by}\isamarkupfalse%
\ force\isanewline
\ \ \isacommand{then}\isamarkupfalse%
\ \isacommand{have}\isamarkupfalse%
\ y{\isacharunderscore}{\kern0pt}form{\isacharcolon}{\kern0pt}\ {\isachardoublequoteopen}{\isacharparenleft}{\kern0pt}{\isasymexists}\ w{\isachardot}{\kern0pt}\ w\ {\isasymin}\isactrlsub c\ {\isasymone}\ {\isasymand}\ y\ {\isacharequal}{\kern0pt}\ left{\isacharunderscore}{\kern0pt}coproj\ {\isasymone}\ {\isasymone}\ {\isasymcirc}\isactrlsub c\ w{\isacharparenright}{\kern0pt}\isanewline
\ \ \ \ \ \ \ \ \ \ \ \ \ \ \ \ \ {\isasymor}\ \ {\isacharparenleft}{\kern0pt}{\isasymexists}\ w{\isachardot}{\kern0pt}\ w\ {\isasymin}\isactrlsub c\ {\isasymone}\ {\isasymand}\ y\ {\isacharequal}{\kern0pt}\ right{\isacharunderscore}{\kern0pt}coproj\ {\isasymone}\ {\isasymone}\ {\isasymcirc}\isactrlsub c\ w{\isacharparenright}{\kern0pt}{\isachardoublequoteclose}\isanewline
\ \ \ \ \isacommand{using}\isamarkupfalse%
\ coprojs{\isacharunderscore}{\kern0pt}jointly{\isacharunderscore}{\kern0pt}surj\ \isacommand{by}\isamarkupfalse%
\ auto\isanewline
\isanewline
\ \ \isacommand{assume}\isamarkupfalse%
\ eqs{\isacharcolon}{\kern0pt}\ {\isachardoublequoteopen}{\isasymlangle}{\isasymt}{\isacharcomma}{\kern0pt}{\isasymf}{\isasymrangle}\ {\isasymamalg}\ {\isasymlangle}{\isasymf}{\isacharcomma}{\kern0pt}{\isasymt}{\isasymrangle}\ {\isasymcirc}\isactrlsub c\ x\ {\isacharequal}{\kern0pt}\ {\isasymlangle}{\isasymt}{\isacharcomma}{\kern0pt}{\isasymf}{\isasymrangle}\ {\isasymamalg}\ {\isasymlangle}{\isasymf}{\isacharcomma}{\kern0pt}{\isasymt}{\isasymrangle}\ {\isasymcirc}\isactrlsub c\ y{\isachardoublequoteclose}\isanewline
\isanewline
\ \ \isacommand{show}\isamarkupfalse%
\ {\isachardoublequoteopen}x\ {\isacharequal}{\kern0pt}\ y{\isachardoublequoteclose}\isanewline
\ \ \isacommand{proof}\isamarkupfalse%
{\isacharparenleft}{\kern0pt}cases\ {\isachardoublequoteopen}{\isasymexists}\ w{\isachardot}{\kern0pt}\ w\ {\isasymin}\isactrlsub c\ {\isasymone}\ {\isasymand}\ x\ {\isacharequal}{\kern0pt}\ left{\isacharunderscore}{\kern0pt}coproj\ {\isasymone}\ {\isasymone}\ {\isasymcirc}\isactrlsub c\ w{\isachardoublequoteclose}{\isacharparenright}{\kern0pt}\isanewline
\ \ \ \ \isacommand{assume}\isamarkupfalse%
\ a{\isadigit{1}}{\isacharcolon}{\kern0pt}\ {\isachardoublequoteopen}{\isasymexists}\ w{\isachardot}{\kern0pt}\ w\ {\isasymin}\isactrlsub c\ {\isasymone}\ {\isasymand}\ x\ {\isacharequal}{\kern0pt}\ left{\isacharunderscore}{\kern0pt}coproj\ {\isasymone}\ {\isasymone}\ {\isasymcirc}\isactrlsub c\ w{\isachardoublequoteclose}\isanewline
\ \ \ \ \isacommand{then}\isamarkupfalse%
\ \isacommand{obtain}\isamarkupfalse%
\ w\ \isakeyword{where}\ x{\isacharunderscore}{\kern0pt}def{\isacharcolon}{\kern0pt}\ {\isachardoublequoteopen}w\ {\isasymin}\isactrlsub c\ {\isasymone}\ {\isasymand}\ x\ {\isacharequal}{\kern0pt}\ left{\isacharunderscore}{\kern0pt}coproj\ {\isasymone}\ {\isasymone}\ {\isasymcirc}\isactrlsub c\ w{\isachardoublequoteclose}\isanewline
\ \ \ \ \ \ \isacommand{by}\isamarkupfalse%
\ blast\isanewline
\ \ \ \ \isacommand{then}\isamarkupfalse%
\ \isacommand{have}\isamarkupfalse%
\ w{\isacharunderscore}{\kern0pt}is{\isacharcolon}{\kern0pt}\ {\isachardoublequoteopen}w\ {\isacharequal}{\kern0pt}\ id{\isacharparenleft}{\kern0pt}{\isasymone}{\isacharparenright}{\kern0pt}{\isachardoublequoteclose}\isanewline
\ \ \ \ \ \ \isacommand{by}\isamarkupfalse%
\ {\isacharparenleft}{\kern0pt}typecheck{\isacharunderscore}{\kern0pt}cfuncs{\isacharcomma}{\kern0pt}\ metis\ terminal{\isacharunderscore}{\kern0pt}func{\isacharunderscore}{\kern0pt}unique\ x{\isacharunderscore}{\kern0pt}def{\isacharparenright}{\kern0pt}\isanewline
\ \ \ \ \isacommand{have}\isamarkupfalse%
\ {\isachardoublequoteopen}{\isasymexists}\ v{\isachardot}{\kern0pt}\ v\ {\isasymin}\isactrlsub c\ {\isasymone}\ {\isasymand}\ y\ {\isacharequal}{\kern0pt}\ left{\isacharunderscore}{\kern0pt}coproj\ {\isasymone}\ {\isasymone}\ {\isasymcirc}\isactrlsub c\ v{\isachardoublequoteclose}\isanewline
\ \ \ \ \isacommand{proof}\isamarkupfalse%
{\isacharparenleft}{\kern0pt}rule\ ccontr{\isacharparenright}{\kern0pt}\isanewline
\ \ \ \ \ \ \isacommand{assume}\isamarkupfalse%
\ a{\isadigit{2}}{\isacharcolon}{\kern0pt}\ {\isachardoublequoteopen}{\isasymnexists}v{\isachardot}{\kern0pt}\ v\ {\isasymin}\isactrlsub c\ {\isasymone}\ {\isasymand}\ y\ {\isacharequal}{\kern0pt}\ left{\isacharunderscore}{\kern0pt}coproj\ {\isasymone}\ {\isasymone}\ {\isasymcirc}\isactrlsub c\ v{\isachardoublequoteclose}\isanewline
\ \ \ \ \ \ \isacommand{then}\isamarkupfalse%
\ \isacommand{obtain}\isamarkupfalse%
\ v\ \isakeyword{where}\ y{\isacharunderscore}{\kern0pt}def{\isacharcolon}{\kern0pt}\ \ {\isachardoublequoteopen}v\ {\isasymin}\isactrlsub c\ {\isasymone}\ {\isasymand}\ y\ {\isacharequal}{\kern0pt}\ right{\isacharunderscore}{\kern0pt}coproj\ {\isasymone}\ {\isasymone}\ {\isasymcirc}\isactrlsub c\ v{\isachardoublequoteclose}\isanewline
\ \ \ \ \ \ \ \ \isacommand{using}\isamarkupfalse%
\ y{\isacharunderscore}{\kern0pt}form\ \isacommand{by}\isamarkupfalse%
\ {\isacharparenleft}{\kern0pt}typecheck{\isacharunderscore}{\kern0pt}cfuncs{\isacharcomma}{\kern0pt}\ blast{\isacharparenright}{\kern0pt}\isanewline
\ \ \ \ \ \ \isacommand{then}\isamarkupfalse%
\ \isacommand{have}\isamarkupfalse%
\ v{\isacharunderscore}{\kern0pt}is{\isacharcolon}{\kern0pt}\ {\isachardoublequoteopen}v\ {\isacharequal}{\kern0pt}\ id{\isacharparenleft}{\kern0pt}{\isasymone}{\isacharparenright}{\kern0pt}{\isachardoublequoteclose}\isanewline
\ \ \ \ \ \ \ \ \isacommand{by}\isamarkupfalse%
\ {\isacharparenleft}{\kern0pt}typecheck{\isacharunderscore}{\kern0pt}cfuncs{\isacharcomma}{\kern0pt}\ metis\ terminal{\isacharunderscore}{\kern0pt}func{\isacharunderscore}{\kern0pt}unique\ y{\isacharunderscore}{\kern0pt}def{\isacharparenright}{\kern0pt}\isanewline
\ \ \ \ \ \ \isacommand{then}\isamarkupfalse%
\ \isacommand{have}\isamarkupfalse%
\ {\isachardoublequoteopen}{\isasymlangle}{\isasymt}{\isacharcomma}{\kern0pt}{\isasymf}{\isasymrangle}\ {\isasymamalg}\ {\isasymlangle}{\isasymf}{\isacharcomma}{\kern0pt}{\isasymt}{\isasymrangle}\ {\isasymcirc}\isactrlsub c\ left{\isacharunderscore}{\kern0pt}coproj\ {\isasymone}\ {\isasymone}\ {\isacharequal}{\kern0pt}\ {\isasymlangle}{\isasymt}{\isacharcomma}{\kern0pt}{\isasymf}{\isasymrangle}\ {\isasymamalg}\ {\isasymlangle}{\isasymf}{\isacharcomma}{\kern0pt}{\isasymt}{\isasymrangle}\ {\isasymcirc}\isactrlsub c\ right{\isacharunderscore}{\kern0pt}coproj\ {\isasymone}\ {\isasymone}{\isachardoublequoteclose}\isanewline
\ \ \ \ \ \ \ \ \isacommand{using}\isamarkupfalse%
\ w{\isacharunderscore}{\kern0pt}is\ eqs\ id{\isacharunderscore}{\kern0pt}right{\isacharunderscore}{\kern0pt}unit{\isadigit{2}}\ x{\isacharunderscore}{\kern0pt}def\ y{\isacharunderscore}{\kern0pt}def\ \isacommand{by}\isamarkupfalse%
\ {\isacharparenleft}{\kern0pt}typecheck{\isacharunderscore}{\kern0pt}cfuncs{\isacharcomma}{\kern0pt}\ force{\isacharparenright}{\kern0pt}\isanewline
\ \ \ \ \ \ \isacommand{then}\isamarkupfalse%
\ \isacommand{have}\isamarkupfalse%
\ {\isachardoublequoteopen}{\isasymlangle}{\isasymt}{\isacharcomma}{\kern0pt}{\isasymf}{\isasymrangle}\ {\isacharequal}{\kern0pt}\ {\isasymlangle}{\isasymf}{\isacharcomma}{\kern0pt}{\isasymt}{\isasymrangle}{\isachardoublequoteclose}\isanewline
\ \ \ \ \ \ \ \ \isacommand{by}\isamarkupfalse%
\ {\isacharparenleft}{\kern0pt}typecheck{\isacharunderscore}{\kern0pt}cfuncs{\isacharcomma}{\kern0pt}\ smt\ {\isacharparenleft}{\kern0pt}z{\isadigit{3}}{\isacharparenright}{\kern0pt}\ cfunc{\isacharunderscore}{\kern0pt}coprod{\isacharunderscore}{\kern0pt}unique\ coprod{\isacharunderscore}{\kern0pt}eq{\isadigit{2}}\ pre{\isacharunderscore}{\kern0pt}XOR{\isacharunderscore}{\kern0pt}type\ right{\isacharunderscore}{\kern0pt}coproj{\isacharunderscore}{\kern0pt}cfunc{\isacharunderscore}{\kern0pt}coprod{\isacharparenright}{\kern0pt}\ \ \ \ \ \ \isanewline
\ \ \ \ \ \ \isacommand{then}\isamarkupfalse%
\ \isacommand{have}\isamarkupfalse%
\ {\isachardoublequoteopen}{\isasymt}\ {\isacharequal}{\kern0pt}\ {\isasymf}\ {\isasymand}\ {\isasymf}\ {\isacharequal}{\kern0pt}\ {\isasymt}{\isachardoublequoteclose}\isanewline
\ \ \ \ \ \ \ \ \isacommand{using}\isamarkupfalse%
\ cart{\isacharunderscore}{\kern0pt}prod{\isacharunderscore}{\kern0pt}eq{\isadigit{2}}\ false{\isacharunderscore}{\kern0pt}func{\isacharunderscore}{\kern0pt}type\ true{\isacharunderscore}{\kern0pt}func{\isacharunderscore}{\kern0pt}type\ \isacommand{by}\isamarkupfalse%
\ blast\isanewline
\ \ \ \ \ \ \isacommand{then}\isamarkupfalse%
\ \isacommand{show}\isamarkupfalse%
\ False\isanewline
\ \ \ \ \ \ \ \ \isacommand{using}\isamarkupfalse%
\ true{\isacharunderscore}{\kern0pt}false{\isacharunderscore}{\kern0pt}distinct\ \isacommand{by}\isamarkupfalse%
\ blast\isanewline
\ \ \ \ \isacommand{qed}\isamarkupfalse%
\isanewline
\ \ \ \ \isacommand{then}\isamarkupfalse%
\ \isacommand{obtain}\isamarkupfalse%
\ v\ \isakeyword{where}\ y{\isacharunderscore}{\kern0pt}def{\isacharcolon}{\kern0pt}\ {\isachardoublequoteopen}v\ {\isasymin}\isactrlsub c\ {\isasymone}\ {\isasymand}\ y\ {\isacharequal}{\kern0pt}\ left{\isacharunderscore}{\kern0pt}coproj\ {\isasymone}\ {\isasymone}\ {\isasymcirc}\isactrlsub c\ v{\isachardoublequoteclose}\isanewline
\ \ \ \ \ \ \isacommand{by}\isamarkupfalse%
\ blast\isanewline
\ \ \ \ \isacommand{then}\isamarkupfalse%
\ \isacommand{have}\isamarkupfalse%
\ {\isachardoublequoteopen}v\ {\isacharequal}{\kern0pt}\ id{\isacharparenleft}{\kern0pt}{\isasymone}{\isacharparenright}{\kern0pt}{\isachardoublequoteclose}\isanewline
\ \ \ \ \ \ \isacommand{by}\isamarkupfalse%
\ {\isacharparenleft}{\kern0pt}typecheck{\isacharunderscore}{\kern0pt}cfuncs{\isacharcomma}{\kern0pt}\ metis\ terminal{\isacharunderscore}{\kern0pt}func{\isacharunderscore}{\kern0pt}unique{\isacharparenright}{\kern0pt}\isanewline
\ \ \ \ \isacommand{then}\isamarkupfalse%
\ \isacommand{show}\isamarkupfalse%
\ {\isacharquery}{\kern0pt}thesis\isanewline
\ \ \ \ \ \ \isacommand{by}\isamarkupfalse%
\ {\isacharparenleft}{\kern0pt}simp\ add{\isacharcolon}{\kern0pt}\ w{\isacharunderscore}{\kern0pt}is\ x{\isacharunderscore}{\kern0pt}def\ y{\isacharunderscore}{\kern0pt}def{\isacharparenright}{\kern0pt}\isanewline
\ \ \isacommand{next}\isamarkupfalse%
\isanewline
\ \ \ \ \isacommand{assume}\isamarkupfalse%
\ {\isachardoublequoteopen}{\isasymnexists}w{\isachardot}{\kern0pt}\ w\ {\isasymin}\isactrlsub c\ {\isasymone}\ {\isasymand}\ x\ {\isacharequal}{\kern0pt}\ left{\isacharunderscore}{\kern0pt}coproj\ {\isasymone}\ {\isasymone}\ {\isasymcirc}\isactrlsub c\ w{\isachardoublequoteclose}\isanewline
\ \ \ \ \isacommand{then}\isamarkupfalse%
\ \isacommand{obtain}\isamarkupfalse%
\ w\ \isakeyword{where}\ x{\isacharunderscore}{\kern0pt}def{\isacharcolon}{\kern0pt}\ {\isachardoublequoteopen}w\ {\isasymin}\isactrlsub c\ {\isasymone}\ {\isasymand}\ x\ {\isacharequal}{\kern0pt}\ right{\isacharunderscore}{\kern0pt}coproj\ {\isasymone}\ {\isasymone}\ {\isasymcirc}\isactrlsub c\ w{\isachardoublequoteclose}\isanewline
\ \ \ \ \ \ \isacommand{using}\isamarkupfalse%
\ x{\isacharunderscore}{\kern0pt}form\ \isacommand{by}\isamarkupfalse%
\ force\isanewline
\ \ \ \ \isacommand{then}\isamarkupfalse%
\ \isacommand{have}\isamarkupfalse%
\ w{\isacharunderscore}{\kern0pt}is{\isacharcolon}{\kern0pt}\ {\isachardoublequoteopen}w\ {\isacharequal}{\kern0pt}\ id\ {\isasymone}{\isachardoublequoteclose}\isanewline
\ \ \ \ \ \ \isacommand{by}\isamarkupfalse%
\ {\isacharparenleft}{\kern0pt}typecheck{\isacharunderscore}{\kern0pt}cfuncs{\isacharcomma}{\kern0pt}\ metis\ terminal{\isacharunderscore}{\kern0pt}func{\isacharunderscore}{\kern0pt}unique\ x{\isacharunderscore}{\kern0pt}def{\isacharparenright}{\kern0pt}\isanewline
\ \ \ \ \isacommand{have}\isamarkupfalse%
\ {\isachardoublequoteopen}{\isasymexists}\ v{\isachardot}{\kern0pt}\ v\ {\isasymin}\isactrlsub c\ {\isasymone}\ {\isasymand}\ y\ {\isacharequal}{\kern0pt}\ right{\isacharunderscore}{\kern0pt}coproj\ {\isasymone}\ {\isasymone}\ {\isasymcirc}\isactrlsub c\ v{\isachardoublequoteclose}\isanewline
\ \ \ \ \isacommand{proof}\isamarkupfalse%
{\isacharparenleft}{\kern0pt}rule\ ccontr{\isacharparenright}{\kern0pt}\isanewline
\ \ \ \ \ \ \isacommand{assume}\isamarkupfalse%
\ a{\isadigit{2}}{\isacharcolon}{\kern0pt}\ {\isachardoublequoteopen}{\isasymnexists}v{\isachardot}{\kern0pt}\ v\ {\isasymin}\isactrlsub c\ {\isasymone}\ {\isasymand}\ y\ {\isacharequal}{\kern0pt}\ right{\isacharunderscore}{\kern0pt}coproj\ {\isasymone}\ {\isasymone}\ {\isasymcirc}\isactrlsub c\ v{\isachardoublequoteclose}\isanewline
\ \ \ \ \ \ \isacommand{then}\isamarkupfalse%
\ \isacommand{obtain}\isamarkupfalse%
\ v\ \isakeyword{where}\ y{\isacharunderscore}{\kern0pt}def{\isacharcolon}{\kern0pt}\ \ {\isachardoublequoteopen}v\ {\isasymin}\isactrlsub c\ {\isasymone}\ {\isasymand}\ y\ {\isacharequal}{\kern0pt}\ left{\isacharunderscore}{\kern0pt}coproj\ {\isasymone}\ {\isasymone}\ {\isasymcirc}\isactrlsub c\ v{\isachardoublequoteclose}\isanewline
\ \ \ \ \ \ \ \ \isacommand{using}\isamarkupfalse%
\ y{\isacharunderscore}{\kern0pt}form\ \isacommand{by}\isamarkupfalse%
\ {\isacharparenleft}{\kern0pt}typecheck{\isacharunderscore}{\kern0pt}cfuncs{\isacharcomma}{\kern0pt}\ blast{\isacharparenright}{\kern0pt}\isanewline
\ \ \ \ \ \ \isacommand{then}\isamarkupfalse%
\ \isacommand{have}\isamarkupfalse%
\ {\isachardoublequoteopen}v\ {\isacharequal}{\kern0pt}\ id\ {\isasymone}{\isachardoublequoteclose}\isanewline
\ \ \ \ \ \ \ \ \isacommand{by}\isamarkupfalse%
\ {\isacharparenleft}{\kern0pt}typecheck{\isacharunderscore}{\kern0pt}cfuncs{\isacharcomma}{\kern0pt}\ metis\ terminal{\isacharunderscore}{\kern0pt}func{\isacharunderscore}{\kern0pt}unique\ y{\isacharunderscore}{\kern0pt}def{\isacharparenright}{\kern0pt}\isanewline
\ \ \ \ \ \ \isacommand{then}\isamarkupfalse%
\ \isacommand{have}\isamarkupfalse%
\ {\isachardoublequoteopen}{\isasymlangle}{\isasymt}{\isacharcomma}{\kern0pt}{\isasymf}{\isasymrangle}\ {\isasymamalg}\ {\isasymlangle}{\isasymf}{\isacharcomma}{\kern0pt}{\isasymt}{\isasymrangle}\ {\isasymcirc}\isactrlsub c\ left{\isacharunderscore}{\kern0pt}coproj\ {\isasymone}\ {\isasymone}\ {\isacharequal}{\kern0pt}\ {\isasymlangle}{\isasymt}{\isacharcomma}{\kern0pt}{\isasymf}{\isasymrangle}\ {\isasymamalg}\ {\isasymlangle}{\isasymf}{\isacharcomma}{\kern0pt}{\isasymt}{\isasymrangle}\ {\isasymcirc}\isactrlsub c\ right{\isacharunderscore}{\kern0pt}coproj\ {\isasymone}\ {\isasymone}{\isachardoublequoteclose}\isanewline
\ \ \ \ \ \ \ \ \isacommand{using}\isamarkupfalse%
\ w{\isacharunderscore}{\kern0pt}is\ eqs\ id{\isacharunderscore}{\kern0pt}right{\isacharunderscore}{\kern0pt}unit{\isadigit{2}}\ x{\isacharunderscore}{\kern0pt}def\ y{\isacharunderscore}{\kern0pt}def\ \isacommand{by}\isamarkupfalse%
\ {\isacharparenleft}{\kern0pt}typecheck{\isacharunderscore}{\kern0pt}cfuncs{\isacharcomma}{\kern0pt}\ force{\isacharparenright}{\kern0pt}\isanewline
\ \ \ \ \ \ \isacommand{then}\isamarkupfalse%
\ \isacommand{have}\isamarkupfalse%
\ {\isachardoublequoteopen}{\isasymlangle}{\isasymt}{\isacharcomma}{\kern0pt}{\isasymf}{\isasymrangle}\ {\isacharequal}{\kern0pt}\ {\isasymlangle}{\isasymf}{\isacharcomma}{\kern0pt}{\isasymt}{\isasymrangle}{\isachardoublequoteclose}\isanewline
\ \ \ \ \ \ \ \ \isacommand{by}\isamarkupfalse%
\ {\isacharparenleft}{\kern0pt}typecheck{\isacharunderscore}{\kern0pt}cfuncs{\isacharcomma}{\kern0pt}\ smt\ {\isacharparenleft}{\kern0pt}z{\isadigit{3}}{\isacharparenright}{\kern0pt}\ \ cfunc{\isacharunderscore}{\kern0pt}coprod{\isacharunderscore}{\kern0pt}unique\ coprod{\isacharunderscore}{\kern0pt}eq{\isadigit{2}}\ pre{\isacharunderscore}{\kern0pt}XOR{\isacharunderscore}{\kern0pt}type\ right{\isacharunderscore}{\kern0pt}coproj{\isacharunderscore}{\kern0pt}cfunc{\isacharunderscore}{\kern0pt}coprod{\isacharparenright}{\kern0pt}\ \ \ \ \ \ \isanewline
\ \ \ \ \ \ \isacommand{then}\isamarkupfalse%
\ \isacommand{have}\isamarkupfalse%
\ {\isachardoublequoteopen}{\isasymt}\ {\isacharequal}{\kern0pt}\ {\isasymf}\ {\isasymand}\ {\isasymf}\ {\isacharequal}{\kern0pt}\ {\isasymt}{\isachardoublequoteclose}\isanewline
\ \ \ \ \ \ \ \ \isacommand{using}\isamarkupfalse%
\ cart{\isacharunderscore}{\kern0pt}prod{\isacharunderscore}{\kern0pt}eq{\isadigit{2}}\ false{\isacharunderscore}{\kern0pt}func{\isacharunderscore}{\kern0pt}type\ true{\isacharunderscore}{\kern0pt}func{\isacharunderscore}{\kern0pt}type\ \isacommand{by}\isamarkupfalse%
\ blast\isanewline
\ \ \ \ \ \ \isacommand{then}\isamarkupfalse%
\ \isacommand{show}\isamarkupfalse%
\ False\isanewline
\ \ \ \ \ \ \ \ \isacommand{using}\isamarkupfalse%
\ true{\isacharunderscore}{\kern0pt}false{\isacharunderscore}{\kern0pt}distinct\ \isacommand{by}\isamarkupfalse%
\ blast\isanewline
\ \ \ \ \isacommand{qed}\isamarkupfalse%
\isanewline
\ \ \ \ \isacommand{then}\isamarkupfalse%
\ \isacommand{obtain}\isamarkupfalse%
\ v\ \isakeyword{where}\ y{\isacharunderscore}{\kern0pt}def{\isacharcolon}{\kern0pt}\ {\isachardoublequoteopen}v\ {\isasymin}\isactrlsub c\ {\isasymone}\ {\isasymand}\ y\ {\isacharequal}{\kern0pt}\ right{\isacharunderscore}{\kern0pt}coproj\ {\isasymone}\ {\isasymone}\ {\isasymcirc}\isactrlsub c\ v{\isachardoublequoteclose}\isanewline
\ \ \ \ \ \ \isacommand{by}\isamarkupfalse%
\ blast\isanewline
\ \ \ \ \isacommand{then}\isamarkupfalse%
\ \isacommand{have}\isamarkupfalse%
\ {\isachardoublequoteopen}v\ {\isacharequal}{\kern0pt}\ id\ {\isasymone}{\isachardoublequoteclose}\isanewline
\ \ \ \ \ \ \isacommand{by}\isamarkupfalse%
\ {\isacharparenleft}{\kern0pt}typecheck{\isacharunderscore}{\kern0pt}cfuncs{\isacharcomma}{\kern0pt}\ metis\ terminal{\isacharunderscore}{\kern0pt}func{\isacharunderscore}{\kern0pt}unique{\isacharparenright}{\kern0pt}\isanewline
\ \ \ \ \isacommand{then}\isamarkupfalse%
\ \isacommand{show}\isamarkupfalse%
\ {\isacharquery}{\kern0pt}thesis\isanewline
\ \ \ \ \ \ \isacommand{by}\isamarkupfalse%
\ {\isacharparenleft}{\kern0pt}simp\ add{\isacharcolon}{\kern0pt}\ w{\isacharunderscore}{\kern0pt}is\ x{\isacharunderscore}{\kern0pt}def\ y{\isacharunderscore}{\kern0pt}def{\isacharparenright}{\kern0pt}\isanewline
\ \ \isacommand{qed}\isamarkupfalse%
\isanewline
\isacommand{qed}\isamarkupfalse%
%
\endisatagproof
{\isafoldproof}%
%
\isadelimproof
\isanewline
%
\endisadelimproof
\isanewline
\isacommand{lemma}\isamarkupfalse%
\ XOR{\isacharunderscore}{\kern0pt}is{\isacharunderscore}{\kern0pt}pullback{\isacharcolon}{\kern0pt}\isanewline
\ \ {\isachardoublequoteopen}is{\isacharunderscore}{\kern0pt}pullback\ {\isacharparenleft}{\kern0pt}{\isasymone}{\isasymCoprod}{\isasymone}{\isacharparenright}{\kern0pt}\ {\isasymone}\ {\isacharparenleft}{\kern0pt}{\isasymOmega}\ {\isasymtimes}\isactrlsub c\ {\isasymOmega}{\isacharparenright}{\kern0pt}\ {\isasymOmega}\ {\isacharparenleft}{\kern0pt}{\isasymbeta}\isactrlbsub {\isacharparenleft}{\kern0pt}{\isasymone}{\isasymCoprod}{\isasymone}{\isacharparenright}{\kern0pt}\isactrlesub {\isacharparenright}{\kern0pt}\ {\isasymt}\ {\isacharparenleft}{\kern0pt}{\isasymlangle}{\isasymt}{\isacharcomma}{\kern0pt}\ {\isasymf}{\isasymrangle}\ {\isasymamalg}\ {\isasymlangle}{\isasymf}{\isacharcomma}{\kern0pt}\ {\isasymt}{\isasymrangle}{\isacharparenright}{\kern0pt}\ XOR{\isachardoublequoteclose}\isanewline
%
\isadelimproof
\ \ %
\endisadelimproof
%
\isatagproof
\isacommand{unfolding}\isamarkupfalse%
\ XOR{\isacharunderscore}{\kern0pt}def\isanewline
\ \ \isacommand{using}\isamarkupfalse%
\ element{\isacharunderscore}{\kern0pt}monomorphism\ characteristic{\isacharunderscore}{\kern0pt}function{\isacharunderscore}{\kern0pt}exists\isanewline
\ \ \isacommand{by}\isamarkupfalse%
\ {\isacharparenleft}{\kern0pt}typecheck{\isacharunderscore}{\kern0pt}cfuncs{\isacharcomma}{\kern0pt}\ simp\ add{\isacharcolon}{\kern0pt}\ the{\isadigit{1}}I{\isadigit{2}}\ injective{\isacharunderscore}{\kern0pt}imp{\isacharunderscore}{\kern0pt}monomorphism\ pre{\isacharunderscore}{\kern0pt}XOR{\isacharunderscore}{\kern0pt}injective{\isacharparenright}{\kern0pt}%
\endisatagproof
{\isafoldproof}%
%
\isadelimproof
\isanewline
%
\endisadelimproof
\ \ \ \ \ \ \isanewline
\isacommand{lemma}\isamarkupfalse%
\ XOR{\isacharunderscore}{\kern0pt}type{\isacharbrackleft}{\kern0pt}type{\isacharunderscore}{\kern0pt}rule{\isacharbrackright}{\kern0pt}{\isacharcolon}{\kern0pt}\isanewline
\ \ {\isachardoublequoteopen}XOR\ {\isacharcolon}{\kern0pt}\ {\isasymOmega}\ {\isasymtimes}\isactrlsub c\ {\isasymOmega}\ {\isasymrightarrow}\ {\isasymOmega}{\isachardoublequoteclose}\isanewline
%
\isadelimproof
\ \ %
\endisadelimproof
%
\isatagproof
\isacommand{unfolding}\isamarkupfalse%
\ XOR{\isacharunderscore}{\kern0pt}def\isanewline
\ \ \isacommand{by}\isamarkupfalse%
\ {\isacharparenleft}{\kern0pt}metis\ XOR{\isacharunderscore}{\kern0pt}def\ XOR{\isacharunderscore}{\kern0pt}is{\isacharunderscore}{\kern0pt}pullback\ is{\isacharunderscore}{\kern0pt}pullback{\isacharunderscore}{\kern0pt}def{\isacharparenright}{\kern0pt}%
\endisatagproof
{\isafoldproof}%
%
\isadelimproof
\isanewline
%
\endisadelimproof
\isanewline
\isacommand{lemma}\isamarkupfalse%
\ XOR{\isacharunderscore}{\kern0pt}only{\isacharunderscore}{\kern0pt}true{\isacharunderscore}{\kern0pt}left{\isacharunderscore}{\kern0pt}is{\isacharunderscore}{\kern0pt}true{\isacharcolon}{\kern0pt}\isanewline
\ \ {\isachardoublequoteopen}XOR\ {\isasymcirc}\isactrlsub c\ \ {\isasymlangle}{\isasymt}{\isacharcomma}{\kern0pt}{\isasymf}{\isasymrangle}\ {\isacharequal}{\kern0pt}\ {\isasymt}{\isachardoublequoteclose}\isanewline
%
\isadelimproof
%
\endisadelimproof
%
\isatagproof
\isacommand{proof}\isamarkupfalse%
\ {\isacharminus}{\kern0pt}\ \ \ \isanewline
\ \ \isacommand{have}\isamarkupfalse%
\ {\isachardoublequoteopen}{\isasymexists}\ j{\isachardot}{\kern0pt}\ j\ {\isasymin}\isactrlsub c\ {\isasymone}{\isasymCoprod}{\isasymone}\ {\isasymand}\ {\isacharparenleft}{\kern0pt}{\isasymlangle}{\isasymt}{\isacharcomma}{\kern0pt}\ {\isasymf}{\isasymrangle}\ {\isasymamalg}{\isasymlangle}{\isasymf}{\isacharcomma}{\kern0pt}\ {\isasymt}{\isasymrangle}{\isacharparenright}{\kern0pt}\ {\isasymcirc}\isactrlsub c\ j\ \ {\isacharequal}{\kern0pt}\ {\isasymlangle}{\isasymt}{\isacharcomma}{\kern0pt}{\isasymf}{\isasymrangle}{\isachardoublequoteclose}\isanewline
\ \ \ \ \isacommand{by}\isamarkupfalse%
\ {\isacharparenleft}{\kern0pt}typecheck{\isacharunderscore}{\kern0pt}cfuncs{\isacharcomma}{\kern0pt}\ meson\ left{\isacharunderscore}{\kern0pt}coproj{\isacharunderscore}{\kern0pt}cfunc{\isacharunderscore}{\kern0pt}coprod\ left{\isacharunderscore}{\kern0pt}proj{\isacharunderscore}{\kern0pt}type{\isacharparenright}{\kern0pt}\isanewline
\ \ \isacommand{then}\isamarkupfalse%
\ \isacommand{show}\isamarkupfalse%
\ {\isacharquery}{\kern0pt}thesis\isanewline
\ \ \ \ \isacommand{by}\isamarkupfalse%
\ {\isacharparenleft}{\kern0pt}smt\ {\isacharparenleft}{\kern0pt}verit{\isacharcomma}{\kern0pt}\ best{\isacharparenright}{\kern0pt}\ XOR{\isacharunderscore}{\kern0pt}is{\isacharunderscore}{\kern0pt}pullback\ comp{\isacharunderscore}{\kern0pt}associative{\isadigit{2}}\ id{\isacharunderscore}{\kern0pt}right{\isacharunderscore}{\kern0pt}unit{\isadigit{2}}\ is{\isacharunderscore}{\kern0pt}pullback{\isacharunderscore}{\kern0pt}def\ terminal{\isacharunderscore}{\kern0pt}func{\isacharunderscore}{\kern0pt}comp{\isacharunderscore}{\kern0pt}elem{\isacharparenright}{\kern0pt}\isanewline
\isacommand{qed}\isamarkupfalse%
%
\endisatagproof
{\isafoldproof}%
%
\isadelimproof
\isanewline
%
\endisadelimproof
\isanewline
\isacommand{lemma}\isamarkupfalse%
\ XOR{\isacharunderscore}{\kern0pt}only{\isacharunderscore}{\kern0pt}true{\isacharunderscore}{\kern0pt}right{\isacharunderscore}{\kern0pt}is{\isacharunderscore}{\kern0pt}true{\isacharcolon}{\kern0pt}\isanewline
\ \ {\isachardoublequoteopen}XOR\ {\isasymcirc}\isactrlsub c\ \ {\isasymlangle}{\isasymf}{\isacharcomma}{\kern0pt}{\isasymt}{\isasymrangle}\ {\isacharequal}{\kern0pt}\ {\isasymt}{\isachardoublequoteclose}\isanewline
%
\isadelimproof
%
\endisadelimproof
%
\isatagproof
\isacommand{proof}\isamarkupfalse%
\ {\isacharminus}{\kern0pt}\ \ \ \isanewline
\ \ \isacommand{have}\isamarkupfalse%
\ {\isachardoublequoteopen}{\isasymexists}\ j{\isachardot}{\kern0pt}\ j\ {\isasymin}\isactrlsub c\ {\isasymone}{\isasymCoprod}{\isasymone}\ {\isasymand}\ {\isacharparenleft}{\kern0pt}{\isasymlangle}{\isasymt}{\isacharcomma}{\kern0pt}\ {\isasymf}{\isasymrangle}\ {\isasymamalg}{\isasymlangle}{\isasymf}{\isacharcomma}{\kern0pt}\ {\isasymt}{\isasymrangle}{\isacharparenright}{\kern0pt}\ {\isasymcirc}\isactrlsub c\ j\ \ {\isacharequal}{\kern0pt}\ {\isasymlangle}{\isasymf}{\isacharcomma}{\kern0pt}{\isasymt}{\isasymrangle}{\isachardoublequoteclose}\isanewline
\ \ \ \ \isacommand{by}\isamarkupfalse%
\ {\isacharparenleft}{\kern0pt}typecheck{\isacharunderscore}{\kern0pt}cfuncs{\isacharcomma}{\kern0pt}\ meson\ right{\isacharunderscore}{\kern0pt}coproj{\isacharunderscore}{\kern0pt}cfunc{\isacharunderscore}{\kern0pt}coprod\ right{\isacharunderscore}{\kern0pt}proj{\isacharunderscore}{\kern0pt}type{\isacharparenright}{\kern0pt}\isanewline
\ \ \isacommand{then}\isamarkupfalse%
\ \isacommand{show}\isamarkupfalse%
\ {\isacharquery}{\kern0pt}thesis\isanewline
\ \ \ \ \isacommand{by}\isamarkupfalse%
\ {\isacharparenleft}{\kern0pt}smt\ {\isacharparenleft}{\kern0pt}verit{\isacharcomma}{\kern0pt}\ best{\isacharparenright}{\kern0pt}\ XOR{\isacharunderscore}{\kern0pt}is{\isacharunderscore}{\kern0pt}pullback\ \ comp{\isacharunderscore}{\kern0pt}associative{\isadigit{2}}\ id{\isacharunderscore}{\kern0pt}right{\isacharunderscore}{\kern0pt}unit{\isadigit{2}}\ is{\isacharunderscore}{\kern0pt}pullback{\isacharunderscore}{\kern0pt}def\ terminal{\isacharunderscore}{\kern0pt}func{\isacharunderscore}{\kern0pt}comp{\isacharunderscore}{\kern0pt}elem{\isacharparenright}{\kern0pt}\isanewline
\isacommand{qed}\isamarkupfalse%
%
\endisatagproof
{\isafoldproof}%
%
\isadelimproof
\isanewline
%
\endisadelimproof
\isanewline
\isacommand{lemma}\isamarkupfalse%
\ XOR{\isacharunderscore}{\kern0pt}false{\isacharunderscore}{\kern0pt}false{\isacharunderscore}{\kern0pt}is{\isacharunderscore}{\kern0pt}false{\isacharcolon}{\kern0pt}\isanewline
\ \ \ {\isachardoublequoteopen}XOR\ {\isasymcirc}\isactrlsub c\ \ {\isasymlangle}{\isasymf}{\isacharcomma}{\kern0pt}{\isasymf}{\isasymrangle}\ {\isacharequal}{\kern0pt}\ {\isasymf}{\isachardoublequoteclose}\isanewline
%
\isadelimproof
%
\endisadelimproof
%
\isatagproof
\isacommand{proof}\isamarkupfalse%
{\isacharparenleft}{\kern0pt}rule\ ccontr{\isacharparenright}{\kern0pt}\isanewline
\ \ \isacommand{assume}\isamarkupfalse%
\ {\isachardoublequoteopen}XOR\ {\isasymcirc}\isactrlsub c\ {\isasymlangle}{\isasymf}{\isacharcomma}{\kern0pt}{\isasymf}{\isasymrangle}\ {\isasymnoteq}\ {\isasymf}{\isachardoublequoteclose}\isanewline
\ \ \isacommand{then}\isamarkupfalse%
\ \isacommand{have}\isamarkupfalse%
\ {\isachardoublequoteopen}XOR\ {\isasymcirc}\isactrlsub c\ {\isasymlangle}{\isasymf}{\isacharcomma}{\kern0pt}{\isasymf}{\isasymrangle}\ \ {\isacharequal}{\kern0pt}\ {\isasymt}{\isachardoublequoteclose}\isanewline
\ \ \ \ \isacommand{by}\isamarkupfalse%
\ {\isacharparenleft}{\kern0pt}metis\ NOR{\isacharunderscore}{\kern0pt}is{\isacharunderscore}{\kern0pt}pullback\ XOR{\isacharunderscore}{\kern0pt}type\ comp{\isacharunderscore}{\kern0pt}type\ is{\isacharunderscore}{\kern0pt}pullback{\isacharunderscore}{\kern0pt}def\ \ true{\isacharunderscore}{\kern0pt}false{\isacharunderscore}{\kern0pt}only{\isacharunderscore}{\kern0pt}truth{\isacharunderscore}{\kern0pt}values{\isacharparenright}{\kern0pt}\isanewline
\ \ \isacommand{then}\isamarkupfalse%
\ \isacommand{obtain}\isamarkupfalse%
\ j\ \isakeyword{where}\ j{\isacharunderscore}{\kern0pt}def{\isacharcolon}{\kern0pt}\ {\isachardoublequoteopen}j\ {\isasymin}\isactrlsub c\ {\isasymone}{\isasymCoprod}{\isasymone}\ {\isasymand}\ {\isacharparenleft}{\kern0pt}{\isasymlangle}{\isasymt}{\isacharcomma}{\kern0pt}\ {\isasymf}{\isasymrangle}\ {\isasymamalg}{\isasymlangle}{\isasymf}{\isacharcomma}{\kern0pt}\ {\isasymt}{\isasymrangle}{\isacharparenright}{\kern0pt}\ {\isasymcirc}\isactrlsub c\ j\ \ {\isacharequal}{\kern0pt}\ {\isasymlangle}{\isasymf}{\isacharcomma}{\kern0pt}{\isasymf}{\isasymrangle}{\isachardoublequoteclose}\isanewline
\ \ \ \ \isacommand{by}\isamarkupfalse%
\ {\isacharparenleft}{\kern0pt}typecheck{\isacharunderscore}{\kern0pt}cfuncs{\isacharcomma}{\kern0pt}\ auto{\isacharcomma}{\kern0pt}\ smt\ {\isacharparenleft}{\kern0pt}verit{\isacharcomma}{\kern0pt}\ ccfv{\isacharunderscore}{\kern0pt}threshold{\isacharparenright}{\kern0pt}\ XOR{\isacharunderscore}{\kern0pt}is{\isacharunderscore}{\kern0pt}pullback\ id{\isacharunderscore}{\kern0pt}right{\isacharunderscore}{\kern0pt}unit{\isadigit{2}}\ id{\isacharunderscore}{\kern0pt}type\ is{\isacharunderscore}{\kern0pt}pullback{\isacharunderscore}{\kern0pt}def{\isacharparenright}{\kern0pt}\isanewline
\ \ \isacommand{show}\isamarkupfalse%
\ False\isanewline
\ \ \isacommand{proof}\isamarkupfalse%
{\isacharparenleft}{\kern0pt}cases\ {\isachardoublequoteopen}j\ {\isacharequal}{\kern0pt}\ left{\isacharunderscore}{\kern0pt}coproj\ {\isasymone}\ {\isasymone}{\isachardoublequoteclose}{\isacharparenright}{\kern0pt}\isanewline
\ \ \ \ \isacommand{assume}\isamarkupfalse%
\ {\isachardoublequoteopen}j\ {\isacharequal}{\kern0pt}\ left{\isacharunderscore}{\kern0pt}coproj\ {\isasymone}\ {\isasymone}{\isachardoublequoteclose}\isanewline
\ \ \ \ \isacommand{then}\isamarkupfalse%
\ \isacommand{have}\isamarkupfalse%
\ {\isachardoublequoteopen}{\isacharparenleft}{\kern0pt}{\isasymlangle}{\isasymt}{\isacharcomma}{\kern0pt}\ {\isasymf}{\isasymrangle}\ {\isasymamalg}{\isasymlangle}{\isasymf}{\isacharcomma}{\kern0pt}\ {\isasymt}{\isasymrangle}{\isacharparenright}{\kern0pt}\ {\isasymcirc}\isactrlsub c\ j\ \ {\isacharequal}{\kern0pt}\ {\isasymlangle}{\isasymt}{\isacharcomma}{\kern0pt}\ {\isasymf}{\isasymrangle}{\isachardoublequoteclose}\isanewline
\ \ \ \ \ \ \isacommand{using}\isamarkupfalse%
\ \ left{\isacharunderscore}{\kern0pt}coproj{\isacharunderscore}{\kern0pt}cfunc{\isacharunderscore}{\kern0pt}coprod\ \isacommand{by}\isamarkupfalse%
\ {\isacharparenleft}{\kern0pt}typecheck{\isacharunderscore}{\kern0pt}cfuncs{\isacharcomma}{\kern0pt}\ presburger{\isacharparenright}{\kern0pt}\isanewline
\ \ \ \ \isacommand{then}\isamarkupfalse%
\ \isacommand{have}\isamarkupfalse%
\ {\isachardoublequoteopen}{\isasymlangle}{\isasymt}{\isacharcomma}{\kern0pt}\ {\isasymf}{\isasymrangle}\ {\isacharequal}{\kern0pt}\ {\isasymlangle}{\isasymf}{\isacharcomma}{\kern0pt}{\isasymf}{\isasymrangle}{\isachardoublequoteclose}\isanewline
\ \ \ \ \ \ \isacommand{using}\isamarkupfalse%
\ j{\isacharunderscore}{\kern0pt}def\ \isacommand{by}\isamarkupfalse%
\ auto\isanewline
\ \ \ \ \isacommand{then}\isamarkupfalse%
\ \isacommand{have}\isamarkupfalse%
\ {\isachardoublequoteopen}{\isasymt}\ {\isacharequal}{\kern0pt}\ {\isasymf}{\isachardoublequoteclose}\isanewline
\ \ \ \ \ \ \isacommand{using}\isamarkupfalse%
\ cart{\isacharunderscore}{\kern0pt}prod{\isacharunderscore}{\kern0pt}eq{\isadigit{2}}\ false{\isacharunderscore}{\kern0pt}func{\isacharunderscore}{\kern0pt}type\ true{\isacharunderscore}{\kern0pt}func{\isacharunderscore}{\kern0pt}type\ \isacommand{by}\isamarkupfalse%
\ auto\isanewline
\ \ \ \ \isacommand{then}\isamarkupfalse%
\ \isacommand{show}\isamarkupfalse%
\ False\isanewline
\ \ \ \ \ \ \isacommand{using}\isamarkupfalse%
\ true{\isacharunderscore}{\kern0pt}false{\isacharunderscore}{\kern0pt}distinct\ \isacommand{by}\isamarkupfalse%
\ auto\isanewline
\ \ \isacommand{next}\isamarkupfalse%
\isanewline
\ \ \ \ \isacommand{assume}\isamarkupfalse%
\ {\isachardoublequoteopen}j\ {\isasymnoteq}\ left{\isacharunderscore}{\kern0pt}coproj\ {\isasymone}\ {\isasymone}{\isachardoublequoteclose}\isanewline
\ \ \ \ \isacommand{then}\isamarkupfalse%
\ \isacommand{have}\isamarkupfalse%
\ {\isachardoublequoteopen}j\ {\isacharequal}{\kern0pt}\ right{\isacharunderscore}{\kern0pt}coproj\ {\isasymone}\ {\isasymone}{\isachardoublequoteclose}\isanewline
\ \ \ \ \ \ \isacommand{by}\isamarkupfalse%
\ {\isacharparenleft}{\kern0pt}meson\ j{\isacharunderscore}{\kern0pt}def\ maps{\isacharunderscore}{\kern0pt}into{\isacharunderscore}{\kern0pt}{\isadigit{1}}u{\isadigit{1}}{\isacharparenright}{\kern0pt}\isanewline
\ \ \ \ \isacommand{then}\isamarkupfalse%
\ \isacommand{have}\isamarkupfalse%
\ {\isachardoublequoteopen}{\isacharparenleft}{\kern0pt}{\isasymlangle}{\isasymt}{\isacharcomma}{\kern0pt}\ {\isasymf}{\isasymrangle}\ {\isasymamalg}{\isasymlangle}{\isasymf}{\isacharcomma}{\kern0pt}\ {\isasymt}{\isasymrangle}{\isacharparenright}{\kern0pt}\ {\isasymcirc}\isactrlsub c\ j\ \ {\isacharequal}{\kern0pt}\ {\isasymlangle}{\isasymf}{\isacharcomma}{\kern0pt}\ {\isasymt}{\isasymrangle}{\isachardoublequoteclose}\isanewline
\ \ \ \ \ \ \isacommand{using}\isamarkupfalse%
\ \ right{\isacharunderscore}{\kern0pt}coproj{\isacharunderscore}{\kern0pt}cfunc{\isacharunderscore}{\kern0pt}coprod\ \isacommand{by}\isamarkupfalse%
\ {\isacharparenleft}{\kern0pt}typecheck{\isacharunderscore}{\kern0pt}cfuncs{\isacharcomma}{\kern0pt}\ presburger{\isacharparenright}{\kern0pt}\isanewline
\ \ \ \ \isacommand{then}\isamarkupfalse%
\ \isacommand{have}\isamarkupfalse%
\ {\isachardoublequoteopen}{\isasymlangle}{\isasymf}{\isacharcomma}{\kern0pt}\ {\isasymt}{\isasymrangle}\ {\isacharequal}{\kern0pt}\ {\isasymlangle}{\isasymf}{\isacharcomma}{\kern0pt}{\isasymf}{\isasymrangle}{\isachardoublequoteclose}\isanewline
\ \ \ \ \ \ \isacommand{using}\isamarkupfalse%
\ j{\isacharunderscore}{\kern0pt}def\ \isacommand{by}\isamarkupfalse%
\ auto\isanewline
\ \ \ \ \isacommand{then}\isamarkupfalse%
\ \isacommand{have}\isamarkupfalse%
\ {\isachardoublequoteopen}{\isasymt}\ {\isacharequal}{\kern0pt}\ {\isasymf}{\isachardoublequoteclose}\isanewline
\ \ \ \ \ \ \isacommand{using}\isamarkupfalse%
\ cart{\isacharunderscore}{\kern0pt}prod{\isacharunderscore}{\kern0pt}eq{\isadigit{2}}\ false{\isacharunderscore}{\kern0pt}func{\isacharunderscore}{\kern0pt}type\ true{\isacharunderscore}{\kern0pt}func{\isacharunderscore}{\kern0pt}type\ \isacommand{by}\isamarkupfalse%
\ auto\isanewline
\ \ \ \ \isacommand{then}\isamarkupfalse%
\ \isacommand{show}\isamarkupfalse%
\ False\isanewline
\ \ \ \ \ \ \isacommand{using}\isamarkupfalse%
\ true{\isacharunderscore}{\kern0pt}false{\isacharunderscore}{\kern0pt}distinct\ \isacommand{by}\isamarkupfalse%
\ auto\isanewline
\ \ \isacommand{qed}\isamarkupfalse%
\isanewline
\isacommand{qed}\isamarkupfalse%
%
\endisatagproof
{\isafoldproof}%
%
\isadelimproof
\isanewline
%
\endisadelimproof
\isanewline
\isacommand{lemma}\isamarkupfalse%
\ XOR{\isacharunderscore}{\kern0pt}true{\isacharunderscore}{\kern0pt}true{\isacharunderscore}{\kern0pt}is{\isacharunderscore}{\kern0pt}false{\isacharcolon}{\kern0pt}\isanewline
\ \ \ {\isachardoublequoteopen}XOR\ {\isasymcirc}\isactrlsub c\ \ {\isasymlangle}{\isasymt}{\isacharcomma}{\kern0pt}{\isasymt}{\isasymrangle}\ {\isacharequal}{\kern0pt}\ {\isasymf}{\isachardoublequoteclose}\isanewline
%
\isadelimproof
%
\endisadelimproof
%
\isatagproof
\isacommand{proof}\isamarkupfalse%
{\isacharparenleft}{\kern0pt}rule\ ccontr{\isacharparenright}{\kern0pt}\isanewline
\ \ \isacommand{assume}\isamarkupfalse%
\ {\isachardoublequoteopen}XOR\ {\isasymcirc}\isactrlsub c\ {\isasymlangle}{\isasymt}{\isacharcomma}{\kern0pt}{\isasymt}{\isasymrangle}\ {\isasymnoteq}\ {\isasymf}{\isachardoublequoteclose}\isanewline
\ \ \isacommand{then}\isamarkupfalse%
\ \isacommand{have}\isamarkupfalse%
\ {\isachardoublequoteopen}XOR\ {\isasymcirc}\isactrlsub c\ {\isasymlangle}{\isasymt}{\isacharcomma}{\kern0pt}{\isasymt}{\isasymrangle}\ \ {\isacharequal}{\kern0pt}\ {\isasymt}{\isachardoublequoteclose}\isanewline
\ \ \ \ \isacommand{by}\isamarkupfalse%
\ {\isacharparenleft}{\kern0pt}metis\ XOR{\isacharunderscore}{\kern0pt}type\ comp{\isacharunderscore}{\kern0pt}type\ diag{\isacharunderscore}{\kern0pt}on{\isacharunderscore}{\kern0pt}elements\ diagonal{\isacharunderscore}{\kern0pt}type\ true{\isacharunderscore}{\kern0pt}false{\isacharunderscore}{\kern0pt}only{\isacharunderscore}{\kern0pt}truth{\isacharunderscore}{\kern0pt}values\ true{\isacharunderscore}{\kern0pt}func{\isacharunderscore}{\kern0pt}type{\isacharparenright}{\kern0pt}\isanewline
\ \ \isacommand{then}\isamarkupfalse%
\ \isacommand{obtain}\isamarkupfalse%
\ j\ \isakeyword{where}\ j{\isacharunderscore}{\kern0pt}def{\isacharcolon}{\kern0pt}\ {\isachardoublequoteopen}j\ {\isasymin}\isactrlsub c\ {\isasymone}{\isasymCoprod}{\isasymone}\ {\isasymand}\ {\isacharparenleft}{\kern0pt}{\isasymlangle}{\isasymt}{\isacharcomma}{\kern0pt}\ {\isasymf}{\isasymrangle}\ {\isasymamalg}{\isasymlangle}{\isasymf}{\isacharcomma}{\kern0pt}\ {\isasymt}{\isasymrangle}{\isacharparenright}{\kern0pt}\ {\isasymcirc}\isactrlsub c\ j\ \ {\isacharequal}{\kern0pt}\ {\isasymlangle}{\isasymt}{\isacharcomma}{\kern0pt}{\isasymt}{\isasymrangle}{\isachardoublequoteclose}\isanewline
\ \ \ \ \isacommand{by}\isamarkupfalse%
\ {\isacharparenleft}{\kern0pt}typecheck{\isacharunderscore}{\kern0pt}cfuncs{\isacharcomma}{\kern0pt}\ auto{\isacharcomma}{\kern0pt}\ smt\ {\isacharparenleft}{\kern0pt}verit{\isacharcomma}{\kern0pt}\ ccfv{\isacharunderscore}{\kern0pt}threshold{\isacharparenright}{\kern0pt}\ XOR{\isacharunderscore}{\kern0pt}is{\isacharunderscore}{\kern0pt}pullback\ id{\isacharunderscore}{\kern0pt}right{\isacharunderscore}{\kern0pt}unit{\isadigit{2}}\ id{\isacharunderscore}{\kern0pt}type\ is{\isacharunderscore}{\kern0pt}pullback{\isacharunderscore}{\kern0pt}def{\isacharparenright}{\kern0pt}\isanewline
\ \ \isacommand{show}\isamarkupfalse%
\ False\isanewline
\ \ \isacommand{proof}\isamarkupfalse%
{\isacharparenleft}{\kern0pt}cases\ {\isachardoublequoteopen}j\ {\isacharequal}{\kern0pt}\ left{\isacharunderscore}{\kern0pt}coproj\ {\isasymone}\ {\isasymone}{\isachardoublequoteclose}{\isacharparenright}{\kern0pt}\isanewline
\ \ \ \ \isacommand{assume}\isamarkupfalse%
\ {\isachardoublequoteopen}j\ {\isacharequal}{\kern0pt}\ left{\isacharunderscore}{\kern0pt}coproj\ {\isasymone}\ {\isasymone}{\isachardoublequoteclose}\isanewline
\ \ \ \ \isacommand{then}\isamarkupfalse%
\ \isacommand{have}\isamarkupfalse%
\ {\isachardoublequoteopen}{\isacharparenleft}{\kern0pt}{\isasymlangle}{\isasymt}{\isacharcomma}{\kern0pt}\ {\isasymf}{\isasymrangle}\ {\isasymamalg}{\isasymlangle}{\isasymf}{\isacharcomma}{\kern0pt}\ {\isasymt}{\isasymrangle}{\isacharparenright}{\kern0pt}\ {\isasymcirc}\isactrlsub c\ j\ \ {\isacharequal}{\kern0pt}\ {\isasymlangle}{\isasymt}{\isacharcomma}{\kern0pt}\ {\isasymf}{\isasymrangle}{\isachardoublequoteclose}\isanewline
\ \ \ \ \ \ \isacommand{using}\isamarkupfalse%
\ \ left{\isacharunderscore}{\kern0pt}coproj{\isacharunderscore}{\kern0pt}cfunc{\isacharunderscore}{\kern0pt}coprod\ \isacommand{by}\isamarkupfalse%
\ {\isacharparenleft}{\kern0pt}typecheck{\isacharunderscore}{\kern0pt}cfuncs{\isacharcomma}{\kern0pt}\ presburger{\isacharparenright}{\kern0pt}\isanewline
\ \ \ \ \isacommand{then}\isamarkupfalse%
\ \isacommand{have}\isamarkupfalse%
\ {\isachardoublequoteopen}{\isasymlangle}{\isasymt}{\isacharcomma}{\kern0pt}\ {\isasymf}{\isasymrangle}\ {\isacharequal}{\kern0pt}\ {\isasymlangle}{\isasymt}{\isacharcomma}{\kern0pt}{\isasymt}{\isasymrangle}{\isachardoublequoteclose}\isanewline
\ \ \ \ \ \ \isacommand{using}\isamarkupfalse%
\ j{\isacharunderscore}{\kern0pt}def\ \isacommand{by}\isamarkupfalse%
\ auto\isanewline
\ \ \ \ \isacommand{then}\isamarkupfalse%
\ \isacommand{have}\isamarkupfalse%
\ {\isachardoublequoteopen}{\isasymt}\ {\isacharequal}{\kern0pt}\ {\isasymf}{\isachardoublequoteclose}\isanewline
\ \ \ \ \ \ \isacommand{using}\isamarkupfalse%
\ cart{\isacharunderscore}{\kern0pt}prod{\isacharunderscore}{\kern0pt}eq{\isadigit{2}}\ false{\isacharunderscore}{\kern0pt}func{\isacharunderscore}{\kern0pt}type\ true{\isacharunderscore}{\kern0pt}func{\isacharunderscore}{\kern0pt}type\ \isacommand{by}\isamarkupfalse%
\ auto\isanewline
\ \ \ \ \isacommand{then}\isamarkupfalse%
\ \isacommand{show}\isamarkupfalse%
\ False\isanewline
\ \ \ \ \ \ \isacommand{using}\isamarkupfalse%
\ true{\isacharunderscore}{\kern0pt}false{\isacharunderscore}{\kern0pt}distinct\ \isacommand{by}\isamarkupfalse%
\ auto\isanewline
\ \ \isacommand{next}\isamarkupfalse%
\isanewline
\ \ \ \ \isacommand{assume}\isamarkupfalse%
\ {\isachardoublequoteopen}j\ {\isasymnoteq}\ left{\isacharunderscore}{\kern0pt}coproj\ {\isasymone}\ {\isasymone}{\isachardoublequoteclose}\isanewline
\ \ \ \ \isacommand{then}\isamarkupfalse%
\ \isacommand{have}\isamarkupfalse%
\ {\isachardoublequoteopen}j\ {\isacharequal}{\kern0pt}\ right{\isacharunderscore}{\kern0pt}coproj\ {\isasymone}\ {\isasymone}{\isachardoublequoteclose}\isanewline
\ \ \ \ \ \ \isacommand{by}\isamarkupfalse%
\ {\isacharparenleft}{\kern0pt}meson\ j{\isacharunderscore}{\kern0pt}def\ maps{\isacharunderscore}{\kern0pt}into{\isacharunderscore}{\kern0pt}{\isadigit{1}}u{\isadigit{1}}{\isacharparenright}{\kern0pt}\isanewline
\ \ \ \ \isacommand{then}\isamarkupfalse%
\ \isacommand{have}\isamarkupfalse%
\ {\isachardoublequoteopen}{\isacharparenleft}{\kern0pt}{\isasymlangle}{\isasymt}{\isacharcomma}{\kern0pt}\ {\isasymf}{\isasymrangle}\ {\isasymamalg}{\isasymlangle}{\isasymf}{\isacharcomma}{\kern0pt}\ {\isasymt}{\isasymrangle}{\isacharparenright}{\kern0pt}\ {\isasymcirc}\isactrlsub c\ j\ \ {\isacharequal}{\kern0pt}\ {\isasymlangle}{\isasymf}{\isacharcomma}{\kern0pt}\ {\isasymt}{\isasymrangle}{\isachardoublequoteclose}\isanewline
\ \ \ \ \ \ \isacommand{using}\isamarkupfalse%
\ \ right{\isacharunderscore}{\kern0pt}coproj{\isacharunderscore}{\kern0pt}cfunc{\isacharunderscore}{\kern0pt}coprod\ \isacommand{by}\isamarkupfalse%
\ {\isacharparenleft}{\kern0pt}typecheck{\isacharunderscore}{\kern0pt}cfuncs{\isacharcomma}{\kern0pt}\ presburger{\isacharparenright}{\kern0pt}\isanewline
\ \ \ \ \isacommand{then}\isamarkupfalse%
\ \isacommand{have}\isamarkupfalse%
\ {\isachardoublequoteopen}{\isasymlangle}{\isasymf}{\isacharcomma}{\kern0pt}\ {\isasymt}{\isasymrangle}\ {\isacharequal}{\kern0pt}\ {\isasymlangle}{\isasymt}{\isacharcomma}{\kern0pt}{\isasymt}{\isasymrangle}{\isachardoublequoteclose}\isanewline
\ \ \ \ \ \ \isacommand{using}\isamarkupfalse%
\ j{\isacharunderscore}{\kern0pt}def\ \isacommand{by}\isamarkupfalse%
\ auto\isanewline
\ \ \ \ \isacommand{then}\isamarkupfalse%
\ \isacommand{have}\isamarkupfalse%
\ {\isachardoublequoteopen}{\isasymt}\ {\isacharequal}{\kern0pt}\ {\isasymf}{\isachardoublequoteclose}\isanewline
\ \ \ \ \ \ \isacommand{using}\isamarkupfalse%
\ cart{\isacharunderscore}{\kern0pt}prod{\isacharunderscore}{\kern0pt}eq{\isadigit{2}}\ false{\isacharunderscore}{\kern0pt}func{\isacharunderscore}{\kern0pt}type\ true{\isacharunderscore}{\kern0pt}func{\isacharunderscore}{\kern0pt}type\ \isacommand{by}\isamarkupfalse%
\ auto\isanewline
\ \ \ \ \isacommand{then}\isamarkupfalse%
\ \isacommand{show}\isamarkupfalse%
\ False\isanewline
\ \ \ \ \ \ \isacommand{using}\isamarkupfalse%
\ true{\isacharunderscore}{\kern0pt}false{\isacharunderscore}{\kern0pt}distinct\ \isacommand{by}\isamarkupfalse%
\ auto\isanewline
\ \ \isacommand{qed}\isamarkupfalse%
\isanewline
\isacommand{qed}\isamarkupfalse%
%
\endisatagproof
{\isafoldproof}%
%
\isadelimproof
%
\endisadelimproof
%
\isadelimdocument
%
\endisadelimdocument
%
\isatagdocument
%
\isamarkupsubsection{NAND%
}
\isamarkuptrue%
%
\endisatagdocument
{\isafolddocument}%
%
\isadelimdocument
%
\endisadelimdocument
\isacommand{definition}\isamarkupfalse%
\ NAND\ {\isacharcolon}{\kern0pt}{\isacharcolon}{\kern0pt}\ {\isachardoublequoteopen}cfunc{\isachardoublequoteclose}\ \isakeyword{where}\isanewline
\ \ {\isachardoublequoteopen}NAND\ {\isacharequal}{\kern0pt}\ {\isacharparenleft}{\kern0pt}THE\ {\isasymchi}{\isachardot}{\kern0pt}\ is{\isacharunderscore}{\kern0pt}pullback\ {\isacharparenleft}{\kern0pt}{\isasymone}{\isasymCoprod}{\isacharparenleft}{\kern0pt}{\isasymone}{\isasymCoprod}{\isasymone}{\isacharparenright}{\kern0pt}{\isacharparenright}{\kern0pt}\ {\isasymone}\ {\isacharparenleft}{\kern0pt}{\isasymOmega}\ {\isasymtimes}\isactrlsub c\ {\isasymOmega}{\isacharparenright}{\kern0pt}\ {\isasymOmega}\ {\isacharparenleft}{\kern0pt}{\isasymbeta}\isactrlbsub {\isacharparenleft}{\kern0pt}{\isasymone}{\isasymCoprod}{\isacharparenleft}{\kern0pt}{\isasymone}{\isasymCoprod}{\isasymone}{\isacharparenright}{\kern0pt}{\isacharparenright}{\kern0pt}\isactrlesub {\isacharparenright}{\kern0pt}\ {\isasymt}\ {\isacharparenleft}{\kern0pt}{\isasymlangle}{\isasymf}{\isacharcomma}{\kern0pt}\ {\isasymf}{\isasymrangle}{\isasymamalg}\ {\isacharparenleft}{\kern0pt}{\isasymlangle}{\isasymt}{\isacharcomma}{\kern0pt}\ {\isasymf}{\isasymrangle}\ {\isasymamalg}{\isasymlangle}{\isasymf}{\isacharcomma}{\kern0pt}\ {\isasymt}{\isasymrangle}{\isacharparenright}{\kern0pt}{\isacharparenright}{\kern0pt}\ {\isasymchi}{\isacharparenright}{\kern0pt}{\isachardoublequoteclose}\isanewline
\isanewline
\isacommand{lemma}\isamarkupfalse%
\ pre{\isacharunderscore}{\kern0pt}NAND{\isacharunderscore}{\kern0pt}type{\isacharbrackleft}{\kern0pt}type{\isacharunderscore}{\kern0pt}rule{\isacharbrackright}{\kern0pt}{\isacharcolon}{\kern0pt}\ \isanewline
\ \ {\isachardoublequoteopen}{\isasymlangle}{\isasymf}{\isacharcomma}{\kern0pt}\ {\isasymf}{\isasymrangle}\ {\isasymamalg}\ {\isacharparenleft}{\kern0pt}{\isasymlangle}{\isasymt}{\isacharcomma}{\kern0pt}\ {\isasymf}{\isasymrangle}\ {\isasymamalg}\ {\isasymlangle}{\isasymf}{\isacharcomma}{\kern0pt}\ {\isasymt}{\isasymrangle}{\isacharparenright}{\kern0pt}\ {\isacharcolon}{\kern0pt}\ {\isasymone}{\isasymCoprod}{\isacharparenleft}{\kern0pt}{\isasymone}{\isasymCoprod}{\isasymone}{\isacharparenright}{\kern0pt}\ {\isasymrightarrow}\ {\isasymOmega}\ {\isasymtimes}\isactrlsub c\ {\isasymOmega}{\isachardoublequoteclose}\isanewline
%
\isadelimproof
\ \ %
\endisadelimproof
%
\isatagproof
\isacommand{by}\isamarkupfalse%
\ typecheck{\isacharunderscore}{\kern0pt}cfuncs%
\endisatagproof
{\isafoldproof}%
%
\isadelimproof
\isanewline
%
\endisadelimproof
\isanewline
\isacommand{lemma}\isamarkupfalse%
\ pre{\isacharunderscore}{\kern0pt}NAND{\isacharunderscore}{\kern0pt}injective{\isacharcolon}{\kern0pt}\isanewline
\ \ {\isachardoublequoteopen}injective{\isacharparenleft}{\kern0pt}{\isasymlangle}{\isasymf}{\isacharcomma}{\kern0pt}\ {\isasymf}{\isasymrangle}\ {\isasymamalg}\ {\isacharparenleft}{\kern0pt}{\isasymlangle}{\isasymt}{\isacharcomma}{\kern0pt}\ {\isasymf}{\isasymrangle}\ {\isasymamalg}\ {\isasymlangle}{\isasymf}{\isacharcomma}{\kern0pt}\ {\isasymt}{\isasymrangle}{\isacharparenright}{\kern0pt}{\isacharparenright}{\kern0pt}{\isachardoublequoteclose}\isanewline
%
\isadelimproof
\ \ %
\endisadelimproof
%
\isatagproof
\isacommand{unfolding}\isamarkupfalse%
\ injective{\isacharunderscore}{\kern0pt}def\isanewline
\isacommand{proof}\isamarkupfalse%
{\isacharparenleft}{\kern0pt}clarify{\isacharparenright}{\kern0pt}\isanewline
\ \ \isacommand{fix}\isamarkupfalse%
\ x\ y\ \isanewline
\ \ \isacommand{assume}\isamarkupfalse%
\ x{\isacharunderscore}{\kern0pt}type{\isacharcolon}{\kern0pt}\ {\isachardoublequoteopen}x\ {\isasymin}\isactrlsub c\ domain\ {\isacharparenleft}{\kern0pt}{\isasymlangle}{\isasymf}{\isacharcomma}{\kern0pt}\ {\isasymf}{\isasymrangle}\ {\isasymamalg}\ {\isasymlangle}{\isasymt}{\isacharcomma}{\kern0pt}{\isasymf}{\isasymrangle}\ {\isasymamalg}\ {\isasymlangle}{\isasymf}{\isacharcomma}{\kern0pt}{\isasymt}{\isasymrangle}{\isacharparenright}{\kern0pt}{\isachardoublequoteclose}\ \isanewline
\ \ \isacommand{then}\isamarkupfalse%
\ \isacommand{have}\isamarkupfalse%
\ x{\isacharunderscore}{\kern0pt}type{\isacharprime}{\kern0pt}{\isacharcolon}{\kern0pt}\ {\isachardoublequoteopen}x\ {\isasymin}\isactrlsub c\ {\isasymone}\ {\isasymCoprod}\ {\isacharparenleft}{\kern0pt}{\isasymone}{\isasymCoprod}{\isasymone}{\isacharparenright}{\kern0pt}{\isachardoublequoteclose}\ \ \isanewline
\ \ \ \ \isacommand{using}\isamarkupfalse%
\ cfunc{\isacharunderscore}{\kern0pt}type{\isacharunderscore}{\kern0pt}def\ pre{\isacharunderscore}{\kern0pt}NAND{\isacharunderscore}{\kern0pt}type\ \isacommand{by}\isamarkupfalse%
\ force\isanewline
\ \ \isacommand{then}\isamarkupfalse%
\ \isacommand{have}\isamarkupfalse%
\ x{\isacharunderscore}{\kern0pt}form{\isacharcolon}{\kern0pt}\ {\isachardoublequoteopen}{\isacharparenleft}{\kern0pt}{\isasymexists}\ w{\isachardot}{\kern0pt}\ w\ {\isasymin}\isactrlsub c\ {\isasymone}\ {\isasymand}\ x\ {\isacharequal}{\kern0pt}\ left{\isacharunderscore}{\kern0pt}coproj\ {\isasymone}\ {\isacharparenleft}{\kern0pt}{\isasymone}{\isasymCoprod}{\isasymone}{\isacharparenright}{\kern0pt}\ {\isasymcirc}\isactrlsub c\ w{\isacharparenright}{\kern0pt}\isanewline
\ \ \ \ \ \ {\isasymor}\ \ {\isacharparenleft}{\kern0pt}{\isasymexists}\ w{\isachardot}{\kern0pt}\ w\ {\isasymin}\isactrlsub c\ {\isasymone}{\isasymCoprod}{\isasymone}\ {\isasymand}\ x\ {\isacharequal}{\kern0pt}\ right{\isacharunderscore}{\kern0pt}coproj\ {\isasymone}\ {\isacharparenleft}{\kern0pt}{\isasymone}{\isasymCoprod}{\isasymone}{\isacharparenright}{\kern0pt}\ {\isasymcirc}\isactrlsub c\ w{\isacharparenright}{\kern0pt}{\isachardoublequoteclose}\isanewline
\ \ \ \ \isacommand{using}\isamarkupfalse%
\ coprojs{\isacharunderscore}{\kern0pt}jointly{\isacharunderscore}{\kern0pt}surj\ \isacommand{by}\isamarkupfalse%
\ auto\isanewline
\isanewline
\ \ \isacommand{assume}\isamarkupfalse%
\ y{\isacharunderscore}{\kern0pt}type{\isacharcolon}{\kern0pt}\ {\isachardoublequoteopen}y\ {\isasymin}\isactrlsub c\ domain\ {\isacharparenleft}{\kern0pt}{\isasymlangle}{\isasymf}{\isacharcomma}{\kern0pt}\ {\isasymf}{\isasymrangle}\ {\isasymamalg}\ {\isasymlangle}{\isasymt}{\isacharcomma}{\kern0pt}{\isasymf}{\isasymrangle}\ {\isasymamalg}\ {\isasymlangle}{\isasymf}{\isacharcomma}{\kern0pt}{\isasymt}{\isasymrangle}{\isacharparenright}{\kern0pt}{\isachardoublequoteclose}\ \isanewline
\ \ \isacommand{then}\isamarkupfalse%
\ \isacommand{have}\isamarkupfalse%
\ y{\isacharunderscore}{\kern0pt}type{\isacharprime}{\kern0pt}{\isacharcolon}{\kern0pt}\ {\isachardoublequoteopen}y\ {\isasymin}\isactrlsub c\ {\isasymone}{\isasymCoprod}\ {\isacharparenleft}{\kern0pt}{\isasymone}{\isasymCoprod}{\isasymone}{\isacharparenright}{\kern0pt}{\isachardoublequoteclose}\ \ \isanewline
\ \ \ \ \isacommand{using}\isamarkupfalse%
\ cfunc{\isacharunderscore}{\kern0pt}type{\isacharunderscore}{\kern0pt}def\ pre{\isacharunderscore}{\kern0pt}NAND{\isacharunderscore}{\kern0pt}type\ \isacommand{by}\isamarkupfalse%
\ force\isanewline
\ \ \isacommand{then}\isamarkupfalse%
\ \isacommand{have}\isamarkupfalse%
\ y{\isacharunderscore}{\kern0pt}form{\isacharcolon}{\kern0pt}\ {\isachardoublequoteopen}{\isacharparenleft}{\kern0pt}{\isasymexists}\ w{\isachardot}{\kern0pt}\ w\ {\isasymin}\isactrlsub c\ {\isasymone}\ {\isasymand}\ y\ {\isacharequal}{\kern0pt}\ left{\isacharunderscore}{\kern0pt}coproj\ {\isasymone}\ {\isacharparenleft}{\kern0pt}{\isasymone}{\isasymCoprod}{\isasymone}{\isacharparenright}{\kern0pt}\ {\isasymcirc}\isactrlsub c\ w{\isacharparenright}{\kern0pt}\isanewline
\ \ \ \ \ \ {\isasymor}\ \ {\isacharparenleft}{\kern0pt}{\isasymexists}\ w{\isachardot}{\kern0pt}\ w\ {\isasymin}\isactrlsub c\ {\isasymone}{\isasymCoprod}{\isasymone}\ {\isasymand}\ y\ {\isacharequal}{\kern0pt}\ right{\isacharunderscore}{\kern0pt}coproj\ {\isasymone}\ {\isacharparenleft}{\kern0pt}{\isasymone}{\isasymCoprod}{\isasymone}{\isacharparenright}{\kern0pt}\ {\isasymcirc}\isactrlsub c\ w{\isacharparenright}{\kern0pt}{\isachardoublequoteclose}\isanewline
\ \ \ \ \isacommand{using}\isamarkupfalse%
\ coprojs{\isacharunderscore}{\kern0pt}jointly{\isacharunderscore}{\kern0pt}surj\ \isacommand{by}\isamarkupfalse%
\ auto\isanewline
\isanewline
\ \ \isacommand{assume}\isamarkupfalse%
\ mx{\isacharunderscore}{\kern0pt}eqs{\isacharunderscore}{\kern0pt}my{\isacharcolon}{\kern0pt}\ {\isachardoublequoteopen}{\isasymlangle}{\isasymf}{\isacharcomma}{\kern0pt}\ {\isasymf}{\isasymrangle}\ {\isasymamalg}\ {\isasymlangle}{\isasymt}{\isacharcomma}{\kern0pt}{\isasymf}{\isasymrangle}\ {\isasymamalg}\ {\isasymlangle}{\isasymf}{\isacharcomma}{\kern0pt}{\isasymt}{\isasymrangle}\ {\isasymcirc}\isactrlsub c\ x\ {\isacharequal}{\kern0pt}\ {\isasymlangle}{\isasymf}{\isacharcomma}{\kern0pt}\ {\isasymf}{\isasymrangle}\ {\isasymamalg}\ {\isasymlangle}{\isasymt}{\isacharcomma}{\kern0pt}{\isasymf}{\isasymrangle}\ {\isasymamalg}\ {\isasymlangle}{\isasymf}{\isacharcomma}{\kern0pt}{\isasymt}{\isasymrangle}\ {\isasymcirc}\isactrlsub c\ y{\isachardoublequoteclose}\isanewline
\isanewline
\ \ \isacommand{have}\isamarkupfalse%
\ f{\isadigit{1}}{\isacharcolon}{\kern0pt}\ {\isachardoublequoteopen}{\isasymlangle}{\isasymf}{\isacharcomma}{\kern0pt}\ {\isasymf}{\isasymrangle}\ {\isasymamalg}\ {\isasymlangle}{\isasymt}{\isacharcomma}{\kern0pt}{\isasymf}{\isasymrangle}\ {\isasymamalg}\ {\isasymlangle}{\isasymf}{\isacharcomma}{\kern0pt}{\isasymt}{\isasymrangle}\ {\isasymcirc}\isactrlsub c\ left{\isacharunderscore}{\kern0pt}coproj\ {\isasymone}\ {\isacharparenleft}{\kern0pt}{\isasymone}\ {\isasymCoprod}\ {\isasymone}{\isacharparenright}{\kern0pt}\ {\isacharequal}{\kern0pt}\ {\isasymlangle}{\isasymf}{\isacharcomma}{\kern0pt}\ {\isasymf}{\isasymrangle}{\isachardoublequoteclose}\isanewline
\ \ \ \ \isacommand{by}\isamarkupfalse%
\ {\isacharparenleft}{\kern0pt}typecheck{\isacharunderscore}{\kern0pt}cfuncs{\isacharcomma}{\kern0pt}\ simp\ add{\isacharcolon}{\kern0pt}\ left{\isacharunderscore}{\kern0pt}coproj{\isacharunderscore}{\kern0pt}cfunc{\isacharunderscore}{\kern0pt}coprod{\isacharparenright}{\kern0pt}\isanewline
\ \ \isacommand{have}\isamarkupfalse%
\ f{\isadigit{2}}{\isacharcolon}{\kern0pt}\ {\isachardoublequoteopen}{\isasymlangle}{\isasymf}{\isacharcomma}{\kern0pt}\ {\isasymf}{\isasymrangle}\ {\isasymamalg}\ {\isasymlangle}{\isasymt}{\isacharcomma}{\kern0pt}{\isasymf}{\isasymrangle}\ {\isasymamalg}\ {\isasymlangle}{\isasymf}{\isacharcomma}{\kern0pt}{\isasymt}{\isasymrangle}\ {\isasymcirc}\isactrlsub c\ {\isacharparenleft}{\kern0pt}right{\isacharunderscore}{\kern0pt}coproj\ {\isasymone}\ {\isacharparenleft}{\kern0pt}{\isasymone}{\isasymCoprod}{\isasymone}{\isacharparenright}{\kern0pt}\ {\isasymcirc}\isactrlsub c\ left{\isacharunderscore}{\kern0pt}coproj\ {\isasymone}\ {\isasymone}{\isacharparenright}{\kern0pt}\ {\isacharequal}{\kern0pt}\ {\isasymlangle}{\isasymt}{\isacharcomma}{\kern0pt}{\isasymf}{\isasymrangle}{\isachardoublequoteclose}\isanewline
\ \ \isacommand{proof}\isamarkupfalse%
{\isacharminus}{\kern0pt}\ \isanewline
\ \ \ \ \isacommand{have}\isamarkupfalse%
\ {\isachardoublequoteopen}{\isasymlangle}{\isasymf}{\isacharcomma}{\kern0pt}\ {\isasymf}{\isasymrangle}\ {\isasymamalg}\ {\isasymlangle}{\isasymt}{\isacharcomma}{\kern0pt}{\isasymf}{\isasymrangle}\ {\isasymamalg}\ {\isasymlangle}{\isasymf}{\isacharcomma}{\kern0pt}{\isasymt}{\isasymrangle}\ {\isasymcirc}\isactrlsub c\ right{\isacharunderscore}{\kern0pt}coproj\ {\isasymone}\ {\isacharparenleft}{\kern0pt}{\isasymone}{\isasymCoprod}{\isasymone}{\isacharparenright}{\kern0pt}\ {\isasymcirc}\isactrlsub c\ left{\isacharunderscore}{\kern0pt}coproj\ {\isasymone}\ {\isasymone}\ {\isacharequal}{\kern0pt}\ \isanewline
\ \ \ \ \ \ \ \ \ \ {\isacharparenleft}{\kern0pt}{\isasymlangle}{\isasymf}{\isacharcomma}{\kern0pt}\ {\isasymf}{\isasymrangle}\ {\isasymamalg}\ {\isasymlangle}{\isasymt}{\isacharcomma}{\kern0pt}{\isasymf}{\isasymrangle}\ {\isasymamalg}\ {\isasymlangle}{\isasymf}{\isacharcomma}{\kern0pt}{\isasymt}{\isasymrangle}\ {\isasymcirc}\isactrlsub c\ right{\isacharunderscore}{\kern0pt}coproj\ {\isasymone}\ {\isacharparenleft}{\kern0pt}{\isasymone}{\isasymCoprod}{\isasymone}{\isacharparenright}{\kern0pt}{\isacharparenright}{\kern0pt}\ {\isasymcirc}\isactrlsub c\ left{\isacharunderscore}{\kern0pt}coproj\ {\isasymone}\ {\isasymone}{\isachardoublequoteclose}\isanewline
\ \ \ \ \ \ \isacommand{by}\isamarkupfalse%
\ {\isacharparenleft}{\kern0pt}typecheck{\isacharunderscore}{\kern0pt}cfuncs{\isacharcomma}{\kern0pt}\ simp\ add{\isacharcolon}{\kern0pt}\ comp{\isacharunderscore}{\kern0pt}associative{\isadigit{2}}{\isacharparenright}{\kern0pt}\isanewline
\ \ \ \ \isacommand{also}\isamarkupfalse%
\ \isacommand{have}\isamarkupfalse%
\ {\isachardoublequoteopen}{\isachardot}{\kern0pt}{\isachardot}{\kern0pt}{\isachardot}{\kern0pt}\ {\isacharequal}{\kern0pt}\ {\isasymlangle}{\isasymt}{\isacharcomma}{\kern0pt}{\isasymf}{\isasymrangle}\ {\isasymamalg}\ {\isasymlangle}{\isasymf}{\isacharcomma}{\kern0pt}{\isasymt}{\isasymrangle}\ {\isasymcirc}\isactrlsub c\ left{\isacharunderscore}{\kern0pt}coproj\ {\isasymone}\ {\isasymone}{\isachardoublequoteclose}\isanewline
\ \ \ \ \ \ \isacommand{using}\isamarkupfalse%
\ right{\isacharunderscore}{\kern0pt}coproj{\isacharunderscore}{\kern0pt}cfunc{\isacharunderscore}{\kern0pt}coprod\ \isacommand{by}\isamarkupfalse%
\ {\isacharparenleft}{\kern0pt}typecheck{\isacharunderscore}{\kern0pt}cfuncs{\isacharcomma}{\kern0pt}\ smt{\isacharparenright}{\kern0pt}\isanewline
\ \ \ \ \isacommand{also}\isamarkupfalse%
\ \isacommand{have}\isamarkupfalse%
\ {\isachardoublequoteopen}{\isachardot}{\kern0pt}{\isachardot}{\kern0pt}{\isachardot}{\kern0pt}\ {\isacharequal}{\kern0pt}\ {\isasymlangle}{\isasymt}{\isacharcomma}{\kern0pt}{\isasymf}{\isasymrangle}{\isachardoublequoteclose}\isanewline
\ \ \ \ \ \ \isacommand{by}\isamarkupfalse%
\ {\isacharparenleft}{\kern0pt}typecheck{\isacharunderscore}{\kern0pt}cfuncs{\isacharcomma}{\kern0pt}\ simp\ add{\isacharcolon}{\kern0pt}\ left{\isacharunderscore}{\kern0pt}coproj{\isacharunderscore}{\kern0pt}cfunc{\isacharunderscore}{\kern0pt}coprod{\isacharparenright}{\kern0pt}\isanewline
\ \ \ \ \isacommand{then}\isamarkupfalse%
\ \isacommand{show}\isamarkupfalse%
\ {\isacharquery}{\kern0pt}thesis\isanewline
\ \ \ \ \ \ \isacommand{by}\isamarkupfalse%
\ {\isacharparenleft}{\kern0pt}simp\ add{\isacharcolon}{\kern0pt}\ calculation{\isacharparenright}{\kern0pt}\isanewline
\ \ \isacommand{qed}\isamarkupfalse%
\isanewline
\ \ \isacommand{have}\isamarkupfalse%
\ f{\isadigit{3}}{\isacharcolon}{\kern0pt}\ {\isachardoublequoteopen}{\isasymlangle}{\isasymf}{\isacharcomma}{\kern0pt}\ {\isasymf}{\isasymrangle}\ {\isasymamalg}\ {\isasymlangle}{\isasymt}{\isacharcomma}{\kern0pt}{\isasymf}{\isasymrangle}\ {\isasymamalg}\ {\isasymlangle}{\isasymf}{\isacharcomma}{\kern0pt}{\isasymt}{\isasymrangle}\ {\isasymcirc}\isactrlsub c\ {\isacharparenleft}{\kern0pt}right{\isacharunderscore}{\kern0pt}coproj\ {\isasymone}\ {\isacharparenleft}{\kern0pt}{\isasymone}{\isasymCoprod}{\isasymone}{\isacharparenright}{\kern0pt}{\isasymcirc}\isactrlsub c\ right{\isacharunderscore}{\kern0pt}coproj\ {\isasymone}\ {\isasymone}{\isacharparenright}{\kern0pt}\ {\isacharequal}{\kern0pt}\ {\isasymlangle}{\isasymf}{\isacharcomma}{\kern0pt}{\isasymt}{\isasymrangle}{\isachardoublequoteclose}\isanewline
\ \ \isacommand{proof}\isamarkupfalse%
{\isacharminus}{\kern0pt}\ \isanewline
\ \ \ \ \isacommand{have}\isamarkupfalse%
\ {\isachardoublequoteopen}{\isasymlangle}{\isasymf}{\isacharcomma}{\kern0pt}\ {\isasymf}{\isasymrangle}\ {\isasymamalg}\ {\isasymlangle}{\isasymt}{\isacharcomma}{\kern0pt}{\isasymf}{\isasymrangle}\ {\isasymamalg}\ {\isasymlangle}{\isasymf}{\isacharcomma}{\kern0pt}{\isasymt}{\isasymrangle}\ {\isasymcirc}\isactrlsub c\ {\isacharparenleft}{\kern0pt}right{\isacharunderscore}{\kern0pt}coproj\ {\isasymone}\ {\isacharparenleft}{\kern0pt}{\isasymone}{\isasymCoprod}{\isasymone}{\isacharparenright}{\kern0pt}{\isasymcirc}\isactrlsub c\ right{\isacharunderscore}{\kern0pt}coproj\ {\isasymone}\ {\isasymone}{\isacharparenright}{\kern0pt}\ {\isacharequal}{\kern0pt}\ \isanewline
\ \ \ \ \ \ \ \ \ \ {\isacharparenleft}{\kern0pt}{\isasymlangle}{\isasymf}{\isacharcomma}{\kern0pt}\ {\isasymf}{\isasymrangle}\ {\isasymamalg}\ {\isasymlangle}{\isasymt}{\isacharcomma}{\kern0pt}{\isasymf}{\isasymrangle}\ {\isasymamalg}\ {\isasymlangle}{\isasymf}{\isacharcomma}{\kern0pt}{\isasymt}{\isasymrangle}\ {\isasymcirc}\isactrlsub c\ right{\isacharunderscore}{\kern0pt}coproj\ {\isasymone}\ {\isacharparenleft}{\kern0pt}{\isasymone}{\isasymCoprod}{\isasymone}{\isacharparenright}{\kern0pt}\ {\isacharparenright}{\kern0pt}{\isasymcirc}\isactrlsub c\ right{\isacharunderscore}{\kern0pt}coproj\ {\isasymone}\ {\isasymone}{\isachardoublequoteclose}\isanewline
\ \ \ \ \ \ \isacommand{by}\isamarkupfalse%
\ {\isacharparenleft}{\kern0pt}typecheck{\isacharunderscore}{\kern0pt}cfuncs{\isacharcomma}{\kern0pt}\ simp\ add{\isacharcolon}{\kern0pt}\ comp{\isacharunderscore}{\kern0pt}associative{\isadigit{2}}{\isacharparenright}{\kern0pt}\isanewline
\ \ \ \ \isacommand{also}\isamarkupfalse%
\ \isacommand{have}\isamarkupfalse%
\ {\isachardoublequoteopen}{\isachardot}{\kern0pt}{\isachardot}{\kern0pt}{\isachardot}{\kern0pt}\ {\isacharequal}{\kern0pt}\ {\isasymlangle}{\isasymt}{\isacharcomma}{\kern0pt}{\isasymf}{\isasymrangle}\ {\isasymamalg}\ {\isasymlangle}{\isasymf}{\isacharcomma}{\kern0pt}{\isasymt}{\isasymrangle}\ {\isasymcirc}\isactrlsub c\ right{\isacharunderscore}{\kern0pt}coproj\ {\isasymone}\ {\isasymone}{\isachardoublequoteclose}\isanewline
\ \ \ \ \ \ \isacommand{using}\isamarkupfalse%
\ right{\isacharunderscore}{\kern0pt}coproj{\isacharunderscore}{\kern0pt}cfunc{\isacharunderscore}{\kern0pt}coprod\ \isacommand{by}\isamarkupfalse%
\ {\isacharparenleft}{\kern0pt}typecheck{\isacharunderscore}{\kern0pt}cfuncs{\isacharcomma}{\kern0pt}\ smt{\isacharparenright}{\kern0pt}\isanewline
\ \ \ \ \isacommand{also}\isamarkupfalse%
\ \isacommand{have}\isamarkupfalse%
\ {\isachardoublequoteopen}{\isachardot}{\kern0pt}{\isachardot}{\kern0pt}{\isachardot}{\kern0pt}\ {\isacharequal}{\kern0pt}\ {\isasymlangle}{\isasymf}{\isacharcomma}{\kern0pt}{\isasymt}{\isasymrangle}{\isachardoublequoteclose}\isanewline
\ \ \ \ \ \ \isacommand{by}\isamarkupfalse%
\ {\isacharparenleft}{\kern0pt}typecheck{\isacharunderscore}{\kern0pt}cfuncs{\isacharcomma}{\kern0pt}\ simp\ add{\isacharcolon}{\kern0pt}\ right{\isacharunderscore}{\kern0pt}coproj{\isacharunderscore}{\kern0pt}cfunc{\isacharunderscore}{\kern0pt}coprod{\isacharparenright}{\kern0pt}\isanewline
\ \ \ \ \isacommand{then}\isamarkupfalse%
\ \isacommand{show}\isamarkupfalse%
\ {\isacharquery}{\kern0pt}thesis\isanewline
\ \ \ \ \ \ \isacommand{by}\isamarkupfalse%
\ {\isacharparenleft}{\kern0pt}simp\ add{\isacharcolon}{\kern0pt}\ calculation{\isacharparenright}{\kern0pt}\isanewline
\ \ \isacommand{qed}\isamarkupfalse%
\isanewline
\ \ \isacommand{show}\isamarkupfalse%
\ {\isachardoublequoteopen}x\ {\isacharequal}{\kern0pt}\ y{\isachardoublequoteclose}\isanewline
\ \ \isacommand{proof}\isamarkupfalse%
{\isacharparenleft}{\kern0pt}cases\ {\isachardoublequoteopen}x\ {\isacharequal}{\kern0pt}\ left{\isacharunderscore}{\kern0pt}coproj\ {\isasymone}\ {\isacharparenleft}{\kern0pt}{\isasymone}\ {\isasymCoprod}\ {\isasymone}{\isacharparenright}{\kern0pt}{\isachardoublequoteclose}{\isacharparenright}{\kern0pt}\isanewline
\ \ \ \ \isacommand{assume}\isamarkupfalse%
\ case{\isadigit{1}}{\isacharcolon}{\kern0pt}\ {\isachardoublequoteopen}x\ {\isacharequal}{\kern0pt}\ left{\isacharunderscore}{\kern0pt}coproj\ {\isasymone}\ {\isacharparenleft}{\kern0pt}{\isasymone}\ {\isasymCoprod}\ {\isasymone}{\isacharparenright}{\kern0pt}{\isachardoublequoteclose}\isanewline
\ \ \ \ \isacommand{then}\isamarkupfalse%
\ \isacommand{show}\isamarkupfalse%
\ {\isachardoublequoteopen}x\ {\isacharequal}{\kern0pt}\ y{\isachardoublequoteclose}\isanewline
\ \ \ \ \ \ \isacommand{by}\isamarkupfalse%
\ {\isacharparenleft}{\kern0pt}typecheck{\isacharunderscore}{\kern0pt}cfuncs{\isacharcomma}{\kern0pt}\ smt\ {\isacharparenleft}{\kern0pt}z{\isadigit{3}}{\isacharparenright}{\kern0pt}\ mx{\isacharunderscore}{\kern0pt}eqs{\isacharunderscore}{\kern0pt}my\ element{\isacharunderscore}{\kern0pt}pair{\isacharunderscore}{\kern0pt}eq\ f{\isadigit{1}}\ f{\isadigit{2}}\ f{\isadigit{3}}\ false{\isacharunderscore}{\kern0pt}func{\isacharunderscore}{\kern0pt}type\ maps{\isacharunderscore}{\kern0pt}into{\isacharunderscore}{\kern0pt}{\isadigit{1}}u{\isadigit{1}}\ terminal{\isacharunderscore}{\kern0pt}func{\isacharunderscore}{\kern0pt}unique\ true{\isacharunderscore}{\kern0pt}false{\isacharunderscore}{\kern0pt}distinct\ true{\isacharunderscore}{\kern0pt}func{\isacharunderscore}{\kern0pt}type\ x{\isacharunderscore}{\kern0pt}form\ y{\isacharunderscore}{\kern0pt}form{\isacharparenright}{\kern0pt}\isanewline
\ \ \isacommand{next}\isamarkupfalse%
\isanewline
\ \ \ \ \isacommand{assume}\isamarkupfalse%
\ not{\isacharunderscore}{\kern0pt}case{\isadigit{1}}{\isacharcolon}{\kern0pt}\ {\isachardoublequoteopen}x\ {\isasymnoteq}\ left{\isacharunderscore}{\kern0pt}coproj\ {\isasymone}\ {\isacharparenleft}{\kern0pt}{\isasymone}\ {\isasymCoprod}\ {\isasymone}{\isacharparenright}{\kern0pt}{\isachardoublequoteclose}\isanewline
\ \ \ \ \isacommand{then}\isamarkupfalse%
\ \isacommand{have}\isamarkupfalse%
\ case{\isadigit{2}}{\isacharunderscore}{\kern0pt}or{\isacharunderscore}{\kern0pt}{\isadigit{3}}{\isacharcolon}{\kern0pt}\ {\isachardoublequoteopen}x\ {\isacharequal}{\kern0pt}\ right{\isacharunderscore}{\kern0pt}coproj\ {\isasymone}\ {\isacharparenleft}{\kern0pt}{\isasymone}{\isasymCoprod}{\isasymone}{\isacharparenright}{\kern0pt}{\isasymcirc}\isactrlsub c\ left{\isacharunderscore}{\kern0pt}coproj\ {\isasymone}\ {\isasymone}\ {\isasymor}\ \isanewline
\ \ \ \ \ \ \ \ \ \ \ \ \ \ \ x\ {\isacharequal}{\kern0pt}\ right{\isacharunderscore}{\kern0pt}coproj\ {\isasymone}\ {\isacharparenleft}{\kern0pt}{\isasymone}{\isasymCoprod}{\isasymone}{\isacharparenright}{\kern0pt}\ {\isasymcirc}\isactrlsub c\ right{\isacharunderscore}{\kern0pt}coproj\ {\isasymone}\ {\isasymone}{\isachardoublequoteclose}\isanewline
\ \ \ \ \ \ \isacommand{by}\isamarkupfalse%
\ {\isacharparenleft}{\kern0pt}metis\ id{\isacharunderscore}{\kern0pt}right{\isacharunderscore}{\kern0pt}unit{\isadigit{2}}\ id{\isacharunderscore}{\kern0pt}type\ left{\isacharunderscore}{\kern0pt}proj{\isacharunderscore}{\kern0pt}type\ maps{\isacharunderscore}{\kern0pt}into{\isacharunderscore}{\kern0pt}{\isadigit{1}}u{\isadigit{1}}\ terminal{\isacharunderscore}{\kern0pt}func{\isacharunderscore}{\kern0pt}unique\ x{\isacharunderscore}{\kern0pt}form{\isacharparenright}{\kern0pt}\isanewline
\ \ \ \ \isacommand{show}\isamarkupfalse%
\ {\isachardoublequoteopen}x\ {\isacharequal}{\kern0pt}\ y{\isachardoublequoteclose}\isanewline
\ \ \ \ \isacommand{proof}\isamarkupfalse%
{\isacharparenleft}{\kern0pt}cases\ {\isachardoublequoteopen}x\ {\isacharequal}{\kern0pt}\ right{\isacharunderscore}{\kern0pt}coproj\ {\isasymone}\ {\isacharparenleft}{\kern0pt}{\isasymone}{\isasymCoprod}{\isasymone}{\isacharparenright}{\kern0pt}{\isasymcirc}\isactrlsub c\ left{\isacharunderscore}{\kern0pt}coproj\ {\isasymone}\ {\isasymone}{\isachardoublequoteclose}{\isacharparenright}{\kern0pt}\isanewline
\ \ \ \ \ \ \isacommand{assume}\isamarkupfalse%
\ case{\isadigit{2}}{\isacharcolon}{\kern0pt}\ {\isachardoublequoteopen}x\ {\isacharequal}{\kern0pt}\ right{\isacharunderscore}{\kern0pt}coproj\ {\isasymone}\ {\isacharparenleft}{\kern0pt}{\isasymone}\ {\isasymCoprod}\ {\isasymone}{\isacharparenright}{\kern0pt}\ {\isasymcirc}\isactrlsub c\ left{\isacharunderscore}{\kern0pt}coproj\ {\isasymone}\ {\isasymone}{\isachardoublequoteclose}\isanewline
\ \ \ \ \ \ \isacommand{then}\isamarkupfalse%
\ \isacommand{show}\isamarkupfalse%
\ {\isachardoublequoteopen}x\ {\isacharequal}{\kern0pt}\ y{\isachardoublequoteclose}\isanewline
\ \ \ \ \ \ \ \ \isacommand{by}\isamarkupfalse%
\ {\isacharparenleft}{\kern0pt}smt\ {\isacharparenleft}{\kern0pt}z{\isadigit{3}}{\isacharparenright}{\kern0pt}\ NOT{\isacharunderscore}{\kern0pt}false{\isacharunderscore}{\kern0pt}is{\isacharunderscore}{\kern0pt}true\ NOT{\isacharunderscore}{\kern0pt}is{\isacharunderscore}{\kern0pt}pullback\ NOT{\isacharunderscore}{\kern0pt}true{\isacharunderscore}{\kern0pt}is{\isacharunderscore}{\kern0pt}false\ NOT{\isacharunderscore}{\kern0pt}type\ x{\isacharunderscore}{\kern0pt}type\ x{\isacharunderscore}{\kern0pt}type{\isacharprime}{\kern0pt}\ cart{\isacharunderscore}{\kern0pt}prod{\isacharunderscore}{\kern0pt}eq{\isadigit{2}}\ case{\isadigit{2}}\ cfunc{\isacharunderscore}{\kern0pt}type{\isacharunderscore}{\kern0pt}def\ characteristic{\isacharunderscore}{\kern0pt}func{\isacharunderscore}{\kern0pt}eq\ characteristic{\isacharunderscore}{\kern0pt}func{\isacharunderscore}{\kern0pt}is{\isacharunderscore}{\kern0pt}pullback\ characteristic{\isacharunderscore}{\kern0pt}function{\isacharunderscore}{\kern0pt}exists\ comp{\isacharunderscore}{\kern0pt}associative\ diag{\isacharunderscore}{\kern0pt}on{\isacharunderscore}{\kern0pt}elements\ diagonal{\isacharunderscore}{\kern0pt}type\ element{\isacharunderscore}{\kern0pt}monomorphism\ f{\isadigit{1}}\ f{\isadigit{2}}\ f{\isadigit{3}}\ false{\isacharunderscore}{\kern0pt}func{\isacharunderscore}{\kern0pt}type\ left{\isacharunderscore}{\kern0pt}proj{\isacharunderscore}{\kern0pt}type\ maps{\isacharunderscore}{\kern0pt}into{\isacharunderscore}{\kern0pt}{\isadigit{1}}u{\isadigit{1}}\ mx{\isacharunderscore}{\kern0pt}eqs{\isacharunderscore}{\kern0pt}my\ terminal{\isacharunderscore}{\kern0pt}func{\isacharunderscore}{\kern0pt}unique\ true{\isacharunderscore}{\kern0pt}false{\isacharunderscore}{\kern0pt}distinct\ true{\isacharunderscore}{\kern0pt}func{\isacharunderscore}{\kern0pt}type\ x{\isacharunderscore}{\kern0pt}type\ y{\isacharunderscore}{\kern0pt}form{\isacharparenright}{\kern0pt}\isanewline
\ \ \ \ \isacommand{next}\isamarkupfalse%
\isanewline
\ \ \ \ \ \ \isacommand{assume}\isamarkupfalse%
\ not{\isacharunderscore}{\kern0pt}case{\isadigit{2}}{\isacharcolon}{\kern0pt}\ {\isachardoublequoteopen}x\ {\isasymnoteq}\ right{\isacharunderscore}{\kern0pt}coproj\ {\isasymone}\ {\isacharparenleft}{\kern0pt}{\isasymone}\ {\isasymCoprod}\ {\isasymone}{\isacharparenright}{\kern0pt}\ {\isasymcirc}\isactrlsub c\ left{\isacharunderscore}{\kern0pt}coproj\ {\isasymone}\ {\isasymone}{\isachardoublequoteclose}\isanewline
\ \ \ \ \ \ \isacommand{then}\isamarkupfalse%
\ \isacommand{have}\isamarkupfalse%
\ case{\isadigit{3}}{\isacharcolon}{\kern0pt}\ {\isachardoublequoteopen}x\ {\isacharequal}{\kern0pt}\ right{\isacharunderscore}{\kern0pt}coproj\ {\isasymone}\ {\isacharparenleft}{\kern0pt}{\isasymone}{\isasymCoprod}{\isasymone}{\isacharparenright}{\kern0pt}\ {\isasymcirc}\isactrlsub c\ right{\isacharunderscore}{\kern0pt}coproj\ {\isasymone}\ {\isasymone}{\isachardoublequoteclose}\isanewline
\ \ \ \ \ \ \ \ \isacommand{using}\isamarkupfalse%
\ case{\isadigit{2}}{\isacharunderscore}{\kern0pt}or{\isacharunderscore}{\kern0pt}{\isadigit{3}}\ \isacommand{by}\isamarkupfalse%
\ blast\isanewline
\ \ \ \ \ \ \isacommand{then}\isamarkupfalse%
\ \isacommand{show}\isamarkupfalse%
\ {\isachardoublequoteopen}x\ {\isacharequal}{\kern0pt}\ y{\isachardoublequoteclose}\isanewline
\ \ \ \ \ \ \ \ \isacommand{by}\isamarkupfalse%
\ {\isacharparenleft}{\kern0pt}smt\ {\isacharparenleft}{\kern0pt}z{\isadigit{3}}{\isacharparenright}{\kern0pt}\ NOT{\isacharunderscore}{\kern0pt}false{\isacharunderscore}{\kern0pt}is{\isacharunderscore}{\kern0pt}true\ NOT{\isacharunderscore}{\kern0pt}is{\isacharunderscore}{\kern0pt}pullback\ NOT{\isacharunderscore}{\kern0pt}true{\isacharunderscore}{\kern0pt}is{\isacharunderscore}{\kern0pt}false\ NOT{\isacharunderscore}{\kern0pt}type\ x{\isacharunderscore}{\kern0pt}type\ x{\isacharunderscore}{\kern0pt}type{\isacharprime}{\kern0pt}\ cart{\isacharunderscore}{\kern0pt}prod{\isacharunderscore}{\kern0pt}eq{\isadigit{2}}\ case{\isadigit{3}}\ cfunc{\isacharunderscore}{\kern0pt}type{\isacharunderscore}{\kern0pt}def\ characteristic{\isacharunderscore}{\kern0pt}func{\isacharunderscore}{\kern0pt}eq\ characteristic{\isacharunderscore}{\kern0pt}func{\isacharunderscore}{\kern0pt}is{\isacharunderscore}{\kern0pt}pullback\ characteristic{\isacharunderscore}{\kern0pt}function{\isacharunderscore}{\kern0pt}exists\ comp{\isacharunderscore}{\kern0pt}associative\ diag{\isacharunderscore}{\kern0pt}on{\isacharunderscore}{\kern0pt}elements\ diagonal{\isacharunderscore}{\kern0pt}type\ element{\isacharunderscore}{\kern0pt}monomorphism\ f{\isadigit{1}}\ f{\isadigit{2}}\ f{\isadigit{3}}\ false{\isacharunderscore}{\kern0pt}func{\isacharunderscore}{\kern0pt}type\ left{\isacharunderscore}{\kern0pt}proj{\isacharunderscore}{\kern0pt}type\ maps{\isacharunderscore}{\kern0pt}into{\isacharunderscore}{\kern0pt}{\isadigit{1}}u{\isadigit{1}}\ mx{\isacharunderscore}{\kern0pt}eqs{\isacharunderscore}{\kern0pt}my\ terminal{\isacharunderscore}{\kern0pt}func{\isacharunderscore}{\kern0pt}unique\ true{\isacharunderscore}{\kern0pt}false{\isacharunderscore}{\kern0pt}distinct\ true{\isacharunderscore}{\kern0pt}func{\isacharunderscore}{\kern0pt}type\ x{\isacharunderscore}{\kern0pt}type\ y{\isacharunderscore}{\kern0pt}form{\isacharparenright}{\kern0pt}\isanewline
\ \ \ \ \isacommand{qed}\isamarkupfalse%
\isanewline
\ \ \isacommand{qed}\isamarkupfalse%
\isanewline
\isacommand{qed}\isamarkupfalse%
%
\endisatagproof
{\isafoldproof}%
%
\isadelimproof
\isanewline
%
\endisadelimproof
\isanewline
\isacommand{lemma}\isamarkupfalse%
\ NAND{\isacharunderscore}{\kern0pt}is{\isacharunderscore}{\kern0pt}pullback{\isacharcolon}{\kern0pt}\isanewline
\ \ {\isachardoublequoteopen}is{\isacharunderscore}{\kern0pt}pullback\ {\isacharparenleft}{\kern0pt}{\isasymone}{\isasymCoprod}{\isacharparenleft}{\kern0pt}{\isasymone}{\isasymCoprod}{\isasymone}{\isacharparenright}{\kern0pt}{\isacharparenright}{\kern0pt}\ {\isasymone}\ {\isacharparenleft}{\kern0pt}{\isasymOmega}{\isasymtimes}\isactrlsub c{\isasymOmega}{\isacharparenright}{\kern0pt}\ {\isasymOmega}\ {\isacharparenleft}{\kern0pt}{\isasymbeta}\isactrlbsub {\isacharparenleft}{\kern0pt}{\isasymone}{\isasymCoprod}{\isacharparenleft}{\kern0pt}{\isasymone}{\isasymCoprod}{\isasymone}{\isacharparenright}{\kern0pt}{\isacharparenright}{\kern0pt}\isactrlesub {\isacharparenright}{\kern0pt}\ {\isasymt}\ {\isacharparenleft}{\kern0pt}{\isasymlangle}{\isasymf}{\isacharcomma}{\kern0pt}\ {\isasymf}{\isasymrangle}{\isasymamalg}\ {\isacharparenleft}{\kern0pt}{\isasymlangle}{\isasymt}{\isacharcomma}{\kern0pt}\ {\isasymf}{\isasymrangle}\ {\isasymamalg}{\isasymlangle}{\isasymf}{\isacharcomma}{\kern0pt}\ {\isasymt}{\isasymrangle}{\isacharparenright}{\kern0pt}{\isacharparenright}{\kern0pt}\ NAND{\isachardoublequoteclose}\isanewline
%
\isadelimproof
\ \ %
\endisadelimproof
%
\isatagproof
\isacommand{unfolding}\isamarkupfalse%
\ NAND{\isacharunderscore}{\kern0pt}def\isanewline
\ \ \isacommand{using}\isamarkupfalse%
\ element{\isacharunderscore}{\kern0pt}monomorphism\ characteristic{\isacharunderscore}{\kern0pt}function{\isacharunderscore}{\kern0pt}exists\isanewline
\ \ \isacommand{by}\isamarkupfalse%
\ {\isacharparenleft}{\kern0pt}typecheck{\isacharunderscore}{\kern0pt}cfuncs{\isacharcomma}{\kern0pt}\ simp\ add{\isacharcolon}{\kern0pt}\ the{\isadigit{1}}I{\isadigit{2}}\ injective{\isacharunderscore}{\kern0pt}imp{\isacharunderscore}{\kern0pt}monomorphism\ pre{\isacharunderscore}{\kern0pt}NAND{\isacharunderscore}{\kern0pt}injective{\isacharparenright}{\kern0pt}%
\endisatagproof
{\isafoldproof}%
%
\isadelimproof
\isanewline
%
\endisadelimproof
\ \ \ \ \ \ \isanewline
\isacommand{lemma}\isamarkupfalse%
\ NAND{\isacharunderscore}{\kern0pt}type{\isacharbrackleft}{\kern0pt}type{\isacharunderscore}{\kern0pt}rule{\isacharbrackright}{\kern0pt}{\isacharcolon}{\kern0pt}\isanewline
\ \ {\isachardoublequoteopen}NAND\ {\isacharcolon}{\kern0pt}\ {\isasymOmega}\ {\isasymtimes}\isactrlsub c\ {\isasymOmega}\ {\isasymrightarrow}\ {\isasymOmega}{\isachardoublequoteclose}\isanewline
%
\isadelimproof
\ \ %
\endisadelimproof
%
\isatagproof
\isacommand{unfolding}\isamarkupfalse%
\ NAND{\isacharunderscore}{\kern0pt}def\isanewline
\ \ \isacommand{by}\isamarkupfalse%
\ {\isacharparenleft}{\kern0pt}metis\ NAND{\isacharunderscore}{\kern0pt}def\ NAND{\isacharunderscore}{\kern0pt}is{\isacharunderscore}{\kern0pt}pullback\ is{\isacharunderscore}{\kern0pt}pullback{\isacharunderscore}{\kern0pt}def{\isacharparenright}{\kern0pt}%
\endisatagproof
{\isafoldproof}%
%
\isadelimproof
\ \isanewline
%
\endisadelimproof
\isanewline
\isacommand{lemma}\isamarkupfalse%
\ NAND{\isacharunderscore}{\kern0pt}left{\isacharunderscore}{\kern0pt}false{\isacharunderscore}{\kern0pt}is{\isacharunderscore}{\kern0pt}true{\isacharcolon}{\kern0pt}\isanewline
\ \ \isakeyword{assumes}\ {\isachardoublequoteopen}p\ {\isasymin}\isactrlsub c\ {\isasymOmega}{\isachardoublequoteclose}\isanewline
\ \ \isakeyword{shows}\ {\isachardoublequoteopen}NAND\ {\isasymcirc}\isactrlsub c\ {\isasymlangle}{\isasymf}{\isacharcomma}{\kern0pt}p{\isasymrangle}\ {\isacharequal}{\kern0pt}\ {\isasymt}{\isachardoublequoteclose}\isanewline
%
\isadelimproof
%
\endisadelimproof
%
\isatagproof
\isacommand{proof}\isamarkupfalse%
\ {\isacharminus}{\kern0pt}\ \isanewline
\ \ \isacommand{have}\isamarkupfalse%
\ {\isachardoublequoteopen}{\isasymexists}\ j{\isachardot}{\kern0pt}\ j\ {\isasymin}\isactrlsub c\ {\isasymone}{\isasymCoprod}{\isacharparenleft}{\kern0pt}{\isasymone}{\isasymCoprod}{\isasymone}{\isacharparenright}{\kern0pt}\ {\isasymand}\ {\isacharparenleft}{\kern0pt}{\isasymlangle}{\isasymf}{\isacharcomma}{\kern0pt}\ {\isasymf}{\isasymrangle}\ {\isasymamalg}\ {\isacharparenleft}{\kern0pt}{\isasymlangle}{\isasymt}{\isacharcomma}{\kern0pt}\ {\isasymf}{\isasymrangle}\ {\isasymamalg}{\isasymlangle}{\isasymf}{\isacharcomma}{\kern0pt}\ {\isasymt}{\isasymrangle}{\isacharparenright}{\kern0pt}{\isacharparenright}{\kern0pt}\ {\isasymcirc}\isactrlsub c\ j\ \ {\isacharequal}{\kern0pt}\ {\isasymlangle}{\isasymf}{\isacharcomma}{\kern0pt}p{\isasymrangle}{\isachardoublequoteclose}\isanewline
\ \ \ \ \isacommand{by}\isamarkupfalse%
\ {\isacharparenleft}{\kern0pt}typecheck{\isacharunderscore}{\kern0pt}cfuncs{\isacharcomma}{\kern0pt}\ smt\ {\isacharparenleft}{\kern0pt}z{\isadigit{3}}{\isacharparenright}{\kern0pt}\ assms\ comp{\isacharunderscore}{\kern0pt}associative{\isadigit{2}}\ comp{\isacharunderscore}{\kern0pt}type\ left{\isacharunderscore}{\kern0pt}coproj{\isacharunderscore}{\kern0pt}cfunc{\isacharunderscore}{\kern0pt}coprod\ left{\isacharunderscore}{\kern0pt}proj{\isacharunderscore}{\kern0pt}type\ right{\isacharunderscore}{\kern0pt}coproj{\isacharunderscore}{\kern0pt}cfunc{\isacharunderscore}{\kern0pt}coprod\ right{\isacharunderscore}{\kern0pt}proj{\isacharunderscore}{\kern0pt}type\ true{\isacharunderscore}{\kern0pt}false{\isacharunderscore}{\kern0pt}only{\isacharunderscore}{\kern0pt}truth{\isacharunderscore}{\kern0pt}values{\isacharparenright}{\kern0pt}\isanewline
\ \ \isacommand{then}\isamarkupfalse%
\ \isacommand{show}\isamarkupfalse%
\ {\isacharquery}{\kern0pt}thesis\ \isanewline
\ \ \ \ \isacommand{by}\isamarkupfalse%
\ {\isacharparenleft}{\kern0pt}typecheck{\isacharunderscore}{\kern0pt}cfuncs{\isacharcomma}{\kern0pt}\ smt\ {\isacharparenleft}{\kern0pt}verit{\isacharcomma}{\kern0pt}\ ccfv{\isacharunderscore}{\kern0pt}threshold{\isacharparenright}{\kern0pt}\ NAND{\isacharunderscore}{\kern0pt}is{\isacharunderscore}{\kern0pt}pullback\ comp{\isacharunderscore}{\kern0pt}associative{\isadigit{2}}\ id{\isacharunderscore}{\kern0pt}right{\isacharunderscore}{\kern0pt}unit{\isadigit{2}}\ is{\isacharunderscore}{\kern0pt}pullback{\isacharunderscore}{\kern0pt}def\ terminal{\isacharunderscore}{\kern0pt}func{\isacharunderscore}{\kern0pt}comp{\isacharunderscore}{\kern0pt}elem{\isacharparenright}{\kern0pt}\isanewline
\isacommand{qed}\isamarkupfalse%
%
\endisatagproof
{\isafoldproof}%
%
\isadelimproof
\isanewline
%
\endisadelimproof
\isanewline
\isacommand{lemma}\isamarkupfalse%
\ NAND{\isacharunderscore}{\kern0pt}right{\isacharunderscore}{\kern0pt}false{\isacharunderscore}{\kern0pt}is{\isacharunderscore}{\kern0pt}true{\isacharcolon}{\kern0pt}\isanewline
\ \ \isakeyword{assumes}\ {\isachardoublequoteopen}p\ {\isasymin}\isactrlsub c\ {\isasymOmega}{\isachardoublequoteclose}\isanewline
\ \ \isakeyword{shows}\ {\isachardoublequoteopen}NAND\ {\isasymcirc}\isactrlsub c\ {\isasymlangle}p{\isacharcomma}{\kern0pt}{\isasymf}{\isasymrangle}\ {\isacharequal}{\kern0pt}\ {\isasymt}{\isachardoublequoteclose}\isanewline
%
\isadelimproof
%
\endisadelimproof
%
\isatagproof
\isacommand{proof}\isamarkupfalse%
\ {\isacharminus}{\kern0pt}\ \isanewline
\ \ \isacommand{have}\isamarkupfalse%
\ {\isachardoublequoteopen}{\isasymexists}\ j{\isachardot}{\kern0pt}\ j\ {\isasymin}\isactrlsub c\ {\isasymone}{\isasymCoprod}{\isacharparenleft}{\kern0pt}{\isasymone}{\isasymCoprod}{\isasymone}{\isacharparenright}{\kern0pt}\ {\isasymand}\ {\isacharparenleft}{\kern0pt}{\isasymlangle}{\isasymf}{\isacharcomma}{\kern0pt}\ {\isasymf}{\isasymrangle}{\isasymamalg}\ {\isacharparenleft}{\kern0pt}{\isasymlangle}{\isasymt}{\isacharcomma}{\kern0pt}\ {\isasymf}{\isasymrangle}\ {\isasymamalg}{\isasymlangle}{\isasymf}{\isacharcomma}{\kern0pt}\ {\isasymt}{\isasymrangle}{\isacharparenright}{\kern0pt}{\isacharparenright}{\kern0pt}\ {\isasymcirc}\isactrlsub c\ j\ \ {\isacharequal}{\kern0pt}\ {\isasymlangle}p{\isacharcomma}{\kern0pt}{\isasymf}{\isasymrangle}{\isachardoublequoteclose}\isanewline
\ \ \ \ \isacommand{by}\isamarkupfalse%
\ {\isacharparenleft}{\kern0pt}typecheck{\isacharunderscore}{\kern0pt}cfuncs{\isacharcomma}{\kern0pt}\ smt\ {\isacharparenleft}{\kern0pt}z{\isadigit{3}}{\isacharparenright}{\kern0pt}\ assms\ comp{\isacharunderscore}{\kern0pt}associative{\isadigit{2}}\ comp{\isacharunderscore}{\kern0pt}type\ left{\isacharunderscore}{\kern0pt}coproj{\isacharunderscore}{\kern0pt}cfunc{\isacharunderscore}{\kern0pt}coprod\ left{\isacharunderscore}{\kern0pt}proj{\isacharunderscore}{\kern0pt}type\ right{\isacharunderscore}{\kern0pt}coproj{\isacharunderscore}{\kern0pt}cfunc{\isacharunderscore}{\kern0pt}coprod\ right{\isacharunderscore}{\kern0pt}proj{\isacharunderscore}{\kern0pt}type\ true{\isacharunderscore}{\kern0pt}false{\isacharunderscore}{\kern0pt}only{\isacharunderscore}{\kern0pt}truth{\isacharunderscore}{\kern0pt}values{\isacharparenright}{\kern0pt}\isanewline
\ \ \isacommand{then}\isamarkupfalse%
\ \isacommand{show}\isamarkupfalse%
\ {\isacharquery}{\kern0pt}thesis\ \isanewline
\ \ \ \ \isacommand{by}\isamarkupfalse%
\ {\isacharparenleft}{\kern0pt}typecheck{\isacharunderscore}{\kern0pt}cfuncs{\isacharcomma}{\kern0pt}\ smt\ {\isacharparenleft}{\kern0pt}verit{\isacharcomma}{\kern0pt}\ ccfv{\isacharunderscore}{\kern0pt}SIG{\isacharparenright}{\kern0pt}\ NAND{\isacharunderscore}{\kern0pt}is{\isacharunderscore}{\kern0pt}pullback\ NOT{\isacharunderscore}{\kern0pt}false{\isacharunderscore}{\kern0pt}is{\isacharunderscore}{\kern0pt}true\ NOT{\isacharunderscore}{\kern0pt}is{\isacharunderscore}{\kern0pt}pullback\ \ comp{\isacharunderscore}{\kern0pt}associative{\isadigit{2}}\ is{\isacharunderscore}{\kern0pt}pullback{\isacharunderscore}{\kern0pt}def\ \ terminal{\isacharunderscore}{\kern0pt}func{\isacharunderscore}{\kern0pt}comp{\isacharparenright}{\kern0pt}\isanewline
\isacommand{qed}\isamarkupfalse%
%
\endisatagproof
{\isafoldproof}%
%
\isadelimproof
\isanewline
%
\endisadelimproof
\isanewline
\isacommand{lemma}\isamarkupfalse%
\ NAND{\isacharunderscore}{\kern0pt}true{\isacharunderscore}{\kern0pt}true{\isacharunderscore}{\kern0pt}is{\isacharunderscore}{\kern0pt}false{\isacharcolon}{\kern0pt}\isanewline
\ {\isachardoublequoteopen}NAND\ {\isasymcirc}\isactrlsub c\ {\isasymlangle}{\isasymt}{\isacharcomma}{\kern0pt}{\isasymt}{\isasymrangle}\ {\isacharequal}{\kern0pt}\ {\isasymf}{\isachardoublequoteclose}\isanewline
%
\isadelimproof
%
\endisadelimproof
%
\isatagproof
\isacommand{proof}\isamarkupfalse%
{\isacharparenleft}{\kern0pt}rule\ ccontr{\isacharparenright}{\kern0pt}\isanewline
\ \ \isacommand{assume}\isamarkupfalse%
\ {\isachardoublequoteopen}NAND\ {\isasymcirc}\isactrlsub c\ {\isasymlangle}{\isasymt}{\isacharcomma}{\kern0pt}{\isasymt}{\isasymrangle}\ {\isasymnoteq}\ {\isasymf}{\isachardoublequoteclose}\isanewline
\ \ \isacommand{then}\isamarkupfalse%
\ \isacommand{have}\isamarkupfalse%
\ {\isachardoublequoteopen}NAND\ {\isasymcirc}\isactrlsub c\ {\isasymlangle}{\isasymt}{\isacharcomma}{\kern0pt}{\isasymt}{\isasymrangle}\ {\isacharequal}{\kern0pt}\ {\isasymt}{\isachardoublequoteclose}\isanewline
\ \ \ \ \isacommand{using}\isamarkupfalse%
\ \ true{\isacharunderscore}{\kern0pt}false{\isacharunderscore}{\kern0pt}only{\isacharunderscore}{\kern0pt}truth{\isacharunderscore}{\kern0pt}values\ \isacommand{by}\isamarkupfalse%
\ {\isacharparenleft}{\kern0pt}typecheck{\isacharunderscore}{\kern0pt}cfuncs{\isacharcomma}{\kern0pt}\ blast{\isacharparenright}{\kern0pt}\isanewline
\ \ \isacommand{then}\isamarkupfalse%
\ \isacommand{obtain}\isamarkupfalse%
\ j\ \isakeyword{where}\ j{\isacharunderscore}{\kern0pt}type{\isacharbrackleft}{\kern0pt}type{\isacharunderscore}{\kern0pt}rule{\isacharbrackright}{\kern0pt}{\isacharcolon}{\kern0pt}\ \ {\isachardoublequoteopen}j\ {\isasymin}\isactrlsub c\ {\isasymone}{\isasymCoprod}{\isacharparenleft}{\kern0pt}{\isasymone}{\isasymCoprod}{\isasymone}{\isacharparenright}{\kern0pt}{\isachardoublequoteclose}\ \isakeyword{and}\ j{\isacharunderscore}{\kern0pt}def{\isacharcolon}{\kern0pt}\ {\isachardoublequoteopen}{\isacharparenleft}{\kern0pt}{\isasymlangle}{\isasymf}{\isacharcomma}{\kern0pt}\ {\isasymf}{\isasymrangle}{\isasymamalg}\ {\isacharparenleft}{\kern0pt}{\isasymlangle}{\isasymt}{\isacharcomma}{\kern0pt}\ {\isasymf}{\isasymrangle}\ {\isasymamalg}{\isasymlangle}{\isasymf}{\isacharcomma}{\kern0pt}\ {\isasymt}{\isasymrangle}{\isacharparenright}{\kern0pt}{\isacharparenright}{\kern0pt}\ {\isasymcirc}\isactrlsub c\ j\ \ {\isacharequal}{\kern0pt}\ {\isasymlangle}{\isasymt}{\isacharcomma}{\kern0pt}{\isasymt}{\isasymrangle}{\isachardoublequoteclose}\isanewline
\ \ \ \ \isacommand{using}\isamarkupfalse%
\ NAND{\isacharunderscore}{\kern0pt}is{\isacharunderscore}{\kern0pt}pullback\ \isacommand{unfolding}\isamarkupfalse%
\ is{\isacharunderscore}{\kern0pt}pullback{\isacharunderscore}{\kern0pt}def\isanewline
\ \ \ \ \isacommand{by}\isamarkupfalse%
\ {\isacharparenleft}{\kern0pt}typecheck{\isacharunderscore}{\kern0pt}cfuncs{\isacharcomma}{\kern0pt}\ smt\ {\isacharparenleft}{\kern0pt}z{\isadigit{3}}{\isacharparenright}{\kern0pt}\ NAND{\isacharunderscore}{\kern0pt}is{\isacharunderscore}{\kern0pt}pullback\ id{\isacharunderscore}{\kern0pt}right{\isacharunderscore}{\kern0pt}unit{\isadigit{2}}\ id{\isacharunderscore}{\kern0pt}type{\isacharparenright}{\kern0pt}\isanewline
\ \ \isacommand{then}\isamarkupfalse%
\ \isacommand{have}\isamarkupfalse%
\ trichotomy{\isacharcolon}{\kern0pt}\ {\isachardoublequoteopen}{\isacharparenleft}{\kern0pt}{\isasymlangle}{\isasymf}{\isacharcomma}{\kern0pt}{\isasymf}{\isasymrangle}\ {\isacharequal}{\kern0pt}\ {\isasymlangle}{\isasymt}{\isacharcomma}{\kern0pt}{\isasymt}{\isasymrangle}{\isacharparenright}{\kern0pt}\ {\isasymor}\ {\isacharparenleft}{\kern0pt}{\isasymlangle}{\isasymt}{\isacharcomma}{\kern0pt}\ {\isasymf}{\isasymrangle}\ {\isacharequal}{\kern0pt}\ {\isasymlangle}{\isasymt}{\isacharcomma}{\kern0pt}{\isasymt}{\isasymrangle}{\isacharparenright}{\kern0pt}\ {\isasymor}\ {\isacharparenleft}{\kern0pt}{\isasymlangle}{\isasymf}{\isacharcomma}{\kern0pt}\ {\isasymt}{\isasymrangle}\ {\isacharequal}{\kern0pt}\ {\isasymlangle}{\isasymt}{\isacharcomma}{\kern0pt}{\isasymt}{\isasymrangle}{\isacharparenright}{\kern0pt}{\isachardoublequoteclose}\isanewline
\ \ \isacommand{proof}\isamarkupfalse%
{\isacharparenleft}{\kern0pt}cases\ {\isachardoublequoteopen}j\ {\isacharequal}{\kern0pt}\ left{\isacharunderscore}{\kern0pt}coproj\ {\isasymone}\ {\isacharparenleft}{\kern0pt}{\isasymone}\ {\isasymCoprod}\ {\isasymone}{\isacharparenright}{\kern0pt}{\isachardoublequoteclose}{\isacharparenright}{\kern0pt}\isanewline
\ \ \ \ \isacommand{assume}\isamarkupfalse%
\ case{\isadigit{1}}{\isacharcolon}{\kern0pt}\ {\isachardoublequoteopen}j\ {\isacharequal}{\kern0pt}\ left{\isacharunderscore}{\kern0pt}coproj\ {\isasymone}\ {\isacharparenleft}{\kern0pt}{\isasymone}\ {\isasymCoprod}\ {\isasymone}{\isacharparenright}{\kern0pt}{\isachardoublequoteclose}\isanewline
\ \ \ \ \isacommand{then}\isamarkupfalse%
\ \isacommand{show}\isamarkupfalse%
\ {\isacharquery}{\kern0pt}thesis\isanewline
\ \ \ \ \ \ \isacommand{by}\isamarkupfalse%
\ {\isacharparenleft}{\kern0pt}metis\ cfunc{\isacharunderscore}{\kern0pt}coprod{\isacharunderscore}{\kern0pt}type\ cfunc{\isacharunderscore}{\kern0pt}prod{\isacharunderscore}{\kern0pt}type\ false{\isacharunderscore}{\kern0pt}func{\isacharunderscore}{\kern0pt}type\ j{\isacharunderscore}{\kern0pt}def\ left{\isacharunderscore}{\kern0pt}coproj{\isacharunderscore}{\kern0pt}cfunc{\isacharunderscore}{\kern0pt}coprod\ true{\isacharunderscore}{\kern0pt}func{\isacharunderscore}{\kern0pt}type{\isacharparenright}{\kern0pt}\isanewline
\ \ \isacommand{next}\isamarkupfalse%
\isanewline
\ \ \ \ \isacommand{assume}\isamarkupfalse%
\ not{\isacharunderscore}{\kern0pt}case{\isadigit{1}}{\isacharcolon}{\kern0pt}\ {\isachardoublequoteopen}j\ {\isasymnoteq}\ left{\isacharunderscore}{\kern0pt}coproj\ {\isasymone}\ {\isacharparenleft}{\kern0pt}{\isasymone}\ {\isasymCoprod}\ {\isasymone}{\isacharparenright}{\kern0pt}{\isachardoublequoteclose}\isanewline
\ \ \ \ \isacommand{then}\isamarkupfalse%
\ \isacommand{have}\isamarkupfalse%
\ case{\isadigit{2}}{\isacharunderscore}{\kern0pt}or{\isacharunderscore}{\kern0pt}{\isadigit{3}}{\isacharcolon}{\kern0pt}\ {\isachardoublequoteopen}j\ {\isacharequal}{\kern0pt}\ right{\isacharunderscore}{\kern0pt}coproj\ {\isasymone}\ {\isacharparenleft}{\kern0pt}{\isasymone}{\isasymCoprod}{\isasymone}{\isacharparenright}{\kern0pt}{\isasymcirc}\isactrlsub c\ left{\isacharunderscore}{\kern0pt}coproj\ {\isasymone}\ {\isasymone}\ {\isasymor}\ \isanewline
\ \ \ \ \ \ \ \ \ \ \ \ \ \ \ j\ {\isacharequal}{\kern0pt}\ right{\isacharunderscore}{\kern0pt}coproj\ {\isasymone}\ {\isacharparenleft}{\kern0pt}{\isasymone}{\isasymCoprod}{\isasymone}{\isacharparenright}{\kern0pt}\ {\isasymcirc}\isactrlsub c\ right{\isacharunderscore}{\kern0pt}coproj\ {\isasymone}\ {\isasymone}{\isachardoublequoteclose}\isanewline
\ \ \ \ \ \ \isacommand{using}\isamarkupfalse%
\ not{\isacharunderscore}{\kern0pt}case{\isadigit{1}}\ set{\isacharunderscore}{\kern0pt}three\ \isacommand{by}\isamarkupfalse%
\ {\isacharparenleft}{\kern0pt}typecheck{\isacharunderscore}{\kern0pt}cfuncs{\isacharcomma}{\kern0pt}\ auto{\isacharparenright}{\kern0pt}\isanewline
\ \ \ \ \isacommand{show}\isamarkupfalse%
\ {\isacharquery}{\kern0pt}thesis\isanewline
\ \ \ \ \isacommand{proof}\isamarkupfalse%
{\isacharparenleft}{\kern0pt}cases\ {\isachardoublequoteopen}j\ {\isacharequal}{\kern0pt}\ right{\isacharunderscore}{\kern0pt}coproj\ {\isasymone}\ {\isacharparenleft}{\kern0pt}{\isasymone}{\isasymCoprod}{\isasymone}{\isacharparenright}{\kern0pt}\ {\isasymcirc}\isactrlsub c\ left{\isacharunderscore}{\kern0pt}coproj\ {\isasymone}\ {\isasymone}{\isachardoublequoteclose}{\isacharparenright}{\kern0pt}\isanewline
\ \ \ \ \ \ \isacommand{assume}\isamarkupfalse%
\ case{\isadigit{2}}{\isacharcolon}{\kern0pt}\ {\isachardoublequoteopen}j\ {\isacharequal}{\kern0pt}\ right{\isacharunderscore}{\kern0pt}coproj\ {\isasymone}\ {\isacharparenleft}{\kern0pt}{\isasymone}\ {\isasymCoprod}\ {\isasymone}{\isacharparenright}{\kern0pt}\ {\isasymcirc}\isactrlsub c\ left{\isacharunderscore}{\kern0pt}coproj\ {\isasymone}\ {\isasymone}{\isachardoublequoteclose}\isanewline
\ \ \ \ \ \ \isacommand{have}\isamarkupfalse%
\ {\isachardoublequoteopen}{\isasymlangle}{\isasymt}{\isacharcomma}{\kern0pt}\ {\isasymf}{\isasymrangle}\ {\isacharequal}{\kern0pt}\ {\isasymlangle}{\isasymt}{\isacharcomma}{\kern0pt}{\isasymt}{\isasymrangle}{\isachardoublequoteclose}\isanewline
\ \ \ \ \ \ \isacommand{proof}\isamarkupfalse%
\ {\isacharminus}{\kern0pt}\ \isanewline
\ \ \ \ \ \ \ \ \isacommand{have}\isamarkupfalse%
\ {\isachardoublequoteopen}{\isacharparenleft}{\kern0pt}{\isasymlangle}{\isasymf}{\isacharcomma}{\kern0pt}\ {\isasymf}{\isasymrangle}{\isasymamalg}\ {\isacharparenleft}{\kern0pt}{\isasymlangle}{\isasymt}{\isacharcomma}{\kern0pt}\ {\isasymf}{\isasymrangle}\ {\isasymamalg}{\isasymlangle}{\isasymf}{\isacharcomma}{\kern0pt}\ {\isasymt}{\isasymrangle}{\isacharparenright}{\kern0pt}{\isacharparenright}{\kern0pt}\ {\isasymcirc}\isactrlsub c\ j\ {\isacharequal}{\kern0pt}\ {\isacharparenleft}{\kern0pt}{\isacharparenleft}{\kern0pt}{\isasymlangle}{\isasymf}{\isacharcomma}{\kern0pt}\ {\isasymf}{\isasymrangle}{\isasymamalg}\ {\isacharparenleft}{\kern0pt}{\isasymlangle}{\isasymt}{\isacharcomma}{\kern0pt}\ {\isasymf}{\isasymrangle}\ {\isasymamalg}{\isasymlangle}{\isasymf}{\isacharcomma}{\kern0pt}\ {\isasymt}{\isasymrangle}{\isacharparenright}{\kern0pt}{\isacharparenright}{\kern0pt}\ {\isasymcirc}\isactrlsub c\ right{\isacharunderscore}{\kern0pt}coproj\ {\isasymone}\ {\isacharparenleft}{\kern0pt}{\isasymone}\ {\isasymCoprod}\ {\isasymone}{\isacharparenright}{\kern0pt}{\isacharparenright}{\kern0pt}\ {\isasymcirc}\isactrlsub c\ left{\isacharunderscore}{\kern0pt}coproj\ {\isasymone}\ {\isasymone}{\isachardoublequoteclose}\isanewline
\ \ \ \ \ \ \ \ \ \ \isacommand{by}\isamarkupfalse%
\ {\isacharparenleft}{\kern0pt}typecheck{\isacharunderscore}{\kern0pt}cfuncs{\isacharcomma}{\kern0pt}\ simp\ add{\isacharcolon}{\kern0pt}\ case{\isadigit{2}}\ comp{\isacharunderscore}{\kern0pt}associative{\isadigit{2}}{\isacharparenright}{\kern0pt}\isanewline
\ \ \ \ \ \ \ \ \isacommand{also}\isamarkupfalse%
\ \isacommand{have}\isamarkupfalse%
\ {\isachardoublequoteopen}{\isachardot}{\kern0pt}{\isachardot}{\kern0pt}{\isachardot}{\kern0pt}\ {\isacharequal}{\kern0pt}\ {\isacharparenleft}{\kern0pt}{\isasymlangle}{\isasymt}{\isacharcomma}{\kern0pt}\ {\isasymf}{\isasymrangle}\ {\isasymamalg}{\isasymlangle}{\isasymf}{\isacharcomma}{\kern0pt}\ {\isasymt}{\isasymrangle}{\isacharparenright}{\kern0pt}\ {\isasymcirc}\isactrlsub c\ left{\isacharunderscore}{\kern0pt}coproj\ {\isasymone}\ {\isasymone}{\isachardoublequoteclose}\isanewline
\ \ \ \ \ \ \ \ \ \ \isacommand{using}\isamarkupfalse%
\ right{\isacharunderscore}{\kern0pt}coproj{\isacharunderscore}{\kern0pt}cfunc{\isacharunderscore}{\kern0pt}coprod\ \isacommand{by}\isamarkupfalse%
\ {\isacharparenleft}{\kern0pt}typecheck{\isacharunderscore}{\kern0pt}cfuncs{\isacharcomma}{\kern0pt}\ presburger{\isacharparenright}{\kern0pt}\isanewline
\ \ \ \ \ \ \ \ \isacommand{also}\isamarkupfalse%
\ \isacommand{have}\isamarkupfalse%
\ {\isachardoublequoteopen}{\isachardot}{\kern0pt}{\isachardot}{\kern0pt}{\isachardot}{\kern0pt}\ {\isacharequal}{\kern0pt}\ {\isasymlangle}{\isasymt}{\isacharcomma}{\kern0pt}\ {\isasymf}{\isasymrangle}{\isachardoublequoteclose}\isanewline
\ \ \ \ \ \ \ \ \ \ \isacommand{by}\isamarkupfalse%
\ {\isacharparenleft}{\kern0pt}typecheck{\isacharunderscore}{\kern0pt}cfuncs{\isacharcomma}{\kern0pt}\ simp\ add{\isacharcolon}{\kern0pt}\ left{\isacharunderscore}{\kern0pt}coproj{\isacharunderscore}{\kern0pt}cfunc{\isacharunderscore}{\kern0pt}coprod{\isacharparenright}{\kern0pt}\isanewline
\ \ \ \ \ \ \ \ \isacommand{then}\isamarkupfalse%
\ \isacommand{show}\isamarkupfalse%
\ {\isacharquery}{\kern0pt}thesis\isanewline
\ \ \ \ \ \ \ \ \ \ \isacommand{using}\isamarkupfalse%
\ calculation\ j{\isacharunderscore}{\kern0pt}def\ \isacommand{by}\isamarkupfalse%
\ presburger\isanewline
\ \ \ \ \ \ \isacommand{qed}\isamarkupfalse%
\isanewline
\ \ \ \ \ \ \isacommand{then}\isamarkupfalse%
\ \isacommand{show}\isamarkupfalse%
\ {\isacharquery}{\kern0pt}thesis\isanewline
\ \ \ \ \ \ \ \ \isacommand{by}\isamarkupfalse%
\ blast\isanewline
\ \ \ \ \isacommand{next}\isamarkupfalse%
\isanewline
\ \ \ \ \ \ \isacommand{assume}\isamarkupfalse%
\ not{\isacharunderscore}{\kern0pt}case{\isadigit{2}}{\isacharcolon}{\kern0pt}\ {\isachardoublequoteopen}j\ {\isasymnoteq}\ right{\isacharunderscore}{\kern0pt}coproj\ {\isasymone}\ {\isacharparenleft}{\kern0pt}{\isasymone}\ {\isasymCoprod}\ {\isasymone}{\isacharparenright}{\kern0pt}\ {\isasymcirc}\isactrlsub c\ left{\isacharunderscore}{\kern0pt}coproj\ {\isasymone}\ {\isasymone}{\isachardoublequoteclose}\isanewline
\ \ \ \ \ \ \isacommand{then}\isamarkupfalse%
\ \isacommand{have}\isamarkupfalse%
\ case{\isadigit{3}}{\isacharcolon}{\kern0pt}\ {\isachardoublequoteopen}j\ {\isacharequal}{\kern0pt}\ right{\isacharunderscore}{\kern0pt}coproj\ {\isasymone}\ {\isacharparenleft}{\kern0pt}{\isasymone}{\isasymCoprod}{\isasymone}{\isacharparenright}{\kern0pt}\ {\isasymcirc}\isactrlsub c\ right{\isacharunderscore}{\kern0pt}coproj\ {\isasymone}\ {\isasymone}{\isachardoublequoteclose}\isanewline
\ \ \ \ \ \ \ \ \isacommand{using}\isamarkupfalse%
\ case{\isadigit{2}}{\isacharunderscore}{\kern0pt}or{\isacharunderscore}{\kern0pt}{\isadigit{3}}\ \isacommand{by}\isamarkupfalse%
\ blast\isanewline
\ \ \ \ \ \ \isacommand{have}\isamarkupfalse%
\ {\isachardoublequoteopen}{\isasymlangle}{\isasymf}{\isacharcomma}{\kern0pt}\ {\isasymt}{\isasymrangle}\ {\isacharequal}{\kern0pt}\ {\isasymlangle}{\isasymt}{\isacharcomma}{\kern0pt}{\isasymt}{\isasymrangle}{\isachardoublequoteclose}\isanewline
\ \ \ \ \ \ \isacommand{proof}\isamarkupfalse%
\ {\isacharminus}{\kern0pt}\ \isanewline
\ \ \ \ \ \ \ \ \isacommand{have}\isamarkupfalse%
\ {\isachardoublequoteopen}{\isacharparenleft}{\kern0pt}{\isasymlangle}{\isasymf}{\isacharcomma}{\kern0pt}\ {\isasymf}{\isasymrangle}{\isasymamalg}\ {\isacharparenleft}{\kern0pt}{\isasymlangle}{\isasymt}{\isacharcomma}{\kern0pt}\ {\isasymf}{\isasymrangle}\ {\isasymamalg}{\isasymlangle}{\isasymf}{\isacharcomma}{\kern0pt}\ {\isasymt}{\isasymrangle}{\isacharparenright}{\kern0pt}{\isacharparenright}{\kern0pt}\ {\isasymcirc}\isactrlsub c\ j\ {\isacharequal}{\kern0pt}\ {\isacharparenleft}{\kern0pt}{\isacharparenleft}{\kern0pt}{\isasymlangle}{\isasymf}{\isacharcomma}{\kern0pt}\ {\isasymf}{\isasymrangle}{\isasymamalg}\ {\isacharparenleft}{\kern0pt}{\isasymlangle}{\isasymt}{\isacharcomma}{\kern0pt}\ {\isasymf}{\isasymrangle}\ {\isasymamalg}{\isasymlangle}{\isasymf}{\isacharcomma}{\kern0pt}\ {\isasymt}{\isasymrangle}{\isacharparenright}{\kern0pt}{\isacharparenright}{\kern0pt}\ {\isasymcirc}\isactrlsub c\ right{\isacharunderscore}{\kern0pt}coproj\ {\isasymone}\ {\isacharparenleft}{\kern0pt}{\isasymone}\ {\isasymCoprod}\ {\isasymone}{\isacharparenright}{\kern0pt}{\isacharparenright}{\kern0pt}\ {\isasymcirc}\isactrlsub c\ right{\isacharunderscore}{\kern0pt}coproj\ {\isasymone}\ {\isasymone}{\isachardoublequoteclose}\isanewline
\ \ \ \ \ \ \ \ \ \ \isacommand{by}\isamarkupfalse%
\ {\isacharparenleft}{\kern0pt}typecheck{\isacharunderscore}{\kern0pt}cfuncs{\isacharcomma}{\kern0pt}\ simp\ add{\isacharcolon}{\kern0pt}\ case{\isadigit{3}}\ comp{\isacharunderscore}{\kern0pt}associative{\isadigit{2}}{\isacharparenright}{\kern0pt}\isanewline
\ \ \ \ \ \ \ \ \isacommand{also}\isamarkupfalse%
\ \isacommand{have}\isamarkupfalse%
\ {\isachardoublequoteopen}{\isachardot}{\kern0pt}{\isachardot}{\kern0pt}{\isachardot}{\kern0pt}\ {\isacharequal}{\kern0pt}\ {\isacharparenleft}{\kern0pt}{\isasymlangle}{\isasymt}{\isacharcomma}{\kern0pt}\ {\isasymf}{\isasymrangle}\ {\isasymamalg}{\isasymlangle}{\isasymf}{\isacharcomma}{\kern0pt}\ {\isasymt}{\isasymrangle}{\isacharparenright}{\kern0pt}\ {\isasymcirc}\isactrlsub c\ right{\isacharunderscore}{\kern0pt}coproj\ {\isasymone}\ {\isasymone}{\isachardoublequoteclose}\isanewline
\ \ \ \ \ \ \ \ \ \ \isacommand{using}\isamarkupfalse%
\ right{\isacharunderscore}{\kern0pt}coproj{\isacharunderscore}{\kern0pt}cfunc{\isacharunderscore}{\kern0pt}coprod\ \isacommand{by}\isamarkupfalse%
\ {\isacharparenleft}{\kern0pt}typecheck{\isacharunderscore}{\kern0pt}cfuncs{\isacharcomma}{\kern0pt}\ presburger{\isacharparenright}{\kern0pt}\isanewline
\ \ \ \ \ \ \ \ \isacommand{also}\isamarkupfalse%
\ \isacommand{have}\isamarkupfalse%
\ {\isachardoublequoteopen}{\isachardot}{\kern0pt}{\isachardot}{\kern0pt}{\isachardot}{\kern0pt}\ {\isacharequal}{\kern0pt}\ {\isasymlangle}{\isasymf}{\isacharcomma}{\kern0pt}\ {\isasymt}{\isasymrangle}{\isachardoublequoteclose}\isanewline
\ \ \ \ \ \ \ \ \ \ \isacommand{by}\isamarkupfalse%
\ {\isacharparenleft}{\kern0pt}typecheck{\isacharunderscore}{\kern0pt}cfuncs{\isacharcomma}{\kern0pt}\ simp\ add{\isacharcolon}{\kern0pt}\ right{\isacharunderscore}{\kern0pt}coproj{\isacharunderscore}{\kern0pt}cfunc{\isacharunderscore}{\kern0pt}coprod{\isacharparenright}{\kern0pt}\isanewline
\ \ \ \ \ \ \ \ \isacommand{then}\isamarkupfalse%
\ \isacommand{show}\isamarkupfalse%
\ {\isacharquery}{\kern0pt}thesis\isanewline
\ \ \ \ \ \ \ \ \ \ \isacommand{using}\isamarkupfalse%
\ calculation\ j{\isacharunderscore}{\kern0pt}def\ \isacommand{by}\isamarkupfalse%
\ presburger\isanewline
\ \ \ \ \ \ \isacommand{qed}\isamarkupfalse%
\isanewline
\ \ \ \ \ \ \isacommand{then}\isamarkupfalse%
\ \isacommand{show}\isamarkupfalse%
\ {\isacharquery}{\kern0pt}thesis\isanewline
\ \ \ \ \ \ \ \ \isacommand{by}\isamarkupfalse%
\ blast\isanewline
\ \ \ \ \isacommand{qed}\isamarkupfalse%
\isanewline
\ \ \isacommand{qed}\isamarkupfalse%
\isanewline
\ \ \ \ \isacommand{then}\isamarkupfalse%
\ \isacommand{have}\isamarkupfalse%
\ {\isachardoublequoteopen}{\isasymt}\ {\isacharequal}{\kern0pt}\ {\isasymf}{\isachardoublequoteclose}\isanewline
\ \ \ \ \ \ \isacommand{using}\isamarkupfalse%
\ trichotomy\ cart{\isacharunderscore}{\kern0pt}prod{\isacharunderscore}{\kern0pt}eq{\isadigit{2}}\ \isacommand{by}\isamarkupfalse%
\ {\isacharparenleft}{\kern0pt}typecheck{\isacharunderscore}{\kern0pt}cfuncs{\isacharcomma}{\kern0pt}\ force{\isacharparenright}{\kern0pt}\isanewline
\ \ \ \ \isacommand{then}\isamarkupfalse%
\ \isacommand{show}\isamarkupfalse%
\ False\isanewline
\ \ \ \ \ \ \isacommand{using}\isamarkupfalse%
\ true{\isacharunderscore}{\kern0pt}false{\isacharunderscore}{\kern0pt}distinct\ \isacommand{by}\isamarkupfalse%
\ auto\ \ \isanewline
\isacommand{qed}\isamarkupfalse%
%
\endisatagproof
{\isafoldproof}%
%
\isadelimproof
\isanewline
%
\endisadelimproof
\isanewline
\isacommand{lemma}\isamarkupfalse%
\ NAND{\isacharunderscore}{\kern0pt}true{\isacharunderscore}{\kern0pt}implies{\isacharunderscore}{\kern0pt}one{\isacharunderscore}{\kern0pt}is{\isacharunderscore}{\kern0pt}false{\isacharcolon}{\kern0pt}\isanewline
\ \ \isakeyword{assumes}\ {\isachardoublequoteopen}p\ {\isasymin}\isactrlsub c\ {\isasymOmega}{\isachardoublequoteclose}\ \isanewline
\ \ \isakeyword{assumes}\ {\isachardoublequoteopen}q\ {\isasymin}\isactrlsub c\ {\isasymOmega}{\isachardoublequoteclose}\isanewline
\ \ \isakeyword{assumes}\ {\isachardoublequoteopen}NAND\ {\isasymcirc}\isactrlsub c\ {\isasymlangle}p{\isacharcomma}{\kern0pt}q{\isasymrangle}\ {\isacharequal}{\kern0pt}\ {\isasymt}{\isachardoublequoteclose}\isanewline
\ \ \isakeyword{shows}\ {\isachardoublequoteopen}p\ {\isacharequal}{\kern0pt}\ {\isasymf}\ {\isasymor}\ q\ {\isacharequal}{\kern0pt}\ {\isasymf}{\isachardoublequoteclose}\isanewline
%
\isadelimproof
\ \ %
\endisadelimproof
%
\isatagproof
\isacommand{by}\isamarkupfalse%
\ {\isacharparenleft}{\kern0pt}metis\ {\isacharparenleft}{\kern0pt}no{\isacharunderscore}{\kern0pt}types{\isacharparenright}{\kern0pt}\ NAND{\isacharunderscore}{\kern0pt}true{\isacharunderscore}{\kern0pt}true{\isacharunderscore}{\kern0pt}is{\isacharunderscore}{\kern0pt}false\ assms\ true{\isacharunderscore}{\kern0pt}false{\isacharunderscore}{\kern0pt}only{\isacharunderscore}{\kern0pt}truth{\isacharunderscore}{\kern0pt}values{\isacharparenright}{\kern0pt}%
\endisatagproof
{\isafoldproof}%
%
\isadelimproof
\isanewline
%
\endisadelimproof
\isanewline
\isacommand{lemma}\isamarkupfalse%
\ NOT{\isacharunderscore}{\kern0pt}AND{\isacharunderscore}{\kern0pt}is{\isacharunderscore}{\kern0pt}NAND{\isacharcolon}{\kern0pt}\isanewline
\ {\isachardoublequoteopen}NAND\ {\isacharequal}{\kern0pt}\ NOT\ {\isasymcirc}\isactrlsub c\ AND{\isachardoublequoteclose}\isanewline
%
\isadelimproof
%
\endisadelimproof
%
\isatagproof
\isacommand{proof}\isamarkupfalse%
{\isacharparenleft}{\kern0pt}etcs{\isacharunderscore}{\kern0pt}rule\ one{\isacharunderscore}{\kern0pt}separator{\isacharparenright}{\kern0pt}\isanewline
\ \ \isacommand{fix}\isamarkupfalse%
\ x\ \isanewline
\ \ \isacommand{assume}\isamarkupfalse%
\ x{\isacharunderscore}{\kern0pt}type{\isacharcolon}{\kern0pt}\ {\isachardoublequoteopen}x\ {\isasymin}\isactrlsub c\ {\isasymOmega}\ {\isasymtimes}\isactrlsub c\ {\isasymOmega}{\isachardoublequoteclose}\isanewline
\ \ \isacommand{then}\isamarkupfalse%
\ \isacommand{obtain}\isamarkupfalse%
\ p\ q\ \isakeyword{where}\ x{\isacharunderscore}{\kern0pt}def{\isacharcolon}{\kern0pt}\ {\isachardoublequoteopen}p\ {\isasymin}\isactrlsub c\ {\isasymOmega}\ {\isasymand}\ q\ {\isasymin}\isactrlsub c\ {\isasymOmega}\ {\isasymand}\ x\ {\isacharequal}{\kern0pt}\ {\isasymlangle}p{\isacharcomma}{\kern0pt}q{\isasymrangle}{\isachardoublequoteclose}\isanewline
\ \ \ \ \isacommand{by}\isamarkupfalse%
\ {\isacharparenleft}{\kern0pt}meson\ cart{\isacharunderscore}{\kern0pt}prod{\isacharunderscore}{\kern0pt}decomp{\isacharparenright}{\kern0pt}\isanewline
\ \ \isacommand{show}\isamarkupfalse%
\ {\isachardoublequoteopen}NAND\ {\isasymcirc}\isactrlsub c\ x\ {\isacharequal}{\kern0pt}\ {\isacharparenleft}{\kern0pt}NOT\ {\isasymcirc}\isactrlsub c\ AND{\isacharparenright}{\kern0pt}\ {\isasymcirc}\isactrlsub c\ x{\isachardoublequoteclose}\isanewline
\ \ \ \ \isacommand{by}\isamarkupfalse%
\ {\isacharparenleft}{\kern0pt}typecheck{\isacharunderscore}{\kern0pt}cfuncs{\isacharcomma}{\kern0pt}\ metis\ AND{\isacharunderscore}{\kern0pt}false{\isacharunderscore}{\kern0pt}left{\isacharunderscore}{\kern0pt}is{\isacharunderscore}{\kern0pt}false\ AND{\isacharunderscore}{\kern0pt}false{\isacharunderscore}{\kern0pt}right{\isacharunderscore}{\kern0pt}is{\isacharunderscore}{\kern0pt}false\ AND{\isacharunderscore}{\kern0pt}true{\isacharunderscore}{\kern0pt}true{\isacharunderscore}{\kern0pt}is{\isacharunderscore}{\kern0pt}true\ NAND{\isacharunderscore}{\kern0pt}left{\isacharunderscore}{\kern0pt}false{\isacharunderscore}{\kern0pt}is{\isacharunderscore}{\kern0pt}true\ NAND{\isacharunderscore}{\kern0pt}right{\isacharunderscore}{\kern0pt}false{\isacharunderscore}{\kern0pt}is{\isacharunderscore}{\kern0pt}true\ NAND{\isacharunderscore}{\kern0pt}true{\isacharunderscore}{\kern0pt}implies{\isacharunderscore}{\kern0pt}one{\isacharunderscore}{\kern0pt}is{\isacharunderscore}{\kern0pt}false\ NOT{\isacharunderscore}{\kern0pt}false{\isacharunderscore}{\kern0pt}is{\isacharunderscore}{\kern0pt}true\ NOT{\isacharunderscore}{\kern0pt}true{\isacharunderscore}{\kern0pt}is{\isacharunderscore}{\kern0pt}false\ comp{\isacharunderscore}{\kern0pt}associative{\isadigit{2}}\ true{\isacharunderscore}{\kern0pt}false{\isacharunderscore}{\kern0pt}only{\isacharunderscore}{\kern0pt}truth{\isacharunderscore}{\kern0pt}values\ x{\isacharunderscore}{\kern0pt}def\ x{\isacharunderscore}{\kern0pt}type{\isacharparenright}{\kern0pt}\isanewline
\isacommand{qed}\isamarkupfalse%
%
\endisatagproof
{\isafoldproof}%
%
\isadelimproof
\isanewline
%
\endisadelimproof
\isanewline
\isacommand{lemma}\isamarkupfalse%
\ NAND{\isacharunderscore}{\kern0pt}not{\isacharunderscore}{\kern0pt}idempotent{\isacharcolon}{\kern0pt}\isanewline
\ \ \isakeyword{assumes}\ {\isachardoublequoteopen}p\ {\isasymin}\isactrlsub c\ {\isasymOmega}{\isachardoublequoteclose}\isanewline
\ \ \isakeyword{shows}\ {\isachardoublequoteopen}NAND\ {\isasymcirc}\isactrlsub c\ {\isasymlangle}p{\isacharcomma}{\kern0pt}p{\isasymrangle}\ {\isacharequal}{\kern0pt}\ NOT\ {\isasymcirc}\isactrlsub c\ p{\isachardoublequoteclose}\isanewline
%
\isadelimproof
\ \ %
\endisadelimproof
%
\isatagproof
\isacommand{using}\isamarkupfalse%
\ NAND{\isacharunderscore}{\kern0pt}right{\isacharunderscore}{\kern0pt}false{\isacharunderscore}{\kern0pt}is{\isacharunderscore}{\kern0pt}true\ NAND{\isacharunderscore}{\kern0pt}true{\isacharunderscore}{\kern0pt}true{\isacharunderscore}{\kern0pt}is{\isacharunderscore}{\kern0pt}false\ NOT{\isacharunderscore}{\kern0pt}false{\isacharunderscore}{\kern0pt}is{\isacharunderscore}{\kern0pt}true\ NOT{\isacharunderscore}{\kern0pt}true{\isacharunderscore}{\kern0pt}is{\isacharunderscore}{\kern0pt}false\ assms\ true{\isacharunderscore}{\kern0pt}false{\isacharunderscore}{\kern0pt}only{\isacharunderscore}{\kern0pt}truth{\isacharunderscore}{\kern0pt}values\ \isacommand{by}\isamarkupfalse%
\ fastforce%
\endisatagproof
{\isafoldproof}%
%
\isadelimproof
%
\endisadelimproof
%
\isadelimdocument
%
\endisadelimdocument
%
\isatagdocument
%
\isamarkupsubsection{IFF%
}
\isamarkuptrue%
%
\endisatagdocument
{\isafolddocument}%
%
\isadelimdocument
%
\endisadelimdocument
\isacommand{definition}\isamarkupfalse%
\ IFF\ {\isacharcolon}{\kern0pt}{\isacharcolon}{\kern0pt}\ {\isachardoublequoteopen}cfunc{\isachardoublequoteclose}\ \isakeyword{where}\isanewline
\ \ {\isachardoublequoteopen}IFF\ {\isacharequal}{\kern0pt}\ {\isacharparenleft}{\kern0pt}THE\ {\isasymchi}{\isachardot}{\kern0pt}\ is{\isacharunderscore}{\kern0pt}pullback\ {\isacharparenleft}{\kern0pt}{\isasymone}{\isasymCoprod}{\isasymone}{\isacharparenright}{\kern0pt}\ {\isasymone}\ {\isacharparenleft}{\kern0pt}{\isasymOmega}\ {\isasymtimes}\isactrlsub c\ {\isasymOmega}{\isacharparenright}{\kern0pt}\ {\isasymOmega}\ {\isacharparenleft}{\kern0pt}{\isasymbeta}\isactrlbsub {\isacharparenleft}{\kern0pt}{\isasymone}{\isasymCoprod}{\isasymone}{\isacharparenright}{\kern0pt}\isactrlesub {\isacharparenright}{\kern0pt}\ {\isasymt}\ {\isacharparenleft}{\kern0pt}{\isasymlangle}{\isasymt}{\isacharcomma}{\kern0pt}\ {\isasymt}{\isasymrangle}\ {\isasymamalg}{\isasymlangle}{\isasymf}{\isacharcomma}{\kern0pt}\ {\isasymf}{\isasymrangle}{\isacharparenright}{\kern0pt}\ {\isasymchi}{\isacharparenright}{\kern0pt}{\isachardoublequoteclose}\isanewline
\isanewline
\isacommand{lemma}\isamarkupfalse%
\ pre{\isacharunderscore}{\kern0pt}IFF{\isacharunderscore}{\kern0pt}type{\isacharbrackleft}{\kern0pt}type{\isacharunderscore}{\kern0pt}rule{\isacharbrackright}{\kern0pt}{\isacharcolon}{\kern0pt}\ \isanewline
\ \ {\isachardoublequoteopen}{\isasymlangle}{\isasymt}{\isacharcomma}{\kern0pt}\ {\isasymt}{\isasymrangle}\ {\isasymamalg}\ {\isasymlangle}{\isasymf}{\isacharcomma}{\kern0pt}\ {\isasymf}{\isasymrangle}\ {\isacharcolon}{\kern0pt}\ {\isasymone}{\isasymCoprod}{\isasymone}\ {\isasymrightarrow}\ {\isasymOmega}\ {\isasymtimes}\isactrlsub c\ {\isasymOmega}{\isachardoublequoteclose}\isanewline
%
\isadelimproof
\ \ %
\endisadelimproof
%
\isatagproof
\isacommand{by}\isamarkupfalse%
\ typecheck{\isacharunderscore}{\kern0pt}cfuncs%
\endisatagproof
{\isafoldproof}%
%
\isadelimproof
\isanewline
%
\endisadelimproof
\isanewline
\isacommand{lemma}\isamarkupfalse%
\ pre{\isacharunderscore}{\kern0pt}IFF{\isacharunderscore}{\kern0pt}injective{\isacharcolon}{\kern0pt}\isanewline
\ {\isachardoublequoteopen}injective{\isacharparenleft}{\kern0pt}{\isasymlangle}{\isasymt}{\isacharcomma}{\kern0pt}\ {\isasymt}{\isasymrangle}\ {\isasymamalg}{\isasymlangle}{\isasymf}{\isacharcomma}{\kern0pt}\ {\isasymf}{\isasymrangle}{\isacharparenright}{\kern0pt}{\isachardoublequoteclose}\isanewline
%
\isadelimproof
\ %
\endisadelimproof
%
\isatagproof
\isacommand{unfolding}\isamarkupfalse%
\ injective{\isacharunderscore}{\kern0pt}def\isanewline
\isacommand{proof}\isamarkupfalse%
{\isacharparenleft}{\kern0pt}clarify{\isacharparenright}{\kern0pt}\isanewline
\ \ \isacommand{fix}\isamarkupfalse%
\ x\ y\ \isanewline
\ \ \isacommand{assume}\isamarkupfalse%
\ {\isachardoublequoteopen}x\ {\isasymin}\isactrlsub c\ domain\ {\isacharparenleft}{\kern0pt}{\isasymlangle}{\isasymt}{\isacharcomma}{\kern0pt}\ {\isasymt}{\isasymrangle}\ {\isasymamalg}{\isasymlangle}{\isasymf}{\isacharcomma}{\kern0pt}\ {\isasymf}{\isasymrangle}{\isacharparenright}{\kern0pt}{\isachardoublequoteclose}\ \isanewline
\ \ \isacommand{then}\isamarkupfalse%
\ \isacommand{have}\isamarkupfalse%
\ x{\isacharunderscore}{\kern0pt}type{\isacharcolon}{\kern0pt}\ {\isachardoublequoteopen}x\ {\isasymin}\isactrlsub c\ {\isacharparenleft}{\kern0pt}{\isasymone}{\isasymCoprod}{\isasymone}{\isacharparenright}{\kern0pt}{\isachardoublequoteclose}\ \ \isanewline
\ \ \ \ \isacommand{using}\isamarkupfalse%
\ cfunc{\isacharunderscore}{\kern0pt}type{\isacharunderscore}{\kern0pt}def\ pre{\isacharunderscore}{\kern0pt}IFF{\isacharunderscore}{\kern0pt}type\ \isacommand{by}\isamarkupfalse%
\ force\isanewline
\ \ \isacommand{then}\isamarkupfalse%
\ \isacommand{have}\isamarkupfalse%
\ x{\isacharunderscore}{\kern0pt}form{\isacharcolon}{\kern0pt}\ {\isachardoublequoteopen}{\isacharparenleft}{\kern0pt}{\isasymexists}\ w{\isachardot}{\kern0pt}\ {\isacharparenleft}{\kern0pt}w\ {\isasymin}\isactrlsub c\ {\isasymone}\ {\isasymand}\ x\ {\isacharequal}{\kern0pt}\ {\isacharparenleft}{\kern0pt}left{\isacharunderscore}{\kern0pt}coproj\ {\isasymone}\ {\isasymone}{\isacharparenright}{\kern0pt}\ {\isasymcirc}\isactrlsub c\ w{\isacharparenright}{\kern0pt}{\isacharparenright}{\kern0pt}\isanewline
\ \ \ \ \ \ {\isasymor}\ \ {\isacharparenleft}{\kern0pt}{\isasymexists}\ w{\isachardot}{\kern0pt}\ {\isacharparenleft}{\kern0pt}w\ {\isasymin}\isactrlsub c\ {\isasymone}\ {\isasymand}\ x\ {\isacharequal}{\kern0pt}\ {\isacharparenleft}{\kern0pt}right{\isacharunderscore}{\kern0pt}coproj\ {\isasymone}\ {\isasymone}{\isacharparenright}{\kern0pt}\ {\isasymcirc}\isactrlsub c\ w{\isacharparenright}{\kern0pt}{\isacharparenright}{\kern0pt}{\isachardoublequoteclose}\isanewline
\ \ \ \ \isacommand{using}\isamarkupfalse%
\ coprojs{\isacharunderscore}{\kern0pt}jointly{\isacharunderscore}{\kern0pt}surj\ \isacommand{by}\isamarkupfalse%
\ auto\isanewline
\isanewline
\ \ \isacommand{assume}\isamarkupfalse%
\ {\isachardoublequoteopen}y\ {\isasymin}\isactrlsub c\ domain\ {\isacharparenleft}{\kern0pt}{\isasymlangle}{\isasymt}{\isacharcomma}{\kern0pt}\ {\isasymt}{\isasymrangle}\ {\isasymamalg}{\isasymlangle}{\isasymf}{\isacharcomma}{\kern0pt}\ {\isasymf}{\isasymrangle}{\isacharparenright}{\kern0pt}{\isachardoublequoteclose}\ \isanewline
\ \ \isacommand{then}\isamarkupfalse%
\ \isacommand{have}\isamarkupfalse%
\ y{\isacharunderscore}{\kern0pt}type{\isacharcolon}{\kern0pt}\ {\isachardoublequoteopen}y\ {\isasymin}\isactrlsub c\ {\isacharparenleft}{\kern0pt}{\isasymone}{\isasymCoprod}{\isasymone}{\isacharparenright}{\kern0pt}{\isachardoublequoteclose}\ \ \isanewline
\ \ \ \ \isacommand{using}\isamarkupfalse%
\ cfunc{\isacharunderscore}{\kern0pt}type{\isacharunderscore}{\kern0pt}def\ pre{\isacharunderscore}{\kern0pt}IFF{\isacharunderscore}{\kern0pt}type\ \isacommand{by}\isamarkupfalse%
\ force\isanewline
\ \ \isacommand{then}\isamarkupfalse%
\ \isacommand{have}\isamarkupfalse%
\ y{\isacharunderscore}{\kern0pt}form{\isacharcolon}{\kern0pt}\ {\isachardoublequoteopen}{\isacharparenleft}{\kern0pt}{\isasymexists}\ w{\isachardot}{\kern0pt}\ {\isacharparenleft}{\kern0pt}w\ {\isasymin}\isactrlsub c\ {\isasymone}\ {\isasymand}\ y\ {\isacharequal}{\kern0pt}\ {\isacharparenleft}{\kern0pt}left{\isacharunderscore}{\kern0pt}coproj\ {\isasymone}\ {\isasymone}{\isacharparenright}{\kern0pt}\ {\isasymcirc}\isactrlsub c\ w{\isacharparenright}{\kern0pt}{\isacharparenright}{\kern0pt}\isanewline
\ \ \ \ \ \ {\isasymor}\ \ {\isacharparenleft}{\kern0pt}{\isasymexists}\ w{\isachardot}{\kern0pt}\ {\isacharparenleft}{\kern0pt}w\ {\isasymin}\isactrlsub c\ {\isasymone}\ {\isasymand}\ y\ {\isacharequal}{\kern0pt}\ {\isacharparenleft}{\kern0pt}right{\isacharunderscore}{\kern0pt}coproj\ {\isasymone}\ {\isasymone}{\isacharparenright}{\kern0pt}\ {\isasymcirc}\isactrlsub c\ w{\isacharparenright}{\kern0pt}{\isacharparenright}{\kern0pt}{\isachardoublequoteclose}\isanewline
\ \ \ \ \isacommand{using}\isamarkupfalse%
\ coprojs{\isacharunderscore}{\kern0pt}jointly{\isacharunderscore}{\kern0pt}surj\ \isacommand{by}\isamarkupfalse%
\ auto\isanewline
\isanewline
\ \ \isacommand{assume}\isamarkupfalse%
\ eqs{\isacharcolon}{\kern0pt}\ {\isachardoublequoteopen}{\isasymlangle}{\isasymt}{\isacharcomma}{\kern0pt}\ {\isasymt}{\isasymrangle}\ {\isasymamalg}{\isasymlangle}{\isasymf}{\isacharcomma}{\kern0pt}\ {\isasymf}{\isasymrangle}\ {\isasymcirc}\isactrlsub c\ x\ {\isacharequal}{\kern0pt}\ {\isasymlangle}{\isasymt}{\isacharcomma}{\kern0pt}\ {\isasymt}{\isasymrangle}\ {\isasymamalg}{\isasymlangle}{\isasymf}{\isacharcomma}{\kern0pt}\ {\isasymf}{\isasymrangle}\ {\isasymcirc}\isactrlsub c\ y{\isachardoublequoteclose}\isanewline
\isanewline
\ \ \isacommand{show}\isamarkupfalse%
\ {\isachardoublequoteopen}x\ {\isacharequal}{\kern0pt}\ y{\isachardoublequoteclose}\isanewline
\ \ \isacommand{proof}\isamarkupfalse%
{\isacharparenleft}{\kern0pt}cases\ {\isachardoublequoteopen}{\isasymexists}\ w{\isachardot}{\kern0pt}\ w\ {\isasymin}\isactrlsub c\ {\isasymone}\ {\isasymand}\ x\ {\isacharequal}{\kern0pt}\ left{\isacharunderscore}{\kern0pt}coproj\ {\isasymone}\ {\isasymone}\ {\isasymcirc}\isactrlsub c\ w{\isachardoublequoteclose}{\isacharparenright}{\kern0pt}\isanewline
\ \ \ \ \isacommand{assume}\isamarkupfalse%
\ a{\isadigit{1}}{\isacharcolon}{\kern0pt}\ {\isachardoublequoteopen}{\isasymexists}\ w{\isachardot}{\kern0pt}\ w\ {\isasymin}\isactrlsub c\ {\isasymone}\ {\isasymand}\ x\ {\isacharequal}{\kern0pt}\ left{\isacharunderscore}{\kern0pt}coproj\ {\isasymone}\ {\isasymone}\ {\isasymcirc}\isactrlsub c\ w{\isachardoublequoteclose}\isanewline
\ \ \ \ \isacommand{then}\isamarkupfalse%
\ \isacommand{obtain}\isamarkupfalse%
\ w\ \isakeyword{where}\ x{\isacharunderscore}{\kern0pt}def{\isacharcolon}{\kern0pt}\ {\isachardoublequoteopen}w\ {\isasymin}\isactrlsub c\ {\isasymone}\ {\isasymand}\ x\ {\isacharequal}{\kern0pt}\ left{\isacharunderscore}{\kern0pt}coproj\ {\isasymone}\ {\isasymone}\ {\isasymcirc}\isactrlsub c\ w{\isachardoublequoteclose}\isanewline
\ \ \ \ \ \ \isacommand{by}\isamarkupfalse%
\ blast\isanewline
\ \ \ \ \isacommand{then}\isamarkupfalse%
\ \isacommand{have}\isamarkupfalse%
\ {\isachardoublequoteopen}w\ {\isacharequal}{\kern0pt}\ id\ {\isasymone}{\isachardoublequoteclose}\isanewline
\ \ \ \ \ \ \isacommand{by}\isamarkupfalse%
\ {\isacharparenleft}{\kern0pt}typecheck{\isacharunderscore}{\kern0pt}cfuncs{\isacharcomma}{\kern0pt}\ metis\ terminal{\isacharunderscore}{\kern0pt}func{\isacharunderscore}{\kern0pt}unique\ x{\isacharunderscore}{\kern0pt}def{\isacharparenright}{\kern0pt}\isanewline
\ \ \ \ \isacommand{have}\isamarkupfalse%
\ {\isachardoublequoteopen}{\isasymexists}\ v{\isachardot}{\kern0pt}\ v\ {\isasymin}\isactrlsub c\ {\isasymone}\ {\isasymand}\ y\ {\isacharequal}{\kern0pt}\ left{\isacharunderscore}{\kern0pt}coproj\ {\isasymone}\ {\isasymone}\ {\isasymcirc}\isactrlsub c\ v{\isachardoublequoteclose}\isanewline
\ \ \ \ \isacommand{proof}\isamarkupfalse%
{\isacharparenleft}{\kern0pt}rule\ ccontr{\isacharparenright}{\kern0pt}\isanewline
\ \ \ \ \ \ \isacommand{assume}\isamarkupfalse%
\ a{\isadigit{2}}{\isacharcolon}{\kern0pt}\ {\isachardoublequoteopen}{\isasymnexists}v{\isachardot}{\kern0pt}\ v\ {\isasymin}\isactrlsub c\ {\isasymone}\ {\isasymand}\ y\ {\isacharequal}{\kern0pt}\ left{\isacharunderscore}{\kern0pt}coproj\ {\isasymone}\ {\isasymone}\ {\isasymcirc}\isactrlsub c\ v{\isachardoublequoteclose}\isanewline
\ \ \ \ \ \ \isacommand{then}\isamarkupfalse%
\ \isacommand{obtain}\isamarkupfalse%
\ v\ \isakeyword{where}\ y{\isacharunderscore}{\kern0pt}def{\isacharcolon}{\kern0pt}\ \ {\isachardoublequoteopen}v\ {\isasymin}\isactrlsub c\ {\isasymone}\ {\isasymand}\ y\ {\isacharequal}{\kern0pt}\ right{\isacharunderscore}{\kern0pt}coproj\ {\isasymone}\ {\isasymone}\ {\isasymcirc}\isactrlsub c\ v{\isachardoublequoteclose}\isanewline
\ \ \ \ \ \ \ \ \isacommand{using}\isamarkupfalse%
\ y{\isacharunderscore}{\kern0pt}form\ \isacommand{by}\isamarkupfalse%
\ {\isacharparenleft}{\kern0pt}typecheck{\isacharunderscore}{\kern0pt}cfuncs{\isacharcomma}{\kern0pt}\ blast{\isacharparenright}{\kern0pt}\isanewline
\ \ \ \ \ \ \isacommand{then}\isamarkupfalse%
\ \isacommand{have}\isamarkupfalse%
\ {\isachardoublequoteopen}v\ {\isacharequal}{\kern0pt}\ id\ {\isasymone}{\isachardoublequoteclose}\isanewline
\ \ \ \ \ \ \ \ \isacommand{by}\isamarkupfalse%
\ {\isacharparenleft}{\kern0pt}typecheck{\isacharunderscore}{\kern0pt}cfuncs{\isacharcomma}{\kern0pt}\ metis\ terminal{\isacharunderscore}{\kern0pt}func{\isacharunderscore}{\kern0pt}unique\ y{\isacharunderscore}{\kern0pt}def{\isacharparenright}{\kern0pt}\isanewline
\ \ \ \ \ \ \isacommand{then}\isamarkupfalse%
\ \isacommand{have}\isamarkupfalse%
\ {\isachardoublequoteopen}{\isasymlangle}{\isasymt}{\isacharcomma}{\kern0pt}\ {\isasymt}{\isasymrangle}\ {\isasymamalg}{\isasymlangle}{\isasymf}{\isacharcomma}{\kern0pt}\ {\isasymf}{\isasymrangle}\ {\isasymcirc}\isactrlsub c\ left{\isacharunderscore}{\kern0pt}coproj\ {\isasymone}\ {\isasymone}\ {\isacharequal}{\kern0pt}\ {\isasymlangle}{\isasymt}{\isacharcomma}{\kern0pt}\ {\isasymt}{\isasymrangle}\ {\isasymamalg}{\isasymlangle}{\isasymf}{\isacharcomma}{\kern0pt}\ {\isasymf}{\isasymrangle}\ {\isasymcirc}\isactrlsub c\ right{\isacharunderscore}{\kern0pt}coproj\ {\isasymone}\ {\isasymone}{\isachardoublequoteclose}\isanewline
\ \ \ \ \ \ \ \ \isacommand{using}\isamarkupfalse%
\ {\isacartoucheopen}v\ {\isacharequal}{\kern0pt}\ id\isactrlsub c\ {\isasymone}{\isacartoucheclose}\ {\isacartoucheopen}w\ {\isacharequal}{\kern0pt}\ id\isactrlsub c\ {\isasymone}{\isacartoucheclose}\ eqs\ id{\isacharunderscore}{\kern0pt}right{\isacharunderscore}{\kern0pt}unit{\isadigit{2}}\ x{\isacharunderscore}{\kern0pt}def\ y{\isacharunderscore}{\kern0pt}def\ \isacommand{by}\isamarkupfalse%
\ {\isacharparenleft}{\kern0pt}typecheck{\isacharunderscore}{\kern0pt}cfuncs{\isacharcomma}{\kern0pt}\ force{\isacharparenright}{\kern0pt}\isanewline
\ \ \ \ \ \ \isacommand{then}\isamarkupfalse%
\ \isacommand{have}\isamarkupfalse%
\ {\isachardoublequoteopen}{\isasymlangle}{\isasymt}{\isacharcomma}{\kern0pt}\ {\isasymt}{\isasymrangle}\ {\isacharequal}{\kern0pt}\ {\isasymlangle}{\isasymf}{\isacharcomma}{\kern0pt}{\isasymf}{\isasymrangle}{\isachardoublequoteclose}\isanewline
\ \ \ \ \ \ \ \ \isacommand{by}\isamarkupfalse%
\ {\isacharparenleft}{\kern0pt}typecheck{\isacharunderscore}{\kern0pt}cfuncs{\isacharcomma}{\kern0pt}\ smt\ {\isacharparenleft}{\kern0pt}z{\isadigit{3}}{\isacharparenright}{\kern0pt}\ \ cfunc{\isacharunderscore}{\kern0pt}coprod{\isacharunderscore}{\kern0pt}unique\ coprod{\isacharunderscore}{\kern0pt}eq{\isadigit{2}}\ pre{\isacharunderscore}{\kern0pt}IFF{\isacharunderscore}{\kern0pt}type\ right{\isacharunderscore}{\kern0pt}coproj{\isacharunderscore}{\kern0pt}cfunc{\isacharunderscore}{\kern0pt}coprod{\isacharparenright}{\kern0pt}\ \ \ \ \ \ \isanewline
\ \ \ \ \ \ \isacommand{then}\isamarkupfalse%
\ \isacommand{have}\isamarkupfalse%
\ {\isachardoublequoteopen}{\isasymt}\ {\isacharequal}{\kern0pt}\ {\isasymf}{\isachardoublequoteclose}\isanewline
\ \ \ \ \ \ \ \ \isacommand{using}\isamarkupfalse%
\ cart{\isacharunderscore}{\kern0pt}prod{\isacharunderscore}{\kern0pt}eq{\isadigit{2}}\ false{\isacharunderscore}{\kern0pt}func{\isacharunderscore}{\kern0pt}type\ true{\isacharunderscore}{\kern0pt}func{\isacharunderscore}{\kern0pt}type\ \isacommand{by}\isamarkupfalse%
\ blast\isanewline
\ \ \ \ \ \ \isacommand{then}\isamarkupfalse%
\ \isacommand{show}\isamarkupfalse%
\ False\isanewline
\ \ \ \ \ \ \ \ \isacommand{using}\isamarkupfalse%
\ true{\isacharunderscore}{\kern0pt}false{\isacharunderscore}{\kern0pt}distinct\ \isacommand{by}\isamarkupfalse%
\ blast\isanewline
\ \ \ \ \isacommand{qed}\isamarkupfalse%
\isanewline
\ \ \ \ \isacommand{then}\isamarkupfalse%
\ \isacommand{obtain}\isamarkupfalse%
\ v\ \isakeyword{where}\ y{\isacharunderscore}{\kern0pt}def{\isacharcolon}{\kern0pt}\ {\isachardoublequoteopen}v\ {\isasymin}\isactrlsub c\ {\isasymone}\ {\isasymand}\ y\ {\isacharequal}{\kern0pt}\ left{\isacharunderscore}{\kern0pt}coproj\ {\isasymone}\ {\isasymone}\ {\isasymcirc}\isactrlsub c\ v{\isachardoublequoteclose}\isanewline
\ \ \ \ \ \ \isacommand{by}\isamarkupfalse%
\ blast\isanewline
\ \ \ \ \isacommand{then}\isamarkupfalse%
\ \isacommand{have}\isamarkupfalse%
\ {\isachardoublequoteopen}v\ {\isacharequal}{\kern0pt}\ id\ {\isasymone}{\isachardoublequoteclose}\isanewline
\ \ \ \ \ \ \isacommand{by}\isamarkupfalse%
\ {\isacharparenleft}{\kern0pt}typecheck{\isacharunderscore}{\kern0pt}cfuncs{\isacharcomma}{\kern0pt}\ metis\ terminal{\isacharunderscore}{\kern0pt}func{\isacharunderscore}{\kern0pt}unique{\isacharparenright}{\kern0pt}\isanewline
\ \ \ \ \isacommand{then}\isamarkupfalse%
\ \isacommand{show}\isamarkupfalse%
\ {\isacharquery}{\kern0pt}thesis\isanewline
\ \ \ \ \ \ \isacommand{by}\isamarkupfalse%
\ {\isacharparenleft}{\kern0pt}simp\ add{\isacharcolon}{\kern0pt}\ {\isacartoucheopen}w\ {\isacharequal}{\kern0pt}\ id\isactrlsub c\ {\isasymone}{\isacartoucheclose}\ x{\isacharunderscore}{\kern0pt}def\ y{\isacharunderscore}{\kern0pt}def{\isacharparenright}{\kern0pt}\isanewline
\ \ \isacommand{next}\isamarkupfalse%
\isanewline
\ \ \ \ \isacommand{assume}\isamarkupfalse%
\ {\isachardoublequoteopen}{\isasymnexists}w{\isachardot}{\kern0pt}\ w\ {\isasymin}\isactrlsub c\ {\isasymone}\ {\isasymand}\ x\ {\isacharequal}{\kern0pt}\ left{\isacharunderscore}{\kern0pt}coproj\ {\isasymone}\ {\isasymone}\ {\isasymcirc}\isactrlsub c\ w{\isachardoublequoteclose}\isanewline
\ \ \ \ \isacommand{then}\isamarkupfalse%
\ \isacommand{obtain}\isamarkupfalse%
\ w\ \isakeyword{where}\ x{\isacharunderscore}{\kern0pt}def{\isacharcolon}{\kern0pt}\ {\isachardoublequoteopen}w\ {\isasymin}\isactrlsub c\ {\isasymone}\ {\isasymand}\ x\ {\isacharequal}{\kern0pt}\ right{\isacharunderscore}{\kern0pt}coproj\ {\isasymone}\ {\isasymone}\ {\isasymcirc}\isactrlsub c\ w{\isachardoublequoteclose}\isanewline
\ \ \ \ \ \ \isacommand{using}\isamarkupfalse%
\ x{\isacharunderscore}{\kern0pt}form\ \isacommand{by}\isamarkupfalse%
\ force\isanewline
\ \ \ \ \isacommand{then}\isamarkupfalse%
\ \isacommand{have}\isamarkupfalse%
\ {\isachardoublequoteopen}w\ {\isacharequal}{\kern0pt}\ id\ {\isasymone}{\isachardoublequoteclose}\isanewline
\ \ \ \ \ \ \isacommand{by}\isamarkupfalse%
\ {\isacharparenleft}{\kern0pt}typecheck{\isacharunderscore}{\kern0pt}cfuncs{\isacharcomma}{\kern0pt}\ metis\ terminal{\isacharunderscore}{\kern0pt}func{\isacharunderscore}{\kern0pt}unique\ x{\isacharunderscore}{\kern0pt}def{\isacharparenright}{\kern0pt}\isanewline
\ \ \ \ \isacommand{have}\isamarkupfalse%
\ {\isachardoublequoteopen}{\isasymexists}\ v{\isachardot}{\kern0pt}\ v\ {\isasymin}\isactrlsub c\ {\isasymone}\ {\isasymand}\ y\ {\isacharequal}{\kern0pt}\ right{\isacharunderscore}{\kern0pt}coproj\ {\isasymone}\ {\isasymone}\ {\isasymcirc}\isactrlsub c\ v{\isachardoublequoteclose}\isanewline
\ \ \ \ \isacommand{proof}\isamarkupfalse%
{\isacharparenleft}{\kern0pt}rule\ ccontr{\isacharparenright}{\kern0pt}\isanewline
\ \ \ \ \ \ \isacommand{assume}\isamarkupfalse%
\ a{\isadigit{2}}{\isacharcolon}{\kern0pt}\ {\isachardoublequoteopen}{\isasymnexists}v{\isachardot}{\kern0pt}\ v\ {\isasymin}\isactrlsub c\ {\isasymone}\ {\isasymand}\ y\ {\isacharequal}{\kern0pt}\ right{\isacharunderscore}{\kern0pt}coproj\ {\isasymone}\ {\isasymone}\ {\isasymcirc}\isactrlsub c\ v{\isachardoublequoteclose}\isanewline
\ \ \ \ \ \ \isacommand{then}\isamarkupfalse%
\ \isacommand{obtain}\isamarkupfalse%
\ v\ \isakeyword{where}\ y{\isacharunderscore}{\kern0pt}def{\isacharcolon}{\kern0pt}\ \ {\isachardoublequoteopen}v\ {\isasymin}\isactrlsub c\ {\isasymone}\ {\isasymand}\ y\ {\isacharequal}{\kern0pt}\ left{\isacharunderscore}{\kern0pt}coproj\ {\isasymone}\ {\isasymone}\ {\isasymcirc}\isactrlsub c\ v{\isachardoublequoteclose}\isanewline
\ \ \ \ \ \ \ \ \isacommand{using}\isamarkupfalse%
\ y{\isacharunderscore}{\kern0pt}form\ \isacommand{by}\isamarkupfalse%
\ {\isacharparenleft}{\kern0pt}typecheck{\isacharunderscore}{\kern0pt}cfuncs{\isacharcomma}{\kern0pt}\ blast{\isacharparenright}{\kern0pt}\isanewline
\ \ \ \ \ \ \isacommand{then}\isamarkupfalse%
\ \isacommand{have}\isamarkupfalse%
\ {\isachardoublequoteopen}v\ {\isacharequal}{\kern0pt}\ id\ {\isasymone}{\isachardoublequoteclose}\isanewline
\ \ \ \ \ \ \ \ \isacommand{by}\isamarkupfalse%
\ {\isacharparenleft}{\kern0pt}typecheck{\isacharunderscore}{\kern0pt}cfuncs{\isacharcomma}{\kern0pt}\ metis\ terminal{\isacharunderscore}{\kern0pt}func{\isacharunderscore}{\kern0pt}unique\ y{\isacharunderscore}{\kern0pt}def{\isacharparenright}{\kern0pt}\isanewline
\ \ \ \ \ \ \isacommand{then}\isamarkupfalse%
\ \isacommand{have}\isamarkupfalse%
\ {\isachardoublequoteopen}{\isasymlangle}{\isasymt}{\isacharcomma}{\kern0pt}\ {\isasymt}{\isasymrangle}\ {\isasymamalg}{\isasymlangle}{\isasymf}{\isacharcomma}{\kern0pt}\ {\isasymf}{\isasymrangle}\ {\isasymcirc}\isactrlsub c\ left{\isacharunderscore}{\kern0pt}coproj\ {\isasymone}\ {\isasymone}\ {\isacharequal}{\kern0pt}\ {\isasymlangle}{\isasymt}{\isacharcomma}{\kern0pt}\ {\isasymt}{\isasymrangle}\ {\isasymamalg}{\isasymlangle}{\isasymf}{\isacharcomma}{\kern0pt}\ {\isasymf}{\isasymrangle}\ {\isasymcirc}\isactrlsub c\ right{\isacharunderscore}{\kern0pt}coproj\ {\isasymone}\ {\isasymone}{\isachardoublequoteclose}\isanewline
\ \ \ \ \ \ \ \ \isacommand{using}\isamarkupfalse%
\ {\isacartoucheopen}v\ {\isacharequal}{\kern0pt}\ id\isactrlsub c\ {\isasymone}{\isacartoucheclose}\ {\isacartoucheopen}w\ {\isacharequal}{\kern0pt}\ id\isactrlsub c\ {\isasymone}{\isacartoucheclose}\ eqs\ id{\isacharunderscore}{\kern0pt}right{\isacharunderscore}{\kern0pt}unit{\isadigit{2}}\ x{\isacharunderscore}{\kern0pt}def\ y{\isacharunderscore}{\kern0pt}def\ \isacommand{by}\isamarkupfalse%
\ {\isacharparenleft}{\kern0pt}typecheck{\isacharunderscore}{\kern0pt}cfuncs{\isacharcomma}{\kern0pt}\ force{\isacharparenright}{\kern0pt}\isanewline
\ \ \ \ \ \ \isacommand{then}\isamarkupfalse%
\ \isacommand{have}\isamarkupfalse%
\ {\isachardoublequoteopen}{\isasymlangle}{\isasymt}{\isacharcomma}{\kern0pt}\ {\isasymt}{\isasymrangle}\ {\isacharequal}{\kern0pt}\ {\isasymlangle}{\isasymf}{\isacharcomma}{\kern0pt}\ {\isasymf}{\isasymrangle}{\isachardoublequoteclose}\isanewline
\ \ \ \ \ \ \ \ \isacommand{by}\isamarkupfalse%
\ {\isacharparenleft}{\kern0pt}typecheck{\isacharunderscore}{\kern0pt}cfuncs{\isacharcomma}{\kern0pt}\ smt\ {\isacharparenleft}{\kern0pt}z{\isadigit{3}}{\isacharparenright}{\kern0pt}\ \ cfunc{\isacharunderscore}{\kern0pt}coprod{\isacharunderscore}{\kern0pt}unique\ coprod{\isacharunderscore}{\kern0pt}eq{\isadigit{2}}\ pre{\isacharunderscore}{\kern0pt}IFF{\isacharunderscore}{\kern0pt}type\ right{\isacharunderscore}{\kern0pt}coproj{\isacharunderscore}{\kern0pt}cfunc{\isacharunderscore}{\kern0pt}coprod{\isacharparenright}{\kern0pt}\ \ \ \ \ \ \isanewline
\ \ \ \ \ \ \isacommand{then}\isamarkupfalse%
\ \isacommand{have}\isamarkupfalse%
\ {\isachardoublequoteopen}{\isasymt}\ {\isacharequal}{\kern0pt}\ {\isasymf}{\isachardoublequoteclose}\isanewline
\ \ \ \ \ \ \ \ \isacommand{using}\isamarkupfalse%
\ cart{\isacharunderscore}{\kern0pt}prod{\isacharunderscore}{\kern0pt}eq{\isadigit{2}}\ false{\isacharunderscore}{\kern0pt}func{\isacharunderscore}{\kern0pt}type\ true{\isacharunderscore}{\kern0pt}func{\isacharunderscore}{\kern0pt}type\ \isacommand{by}\isamarkupfalse%
\ blast\isanewline
\ \ \ \ \ \ \isacommand{then}\isamarkupfalse%
\ \isacommand{show}\isamarkupfalse%
\ False\isanewline
\ \ \ \ \ \ \ \ \isacommand{using}\isamarkupfalse%
\ true{\isacharunderscore}{\kern0pt}false{\isacharunderscore}{\kern0pt}distinct\ \isacommand{by}\isamarkupfalse%
\ blast\isanewline
\ \ \ \ \isacommand{qed}\isamarkupfalse%
\isanewline
\ \ \ \ \isacommand{then}\isamarkupfalse%
\ \isacommand{obtain}\isamarkupfalse%
\ v\ \isakeyword{where}\ y{\isacharunderscore}{\kern0pt}def{\isacharcolon}{\kern0pt}\ {\isachardoublequoteopen}v\ {\isasymin}\isactrlsub c\ {\isasymone}\ {\isasymand}\ y\ {\isacharequal}{\kern0pt}\ {\isacharparenleft}{\kern0pt}right{\isacharunderscore}{\kern0pt}coproj\ {\isasymone}\ {\isasymone}{\isacharparenright}{\kern0pt}\ {\isasymcirc}\isactrlsub c\ v{\isachardoublequoteclose}\isanewline
\ \ \ \ \ \ \isacommand{by}\isamarkupfalse%
\ blast\isanewline
\ \ \ \ \isacommand{then}\isamarkupfalse%
\ \isacommand{have}\isamarkupfalse%
\ {\isachardoublequoteopen}v\ {\isacharequal}{\kern0pt}\ id\ {\isasymone}{\isachardoublequoteclose}\isanewline
\ \ \ \ \ \ \isacommand{by}\isamarkupfalse%
\ {\isacharparenleft}{\kern0pt}typecheck{\isacharunderscore}{\kern0pt}cfuncs{\isacharcomma}{\kern0pt}\ metis\ terminal{\isacharunderscore}{\kern0pt}func{\isacharunderscore}{\kern0pt}unique{\isacharparenright}{\kern0pt}\isanewline
\ \ \ \ \isacommand{then}\isamarkupfalse%
\ \isacommand{show}\isamarkupfalse%
\ {\isacharquery}{\kern0pt}thesis\isanewline
\ \ \ \ \ \ \isacommand{by}\isamarkupfalse%
\ {\isacharparenleft}{\kern0pt}simp\ add{\isacharcolon}{\kern0pt}\ {\isacartoucheopen}w\ {\isacharequal}{\kern0pt}\ id\isactrlsub c\ {\isasymone}{\isacartoucheclose}\ x{\isacharunderscore}{\kern0pt}def\ y{\isacharunderscore}{\kern0pt}def{\isacharparenright}{\kern0pt}\isanewline
\ \ \isacommand{qed}\isamarkupfalse%
\isanewline
\isacommand{qed}\isamarkupfalse%
%
\endisatagproof
{\isafoldproof}%
%
\isadelimproof
\isanewline
%
\endisadelimproof
\isanewline
\isacommand{lemma}\isamarkupfalse%
\ IFF{\isacharunderscore}{\kern0pt}is{\isacharunderscore}{\kern0pt}pullback{\isacharcolon}{\kern0pt}\isanewline
\ \ {\isachardoublequoteopen}is{\isacharunderscore}{\kern0pt}pullback\ {\isacharparenleft}{\kern0pt}{\isasymone}{\isasymCoprod}{\isasymone}{\isacharparenright}{\kern0pt}\ {\isasymone}\ {\isacharparenleft}{\kern0pt}{\isasymOmega}{\isasymtimes}\isactrlsub c{\isasymOmega}{\isacharparenright}{\kern0pt}\ {\isasymOmega}\ {\isacharparenleft}{\kern0pt}{\isasymbeta}\isactrlbsub {\isacharparenleft}{\kern0pt}{\isasymone}{\isasymCoprod}{\isasymone}{\isacharparenright}{\kern0pt}\isactrlesub {\isacharparenright}{\kern0pt}\ {\isasymt}\ {\isacharparenleft}{\kern0pt}{\isasymlangle}{\isasymt}{\isacharcomma}{\kern0pt}\ {\isasymt}{\isasymrangle}\ {\isasymamalg}{\isasymlangle}{\isasymf}{\isacharcomma}{\kern0pt}\ {\isasymf}{\isasymrangle}{\isacharparenright}{\kern0pt}\ IFF{\isachardoublequoteclose}\isanewline
%
\isadelimproof
\ \ %
\endisadelimproof
%
\isatagproof
\isacommand{unfolding}\isamarkupfalse%
\ IFF{\isacharunderscore}{\kern0pt}def\isanewline
\ \ \isacommand{using}\isamarkupfalse%
\ element{\isacharunderscore}{\kern0pt}monomorphism\ characteristic{\isacharunderscore}{\kern0pt}function{\isacharunderscore}{\kern0pt}exists\isanewline
\ \ \isacommand{by}\isamarkupfalse%
\ {\isacharparenleft}{\kern0pt}typecheck{\isacharunderscore}{\kern0pt}cfuncs{\isacharcomma}{\kern0pt}\ simp\ add{\isacharcolon}{\kern0pt}\ the{\isadigit{1}}I{\isadigit{2}}\ injective{\isacharunderscore}{\kern0pt}imp{\isacharunderscore}{\kern0pt}monomorphism\ pre{\isacharunderscore}{\kern0pt}IFF{\isacharunderscore}{\kern0pt}injective{\isacharparenright}{\kern0pt}%
\endisatagproof
{\isafoldproof}%
%
\isadelimproof
\isanewline
%
\endisadelimproof
\isanewline
\isacommand{lemma}\isamarkupfalse%
\ IFF{\isacharunderscore}{\kern0pt}type{\isacharbrackleft}{\kern0pt}type{\isacharunderscore}{\kern0pt}rule{\isacharbrackright}{\kern0pt}{\isacharcolon}{\kern0pt}\isanewline
\ \ {\isachardoublequoteopen}IFF\ {\isacharcolon}{\kern0pt}\ {\isasymOmega}\ {\isasymtimes}\isactrlsub c\ {\isasymOmega}\ {\isasymrightarrow}\ {\isasymOmega}{\isachardoublequoteclose}\isanewline
%
\isadelimproof
\ \ %
\endisadelimproof
%
\isatagproof
\isacommand{unfolding}\isamarkupfalse%
\ IFF{\isacharunderscore}{\kern0pt}def\isanewline
\ \ \isacommand{by}\isamarkupfalse%
\ {\isacharparenleft}{\kern0pt}metis\ IFF{\isacharunderscore}{\kern0pt}def\ IFF{\isacharunderscore}{\kern0pt}is{\isacharunderscore}{\kern0pt}pullback\ is{\isacharunderscore}{\kern0pt}pullback{\isacharunderscore}{\kern0pt}def{\isacharparenright}{\kern0pt}%
\endisatagproof
{\isafoldproof}%
%
\isadelimproof
\isanewline
%
\endisadelimproof
\isanewline
\isacommand{lemma}\isamarkupfalse%
\ IFF{\isacharunderscore}{\kern0pt}true{\isacharunderscore}{\kern0pt}true{\isacharunderscore}{\kern0pt}is{\isacharunderscore}{\kern0pt}true{\isacharcolon}{\kern0pt}\isanewline
\ {\isachardoublequoteopen}IFF\ {\isasymcirc}\isactrlsub c\ {\isasymlangle}{\isasymt}{\isacharcomma}{\kern0pt}{\isasymt}{\isasymrangle}\ {\isacharequal}{\kern0pt}\ {\isasymt}{\isachardoublequoteclose}\isanewline
%
\isadelimproof
%
\endisadelimproof
%
\isatagproof
\isacommand{proof}\isamarkupfalse%
\ {\isacharminus}{\kern0pt}\ \isanewline
\ \ \isacommand{have}\isamarkupfalse%
\ {\isachardoublequoteopen}{\isasymexists}\ j{\isachardot}{\kern0pt}\ j\ {\isasymin}\isactrlsub c\ {\isacharparenleft}{\kern0pt}{\isasymone}{\isasymCoprod}{\isasymone}{\isacharparenright}{\kern0pt}\ {\isasymand}\ {\isacharparenleft}{\kern0pt}{\isasymlangle}{\isasymt}{\isacharcomma}{\kern0pt}\ {\isasymt}{\isasymrangle}\ {\isasymamalg}{\isasymlangle}{\isasymf}{\isacharcomma}{\kern0pt}\ {\isasymf}{\isasymrangle}{\isacharparenright}{\kern0pt}\ {\isasymcirc}\isactrlsub c\ j\ \ {\isacharequal}{\kern0pt}\ {\isasymlangle}{\isasymt}{\isacharcomma}{\kern0pt}{\isasymt}{\isasymrangle}{\isachardoublequoteclose}\isanewline
\ \ \ \ \isacommand{by}\isamarkupfalse%
\ {\isacharparenleft}{\kern0pt}typecheck{\isacharunderscore}{\kern0pt}cfuncs{\isacharcomma}{\kern0pt}\ smt\ {\isacharparenleft}{\kern0pt}z{\isadigit{3}}{\isacharparenright}{\kern0pt}\ \ comp{\isacharunderscore}{\kern0pt}associative{\isadigit{2}}\ comp{\isacharunderscore}{\kern0pt}type\ left{\isacharunderscore}{\kern0pt}coproj{\isacharunderscore}{\kern0pt}cfunc{\isacharunderscore}{\kern0pt}coprod\ left{\isacharunderscore}{\kern0pt}proj{\isacharunderscore}{\kern0pt}type\ right{\isacharunderscore}{\kern0pt}coproj{\isacharunderscore}{\kern0pt}cfunc{\isacharunderscore}{\kern0pt}coprod\ right{\isacharunderscore}{\kern0pt}proj{\isacharunderscore}{\kern0pt}type\ true{\isacharunderscore}{\kern0pt}false{\isacharunderscore}{\kern0pt}only{\isacharunderscore}{\kern0pt}truth{\isacharunderscore}{\kern0pt}values{\isacharparenright}{\kern0pt}\isanewline
\ \ \isacommand{then}\isamarkupfalse%
\ \isacommand{show}\isamarkupfalse%
\ {\isacharquery}{\kern0pt}thesis\ \isanewline
\ \ \ \ \isacommand{by}\isamarkupfalse%
\ {\isacharparenleft}{\kern0pt}smt\ {\isacharparenleft}{\kern0pt}verit{\isacharcomma}{\kern0pt}\ ccfv{\isacharunderscore}{\kern0pt}threshold{\isacharparenright}{\kern0pt}\ AND{\isacharunderscore}{\kern0pt}is{\isacharunderscore}{\kern0pt}pullback\ AND{\isacharunderscore}{\kern0pt}true{\isacharunderscore}{\kern0pt}true{\isacharunderscore}{\kern0pt}is{\isacharunderscore}{\kern0pt}true\ IFF{\isacharunderscore}{\kern0pt}is{\isacharunderscore}{\kern0pt}pullback\ comp{\isacharunderscore}{\kern0pt}associative{\isadigit{2}}\ is{\isacharunderscore}{\kern0pt}pullback{\isacharunderscore}{\kern0pt}def\ \ terminal{\isacharunderscore}{\kern0pt}func{\isacharunderscore}{\kern0pt}comp{\isacharparenright}{\kern0pt}\isanewline
\isacommand{qed}\isamarkupfalse%
%
\endisatagproof
{\isafoldproof}%
%
\isadelimproof
\isanewline
%
\endisadelimproof
\isanewline
\isacommand{lemma}\isamarkupfalse%
\ IFF{\isacharunderscore}{\kern0pt}false{\isacharunderscore}{\kern0pt}false{\isacharunderscore}{\kern0pt}is{\isacharunderscore}{\kern0pt}true{\isacharcolon}{\kern0pt}\isanewline
\ {\isachardoublequoteopen}IFF\ {\isasymcirc}\isactrlsub c\ {\isasymlangle}{\isasymf}{\isacharcomma}{\kern0pt}{\isasymf}{\isasymrangle}\ {\isacharequal}{\kern0pt}\ {\isasymt}{\isachardoublequoteclose}\isanewline
%
\isadelimproof
%
\endisadelimproof
%
\isatagproof
\isacommand{proof}\isamarkupfalse%
\ {\isacharminus}{\kern0pt}\ \isanewline
\ \ \isacommand{have}\isamarkupfalse%
\ {\isachardoublequoteopen}{\isasymexists}\ j{\isachardot}{\kern0pt}\ j\ {\isasymin}\isactrlsub c\ {\isacharparenleft}{\kern0pt}{\isasymone}{\isasymCoprod}{\isasymone}{\isacharparenright}{\kern0pt}\ {\isasymand}\ {\isacharparenleft}{\kern0pt}{\isasymlangle}{\isasymt}{\isacharcomma}{\kern0pt}\ {\isasymt}{\isasymrangle}\ {\isasymamalg}{\isasymlangle}{\isasymf}{\isacharcomma}{\kern0pt}\ {\isasymf}{\isasymrangle}{\isacharparenright}{\kern0pt}\ {\isasymcirc}\isactrlsub c\ j\ \ {\isacharequal}{\kern0pt}\ {\isasymlangle}{\isasymf}{\isacharcomma}{\kern0pt}{\isasymf}{\isasymrangle}{\isachardoublequoteclose}\isanewline
\ \ \ \ \isacommand{by}\isamarkupfalse%
\ {\isacharparenleft}{\kern0pt}typecheck{\isacharunderscore}{\kern0pt}cfuncs{\isacharcomma}{\kern0pt}\ smt\ {\isacharparenleft}{\kern0pt}z{\isadigit{3}}{\isacharparenright}{\kern0pt}\ \ comp{\isacharunderscore}{\kern0pt}associative{\isadigit{2}}\ comp{\isacharunderscore}{\kern0pt}type\ left{\isacharunderscore}{\kern0pt}coproj{\isacharunderscore}{\kern0pt}cfunc{\isacharunderscore}{\kern0pt}coprod\ left{\isacharunderscore}{\kern0pt}proj{\isacharunderscore}{\kern0pt}type\ right{\isacharunderscore}{\kern0pt}coproj{\isacharunderscore}{\kern0pt}cfunc{\isacharunderscore}{\kern0pt}coprod\ right{\isacharunderscore}{\kern0pt}proj{\isacharunderscore}{\kern0pt}type\ true{\isacharunderscore}{\kern0pt}false{\isacharunderscore}{\kern0pt}only{\isacharunderscore}{\kern0pt}truth{\isacharunderscore}{\kern0pt}values{\isacharparenright}{\kern0pt}\isanewline
\ \ \isacommand{then}\isamarkupfalse%
\ \isacommand{show}\isamarkupfalse%
\ {\isacharquery}{\kern0pt}thesis\ \isanewline
\ \ \ \ \isacommand{by}\isamarkupfalse%
\ {\isacharparenleft}{\kern0pt}smt\ {\isacharparenleft}{\kern0pt}verit{\isacharcomma}{\kern0pt}\ ccfv{\isacharunderscore}{\kern0pt}threshold{\isacharparenright}{\kern0pt}\ AND{\isacharunderscore}{\kern0pt}is{\isacharunderscore}{\kern0pt}pullback\ AND{\isacharunderscore}{\kern0pt}true{\isacharunderscore}{\kern0pt}true{\isacharunderscore}{\kern0pt}is{\isacharunderscore}{\kern0pt}true\ IFF{\isacharunderscore}{\kern0pt}is{\isacharunderscore}{\kern0pt}pullback\ comp{\isacharunderscore}{\kern0pt}associative{\isadigit{2}}\ is{\isacharunderscore}{\kern0pt}pullback{\isacharunderscore}{\kern0pt}def\ \ terminal{\isacharunderscore}{\kern0pt}func{\isacharunderscore}{\kern0pt}comp{\isacharparenright}{\kern0pt}\isanewline
\isacommand{qed}\isamarkupfalse%
%
\endisatagproof
{\isafoldproof}%
%
\isadelimproof
\isanewline
%
\endisadelimproof
\isanewline
\isacommand{lemma}\isamarkupfalse%
\ IFF{\isacharunderscore}{\kern0pt}true{\isacharunderscore}{\kern0pt}false{\isacharunderscore}{\kern0pt}is{\isacharunderscore}{\kern0pt}false{\isacharcolon}{\kern0pt}\isanewline
\ {\isachardoublequoteopen}IFF\ {\isasymcirc}\isactrlsub c\ {\isasymlangle}{\isasymt}{\isacharcomma}{\kern0pt}{\isasymf}{\isasymrangle}\ {\isacharequal}{\kern0pt}\ {\isasymf}{\isachardoublequoteclose}\isanewline
%
\isadelimproof
%
\endisadelimproof
%
\isatagproof
\isacommand{proof}\isamarkupfalse%
{\isacharparenleft}{\kern0pt}rule\ ccontr{\isacharparenright}{\kern0pt}\isanewline
\ \ \isacommand{assume}\isamarkupfalse%
\ {\isachardoublequoteopen}IFF\ {\isasymcirc}\isactrlsub c\ {\isasymlangle}{\isasymt}{\isacharcomma}{\kern0pt}{\isasymf}{\isasymrangle}\ {\isasymnoteq}\ {\isasymf}{\isachardoublequoteclose}\isanewline
\ \ \isacommand{then}\isamarkupfalse%
\ \isacommand{have}\isamarkupfalse%
\ {\isachardoublequoteopen}IFF\ {\isasymcirc}\isactrlsub c\ {\isasymlangle}{\isasymt}{\isacharcomma}{\kern0pt}{\isasymf}{\isasymrangle}\ \ {\isacharequal}{\kern0pt}\ {\isasymt}{\isachardoublequoteclose}\isanewline
\ \ \ \ \isacommand{using}\isamarkupfalse%
\ true{\isacharunderscore}{\kern0pt}false{\isacharunderscore}{\kern0pt}only{\isacharunderscore}{\kern0pt}truth{\isacharunderscore}{\kern0pt}values\ \isacommand{by}\isamarkupfalse%
\ {\isacharparenleft}{\kern0pt}typecheck{\isacharunderscore}{\kern0pt}cfuncs{\isacharcomma}{\kern0pt}\ blast{\isacharparenright}{\kern0pt}\ \ \ \ \isanewline
\ \ \isacommand{then}\isamarkupfalse%
\ \isacommand{obtain}\isamarkupfalse%
\ j\ \isakeyword{where}\ j{\isacharunderscore}{\kern0pt}type{\isacharbrackleft}{\kern0pt}type{\isacharunderscore}{\kern0pt}rule{\isacharbrackright}{\kern0pt}{\isacharcolon}{\kern0pt}\ {\isachardoublequoteopen}j\ {\isasymin}\isactrlsub c\ {\isasymone}{\isasymCoprod}{\isasymone}\ {\isasymand}\ {\isacharparenleft}{\kern0pt}{\isasymlangle}{\isasymt}{\isacharcomma}{\kern0pt}\ {\isasymt}{\isasymrangle}\ {\isasymamalg}{\isasymlangle}{\isasymf}{\isacharcomma}{\kern0pt}\ {\isasymf}{\isasymrangle}{\isacharparenright}{\kern0pt}\ {\isasymcirc}\isactrlsub c\ j\ \ {\isacharequal}{\kern0pt}\ {\isasymlangle}{\isasymt}{\isacharcomma}{\kern0pt}{\isasymf}{\isasymrangle}{\isachardoublequoteclose}\isanewline
\ \ \ \ \isacommand{by}\isamarkupfalse%
\ {\isacharparenleft}{\kern0pt}typecheck{\isacharunderscore}{\kern0pt}cfuncs{\isacharcomma}{\kern0pt}\ smt\ {\isacharparenleft}{\kern0pt}verit{\isacharcomma}{\kern0pt}\ ccfv{\isacharunderscore}{\kern0pt}threshold{\isacharparenright}{\kern0pt}\ IFF{\isacharunderscore}{\kern0pt}is{\isacharunderscore}{\kern0pt}pullback\ characteristic{\isacharunderscore}{\kern0pt}function{\isacharunderscore}{\kern0pt}exists\ element{\isacharunderscore}{\kern0pt}monomorphism\ is{\isacharunderscore}{\kern0pt}pullback{\isacharunderscore}{\kern0pt}def{\isacharparenright}{\kern0pt}\isanewline
\ \ \isacommand{show}\isamarkupfalse%
\ False\isanewline
\ \ \isacommand{proof}\isamarkupfalse%
{\isacharparenleft}{\kern0pt}cases\ {\isachardoublequoteopen}j\ {\isacharequal}{\kern0pt}\ left{\isacharunderscore}{\kern0pt}coproj\ {\isasymone}\ {\isasymone}{\isachardoublequoteclose}{\isacharparenright}{\kern0pt}\isanewline
\ \ \ \ \isacommand{assume}\isamarkupfalse%
\ {\isachardoublequoteopen}j\ {\isacharequal}{\kern0pt}\ left{\isacharunderscore}{\kern0pt}coproj\ {\isasymone}\ {\isasymone}{\isachardoublequoteclose}\isanewline
\ \ \ \ \isacommand{then}\isamarkupfalse%
\ \isacommand{have}\isamarkupfalse%
\ {\isachardoublequoteopen}{\isacharparenleft}{\kern0pt}{\isasymlangle}{\isasymt}{\isacharcomma}{\kern0pt}\ {\isasymt}{\isasymrangle}\ {\isasymamalg}{\isasymlangle}{\isasymf}{\isacharcomma}{\kern0pt}\ {\isasymf}{\isasymrangle}{\isacharparenright}{\kern0pt}\ {\isasymcirc}\isactrlsub c\ j\ \ {\isacharequal}{\kern0pt}\ {\isasymlangle}{\isasymt}{\isacharcomma}{\kern0pt}\ {\isasymt}{\isasymrangle}{\isachardoublequoteclose}\isanewline
\ \ \ \ \ \ \isacommand{using}\isamarkupfalse%
\ \ left{\isacharunderscore}{\kern0pt}coproj{\isacharunderscore}{\kern0pt}cfunc{\isacharunderscore}{\kern0pt}coprod\ \isacommand{by}\isamarkupfalse%
\ {\isacharparenleft}{\kern0pt}typecheck{\isacharunderscore}{\kern0pt}cfuncs{\isacharcomma}{\kern0pt}\ presburger{\isacharparenright}{\kern0pt}\isanewline
\ \ \ \ \isacommand{then}\isamarkupfalse%
\ \isacommand{have}\isamarkupfalse%
\ {\isachardoublequoteopen}{\isasymlangle}{\isasymt}{\isacharcomma}{\kern0pt}\ {\isasymf}{\isasymrangle}\ {\isacharequal}{\kern0pt}\ {\isasymlangle}{\isasymt}{\isacharcomma}{\kern0pt}{\isasymt}{\isasymrangle}{\isachardoublequoteclose}\isanewline
\ \ \ \ \ \ \isacommand{using}\isamarkupfalse%
\ \ j{\isacharunderscore}{\kern0pt}type\ \isacommand{by}\isamarkupfalse%
\ argo\isanewline
\ \ \ \ \isacommand{then}\isamarkupfalse%
\ \isacommand{have}\isamarkupfalse%
\ {\isachardoublequoteopen}{\isasymt}\ {\isacharequal}{\kern0pt}\ {\isasymf}{\isachardoublequoteclose}\isanewline
\ \ \ \ \ \ \isacommand{using}\isamarkupfalse%
\ cart{\isacharunderscore}{\kern0pt}prod{\isacharunderscore}{\kern0pt}eq{\isadigit{2}}\ false{\isacharunderscore}{\kern0pt}func{\isacharunderscore}{\kern0pt}type\ true{\isacharunderscore}{\kern0pt}func{\isacharunderscore}{\kern0pt}type\ \isacommand{by}\isamarkupfalse%
\ auto\isanewline
\ \ \ \ \isacommand{then}\isamarkupfalse%
\ \isacommand{show}\isamarkupfalse%
\ False\isanewline
\ \ \ \ \ \ \isacommand{using}\isamarkupfalse%
\ true{\isacharunderscore}{\kern0pt}false{\isacharunderscore}{\kern0pt}distinct\ \isacommand{by}\isamarkupfalse%
\ auto\isanewline
\ \ \isacommand{next}\isamarkupfalse%
\isanewline
\ \ \ \ \isacommand{assume}\isamarkupfalse%
\ {\isachardoublequoteopen}j\ {\isasymnoteq}\ left{\isacharunderscore}{\kern0pt}coproj\ {\isasymone}\ {\isasymone}{\isachardoublequoteclose}\isanewline
\ \ \ \ \isacommand{then}\isamarkupfalse%
\ \isacommand{have}\isamarkupfalse%
\ {\isachardoublequoteopen}j\ {\isacharequal}{\kern0pt}\ right{\isacharunderscore}{\kern0pt}coproj\ {\isasymone}\ {\isasymone}{\isachardoublequoteclose}\isanewline
\ \ \ \ \ \ \isacommand{using}\isamarkupfalse%
\ j{\isacharunderscore}{\kern0pt}type\ maps{\isacharunderscore}{\kern0pt}into{\isacharunderscore}{\kern0pt}{\isadigit{1}}u{\isadigit{1}}\ \isacommand{by}\isamarkupfalse%
\ auto\isanewline
\ \ \ \ \isacommand{then}\isamarkupfalse%
\ \isacommand{have}\isamarkupfalse%
\ {\isachardoublequoteopen}{\isacharparenleft}{\kern0pt}{\isasymlangle}{\isasymt}{\isacharcomma}{\kern0pt}\ {\isasymt}{\isasymrangle}\ {\isasymamalg}{\isasymlangle}{\isasymf}{\isacharcomma}{\kern0pt}\ {\isasymf}{\isasymrangle}{\isacharparenright}{\kern0pt}\ {\isasymcirc}\isactrlsub c\ j\ \ {\isacharequal}{\kern0pt}\ {\isasymlangle}{\isasymf}{\isacharcomma}{\kern0pt}\ {\isasymf}{\isasymrangle}{\isachardoublequoteclose}\isanewline
\ \ \ \ \ \ \isacommand{using}\isamarkupfalse%
\ \ right{\isacharunderscore}{\kern0pt}coproj{\isacharunderscore}{\kern0pt}cfunc{\isacharunderscore}{\kern0pt}coprod\ \isacommand{by}\isamarkupfalse%
\ {\isacharparenleft}{\kern0pt}typecheck{\isacharunderscore}{\kern0pt}cfuncs{\isacharcomma}{\kern0pt}\ presburger{\isacharparenright}{\kern0pt}\isanewline
\ \ \ \ \isacommand{then}\isamarkupfalse%
\ \isacommand{have}\isamarkupfalse%
\ {\isachardoublequoteopen}{\isasymlangle}{\isasymf}{\isacharcomma}{\kern0pt}\ {\isasymt}{\isasymrangle}\ {\isacharequal}{\kern0pt}\ {\isasymlangle}{\isasymf}{\isacharcomma}{\kern0pt}\ {\isasymf}{\isasymrangle}{\isachardoublequoteclose}\isanewline
\ \ \ \ \ \ \isacommand{using}\isamarkupfalse%
\ XOR{\isacharunderscore}{\kern0pt}false{\isacharunderscore}{\kern0pt}false{\isacharunderscore}{\kern0pt}is{\isacharunderscore}{\kern0pt}false\ XOR{\isacharunderscore}{\kern0pt}only{\isacharunderscore}{\kern0pt}true{\isacharunderscore}{\kern0pt}left{\isacharunderscore}{\kern0pt}is{\isacharunderscore}{\kern0pt}true\ j{\isacharunderscore}{\kern0pt}type\ \isacommand{by}\isamarkupfalse%
\ argo\isanewline
\ \ \ \ \isacommand{then}\isamarkupfalse%
\ \isacommand{have}\isamarkupfalse%
\ {\isachardoublequoteopen}{\isasymt}\ {\isacharequal}{\kern0pt}\ {\isasymf}{\isachardoublequoteclose}\isanewline
\ \ \ \ \ \ \isacommand{using}\isamarkupfalse%
\ cart{\isacharunderscore}{\kern0pt}prod{\isacharunderscore}{\kern0pt}eq{\isadigit{2}}\ false{\isacharunderscore}{\kern0pt}func{\isacharunderscore}{\kern0pt}type\ true{\isacharunderscore}{\kern0pt}func{\isacharunderscore}{\kern0pt}type\ \isacommand{by}\isamarkupfalse%
\ auto\isanewline
\ \ \ \ \isacommand{then}\isamarkupfalse%
\ \isacommand{show}\isamarkupfalse%
\ False\isanewline
\ \ \ \ \ \ \isacommand{using}\isamarkupfalse%
\ true{\isacharunderscore}{\kern0pt}false{\isacharunderscore}{\kern0pt}distinct\ \isacommand{by}\isamarkupfalse%
\ auto\isanewline
\ \isacommand{qed}\isamarkupfalse%
\isanewline
\isacommand{qed}\isamarkupfalse%
%
\endisatagproof
{\isafoldproof}%
%
\isadelimproof
\isanewline
%
\endisadelimproof
\isanewline
\isacommand{lemma}\isamarkupfalse%
\ IFF{\isacharunderscore}{\kern0pt}false{\isacharunderscore}{\kern0pt}true{\isacharunderscore}{\kern0pt}is{\isacharunderscore}{\kern0pt}false{\isacharcolon}{\kern0pt}\isanewline
\ {\isachardoublequoteopen}IFF\ {\isasymcirc}\isactrlsub c\ {\isasymlangle}{\isasymf}{\isacharcomma}{\kern0pt}{\isasymt}{\isasymrangle}\ {\isacharequal}{\kern0pt}\ {\isasymf}{\isachardoublequoteclose}\isanewline
%
\isadelimproof
%
\endisadelimproof
%
\isatagproof
\isacommand{proof}\isamarkupfalse%
{\isacharparenleft}{\kern0pt}rule\ ccontr{\isacharparenright}{\kern0pt}\isanewline
\ \ \isacommand{assume}\isamarkupfalse%
\ {\isachardoublequoteopen}IFF\ {\isasymcirc}\isactrlsub c\ {\isasymlangle}{\isasymf}{\isacharcomma}{\kern0pt}{\isasymt}{\isasymrangle}\ {\isasymnoteq}\ {\isasymf}{\isachardoublequoteclose}\isanewline
\ \ \isacommand{then}\isamarkupfalse%
\ \isacommand{have}\isamarkupfalse%
\ {\isachardoublequoteopen}IFF\ {\isasymcirc}\isactrlsub c\ {\isasymlangle}{\isasymf}{\isacharcomma}{\kern0pt}{\isasymt}{\isasymrangle}\ \ {\isacharequal}{\kern0pt}\ {\isasymt}{\isachardoublequoteclose}\isanewline
\ \ \ \ \isacommand{using}\isamarkupfalse%
\ true{\isacharunderscore}{\kern0pt}false{\isacharunderscore}{\kern0pt}only{\isacharunderscore}{\kern0pt}truth{\isacharunderscore}{\kern0pt}values\ \isacommand{by}\isamarkupfalse%
\ {\isacharparenleft}{\kern0pt}typecheck{\isacharunderscore}{\kern0pt}cfuncs{\isacharcomma}{\kern0pt}\ blast{\isacharparenright}{\kern0pt}\isanewline
\ \ \isacommand{then}\isamarkupfalse%
\ \isacommand{obtain}\isamarkupfalse%
\ j\ \isakeyword{where}\ j{\isacharunderscore}{\kern0pt}type{\isacharbrackleft}{\kern0pt}type{\isacharunderscore}{\kern0pt}rule{\isacharbrackright}{\kern0pt}{\isacharcolon}{\kern0pt}\ {\isachardoublequoteopen}j\ {\isasymin}\isactrlsub c\ {\isasymone}{\isasymCoprod}{\isasymone}{\isachardoublequoteclose}\ \isakeyword{and}\ j{\isacharunderscore}{\kern0pt}def{\isacharcolon}{\kern0pt}\ \ {\isachardoublequoteopen}{\isacharparenleft}{\kern0pt}{\isasymlangle}{\isasymt}{\isacharcomma}{\kern0pt}\ {\isasymt}{\isasymrangle}\ {\isasymamalg}{\isasymlangle}{\isasymf}{\isacharcomma}{\kern0pt}\ {\isasymf}{\isasymrangle}{\isacharparenright}{\kern0pt}\ {\isasymcirc}\isactrlsub c\ j\ \ {\isacharequal}{\kern0pt}\ {\isasymlangle}{\isasymf}{\isacharcomma}{\kern0pt}{\isasymt}{\isasymrangle}{\isachardoublequoteclose}\isanewline
\ \ \ \ \isacommand{by}\isamarkupfalse%
\ {\isacharparenleft}{\kern0pt}typecheck{\isacharunderscore}{\kern0pt}cfuncs{\isacharcomma}{\kern0pt}\ smt\ {\isacharparenleft}{\kern0pt}verit{\isacharcomma}{\kern0pt}\ ccfv{\isacharunderscore}{\kern0pt}threshold{\isacharparenright}{\kern0pt}\ IFF{\isacharunderscore}{\kern0pt}is{\isacharunderscore}{\kern0pt}pullback\ id{\isacharunderscore}{\kern0pt}right{\isacharunderscore}{\kern0pt}unit{\isadigit{2}}\ is{\isacharunderscore}{\kern0pt}pullback{\isacharunderscore}{\kern0pt}def\ one{\isacharunderscore}{\kern0pt}unique{\isacharunderscore}{\kern0pt}element\ terminal{\isacharunderscore}{\kern0pt}func{\isacharunderscore}{\kern0pt}comp\ terminal{\isacharunderscore}{\kern0pt}func{\isacharunderscore}{\kern0pt}comp{\isacharunderscore}{\kern0pt}elem\ terminal{\isacharunderscore}{\kern0pt}func{\isacharunderscore}{\kern0pt}unique{\isacharparenright}{\kern0pt}\isanewline
\ \ \isacommand{show}\isamarkupfalse%
\ False\isanewline
\ \ \isacommand{proof}\isamarkupfalse%
{\isacharparenleft}{\kern0pt}cases\ {\isachardoublequoteopen}j\ {\isacharequal}{\kern0pt}\ left{\isacharunderscore}{\kern0pt}coproj\ {\isasymone}\ {\isasymone}{\isachardoublequoteclose}{\isacharparenright}{\kern0pt}\isanewline
\ \ \ \ \isacommand{assume}\isamarkupfalse%
\ {\isachardoublequoteopen}j\ {\isacharequal}{\kern0pt}\ left{\isacharunderscore}{\kern0pt}coproj\ {\isasymone}\ {\isasymone}{\isachardoublequoteclose}\isanewline
\ \ \ \ \isacommand{then}\isamarkupfalse%
\ \isacommand{have}\isamarkupfalse%
\ {\isachardoublequoteopen}{\isacharparenleft}{\kern0pt}{\isasymlangle}{\isasymt}{\isacharcomma}{\kern0pt}\ {\isasymt}{\isasymrangle}\ {\isasymamalg}{\isasymlangle}{\isasymf}{\isacharcomma}{\kern0pt}\ {\isasymf}{\isasymrangle}{\isacharparenright}{\kern0pt}\ {\isasymcirc}\isactrlsub c\ j\ \ {\isacharequal}{\kern0pt}\ {\isasymlangle}{\isasymt}{\isacharcomma}{\kern0pt}\ {\isasymt}{\isasymrangle}{\isachardoublequoteclose}\isanewline
\ \ \ \ \ \ \isacommand{using}\isamarkupfalse%
\ \ left{\isacharunderscore}{\kern0pt}coproj{\isacharunderscore}{\kern0pt}cfunc{\isacharunderscore}{\kern0pt}coprod\ \isacommand{by}\isamarkupfalse%
\ {\isacharparenleft}{\kern0pt}typecheck{\isacharunderscore}{\kern0pt}cfuncs{\isacharcomma}{\kern0pt}\ presburger{\isacharparenright}{\kern0pt}\isanewline
\ \ \ \ \isacommand{then}\isamarkupfalse%
\ \isacommand{have}\isamarkupfalse%
\ {\isachardoublequoteopen}{\isasymlangle}{\isasymf}{\isacharcomma}{\kern0pt}{\isasymt}{\isasymrangle}\ {\isacharequal}{\kern0pt}\ {\isasymlangle}{\isasymt}{\isacharcomma}{\kern0pt}{\isasymt}{\isasymrangle}{\isachardoublequoteclose}\isanewline
\ \ \ \ \ \ \isacommand{using}\isamarkupfalse%
\ j{\isacharunderscore}{\kern0pt}def\ \isacommand{by}\isamarkupfalse%
\ auto\isanewline
\ \ \ \ \isacommand{then}\isamarkupfalse%
\ \isacommand{have}\isamarkupfalse%
\ {\isachardoublequoteopen}{\isasymt}\ {\isacharequal}{\kern0pt}\ {\isasymf}{\isachardoublequoteclose}\isanewline
\ \ \ \ \ \ \isacommand{using}\isamarkupfalse%
\ cart{\isacharunderscore}{\kern0pt}prod{\isacharunderscore}{\kern0pt}eq{\isadigit{2}}\ false{\isacharunderscore}{\kern0pt}func{\isacharunderscore}{\kern0pt}type\ true{\isacharunderscore}{\kern0pt}func{\isacharunderscore}{\kern0pt}type\ \isacommand{by}\isamarkupfalse%
\ auto\isanewline
\ \ \ \ \isacommand{then}\isamarkupfalse%
\ \isacommand{show}\isamarkupfalse%
\ False\isanewline
\ \ \ \ \ \ \isacommand{using}\isamarkupfalse%
\ true{\isacharunderscore}{\kern0pt}false{\isacharunderscore}{\kern0pt}distinct\ \isacommand{by}\isamarkupfalse%
\ auto\isanewline
\ \ \isacommand{next}\isamarkupfalse%
\isanewline
\ \ \ \ \isacommand{assume}\isamarkupfalse%
\ {\isachardoublequoteopen}j\ {\isasymnoteq}\ left{\isacharunderscore}{\kern0pt}coproj\ {\isasymone}\ {\isasymone}{\isachardoublequoteclose}\isanewline
\ \ \ \ \isacommand{then}\isamarkupfalse%
\ \isacommand{have}\isamarkupfalse%
\ {\isachardoublequoteopen}j\ {\isacharequal}{\kern0pt}\ right{\isacharunderscore}{\kern0pt}coproj\ {\isasymone}\ {\isasymone}{\isachardoublequoteclose}\isanewline
\ \ \ \ \ \ \isacommand{using}\isamarkupfalse%
\ j{\isacharunderscore}{\kern0pt}type\ maps{\isacharunderscore}{\kern0pt}into{\isacharunderscore}{\kern0pt}{\isadigit{1}}u{\isadigit{1}}\ \isacommand{by}\isamarkupfalse%
\ blast\isanewline
\ \ \ \ \isacommand{then}\isamarkupfalse%
\ \isacommand{have}\isamarkupfalse%
\ {\isachardoublequoteopen}{\isacharparenleft}{\kern0pt}{\isasymlangle}{\isasymt}{\isacharcomma}{\kern0pt}\ {\isasymt}{\isasymrangle}\ {\isasymamalg}{\isasymlangle}{\isasymf}{\isacharcomma}{\kern0pt}\ {\isasymf}{\isasymrangle}{\isacharparenright}{\kern0pt}\ {\isasymcirc}\isactrlsub c\ j\ \ {\isacharequal}{\kern0pt}\ {\isasymlangle}{\isasymf}{\isacharcomma}{\kern0pt}\ {\isasymf}{\isasymrangle}{\isachardoublequoteclose}\isanewline
\ \ \ \ \ \ \isacommand{using}\isamarkupfalse%
\ \ right{\isacharunderscore}{\kern0pt}coproj{\isacharunderscore}{\kern0pt}cfunc{\isacharunderscore}{\kern0pt}coprod\ \isacommand{by}\isamarkupfalse%
\ {\isacharparenleft}{\kern0pt}typecheck{\isacharunderscore}{\kern0pt}cfuncs{\isacharcomma}{\kern0pt}\ presburger{\isacharparenright}{\kern0pt}\isanewline
\ \ \ \ \isacommand{then}\isamarkupfalse%
\ \isacommand{have}\isamarkupfalse%
\ {\isachardoublequoteopen}{\isasymlangle}{\isasymf}{\isacharcomma}{\kern0pt}{\isasymt}{\isasymrangle}\ {\isacharequal}{\kern0pt}\ {\isasymlangle}{\isasymf}{\isacharcomma}{\kern0pt}\ {\isasymf}{\isasymrangle}{\isachardoublequoteclose}\isanewline
\ \ \ \ \ \ \isacommand{using}\isamarkupfalse%
\ XOR{\isacharunderscore}{\kern0pt}false{\isacharunderscore}{\kern0pt}false{\isacharunderscore}{\kern0pt}is{\isacharunderscore}{\kern0pt}false\ XOR{\isacharunderscore}{\kern0pt}only{\isacharunderscore}{\kern0pt}true{\isacharunderscore}{\kern0pt}left{\isacharunderscore}{\kern0pt}is{\isacharunderscore}{\kern0pt}true\ j{\isacharunderscore}{\kern0pt}def\ \isacommand{by}\isamarkupfalse%
\ fastforce\isanewline
\ \ \ \ \isacommand{then}\isamarkupfalse%
\ \isacommand{have}\isamarkupfalse%
\ {\isachardoublequoteopen}{\isasymt}\ {\isacharequal}{\kern0pt}\ {\isasymf}{\isachardoublequoteclose}\isanewline
\ \ \ \ \ \ \isacommand{using}\isamarkupfalse%
\ cart{\isacharunderscore}{\kern0pt}prod{\isacharunderscore}{\kern0pt}eq{\isadigit{2}}\ false{\isacharunderscore}{\kern0pt}func{\isacharunderscore}{\kern0pt}type\ true{\isacharunderscore}{\kern0pt}func{\isacharunderscore}{\kern0pt}type\ \isacommand{by}\isamarkupfalse%
\ auto\isanewline
\ \ \ \ \isacommand{then}\isamarkupfalse%
\ \isacommand{show}\isamarkupfalse%
\ False\isanewline
\ \ \ \ \ \ \isacommand{using}\isamarkupfalse%
\ true{\isacharunderscore}{\kern0pt}false{\isacharunderscore}{\kern0pt}distinct\ \isacommand{by}\isamarkupfalse%
\ auto\isanewline
\ \isacommand{qed}\isamarkupfalse%
\isanewline
\isacommand{qed}\isamarkupfalse%
%
\endisatagproof
{\isafoldproof}%
%
\isadelimproof
\isanewline
%
\endisadelimproof
\isanewline
\isacommand{lemma}\isamarkupfalse%
\ NOT{\isacharunderscore}{\kern0pt}IFF{\isacharunderscore}{\kern0pt}is{\isacharunderscore}{\kern0pt}XOR{\isacharcolon}{\kern0pt}\ \isanewline
\ \ {\isachardoublequoteopen}NOT\ {\isasymcirc}\isactrlsub c\ IFF\ {\isacharequal}{\kern0pt}\ XOR{\isachardoublequoteclose}\isanewline
%
\isadelimproof
%
\endisadelimproof
%
\isatagproof
\isacommand{proof}\isamarkupfalse%
{\isacharparenleft}{\kern0pt}etcs{\isacharunderscore}{\kern0pt}rule\ one{\isacharunderscore}{\kern0pt}separator{\isacharparenright}{\kern0pt}\isanewline
\ \ \isacommand{fix}\isamarkupfalse%
\ x\ \ \ \isanewline
\ \ \isacommand{assume}\isamarkupfalse%
\ x{\isacharunderscore}{\kern0pt}type{\isacharcolon}{\kern0pt}\ {\isachardoublequoteopen}x\ {\isasymin}\isactrlsub c\ {\isasymOmega}\ {\isasymtimes}\isactrlsub c\ {\isasymOmega}{\isachardoublequoteclose}\isanewline
\ \ \isacommand{then}\isamarkupfalse%
\ \isacommand{obtain}\isamarkupfalse%
\ u\ w\ \isakeyword{where}\ x{\isacharunderscore}{\kern0pt}def{\isacharcolon}{\kern0pt}\ {\isachardoublequoteopen}u\ {\isasymin}\isactrlsub c\ {\isasymOmega}\ {\isasymand}\ w\ {\isasymin}\isactrlsub c\ {\isasymOmega}\ {\isasymand}\ x\ {\isacharequal}{\kern0pt}\ {\isasymlangle}u{\isacharcomma}{\kern0pt}w{\isasymrangle}{\isachardoublequoteclose}\isanewline
\ \ \ \ \isacommand{using}\isamarkupfalse%
\ cart{\isacharunderscore}{\kern0pt}prod{\isacharunderscore}{\kern0pt}decomp\ \isacommand{by}\isamarkupfalse%
\ blast\isanewline
\ \ \isacommand{show}\isamarkupfalse%
\ {\isachardoublequoteopen}{\isacharparenleft}{\kern0pt}NOT\ {\isasymcirc}\isactrlsub c\ IFF{\isacharparenright}{\kern0pt}\ {\isasymcirc}\isactrlsub c\ x\ {\isacharequal}{\kern0pt}\ XOR\ {\isasymcirc}\isactrlsub c\ x{\isachardoublequoteclose}\isanewline
\ \ \isacommand{proof}\isamarkupfalse%
{\isacharparenleft}{\kern0pt}cases\ {\isachardoublequoteopen}u\ {\isacharequal}{\kern0pt}\ {\isasymt}{\isachardoublequoteclose}{\isacharparenright}{\kern0pt}\isanewline
\ \ \ \ \isacommand{show}\isamarkupfalse%
\ {\isachardoublequoteopen}{\isacharparenleft}{\kern0pt}NOT\ {\isasymcirc}\isactrlsub c\ IFF{\isacharparenright}{\kern0pt}\ {\isasymcirc}\isactrlsub c\ x\ {\isacharequal}{\kern0pt}\ XOR\ {\isasymcirc}\isactrlsub c\ x{\isachardoublequoteclose}\isanewline
\ \ \ \ \isacommand{proof}\isamarkupfalse%
{\isacharparenleft}{\kern0pt}cases\ {\isachardoublequoteopen}w\ {\isacharequal}{\kern0pt}\ {\isasymt}{\isachardoublequoteclose}{\isacharparenright}{\kern0pt}\isanewline
\ \ \ \ \ \ \isacommand{show}\isamarkupfalse%
\ {\isachardoublequoteopen}{\isacharparenleft}{\kern0pt}NOT\ {\isasymcirc}\isactrlsub c\ IFF{\isacharparenright}{\kern0pt}\ {\isasymcirc}\isactrlsub c\ x\ {\isacharequal}{\kern0pt}\ XOR\ {\isasymcirc}\isactrlsub c\ x{\isachardoublequoteclose}\isanewline
\ \ \ \ \ \ \ \ \isacommand{by}\isamarkupfalse%
\ {\isacharparenleft}{\kern0pt}metis\ IFF{\isacharunderscore}{\kern0pt}false{\isacharunderscore}{\kern0pt}false{\isacharunderscore}{\kern0pt}is{\isacharunderscore}{\kern0pt}true\ IFF{\isacharunderscore}{\kern0pt}false{\isacharunderscore}{\kern0pt}true{\isacharunderscore}{\kern0pt}is{\isacharunderscore}{\kern0pt}false\ IFF{\isacharunderscore}{\kern0pt}true{\isacharunderscore}{\kern0pt}false{\isacharunderscore}{\kern0pt}is{\isacharunderscore}{\kern0pt}false\ IFF{\isacharunderscore}{\kern0pt}true{\isacharunderscore}{\kern0pt}true{\isacharunderscore}{\kern0pt}is{\isacharunderscore}{\kern0pt}true\ IFF{\isacharunderscore}{\kern0pt}type\ NOT{\isacharunderscore}{\kern0pt}false{\isacharunderscore}{\kern0pt}is{\isacharunderscore}{\kern0pt}true\ NOT{\isacharunderscore}{\kern0pt}true{\isacharunderscore}{\kern0pt}is{\isacharunderscore}{\kern0pt}false\ NOT{\isacharunderscore}{\kern0pt}type\ XOR{\isacharunderscore}{\kern0pt}false{\isacharunderscore}{\kern0pt}false{\isacharunderscore}{\kern0pt}is{\isacharunderscore}{\kern0pt}false\ XOR{\isacharunderscore}{\kern0pt}only{\isacharunderscore}{\kern0pt}true{\isacharunderscore}{\kern0pt}left{\isacharunderscore}{\kern0pt}is{\isacharunderscore}{\kern0pt}true\ XOR{\isacharunderscore}{\kern0pt}only{\isacharunderscore}{\kern0pt}true{\isacharunderscore}{\kern0pt}right{\isacharunderscore}{\kern0pt}is{\isacharunderscore}{\kern0pt}true\ XOR{\isacharunderscore}{\kern0pt}true{\isacharunderscore}{\kern0pt}true{\isacharunderscore}{\kern0pt}is{\isacharunderscore}{\kern0pt}false\ cfunc{\isacharunderscore}{\kern0pt}type{\isacharunderscore}{\kern0pt}def\ comp{\isacharunderscore}{\kern0pt}associative\ true{\isacharunderscore}{\kern0pt}false{\isacharunderscore}{\kern0pt}only{\isacharunderscore}{\kern0pt}truth{\isacharunderscore}{\kern0pt}values\ x{\isacharunderscore}{\kern0pt}def\ x{\isacharunderscore}{\kern0pt}type{\isacharparenright}{\kern0pt}\isanewline
\ \ \ \ \isacommand{next}\isamarkupfalse%
\ \isanewline
\ \ \ \ \ \ \isacommand{assume}\isamarkupfalse%
\ {\isachardoublequoteopen}w\ {\isasymnoteq}\ {\isasymt}{\isachardoublequoteclose}\isanewline
\ \ \ \ \ \ \isacommand{then}\isamarkupfalse%
\ \isacommand{have}\isamarkupfalse%
\ {\isachardoublequoteopen}w\ {\isacharequal}{\kern0pt}\ {\isasymf}{\isachardoublequoteclose}\isanewline
\ \ \ \ \ \ \ \ \isacommand{by}\isamarkupfalse%
\ {\isacharparenleft}{\kern0pt}metis\ true{\isacharunderscore}{\kern0pt}false{\isacharunderscore}{\kern0pt}only{\isacharunderscore}{\kern0pt}truth{\isacharunderscore}{\kern0pt}values\ x{\isacharunderscore}{\kern0pt}def{\isacharparenright}{\kern0pt}\isanewline
\ \ \ \ \ \ \isacommand{then}\isamarkupfalse%
\ \isacommand{show}\isamarkupfalse%
\ {\isachardoublequoteopen}{\isacharparenleft}{\kern0pt}NOT\ {\isasymcirc}\isactrlsub c\ IFF{\isacharparenright}{\kern0pt}\ {\isasymcirc}\isactrlsub c\ x\ {\isacharequal}{\kern0pt}\ XOR\ {\isasymcirc}\isactrlsub c\ x{\isachardoublequoteclose}\isanewline
\ \ \ \ \ \ \ \ \isacommand{by}\isamarkupfalse%
\ {\isacharparenleft}{\kern0pt}metis\ IFF{\isacharunderscore}{\kern0pt}false{\isacharunderscore}{\kern0pt}false{\isacharunderscore}{\kern0pt}is{\isacharunderscore}{\kern0pt}true\ IFF{\isacharunderscore}{\kern0pt}true{\isacharunderscore}{\kern0pt}false{\isacharunderscore}{\kern0pt}is{\isacharunderscore}{\kern0pt}false\ IFF{\isacharunderscore}{\kern0pt}type\ NOT{\isacharunderscore}{\kern0pt}false{\isacharunderscore}{\kern0pt}is{\isacharunderscore}{\kern0pt}true\ NOT{\isacharunderscore}{\kern0pt}true{\isacharunderscore}{\kern0pt}is{\isacharunderscore}{\kern0pt}false\ NOT{\isacharunderscore}{\kern0pt}type\ XOR{\isacharunderscore}{\kern0pt}false{\isacharunderscore}{\kern0pt}false{\isacharunderscore}{\kern0pt}is{\isacharunderscore}{\kern0pt}false\ XOR{\isacharunderscore}{\kern0pt}only{\isacharunderscore}{\kern0pt}true{\isacharunderscore}{\kern0pt}left{\isacharunderscore}{\kern0pt}is{\isacharunderscore}{\kern0pt}true\ comp{\isacharunderscore}{\kern0pt}associative{\isadigit{2}}\ true{\isacharunderscore}{\kern0pt}false{\isacharunderscore}{\kern0pt}only{\isacharunderscore}{\kern0pt}truth{\isacharunderscore}{\kern0pt}values\ x{\isacharunderscore}{\kern0pt}def\ x{\isacharunderscore}{\kern0pt}type{\isacharparenright}{\kern0pt}\isanewline
\ \ \ \ \isacommand{qed}\isamarkupfalse%
\isanewline
\ \ \isacommand{next}\isamarkupfalse%
\ \isanewline
\ \ \ \ \isacommand{assume}\isamarkupfalse%
\ {\isachardoublequoteopen}u\ {\isasymnoteq}\ {\isasymt}{\isachardoublequoteclose}\isanewline
\ \ \ \ \isacommand{then}\isamarkupfalse%
\ \isacommand{have}\isamarkupfalse%
\ {\isachardoublequoteopen}u\ {\isacharequal}{\kern0pt}\ {\isasymf}{\isachardoublequoteclose}\isanewline
\ \ \ \ \ \ \isacommand{by}\isamarkupfalse%
\ {\isacharparenleft}{\kern0pt}metis\ true{\isacharunderscore}{\kern0pt}false{\isacharunderscore}{\kern0pt}only{\isacharunderscore}{\kern0pt}truth{\isacharunderscore}{\kern0pt}values\ x{\isacharunderscore}{\kern0pt}def{\isacharparenright}{\kern0pt}\isanewline
\ \ \ \ \isacommand{show}\isamarkupfalse%
\ {\isachardoublequoteopen}{\isacharparenleft}{\kern0pt}NOT\ {\isasymcirc}\isactrlsub c\ IFF{\isacharparenright}{\kern0pt}\ {\isasymcirc}\isactrlsub c\ x\ {\isacharequal}{\kern0pt}\ XOR\ {\isasymcirc}\isactrlsub c\ x{\isachardoublequoteclose}\isanewline
\ \ \ \ \isacommand{proof}\isamarkupfalse%
{\isacharparenleft}{\kern0pt}cases\ {\isachardoublequoteopen}w\ {\isacharequal}{\kern0pt}\ {\isasymt}{\isachardoublequoteclose}{\isacharparenright}{\kern0pt}\isanewline
\ \ \ \ \ \ \isacommand{show}\isamarkupfalse%
\ {\isachardoublequoteopen}{\isacharparenleft}{\kern0pt}NOT\ {\isasymcirc}\isactrlsub c\ IFF{\isacharparenright}{\kern0pt}\ {\isasymcirc}\isactrlsub c\ x\ {\isacharequal}{\kern0pt}\ XOR\ {\isasymcirc}\isactrlsub c\ x{\isachardoublequoteclose}\isanewline
\ \ \ \ \ \ \ \ \isacommand{by}\isamarkupfalse%
\ {\isacharparenleft}{\kern0pt}metis\ IFF{\isacharunderscore}{\kern0pt}false{\isacharunderscore}{\kern0pt}false{\isacharunderscore}{\kern0pt}is{\isacharunderscore}{\kern0pt}true\ IFF{\isacharunderscore}{\kern0pt}false{\isacharunderscore}{\kern0pt}true{\isacharunderscore}{\kern0pt}is{\isacharunderscore}{\kern0pt}false\ IFF{\isacharunderscore}{\kern0pt}type\ NOT{\isacharunderscore}{\kern0pt}false{\isacharunderscore}{\kern0pt}is{\isacharunderscore}{\kern0pt}true\ NOT{\isacharunderscore}{\kern0pt}true{\isacharunderscore}{\kern0pt}is{\isacharunderscore}{\kern0pt}false\ NOT{\isacharunderscore}{\kern0pt}type\ XOR{\isacharunderscore}{\kern0pt}false{\isacharunderscore}{\kern0pt}false{\isacharunderscore}{\kern0pt}is{\isacharunderscore}{\kern0pt}false\ XOR{\isacharunderscore}{\kern0pt}only{\isacharunderscore}{\kern0pt}true{\isacharunderscore}{\kern0pt}right{\isacharunderscore}{\kern0pt}is{\isacharunderscore}{\kern0pt}true\ {\isacartoucheopen}u\ {\isacharequal}{\kern0pt}\ {\isasymf}{\isacartoucheclose}\ comp{\isacharunderscore}{\kern0pt}associative{\isadigit{2}}\ true{\isacharunderscore}{\kern0pt}false{\isacharunderscore}{\kern0pt}only{\isacharunderscore}{\kern0pt}truth{\isacharunderscore}{\kern0pt}values\ x{\isacharunderscore}{\kern0pt}def\ x{\isacharunderscore}{\kern0pt}type{\isacharparenright}{\kern0pt}\isanewline
\ \ \ \ \isacommand{next}\isamarkupfalse%
\isanewline
\ \ \ \ \ \ \isacommand{assume}\isamarkupfalse%
\ {\isachardoublequoteopen}w\ {\isasymnoteq}\ {\isasymt}{\isachardoublequoteclose}\isanewline
\ \ \ \ \ \ \isacommand{then}\isamarkupfalse%
\ \isacommand{have}\isamarkupfalse%
\ {\isachardoublequoteopen}w\ {\isacharequal}{\kern0pt}\ {\isasymf}{\isachardoublequoteclose}\isanewline
\ \ \ \ \ \ \ \ \isacommand{by}\isamarkupfalse%
\ {\isacharparenleft}{\kern0pt}metis\ true{\isacharunderscore}{\kern0pt}false{\isacharunderscore}{\kern0pt}only{\isacharunderscore}{\kern0pt}truth{\isacharunderscore}{\kern0pt}values\ x{\isacharunderscore}{\kern0pt}def{\isacharparenright}{\kern0pt}\isanewline
\ \ \ \ \ \ \isacommand{then}\isamarkupfalse%
\ \isacommand{show}\isamarkupfalse%
\ {\isachardoublequoteopen}{\isacharparenleft}{\kern0pt}NOT\ {\isasymcirc}\isactrlsub c\ IFF{\isacharparenright}{\kern0pt}\ {\isasymcirc}\isactrlsub c\ x\ {\isacharequal}{\kern0pt}\ XOR\ {\isasymcirc}\isactrlsub c\ x{\isachardoublequoteclose}\isanewline
\ \ \ \ \ \ \ \ \isacommand{by}\isamarkupfalse%
\ {\isacharparenleft}{\kern0pt}metis\ IFF{\isacharunderscore}{\kern0pt}false{\isacharunderscore}{\kern0pt}false{\isacharunderscore}{\kern0pt}is{\isacharunderscore}{\kern0pt}true\ IFF{\isacharunderscore}{\kern0pt}type\ NOT{\isacharunderscore}{\kern0pt}true{\isacharunderscore}{\kern0pt}is{\isacharunderscore}{\kern0pt}false\ NOT{\isacharunderscore}{\kern0pt}type\ XOR{\isacharunderscore}{\kern0pt}false{\isacharunderscore}{\kern0pt}false{\isacharunderscore}{\kern0pt}is{\isacharunderscore}{\kern0pt}false\ {\isacartoucheopen}u\ {\isacharequal}{\kern0pt}\ {\isasymf}{\isacartoucheclose}\ cfunc{\isacharunderscore}{\kern0pt}type{\isacharunderscore}{\kern0pt}def\ comp{\isacharunderscore}{\kern0pt}associative\ x{\isacharunderscore}{\kern0pt}def\ x{\isacharunderscore}{\kern0pt}type{\isacharparenright}{\kern0pt}\isanewline
\ \ \ \ \isacommand{qed}\isamarkupfalse%
\isanewline
\ \ \isacommand{qed}\isamarkupfalse%
\isanewline
\isacommand{qed}\isamarkupfalse%
%
\endisatagproof
{\isafoldproof}%
%
\isadelimproof
%
\endisadelimproof
%
\isadelimdocument
%
\endisadelimdocument
%
\isatagdocument
%
\isamarkupsubsection{IMPLIES%
}
\isamarkuptrue%
%
\endisatagdocument
{\isafolddocument}%
%
\isadelimdocument
%
\endisadelimdocument
\isacommand{definition}\isamarkupfalse%
\ IMPLIES\ {\isacharcolon}{\kern0pt}{\isacharcolon}{\kern0pt}\ {\isachardoublequoteopen}cfunc{\isachardoublequoteclose}\ \isakeyword{where}\isanewline
\ \ {\isachardoublequoteopen}IMPLIES\ {\isacharequal}{\kern0pt}\ {\isacharparenleft}{\kern0pt}THE\ {\isasymchi}{\isachardot}{\kern0pt}\ is{\isacharunderscore}{\kern0pt}pullback\ {\isacharparenleft}{\kern0pt}{\isasymone}{\isasymCoprod}{\isacharparenleft}{\kern0pt}{\isasymone}{\isasymCoprod}{\isasymone}{\isacharparenright}{\kern0pt}{\isacharparenright}{\kern0pt}\ {\isasymone}\ {\isacharparenleft}{\kern0pt}{\isasymOmega}{\isasymtimes}\isactrlsub c{\isasymOmega}{\isacharparenright}{\kern0pt}\ {\isasymOmega}\ {\isacharparenleft}{\kern0pt}{\isasymbeta}\isactrlbsub {\isacharparenleft}{\kern0pt}{\isasymone}{\isasymCoprod}{\isacharparenleft}{\kern0pt}{\isasymone}{\isasymCoprod}{\isasymone}{\isacharparenright}{\kern0pt}{\isacharparenright}{\kern0pt}\isactrlesub {\isacharparenright}{\kern0pt}\ {\isasymt}\ {\isacharparenleft}{\kern0pt}{\isasymlangle}{\isasymt}{\isacharcomma}{\kern0pt}\ {\isasymt}{\isasymrangle}{\isasymamalg}\ {\isacharparenleft}{\kern0pt}{\isasymlangle}{\isasymf}{\isacharcomma}{\kern0pt}\ {\isasymf}{\isasymrangle}\ {\isasymamalg}{\isasymlangle}{\isasymf}{\isacharcomma}{\kern0pt}\ {\isasymt}{\isasymrangle}{\isacharparenright}{\kern0pt}{\isacharparenright}{\kern0pt}\ {\isasymchi}{\isacharparenright}{\kern0pt}{\isachardoublequoteclose}\isanewline
\isanewline
\isacommand{lemma}\isamarkupfalse%
\ pre{\isacharunderscore}{\kern0pt}IMPLIES{\isacharunderscore}{\kern0pt}type{\isacharbrackleft}{\kern0pt}type{\isacharunderscore}{\kern0pt}rule{\isacharbrackright}{\kern0pt}{\isacharcolon}{\kern0pt}\ \isanewline
\ \ {\isachardoublequoteopen}{\isasymlangle}{\isasymt}{\isacharcomma}{\kern0pt}\ {\isasymt}{\isasymrangle}\ {\isasymamalg}\ {\isacharparenleft}{\kern0pt}{\isasymlangle}{\isasymf}{\isacharcomma}{\kern0pt}\ {\isasymf}{\isasymrangle}\ {\isasymamalg}\ {\isasymlangle}{\isasymf}{\isacharcomma}{\kern0pt}\ {\isasymt}{\isasymrangle}{\isacharparenright}{\kern0pt}\ {\isacharcolon}{\kern0pt}\ {\isasymone}\ {\isasymCoprod}\ {\isacharparenleft}{\kern0pt}{\isasymone}\ {\isasymCoprod}\ {\isasymone}{\isacharparenright}{\kern0pt}\ {\isasymrightarrow}\ {\isasymOmega}\ {\isasymtimes}\isactrlsub c\ {\isasymOmega}{\isachardoublequoteclose}\isanewline
%
\isadelimproof
\ \ %
\endisadelimproof
%
\isatagproof
\isacommand{by}\isamarkupfalse%
\ typecheck{\isacharunderscore}{\kern0pt}cfuncs%
\endisatagproof
{\isafoldproof}%
%
\isadelimproof
\isanewline
%
\endisadelimproof
\isanewline
\isacommand{lemma}\isamarkupfalse%
\ pre{\isacharunderscore}{\kern0pt}IMPLIES{\isacharunderscore}{\kern0pt}injective{\isacharcolon}{\kern0pt}\isanewline
\ \ {\isachardoublequoteopen}injective{\isacharparenleft}{\kern0pt}{\isasymlangle}{\isasymt}{\isacharcomma}{\kern0pt}\ {\isasymt}{\isasymrangle}\ {\isasymamalg}\ {\isacharparenleft}{\kern0pt}{\isasymlangle}{\isasymf}{\isacharcomma}{\kern0pt}\ {\isasymf}{\isasymrangle}\ {\isasymamalg}{\isasymlangle}{\isasymf}{\isacharcomma}{\kern0pt}\ {\isasymt}{\isasymrangle}{\isacharparenright}{\kern0pt}{\isacharparenright}{\kern0pt}{\isachardoublequoteclose}\isanewline
%
\isadelimproof
\ \ %
\endisadelimproof
%
\isatagproof
\isacommand{unfolding}\isamarkupfalse%
\ injective{\isacharunderscore}{\kern0pt}def\isanewline
\isacommand{proof}\isamarkupfalse%
{\isacharparenleft}{\kern0pt}clarify{\isacharparenright}{\kern0pt}\isanewline
\ \ \isacommand{fix}\isamarkupfalse%
\ x\ y\ \isanewline
\ \ \isacommand{assume}\isamarkupfalse%
\ a{\isadigit{1}}{\isacharcolon}{\kern0pt}\ {\isachardoublequoteopen}x\ {\isasymin}\isactrlsub c\ domain\ {\isacharparenleft}{\kern0pt}{\isasymlangle}{\isasymt}{\isacharcomma}{\kern0pt}{\isasymt}{\isasymrangle}\ {\isasymamalg}\ {\isasymlangle}{\isasymf}{\isacharcomma}{\kern0pt}\ {\isasymf}{\isasymrangle}\ {\isasymamalg}\ {\isasymlangle}{\isasymf}{\isacharcomma}{\kern0pt}{\isasymt}{\isasymrangle}{\isacharparenright}{\kern0pt}{\isachardoublequoteclose}\ \isanewline
\ \ \isacommand{then}\isamarkupfalse%
\ \isacommand{have}\isamarkupfalse%
\ x{\isacharunderscore}{\kern0pt}type{\isacharbrackleft}{\kern0pt}type{\isacharunderscore}{\kern0pt}rule{\isacharbrackright}{\kern0pt}{\isacharcolon}{\kern0pt}\ {\isachardoublequoteopen}x\ {\isasymin}\isactrlsub c\ {\isacharparenleft}{\kern0pt}{\isasymone}{\isasymCoprod}{\isacharparenleft}{\kern0pt}{\isasymone}{\isasymCoprod}{\isasymone}{\isacharparenright}{\kern0pt}{\isacharparenright}{\kern0pt}{\isachardoublequoteclose}\ \ \isanewline
\ \ \ \ \isacommand{using}\isamarkupfalse%
\ cfunc{\isacharunderscore}{\kern0pt}type{\isacharunderscore}{\kern0pt}def\ pre{\isacharunderscore}{\kern0pt}IMPLIES{\isacharunderscore}{\kern0pt}type\ \isacommand{by}\isamarkupfalse%
\ force\isanewline
\ \ \isacommand{then}\isamarkupfalse%
\ \isacommand{have}\isamarkupfalse%
\ x{\isacharunderscore}{\kern0pt}form{\isacharcolon}{\kern0pt}\ {\isachardoublequoteopen}{\isacharparenleft}{\kern0pt}{\isasymexists}\ w{\isachardot}{\kern0pt}\ {\isacharparenleft}{\kern0pt}w\ {\isasymin}\isactrlsub c\ {\isasymone}\ {\isasymand}\ x\ {\isacharequal}{\kern0pt}\ {\isacharparenleft}{\kern0pt}left{\isacharunderscore}{\kern0pt}coproj\ {\isasymone}\ {\isacharparenleft}{\kern0pt}{\isasymone}{\isasymCoprod}{\isasymone}{\isacharparenright}{\kern0pt}{\isacharparenright}{\kern0pt}\ {\isasymcirc}\isactrlsub c\ w{\isacharparenright}{\kern0pt}{\isacharparenright}{\kern0pt}\isanewline
\ \ \ \ \ \ {\isasymor}\ \ {\isacharparenleft}{\kern0pt}{\isasymexists}\ w{\isachardot}{\kern0pt}\ {\isacharparenleft}{\kern0pt}w\ {\isasymin}\isactrlsub c\ {\isacharparenleft}{\kern0pt}{\isasymone}{\isasymCoprod}{\isasymone}{\isacharparenright}{\kern0pt}\ {\isasymand}\ x\ {\isacharequal}{\kern0pt}\ {\isacharparenleft}{\kern0pt}right{\isacharunderscore}{\kern0pt}coproj\ {\isasymone}\ {\isacharparenleft}{\kern0pt}{\isasymone}{\isasymCoprod}{\isasymone}{\isacharparenright}{\kern0pt}{\isacharparenright}{\kern0pt}\ {\isasymcirc}\isactrlsub c\ w{\isacharparenright}{\kern0pt}{\isacharparenright}{\kern0pt}{\isachardoublequoteclose}\isanewline
\ \ \ \ \isacommand{using}\isamarkupfalse%
\ coprojs{\isacharunderscore}{\kern0pt}jointly{\isacharunderscore}{\kern0pt}surj\ \isacommand{by}\isamarkupfalse%
\ auto\isanewline
\isanewline
\ \ \isacommand{assume}\isamarkupfalse%
\ {\isachardoublequoteopen}y\ {\isasymin}\isactrlsub c\ domain\ {\isacharparenleft}{\kern0pt}{\isasymlangle}{\isasymt}{\isacharcomma}{\kern0pt}{\isasymt}{\isasymrangle}\ {\isasymamalg}\ {\isasymlangle}{\isasymf}{\isacharcomma}{\kern0pt}\ {\isasymf}{\isasymrangle}\ {\isasymamalg}\ {\isasymlangle}{\isasymf}{\isacharcomma}{\kern0pt}{\isasymt}{\isasymrangle}{\isacharparenright}{\kern0pt}{\isachardoublequoteclose}\ \isanewline
\ \ \isacommand{then}\isamarkupfalse%
\ \isacommand{have}\isamarkupfalse%
\ y{\isacharunderscore}{\kern0pt}type{\isacharcolon}{\kern0pt}\ {\isachardoublequoteopen}y\ {\isasymin}\isactrlsub c\ {\isacharparenleft}{\kern0pt}{\isasymone}{\isasymCoprod}{\isacharparenleft}{\kern0pt}{\isasymone}{\isasymCoprod}{\isasymone}{\isacharparenright}{\kern0pt}{\isacharparenright}{\kern0pt}{\isachardoublequoteclose}\ \ \isanewline
\ \ \ \ \isacommand{using}\isamarkupfalse%
\ cfunc{\isacharunderscore}{\kern0pt}type{\isacharunderscore}{\kern0pt}def\ pre{\isacharunderscore}{\kern0pt}IMPLIES{\isacharunderscore}{\kern0pt}type\ \isacommand{by}\isamarkupfalse%
\ force\isanewline
\ \ \isacommand{then}\isamarkupfalse%
\ \isacommand{have}\isamarkupfalse%
\ y{\isacharunderscore}{\kern0pt}form{\isacharcolon}{\kern0pt}\ {\isachardoublequoteopen}{\isacharparenleft}{\kern0pt}{\isasymexists}\ w{\isachardot}{\kern0pt}\ {\isacharparenleft}{\kern0pt}w\ {\isasymin}\isactrlsub c\ {\isasymone}\ {\isasymand}\ y\ {\isacharequal}{\kern0pt}\ {\isacharparenleft}{\kern0pt}left{\isacharunderscore}{\kern0pt}coproj\ {\isasymone}\ {\isacharparenleft}{\kern0pt}{\isasymone}{\isasymCoprod}{\isasymone}{\isacharparenright}{\kern0pt}{\isacharparenright}{\kern0pt}\ {\isasymcirc}\isactrlsub c\ w{\isacharparenright}{\kern0pt}{\isacharparenright}{\kern0pt}\isanewline
\ \ \ \ \ \ {\isasymor}\ \ {\isacharparenleft}{\kern0pt}{\isasymexists}\ w{\isachardot}{\kern0pt}\ {\isacharparenleft}{\kern0pt}w\ {\isasymin}\isactrlsub c\ {\isacharparenleft}{\kern0pt}{\isasymone}{\isasymCoprod}{\isasymone}{\isacharparenright}{\kern0pt}\ {\isasymand}\ y\ {\isacharequal}{\kern0pt}\ {\isacharparenleft}{\kern0pt}right{\isacharunderscore}{\kern0pt}coproj\ {\isasymone}\ {\isacharparenleft}{\kern0pt}{\isasymone}{\isasymCoprod}{\isasymone}{\isacharparenright}{\kern0pt}{\isacharparenright}{\kern0pt}\ {\isasymcirc}\isactrlsub c\ w{\isacharparenright}{\kern0pt}{\isacharparenright}{\kern0pt}{\isachardoublequoteclose}\isanewline
\ \ \ \ \isacommand{using}\isamarkupfalse%
\ coprojs{\isacharunderscore}{\kern0pt}jointly{\isacharunderscore}{\kern0pt}surj\ \isacommand{by}\isamarkupfalse%
\ auto\isanewline
\isanewline
\ \ \isacommand{assume}\isamarkupfalse%
\ mx{\isacharunderscore}{\kern0pt}eqs{\isacharunderscore}{\kern0pt}my{\isacharcolon}{\kern0pt}\ {\isachardoublequoteopen}{\isasymlangle}{\isasymt}{\isacharcomma}{\kern0pt}{\isasymt}{\isasymrangle}\ {\isasymamalg}\ {\isasymlangle}{\isasymf}{\isacharcomma}{\kern0pt}\ {\isasymf}{\isasymrangle}\ {\isasymamalg}\ {\isasymlangle}{\isasymf}{\isacharcomma}{\kern0pt}{\isasymt}{\isasymrangle}\ {\isasymcirc}\isactrlsub c\ x\ {\isacharequal}{\kern0pt}\ {\isasymlangle}{\isasymt}{\isacharcomma}{\kern0pt}{\isasymt}{\isasymrangle}\ {\isasymamalg}\ {\isasymlangle}{\isasymf}{\isacharcomma}{\kern0pt}\ {\isasymf}{\isasymrangle}\ {\isasymamalg}\ {\isasymlangle}{\isasymf}{\isacharcomma}{\kern0pt}{\isasymt}{\isasymrangle}\ {\isasymcirc}\isactrlsub c\ y{\isachardoublequoteclose}\isanewline
\isanewline
\ \ \isacommand{have}\isamarkupfalse%
\ f{\isadigit{1}}{\isacharcolon}{\kern0pt}\ {\isachardoublequoteopen}{\isasymlangle}{\isasymt}{\isacharcomma}{\kern0pt}{\isasymt}{\isasymrangle}\ {\isasymamalg}\ {\isasymlangle}{\isasymf}{\isacharcomma}{\kern0pt}\ {\isasymf}{\isasymrangle}\ {\isasymamalg}\ {\isasymlangle}{\isasymf}{\isacharcomma}{\kern0pt}{\isasymt}{\isasymrangle}\ {\isasymcirc}\isactrlsub c\ left{\isacharunderscore}{\kern0pt}coproj\ {\isasymone}\ {\isacharparenleft}{\kern0pt}{\isasymone}\ {\isasymCoprod}\ {\isasymone}{\isacharparenright}{\kern0pt}\ {\isacharequal}{\kern0pt}\ {\isasymlangle}{\isasymt}{\isacharcomma}{\kern0pt}{\isasymt}{\isasymrangle}{\isachardoublequoteclose}\isanewline
\ \ \ \ \isacommand{by}\isamarkupfalse%
\ {\isacharparenleft}{\kern0pt}typecheck{\isacharunderscore}{\kern0pt}cfuncs{\isacharcomma}{\kern0pt}\ simp\ add{\isacharcolon}{\kern0pt}\ left{\isacharunderscore}{\kern0pt}coproj{\isacharunderscore}{\kern0pt}cfunc{\isacharunderscore}{\kern0pt}coprod{\isacharparenright}{\kern0pt}\isanewline
\ \ \isacommand{have}\isamarkupfalse%
\ f{\isadigit{2}}{\isacharcolon}{\kern0pt}\ {\isachardoublequoteopen}{\isasymlangle}{\isasymt}{\isacharcomma}{\kern0pt}{\isasymt}{\isasymrangle}\ {\isasymamalg}\ {\isasymlangle}{\isasymf}{\isacharcomma}{\kern0pt}\ {\isasymf}{\isasymrangle}\ {\isasymamalg}\ {\isasymlangle}{\isasymf}{\isacharcomma}{\kern0pt}{\isasymt}{\isasymrangle}\ {\isasymcirc}\isactrlsub c\ {\isacharparenleft}{\kern0pt}right{\isacharunderscore}{\kern0pt}coproj\ {\isasymone}\ {\isacharparenleft}{\kern0pt}{\isasymone}{\isasymCoprod}{\isasymone}{\isacharparenright}{\kern0pt}{\isasymcirc}\isactrlsub c\ left{\isacharunderscore}{\kern0pt}coproj\ {\isasymone}\ {\isasymone}{\isacharparenright}{\kern0pt}\ {\isacharequal}{\kern0pt}\ {\isasymlangle}{\isasymf}{\isacharcomma}{\kern0pt}\ {\isasymf}{\isasymrangle}{\isachardoublequoteclose}\isanewline
\ \ \isacommand{proof}\isamarkupfalse%
{\isacharminus}{\kern0pt}\ \isanewline
\ \ \ \ \isacommand{have}\isamarkupfalse%
\ {\isachardoublequoteopen}{\isasymlangle}{\isasymt}{\isacharcomma}{\kern0pt}{\isasymt}{\isasymrangle}\ {\isasymamalg}\ {\isasymlangle}{\isasymf}{\isacharcomma}{\kern0pt}\ {\isasymf}{\isasymrangle}\ {\isasymamalg}\ {\isasymlangle}{\isasymf}{\isacharcomma}{\kern0pt}{\isasymt}{\isasymrangle}\ {\isasymcirc}\isactrlsub c\ {\isacharparenleft}{\kern0pt}right{\isacharunderscore}{\kern0pt}coproj\ {\isasymone}\ {\isacharparenleft}{\kern0pt}{\isasymone}{\isasymCoprod}{\isasymone}{\isacharparenright}{\kern0pt}{\isasymcirc}\isactrlsub c\ left{\isacharunderscore}{\kern0pt}coproj\ {\isasymone}\ {\isasymone}{\isacharparenright}{\kern0pt}\ {\isacharequal}{\kern0pt}\ \isanewline
\ \ \ \ \ \ \ \ \ \ {\isacharparenleft}{\kern0pt}{\isasymlangle}{\isasymt}{\isacharcomma}{\kern0pt}{\isasymt}{\isasymrangle}\ {\isasymamalg}\ {\isasymlangle}{\isasymf}{\isacharcomma}{\kern0pt}\ {\isasymf}{\isasymrangle}\ {\isasymamalg}\ {\isasymlangle}{\isasymf}{\isacharcomma}{\kern0pt}{\isasymt}{\isasymrangle}\ {\isasymcirc}\isactrlsub c\ right{\isacharunderscore}{\kern0pt}coproj\ {\isasymone}\ {\isacharparenleft}{\kern0pt}{\isasymone}{\isasymCoprod}{\isasymone}{\isacharparenright}{\kern0pt}\ {\isacharparenright}{\kern0pt}{\isasymcirc}\isactrlsub c\ left{\isacharunderscore}{\kern0pt}coproj\ {\isasymone}\ {\isasymone}{\isachardoublequoteclose}\isanewline
\ \ \ \ \ \ \isacommand{by}\isamarkupfalse%
\ {\isacharparenleft}{\kern0pt}typecheck{\isacharunderscore}{\kern0pt}cfuncs{\isacharcomma}{\kern0pt}\ simp\ add{\isacharcolon}{\kern0pt}\ comp{\isacharunderscore}{\kern0pt}associative{\isadigit{2}}{\isacharparenright}{\kern0pt}\isanewline
\ \ \ \ \isacommand{also}\isamarkupfalse%
\ \isacommand{have}\isamarkupfalse%
\ {\isachardoublequoteopen}{\isachardot}{\kern0pt}{\isachardot}{\kern0pt}{\isachardot}{\kern0pt}\ {\isacharequal}{\kern0pt}\ {\isasymlangle}{\isasymf}{\isacharcomma}{\kern0pt}\ {\isasymf}{\isasymrangle}\ {\isasymamalg}\ {\isasymlangle}{\isasymf}{\isacharcomma}{\kern0pt}{\isasymt}{\isasymrangle}\ {\isasymcirc}\isactrlsub c\ left{\isacharunderscore}{\kern0pt}coproj\ {\isasymone}\ {\isasymone}{\isachardoublequoteclose}\isanewline
\ \ \ \ \ \ \isacommand{using}\isamarkupfalse%
\ right{\isacharunderscore}{\kern0pt}coproj{\isacharunderscore}{\kern0pt}cfunc{\isacharunderscore}{\kern0pt}coprod\ \isacommand{by}\isamarkupfalse%
\ {\isacharparenleft}{\kern0pt}typecheck{\isacharunderscore}{\kern0pt}cfuncs{\isacharcomma}{\kern0pt}\ smt{\isacharparenright}{\kern0pt}\isanewline
\ \ \ \ \isacommand{also}\isamarkupfalse%
\ \isacommand{have}\isamarkupfalse%
\ {\isachardoublequoteopen}{\isachardot}{\kern0pt}{\isachardot}{\kern0pt}{\isachardot}{\kern0pt}\ {\isacharequal}{\kern0pt}\ {\isasymlangle}{\isasymf}{\isacharcomma}{\kern0pt}\ {\isasymf}{\isasymrangle}{\isachardoublequoteclose}\isanewline
\ \ \ \ \ \ \isacommand{by}\isamarkupfalse%
\ {\isacharparenleft}{\kern0pt}typecheck{\isacharunderscore}{\kern0pt}cfuncs{\isacharcomma}{\kern0pt}\ simp\ add{\isacharcolon}{\kern0pt}\ left{\isacharunderscore}{\kern0pt}coproj{\isacharunderscore}{\kern0pt}cfunc{\isacharunderscore}{\kern0pt}coprod{\isacharparenright}{\kern0pt}\isanewline
\ \ \ \ \isacommand{then}\isamarkupfalse%
\ \isacommand{show}\isamarkupfalse%
\ {\isacharquery}{\kern0pt}thesis\isanewline
\ \ \ \ \ \ \isacommand{by}\isamarkupfalse%
\ {\isacharparenleft}{\kern0pt}simp\ add{\isacharcolon}{\kern0pt}\ calculation{\isacharparenright}{\kern0pt}\isanewline
\ \ \isacommand{qed}\isamarkupfalse%
\isanewline
\ \ \isacommand{have}\isamarkupfalse%
\ f{\isadigit{3}}{\isacharcolon}{\kern0pt}\ {\isachardoublequoteopen}{\isasymlangle}{\isasymt}{\isacharcomma}{\kern0pt}{\isasymt}{\isasymrangle}\ {\isasymamalg}\ {\isasymlangle}{\isasymf}{\isacharcomma}{\kern0pt}\ {\isasymf}{\isasymrangle}\ {\isasymamalg}\ {\isasymlangle}{\isasymf}{\isacharcomma}{\kern0pt}{\isasymt}{\isasymrangle}\ {\isasymcirc}\isactrlsub c\ {\isacharparenleft}{\kern0pt}right{\isacharunderscore}{\kern0pt}coproj\ {\isasymone}\ {\isacharparenleft}{\kern0pt}{\isasymone}{\isasymCoprod}{\isasymone}{\isacharparenright}{\kern0pt}{\isasymcirc}\isactrlsub c\ right{\isacharunderscore}{\kern0pt}coproj\ {\isasymone}\ {\isasymone}{\isacharparenright}{\kern0pt}\ {\isacharequal}{\kern0pt}\ {\isasymlangle}{\isasymf}{\isacharcomma}{\kern0pt}{\isasymt}{\isasymrangle}{\isachardoublequoteclose}\isanewline
\ \ \isacommand{proof}\isamarkupfalse%
{\isacharminus}{\kern0pt}\ \isanewline
\ \ \ \ \isacommand{have}\isamarkupfalse%
\ {\isachardoublequoteopen}{\isasymlangle}{\isasymt}{\isacharcomma}{\kern0pt}{\isasymt}{\isasymrangle}\ {\isasymamalg}\ {\isasymlangle}{\isasymf}{\isacharcomma}{\kern0pt}\ {\isasymf}{\isasymrangle}\ {\isasymamalg}\ {\isasymlangle}{\isasymf}{\isacharcomma}{\kern0pt}{\isasymt}{\isasymrangle}\ {\isasymcirc}\isactrlsub c\ right{\isacharunderscore}{\kern0pt}coproj\ {\isasymone}\ {\isacharparenleft}{\kern0pt}{\isasymone}{\isasymCoprod}{\isasymone}{\isacharparenright}{\kern0pt}{\isasymcirc}\isactrlsub c\ right{\isacharunderscore}{\kern0pt}coproj\ {\isasymone}\ {\isasymone}\ {\isacharequal}{\kern0pt}\ \isanewline
\ \ \ \ \ \ \ \ \ \ {\isacharparenleft}{\kern0pt}{\isasymlangle}{\isasymt}{\isacharcomma}{\kern0pt}{\isasymt}{\isasymrangle}\ {\isasymamalg}\ {\isasymlangle}{\isasymf}{\isacharcomma}{\kern0pt}\ {\isasymf}{\isasymrangle}\ {\isasymamalg}\ {\isasymlangle}{\isasymf}{\isacharcomma}{\kern0pt}{\isasymt}{\isasymrangle}\ {\isasymcirc}\isactrlsub c\ right{\isacharunderscore}{\kern0pt}coproj\ {\isasymone}\ {\isacharparenleft}{\kern0pt}{\isasymone}{\isasymCoprod}{\isasymone}{\isacharparenright}{\kern0pt}{\isacharparenright}{\kern0pt}{\isasymcirc}\isactrlsub c\ right{\isacharunderscore}{\kern0pt}coproj\ {\isasymone}\ {\isasymone}{\isachardoublequoteclose}\isanewline
\ \ \ \ \ \ \isacommand{by}\isamarkupfalse%
\ {\isacharparenleft}{\kern0pt}typecheck{\isacharunderscore}{\kern0pt}cfuncs{\isacharcomma}{\kern0pt}\ simp\ add{\isacharcolon}{\kern0pt}\ comp{\isacharunderscore}{\kern0pt}associative{\isadigit{2}}{\isacharparenright}{\kern0pt}\isanewline
\ \ \ \ \isacommand{also}\isamarkupfalse%
\ \isacommand{have}\isamarkupfalse%
\ {\isachardoublequoteopen}{\isachardot}{\kern0pt}{\isachardot}{\kern0pt}{\isachardot}{\kern0pt}\ {\isacharequal}{\kern0pt}\ {\isasymlangle}{\isasymf}{\isacharcomma}{\kern0pt}\ {\isasymf}{\isasymrangle}\ {\isasymamalg}\ {\isasymlangle}{\isasymf}{\isacharcomma}{\kern0pt}{\isasymt}{\isasymrangle}\ {\isasymcirc}\isactrlsub c\ right{\isacharunderscore}{\kern0pt}coproj\ {\isasymone}\ {\isasymone}{\isachardoublequoteclose}\isanewline
\ \ \ \ \ \ \isacommand{using}\isamarkupfalse%
\ right{\isacharunderscore}{\kern0pt}coproj{\isacharunderscore}{\kern0pt}cfunc{\isacharunderscore}{\kern0pt}coprod\ \isacommand{by}\isamarkupfalse%
\ {\isacharparenleft}{\kern0pt}typecheck{\isacharunderscore}{\kern0pt}cfuncs{\isacharcomma}{\kern0pt}\ smt{\isacharparenright}{\kern0pt}\isanewline
\ \ \ \ \isacommand{also}\isamarkupfalse%
\ \isacommand{have}\isamarkupfalse%
\ {\isachardoublequoteopen}{\isachardot}{\kern0pt}{\isachardot}{\kern0pt}{\isachardot}{\kern0pt}\ {\isacharequal}{\kern0pt}\ {\isasymlangle}{\isasymf}{\isacharcomma}{\kern0pt}{\isasymt}{\isasymrangle}{\isachardoublequoteclose}\isanewline
\ \ \ \ \ \ \isacommand{by}\isamarkupfalse%
\ {\isacharparenleft}{\kern0pt}typecheck{\isacharunderscore}{\kern0pt}cfuncs{\isacharcomma}{\kern0pt}\ simp\ add{\isacharcolon}{\kern0pt}\ right{\isacharunderscore}{\kern0pt}coproj{\isacharunderscore}{\kern0pt}cfunc{\isacharunderscore}{\kern0pt}coprod{\isacharparenright}{\kern0pt}\isanewline
\ \ \ \ \isacommand{then}\isamarkupfalse%
\ \isacommand{show}\isamarkupfalse%
\ {\isacharquery}{\kern0pt}thesis\isanewline
\ \ \ \ \ \ \isacommand{by}\isamarkupfalse%
\ {\isacharparenleft}{\kern0pt}simp\ add{\isacharcolon}{\kern0pt}\ calculation{\isacharparenright}{\kern0pt}\isanewline
\ \ \isacommand{qed}\isamarkupfalse%
\isanewline
\ \ \isacommand{show}\isamarkupfalse%
\ {\isachardoublequoteopen}x\ {\isacharequal}{\kern0pt}\ y{\isachardoublequoteclose}\isanewline
\ \ \isacommand{proof}\isamarkupfalse%
{\isacharparenleft}{\kern0pt}cases\ {\isachardoublequoteopen}x\ {\isacharequal}{\kern0pt}\ left{\isacharunderscore}{\kern0pt}coproj\ {\isasymone}\ {\isacharparenleft}{\kern0pt}{\isasymone}\ {\isasymCoprod}\ {\isasymone}{\isacharparenright}{\kern0pt}{\isachardoublequoteclose}{\isacharparenright}{\kern0pt}\isanewline
\ \ \ \ \isacommand{assume}\isamarkupfalse%
\ case{\isadigit{1}}{\isacharcolon}{\kern0pt}\ {\isachardoublequoteopen}x\ {\isacharequal}{\kern0pt}\ left{\isacharunderscore}{\kern0pt}coproj\ {\isasymone}\ {\isacharparenleft}{\kern0pt}{\isasymone}\ {\isasymCoprod}\ {\isasymone}{\isacharparenright}{\kern0pt}{\isachardoublequoteclose}\isanewline
\ \ \ \ \isacommand{then}\isamarkupfalse%
\ \isacommand{show}\isamarkupfalse%
\ {\isachardoublequoteopen}x\ {\isacharequal}{\kern0pt}\ y{\isachardoublequoteclose}\isanewline
\ \ \ \ \ \ \isacommand{by}\isamarkupfalse%
\ {\isacharparenleft}{\kern0pt}typecheck{\isacharunderscore}{\kern0pt}cfuncs{\isacharcomma}{\kern0pt}\ smt\ {\isacharparenleft}{\kern0pt}z{\isadigit{3}}{\isacharparenright}{\kern0pt}\ mx{\isacharunderscore}{\kern0pt}eqs{\isacharunderscore}{\kern0pt}my\ element{\isacharunderscore}{\kern0pt}pair{\isacharunderscore}{\kern0pt}eq\ f{\isadigit{1}}\ f{\isadigit{2}}\ f{\isadigit{3}}\ false{\isacharunderscore}{\kern0pt}func{\isacharunderscore}{\kern0pt}type\ maps{\isacharunderscore}{\kern0pt}into{\isacharunderscore}{\kern0pt}{\isadigit{1}}u{\isadigit{1}}\ terminal{\isacharunderscore}{\kern0pt}func{\isacharunderscore}{\kern0pt}unique\ true{\isacharunderscore}{\kern0pt}false{\isacharunderscore}{\kern0pt}distinct\ true{\isacharunderscore}{\kern0pt}func{\isacharunderscore}{\kern0pt}type\ x{\isacharunderscore}{\kern0pt}form\ y{\isacharunderscore}{\kern0pt}form{\isacharparenright}{\kern0pt}\isanewline
\ \ \isacommand{next}\isamarkupfalse%
\isanewline
\ \ \ \ \isacommand{assume}\isamarkupfalse%
\ not{\isacharunderscore}{\kern0pt}case{\isadigit{1}}{\isacharcolon}{\kern0pt}\ {\isachardoublequoteopen}x\ {\isasymnoteq}\ left{\isacharunderscore}{\kern0pt}coproj\ {\isasymone}\ {\isacharparenleft}{\kern0pt}{\isasymone}\ {\isasymCoprod}\ {\isasymone}{\isacharparenright}{\kern0pt}{\isachardoublequoteclose}\isanewline
\ \ \ \ \isacommand{then}\isamarkupfalse%
\ \isacommand{have}\isamarkupfalse%
\ case{\isadigit{2}}{\isacharunderscore}{\kern0pt}or{\isacharunderscore}{\kern0pt}{\isadigit{3}}{\isacharcolon}{\kern0pt}\ {\isachardoublequoteopen}x\ {\isacharequal}{\kern0pt}\ {\isacharparenleft}{\kern0pt}right{\isacharunderscore}{\kern0pt}coproj\ {\isasymone}\ {\isacharparenleft}{\kern0pt}{\isasymone}{\isasymCoprod}{\isasymone}{\isacharparenright}{\kern0pt}{\isasymcirc}\isactrlsub c\ left{\isacharunderscore}{\kern0pt}coproj\ {\isasymone}\ {\isasymone}{\isacharparenright}{\kern0pt}{\isasymor}\ \isanewline
\ \ \ \ \ \ \ \ \ \ \ \ \ \ \ x\ {\isacharequal}{\kern0pt}\ right{\isacharunderscore}{\kern0pt}coproj\ {\isasymone}\ {\isacharparenleft}{\kern0pt}{\isasymone}{\isasymCoprod}{\isasymone}{\isacharparenright}{\kern0pt}\ {\isasymcirc}\isactrlsub c{\isacharparenleft}{\kern0pt}right{\isacharunderscore}{\kern0pt}coproj\ {\isasymone}\ {\isasymone}{\isacharparenright}{\kern0pt}{\isachardoublequoteclose}\isanewline
\ \ \ \ \ \ \isacommand{by}\isamarkupfalse%
\ {\isacharparenleft}{\kern0pt}metis\ id{\isacharunderscore}{\kern0pt}right{\isacharunderscore}{\kern0pt}unit{\isadigit{2}}\ id{\isacharunderscore}{\kern0pt}type\ left{\isacharunderscore}{\kern0pt}proj{\isacharunderscore}{\kern0pt}type\ maps{\isacharunderscore}{\kern0pt}into{\isacharunderscore}{\kern0pt}{\isadigit{1}}u{\isadigit{1}}\ terminal{\isacharunderscore}{\kern0pt}func{\isacharunderscore}{\kern0pt}unique\ x{\isacharunderscore}{\kern0pt}form{\isacharparenright}{\kern0pt}\isanewline
\ \ \ \ \isacommand{show}\isamarkupfalse%
\ {\isachardoublequoteopen}x\ {\isacharequal}{\kern0pt}\ y{\isachardoublequoteclose}\isanewline
\ \ \ \ \isacommand{proof}\isamarkupfalse%
{\isacharparenleft}{\kern0pt}cases\ {\isachardoublequoteopen}x\ {\isacharequal}{\kern0pt}\ right{\isacharunderscore}{\kern0pt}coproj\ {\isasymone}\ {\isacharparenleft}{\kern0pt}{\isasymone}{\isasymCoprod}{\isasymone}{\isacharparenright}{\kern0pt}{\isasymcirc}\isactrlsub c\ left{\isacharunderscore}{\kern0pt}coproj\ {\isasymone}\ {\isasymone}{\isachardoublequoteclose}{\isacharparenright}{\kern0pt}\isanewline
\ \ \ \ \ \ \isacommand{assume}\isamarkupfalse%
\ case{\isadigit{2}}{\isacharcolon}{\kern0pt}\ {\isachardoublequoteopen}x\ {\isacharequal}{\kern0pt}\ right{\isacharunderscore}{\kern0pt}coproj\ {\isasymone}\ {\isacharparenleft}{\kern0pt}{\isasymone}\ {\isasymCoprod}\ {\isasymone}{\isacharparenright}{\kern0pt}\ {\isasymcirc}\isactrlsub c\ left{\isacharunderscore}{\kern0pt}coproj\ {\isasymone}\ {\isasymone}{\isachardoublequoteclose}\isanewline
\ \ \ \ \ \ \isacommand{then}\isamarkupfalse%
\ \isacommand{show}\isamarkupfalse%
\ {\isachardoublequoteopen}x\ {\isacharequal}{\kern0pt}\ y{\isachardoublequoteclose}\isanewline
\ \ \ \ \ \ \ \ \isacommand{by}\isamarkupfalse%
\ {\isacharparenleft}{\kern0pt}typecheck{\isacharunderscore}{\kern0pt}cfuncs{\isacharcomma}{\kern0pt}\ smt\ {\isacharparenleft}{\kern0pt}z{\isadigit{3}}{\isacharparenright}{\kern0pt}\ a{\isadigit{1}}\ NOT{\isacharunderscore}{\kern0pt}false{\isacharunderscore}{\kern0pt}is{\isacharunderscore}{\kern0pt}true\ NOT{\isacharunderscore}{\kern0pt}is{\isacharunderscore}{\kern0pt}pullback\ \ cart{\isacharunderscore}{\kern0pt}prod{\isacharunderscore}{\kern0pt}eq{\isadigit{2}}\ cfunc{\isacharunderscore}{\kern0pt}prod{\isacharunderscore}{\kern0pt}comp\ cfunc{\isacharunderscore}{\kern0pt}type{\isacharunderscore}{\kern0pt}def\ characteristic{\isacharunderscore}{\kern0pt}func{\isacharunderscore}{\kern0pt}eq\ characteristic{\isacharunderscore}{\kern0pt}func{\isacharunderscore}{\kern0pt}is{\isacharunderscore}{\kern0pt}pullback\ characteristic{\isacharunderscore}{\kern0pt}function{\isacharunderscore}{\kern0pt}exists\ comp{\isacharunderscore}{\kern0pt}associative\ element{\isacharunderscore}{\kern0pt}monomorphism\ f{\isadigit{1}}\ f{\isadigit{2}}\ f{\isadigit{3}}\ false{\isacharunderscore}{\kern0pt}func{\isacharunderscore}{\kern0pt}type\ left{\isacharunderscore}{\kern0pt}proj{\isacharunderscore}{\kern0pt}type\ maps{\isacharunderscore}{\kern0pt}into{\isacharunderscore}{\kern0pt}{\isadigit{1}}u{\isadigit{1}}\ mx{\isacharunderscore}{\kern0pt}eqs{\isacharunderscore}{\kern0pt}my\ terminal{\isacharunderscore}{\kern0pt}func{\isacharunderscore}{\kern0pt}unique\ true{\isacharunderscore}{\kern0pt}false{\isacharunderscore}{\kern0pt}distinct\ true{\isacharunderscore}{\kern0pt}func{\isacharunderscore}{\kern0pt}type\ y{\isacharunderscore}{\kern0pt}form{\isacharparenright}{\kern0pt}\isanewline
\ \ \ \ \isacommand{next}\isamarkupfalse%
\isanewline
\ \ \ \ \ \ \isacommand{assume}\isamarkupfalse%
\ not{\isacharunderscore}{\kern0pt}case{\isadigit{2}}{\isacharcolon}{\kern0pt}\ {\isachardoublequoteopen}x\ {\isasymnoteq}\ right{\isacharunderscore}{\kern0pt}coproj\ {\isasymone}\ {\isacharparenleft}{\kern0pt}{\isasymone}\ {\isasymCoprod}\ {\isasymone}{\isacharparenright}{\kern0pt}\ {\isasymcirc}\isactrlsub c\ left{\isacharunderscore}{\kern0pt}coproj\ {\isasymone}\ {\isasymone}{\isachardoublequoteclose}\isanewline
\ \ \ \ \ \ \isacommand{then}\isamarkupfalse%
\ \isacommand{have}\isamarkupfalse%
\ case{\isadigit{3}}{\isacharcolon}{\kern0pt}\ {\isachardoublequoteopen}x\ {\isacharequal}{\kern0pt}\ right{\isacharunderscore}{\kern0pt}coproj\ {\isasymone}\ {\isacharparenleft}{\kern0pt}{\isasymone}{\isasymCoprod}{\isasymone}{\isacharparenright}{\kern0pt}\ {\isasymcirc}\isactrlsub c{\isacharparenleft}{\kern0pt}right{\isacharunderscore}{\kern0pt}coproj\ {\isasymone}\ {\isasymone}{\isacharparenright}{\kern0pt}{\isachardoublequoteclose}\isanewline
\ \ \ \ \ \ \ \ \isacommand{using}\isamarkupfalse%
\ case{\isadigit{2}}{\isacharunderscore}{\kern0pt}or{\isacharunderscore}{\kern0pt}{\isadigit{3}}\ \isacommand{by}\isamarkupfalse%
\ blast\isanewline
\ \ \ \ \ \ \isacommand{then}\isamarkupfalse%
\ \isacommand{show}\isamarkupfalse%
\ {\isachardoublequoteopen}x\ {\isacharequal}{\kern0pt}\ y{\isachardoublequoteclose}\isanewline
\ \ \ \ \ \ \ \ \isacommand{by}\isamarkupfalse%
\ {\isacharparenleft}{\kern0pt}smt\ {\isacharparenleft}{\kern0pt}z{\isadigit{3}}{\isacharparenright}{\kern0pt}\ NOT{\isacharunderscore}{\kern0pt}false{\isacharunderscore}{\kern0pt}is{\isacharunderscore}{\kern0pt}true\ NOT{\isacharunderscore}{\kern0pt}is{\isacharunderscore}{\kern0pt}pullback\ a{\isadigit{1}}\ cart{\isacharunderscore}{\kern0pt}prod{\isacharunderscore}{\kern0pt}eq{\isadigit{2}}\ cfunc{\isacharunderscore}{\kern0pt}type{\isacharunderscore}{\kern0pt}def\ characteristic{\isacharunderscore}{\kern0pt}func{\isacharunderscore}{\kern0pt}eq\ characteristic{\isacharunderscore}{\kern0pt}func{\isacharunderscore}{\kern0pt}is{\isacharunderscore}{\kern0pt}pullback\ characteristic{\isacharunderscore}{\kern0pt}function{\isacharunderscore}{\kern0pt}exists\ comp{\isacharunderscore}{\kern0pt}associative\ diag{\isacharunderscore}{\kern0pt}on{\isacharunderscore}{\kern0pt}elements\ diagonal{\isacharunderscore}{\kern0pt}type\ element{\isacharunderscore}{\kern0pt}monomorphism\ f{\isadigit{1}}\ f{\isadigit{2}}\ f{\isadigit{3}}\ false{\isacharunderscore}{\kern0pt}func{\isacharunderscore}{\kern0pt}type\ left{\isacharunderscore}{\kern0pt}proj{\isacharunderscore}{\kern0pt}type\ maps{\isacharunderscore}{\kern0pt}into{\isacharunderscore}{\kern0pt}{\isadigit{1}}u{\isadigit{1}}\ mx{\isacharunderscore}{\kern0pt}eqs{\isacharunderscore}{\kern0pt}my\ terminal{\isacharunderscore}{\kern0pt}func{\isacharunderscore}{\kern0pt}unique\ true{\isacharunderscore}{\kern0pt}false{\isacharunderscore}{\kern0pt}distinct\ true{\isacharunderscore}{\kern0pt}func{\isacharunderscore}{\kern0pt}type\ x{\isacharunderscore}{\kern0pt}type\ y{\isacharunderscore}{\kern0pt}form{\isacharparenright}{\kern0pt}\isanewline
\ \ \ \ \isacommand{qed}\isamarkupfalse%
\isanewline
\ \ \isacommand{qed}\isamarkupfalse%
\isanewline
\isacommand{qed}\isamarkupfalse%
%
\endisatagproof
{\isafoldproof}%
%
\isadelimproof
\isanewline
%
\endisadelimproof
\isanewline
\isacommand{lemma}\isamarkupfalse%
\ IMPLIES{\isacharunderscore}{\kern0pt}is{\isacharunderscore}{\kern0pt}pullback{\isacharcolon}{\kern0pt}\isanewline
\ \ {\isachardoublequoteopen}is{\isacharunderscore}{\kern0pt}pullback\ {\isacharparenleft}{\kern0pt}{\isasymone}{\isasymCoprod}{\isacharparenleft}{\kern0pt}{\isasymone}{\isasymCoprod}{\isasymone}{\isacharparenright}{\kern0pt}{\isacharparenright}{\kern0pt}\ {\isasymone}\ {\isacharparenleft}{\kern0pt}{\isasymOmega}{\isasymtimes}\isactrlsub c{\isasymOmega}{\isacharparenright}{\kern0pt}\ {\isasymOmega}\ {\isacharparenleft}{\kern0pt}{\isasymbeta}\isactrlbsub {\isacharparenleft}{\kern0pt}{\isasymone}{\isasymCoprod}{\isacharparenleft}{\kern0pt}{\isasymone}{\isasymCoprod}{\isasymone}{\isacharparenright}{\kern0pt}{\isacharparenright}{\kern0pt}\isactrlesub {\isacharparenright}{\kern0pt}\ {\isasymt}\ {\isacharparenleft}{\kern0pt}{\isasymlangle}{\isasymt}{\isacharcomma}{\kern0pt}\ {\isasymt}{\isasymrangle}{\isasymamalg}\ {\isacharparenleft}{\kern0pt}{\isasymlangle}{\isasymf}{\isacharcomma}{\kern0pt}\ {\isasymf}{\isasymrangle}\ {\isasymamalg}{\isasymlangle}{\isasymf}{\isacharcomma}{\kern0pt}\ {\isasymt}{\isasymrangle}{\isacharparenright}{\kern0pt}{\isacharparenright}{\kern0pt}\ IMPLIES{\isachardoublequoteclose}\isanewline
%
\isadelimproof
\ \ %
\endisadelimproof
%
\isatagproof
\isacommand{unfolding}\isamarkupfalse%
\ IMPLIES{\isacharunderscore}{\kern0pt}def\isanewline
\ \ \isacommand{using}\isamarkupfalse%
\ element{\isacharunderscore}{\kern0pt}monomorphism\ characteristic{\isacharunderscore}{\kern0pt}function{\isacharunderscore}{\kern0pt}exists\isanewline
\ \ \isacommand{by}\isamarkupfalse%
\ {\isacharparenleft}{\kern0pt}typecheck{\isacharunderscore}{\kern0pt}cfuncs{\isacharcomma}{\kern0pt}\ simp\ add{\isacharcolon}{\kern0pt}\ the{\isadigit{1}}I{\isadigit{2}}\ injective{\isacharunderscore}{\kern0pt}imp{\isacharunderscore}{\kern0pt}monomorphism\ pre{\isacharunderscore}{\kern0pt}IMPLIES{\isacharunderscore}{\kern0pt}injective{\isacharparenright}{\kern0pt}%
\endisatagproof
{\isafoldproof}%
%
\isadelimproof
\isanewline
%
\endisadelimproof
\ \ \ \ \ \ \isanewline
\isacommand{lemma}\isamarkupfalse%
\ IMPLIES{\isacharunderscore}{\kern0pt}type{\isacharbrackleft}{\kern0pt}type{\isacharunderscore}{\kern0pt}rule{\isacharbrackright}{\kern0pt}{\isacharcolon}{\kern0pt}\isanewline
\ \ {\isachardoublequoteopen}IMPLIES\ {\isacharcolon}{\kern0pt}\ {\isasymOmega}\ {\isasymtimes}\isactrlsub c\ {\isasymOmega}\ {\isasymrightarrow}\ {\isasymOmega}{\isachardoublequoteclose}\isanewline
%
\isadelimproof
\ \ %
\endisadelimproof
%
\isatagproof
\isacommand{unfolding}\isamarkupfalse%
\ IMPLIES{\isacharunderscore}{\kern0pt}def\isanewline
\ \ \isacommand{by}\isamarkupfalse%
\ {\isacharparenleft}{\kern0pt}metis\ IMPLIES{\isacharunderscore}{\kern0pt}def\ IMPLIES{\isacharunderscore}{\kern0pt}is{\isacharunderscore}{\kern0pt}pullback\ is{\isacharunderscore}{\kern0pt}pullback{\isacharunderscore}{\kern0pt}def{\isacharparenright}{\kern0pt}%
\endisatagproof
{\isafoldproof}%
%
\isadelimproof
\isanewline
%
\endisadelimproof
\isanewline
\isacommand{lemma}\isamarkupfalse%
\ IMPLIES{\isacharunderscore}{\kern0pt}true{\isacharunderscore}{\kern0pt}true{\isacharunderscore}{\kern0pt}is{\isacharunderscore}{\kern0pt}true{\isacharcolon}{\kern0pt}\isanewline
\ \ {\isachardoublequoteopen}IMPLIES\ {\isasymcirc}\isactrlsub c\ {\isasymlangle}{\isasymt}{\isacharcomma}{\kern0pt}{\isasymt}{\isasymrangle}\ {\isacharequal}{\kern0pt}\ {\isasymt}{\isachardoublequoteclose}\isanewline
%
\isadelimproof
%
\endisadelimproof
%
\isatagproof
\isacommand{proof}\isamarkupfalse%
\ {\isacharminus}{\kern0pt}\ \ \ \isanewline
\ \ \isacommand{have}\isamarkupfalse%
\ {\isachardoublequoteopen}{\isasymexists}\ j{\isachardot}{\kern0pt}\ j\ {\isasymin}\isactrlsub c\ {\isasymone}\ {\isasymCoprod}\ {\isacharparenleft}{\kern0pt}{\isasymone}{\isasymCoprod}{\isasymone}{\isacharparenright}{\kern0pt}\ {\isasymand}\ {\isacharparenleft}{\kern0pt}{\isasymlangle}{\isasymt}{\isacharcomma}{\kern0pt}\ {\isasymt}{\isasymrangle}{\isasymamalg}\ {\isacharparenleft}{\kern0pt}{\isasymlangle}{\isasymf}{\isacharcomma}{\kern0pt}\ {\isasymf}{\isasymrangle}\ {\isasymamalg}{\isasymlangle}{\isasymf}{\isacharcomma}{\kern0pt}\ {\isasymt}{\isasymrangle}{\isacharparenright}{\kern0pt}{\isacharparenright}{\kern0pt}\ {\isasymcirc}\isactrlsub c\ j\ \ {\isacharequal}{\kern0pt}\ {\isasymlangle}{\isasymt}{\isacharcomma}{\kern0pt}{\isasymt}{\isasymrangle}{\isachardoublequoteclose}\isanewline
\ \ \ \ \isacommand{by}\isamarkupfalse%
\ {\isacharparenleft}{\kern0pt}typecheck{\isacharunderscore}{\kern0pt}cfuncs{\isacharcomma}{\kern0pt}\ meson\ left{\isacharunderscore}{\kern0pt}coproj{\isacharunderscore}{\kern0pt}cfunc{\isacharunderscore}{\kern0pt}coprod\ left{\isacharunderscore}{\kern0pt}proj{\isacharunderscore}{\kern0pt}type{\isacharparenright}{\kern0pt}\isanewline
\ \ \isacommand{then}\isamarkupfalse%
\ \isacommand{show}\isamarkupfalse%
\ {\isacharquery}{\kern0pt}thesis\isanewline
\ \ \ \ \isacommand{by}\isamarkupfalse%
\ {\isacharparenleft}{\kern0pt}smt\ {\isacharparenleft}{\kern0pt}verit{\isacharcomma}{\kern0pt}\ ccfv{\isacharunderscore}{\kern0pt}threshold{\isacharparenright}{\kern0pt}\ IMPLIES{\isacharunderscore}{\kern0pt}is{\isacharunderscore}{\kern0pt}pullback\ NOT{\isacharunderscore}{\kern0pt}false{\isacharunderscore}{\kern0pt}is{\isacharunderscore}{\kern0pt}true\ NOT{\isacharunderscore}{\kern0pt}is{\isacharunderscore}{\kern0pt}pullback\ comp{\isacharunderscore}{\kern0pt}associative{\isadigit{2}}\ is{\isacharunderscore}{\kern0pt}pullback{\isacharunderscore}{\kern0pt}def\ \ terminal{\isacharunderscore}{\kern0pt}func{\isacharunderscore}{\kern0pt}comp{\isacharparenright}{\kern0pt}\isanewline
\isacommand{qed}\isamarkupfalse%
%
\endisatagproof
{\isafoldproof}%
%
\isadelimproof
\ \isanewline
%
\endisadelimproof
\isanewline
\isacommand{lemma}\isamarkupfalse%
\ IMPLIES{\isacharunderscore}{\kern0pt}false{\isacharunderscore}{\kern0pt}true{\isacharunderscore}{\kern0pt}is{\isacharunderscore}{\kern0pt}true{\isacharcolon}{\kern0pt}\isanewline
\ \ {\isachardoublequoteopen}IMPLIES\ {\isasymcirc}\isactrlsub c\ {\isasymlangle}{\isasymf}{\isacharcomma}{\kern0pt}{\isasymt}{\isasymrangle}\ {\isacharequal}{\kern0pt}\ {\isasymt}{\isachardoublequoteclose}\isanewline
%
\isadelimproof
%
\endisadelimproof
%
\isatagproof
\isacommand{proof}\isamarkupfalse%
\ {\isacharminus}{\kern0pt}\ \ \ \isanewline
\ \ \isacommand{have}\isamarkupfalse%
\ {\isachardoublequoteopen}{\isasymexists}\ j{\isachardot}{\kern0pt}\ j\ {\isasymin}\isactrlsub c\ {\isasymone}{\isasymCoprod}{\isacharparenleft}{\kern0pt}{\isasymone}{\isasymCoprod}{\isasymone}{\isacharparenright}{\kern0pt}\ {\isasymand}\ {\isacharparenleft}{\kern0pt}{\isasymlangle}{\isasymt}{\isacharcomma}{\kern0pt}\ {\isasymt}{\isasymrangle}{\isasymamalg}\ {\isacharparenleft}{\kern0pt}{\isasymlangle}{\isasymf}{\isacharcomma}{\kern0pt}\ {\isasymf}{\isasymrangle}\ {\isasymamalg}{\isasymlangle}{\isasymf}{\isacharcomma}{\kern0pt}\ {\isasymt}{\isasymrangle}{\isacharparenright}{\kern0pt}{\isacharparenright}{\kern0pt}\ {\isasymcirc}\isactrlsub c\ j\ \ {\isacharequal}{\kern0pt}\ {\isasymlangle}{\isasymf}{\isacharcomma}{\kern0pt}{\isasymt}{\isasymrangle}{\isachardoublequoteclose}\isanewline
\ \ \ \ \isacommand{by}\isamarkupfalse%
\ {\isacharparenleft}{\kern0pt}typecheck{\isacharunderscore}{\kern0pt}cfuncs{\isacharcomma}{\kern0pt}\ smt\ {\isacharparenleft}{\kern0pt}z{\isadigit{3}}{\isacharparenright}{\kern0pt}\ comp{\isacharunderscore}{\kern0pt}associative{\isadigit{2}}\ comp{\isacharunderscore}{\kern0pt}type\ right{\isacharunderscore}{\kern0pt}coproj{\isacharunderscore}{\kern0pt}cfunc{\isacharunderscore}{\kern0pt}coprod\ right{\isacharunderscore}{\kern0pt}proj{\isacharunderscore}{\kern0pt}type{\isacharparenright}{\kern0pt}\isanewline
\ \ \isacommand{then}\isamarkupfalse%
\ \isacommand{show}\isamarkupfalse%
\ {\isacharquery}{\kern0pt}thesis\isanewline
\ \ \ \ \isacommand{by}\isamarkupfalse%
\ {\isacharparenleft}{\kern0pt}smt\ {\isacharparenleft}{\kern0pt}verit{\isacharcomma}{\kern0pt}\ ccfv{\isacharunderscore}{\kern0pt}threshold{\isacharparenright}{\kern0pt}\ IMPLIES{\isacharunderscore}{\kern0pt}is{\isacharunderscore}{\kern0pt}pullback\ NOT{\isacharunderscore}{\kern0pt}false{\isacharunderscore}{\kern0pt}is{\isacharunderscore}{\kern0pt}true\ NOT{\isacharunderscore}{\kern0pt}is{\isacharunderscore}{\kern0pt}pullback\ comp{\isacharunderscore}{\kern0pt}associative{\isadigit{2}}\ is{\isacharunderscore}{\kern0pt}pullback{\isacharunderscore}{\kern0pt}def\ \ terminal{\isacharunderscore}{\kern0pt}func{\isacharunderscore}{\kern0pt}comp{\isacharparenright}{\kern0pt}\isanewline
\isacommand{qed}\isamarkupfalse%
%
\endisatagproof
{\isafoldproof}%
%
\isadelimproof
\ \isanewline
%
\endisadelimproof
\isanewline
\isacommand{lemma}\isamarkupfalse%
\ IMPLIES{\isacharunderscore}{\kern0pt}false{\isacharunderscore}{\kern0pt}false{\isacharunderscore}{\kern0pt}is{\isacharunderscore}{\kern0pt}true{\isacharcolon}{\kern0pt}\isanewline
\ \ {\isachardoublequoteopen}IMPLIES\ {\isasymcirc}\isactrlsub c\ \ {\isasymlangle}{\isasymf}{\isacharcomma}{\kern0pt}{\isasymf}{\isasymrangle}\ {\isacharequal}{\kern0pt}\ {\isasymt}{\isachardoublequoteclose}\isanewline
%
\isadelimproof
%
\endisadelimproof
%
\isatagproof
\isacommand{proof}\isamarkupfalse%
\ {\isacharminus}{\kern0pt}\ \ \ \isanewline
\ \ \isacommand{have}\isamarkupfalse%
\ {\isachardoublequoteopen}{\isasymexists}\ j{\isachardot}{\kern0pt}\ j\ {\isasymin}\isactrlsub c\ {\isasymone}{\isasymCoprod}{\isacharparenleft}{\kern0pt}{\isasymone}{\isasymCoprod}{\isasymone}{\isacharparenright}{\kern0pt}\ {\isasymand}\ {\isacharparenleft}{\kern0pt}{\isasymlangle}{\isasymt}{\isacharcomma}{\kern0pt}\ {\isasymt}{\isasymrangle}{\isasymamalg}\ {\isacharparenleft}{\kern0pt}{\isasymlangle}{\isasymf}{\isacharcomma}{\kern0pt}\ {\isasymf}{\isasymrangle}\ {\isasymamalg}{\isasymlangle}{\isasymf}{\isacharcomma}{\kern0pt}\ {\isasymt}{\isasymrangle}{\isacharparenright}{\kern0pt}{\isacharparenright}{\kern0pt}\ {\isasymcirc}\isactrlsub c\ j\ \ {\isacharequal}{\kern0pt}\ {\isasymlangle}{\isasymf}{\isacharcomma}{\kern0pt}{\isasymf}{\isasymrangle}{\isachardoublequoteclose}\isanewline
\ \ \ \ \isacommand{by}\isamarkupfalse%
\ {\isacharparenleft}{\kern0pt}typecheck{\isacharunderscore}{\kern0pt}cfuncs{\isacharcomma}{\kern0pt}\ smt\ {\isacharparenleft}{\kern0pt}verit{\isacharcomma}{\kern0pt}\ ccfv{\isacharunderscore}{\kern0pt}SIG{\isacharparenright}{\kern0pt}\ cfunc{\isacharunderscore}{\kern0pt}type{\isacharunderscore}{\kern0pt}def\ comp{\isacharunderscore}{\kern0pt}associative\ comp{\isacharunderscore}{\kern0pt}type\ left{\isacharunderscore}{\kern0pt}coproj{\isacharunderscore}{\kern0pt}cfunc{\isacharunderscore}{\kern0pt}coprod\ left{\isacharunderscore}{\kern0pt}proj{\isacharunderscore}{\kern0pt}type\ right{\isacharunderscore}{\kern0pt}coproj{\isacharunderscore}{\kern0pt}cfunc{\isacharunderscore}{\kern0pt}coprod\ right{\isacharunderscore}{\kern0pt}proj{\isacharunderscore}{\kern0pt}type{\isacharparenright}{\kern0pt}\isanewline
\ \ \isacommand{then}\isamarkupfalse%
\ \isacommand{show}\isamarkupfalse%
\ {\isacharquery}{\kern0pt}thesis\isanewline
\ \ \ \ \isacommand{by}\isamarkupfalse%
\ {\isacharparenleft}{\kern0pt}smt\ {\isacharparenleft}{\kern0pt}verit{\isacharcomma}{\kern0pt}\ ccfv{\isacharunderscore}{\kern0pt}threshold{\isacharparenright}{\kern0pt}\ IMPLIES{\isacharunderscore}{\kern0pt}is{\isacharunderscore}{\kern0pt}pullback\ NOT{\isacharunderscore}{\kern0pt}false{\isacharunderscore}{\kern0pt}is{\isacharunderscore}{\kern0pt}true\ NOT{\isacharunderscore}{\kern0pt}is{\isacharunderscore}{\kern0pt}pullback\ comp{\isacharunderscore}{\kern0pt}associative{\isadigit{2}}\ is{\isacharunderscore}{\kern0pt}pullback{\isacharunderscore}{\kern0pt}def\ \ terminal{\isacharunderscore}{\kern0pt}func{\isacharunderscore}{\kern0pt}comp{\isacharparenright}{\kern0pt}\isanewline
\isacommand{qed}\isamarkupfalse%
%
\endisatagproof
{\isafoldproof}%
%
\isadelimproof
\ \isanewline
%
\endisadelimproof
\isanewline
\isacommand{lemma}\isamarkupfalse%
\ IMPLIES{\isacharunderscore}{\kern0pt}true{\isacharunderscore}{\kern0pt}false{\isacharunderscore}{\kern0pt}is{\isacharunderscore}{\kern0pt}false{\isacharcolon}{\kern0pt}\isanewline
\ \ {\isachardoublequoteopen}IMPLIES\ {\isasymcirc}\isactrlsub c\ \ {\isasymlangle}{\isasymt}{\isacharcomma}{\kern0pt}{\isasymf}{\isasymrangle}\ {\isacharequal}{\kern0pt}\ {\isasymf}{\isachardoublequoteclose}\isanewline
%
\isadelimproof
%
\endisadelimproof
%
\isatagproof
\isacommand{proof}\isamarkupfalse%
{\isacharparenleft}{\kern0pt}rule\ ccontr{\isacharparenright}{\kern0pt}\ \ \isanewline
\ \ \isacommand{assume}\isamarkupfalse%
\ {\isachardoublequoteopen}IMPLIES\ {\isasymcirc}\isactrlsub c\ {\isasymlangle}{\isasymt}{\isacharcomma}{\kern0pt}{\isasymf}{\isasymrangle}\ {\isasymnoteq}\ {\isasymf}{\isachardoublequoteclose}\isanewline
\ \ \isacommand{then}\isamarkupfalse%
\ \isacommand{have}\isamarkupfalse%
\ {\isachardoublequoteopen}IMPLIES\ {\isasymcirc}\isactrlsub c\ {\isasymlangle}{\isasymt}{\isacharcomma}{\kern0pt}{\isasymf}{\isasymrangle}\ {\isacharequal}{\kern0pt}\ {\isasymt}{\isachardoublequoteclose}\isanewline
\ \ \ \ \isacommand{using}\isamarkupfalse%
\ true{\isacharunderscore}{\kern0pt}false{\isacharunderscore}{\kern0pt}only{\isacharunderscore}{\kern0pt}truth{\isacharunderscore}{\kern0pt}values\ \isacommand{by}\isamarkupfalse%
\ {\isacharparenleft}{\kern0pt}typecheck{\isacharunderscore}{\kern0pt}cfuncs{\isacharcomma}{\kern0pt}\ blast{\isacharparenright}{\kern0pt}\isanewline
\ \ \isacommand{then}\isamarkupfalse%
\ \isacommand{obtain}\isamarkupfalse%
\ j\ \isakeyword{where}\ j{\isacharunderscore}{\kern0pt}def{\isacharcolon}{\kern0pt}\ {\isachardoublequoteopen}j\ {\isasymin}\isactrlsub c\ {\isasymone}{\isasymCoprod}{\isacharparenleft}{\kern0pt}{\isasymone}{\isasymCoprod}{\isasymone}{\isacharparenright}{\kern0pt}\ {\isasymand}\ {\isacharparenleft}{\kern0pt}{\isasymlangle}{\isasymt}{\isacharcomma}{\kern0pt}\ {\isasymt}{\isasymrangle}{\isasymamalg}\ {\isacharparenleft}{\kern0pt}{\isasymlangle}{\isasymf}{\isacharcomma}{\kern0pt}\ {\isasymf}{\isasymrangle}\ {\isasymamalg}{\isasymlangle}{\isasymf}{\isacharcomma}{\kern0pt}\ {\isasymt}{\isasymrangle}{\isacharparenright}{\kern0pt}{\isacharparenright}{\kern0pt}\ {\isasymcirc}\isactrlsub c\ j\ \ {\isacharequal}{\kern0pt}\ {\isasymlangle}{\isasymt}{\isacharcomma}{\kern0pt}{\isasymf}{\isasymrangle}{\isachardoublequoteclose}\isanewline
\ \ \ \ \isacommand{by}\isamarkupfalse%
\ {\isacharparenleft}{\kern0pt}typecheck{\isacharunderscore}{\kern0pt}cfuncs{\isacharcomma}{\kern0pt}\ smt\ {\isacharparenleft}{\kern0pt}verit{\isacharcomma}{\kern0pt}\ ccfv{\isacharunderscore}{\kern0pt}threshold{\isacharparenright}{\kern0pt}\ IMPLIES{\isacharunderscore}{\kern0pt}is{\isacharunderscore}{\kern0pt}pullback\ \ id{\isacharunderscore}{\kern0pt}right{\isacharunderscore}{\kern0pt}unit{\isadigit{2}}\ is{\isacharunderscore}{\kern0pt}pullback{\isacharunderscore}{\kern0pt}def\ one{\isacharunderscore}{\kern0pt}unique{\isacharunderscore}{\kern0pt}element\ terminal{\isacharunderscore}{\kern0pt}func{\isacharunderscore}{\kern0pt}comp\ terminal{\isacharunderscore}{\kern0pt}func{\isacharunderscore}{\kern0pt}comp{\isacharunderscore}{\kern0pt}elem\ terminal{\isacharunderscore}{\kern0pt}func{\isacharunderscore}{\kern0pt}unique{\isacharparenright}{\kern0pt}\isanewline
\ \ \isacommand{show}\isamarkupfalse%
\ False\isanewline
\ \ \isacommand{proof}\isamarkupfalse%
{\isacharparenleft}{\kern0pt}cases\ {\isachardoublequoteopen}j\ {\isacharequal}{\kern0pt}\ left{\isacharunderscore}{\kern0pt}coproj\ {\isasymone}\ {\isacharparenleft}{\kern0pt}{\isasymone}{\isasymCoprod}{\isasymone}{\isacharparenright}{\kern0pt}{\isachardoublequoteclose}{\isacharparenright}{\kern0pt}\isanewline
\ \ \ \ \isacommand{assume}\isamarkupfalse%
\ case{\isadigit{1}}{\isacharcolon}{\kern0pt}\ {\isachardoublequoteopen}j\ {\isacharequal}{\kern0pt}\ left{\isacharunderscore}{\kern0pt}coproj\ {\isasymone}\ {\isacharparenleft}{\kern0pt}{\isasymone}{\isasymCoprod}{\isasymone}{\isacharparenright}{\kern0pt}{\isachardoublequoteclose}\isanewline
\ \ \ \ \isacommand{show}\isamarkupfalse%
\ False\isanewline
\ \ \ \ \isacommand{proof}\isamarkupfalse%
\ {\isacharminus}{\kern0pt}\ \isanewline
\ \ \ \ \ \ \isacommand{have}\isamarkupfalse%
\ {\isachardoublequoteopen}{\isacharparenleft}{\kern0pt}{\isasymlangle}{\isasymt}{\isacharcomma}{\kern0pt}\ {\isasymt}{\isasymrangle}{\isasymamalg}\ {\isacharparenleft}{\kern0pt}{\isasymlangle}{\isasymf}{\isacharcomma}{\kern0pt}\ {\isasymf}{\isasymrangle}\ {\isasymamalg}{\isasymlangle}{\isasymf}{\isacharcomma}{\kern0pt}\ {\isasymt}{\isasymrangle}{\isacharparenright}{\kern0pt}{\isacharparenright}{\kern0pt}\ {\isasymcirc}\isactrlsub c\ j\ {\isacharequal}{\kern0pt}\ {\isasymlangle}{\isasymt}{\isacharcomma}{\kern0pt}\ {\isasymt}{\isasymrangle}{\isachardoublequoteclose}\isanewline
\ \ \ \ \ \ \ \ \isacommand{by}\isamarkupfalse%
\ {\isacharparenleft}{\kern0pt}typecheck{\isacharunderscore}{\kern0pt}cfuncs{\isacharcomma}{\kern0pt}\ simp\ add{\isacharcolon}{\kern0pt}\ case{\isadigit{1}}\ left{\isacharunderscore}{\kern0pt}coproj{\isacharunderscore}{\kern0pt}cfunc{\isacharunderscore}{\kern0pt}coprod{\isacharparenright}{\kern0pt}\isanewline
\ \ \ \ \ \ \isacommand{then}\isamarkupfalse%
\ \isacommand{have}\isamarkupfalse%
\ {\isachardoublequoteopen}{\isasymlangle}{\isasymt}{\isacharcomma}{\kern0pt}\ {\isasymt}{\isasymrangle}\ {\isacharequal}{\kern0pt}\ {\isasymlangle}{\isasymt}{\isacharcomma}{\kern0pt}{\isasymf}{\isasymrangle}{\isachardoublequoteclose}\isanewline
\ \ \ \ \ \ \ \ \isacommand{using}\isamarkupfalse%
\ j{\isacharunderscore}{\kern0pt}def\ \isacommand{by}\isamarkupfalse%
\ presburger\isanewline
\ \ \ \ \ \ \isacommand{then}\isamarkupfalse%
\ \isacommand{have}\isamarkupfalse%
\ {\isachardoublequoteopen}{\isasymt}\ {\isacharequal}{\kern0pt}\ {\isasymf}{\isachardoublequoteclose}\isanewline
\ \ \ \ \ \ \ \ \isacommand{using}\isamarkupfalse%
\ IFF{\isacharunderscore}{\kern0pt}true{\isacharunderscore}{\kern0pt}false{\isacharunderscore}{\kern0pt}is{\isacharunderscore}{\kern0pt}false\ IFF{\isacharunderscore}{\kern0pt}true{\isacharunderscore}{\kern0pt}true{\isacharunderscore}{\kern0pt}is{\isacharunderscore}{\kern0pt}true\ \isacommand{by}\isamarkupfalse%
\ auto\isanewline
\ \ \ \ \ \ \isacommand{then}\isamarkupfalse%
\ \isacommand{show}\isamarkupfalse%
\ False\isanewline
\ \ \ \ \ \ \ \ \isacommand{using}\isamarkupfalse%
\ true{\isacharunderscore}{\kern0pt}false{\isacharunderscore}{\kern0pt}distinct\ \isacommand{by}\isamarkupfalse%
\ blast\isanewline
\ \ \ \ \isacommand{qed}\isamarkupfalse%
\isanewline
\ \ \isacommand{next}\isamarkupfalse%
\isanewline
\ \ \ \ \isacommand{assume}\isamarkupfalse%
\ {\isachardoublequoteopen}j\ {\isasymnoteq}\ left{\isacharunderscore}{\kern0pt}coproj\ {\isasymone}\ {\isacharparenleft}{\kern0pt}{\isasymone}\ {\isasymCoprod}\ {\isasymone}{\isacharparenright}{\kern0pt}{\isachardoublequoteclose}\isanewline
\ \ \ \ \isacommand{then}\isamarkupfalse%
\ \isacommand{have}\isamarkupfalse%
\ case{\isadigit{2}}{\isacharunderscore}{\kern0pt}or{\isacharunderscore}{\kern0pt}{\isadigit{3}}{\isacharcolon}{\kern0pt}\ {\isachardoublequoteopen}j\ {\isacharequal}{\kern0pt}\ right{\isacharunderscore}{\kern0pt}coproj\ {\isasymone}\ {\isacharparenleft}{\kern0pt}{\isasymone}{\isasymCoprod}{\isasymone}{\isacharparenright}{\kern0pt}{\isasymcirc}\isactrlsub c\ left{\isacharunderscore}{\kern0pt}coproj\ {\isasymone}\ {\isasymone}\ {\isasymor}\ \isanewline
\ \ \ \ \ \ \ \ \ \ \ \ \ \ \ \ \ \ \ \ \ \ j\ {\isacharequal}{\kern0pt}\ right{\isacharunderscore}{\kern0pt}coproj\ {\isasymone}\ {\isacharparenleft}{\kern0pt}{\isasymone}{\isasymCoprod}{\isasymone}{\isacharparenright}{\kern0pt}\ {\isasymcirc}\isactrlsub c\ right{\isacharunderscore}{\kern0pt}coproj\ {\isasymone}\ {\isasymone}{\isachardoublequoteclose}\isanewline
\ \ \ \ \ \ \isacommand{by}\isamarkupfalse%
\ {\isacharparenleft}{\kern0pt}metis\ coprojs{\isacharunderscore}{\kern0pt}jointly{\isacharunderscore}{\kern0pt}surj\ id{\isacharunderscore}{\kern0pt}right{\isacharunderscore}{\kern0pt}unit{\isadigit{2}}\ id{\isacharunderscore}{\kern0pt}type\ j{\isacharunderscore}{\kern0pt}def\ left{\isacharunderscore}{\kern0pt}proj{\isacharunderscore}{\kern0pt}type\ maps{\isacharunderscore}{\kern0pt}into{\isacharunderscore}{\kern0pt}{\isadigit{1}}u{\isadigit{1}}\ one{\isacharunderscore}{\kern0pt}unique{\isacharunderscore}{\kern0pt}element{\isacharparenright}{\kern0pt}\isanewline
\ \ \ \ \isacommand{show}\isamarkupfalse%
\ False\isanewline
\ \ \ \ \isacommand{proof}\isamarkupfalse%
{\isacharparenleft}{\kern0pt}cases\ {\isachardoublequoteopen}j\ {\isacharequal}{\kern0pt}\ right{\isacharunderscore}{\kern0pt}coproj\ {\isasymone}\ {\isacharparenleft}{\kern0pt}{\isasymone}{\isasymCoprod}{\isasymone}{\isacharparenright}{\kern0pt}{\isasymcirc}\isactrlsub c\ left{\isacharunderscore}{\kern0pt}coproj\ {\isasymone}\ {\isasymone}{\isachardoublequoteclose}{\isacharparenright}{\kern0pt}\isanewline
\ \ \ \ \ \ \isacommand{assume}\isamarkupfalse%
\ case{\isadigit{2}}{\isacharcolon}{\kern0pt}\ {\isachardoublequoteopen}j\ {\isacharequal}{\kern0pt}\ right{\isacharunderscore}{\kern0pt}coproj\ {\isasymone}\ {\isacharparenleft}{\kern0pt}{\isasymone}{\isasymCoprod}{\isasymone}{\isacharparenright}{\kern0pt}{\isasymcirc}\isactrlsub c\ left{\isacharunderscore}{\kern0pt}coproj\ {\isasymone}\ {\isasymone}{\isachardoublequoteclose}\isanewline
\ \ \ \ \ \ \isacommand{show}\isamarkupfalse%
\ False\isanewline
\ \ \ \ \ \ \isacommand{proof}\isamarkupfalse%
\ {\isacharminus}{\kern0pt}\ \isanewline
\ \ \ \ \ \ \ \ \isacommand{have}\isamarkupfalse%
\ {\isachardoublequoteopen}{\isacharparenleft}{\kern0pt}{\isasymlangle}{\isasymt}{\isacharcomma}{\kern0pt}\ {\isasymt}{\isasymrangle}{\isasymamalg}\ {\isacharparenleft}{\kern0pt}{\isasymlangle}{\isasymf}{\isacharcomma}{\kern0pt}\ {\isasymf}{\isasymrangle}\ {\isasymamalg}{\isasymlangle}{\isasymf}{\isacharcomma}{\kern0pt}\ {\isasymt}{\isasymrangle}{\isacharparenright}{\kern0pt}{\isacharparenright}{\kern0pt}\ {\isasymcirc}\isactrlsub c\ j\ {\isacharequal}{\kern0pt}\ {\isasymlangle}{\isasymf}{\isacharcomma}{\kern0pt}\ {\isasymf}{\isasymrangle}{\isachardoublequoteclose}\isanewline
\ \ \ \ \ \ \ \ \ \ \isacommand{by}\isamarkupfalse%
\ {\isacharparenleft}{\kern0pt}typecheck{\isacharunderscore}{\kern0pt}cfuncs{\isacharcomma}{\kern0pt}\ smt\ {\isacharparenleft}{\kern0pt}z{\isadigit{3}}{\isacharparenright}{\kern0pt}\ case{\isadigit{2}}\ comp{\isacharunderscore}{\kern0pt}associative{\isadigit{2}}\ left{\isacharunderscore}{\kern0pt}coproj{\isacharunderscore}{\kern0pt}cfunc{\isacharunderscore}{\kern0pt}coprod\ left{\isacharunderscore}{\kern0pt}proj{\isacharunderscore}{\kern0pt}type\ right{\isacharunderscore}{\kern0pt}coproj{\isacharunderscore}{\kern0pt}cfunc{\isacharunderscore}{\kern0pt}coprod\ right{\isacharunderscore}{\kern0pt}proj{\isacharunderscore}{\kern0pt}type{\isacharparenright}{\kern0pt}\isanewline
\ \ \ \ \ \ \ \ \isacommand{then}\isamarkupfalse%
\ \isacommand{have}\isamarkupfalse%
\ {\isachardoublequoteopen}{\isasymlangle}{\isasymt}{\isacharcomma}{\kern0pt}\ {\isasymt}{\isasymrangle}\ {\isacharequal}{\kern0pt}\ {\isasymlangle}{\isasymf}{\isacharcomma}{\kern0pt}{\isasymf}{\isasymrangle}{\isachardoublequoteclose}\isanewline
\ \ \ \ \ \ \ \ \ \ \isacommand{using}\isamarkupfalse%
\ XOR{\isacharunderscore}{\kern0pt}false{\isacharunderscore}{\kern0pt}false{\isacharunderscore}{\kern0pt}is{\isacharunderscore}{\kern0pt}false\ XOR{\isacharunderscore}{\kern0pt}only{\isacharunderscore}{\kern0pt}true{\isacharunderscore}{\kern0pt}left{\isacharunderscore}{\kern0pt}is{\isacharunderscore}{\kern0pt}true\ j{\isacharunderscore}{\kern0pt}def\ \isacommand{by}\isamarkupfalse%
\ auto\isanewline
\ \ \ \ \ \ \ \ \isacommand{then}\isamarkupfalse%
\ \isacommand{have}\isamarkupfalse%
\ {\isachardoublequoteopen}{\isasymt}\ {\isacharequal}{\kern0pt}\ {\isasymf}{\isachardoublequoteclose}\isanewline
\ \ \ \ \ \ \ \ \ \ \isacommand{by}\isamarkupfalse%
\ {\isacharparenleft}{\kern0pt}metis\ XOR{\isacharunderscore}{\kern0pt}only{\isacharunderscore}{\kern0pt}true{\isacharunderscore}{\kern0pt}left{\isacharunderscore}{\kern0pt}is{\isacharunderscore}{\kern0pt}true\ XOR{\isacharunderscore}{\kern0pt}true{\isacharunderscore}{\kern0pt}true{\isacharunderscore}{\kern0pt}is{\isacharunderscore}{\kern0pt}false\ {\isacartoucheopen}{\isasymlangle}{\isasymt}{\isacharcomma}{\kern0pt}{\isasymt}{\isasymrangle}\ {\isasymamalg}\ {\isasymlangle}{\isasymf}{\isacharcomma}{\kern0pt}{\isasymf}{\isasymrangle}\ {\isasymamalg}\ {\isasymlangle}{\isasymf}{\isacharcomma}{\kern0pt}{\isasymt}{\isasymrangle}\ {\isasymcirc}\isactrlsub c\ j\ {\isacharequal}{\kern0pt}\ {\isasymlangle}{\isasymf}{\isacharcomma}{\kern0pt}{\isasymf}{\isasymrangle}{\isacartoucheclose}\ j{\isacharunderscore}{\kern0pt}def{\isacharparenright}{\kern0pt}\isanewline
\ \ \ \ \ \ \ \ \isacommand{then}\isamarkupfalse%
\ \isacommand{show}\isamarkupfalse%
\ False\isanewline
\ \ \ \ \ \ \ \ \ \ \isacommand{using}\isamarkupfalse%
\ true{\isacharunderscore}{\kern0pt}false{\isacharunderscore}{\kern0pt}distinct\ \isacommand{by}\isamarkupfalse%
\ blast\isanewline
\ \ \ \ \ \ \isacommand{qed}\isamarkupfalse%
\isanewline
\ \ \ \ \isacommand{next}\isamarkupfalse%
\isanewline
\ \ \ \ \ \ \isacommand{assume}\isamarkupfalse%
\ {\isachardoublequoteopen}j\ {\isasymnoteq}\ right{\isacharunderscore}{\kern0pt}coproj\ {\isasymone}\ {\isacharparenleft}{\kern0pt}{\isasymone}\ {\isasymCoprod}\ {\isasymone}{\isacharparenright}{\kern0pt}\ {\isasymcirc}\isactrlsub c\ left{\isacharunderscore}{\kern0pt}coproj\ {\isasymone}\ {\isasymone}{\isachardoublequoteclose}\isanewline
\ \ \ \ \ \ \isacommand{then}\isamarkupfalse%
\ \isacommand{have}\isamarkupfalse%
\ case{\isadigit{3}}{\isacharcolon}{\kern0pt}\ {\isachardoublequoteopen}j\ {\isacharequal}{\kern0pt}\ right{\isacharunderscore}{\kern0pt}coproj\ {\isasymone}\ {\isacharparenleft}{\kern0pt}{\isasymone}{\isasymCoprod}{\isasymone}{\isacharparenright}{\kern0pt}\ {\isasymcirc}\isactrlsub c\ right{\isacharunderscore}{\kern0pt}coproj\ {\isasymone}\ {\isasymone}{\isachardoublequoteclose}\isanewline
\ \ \ \ \ \ \ \ \isacommand{using}\isamarkupfalse%
\ case{\isadigit{2}}{\isacharunderscore}{\kern0pt}or{\isacharunderscore}{\kern0pt}{\isadigit{3}}\ \isacommand{by}\isamarkupfalse%
\ blast\isanewline
\ \ \ \ \ \ \isacommand{show}\isamarkupfalse%
\ False\isanewline
\ \ \ \ \ \ \isacommand{proof}\isamarkupfalse%
\ {\isacharminus}{\kern0pt}\ \isanewline
\ \ \ \ \ \ \ \ \isacommand{have}\isamarkupfalse%
\ {\isachardoublequoteopen}{\isacharparenleft}{\kern0pt}{\isasymlangle}{\isasymt}{\isacharcomma}{\kern0pt}\ {\isasymt}{\isasymrangle}{\isasymamalg}\ {\isacharparenleft}{\kern0pt}{\isasymlangle}{\isasymf}{\isacharcomma}{\kern0pt}\ {\isasymf}{\isasymrangle}\ {\isasymamalg}{\isasymlangle}{\isasymf}{\isacharcomma}{\kern0pt}\ {\isasymt}{\isasymrangle}{\isacharparenright}{\kern0pt}{\isacharparenright}{\kern0pt}\ {\isasymcirc}\isactrlsub c\ j\ {\isacharequal}{\kern0pt}\ {\isasymlangle}{\isasymf}{\isacharcomma}{\kern0pt}\ {\isasymt}{\isasymrangle}{\isachardoublequoteclose}\isanewline
\ \ \ \ \ \ \ \ \ \ \isacommand{by}\isamarkupfalse%
\ {\isacharparenleft}{\kern0pt}typecheck{\isacharunderscore}{\kern0pt}cfuncs{\isacharcomma}{\kern0pt}\ smt\ {\isacharparenleft}{\kern0pt}z{\isadigit{3}}{\isacharparenright}{\kern0pt}\ case{\isadigit{3}}\ comp{\isacharunderscore}{\kern0pt}associative{\isadigit{2}}\ left{\isacharunderscore}{\kern0pt}coproj{\isacharunderscore}{\kern0pt}cfunc{\isacharunderscore}{\kern0pt}coprod\ left{\isacharunderscore}{\kern0pt}proj{\isacharunderscore}{\kern0pt}type\ right{\isacharunderscore}{\kern0pt}coproj{\isacharunderscore}{\kern0pt}cfunc{\isacharunderscore}{\kern0pt}coprod\ right{\isacharunderscore}{\kern0pt}proj{\isacharunderscore}{\kern0pt}type{\isacharparenright}{\kern0pt}\isanewline
\ \ \ \ \ \ \ \ \isacommand{then}\isamarkupfalse%
\ \isacommand{have}\isamarkupfalse%
\ {\isachardoublequoteopen}{\isasymlangle}{\isasymt}{\isacharcomma}{\kern0pt}\ {\isasymt}{\isasymrangle}\ {\isacharequal}{\kern0pt}\ {\isasymlangle}{\isasymf}{\isacharcomma}{\kern0pt}\ {\isasymt}{\isasymrangle}{\isachardoublequoteclose}\isanewline
\ \ \ \ \ \ \ \ \ \ \isacommand{by}\isamarkupfalse%
\ {\isacharparenleft}{\kern0pt}metis\ cart{\isacharunderscore}{\kern0pt}prod{\isacharunderscore}{\kern0pt}eq{\isadigit{2}}\ false{\isacharunderscore}{\kern0pt}func{\isacharunderscore}{\kern0pt}type\ j{\isacharunderscore}{\kern0pt}def\ true{\isacharunderscore}{\kern0pt}func{\isacharunderscore}{\kern0pt}type{\isacharparenright}{\kern0pt}\isanewline
\ \ \ \ \ \ \ \ \isacommand{then}\isamarkupfalse%
\ \isacommand{have}\isamarkupfalse%
\ {\isachardoublequoteopen}{\isasymt}\ {\isacharequal}{\kern0pt}\ {\isasymf}{\isachardoublequoteclose}\isanewline
\ \ \ \ \ \ \ \ \ \ \isacommand{using}\isamarkupfalse%
\ XOR{\isacharunderscore}{\kern0pt}only{\isacharunderscore}{\kern0pt}true{\isacharunderscore}{\kern0pt}right{\isacharunderscore}{\kern0pt}is{\isacharunderscore}{\kern0pt}true\ XOR{\isacharunderscore}{\kern0pt}true{\isacharunderscore}{\kern0pt}true{\isacharunderscore}{\kern0pt}is{\isacharunderscore}{\kern0pt}false\ \isacommand{by}\isamarkupfalse%
\ auto\isanewline
\ \ \ \ \ \ \ \ \isacommand{then}\isamarkupfalse%
\ \isacommand{show}\isamarkupfalse%
\ False\isanewline
\ \ \ \ \ \ \ \ \ \ \isacommand{using}\isamarkupfalse%
\ true{\isacharunderscore}{\kern0pt}false{\isacharunderscore}{\kern0pt}distinct\ \isacommand{by}\isamarkupfalse%
\ blast\isanewline
\ \ \ \ \ \ \isacommand{qed}\isamarkupfalse%
\isanewline
\ \ \ \ \isacommand{qed}\isamarkupfalse%
\isanewline
\ \ \isacommand{qed}\isamarkupfalse%
\isanewline
\isacommand{qed}\isamarkupfalse%
%
\endisatagproof
{\isafoldproof}%
%
\isadelimproof
\isanewline
%
\endisadelimproof
\isanewline
\isacommand{lemma}\isamarkupfalse%
\ IMPLIES{\isacharunderscore}{\kern0pt}false{\isacharunderscore}{\kern0pt}is{\isacharunderscore}{\kern0pt}true{\isacharunderscore}{\kern0pt}false{\isacharcolon}{\kern0pt}\isanewline
\ \ \isakeyword{assumes}\ {\isachardoublequoteopen}p\ {\isasymin}\isactrlsub c\ {\isasymOmega}{\isachardoublequoteclose}\isanewline
\ \ \isakeyword{assumes}\ {\isachardoublequoteopen}q\ {\isasymin}\isactrlsub c\ {\isasymOmega}{\isachardoublequoteclose}\ \ \isanewline
\ \ \isakeyword{assumes}\ {\isachardoublequoteopen}IMPLIES\ {\isasymcirc}\isactrlsub c\ \ {\isasymlangle}p{\isacharcomma}{\kern0pt}q{\isasymrangle}\ {\isacharequal}{\kern0pt}\ {\isasymf}{\isachardoublequoteclose}\isanewline
\ \ \isakeyword{shows}\ {\isachardoublequoteopen}p\ {\isacharequal}{\kern0pt}\ {\isasymt}\ {\isasymand}\ q\ {\isacharequal}{\kern0pt}\ {\isasymf}{\isachardoublequoteclose}\isanewline
%
\isadelimproof
\ \ %
\endisadelimproof
%
\isatagproof
\isacommand{by}\isamarkupfalse%
\ {\isacharparenleft}{\kern0pt}metis\ IMPLIES{\isacharunderscore}{\kern0pt}false{\isacharunderscore}{\kern0pt}false{\isacharunderscore}{\kern0pt}is{\isacharunderscore}{\kern0pt}true\ IMPLIES{\isacharunderscore}{\kern0pt}false{\isacharunderscore}{\kern0pt}true{\isacharunderscore}{\kern0pt}is{\isacharunderscore}{\kern0pt}true\ IMPLIES{\isacharunderscore}{\kern0pt}true{\isacharunderscore}{\kern0pt}true{\isacharunderscore}{\kern0pt}is{\isacharunderscore}{\kern0pt}true\ assms\ true{\isacharunderscore}{\kern0pt}false{\isacharunderscore}{\kern0pt}only{\isacharunderscore}{\kern0pt}truth{\isacharunderscore}{\kern0pt}values{\isacharparenright}{\kern0pt}%
\endisatagproof
{\isafoldproof}%
%
\isadelimproof
%
\endisadelimproof
%
\begin{isamarkuptext}%
ETCS analog to $(A \iff B) = (A \implies B) \land (B \implies A)$%
\end{isamarkuptext}\isamarkuptrue%
\isacommand{lemma}\isamarkupfalse%
\ iff{\isacharunderscore}{\kern0pt}is{\isacharunderscore}{\kern0pt}and{\isacharunderscore}{\kern0pt}implies{\isacharunderscore}{\kern0pt}implies{\isacharunderscore}{\kern0pt}swap{\isacharcolon}{\kern0pt}\isanewline
{\isachardoublequoteopen}IFF\ {\isacharequal}{\kern0pt}\ AND\ {\isasymcirc}\isactrlsub c\ \ {\isasymlangle}IMPLIES{\isacharcomma}{\kern0pt}\ IMPLIES\ {\isasymcirc}\isactrlsub c\ \ swap\ {\isasymOmega}\ {\isasymOmega}{\isasymrangle}{\isachardoublequoteclose}\isanewline
%
\isadelimproof
%
\endisadelimproof
%
\isatagproof
\isacommand{proof}\isamarkupfalse%
{\isacharparenleft}{\kern0pt}etcs{\isacharunderscore}{\kern0pt}rule\ one{\isacharunderscore}{\kern0pt}separator{\isacharparenright}{\kern0pt}\isanewline
\ \ \isacommand{fix}\isamarkupfalse%
\ x\ \isanewline
\ \ \isacommand{assume}\isamarkupfalse%
\ x{\isacharunderscore}{\kern0pt}type{\isacharcolon}{\kern0pt}\ {\isachardoublequoteopen}x\ {\isasymin}\isactrlsub c\ {\isasymOmega}\ {\isasymtimes}\isactrlsub c\ {\isasymOmega}{\isachardoublequoteclose}\isanewline
\ \ \isacommand{then}\isamarkupfalse%
\ \isacommand{obtain}\isamarkupfalse%
\ p\ q\ \isakeyword{where}\ x{\isacharunderscore}{\kern0pt}def{\isacharcolon}{\kern0pt}\ {\isachardoublequoteopen}p\ {\isasymin}\isactrlsub c\ {\isasymOmega}\ {\isasymand}\ q\ {\isasymin}\isactrlsub c\ {\isasymOmega}\ {\isasymand}\ x\ {\isacharequal}{\kern0pt}\ {\isasymlangle}p{\isacharcomma}{\kern0pt}q{\isasymrangle}{\isachardoublequoteclose}\isanewline
\ \ \ \ \isacommand{by}\isamarkupfalse%
\ {\isacharparenleft}{\kern0pt}meson\ cart{\isacharunderscore}{\kern0pt}prod{\isacharunderscore}{\kern0pt}decomp{\isacharparenright}{\kern0pt}\isanewline
\ \ \isacommand{show}\isamarkupfalse%
\ {\isachardoublequoteopen}IFF\ {\isasymcirc}\isactrlsub c\ x\ {\isacharequal}{\kern0pt}\ {\isacharparenleft}{\kern0pt}AND\ {\isasymcirc}\isactrlsub c\ {\isasymlangle}IMPLIES{\isacharcomma}{\kern0pt}IMPLIES\ {\isasymcirc}\isactrlsub c\ swap\ {\isasymOmega}\ {\isasymOmega}{\isasymrangle}{\isacharparenright}{\kern0pt}\ {\isasymcirc}\isactrlsub c\ x{\isachardoublequoteclose}\isanewline
\ \ \isacommand{proof}\isamarkupfalse%
{\isacharparenleft}{\kern0pt}cases\ {\isachardoublequoteopen}p\ {\isacharequal}{\kern0pt}\ {\isasymt}{\isachardoublequoteclose}{\isacharparenright}{\kern0pt}\isanewline
\ \ \ \ \isacommand{assume}\isamarkupfalse%
\ {\isachardoublequoteopen}p\ {\isacharequal}{\kern0pt}\ {\isasymt}{\isachardoublequoteclose}\isanewline
\ \ \ \ \isacommand{show}\isamarkupfalse%
\ {\isacharquery}{\kern0pt}thesis\isanewline
\ \ \ \ \isacommand{proof}\isamarkupfalse%
{\isacharparenleft}{\kern0pt}cases\ {\isachardoublequoteopen}q\ {\isacharequal}{\kern0pt}\ {\isasymt}{\isachardoublequoteclose}{\isacharparenright}{\kern0pt}\isanewline
\ \ \ \ \ \ \isacommand{assume}\isamarkupfalse%
\ {\isachardoublequoteopen}q\ {\isacharequal}{\kern0pt}\ {\isasymt}{\isachardoublequoteclose}\isanewline
\ \ \ \ \ \ \isacommand{show}\isamarkupfalse%
\ {\isacharquery}{\kern0pt}thesis\isanewline
\ \ \ \ \ \ \isacommand{proof}\isamarkupfalse%
\ {\isacharminus}{\kern0pt}\ \isanewline
\ \ \ \ \ \ \ \ \isacommand{have}\isamarkupfalse%
\ {\isachardoublequoteopen}{\isacharparenleft}{\kern0pt}AND\ {\isasymcirc}\isactrlsub c\ {\isasymlangle}IMPLIES{\isacharcomma}{\kern0pt}IMPLIES\ {\isasymcirc}\isactrlsub c\ swap\ {\isasymOmega}\ {\isasymOmega}{\isasymrangle}{\isacharparenright}{\kern0pt}\ {\isasymcirc}\isactrlsub c\ x\ {\isacharequal}{\kern0pt}\ \ \ \ \isanewline
\ \ \ \ \ \ \ \ \ \ \ \ \ \ \ AND\ {\isasymcirc}\isactrlsub c\ {\isasymlangle}IMPLIES{\isacharcomma}{\kern0pt}IMPLIES\ {\isasymcirc}\isactrlsub c\ swap\ {\isasymOmega}\ {\isasymOmega}{\isasymrangle}\ \ {\isasymcirc}\isactrlsub c\ x{\isachardoublequoteclose}\isanewline
\ \ \ \ \ \ \ \ \ \ \isacommand{using}\isamarkupfalse%
\ comp{\isacharunderscore}{\kern0pt}associative{\isadigit{2}}\ x{\isacharunderscore}{\kern0pt}type\ \isacommand{by}\isamarkupfalse%
\ {\isacharparenleft}{\kern0pt}typecheck{\isacharunderscore}{\kern0pt}cfuncs{\isacharcomma}{\kern0pt}\ force{\isacharparenright}{\kern0pt}\isanewline
\ \ \ \ \ \ \ \ \isacommand{also}\isamarkupfalse%
\ \isacommand{have}\isamarkupfalse%
\ {\isachardoublequoteopen}{\isachardot}{\kern0pt}{\isachardot}{\kern0pt}{\isachardot}{\kern0pt}\ {\isacharequal}{\kern0pt}\ AND\ {\isasymcirc}\isactrlsub c\ {\isasymlangle}IMPLIES\ {\isasymcirc}\isactrlsub c\ x{\isacharcomma}{\kern0pt}IMPLIES\ {\isasymcirc}\isactrlsub c\ swap\ {\isasymOmega}\ {\isasymOmega}\ {\isasymcirc}\isactrlsub c\ x{\isasymrangle}{\isachardoublequoteclose}\isanewline
\ \ \ \ \ \ \ \ \ \ \isacommand{using}\isamarkupfalse%
\ cfunc{\isacharunderscore}{\kern0pt}prod{\isacharunderscore}{\kern0pt}comp\ comp{\isacharunderscore}{\kern0pt}associative{\isadigit{2}}\ x{\isacharunderscore}{\kern0pt}type\ \isacommand{by}\isamarkupfalse%
\ {\isacharparenleft}{\kern0pt}typecheck{\isacharunderscore}{\kern0pt}cfuncs{\isacharcomma}{\kern0pt}\ force{\isacharparenright}{\kern0pt}\isanewline
\ \ \ \ \ \ \ \ \isacommand{also}\isamarkupfalse%
\ \isacommand{have}\isamarkupfalse%
\ {\isachardoublequoteopen}{\isachardot}{\kern0pt}{\isachardot}{\kern0pt}{\isachardot}{\kern0pt}\ {\isacharequal}{\kern0pt}\ AND\ {\isasymcirc}\isactrlsub c\ {\isasymlangle}IMPLIES\ {\isasymcirc}\isactrlsub c\ {\isasymlangle}{\isasymt}{\isacharcomma}{\kern0pt}{\isasymt}{\isasymrangle}{\isacharcomma}{\kern0pt}\ IMPLIES\ {\isasymcirc}\isactrlsub c\ {\isasymlangle}{\isasymt}{\isacharcomma}{\kern0pt}{\isasymt}{\isasymrangle}{\isasymrangle}{\isachardoublequoteclose}\isanewline
\ \ \ \ \ \ \ \ \ \ \isacommand{using}\isamarkupfalse%
\ {\isacartoucheopen}p\ {\isacharequal}{\kern0pt}\ {\isasymt}{\isacartoucheclose}\ {\isacartoucheopen}q\ {\isacharequal}{\kern0pt}\ {\isasymt}{\isacartoucheclose}\ swap{\isacharunderscore}{\kern0pt}ap\ x{\isacharunderscore}{\kern0pt}def\ \isacommand{by}\isamarkupfalse%
\ {\isacharparenleft}{\kern0pt}typecheck{\isacharunderscore}{\kern0pt}cfuncs{\isacharcomma}{\kern0pt}\ presburger{\isacharparenright}{\kern0pt}\isanewline
\ \ \ \ \ \ \ \ \isacommand{also}\isamarkupfalse%
\ \isacommand{have}\isamarkupfalse%
\ {\isachardoublequoteopen}{\isachardot}{\kern0pt}{\isachardot}{\kern0pt}{\isachardot}{\kern0pt}\ {\isacharequal}{\kern0pt}\ AND\ {\isasymcirc}\isactrlsub c\ {\isasymlangle}{\isasymt}{\isacharcomma}{\kern0pt}\ {\isasymt}{\isasymrangle}{\isachardoublequoteclose}\isanewline
\ \ \ \ \ \ \ \ \ \ \isacommand{using}\isamarkupfalse%
\ IMPLIES{\isacharunderscore}{\kern0pt}true{\isacharunderscore}{\kern0pt}true{\isacharunderscore}{\kern0pt}is{\isacharunderscore}{\kern0pt}true\ \isacommand{by}\isamarkupfalse%
\ presburger\isanewline
\ \ \ \ \ \ \ \ \isacommand{also}\isamarkupfalse%
\ \isacommand{have}\isamarkupfalse%
\ {\isachardoublequoteopen}{\isachardot}{\kern0pt}{\isachardot}{\kern0pt}{\isachardot}{\kern0pt}\ {\isacharequal}{\kern0pt}\ {\isasymt}{\isachardoublequoteclose}\isanewline
\ \ \ \ \ \ \ \ \ \ \isacommand{by}\isamarkupfalse%
\ {\isacharparenleft}{\kern0pt}simp\ add{\isacharcolon}{\kern0pt}\ AND{\isacharunderscore}{\kern0pt}true{\isacharunderscore}{\kern0pt}true{\isacharunderscore}{\kern0pt}is{\isacharunderscore}{\kern0pt}true{\isacharparenright}{\kern0pt}\isanewline
\ \ \ \ \ \ \ \ \isacommand{also}\isamarkupfalse%
\ \isacommand{have}\isamarkupfalse%
\ {\isachardoublequoteopen}{\isachardot}{\kern0pt}{\isachardot}{\kern0pt}{\isachardot}{\kern0pt}\ {\isacharequal}{\kern0pt}\ IFF\ {\isasymcirc}\isactrlsub c\ x{\isachardoublequoteclose}\isanewline
\ \ \ \ \ \ \ \ \ \ \isacommand{by}\isamarkupfalse%
\ {\isacharparenleft}{\kern0pt}simp\ add{\isacharcolon}{\kern0pt}\ IFF{\isacharunderscore}{\kern0pt}true{\isacharunderscore}{\kern0pt}true{\isacharunderscore}{\kern0pt}is{\isacharunderscore}{\kern0pt}true\ {\isacartoucheopen}p\ {\isacharequal}{\kern0pt}\ {\isasymt}{\isacartoucheclose}\ {\isacartoucheopen}q\ {\isacharequal}{\kern0pt}\ {\isasymt}{\isacartoucheclose}\ x{\isacharunderscore}{\kern0pt}def{\isacharparenright}{\kern0pt}\isanewline
\ \ \ \ \ \ \ \ \isacommand{then}\isamarkupfalse%
\ \isacommand{show}\isamarkupfalse%
\ {\isacharquery}{\kern0pt}thesis\isanewline
\ \ \ \ \ \ \ \ \ \ \isacommand{by}\isamarkupfalse%
\ {\isacharparenleft}{\kern0pt}simp\ add{\isacharcolon}{\kern0pt}\ calculation{\isacharparenright}{\kern0pt}\isanewline
\ \ \ \ \ \ \isacommand{qed}\isamarkupfalse%
\isanewline
\ \ \ \ \isacommand{next}\isamarkupfalse%
\isanewline
\ \ \ \ \ \ \isacommand{assume}\isamarkupfalse%
\ {\isachardoublequoteopen}q\ {\isasymnoteq}\ {\isasymt}{\isachardoublequoteclose}\isanewline
\ \ \ \ \ \ \isacommand{then}\isamarkupfalse%
\ \isacommand{have}\isamarkupfalse%
\ {\isachardoublequoteopen}q\ {\isacharequal}{\kern0pt}\ {\isasymf}{\isachardoublequoteclose}\isanewline
\ \ \ \ \ \ \ \ \isacommand{by}\isamarkupfalse%
\ {\isacharparenleft}{\kern0pt}meson\ true{\isacharunderscore}{\kern0pt}false{\isacharunderscore}{\kern0pt}only{\isacharunderscore}{\kern0pt}truth{\isacharunderscore}{\kern0pt}values\ x{\isacharunderscore}{\kern0pt}def{\isacharparenright}{\kern0pt}\isanewline
\ \ \ \ \ \ \isacommand{show}\isamarkupfalse%
\ {\isacharquery}{\kern0pt}thesis\isanewline
\ \ \ \ \ \ \isacommand{proof}\isamarkupfalse%
\ {\isacharminus}{\kern0pt}\ \isanewline
\ \ \ \ \ \ \ \ \isacommand{have}\isamarkupfalse%
\ {\isachardoublequoteopen}{\isacharparenleft}{\kern0pt}AND\ {\isasymcirc}\isactrlsub c\ {\isasymlangle}IMPLIES{\isacharcomma}{\kern0pt}IMPLIES\ {\isasymcirc}\isactrlsub c\ swap\ {\isasymOmega}\ {\isasymOmega}{\isasymrangle}{\isacharparenright}{\kern0pt}\ {\isasymcirc}\isactrlsub c\ x\ {\isacharequal}{\kern0pt}\ \ \ \ \isanewline
\ \ \ \ \ \ \ \ \ \ \ \ \ \ \ AND\ {\isasymcirc}\isactrlsub c\ {\isasymlangle}IMPLIES{\isacharcomma}{\kern0pt}IMPLIES\ {\isasymcirc}\isactrlsub c\ swap\ {\isasymOmega}\ {\isasymOmega}{\isasymrangle}\ \ {\isasymcirc}\isactrlsub c\ x{\isachardoublequoteclose}\isanewline
\ \ \ \ \ \ \ \ \ \ \isacommand{using}\isamarkupfalse%
\ comp{\isacharunderscore}{\kern0pt}associative{\isadigit{2}}\ x{\isacharunderscore}{\kern0pt}type\ \isacommand{by}\isamarkupfalse%
\ {\isacharparenleft}{\kern0pt}typecheck{\isacharunderscore}{\kern0pt}cfuncs{\isacharcomma}{\kern0pt}\ force{\isacharparenright}{\kern0pt}\isanewline
\ \ \ \ \ \ \ \ \isacommand{also}\isamarkupfalse%
\ \isacommand{have}\isamarkupfalse%
\ {\isachardoublequoteopen}{\isachardot}{\kern0pt}{\isachardot}{\kern0pt}{\isachardot}{\kern0pt}\ {\isacharequal}{\kern0pt}\ AND\ {\isasymcirc}\isactrlsub c\ {\isasymlangle}IMPLIES\ {\isasymcirc}\isactrlsub c\ x{\isacharcomma}{\kern0pt}IMPLIES\ {\isasymcirc}\isactrlsub c\ swap\ {\isasymOmega}\ {\isasymOmega}\ {\isasymcirc}\isactrlsub c\ x{\isasymrangle}{\isachardoublequoteclose}\isanewline
\ \ \ \ \ \ \ \ \ \ \isacommand{using}\isamarkupfalse%
\ cfunc{\isacharunderscore}{\kern0pt}prod{\isacharunderscore}{\kern0pt}comp\ comp{\isacharunderscore}{\kern0pt}associative{\isadigit{2}}\ x{\isacharunderscore}{\kern0pt}type\ \isacommand{by}\isamarkupfalse%
\ {\isacharparenleft}{\kern0pt}typecheck{\isacharunderscore}{\kern0pt}cfuncs{\isacharcomma}{\kern0pt}\ force{\isacharparenright}{\kern0pt}\isanewline
\ \ \ \ \ \ \ \ \isacommand{also}\isamarkupfalse%
\ \isacommand{have}\isamarkupfalse%
\ {\isachardoublequoteopen}{\isachardot}{\kern0pt}{\isachardot}{\kern0pt}{\isachardot}{\kern0pt}\ {\isacharequal}{\kern0pt}\ AND\ {\isasymcirc}\isactrlsub c\ {\isasymlangle}IMPLIES\ {\isasymcirc}\isactrlsub c\ {\isasymlangle}{\isasymt}{\isacharcomma}{\kern0pt}{\isasymf}{\isasymrangle}{\isacharcomma}{\kern0pt}\ IMPLIES\ {\isasymcirc}\isactrlsub c\ {\isasymlangle}{\isasymf}{\isacharcomma}{\kern0pt}{\isasymt}{\isasymrangle}{\isasymrangle}{\isachardoublequoteclose}\isanewline
\ \ \ \ \ \ \ \ \ \ \isacommand{using}\isamarkupfalse%
\ {\isacartoucheopen}p\ {\isacharequal}{\kern0pt}\ {\isasymt}{\isacartoucheclose}\ {\isacartoucheopen}q\ {\isacharequal}{\kern0pt}\ {\isasymf}{\isacartoucheclose}\ swap{\isacharunderscore}{\kern0pt}ap\ x{\isacharunderscore}{\kern0pt}def\ \isacommand{by}\isamarkupfalse%
\ {\isacharparenleft}{\kern0pt}typecheck{\isacharunderscore}{\kern0pt}cfuncs{\isacharcomma}{\kern0pt}\ presburger{\isacharparenright}{\kern0pt}\isanewline
\ \ \ \ \ \ \ \ \isacommand{also}\isamarkupfalse%
\ \isacommand{have}\isamarkupfalse%
\ {\isachardoublequoteopen}{\isachardot}{\kern0pt}{\isachardot}{\kern0pt}{\isachardot}{\kern0pt}\ {\isacharequal}{\kern0pt}\ AND\ {\isasymcirc}\isactrlsub c\ {\isasymlangle}{\isasymf}{\isacharcomma}{\kern0pt}\ {\isasymt}{\isasymrangle}{\isachardoublequoteclose}\isanewline
\ \ \ \ \ \ \ \ \ \ \isacommand{using}\isamarkupfalse%
\ IMPLIES{\isacharunderscore}{\kern0pt}false{\isacharunderscore}{\kern0pt}true{\isacharunderscore}{\kern0pt}is{\isacharunderscore}{\kern0pt}true\ IMPLIES{\isacharunderscore}{\kern0pt}true{\isacharunderscore}{\kern0pt}false{\isacharunderscore}{\kern0pt}is{\isacharunderscore}{\kern0pt}false\ \isacommand{by}\isamarkupfalse%
\ presburger\isanewline
\ \ \ \ \ \ \ \ \isacommand{also}\isamarkupfalse%
\ \isacommand{have}\isamarkupfalse%
\ {\isachardoublequoteopen}{\isachardot}{\kern0pt}{\isachardot}{\kern0pt}{\isachardot}{\kern0pt}\ {\isacharequal}{\kern0pt}\ {\isasymf}{\isachardoublequoteclose}\isanewline
\ \ \ \ \ \ \ \ \ \ \isacommand{by}\isamarkupfalse%
\ {\isacharparenleft}{\kern0pt}simp\ add{\isacharcolon}{\kern0pt}\ AND{\isacharunderscore}{\kern0pt}false{\isacharunderscore}{\kern0pt}left{\isacharunderscore}{\kern0pt}is{\isacharunderscore}{\kern0pt}false\ true{\isacharunderscore}{\kern0pt}func{\isacharunderscore}{\kern0pt}type{\isacharparenright}{\kern0pt}\isanewline
\ \ \ \ \ \ \ \ \isacommand{also}\isamarkupfalse%
\ \isacommand{have}\isamarkupfalse%
\ {\isachardoublequoteopen}{\isachardot}{\kern0pt}{\isachardot}{\kern0pt}{\isachardot}{\kern0pt}\ {\isacharequal}{\kern0pt}\ IFF\ {\isasymcirc}\isactrlsub c\ x{\isachardoublequoteclose}\isanewline
\ \ \ \ \ \ \ \ \ \ \isacommand{by}\isamarkupfalse%
\ {\isacharparenleft}{\kern0pt}simp\ add{\isacharcolon}{\kern0pt}\ IFF{\isacharunderscore}{\kern0pt}true{\isacharunderscore}{\kern0pt}false{\isacharunderscore}{\kern0pt}is{\isacharunderscore}{\kern0pt}false\ {\isacartoucheopen}p\ {\isacharequal}{\kern0pt}\ {\isasymt}{\isacartoucheclose}\ {\isacartoucheopen}q\ {\isacharequal}{\kern0pt}\ {\isasymf}{\isacartoucheclose}\ x{\isacharunderscore}{\kern0pt}def{\isacharparenright}{\kern0pt}\isanewline
\ \ \ \ \ \ \ \ \isacommand{then}\isamarkupfalse%
\ \isacommand{show}\isamarkupfalse%
\ {\isacharquery}{\kern0pt}thesis\isanewline
\ \ \ \ \ \ \ \ \ \ \isacommand{by}\isamarkupfalse%
\ {\isacharparenleft}{\kern0pt}simp\ add{\isacharcolon}{\kern0pt}\ calculation{\isacharparenright}{\kern0pt}\isanewline
\ \ \ \ \ \ \isacommand{qed}\isamarkupfalse%
\isanewline
\ \ \ \ \isacommand{qed}\isamarkupfalse%
\isanewline
\ \ \isacommand{next}\isamarkupfalse%
\isanewline
\ \ \ \ \isacommand{assume}\isamarkupfalse%
\ {\isachardoublequoteopen}p\ {\isasymnoteq}\ {\isasymt}{\isachardoublequoteclose}\isanewline
\ \ \ \ \isacommand{then}\isamarkupfalse%
\ \isacommand{have}\isamarkupfalse%
\ {\isachardoublequoteopen}p\ {\isacharequal}{\kern0pt}\ {\isasymf}{\isachardoublequoteclose}\isanewline
\ \ \ \ \ \ \isacommand{using}\isamarkupfalse%
\ true{\isacharunderscore}{\kern0pt}false{\isacharunderscore}{\kern0pt}only{\isacharunderscore}{\kern0pt}truth{\isacharunderscore}{\kern0pt}values\ x{\isacharunderscore}{\kern0pt}def\ \isacommand{by}\isamarkupfalse%
\ blast\isanewline
\ \ \ \ \isacommand{show}\isamarkupfalse%
\ {\isacharquery}{\kern0pt}thesis\isanewline
\ \ \ \ \isacommand{proof}\isamarkupfalse%
{\isacharparenleft}{\kern0pt}cases\ {\isachardoublequoteopen}q\ {\isacharequal}{\kern0pt}\ {\isasymt}{\isachardoublequoteclose}{\isacharparenright}{\kern0pt}\isanewline
\ \ \ \ \ \ \isacommand{assume}\isamarkupfalse%
\ {\isachardoublequoteopen}q\ {\isacharequal}{\kern0pt}\ {\isasymt}{\isachardoublequoteclose}\isanewline
\ \ \ \ \ \ \isacommand{show}\isamarkupfalse%
\ {\isacharquery}{\kern0pt}thesis\isanewline
\ \ \ \ \ \ \isacommand{proof}\isamarkupfalse%
\ {\isacharminus}{\kern0pt}\ \isanewline
\ \ \ \ \ \ \ \ \isacommand{have}\isamarkupfalse%
\ {\isachardoublequoteopen}{\isacharparenleft}{\kern0pt}AND\ {\isasymcirc}\isactrlsub c\ {\isasymlangle}IMPLIES{\isacharcomma}{\kern0pt}IMPLIES\ {\isasymcirc}\isactrlsub c\ swap\ {\isasymOmega}\ {\isasymOmega}{\isasymrangle}{\isacharparenright}{\kern0pt}\ {\isasymcirc}\isactrlsub c\ x\ {\isacharequal}{\kern0pt}\ \ \ \ \isanewline
\ \ \ \ \ \ \ \ \ \ \ \ \ \ \ AND\ {\isasymcirc}\isactrlsub c\ {\isasymlangle}IMPLIES{\isacharcomma}{\kern0pt}IMPLIES\ {\isasymcirc}\isactrlsub c\ swap\ {\isasymOmega}\ {\isasymOmega}{\isasymrangle}\ \ {\isasymcirc}\isactrlsub c\ x{\isachardoublequoteclose}\isanewline
\ \ \ \ \ \ \ \ \ \ \isacommand{using}\isamarkupfalse%
\ comp{\isacharunderscore}{\kern0pt}associative{\isadigit{2}}\ x{\isacharunderscore}{\kern0pt}type\ \isacommand{by}\isamarkupfalse%
\ {\isacharparenleft}{\kern0pt}typecheck{\isacharunderscore}{\kern0pt}cfuncs{\isacharcomma}{\kern0pt}\ force{\isacharparenright}{\kern0pt}\isanewline
\ \ \ \ \ \ \ \ \isacommand{also}\isamarkupfalse%
\ \isacommand{have}\isamarkupfalse%
\ {\isachardoublequoteopen}{\isachardot}{\kern0pt}{\isachardot}{\kern0pt}{\isachardot}{\kern0pt}\ {\isacharequal}{\kern0pt}\ AND\ {\isasymcirc}\isactrlsub c\ {\isasymlangle}IMPLIES\ {\isasymcirc}\isactrlsub c\ x{\isacharcomma}{\kern0pt}IMPLIES\ {\isasymcirc}\isactrlsub c\ swap\ {\isasymOmega}\ {\isasymOmega}\ {\isasymcirc}\isactrlsub c\ x{\isasymrangle}{\isachardoublequoteclose}\isanewline
\ \ \ \ \ \ \ \ \ \ \isacommand{using}\isamarkupfalse%
\ cfunc{\isacharunderscore}{\kern0pt}prod{\isacharunderscore}{\kern0pt}comp\ comp{\isacharunderscore}{\kern0pt}associative{\isadigit{2}}\ x{\isacharunderscore}{\kern0pt}type\ \isacommand{by}\isamarkupfalse%
\ {\isacharparenleft}{\kern0pt}typecheck{\isacharunderscore}{\kern0pt}cfuncs{\isacharcomma}{\kern0pt}\ force{\isacharparenright}{\kern0pt}\isanewline
\ \ \ \ \ \ \ \ \isacommand{also}\isamarkupfalse%
\ \isacommand{have}\isamarkupfalse%
\ {\isachardoublequoteopen}{\isachardot}{\kern0pt}{\isachardot}{\kern0pt}{\isachardot}{\kern0pt}\ {\isacharequal}{\kern0pt}\ AND\ {\isasymcirc}\isactrlsub c\ {\isasymlangle}IMPLIES\ {\isasymcirc}\isactrlsub c\ {\isasymlangle}{\isasymf}{\isacharcomma}{\kern0pt}{\isasymt}{\isasymrangle}{\isacharcomma}{\kern0pt}\ IMPLIES\ {\isasymcirc}\isactrlsub c\ {\isasymlangle}{\isasymt}{\isacharcomma}{\kern0pt}{\isasymf}{\isasymrangle}{\isasymrangle}{\isachardoublequoteclose}\isanewline
\ \ \ \ \ \ \ \ \ \ \isacommand{using}\isamarkupfalse%
\ {\isacartoucheopen}p\ {\isacharequal}{\kern0pt}\ {\isasymf}{\isacartoucheclose}\ {\isacartoucheopen}q\ {\isacharequal}{\kern0pt}\ {\isasymt}{\isacartoucheclose}\ swap{\isacharunderscore}{\kern0pt}ap\ x{\isacharunderscore}{\kern0pt}def\ \isacommand{by}\isamarkupfalse%
\ {\isacharparenleft}{\kern0pt}typecheck{\isacharunderscore}{\kern0pt}cfuncs{\isacharcomma}{\kern0pt}\ presburger{\isacharparenright}{\kern0pt}\isanewline
\ \ \ \ \ \ \ \ \isacommand{also}\isamarkupfalse%
\ \isacommand{have}\isamarkupfalse%
\ {\isachardoublequoteopen}{\isachardot}{\kern0pt}{\isachardot}{\kern0pt}{\isachardot}{\kern0pt}\ {\isacharequal}{\kern0pt}\ AND\ {\isasymcirc}\isactrlsub c\ {\isasymlangle}{\isasymt}{\isacharcomma}{\kern0pt}\ {\isasymf}{\isasymrangle}{\isachardoublequoteclose}\isanewline
\ \ \ \ \ \ \ \ \ \ \isacommand{by}\isamarkupfalse%
\ {\isacharparenleft}{\kern0pt}simp\ add{\isacharcolon}{\kern0pt}\ IMPLIES{\isacharunderscore}{\kern0pt}false{\isacharunderscore}{\kern0pt}true{\isacharunderscore}{\kern0pt}is{\isacharunderscore}{\kern0pt}true\ IMPLIES{\isacharunderscore}{\kern0pt}true{\isacharunderscore}{\kern0pt}false{\isacharunderscore}{\kern0pt}is{\isacharunderscore}{\kern0pt}false{\isacharparenright}{\kern0pt}\isanewline
\ \ \ \ \ \ \ \ \isacommand{also}\isamarkupfalse%
\ \isacommand{have}\isamarkupfalse%
\ {\isachardoublequoteopen}{\isachardot}{\kern0pt}{\isachardot}{\kern0pt}{\isachardot}{\kern0pt}\ {\isacharequal}{\kern0pt}\ {\isasymf}{\isachardoublequoteclose}\isanewline
\ \ \ \ \ \ \ \ \ \ \isacommand{by}\isamarkupfalse%
\ {\isacharparenleft}{\kern0pt}simp\ add{\isacharcolon}{\kern0pt}\ AND{\isacharunderscore}{\kern0pt}false{\isacharunderscore}{\kern0pt}right{\isacharunderscore}{\kern0pt}is{\isacharunderscore}{\kern0pt}false\ true{\isacharunderscore}{\kern0pt}func{\isacharunderscore}{\kern0pt}type{\isacharparenright}{\kern0pt}\isanewline
\ \ \ \ \ \ \ \ \isacommand{also}\isamarkupfalse%
\ \isacommand{have}\isamarkupfalse%
\ {\isachardoublequoteopen}{\isachardot}{\kern0pt}{\isachardot}{\kern0pt}{\isachardot}{\kern0pt}\ {\isacharequal}{\kern0pt}\ IFF\ {\isasymcirc}\isactrlsub c\ x{\isachardoublequoteclose}\isanewline
\ \ \ \ \ \ \ \ \ \ \isacommand{by}\isamarkupfalse%
\ {\isacharparenleft}{\kern0pt}simp\ add{\isacharcolon}{\kern0pt}\ IFF{\isacharunderscore}{\kern0pt}false{\isacharunderscore}{\kern0pt}true{\isacharunderscore}{\kern0pt}is{\isacharunderscore}{\kern0pt}false\ {\isacartoucheopen}p\ {\isacharequal}{\kern0pt}\ {\isasymf}{\isacartoucheclose}\ {\isacartoucheopen}q\ {\isacharequal}{\kern0pt}\ {\isasymt}{\isacartoucheclose}\ x{\isacharunderscore}{\kern0pt}def{\isacharparenright}{\kern0pt}\isanewline
\ \ \ \ \ \ \ \ \isacommand{then}\isamarkupfalse%
\ \isacommand{show}\isamarkupfalse%
\ {\isacharquery}{\kern0pt}thesis\isanewline
\ \ \ \ \ \ \ \ \ \ \isacommand{by}\isamarkupfalse%
\ {\isacharparenleft}{\kern0pt}simp\ add{\isacharcolon}{\kern0pt}\ calculation{\isacharparenright}{\kern0pt}\isanewline
\ \ \ \ \ \ \isacommand{qed}\isamarkupfalse%
\isanewline
\ \ \ \ \isacommand{next}\isamarkupfalse%
\isanewline
\ \ \ \ \ \ \isacommand{assume}\isamarkupfalse%
\ {\isachardoublequoteopen}q\ {\isasymnoteq}\ {\isasymt}{\isachardoublequoteclose}\isanewline
\ \ \ \ \ \ \isacommand{then}\isamarkupfalse%
\ \isacommand{have}\isamarkupfalse%
\ {\isachardoublequoteopen}q\ {\isacharequal}{\kern0pt}\ {\isasymf}{\isachardoublequoteclose}\isanewline
\ \ \ \ \ \ \ \ \isacommand{by}\isamarkupfalse%
\ {\isacharparenleft}{\kern0pt}meson\ true{\isacharunderscore}{\kern0pt}false{\isacharunderscore}{\kern0pt}only{\isacharunderscore}{\kern0pt}truth{\isacharunderscore}{\kern0pt}values\ x{\isacharunderscore}{\kern0pt}def{\isacharparenright}{\kern0pt}\isanewline
\ \ \ \ \ \ \isacommand{show}\isamarkupfalse%
\ {\isacharquery}{\kern0pt}thesis\isanewline
\ \ \ \ \ \ \isacommand{proof}\isamarkupfalse%
\ {\isacharminus}{\kern0pt}\ \isanewline
\ \ \ \ \ \ \ \ \isacommand{have}\isamarkupfalse%
\ {\isachardoublequoteopen}{\isacharparenleft}{\kern0pt}AND\ {\isasymcirc}\isactrlsub c\ {\isasymlangle}IMPLIES{\isacharcomma}{\kern0pt}IMPLIES\ {\isasymcirc}\isactrlsub c\ swap\ {\isasymOmega}\ {\isasymOmega}{\isasymrangle}{\isacharparenright}{\kern0pt}\ {\isasymcirc}\isactrlsub c\ x\ {\isacharequal}{\kern0pt}\ \ \ \ \isanewline
\ \ \ \ \ \ \ \ \ \ \ \ \ \ \ AND\ {\isasymcirc}\isactrlsub c\ {\isasymlangle}IMPLIES{\isacharcomma}{\kern0pt}IMPLIES\ {\isasymcirc}\isactrlsub c\ swap\ {\isasymOmega}\ {\isasymOmega}{\isasymrangle}\ \ {\isasymcirc}\isactrlsub c\ x{\isachardoublequoteclose}\isanewline
\ \ \ \ \ \ \ \ \ \ \isacommand{using}\isamarkupfalse%
\ comp{\isacharunderscore}{\kern0pt}associative{\isadigit{2}}\ x{\isacharunderscore}{\kern0pt}type\ \isacommand{by}\isamarkupfalse%
\ {\isacharparenleft}{\kern0pt}typecheck{\isacharunderscore}{\kern0pt}cfuncs{\isacharcomma}{\kern0pt}\ force{\isacharparenright}{\kern0pt}\isanewline
\ \ \ \ \ \ \ \ \isacommand{also}\isamarkupfalse%
\ \isacommand{have}\isamarkupfalse%
\ {\isachardoublequoteopen}{\isachardot}{\kern0pt}{\isachardot}{\kern0pt}{\isachardot}{\kern0pt}\ {\isacharequal}{\kern0pt}\ AND\ {\isasymcirc}\isactrlsub c\ {\isasymlangle}IMPLIES\ {\isasymcirc}\isactrlsub c\ x{\isacharcomma}{\kern0pt}IMPLIES\ {\isasymcirc}\isactrlsub c\ swap\ {\isasymOmega}\ {\isasymOmega}\ {\isasymcirc}\isactrlsub c\ x{\isasymrangle}{\isachardoublequoteclose}\isanewline
\ \ \ \ \ \ \ \ \ \ \isacommand{using}\isamarkupfalse%
\ cfunc{\isacharunderscore}{\kern0pt}prod{\isacharunderscore}{\kern0pt}comp\ comp{\isacharunderscore}{\kern0pt}associative{\isadigit{2}}\ x{\isacharunderscore}{\kern0pt}type\ \isacommand{by}\isamarkupfalse%
\ {\isacharparenleft}{\kern0pt}typecheck{\isacharunderscore}{\kern0pt}cfuncs{\isacharcomma}{\kern0pt}\ force{\isacharparenright}{\kern0pt}\isanewline
\ \ \ \ \ \ \ \ \isacommand{also}\isamarkupfalse%
\ \isacommand{have}\isamarkupfalse%
\ {\isachardoublequoteopen}{\isachardot}{\kern0pt}{\isachardot}{\kern0pt}{\isachardot}{\kern0pt}\ {\isacharequal}{\kern0pt}\ AND\ {\isasymcirc}\isactrlsub c\ {\isasymlangle}IMPLIES\ {\isasymcirc}\isactrlsub c\ {\isasymlangle}{\isasymf}{\isacharcomma}{\kern0pt}{\isasymf}{\isasymrangle}{\isacharcomma}{\kern0pt}\ IMPLIES\ {\isasymcirc}\isactrlsub c\ {\isasymlangle}{\isasymf}{\isacharcomma}{\kern0pt}{\isasymf}{\isasymrangle}{\isasymrangle}{\isachardoublequoteclose}\isanewline
\ \ \ \ \ \ \ \ \ \ \isacommand{using}\isamarkupfalse%
\ {\isacartoucheopen}p\ {\isacharequal}{\kern0pt}\ {\isasymf}{\isacartoucheclose}\ {\isacartoucheopen}q\ {\isacharequal}{\kern0pt}\ {\isasymf}{\isacartoucheclose}\ swap{\isacharunderscore}{\kern0pt}ap\ x{\isacharunderscore}{\kern0pt}def\ \isacommand{by}\isamarkupfalse%
\ {\isacharparenleft}{\kern0pt}typecheck{\isacharunderscore}{\kern0pt}cfuncs{\isacharcomma}{\kern0pt}\ presburger{\isacharparenright}{\kern0pt}\isanewline
\ \ \ \ \ \ \ \ \isacommand{also}\isamarkupfalse%
\ \isacommand{have}\isamarkupfalse%
\ {\isachardoublequoteopen}{\isachardot}{\kern0pt}{\isachardot}{\kern0pt}{\isachardot}{\kern0pt}\ {\isacharequal}{\kern0pt}\ AND\ {\isasymcirc}\isactrlsub c\ {\isasymlangle}{\isasymt}{\isacharcomma}{\kern0pt}\ {\isasymt}{\isasymrangle}{\isachardoublequoteclose}\isanewline
\ \ \ \ \ \ \ \ \ \ \isacommand{by}\isamarkupfalse%
\ {\isacharparenleft}{\kern0pt}simp\ add{\isacharcolon}{\kern0pt}\ IMPLIES{\isacharunderscore}{\kern0pt}false{\isacharunderscore}{\kern0pt}false{\isacharunderscore}{\kern0pt}is{\isacharunderscore}{\kern0pt}true{\isacharparenright}{\kern0pt}\isanewline
\ \ \ \ \ \ \ \ \isacommand{also}\isamarkupfalse%
\ \isacommand{have}\isamarkupfalse%
\ {\isachardoublequoteopen}{\isachardot}{\kern0pt}{\isachardot}{\kern0pt}{\isachardot}{\kern0pt}\ {\isacharequal}{\kern0pt}\ {\isasymt}{\isachardoublequoteclose}\isanewline
\ \ \ \ \ \ \ \ \ \ \isacommand{by}\isamarkupfalse%
\ {\isacharparenleft}{\kern0pt}simp\ add{\isacharcolon}{\kern0pt}\ AND{\isacharunderscore}{\kern0pt}true{\isacharunderscore}{\kern0pt}true{\isacharunderscore}{\kern0pt}is{\isacharunderscore}{\kern0pt}true{\isacharparenright}{\kern0pt}\isanewline
\ \ \ \ \ \ \ \ \isacommand{also}\isamarkupfalse%
\ \isacommand{have}\isamarkupfalse%
\ {\isachardoublequoteopen}{\isachardot}{\kern0pt}{\isachardot}{\kern0pt}{\isachardot}{\kern0pt}\ {\isacharequal}{\kern0pt}\ IFF\ {\isasymcirc}\isactrlsub c\ x{\isachardoublequoteclose}\isanewline
\ \ \ \ \ \ \ \ \ \ \isacommand{by}\isamarkupfalse%
\ {\isacharparenleft}{\kern0pt}simp\ add{\isacharcolon}{\kern0pt}\ IFF{\isacharunderscore}{\kern0pt}false{\isacharunderscore}{\kern0pt}false{\isacharunderscore}{\kern0pt}is{\isacharunderscore}{\kern0pt}true\ {\isacartoucheopen}p\ {\isacharequal}{\kern0pt}\ {\isasymf}{\isacartoucheclose}\ {\isacartoucheopen}q\ {\isacharequal}{\kern0pt}\ {\isasymf}{\isacartoucheclose}\ x{\isacharunderscore}{\kern0pt}def{\isacharparenright}{\kern0pt}\isanewline
\ \ \ \ \ \ \ \ \isacommand{then}\isamarkupfalse%
\ \isacommand{show}\isamarkupfalse%
\ {\isacharquery}{\kern0pt}thesis\isanewline
\ \ \ \ \ \ \ \ \ \ \isacommand{by}\isamarkupfalse%
\ {\isacharparenleft}{\kern0pt}simp\ add{\isacharcolon}{\kern0pt}\ calculation{\isacharparenright}{\kern0pt}\isanewline
\ \ \ \ \ \ \isacommand{qed}\isamarkupfalse%
\isanewline
\ \ \ \ \isacommand{qed}\isamarkupfalse%
\isanewline
\ \ \isacommand{qed}\isamarkupfalse%
\isanewline
\isacommand{qed}\isamarkupfalse%
%
\endisatagproof
{\isafoldproof}%
%
\isadelimproof
\isanewline
%
\endisadelimproof
\isanewline
\isacommand{lemma}\isamarkupfalse%
\ IMPLIES{\isacharunderscore}{\kern0pt}is{\isacharunderscore}{\kern0pt}OR{\isacharunderscore}{\kern0pt}NOT{\isacharunderscore}{\kern0pt}id{\isacharcolon}{\kern0pt}\isanewline
\ \ {\isachardoublequoteopen}IMPLIES\ {\isacharequal}{\kern0pt}\ OR\ {\isasymcirc}\isactrlsub c\ {\isacharparenleft}{\kern0pt}NOT\ {\isasymtimes}\isactrlsub f\ id{\isacharparenleft}{\kern0pt}{\isasymOmega}{\isacharparenright}{\kern0pt}{\isacharparenright}{\kern0pt}{\isachardoublequoteclose}\isanewline
%
\isadelimproof
%
\endisadelimproof
%
\isatagproof
\isacommand{proof}\isamarkupfalse%
{\isacharparenleft}{\kern0pt}etcs{\isacharunderscore}{\kern0pt}rule\ one{\isacharunderscore}{\kern0pt}separator{\isacharparenright}{\kern0pt}\isanewline
\ \ \isacommand{fix}\isamarkupfalse%
\ x\ \isanewline
\ \ \isacommand{assume}\isamarkupfalse%
\ x{\isacharunderscore}{\kern0pt}type{\isacharcolon}{\kern0pt}\ {\isachardoublequoteopen}x\ {\isasymin}\isactrlsub c\ {\isasymOmega}\ {\isasymtimes}\isactrlsub c\ {\isasymOmega}{\isachardoublequoteclose}\isanewline
\ \ \isacommand{then}\isamarkupfalse%
\ \isacommand{obtain}\isamarkupfalse%
\ u\ v\ \isakeyword{where}\ x{\isacharunderscore}{\kern0pt}form{\isacharcolon}{\kern0pt}\ {\isachardoublequoteopen}u\ {\isasymin}\isactrlsub c\ {\isasymOmega}\ {\isasymand}\ v\ {\isasymin}\isactrlsub c\ {\isasymOmega}\ {\isasymand}\ x\ {\isacharequal}{\kern0pt}\ {\isasymlangle}u{\isacharcomma}{\kern0pt}\ v{\isasymrangle}{\isachardoublequoteclose}\isanewline
\ \ \ \ \isacommand{using}\isamarkupfalse%
\ cart{\isacharunderscore}{\kern0pt}prod{\isacharunderscore}{\kern0pt}decomp\ \isacommand{by}\isamarkupfalse%
\ blast\isanewline
\ \ \isacommand{show}\isamarkupfalse%
\ {\isachardoublequoteopen}IMPLIES\ {\isasymcirc}\isactrlsub c\ x\ {\isacharequal}{\kern0pt}\ {\isacharparenleft}{\kern0pt}OR\ {\isasymcirc}\isactrlsub c\ NOT\ {\isasymtimes}\isactrlsub f\ id\isactrlsub c\ {\isasymOmega}{\isacharparenright}{\kern0pt}\ {\isasymcirc}\isactrlsub c\ x{\isachardoublequoteclose}\isanewline
\ \ \isacommand{proof}\isamarkupfalse%
{\isacharparenleft}{\kern0pt}cases\ {\isachardoublequoteopen}u\ {\isacharequal}{\kern0pt}\ {\isasymt}{\isachardoublequoteclose}{\isacharparenright}{\kern0pt}\isanewline
\ \ \ \ \isacommand{assume}\isamarkupfalse%
\ {\isachardoublequoteopen}u\ {\isacharequal}{\kern0pt}\ {\isasymt}{\isachardoublequoteclose}\isanewline
\ \ \ \ \isacommand{show}\isamarkupfalse%
\ {\isacharquery}{\kern0pt}thesis\isanewline
\ \ \ \ \isacommand{proof}\isamarkupfalse%
{\isacharparenleft}{\kern0pt}cases\ {\isachardoublequoteopen}v\ {\isacharequal}{\kern0pt}\ {\isasymt}{\isachardoublequoteclose}{\isacharparenright}{\kern0pt}\isanewline
\ \ \ \ \ \ \isacommand{assume}\isamarkupfalse%
\ {\isachardoublequoteopen}v\ {\isacharequal}{\kern0pt}\ {\isasymt}{\isachardoublequoteclose}\isanewline
\ \ \ \ \ \ \isacommand{have}\isamarkupfalse%
\ {\isachardoublequoteopen}{\isacharparenleft}{\kern0pt}OR\ {\isasymcirc}\isactrlsub c\ NOT\ {\isasymtimes}\isactrlsub f\ id\isactrlsub c\ {\isasymOmega}{\isacharparenright}{\kern0pt}\ {\isasymcirc}\isactrlsub c\ x\ {\isacharequal}{\kern0pt}\ OR\ {\isasymcirc}\isactrlsub c\ {\isacharparenleft}{\kern0pt}NOT\ {\isasymtimes}\isactrlsub f\ id\isactrlsub c\ {\isasymOmega}{\isacharparenright}{\kern0pt}\ {\isasymcirc}\isactrlsub c\ x{\isachardoublequoteclose}\isanewline
\ \ \ \ \ \ \ \ \isacommand{using}\isamarkupfalse%
\ comp{\isacharunderscore}{\kern0pt}associative{\isadigit{2}}\ x{\isacharunderscore}{\kern0pt}type\ \isacommand{by}\isamarkupfalse%
\ {\isacharparenleft}{\kern0pt}typecheck{\isacharunderscore}{\kern0pt}cfuncs{\isacharcomma}{\kern0pt}\ force{\isacharparenright}{\kern0pt}\isanewline
\ \ \ \ \ \ \isacommand{also}\isamarkupfalse%
\ \isacommand{have}\isamarkupfalse%
\ {\isachardoublequoteopen}{\isachardot}{\kern0pt}{\isachardot}{\kern0pt}{\isachardot}{\kern0pt}\ {\isacharequal}{\kern0pt}\ OR\ {\isasymcirc}\isactrlsub c\ {\isasymlangle}NOT\ {\isasymcirc}\isactrlsub c\ {\isasymt}{\isacharcomma}{\kern0pt}\ id\isactrlsub c\ {\isasymOmega}\ {\isasymcirc}\isactrlsub c\ {\isasymt}{\isasymrangle}{\isachardoublequoteclose}\isanewline
\ \ \ \ \ \ \ \ \isacommand{by}\isamarkupfalse%
\ {\isacharparenleft}{\kern0pt}typecheck{\isacharunderscore}{\kern0pt}cfuncs{\isacharcomma}{\kern0pt}\ simp\ add{\isacharcolon}{\kern0pt}\ {\isacartoucheopen}u\ {\isacharequal}{\kern0pt}\ {\isasymt}{\isacartoucheclose}\ {\isacartoucheopen}v\ {\isacharequal}{\kern0pt}\ {\isasymt}{\isacartoucheclose}\ cfunc{\isacharunderscore}{\kern0pt}cross{\isacharunderscore}{\kern0pt}prod{\isacharunderscore}{\kern0pt}comp{\isacharunderscore}{\kern0pt}cfunc{\isacharunderscore}{\kern0pt}prod\ x{\isacharunderscore}{\kern0pt}form{\isacharparenright}{\kern0pt}\isanewline
\ \ \ \ \ \ \isacommand{also}\isamarkupfalse%
\ \isacommand{have}\isamarkupfalse%
\ {\isachardoublequoteopen}{\isachardot}{\kern0pt}{\isachardot}{\kern0pt}{\isachardot}{\kern0pt}\ {\isacharequal}{\kern0pt}\ OR\ {\isasymcirc}\isactrlsub c\ {\isasymlangle}{\isasymf}{\isacharcomma}{\kern0pt}\ {\isasymt}{\isasymrangle}{\isachardoublequoteclose}\isanewline
\ \ \ \ \ \ \ \ \isacommand{by}\isamarkupfalse%
\ {\isacharparenleft}{\kern0pt}typecheck{\isacharunderscore}{\kern0pt}cfuncs{\isacharcomma}{\kern0pt}\ simp\ add{\isacharcolon}{\kern0pt}\ NOT{\isacharunderscore}{\kern0pt}true{\isacharunderscore}{\kern0pt}is{\isacharunderscore}{\kern0pt}false\ id{\isacharunderscore}{\kern0pt}left{\isacharunderscore}{\kern0pt}unit{\isadigit{2}}{\isacharparenright}{\kern0pt}\isanewline
\ \ \ \ \ \ \isacommand{also}\isamarkupfalse%
\ \isacommand{have}\isamarkupfalse%
\ {\isachardoublequoteopen}{\isachardot}{\kern0pt}{\isachardot}{\kern0pt}{\isachardot}{\kern0pt}\ {\isacharequal}{\kern0pt}\ {\isasymt}{\isachardoublequoteclose}\isanewline
\ \ \ \ \ \ \ \ \isacommand{by}\isamarkupfalse%
\ {\isacharparenleft}{\kern0pt}simp\ add{\isacharcolon}{\kern0pt}\ OR{\isacharunderscore}{\kern0pt}true{\isacharunderscore}{\kern0pt}right{\isacharunderscore}{\kern0pt}is{\isacharunderscore}{\kern0pt}true\ false{\isacharunderscore}{\kern0pt}func{\isacharunderscore}{\kern0pt}type{\isacharparenright}{\kern0pt}\isanewline
\ \ \ \ \ \ \isacommand{also}\isamarkupfalse%
\ \isacommand{have}\isamarkupfalse%
\ {\isachardoublequoteopen}{\isachardot}{\kern0pt}{\isachardot}{\kern0pt}{\isachardot}{\kern0pt}\ {\isacharequal}{\kern0pt}\ IMPLIES\ {\isasymcirc}\isactrlsub c\ x{\isachardoublequoteclose}\isanewline
\ \ \ \ \ \ \ \ \isacommand{by}\isamarkupfalse%
\ {\isacharparenleft}{\kern0pt}simp\ add{\isacharcolon}{\kern0pt}\ IMPLIES{\isacharunderscore}{\kern0pt}true{\isacharunderscore}{\kern0pt}true{\isacharunderscore}{\kern0pt}is{\isacharunderscore}{\kern0pt}true\ {\isacartoucheopen}u\ {\isacharequal}{\kern0pt}\ {\isasymt}{\isacartoucheclose}\ {\isacartoucheopen}v\ {\isacharequal}{\kern0pt}\ {\isasymt}{\isacartoucheclose}\ x{\isacharunderscore}{\kern0pt}form{\isacharparenright}{\kern0pt}\isanewline
\ \ \ \ \ \ \isacommand{then}\isamarkupfalse%
\ \isacommand{show}\isamarkupfalse%
\ {\isacharquery}{\kern0pt}thesis\isanewline
\ \ \ \ \ \ \ \ \isacommand{by}\isamarkupfalse%
\ {\isacharparenleft}{\kern0pt}simp\ add{\isacharcolon}{\kern0pt}\ calculation{\isacharparenright}{\kern0pt}\isanewline
\ \ \ \ \isacommand{next}\isamarkupfalse%
\isanewline
\ \ \ \ \ \ \isacommand{assume}\isamarkupfalse%
\ {\isachardoublequoteopen}v\ {\isasymnoteq}\ {\isasymt}{\isachardoublequoteclose}\isanewline
\ \ \ \ \ \ \isacommand{then}\isamarkupfalse%
\ \isacommand{have}\isamarkupfalse%
\ {\isachardoublequoteopen}v\ {\isacharequal}{\kern0pt}\ {\isasymf}{\isachardoublequoteclose}\isanewline
\ \ \ \ \ \ \ \ \isacommand{by}\isamarkupfalse%
\ {\isacharparenleft}{\kern0pt}metis\ true{\isacharunderscore}{\kern0pt}false{\isacharunderscore}{\kern0pt}only{\isacharunderscore}{\kern0pt}truth{\isacharunderscore}{\kern0pt}values\ x{\isacharunderscore}{\kern0pt}form{\isacharparenright}{\kern0pt}\isanewline
\ \ \ \ \ \ \isacommand{have}\isamarkupfalse%
\ {\isachardoublequoteopen}{\isacharparenleft}{\kern0pt}OR\ {\isasymcirc}\isactrlsub c\ NOT\ {\isasymtimes}\isactrlsub f\ id\isactrlsub c\ {\isasymOmega}{\isacharparenright}{\kern0pt}\ {\isasymcirc}\isactrlsub c\ x\ {\isacharequal}{\kern0pt}\ OR\ {\isasymcirc}\isactrlsub c\ {\isacharparenleft}{\kern0pt}NOT\ {\isasymtimes}\isactrlsub f\ id\isactrlsub c\ {\isasymOmega}{\isacharparenright}{\kern0pt}\ {\isasymcirc}\isactrlsub c\ x{\isachardoublequoteclose}\isanewline
\ \ \ \ \ \ \ \ \isacommand{using}\isamarkupfalse%
\ comp{\isacharunderscore}{\kern0pt}associative{\isadigit{2}}\ x{\isacharunderscore}{\kern0pt}type\ \isacommand{by}\isamarkupfalse%
\ {\isacharparenleft}{\kern0pt}typecheck{\isacharunderscore}{\kern0pt}cfuncs{\isacharcomma}{\kern0pt}\ force{\isacharparenright}{\kern0pt}\isanewline
\ \ \ \ \ \ \isacommand{also}\isamarkupfalse%
\ \isacommand{have}\isamarkupfalse%
\ {\isachardoublequoteopen}{\isachardot}{\kern0pt}{\isachardot}{\kern0pt}{\isachardot}{\kern0pt}\ {\isacharequal}{\kern0pt}\ OR\ {\isasymcirc}\isactrlsub c\ {\isasymlangle}NOT\ {\isasymcirc}\isactrlsub c\ {\isasymt}{\isacharcomma}{\kern0pt}\ id\isactrlsub c\ {\isasymOmega}\ {\isasymcirc}\isactrlsub c\ {\isasymf}{\isasymrangle}{\isachardoublequoteclose}\isanewline
\ \ \ \ \ \ \ \ \isacommand{by}\isamarkupfalse%
\ {\isacharparenleft}{\kern0pt}typecheck{\isacharunderscore}{\kern0pt}cfuncs{\isacharcomma}{\kern0pt}\ simp\ add{\isacharcolon}{\kern0pt}\ {\isacartoucheopen}u\ {\isacharequal}{\kern0pt}\ {\isasymt}{\isacartoucheclose}\ {\isacartoucheopen}v\ {\isacharequal}{\kern0pt}\ {\isasymf}{\isacartoucheclose}\ cfunc{\isacharunderscore}{\kern0pt}cross{\isacharunderscore}{\kern0pt}prod{\isacharunderscore}{\kern0pt}comp{\isacharunderscore}{\kern0pt}cfunc{\isacharunderscore}{\kern0pt}prod\ x{\isacharunderscore}{\kern0pt}form{\isacharparenright}{\kern0pt}\isanewline
\ \ \ \ \ \ \isacommand{also}\isamarkupfalse%
\ \isacommand{have}\isamarkupfalse%
\ {\isachardoublequoteopen}{\isachardot}{\kern0pt}{\isachardot}{\kern0pt}{\isachardot}{\kern0pt}\ {\isacharequal}{\kern0pt}\ OR\ {\isasymcirc}\isactrlsub c\ {\isasymlangle}{\isasymf}{\isacharcomma}{\kern0pt}\ {\isasymf}{\isasymrangle}{\isachardoublequoteclose}\isanewline
\ \ \ \ \ \ \ \ \isacommand{by}\isamarkupfalse%
\ {\isacharparenleft}{\kern0pt}typecheck{\isacharunderscore}{\kern0pt}cfuncs{\isacharcomma}{\kern0pt}\ simp\ add{\isacharcolon}{\kern0pt}\ NOT{\isacharunderscore}{\kern0pt}true{\isacharunderscore}{\kern0pt}is{\isacharunderscore}{\kern0pt}false\ id{\isacharunderscore}{\kern0pt}left{\isacharunderscore}{\kern0pt}unit{\isadigit{2}}{\isacharparenright}{\kern0pt}\isanewline
\ \ \ \ \ \ \isacommand{also}\isamarkupfalse%
\ \isacommand{have}\isamarkupfalse%
\ {\isachardoublequoteopen}{\isachardot}{\kern0pt}{\isachardot}{\kern0pt}{\isachardot}{\kern0pt}\ {\isacharequal}{\kern0pt}\ {\isasymf}{\isachardoublequoteclose}\isanewline
\ \ \ \ \ \ \ \ \isacommand{by}\isamarkupfalse%
\ {\isacharparenleft}{\kern0pt}simp\ add{\isacharcolon}{\kern0pt}\ OR{\isacharunderscore}{\kern0pt}false{\isacharunderscore}{\kern0pt}false{\isacharunderscore}{\kern0pt}is{\isacharunderscore}{\kern0pt}false\ false{\isacharunderscore}{\kern0pt}func{\isacharunderscore}{\kern0pt}type{\isacharparenright}{\kern0pt}\isanewline
\ \ \ \ \ \ \isacommand{also}\isamarkupfalse%
\ \isacommand{have}\isamarkupfalse%
\ {\isachardoublequoteopen}{\isachardot}{\kern0pt}{\isachardot}{\kern0pt}{\isachardot}{\kern0pt}\ {\isacharequal}{\kern0pt}\ IMPLIES\ {\isasymcirc}\isactrlsub c\ x{\isachardoublequoteclose}\isanewline
\ \ \ \ \ \ \ \ \isacommand{by}\isamarkupfalse%
\ {\isacharparenleft}{\kern0pt}simp\ add{\isacharcolon}{\kern0pt}\ IMPLIES{\isacharunderscore}{\kern0pt}true{\isacharunderscore}{\kern0pt}false{\isacharunderscore}{\kern0pt}is{\isacharunderscore}{\kern0pt}false\ {\isacartoucheopen}u\ {\isacharequal}{\kern0pt}\ {\isasymt}{\isacartoucheclose}\ {\isacartoucheopen}v\ {\isacharequal}{\kern0pt}\ {\isasymf}{\isacartoucheclose}\ x{\isacharunderscore}{\kern0pt}form{\isacharparenright}{\kern0pt}\isanewline
\ \ \ \ \ \ \isacommand{then}\isamarkupfalse%
\ \isacommand{show}\isamarkupfalse%
\ {\isacharquery}{\kern0pt}thesis\isanewline
\ \ \ \ \ \ \ \ \isacommand{by}\isamarkupfalse%
\ {\isacharparenleft}{\kern0pt}simp\ add{\isacharcolon}{\kern0pt}\ calculation{\isacharparenright}{\kern0pt}\isanewline
\ \ \ \ \isacommand{qed}\isamarkupfalse%
\isanewline
\ \ \isacommand{next}\isamarkupfalse%
\isanewline
\ \ \ \ \isacommand{assume}\isamarkupfalse%
\ {\isachardoublequoteopen}u\ {\isasymnoteq}\ {\isasymt}{\isachardoublequoteclose}\isanewline
\ \ \ \ \isacommand{then}\isamarkupfalse%
\ \isacommand{have}\isamarkupfalse%
\ {\isachardoublequoteopen}u\ {\isacharequal}{\kern0pt}\ {\isasymf}{\isachardoublequoteclose}\isanewline
\ \ \ \ \ \ \ \ \isacommand{by}\isamarkupfalse%
\ {\isacharparenleft}{\kern0pt}metis\ true{\isacharunderscore}{\kern0pt}false{\isacharunderscore}{\kern0pt}only{\isacharunderscore}{\kern0pt}truth{\isacharunderscore}{\kern0pt}values\ x{\isacharunderscore}{\kern0pt}form{\isacharparenright}{\kern0pt}\isanewline
\ \ \ \ \isacommand{show}\isamarkupfalse%
\ {\isacharquery}{\kern0pt}thesis\ \isanewline
\ \ \ \ \isacommand{proof}\isamarkupfalse%
{\isacharparenleft}{\kern0pt}cases\ {\isachardoublequoteopen}v\ {\isacharequal}{\kern0pt}\ {\isasymt}{\isachardoublequoteclose}{\isacharparenright}{\kern0pt}\isanewline
\ \ \ \ \ \ \isacommand{assume}\isamarkupfalse%
\ {\isachardoublequoteopen}v\ {\isacharequal}{\kern0pt}\ {\isasymt}{\isachardoublequoteclose}\isanewline
\ \ \ \ \ \ \isacommand{have}\isamarkupfalse%
\ {\isachardoublequoteopen}{\isacharparenleft}{\kern0pt}OR\ {\isasymcirc}\isactrlsub c\ NOT\ {\isasymtimes}\isactrlsub f\ id\isactrlsub c\ {\isasymOmega}{\isacharparenright}{\kern0pt}\ {\isasymcirc}\isactrlsub c\ x\ {\isacharequal}{\kern0pt}\ OR\ {\isasymcirc}\isactrlsub c\ {\isacharparenleft}{\kern0pt}NOT\ {\isasymtimes}\isactrlsub f\ id\isactrlsub c\ {\isasymOmega}{\isacharparenright}{\kern0pt}\ {\isasymcirc}\isactrlsub c\ x{\isachardoublequoteclose}\isanewline
\ \ \ \ \ \ \ \ \isacommand{using}\isamarkupfalse%
\ comp{\isacharunderscore}{\kern0pt}associative{\isadigit{2}}\ x{\isacharunderscore}{\kern0pt}type\ \isacommand{by}\isamarkupfalse%
\ {\isacharparenleft}{\kern0pt}typecheck{\isacharunderscore}{\kern0pt}cfuncs{\isacharcomma}{\kern0pt}\ force{\isacharparenright}{\kern0pt}\isanewline
\ \ \ \ \ \ \isacommand{also}\isamarkupfalse%
\ \isacommand{have}\isamarkupfalse%
\ {\isachardoublequoteopen}{\isachardot}{\kern0pt}{\isachardot}{\kern0pt}{\isachardot}{\kern0pt}\ {\isacharequal}{\kern0pt}\ OR\ {\isasymcirc}\isactrlsub c\ {\isasymlangle}NOT\ {\isasymcirc}\isactrlsub c\ {\isasymf}{\isacharcomma}{\kern0pt}\ id\isactrlsub c\ {\isasymOmega}\ {\isasymcirc}\isactrlsub c\ {\isasymt}{\isasymrangle}{\isachardoublequoteclose}\isanewline
\ \ \ \ \ \ \ \ \isacommand{by}\isamarkupfalse%
\ {\isacharparenleft}{\kern0pt}typecheck{\isacharunderscore}{\kern0pt}cfuncs{\isacharcomma}{\kern0pt}\ simp\ add{\isacharcolon}{\kern0pt}\ {\isacartoucheopen}u\ {\isacharequal}{\kern0pt}\ {\isasymf}{\isacartoucheclose}\ {\isacartoucheopen}v\ {\isacharequal}{\kern0pt}\ {\isasymt}{\isacartoucheclose}\ cfunc{\isacharunderscore}{\kern0pt}cross{\isacharunderscore}{\kern0pt}prod{\isacharunderscore}{\kern0pt}comp{\isacharunderscore}{\kern0pt}cfunc{\isacharunderscore}{\kern0pt}prod\ x{\isacharunderscore}{\kern0pt}form{\isacharparenright}{\kern0pt}\isanewline
\ \ \ \ \ \ \isacommand{also}\isamarkupfalse%
\ \isacommand{have}\isamarkupfalse%
\ {\isachardoublequoteopen}{\isachardot}{\kern0pt}{\isachardot}{\kern0pt}{\isachardot}{\kern0pt}\ {\isacharequal}{\kern0pt}\ OR\ {\isasymcirc}\isactrlsub c\ {\isasymlangle}{\isasymt}{\isacharcomma}{\kern0pt}\ {\isasymt}{\isasymrangle}{\isachardoublequoteclose}\isanewline
\ \ \ \ \ \ \ \ \isacommand{using}\isamarkupfalse%
\ NOT{\isacharunderscore}{\kern0pt}false{\isacharunderscore}{\kern0pt}is{\isacharunderscore}{\kern0pt}true\ id{\isacharunderscore}{\kern0pt}left{\isacharunderscore}{\kern0pt}unit{\isadigit{2}}\ true{\isacharunderscore}{\kern0pt}func{\isacharunderscore}{\kern0pt}type\ \isacommand{by}\isamarkupfalse%
\ smt\isanewline
\ \ \ \ \ \ \isacommand{also}\isamarkupfalse%
\ \isacommand{have}\isamarkupfalse%
\ {\isachardoublequoteopen}{\isachardot}{\kern0pt}{\isachardot}{\kern0pt}{\isachardot}{\kern0pt}\ {\isacharequal}{\kern0pt}\ {\isasymt}{\isachardoublequoteclose}\isanewline
\ \ \ \ \ \ \ \ \isacommand{by}\isamarkupfalse%
\ {\isacharparenleft}{\kern0pt}simp\ add{\isacharcolon}{\kern0pt}\ OR{\isacharunderscore}{\kern0pt}true{\isacharunderscore}{\kern0pt}right{\isacharunderscore}{\kern0pt}is{\isacharunderscore}{\kern0pt}true\ true{\isacharunderscore}{\kern0pt}func{\isacharunderscore}{\kern0pt}type{\isacharparenright}{\kern0pt}\isanewline
\ \ \ \ \ \ \isacommand{also}\isamarkupfalse%
\ \isacommand{have}\isamarkupfalse%
\ {\isachardoublequoteopen}{\isachardot}{\kern0pt}{\isachardot}{\kern0pt}{\isachardot}{\kern0pt}\ {\isacharequal}{\kern0pt}\ IMPLIES\ {\isasymcirc}\isactrlsub c\ x{\isachardoublequoteclose}\isanewline
\ \ \ \ \ \ \ \ \isacommand{by}\isamarkupfalse%
\ {\isacharparenleft}{\kern0pt}simp\ add{\isacharcolon}{\kern0pt}\ IMPLIES{\isacharunderscore}{\kern0pt}false{\isacharunderscore}{\kern0pt}true{\isacharunderscore}{\kern0pt}is{\isacharunderscore}{\kern0pt}true\ {\isacartoucheopen}u\ {\isacharequal}{\kern0pt}\ {\isasymf}{\isacartoucheclose}\ {\isacartoucheopen}v\ {\isacharequal}{\kern0pt}\ {\isasymt}{\isacartoucheclose}\ x{\isacharunderscore}{\kern0pt}form{\isacharparenright}{\kern0pt}\isanewline
\ \ \ \ \ \ \isacommand{then}\isamarkupfalse%
\ \isacommand{show}\isamarkupfalse%
\ {\isacharquery}{\kern0pt}thesis\isanewline
\ \ \ \ \ \ \ \ \isacommand{by}\isamarkupfalse%
\ {\isacharparenleft}{\kern0pt}simp\ add{\isacharcolon}{\kern0pt}\ calculation{\isacharparenright}{\kern0pt}\isanewline
\ \ \ \ \isacommand{next}\isamarkupfalse%
\isanewline
\ \ \ \ \ \ \isacommand{assume}\isamarkupfalse%
\ {\isachardoublequoteopen}v\ {\isasymnoteq}\ {\isasymt}{\isachardoublequoteclose}\isanewline
\ \ \ \ \ \ \isacommand{then}\isamarkupfalse%
\ \isacommand{have}\isamarkupfalse%
\ {\isachardoublequoteopen}v\ {\isacharequal}{\kern0pt}\ {\isasymf}{\isachardoublequoteclose}\isanewline
\ \ \ \ \ \ \ \ \isacommand{by}\isamarkupfalse%
\ {\isacharparenleft}{\kern0pt}metis\ true{\isacharunderscore}{\kern0pt}false{\isacharunderscore}{\kern0pt}only{\isacharunderscore}{\kern0pt}truth{\isacharunderscore}{\kern0pt}values\ x{\isacharunderscore}{\kern0pt}form{\isacharparenright}{\kern0pt}\isanewline
\ \ \ \ \ \ \isacommand{have}\isamarkupfalse%
\ {\isachardoublequoteopen}{\isacharparenleft}{\kern0pt}OR\ {\isasymcirc}\isactrlsub c\ NOT\ {\isasymtimes}\isactrlsub f\ id\isactrlsub c\ {\isasymOmega}{\isacharparenright}{\kern0pt}\ {\isasymcirc}\isactrlsub c\ x\ {\isacharequal}{\kern0pt}\ OR\ {\isasymcirc}\isactrlsub c\ {\isacharparenleft}{\kern0pt}NOT\ {\isasymtimes}\isactrlsub f\ id\isactrlsub c\ {\isasymOmega}{\isacharparenright}{\kern0pt}\ {\isasymcirc}\isactrlsub c\ x{\isachardoublequoteclose}\isanewline
\ \ \ \ \ \ \ \ \isacommand{using}\isamarkupfalse%
\ comp{\isacharunderscore}{\kern0pt}associative{\isadigit{2}}\ x{\isacharunderscore}{\kern0pt}type\ \isacommand{by}\isamarkupfalse%
\ {\isacharparenleft}{\kern0pt}typecheck{\isacharunderscore}{\kern0pt}cfuncs{\isacharcomma}{\kern0pt}\ force{\isacharparenright}{\kern0pt}\isanewline
\ \ \ \ \ \ \isacommand{also}\isamarkupfalse%
\ \isacommand{have}\isamarkupfalse%
\ {\isachardoublequoteopen}{\isachardot}{\kern0pt}{\isachardot}{\kern0pt}{\isachardot}{\kern0pt}\ {\isacharequal}{\kern0pt}\ OR\ {\isasymcirc}\isactrlsub c\ {\isasymlangle}NOT\ {\isasymcirc}\isactrlsub c\ {\isasymf}{\isacharcomma}{\kern0pt}\ id\isactrlsub c\ {\isasymOmega}\ {\isasymcirc}\isactrlsub c\ {\isasymf}{\isasymrangle}{\isachardoublequoteclose}\isanewline
\ \ \ \ \ \ \ \ \isacommand{by}\isamarkupfalse%
\ {\isacharparenleft}{\kern0pt}typecheck{\isacharunderscore}{\kern0pt}cfuncs{\isacharcomma}{\kern0pt}\ simp\ add{\isacharcolon}{\kern0pt}\ {\isacartoucheopen}u\ {\isacharequal}{\kern0pt}\ {\isasymf}{\isacartoucheclose}\ {\isacartoucheopen}v\ {\isacharequal}{\kern0pt}\ {\isasymf}{\isacartoucheclose}\ cfunc{\isacharunderscore}{\kern0pt}cross{\isacharunderscore}{\kern0pt}prod{\isacharunderscore}{\kern0pt}comp{\isacharunderscore}{\kern0pt}cfunc{\isacharunderscore}{\kern0pt}prod\ x{\isacharunderscore}{\kern0pt}form{\isacharparenright}{\kern0pt}\isanewline
\ \ \ \ \ \ \isacommand{also}\isamarkupfalse%
\ \isacommand{have}\isamarkupfalse%
\ {\isachardoublequoteopen}{\isachardot}{\kern0pt}{\isachardot}{\kern0pt}{\isachardot}{\kern0pt}\ {\isacharequal}{\kern0pt}\ OR\ {\isasymcirc}\isactrlsub c\ {\isasymlangle}{\isasymt}{\isacharcomma}{\kern0pt}\ {\isasymf}{\isasymrangle}{\isachardoublequoteclose}\isanewline
\ \ \ \ \ \ \ \ \isacommand{using}\isamarkupfalse%
\ NOT{\isacharunderscore}{\kern0pt}false{\isacharunderscore}{\kern0pt}is{\isacharunderscore}{\kern0pt}true\ false{\isacharunderscore}{\kern0pt}func{\isacharunderscore}{\kern0pt}type\ id{\isacharunderscore}{\kern0pt}left{\isacharunderscore}{\kern0pt}unit{\isadigit{2}}\ \isacommand{by}\isamarkupfalse%
\ presburger\isanewline
\ \ \ \ \ \ \isacommand{also}\isamarkupfalse%
\ \isacommand{have}\isamarkupfalse%
\ {\isachardoublequoteopen}{\isachardot}{\kern0pt}{\isachardot}{\kern0pt}{\isachardot}{\kern0pt}\ {\isacharequal}{\kern0pt}\ {\isasymt}{\isachardoublequoteclose}\isanewline
\ \ \ \ \ \ \ \ \isacommand{by}\isamarkupfalse%
\ {\isacharparenleft}{\kern0pt}simp\ add{\isacharcolon}{\kern0pt}\ OR{\isacharunderscore}{\kern0pt}true{\isacharunderscore}{\kern0pt}left{\isacharunderscore}{\kern0pt}is{\isacharunderscore}{\kern0pt}true\ false{\isacharunderscore}{\kern0pt}func{\isacharunderscore}{\kern0pt}type{\isacharparenright}{\kern0pt}\isanewline
\ \ \ \ \ \ \isacommand{also}\isamarkupfalse%
\ \isacommand{have}\isamarkupfalse%
\ {\isachardoublequoteopen}{\isachardot}{\kern0pt}{\isachardot}{\kern0pt}{\isachardot}{\kern0pt}\ {\isacharequal}{\kern0pt}\ IMPLIES\ {\isasymcirc}\isactrlsub c\ x{\isachardoublequoteclose}\isanewline
\ \ \ \ \ \ \ \ \isacommand{by}\isamarkupfalse%
\ {\isacharparenleft}{\kern0pt}simp\ add{\isacharcolon}{\kern0pt}\ IMPLIES{\isacharunderscore}{\kern0pt}false{\isacharunderscore}{\kern0pt}false{\isacharunderscore}{\kern0pt}is{\isacharunderscore}{\kern0pt}true\ {\isacartoucheopen}u\ {\isacharequal}{\kern0pt}\ {\isasymf}{\isacartoucheclose}\ {\isacartoucheopen}v\ {\isacharequal}{\kern0pt}\ {\isasymf}{\isacartoucheclose}\ x{\isacharunderscore}{\kern0pt}form{\isacharparenright}{\kern0pt}\isanewline
\ \ \ \ \ \ \isacommand{then}\isamarkupfalse%
\ \isacommand{show}\isamarkupfalse%
\ {\isacharquery}{\kern0pt}thesis\isanewline
\ \ \ \ \ \ \ \ \isacommand{by}\isamarkupfalse%
\ {\isacharparenleft}{\kern0pt}simp\ add{\isacharcolon}{\kern0pt}\ calculation{\isacharparenright}{\kern0pt}\isanewline
\ \ \ \ \isacommand{qed}\isamarkupfalse%
\isanewline
\ \ \isacommand{qed}\isamarkupfalse%
\isanewline
\isacommand{qed}\isamarkupfalse%
%
\endisatagproof
{\isafoldproof}%
%
\isadelimproof
\isanewline
%
\endisadelimproof
\isanewline
\isacommand{lemma}\isamarkupfalse%
\ IMPLIES{\isacharunderscore}{\kern0pt}implies{\isacharunderscore}{\kern0pt}implies{\isacharcolon}{\kern0pt}\isanewline
\ \ \isakeyword{assumes}\ P{\isacharunderscore}{\kern0pt}type{\isacharbrackleft}{\kern0pt}type{\isacharunderscore}{\kern0pt}rule{\isacharbrackright}{\kern0pt}{\isacharcolon}{\kern0pt}\ {\isachardoublequoteopen}P\ {\isacharcolon}{\kern0pt}\ X\ {\isasymrightarrow}\ {\isasymOmega}{\isachardoublequoteclose}\ \isakeyword{and}\ Q{\isacharunderscore}{\kern0pt}type{\isacharbrackleft}{\kern0pt}type{\isacharunderscore}{\kern0pt}rule{\isacharbrackright}{\kern0pt}{\isacharcolon}{\kern0pt}\ {\isachardoublequoteopen}Q\ {\isacharcolon}{\kern0pt}\ Y\ {\isasymrightarrow}\ {\isasymOmega}{\isachardoublequoteclose}\isanewline
\ \ \isakeyword{assumes}\ X{\isacharunderscore}{\kern0pt}nonempty{\isacharcolon}{\kern0pt}\ {\isachardoublequoteopen}{\isasymexists}x{\isachardot}{\kern0pt}\ x\ {\isasymin}\isactrlsub c\ X{\isachardoublequoteclose}\isanewline
\ \ \isakeyword{assumes}\ IMPLIES{\isacharunderscore}{\kern0pt}true{\isacharcolon}{\kern0pt}\ {\isachardoublequoteopen}IMPLIES\ {\isasymcirc}\isactrlsub c\ {\isacharparenleft}{\kern0pt}P\ {\isasymtimes}\isactrlsub f\ Q{\isacharparenright}{\kern0pt}\ {\isacharequal}{\kern0pt}\ {\isasymt}\ {\isasymcirc}\isactrlsub c\ {\isasymbeta}\isactrlbsub X\ {\isasymtimes}\isactrlsub c\ Y\isactrlesub {\isachardoublequoteclose}\isanewline
\ \ \isakeyword{shows}\ {\isachardoublequoteopen}P\ {\isacharequal}{\kern0pt}\ {\isasymt}\ {\isasymcirc}\isactrlsub c\ {\isasymbeta}\isactrlbsub X\isactrlesub \ {\isasymLongrightarrow}\ Q\ {\isacharequal}{\kern0pt}\ {\isasymt}\ {\isasymcirc}\isactrlsub c\ {\isasymbeta}\isactrlbsub Y\isactrlesub {\isachardoublequoteclose}\isanewline
%
\isadelimproof
%
\endisadelimproof
%
\isatagproof
\isacommand{proof}\isamarkupfalse%
\ {\isacharminus}{\kern0pt}\isanewline
\ \ \isacommand{obtain}\isamarkupfalse%
\ z\ \isakeyword{where}\ z{\isacharunderscore}{\kern0pt}type{\isacharbrackleft}{\kern0pt}type{\isacharunderscore}{\kern0pt}rule{\isacharbrackright}{\kern0pt}{\isacharcolon}{\kern0pt}\ {\isachardoublequoteopen}z\ {\isacharcolon}{\kern0pt}\ X\ {\isasymtimes}\isactrlsub c\ Y\ {\isasymrightarrow}\ {\isasymone}\ {\isasymCoprod}\ {\isasymone}\ {\isasymCoprod}\ {\isasymone}{\isachardoublequoteclose}\isanewline
\ \ \ \ \isakeyword{and}\ z{\isacharunderscore}{\kern0pt}eq{\isacharcolon}{\kern0pt}\ {\isachardoublequoteopen}P\ {\isasymtimes}\isactrlsub f\ Q\ {\isacharequal}{\kern0pt}\ {\isacharparenleft}{\kern0pt}{\isasymlangle}{\isasymt}{\isacharcomma}{\kern0pt}{\isasymt}{\isasymrangle}\ {\isasymamalg}\ {\isasymlangle}{\isasymf}{\isacharcomma}{\kern0pt}{\isasymf}{\isasymrangle}\ {\isasymamalg}\ {\isasymlangle}{\isasymf}{\isacharcomma}{\kern0pt}{\isasymt}{\isasymrangle}{\isacharparenright}{\kern0pt}\ {\isasymcirc}\isactrlsub c\ z{\isachardoublequoteclose}\isanewline
\ \ \ \ \isacommand{using}\isamarkupfalse%
\ IMPLIES{\isacharunderscore}{\kern0pt}is{\isacharunderscore}{\kern0pt}pullback\ \isacommand{unfolding}\isamarkupfalse%
\ is{\isacharunderscore}{\kern0pt}pullback{\isacharunderscore}{\kern0pt}def\isanewline
\ \ \ \ \isacommand{by}\isamarkupfalse%
\ {\isacharparenleft}{\kern0pt}auto{\isacharcomma}{\kern0pt}\ typecheck{\isacharunderscore}{\kern0pt}cfuncs{\isacharcomma}{\kern0pt}\ metis\ IMPLIES{\isacharunderscore}{\kern0pt}true\ terminal{\isacharunderscore}{\kern0pt}func{\isacharunderscore}{\kern0pt}type{\isacharparenright}{\kern0pt}\ \ \isanewline
\ \ \isacommand{assume}\isamarkupfalse%
\ P{\isacharunderscore}{\kern0pt}true{\isacharcolon}{\kern0pt}\ {\isachardoublequoteopen}P\ {\isacharequal}{\kern0pt}\ {\isasymt}\ {\isasymcirc}\isactrlsub c\ {\isasymbeta}\isactrlbsub X\isactrlesub {\isachardoublequoteclose}\isanewline
\ \ \isanewline
\ \ \isacommand{have}\isamarkupfalse%
\ {\isachardoublequoteopen}left{\isacharunderscore}{\kern0pt}cart{\isacharunderscore}{\kern0pt}proj\ {\isasymOmega}\ {\isasymOmega}\ {\isasymcirc}\isactrlsub c\ {\isacharparenleft}{\kern0pt}P\ {\isasymtimes}\isactrlsub f\ Q{\isacharparenright}{\kern0pt}\ {\isacharequal}{\kern0pt}\ left{\isacharunderscore}{\kern0pt}cart{\isacharunderscore}{\kern0pt}proj\ {\isasymOmega}\ {\isasymOmega}\ {\isasymcirc}\isactrlsub c\ {\isacharparenleft}{\kern0pt}{\isasymlangle}{\isasymt}{\isacharcomma}{\kern0pt}{\isasymt}{\isasymrangle}\ {\isasymamalg}\ {\isasymlangle}{\isasymf}{\isacharcomma}{\kern0pt}{\isasymf}{\isasymrangle}\ {\isasymamalg}\ {\isasymlangle}{\isasymf}{\isacharcomma}{\kern0pt}{\isasymt}{\isasymrangle}{\isacharparenright}{\kern0pt}\ {\isasymcirc}\isactrlsub c\ z{\isachardoublequoteclose}\isanewline
\ \ \ \ \isacommand{using}\isamarkupfalse%
\ z{\isacharunderscore}{\kern0pt}eq\ \isacommand{by}\isamarkupfalse%
\ simp\isanewline
\ \ \isacommand{then}\isamarkupfalse%
\ \isacommand{have}\isamarkupfalse%
\ {\isachardoublequoteopen}P\ {\isasymcirc}\isactrlsub c\ left{\isacharunderscore}{\kern0pt}cart{\isacharunderscore}{\kern0pt}proj\ X\ Y\ {\isacharequal}{\kern0pt}\ {\isacharparenleft}{\kern0pt}left{\isacharunderscore}{\kern0pt}cart{\isacharunderscore}{\kern0pt}proj\ {\isasymOmega}\ {\isasymOmega}\ {\isasymcirc}\isactrlsub c\ {\isacharparenleft}{\kern0pt}{\isasymlangle}{\isasymt}{\isacharcomma}{\kern0pt}{\isasymt}{\isasymrangle}\ {\isasymamalg}\ {\isasymlangle}{\isasymf}{\isacharcomma}{\kern0pt}{\isasymf}{\isasymrangle}\ {\isasymamalg}\ {\isasymlangle}{\isasymf}{\isacharcomma}{\kern0pt}{\isasymt}{\isasymrangle}{\isacharparenright}{\kern0pt}{\isacharparenright}{\kern0pt}\ {\isasymcirc}\isactrlsub c\ z{\isachardoublequoteclose}\isanewline
\ \ \ \ \isacommand{using}\isamarkupfalse%
\ Q{\isacharunderscore}{\kern0pt}type\ comp{\isacharunderscore}{\kern0pt}associative{\isadigit{2}}\ left{\isacharunderscore}{\kern0pt}cart{\isacharunderscore}{\kern0pt}proj{\isacharunderscore}{\kern0pt}cfunc{\isacharunderscore}{\kern0pt}cross{\isacharunderscore}{\kern0pt}prod\ \isacommand{by}\isamarkupfalse%
\ {\isacharparenleft}{\kern0pt}typecheck{\isacharunderscore}{\kern0pt}cfuncs{\isacharcomma}{\kern0pt}\ force{\isacharparenright}{\kern0pt}\isanewline
\ \ \isacommand{then}\isamarkupfalse%
\ \isacommand{have}\isamarkupfalse%
\ {\isachardoublequoteopen}P\ {\isasymcirc}\isactrlsub c\ left{\isacharunderscore}{\kern0pt}cart{\isacharunderscore}{\kern0pt}proj\ X\ Y\isanewline
\ \ \ \ {\isacharequal}{\kern0pt}\ {\isacharparenleft}{\kern0pt}{\isacharparenleft}{\kern0pt}left{\isacharunderscore}{\kern0pt}cart{\isacharunderscore}{\kern0pt}proj\ {\isasymOmega}\ {\isasymOmega}\ {\isasymcirc}\isactrlsub c\ {\isasymlangle}{\isasymt}{\isacharcomma}{\kern0pt}{\isasymt}{\isasymrangle}{\isacharparenright}{\kern0pt}\ {\isasymamalg}\ {\isacharparenleft}{\kern0pt}left{\isacharunderscore}{\kern0pt}cart{\isacharunderscore}{\kern0pt}proj\ {\isasymOmega}\ {\isasymOmega}\ {\isasymcirc}\isactrlsub c\ {\isasymlangle}{\isasymf}{\isacharcomma}{\kern0pt}{\isasymf}{\isasymrangle}{\isacharparenright}{\kern0pt}\ {\isasymamalg}\ {\isacharparenleft}{\kern0pt}left{\isacharunderscore}{\kern0pt}cart{\isacharunderscore}{\kern0pt}proj\ {\isasymOmega}\ {\isasymOmega}\ {\isasymcirc}\isactrlsub c\ {\isasymlangle}{\isasymf}{\isacharcomma}{\kern0pt}{\isasymt}{\isasymrangle}{\isacharparenright}{\kern0pt}{\isacharparenright}{\kern0pt}\ {\isasymcirc}\isactrlsub c\ z{\isachardoublequoteclose}\isanewline
\ \ \ \ \isacommand{by}\isamarkupfalse%
\ {\isacharparenleft}{\kern0pt}typecheck{\isacharunderscore}{\kern0pt}cfuncs{\isacharunderscore}{\kern0pt}prems{\isacharcomma}{\kern0pt}\ simp\ add{\isacharcolon}{\kern0pt}\ cfunc{\isacharunderscore}{\kern0pt}coprod{\isacharunderscore}{\kern0pt}comp{\isacharparenright}{\kern0pt}\isanewline
\ \ \isacommand{then}\isamarkupfalse%
\ \isacommand{have}\isamarkupfalse%
\ {\isachardoublequoteopen}P\ {\isasymcirc}\isactrlsub c\ left{\isacharunderscore}{\kern0pt}cart{\isacharunderscore}{\kern0pt}proj\ X\ Y\ {\isacharequal}{\kern0pt}\ {\isacharparenleft}{\kern0pt}{\isasymt}\ {\isasymamalg}\ {\isasymf}\ {\isasymamalg}\ {\isasymf}{\isacharparenright}{\kern0pt}\ {\isasymcirc}\isactrlsub c\ z{\isachardoublequoteclose}\isanewline
\ \ \ \ \isacommand{by}\isamarkupfalse%
\ {\isacharparenleft}{\kern0pt}typecheck{\isacharunderscore}{\kern0pt}cfuncs{\isacharunderscore}{\kern0pt}prems{\isacharcomma}{\kern0pt}\ smt\ left{\isacharunderscore}{\kern0pt}cart{\isacharunderscore}{\kern0pt}proj{\isacharunderscore}{\kern0pt}cfunc{\isacharunderscore}{\kern0pt}prod{\isacharparenright}{\kern0pt}\isanewline
\isanewline
\ \ \isacommand{show}\isamarkupfalse%
\ {\isachardoublequoteopen}Q\ {\isacharequal}{\kern0pt}\ {\isasymt}\ {\isasymcirc}\isactrlsub c\ {\isasymbeta}\isactrlbsub Y\isactrlesub {\isachardoublequoteclose}\isanewline
\ \ \isacommand{proof}\isamarkupfalse%
\ {\isacharparenleft}{\kern0pt}etcs{\isacharunderscore}{\kern0pt}rule\ one{\isacharunderscore}{\kern0pt}separator{\isacharparenright}{\kern0pt}\isanewline
\ \ \ \ \isacommand{fix}\isamarkupfalse%
\ y\isanewline
\ \ \ \ \isacommand{assume}\isamarkupfalse%
\ y{\isacharunderscore}{\kern0pt}in{\isacharunderscore}{\kern0pt}Y{\isacharbrackleft}{\kern0pt}type{\isacharunderscore}{\kern0pt}rule{\isacharbrackright}{\kern0pt}{\isacharcolon}{\kern0pt}\ {\isachardoublequoteopen}y\ {\isasymin}\isactrlsub c\ Y{\isachardoublequoteclose}\isanewline
\ \ \ \ \isacommand{obtain}\isamarkupfalse%
\ x\ \isakeyword{where}\ x{\isacharunderscore}{\kern0pt}in{\isacharunderscore}{\kern0pt}X{\isacharbrackleft}{\kern0pt}type{\isacharunderscore}{\kern0pt}rule{\isacharbrackright}{\kern0pt}{\isacharcolon}{\kern0pt}\ {\isachardoublequoteopen}x\ {\isasymin}\isactrlsub c\ X{\isachardoublequoteclose}\isanewline
\ \ \ \ \ \ \isacommand{using}\isamarkupfalse%
\ X{\isacharunderscore}{\kern0pt}nonempty\ \isacommand{by}\isamarkupfalse%
\ blast\isanewline
\isanewline
\ \ \ \ \isacommand{have}\isamarkupfalse%
\ {\isachardoublequoteopen}z\ {\isasymcirc}\isactrlsub c\ {\isasymlangle}x{\isacharcomma}{\kern0pt}y{\isasymrangle}\ {\isacharequal}{\kern0pt}\ left{\isacharunderscore}{\kern0pt}coproj\ {\isasymone}\ {\isacharparenleft}{\kern0pt}{\isasymone}\ {\isasymCoprod}\ {\isasymone}{\isacharparenright}{\kern0pt}\isanewline
\ \ \ \ \ \ \ \ {\isasymor}\ z\ {\isasymcirc}\isactrlsub c\ {\isasymlangle}x{\isacharcomma}{\kern0pt}y{\isasymrangle}\ {\isacharequal}{\kern0pt}\ right{\isacharunderscore}{\kern0pt}coproj\ {\isasymone}\ {\isacharparenleft}{\kern0pt}{\isasymone}\ {\isasymCoprod}\ {\isasymone}{\isacharparenright}{\kern0pt}\ {\isasymcirc}\isactrlsub c\ left{\isacharunderscore}{\kern0pt}coproj\ {\isasymone}\ {\isasymone}\isanewline
\ \ \ \ \ \ \ \ {\isasymor}\ z\ {\isasymcirc}\isactrlsub c\ {\isasymlangle}x{\isacharcomma}{\kern0pt}y{\isasymrangle}\ {\isacharequal}{\kern0pt}\ right{\isacharunderscore}{\kern0pt}coproj\ {\isasymone}\ {\isacharparenleft}{\kern0pt}{\isasymone}\ {\isasymCoprod}\ {\isasymone}{\isacharparenright}{\kern0pt}\ {\isasymcirc}\isactrlsub c\ right{\isacharunderscore}{\kern0pt}coproj\ {\isasymone}\ {\isasymone}{\isachardoublequoteclose}\isanewline
\ \ \ \ \ \ \isacommand{by}\isamarkupfalse%
\ {\isacharparenleft}{\kern0pt}typecheck{\isacharunderscore}{\kern0pt}cfuncs{\isacharcomma}{\kern0pt}\ smt\ comp{\isacharunderscore}{\kern0pt}associative{\isadigit{2}}\ coprojs{\isacharunderscore}{\kern0pt}jointly{\isacharunderscore}{\kern0pt}surj\ one{\isacharunderscore}{\kern0pt}unique{\isacharunderscore}{\kern0pt}element{\isacharparenright}{\kern0pt}\isanewline
\ \ \ \ \isacommand{then}\isamarkupfalse%
\ \isacommand{show}\isamarkupfalse%
\ {\isachardoublequoteopen}Q\ {\isasymcirc}\isactrlsub c\ y\ {\isacharequal}{\kern0pt}\ {\isacharparenleft}{\kern0pt}{\isasymt}\ {\isasymcirc}\isactrlsub c\ {\isasymbeta}\isactrlbsub Y\isactrlesub {\isacharparenright}{\kern0pt}\ {\isasymcirc}\isactrlsub c\ y{\isachardoublequoteclose}\isanewline
\ \ \ \ \isacommand{proof}\isamarkupfalse%
\ safe\isanewline
\ \ \ \ \ \ \isacommand{assume}\isamarkupfalse%
\ {\isachardoublequoteopen}z\ {\isasymcirc}\isactrlsub c\ {\isasymlangle}x{\isacharcomma}{\kern0pt}y{\isasymrangle}\ {\isacharequal}{\kern0pt}\ left{\isacharunderscore}{\kern0pt}coproj\ {\isasymone}\ {\isacharparenleft}{\kern0pt}{\isasymone}\ {\isasymCoprod}\ {\isasymone}{\isacharparenright}{\kern0pt}{\isachardoublequoteclose}\isanewline
\ \ \ \ \ \ \isacommand{then}\isamarkupfalse%
\ \isacommand{have}\isamarkupfalse%
\ {\isachardoublequoteopen}{\isacharparenleft}{\kern0pt}P\ {\isasymtimes}\isactrlsub f\ Q{\isacharparenright}{\kern0pt}\ {\isasymcirc}\isactrlsub c\ {\isasymlangle}x{\isacharcomma}{\kern0pt}y{\isasymrangle}\ {\isacharequal}{\kern0pt}\ {\isacharparenleft}{\kern0pt}{\isasymlangle}{\isasymt}{\isacharcomma}{\kern0pt}{\isasymt}{\isasymrangle}\ {\isasymamalg}\ {\isasymlangle}{\isasymf}{\isacharcomma}{\kern0pt}{\isasymf}{\isasymrangle}\ {\isasymamalg}\ {\isasymlangle}{\isasymf}{\isacharcomma}{\kern0pt}{\isasymt}{\isasymrangle}{\isacharparenright}{\kern0pt}\ {\isasymcirc}\isactrlsub c\ left{\isacharunderscore}{\kern0pt}coproj\ {\isasymone}\ {\isacharparenleft}{\kern0pt}{\isasymone}\ {\isasymCoprod}\ {\isasymone}{\isacharparenright}{\kern0pt}{\isachardoublequoteclose}\isanewline
\ \ \ \ \ \ \ \ \isacommand{by}\isamarkupfalse%
\ {\isacharparenleft}{\kern0pt}typecheck{\isacharunderscore}{\kern0pt}cfuncs{\isacharcomma}{\kern0pt}\ smt\ comp{\isacharunderscore}{\kern0pt}associative{\isadigit{2}}\ z{\isacharunderscore}{\kern0pt}eq\ z{\isacharunderscore}{\kern0pt}type{\isacharparenright}{\kern0pt}\isanewline
\ \ \ \ \ \ \isacommand{then}\isamarkupfalse%
\ \isacommand{have}\isamarkupfalse%
\ {\isachardoublequoteopen}{\isacharparenleft}{\kern0pt}P\ {\isasymtimes}\isactrlsub f\ Q{\isacharparenright}{\kern0pt}\ {\isasymcirc}\isactrlsub c\ {\isasymlangle}x{\isacharcomma}{\kern0pt}y{\isasymrangle}\ {\isacharequal}{\kern0pt}\ {\isasymlangle}{\isasymt}{\isacharcomma}{\kern0pt}{\isasymt}{\isasymrangle}{\isachardoublequoteclose}\isanewline
\ \ \ \ \ \ \ \ \isacommand{by}\isamarkupfalse%
\ {\isacharparenleft}{\kern0pt}typecheck{\isacharunderscore}{\kern0pt}cfuncs{\isacharunderscore}{\kern0pt}prems{\isacharcomma}{\kern0pt}\ smt\ left{\isacharunderscore}{\kern0pt}coproj{\isacharunderscore}{\kern0pt}cfunc{\isacharunderscore}{\kern0pt}coprod{\isacharparenright}{\kern0pt}\isanewline
\ \ \ \ \ \ \isacommand{then}\isamarkupfalse%
\ \isacommand{have}\isamarkupfalse%
\ {\isachardoublequoteopen}Q\ {\isasymcirc}\isactrlsub c\ y\ {\isacharequal}{\kern0pt}\ {\isasymt}{\isachardoublequoteclose}\isanewline
\ \ \ \ \ \ \ \ \isacommand{by}\isamarkupfalse%
\ {\isacharparenleft}{\kern0pt}typecheck{\isacharunderscore}{\kern0pt}cfuncs{\isacharunderscore}{\kern0pt}prems{\isacharcomma}{\kern0pt}\ smt\ {\isacharparenleft}{\kern0pt}verit{\isacharcomma}{\kern0pt}\ best{\isacharparenright}{\kern0pt}\ cfunc{\isacharunderscore}{\kern0pt}cross{\isacharunderscore}{\kern0pt}prod{\isacharunderscore}{\kern0pt}comp{\isacharunderscore}{\kern0pt}cfunc{\isacharunderscore}{\kern0pt}prod\ comp{\isacharunderscore}{\kern0pt}associative{\isadigit{2}}\ comp{\isacharunderscore}{\kern0pt}type\ id{\isacharunderscore}{\kern0pt}right{\isacharunderscore}{\kern0pt}unit{\isadigit{2}}\ right{\isacharunderscore}{\kern0pt}cart{\isacharunderscore}{\kern0pt}proj{\isacharunderscore}{\kern0pt}cfunc{\isacharunderscore}{\kern0pt}prod{\isacharparenright}{\kern0pt}\isanewline
\ \ \ \ \ \ \isacommand{then}\isamarkupfalse%
\ \isacommand{show}\isamarkupfalse%
\ {\isachardoublequoteopen}Q\ {\isasymcirc}\isactrlsub c\ y\ {\isacharequal}{\kern0pt}\ {\isacharparenleft}{\kern0pt}{\isasymt}\ {\isasymcirc}\isactrlsub c\ {\isasymbeta}\isactrlbsub Y\isactrlesub {\isacharparenright}{\kern0pt}\ {\isasymcirc}\isactrlsub c\ y{\isachardoublequoteclose}\isanewline
\ \ \ \ \ \ \ \ \isacommand{by}\isamarkupfalse%
\ {\isacharparenleft}{\kern0pt}smt\ {\isacharparenleft}{\kern0pt}verit{\isacharcomma}{\kern0pt}\ best{\isacharparenright}{\kern0pt}\ comp{\isacharunderscore}{\kern0pt}associative{\isadigit{2}}\ id{\isacharunderscore}{\kern0pt}right{\isacharunderscore}{\kern0pt}unit{\isadigit{2}}\ terminal{\isacharunderscore}{\kern0pt}func{\isacharunderscore}{\kern0pt}comp{\isacharunderscore}{\kern0pt}elem\ terminal{\isacharunderscore}{\kern0pt}func{\isacharunderscore}{\kern0pt}type\ true{\isacharunderscore}{\kern0pt}func{\isacharunderscore}{\kern0pt}type\ y{\isacharunderscore}{\kern0pt}in{\isacharunderscore}{\kern0pt}Y{\isacharparenright}{\kern0pt}\isanewline
\ \ \ \ \isacommand{next}\isamarkupfalse%
\isanewline
\ \ \ \ \ \ \isacommand{assume}\isamarkupfalse%
\ {\isachardoublequoteopen}z\ {\isasymcirc}\isactrlsub c\ {\isasymlangle}x{\isacharcomma}{\kern0pt}y{\isasymrangle}\ {\isacharequal}{\kern0pt}\ right{\isacharunderscore}{\kern0pt}coproj\ {\isasymone}\ {\isacharparenleft}{\kern0pt}{\isasymone}\ {\isasymCoprod}\ {\isasymone}{\isacharparenright}{\kern0pt}\ {\isasymcirc}\isactrlsub c\ left{\isacharunderscore}{\kern0pt}coproj\ {\isasymone}\ {\isasymone}{\isachardoublequoteclose}\isanewline
\ \ \ \ \ \ \isacommand{then}\isamarkupfalse%
\ \isacommand{have}\isamarkupfalse%
\ {\isachardoublequoteopen}{\isacharparenleft}{\kern0pt}P\ {\isasymtimes}\isactrlsub f\ Q{\isacharparenright}{\kern0pt}\ {\isasymcirc}\isactrlsub c\ {\isasymlangle}x{\isacharcomma}{\kern0pt}y{\isasymrangle}\ {\isacharequal}{\kern0pt}\ {\isacharparenleft}{\kern0pt}{\isasymlangle}{\isasymt}{\isacharcomma}{\kern0pt}{\isasymt}{\isasymrangle}\ {\isasymamalg}\ {\isasymlangle}{\isasymf}{\isacharcomma}{\kern0pt}{\isasymf}{\isasymrangle}\ {\isasymamalg}\ {\isasymlangle}{\isasymf}{\isacharcomma}{\kern0pt}{\isasymt}{\isasymrangle}{\isacharparenright}{\kern0pt}\ {\isasymcirc}\isactrlsub c\ right{\isacharunderscore}{\kern0pt}coproj\ {\isasymone}\ {\isacharparenleft}{\kern0pt}{\isasymone}\ {\isasymCoprod}\ {\isasymone}{\isacharparenright}{\kern0pt}\ {\isasymcirc}\isactrlsub c\ left{\isacharunderscore}{\kern0pt}coproj\ {\isasymone}\ {\isasymone}{\isachardoublequoteclose}\isanewline
\ \ \ \ \ \ \ \ \isacommand{by}\isamarkupfalse%
\ {\isacharparenleft}{\kern0pt}typecheck{\isacharunderscore}{\kern0pt}cfuncs{\isacharcomma}{\kern0pt}\ smt\ comp{\isacharunderscore}{\kern0pt}associative{\isadigit{2}}\ z{\isacharunderscore}{\kern0pt}eq\ z{\isacharunderscore}{\kern0pt}type{\isacharparenright}{\kern0pt}\isanewline
\ \ \ \ \ \ \isacommand{then}\isamarkupfalse%
\ \isacommand{have}\isamarkupfalse%
\ {\isachardoublequoteopen}{\isacharparenleft}{\kern0pt}P\ {\isasymtimes}\isactrlsub f\ Q{\isacharparenright}{\kern0pt}\ {\isasymcirc}\isactrlsub c\ {\isasymlangle}x{\isacharcomma}{\kern0pt}y{\isasymrangle}\ {\isacharequal}{\kern0pt}\ {\isacharparenleft}{\kern0pt}{\isasymlangle}{\isasymf}{\isacharcomma}{\kern0pt}{\isasymf}{\isasymrangle}\ {\isasymamalg}\ {\isasymlangle}{\isasymf}{\isacharcomma}{\kern0pt}{\isasymt}{\isasymrangle}{\isacharparenright}{\kern0pt}\ {\isasymcirc}\isactrlsub c\ left{\isacharunderscore}{\kern0pt}coproj\ {\isasymone}\ {\isasymone}{\isachardoublequoteclose}\isanewline
\ \ \ \ \ \ \ \ \isacommand{by}\isamarkupfalse%
\ {\isacharparenleft}{\kern0pt}typecheck{\isacharunderscore}{\kern0pt}cfuncs{\isacharunderscore}{\kern0pt}prems{\isacharcomma}{\kern0pt}\ smt\ right{\isacharunderscore}{\kern0pt}coproj{\isacharunderscore}{\kern0pt}cfunc{\isacharunderscore}{\kern0pt}coprod\ comp{\isacharunderscore}{\kern0pt}associative{\isadigit{2}}{\isacharparenright}{\kern0pt}\isanewline
\ \ \ \ \ \ \isacommand{then}\isamarkupfalse%
\ \isacommand{have}\isamarkupfalse%
\ {\isachardoublequoteopen}{\isacharparenleft}{\kern0pt}P\ {\isasymtimes}\isactrlsub f\ Q{\isacharparenright}{\kern0pt}\ {\isasymcirc}\isactrlsub c\ {\isasymlangle}x{\isacharcomma}{\kern0pt}y{\isasymrangle}\ {\isacharequal}{\kern0pt}\ {\isasymlangle}{\isasymf}{\isacharcomma}{\kern0pt}{\isasymf}{\isasymrangle}{\isachardoublequoteclose}\isanewline
\ \ \ \ \ \ \ \ \isacommand{by}\isamarkupfalse%
\ {\isacharparenleft}{\kern0pt}typecheck{\isacharunderscore}{\kern0pt}cfuncs{\isacharunderscore}{\kern0pt}prems{\isacharcomma}{\kern0pt}\ smt\ left{\isacharunderscore}{\kern0pt}coproj{\isacharunderscore}{\kern0pt}cfunc{\isacharunderscore}{\kern0pt}coprod{\isacharparenright}{\kern0pt}\isanewline
\ \ \ \ \ \ \isacommand{then}\isamarkupfalse%
\ \isacommand{have}\isamarkupfalse%
\ {\isachardoublequoteopen}P\ {\isasymcirc}\isactrlsub c\ x\ {\isacharequal}{\kern0pt}\ {\isasymf}{\isachardoublequoteclose}\isanewline
\ \ \ \ \ \ \ \ \isacommand{by}\isamarkupfalse%
\ {\isacharparenleft}{\kern0pt}typecheck{\isacharunderscore}{\kern0pt}cfuncs{\isacharunderscore}{\kern0pt}prems{\isacharcomma}{\kern0pt}\ smt\ {\isacharparenleft}{\kern0pt}verit{\isacharcomma}{\kern0pt}\ best{\isacharparenright}{\kern0pt}\ cfunc{\isacharunderscore}{\kern0pt}cross{\isacharunderscore}{\kern0pt}prod{\isacharunderscore}{\kern0pt}comp{\isacharunderscore}{\kern0pt}cfunc{\isacharunderscore}{\kern0pt}prod\ comp{\isacharunderscore}{\kern0pt}associative{\isadigit{2}}\ comp{\isacharunderscore}{\kern0pt}type\ id{\isacharunderscore}{\kern0pt}right{\isacharunderscore}{\kern0pt}unit{\isadigit{2}}\ left{\isacharunderscore}{\kern0pt}cart{\isacharunderscore}{\kern0pt}proj{\isacharunderscore}{\kern0pt}cfunc{\isacharunderscore}{\kern0pt}prod{\isacharparenright}{\kern0pt}\isanewline
\ \ \ \ \ \ \isacommand{also}\isamarkupfalse%
\ \isacommand{have}\isamarkupfalse%
\ {\isachardoublequoteopen}P\ {\isasymcirc}\isactrlsub c\ x\ {\isacharequal}{\kern0pt}\ {\isasymt}{\isachardoublequoteclose}\isanewline
\ \ \ \ \ \ \ \ \isacommand{using}\isamarkupfalse%
\ P{\isacharunderscore}{\kern0pt}true\ \isacommand{by}\isamarkupfalse%
\ {\isacharparenleft}{\kern0pt}typecheck{\isacharunderscore}{\kern0pt}cfuncs{\isacharunderscore}{\kern0pt}prems{\isacharcomma}{\kern0pt}\ smt\ {\isacharparenleft}{\kern0pt}z{\isadigit{3}}{\isacharparenright}{\kern0pt}\ comp{\isacharunderscore}{\kern0pt}associative{\isadigit{2}}\ id{\isacharunderscore}{\kern0pt}right{\isacharunderscore}{\kern0pt}unit{\isadigit{2}}\ id{\isacharunderscore}{\kern0pt}type\ one{\isacharunderscore}{\kern0pt}unique{\isacharunderscore}{\kern0pt}element\ terminal{\isacharunderscore}{\kern0pt}func{\isacharunderscore}{\kern0pt}comp\ terminal{\isacharunderscore}{\kern0pt}func{\isacharunderscore}{\kern0pt}type\ x{\isacharunderscore}{\kern0pt}in{\isacharunderscore}{\kern0pt}X{\isacharparenright}{\kern0pt}\isanewline
\ \ \ \ \ \ \isacommand{then}\isamarkupfalse%
\ \isacommand{have}\isamarkupfalse%
\ False\isanewline
\ \ \ \ \ \ \ \ \isacommand{using}\isamarkupfalse%
\ calculation\ true{\isacharunderscore}{\kern0pt}false{\isacharunderscore}{\kern0pt}distinct\ \isacommand{by}\isamarkupfalse%
\ auto\isanewline
\ \ \ \ \ \ \isacommand{then}\isamarkupfalse%
\ \isacommand{show}\isamarkupfalse%
\ {\isachardoublequoteopen}Q\ {\isasymcirc}\isactrlsub c\ y\ {\isacharequal}{\kern0pt}\ {\isacharparenleft}{\kern0pt}{\isasymt}\ {\isasymcirc}\isactrlsub c\ {\isasymbeta}\isactrlbsub Y\isactrlesub {\isacharparenright}{\kern0pt}\ {\isasymcirc}\isactrlsub c\ y{\isachardoublequoteclose}\isanewline
\ \ \ \ \ \ \ \ \isacommand{by}\isamarkupfalse%
\ simp\isanewline
\ \ \ \ \isacommand{next}\isamarkupfalse%
\isanewline
\ \ \ \ \ \ \isacommand{assume}\isamarkupfalse%
\ {\isachardoublequoteopen}z\ {\isasymcirc}\isactrlsub c\ {\isasymlangle}x{\isacharcomma}{\kern0pt}y{\isasymrangle}\ {\isacharequal}{\kern0pt}\ right{\isacharunderscore}{\kern0pt}coproj\ {\isasymone}\ {\isacharparenleft}{\kern0pt}{\isasymone}\ {\isasymCoprod}\ {\isasymone}{\isacharparenright}{\kern0pt}\ {\isasymcirc}\isactrlsub c\ right{\isacharunderscore}{\kern0pt}coproj\ {\isasymone}\ {\isasymone}{\isachardoublequoteclose}\isanewline
\ \ \ \ \ \ \isacommand{then}\isamarkupfalse%
\ \isacommand{have}\isamarkupfalse%
\ {\isachardoublequoteopen}{\isacharparenleft}{\kern0pt}P\ {\isasymtimes}\isactrlsub f\ Q{\isacharparenright}{\kern0pt}\ {\isasymcirc}\isactrlsub c\ {\isasymlangle}x{\isacharcomma}{\kern0pt}y{\isasymrangle}\ {\isacharequal}{\kern0pt}\ {\isacharparenleft}{\kern0pt}{\isasymlangle}{\isasymt}{\isacharcomma}{\kern0pt}{\isasymt}{\isasymrangle}\ {\isasymamalg}\ {\isasymlangle}{\isasymf}{\isacharcomma}{\kern0pt}{\isasymf}{\isasymrangle}\ {\isasymamalg}\ {\isasymlangle}{\isasymf}{\isacharcomma}{\kern0pt}{\isasymt}{\isasymrangle}{\isacharparenright}{\kern0pt}\ {\isasymcirc}\isactrlsub c\ right{\isacharunderscore}{\kern0pt}coproj\ {\isasymone}\ {\isacharparenleft}{\kern0pt}{\isasymone}\ {\isasymCoprod}\ {\isasymone}{\isacharparenright}{\kern0pt}\ {\isasymcirc}\isactrlsub c\ right{\isacharunderscore}{\kern0pt}coproj\ {\isasymone}\ {\isasymone}{\isachardoublequoteclose}\isanewline
\ \ \ \ \ \ \ \ \isacommand{by}\isamarkupfalse%
\ {\isacharparenleft}{\kern0pt}typecheck{\isacharunderscore}{\kern0pt}cfuncs{\isacharcomma}{\kern0pt}\ smt\ comp{\isacharunderscore}{\kern0pt}associative{\isadigit{2}}\ z{\isacharunderscore}{\kern0pt}eq\ z{\isacharunderscore}{\kern0pt}type{\isacharparenright}{\kern0pt}\isanewline
\ \ \ \ \ \ \isacommand{then}\isamarkupfalse%
\ \isacommand{have}\isamarkupfalse%
\ {\isachardoublequoteopen}{\isacharparenleft}{\kern0pt}P\ {\isasymtimes}\isactrlsub f\ Q{\isacharparenright}{\kern0pt}\ {\isasymcirc}\isactrlsub c\ {\isasymlangle}x{\isacharcomma}{\kern0pt}y{\isasymrangle}\ {\isacharequal}{\kern0pt}\ {\isacharparenleft}{\kern0pt}{\isasymlangle}{\isasymf}{\isacharcomma}{\kern0pt}{\isasymf}{\isasymrangle}\ {\isasymamalg}\ {\isasymlangle}{\isasymf}{\isacharcomma}{\kern0pt}{\isasymt}{\isasymrangle}{\isacharparenright}{\kern0pt}\ {\isasymcirc}\isactrlsub c\ right{\isacharunderscore}{\kern0pt}coproj\ {\isasymone}\ {\isasymone}{\isachardoublequoteclose}\isanewline
\ \ \ \ \ \ \ \ \isacommand{by}\isamarkupfalse%
\ {\isacharparenleft}{\kern0pt}typecheck{\isacharunderscore}{\kern0pt}cfuncs{\isacharunderscore}{\kern0pt}prems{\isacharcomma}{\kern0pt}\ smt\ right{\isacharunderscore}{\kern0pt}coproj{\isacharunderscore}{\kern0pt}cfunc{\isacharunderscore}{\kern0pt}coprod\ comp{\isacharunderscore}{\kern0pt}associative{\isadigit{2}}{\isacharparenright}{\kern0pt}\isanewline
\ \ \ \ \ \ \isacommand{then}\isamarkupfalse%
\ \isacommand{have}\isamarkupfalse%
\ {\isachardoublequoteopen}{\isacharparenleft}{\kern0pt}P\ {\isasymtimes}\isactrlsub f\ Q{\isacharparenright}{\kern0pt}\ {\isasymcirc}\isactrlsub c\ {\isasymlangle}x{\isacharcomma}{\kern0pt}y{\isasymrangle}\ {\isacharequal}{\kern0pt}\ {\isasymlangle}{\isasymf}{\isacharcomma}{\kern0pt}{\isasymt}{\isasymrangle}{\isachardoublequoteclose}\isanewline
\ \ \ \ \ \ \ \ \isacommand{by}\isamarkupfalse%
\ {\isacharparenleft}{\kern0pt}typecheck{\isacharunderscore}{\kern0pt}cfuncs{\isacharunderscore}{\kern0pt}prems{\isacharcomma}{\kern0pt}\ smt\ right{\isacharunderscore}{\kern0pt}coproj{\isacharunderscore}{\kern0pt}cfunc{\isacharunderscore}{\kern0pt}coprod{\isacharparenright}{\kern0pt}\isanewline
\ \ \ \ \ \ \isacommand{then}\isamarkupfalse%
\ \isacommand{have}\isamarkupfalse%
\ {\isachardoublequoteopen}Q\ {\isasymcirc}\isactrlsub c\ y\ {\isacharequal}{\kern0pt}\ {\isasymt}{\isachardoublequoteclose}\isanewline
\ \ \ \ \ \ \ \ \isacommand{by}\isamarkupfalse%
\ {\isacharparenleft}{\kern0pt}typecheck{\isacharunderscore}{\kern0pt}cfuncs{\isacharunderscore}{\kern0pt}prems{\isacharcomma}{\kern0pt}\ smt\ {\isacharparenleft}{\kern0pt}verit{\isacharcomma}{\kern0pt}\ best{\isacharparenright}{\kern0pt}\ cfunc{\isacharunderscore}{\kern0pt}cross{\isacharunderscore}{\kern0pt}prod{\isacharunderscore}{\kern0pt}comp{\isacharunderscore}{\kern0pt}cfunc{\isacharunderscore}{\kern0pt}prod\ comp{\isacharunderscore}{\kern0pt}associative{\isadigit{2}}\ comp{\isacharunderscore}{\kern0pt}type\ id{\isacharunderscore}{\kern0pt}right{\isacharunderscore}{\kern0pt}unit{\isadigit{2}}\ right{\isacharunderscore}{\kern0pt}cart{\isacharunderscore}{\kern0pt}proj{\isacharunderscore}{\kern0pt}cfunc{\isacharunderscore}{\kern0pt}prod{\isacharparenright}{\kern0pt}\isanewline
\ \ \ \ \ \ \isacommand{then}\isamarkupfalse%
\ \isacommand{show}\isamarkupfalse%
\ {\isachardoublequoteopen}Q\ {\isasymcirc}\isactrlsub c\ y\ {\isacharequal}{\kern0pt}\ {\isacharparenleft}{\kern0pt}{\isasymt}\ {\isasymcirc}\isactrlsub c\ {\isasymbeta}\isactrlbsub Y\isactrlesub {\isacharparenright}{\kern0pt}\ {\isasymcirc}\isactrlsub c\ y{\isachardoublequoteclose}\isanewline
\ \ \ \ \ \ \ \ \isacommand{by}\isamarkupfalse%
\ {\isacharparenleft}{\kern0pt}typecheck{\isacharunderscore}{\kern0pt}cfuncs{\isacharcomma}{\kern0pt}\ smt\ {\isacharparenleft}{\kern0pt}z{\isadigit{3}}{\isacharparenright}{\kern0pt}\ comp{\isacharunderscore}{\kern0pt}associative{\isadigit{2}}\ id{\isacharunderscore}{\kern0pt}right{\isacharunderscore}{\kern0pt}unit{\isadigit{2}}\ id{\isacharunderscore}{\kern0pt}type\ one{\isacharunderscore}{\kern0pt}unique{\isacharunderscore}{\kern0pt}element\ terminal{\isacharunderscore}{\kern0pt}func{\isacharunderscore}{\kern0pt}comp\ terminal{\isacharunderscore}{\kern0pt}func{\isacharunderscore}{\kern0pt}type{\isacharparenright}{\kern0pt}\isanewline
\ \ \ \ \isacommand{qed}\isamarkupfalse%
\isanewline
\ \ \isacommand{qed}\isamarkupfalse%
\isanewline
\isacommand{qed}\isamarkupfalse%
%
\endisatagproof
{\isafoldproof}%
%
\isadelimproof
\isanewline
%
\endisadelimproof
\isanewline
\isacommand{lemma}\isamarkupfalse%
\ IMPLIES{\isacharunderscore}{\kern0pt}elim{\isacharcolon}{\kern0pt}\isanewline
\ \ \isakeyword{assumes}\ IMPLIES{\isacharunderscore}{\kern0pt}true{\isacharcolon}{\kern0pt}\ {\isachardoublequoteopen}IMPLIES\ {\isasymcirc}\isactrlsub c\ {\isacharparenleft}{\kern0pt}P\ {\isasymtimes}\isactrlsub f\ Q{\isacharparenright}{\kern0pt}\ {\isacharequal}{\kern0pt}\ {\isasymt}\ {\isasymcirc}\isactrlsub c\ {\isasymbeta}\isactrlbsub X\ {\isasymtimes}\isactrlsub c\ Y\isactrlesub {\isachardoublequoteclose}\isanewline
\ \ \isakeyword{assumes}\ P{\isacharunderscore}{\kern0pt}type{\isacharbrackleft}{\kern0pt}type{\isacharunderscore}{\kern0pt}rule{\isacharbrackright}{\kern0pt}{\isacharcolon}{\kern0pt}\ {\isachardoublequoteopen}P\ {\isacharcolon}{\kern0pt}\ X\ {\isasymrightarrow}\ {\isasymOmega}{\isachardoublequoteclose}\ \isakeyword{and}\ Q{\isacharunderscore}{\kern0pt}type{\isacharbrackleft}{\kern0pt}type{\isacharunderscore}{\kern0pt}rule{\isacharbrackright}{\kern0pt}{\isacharcolon}{\kern0pt}\ {\isachardoublequoteopen}Q\ {\isacharcolon}{\kern0pt}\ Y\ {\isasymrightarrow}\ {\isasymOmega}{\isachardoublequoteclose}\isanewline
\ \ \isakeyword{assumes}\ X{\isacharunderscore}{\kern0pt}nonempty{\isacharcolon}{\kern0pt}\ {\isachardoublequoteopen}{\isasymexists}x{\isachardot}{\kern0pt}\ x\ {\isasymin}\isactrlsub c\ X{\isachardoublequoteclose}\isanewline
\ \ \isakeyword{shows}\ {\isachardoublequoteopen}{\isacharparenleft}{\kern0pt}P\ {\isacharequal}{\kern0pt}\ {\isasymt}\ {\isasymcirc}\isactrlsub c\ {\isasymbeta}\isactrlbsub X\isactrlesub {\isacharparenright}{\kern0pt}\ {\isasymLongrightarrow}\ {\isacharparenleft}{\kern0pt}{\isacharparenleft}{\kern0pt}Q\ {\isacharequal}{\kern0pt}\ {\isasymt}\ {\isasymcirc}\isactrlsub c\ {\isasymbeta}\isactrlbsub Y\isactrlesub {\isacharparenright}{\kern0pt}\ {\isasymLongrightarrow}\ R{\isacharparenright}{\kern0pt}\ {\isasymLongrightarrow}\ R{\isachardoublequoteclose}\isanewline
%
\isadelimproof
\ \ %
\endisadelimproof
%
\isatagproof
\isacommand{using}\isamarkupfalse%
\ IMPLIES{\isacharunderscore}{\kern0pt}implies{\isacharunderscore}{\kern0pt}implies\ assms\ \isacommand{by}\isamarkupfalse%
\ blast%
\endisatagproof
{\isafoldproof}%
%
\isadelimproof
\isanewline
%
\endisadelimproof
\isanewline
\isacommand{lemma}\isamarkupfalse%
\ IMPLIES{\isacharunderscore}{\kern0pt}elim{\isacharprime}{\kern0pt}{\isacharprime}{\kern0pt}{\isacharcolon}{\kern0pt}\isanewline
\ \ \isakeyword{assumes}\ IMPLIES{\isacharunderscore}{\kern0pt}true{\isacharcolon}{\kern0pt}\ {\isachardoublequoteopen}IMPLIES\ {\isasymcirc}\isactrlsub c\ {\isacharparenleft}{\kern0pt}P\ {\isasymtimes}\isactrlsub f\ Q{\isacharparenright}{\kern0pt}\ {\isacharequal}{\kern0pt}\ {\isasymt}{\isachardoublequoteclose}\isanewline
\ \ \isakeyword{assumes}\ P{\isacharunderscore}{\kern0pt}type{\isacharbrackleft}{\kern0pt}type{\isacharunderscore}{\kern0pt}rule{\isacharbrackright}{\kern0pt}{\isacharcolon}{\kern0pt}\ {\isachardoublequoteopen}P\ {\isacharcolon}{\kern0pt}\ {\isasymone}\ {\isasymrightarrow}\ {\isasymOmega}{\isachardoublequoteclose}\ \isakeyword{and}\ Q{\isacharunderscore}{\kern0pt}type{\isacharbrackleft}{\kern0pt}type{\isacharunderscore}{\kern0pt}rule{\isacharbrackright}{\kern0pt}{\isacharcolon}{\kern0pt}\ {\isachardoublequoteopen}Q\ {\isacharcolon}{\kern0pt}\ {\isasymone}\ {\isasymrightarrow}\ {\isasymOmega}{\isachardoublequoteclose}\isanewline
\ \ \isakeyword{shows}\ {\isachardoublequoteopen}{\isacharparenleft}{\kern0pt}P\ {\isacharequal}{\kern0pt}\ {\isasymt}{\isacharparenright}{\kern0pt}\ {\isasymLongrightarrow}\ {\isacharparenleft}{\kern0pt}{\isacharparenleft}{\kern0pt}Q\ {\isacharequal}{\kern0pt}\ {\isasymt}{\isacharparenright}{\kern0pt}\ {\isasymLongrightarrow}\ R{\isacharparenright}{\kern0pt}\ {\isasymLongrightarrow}\ R{\isachardoublequoteclose}\isanewline
%
\isadelimproof
%
\endisadelimproof
%
\isatagproof
\isacommand{proof}\isamarkupfalse%
\ {\isacharminus}{\kern0pt}\isanewline
\ \ \isacommand{have}\isamarkupfalse%
\ one{\isacharunderscore}{\kern0pt}nonempty{\isacharcolon}{\kern0pt}\ {\isachardoublequoteopen}{\isasymexists}x{\isachardot}{\kern0pt}\ x\ {\isasymin}\isactrlsub c\ {\isasymone}{\isachardoublequoteclose}\isanewline
\ \ \ \ \isacommand{using}\isamarkupfalse%
\ one{\isacharunderscore}{\kern0pt}unique{\isacharunderscore}{\kern0pt}element\ \isacommand{by}\isamarkupfalse%
\ blast\isanewline
\ \ \isacommand{have}\isamarkupfalse%
\ {\isachardoublequoteopen}{\isacharparenleft}{\kern0pt}IMPLIES\ {\isasymcirc}\isactrlsub c\ {\isacharparenleft}{\kern0pt}P\ {\isasymtimes}\isactrlsub f\ Q{\isacharparenright}{\kern0pt}\ {\isacharequal}{\kern0pt}\ {\isasymt}\ {\isasymcirc}\isactrlsub c\ {\isasymbeta}\isactrlbsub {\isasymone}\ {\isasymtimes}\isactrlsub c\ {\isasymone}\isactrlesub {\isacharparenright}{\kern0pt}{\isachardoublequoteclose}\isanewline
\ \ \ \ \isacommand{by}\isamarkupfalse%
\ {\isacharparenleft}{\kern0pt}typecheck{\isacharunderscore}{\kern0pt}cfuncs{\isacharcomma}{\kern0pt}\ metis\ IMPLIES{\isacharunderscore}{\kern0pt}true\ id{\isacharunderscore}{\kern0pt}right{\isacharunderscore}{\kern0pt}unit{\isadigit{2}}\ id{\isacharunderscore}{\kern0pt}type\ one{\isacharunderscore}{\kern0pt}unique{\isacharunderscore}{\kern0pt}element\ terminal{\isacharunderscore}{\kern0pt}func{\isacharunderscore}{\kern0pt}comp\ terminal{\isacharunderscore}{\kern0pt}func{\isacharunderscore}{\kern0pt}type{\isacharparenright}{\kern0pt}\isanewline
\ \ \isacommand{then}\isamarkupfalse%
\ \isacommand{have}\isamarkupfalse%
\ {\isachardoublequoteopen}{\isacharparenleft}{\kern0pt}P\ {\isacharequal}{\kern0pt}\ {\isasymt}\ {\isasymcirc}\isactrlsub c\ {\isasymbeta}\isactrlbsub {\isasymone}\isactrlesub {\isacharparenright}{\kern0pt}\ {\isasymLongrightarrow}\ {\isacharparenleft}{\kern0pt}{\isacharparenleft}{\kern0pt}Q\ {\isacharequal}{\kern0pt}\ {\isasymt}\ {\isasymcirc}\isactrlsub c\ {\isasymbeta}\isactrlbsub {\isasymone}\isactrlesub {\isacharparenright}{\kern0pt}\ {\isasymLongrightarrow}\ R{\isacharparenright}{\kern0pt}\ {\isasymLongrightarrow}\ R{\isachardoublequoteclose}\isanewline
\ \ \ \ \isacommand{using}\isamarkupfalse%
\ one{\isacharunderscore}{\kern0pt}nonempty\ \isacommand{by}\isamarkupfalse%
\ {\isacharparenleft}{\kern0pt}{\isacharminus}{\kern0pt}{\isacharcomma}{\kern0pt}\ etcs{\isacharunderscore}{\kern0pt}erule\ IMPLIES{\isacharunderscore}{\kern0pt}elim{\isacharcomma}{\kern0pt}\ auto{\isacharparenright}{\kern0pt}\isanewline
\ \ \isacommand{then}\isamarkupfalse%
\ \isacommand{show}\isamarkupfalse%
\ {\isachardoublequoteopen}{\isacharparenleft}{\kern0pt}P\ {\isacharequal}{\kern0pt}\ {\isasymt}{\isacharparenright}{\kern0pt}\ {\isasymLongrightarrow}\ {\isacharparenleft}{\kern0pt}{\isacharparenleft}{\kern0pt}Q\ {\isacharequal}{\kern0pt}\ {\isasymt}{\isacharparenright}{\kern0pt}\ {\isasymLongrightarrow}\ R{\isacharparenright}{\kern0pt}\ {\isasymLongrightarrow}\ R{\isachardoublequoteclose}\isanewline
\ \ \ \ \isacommand{by}\isamarkupfalse%
\ {\isacharparenleft}{\kern0pt}typecheck{\isacharunderscore}{\kern0pt}cfuncs{\isacharcomma}{\kern0pt}\ metis\ id{\isacharunderscore}{\kern0pt}right{\isacharunderscore}{\kern0pt}unit{\isadigit{2}}\ id{\isacharunderscore}{\kern0pt}type\ one{\isacharunderscore}{\kern0pt}unique{\isacharunderscore}{\kern0pt}element\ terminal{\isacharunderscore}{\kern0pt}func{\isacharunderscore}{\kern0pt}type{\isacharparenright}{\kern0pt}\isanewline
\isacommand{qed}\isamarkupfalse%
%
\endisatagproof
{\isafoldproof}%
%
\isadelimproof
\isanewline
%
\endisadelimproof
\isanewline
\isacommand{lemma}\isamarkupfalse%
\ IMPLIES{\isacharunderscore}{\kern0pt}elim{\isacharprime}{\kern0pt}{\isacharcolon}{\kern0pt}\isanewline
\ \ \isakeyword{assumes}\ IMPLIES{\isacharunderscore}{\kern0pt}true{\isacharcolon}{\kern0pt}\ {\isachardoublequoteopen}IMPLIES\ {\isasymcirc}\isactrlsub c\ {\isasymlangle}P{\isacharcomma}{\kern0pt}\ Q{\isasymrangle}\ {\isacharequal}{\kern0pt}\ {\isasymt}{\isachardoublequoteclose}\isanewline
\ \ \isakeyword{assumes}\ P{\isacharunderscore}{\kern0pt}type{\isacharbrackleft}{\kern0pt}type{\isacharunderscore}{\kern0pt}rule{\isacharbrackright}{\kern0pt}{\isacharcolon}{\kern0pt}\ {\isachardoublequoteopen}P\ {\isacharcolon}{\kern0pt}\ {\isasymone}\ {\isasymrightarrow}\ {\isasymOmega}{\isachardoublequoteclose}\ \isakeyword{and}\ Q{\isacharunderscore}{\kern0pt}type{\isacharbrackleft}{\kern0pt}type{\isacharunderscore}{\kern0pt}rule{\isacharbrackright}{\kern0pt}{\isacharcolon}{\kern0pt}\ {\isachardoublequoteopen}Q\ {\isacharcolon}{\kern0pt}\ {\isasymone}\ {\isasymrightarrow}\ {\isasymOmega}{\isachardoublequoteclose}\isanewline
\ \ \isakeyword{shows}\ {\isachardoublequoteopen}{\isacharparenleft}{\kern0pt}P\ {\isacharequal}{\kern0pt}\ {\isasymt}{\isacharparenright}{\kern0pt}\ {\isasymLongrightarrow}\ {\isacharparenleft}{\kern0pt}{\isacharparenleft}{\kern0pt}Q\ {\isacharequal}{\kern0pt}\ {\isasymt}{\isacharparenright}{\kern0pt}\ {\isasymLongrightarrow}\ R{\isacharparenright}{\kern0pt}\ {\isasymLongrightarrow}\ R{\isachardoublequoteclose}\isanewline
%
\isadelimproof
\ \ %
\endisadelimproof
%
\isatagproof
\isacommand{using}\isamarkupfalse%
\ IMPLIES{\isacharunderscore}{\kern0pt}true\ IMPLIES{\isacharunderscore}{\kern0pt}true{\isacharunderscore}{\kern0pt}false{\isacharunderscore}{\kern0pt}is{\isacharunderscore}{\kern0pt}false\ Q{\isacharunderscore}{\kern0pt}type\ true{\isacharunderscore}{\kern0pt}false{\isacharunderscore}{\kern0pt}only{\isacharunderscore}{\kern0pt}truth{\isacharunderscore}{\kern0pt}values\ \isacommand{by}\isamarkupfalse%
\ force%
\endisatagproof
{\isafoldproof}%
%
\isadelimproof
\isanewline
%
\endisadelimproof
\isanewline
\isacommand{lemma}\isamarkupfalse%
\ implies{\isacharunderscore}{\kern0pt}implies{\isacharunderscore}{\kern0pt}IMPLIES{\isacharcolon}{\kern0pt}\isanewline
\ \ \isakeyword{assumes}\ P{\isacharunderscore}{\kern0pt}type{\isacharbrackleft}{\kern0pt}type{\isacharunderscore}{\kern0pt}rule{\isacharbrackright}{\kern0pt}{\isacharcolon}{\kern0pt}\ {\isachardoublequoteopen}P\ {\isacharcolon}{\kern0pt}\ {\isasymone}\ {\isasymrightarrow}\ {\isasymOmega}{\isachardoublequoteclose}\ \isakeyword{and}\ Q{\isacharunderscore}{\kern0pt}type{\isacharbrackleft}{\kern0pt}type{\isacharunderscore}{\kern0pt}rule{\isacharbrackright}{\kern0pt}{\isacharcolon}{\kern0pt}\ {\isachardoublequoteopen}Q\ {\isacharcolon}{\kern0pt}\ {\isasymone}\ {\isasymrightarrow}\ {\isasymOmega}{\isachardoublequoteclose}\isanewline
\ \ \isakeyword{shows}\ \ {\isachardoublequoteopen}{\isacharparenleft}{\kern0pt}P\ {\isacharequal}{\kern0pt}\ {\isasymt}\ {\isasymLongrightarrow}\ Q\ {\isacharequal}{\kern0pt}\ {\isasymt}{\isacharparenright}{\kern0pt}\ {\isasymLongrightarrow}\ IMPLIES\ {\isasymcirc}\isactrlsub c\ {\isasymlangle}P{\isacharcomma}{\kern0pt}\ Q{\isasymrangle}\ {\isacharequal}{\kern0pt}\ {\isasymt}{\isachardoublequoteclose}\isanewline
%
\isadelimproof
\ \ %
\endisadelimproof
%
\isatagproof
\isacommand{by}\isamarkupfalse%
\ {\isacharparenleft}{\kern0pt}typecheck{\isacharunderscore}{\kern0pt}cfuncs{\isacharcomma}{\kern0pt}\ metis\ IMPLIES{\isacharunderscore}{\kern0pt}false{\isacharunderscore}{\kern0pt}is{\isacharunderscore}{\kern0pt}true{\isacharunderscore}{\kern0pt}false\ true{\isacharunderscore}{\kern0pt}false{\isacharunderscore}{\kern0pt}only{\isacharunderscore}{\kern0pt}truth{\isacharunderscore}{\kern0pt}values{\isacharparenright}{\kern0pt}%
\endisatagproof
{\isafoldproof}%
%
\isadelimproof
%
\endisadelimproof
%
\isadelimdocument
%
\endisadelimdocument
%
\isatagdocument
%
\isamarkupsubsection{Other Boolean Identities%
}
\isamarkuptrue%
%
\endisatagdocument
{\isafolddocument}%
%
\isadelimdocument
%
\endisadelimdocument
\isacommand{lemma}\isamarkupfalse%
\ AND{\isacharunderscore}{\kern0pt}OR{\isacharunderscore}{\kern0pt}distributive{\isacharcolon}{\kern0pt}\isanewline
\ \ \isakeyword{assumes}\ {\isachardoublequoteopen}p\ {\isasymin}\isactrlsub c\ {\isasymOmega}{\isachardoublequoteclose}\isanewline
\ \ \isakeyword{assumes}\ {\isachardoublequoteopen}q\ {\isasymin}\isactrlsub c\ {\isasymOmega}{\isachardoublequoteclose}\isanewline
\ \ \isakeyword{assumes}\ {\isachardoublequoteopen}r\ {\isasymin}\isactrlsub c\ {\isasymOmega}{\isachardoublequoteclose}\isanewline
\ \ \isakeyword{shows}\ {\isachardoublequoteopen}AND\ {\isasymcirc}\isactrlsub c\ {\isasymlangle}p{\isacharcomma}{\kern0pt}\ OR\ {\isasymcirc}\isactrlsub c\ {\isasymlangle}q{\isacharcomma}{\kern0pt}r{\isasymrangle}{\isasymrangle}\ {\isacharequal}{\kern0pt}\ OR\ {\isasymcirc}\isactrlsub c\ {\isasymlangle}AND\ {\isasymcirc}\isactrlsub c\ {\isasymlangle}p{\isacharcomma}{\kern0pt}q{\isasymrangle}{\isacharcomma}{\kern0pt}\ AND\ {\isasymcirc}\isactrlsub c\ {\isasymlangle}p{\isacharcomma}{\kern0pt}r{\isasymrangle}{\isasymrangle}{\isachardoublequoteclose}\isanewline
%
\isadelimproof
\ \ %
\endisadelimproof
%
\isatagproof
\isacommand{by}\isamarkupfalse%
\ {\isacharparenleft}{\kern0pt}metis\ AND{\isacharunderscore}{\kern0pt}commutative\ AND{\isacharunderscore}{\kern0pt}false{\isacharunderscore}{\kern0pt}right{\isacharunderscore}{\kern0pt}is{\isacharunderscore}{\kern0pt}false\ AND{\isacharunderscore}{\kern0pt}true{\isacharunderscore}{\kern0pt}true{\isacharunderscore}{\kern0pt}is{\isacharunderscore}{\kern0pt}true\ OR{\isacharunderscore}{\kern0pt}false{\isacharunderscore}{\kern0pt}false{\isacharunderscore}{\kern0pt}is{\isacharunderscore}{\kern0pt}false\ OR{\isacharunderscore}{\kern0pt}true{\isacharunderscore}{\kern0pt}left{\isacharunderscore}{\kern0pt}is{\isacharunderscore}{\kern0pt}true\ OR{\isacharunderscore}{\kern0pt}true{\isacharunderscore}{\kern0pt}right{\isacharunderscore}{\kern0pt}is{\isacharunderscore}{\kern0pt}true\ assms\ true{\isacharunderscore}{\kern0pt}false{\isacharunderscore}{\kern0pt}only{\isacharunderscore}{\kern0pt}truth{\isacharunderscore}{\kern0pt}values{\isacharparenright}{\kern0pt}%
\endisatagproof
{\isafoldproof}%
%
\isadelimproof
\isanewline
%
\endisadelimproof
\isanewline
\isacommand{lemma}\isamarkupfalse%
\ OR{\isacharunderscore}{\kern0pt}AND{\isacharunderscore}{\kern0pt}distributive{\isacharcolon}{\kern0pt}\isanewline
\ \ \isakeyword{assumes}\ {\isachardoublequoteopen}p\ {\isasymin}\isactrlsub c\ {\isasymOmega}{\isachardoublequoteclose}\isanewline
\ \ \isakeyword{assumes}\ {\isachardoublequoteopen}q\ {\isasymin}\isactrlsub c\ {\isasymOmega}{\isachardoublequoteclose}\isanewline
\ \ \isakeyword{assumes}\ {\isachardoublequoteopen}r\ {\isasymin}\isactrlsub c\ {\isasymOmega}{\isachardoublequoteclose}\isanewline
\ \ \isakeyword{shows}\ {\isachardoublequoteopen}OR\ {\isasymcirc}\isactrlsub c\ {\isasymlangle}p{\isacharcomma}{\kern0pt}\ AND\ {\isasymcirc}\isactrlsub c\ {\isasymlangle}q{\isacharcomma}{\kern0pt}r{\isasymrangle}{\isasymrangle}\ {\isacharequal}{\kern0pt}\ AND\ {\isasymcirc}\isactrlsub c\ {\isasymlangle}OR\ {\isasymcirc}\isactrlsub c\ {\isasymlangle}p{\isacharcomma}{\kern0pt}q{\isasymrangle}{\isacharcomma}{\kern0pt}\ OR\ {\isasymcirc}\isactrlsub c\ {\isasymlangle}p{\isacharcomma}{\kern0pt}r{\isasymrangle}{\isasymrangle}{\isachardoublequoteclose}\isanewline
%
\isadelimproof
\ \ %
\endisadelimproof
%
\isatagproof
\isacommand{by}\isamarkupfalse%
\ {\isacharparenleft}{\kern0pt}smt\ {\isacharparenleft}{\kern0pt}z{\isadigit{3}}{\isacharparenright}{\kern0pt}\ AND{\isacharunderscore}{\kern0pt}commutative\ AND{\isacharunderscore}{\kern0pt}false{\isacharunderscore}{\kern0pt}right{\isacharunderscore}{\kern0pt}is{\isacharunderscore}{\kern0pt}false\ AND{\isacharunderscore}{\kern0pt}true{\isacharunderscore}{\kern0pt}true{\isacharunderscore}{\kern0pt}is{\isacharunderscore}{\kern0pt}true\ OR{\isacharunderscore}{\kern0pt}commutative\ OR{\isacharunderscore}{\kern0pt}false{\isacharunderscore}{\kern0pt}false{\isacharunderscore}{\kern0pt}is{\isacharunderscore}{\kern0pt}false\ OR{\isacharunderscore}{\kern0pt}true{\isacharunderscore}{\kern0pt}right{\isacharunderscore}{\kern0pt}is{\isacharunderscore}{\kern0pt}true\ assms\ true{\isacharunderscore}{\kern0pt}false{\isacharunderscore}{\kern0pt}only{\isacharunderscore}{\kern0pt}truth{\isacharunderscore}{\kern0pt}values{\isacharparenright}{\kern0pt}%
\endisatagproof
{\isafoldproof}%
%
\isadelimproof
\isanewline
%
\endisadelimproof
\isanewline
\isacommand{lemma}\isamarkupfalse%
\ OR{\isacharunderscore}{\kern0pt}AND{\isacharunderscore}{\kern0pt}absorption{\isacharcolon}{\kern0pt}\isanewline
\ \ \isakeyword{assumes}\ {\isachardoublequoteopen}p\ {\isasymin}\isactrlsub c\ {\isasymOmega}{\isachardoublequoteclose}\isanewline
\ \ \isakeyword{assumes}\ {\isachardoublequoteopen}q\ {\isasymin}\isactrlsub c\ {\isasymOmega}{\isachardoublequoteclose}\isanewline
\ \ \isakeyword{shows}\ {\isachardoublequoteopen}OR\ {\isasymcirc}\isactrlsub c\ {\isasymlangle}p{\isacharcomma}{\kern0pt}\ AND\ {\isasymcirc}\isactrlsub c\ {\isasymlangle}p{\isacharcomma}{\kern0pt}q{\isasymrangle}{\isasymrangle}\ {\isacharequal}{\kern0pt}\ p{\isachardoublequoteclose}\isanewline
%
\isadelimproof
\ \ %
\endisadelimproof
%
\isatagproof
\isacommand{by}\isamarkupfalse%
\ {\isacharparenleft}{\kern0pt}metis\ AND{\isacharunderscore}{\kern0pt}commutative\ AND{\isacharunderscore}{\kern0pt}complementary\ AND{\isacharunderscore}{\kern0pt}idempotent\ NOT{\isacharunderscore}{\kern0pt}true{\isacharunderscore}{\kern0pt}is{\isacharunderscore}{\kern0pt}false\ OR{\isacharunderscore}{\kern0pt}false{\isacharunderscore}{\kern0pt}false{\isacharunderscore}{\kern0pt}is{\isacharunderscore}{\kern0pt}false\ OR{\isacharunderscore}{\kern0pt}true{\isacharunderscore}{\kern0pt}left{\isacharunderscore}{\kern0pt}is{\isacharunderscore}{\kern0pt}true\ assms\ true{\isacharunderscore}{\kern0pt}false{\isacharunderscore}{\kern0pt}only{\isacharunderscore}{\kern0pt}truth{\isacharunderscore}{\kern0pt}values{\isacharparenright}{\kern0pt}%
\endisatagproof
{\isafoldproof}%
%
\isadelimproof
\isanewline
%
\endisadelimproof
\isanewline
\isacommand{lemma}\isamarkupfalse%
\ AND{\isacharunderscore}{\kern0pt}OR{\isacharunderscore}{\kern0pt}absorption{\isacharcolon}{\kern0pt}\isanewline
\ \ \isakeyword{assumes}\ {\isachardoublequoteopen}p\ {\isasymin}\isactrlsub c\ {\isasymOmega}{\isachardoublequoteclose}\isanewline
\ \ \isakeyword{assumes}\ {\isachardoublequoteopen}q\ {\isasymin}\isactrlsub c\ {\isasymOmega}{\isachardoublequoteclose}\isanewline
\ \ \isakeyword{shows}\ {\isachardoublequoteopen}AND\ {\isasymcirc}\isactrlsub c\ {\isasymlangle}p{\isacharcomma}{\kern0pt}\ OR\ {\isasymcirc}\isactrlsub c\ {\isasymlangle}p{\isacharcomma}{\kern0pt}q{\isasymrangle}{\isasymrangle}\ {\isacharequal}{\kern0pt}\ p{\isachardoublequoteclose}\isanewline
%
\isadelimproof
\ \ %
\endisadelimproof
%
\isatagproof
\isacommand{by}\isamarkupfalse%
\ {\isacharparenleft}{\kern0pt}metis\ AND{\isacharunderscore}{\kern0pt}commutative\ AND{\isacharunderscore}{\kern0pt}complementary\ AND{\isacharunderscore}{\kern0pt}idempotent\ NOT{\isacharunderscore}{\kern0pt}true{\isacharunderscore}{\kern0pt}is{\isacharunderscore}{\kern0pt}false\ OR{\isacharunderscore}{\kern0pt}AND{\isacharunderscore}{\kern0pt}absorption\ OR{\isacharunderscore}{\kern0pt}commutative\ assms\ true{\isacharunderscore}{\kern0pt}false{\isacharunderscore}{\kern0pt}only{\isacharunderscore}{\kern0pt}truth{\isacharunderscore}{\kern0pt}values{\isacharparenright}{\kern0pt}%
\endisatagproof
{\isafoldproof}%
%
\isadelimproof
\isanewline
%
\endisadelimproof
\isanewline
\isacommand{lemma}\isamarkupfalse%
\ deMorgan{\isacharunderscore}{\kern0pt}Law{\isadigit{1}}{\isacharcolon}{\kern0pt}\isanewline
\ \ \isakeyword{assumes}\ {\isachardoublequoteopen}p\ {\isasymin}\isactrlsub c\ {\isasymOmega}{\isachardoublequoteclose}\isanewline
\ \ \isakeyword{assumes}\ {\isachardoublequoteopen}q\ {\isasymin}\isactrlsub c\ {\isasymOmega}{\isachardoublequoteclose}\isanewline
\ \ \isakeyword{shows}\ {\isachardoublequoteopen}NOT\ {\isasymcirc}\isactrlsub c\ OR\ {\isasymcirc}\isactrlsub c\ {\isasymlangle}p{\isacharcomma}{\kern0pt}q{\isasymrangle}\ {\isacharequal}{\kern0pt}\ AND\ {\isasymcirc}\isactrlsub c\ {\isasymlangle}NOT\ {\isasymcirc}\isactrlsub c\ p{\isacharcomma}{\kern0pt}\ NOT\ {\isasymcirc}\isactrlsub c\ q{\isasymrangle}{\isachardoublequoteclose}\isanewline
%
\isadelimproof
\ \ %
\endisadelimproof
%
\isatagproof
\isacommand{by}\isamarkupfalse%
\ {\isacharparenleft}{\kern0pt}metis\ AND{\isacharunderscore}{\kern0pt}OR{\isacharunderscore}{\kern0pt}absorption\ AND{\isacharunderscore}{\kern0pt}complementary\ AND{\isacharunderscore}{\kern0pt}true{\isacharunderscore}{\kern0pt}true{\isacharunderscore}{\kern0pt}is{\isacharunderscore}{\kern0pt}true\ NOT{\isacharunderscore}{\kern0pt}false{\isacharunderscore}{\kern0pt}is{\isacharunderscore}{\kern0pt}true\ NOT{\isacharunderscore}{\kern0pt}true{\isacharunderscore}{\kern0pt}is{\isacharunderscore}{\kern0pt}false\ OR{\isacharunderscore}{\kern0pt}AND{\isacharunderscore}{\kern0pt}absorption\ OR{\isacharunderscore}{\kern0pt}commutative\ OR{\isacharunderscore}{\kern0pt}idempotent\ assms\ false{\isacharunderscore}{\kern0pt}func{\isacharunderscore}{\kern0pt}type\ true{\isacharunderscore}{\kern0pt}false{\isacharunderscore}{\kern0pt}only{\isacharunderscore}{\kern0pt}truth{\isacharunderscore}{\kern0pt}values{\isacharparenright}{\kern0pt}%
\endisatagproof
{\isafoldproof}%
%
\isadelimproof
\isanewline
%
\endisadelimproof
\isanewline
\isacommand{lemma}\isamarkupfalse%
\ deMorgan{\isacharunderscore}{\kern0pt}Law{\isadigit{2}}{\isacharcolon}{\kern0pt}\isanewline
\ \ \isakeyword{assumes}\ {\isachardoublequoteopen}p\ {\isasymin}\isactrlsub c\ {\isasymOmega}{\isachardoublequoteclose}\isanewline
\ \ \isakeyword{assumes}\ {\isachardoublequoteopen}q\ {\isasymin}\isactrlsub c\ {\isasymOmega}{\isachardoublequoteclose}\isanewline
\ \ \isakeyword{shows}\ {\isachardoublequoteopen}NOT\ {\isasymcirc}\isactrlsub c\ AND\ {\isasymcirc}\isactrlsub c\ {\isasymlangle}p{\isacharcomma}{\kern0pt}q{\isasymrangle}\ {\isacharequal}{\kern0pt}\ OR\ {\isasymcirc}\isactrlsub c\ {\isasymlangle}NOT\ {\isasymcirc}\isactrlsub c\ p{\isacharcomma}{\kern0pt}\ NOT\ {\isasymcirc}\isactrlsub c\ q{\isasymrangle}{\isachardoublequoteclose}\isanewline
%
\isadelimproof
\ \ %
\endisadelimproof
%
\isatagproof
\isacommand{by}\isamarkupfalse%
\ {\isacharparenleft}{\kern0pt}metis\ AND{\isacharunderscore}{\kern0pt}complementary\ AND{\isacharunderscore}{\kern0pt}idempotent\ NOT{\isacharunderscore}{\kern0pt}false{\isacharunderscore}{\kern0pt}is{\isacharunderscore}{\kern0pt}true\ NOT{\isacharunderscore}{\kern0pt}true{\isacharunderscore}{\kern0pt}is{\isacharunderscore}{\kern0pt}false\ OR{\isacharunderscore}{\kern0pt}complementary\ OR{\isacharunderscore}{\kern0pt}false{\isacharunderscore}{\kern0pt}false{\isacharunderscore}{\kern0pt}is{\isacharunderscore}{\kern0pt}false\ OR{\isacharunderscore}{\kern0pt}idempotent\ assms\ true{\isacharunderscore}{\kern0pt}false{\isacharunderscore}{\kern0pt}only{\isacharunderscore}{\kern0pt}truth{\isacharunderscore}{\kern0pt}values\ true{\isacharunderscore}{\kern0pt}func{\isacharunderscore}{\kern0pt}type{\isacharparenright}{\kern0pt}%
\endisatagproof
{\isafoldproof}%
%
\isadelimproof
\isanewline
%
\endisadelimproof
%
\isadelimtheory
\ \isanewline
%
\endisadelimtheory
%
\isatagtheory
\isacommand{end}\isamarkupfalse%
%
\endisatagtheory
{\isafoldtheory}%
%
\isadelimtheory
%
\endisadelimtheory
%
\end{isabellebody}%
\endinput
%:%file=~/ETCS/Category_Set/Pred_Logic.thy%:%
%:%11=1%:%
%:%27=3%:%
%:%28=3%:%
%:%29=4%:%
%:%30=5%:%
%:%44=7%:%
%:%54=9%:%
%:%55=9%:%
%:%56=10%:%
%:%57=11%:%
%:%58=12%:%
%:%59=12%:%
%:%60=13%:%
%:%63=14%:%
%:%67=14%:%
%:%68=14%:%
%:%69=15%:%
%:%70=15%:%
%:%71=16%:%
%:%72=16%:%
%:%77=16%:%
%:%80=17%:%
%:%81=18%:%
%:%82=18%:%
%:%83=19%:%
%:%86=20%:%
%:%90=20%:%
%:%91=20%:%
%:%92=20%:%
%:%93=20%:%
%:%98=20%:%
%:%101=21%:%
%:%102=22%:%
%:%103=22%:%
%:%104=23%:%
%:%107=24%:%
%:%111=24%:%
%:%112=24%:%
%:%113=24%:%
%:%114=25%:%
%:%115=25%:%
%:%120=25%:%
%:%123=26%:%
%:%124=27%:%
%:%125=27%:%
%:%126=28%:%
%:%133=29%:%
%:%134=29%:%
%:%135=30%:%
%:%136=30%:%
%:%137=31%:%
%:%138=31%:%
%:%139=31%:%
%:%140=32%:%
%:%141=32%:%
%:%142=32%:%
%:%143=33%:%
%:%144=33%:%
%:%145=33%:%
%:%146=34%:%
%:%147=34%:%
%:%148=34%:%
%:%149=35%:%
%:%150=35%:%
%:%151=35%:%
%:%152=36%:%
%:%153=36%:%
%:%154=36%:%
%:%155=36%:%
%:%156=37%:%
%:%157=37%:%
%:%158=37%:%
%:%159=38%:%
%:%160=38%:%
%:%161=38%:%
%:%162=39%:%
%:%163=39%:%
%:%164=39%:%
%:%165=40%:%
%:%166=40%:%
%:%167=40%:%
%:%168=41%:%
%:%169=41%:%
%:%170=41%:%
%:%171=42%:%
%:%172=42%:%
%:%173=42%:%
%:%174=43%:%
%:%180=43%:%
%:%183=44%:%
%:%184=45%:%
%:%185=45%:%
%:%186=46%:%
%:%187=47%:%
%:%190=48%:%
%:%194=48%:%
%:%195=48%:%
%:%196=48%:%
%:%201=48%:%
%:%204=49%:%
%:%205=50%:%
%:%206=50%:%
%:%207=51%:%
%:%208=52%:%
%:%211=53%:%
%:%215=53%:%
%:%216=53%:%
%:%217=53%:%
%:%222=53%:%
%:%225=54%:%
%:%226=55%:%
%:%227=55%:%
%:%228=56%:%
%:%231=57%:%
%:%235=57%:%
%:%236=57%:%
%:%237=58%:%
%:%238=59%:%
%:%252=61%:%
%:%262=63%:%
%:%263=63%:%
%:%264=64%:%
%:%265=65%:%
%:%266=66%:%
%:%267=66%:%
%:%268=67%:%
%:%271=68%:%
%:%275=68%:%
%:%276=68%:%
%:%277=69%:%
%:%278=69%:%
%:%279=70%:%
%:%280=70%:%
%:%285=70%:%
%:%288=71%:%
%:%289=72%:%
%:%290=72%:%
%:%291=73%:%
%:%294=74%:%
%:%298=74%:%
%:%299=74%:%
%:%300=74%:%
%:%301=74%:%
%:%306=74%:%
%:%309=75%:%
%:%310=76%:%
%:%311=76%:%
%:%312=77%:%
%:%315=78%:%
%:%319=78%:%
%:%320=78%:%
%:%321=78%:%
%:%322=79%:%
%:%323=79%:%
%:%328=79%:%
%:%331=80%:%
%:%332=81%:%
%:%333=81%:%
%:%334=82%:%
%:%335=83%:%
%:%342=84%:%
%:%343=84%:%
%:%344=85%:%
%:%345=85%:%
%:%346=86%:%
%:%347=86%:%
%:%348=86%:%
%:%349=87%:%
%:%350=87%:%
%:%351=87%:%
%:%352=88%:%
%:%353=88%:%
%:%354=88%:%
%:%355=89%:%
%:%356=89%:%
%:%357=89%:%
%:%358=90%:%
%:%359=90%:%
%:%360=90%:%
%:%361=91%:%
%:%362=91%:%
%:%363=91%:%
%:%364=91%:%
%:%365=92%:%
%:%366=92%:%
%:%367=92%:%
%:%368=93%:%
%:%369=93%:%
%:%370=93%:%
%:%371=94%:%
%:%372=94%:%
%:%373=94%:%
%:%374=95%:%
%:%375=95%:%
%:%376=96%:%
%:%377=96%:%
%:%378=96%:%
%:%379=97%:%
%:%380=97%:%
%:%381=97%:%
%:%382=98%:%
%:%383=98%:%
%:%384=98%:%
%:%385=99%:%
%:%386=99%:%
%:%387=99%:%
%:%388=100%:%
%:%394=100%:%
%:%397=101%:%
%:%398=102%:%
%:%399=102%:%
%:%400=103%:%
%:%401=104%:%
%:%408=105%:%
%:%409=105%:%
%:%410=106%:%
%:%411=106%:%
%:%412=107%:%
%:%413=107%:%
%:%414=107%:%
%:%415=108%:%
%:%416=108%:%
%:%417=108%:%
%:%418=109%:%
%:%419=109%:%
%:%420=109%:%
%:%421=110%:%
%:%422=110%:%
%:%423=110%:%
%:%424=111%:%
%:%425=111%:%
%:%426=111%:%
%:%427=112%:%
%:%428=112%:%
%:%429=112%:%
%:%430=112%:%
%:%431=113%:%
%:%432=113%:%
%:%433=113%:%
%:%434=114%:%
%:%435=114%:%
%:%436=114%:%
%:%437=115%:%
%:%438=115%:%
%:%439=115%:%
%:%440=116%:%
%:%441=116%:%
%:%442=117%:%
%:%443=117%:%
%:%444=117%:%
%:%445=118%:%
%:%446=118%:%
%:%447=118%:%
%:%448=119%:%
%:%449=119%:%
%:%450=119%:%
%:%451=120%:%
%:%452=120%:%
%:%453=120%:%
%:%454=121%:%
%:%460=121%:%
%:%463=122%:%
%:%464=123%:%
%:%465=123%:%
%:%466=124%:%
%:%467=125%:%
%:%468=126%:%
%:%471=127%:%
%:%475=127%:%
%:%476=127%:%
%:%481=127%:%
%:%484=128%:%
%:%485=129%:%
%:%486=129%:%
%:%487=130%:%
%:%488=131%:%
%:%491=132%:%
%:%495=132%:%
%:%496=132%:%
%:%497=132%:%
%:%502=132%:%
%:%505=133%:%
%:%506=134%:%
%:%507=134%:%
%:%508=135%:%
%:%509=136%:%
%:%510=137%:%
%:%511=138%:%
%:%514=139%:%
%:%518=139%:%
%:%519=139%:%
%:%524=139%:%
%:%527=140%:%
%:%528=141%:%
%:%529=141%:%
%:%530=142%:%
%:%531=143%:%
%:%534=144%:%
%:%538=144%:%
%:%539=144%:%
%:%553=146%:%
%:%563=148%:%
%:%564=148%:%
%:%565=149%:%
%:%566=150%:%
%:%567=151%:%
%:%568=151%:%
%:%569=152%:%
%:%572=153%:%
%:%576=153%:%
%:%577=153%:%
%:%578=154%:%
%:%579=154%:%
%:%580=155%:%
%:%581=155%:%
%:%586=155%:%
%:%589=156%:%
%:%590=157%:%
%:%591=157%:%
%:%592=158%:%
%:%595=159%:%
%:%599=159%:%
%:%600=159%:%
%:%601=159%:%
%:%602=159%:%
%:%607=159%:%
%:%610=160%:%
%:%611=161%:%
%:%612=161%:%
%:%613=162%:%
%:%616=163%:%
%:%620=163%:%
%:%621=163%:%
%:%622=163%:%
%:%623=164%:%
%:%624=164%:%
%:%629=164%:%
%:%632=165%:%
%:%633=166%:%
%:%634=166%:%
%:%635=167%:%
%:%636=168%:%
%:%643=169%:%
%:%644=169%:%
%:%645=170%:%
%:%646=170%:%
%:%647=171%:%
%:%648=171%:%
%:%649=171%:%
%:%650=172%:%
%:%651=172%:%
%:%652=172%:%
%:%653=173%:%
%:%654=173%:%
%:%655=173%:%
%:%656=174%:%
%:%657=174%:%
%:%658=174%:%
%:%659=175%:%
%:%660=175%:%
%:%661=175%:%
%:%662=176%:%
%:%663=176%:%
%:%664=176%:%
%:%665=176%:%
%:%666=177%:%
%:%667=177%:%
%:%668=177%:%
%:%669=178%:%
%:%670=178%:%
%:%671=178%:%
%:%672=179%:%
%:%673=179%:%
%:%674=179%:%
%:%675=180%:%
%:%676=180%:%
%:%677=180%:%
%:%678=181%:%
%:%679=181%:%
%:%680=181%:%
%:%681=182%:%
%:%682=182%:%
%:%683=182%:%
%:%684=183%:%
%:%685=183%:%
%:%686=183%:%
%:%687=184%:%
%:%688=184%:%
%:%689=184%:%
%:%690=185%:%
%:%696=185%:%
%:%699=186%:%
%:%700=187%:%
%:%701=187%:%
%:%702=188%:%
%:%703=189%:%
%:%710=190%:%
%:%711=190%:%
%:%712=191%:%
%:%713=191%:%
%:%714=192%:%
%:%715=192%:%
%:%716=192%:%
%:%717=193%:%
%:%718=193%:%
%:%719=193%:%
%:%720=194%:%
%:%721=194%:%
%:%722=194%:%
%:%723=195%:%
%:%724=195%:%
%:%725=195%:%
%:%726=196%:%
%:%727=196%:%
%:%728=196%:%
%:%729=197%:%
%:%730=197%:%
%:%731=197%:%
%:%732=197%:%
%:%733=198%:%
%:%734=198%:%
%:%735=198%:%
%:%736=199%:%
%:%737=199%:%
%:%738=199%:%
%:%739=200%:%
%:%740=200%:%
%:%741=200%:%
%:%742=201%:%
%:%743=201%:%
%:%744=201%:%
%:%745=202%:%
%:%746=202%:%
%:%747=202%:%
%:%748=203%:%
%:%749=203%:%
%:%750=203%:%
%:%751=204%:%
%:%752=204%:%
%:%753=204%:%
%:%754=205%:%
%:%755=205%:%
%:%756=205%:%
%:%757=206%:%
%:%763=206%:%
%:%766=207%:%
%:%767=208%:%
%:%768=208%:%
%:%769=209%:%
%:%770=210%:%
%:%771=211%:%
%:%772=212%:%
%:%779=213%:%
%:%780=213%:%
%:%781=214%:%
%:%782=214%:%
%:%783=215%:%
%:%784=215%:%
%:%785=215%:%
%:%786=216%:%
%:%787=216%:%
%:%788=217%:%
%:%789=217%:%
%:%790=217%:%
%:%791=218%:%
%:%792=218%:%
%:%793=218%:%
%:%794=219%:%
%:%795=219%:%
%:%796=219%:%
%:%797=220%:%
%:%798=220%:%
%:%799=221%:%
%:%800=221%:%
%:%801=221%:%
%:%802=222%:%
%:%803=222%:%
%:%804=223%:%
%:%805=223%:%
%:%806=223%:%
%:%807=224%:%
%:%808=225%:%
%:%809=225%:%
%:%810=226%:%
%:%811=226%:%
%:%812=226%:%
%:%813=227%:%
%:%814=228%:%
%:%815=228%:%
%:%816=228%:%
%:%817=229%:%
%:%818=229%:%
%:%819=229%:%
%:%820=230%:%
%:%821=230%:%
%:%822=231%:%
%:%823=231%:%
%:%824=232%:%
%:%825=232%:%
%:%826=232%:%
%:%827=233%:%
%:%828=233%:%
%:%829=234%:%
%:%830=234%:%
%:%831=235%:%
%:%832=235%:%
%:%833=236%:%
%:%834=236%:%
%:%835=237%:%
%:%836=237%:%
%:%837=238%:%
%:%838=238%:%
%:%839=239%:%
%:%840=239%:%
%:%841=239%:%
%:%842=240%:%
%:%843=240%:%
%:%844=240%:%
%:%845=241%:%
%:%851=241%:%
%:%854=242%:%
%:%855=243%:%
%:%856=243%:%
%:%857=244%:%
%:%858=245%:%
%:%859=246%:%
%:%860=247%:%
%:%863=248%:%
%:%867=248%:%
%:%868=248%:%
%:%882=250%:%
%:%892=252%:%
%:%893=252%:%
%:%894=253%:%
%:%895=254%:%
%:%896=255%:%
%:%897=255%:%
%:%898=256%:%
%:%901=257%:%
%:%905=257%:%
%:%906=257%:%
%:%911=257%:%
%:%914=258%:%
%:%915=259%:%
%:%916=259%:%
%:%917=260%:%
%:%920=263%:%
%:%923=264%:%
%:%927=264%:%
%:%928=264%:%
%:%933=264%:%
%:%936=265%:%
%:%937=266%:%
%:%938=266%:%
%:%939=267%:%
%:%946=268%:%
%:%947=268%:%
%:%948=269%:%
%:%949=269%:%
%:%950=270%:%
%:%951=270%:%
%:%952=271%:%
%:%953=271%:%
%:%954=272%:%
%:%955=272%:%
%:%956=273%:%
%:%957=273%:%
%:%958=274%:%
%:%959=274%:%
%:%960=275%:%
%:%961=275%:%
%:%962=276%:%
%:%963=276%:%
%:%964=277%:%
%:%970=277%:%
%:%973=278%:%
%:%974=279%:%
%:%975=279%:%
%:%976=280%:%
%:%979=281%:%
%:%983=281%:%
%:%984=281%:%
%:%985=282%:%
%:%986=282%:%
%:%987=283%:%
%:%988=283%:%
%:%989=284%:%
%:%990=284%:%
%:%991=285%:%
%:%992=285%:%
%:%993=285%:%
%:%994=286%:%
%:%995=286%:%
%:%996=286%:%
%:%997=287%:%
%:%998=287%:%
%:%999=287%:%
%:%1000=288%:%
%:%1001=289%:%
%:%1002=289%:%
%:%1003=289%:%
%:%1004=290%:%
%:%1005=291%:%
%:%1006=291%:%
%:%1007=292%:%
%:%1008=292%:%
%:%1009=292%:%
%:%1010=293%:%
%:%1011=293%:%
%:%1012=293%:%
%:%1013=294%:%
%:%1014=294%:%
%:%1015=294%:%
%:%1016=295%:%
%:%1017=296%:%
%:%1018=296%:%
%:%1019=296%:%
%:%1020=297%:%
%:%1021=298%:%
%:%1022=298%:%
%:%1023=299%:%
%:%1024=300%:%
%:%1025=300%:%
%:%1026=301%:%
%:%1027=301%:%
%:%1028=302%:%
%:%1029=302%:%
%:%1030=303%:%
%:%1031=303%:%
%:%1032=304%:%
%:%1033=304%:%
%:%1034=305%:%
%:%1035=306%:%
%:%1036=306%:%
%:%1037=307%:%
%:%1038=307%:%
%:%1039=307%:%
%:%1040=308%:%
%:%1041=308%:%
%:%1042=308%:%
%:%1043=309%:%
%:%1044=309%:%
%:%1045=309%:%
%:%1046=310%:%
%:%1047=310%:%
%:%1048=311%:%
%:%1049=311%:%
%:%1050=311%:%
%:%1051=312%:%
%:%1052=312%:%
%:%1053=313%:%
%:%1054=313%:%
%:%1055=314%:%
%:%1056=314%:%
%:%1057=315%:%
%:%1058=315%:%
%:%1059=316%:%
%:%1060=316%:%
%:%1061=317%:%
%:%1062=318%:%
%:%1063=318%:%
%:%1064=319%:%
%:%1065=319%:%
%:%1066=319%:%
%:%1067=320%:%
%:%1068=320%:%
%:%1069=320%:%
%:%1070=321%:%
%:%1071=321%:%
%:%1072=321%:%
%:%1073=322%:%
%:%1074=322%:%
%:%1075=323%:%
%:%1076=323%:%
%:%1077=323%:%
%:%1078=324%:%
%:%1079=324%:%
%:%1080=325%:%
%:%1081=325%:%
%:%1082=326%:%
%:%1083=326%:%
%:%1084=327%:%
%:%1085=327%:%
%:%1086=328%:%
%:%1087=328%:%
%:%1088=329%:%
%:%1089=329%:%
%:%1090=329%:%
%:%1091=330%:%
%:%1092=330%:%
%:%1093=331%:%
%:%1094=331%:%
%:%1095=332%:%
%:%1096=332%:%
%:%1097=333%:%
%:%1098=333%:%
%:%1099=333%:%
%:%1100=334%:%
%:%1101=335%:%
%:%1102=335%:%
%:%1103=336%:%
%:%1104=336%:%
%:%1105=337%:%
%:%1106=337%:%
%:%1107=338%:%
%:%1108=338%:%
%:%1109=339%:%
%:%1110=339%:%
%:%1111=339%:%
%:%1112=340%:%
%:%1113=340%:%
%:%1114=341%:%
%:%1115=341%:%
%:%1116=342%:%
%:%1117=342%:%
%:%1118=343%:%
%:%1119=343%:%
%:%1120=343%:%
%:%1121=344%:%
%:%1122=344%:%
%:%1123=344%:%
%:%1124=345%:%
%:%1125=345%:%
%:%1126=345%:%
%:%1127=346%:%
%:%1128=346%:%
%:%1129=347%:%
%:%1130=347%:%
%:%1131=348%:%
%:%1132=348%:%
%:%1133=349%:%
%:%1139=349%:%
%:%1142=350%:%
%:%1143=351%:%
%:%1144=351%:%
%:%1145=352%:%
%:%1148=353%:%
%:%1152=353%:%
%:%1153=353%:%
%:%1154=354%:%
%:%1155=354%:%
%:%1156=355%:%
%:%1157=355%:%
%:%1162=355%:%
%:%1165=356%:%
%:%1166=357%:%
%:%1167=357%:%
%:%1168=358%:%
%:%1171=359%:%
%:%1175=359%:%
%:%1176=359%:%
%:%1177=360%:%
%:%1178=360%:%
%:%1183=360%:%
%:%1186=361%:%
%:%1187=362%:%
%:%1188=362%:%
%:%1189=363%:%
%:%1190=364%:%
%:%1197=365%:%
%:%1198=365%:%
%:%1199=366%:%
%:%1200=366%:%
%:%1201=367%:%
%:%1202=367%:%
%:%1203=368%:%
%:%1204=369%:%
%:%1205=369%:%
%:%1206=369%:%
%:%1207=370%:%
%:%1208=370%:%
%:%1209=371%:%
%:%1210=372%:%
%:%1216=372%:%
%:%1219=373%:%
%:%1220=374%:%
%:%1221=374%:%
%:%1222=375%:%
%:%1223=376%:%
%:%1230=377%:%
%:%1231=377%:%
%:%1232=378%:%
%:%1233=378%:%
%:%1234=379%:%
%:%1235=379%:%
%:%1236=380%:%
%:%1237=381%:%
%:%1238=381%:%
%:%1239=381%:%
%:%1240=382%:%
%:%1241=382%:%
%:%1242=383%:%
%:%1243=384%:%
%:%1249=384%:%
%:%1252=385%:%
%:%1253=386%:%
%:%1254=386%:%
%:%1255=387%:%
%:%1262=388%:%
%:%1263=388%:%
%:%1264=389%:%
%:%1265=389%:%
%:%1266=390%:%
%:%1267=390%:%
%:%1268=390%:%
%:%1269=391%:%
%:%1270=391%:%
%:%1271=391%:%
%:%1272=392%:%
%:%1273=392%:%
%:%1274=392%:%
%:%1275=393%:%
%:%1276=393%:%
%:%1277=393%:%
%:%1278=394%:%
%:%1279=394%:%
%:%1280=395%:%
%:%1281=395%:%
%:%1282=396%:%
%:%1283=396%:%
%:%1284=397%:%
%:%1285=397%:%
%:%1286=398%:%
%:%1287=398%:%
%:%1288=398%:%
%:%1289=399%:%
%:%1290=399%:%
%:%1291=399%:%
%:%1292=400%:%
%:%1293=400%:%
%:%1294=401%:%
%:%1295=401%:%
%:%1296=402%:%
%:%1297=402%:%
%:%1298=402%:%
%:%1299=403%:%
%:%1300=404%:%
%:%1301=404%:%
%:%1302=404%:%
%:%1303=405%:%
%:%1304=405%:%
%:%1305=406%:%
%:%1306=406%:%
%:%1307=407%:%
%:%1308=407%:%
%:%1309=408%:%
%:%1310=408%:%
%:%1311=409%:%
%:%1312=409%:%
%:%1313=410%:%
%:%1314=410%:%
%:%1315=411%:%
%:%1316=411%:%
%:%1317=412%:%
%:%1318=412%:%
%:%1319=412%:%
%:%1320=413%:%
%:%1321=413%:%
%:%1322=413%:%
%:%1323=414%:%
%:%1324=414%:%
%:%1325=414%:%
%:%1326=415%:%
%:%1327=415%:%
%:%1328=416%:%
%:%1329=416%:%
%:%1330=416%:%
%:%1331=417%:%
%:%1332=417%:%
%:%1333=417%:%
%:%1334=418%:%
%:%1335=418%:%
%:%1336=419%:%
%:%1337=419%:%
%:%1338=419%:%
%:%1339=420%:%
%:%1340=420%:%
%:%1341=421%:%
%:%1342=421%:%
%:%1343=422%:%
%:%1344=422%:%
%:%1345=423%:%
%:%1346=423%:%
%:%1347=423%:%
%:%1348=424%:%
%:%1349=424%:%
%:%1350=424%:%
%:%1351=425%:%
%:%1352=425%:%
%:%1353=426%:%
%:%1354=426%:%
%:%1355=427%:%
%:%1356=427%:%
%:%1357=428%:%
%:%1358=428%:%
%:%1359=429%:%
%:%1360=429%:%
%:%1361=429%:%
%:%1362=430%:%
%:%1363=430%:%
%:%1364=430%:%
%:%1365=431%:%
%:%1366=431%:%
%:%1367=431%:%
%:%1368=432%:%
%:%1369=432%:%
%:%1370=433%:%
%:%1371=433%:%
%:%1372=433%:%
%:%1373=434%:%
%:%1374=434%:%
%:%1375=434%:%
%:%1376=435%:%
%:%1377=435%:%
%:%1378=436%:%
%:%1379=436%:%
%:%1380=436%:%
%:%1381=437%:%
%:%1382=437%:%
%:%1383=438%:%
%:%1384=438%:%
%:%1385=439%:%
%:%1386=439%:%
%:%1387=440%:%
%:%1388=440%:%
%:%1389=440%:%
%:%1390=441%:%
%:%1391=441%:%
%:%1392=441%:%
%:%1393=442%:%
%:%1394=442%:%
%:%1395=442%:%
%:%1396=443%:%
%:%1397=443%:%
%:%1398=443%:%
%:%1399=444%:%
%:%1405=444%:%
%:%1408=445%:%
%:%1409=446%:%
%:%1410=446%:%
%:%1411=447%:%
%:%1412=448%:%
%:%1413=449%:%
%:%1414=450%:%
%:%1417=451%:%
%:%1421=451%:%
%:%1422=451%:%
%:%1427=451%:%
%:%1430=452%:%
%:%1431=453%:%
%:%1432=453%:%
%:%1433=454%:%
%:%1440=455%:%
%:%1441=455%:%
%:%1442=456%:%
%:%1443=456%:%
%:%1444=457%:%
%:%1445=457%:%
%:%1446=458%:%
%:%1447=458%:%
%:%1448=458%:%
%:%1449=459%:%
%:%1450=459%:%
%:%1451=460%:%
%:%1452=460%:%
%:%1453=461%:%
%:%1454=461%:%
%:%1455=462%:%
%:%1456=462%:%
%:%1457=463%:%
%:%1458=463%:%
%:%1459=464%:%
%:%1460=464%:%
%:%1461=465%:%
%:%1462=465%:%
%:%1463=466%:%
%:%1464=466%:%
%:%1465=466%:%
%:%1466=467%:%
%:%1467=467%:%
%:%1468=467%:%
%:%1469=468%:%
%:%1470=468%:%
%:%1471=469%:%
%:%1472=469%:%
%:%1473=470%:%
%:%1474=470%:%
%:%1475=471%:%
%:%1476=471%:%
%:%1477=472%:%
%:%1478=473%:%
%:%1479=473%:%
%:%1480=474%:%
%:%1481=474%:%
%:%1482=475%:%
%:%1483=475%:%
%:%1484=475%:%
%:%1485=476%:%
%:%1486=476%:%
%:%1487=477%:%
%:%1488=478%:%
%:%1489=478%:%
%:%1490=479%:%
%:%1491=479%:%
%:%1492=480%:%
%:%1498=480%:%
%:%1501=481%:%
%:%1502=482%:%
%:%1503=482%:%
%:%1504=483%:%
%:%1505=484%:%
%:%1506=485%:%
%:%1509=486%:%
%:%1513=486%:%
%:%1514=486%:%
%:%1519=486%:%
%:%1522=487%:%
%:%1523=488%:%
%:%1524=488%:%
%:%1525=489%:%
%:%1526=490%:%
%:%1529=491%:%
%:%1533=491%:%
%:%1534=491%:%
%:%1535=491%:%
%:%1540=491%:%
%:%1543=492%:%
%:%1544=493%:%
%:%1545=493%:%
%:%1546=494%:%
%:%1547=495%:%
%:%1548=496%:%
%:%1549=497%:%
%:%1552=498%:%
%:%1556=498%:%
%:%1557=498%:%
%:%1562=498%:%
%:%1565=499%:%
%:%1566=500%:%
%:%1567=500%:%
%:%1568=501%:%
%:%1569=502%:%
%:%1572=503%:%
%:%1576=503%:%
%:%1577=503%:%
%:%1591=505%:%
%:%1601=507%:%
%:%1602=507%:%
%:%1603=508%:%
%:%1604=509%:%
%:%1605=510%:%
%:%1606=510%:%
%:%1607=511%:%
%:%1610=512%:%
%:%1614=512%:%
%:%1615=512%:%
%:%1620=512%:%
%:%1623=513%:%
%:%1624=514%:%
%:%1625=514%:%
%:%1626=515%:%
%:%1629=516%:%
%:%1633=516%:%
%:%1634=516%:%
%:%1635=517%:%
%:%1636=517%:%
%:%1637=518%:%
%:%1638=518%:%
%:%1639=519%:%
%:%1640=519%:%
%:%1641=520%:%
%:%1642=520%:%
%:%1643=520%:%
%:%1644=521%:%
%:%1645=521%:%
%:%1646=521%:%
%:%1647=522%:%
%:%1648=522%:%
%:%1649=522%:%
%:%1650=523%:%
%:%1651=524%:%
%:%1652=524%:%
%:%1653=524%:%
%:%1654=525%:%
%:%1655=526%:%
%:%1656=526%:%
%:%1657=527%:%
%:%1658=527%:%
%:%1659=527%:%
%:%1660=528%:%
%:%1661=528%:%
%:%1662=528%:%
%:%1663=529%:%
%:%1664=529%:%
%:%1665=529%:%
%:%1666=530%:%
%:%1667=531%:%
%:%1668=531%:%
%:%1669=531%:%
%:%1670=532%:%
%:%1671=533%:%
%:%1672=533%:%
%:%1673=534%:%
%:%1674=535%:%
%:%1675=535%:%
%:%1676=536%:%
%:%1677=536%:%
%:%1678=537%:%
%:%1679=537%:%
%:%1680=538%:%
%:%1681=538%:%
%:%1682=538%:%
%:%1683=539%:%
%:%1684=539%:%
%:%1685=540%:%
%:%1686=540%:%
%:%1687=540%:%
%:%1688=541%:%
%:%1689=541%:%
%:%1690=542%:%
%:%1691=542%:%
%:%1692=543%:%
%:%1693=543%:%
%:%1694=544%:%
%:%1695=544%:%
%:%1696=545%:%
%:%1697=545%:%
%:%1698=545%:%
%:%1699=546%:%
%:%1700=546%:%
%:%1701=546%:%
%:%1702=547%:%
%:%1703=547%:%
%:%1704=547%:%
%:%1705=548%:%
%:%1706=548%:%
%:%1707=549%:%
%:%1708=549%:%
%:%1709=549%:%
%:%1710=550%:%
%:%1711=550%:%
%:%1712=550%:%
%:%1713=551%:%
%:%1714=551%:%
%:%1715=551%:%
%:%1716=552%:%
%:%1717=552%:%
%:%1718=553%:%
%:%1719=553%:%
%:%1720=553%:%
%:%1721=554%:%
%:%1722=554%:%
%:%1723=554%:%
%:%1724=555%:%
%:%1725=555%:%
%:%1726=555%:%
%:%1727=556%:%
%:%1728=556%:%
%:%1729=556%:%
%:%1730=557%:%
%:%1731=557%:%
%:%1732=558%:%
%:%1733=558%:%
%:%1734=558%:%
%:%1735=559%:%
%:%1736=559%:%
%:%1737=560%:%
%:%1738=560%:%
%:%1739=560%:%
%:%1740=561%:%
%:%1741=561%:%
%:%1742=562%:%
%:%1743=562%:%
%:%1744=562%:%
%:%1745=563%:%
%:%1746=563%:%
%:%1747=564%:%
%:%1748=564%:%
%:%1749=565%:%
%:%1750=565%:%
%:%1751=566%:%
%:%1752=566%:%
%:%1753=566%:%
%:%1754=567%:%
%:%1755=567%:%
%:%1756=567%:%
%:%1757=568%:%
%:%1758=568%:%
%:%1759=568%:%
%:%1760=569%:%
%:%1761=569%:%
%:%1762=570%:%
%:%1763=570%:%
%:%1764=571%:%
%:%1765=571%:%
%:%1766=572%:%
%:%1767=572%:%
%:%1768=573%:%
%:%1769=573%:%
%:%1770=573%:%
%:%1771=574%:%
%:%1772=574%:%
%:%1773=574%:%
%:%1774=575%:%
%:%1775=575%:%
%:%1776=575%:%
%:%1777=576%:%
%:%1778=576%:%
%:%1779=577%:%
%:%1780=577%:%
%:%1781=577%:%
%:%1782=578%:%
%:%1783=578%:%
%:%1784=578%:%
%:%1785=579%:%
%:%1786=579%:%
%:%1787=579%:%
%:%1788=580%:%
%:%1789=580%:%
%:%1790=581%:%
%:%1791=581%:%
%:%1792=581%:%
%:%1793=582%:%
%:%1794=582%:%
%:%1795=582%:%
%:%1796=583%:%
%:%1797=583%:%
%:%1798=583%:%
%:%1799=584%:%
%:%1800=584%:%
%:%1801=584%:%
%:%1802=585%:%
%:%1803=585%:%
%:%1804=586%:%
%:%1805=586%:%
%:%1806=586%:%
%:%1807=587%:%
%:%1808=587%:%
%:%1809=588%:%
%:%1810=588%:%
%:%1811=588%:%
%:%1812=589%:%
%:%1813=589%:%
%:%1814=590%:%
%:%1815=590%:%
%:%1816=590%:%
%:%1817=591%:%
%:%1818=591%:%
%:%1819=592%:%
%:%1820=592%:%
%:%1821=593%:%
%:%1827=593%:%
%:%1830=594%:%
%:%1831=595%:%
%:%1832=595%:%
%:%1833=596%:%
%:%1836=597%:%
%:%1840=597%:%
%:%1841=597%:%
%:%1842=598%:%
%:%1843=598%:%
%:%1844=599%:%
%:%1845=599%:%
%:%1850=599%:%
%:%1853=600%:%
%:%1854=601%:%
%:%1855=601%:%
%:%1856=602%:%
%:%1859=603%:%
%:%1863=603%:%
%:%1864=603%:%
%:%1865=604%:%
%:%1866=604%:%
%:%1871=604%:%
%:%1874=605%:%
%:%1875=606%:%
%:%1876=606%:%
%:%1877=607%:%
%:%1884=608%:%
%:%1885=608%:%
%:%1886=609%:%
%:%1887=609%:%
%:%1888=610%:%
%:%1889=610%:%
%:%1890=611%:%
%:%1891=611%:%
%:%1892=611%:%
%:%1893=612%:%
%:%1894=612%:%
%:%1895=613%:%
%:%1901=613%:%
%:%1904=614%:%
%:%1905=615%:%
%:%1906=615%:%
%:%1907=616%:%
%:%1914=617%:%
%:%1915=617%:%
%:%1916=618%:%
%:%1917=618%:%
%:%1918=619%:%
%:%1919=619%:%
%:%1920=620%:%
%:%1921=620%:%
%:%1922=620%:%
%:%1923=621%:%
%:%1924=621%:%
%:%1925=622%:%
%:%1931=622%:%
%:%1934=623%:%
%:%1935=624%:%
%:%1936=624%:%
%:%1937=625%:%
%:%1944=626%:%
%:%1945=626%:%
%:%1946=627%:%
%:%1947=627%:%
%:%1948=628%:%
%:%1949=628%:%
%:%1950=628%:%
%:%1951=629%:%
%:%1952=629%:%
%:%1953=630%:%
%:%1954=630%:%
%:%1955=630%:%
%:%1956=631%:%
%:%1957=631%:%
%:%1958=632%:%
%:%1959=632%:%
%:%1960=633%:%
%:%1961=633%:%
%:%1962=634%:%
%:%1963=634%:%
%:%1964=635%:%
%:%1965=635%:%
%:%1966=635%:%
%:%1967=636%:%
%:%1968=636%:%
%:%1969=636%:%
%:%1970=637%:%
%:%1971=637%:%
%:%1972=637%:%
%:%1973=638%:%
%:%1974=638%:%
%:%1975=638%:%
%:%1976=639%:%
%:%1977=639%:%
%:%1978=639%:%
%:%1979=640%:%
%:%1980=640%:%
%:%1981=640%:%
%:%1982=641%:%
%:%1983=641%:%
%:%1984=641%:%
%:%1985=642%:%
%:%1986=642%:%
%:%1987=642%:%
%:%1988=643%:%
%:%1989=643%:%
%:%1990=644%:%
%:%1991=644%:%
%:%1992=645%:%
%:%1993=645%:%
%:%1994=645%:%
%:%1995=646%:%
%:%1996=646%:%
%:%1997=647%:%
%:%1998=647%:%
%:%1999=647%:%
%:%2000=648%:%
%:%2001=648%:%
%:%2002=648%:%
%:%2003=649%:%
%:%2004=649%:%
%:%2005=649%:%
%:%2006=650%:%
%:%2007=650%:%
%:%2008=650%:%
%:%2009=651%:%
%:%2010=651%:%
%:%2011=651%:%
%:%2012=652%:%
%:%2013=652%:%
%:%2014=652%:%
%:%2015=653%:%
%:%2016=653%:%
%:%2017=653%:%
%:%2018=654%:%
%:%2019=654%:%
%:%2020=654%:%
%:%2021=655%:%
%:%2022=655%:%
%:%2023=656%:%
%:%2029=656%:%
%:%2032=657%:%
%:%2033=658%:%
%:%2034=658%:%
%:%2035=659%:%
%:%2042=660%:%
%:%2043=660%:%
%:%2044=661%:%
%:%2045=661%:%
%:%2046=662%:%
%:%2047=662%:%
%:%2048=662%:%
%:%2049=663%:%
%:%2050=663%:%
%:%2051=664%:%
%:%2052=664%:%
%:%2053=664%:%
%:%2054=665%:%
%:%2055=665%:%
%:%2056=666%:%
%:%2057=666%:%
%:%2058=667%:%
%:%2059=667%:%
%:%2060=668%:%
%:%2061=668%:%
%:%2062=669%:%
%:%2063=669%:%
%:%2064=669%:%
%:%2065=670%:%
%:%2066=670%:%
%:%2067=670%:%
%:%2068=671%:%
%:%2069=671%:%
%:%2070=671%:%
%:%2071=672%:%
%:%2072=672%:%
%:%2073=672%:%
%:%2074=673%:%
%:%2075=673%:%
%:%2076=673%:%
%:%2077=674%:%
%:%2078=674%:%
%:%2079=674%:%
%:%2080=675%:%
%:%2081=675%:%
%:%2082=675%:%
%:%2083=676%:%
%:%2084=676%:%
%:%2085=676%:%
%:%2086=677%:%
%:%2087=677%:%
%:%2088=678%:%
%:%2089=678%:%
%:%2090=679%:%
%:%2091=679%:%
%:%2092=679%:%
%:%2093=680%:%
%:%2094=680%:%
%:%2095=681%:%
%:%2096=681%:%
%:%2097=681%:%
%:%2098=682%:%
%:%2099=682%:%
%:%2100=682%:%
%:%2101=683%:%
%:%2102=683%:%
%:%2103=683%:%
%:%2104=684%:%
%:%2105=684%:%
%:%2106=684%:%
%:%2107=685%:%
%:%2108=685%:%
%:%2109=685%:%
%:%2110=686%:%
%:%2111=686%:%
%:%2112=686%:%
%:%2113=687%:%
%:%2114=687%:%
%:%2115=687%:%
%:%2116=688%:%
%:%2117=688%:%
%:%2118=688%:%
%:%2119=689%:%
%:%2120=689%:%
%:%2121=690%:%
%:%2136=692%:%
%:%2146=693%:%
%:%2147=693%:%
%:%2148=694%:%
%:%2149=695%:%
%:%2150=696%:%
%:%2151=696%:%
%:%2152=697%:%
%:%2155=698%:%
%:%2159=698%:%
%:%2160=698%:%
%:%2165=698%:%
%:%2168=699%:%
%:%2169=700%:%
%:%2170=700%:%
%:%2171=701%:%
%:%2174=702%:%
%:%2178=702%:%
%:%2179=702%:%
%:%2180=703%:%
%:%2181=703%:%
%:%2182=704%:%
%:%2183=704%:%
%:%2184=705%:%
%:%2185=705%:%
%:%2186=706%:%
%:%2187=706%:%
%:%2188=706%:%
%:%2189=707%:%
%:%2190=707%:%
%:%2191=707%:%
%:%2192=708%:%
%:%2193=708%:%
%:%2194=708%:%
%:%2195=709%:%
%:%2196=710%:%
%:%2197=710%:%
%:%2198=710%:%
%:%2199=711%:%
%:%2200=712%:%
%:%2201=712%:%
%:%2202=713%:%
%:%2203=713%:%
%:%2204=713%:%
%:%2205=714%:%
%:%2206=714%:%
%:%2207=714%:%
%:%2208=715%:%
%:%2209=715%:%
%:%2210=715%:%
%:%2211=716%:%
%:%2212=717%:%
%:%2213=717%:%
%:%2214=717%:%
%:%2215=718%:%
%:%2216=719%:%
%:%2217=719%:%
%:%2218=720%:%
%:%2219=721%:%
%:%2220=721%:%
%:%2221=722%:%
%:%2222=722%:%
%:%2223=723%:%
%:%2224=723%:%
%:%2225=724%:%
%:%2226=724%:%
%:%2227=725%:%
%:%2228=725%:%
%:%2229=726%:%
%:%2230=727%:%
%:%2231=727%:%
%:%2232=728%:%
%:%2233=728%:%
%:%2234=728%:%
%:%2235=729%:%
%:%2236=729%:%
%:%2237=729%:%
%:%2238=730%:%
%:%2239=730%:%
%:%2240=730%:%
%:%2241=731%:%
%:%2242=731%:%
%:%2243=732%:%
%:%2244=732%:%
%:%2245=732%:%
%:%2246=733%:%
%:%2247=733%:%
%:%2248=734%:%
%:%2249=734%:%
%:%2250=735%:%
%:%2251=735%:%
%:%2252=736%:%
%:%2253=736%:%
%:%2254=737%:%
%:%2255=737%:%
%:%2256=738%:%
%:%2257=739%:%
%:%2258=739%:%
%:%2259=740%:%
%:%2260=740%:%
%:%2261=740%:%
%:%2262=741%:%
%:%2263=741%:%
%:%2264=741%:%
%:%2265=742%:%
%:%2266=742%:%
%:%2267=742%:%
%:%2268=743%:%
%:%2269=743%:%
%:%2270=744%:%
%:%2271=744%:%
%:%2272=744%:%
%:%2273=745%:%
%:%2274=745%:%
%:%2275=746%:%
%:%2276=746%:%
%:%2277=747%:%
%:%2278=747%:%
%:%2279=748%:%
%:%2280=748%:%
%:%2281=749%:%
%:%2282=749%:%
%:%2283=750%:%
%:%2284=750%:%
%:%2285=750%:%
%:%2286=751%:%
%:%2287=751%:%
%:%2288=752%:%
%:%2289=752%:%
%:%2290=753%:%
%:%2291=753%:%
%:%2292=754%:%
%:%2293=754%:%
%:%2294=754%:%
%:%2295=755%:%
%:%2296=756%:%
%:%2297=756%:%
%:%2298=757%:%
%:%2299=757%:%
%:%2300=758%:%
%:%2301=758%:%
%:%2302=759%:%
%:%2303=759%:%
%:%2304=760%:%
%:%2305=760%:%
%:%2306=760%:%
%:%2307=761%:%
%:%2308=761%:%
%:%2309=762%:%
%:%2310=762%:%
%:%2311=763%:%
%:%2312=763%:%
%:%2313=764%:%
%:%2314=764%:%
%:%2315=764%:%
%:%2316=765%:%
%:%2317=765%:%
%:%2318=765%:%
%:%2319=766%:%
%:%2320=766%:%
%:%2321=766%:%
%:%2322=767%:%
%:%2323=767%:%
%:%2324=768%:%
%:%2325=768%:%
%:%2326=769%:%
%:%2327=769%:%
%:%2328=770%:%
%:%2334=770%:%
%:%2337=771%:%
%:%2338=772%:%
%:%2339=772%:%
%:%2340=773%:%
%:%2343=774%:%
%:%2347=774%:%
%:%2348=774%:%
%:%2349=775%:%
%:%2350=775%:%
%:%2351=776%:%
%:%2352=776%:%
%:%2357=776%:%
%:%2360=777%:%
%:%2361=778%:%
%:%2362=778%:%
%:%2363=779%:%
%:%2366=780%:%
%:%2370=780%:%
%:%2371=780%:%
%:%2372=781%:%
%:%2373=781%:%
%:%2378=781%:%
%:%2381=782%:%
%:%2382=783%:%
%:%2383=783%:%
%:%2384=784%:%
%:%2385=785%:%
%:%2392=786%:%
%:%2393=786%:%
%:%2394=787%:%
%:%2395=787%:%
%:%2396=788%:%
%:%2397=788%:%
%:%2398=789%:%
%:%2399=789%:%
%:%2400=789%:%
%:%2401=790%:%
%:%2402=790%:%
%:%2403=791%:%
%:%2409=791%:%
%:%2412=792%:%
%:%2413=793%:%
%:%2414=793%:%
%:%2415=794%:%
%:%2416=795%:%
%:%2423=796%:%
%:%2424=796%:%
%:%2425=797%:%
%:%2426=797%:%
%:%2427=798%:%
%:%2428=798%:%
%:%2429=799%:%
%:%2430=799%:%
%:%2431=799%:%
%:%2432=800%:%
%:%2433=800%:%
%:%2434=801%:%
%:%2440=801%:%
%:%2443=802%:%
%:%2444=803%:%
%:%2445=803%:%
%:%2446=804%:%
%:%2453=805%:%
%:%2454=805%:%
%:%2455=806%:%
%:%2456=806%:%
%:%2457=807%:%
%:%2458=807%:%
%:%2459=807%:%
%:%2460=808%:%
%:%2461=808%:%
%:%2462=808%:%
%:%2463=809%:%
%:%2464=809%:%
%:%2465=809%:%
%:%2466=810%:%
%:%2467=810%:%
%:%2468=810%:%
%:%2469=811%:%
%:%2470=811%:%
%:%2471=812%:%
%:%2472=812%:%
%:%2473=812%:%
%:%2474=813%:%
%:%2475=813%:%
%:%2476=814%:%
%:%2477=814%:%
%:%2478=815%:%
%:%2479=815%:%
%:%2480=815%:%
%:%2481=816%:%
%:%2482=816%:%
%:%2483=817%:%
%:%2484=817%:%
%:%2485=818%:%
%:%2486=818%:%
%:%2487=819%:%
%:%2488=819%:%
%:%2489=819%:%
%:%2490=820%:%
%:%2491=821%:%
%:%2492=821%:%
%:%2493=821%:%
%:%2494=822%:%
%:%2495=822%:%
%:%2496=823%:%
%:%2497=823%:%
%:%2498=824%:%
%:%2499=824%:%
%:%2500=825%:%
%:%2501=825%:%
%:%2502=826%:%
%:%2503=826%:%
%:%2504=827%:%
%:%2505=827%:%
%:%2506=828%:%
%:%2507=828%:%
%:%2508=829%:%
%:%2509=829%:%
%:%2510=829%:%
%:%2511=830%:%
%:%2512=830%:%
%:%2513=830%:%
%:%2514=831%:%
%:%2515=831%:%
%:%2516=831%:%
%:%2517=832%:%
%:%2518=832%:%
%:%2519=833%:%
%:%2520=833%:%
%:%2521=833%:%
%:%2522=834%:%
%:%2523=834%:%
%:%2524=834%:%
%:%2525=835%:%
%:%2526=835%:%
%:%2527=836%:%
%:%2528=836%:%
%:%2529=836%:%
%:%2530=837%:%
%:%2531=837%:%
%:%2532=838%:%
%:%2533=838%:%
%:%2534=839%:%
%:%2535=839%:%
%:%2536=840%:%
%:%2537=840%:%
%:%2538=840%:%
%:%2539=841%:%
%:%2540=841%:%
%:%2541=841%:%
%:%2542=842%:%
%:%2543=842%:%
%:%2544=843%:%
%:%2545=843%:%
%:%2546=844%:%
%:%2547=844%:%
%:%2548=845%:%
%:%2549=845%:%
%:%2550=846%:%
%:%2551=846%:%
%:%2552=846%:%
%:%2553=847%:%
%:%2554=847%:%
%:%2555=847%:%
%:%2556=848%:%
%:%2557=848%:%
%:%2558=848%:%
%:%2559=849%:%
%:%2560=849%:%
%:%2561=850%:%
%:%2562=850%:%
%:%2563=850%:%
%:%2564=851%:%
%:%2565=851%:%
%:%2566=851%:%
%:%2567=852%:%
%:%2568=852%:%
%:%2569=853%:%
%:%2570=853%:%
%:%2571=853%:%
%:%2572=854%:%
%:%2573=854%:%
%:%2574=855%:%
%:%2575=855%:%
%:%2576=856%:%
%:%2577=856%:%
%:%2578=857%:%
%:%2579=857%:%
%:%2580=857%:%
%:%2581=858%:%
%:%2582=858%:%
%:%2583=858%:%
%:%2584=859%:%
%:%2585=859%:%
%:%2586=859%:%
%:%2587=860%:%
%:%2588=860%:%
%:%2589=860%:%
%:%2590=861%:%
%:%2596=861%:%
%:%2599=862%:%
%:%2600=863%:%
%:%2601=863%:%
%:%2602=864%:%
%:%2603=865%:%
%:%2604=866%:%
%:%2605=867%:%
%:%2608=868%:%
%:%2612=868%:%
%:%2613=868%:%
%:%2618=868%:%
%:%2621=869%:%
%:%2622=870%:%
%:%2623=870%:%
%:%2624=871%:%
%:%2631=872%:%
%:%2632=872%:%
%:%2633=873%:%
%:%2634=873%:%
%:%2635=874%:%
%:%2636=874%:%
%:%2637=875%:%
%:%2638=875%:%
%:%2639=875%:%
%:%2640=876%:%
%:%2641=876%:%
%:%2642=877%:%
%:%2643=877%:%
%:%2644=878%:%
%:%2645=878%:%
%:%2646=879%:%
%:%2652=879%:%
%:%2655=880%:%
%:%2656=881%:%
%:%2657=881%:%
%:%2658=882%:%
%:%2659=883%:%
%:%2662=884%:%
%:%2666=884%:%
%:%2667=884%:%
%:%2668=884%:%
%:%2682=886%:%
%:%2692=888%:%
%:%2693=888%:%
%:%2694=889%:%
%:%2695=890%:%
%:%2696=891%:%
%:%2697=891%:%
%:%2698=892%:%
%:%2701=893%:%
%:%2705=893%:%
%:%2706=893%:%
%:%2711=893%:%
%:%2714=894%:%
%:%2715=895%:%
%:%2716=895%:%
%:%2717=896%:%
%:%2720=897%:%
%:%2724=897%:%
%:%2725=897%:%
%:%2726=898%:%
%:%2727=898%:%
%:%2728=899%:%
%:%2729=899%:%
%:%2730=900%:%
%:%2731=900%:%
%:%2732=901%:%
%:%2733=901%:%
%:%2734=901%:%
%:%2735=902%:%
%:%2736=902%:%
%:%2737=902%:%
%:%2738=903%:%
%:%2739=903%:%
%:%2740=903%:%
%:%2741=904%:%
%:%2742=905%:%
%:%2743=905%:%
%:%2744=905%:%
%:%2745=906%:%
%:%2746=907%:%
%:%2747=907%:%
%:%2748=908%:%
%:%2749=908%:%
%:%2750=908%:%
%:%2751=909%:%
%:%2752=909%:%
%:%2753=909%:%
%:%2754=910%:%
%:%2755=910%:%
%:%2756=910%:%
%:%2757=911%:%
%:%2758=912%:%
%:%2759=912%:%
%:%2760=912%:%
%:%2761=913%:%
%:%2762=914%:%
%:%2763=914%:%
%:%2764=915%:%
%:%2765=916%:%
%:%2766=916%:%
%:%2767=917%:%
%:%2768=917%:%
%:%2769=918%:%
%:%2770=918%:%
%:%2771=919%:%
%:%2772=919%:%
%:%2773=919%:%
%:%2774=920%:%
%:%2775=920%:%
%:%2776=921%:%
%:%2777=921%:%
%:%2778=921%:%
%:%2779=922%:%
%:%2780=922%:%
%:%2781=923%:%
%:%2782=923%:%
%:%2783=924%:%
%:%2784=924%:%
%:%2785=925%:%
%:%2786=925%:%
%:%2787=926%:%
%:%2788=926%:%
%:%2789=926%:%
%:%2790=927%:%
%:%2791=927%:%
%:%2792=927%:%
%:%2793=928%:%
%:%2794=928%:%
%:%2795=928%:%
%:%2796=929%:%
%:%2797=929%:%
%:%2798=930%:%
%:%2799=930%:%
%:%2800=930%:%
%:%2801=931%:%
%:%2802=931%:%
%:%2803=931%:%
%:%2804=932%:%
%:%2805=932%:%
%:%2806=932%:%
%:%2807=933%:%
%:%2808=933%:%
%:%2809=934%:%
%:%2810=934%:%
%:%2811=934%:%
%:%2812=935%:%
%:%2813=935%:%
%:%2814=935%:%
%:%2815=936%:%
%:%2816=936%:%
%:%2817=936%:%
%:%2818=937%:%
%:%2819=937%:%
%:%2820=937%:%
%:%2821=938%:%
%:%2822=938%:%
%:%2823=939%:%
%:%2824=939%:%
%:%2825=939%:%
%:%2826=940%:%
%:%2827=940%:%
%:%2828=941%:%
%:%2829=941%:%
%:%2830=941%:%
%:%2831=942%:%
%:%2832=942%:%
%:%2833=943%:%
%:%2834=943%:%
%:%2835=943%:%
%:%2836=944%:%
%:%2837=944%:%
%:%2838=945%:%
%:%2839=945%:%
%:%2840=946%:%
%:%2841=946%:%
%:%2842=947%:%
%:%2843=947%:%
%:%2844=947%:%
%:%2845=948%:%
%:%2846=948%:%
%:%2847=948%:%
%:%2848=949%:%
%:%2849=949%:%
%:%2850=949%:%
%:%2851=950%:%
%:%2852=950%:%
%:%2853=951%:%
%:%2854=951%:%
%:%2855=952%:%
%:%2856=952%:%
%:%2857=953%:%
%:%2858=953%:%
%:%2859=954%:%
%:%2860=954%:%
%:%2861=954%:%
%:%2862=955%:%
%:%2863=955%:%
%:%2864=955%:%
%:%2865=956%:%
%:%2866=956%:%
%:%2867=956%:%
%:%2868=957%:%
%:%2869=957%:%
%:%2870=958%:%
%:%2871=958%:%
%:%2872=958%:%
%:%2873=959%:%
%:%2874=959%:%
%:%2875=959%:%
%:%2876=960%:%
%:%2877=960%:%
%:%2878=960%:%
%:%2879=961%:%
%:%2880=961%:%
%:%2881=962%:%
%:%2882=962%:%
%:%2883=962%:%
%:%2884=963%:%
%:%2885=963%:%
%:%2886=963%:%
%:%2887=964%:%
%:%2888=964%:%
%:%2889=964%:%
%:%2890=965%:%
%:%2891=965%:%
%:%2892=965%:%
%:%2893=966%:%
%:%2894=966%:%
%:%2895=967%:%
%:%2896=967%:%
%:%2897=967%:%
%:%2898=968%:%
%:%2899=968%:%
%:%2900=969%:%
%:%2901=969%:%
%:%2902=969%:%
%:%2903=970%:%
%:%2904=970%:%
%:%2905=971%:%
%:%2906=971%:%
%:%2907=971%:%
%:%2908=972%:%
%:%2909=972%:%
%:%2910=973%:%
%:%2911=973%:%
%:%2912=974%:%
%:%2918=974%:%
%:%2921=975%:%
%:%2922=976%:%
%:%2923=976%:%
%:%2924=977%:%
%:%2927=978%:%
%:%2931=978%:%
%:%2932=978%:%
%:%2933=979%:%
%:%2934=979%:%
%:%2935=980%:%
%:%2936=980%:%
%:%2941=980%:%
%:%2944=981%:%
%:%2945=982%:%
%:%2946=982%:%
%:%2947=983%:%
%:%2950=984%:%
%:%2954=984%:%
%:%2955=984%:%
%:%2956=985%:%
%:%2957=985%:%
%:%2962=985%:%
%:%2965=986%:%
%:%2966=987%:%
%:%2967=987%:%
%:%2968=988%:%
%:%2975=989%:%
%:%2976=989%:%
%:%2977=990%:%
%:%2978=990%:%
%:%2979=991%:%
%:%2980=991%:%
%:%2981=992%:%
%:%2982=992%:%
%:%2983=992%:%
%:%2984=993%:%
%:%2985=993%:%
%:%2986=994%:%
%:%2992=994%:%
%:%2995=995%:%
%:%2996=996%:%
%:%2997=996%:%
%:%2998=997%:%
%:%3005=998%:%
%:%3006=998%:%
%:%3007=999%:%
%:%3008=999%:%
%:%3009=1000%:%
%:%3010=1000%:%
%:%3011=1001%:%
%:%3012=1001%:%
%:%3013=1001%:%
%:%3014=1002%:%
%:%3015=1002%:%
%:%3016=1003%:%
%:%3022=1003%:%
%:%3025=1004%:%
%:%3026=1005%:%
%:%3027=1005%:%
%:%3028=1006%:%
%:%3035=1007%:%
%:%3036=1007%:%
%:%3037=1008%:%
%:%3038=1008%:%
%:%3039=1009%:%
%:%3040=1009%:%
%:%3041=1009%:%
%:%3042=1010%:%
%:%3043=1010%:%
%:%3044=1010%:%
%:%3045=1011%:%
%:%3046=1011%:%
%:%3047=1011%:%
%:%3048=1012%:%
%:%3049=1012%:%
%:%3050=1013%:%
%:%3051=1013%:%
%:%3052=1014%:%
%:%3053=1014%:%
%:%3054=1015%:%
%:%3055=1015%:%
%:%3056=1016%:%
%:%3057=1016%:%
%:%3058=1016%:%
%:%3059=1017%:%
%:%3060=1017%:%
%:%3061=1017%:%
%:%3062=1018%:%
%:%3063=1018%:%
%:%3064=1018%:%
%:%3065=1019%:%
%:%3066=1019%:%
%:%3067=1019%:%
%:%3068=1020%:%
%:%3069=1020%:%
%:%3070=1020%:%
%:%3071=1021%:%
%:%3072=1021%:%
%:%3073=1021%:%
%:%3074=1022%:%
%:%3075=1022%:%
%:%3076=1022%:%
%:%3077=1023%:%
%:%3078=1023%:%
%:%3079=1023%:%
%:%3080=1024%:%
%:%3081=1024%:%
%:%3082=1025%:%
%:%3083=1025%:%
%:%3084=1026%:%
%:%3085=1026%:%
%:%3086=1026%:%
%:%3087=1027%:%
%:%3088=1027%:%
%:%3089=1027%:%
%:%3090=1028%:%
%:%3091=1028%:%
%:%3092=1028%:%
%:%3093=1029%:%
%:%3094=1029%:%
%:%3095=1029%:%
%:%3096=1030%:%
%:%3097=1030%:%
%:%3098=1030%:%
%:%3099=1031%:%
%:%3100=1031%:%
%:%3101=1031%:%
%:%3102=1032%:%
%:%3103=1032%:%
%:%3104=1032%:%
%:%3105=1033%:%
%:%3106=1033%:%
%:%3107=1033%:%
%:%3108=1034%:%
%:%3109=1034%:%
%:%3110=1034%:%
%:%3111=1035%:%
%:%3112=1035%:%
%:%3113=1035%:%
%:%3114=1036%:%
%:%3115=1036%:%
%:%3116=1037%:%
%:%3122=1037%:%
%:%3125=1038%:%
%:%3126=1039%:%
%:%3127=1039%:%
%:%3128=1040%:%
%:%3135=1041%:%
%:%3136=1041%:%
%:%3137=1042%:%
%:%3138=1042%:%
%:%3139=1043%:%
%:%3140=1043%:%
%:%3141=1043%:%
%:%3142=1044%:%
%:%3143=1044%:%
%:%3144=1044%:%
%:%3145=1045%:%
%:%3146=1045%:%
%:%3147=1045%:%
%:%3148=1046%:%
%:%3149=1046%:%
%:%3150=1047%:%
%:%3151=1047%:%
%:%3152=1048%:%
%:%3153=1048%:%
%:%3154=1049%:%
%:%3155=1049%:%
%:%3156=1050%:%
%:%3157=1050%:%
%:%3158=1050%:%
%:%3159=1051%:%
%:%3160=1051%:%
%:%3161=1051%:%
%:%3162=1052%:%
%:%3163=1052%:%
%:%3164=1052%:%
%:%3165=1053%:%
%:%3166=1053%:%
%:%3167=1053%:%
%:%3168=1054%:%
%:%3169=1054%:%
%:%3170=1054%:%
%:%3171=1055%:%
%:%3172=1055%:%
%:%3173=1055%:%
%:%3174=1056%:%
%:%3175=1056%:%
%:%3176=1056%:%
%:%3177=1057%:%
%:%3178=1057%:%
%:%3179=1057%:%
%:%3180=1058%:%
%:%3181=1058%:%
%:%3182=1059%:%
%:%3183=1059%:%
%:%3184=1060%:%
%:%3185=1060%:%
%:%3186=1060%:%
%:%3187=1061%:%
%:%3188=1061%:%
%:%3189=1061%:%
%:%3190=1062%:%
%:%3191=1062%:%
%:%3192=1062%:%
%:%3193=1063%:%
%:%3194=1063%:%
%:%3195=1063%:%
%:%3196=1064%:%
%:%3197=1064%:%
%:%3198=1064%:%
%:%3199=1065%:%
%:%3200=1065%:%
%:%3201=1065%:%
%:%3202=1066%:%
%:%3203=1066%:%
%:%3204=1066%:%
%:%3205=1067%:%
%:%3206=1067%:%
%:%3207=1067%:%
%:%3208=1068%:%
%:%3209=1068%:%
%:%3210=1068%:%
%:%3211=1069%:%
%:%3212=1069%:%
%:%3213=1069%:%
%:%3214=1070%:%
%:%3215=1070%:%
%:%3216=1071%:%
%:%3222=1071%:%
%:%3225=1072%:%
%:%3226=1073%:%
%:%3227=1073%:%
%:%3228=1074%:%
%:%3235=1075%:%
%:%3236=1075%:%
%:%3237=1076%:%
%:%3238=1076%:%
%:%3239=1077%:%
%:%3240=1077%:%
%:%3241=1078%:%
%:%3242=1078%:%
%:%3243=1078%:%
%:%3244=1079%:%
%:%3245=1079%:%
%:%3246=1079%:%
%:%3247=1080%:%
%:%3248=1080%:%
%:%3249=1081%:%
%:%3250=1081%:%
%:%3251=1082%:%
%:%3252=1082%:%
%:%3253=1083%:%
%:%3254=1083%:%
%:%3255=1084%:%
%:%3256=1084%:%
%:%3257=1085%:%
%:%3258=1085%:%
%:%3259=1086%:%
%:%3260=1086%:%
%:%3261=1087%:%
%:%3262=1087%:%
%:%3263=1088%:%
%:%3264=1088%:%
%:%3265=1088%:%
%:%3266=1089%:%
%:%3267=1089%:%
%:%3268=1090%:%
%:%3269=1090%:%
%:%3270=1090%:%
%:%3271=1091%:%
%:%3272=1091%:%
%:%3273=1092%:%
%:%3274=1092%:%
%:%3275=1093%:%
%:%3276=1093%:%
%:%3277=1094%:%
%:%3278=1094%:%
%:%3279=1095%:%
%:%3280=1095%:%
%:%3281=1095%:%
%:%3282=1096%:%
%:%3283=1096%:%
%:%3284=1097%:%
%:%3285=1097%:%
%:%3286=1098%:%
%:%3287=1098%:%
%:%3288=1099%:%
%:%3289=1099%:%
%:%3290=1100%:%
%:%3291=1100%:%
%:%3292=1101%:%
%:%3293=1101%:%
%:%3294=1102%:%
%:%3295=1102%:%
%:%3296=1103%:%
%:%3297=1103%:%
%:%3298=1103%:%
%:%3299=1104%:%
%:%3300=1104%:%
%:%3301=1105%:%
%:%3302=1105%:%
%:%3303=1105%:%
%:%3304=1106%:%
%:%3305=1106%:%
%:%3306=1107%:%
%:%3307=1107%:%
%:%3308=1108%:%
%:%3309=1108%:%
%:%3310=1109%:%
%:%3325=1111%:%
%:%3335=1113%:%
%:%3336=1113%:%
%:%3337=1114%:%
%:%3338=1115%:%
%:%3339=1116%:%
%:%3340=1116%:%
%:%3341=1117%:%
%:%3344=1118%:%
%:%3348=1118%:%
%:%3349=1118%:%
%:%3354=1118%:%
%:%3357=1119%:%
%:%3358=1120%:%
%:%3359=1120%:%
%:%3360=1121%:%
%:%3363=1122%:%
%:%3367=1122%:%
%:%3368=1122%:%
%:%3369=1123%:%
%:%3370=1123%:%
%:%3371=1124%:%
%:%3372=1124%:%
%:%3373=1125%:%
%:%3374=1125%:%
%:%3375=1126%:%
%:%3376=1126%:%
%:%3377=1126%:%
%:%3378=1127%:%
%:%3379=1127%:%
%:%3380=1127%:%
%:%3381=1128%:%
%:%3382=1128%:%
%:%3383=1128%:%
%:%3384=1129%:%
%:%3385=1130%:%
%:%3386=1130%:%
%:%3387=1130%:%
%:%3388=1131%:%
%:%3389=1132%:%
%:%3390=1132%:%
%:%3391=1133%:%
%:%3392=1133%:%
%:%3393=1133%:%
%:%3394=1134%:%
%:%3395=1134%:%
%:%3396=1134%:%
%:%3397=1135%:%
%:%3398=1135%:%
%:%3399=1135%:%
%:%3400=1136%:%
%:%3401=1137%:%
%:%3402=1137%:%
%:%3403=1137%:%
%:%3404=1138%:%
%:%3405=1139%:%
%:%3406=1139%:%
%:%3407=1140%:%
%:%3408=1141%:%
%:%3409=1141%:%
%:%3410=1142%:%
%:%3411=1142%:%
%:%3412=1143%:%
%:%3413=1143%:%
%:%3414=1144%:%
%:%3415=1144%:%
%:%3416=1145%:%
%:%3417=1145%:%
%:%3418=1146%:%
%:%3419=1147%:%
%:%3420=1147%:%
%:%3421=1148%:%
%:%3422=1148%:%
%:%3423=1148%:%
%:%3424=1149%:%
%:%3425=1149%:%
%:%3426=1149%:%
%:%3427=1150%:%
%:%3428=1150%:%
%:%3429=1150%:%
%:%3430=1151%:%
%:%3431=1151%:%
%:%3432=1152%:%
%:%3433=1152%:%
%:%3434=1152%:%
%:%3435=1153%:%
%:%3436=1153%:%
%:%3437=1154%:%
%:%3438=1154%:%
%:%3439=1155%:%
%:%3440=1155%:%
%:%3441=1156%:%
%:%3442=1156%:%
%:%3443=1157%:%
%:%3444=1157%:%
%:%3445=1158%:%
%:%3446=1159%:%
%:%3447=1159%:%
%:%3448=1160%:%
%:%3449=1160%:%
%:%3450=1160%:%
%:%3451=1161%:%
%:%3452=1161%:%
%:%3453=1161%:%
%:%3454=1162%:%
%:%3455=1162%:%
%:%3456=1162%:%
%:%3457=1163%:%
%:%3458=1163%:%
%:%3459=1164%:%
%:%3460=1164%:%
%:%3461=1164%:%
%:%3462=1165%:%
%:%3463=1165%:%
%:%3464=1166%:%
%:%3465=1166%:%
%:%3466=1167%:%
%:%3467=1167%:%
%:%3468=1168%:%
%:%3469=1168%:%
%:%3470=1169%:%
%:%3471=1169%:%
%:%3472=1170%:%
%:%3473=1170%:%
%:%3474=1170%:%
%:%3475=1171%:%
%:%3476=1171%:%
%:%3477=1172%:%
%:%3478=1172%:%
%:%3479=1173%:%
%:%3480=1173%:%
%:%3481=1174%:%
%:%3482=1174%:%
%:%3483=1174%:%
%:%3484=1175%:%
%:%3485=1176%:%
%:%3486=1176%:%
%:%3487=1177%:%
%:%3488=1177%:%
%:%3489=1178%:%
%:%3490=1178%:%
%:%3491=1179%:%
%:%3492=1179%:%
%:%3493=1180%:%
%:%3494=1180%:%
%:%3495=1180%:%
%:%3496=1181%:%
%:%3497=1181%:%
%:%3498=1182%:%
%:%3499=1182%:%
%:%3500=1183%:%
%:%3501=1183%:%
%:%3502=1184%:%
%:%3503=1184%:%
%:%3504=1184%:%
%:%3505=1185%:%
%:%3506=1185%:%
%:%3507=1185%:%
%:%3508=1186%:%
%:%3509=1186%:%
%:%3510=1186%:%
%:%3511=1187%:%
%:%3512=1187%:%
%:%3513=1188%:%
%:%3514=1188%:%
%:%3515=1189%:%
%:%3516=1189%:%
%:%3517=1190%:%
%:%3523=1190%:%
%:%3526=1191%:%
%:%3527=1192%:%
%:%3528=1192%:%
%:%3529=1193%:%
%:%3532=1194%:%
%:%3536=1194%:%
%:%3537=1194%:%
%:%3538=1195%:%
%:%3539=1195%:%
%:%3540=1196%:%
%:%3541=1196%:%
%:%3546=1196%:%
%:%3549=1197%:%
%:%3550=1198%:%
%:%3551=1198%:%
%:%3552=1199%:%
%:%3555=1200%:%
%:%3559=1200%:%
%:%3560=1200%:%
%:%3561=1201%:%
%:%3562=1201%:%
%:%3567=1201%:%
%:%3570=1202%:%
%:%3571=1203%:%
%:%3572=1203%:%
%:%3573=1204%:%
%:%3580=1205%:%
%:%3581=1205%:%
%:%3582=1206%:%
%:%3583=1206%:%
%:%3584=1207%:%
%:%3585=1207%:%
%:%3586=1208%:%
%:%3587=1208%:%
%:%3588=1208%:%
%:%3589=1209%:%
%:%3590=1209%:%
%:%3591=1210%:%
%:%3597=1210%:%
%:%3600=1211%:%
%:%3601=1212%:%
%:%3602=1212%:%
%:%3603=1213%:%
%:%3610=1214%:%
%:%3611=1214%:%
%:%3612=1215%:%
%:%3613=1215%:%
%:%3614=1216%:%
%:%3615=1216%:%
%:%3616=1217%:%
%:%3617=1217%:%
%:%3618=1217%:%
%:%3619=1218%:%
%:%3620=1218%:%
%:%3621=1219%:%
%:%3627=1219%:%
%:%3630=1220%:%
%:%3631=1221%:%
%:%3632=1221%:%
%:%3633=1222%:%
%:%3640=1223%:%
%:%3641=1223%:%
%:%3642=1224%:%
%:%3643=1224%:%
%:%3644=1225%:%
%:%3645=1225%:%
%:%3646=1226%:%
%:%3647=1226%:%
%:%3648=1226%:%
%:%3649=1227%:%
%:%3650=1227%:%
%:%3651=1228%:%
%:%3657=1228%:%
%:%3660=1229%:%
%:%3661=1230%:%
%:%3662=1230%:%
%:%3663=1231%:%
%:%3670=1232%:%
%:%3671=1232%:%
%:%3672=1233%:%
%:%3673=1233%:%
%:%3674=1234%:%
%:%3675=1234%:%
%:%3676=1234%:%
%:%3677=1235%:%
%:%3678=1235%:%
%:%3679=1235%:%
%:%3680=1236%:%
%:%3681=1236%:%
%:%3682=1236%:%
%:%3683=1237%:%
%:%3684=1237%:%
%:%3685=1238%:%
%:%3686=1238%:%
%:%3687=1239%:%
%:%3688=1239%:%
%:%3689=1240%:%
%:%3690=1240%:%
%:%3691=1241%:%
%:%3692=1241%:%
%:%3693=1242%:%
%:%3694=1242%:%
%:%3695=1243%:%
%:%3696=1243%:%
%:%3697=1244%:%
%:%3698=1244%:%
%:%3699=1245%:%
%:%3700=1245%:%
%:%3701=1245%:%
%:%3702=1246%:%
%:%3703=1246%:%
%:%3704=1246%:%
%:%3705=1247%:%
%:%3706=1247%:%
%:%3707=1247%:%
%:%3708=1248%:%
%:%3709=1248%:%
%:%3710=1248%:%
%:%3711=1249%:%
%:%3712=1249%:%
%:%3713=1249%:%
%:%3714=1250%:%
%:%3715=1250%:%
%:%3716=1250%:%
%:%3717=1251%:%
%:%3718=1251%:%
%:%3719=1252%:%
%:%3720=1252%:%
%:%3721=1253%:%
%:%3722=1253%:%
%:%3723=1254%:%
%:%3724=1254%:%
%:%3725=1254%:%
%:%3726=1255%:%
%:%3727=1256%:%
%:%3728=1256%:%
%:%3729=1257%:%
%:%3730=1257%:%
%:%3731=1258%:%
%:%3732=1258%:%
%:%3733=1259%:%
%:%3734=1259%:%
%:%3735=1260%:%
%:%3736=1260%:%
%:%3737=1261%:%
%:%3738=1261%:%
%:%3739=1262%:%
%:%3740=1262%:%
%:%3741=1263%:%
%:%3742=1263%:%
%:%3743=1264%:%
%:%3744=1264%:%
%:%3745=1264%:%
%:%3746=1265%:%
%:%3747=1265%:%
%:%3748=1265%:%
%:%3749=1266%:%
%:%3750=1266%:%
%:%3751=1266%:%
%:%3752=1267%:%
%:%3753=1267%:%
%:%3754=1268%:%
%:%3755=1268%:%
%:%3756=1268%:%
%:%3757=1269%:%
%:%3758=1269%:%
%:%3759=1269%:%
%:%3760=1270%:%
%:%3761=1270%:%
%:%3762=1271%:%
%:%3763=1271%:%
%:%3764=1272%:%
%:%3765=1272%:%
%:%3766=1273%:%
%:%3767=1273%:%
%:%3768=1273%:%
%:%3769=1274%:%
%:%3770=1274%:%
%:%3771=1274%:%
%:%3772=1275%:%
%:%3773=1275%:%
%:%3774=1276%:%
%:%3775=1276%:%
%:%3776=1277%:%
%:%3777=1277%:%
%:%3778=1278%:%
%:%3779=1278%:%
%:%3780=1279%:%
%:%3781=1279%:%
%:%3782=1279%:%
%:%3783=1280%:%
%:%3784=1280%:%
%:%3785=1281%:%
%:%3786=1281%:%
%:%3787=1281%:%
%:%3788=1282%:%
%:%3789=1282%:%
%:%3790=1282%:%
%:%3791=1283%:%
%:%3792=1283%:%
%:%3793=1283%:%
%:%3794=1284%:%
%:%3795=1284%:%
%:%3796=1284%:%
%:%3797=1285%:%
%:%3798=1285%:%
%:%3799=1286%:%
%:%3800=1286%:%
%:%3801=1287%:%
%:%3802=1287%:%
%:%3803=1288%:%
%:%3809=1288%:%
%:%3812=1289%:%
%:%3813=1290%:%
%:%3814=1290%:%
%:%3815=1291%:%
%:%3816=1292%:%
%:%3817=1293%:%
%:%3818=1294%:%
%:%3821=1295%:%
%:%3825=1295%:%
%:%3826=1295%:%
%:%3835=1297%:%
%:%3837=1298%:%
%:%3838=1298%:%
%:%3839=1299%:%
%:%3846=1300%:%
%:%3847=1300%:%
%:%3848=1301%:%
%:%3849=1301%:%
%:%3850=1302%:%
%:%3851=1302%:%
%:%3852=1303%:%
%:%3853=1303%:%
%:%3854=1303%:%
%:%3855=1304%:%
%:%3856=1304%:%
%:%3857=1305%:%
%:%3858=1305%:%
%:%3859=1306%:%
%:%3860=1306%:%
%:%3861=1307%:%
%:%3862=1307%:%
%:%3863=1308%:%
%:%3864=1308%:%
%:%3865=1309%:%
%:%3866=1309%:%
%:%3867=1310%:%
%:%3868=1310%:%
%:%3869=1311%:%
%:%3870=1311%:%
%:%3871=1312%:%
%:%3872=1312%:%
%:%3873=1313%:%
%:%3874=1313%:%
%:%3875=1314%:%
%:%3876=1315%:%
%:%3877=1315%:%
%:%3878=1315%:%
%:%3879=1316%:%
%:%3880=1316%:%
%:%3881=1316%:%
%:%3882=1317%:%
%:%3883=1317%:%
%:%3884=1317%:%
%:%3885=1318%:%
%:%3886=1318%:%
%:%3887=1318%:%
%:%3888=1319%:%
%:%3889=1319%:%
%:%3890=1319%:%
%:%3891=1320%:%
%:%3892=1320%:%
%:%3893=1320%:%
%:%3894=1321%:%
%:%3895=1321%:%
%:%3896=1321%:%
%:%3897=1322%:%
%:%3898=1322%:%
%:%3899=1322%:%
%:%3900=1323%:%
%:%3901=1323%:%
%:%3902=1324%:%
%:%3903=1324%:%
%:%3904=1324%:%
%:%3905=1325%:%
%:%3906=1325%:%
%:%3907=1326%:%
%:%3908=1326%:%
%:%3909=1326%:%
%:%3910=1327%:%
%:%3911=1327%:%
%:%3912=1328%:%
%:%3913=1328%:%
%:%3914=1329%:%
%:%3915=1329%:%
%:%3916=1330%:%
%:%3917=1330%:%
%:%3918=1331%:%
%:%3919=1331%:%
%:%3920=1331%:%
%:%3921=1332%:%
%:%3922=1332%:%
%:%3923=1333%:%
%:%3924=1333%:%
%:%3925=1334%:%
%:%3926=1334%:%
%:%3927=1335%:%
%:%3928=1335%:%
%:%3929=1336%:%
%:%3930=1337%:%
%:%3931=1337%:%
%:%3932=1337%:%
%:%3933=1338%:%
%:%3934=1338%:%
%:%3935=1338%:%
%:%3936=1339%:%
%:%3937=1339%:%
%:%3938=1339%:%
%:%3939=1340%:%
%:%3940=1340%:%
%:%3941=1340%:%
%:%3942=1341%:%
%:%3943=1341%:%
%:%3944=1341%:%
%:%3945=1342%:%
%:%3946=1342%:%
%:%3947=1342%:%
%:%3948=1343%:%
%:%3949=1343%:%
%:%3950=1343%:%
%:%3951=1344%:%
%:%3952=1344%:%
%:%3953=1344%:%
%:%3954=1345%:%
%:%3955=1345%:%
%:%3956=1346%:%
%:%3957=1346%:%
%:%3958=1346%:%
%:%3959=1347%:%
%:%3960=1347%:%
%:%3961=1348%:%
%:%3962=1348%:%
%:%3963=1348%:%
%:%3964=1349%:%
%:%3965=1349%:%
%:%3966=1350%:%
%:%3967=1350%:%
%:%3968=1351%:%
%:%3969=1351%:%
%:%3970=1352%:%
%:%3971=1352%:%
%:%3972=1353%:%
%:%3973=1353%:%
%:%3974=1354%:%
%:%3975=1354%:%
%:%3976=1354%:%
%:%3977=1355%:%
%:%3978=1355%:%
%:%3979=1355%:%
%:%3980=1356%:%
%:%3981=1356%:%
%:%3982=1357%:%
%:%3983=1357%:%
%:%3984=1358%:%
%:%3985=1358%:%
%:%3986=1359%:%
%:%3987=1359%:%
%:%3988=1360%:%
%:%3989=1360%:%
%:%3990=1361%:%
%:%3991=1361%:%
%:%3992=1362%:%
%:%3993=1363%:%
%:%3994=1363%:%
%:%3995=1363%:%
%:%3996=1364%:%
%:%3997=1364%:%
%:%3998=1364%:%
%:%3999=1365%:%
%:%4000=1365%:%
%:%4001=1365%:%
%:%4002=1366%:%
%:%4003=1366%:%
%:%4004=1366%:%
%:%4005=1367%:%
%:%4006=1367%:%
%:%4007=1367%:%
%:%4008=1368%:%
%:%4009=1368%:%
%:%4010=1368%:%
%:%4011=1369%:%
%:%4012=1369%:%
%:%4013=1370%:%
%:%4014=1370%:%
%:%4015=1370%:%
%:%4016=1371%:%
%:%4017=1371%:%
%:%4018=1372%:%
%:%4019=1372%:%
%:%4020=1372%:%
%:%4021=1373%:%
%:%4022=1373%:%
%:%4023=1374%:%
%:%4024=1374%:%
%:%4025=1374%:%
%:%4026=1375%:%
%:%4027=1375%:%
%:%4028=1376%:%
%:%4029=1376%:%
%:%4030=1377%:%
%:%4031=1377%:%
%:%4032=1378%:%
%:%4033=1378%:%
%:%4034=1379%:%
%:%4035=1379%:%
%:%4036=1379%:%
%:%4037=1380%:%
%:%4038=1380%:%
%:%4039=1381%:%
%:%4040=1381%:%
%:%4041=1382%:%
%:%4042=1382%:%
%:%4043=1383%:%
%:%4044=1383%:%
%:%4045=1384%:%
%:%4046=1385%:%
%:%4047=1385%:%
%:%4048=1385%:%
%:%4049=1386%:%
%:%4050=1386%:%
%:%4051=1386%:%
%:%4052=1387%:%
%:%4053=1387%:%
%:%4054=1387%:%
%:%4055=1388%:%
%:%4056=1388%:%
%:%4057=1388%:%
%:%4058=1389%:%
%:%4059=1389%:%
%:%4060=1389%:%
%:%4061=1390%:%
%:%4062=1390%:%
%:%4063=1390%:%
%:%4064=1391%:%
%:%4065=1391%:%
%:%4066=1392%:%
%:%4067=1392%:%
%:%4068=1392%:%
%:%4069=1393%:%
%:%4070=1393%:%
%:%4071=1394%:%
%:%4072=1394%:%
%:%4073=1394%:%
%:%4074=1395%:%
%:%4075=1395%:%
%:%4076=1396%:%
%:%4077=1396%:%
%:%4078=1396%:%
%:%4079=1397%:%
%:%4080=1397%:%
%:%4081=1398%:%
%:%4082=1398%:%
%:%4083=1399%:%
%:%4084=1399%:%
%:%4085=1400%:%
%:%4086=1400%:%
%:%4087=1401%:%
%:%4093=1401%:%
%:%4096=1402%:%
%:%4097=1403%:%
%:%4098=1403%:%
%:%4099=1404%:%
%:%4106=1405%:%
%:%4107=1405%:%
%:%4108=1406%:%
%:%4109=1406%:%
%:%4110=1407%:%
%:%4111=1407%:%
%:%4112=1408%:%
%:%4113=1408%:%
%:%4114=1408%:%
%:%4115=1409%:%
%:%4116=1409%:%
%:%4117=1409%:%
%:%4118=1410%:%
%:%4119=1410%:%
%:%4120=1411%:%
%:%4121=1411%:%
%:%4122=1412%:%
%:%4123=1412%:%
%:%4124=1413%:%
%:%4125=1413%:%
%:%4126=1414%:%
%:%4127=1414%:%
%:%4128=1415%:%
%:%4129=1415%:%
%:%4130=1416%:%
%:%4131=1416%:%
%:%4132=1417%:%
%:%4133=1417%:%
%:%4134=1417%:%
%:%4135=1418%:%
%:%4136=1418%:%
%:%4137=1418%:%
%:%4138=1419%:%
%:%4139=1419%:%
%:%4140=1420%:%
%:%4141=1420%:%
%:%4142=1420%:%
%:%4143=1421%:%
%:%4144=1421%:%
%:%4145=1422%:%
%:%4146=1422%:%
%:%4147=1422%:%
%:%4148=1423%:%
%:%4149=1423%:%
%:%4150=1424%:%
%:%4151=1424%:%
%:%4152=1424%:%
%:%4153=1425%:%
%:%4154=1425%:%
%:%4155=1426%:%
%:%4156=1426%:%
%:%4157=1426%:%
%:%4158=1427%:%
%:%4159=1427%:%
%:%4160=1428%:%
%:%4161=1428%:%
%:%4162=1429%:%
%:%4163=1429%:%
%:%4164=1430%:%
%:%4165=1430%:%
%:%4166=1430%:%
%:%4167=1431%:%
%:%4168=1431%:%
%:%4169=1432%:%
%:%4170=1432%:%
%:%4171=1433%:%
%:%4172=1433%:%
%:%4173=1433%:%
%:%4174=1434%:%
%:%4175=1434%:%
%:%4176=1434%:%
%:%4177=1435%:%
%:%4178=1435%:%
%:%4179=1436%:%
%:%4180=1436%:%
%:%4181=1436%:%
%:%4182=1437%:%
%:%4183=1437%:%
%:%4184=1438%:%
%:%4185=1438%:%
%:%4186=1438%:%
%:%4187=1439%:%
%:%4188=1439%:%
%:%4189=1440%:%
%:%4190=1440%:%
%:%4191=1440%:%
%:%4192=1441%:%
%:%4193=1441%:%
%:%4194=1442%:%
%:%4195=1442%:%
%:%4196=1442%:%
%:%4197=1443%:%
%:%4198=1443%:%
%:%4199=1444%:%
%:%4200=1444%:%
%:%4201=1445%:%
%:%4202=1445%:%
%:%4203=1446%:%
%:%4204=1446%:%
%:%4205=1447%:%
%:%4206=1447%:%
%:%4207=1447%:%
%:%4208=1448%:%
%:%4209=1448%:%
%:%4210=1449%:%
%:%4211=1449%:%
%:%4212=1450%:%
%:%4213=1450%:%
%:%4214=1451%:%
%:%4215=1451%:%
%:%4216=1452%:%
%:%4217=1452%:%
%:%4218=1453%:%
%:%4219=1453%:%
%:%4220=1453%:%
%:%4221=1454%:%
%:%4222=1454%:%
%:%4223=1454%:%
%:%4224=1455%:%
%:%4225=1455%:%
%:%4226=1456%:%
%:%4227=1456%:%
%:%4228=1456%:%
%:%4229=1457%:%
%:%4230=1457%:%
%:%4231=1457%:%
%:%4232=1458%:%
%:%4233=1458%:%
%:%4234=1458%:%
%:%4235=1459%:%
%:%4236=1459%:%
%:%4237=1460%:%
%:%4238=1460%:%
%:%4239=1460%:%
%:%4240=1461%:%
%:%4241=1461%:%
%:%4242=1462%:%
%:%4243=1462%:%
%:%4244=1462%:%
%:%4245=1463%:%
%:%4246=1463%:%
%:%4247=1464%:%
%:%4248=1464%:%
%:%4249=1465%:%
%:%4250=1465%:%
%:%4251=1466%:%
%:%4252=1466%:%
%:%4253=1466%:%
%:%4254=1467%:%
%:%4255=1467%:%
%:%4256=1468%:%
%:%4257=1468%:%
%:%4258=1469%:%
%:%4259=1469%:%
%:%4260=1469%:%
%:%4261=1470%:%
%:%4262=1470%:%
%:%4263=1470%:%
%:%4264=1471%:%
%:%4265=1471%:%
%:%4266=1472%:%
%:%4267=1472%:%
%:%4268=1472%:%
%:%4269=1473%:%
%:%4270=1473%:%
%:%4271=1473%:%
%:%4272=1474%:%
%:%4273=1474%:%
%:%4274=1474%:%
%:%4275=1475%:%
%:%4276=1475%:%
%:%4277=1476%:%
%:%4278=1476%:%
%:%4279=1476%:%
%:%4280=1477%:%
%:%4281=1477%:%
%:%4282=1478%:%
%:%4283=1478%:%
%:%4284=1478%:%
%:%4285=1479%:%
%:%4286=1479%:%
%:%4287=1480%:%
%:%4288=1480%:%
%:%4289=1481%:%
%:%4290=1481%:%
%:%4291=1482%:%
%:%4297=1482%:%
%:%4300=1483%:%
%:%4301=1484%:%
%:%4302=1484%:%
%:%4303=1485%:%
%:%4304=1486%:%
%:%4305=1487%:%
%:%4306=1488%:%
%:%4313=1489%:%
%:%4314=1489%:%
%:%4315=1490%:%
%:%4316=1490%:%
%:%4317=1491%:%
%:%4318=1492%:%
%:%4319=1492%:%
%:%4320=1492%:%
%:%4321=1493%:%
%:%4322=1493%:%
%:%4323=1494%:%
%:%4324=1494%:%
%:%4325=1495%:%
%:%4326=1496%:%
%:%4327=1496%:%
%:%4328=1497%:%
%:%4329=1497%:%
%:%4330=1497%:%
%:%4331=1498%:%
%:%4332=1498%:%
%:%4333=1498%:%
%:%4334=1499%:%
%:%4335=1499%:%
%:%4336=1499%:%
%:%4337=1500%:%
%:%4338=1500%:%
%:%4339=1500%:%
%:%4340=1501%:%
%:%4341=1502%:%
%:%4342=1502%:%
%:%4343=1503%:%
%:%4344=1503%:%
%:%4345=1503%:%
%:%4346=1504%:%
%:%4347=1504%:%
%:%4348=1505%:%
%:%4349=1506%:%
%:%4350=1506%:%
%:%4351=1507%:%
%:%4352=1507%:%
%:%4353=1508%:%
%:%4354=1508%:%
%:%4355=1509%:%
%:%4356=1509%:%
%:%4357=1510%:%
%:%4358=1510%:%
%:%4359=1511%:%
%:%4360=1511%:%
%:%4361=1511%:%
%:%4362=1512%:%
%:%4363=1513%:%
%:%4364=1513%:%
%:%4366=1515%:%
%:%4367=1516%:%
%:%4368=1516%:%
%:%4369=1517%:%
%:%4370=1517%:%
%:%4371=1517%:%
%:%4372=1518%:%
%:%4373=1518%:%
%:%4374=1519%:%
%:%4375=1519%:%
%:%4376=1520%:%
%:%4377=1520%:%
%:%4378=1520%:%
%:%4379=1521%:%
%:%4380=1521%:%
%:%4381=1522%:%
%:%4382=1522%:%
%:%4383=1522%:%
%:%4384=1523%:%
%:%4385=1523%:%
%:%4386=1524%:%
%:%4387=1524%:%
%:%4388=1524%:%
%:%4389=1525%:%
%:%4390=1525%:%
%:%4391=1526%:%
%:%4392=1526%:%
%:%4393=1526%:%
%:%4394=1527%:%
%:%4395=1527%:%
%:%4396=1528%:%
%:%4397=1528%:%
%:%4398=1529%:%
%:%4399=1529%:%
%:%4400=1530%:%
%:%4401=1530%:%
%:%4402=1530%:%
%:%4403=1531%:%
%:%4404=1531%:%
%:%4405=1532%:%
%:%4406=1532%:%
%:%4407=1532%:%
%:%4408=1533%:%
%:%4409=1533%:%
%:%4410=1534%:%
%:%4411=1534%:%
%:%4412=1534%:%
%:%4413=1535%:%
%:%4414=1535%:%
%:%4415=1536%:%
%:%4416=1536%:%
%:%4417=1536%:%
%:%4418=1537%:%
%:%4419=1537%:%
%:%4420=1538%:%
%:%4421=1538%:%
%:%4422=1538%:%
%:%4423=1539%:%
%:%4424=1539%:%
%:%4425=1539%:%
%:%4426=1540%:%
%:%4427=1540%:%
%:%4428=1540%:%
%:%4429=1541%:%
%:%4430=1541%:%
%:%4431=1541%:%
%:%4432=1542%:%
%:%4433=1542%:%
%:%4434=1542%:%
%:%4435=1543%:%
%:%4436=1543%:%
%:%4437=1544%:%
%:%4438=1544%:%
%:%4439=1545%:%
%:%4440=1545%:%
%:%4441=1546%:%
%:%4442=1546%:%
%:%4443=1546%:%
%:%4444=1547%:%
%:%4445=1547%:%
%:%4446=1548%:%
%:%4447=1548%:%
%:%4448=1548%:%
%:%4449=1549%:%
%:%4450=1549%:%
%:%4451=1550%:%
%:%4452=1550%:%
%:%4453=1550%:%
%:%4454=1551%:%
%:%4455=1551%:%
%:%4456=1552%:%
%:%4457=1552%:%
%:%4458=1552%:%
%:%4459=1553%:%
%:%4460=1553%:%
%:%4461=1554%:%
%:%4462=1554%:%
%:%4463=1554%:%
%:%4464=1555%:%
%:%4465=1555%:%
%:%4466=1556%:%
%:%4467=1556%:%
%:%4468=1557%:%
%:%4469=1557%:%
%:%4470=1558%:%
%:%4476=1558%:%
%:%4479=1559%:%
%:%4480=1560%:%
%:%4481=1560%:%
%:%4482=1561%:%
%:%4483=1562%:%
%:%4484=1563%:%
%:%4485=1564%:%
%:%4488=1565%:%
%:%4492=1565%:%
%:%4493=1565%:%
%:%4494=1565%:%
%:%4499=1565%:%
%:%4502=1566%:%
%:%4503=1567%:%
%:%4504=1567%:%
%:%4505=1568%:%
%:%4506=1569%:%
%:%4507=1570%:%
%:%4514=1571%:%
%:%4515=1571%:%
%:%4516=1572%:%
%:%4517=1572%:%
%:%4518=1573%:%
%:%4519=1573%:%
%:%4520=1573%:%
%:%4521=1574%:%
%:%4522=1574%:%
%:%4523=1575%:%
%:%4524=1575%:%
%:%4525=1576%:%
%:%4526=1576%:%
%:%4527=1576%:%
%:%4528=1577%:%
%:%4529=1577%:%
%:%4530=1577%:%
%:%4531=1578%:%
%:%4532=1578%:%
%:%4533=1578%:%
%:%4534=1579%:%
%:%4535=1579%:%
%:%4536=1580%:%
%:%4542=1580%:%
%:%4545=1581%:%
%:%4546=1582%:%
%:%4547=1582%:%
%:%4548=1583%:%
%:%4549=1584%:%
%:%4550=1585%:%
%:%4553=1586%:%
%:%4557=1586%:%
%:%4558=1586%:%
%:%4559=1586%:%
%:%4564=1586%:%
%:%4567=1587%:%
%:%4568=1588%:%
%:%4569=1588%:%
%:%4570=1589%:%
%:%4571=1590%:%
%:%4574=1591%:%
%:%4578=1591%:%
%:%4579=1591%:%
%:%4593=1593%:%
%:%4603=1595%:%
%:%4604=1595%:%
%:%4605=1596%:%
%:%4606=1597%:%
%:%4607=1598%:%
%:%4608=1599%:%
%:%4611=1600%:%
%:%4615=1600%:%
%:%4616=1600%:%
%:%4621=1600%:%
%:%4624=1601%:%
%:%4625=1602%:%
%:%4626=1602%:%
%:%4627=1603%:%
%:%4628=1604%:%
%:%4629=1605%:%
%:%4630=1606%:%
%:%4633=1607%:%
%:%4637=1607%:%
%:%4638=1607%:%
%:%4643=1607%:%
%:%4646=1608%:%
%:%4647=1609%:%
%:%4648=1609%:%
%:%4649=1610%:%
%:%4650=1611%:%
%:%4651=1612%:%
%:%4654=1613%:%
%:%4658=1613%:%
%:%4659=1613%:%
%:%4664=1613%:%
%:%4667=1614%:%
%:%4668=1615%:%
%:%4669=1615%:%
%:%4670=1616%:%
%:%4671=1617%:%
%:%4672=1618%:%
%:%4675=1619%:%
%:%4679=1619%:%
%:%4680=1619%:%
%:%4685=1619%:%
%:%4688=1620%:%
%:%4689=1621%:%
%:%4690=1621%:%
%:%4691=1622%:%
%:%4692=1623%:%
%:%4693=1624%:%
%:%4696=1625%:%
%:%4700=1625%:%
%:%4701=1625%:%
%:%4706=1625%:%
%:%4709=1626%:%
%:%4710=1627%:%
%:%4711=1627%:%
%:%4712=1628%:%
%:%4713=1629%:%
%:%4714=1630%:%
%:%4717=1631%:%
%:%4721=1631%:%
%:%4722=1631%:%
%:%4727=1631%:%
%:%4732=1632%:%
%:%4737=1633%:%

%
\begin{isabellebody}%
\setisabellecontext{Quant{\isacharunderscore}{\kern0pt}Logic}%
%
\isadelimdocument
%
\endisadelimdocument
%
\isatagdocument
%
\isamarkupsection{Quantifiers%
}
\isamarkuptrue%
%
\endisatagdocument
{\isafolddocument}%
%
\isadelimdocument
%
\endisadelimdocument
%
\isadelimtheory
%
\endisadelimtheory
%
\isatagtheory
\isacommand{theory}\isamarkupfalse%
\ Quant{\isacharunderscore}{\kern0pt}Logic\isanewline
\ \ \isakeyword{imports}\ Pred{\isacharunderscore}{\kern0pt}Logic\ Exponential{\isacharunderscore}{\kern0pt}Objects\isanewline
\isakeyword{begin}%
\endisatagtheory
{\isafoldtheory}%
%
\isadelimtheory
%
\endisadelimtheory
%
\isadelimdocument
%
\endisadelimdocument
%
\isatagdocument
%
\isamarkupsubsection{Universal Quantification%
}
\isamarkuptrue%
%
\endisatagdocument
{\isafolddocument}%
%
\isadelimdocument
%
\endisadelimdocument
\isacommand{definition}\isamarkupfalse%
\ FORALL\ {\isacharcolon}{\kern0pt}{\isacharcolon}{\kern0pt}\ {\isachardoublequoteopen}cset\ {\isasymRightarrow}\ cfunc{\isachardoublequoteclose}\ \isakeyword{where}\isanewline
\ \ {\isachardoublequoteopen}FORALL\ X\ {\isacharequal}{\kern0pt}\ {\isacharparenleft}{\kern0pt}THE\ {\isasymchi}{\isachardot}{\kern0pt}\ is{\isacharunderscore}{\kern0pt}pullback\ {\isasymone}\ {\isasymone}\ {\isacharparenleft}{\kern0pt}{\isasymOmega}\isactrlbsup X\isactrlesup {\isacharparenright}{\kern0pt}\ {\isasymOmega}\ {\isacharparenleft}{\kern0pt}{\isasymbeta}\isactrlbsub {\isasymone}\isactrlesub {\isacharparenright}{\kern0pt}\ {\isasymt}\ {\isacharparenleft}{\kern0pt}{\isacharparenleft}{\kern0pt}{\isasymt}\ {\isasymcirc}\isactrlsub c\ {\isasymbeta}\isactrlbsub X\ {\isasymtimes}\isactrlsub c\ {\isasymone}\isactrlesub {\isacharparenright}{\kern0pt}\isactrlsup {\isasymsharp}{\isacharparenright}{\kern0pt}\ {\isasymchi}{\isacharparenright}{\kern0pt}{\isachardoublequoteclose}\isanewline
\isanewline
\isacommand{lemma}\isamarkupfalse%
\ FORALL{\isacharunderscore}{\kern0pt}is{\isacharunderscore}{\kern0pt}pullback{\isacharcolon}{\kern0pt}\isanewline
\ \ {\isachardoublequoteopen}is{\isacharunderscore}{\kern0pt}pullback\ {\isasymone}\ {\isasymone}\ {\isacharparenleft}{\kern0pt}{\isasymOmega}\isactrlbsup X\isactrlesup {\isacharparenright}{\kern0pt}\ {\isasymOmega}\ {\isacharparenleft}{\kern0pt}{\isasymbeta}\isactrlbsub {\isasymone}\isactrlesub {\isacharparenright}{\kern0pt}\ {\isasymt}\ {\isacharparenleft}{\kern0pt}{\isacharparenleft}{\kern0pt}{\isasymt}\ {\isasymcirc}\isactrlsub c\ {\isasymbeta}\isactrlbsub X\ {\isasymtimes}\isactrlsub c\ {\isasymone}\isactrlesub {\isacharparenright}{\kern0pt}\isactrlsup {\isasymsharp}{\isacharparenright}{\kern0pt}\ {\isacharparenleft}{\kern0pt}FORALL\ X{\isacharparenright}{\kern0pt}{\isachardoublequoteclose}\isanewline
%
\isadelimproof
\ \ %
\endisadelimproof
%
\isatagproof
\isacommand{unfolding}\isamarkupfalse%
\ FORALL{\isacharunderscore}{\kern0pt}def\isanewline
\ \ \isacommand{using}\isamarkupfalse%
\ characteristic{\isacharunderscore}{\kern0pt}function{\isacharunderscore}{\kern0pt}exists\ element{\isacharunderscore}{\kern0pt}monomorphism\isanewline
\ \ \isacommand{by}\isamarkupfalse%
\ {\isacharparenleft}{\kern0pt}typecheck{\isacharunderscore}{\kern0pt}cfuncs{\isacharcomma}{\kern0pt}\ simp\ add{\isacharcolon}{\kern0pt}\ the{\isadigit{1}}I{\isadigit{2}}{\isacharparenright}{\kern0pt}%
\endisatagproof
{\isafoldproof}%
%
\isadelimproof
\isanewline
%
\endisadelimproof
\isanewline
\isacommand{lemma}\isamarkupfalse%
\ FORALL{\isacharunderscore}{\kern0pt}type{\isacharbrackleft}{\kern0pt}type{\isacharunderscore}{\kern0pt}rule{\isacharbrackright}{\kern0pt}{\isacharcolon}{\kern0pt}\isanewline
\ \ {\isachardoublequoteopen}FORALL\ X\ {\isacharcolon}{\kern0pt}\ {\isasymOmega}\isactrlbsup X\isactrlesup \ {\isasymrightarrow}\ {\isasymOmega}{\isachardoublequoteclose}\isanewline
%
\isadelimproof
\ \ %
\endisadelimproof
%
\isatagproof
\isacommand{using}\isamarkupfalse%
\ FORALL{\isacharunderscore}{\kern0pt}is{\isacharunderscore}{\kern0pt}pullback\ \isacommand{unfolding}\isamarkupfalse%
\ is{\isacharunderscore}{\kern0pt}pullback{\isacharunderscore}{\kern0pt}def\ \ \isacommand{by}\isamarkupfalse%
\ auto%
\endisatagproof
{\isafoldproof}%
%
\isadelimproof
\isanewline
%
\endisadelimproof
\isanewline
\isacommand{lemma}\isamarkupfalse%
\ all{\isacharunderscore}{\kern0pt}true{\isacharunderscore}{\kern0pt}implies{\isacharunderscore}{\kern0pt}FORALL{\isacharunderscore}{\kern0pt}true{\isacharcolon}{\kern0pt}\isanewline
\ \ \isakeyword{assumes}\ p{\isacharunderscore}{\kern0pt}type{\isacharbrackleft}{\kern0pt}type{\isacharunderscore}{\kern0pt}rule{\isacharbrackright}{\kern0pt}{\isacharcolon}{\kern0pt}\ {\isachardoublequoteopen}p\ {\isacharcolon}{\kern0pt}\ X\ {\isasymrightarrow}\ {\isasymOmega}{\isachardoublequoteclose}\ \isakeyword{and}\ all{\isacharunderscore}{\kern0pt}p{\isacharunderscore}{\kern0pt}true{\isacharcolon}{\kern0pt}\ {\isachardoublequoteopen}{\isasymAnd}\ x{\isachardot}{\kern0pt}\ x\ {\isasymin}\isactrlsub c\ X\ {\isasymLongrightarrow}\ p\ {\isasymcirc}\isactrlsub c\ x\ {\isacharequal}{\kern0pt}\ {\isasymt}{\isachardoublequoteclose}\isanewline
\ \ \isakeyword{shows}\ {\isachardoublequoteopen}FORALL\ X\ {\isasymcirc}\isactrlsub c\ {\isacharparenleft}{\kern0pt}p\ {\isasymcirc}\isactrlsub c\ left{\isacharunderscore}{\kern0pt}cart{\isacharunderscore}{\kern0pt}proj\ X\ {\isasymone}{\isacharparenright}{\kern0pt}\isactrlsup {\isasymsharp}\ {\isacharequal}{\kern0pt}\ {\isasymt}{\isachardoublequoteclose}\isanewline
%
\isadelimproof
%
\endisadelimproof
%
\isatagproof
\isacommand{proof}\isamarkupfalse%
\ {\isacharminus}{\kern0pt}\isanewline
\ \ \isacommand{have}\isamarkupfalse%
\ {\isachardoublequoteopen}p\ {\isasymcirc}\isactrlsub c\ left{\isacharunderscore}{\kern0pt}cart{\isacharunderscore}{\kern0pt}proj\ X\ {\isasymone}\ {\isacharequal}{\kern0pt}\ {\isasymt}\ {\isasymcirc}\isactrlsub c\ {\isasymbeta}\isactrlbsub X\ {\isasymtimes}\isactrlsub c\ {\isasymone}\isactrlesub {\isachardoublequoteclose}\isanewline
\ \ \isacommand{proof}\isamarkupfalse%
\ {\isacharparenleft}{\kern0pt}etcs{\isacharunderscore}{\kern0pt}rule\ one{\isacharunderscore}{\kern0pt}separator{\isacharparenright}{\kern0pt}\isanewline
\ \ \ \ \isacommand{fix}\isamarkupfalse%
\ x\isanewline
\ \ \ \ \isacommand{assume}\isamarkupfalse%
\ x{\isacharunderscore}{\kern0pt}type{\isacharcolon}{\kern0pt}\ {\isachardoublequoteopen}x\ {\isasymin}\isactrlsub c\ X\ {\isasymtimes}\isactrlsub c\ {\isasymone}{\isachardoublequoteclose}\isanewline
\isanewline
\ \ \ \ \isacommand{have}\isamarkupfalse%
\ {\isachardoublequoteopen}{\isacharparenleft}{\kern0pt}p\ {\isasymcirc}\isactrlsub c\ left{\isacharunderscore}{\kern0pt}cart{\isacharunderscore}{\kern0pt}proj\ X\ {\isasymone}{\isacharparenright}{\kern0pt}\ {\isasymcirc}\isactrlsub c\ x\ {\isacharequal}{\kern0pt}\ p\ {\isasymcirc}\isactrlsub c\ {\isacharparenleft}{\kern0pt}left{\isacharunderscore}{\kern0pt}cart{\isacharunderscore}{\kern0pt}proj\ X\ {\isasymone}\ {\isasymcirc}\isactrlsub c\ x{\isacharparenright}{\kern0pt}{\isachardoublequoteclose}\isanewline
\ \ \ \ \ \ \isacommand{using}\isamarkupfalse%
\ x{\isacharunderscore}{\kern0pt}type\ p{\isacharunderscore}{\kern0pt}type\ comp{\isacharunderscore}{\kern0pt}associative{\isadigit{2}}\ \isacommand{by}\isamarkupfalse%
\ {\isacharparenleft}{\kern0pt}typecheck{\isacharunderscore}{\kern0pt}cfuncs{\isacharcomma}{\kern0pt}\ auto{\isacharparenright}{\kern0pt}\isanewline
\ \ \ \ \isacommand{also}\isamarkupfalse%
\ \isacommand{have}\isamarkupfalse%
\ {\isachardoublequoteopen}{\isachardot}{\kern0pt}{\isachardot}{\kern0pt}{\isachardot}{\kern0pt}\ {\isacharequal}{\kern0pt}\ {\isasymt}{\isachardoublequoteclose}\isanewline
\ \ \ \ \ \ \isacommand{using}\isamarkupfalse%
\ x{\isacharunderscore}{\kern0pt}type\ all{\isacharunderscore}{\kern0pt}p{\isacharunderscore}{\kern0pt}true\ \isacommand{by}\isamarkupfalse%
\ {\isacharparenleft}{\kern0pt}typecheck{\isacharunderscore}{\kern0pt}cfuncs{\isacharcomma}{\kern0pt}\ auto{\isacharparenright}{\kern0pt}\isanewline
\ \ \ \ \isacommand{also}\isamarkupfalse%
\ \isacommand{have}\isamarkupfalse%
\ {\isachardoublequoteopen}{\isachardot}{\kern0pt}{\isachardot}{\kern0pt}{\isachardot}{\kern0pt}\ {\isacharequal}{\kern0pt}\ {\isasymt}\ {\isasymcirc}\isactrlsub c\ {\isasymbeta}\isactrlbsub X\ {\isasymtimes}\isactrlsub c\ {\isasymone}\isactrlesub \ {\isasymcirc}\isactrlsub c\ x\ {\isachardoublequoteclose}\isanewline
\ \ \ \ \ \ \isacommand{using}\isamarkupfalse%
\ x{\isacharunderscore}{\kern0pt}type\ \isacommand{by}\isamarkupfalse%
\ {\isacharparenleft}{\kern0pt}typecheck{\isacharunderscore}{\kern0pt}cfuncs{\isacharcomma}{\kern0pt}\ metis\ id{\isacharunderscore}{\kern0pt}right{\isacharunderscore}{\kern0pt}unit{\isadigit{2}}\ id{\isacharunderscore}{\kern0pt}type\ one{\isacharunderscore}{\kern0pt}unique{\isacharunderscore}{\kern0pt}element{\isacharparenright}{\kern0pt}\isanewline
\ \ \ \ \isacommand{also}\isamarkupfalse%
\ \isacommand{have}\isamarkupfalse%
\ {\isachardoublequoteopen}{\isachardot}{\kern0pt}{\isachardot}{\kern0pt}{\isachardot}{\kern0pt}\ {\isacharequal}{\kern0pt}\ {\isacharparenleft}{\kern0pt}{\isasymt}\ {\isasymcirc}\isactrlsub c\ {\isasymbeta}\isactrlbsub X\ {\isasymtimes}\isactrlsub c\ {\isasymone}\isactrlesub {\isacharparenright}{\kern0pt}\ {\isasymcirc}\isactrlsub c\ x{\isachardoublequoteclose}\isanewline
\ \ \ \ \ \ \isacommand{using}\isamarkupfalse%
\ x{\isacharunderscore}{\kern0pt}type\ comp{\isacharunderscore}{\kern0pt}associative{\isadigit{2}}\ \isacommand{by}\isamarkupfalse%
\ {\isacharparenleft}{\kern0pt}typecheck{\isacharunderscore}{\kern0pt}cfuncs{\isacharcomma}{\kern0pt}\ auto{\isacharparenright}{\kern0pt}\isanewline
\ \ \ \ \isanewline
\ \ \ \ \isacommand{then}\isamarkupfalse%
\ \isacommand{show}\isamarkupfalse%
\ {\isachardoublequoteopen}{\isacharparenleft}{\kern0pt}p\ {\isasymcirc}\isactrlsub c\ left{\isacharunderscore}{\kern0pt}cart{\isacharunderscore}{\kern0pt}proj\ X\ {\isasymone}{\isacharparenright}{\kern0pt}\ {\isasymcirc}\isactrlsub c\ x\ {\isacharequal}{\kern0pt}\ {\isacharparenleft}{\kern0pt}{\isasymt}\ {\isasymcirc}\isactrlsub c\ {\isasymbeta}\isactrlbsub X\ {\isasymtimes}\isactrlsub c\ {\isasymone}\isactrlesub {\isacharparenright}{\kern0pt}\ {\isasymcirc}\isactrlsub c\ x{\isachardoublequoteclose}\isanewline
\ \ \ \ \ \ \isacommand{using}\isamarkupfalse%
\ calculation\ \isacommand{by}\isamarkupfalse%
\ auto\isanewline
\ \ \isacommand{qed}\isamarkupfalse%
\isanewline
\ \ \isacommand{then}\isamarkupfalse%
\ \isacommand{have}\isamarkupfalse%
\ {\isachardoublequoteopen}{\isacharparenleft}{\kern0pt}p\ {\isasymcirc}\isactrlsub c\ left{\isacharunderscore}{\kern0pt}cart{\isacharunderscore}{\kern0pt}proj\ X\ {\isasymone}{\isacharparenright}{\kern0pt}\isactrlsup {\isasymsharp}\ {\isacharequal}{\kern0pt}\ {\isacharparenleft}{\kern0pt}{\isasymt}\ {\isasymcirc}\isactrlsub c\ {\isasymbeta}\isactrlbsub X\ {\isasymtimes}\isactrlsub c\ {\isasymone}\isactrlesub {\isacharparenright}{\kern0pt}\isactrlsup {\isasymsharp}{\isachardoublequoteclose}\isanewline
\ \ \ \ \isacommand{by}\isamarkupfalse%
\ simp\isanewline
\ \ \isacommand{then}\isamarkupfalse%
\ \isacommand{have}\isamarkupfalse%
\ {\isachardoublequoteopen}FORALL\ X\ {\isasymcirc}\isactrlsub c\ {\isacharparenleft}{\kern0pt}p\ {\isasymcirc}\isactrlsub c\ left{\isacharunderscore}{\kern0pt}cart{\isacharunderscore}{\kern0pt}proj\ X\ {\isasymone}{\isacharparenright}{\kern0pt}\isactrlsup {\isasymsharp}\ {\isacharequal}{\kern0pt}\ {\isasymt}\ {\isasymcirc}\isactrlsub c\ {\isasymbeta}\isactrlbsub {\isasymone}\isactrlesub {\isachardoublequoteclose}\isanewline
\ \ \ \ \isacommand{using}\isamarkupfalse%
\ FORALL{\isacharunderscore}{\kern0pt}is{\isacharunderscore}{\kern0pt}pullback\ \isacommand{unfolding}\isamarkupfalse%
\ is{\isacharunderscore}{\kern0pt}pullback{\isacharunderscore}{\kern0pt}def\ \ \isacommand{by}\isamarkupfalse%
\ auto\isanewline
\ \ \isacommand{then}\isamarkupfalse%
\ \isacommand{show}\isamarkupfalse%
\ {\isachardoublequoteopen}FORALL\ X\ {\isasymcirc}\isactrlsub c\ {\isacharparenleft}{\kern0pt}p\ {\isasymcirc}\isactrlsub c\ left{\isacharunderscore}{\kern0pt}cart{\isacharunderscore}{\kern0pt}proj\ X\ {\isasymone}{\isacharparenright}{\kern0pt}\isactrlsup {\isasymsharp}\ {\isacharequal}{\kern0pt}\ {\isasymt}{\isachardoublequoteclose}\isanewline
\ \ \ \ \isacommand{using}\isamarkupfalse%
\ NOT{\isacharunderscore}{\kern0pt}false{\isacharunderscore}{\kern0pt}is{\isacharunderscore}{\kern0pt}true\ NOT{\isacharunderscore}{\kern0pt}is{\isacharunderscore}{\kern0pt}pullback\ is{\isacharunderscore}{\kern0pt}pullback{\isacharunderscore}{\kern0pt}def\ \ \isacommand{by}\isamarkupfalse%
\ auto\isanewline
\isacommand{qed}\isamarkupfalse%
%
\endisatagproof
{\isafoldproof}%
%
\isadelimproof
\isanewline
%
\endisadelimproof
\isanewline
\isacommand{lemma}\isamarkupfalse%
\ all{\isacharunderscore}{\kern0pt}true{\isacharunderscore}{\kern0pt}implies{\isacharunderscore}{\kern0pt}FORALL{\isacharunderscore}{\kern0pt}true{\isadigit{2}}{\isacharcolon}{\kern0pt}\isanewline
\ \ \isakeyword{assumes}\ p{\isacharunderscore}{\kern0pt}type{\isacharbrackleft}{\kern0pt}type{\isacharunderscore}{\kern0pt}rule{\isacharbrackright}{\kern0pt}{\isacharcolon}{\kern0pt}\ {\isachardoublequoteopen}p\ {\isacharcolon}{\kern0pt}\ X\ {\isasymtimes}\isactrlsub c\ Y\ {\isasymrightarrow}\ {\isasymOmega}{\isachardoublequoteclose}\ \isakeyword{and}\ all{\isacharunderscore}{\kern0pt}p{\isacharunderscore}{\kern0pt}true{\isacharcolon}{\kern0pt}\ {\isachardoublequoteopen}{\isasymAnd}\ xy{\isachardot}{\kern0pt}\ xy\ {\isasymin}\isactrlsub c\ X\ {\isasymtimes}\isactrlsub c\ Y\ {\isasymLongrightarrow}\ p\ {\isasymcirc}\isactrlsub c\ xy\ {\isacharequal}{\kern0pt}\ {\isasymt}{\isachardoublequoteclose}\isanewline
\ \ \isakeyword{shows}\ {\isachardoublequoteopen}FORALL\ X\ {\isasymcirc}\isactrlsub c\ p\isactrlsup {\isasymsharp}\ {\isacharequal}{\kern0pt}\ {\isasymt}\ {\isasymcirc}\isactrlsub c\ {\isasymbeta}\isactrlbsub Y\isactrlesub {\isachardoublequoteclose}\isanewline
%
\isadelimproof
%
\endisadelimproof
%
\isatagproof
\isacommand{proof}\isamarkupfalse%
\ {\isacharminus}{\kern0pt}\isanewline
\ \ \isacommand{have}\isamarkupfalse%
\ {\isachardoublequoteopen}p\ {\isacharequal}{\kern0pt}\ {\isasymt}\ {\isasymcirc}\isactrlsub c\ {\isasymbeta}\isactrlbsub X\ {\isasymtimes}\isactrlsub c\ Y\isactrlesub {\isachardoublequoteclose}\isanewline
\ \ \isacommand{proof}\isamarkupfalse%
\ {\isacharparenleft}{\kern0pt}etcs{\isacharunderscore}{\kern0pt}rule\ one{\isacharunderscore}{\kern0pt}separator{\isacharparenright}{\kern0pt}\isanewline
\ \ \ \ \isacommand{fix}\isamarkupfalse%
\ xy\isanewline
\ \ \ \ \isacommand{assume}\isamarkupfalse%
\ xy{\isacharunderscore}{\kern0pt}type{\isacharbrackleft}{\kern0pt}type{\isacharunderscore}{\kern0pt}rule{\isacharbrackright}{\kern0pt}{\isacharcolon}{\kern0pt}\ {\isachardoublequoteopen}xy\ {\isasymin}\isactrlsub c\ X\ {\isasymtimes}\isactrlsub c\ Y{\isachardoublequoteclose}\isanewline
\ \ \ \ \isacommand{then}\isamarkupfalse%
\ \isacommand{have}\isamarkupfalse%
\ {\isachardoublequoteopen}p\ {\isasymcirc}\isactrlsub c\ xy\ {\isacharequal}{\kern0pt}\ {\isasymt}{\isachardoublequoteclose}\isanewline
\ \ \ \ \ \ \isacommand{using}\isamarkupfalse%
\ all{\isacharunderscore}{\kern0pt}p{\isacharunderscore}{\kern0pt}true\ \isacommand{by}\isamarkupfalse%
\ blast\isanewline
\ \ \ \ \isacommand{then}\isamarkupfalse%
\ \isacommand{have}\isamarkupfalse%
\ {\isachardoublequoteopen}p\ {\isasymcirc}\isactrlsub c\ xy\ {\isacharequal}{\kern0pt}\ {\isasymt}\ {\isasymcirc}\isactrlsub c\ {\isacharparenleft}{\kern0pt}{\isasymbeta}\isactrlbsub X\ {\isasymtimes}\isactrlsub c\ Y\isactrlesub \ {\isasymcirc}\isactrlsub c\ xy{\isacharparenright}{\kern0pt}{\isachardoublequoteclose}\isanewline
\ \ \ \ \ \ \isacommand{by}\isamarkupfalse%
\ {\isacharparenleft}{\kern0pt}typecheck{\isacharunderscore}{\kern0pt}cfuncs{\isacharcomma}{\kern0pt}\ metis\ id{\isacharunderscore}{\kern0pt}right{\isacharunderscore}{\kern0pt}unit{\isadigit{2}}\ id{\isacharunderscore}{\kern0pt}type\ one{\isacharunderscore}{\kern0pt}unique{\isacharunderscore}{\kern0pt}element{\isacharparenright}{\kern0pt}\isanewline
\ \ \ \ \isacommand{then}\isamarkupfalse%
\ \isacommand{show}\isamarkupfalse%
\ {\isachardoublequoteopen}p\ {\isasymcirc}\isactrlsub c\ xy\ {\isacharequal}{\kern0pt}\ {\isacharparenleft}{\kern0pt}{\isasymt}\ {\isasymcirc}\isactrlsub c\ {\isasymbeta}\isactrlbsub X\ {\isasymtimes}\isactrlsub c\ Y\isactrlesub {\isacharparenright}{\kern0pt}\ {\isasymcirc}\isactrlsub c\ xy{\isachardoublequoteclose}\isanewline
\ \ \ \ \ \ \isacommand{by}\isamarkupfalse%
\ {\isacharparenleft}{\kern0pt}typecheck{\isacharunderscore}{\kern0pt}cfuncs{\isacharcomma}{\kern0pt}\ smt\ comp{\isacharunderscore}{\kern0pt}associative{\isadigit{2}}{\isacharparenright}{\kern0pt}\isanewline
\ \ \isacommand{qed}\isamarkupfalse%
\isanewline
\ \ \isacommand{then}\isamarkupfalse%
\ \isacommand{have}\isamarkupfalse%
\ {\isachardoublequoteopen}p\isactrlsup {\isasymsharp}\ {\isacharequal}{\kern0pt}\ {\isacharparenleft}{\kern0pt}{\isasymt}\ {\isasymcirc}\isactrlsub c\ {\isasymbeta}\isactrlbsub X\ {\isasymtimes}\isactrlsub c\ Y\isactrlesub {\isacharparenright}{\kern0pt}\isactrlsup {\isasymsharp}{\isachardoublequoteclose}\isanewline
\ \ \ \ \isacommand{by}\isamarkupfalse%
\ blast\isanewline
\ \ \isacommand{then}\isamarkupfalse%
\ \isacommand{have}\isamarkupfalse%
\ {\isachardoublequoteopen}p\isactrlsup {\isasymsharp}\ {\isacharequal}{\kern0pt}\ {\isacharparenleft}{\kern0pt}{\isasymt}\ {\isasymcirc}\isactrlsub c\ {\isasymbeta}\isactrlbsub X\ {\isasymtimes}\isactrlsub c\ {\isasymone}\isactrlesub \ {\isasymcirc}\isactrlsub c\ {\isacharparenleft}{\kern0pt}id\ X\ {\isasymtimes}\isactrlsub f\ {\isasymbeta}\isactrlbsub Y\isactrlesub {\isacharparenright}{\kern0pt}{\isacharparenright}{\kern0pt}\isactrlsup {\isasymsharp}{\isachardoublequoteclose}\isanewline
\ \ \ \ \isacommand{by}\isamarkupfalse%
\ {\isacharparenleft}{\kern0pt}typecheck{\isacharunderscore}{\kern0pt}cfuncs{\isacharcomma}{\kern0pt}\ metis\ terminal{\isacharunderscore}{\kern0pt}func{\isacharunderscore}{\kern0pt}unique{\isacharparenright}{\kern0pt}\isanewline
\ \ \isacommand{then}\isamarkupfalse%
\ \isacommand{have}\isamarkupfalse%
\ {\isachardoublequoteopen}p\isactrlsup {\isasymsharp}\ {\isacharequal}{\kern0pt}\ {\isacharparenleft}{\kern0pt}{\isacharparenleft}{\kern0pt}{\isasymt}\ {\isasymcirc}\isactrlsub c\ {\isasymbeta}\isactrlbsub X\ {\isasymtimes}\isactrlsub c\ {\isasymone}\isactrlesub {\isacharparenright}{\kern0pt}\ {\isasymcirc}\isactrlsub c\ {\isacharparenleft}{\kern0pt}id\ X\ {\isasymtimes}\isactrlsub f\ {\isasymbeta}\isactrlbsub Y\isactrlesub {\isacharparenright}{\kern0pt}{\isacharparenright}{\kern0pt}\isactrlsup {\isasymsharp}{\isachardoublequoteclose}\isanewline
\ \ \ \ \isacommand{by}\isamarkupfalse%
\ {\isacharparenleft}{\kern0pt}typecheck{\isacharunderscore}{\kern0pt}cfuncs{\isacharcomma}{\kern0pt}\ smt\ comp{\isacharunderscore}{\kern0pt}associative{\isadigit{2}}{\isacharparenright}{\kern0pt}\isanewline
\ \ \isacommand{then}\isamarkupfalse%
\ \isacommand{have}\isamarkupfalse%
\ {\isachardoublequoteopen}p\isactrlsup {\isasymsharp}\ {\isacharequal}{\kern0pt}\ {\isacharparenleft}{\kern0pt}{\isasymt}\ {\isasymcirc}\isactrlsub c\ {\isasymbeta}\isactrlbsub X\ {\isasymtimes}\isactrlsub c\ {\isasymone}\isactrlesub {\isacharparenright}{\kern0pt}\isactrlsup {\isasymsharp}\ {\isasymcirc}\isactrlsub c\ {\isasymbeta}\isactrlbsub Y\isactrlesub {\isachardoublequoteclose}\isanewline
\ \ \ \ \isacommand{by}\isamarkupfalse%
\ {\isacharparenleft}{\kern0pt}typecheck{\isacharunderscore}{\kern0pt}cfuncs{\isacharcomma}{\kern0pt}\ simp\ add{\isacharcolon}{\kern0pt}\ sharp{\isacharunderscore}{\kern0pt}comp{\isacharparenright}{\kern0pt}\isanewline
\ \ \isacommand{then}\isamarkupfalse%
\ \isacommand{have}\isamarkupfalse%
\ {\isachardoublequoteopen}FORALL\ X\ {\isasymcirc}\isactrlsub c\ p\isactrlsup {\isasymsharp}\ {\isacharequal}{\kern0pt}\ {\isacharparenleft}{\kern0pt}FORALL\ X\ {\isasymcirc}\isactrlsub c\ {\isacharparenleft}{\kern0pt}{\isasymt}\ {\isasymcirc}\isactrlsub c\ {\isasymbeta}\isactrlbsub X\ {\isasymtimes}\isactrlsub c\ {\isasymone}\isactrlesub {\isacharparenright}{\kern0pt}\isactrlsup {\isasymsharp}{\isacharparenright}{\kern0pt}\ {\isasymcirc}\isactrlsub c\ {\isasymbeta}\isactrlbsub Y\isactrlesub {\isachardoublequoteclose}\isanewline
\ \ \ \ \isacommand{by}\isamarkupfalse%
\ {\isacharparenleft}{\kern0pt}typecheck{\isacharunderscore}{\kern0pt}cfuncs{\isacharcomma}{\kern0pt}\ smt\ comp{\isacharunderscore}{\kern0pt}associative{\isadigit{2}}{\isacharparenright}{\kern0pt}\isanewline
\ \ \isacommand{then}\isamarkupfalse%
\ \isacommand{have}\isamarkupfalse%
\ {\isachardoublequoteopen}FORALL\ X\ {\isasymcirc}\isactrlsub c\ p\isactrlsup {\isasymsharp}\ {\isacharequal}{\kern0pt}\ {\isacharparenleft}{\kern0pt}{\isasymt}\ {\isasymcirc}\isactrlsub c\ {\isasymbeta}\isactrlbsub {\isasymone}\isactrlesub {\isacharparenright}{\kern0pt}\ {\isasymcirc}\isactrlsub c\ {\isasymbeta}\isactrlbsub Y\isactrlesub {\isachardoublequoteclose}\isanewline
\ \ \ \ \isacommand{using}\isamarkupfalse%
\ FORALL{\isacharunderscore}{\kern0pt}is{\isacharunderscore}{\kern0pt}pullback\ \isacommand{unfolding}\isamarkupfalse%
\ is{\isacharunderscore}{\kern0pt}pullback{\isacharunderscore}{\kern0pt}def\ \ \isacommand{by}\isamarkupfalse%
\ auto\isanewline
\ \ \isacommand{then}\isamarkupfalse%
\ \isacommand{show}\isamarkupfalse%
\ {\isachardoublequoteopen}FORALL\ X\ {\isasymcirc}\isactrlsub c\ p\isactrlsup {\isasymsharp}\ {\isacharequal}{\kern0pt}\ {\isasymt}\ {\isasymcirc}\isactrlsub c\ {\isasymbeta}\isactrlbsub Y\isactrlesub {\isachardoublequoteclose}\isanewline
\ \ \ \ \isacommand{by}\isamarkupfalse%
\ {\isacharparenleft}{\kern0pt}metis\ id{\isacharunderscore}{\kern0pt}right{\isacharunderscore}{\kern0pt}unit{\isadigit{2}}\ id{\isacharunderscore}{\kern0pt}type\ terminal{\isacharunderscore}{\kern0pt}func{\isacharunderscore}{\kern0pt}unique\ true{\isacharunderscore}{\kern0pt}func{\isacharunderscore}{\kern0pt}type{\isacharparenright}{\kern0pt}\isanewline
\isacommand{qed}\isamarkupfalse%
%
\endisatagproof
{\isafoldproof}%
%
\isadelimproof
\isanewline
%
\endisadelimproof
\isanewline
\isacommand{lemma}\isamarkupfalse%
\ all{\isacharunderscore}{\kern0pt}true{\isacharunderscore}{\kern0pt}implies{\isacharunderscore}{\kern0pt}FORALL{\isacharunderscore}{\kern0pt}true{\isadigit{3}}{\isacharcolon}{\kern0pt}\isanewline
\ \ \isakeyword{assumes}\ p{\isacharunderscore}{\kern0pt}type{\isacharbrackleft}{\kern0pt}type{\isacharunderscore}{\kern0pt}rule{\isacharbrackright}{\kern0pt}{\isacharcolon}{\kern0pt}\ {\isachardoublequoteopen}p\ {\isacharcolon}{\kern0pt}\ X\ {\isasymtimes}\isactrlsub c\ {\isasymone}\ {\isasymrightarrow}\ {\isasymOmega}{\isachardoublequoteclose}\ \isakeyword{and}\ all{\isacharunderscore}{\kern0pt}p{\isacharunderscore}{\kern0pt}true{\isacharcolon}{\kern0pt}\ {\isachardoublequoteopen}{\isasymAnd}\ x{\isachardot}{\kern0pt}\ x\ {\isasymin}\isactrlsub c\ X\ {\isasymLongrightarrow}\ p\ {\isasymcirc}\isactrlsub c\ {\isasymlangle}x{\isacharcomma}{\kern0pt}\ id\ {\isasymone}{\isasymrangle}\ {\isacharequal}{\kern0pt}\ {\isasymt}{\isachardoublequoteclose}\isanewline
\ \ \isakeyword{shows}\ {\isachardoublequoteopen}FORALL\ X\ {\isasymcirc}\isactrlsub c\ p\isactrlsup {\isasymsharp}\ {\isacharequal}{\kern0pt}\ {\isasymt}{\isachardoublequoteclose}\isanewline
%
\isadelimproof
%
\endisadelimproof
%
\isatagproof
\isacommand{proof}\isamarkupfalse%
\ {\isacharminus}{\kern0pt}\isanewline
\ \ \isacommand{have}\isamarkupfalse%
\ {\isachardoublequoteopen}FORALL\ X\ {\isasymcirc}\isactrlsub c\ p\isactrlsup {\isasymsharp}\ {\isacharequal}{\kern0pt}\ {\isasymt}\ {\isasymcirc}\isactrlsub c\ {\isasymbeta}\isactrlbsub {\isasymone}\isactrlesub {\isachardoublequoteclose}\isanewline
\ \ \ \ \isacommand{by}\isamarkupfalse%
\ {\isacharparenleft}{\kern0pt}etcs{\isacharunderscore}{\kern0pt}rule\ all{\isacharunderscore}{\kern0pt}true{\isacharunderscore}{\kern0pt}implies{\isacharunderscore}{\kern0pt}FORALL{\isacharunderscore}{\kern0pt}true{\isadigit{2}}{\isacharcomma}{\kern0pt}\ metis\ all{\isacharunderscore}{\kern0pt}p{\isacharunderscore}{\kern0pt}true\ cart{\isacharunderscore}{\kern0pt}prod{\isacharunderscore}{\kern0pt}decomp\ id{\isacharunderscore}{\kern0pt}type\ one{\isacharunderscore}{\kern0pt}unique{\isacharunderscore}{\kern0pt}element{\isacharparenright}{\kern0pt}\isanewline
\ \ \isacommand{then}\isamarkupfalse%
\ \isacommand{show}\isamarkupfalse%
\ {\isacharquery}{\kern0pt}thesis\isanewline
\ \ \ \ \isacommand{by}\isamarkupfalse%
\ {\isacharparenleft}{\kern0pt}metis\ id{\isacharunderscore}{\kern0pt}right{\isacharunderscore}{\kern0pt}unit{\isadigit{2}}\ id{\isacharunderscore}{\kern0pt}type\ terminal{\isacharunderscore}{\kern0pt}func{\isacharunderscore}{\kern0pt}unique\ true{\isacharunderscore}{\kern0pt}func{\isacharunderscore}{\kern0pt}type{\isacharparenright}{\kern0pt}\isanewline
\isacommand{qed}\isamarkupfalse%
%
\endisatagproof
{\isafoldproof}%
%
\isadelimproof
\isanewline
%
\endisadelimproof
\isanewline
\isacommand{lemma}\isamarkupfalse%
\ FORALL{\isacharunderscore}{\kern0pt}true{\isacharunderscore}{\kern0pt}implies{\isacharunderscore}{\kern0pt}all{\isacharunderscore}{\kern0pt}true{\isacharcolon}{\kern0pt}\isanewline
\ \ \isakeyword{assumes}\ p{\isacharunderscore}{\kern0pt}type{\isacharcolon}{\kern0pt}\ {\isachardoublequoteopen}p\ {\isacharcolon}{\kern0pt}\ X\ {\isasymrightarrow}\ {\isasymOmega}{\isachardoublequoteclose}\ \isakeyword{and}\ FORALL{\isacharunderscore}{\kern0pt}p{\isacharunderscore}{\kern0pt}true{\isacharcolon}{\kern0pt}\ {\isachardoublequoteopen}FORALL\ X\ {\isasymcirc}\isactrlsub c\ {\isacharparenleft}{\kern0pt}p\ {\isasymcirc}\isactrlsub c\ left{\isacharunderscore}{\kern0pt}cart{\isacharunderscore}{\kern0pt}proj\ X\ {\isasymone}{\isacharparenright}{\kern0pt}\isactrlsup {\isasymsharp}\ {\isacharequal}{\kern0pt}\ {\isasymt}{\isachardoublequoteclose}\isanewline
\ \ \isakeyword{shows}\ {\isachardoublequoteopen}{\isasymAnd}\ x{\isachardot}{\kern0pt}\ x\ {\isasymin}\isactrlsub c\ X\ {\isasymLongrightarrow}\ p\ {\isasymcirc}\isactrlsub c\ x\ {\isacharequal}{\kern0pt}\ {\isasymt}{\isachardoublequoteclose}\isanewline
%
\isadelimproof
%
\endisadelimproof
%
\isatagproof
\isacommand{proof}\isamarkupfalse%
\ {\isacharparenleft}{\kern0pt}rule\ ccontr{\isacharparenright}{\kern0pt}\isanewline
\ \ \isacommand{fix}\isamarkupfalse%
\ x\isanewline
\ \ \isacommand{assume}\isamarkupfalse%
\ x{\isacharunderscore}{\kern0pt}type{\isacharcolon}{\kern0pt}\ {\isachardoublequoteopen}x\ {\isasymin}\isactrlsub c\ X{\isachardoublequoteclose}\isanewline
\ \ \isacommand{assume}\isamarkupfalse%
\ {\isachardoublequoteopen}p\ {\isasymcirc}\isactrlsub c\ x\ {\isasymnoteq}\ {\isasymt}{\isachardoublequoteclose}\isanewline
\ \ \isacommand{then}\isamarkupfalse%
\ \isacommand{have}\isamarkupfalse%
\ {\isachardoublequoteopen}p\ {\isasymcirc}\isactrlsub c\ x\ {\isacharequal}{\kern0pt}\ {\isasymf}{\isachardoublequoteclose}\isanewline
\ \ \ \ \isacommand{using}\isamarkupfalse%
\ comp{\isacharunderscore}{\kern0pt}type\ p{\isacharunderscore}{\kern0pt}type\ true{\isacharunderscore}{\kern0pt}false{\isacharunderscore}{\kern0pt}only{\isacharunderscore}{\kern0pt}truth{\isacharunderscore}{\kern0pt}values\ x{\isacharunderscore}{\kern0pt}type\ \isacommand{by}\isamarkupfalse%
\ blast\isanewline
\ \ \isacommand{then}\isamarkupfalse%
\ \isacommand{have}\isamarkupfalse%
\ {\isachardoublequoteopen}p\ {\isasymcirc}\isactrlsub c\ left{\isacharunderscore}{\kern0pt}cart{\isacharunderscore}{\kern0pt}proj\ X\ {\isasymone}\ {\isasymcirc}\isactrlsub c\ {\isasymlangle}x{\isacharcomma}{\kern0pt}\ id\ {\isasymone}{\isasymrangle}\ {\isacharequal}{\kern0pt}\ {\isasymf}{\isachardoublequoteclose}\isanewline
\ \ \ \ \isacommand{using}\isamarkupfalse%
\ id{\isacharunderscore}{\kern0pt}type\ left{\isacharunderscore}{\kern0pt}cart{\isacharunderscore}{\kern0pt}proj{\isacharunderscore}{\kern0pt}cfunc{\isacharunderscore}{\kern0pt}prod\ x{\isacharunderscore}{\kern0pt}type\ \isacommand{by}\isamarkupfalse%
\ auto\isanewline
\ \ \isacommand{then}\isamarkupfalse%
\ \isacommand{have}\isamarkupfalse%
\ p{\isacharunderscore}{\kern0pt}left{\isacharunderscore}{\kern0pt}proj{\isacharunderscore}{\kern0pt}false{\isacharcolon}{\kern0pt}\ {\isachardoublequoteopen}p\ {\isasymcirc}\isactrlsub c\ left{\isacharunderscore}{\kern0pt}cart{\isacharunderscore}{\kern0pt}proj\ X\ {\isasymone}\ {\isasymcirc}\isactrlsub c\ {\isasymlangle}x{\isacharcomma}{\kern0pt}\ id\ {\isasymone}{\isasymrangle}\ {\isacharequal}{\kern0pt}\ {\isasymf}\ {\isasymcirc}\isactrlsub c\ {\isasymbeta}\isactrlbsub X\ {\isasymtimes}\isactrlsub c\ {\isasymone}\isactrlesub \ {\isasymcirc}\isactrlsub c\ {\isasymlangle}x{\isacharcomma}{\kern0pt}\ id\ {\isasymone}{\isasymrangle}{\isachardoublequoteclose}\isanewline
\ \ \ \ \isacommand{using}\isamarkupfalse%
\ x{\isacharunderscore}{\kern0pt}type\ \isacommand{by}\isamarkupfalse%
\ {\isacharparenleft}{\kern0pt}typecheck{\isacharunderscore}{\kern0pt}cfuncs{\isacharcomma}{\kern0pt}\ metis\ id{\isacharunderscore}{\kern0pt}right{\isacharunderscore}{\kern0pt}unit{\isadigit{2}}\ one{\isacharunderscore}{\kern0pt}unique{\isacharunderscore}{\kern0pt}element{\isacharparenright}{\kern0pt}\isanewline
\isanewline
\ \ \isacommand{have}\isamarkupfalse%
\ {\isachardoublequoteopen}{\isasymt}\ {\isasymcirc}\isactrlsub c\ id\ {\isasymone}\ {\isacharequal}{\kern0pt}\ FORALL\ X\ {\isasymcirc}\isactrlsub c\ {\isacharparenleft}{\kern0pt}p\ {\isasymcirc}\isactrlsub c\ left{\isacharunderscore}{\kern0pt}cart{\isacharunderscore}{\kern0pt}proj\ X\ {\isasymone}{\isacharparenright}{\kern0pt}\isactrlsup {\isasymsharp}{\isachardoublequoteclose}\isanewline
\ \ \ \ \isacommand{using}\isamarkupfalse%
\ FORALL{\isacharunderscore}{\kern0pt}p{\isacharunderscore}{\kern0pt}true\ id{\isacharunderscore}{\kern0pt}right{\isacharunderscore}{\kern0pt}unit{\isadigit{2}}\ true{\isacharunderscore}{\kern0pt}func{\isacharunderscore}{\kern0pt}type\ \isacommand{by}\isamarkupfalse%
\ auto\isanewline
\ \ \isacommand{then}\isamarkupfalse%
\ \isacommand{obtain}\isamarkupfalse%
\ j\ \isakeyword{where}\ \isanewline
\ \ \ \ \ \ j{\isacharunderscore}{\kern0pt}type{\isacharcolon}{\kern0pt}\ {\isachardoublequoteopen}j\ {\isasymin}\isactrlsub c\ {\isasymone}{\isachardoublequoteclose}\ \isakeyword{and}\ \isanewline
\ \ \ \ \ \ j{\isacharunderscore}{\kern0pt}id{\isacharcolon}{\kern0pt}\ {\isachardoublequoteopen}{\isasymbeta}\isactrlbsub {\isasymone}\isactrlesub \ {\isasymcirc}\isactrlsub c\ j\ {\isacharequal}{\kern0pt}\ id\ {\isasymone}{\isachardoublequoteclose}\ \isakeyword{and}\isanewline
\ \ \ \ \ \ t{\isacharunderscore}{\kern0pt}j{\isacharunderscore}{\kern0pt}eq{\isacharunderscore}{\kern0pt}p{\isacharunderscore}{\kern0pt}left{\isacharunderscore}{\kern0pt}proj{\isacharcolon}{\kern0pt}\ {\isachardoublequoteopen}{\isacharparenleft}{\kern0pt}{\isasymt}\ {\isasymcirc}\isactrlsub c\ {\isasymbeta}\isactrlbsub X\ {\isasymtimes}\isactrlsub c\ {\isasymone}\isactrlesub {\isacharparenright}{\kern0pt}\isactrlsup {\isasymsharp}\ {\isasymcirc}\isactrlsub c\ j\ {\isacharequal}{\kern0pt}\ {\isacharparenleft}{\kern0pt}p\ {\isasymcirc}\isactrlsub c\ left{\isacharunderscore}{\kern0pt}cart{\isacharunderscore}{\kern0pt}proj\ X\ {\isasymone}{\isacharparenright}{\kern0pt}\isactrlsup {\isasymsharp}{\isachardoublequoteclose}\isanewline
\ \ \ \ \isacommand{using}\isamarkupfalse%
\ FORALL{\isacharunderscore}{\kern0pt}is{\isacharunderscore}{\kern0pt}pullback\ p{\isacharunderscore}{\kern0pt}type\ \isacommand{unfolding}\isamarkupfalse%
\ is{\isacharunderscore}{\kern0pt}pullback{\isacharunderscore}{\kern0pt}def\ \isacommand{by}\isamarkupfalse%
\ {\isacharparenleft}{\kern0pt}typecheck{\isacharunderscore}{\kern0pt}cfuncs{\isacharcomma}{\kern0pt}\ blast{\isacharparenright}{\kern0pt}\isanewline
\ \ \isacommand{then}\isamarkupfalse%
\ \isacommand{have}\isamarkupfalse%
\ {\isachardoublequoteopen}j\ {\isacharequal}{\kern0pt}\ id\ {\isasymone}{\isachardoublequoteclose}\isanewline
\ \ \ \ \isacommand{using}\isamarkupfalse%
\ id{\isacharunderscore}{\kern0pt}type\ one{\isacharunderscore}{\kern0pt}unique{\isacharunderscore}{\kern0pt}element\ \isacommand{by}\isamarkupfalse%
\ blast\isanewline
\ \ \isacommand{then}\isamarkupfalse%
\ \isacommand{have}\isamarkupfalse%
\ {\isachardoublequoteopen}{\isacharparenleft}{\kern0pt}{\isasymt}\ {\isasymcirc}\isactrlsub c\ {\isasymbeta}\isactrlbsub X\ {\isasymtimes}\isactrlsub c\ {\isasymone}\isactrlesub {\isacharparenright}{\kern0pt}\isactrlsup {\isasymsharp}\ {\isacharequal}{\kern0pt}\ {\isacharparenleft}{\kern0pt}p\ {\isasymcirc}\isactrlsub c\ left{\isacharunderscore}{\kern0pt}cart{\isacharunderscore}{\kern0pt}proj\ X\ {\isasymone}{\isacharparenright}{\kern0pt}\isactrlsup {\isasymsharp}{\isachardoublequoteclose}\isanewline
\ \ \ \ \isacommand{using}\isamarkupfalse%
\ id{\isacharunderscore}{\kern0pt}right{\isacharunderscore}{\kern0pt}unit{\isadigit{2}}\ t{\isacharunderscore}{\kern0pt}j{\isacharunderscore}{\kern0pt}eq{\isacharunderscore}{\kern0pt}p{\isacharunderscore}{\kern0pt}left{\isacharunderscore}{\kern0pt}proj\ p{\isacharunderscore}{\kern0pt}type\ \isacommand{by}\isamarkupfalse%
\ {\isacharparenleft}{\kern0pt}typecheck{\isacharunderscore}{\kern0pt}cfuncs{\isacharcomma}{\kern0pt}\ auto{\isacharparenright}{\kern0pt}\isanewline
\ \ \isacommand{then}\isamarkupfalse%
\ \isacommand{have}\isamarkupfalse%
\ {\isachardoublequoteopen}{\isasymt}\ {\isasymcirc}\isactrlsub c\ {\isasymbeta}\isactrlbsub X\ {\isasymtimes}\isactrlsub c\ {\isasymone}\isactrlesub \ {\isacharequal}{\kern0pt}\ p\ {\isasymcirc}\isactrlsub c\ left{\isacharunderscore}{\kern0pt}cart{\isacharunderscore}{\kern0pt}proj\ X\ {\isasymone}{\isachardoublequoteclose}\isanewline
\ \ \ \ \isacommand{using}\isamarkupfalse%
\ p{\isacharunderscore}{\kern0pt}type\ \isacommand{by}\isamarkupfalse%
\ {\isacharparenleft}{\kern0pt}typecheck{\isacharunderscore}{\kern0pt}cfuncs{\isacharcomma}{\kern0pt}\ metis\ flat{\isacharunderscore}{\kern0pt}cancels{\isacharunderscore}{\kern0pt}sharp{\isacharparenright}{\kern0pt}\isanewline
\ \ \isacommand{then}\isamarkupfalse%
\ \isacommand{have}\isamarkupfalse%
\ p{\isacharunderscore}{\kern0pt}left{\isacharunderscore}{\kern0pt}proj{\isacharunderscore}{\kern0pt}true{\isacharcolon}{\kern0pt}\ {\isachardoublequoteopen}{\isasymt}\ {\isasymcirc}\isactrlsub c\ {\isasymbeta}\isactrlbsub X\ {\isasymtimes}\isactrlsub c\ {\isasymone}\isactrlesub \ {\isasymcirc}\isactrlsub c\ {\isasymlangle}x{\isacharcomma}{\kern0pt}\ id\ {\isasymone}{\isasymrangle}\ {\isacharequal}{\kern0pt}\ p\ {\isasymcirc}\isactrlsub c\ left{\isacharunderscore}{\kern0pt}cart{\isacharunderscore}{\kern0pt}proj\ X\ {\isasymone}\ {\isasymcirc}\isactrlsub c\ {\isasymlangle}x{\isacharcomma}{\kern0pt}\ id\ {\isasymone}{\isasymrangle}{\isachardoublequoteclose}\isanewline
\ \ \ \ \isacommand{using}\isamarkupfalse%
\ p{\isacharunderscore}{\kern0pt}type\ x{\isacharunderscore}{\kern0pt}type\ comp{\isacharunderscore}{\kern0pt}associative{\isadigit{2}}\ \isacommand{by}\isamarkupfalse%
\ {\isacharparenleft}{\kern0pt}typecheck{\isacharunderscore}{\kern0pt}cfuncs{\isacharcomma}{\kern0pt}\ auto{\isacharparenright}{\kern0pt}\isanewline
\isanewline
\ \ \isacommand{have}\isamarkupfalse%
\ {\isachardoublequoteopen}{\isasymt}\ {\isasymcirc}\isactrlsub c\ {\isasymbeta}\isactrlbsub X\ {\isasymtimes}\isactrlsub c\ {\isasymone}\isactrlesub \ {\isasymcirc}\isactrlsub c\ {\isasymlangle}x{\isacharcomma}{\kern0pt}\ id\ {\isasymone}{\isasymrangle}\ {\isacharequal}{\kern0pt}\ {\isasymf}\ {\isasymcirc}\isactrlsub c\ {\isasymbeta}\isactrlbsub X\ {\isasymtimes}\isactrlsub c\ {\isasymone}\isactrlesub \ {\isasymcirc}\isactrlsub c\ {\isasymlangle}x{\isacharcomma}{\kern0pt}\ id\ {\isasymone}{\isasymrangle}{\isachardoublequoteclose}\isanewline
\ \ \ \ \isacommand{using}\isamarkupfalse%
\ p{\isacharunderscore}{\kern0pt}left{\isacharunderscore}{\kern0pt}proj{\isacharunderscore}{\kern0pt}false\ p{\isacharunderscore}{\kern0pt}left{\isacharunderscore}{\kern0pt}proj{\isacharunderscore}{\kern0pt}true\ \isacommand{by}\isamarkupfalse%
\ auto\isanewline
\ \ \isacommand{then}\isamarkupfalse%
\ \isacommand{have}\isamarkupfalse%
\ {\isachardoublequoteopen}{\isasymt}\ {\isasymcirc}\isactrlsub c\ id\ {\isasymone}\ {\isacharequal}{\kern0pt}\ {\isasymf}\ {\isasymcirc}\isactrlsub c\ id\ {\isasymone}{\isachardoublequoteclose}\isanewline
\ \ \ \ \isacommand{by}\isamarkupfalse%
\ {\isacharparenleft}{\kern0pt}metis\ id{\isacharunderscore}{\kern0pt}type\ right{\isacharunderscore}{\kern0pt}cart{\isacharunderscore}{\kern0pt}proj{\isacharunderscore}{\kern0pt}cfunc{\isacharunderscore}{\kern0pt}prod\ right{\isacharunderscore}{\kern0pt}cart{\isacharunderscore}{\kern0pt}proj{\isacharunderscore}{\kern0pt}type\ terminal{\isacharunderscore}{\kern0pt}func{\isacharunderscore}{\kern0pt}unique\ x{\isacharunderscore}{\kern0pt}type{\isacharparenright}{\kern0pt}\isanewline
\ \ \isacommand{then}\isamarkupfalse%
\ \isacommand{have}\isamarkupfalse%
\ {\isachardoublequoteopen}{\isasymt}\ {\isacharequal}{\kern0pt}\ {\isasymf}{\isachardoublequoteclose}\isanewline
\ \ \ \ \isacommand{using}\isamarkupfalse%
\ true{\isacharunderscore}{\kern0pt}func{\isacharunderscore}{\kern0pt}type\ false{\isacharunderscore}{\kern0pt}func{\isacharunderscore}{\kern0pt}type\ id{\isacharunderscore}{\kern0pt}right{\isacharunderscore}{\kern0pt}unit{\isadigit{2}}\ \isacommand{by}\isamarkupfalse%
\ auto\isanewline
\ \ \isacommand{then}\isamarkupfalse%
\ \isacommand{show}\isamarkupfalse%
\ False\isanewline
\ \ \ \ \isacommand{using}\isamarkupfalse%
\ true{\isacharunderscore}{\kern0pt}false{\isacharunderscore}{\kern0pt}distinct\ \isacommand{by}\isamarkupfalse%
\ auto\isanewline
\isacommand{qed}\isamarkupfalse%
%
\endisatagproof
{\isafoldproof}%
%
\isadelimproof
\isanewline
%
\endisadelimproof
\isanewline
\isacommand{lemma}\isamarkupfalse%
\ FORALL{\isacharunderscore}{\kern0pt}true{\isacharunderscore}{\kern0pt}implies{\isacharunderscore}{\kern0pt}all{\isacharunderscore}{\kern0pt}true{\isadigit{2}}{\isacharcolon}{\kern0pt}\isanewline
\ \ \isakeyword{assumes}\ p{\isacharunderscore}{\kern0pt}type{\isacharbrackleft}{\kern0pt}type{\isacharunderscore}{\kern0pt}rule{\isacharbrackright}{\kern0pt}{\isacharcolon}{\kern0pt}\ {\isachardoublequoteopen}p\ {\isacharcolon}{\kern0pt}\ X\ {\isasymtimes}\isactrlsub c\ Y\ {\isasymrightarrow}\ {\isasymOmega}{\isachardoublequoteclose}\ \isakeyword{and}\ FORALL{\isacharunderscore}{\kern0pt}p{\isacharunderscore}{\kern0pt}true{\isacharcolon}{\kern0pt}\ {\isachardoublequoteopen}FORALL\ X\ {\isasymcirc}\isactrlsub c\ p\isactrlsup {\isasymsharp}\ {\isacharequal}{\kern0pt}\ {\isasymt}\ {\isasymcirc}\isactrlsub c\ {\isasymbeta}\isactrlbsub Y\isactrlesub {\isachardoublequoteclose}\isanewline
\ \ \isakeyword{shows}\ {\isachardoublequoteopen}{\isasymAnd}\ x\ y{\isachardot}{\kern0pt}\ x\ {\isasymin}\isactrlsub c\ X\ {\isasymLongrightarrow}\ y\ {\isasymin}\isactrlsub c\ Y\ {\isasymLongrightarrow}\ p\ {\isasymcirc}\isactrlsub c\ {\isasymlangle}x{\isacharcomma}{\kern0pt}\ y{\isasymrangle}\ {\isacharequal}{\kern0pt}\ {\isasymt}{\isachardoublequoteclose}\isanewline
%
\isadelimproof
%
\endisadelimproof
%
\isatagproof
\isacommand{proof}\isamarkupfalse%
\ {\isacharminus}{\kern0pt}\isanewline
\ \ \isacommand{have}\isamarkupfalse%
\ {\isachardoublequoteopen}p\isactrlsup {\isasymsharp}\ {\isacharequal}{\kern0pt}\ {\isacharparenleft}{\kern0pt}{\isasymt}\ {\isasymcirc}\isactrlsub c\ {\isasymbeta}\isactrlbsub X\ {\isasymtimes}\isactrlsub c\ {\isasymone}\isactrlesub {\isacharparenright}{\kern0pt}\isactrlsup {\isasymsharp}\ {\isasymcirc}\isactrlsub c\ {\isasymbeta}\isactrlbsub Y\isactrlesub {\isachardoublequoteclose}\isanewline
\ \ \ \ \isacommand{using}\isamarkupfalse%
\ FORALL{\isacharunderscore}{\kern0pt}is{\isacharunderscore}{\kern0pt}pullback\ FORALL{\isacharunderscore}{\kern0pt}p{\isacharunderscore}{\kern0pt}true\ \isacommand{unfolding}\isamarkupfalse%
\ is{\isacharunderscore}{\kern0pt}pullback{\isacharunderscore}{\kern0pt}def\ \isanewline
\ \ \ \ \isacommand{by}\isamarkupfalse%
\ {\isacharparenleft}{\kern0pt}typecheck{\isacharunderscore}{\kern0pt}cfuncs{\isacharcomma}{\kern0pt}\ metis\ terminal{\isacharunderscore}{\kern0pt}func{\isacharunderscore}{\kern0pt}unique{\isacharparenright}{\kern0pt}\isanewline
\ \ \isacommand{then}\isamarkupfalse%
\ \isacommand{have}\isamarkupfalse%
\ {\isachardoublequoteopen}p\isactrlsup {\isasymsharp}\ {\isacharequal}{\kern0pt}\ {\isacharparenleft}{\kern0pt}{\isacharparenleft}{\kern0pt}{\isasymt}\ {\isasymcirc}\isactrlsub c\ {\isasymbeta}\isactrlbsub X\ {\isasymtimes}\isactrlsub c\ {\isasymone}\isactrlesub {\isacharparenright}{\kern0pt}\ {\isasymcirc}\isactrlsub c\ {\isacharparenleft}{\kern0pt}id\ X\ {\isasymtimes}\isactrlsub f\ {\isasymbeta}\isactrlbsub Y\isactrlesub {\isacharparenright}{\kern0pt}{\isacharparenright}{\kern0pt}\isactrlsup {\isasymsharp}{\isachardoublequoteclose}\isanewline
\ \ \ \ \isacommand{by}\isamarkupfalse%
\ {\isacharparenleft}{\kern0pt}typecheck{\isacharunderscore}{\kern0pt}cfuncs{\isacharcomma}{\kern0pt}\ simp\ add{\isacharcolon}{\kern0pt}\ sharp{\isacharunderscore}{\kern0pt}comp{\isacharparenright}{\kern0pt}\isanewline
\ \ \isacommand{then}\isamarkupfalse%
\ \isacommand{have}\isamarkupfalse%
\ {\isachardoublequoteopen}p\isactrlsup {\isasymsharp}\ {\isacharequal}{\kern0pt}\ {\isacharparenleft}{\kern0pt}{\isasymt}\ {\isasymcirc}\isactrlsub c\ {\isasymbeta}\isactrlbsub X\ {\isasymtimes}\isactrlsub c\ Y\isactrlesub {\isacharparenright}{\kern0pt}\isactrlsup {\isasymsharp}{\isachardoublequoteclose}\isanewline
\ \ \ \ \isacommand{by}\isamarkupfalse%
\ {\isacharparenleft}{\kern0pt}typecheck{\isacharunderscore}{\kern0pt}cfuncs{\isacharunderscore}{\kern0pt}prems{\isacharcomma}{\kern0pt}\ smt\ {\isacharparenleft}{\kern0pt}z{\isadigit{3}}{\isacharparenright}{\kern0pt}\ comp{\isacharunderscore}{\kern0pt}associative{\isadigit{2}}\ terminal{\isacharunderscore}{\kern0pt}func{\isacharunderscore}{\kern0pt}comp{\isacharparenright}{\kern0pt}\isanewline
\ \ \isacommand{then}\isamarkupfalse%
\ \isacommand{have}\isamarkupfalse%
\ {\isachardoublequoteopen}p\ {\isacharequal}{\kern0pt}\ {\isasymt}\ {\isasymcirc}\isactrlsub c\ {\isasymbeta}\isactrlbsub X\ {\isasymtimes}\isactrlsub c\ Y\isactrlesub {\isachardoublequoteclose}\isanewline
\ \ \ \ \isacommand{by}\isamarkupfalse%
\ {\isacharparenleft}{\kern0pt}typecheck{\isacharunderscore}{\kern0pt}cfuncs{\isacharcomma}{\kern0pt}\ metis\ flat{\isacharunderscore}{\kern0pt}cancels{\isacharunderscore}{\kern0pt}sharp{\isacharparenright}{\kern0pt}\isanewline
\ \ \isacommand{then}\isamarkupfalse%
\ \isacommand{have}\isamarkupfalse%
\ {\isachardoublequoteopen}{\isasymAnd}\ x\ y{\isachardot}{\kern0pt}\ x\ {\isasymin}\isactrlsub c\ X\ {\isasymLongrightarrow}\ y\ {\isasymin}\isactrlsub c\ Y\ {\isasymLongrightarrow}\ p\ {\isasymcirc}\isactrlsub c\ {\isasymlangle}x{\isacharcomma}{\kern0pt}\ y{\isasymrangle}\ {\isacharequal}{\kern0pt}\ {\isacharparenleft}{\kern0pt}{\isasymt}\ {\isasymcirc}\isactrlsub c\ {\isasymbeta}\isactrlbsub X\ {\isasymtimes}\isactrlsub c\ Y\isactrlesub {\isacharparenright}{\kern0pt}\ {\isasymcirc}\isactrlsub c\ {\isasymlangle}x{\isacharcomma}{\kern0pt}\ y{\isasymrangle}{\isachardoublequoteclose}\isanewline
\ \ \ \ \isacommand{by}\isamarkupfalse%
\ auto\isanewline
\ \ \isacommand{then}\isamarkupfalse%
\ \isacommand{show}\isamarkupfalse%
\ {\isachardoublequoteopen}{\isasymAnd}\ x\ y{\isachardot}{\kern0pt}\ x\ {\isasymin}\isactrlsub c\ X\ {\isasymLongrightarrow}\ y\ {\isasymin}\isactrlsub c\ Y\ {\isasymLongrightarrow}\ p\ {\isasymcirc}\isactrlsub c\ {\isasymlangle}x{\isacharcomma}{\kern0pt}\ y{\isasymrangle}\ {\isacharequal}{\kern0pt}\ {\isasymt}{\isachardoublequoteclose}\isanewline
\ \ \isacommand{proof}\isamarkupfalse%
\ {\isacharminus}{\kern0pt}\isanewline
\ \ \ \ \isacommand{fix}\isamarkupfalse%
\ x\ y\isanewline
\ \ \ \ \isacommand{assume}\isamarkupfalse%
\ xy{\isacharunderscore}{\kern0pt}types{\isacharbrackleft}{\kern0pt}type{\isacharunderscore}{\kern0pt}rule{\isacharbrackright}{\kern0pt}{\isacharcolon}{\kern0pt}\ {\isachardoublequoteopen}x\ {\isasymin}\isactrlsub c\ X{\isachardoublequoteclose}\ {\isachardoublequoteopen}y\ {\isasymin}\isactrlsub c\ Y{\isachardoublequoteclose}\isanewline
\ \ \ \ \isacommand{assume}\isamarkupfalse%
\ {\isachardoublequoteopen}{\isasymAnd}x\ y{\isachardot}{\kern0pt}\ x\ {\isasymin}\isactrlsub c\ X\ {\isasymLongrightarrow}\ y\ {\isasymin}\isactrlsub c\ Y\ {\isasymLongrightarrow}\ p\ {\isasymcirc}\isactrlsub c\ {\isasymlangle}x{\isacharcomma}{\kern0pt}y{\isasymrangle}\ {\isacharequal}{\kern0pt}\ {\isacharparenleft}{\kern0pt}{\isasymt}\ {\isasymcirc}\isactrlsub c\ {\isasymbeta}\isactrlbsub X\ {\isasymtimes}\isactrlsub c\ Y\isactrlesub {\isacharparenright}{\kern0pt}\ {\isasymcirc}\isactrlsub c\ {\isasymlangle}x{\isacharcomma}{\kern0pt}y{\isasymrangle}{\isachardoublequoteclose}\isanewline
\ \ \ \ \isacommand{then}\isamarkupfalse%
\ \isacommand{have}\isamarkupfalse%
\ {\isachardoublequoteopen}p\ {\isasymcirc}\isactrlsub c\ {\isasymlangle}x{\isacharcomma}{\kern0pt}y{\isasymrangle}\ {\isacharequal}{\kern0pt}\ {\isacharparenleft}{\kern0pt}{\isasymt}\ {\isasymcirc}\isactrlsub c\ {\isasymbeta}\isactrlbsub X\ {\isasymtimes}\isactrlsub c\ Y\isactrlesub {\isacharparenright}{\kern0pt}\ {\isasymcirc}\isactrlsub c\ {\isasymlangle}x{\isacharcomma}{\kern0pt}y{\isasymrangle}{\isachardoublequoteclose}\isanewline
\ \ \ \ \ \ \isacommand{using}\isamarkupfalse%
\ xy{\isacharunderscore}{\kern0pt}types\ \isacommand{by}\isamarkupfalse%
\ auto\isanewline
\ \ \ \ \isacommand{then}\isamarkupfalse%
\ \isacommand{have}\isamarkupfalse%
\ {\isachardoublequoteopen}p\ {\isasymcirc}\isactrlsub c\ {\isasymlangle}x{\isacharcomma}{\kern0pt}y{\isasymrangle}\ {\isacharequal}{\kern0pt}\ {\isasymt}\ {\isasymcirc}\isactrlsub c\ {\isacharparenleft}{\kern0pt}{\isasymbeta}\isactrlbsub X\ {\isasymtimes}\isactrlsub c\ Y\isactrlesub \ {\isasymcirc}\isactrlsub c\ {\isasymlangle}x{\isacharcomma}{\kern0pt}y{\isasymrangle}{\isacharparenright}{\kern0pt}{\isachardoublequoteclose}\isanewline
\ \ \ \ \ \ \isacommand{by}\isamarkupfalse%
\ {\isacharparenleft}{\kern0pt}typecheck{\isacharunderscore}{\kern0pt}cfuncs{\isacharcomma}{\kern0pt}\ smt\ comp{\isacharunderscore}{\kern0pt}associative{\isadigit{2}}{\isacharparenright}{\kern0pt}\isanewline
\ \ \ \ \isacommand{then}\isamarkupfalse%
\ \isacommand{show}\isamarkupfalse%
\ {\isachardoublequoteopen}p\ {\isasymcirc}\isactrlsub c\ {\isasymlangle}x{\isacharcomma}{\kern0pt}\ y{\isasymrangle}\ {\isacharequal}{\kern0pt}\ {\isasymt}{\isachardoublequoteclose}\isanewline
\ \ \ \ \ \ \isacommand{by}\isamarkupfalse%
\ {\isacharparenleft}{\kern0pt}typecheck{\isacharunderscore}{\kern0pt}cfuncs{\isacharunderscore}{\kern0pt}prems{\isacharcomma}{\kern0pt}\ metis\ id{\isacharunderscore}{\kern0pt}right{\isacharunderscore}{\kern0pt}unit{\isadigit{2}}\ id{\isacharunderscore}{\kern0pt}type\ one{\isacharunderscore}{\kern0pt}unique{\isacharunderscore}{\kern0pt}element{\isacharparenright}{\kern0pt}\isanewline
\ \ \isacommand{qed}\isamarkupfalse%
\isanewline
\isacommand{qed}\isamarkupfalse%
%
\endisatagproof
{\isafoldproof}%
%
\isadelimproof
\isanewline
%
\endisadelimproof
\isanewline
\isacommand{lemma}\isamarkupfalse%
\ FORALL{\isacharunderscore}{\kern0pt}true{\isacharunderscore}{\kern0pt}implies{\isacharunderscore}{\kern0pt}all{\isacharunderscore}{\kern0pt}true{\isadigit{3}}{\isacharcolon}{\kern0pt}\isanewline
\ \ \isakeyword{assumes}\ p{\isacharunderscore}{\kern0pt}type{\isacharbrackleft}{\kern0pt}type{\isacharunderscore}{\kern0pt}rule{\isacharbrackright}{\kern0pt}{\isacharcolon}{\kern0pt}\ {\isachardoublequoteopen}p\ {\isacharcolon}{\kern0pt}\ X\ {\isasymtimes}\isactrlsub c\ {\isasymone}\ {\isasymrightarrow}\ {\isasymOmega}{\isachardoublequoteclose}\ \isakeyword{and}\ FORALL{\isacharunderscore}{\kern0pt}p{\isacharunderscore}{\kern0pt}true{\isacharcolon}{\kern0pt}\ {\isachardoublequoteopen}FORALL\ X\ {\isasymcirc}\isactrlsub c\ p\isactrlsup {\isasymsharp}\ {\isacharequal}{\kern0pt}\ {\isasymt}{\isachardoublequoteclose}\isanewline
\ \ \isakeyword{shows}\ {\isachardoublequoteopen}{\isasymAnd}\ x{\isachardot}{\kern0pt}\ x\ {\isasymin}\isactrlsub c\ X\ \ {\isasymLongrightarrow}\ p\ {\isasymcirc}\isactrlsub c\ {\isasymlangle}x{\isacharcomma}{\kern0pt}\ id\ {\isasymone}{\isasymrangle}\ {\isacharequal}{\kern0pt}\ {\isasymt}{\isachardoublequoteclose}\isanewline
%
\isadelimproof
\ \ %
\endisadelimproof
%
\isatagproof
\isacommand{using}\isamarkupfalse%
\ FORALL{\isacharunderscore}{\kern0pt}p{\isacharunderscore}{\kern0pt}true\ FORALL{\isacharunderscore}{\kern0pt}true{\isacharunderscore}{\kern0pt}implies{\isacharunderscore}{\kern0pt}all{\isacharunderscore}{\kern0pt}true{\isadigit{2}}\ id{\isacharunderscore}{\kern0pt}right{\isacharunderscore}{\kern0pt}unit{\isadigit{2}}\ terminal{\isacharunderscore}{\kern0pt}func{\isacharunderscore}{\kern0pt}unique\ \isacommand{by}\isamarkupfalse%
\ {\isacharparenleft}{\kern0pt}typecheck{\isacharunderscore}{\kern0pt}cfuncs{\isacharcomma}{\kern0pt}\ auto{\isacharparenright}{\kern0pt}%
\endisatagproof
{\isafoldproof}%
%
\isadelimproof
\isanewline
%
\endisadelimproof
\isanewline
\isacommand{lemma}\isamarkupfalse%
\ FORALL{\isacharunderscore}{\kern0pt}elim{\isacharcolon}{\kern0pt}\isanewline
\ \ \isakeyword{assumes}\ FORALL{\isacharunderscore}{\kern0pt}p{\isacharunderscore}{\kern0pt}true{\isacharcolon}{\kern0pt}\ {\isachardoublequoteopen}FORALL\ X\ {\isasymcirc}\isactrlsub c\ p\isactrlsup {\isasymsharp}\ {\isacharequal}{\kern0pt}\ {\isasymt}{\isachardoublequoteclose}\ \isakeyword{and}\ p{\isacharunderscore}{\kern0pt}type{\isacharbrackleft}{\kern0pt}type{\isacharunderscore}{\kern0pt}rule{\isacharbrackright}{\kern0pt}{\isacharcolon}{\kern0pt}\ {\isachardoublequoteopen}p\ {\isacharcolon}{\kern0pt}\ X\ {\isasymtimes}\isactrlsub c\ {\isasymone}\ {\isasymrightarrow}\ {\isasymOmega}{\isachardoublequoteclose}\isanewline
\ \ \isakeyword{assumes}\ x{\isacharunderscore}{\kern0pt}type{\isacharbrackleft}{\kern0pt}type{\isacharunderscore}{\kern0pt}rule{\isacharbrackright}{\kern0pt}{\isacharcolon}{\kern0pt}\ {\isachardoublequoteopen}x\ {\isasymin}\isactrlsub c\ X{\isachardoublequoteclose}\isanewline
\ \ \isakeyword{shows}\ {\isachardoublequoteopen}{\isacharparenleft}{\kern0pt}p\ {\isasymcirc}\isactrlsub c\ {\isasymlangle}x{\isacharcomma}{\kern0pt}\ id\ {\isasymone}{\isasymrangle}\ {\isacharequal}{\kern0pt}\ {\isasymt}\ {\isasymLongrightarrow}\ P{\isacharparenright}{\kern0pt}\ {\isasymLongrightarrow}\ P{\isachardoublequoteclose}\isanewline
%
\isadelimproof
\ \ %
\endisadelimproof
%
\isatagproof
\isacommand{using}\isamarkupfalse%
\ FORALL{\isacharunderscore}{\kern0pt}p{\isacharunderscore}{\kern0pt}true\ FORALL{\isacharunderscore}{\kern0pt}true{\isacharunderscore}{\kern0pt}implies{\isacharunderscore}{\kern0pt}all{\isacharunderscore}{\kern0pt}true{\isadigit{3}}\ p{\isacharunderscore}{\kern0pt}type\ x{\isacharunderscore}{\kern0pt}type\ \isacommand{by}\isamarkupfalse%
\ blast%
\endisatagproof
{\isafoldproof}%
%
\isadelimproof
\isanewline
%
\endisadelimproof
\isanewline
\isacommand{lemma}\isamarkupfalse%
\ FORALL{\isacharunderscore}{\kern0pt}elim{\isacharprime}{\kern0pt}{\isacharcolon}{\kern0pt}\isanewline
\ \ \isakeyword{assumes}\ FORALL{\isacharunderscore}{\kern0pt}p{\isacharunderscore}{\kern0pt}true{\isacharcolon}{\kern0pt}\ {\isachardoublequoteopen}FORALL\ X\ {\isasymcirc}\isactrlsub c\ p\isactrlsup {\isasymsharp}\ {\isacharequal}{\kern0pt}\ {\isasymt}{\isachardoublequoteclose}\ \isakeyword{and}\ p{\isacharunderscore}{\kern0pt}type{\isacharbrackleft}{\kern0pt}type{\isacharunderscore}{\kern0pt}rule{\isacharbrackright}{\kern0pt}{\isacharcolon}{\kern0pt}\ {\isachardoublequoteopen}p\ {\isacharcolon}{\kern0pt}\ X\ {\isasymtimes}\isactrlsub c\ {\isasymone}\ {\isasymrightarrow}\ {\isasymOmega}{\isachardoublequoteclose}\isanewline
\ \ \isakeyword{shows}\ {\isachardoublequoteopen}{\isacharparenleft}{\kern0pt}{\isacharparenleft}{\kern0pt}{\isasymAnd}x{\isachardot}{\kern0pt}\ x\ {\isasymin}\isactrlsub c\ X\ {\isasymLongrightarrow}\ p\ {\isasymcirc}\isactrlsub c\ {\isasymlangle}x{\isacharcomma}{\kern0pt}\ id\ {\isasymone}{\isasymrangle}\ {\isacharequal}{\kern0pt}\ {\isasymt}{\isacharparenright}{\kern0pt}\ {\isasymLongrightarrow}\ P{\isacharparenright}{\kern0pt}\ {\isasymLongrightarrow}\ P{\isachardoublequoteclose}\isanewline
%
\isadelimproof
\ \ %
\endisadelimproof
%
\isatagproof
\isacommand{using}\isamarkupfalse%
\ FORALL{\isacharunderscore}{\kern0pt}p{\isacharunderscore}{\kern0pt}true\ FORALL{\isacharunderscore}{\kern0pt}true{\isacharunderscore}{\kern0pt}implies{\isacharunderscore}{\kern0pt}all{\isacharunderscore}{\kern0pt}true{\isadigit{3}}\ p{\isacharunderscore}{\kern0pt}type\ \isacommand{by}\isamarkupfalse%
\ auto%
\endisatagproof
{\isafoldproof}%
%
\isadelimproof
%
\endisadelimproof
%
\isadelimdocument
%
\endisadelimdocument
%
\isatagdocument
%
\isamarkupsubsection{Existential Quantification%
}
\isamarkuptrue%
%
\endisatagdocument
{\isafolddocument}%
%
\isadelimdocument
%
\endisadelimdocument
\isacommand{definition}\isamarkupfalse%
\ EXISTS\ {\isacharcolon}{\kern0pt}{\isacharcolon}{\kern0pt}\ {\isachardoublequoteopen}cset\ {\isasymRightarrow}\ cfunc{\isachardoublequoteclose}\ \isakeyword{where}\isanewline
\ \ {\isachardoublequoteopen}EXISTS\ X\ {\isacharequal}{\kern0pt}\ NOT\ {\isasymcirc}\isactrlsub c\ FORALL\ X\ {\isasymcirc}\isactrlsub c\ NOT\isactrlbsup X\isactrlesup \isactrlsub f{\isachardoublequoteclose}\isanewline
\isanewline
\isacommand{lemma}\isamarkupfalse%
\ EXISTS{\isacharunderscore}{\kern0pt}type{\isacharbrackleft}{\kern0pt}type{\isacharunderscore}{\kern0pt}rule{\isacharbrackright}{\kern0pt}{\isacharcolon}{\kern0pt}\isanewline
\ \ {\isachardoublequoteopen}EXISTS\ X\ {\isacharcolon}{\kern0pt}\ {\isasymOmega}\isactrlbsup X\isactrlesup \ {\isasymrightarrow}\ {\isasymOmega}{\isachardoublequoteclose}\isanewline
%
\isadelimproof
\ \ %
\endisadelimproof
%
\isatagproof
\isacommand{unfolding}\isamarkupfalse%
\ EXISTS{\isacharunderscore}{\kern0pt}def\ \isacommand{by}\isamarkupfalse%
\ typecheck{\isacharunderscore}{\kern0pt}cfuncs%
\endisatagproof
{\isafoldproof}%
%
\isadelimproof
\isanewline
%
\endisadelimproof
\isanewline
\isacommand{lemma}\isamarkupfalse%
\ EXISTS{\isacharunderscore}{\kern0pt}true{\isacharunderscore}{\kern0pt}implies{\isacharunderscore}{\kern0pt}exists{\isacharunderscore}{\kern0pt}true{\isacharcolon}{\kern0pt}\isanewline
\ \ \isakeyword{assumes}\ p{\isacharunderscore}{\kern0pt}type{\isacharcolon}{\kern0pt}\ {\isachardoublequoteopen}p\ {\isacharcolon}{\kern0pt}\ X\ {\isasymrightarrow}\ {\isasymOmega}{\isachardoublequoteclose}\ \isakeyword{and}\ EXISTS{\isacharunderscore}{\kern0pt}p{\isacharunderscore}{\kern0pt}true{\isacharcolon}{\kern0pt}\ {\isachardoublequoteopen}EXISTS\ X\ {\isasymcirc}\isactrlsub c\ {\isacharparenleft}{\kern0pt}p\ {\isasymcirc}\isactrlsub c\ left{\isacharunderscore}{\kern0pt}cart{\isacharunderscore}{\kern0pt}proj\ X\ {\isasymone}{\isacharparenright}{\kern0pt}\isactrlsup {\isasymsharp}\ {\isacharequal}{\kern0pt}\ {\isasymt}{\isachardoublequoteclose}\isanewline
\ \ \isakeyword{shows}\ {\isachardoublequoteopen}{\isasymexists}\ x{\isachardot}{\kern0pt}\ x\ {\isasymin}\isactrlsub c\ X\ {\isasymand}\ p\ {\isasymcirc}\isactrlsub c\ x\ {\isacharequal}{\kern0pt}\ {\isasymt}{\isachardoublequoteclose}\isanewline
%
\isadelimproof
%
\endisadelimproof
%
\isatagproof
\isacommand{proof}\isamarkupfalse%
\ {\isacharminus}{\kern0pt}\isanewline
\ \ \isacommand{have}\isamarkupfalse%
\ {\isachardoublequoteopen}NOT\ {\isasymcirc}\isactrlsub c\ FORALL\ X\ {\isasymcirc}\isactrlsub c\ NOT\isactrlbsup X\isactrlesup \isactrlsub f\ {\isasymcirc}\isactrlsub c\ {\isacharparenleft}{\kern0pt}p\ {\isasymcirc}\isactrlsub c\ left{\isacharunderscore}{\kern0pt}cart{\isacharunderscore}{\kern0pt}proj\ X\ {\isasymone}{\isacharparenright}{\kern0pt}\isactrlsup {\isasymsharp}\ {\isacharequal}{\kern0pt}\ {\isasymt}{\isachardoublequoteclose}\isanewline
\ \ \ \ \isacommand{using}\isamarkupfalse%
\ p{\isacharunderscore}{\kern0pt}type\ EXISTS{\isacharunderscore}{\kern0pt}p{\isacharunderscore}{\kern0pt}true\ cfunc{\isacharunderscore}{\kern0pt}type{\isacharunderscore}{\kern0pt}def\ comp{\isacharunderscore}{\kern0pt}associative\ comp{\isacharunderscore}{\kern0pt}type\isanewline
\ \ \ \ \isacommand{unfolding}\isamarkupfalse%
\ EXISTS{\isacharunderscore}{\kern0pt}def\isanewline
\ \ \ \ \isacommand{by}\isamarkupfalse%
\ {\isacharparenleft}{\kern0pt}typecheck{\isacharunderscore}{\kern0pt}cfuncs{\isacharcomma}{\kern0pt}\ auto{\isacharparenright}{\kern0pt}\isanewline
\ \ \isacommand{then}\isamarkupfalse%
\ \isacommand{have}\isamarkupfalse%
\ {\isachardoublequoteopen}NOT\ {\isasymcirc}\isactrlsub c\ FORALL\ X\ {\isasymcirc}\isactrlsub c\ {\isacharparenleft}{\kern0pt}NOT\ {\isasymcirc}\isactrlsub c\ p\ {\isasymcirc}\isactrlsub c\ left{\isacharunderscore}{\kern0pt}cart{\isacharunderscore}{\kern0pt}proj\ X\ {\isasymone}{\isacharparenright}{\kern0pt}\isactrlsup {\isasymsharp}\ {\isacharequal}{\kern0pt}\ {\isasymt}{\isachardoublequoteclose}\isanewline
\ \ \ \ \isacommand{using}\isamarkupfalse%
\ p{\isacharunderscore}{\kern0pt}type\ transpose{\isacharunderscore}{\kern0pt}of{\isacharunderscore}{\kern0pt}comp\ \isacommand{by}\isamarkupfalse%
\ {\isacharparenleft}{\kern0pt}typecheck{\isacharunderscore}{\kern0pt}cfuncs{\isacharcomma}{\kern0pt}\ auto{\isacharparenright}{\kern0pt}\isanewline
\ \ \isacommand{then}\isamarkupfalse%
\ \isacommand{have}\isamarkupfalse%
\ {\isachardoublequoteopen}FORALL\ X\ {\isasymcirc}\isactrlsub c\ {\isacharparenleft}{\kern0pt}NOT\ {\isasymcirc}\isactrlsub c\ p\ {\isasymcirc}\isactrlsub c\ left{\isacharunderscore}{\kern0pt}cart{\isacharunderscore}{\kern0pt}proj\ X\ {\isasymone}{\isacharparenright}{\kern0pt}\isactrlsup {\isasymsharp}\ {\isasymnoteq}\ {\isasymt}{\isachardoublequoteclose}\isanewline
\ \ \ \ \isacommand{using}\isamarkupfalse%
\ NOT{\isacharunderscore}{\kern0pt}true{\isacharunderscore}{\kern0pt}is{\isacharunderscore}{\kern0pt}false\ true{\isacharunderscore}{\kern0pt}false{\isacharunderscore}{\kern0pt}distinct\ \isacommand{by}\isamarkupfalse%
\ auto\isanewline
\ \ \isacommand{then}\isamarkupfalse%
\ \isacommand{have}\isamarkupfalse%
\ {\isachardoublequoteopen}FORALL\ X\ {\isasymcirc}\isactrlsub c\ {\isacharparenleft}{\kern0pt}{\isacharparenleft}{\kern0pt}NOT\ {\isasymcirc}\isactrlsub c\ p{\isacharparenright}{\kern0pt}\ {\isasymcirc}\isactrlsub c\ left{\isacharunderscore}{\kern0pt}cart{\isacharunderscore}{\kern0pt}proj\ X\ {\isasymone}{\isacharparenright}{\kern0pt}\isactrlsup {\isasymsharp}\ {\isasymnoteq}\ {\isasymt}{\isachardoublequoteclose}\isanewline
\ \ \ \ \isacommand{using}\isamarkupfalse%
\ p{\isacharunderscore}{\kern0pt}type\ comp{\isacharunderscore}{\kern0pt}associative{\isadigit{2}}\ \isacommand{by}\isamarkupfalse%
\ {\isacharparenleft}{\kern0pt}typecheck{\isacharunderscore}{\kern0pt}cfuncs{\isacharcomma}{\kern0pt}\ auto{\isacharparenright}{\kern0pt}\isanewline
\ \ \isacommand{then}\isamarkupfalse%
\ \isacommand{have}\isamarkupfalse%
\ {\isachardoublequoteopen}{\isasymnot}\ {\isacharparenleft}{\kern0pt}{\isasymforall}\ x{\isachardot}{\kern0pt}\ x\ {\isasymin}\isactrlsub c\ X\ {\isasymlongrightarrow}\ {\isacharparenleft}{\kern0pt}NOT\ {\isasymcirc}\isactrlsub c\ p{\isacharparenright}{\kern0pt}\ {\isasymcirc}\isactrlsub c\ x\ {\isacharequal}{\kern0pt}\ {\isasymt}{\isacharparenright}{\kern0pt}{\isachardoublequoteclose}\isanewline
\ \ \ \ \isacommand{using}\isamarkupfalse%
\ NOT{\isacharunderscore}{\kern0pt}type\ all{\isacharunderscore}{\kern0pt}true{\isacharunderscore}{\kern0pt}implies{\isacharunderscore}{\kern0pt}FORALL{\isacharunderscore}{\kern0pt}true\ comp{\isacharunderscore}{\kern0pt}type\ p{\isacharunderscore}{\kern0pt}type\ \isacommand{by}\isamarkupfalse%
\ blast\isanewline
\ \ \isacommand{then}\isamarkupfalse%
\ \isacommand{have}\isamarkupfalse%
\ {\isachardoublequoteopen}{\isasymnot}\ {\isacharparenleft}{\kern0pt}{\isasymforall}\ x{\isachardot}{\kern0pt}\ x\ {\isasymin}\isactrlsub c\ X\ {\isasymlongrightarrow}\ NOT\ {\isasymcirc}\isactrlsub c\ {\isacharparenleft}{\kern0pt}p\ {\isasymcirc}\isactrlsub c\ x{\isacharparenright}{\kern0pt}\ {\isacharequal}{\kern0pt}\ {\isasymt}{\isacharparenright}{\kern0pt}{\isachardoublequoteclose}\isanewline
\ \ \ \ \isacommand{using}\isamarkupfalse%
\ p{\isacharunderscore}{\kern0pt}type\ comp{\isacharunderscore}{\kern0pt}associative{\isadigit{2}}\ \isacommand{by}\isamarkupfalse%
\ {\isacharparenleft}{\kern0pt}typecheck{\isacharunderscore}{\kern0pt}cfuncs{\isacharcomma}{\kern0pt}\ auto{\isacharparenright}{\kern0pt}\isanewline
\ \ \isacommand{then}\isamarkupfalse%
\ \isacommand{have}\isamarkupfalse%
\ {\isachardoublequoteopen}{\isasymnot}\ {\isacharparenleft}{\kern0pt}{\isasymforall}\ x{\isachardot}{\kern0pt}\ x\ {\isasymin}\isactrlsub c\ X\ {\isasymlongrightarrow}\ p\ {\isasymcirc}\isactrlsub c\ x\ {\isasymnoteq}\ {\isasymt}{\isacharparenright}{\kern0pt}{\isachardoublequoteclose}\isanewline
\ \ \ \ \isacommand{using}\isamarkupfalse%
\ NOT{\isacharunderscore}{\kern0pt}false{\isacharunderscore}{\kern0pt}is{\isacharunderscore}{\kern0pt}true\ comp{\isacharunderscore}{\kern0pt}type\ p{\isacharunderscore}{\kern0pt}type\ true{\isacharunderscore}{\kern0pt}false{\isacharunderscore}{\kern0pt}only{\isacharunderscore}{\kern0pt}truth{\isacharunderscore}{\kern0pt}values\ \isacommand{by}\isamarkupfalse%
\ fastforce\isanewline
\ \ \isacommand{then}\isamarkupfalse%
\ \isacommand{show}\isamarkupfalse%
\ {\isachardoublequoteopen}{\isasymexists}x{\isachardot}{\kern0pt}\ x\ {\isasymin}\isactrlsub c\ X\ {\isasymand}\ p\ {\isasymcirc}\isactrlsub c\ x\ {\isacharequal}{\kern0pt}\ {\isasymt}{\isachardoublequoteclose}\isanewline
\ \ \ \ \isacommand{by}\isamarkupfalse%
\ blast\isanewline
\isacommand{qed}\isamarkupfalse%
%
\endisatagproof
{\isafoldproof}%
%
\isadelimproof
\isanewline
%
\endisadelimproof
\isanewline
\isacommand{lemma}\isamarkupfalse%
\ EXISTS{\isacharunderscore}{\kern0pt}elim{\isacharcolon}{\kern0pt}\isanewline
\ \ \isakeyword{assumes}\ EXISTS{\isacharunderscore}{\kern0pt}p{\isacharunderscore}{\kern0pt}true{\isacharcolon}{\kern0pt}\ {\isachardoublequoteopen}EXISTS\ X\ {\isasymcirc}\isactrlsub c\ {\isacharparenleft}{\kern0pt}p\ {\isasymcirc}\isactrlsub c\ left{\isacharunderscore}{\kern0pt}cart{\isacharunderscore}{\kern0pt}proj\ X\ {\isasymone}{\isacharparenright}{\kern0pt}\isactrlsup {\isasymsharp}\ {\isacharequal}{\kern0pt}\ {\isasymt}{\isachardoublequoteclose}\ \isakeyword{and}\ p{\isacharunderscore}{\kern0pt}type{\isacharcolon}{\kern0pt}\ {\isachardoublequoteopen}p\ {\isacharcolon}{\kern0pt}\ X\ {\isasymrightarrow}\ {\isasymOmega}{\isachardoublequoteclose}\isanewline
\ \ \isakeyword{shows}\ {\isachardoublequoteopen}{\isacharparenleft}{\kern0pt}{\isasymAnd}\ x{\isachardot}{\kern0pt}\ x\ {\isasymin}\isactrlsub c\ X\ {\isasymLongrightarrow}\ p\ {\isasymcirc}\isactrlsub c\ x\ {\isacharequal}{\kern0pt}\ {\isasymt}\ {\isasymLongrightarrow}\ Q{\isacharparenright}{\kern0pt}\ {\isasymLongrightarrow}\ Q{\isachardoublequoteclose}\isanewline
%
\isadelimproof
\ \ %
\endisadelimproof
%
\isatagproof
\isacommand{using}\isamarkupfalse%
\ EXISTS{\isacharunderscore}{\kern0pt}p{\isacharunderscore}{\kern0pt}true\ EXISTS{\isacharunderscore}{\kern0pt}true{\isacharunderscore}{\kern0pt}implies{\isacharunderscore}{\kern0pt}exists{\isacharunderscore}{\kern0pt}true\ p{\isacharunderscore}{\kern0pt}type\ \isacommand{by}\isamarkupfalse%
\ auto%
\endisatagproof
{\isafoldproof}%
%
\isadelimproof
\isanewline
%
\endisadelimproof
\isanewline
\isacommand{lemma}\isamarkupfalse%
\ exists{\isacharunderscore}{\kern0pt}true{\isacharunderscore}{\kern0pt}implies{\isacharunderscore}{\kern0pt}EXISTS{\isacharunderscore}{\kern0pt}true{\isacharcolon}{\kern0pt}\isanewline
\ \ \isakeyword{assumes}\ p{\isacharunderscore}{\kern0pt}type{\isacharcolon}{\kern0pt}\ {\isachardoublequoteopen}p\ {\isacharcolon}{\kern0pt}\ X\ {\isasymrightarrow}\ {\isasymOmega}{\isachardoublequoteclose}\ \isakeyword{and}\ exists{\isacharunderscore}{\kern0pt}p{\isacharunderscore}{\kern0pt}true{\isacharcolon}{\kern0pt}\ {\isachardoublequoteopen}{\isasymexists}\ x{\isachardot}{\kern0pt}\ x\ {\isasymin}\isactrlsub c\ X\ {\isasymand}\ p\ {\isasymcirc}\isactrlsub c\ x\ {\isacharequal}{\kern0pt}\ {\isasymt}{\isachardoublequoteclose}\isanewline
\ \ \isakeyword{shows}\ \ {\isachardoublequoteopen}EXISTS\ X\ {\isasymcirc}\isactrlsub c\ {\isacharparenleft}{\kern0pt}p\ {\isasymcirc}\isactrlsub c\ left{\isacharunderscore}{\kern0pt}cart{\isacharunderscore}{\kern0pt}proj\ X\ {\isasymone}{\isacharparenright}{\kern0pt}\isactrlsup {\isasymsharp}\ {\isacharequal}{\kern0pt}\ {\isasymt}{\isachardoublequoteclose}\isanewline
%
\isadelimproof
%
\endisadelimproof
%
\isatagproof
\isacommand{proof}\isamarkupfalse%
\ {\isacharminus}{\kern0pt}\isanewline
\ \isacommand{have}\isamarkupfalse%
\ {\isachardoublequoteopen}{\isasymnot}\ {\isacharparenleft}{\kern0pt}{\isasymforall}\ x{\isachardot}{\kern0pt}\ x\ {\isasymin}\isactrlsub c\ X\ {\isasymlongrightarrow}\ p\ {\isasymcirc}\isactrlsub c\ x\ {\isasymnoteq}\ {\isasymt}{\isacharparenright}{\kern0pt}{\isachardoublequoteclose}\isanewline
\ \ \ \isacommand{using}\isamarkupfalse%
\ exists{\isacharunderscore}{\kern0pt}p{\isacharunderscore}{\kern0pt}true\ \isacommand{by}\isamarkupfalse%
\ blast\isanewline
\ \isacommand{then}\isamarkupfalse%
\ \isacommand{have}\isamarkupfalse%
\ {\isachardoublequoteopen}{\isasymnot}\ {\isacharparenleft}{\kern0pt}{\isasymforall}\ x{\isachardot}{\kern0pt}\ x\ {\isasymin}\isactrlsub c\ X\ {\isasymlongrightarrow}\ NOT\ {\isasymcirc}\isactrlsub c\ {\isacharparenleft}{\kern0pt}p\ {\isasymcirc}\isactrlsub c\ x{\isacharparenright}{\kern0pt}\ {\isacharequal}{\kern0pt}\ {\isasymt}{\isacharparenright}{\kern0pt}{\isachardoublequoteclose}\isanewline
\ \ \ \isacommand{using}\isamarkupfalse%
\ NOT{\isacharunderscore}{\kern0pt}true{\isacharunderscore}{\kern0pt}is{\isacharunderscore}{\kern0pt}false\ true{\isacharunderscore}{\kern0pt}false{\isacharunderscore}{\kern0pt}distinct\ \isacommand{by}\isamarkupfalse%
\ auto\isanewline
\ \isacommand{then}\isamarkupfalse%
\ \isacommand{have}\isamarkupfalse%
\ {\isachardoublequoteopen}{\isasymnot}\ {\isacharparenleft}{\kern0pt}{\isasymforall}\ x{\isachardot}{\kern0pt}\ x\ {\isasymin}\isactrlsub c\ X\ {\isasymlongrightarrow}\ {\isacharparenleft}{\kern0pt}NOT\ {\isasymcirc}\isactrlsub c\ p{\isacharparenright}{\kern0pt}\ {\isasymcirc}\isactrlsub c\ x\ {\isacharequal}{\kern0pt}\ {\isasymt}{\isacharparenright}{\kern0pt}{\isachardoublequoteclose}\isanewline
\ \ \ \isacommand{using}\isamarkupfalse%
\ p{\isacharunderscore}{\kern0pt}type\ \isacommand{by}\isamarkupfalse%
\ {\isacharparenleft}{\kern0pt}typecheck{\isacharunderscore}{\kern0pt}cfuncs{\isacharcomma}{\kern0pt}\ metis\ NOT{\isacharunderscore}{\kern0pt}true{\isacharunderscore}{\kern0pt}is{\isacharunderscore}{\kern0pt}false\ cfunc{\isacharunderscore}{\kern0pt}type{\isacharunderscore}{\kern0pt}def\ comp{\isacharunderscore}{\kern0pt}associative\ exists{\isacharunderscore}{\kern0pt}p{\isacharunderscore}{\kern0pt}true\ true{\isacharunderscore}{\kern0pt}false{\isacharunderscore}{\kern0pt}distinct{\isacharparenright}{\kern0pt}\isanewline
\ \isacommand{then}\isamarkupfalse%
\ \isacommand{have}\isamarkupfalse%
\ {\isachardoublequoteopen}FORALL\ X\ {\isasymcirc}\isactrlsub c\ {\isacharparenleft}{\kern0pt}{\isacharparenleft}{\kern0pt}NOT\ {\isasymcirc}\isactrlsub c\ p{\isacharparenright}{\kern0pt}\ {\isasymcirc}\isactrlsub c\ left{\isacharunderscore}{\kern0pt}cart{\isacharunderscore}{\kern0pt}proj\ X\ {\isasymone}{\isacharparenright}{\kern0pt}\isactrlsup {\isasymsharp}\ {\isasymnoteq}\ {\isasymt}{\isachardoublequoteclose}\isanewline
\ \ \ \isacommand{using}\isamarkupfalse%
\ FORALL{\isacharunderscore}{\kern0pt}true{\isacharunderscore}{\kern0pt}implies{\isacharunderscore}{\kern0pt}all{\isacharunderscore}{\kern0pt}true\ NOT{\isacharunderscore}{\kern0pt}type\ comp{\isacharunderscore}{\kern0pt}type\ p{\isacharunderscore}{\kern0pt}type\ \isacommand{by}\isamarkupfalse%
\ blast\isanewline
\ \isacommand{then}\isamarkupfalse%
\ \isacommand{have}\isamarkupfalse%
\ {\isachardoublequoteopen}FORALL\ X\ {\isasymcirc}\isactrlsub c\ {\isacharparenleft}{\kern0pt}NOT\ {\isasymcirc}\isactrlsub c\ p\ {\isasymcirc}\isactrlsub c\ left{\isacharunderscore}{\kern0pt}cart{\isacharunderscore}{\kern0pt}proj\ X\ {\isasymone}{\isacharparenright}{\kern0pt}\isactrlsup {\isasymsharp}\ {\isasymnoteq}\ {\isasymt}{\isachardoublequoteclose}\isanewline
\ \ \ \isacommand{using}\isamarkupfalse%
\ NOT{\isacharunderscore}{\kern0pt}type\ cfunc{\isacharunderscore}{\kern0pt}type{\isacharunderscore}{\kern0pt}def\ comp{\isacharunderscore}{\kern0pt}associative\ left{\isacharunderscore}{\kern0pt}cart{\isacharunderscore}{\kern0pt}proj{\isacharunderscore}{\kern0pt}type\ p{\isacharunderscore}{\kern0pt}type\ \isacommand{by}\isamarkupfalse%
\ auto\isanewline
\ \isacommand{then}\isamarkupfalse%
\ \isacommand{have}\isamarkupfalse%
\ {\isachardoublequoteopen}NOT\ {\isasymcirc}\isactrlsub c\ FORALL\ X\ {\isasymcirc}\isactrlsub c\ {\isacharparenleft}{\kern0pt}NOT\ {\isasymcirc}\isactrlsub c\ p\ {\isasymcirc}\isactrlsub c\ left{\isacharunderscore}{\kern0pt}cart{\isacharunderscore}{\kern0pt}proj\ X\ {\isasymone}{\isacharparenright}{\kern0pt}\isactrlsup {\isasymsharp}\ {\isacharequal}{\kern0pt}\ {\isasymt}{\isachardoublequoteclose}\isanewline
\ \ \ \isacommand{using}\isamarkupfalse%
\ assms\ NOT{\isacharunderscore}{\kern0pt}is{\isacharunderscore}{\kern0pt}false{\isacharunderscore}{\kern0pt}implies{\isacharunderscore}{\kern0pt}true\ true{\isacharunderscore}{\kern0pt}false{\isacharunderscore}{\kern0pt}only{\isacharunderscore}{\kern0pt}truth{\isacharunderscore}{\kern0pt}values\ \isacommand{by}\isamarkupfalse%
\ {\isacharparenleft}{\kern0pt}typecheck{\isacharunderscore}{\kern0pt}cfuncs{\isacharcomma}{\kern0pt}\ blast{\isacharparenright}{\kern0pt}\isanewline
\ \isacommand{then}\isamarkupfalse%
\ \isacommand{have}\isamarkupfalse%
\ {\isachardoublequoteopen}NOT\ {\isasymcirc}\isactrlsub c\ FORALL\ X\ {\isasymcirc}\isactrlsub c\ NOT\isactrlbsup X\isactrlesup \isactrlsub f\ {\isasymcirc}\isactrlsub c\ {\isacharparenleft}{\kern0pt}p\ {\isasymcirc}\isactrlsub c\ left{\isacharunderscore}{\kern0pt}cart{\isacharunderscore}{\kern0pt}proj\ X\ {\isasymone}{\isacharparenright}{\kern0pt}\isactrlsup {\isasymsharp}\ {\isacharequal}{\kern0pt}\ {\isasymt}{\isachardoublequoteclose}\isanewline
\ \ \ \isacommand{using}\isamarkupfalse%
\ assms\ transpose{\isacharunderscore}{\kern0pt}of{\isacharunderscore}{\kern0pt}comp\ \isacommand{by}\isamarkupfalse%
\ {\isacharparenleft}{\kern0pt}typecheck{\isacharunderscore}{\kern0pt}cfuncs{\isacharcomma}{\kern0pt}\ auto{\isacharparenright}{\kern0pt}\isanewline
\ \isacommand{then}\isamarkupfalse%
\ \isacommand{have}\isamarkupfalse%
\ {\isachardoublequoteopen}{\isacharparenleft}{\kern0pt}NOT\ {\isasymcirc}\isactrlsub c\ FORALL\ X\ {\isasymcirc}\isactrlsub c\ NOT\isactrlbsup X\isactrlesup \isactrlsub f{\isacharparenright}{\kern0pt}\ {\isasymcirc}\isactrlsub c\ {\isacharparenleft}{\kern0pt}p\ {\isasymcirc}\isactrlsub c\ left{\isacharunderscore}{\kern0pt}cart{\isacharunderscore}{\kern0pt}proj\ X\ {\isasymone}{\isacharparenright}{\kern0pt}\isactrlsup {\isasymsharp}\ {\isacharequal}{\kern0pt}\ {\isasymt}{\isachardoublequoteclose}\isanewline
\ \ \ \ \isacommand{using}\isamarkupfalse%
\ assms\ cfunc{\isacharunderscore}{\kern0pt}type{\isacharunderscore}{\kern0pt}def\ comp{\isacharunderscore}{\kern0pt}associative\ \isacommand{by}\isamarkupfalse%
\ {\isacharparenleft}{\kern0pt}typecheck{\isacharunderscore}{\kern0pt}cfuncs{\isacharcomma}{\kern0pt}auto{\isacharparenright}{\kern0pt}\isanewline
\ \isacommand{then}\isamarkupfalse%
\ \isacommand{show}\isamarkupfalse%
\ {\isachardoublequoteopen}EXISTS\ X\ {\isasymcirc}\isactrlsub c\ {\isacharparenleft}{\kern0pt}p\ {\isasymcirc}\isactrlsub c\ left{\isacharunderscore}{\kern0pt}cart{\isacharunderscore}{\kern0pt}proj\ X\ {\isasymone}{\isacharparenright}{\kern0pt}\isactrlsup {\isasymsharp}\ {\isacharequal}{\kern0pt}\ {\isasymt}{\isachardoublequoteclose}\isanewline
\ \ \isacommand{by}\isamarkupfalse%
\ {\isacharparenleft}{\kern0pt}simp\ add{\isacharcolon}{\kern0pt}\ EXISTS{\isacharunderscore}{\kern0pt}def{\isacharparenright}{\kern0pt}\isanewline
\isacommand{qed}\isamarkupfalse%
%
\endisatagproof
{\isafoldproof}%
%
\isadelimproof
\isanewline
%
\endisadelimproof
%
\isadelimtheory
\isanewline
%
\endisadelimtheory
%
\isatagtheory
\isacommand{end}\isamarkupfalse%
%
\endisatagtheory
{\isafoldtheory}%
%
\isadelimtheory
%
\endisadelimtheory
%
\end{isabellebody}%
\endinput
%:%file=~/ETCS/Category_Set/Quant_Logic.thy%:%
%:%11=1%:%
%:%27=3%:%
%:%28=3%:%
%:%29=4%:%
%:%30=5%:%
%:%44=7%:%
%:%54=9%:%
%:%55=9%:%
%:%56=10%:%
%:%57=11%:%
%:%58=12%:%
%:%59=12%:%
%:%60=13%:%
%:%63=14%:%
%:%67=14%:%
%:%68=14%:%
%:%69=15%:%
%:%70=15%:%
%:%71=16%:%
%:%72=16%:%
%:%77=16%:%
%:%80=17%:%
%:%81=18%:%
%:%82=18%:%
%:%83=19%:%
%:%86=20%:%
%:%90=20%:%
%:%91=20%:%
%:%92=20%:%
%:%93=20%:%
%:%98=20%:%
%:%101=21%:%
%:%102=22%:%
%:%103=22%:%
%:%104=23%:%
%:%105=24%:%
%:%112=25%:%
%:%113=25%:%
%:%114=26%:%
%:%115=26%:%
%:%116=27%:%
%:%117=27%:%
%:%118=28%:%
%:%119=28%:%
%:%120=29%:%
%:%121=29%:%
%:%122=30%:%
%:%123=31%:%
%:%124=31%:%
%:%125=32%:%
%:%126=32%:%
%:%127=32%:%
%:%128=33%:%
%:%129=33%:%
%:%130=33%:%
%:%131=34%:%
%:%132=34%:%
%:%133=34%:%
%:%134=35%:%
%:%135=35%:%
%:%136=35%:%
%:%137=36%:%
%:%138=36%:%
%:%139=36%:%
%:%140=37%:%
%:%141=37%:%
%:%142=37%:%
%:%143=38%:%
%:%144=38%:%
%:%145=38%:%
%:%146=39%:%
%:%147=40%:%
%:%148=40%:%
%:%149=40%:%
%:%150=41%:%
%:%151=41%:%
%:%152=41%:%
%:%153=42%:%
%:%154=42%:%
%:%155=43%:%
%:%156=43%:%
%:%157=43%:%
%:%158=44%:%
%:%159=44%:%
%:%160=45%:%
%:%161=45%:%
%:%162=45%:%
%:%163=46%:%
%:%164=46%:%
%:%165=46%:%
%:%166=46%:%
%:%167=47%:%
%:%168=47%:%
%:%169=47%:%
%:%170=48%:%
%:%171=48%:%
%:%172=48%:%
%:%173=49%:%
%:%179=49%:%
%:%182=50%:%
%:%183=51%:%
%:%184=51%:%
%:%185=52%:%
%:%186=53%:%
%:%193=54%:%
%:%194=54%:%
%:%195=55%:%
%:%196=55%:%
%:%197=56%:%
%:%198=56%:%
%:%199=57%:%
%:%200=57%:%
%:%201=58%:%
%:%202=58%:%
%:%203=59%:%
%:%204=59%:%
%:%205=59%:%
%:%206=60%:%
%:%207=60%:%
%:%208=60%:%
%:%209=61%:%
%:%210=61%:%
%:%211=61%:%
%:%212=62%:%
%:%213=62%:%
%:%214=63%:%
%:%215=63%:%
%:%216=63%:%
%:%217=64%:%
%:%218=64%:%
%:%219=65%:%
%:%220=65%:%
%:%221=66%:%
%:%222=66%:%
%:%223=66%:%
%:%224=67%:%
%:%225=67%:%
%:%226=68%:%
%:%227=68%:%
%:%228=68%:%
%:%229=69%:%
%:%230=69%:%
%:%231=70%:%
%:%232=70%:%
%:%233=70%:%
%:%234=71%:%
%:%235=71%:%
%:%236=72%:%
%:%237=72%:%
%:%238=72%:%
%:%239=73%:%
%:%240=73%:%
%:%241=74%:%
%:%242=74%:%
%:%243=74%:%
%:%244=75%:%
%:%245=75%:%
%:%246=76%:%
%:%247=76%:%
%:%248=76%:%
%:%249=77%:%
%:%250=77%:%
%:%251=77%:%
%:%252=77%:%
%:%253=78%:%
%:%254=78%:%
%:%255=78%:%
%:%256=79%:%
%:%257=79%:%
%:%258=80%:%
%:%264=80%:%
%:%267=81%:%
%:%268=82%:%
%:%269=82%:%
%:%270=83%:%
%:%271=84%:%
%:%278=85%:%
%:%279=85%:%
%:%280=86%:%
%:%281=86%:%
%:%282=87%:%
%:%283=87%:%
%:%284=88%:%
%:%285=88%:%
%:%286=88%:%
%:%287=89%:%
%:%288=89%:%
%:%289=90%:%
%:%295=90%:%
%:%298=91%:%
%:%299=92%:%
%:%300=92%:%
%:%301=93%:%
%:%302=94%:%
%:%309=95%:%
%:%310=95%:%
%:%311=96%:%
%:%312=96%:%
%:%313=97%:%
%:%314=97%:%
%:%315=98%:%
%:%316=98%:%
%:%317=99%:%
%:%318=99%:%
%:%319=99%:%
%:%320=100%:%
%:%321=100%:%
%:%322=100%:%
%:%323=101%:%
%:%324=101%:%
%:%325=101%:%
%:%326=102%:%
%:%327=102%:%
%:%328=102%:%
%:%329=103%:%
%:%330=103%:%
%:%331=103%:%
%:%332=104%:%
%:%333=104%:%
%:%334=104%:%
%:%335=105%:%
%:%336=106%:%
%:%337=106%:%
%:%338=107%:%
%:%339=107%:%
%:%340=107%:%
%:%341=108%:%
%:%342=108%:%
%:%343=108%:%
%:%344=109%:%
%:%345=110%:%
%:%346=111%:%
%:%347=112%:%
%:%348=112%:%
%:%349=112%:%
%:%350=112%:%
%:%351=113%:%
%:%352=113%:%
%:%353=113%:%
%:%354=114%:%
%:%355=114%:%
%:%356=114%:%
%:%357=115%:%
%:%358=115%:%
%:%359=115%:%
%:%360=116%:%
%:%361=116%:%
%:%362=116%:%
%:%363=117%:%
%:%364=117%:%
%:%365=117%:%
%:%366=118%:%
%:%367=118%:%
%:%368=118%:%
%:%369=119%:%
%:%370=119%:%
%:%371=119%:%
%:%372=120%:%
%:%373=120%:%
%:%374=120%:%
%:%375=121%:%
%:%376=122%:%
%:%377=122%:%
%:%378=123%:%
%:%379=123%:%
%:%380=123%:%
%:%381=124%:%
%:%382=124%:%
%:%383=124%:%
%:%384=125%:%
%:%385=125%:%
%:%386=126%:%
%:%387=126%:%
%:%388=126%:%
%:%389=127%:%
%:%390=127%:%
%:%391=127%:%
%:%392=128%:%
%:%393=128%:%
%:%394=128%:%
%:%395=129%:%
%:%396=129%:%
%:%397=129%:%
%:%398=130%:%
%:%404=130%:%
%:%407=131%:%
%:%408=132%:%
%:%409=132%:%
%:%410=133%:%
%:%411=134%:%
%:%418=135%:%
%:%419=135%:%
%:%420=136%:%
%:%421=136%:%
%:%422=137%:%
%:%423=137%:%
%:%424=137%:%
%:%425=138%:%
%:%426=138%:%
%:%427=139%:%
%:%428=139%:%
%:%429=139%:%
%:%430=140%:%
%:%431=140%:%
%:%432=141%:%
%:%433=141%:%
%:%434=141%:%
%:%435=142%:%
%:%436=142%:%
%:%437=143%:%
%:%438=143%:%
%:%439=143%:%
%:%440=144%:%
%:%441=144%:%
%:%442=145%:%
%:%443=145%:%
%:%444=145%:%
%:%445=146%:%
%:%446=146%:%
%:%447=147%:%
%:%448=147%:%
%:%449=147%:%
%:%450=148%:%
%:%451=148%:%
%:%452=149%:%
%:%453=149%:%
%:%454=150%:%
%:%455=150%:%
%:%456=151%:%
%:%457=151%:%
%:%458=152%:%
%:%459=152%:%
%:%460=152%:%
%:%461=153%:%
%:%462=153%:%
%:%463=153%:%
%:%464=154%:%
%:%465=154%:%
%:%466=154%:%
%:%467=155%:%
%:%468=155%:%
%:%469=156%:%
%:%470=156%:%
%:%471=156%:%
%:%472=157%:%
%:%473=157%:%
%:%474=158%:%
%:%475=158%:%
%:%476=159%:%
%:%482=159%:%
%:%485=160%:%
%:%486=161%:%
%:%487=161%:%
%:%488=162%:%
%:%489=163%:%
%:%492=164%:%
%:%496=164%:%
%:%497=164%:%
%:%498=164%:%
%:%503=164%:%
%:%506=165%:%
%:%507=166%:%
%:%508=166%:%
%:%509=167%:%
%:%510=168%:%
%:%511=169%:%
%:%514=170%:%
%:%518=170%:%
%:%519=170%:%
%:%520=170%:%
%:%525=170%:%
%:%528=171%:%
%:%529=172%:%
%:%530=172%:%
%:%531=173%:%
%:%532=174%:%
%:%535=175%:%
%:%539=175%:%
%:%540=175%:%
%:%541=175%:%
%:%555=177%:%
%:%565=179%:%
%:%566=179%:%
%:%567=180%:%
%:%568=181%:%
%:%569=182%:%
%:%570=182%:%
%:%571=183%:%
%:%574=184%:%
%:%578=184%:%
%:%579=184%:%
%:%580=184%:%
%:%585=184%:%
%:%588=185%:%
%:%589=186%:%
%:%590=186%:%
%:%591=187%:%
%:%592=188%:%
%:%599=189%:%
%:%600=189%:%
%:%601=190%:%
%:%602=190%:%
%:%603=191%:%
%:%604=191%:%
%:%605=192%:%
%:%606=192%:%
%:%607=193%:%
%:%608=193%:%
%:%609=194%:%
%:%610=194%:%
%:%611=194%:%
%:%612=195%:%
%:%613=195%:%
%:%614=195%:%
%:%615=196%:%
%:%616=196%:%
%:%617=196%:%
%:%618=197%:%
%:%619=197%:%
%:%620=197%:%
%:%621=198%:%
%:%622=198%:%
%:%623=198%:%
%:%624=199%:%
%:%625=199%:%
%:%626=199%:%
%:%627=200%:%
%:%628=200%:%
%:%629=200%:%
%:%630=201%:%
%:%631=201%:%
%:%632=201%:%
%:%633=202%:%
%:%634=202%:%
%:%635=202%:%
%:%636=203%:%
%:%637=203%:%
%:%638=203%:%
%:%639=204%:%
%:%640=204%:%
%:%641=204%:%
%:%642=205%:%
%:%643=205%:%
%:%644=205%:%
%:%645=206%:%
%:%646=206%:%
%:%647=206%:%
%:%648=207%:%
%:%649=207%:%
%:%650=208%:%
%:%656=208%:%
%:%659=209%:%
%:%660=210%:%
%:%661=210%:%
%:%662=211%:%
%:%663=212%:%
%:%666=213%:%
%:%670=213%:%
%:%671=213%:%
%:%672=213%:%
%:%677=213%:%
%:%680=214%:%
%:%681=215%:%
%:%682=215%:%
%:%683=216%:%
%:%684=217%:%
%:%691=218%:%
%:%692=218%:%
%:%693=219%:%
%:%694=219%:%
%:%695=220%:%
%:%696=220%:%
%:%697=220%:%
%:%698=221%:%
%:%699=221%:%
%:%700=221%:%
%:%701=222%:%
%:%702=222%:%
%:%703=222%:%
%:%704=223%:%
%:%705=223%:%
%:%706=223%:%
%:%707=224%:%
%:%708=224%:%
%:%709=224%:%
%:%710=225%:%
%:%711=225%:%
%:%712=225%:%
%:%713=226%:%
%:%714=226%:%
%:%715=226%:%
%:%716=227%:%
%:%717=227%:%
%:%718=227%:%
%:%719=228%:%
%:%720=228%:%
%:%721=228%:%
%:%722=229%:%
%:%723=229%:%
%:%724=229%:%
%:%725=230%:%
%:%726=230%:%
%:%727=230%:%
%:%728=231%:%
%:%729=231%:%
%:%730=231%:%
%:%731=232%:%
%:%732=232%:%
%:%733=232%:%
%:%734=233%:%
%:%735=233%:%
%:%736=233%:%
%:%737=234%:%
%:%738=234%:%
%:%739=234%:%
%:%740=235%:%
%:%741=235%:%
%:%742=235%:%
%:%743=236%:%
%:%744=236%:%
%:%745=237%:%
%:%751=237%:%
%:%756=238%:%
%:%761=239%:%

%
\begin{isabellebody}%
\setisabellecontext{Nat{\isacharunderscore}{\kern0pt}Parity}%
%
\isadelimdocument
%
\endisadelimdocument
%
\isatagdocument
%
\isamarkupsection{Natural Number Parity and Halving%
}
\isamarkuptrue%
%
\endisatagdocument
{\isafolddocument}%
%
\isadelimdocument
%
\endisadelimdocument
%
\isadelimtheory
%
\endisadelimtheory
%
\isatagtheory
\isacommand{theory}\isamarkupfalse%
\ Nat{\isacharunderscore}{\kern0pt}Parity\isanewline
\ \ \isakeyword{imports}\ Nats\ Quant{\isacharunderscore}{\kern0pt}Logic\isanewline
\isakeyword{begin}%
\endisatagtheory
{\isafoldtheory}%
%
\isadelimtheory
%
\endisadelimtheory
%
\isadelimdocument
%
\endisadelimdocument
%
\isatagdocument
%
\isamarkupsubsection{Nth Even Number%
}
\isamarkuptrue%
%
\endisatagdocument
{\isafolddocument}%
%
\isadelimdocument
%
\endisadelimdocument
\isacommand{definition}\isamarkupfalse%
\ nth{\isacharunderscore}{\kern0pt}even\ {\isacharcolon}{\kern0pt}{\isacharcolon}{\kern0pt}\ {\isachardoublequoteopen}cfunc{\isachardoublequoteclose}\ \isakeyword{where}\isanewline
\ \ {\isachardoublequoteopen}nth{\isacharunderscore}{\kern0pt}even\ {\isacharequal}{\kern0pt}\ {\isacharparenleft}{\kern0pt}THE\ u{\isachardot}{\kern0pt}\ u{\isacharcolon}{\kern0pt}\ {\isasymnat}\isactrlsub c\ {\isasymrightarrow}\ {\isasymnat}\isactrlsub c\ {\isasymand}\ \isanewline
\ \ \ \ u\ {\isasymcirc}\isactrlsub c\ zero\ {\isacharequal}{\kern0pt}\ zero\ {\isasymand}\isanewline
\ \ \ \ {\isacharparenleft}{\kern0pt}successor\ {\isasymcirc}\isactrlsub c\ successor{\isacharparenright}{\kern0pt}\ {\isasymcirc}\isactrlsub c\ u\ {\isacharequal}{\kern0pt}\ u\ {\isasymcirc}\isactrlsub c\ successor{\isacharparenright}{\kern0pt}{\isachardoublequoteclose}\isanewline
\isanewline
\isacommand{lemma}\isamarkupfalse%
\ nth{\isacharunderscore}{\kern0pt}even{\isacharunderscore}{\kern0pt}def{\isadigit{2}}{\isacharcolon}{\kern0pt}\isanewline
\ \ {\isachardoublequoteopen}nth{\isacharunderscore}{\kern0pt}even{\isacharcolon}{\kern0pt}\ {\isasymnat}\isactrlsub c\ {\isasymrightarrow}\ {\isasymnat}\isactrlsub c\ {\isasymand}\ nth{\isacharunderscore}{\kern0pt}even\ {\isasymcirc}\isactrlsub c\ zero\ {\isacharequal}{\kern0pt}\ zero\ {\isasymand}\ {\isacharparenleft}{\kern0pt}successor\ {\isasymcirc}\isactrlsub c\ successor{\isacharparenright}{\kern0pt}\ {\isasymcirc}\isactrlsub c\ nth{\isacharunderscore}{\kern0pt}even\ {\isacharequal}{\kern0pt}\ nth{\isacharunderscore}{\kern0pt}even\ {\isasymcirc}\isactrlsub c\ successor{\isachardoublequoteclose}\isanewline
%
\isadelimproof
\ \ %
\endisadelimproof
%
\isatagproof
\isacommand{unfolding}\isamarkupfalse%
\ nth{\isacharunderscore}{\kern0pt}even{\isacharunderscore}{\kern0pt}def\ \isacommand{by}\isamarkupfalse%
\ {\isacharparenleft}{\kern0pt}rule\ theI{\isacharprime}{\kern0pt}{\isacharcomma}{\kern0pt}\ etcs{\isacharunderscore}{\kern0pt}rule\ natural{\isacharunderscore}{\kern0pt}number{\isacharunderscore}{\kern0pt}object{\isacharunderscore}{\kern0pt}property{\isadigit{2}}{\isacharparenright}{\kern0pt}%
\endisatagproof
{\isafoldproof}%
%
\isadelimproof
\isanewline
%
\endisadelimproof
\isanewline
\isacommand{lemma}\isamarkupfalse%
\ nth{\isacharunderscore}{\kern0pt}even{\isacharunderscore}{\kern0pt}type{\isacharbrackleft}{\kern0pt}type{\isacharunderscore}{\kern0pt}rule{\isacharbrackright}{\kern0pt}{\isacharcolon}{\kern0pt}\isanewline
\ \ {\isachardoublequoteopen}nth{\isacharunderscore}{\kern0pt}even{\isacharcolon}{\kern0pt}\ {\isasymnat}\isactrlsub c\ {\isasymrightarrow}\ {\isasymnat}\isactrlsub c{\isachardoublequoteclose}\isanewline
%
\isadelimproof
\ \ %
\endisadelimproof
%
\isatagproof
\isacommand{by}\isamarkupfalse%
\ {\isacharparenleft}{\kern0pt}simp\ add{\isacharcolon}{\kern0pt}\ nth{\isacharunderscore}{\kern0pt}even{\isacharunderscore}{\kern0pt}def{\isadigit{2}}{\isacharparenright}{\kern0pt}%
\endisatagproof
{\isafoldproof}%
%
\isadelimproof
\isanewline
%
\endisadelimproof
\isanewline
\isacommand{lemma}\isamarkupfalse%
\ nth{\isacharunderscore}{\kern0pt}even{\isacharunderscore}{\kern0pt}zero{\isacharcolon}{\kern0pt}\isanewline
\ \ {\isachardoublequoteopen}nth{\isacharunderscore}{\kern0pt}even\ {\isasymcirc}\isactrlsub c\ zero\ {\isacharequal}{\kern0pt}\ zero{\isachardoublequoteclose}\isanewline
%
\isadelimproof
\ \ %
\endisadelimproof
%
\isatagproof
\isacommand{by}\isamarkupfalse%
\ {\isacharparenleft}{\kern0pt}simp\ add{\isacharcolon}{\kern0pt}\ nth{\isacharunderscore}{\kern0pt}even{\isacharunderscore}{\kern0pt}def{\isadigit{2}}{\isacharparenright}{\kern0pt}%
\endisatagproof
{\isafoldproof}%
%
\isadelimproof
\isanewline
%
\endisadelimproof
\isanewline
\isacommand{lemma}\isamarkupfalse%
\ nth{\isacharunderscore}{\kern0pt}even{\isacharunderscore}{\kern0pt}successor{\isacharcolon}{\kern0pt}\isanewline
\ \ {\isachardoublequoteopen}nth{\isacharunderscore}{\kern0pt}even\ {\isasymcirc}\isactrlsub c\ successor\ {\isacharequal}{\kern0pt}\ {\isacharparenleft}{\kern0pt}successor\ {\isasymcirc}\isactrlsub c\ successor{\isacharparenright}{\kern0pt}\ {\isasymcirc}\isactrlsub c\ nth{\isacharunderscore}{\kern0pt}even{\isachardoublequoteclose}\isanewline
%
\isadelimproof
\ \ %
\endisadelimproof
%
\isatagproof
\isacommand{by}\isamarkupfalse%
\ {\isacharparenleft}{\kern0pt}simp\ add{\isacharcolon}{\kern0pt}\ nth{\isacharunderscore}{\kern0pt}even{\isacharunderscore}{\kern0pt}def{\isadigit{2}}{\isacharparenright}{\kern0pt}%
\endisatagproof
{\isafoldproof}%
%
\isadelimproof
\isanewline
%
\endisadelimproof
\isanewline
\isacommand{lemma}\isamarkupfalse%
\ nth{\isacharunderscore}{\kern0pt}even{\isacharunderscore}{\kern0pt}successor{\isadigit{2}}{\isacharcolon}{\kern0pt}\isanewline
\ \ {\isachardoublequoteopen}nth{\isacharunderscore}{\kern0pt}even\ {\isasymcirc}\isactrlsub c\ successor\ {\isacharequal}{\kern0pt}\ successor\ {\isasymcirc}\isactrlsub c\ successor\ {\isasymcirc}\isactrlsub c\ nth{\isacharunderscore}{\kern0pt}even{\isachardoublequoteclose}\isanewline
%
\isadelimproof
\ \ %
\endisadelimproof
%
\isatagproof
\isacommand{using}\isamarkupfalse%
\ comp{\isacharunderscore}{\kern0pt}associative{\isadigit{2}}\ nth{\isacharunderscore}{\kern0pt}even{\isacharunderscore}{\kern0pt}def{\isadigit{2}}\ \isacommand{by}\isamarkupfalse%
\ {\isacharparenleft}{\kern0pt}typecheck{\isacharunderscore}{\kern0pt}cfuncs{\isacharcomma}{\kern0pt}\ auto{\isacharparenright}{\kern0pt}%
\endisatagproof
{\isafoldproof}%
%
\isadelimproof
%
\endisadelimproof
%
\isadelimdocument
%
\endisadelimdocument
%
\isatagdocument
%
\isamarkupsubsection{Nth Odd Number%
}
\isamarkuptrue%
%
\endisatagdocument
{\isafolddocument}%
%
\isadelimdocument
%
\endisadelimdocument
\isacommand{definition}\isamarkupfalse%
\ nth{\isacharunderscore}{\kern0pt}odd\ {\isacharcolon}{\kern0pt}{\isacharcolon}{\kern0pt}\ {\isachardoublequoteopen}cfunc{\isachardoublequoteclose}\ \isakeyword{where}\isanewline
\ \ {\isachardoublequoteopen}nth{\isacharunderscore}{\kern0pt}odd\ {\isacharequal}{\kern0pt}\ {\isacharparenleft}{\kern0pt}THE\ u{\isachardot}{\kern0pt}\ u{\isacharcolon}{\kern0pt}\ {\isasymnat}\isactrlsub c\ {\isasymrightarrow}\ {\isasymnat}\isactrlsub c\ {\isasymand}\ \isanewline
\ \ \ \ u\ {\isasymcirc}\isactrlsub c\ zero\ {\isacharequal}{\kern0pt}\ successor\ {\isasymcirc}\isactrlsub c\ zero\ {\isasymand}\isanewline
\ \ \ \ {\isacharparenleft}{\kern0pt}successor\ {\isasymcirc}\isactrlsub c\ successor{\isacharparenright}{\kern0pt}\ {\isasymcirc}\isactrlsub c\ u\ {\isacharequal}{\kern0pt}\ u\ {\isasymcirc}\isactrlsub c\ successor{\isacharparenright}{\kern0pt}{\isachardoublequoteclose}\isanewline
\isanewline
\isacommand{lemma}\isamarkupfalse%
\ nth{\isacharunderscore}{\kern0pt}odd{\isacharunderscore}{\kern0pt}def{\isadigit{2}}{\isacharcolon}{\kern0pt}\isanewline
\ \ {\isachardoublequoteopen}nth{\isacharunderscore}{\kern0pt}odd{\isacharcolon}{\kern0pt}\ {\isasymnat}\isactrlsub c\ {\isasymrightarrow}\ {\isasymnat}\isactrlsub c\ {\isasymand}\ nth{\isacharunderscore}{\kern0pt}odd\ {\isasymcirc}\isactrlsub c\ zero\ {\isacharequal}{\kern0pt}\ successor\ {\isasymcirc}\isactrlsub c\ zero\ {\isasymand}\ {\isacharparenleft}{\kern0pt}successor\ {\isasymcirc}\isactrlsub c\ successor{\isacharparenright}{\kern0pt}\ {\isasymcirc}\isactrlsub c\ nth{\isacharunderscore}{\kern0pt}odd\ {\isacharequal}{\kern0pt}\ nth{\isacharunderscore}{\kern0pt}odd\ {\isasymcirc}\isactrlsub c\ successor{\isachardoublequoteclose}\isanewline
%
\isadelimproof
\ \ %
\endisadelimproof
%
\isatagproof
\isacommand{unfolding}\isamarkupfalse%
\ nth{\isacharunderscore}{\kern0pt}odd{\isacharunderscore}{\kern0pt}def\ \isacommand{by}\isamarkupfalse%
\ {\isacharparenleft}{\kern0pt}rule\ theI{\isacharprime}{\kern0pt}{\isacharcomma}{\kern0pt}\ etcs{\isacharunderscore}{\kern0pt}rule\ natural{\isacharunderscore}{\kern0pt}number{\isacharunderscore}{\kern0pt}object{\isacharunderscore}{\kern0pt}property{\isadigit{2}}{\isacharparenright}{\kern0pt}%
\endisatagproof
{\isafoldproof}%
%
\isadelimproof
\isanewline
%
\endisadelimproof
\isanewline
\isacommand{lemma}\isamarkupfalse%
\ nth{\isacharunderscore}{\kern0pt}odd{\isacharunderscore}{\kern0pt}type{\isacharbrackleft}{\kern0pt}type{\isacharunderscore}{\kern0pt}rule{\isacharbrackright}{\kern0pt}{\isacharcolon}{\kern0pt}\isanewline
\ \ {\isachardoublequoteopen}nth{\isacharunderscore}{\kern0pt}odd{\isacharcolon}{\kern0pt}\ {\isasymnat}\isactrlsub c\ {\isasymrightarrow}\ {\isasymnat}\isactrlsub c{\isachardoublequoteclose}\isanewline
%
\isadelimproof
\ \ %
\endisadelimproof
%
\isatagproof
\isacommand{by}\isamarkupfalse%
\ {\isacharparenleft}{\kern0pt}simp\ add{\isacharcolon}{\kern0pt}\ nth{\isacharunderscore}{\kern0pt}odd{\isacharunderscore}{\kern0pt}def{\isadigit{2}}{\isacharparenright}{\kern0pt}%
\endisatagproof
{\isafoldproof}%
%
\isadelimproof
\isanewline
%
\endisadelimproof
\isanewline
\isacommand{lemma}\isamarkupfalse%
\ nth{\isacharunderscore}{\kern0pt}odd{\isacharunderscore}{\kern0pt}zero{\isacharcolon}{\kern0pt}\isanewline
\ \ {\isachardoublequoteopen}nth{\isacharunderscore}{\kern0pt}odd\ {\isasymcirc}\isactrlsub c\ zero\ {\isacharequal}{\kern0pt}\ successor\ {\isasymcirc}\isactrlsub c\ zero{\isachardoublequoteclose}\isanewline
%
\isadelimproof
\ \ %
\endisadelimproof
%
\isatagproof
\isacommand{by}\isamarkupfalse%
\ {\isacharparenleft}{\kern0pt}simp\ add{\isacharcolon}{\kern0pt}\ nth{\isacharunderscore}{\kern0pt}odd{\isacharunderscore}{\kern0pt}def{\isadigit{2}}{\isacharparenright}{\kern0pt}%
\endisatagproof
{\isafoldproof}%
%
\isadelimproof
\isanewline
%
\endisadelimproof
\isanewline
\isacommand{lemma}\isamarkupfalse%
\ nth{\isacharunderscore}{\kern0pt}odd{\isacharunderscore}{\kern0pt}successor{\isacharcolon}{\kern0pt}\isanewline
\ \ {\isachardoublequoteopen}nth{\isacharunderscore}{\kern0pt}odd\ {\isasymcirc}\isactrlsub c\ successor\ {\isacharequal}{\kern0pt}\ {\isacharparenleft}{\kern0pt}successor\ {\isasymcirc}\isactrlsub c\ successor{\isacharparenright}{\kern0pt}\ {\isasymcirc}\isactrlsub c\ nth{\isacharunderscore}{\kern0pt}odd{\isachardoublequoteclose}\isanewline
%
\isadelimproof
\ \ %
\endisadelimproof
%
\isatagproof
\isacommand{by}\isamarkupfalse%
\ {\isacharparenleft}{\kern0pt}simp\ add{\isacharcolon}{\kern0pt}\ nth{\isacharunderscore}{\kern0pt}odd{\isacharunderscore}{\kern0pt}def{\isadigit{2}}{\isacharparenright}{\kern0pt}%
\endisatagproof
{\isafoldproof}%
%
\isadelimproof
\isanewline
%
\endisadelimproof
\isanewline
\isacommand{lemma}\isamarkupfalse%
\ nth{\isacharunderscore}{\kern0pt}odd{\isacharunderscore}{\kern0pt}successor{\isadigit{2}}{\isacharcolon}{\kern0pt}\isanewline
\ \ {\isachardoublequoteopen}nth{\isacharunderscore}{\kern0pt}odd\ {\isasymcirc}\isactrlsub c\ successor\ {\isacharequal}{\kern0pt}\ successor\ {\isasymcirc}\isactrlsub c\ successor\ {\isasymcirc}\isactrlsub c\ nth{\isacharunderscore}{\kern0pt}odd{\isachardoublequoteclose}\isanewline
%
\isadelimproof
\ \ %
\endisadelimproof
%
\isatagproof
\isacommand{using}\isamarkupfalse%
\ comp{\isacharunderscore}{\kern0pt}associative{\isadigit{2}}\ nth{\isacharunderscore}{\kern0pt}odd{\isacharunderscore}{\kern0pt}def{\isadigit{2}}\ \isacommand{by}\isamarkupfalse%
\ {\isacharparenleft}{\kern0pt}typecheck{\isacharunderscore}{\kern0pt}cfuncs{\isacharcomma}{\kern0pt}\ auto{\isacharparenright}{\kern0pt}%
\endisatagproof
{\isafoldproof}%
%
\isadelimproof
\isanewline
%
\endisadelimproof
\isanewline
\isacommand{lemma}\isamarkupfalse%
\ nth{\isacharunderscore}{\kern0pt}odd{\isacharunderscore}{\kern0pt}is{\isacharunderscore}{\kern0pt}succ{\isacharunderscore}{\kern0pt}nth{\isacharunderscore}{\kern0pt}even{\isacharcolon}{\kern0pt}\isanewline
\ \ {\isachardoublequoteopen}nth{\isacharunderscore}{\kern0pt}odd\ {\isacharequal}{\kern0pt}\ successor\ {\isasymcirc}\isactrlsub c\ nth{\isacharunderscore}{\kern0pt}even{\isachardoublequoteclose}\isanewline
%
\isadelimproof
%
\endisadelimproof
%
\isatagproof
\isacommand{proof}\isamarkupfalse%
\ {\isacharparenleft}{\kern0pt}etcs{\isacharunderscore}{\kern0pt}rule\ natural{\isacharunderscore}{\kern0pt}number{\isacharunderscore}{\kern0pt}object{\isacharunderscore}{\kern0pt}func{\isacharunderscore}{\kern0pt}unique{\isacharbrackleft}{\kern0pt}\isakeyword{where}\ X{\isacharequal}{\kern0pt}{\isachardoublequoteopen}{\isasymnat}\isactrlsub c{\isachardoublequoteclose}{\isacharcomma}{\kern0pt}\ \isakeyword{where}\ f{\isacharequal}{\kern0pt}{\isachardoublequoteopen}successor\ {\isasymcirc}\isactrlsub c\ successor{\isachardoublequoteclose}{\isacharbrackright}{\kern0pt}{\isacharparenright}{\kern0pt}\isanewline
\ \ \isacommand{show}\isamarkupfalse%
\ {\isachardoublequoteopen}nth{\isacharunderscore}{\kern0pt}odd\ {\isasymcirc}\isactrlsub c\ zero\ {\isacharequal}{\kern0pt}\ {\isacharparenleft}{\kern0pt}successor\ {\isasymcirc}\isactrlsub c\ nth{\isacharunderscore}{\kern0pt}even{\isacharparenright}{\kern0pt}\ {\isasymcirc}\isactrlsub c\ zero{\isachardoublequoteclose}\isanewline
\ \ \isacommand{proof}\isamarkupfalse%
\ {\isacharminus}{\kern0pt}\isanewline
\ \ \ \ \isacommand{have}\isamarkupfalse%
\ {\isachardoublequoteopen}nth{\isacharunderscore}{\kern0pt}odd\ {\isasymcirc}\isactrlsub c\ zero\ {\isacharequal}{\kern0pt}\ successor\ {\isasymcirc}\isactrlsub c\ zero{\isachardoublequoteclose}\isanewline
\ \ \ \ \ \ \isacommand{by}\isamarkupfalse%
\ {\isacharparenleft}{\kern0pt}simp\ add{\isacharcolon}{\kern0pt}\ nth{\isacharunderscore}{\kern0pt}odd{\isacharunderscore}{\kern0pt}zero{\isacharparenright}{\kern0pt}\isanewline
\ \ \ \ \isacommand{also}\isamarkupfalse%
\ \isacommand{have}\isamarkupfalse%
\ {\isachardoublequoteopen}{\isachardot}{\kern0pt}{\isachardot}{\kern0pt}{\isachardot}{\kern0pt}\ {\isacharequal}{\kern0pt}\ {\isacharparenleft}{\kern0pt}successor\ {\isasymcirc}\isactrlsub c\ nth{\isacharunderscore}{\kern0pt}even{\isacharparenright}{\kern0pt}\ {\isasymcirc}\isactrlsub c\ zero{\isachardoublequoteclose}\isanewline
\ \ \ \ \ \ \isacommand{using}\isamarkupfalse%
\ comp{\isacharunderscore}{\kern0pt}associative{\isadigit{2}}\ nth{\isacharunderscore}{\kern0pt}even{\isacharunderscore}{\kern0pt}def{\isadigit{2}}\ successor{\isacharunderscore}{\kern0pt}type\ zero{\isacharunderscore}{\kern0pt}type\ \isacommand{by}\isamarkupfalse%
\ fastforce\isanewline
\ \ \ \ \isacommand{then}\isamarkupfalse%
\ \isacommand{show}\isamarkupfalse%
\ {\isacharquery}{\kern0pt}thesis\isanewline
\ \ \ \ \ \ \isacommand{using}\isamarkupfalse%
\ calculation\ \isacommand{by}\isamarkupfalse%
\ auto\isanewline
\ \ \isacommand{qed}\isamarkupfalse%
\isanewline
\isanewline
\ \ \isacommand{show}\isamarkupfalse%
\ {\isachardoublequoteopen}nth{\isacharunderscore}{\kern0pt}odd\ {\isasymcirc}\isactrlsub c\ successor\ {\isacharequal}{\kern0pt}\ {\isacharparenleft}{\kern0pt}successor\ {\isasymcirc}\isactrlsub c\ successor{\isacharparenright}{\kern0pt}\ {\isasymcirc}\isactrlsub c\ nth{\isacharunderscore}{\kern0pt}odd{\isachardoublequoteclose}\isanewline
\ \ \ \ \isacommand{by}\isamarkupfalse%
\ {\isacharparenleft}{\kern0pt}simp\ add{\isacharcolon}{\kern0pt}\ nth{\isacharunderscore}{\kern0pt}odd{\isacharunderscore}{\kern0pt}successor{\isacharparenright}{\kern0pt}\isanewline
\isanewline
\ \ \isacommand{show}\isamarkupfalse%
\ {\isachardoublequoteopen}{\isacharparenleft}{\kern0pt}successor\ {\isasymcirc}\isactrlsub c\ nth{\isacharunderscore}{\kern0pt}even{\isacharparenright}{\kern0pt}\ {\isasymcirc}\isactrlsub c\ successor\ {\isacharequal}{\kern0pt}\ {\isacharparenleft}{\kern0pt}successor\ {\isasymcirc}\isactrlsub c\ successor{\isacharparenright}{\kern0pt}\ {\isasymcirc}\isactrlsub c\ successor\ {\isasymcirc}\isactrlsub c\ nth{\isacharunderscore}{\kern0pt}even{\isachardoublequoteclose}\isanewline
\ \ \isacommand{proof}\isamarkupfalse%
\ {\isacharminus}{\kern0pt}\isanewline
\ \ \ \ \isacommand{have}\isamarkupfalse%
\ {\isachardoublequoteopen}{\isacharparenleft}{\kern0pt}successor\ {\isasymcirc}\isactrlsub c\ nth{\isacharunderscore}{\kern0pt}even{\isacharparenright}{\kern0pt}\ {\isasymcirc}\isactrlsub c\ successor\ {\isacharequal}{\kern0pt}\ successor\ {\isasymcirc}\isactrlsub c\ nth{\isacharunderscore}{\kern0pt}even\ {\isasymcirc}\isactrlsub c\ successor{\isachardoublequoteclose}\isanewline
\ \ \ \ \ \ \isacommand{by}\isamarkupfalse%
\ {\isacharparenleft}{\kern0pt}typecheck{\isacharunderscore}{\kern0pt}cfuncs{\isacharcomma}{\kern0pt}\ simp\ add{\isacharcolon}{\kern0pt}\ comp{\isacharunderscore}{\kern0pt}associative{\isadigit{2}}{\isacharparenright}{\kern0pt}\isanewline
\ \ \ \ \isacommand{also}\isamarkupfalse%
\ \isacommand{have}\isamarkupfalse%
\ {\isachardoublequoteopen}{\isachardot}{\kern0pt}{\isachardot}{\kern0pt}{\isachardot}{\kern0pt}\ {\isacharequal}{\kern0pt}\ successor\ {\isasymcirc}\isactrlsub c\ successor\ {\isasymcirc}\isactrlsub c\ successor\ {\isasymcirc}\isactrlsub c\ nth{\isacharunderscore}{\kern0pt}even{\isachardoublequoteclose}\isanewline
\ \ \ \ \ \ \isacommand{by}\isamarkupfalse%
\ {\isacharparenleft}{\kern0pt}simp\ add{\isacharcolon}{\kern0pt}\ nth{\isacharunderscore}{\kern0pt}even{\isacharunderscore}{\kern0pt}successor{\isadigit{2}}{\isacharparenright}{\kern0pt}\isanewline
\ \ \ \ \isacommand{also}\isamarkupfalse%
\ \isacommand{have}\isamarkupfalse%
\ {\isachardoublequoteopen}{\isachardot}{\kern0pt}{\isachardot}{\kern0pt}{\isachardot}{\kern0pt}\ {\isacharequal}{\kern0pt}\ {\isacharparenleft}{\kern0pt}successor\ {\isasymcirc}\isactrlsub c\ successor{\isacharparenright}{\kern0pt}\ {\isasymcirc}\isactrlsub c\ successor\ {\isasymcirc}\isactrlsub c\ nth{\isacharunderscore}{\kern0pt}even{\isachardoublequoteclose}\isanewline
\ \ \ \ \ \ \isacommand{by}\isamarkupfalse%
\ {\isacharparenleft}{\kern0pt}typecheck{\isacharunderscore}{\kern0pt}cfuncs{\isacharcomma}{\kern0pt}\ simp\ add{\isacharcolon}{\kern0pt}\ comp{\isacharunderscore}{\kern0pt}associative{\isadigit{2}}{\isacharparenright}{\kern0pt}\isanewline
\ \ \ \ \isacommand{then}\isamarkupfalse%
\ \isacommand{show}\isamarkupfalse%
\ {\isacharquery}{\kern0pt}thesis\isanewline
\ \ \ \ \ \ \isacommand{using}\isamarkupfalse%
\ calculation\ \isacommand{by}\isamarkupfalse%
\ auto\isanewline
\ \ \isacommand{qed}\isamarkupfalse%
\isanewline
\isacommand{qed}\isamarkupfalse%
%
\endisatagproof
{\isafoldproof}%
%
\isadelimproof
\isanewline
%
\endisadelimproof
\isanewline
\isacommand{lemma}\isamarkupfalse%
\ succ{\isacharunderscore}{\kern0pt}nth{\isacharunderscore}{\kern0pt}odd{\isacharunderscore}{\kern0pt}is{\isacharunderscore}{\kern0pt}nth{\isacharunderscore}{\kern0pt}even{\isacharunderscore}{\kern0pt}succ{\isacharcolon}{\kern0pt}\isanewline
\ \ {\isachardoublequoteopen}successor\ {\isasymcirc}\isactrlsub c\ nth{\isacharunderscore}{\kern0pt}odd\ {\isacharequal}{\kern0pt}\ nth{\isacharunderscore}{\kern0pt}even\ {\isasymcirc}\isactrlsub c\ successor{\isachardoublequoteclose}\isanewline
%
\isadelimproof
%
\endisadelimproof
%
\isatagproof
\isacommand{proof}\isamarkupfalse%
\ {\isacharparenleft}{\kern0pt}etcs{\isacharunderscore}{\kern0pt}rule\ natural{\isacharunderscore}{\kern0pt}number{\isacharunderscore}{\kern0pt}object{\isacharunderscore}{\kern0pt}func{\isacharunderscore}{\kern0pt}unique{\isacharbrackleft}{\kern0pt}\isakeyword{where}\ f{\isacharequal}{\kern0pt}{\isachardoublequoteopen}successor\ {\isasymcirc}\isactrlsub c\ successor{\isachardoublequoteclose}{\isacharbrackright}{\kern0pt}{\isacharparenright}{\kern0pt}\isanewline
\ \ \isacommand{show}\isamarkupfalse%
\ {\isachardoublequoteopen}{\isacharparenleft}{\kern0pt}successor\ {\isasymcirc}\isactrlsub c\ nth{\isacharunderscore}{\kern0pt}odd{\isacharparenright}{\kern0pt}\ {\isasymcirc}\isactrlsub c\ zero\ {\isacharequal}{\kern0pt}\ {\isacharparenleft}{\kern0pt}nth{\isacharunderscore}{\kern0pt}even\ {\isasymcirc}\isactrlsub c\ successor{\isacharparenright}{\kern0pt}\ {\isasymcirc}\isactrlsub c\ zero{\isachardoublequoteclose}\isanewline
\ \ \isacommand{proof}\isamarkupfalse%
\ {\isacharminus}{\kern0pt}\isanewline
\ \ \ \ \isacommand{have}\isamarkupfalse%
\ {\isachardoublequoteopen}{\isacharparenleft}{\kern0pt}successor\ {\isasymcirc}\isactrlsub c\ nth{\isacharunderscore}{\kern0pt}odd{\isacharparenright}{\kern0pt}\ {\isasymcirc}\isactrlsub c\ zero\ {\isacharequal}{\kern0pt}\ successor\ {\isasymcirc}\isactrlsub c\ successor\ {\isasymcirc}\isactrlsub c\ zero{\isachardoublequoteclose}\isanewline
\ \ \ \ \ \ \isacommand{using}\isamarkupfalse%
\ comp{\isacharunderscore}{\kern0pt}associative{\isadigit{2}}\ nth{\isacharunderscore}{\kern0pt}odd{\isacharunderscore}{\kern0pt}def{\isadigit{2}}\ successor{\isacharunderscore}{\kern0pt}type\ zero{\isacharunderscore}{\kern0pt}type\ \isacommand{by}\isamarkupfalse%
\ fastforce\isanewline
\ \ \ \ \isacommand{also}\isamarkupfalse%
\ \isacommand{have}\isamarkupfalse%
\ {\isachardoublequoteopen}{\isachardot}{\kern0pt}{\isachardot}{\kern0pt}{\isachardot}{\kern0pt}\ {\isacharequal}{\kern0pt}\ {\isacharparenleft}{\kern0pt}nth{\isacharunderscore}{\kern0pt}even\ {\isasymcirc}\isactrlsub c\ successor{\isacharparenright}{\kern0pt}\ {\isasymcirc}\isactrlsub c\ zero{\isachardoublequoteclose}\isanewline
\ \ \ \ \ \ \isacommand{using}\isamarkupfalse%
\ calculation\ nth{\isacharunderscore}{\kern0pt}even{\isacharunderscore}{\kern0pt}successor{\isadigit{2}}\ nth{\isacharunderscore}{\kern0pt}odd{\isacharunderscore}{\kern0pt}is{\isacharunderscore}{\kern0pt}succ{\isacharunderscore}{\kern0pt}nth{\isacharunderscore}{\kern0pt}even\ \isacommand{by}\isamarkupfalse%
\ auto\isanewline
\ \ \ \ \isacommand{then}\isamarkupfalse%
\ \isacommand{show}\isamarkupfalse%
\ {\isacharquery}{\kern0pt}thesis\isanewline
\ \ \ \ \ \ \isacommand{using}\isamarkupfalse%
\ calculation\ \isacommand{by}\isamarkupfalse%
\ auto\isanewline
\ \ \isacommand{qed}\isamarkupfalse%
\isanewline
\isanewline
\ \ \isacommand{show}\isamarkupfalse%
\ {\isachardoublequoteopen}{\isacharparenleft}{\kern0pt}successor\ {\isasymcirc}\isactrlsub c\ nth{\isacharunderscore}{\kern0pt}odd{\isacharparenright}{\kern0pt}\ {\isasymcirc}\isactrlsub c\ successor\ {\isacharequal}{\kern0pt}\ {\isacharparenleft}{\kern0pt}successor\ {\isasymcirc}\isactrlsub c\ successor{\isacharparenright}{\kern0pt}\ {\isasymcirc}\isactrlsub c\ successor\ {\isasymcirc}\isactrlsub c\ nth{\isacharunderscore}{\kern0pt}odd{\isachardoublequoteclose}\isanewline
\ \ \ \ \isacommand{by}\isamarkupfalse%
\ {\isacharparenleft}{\kern0pt}metis\ cfunc{\isacharunderscore}{\kern0pt}type{\isacharunderscore}{\kern0pt}def\ codomain{\isacharunderscore}{\kern0pt}comp\ comp{\isacharunderscore}{\kern0pt}associative\ nth{\isacharunderscore}{\kern0pt}odd{\isacharunderscore}{\kern0pt}def{\isadigit{2}}\ successor{\isacharunderscore}{\kern0pt}type{\isacharparenright}{\kern0pt}\isanewline
\ \ \isacommand{then}\isamarkupfalse%
\ \isacommand{show}\isamarkupfalse%
\ {\isachardoublequoteopen}{\isacharparenleft}{\kern0pt}nth{\isacharunderscore}{\kern0pt}even\ {\isasymcirc}\isactrlsub c\ successor{\isacharparenright}{\kern0pt}\ {\isasymcirc}\isactrlsub c\ successor\ {\isacharequal}{\kern0pt}\ {\isacharparenleft}{\kern0pt}successor\ {\isasymcirc}\isactrlsub c\ successor{\isacharparenright}{\kern0pt}\ {\isasymcirc}\isactrlsub c\ nth{\isacharunderscore}{\kern0pt}even\ {\isasymcirc}\isactrlsub c\ successor{\isachardoublequoteclose}\isanewline
\ \ \ \ \isacommand{using}\isamarkupfalse%
\ nth{\isacharunderscore}{\kern0pt}even{\isacharunderscore}{\kern0pt}successor{\isadigit{2}}\ nth{\isacharunderscore}{\kern0pt}odd{\isacharunderscore}{\kern0pt}is{\isacharunderscore}{\kern0pt}succ{\isacharunderscore}{\kern0pt}nth{\isacharunderscore}{\kern0pt}even\ \isacommand{by}\isamarkupfalse%
\ auto\isanewline
\isacommand{qed}\isamarkupfalse%
%
\endisatagproof
{\isafoldproof}%
%
\isadelimproof
%
\endisadelimproof
%
\isadelimdocument
%
\endisadelimdocument
%
\isatagdocument
%
\isamarkupsubsection{Checking if a Number is Even%
}
\isamarkuptrue%
%
\endisatagdocument
{\isafolddocument}%
%
\isadelimdocument
%
\endisadelimdocument
\isacommand{definition}\isamarkupfalse%
\ is{\isacharunderscore}{\kern0pt}even\ {\isacharcolon}{\kern0pt}{\isacharcolon}{\kern0pt}\ {\isachardoublequoteopen}cfunc{\isachardoublequoteclose}\ \isakeyword{where}\isanewline
\ \ {\isachardoublequoteopen}is{\isacharunderscore}{\kern0pt}even\ {\isacharequal}{\kern0pt}\ {\isacharparenleft}{\kern0pt}THE\ u{\isachardot}{\kern0pt}\ u{\isacharcolon}{\kern0pt}\ {\isasymnat}\isactrlsub c\ {\isasymrightarrow}\ {\isasymOmega}\ {\isasymand}\ u\ {\isasymcirc}\isactrlsub c\ zero\ {\isacharequal}{\kern0pt}\ {\isasymt}\ {\isasymand}\ NOT\ {\isasymcirc}\isactrlsub c\ u\ {\isacharequal}{\kern0pt}\ u\ {\isasymcirc}\isactrlsub c\ successor{\isacharparenright}{\kern0pt}{\isachardoublequoteclose}\isanewline
\isanewline
\isacommand{lemma}\isamarkupfalse%
\ is{\isacharunderscore}{\kern0pt}even{\isacharunderscore}{\kern0pt}def{\isadigit{2}}{\isacharcolon}{\kern0pt}\isanewline
\ \ {\isachardoublequoteopen}is{\isacharunderscore}{\kern0pt}even\ {\isacharcolon}{\kern0pt}\ {\isasymnat}\isactrlsub c\ {\isasymrightarrow}\ {\isasymOmega}\ {\isasymand}\ is{\isacharunderscore}{\kern0pt}even\ {\isasymcirc}\isactrlsub c\ zero\ {\isacharequal}{\kern0pt}\ {\isasymt}\ {\isasymand}\ NOT\ {\isasymcirc}\isactrlsub c\ is{\isacharunderscore}{\kern0pt}even\ {\isacharequal}{\kern0pt}\ is{\isacharunderscore}{\kern0pt}even\ {\isasymcirc}\isactrlsub c\ successor{\isachardoublequoteclose}\isanewline
%
\isadelimproof
\ \ %
\endisadelimproof
%
\isatagproof
\isacommand{unfolding}\isamarkupfalse%
\ is{\isacharunderscore}{\kern0pt}even{\isacharunderscore}{\kern0pt}def\ \isacommand{by}\isamarkupfalse%
\ {\isacharparenleft}{\kern0pt}rule\ theI{\isacharprime}{\kern0pt}{\isacharcomma}{\kern0pt}\ etcs{\isacharunderscore}{\kern0pt}rule\ natural{\isacharunderscore}{\kern0pt}number{\isacharunderscore}{\kern0pt}object{\isacharunderscore}{\kern0pt}property{\isadigit{2}}{\isacharparenright}{\kern0pt}%
\endisatagproof
{\isafoldproof}%
%
\isadelimproof
\isanewline
%
\endisadelimproof
\isanewline
\isacommand{lemma}\isamarkupfalse%
\ is{\isacharunderscore}{\kern0pt}even{\isacharunderscore}{\kern0pt}type{\isacharbrackleft}{\kern0pt}type{\isacharunderscore}{\kern0pt}rule{\isacharbrackright}{\kern0pt}{\isacharcolon}{\kern0pt}\isanewline
\ \ {\isachardoublequoteopen}is{\isacharunderscore}{\kern0pt}even\ {\isacharcolon}{\kern0pt}\ {\isasymnat}\isactrlsub c\ {\isasymrightarrow}\ {\isasymOmega}{\isachardoublequoteclose}\isanewline
%
\isadelimproof
\ \ %
\endisadelimproof
%
\isatagproof
\isacommand{by}\isamarkupfalse%
\ {\isacharparenleft}{\kern0pt}simp\ add{\isacharcolon}{\kern0pt}\ is{\isacharunderscore}{\kern0pt}even{\isacharunderscore}{\kern0pt}def{\isadigit{2}}{\isacharparenright}{\kern0pt}%
\endisatagproof
{\isafoldproof}%
%
\isadelimproof
\isanewline
%
\endisadelimproof
\isanewline
\isacommand{lemma}\isamarkupfalse%
\ is{\isacharunderscore}{\kern0pt}even{\isacharunderscore}{\kern0pt}zero{\isacharcolon}{\kern0pt}\isanewline
\ \ {\isachardoublequoteopen}is{\isacharunderscore}{\kern0pt}even\ {\isasymcirc}\isactrlsub c\ zero\ {\isacharequal}{\kern0pt}\ {\isasymt}{\isachardoublequoteclose}\isanewline
%
\isadelimproof
\ \ %
\endisadelimproof
%
\isatagproof
\isacommand{by}\isamarkupfalse%
\ {\isacharparenleft}{\kern0pt}simp\ add{\isacharcolon}{\kern0pt}\ is{\isacharunderscore}{\kern0pt}even{\isacharunderscore}{\kern0pt}def{\isadigit{2}}{\isacharparenright}{\kern0pt}%
\endisatagproof
{\isafoldproof}%
%
\isadelimproof
\isanewline
%
\endisadelimproof
\isanewline
\isacommand{lemma}\isamarkupfalse%
\ is{\isacharunderscore}{\kern0pt}even{\isacharunderscore}{\kern0pt}successor{\isacharcolon}{\kern0pt}\isanewline
\ \ {\isachardoublequoteopen}is{\isacharunderscore}{\kern0pt}even\ {\isasymcirc}\isactrlsub c\ successor\ {\isacharequal}{\kern0pt}\ NOT\ {\isasymcirc}\isactrlsub c\ is{\isacharunderscore}{\kern0pt}even{\isachardoublequoteclose}\isanewline
%
\isadelimproof
\ \ %
\endisadelimproof
%
\isatagproof
\isacommand{by}\isamarkupfalse%
\ {\isacharparenleft}{\kern0pt}simp\ add{\isacharcolon}{\kern0pt}\ is{\isacharunderscore}{\kern0pt}even{\isacharunderscore}{\kern0pt}def{\isadigit{2}}{\isacharparenright}{\kern0pt}%
\endisatagproof
{\isafoldproof}%
%
\isadelimproof
%
\endisadelimproof
%
\isadelimdocument
%
\endisadelimdocument
%
\isatagdocument
%
\isamarkupsubsection{Checking if a Number is Odd%
}
\isamarkuptrue%
%
\endisatagdocument
{\isafolddocument}%
%
\isadelimdocument
%
\endisadelimdocument
\isacommand{definition}\isamarkupfalse%
\ is{\isacharunderscore}{\kern0pt}odd\ {\isacharcolon}{\kern0pt}{\isacharcolon}{\kern0pt}\ {\isachardoublequoteopen}cfunc{\isachardoublequoteclose}\ \isakeyword{where}\isanewline
\ \ {\isachardoublequoteopen}is{\isacharunderscore}{\kern0pt}odd\ {\isacharequal}{\kern0pt}\ {\isacharparenleft}{\kern0pt}THE\ u{\isachardot}{\kern0pt}\ u{\isacharcolon}{\kern0pt}\ {\isasymnat}\isactrlsub c\ {\isasymrightarrow}\ {\isasymOmega}\ {\isasymand}\ u\ {\isasymcirc}\isactrlsub c\ zero\ {\isacharequal}{\kern0pt}\ {\isasymf}\ {\isasymand}\ NOT\ {\isasymcirc}\isactrlsub c\ u\ {\isacharequal}{\kern0pt}\ u\ {\isasymcirc}\isactrlsub c\ successor{\isacharparenright}{\kern0pt}{\isachardoublequoteclose}\isanewline
\isanewline
\isacommand{lemma}\isamarkupfalse%
\ is{\isacharunderscore}{\kern0pt}odd{\isacharunderscore}{\kern0pt}def{\isadigit{2}}{\isacharcolon}{\kern0pt}\isanewline
\ \ {\isachardoublequoteopen}is{\isacharunderscore}{\kern0pt}odd\ {\isacharcolon}{\kern0pt}\ {\isasymnat}\isactrlsub c\ {\isasymrightarrow}\ {\isasymOmega}\ {\isasymand}\ is{\isacharunderscore}{\kern0pt}odd\ {\isasymcirc}\isactrlsub c\ zero\ {\isacharequal}{\kern0pt}\ {\isasymf}\ {\isasymand}\ NOT\ {\isasymcirc}\isactrlsub c\ is{\isacharunderscore}{\kern0pt}odd\ {\isacharequal}{\kern0pt}\ is{\isacharunderscore}{\kern0pt}odd\ {\isasymcirc}\isactrlsub c\ successor{\isachardoublequoteclose}\isanewline
%
\isadelimproof
\ \ %
\endisadelimproof
%
\isatagproof
\isacommand{unfolding}\isamarkupfalse%
\ is{\isacharunderscore}{\kern0pt}odd{\isacharunderscore}{\kern0pt}def\ \isacommand{by}\isamarkupfalse%
\ {\isacharparenleft}{\kern0pt}rule\ theI{\isacharprime}{\kern0pt}{\isacharcomma}{\kern0pt}\ etcs{\isacharunderscore}{\kern0pt}rule\ natural{\isacharunderscore}{\kern0pt}number{\isacharunderscore}{\kern0pt}object{\isacharunderscore}{\kern0pt}property{\isadigit{2}}{\isacharparenright}{\kern0pt}%
\endisatagproof
{\isafoldproof}%
%
\isadelimproof
\isanewline
%
\endisadelimproof
\isanewline
\isacommand{lemma}\isamarkupfalse%
\ is{\isacharunderscore}{\kern0pt}odd{\isacharunderscore}{\kern0pt}type{\isacharbrackleft}{\kern0pt}type{\isacharunderscore}{\kern0pt}rule{\isacharbrackright}{\kern0pt}{\isacharcolon}{\kern0pt}\isanewline
\ \ {\isachardoublequoteopen}is{\isacharunderscore}{\kern0pt}odd\ {\isacharcolon}{\kern0pt}\ {\isasymnat}\isactrlsub c\ {\isasymrightarrow}\ {\isasymOmega}{\isachardoublequoteclose}\isanewline
%
\isadelimproof
\ \ %
\endisadelimproof
%
\isatagproof
\isacommand{by}\isamarkupfalse%
\ {\isacharparenleft}{\kern0pt}simp\ add{\isacharcolon}{\kern0pt}\ is{\isacharunderscore}{\kern0pt}odd{\isacharunderscore}{\kern0pt}def{\isadigit{2}}{\isacharparenright}{\kern0pt}%
\endisatagproof
{\isafoldproof}%
%
\isadelimproof
\isanewline
%
\endisadelimproof
\isanewline
\isacommand{lemma}\isamarkupfalse%
\ is{\isacharunderscore}{\kern0pt}odd{\isacharunderscore}{\kern0pt}zero{\isacharcolon}{\kern0pt}\isanewline
\ \ {\isachardoublequoteopen}is{\isacharunderscore}{\kern0pt}odd\ {\isasymcirc}\isactrlsub c\ zero\ {\isacharequal}{\kern0pt}\ {\isasymf}{\isachardoublequoteclose}\isanewline
%
\isadelimproof
\ \ %
\endisadelimproof
%
\isatagproof
\isacommand{by}\isamarkupfalse%
\ {\isacharparenleft}{\kern0pt}simp\ add{\isacharcolon}{\kern0pt}\ is{\isacharunderscore}{\kern0pt}odd{\isacharunderscore}{\kern0pt}def{\isadigit{2}}{\isacharparenright}{\kern0pt}%
\endisatagproof
{\isafoldproof}%
%
\isadelimproof
\isanewline
%
\endisadelimproof
\isanewline
\isacommand{lemma}\isamarkupfalse%
\ is{\isacharunderscore}{\kern0pt}odd{\isacharunderscore}{\kern0pt}successor{\isacharcolon}{\kern0pt}\isanewline
\ \ {\isachardoublequoteopen}is{\isacharunderscore}{\kern0pt}odd\ {\isasymcirc}\isactrlsub c\ successor\ {\isacharequal}{\kern0pt}\ NOT\ {\isasymcirc}\isactrlsub c\ is{\isacharunderscore}{\kern0pt}odd{\isachardoublequoteclose}\isanewline
%
\isadelimproof
\ \ %
\endisadelimproof
%
\isatagproof
\isacommand{by}\isamarkupfalse%
\ {\isacharparenleft}{\kern0pt}simp\ add{\isacharcolon}{\kern0pt}\ is{\isacharunderscore}{\kern0pt}odd{\isacharunderscore}{\kern0pt}def{\isadigit{2}}{\isacharparenright}{\kern0pt}%
\endisatagproof
{\isafoldproof}%
%
\isadelimproof
\isanewline
%
\endisadelimproof
\isanewline
\isacommand{lemma}\isamarkupfalse%
\ is{\isacharunderscore}{\kern0pt}even{\isacharunderscore}{\kern0pt}not{\isacharunderscore}{\kern0pt}is{\isacharunderscore}{\kern0pt}odd{\isacharcolon}{\kern0pt}\isanewline
\ \ {\isachardoublequoteopen}is{\isacharunderscore}{\kern0pt}even\ {\isacharequal}{\kern0pt}\ NOT\ {\isasymcirc}\isactrlsub c\ is{\isacharunderscore}{\kern0pt}odd{\isachardoublequoteclose}\isanewline
%
\isadelimproof
%
\endisadelimproof
%
\isatagproof
\isacommand{proof}\isamarkupfalse%
\ {\isacharparenleft}{\kern0pt}typecheck{\isacharunderscore}{\kern0pt}cfuncs{\isacharcomma}{\kern0pt}\ rule\ natural{\isacharunderscore}{\kern0pt}number{\isacharunderscore}{\kern0pt}object{\isacharunderscore}{\kern0pt}func{\isacharunderscore}{\kern0pt}unique{\isacharbrackleft}{\kern0pt}\isakeyword{where}\ f{\isacharequal}{\kern0pt}{\isachardoublequoteopen}NOT{\isachardoublequoteclose}{\isacharcomma}{\kern0pt}\ \isakeyword{where}\ X{\isacharequal}{\kern0pt}{\isachardoublequoteopen}{\isasymOmega}{\isachardoublequoteclose}{\isacharbrackright}{\kern0pt}{\isacharcomma}{\kern0pt}\ clarify{\isacharparenright}{\kern0pt}\isanewline
\ \ \isacommand{show}\isamarkupfalse%
\ {\isachardoublequoteopen}is{\isacharunderscore}{\kern0pt}even\ {\isasymcirc}\isactrlsub c\ zero\ {\isacharequal}{\kern0pt}\ {\isacharparenleft}{\kern0pt}NOT\ {\isasymcirc}\isactrlsub c\ is{\isacharunderscore}{\kern0pt}odd{\isacharparenright}{\kern0pt}\ {\isasymcirc}\isactrlsub c\ zero{\isachardoublequoteclose}\isanewline
\ \ \ \ \isacommand{by}\isamarkupfalse%
\ {\isacharparenleft}{\kern0pt}typecheck{\isacharunderscore}{\kern0pt}cfuncs{\isacharcomma}{\kern0pt}\ metis\ NOT{\isacharunderscore}{\kern0pt}false{\isacharunderscore}{\kern0pt}is{\isacharunderscore}{\kern0pt}true\ cfunc{\isacharunderscore}{\kern0pt}type{\isacharunderscore}{\kern0pt}def\ comp{\isacharunderscore}{\kern0pt}associative\ is{\isacharunderscore}{\kern0pt}even{\isacharunderscore}{\kern0pt}def{\isadigit{2}}\ is{\isacharunderscore}{\kern0pt}odd{\isacharunderscore}{\kern0pt}def{\isadigit{2}}{\isacharparenright}{\kern0pt}\isanewline
\isanewline
\ \ \isacommand{show}\isamarkupfalse%
\ {\isachardoublequoteopen}is{\isacharunderscore}{\kern0pt}even\ {\isasymcirc}\isactrlsub c\ successor\ {\isacharequal}{\kern0pt}\ NOT\ {\isasymcirc}\isactrlsub c\ is{\isacharunderscore}{\kern0pt}even{\isachardoublequoteclose}\isanewline
\ \ \ \ \isacommand{by}\isamarkupfalse%
\ {\isacharparenleft}{\kern0pt}simp\ add{\isacharcolon}{\kern0pt}\ is{\isacharunderscore}{\kern0pt}even{\isacharunderscore}{\kern0pt}successor{\isacharparenright}{\kern0pt}\isanewline
\isanewline
\ \ \isacommand{show}\isamarkupfalse%
\ {\isachardoublequoteopen}{\isacharparenleft}{\kern0pt}NOT\ {\isasymcirc}\isactrlsub c\ is{\isacharunderscore}{\kern0pt}odd{\isacharparenright}{\kern0pt}\ {\isasymcirc}\isactrlsub c\ successor\ {\isacharequal}{\kern0pt}\ NOT\ {\isasymcirc}\isactrlsub c\ NOT\ {\isasymcirc}\isactrlsub c\ is{\isacharunderscore}{\kern0pt}odd{\isachardoublequoteclose}\isanewline
\ \ \ \ \isacommand{by}\isamarkupfalse%
\ {\isacharparenleft}{\kern0pt}typecheck{\isacharunderscore}{\kern0pt}cfuncs{\isacharcomma}{\kern0pt}\ simp\ add{\isacharcolon}{\kern0pt}\ cfunc{\isacharunderscore}{\kern0pt}type{\isacharunderscore}{\kern0pt}def\ comp{\isacharunderscore}{\kern0pt}associative\ is{\isacharunderscore}{\kern0pt}odd{\isacharunderscore}{\kern0pt}def{\isadigit{2}}{\isacharparenright}{\kern0pt}\isanewline
\isacommand{qed}\isamarkupfalse%
%
\endisatagproof
{\isafoldproof}%
%
\isadelimproof
\isanewline
%
\endisadelimproof
\isanewline
\isacommand{lemma}\isamarkupfalse%
\ is{\isacharunderscore}{\kern0pt}odd{\isacharunderscore}{\kern0pt}not{\isacharunderscore}{\kern0pt}is{\isacharunderscore}{\kern0pt}even{\isacharcolon}{\kern0pt}\isanewline
\ \ {\isachardoublequoteopen}is{\isacharunderscore}{\kern0pt}odd\ {\isacharequal}{\kern0pt}\ NOT\ {\isasymcirc}\isactrlsub c\ is{\isacharunderscore}{\kern0pt}even{\isachardoublequoteclose}\isanewline
%
\isadelimproof
%
\endisadelimproof
%
\isatagproof
\isacommand{proof}\isamarkupfalse%
\ {\isacharparenleft}{\kern0pt}typecheck{\isacharunderscore}{\kern0pt}cfuncs{\isacharcomma}{\kern0pt}\ rule\ natural{\isacharunderscore}{\kern0pt}number{\isacharunderscore}{\kern0pt}object{\isacharunderscore}{\kern0pt}func{\isacharunderscore}{\kern0pt}unique{\isacharbrackleft}{\kern0pt}\isakeyword{where}\ f{\isacharequal}{\kern0pt}{\isachardoublequoteopen}NOT{\isachardoublequoteclose}{\isacharcomma}{\kern0pt}\ \isakeyword{where}\ X{\isacharequal}{\kern0pt}{\isachardoublequoteopen}{\isasymOmega}{\isachardoublequoteclose}{\isacharbrackright}{\kern0pt}{\isacharcomma}{\kern0pt}\ clarify{\isacharparenright}{\kern0pt}\isanewline
\ \ \isacommand{show}\isamarkupfalse%
\ {\isachardoublequoteopen}is{\isacharunderscore}{\kern0pt}odd\ {\isasymcirc}\isactrlsub c\ zero\ {\isacharequal}{\kern0pt}\ {\isacharparenleft}{\kern0pt}NOT\ {\isasymcirc}\isactrlsub c\ is{\isacharunderscore}{\kern0pt}even{\isacharparenright}{\kern0pt}\ {\isasymcirc}\isactrlsub c\ zero{\isachardoublequoteclose}\isanewline
\ \ \ \ \isacommand{by}\isamarkupfalse%
\ {\isacharparenleft}{\kern0pt}typecheck{\isacharunderscore}{\kern0pt}cfuncs{\isacharcomma}{\kern0pt}\ metis\ NOT{\isacharunderscore}{\kern0pt}true{\isacharunderscore}{\kern0pt}is{\isacharunderscore}{\kern0pt}false\ cfunc{\isacharunderscore}{\kern0pt}type{\isacharunderscore}{\kern0pt}def\ comp{\isacharunderscore}{\kern0pt}associative\ is{\isacharunderscore}{\kern0pt}even{\isacharunderscore}{\kern0pt}def{\isadigit{2}}\ is{\isacharunderscore}{\kern0pt}odd{\isacharunderscore}{\kern0pt}def{\isadigit{2}}{\isacharparenright}{\kern0pt}\isanewline
\isanewline
\ \ \isacommand{show}\isamarkupfalse%
\ {\isachardoublequoteopen}is{\isacharunderscore}{\kern0pt}odd\ {\isasymcirc}\isactrlsub c\ successor\ {\isacharequal}{\kern0pt}\ NOT\ {\isasymcirc}\isactrlsub c\ is{\isacharunderscore}{\kern0pt}odd{\isachardoublequoteclose}\isanewline
\ \ \ \ \isacommand{by}\isamarkupfalse%
\ {\isacharparenleft}{\kern0pt}simp\ add{\isacharcolon}{\kern0pt}\ is{\isacharunderscore}{\kern0pt}odd{\isacharunderscore}{\kern0pt}successor{\isacharparenright}{\kern0pt}\isanewline
\isanewline
\ \ \isacommand{show}\isamarkupfalse%
\ {\isachardoublequoteopen}{\isacharparenleft}{\kern0pt}NOT\ {\isasymcirc}\isactrlsub c\ is{\isacharunderscore}{\kern0pt}even{\isacharparenright}{\kern0pt}\ {\isasymcirc}\isactrlsub c\ successor\ {\isacharequal}{\kern0pt}\ NOT\ {\isasymcirc}\isactrlsub c\ NOT\ {\isasymcirc}\isactrlsub c\ is{\isacharunderscore}{\kern0pt}even{\isachardoublequoteclose}\isanewline
\ \ \ \ \isacommand{by}\isamarkupfalse%
\ {\isacharparenleft}{\kern0pt}typecheck{\isacharunderscore}{\kern0pt}cfuncs{\isacharcomma}{\kern0pt}\ simp\ add{\isacharcolon}{\kern0pt}\ cfunc{\isacharunderscore}{\kern0pt}type{\isacharunderscore}{\kern0pt}def\ comp{\isacharunderscore}{\kern0pt}associative\ is{\isacharunderscore}{\kern0pt}even{\isacharunderscore}{\kern0pt}def{\isadigit{2}}{\isacharparenright}{\kern0pt}\isanewline
\isacommand{qed}\isamarkupfalse%
%
\endisatagproof
{\isafoldproof}%
%
\isadelimproof
\isanewline
%
\endisadelimproof
\isanewline
\isacommand{lemma}\isamarkupfalse%
\ not{\isacharunderscore}{\kern0pt}even{\isacharunderscore}{\kern0pt}and{\isacharunderscore}{\kern0pt}odd{\isacharcolon}{\kern0pt}\isanewline
\ \ \isakeyword{assumes}\ {\isachardoublequoteopen}m\ {\isasymin}\isactrlsub c\ {\isasymnat}\isactrlsub c{\isachardoublequoteclose}\isanewline
\ \ \isakeyword{shows}\ {\isachardoublequoteopen}{\isasymnot}{\isacharparenleft}{\kern0pt}is{\isacharunderscore}{\kern0pt}even\ {\isasymcirc}\isactrlsub c\ m\ {\isacharequal}{\kern0pt}\ {\isasymt}\ {\isasymand}\ is{\isacharunderscore}{\kern0pt}odd\ {\isasymcirc}\isactrlsub c\ m\ {\isacharequal}{\kern0pt}\ {\isasymt}{\isacharparenright}{\kern0pt}{\isachardoublequoteclose}\isanewline
%
\isadelimproof
\ \ %
\endisadelimproof
%
\isatagproof
\isacommand{using}\isamarkupfalse%
\ assms\ NOT{\isacharunderscore}{\kern0pt}true{\isacharunderscore}{\kern0pt}is{\isacharunderscore}{\kern0pt}false\ NOT{\isacharunderscore}{\kern0pt}type\ comp{\isacharunderscore}{\kern0pt}associative{\isadigit{2}}\ is{\isacharunderscore}{\kern0pt}even{\isacharunderscore}{\kern0pt}not{\isacharunderscore}{\kern0pt}is{\isacharunderscore}{\kern0pt}odd\ true{\isacharunderscore}{\kern0pt}false{\isacharunderscore}{\kern0pt}distinct\ \isacommand{by}\isamarkupfalse%
\ {\isacharparenleft}{\kern0pt}typecheck{\isacharunderscore}{\kern0pt}cfuncs{\isacharcomma}{\kern0pt}\ fastforce{\isacharparenright}{\kern0pt}%
\endisatagproof
{\isafoldproof}%
%
\isadelimproof
\isanewline
%
\endisadelimproof
\isanewline
\isacommand{lemma}\isamarkupfalse%
\ even{\isacharunderscore}{\kern0pt}or{\isacharunderscore}{\kern0pt}odd{\isacharcolon}{\kern0pt}\isanewline
\ \ \isakeyword{assumes}\ {\isachardoublequoteopen}n\ {\isasymin}\isactrlsub c\ {\isasymnat}\isactrlsub c{\isachardoublequoteclose}\isanewline
\ \ \isakeyword{shows}\ {\isachardoublequoteopen}is{\isacharunderscore}{\kern0pt}even\ {\isasymcirc}\isactrlsub c\ n\ {\isacharequal}{\kern0pt}\ {\isasymt}\ {\isasymor}\ is{\isacharunderscore}{\kern0pt}odd\ {\isasymcirc}\isactrlsub c\ n\ {\isacharequal}{\kern0pt}\ {\isasymt}{\isachardoublequoteclose}\isanewline
%
\isadelimproof
\ \ %
\endisadelimproof
%
\isatagproof
\isacommand{by}\isamarkupfalse%
\ {\isacharparenleft}{\kern0pt}typecheck{\isacharunderscore}{\kern0pt}cfuncs{\isacharcomma}{\kern0pt}\ metis\ NOT{\isacharunderscore}{\kern0pt}false{\isacharunderscore}{\kern0pt}is{\isacharunderscore}{\kern0pt}true\ NOT{\isacharunderscore}{\kern0pt}type\ comp{\isacharunderscore}{\kern0pt}associative{\isadigit{2}}\ is{\isacharunderscore}{\kern0pt}even{\isacharunderscore}{\kern0pt}not{\isacharunderscore}{\kern0pt}is{\isacharunderscore}{\kern0pt}odd\ true{\isacharunderscore}{\kern0pt}false{\isacharunderscore}{\kern0pt}only{\isacharunderscore}{\kern0pt}truth{\isacharunderscore}{\kern0pt}values\ assms{\isacharparenright}{\kern0pt}%
\endisatagproof
{\isafoldproof}%
%
\isadelimproof
\isanewline
%
\endisadelimproof
\isanewline
\isacommand{lemma}\isamarkupfalse%
\ is{\isacharunderscore}{\kern0pt}even{\isacharunderscore}{\kern0pt}nth{\isacharunderscore}{\kern0pt}even{\isacharunderscore}{\kern0pt}true{\isacharcolon}{\kern0pt}\isanewline
\ \ {\isachardoublequoteopen}is{\isacharunderscore}{\kern0pt}even\ {\isasymcirc}\isactrlsub c\ nth{\isacharunderscore}{\kern0pt}even\ {\isacharequal}{\kern0pt}\ {\isasymt}\ {\isasymcirc}\isactrlsub c\ {\isasymbeta}\isactrlbsub {\isasymnat}\isactrlsub c\isactrlesub {\isachardoublequoteclose}\isanewline
%
\isadelimproof
%
\endisadelimproof
%
\isatagproof
\isacommand{proof}\isamarkupfalse%
\ {\isacharparenleft}{\kern0pt}etcs{\isacharunderscore}{\kern0pt}rule\ natural{\isacharunderscore}{\kern0pt}number{\isacharunderscore}{\kern0pt}object{\isacharunderscore}{\kern0pt}func{\isacharunderscore}{\kern0pt}unique{\isacharbrackleft}{\kern0pt}\isakeyword{where}\ f{\isacharequal}{\kern0pt}{\isachardoublequoteopen}id\ {\isasymOmega}{\isachardoublequoteclose}{\isacharcomma}{\kern0pt}\ \isakeyword{where}\ X{\isacharequal}{\kern0pt}{\isasymOmega}{\isacharbrackright}{\kern0pt}{\isacharparenright}{\kern0pt}\isanewline
\ \ \isacommand{show}\isamarkupfalse%
\ {\isachardoublequoteopen}{\isacharparenleft}{\kern0pt}is{\isacharunderscore}{\kern0pt}even\ {\isasymcirc}\isactrlsub c\ nth{\isacharunderscore}{\kern0pt}even{\isacharparenright}{\kern0pt}\ {\isasymcirc}\isactrlsub c\ zero\ {\isacharequal}{\kern0pt}\ {\isacharparenleft}{\kern0pt}{\isasymt}\ {\isasymcirc}\isactrlsub c\ {\isasymbeta}\isactrlbsub {\isasymnat}\isactrlsub c\isactrlesub {\isacharparenright}{\kern0pt}\ {\isasymcirc}\isactrlsub c\ zero{\isachardoublequoteclose}\isanewline
\ \ \isacommand{proof}\isamarkupfalse%
\ {\isacharminus}{\kern0pt}\isanewline
\ \ \ \ \isacommand{have}\isamarkupfalse%
\ {\isachardoublequoteopen}{\isacharparenleft}{\kern0pt}is{\isacharunderscore}{\kern0pt}even\ {\isasymcirc}\isactrlsub c\ nth{\isacharunderscore}{\kern0pt}even{\isacharparenright}{\kern0pt}\ {\isasymcirc}\isactrlsub c\ zero\ {\isacharequal}{\kern0pt}\ is{\isacharunderscore}{\kern0pt}even\ {\isasymcirc}\isactrlsub c\ nth{\isacharunderscore}{\kern0pt}even\ {\isasymcirc}\isactrlsub c\ zero{\isachardoublequoteclose}\isanewline
\ \ \ \ \ \ \isacommand{by}\isamarkupfalse%
\ {\isacharparenleft}{\kern0pt}typecheck{\isacharunderscore}{\kern0pt}cfuncs{\isacharcomma}{\kern0pt}\ simp\ add{\isacharcolon}{\kern0pt}\ comp{\isacharunderscore}{\kern0pt}associative{\isadigit{2}}{\isacharparenright}{\kern0pt}\isanewline
\ \ \ \ \isacommand{also}\isamarkupfalse%
\ \isacommand{have}\isamarkupfalse%
\ {\isachardoublequoteopen}{\isachardot}{\kern0pt}{\isachardot}{\kern0pt}{\isachardot}{\kern0pt}\ {\isacharequal}{\kern0pt}\ {\isasymt}{\isachardoublequoteclose}\isanewline
\ \ \ \ \ \ \isacommand{by}\isamarkupfalse%
\ {\isacharparenleft}{\kern0pt}simp\ add{\isacharcolon}{\kern0pt}\ is{\isacharunderscore}{\kern0pt}even{\isacharunderscore}{\kern0pt}zero\ nth{\isacharunderscore}{\kern0pt}even{\isacharunderscore}{\kern0pt}zero{\isacharparenright}{\kern0pt}\isanewline
\ \ \ \ \isacommand{also}\isamarkupfalse%
\ \isacommand{have}\isamarkupfalse%
\ {\isachardoublequoteopen}{\isachardot}{\kern0pt}{\isachardot}{\kern0pt}{\isachardot}{\kern0pt}\ {\isacharequal}{\kern0pt}\ {\isacharparenleft}{\kern0pt}{\isasymt}\ {\isasymcirc}\isactrlsub c\ {\isasymbeta}\isactrlbsub {\isasymnat}\isactrlsub c\isactrlesub {\isacharparenright}{\kern0pt}\ {\isasymcirc}\isactrlsub c\ zero{\isachardoublequoteclose}\isanewline
\ \ \ \ \ \ \isacommand{by}\isamarkupfalse%
\ {\isacharparenleft}{\kern0pt}typecheck{\isacharunderscore}{\kern0pt}cfuncs{\isacharcomma}{\kern0pt}\ metis\ comp{\isacharunderscore}{\kern0pt}associative{\isadigit{2}}\ id{\isacharunderscore}{\kern0pt}right{\isacharunderscore}{\kern0pt}unit{\isadigit{2}}\ terminal{\isacharunderscore}{\kern0pt}func{\isacharunderscore}{\kern0pt}comp{\isacharunderscore}{\kern0pt}elem{\isacharparenright}{\kern0pt}\isanewline
\ \ \ \ \isacommand{then}\isamarkupfalse%
\ \isacommand{show}\isamarkupfalse%
\ {\isacharquery}{\kern0pt}thesis\isanewline
\ \ \ \ \ \ \isacommand{using}\isamarkupfalse%
\ calculation\ \isacommand{by}\isamarkupfalse%
\ auto\isanewline
\ \ \isacommand{qed}\isamarkupfalse%
\isanewline
\isanewline
\ \ \isacommand{show}\isamarkupfalse%
\ {\isachardoublequoteopen}{\isacharparenleft}{\kern0pt}is{\isacharunderscore}{\kern0pt}even\ {\isasymcirc}\isactrlsub c\ nth{\isacharunderscore}{\kern0pt}even{\isacharparenright}{\kern0pt}\ {\isasymcirc}\isactrlsub c\ successor\ {\isacharequal}{\kern0pt}\ id\isactrlsub c\ {\isasymOmega}\ {\isasymcirc}\isactrlsub c\ is{\isacharunderscore}{\kern0pt}even\ {\isasymcirc}\isactrlsub c\ nth{\isacharunderscore}{\kern0pt}even{\isachardoublequoteclose}\isanewline
\ \ \isacommand{proof}\isamarkupfalse%
\ {\isacharminus}{\kern0pt}\isanewline
\ \ \ \ \isacommand{have}\isamarkupfalse%
\ {\isachardoublequoteopen}{\isacharparenleft}{\kern0pt}is{\isacharunderscore}{\kern0pt}even\ {\isasymcirc}\isactrlsub c\ nth{\isacharunderscore}{\kern0pt}even{\isacharparenright}{\kern0pt}\ {\isasymcirc}\isactrlsub c\ successor\ {\isacharequal}{\kern0pt}\ is{\isacharunderscore}{\kern0pt}even\ {\isasymcirc}\isactrlsub c\ nth{\isacharunderscore}{\kern0pt}even\ {\isasymcirc}\isactrlsub c\ successor{\isachardoublequoteclose}\isanewline
\ \ \ \ \ \ \isacommand{by}\isamarkupfalse%
\ {\isacharparenleft}{\kern0pt}typecheck{\isacharunderscore}{\kern0pt}cfuncs{\isacharcomma}{\kern0pt}\ simp\ add{\isacharcolon}{\kern0pt}\ comp{\isacharunderscore}{\kern0pt}associative{\isadigit{2}}{\isacharparenright}{\kern0pt}\isanewline
\ \ \ \ \isacommand{also}\isamarkupfalse%
\ \isacommand{have}\isamarkupfalse%
\ {\isachardoublequoteopen}{\isachardot}{\kern0pt}{\isachardot}{\kern0pt}{\isachardot}{\kern0pt}\ {\isacharequal}{\kern0pt}\ is{\isacharunderscore}{\kern0pt}even\ {\isasymcirc}\isactrlsub c\ successor\ {\isasymcirc}\isactrlsub c\ successor\ {\isasymcirc}\isactrlsub c\ nth{\isacharunderscore}{\kern0pt}even{\isachardoublequoteclose}\isanewline
\ \ \ \ \ \ \isacommand{by}\isamarkupfalse%
\ {\isacharparenleft}{\kern0pt}simp\ add{\isacharcolon}{\kern0pt}\ nth{\isacharunderscore}{\kern0pt}even{\isacharunderscore}{\kern0pt}successor{\isadigit{2}}{\isacharparenright}{\kern0pt}\isanewline
\ \ \ \ \isacommand{also}\isamarkupfalse%
\ \isacommand{have}\isamarkupfalse%
\ {\isachardoublequoteopen}{\isachardot}{\kern0pt}{\isachardot}{\kern0pt}{\isachardot}{\kern0pt}\ {\isacharequal}{\kern0pt}\ {\isacharparenleft}{\kern0pt}{\isacharparenleft}{\kern0pt}is{\isacharunderscore}{\kern0pt}even\ {\isasymcirc}\isactrlsub c\ successor{\isacharparenright}{\kern0pt}\ {\isasymcirc}\isactrlsub c\ successor{\isacharparenright}{\kern0pt}\ {\isasymcirc}\isactrlsub c\ nth{\isacharunderscore}{\kern0pt}even{\isachardoublequoteclose}\isanewline
\ \ \ \ \ \ \isacommand{by}\isamarkupfalse%
\ {\isacharparenleft}{\kern0pt}typecheck{\isacharunderscore}{\kern0pt}cfuncs{\isacharcomma}{\kern0pt}\ smt\ comp{\isacharunderscore}{\kern0pt}associative{\isadigit{2}}{\isacharparenright}{\kern0pt}\isanewline
\ \ \ \ \isacommand{also}\isamarkupfalse%
\ \isacommand{have}\isamarkupfalse%
\ {\isachardoublequoteopen}{\isachardot}{\kern0pt}{\isachardot}{\kern0pt}{\isachardot}{\kern0pt}\ {\isacharequal}{\kern0pt}\ \ is{\isacharunderscore}{\kern0pt}even\ {\isasymcirc}\isactrlsub c\ nth{\isacharunderscore}{\kern0pt}even{\isachardoublequoteclose}\isanewline
\ \ \ \ \ \ \isacommand{using}\isamarkupfalse%
\ is{\isacharunderscore}{\kern0pt}even{\isacharunderscore}{\kern0pt}def{\isadigit{2}}\ is{\isacharunderscore}{\kern0pt}even{\isacharunderscore}{\kern0pt}not{\isacharunderscore}{\kern0pt}is{\isacharunderscore}{\kern0pt}odd\ is{\isacharunderscore}{\kern0pt}odd{\isacharunderscore}{\kern0pt}def{\isadigit{2}}\ is{\isacharunderscore}{\kern0pt}odd{\isacharunderscore}{\kern0pt}not{\isacharunderscore}{\kern0pt}is{\isacharunderscore}{\kern0pt}even\ \isacommand{by}\isamarkupfalse%
\ {\isacharparenleft}{\kern0pt}typecheck{\isacharunderscore}{\kern0pt}cfuncs{\isacharcomma}{\kern0pt}\ auto{\isacharparenright}{\kern0pt}\isanewline
\ \ \ \ \isacommand{also}\isamarkupfalse%
\ \isacommand{have}\isamarkupfalse%
\ {\isachardoublequoteopen}{\isachardot}{\kern0pt}{\isachardot}{\kern0pt}{\isachardot}{\kern0pt}\ {\isacharequal}{\kern0pt}\ id\ {\isasymOmega}\ {\isasymcirc}\isactrlsub c\ is{\isacharunderscore}{\kern0pt}even\ {\isasymcirc}\isactrlsub c\ nth{\isacharunderscore}{\kern0pt}even{\isachardoublequoteclose}\isanewline
\ \ \ \ \ \ \isacommand{by}\isamarkupfalse%
\ {\isacharparenleft}{\kern0pt}typecheck{\isacharunderscore}{\kern0pt}cfuncs{\isacharcomma}{\kern0pt}\ simp\ add{\isacharcolon}{\kern0pt}\ id{\isacharunderscore}{\kern0pt}left{\isacharunderscore}{\kern0pt}unit{\isadigit{2}}{\isacharparenright}{\kern0pt}\isanewline
\ \ \ \ \isacommand{then}\isamarkupfalse%
\ \isacommand{show}\isamarkupfalse%
\ {\isacharquery}{\kern0pt}thesis\isanewline
\ \ \ \ \ \ \isacommand{using}\isamarkupfalse%
\ calculation\ \isacommand{by}\isamarkupfalse%
\ auto\isanewline
\ \ \isacommand{qed}\isamarkupfalse%
\isanewline
\isanewline
\ \ \isacommand{show}\isamarkupfalse%
\ {\isachardoublequoteopen}{\isacharparenleft}{\kern0pt}{\isasymt}\ {\isasymcirc}\isactrlsub c\ {\isasymbeta}\isactrlbsub {\isasymnat}\isactrlsub c\isactrlesub {\isacharparenright}{\kern0pt}\ {\isasymcirc}\isactrlsub c\ successor\ {\isacharequal}{\kern0pt}\ id\isactrlsub c\ {\isasymOmega}\ {\isasymcirc}\isactrlsub c\ {\isasymt}\ {\isasymcirc}\isactrlsub c\ {\isasymbeta}\isactrlbsub {\isasymnat}\isactrlsub c\isactrlesub {\isachardoublequoteclose}\isanewline
\ \ \ \ \isacommand{by}\isamarkupfalse%
\ {\isacharparenleft}{\kern0pt}typecheck{\isacharunderscore}{\kern0pt}cfuncs{\isacharcomma}{\kern0pt}\ smt\ comp{\isacharunderscore}{\kern0pt}associative{\isadigit{2}}\ id{\isacharunderscore}{\kern0pt}left{\isacharunderscore}{\kern0pt}unit{\isadigit{2}}\ terminal{\isacharunderscore}{\kern0pt}func{\isacharunderscore}{\kern0pt}comp{\isacharparenright}{\kern0pt}\isanewline
\isacommand{qed}\isamarkupfalse%
%
\endisatagproof
{\isafoldproof}%
%
\isadelimproof
\isanewline
%
\endisadelimproof
\isanewline
\isacommand{lemma}\isamarkupfalse%
\ is{\isacharunderscore}{\kern0pt}odd{\isacharunderscore}{\kern0pt}nth{\isacharunderscore}{\kern0pt}odd{\isacharunderscore}{\kern0pt}true{\isacharcolon}{\kern0pt}\isanewline
\ \ {\isachardoublequoteopen}is{\isacharunderscore}{\kern0pt}odd\ {\isasymcirc}\isactrlsub c\ nth{\isacharunderscore}{\kern0pt}odd\ {\isacharequal}{\kern0pt}\ {\isasymt}\ {\isasymcirc}\isactrlsub c\ {\isasymbeta}\isactrlbsub {\isasymnat}\isactrlsub c\isactrlesub {\isachardoublequoteclose}\isanewline
%
\isadelimproof
%
\endisadelimproof
%
\isatagproof
\isacommand{proof}\isamarkupfalse%
\ {\isacharparenleft}{\kern0pt}etcs{\isacharunderscore}{\kern0pt}rule\ natural{\isacharunderscore}{\kern0pt}number{\isacharunderscore}{\kern0pt}object{\isacharunderscore}{\kern0pt}func{\isacharunderscore}{\kern0pt}unique{\isacharbrackleft}{\kern0pt}\isakeyword{where}\ f{\isacharequal}{\kern0pt}{\isachardoublequoteopen}id\ {\isasymOmega}{\isachardoublequoteclose}{\isacharcomma}{\kern0pt}\ \isakeyword{where}\ X{\isacharequal}{\kern0pt}{\isasymOmega}{\isacharbrackright}{\kern0pt}{\isacharparenright}{\kern0pt}\isanewline
\ \ \isacommand{show}\isamarkupfalse%
\ {\isachardoublequoteopen}{\isacharparenleft}{\kern0pt}is{\isacharunderscore}{\kern0pt}odd\ {\isasymcirc}\isactrlsub c\ nth{\isacharunderscore}{\kern0pt}odd{\isacharparenright}{\kern0pt}\ {\isasymcirc}\isactrlsub c\ zero\ {\isacharequal}{\kern0pt}\ {\isacharparenleft}{\kern0pt}{\isasymt}\ {\isasymcirc}\isactrlsub c\ {\isasymbeta}\isactrlbsub {\isasymnat}\isactrlsub c\isactrlesub {\isacharparenright}{\kern0pt}\ {\isasymcirc}\isactrlsub c\ zero{\isachardoublequoteclose}\isanewline
\ \ \isacommand{proof}\isamarkupfalse%
\ {\isacharminus}{\kern0pt}\isanewline
\ \ \ \ \isacommand{have}\isamarkupfalse%
\ {\isachardoublequoteopen}{\isacharparenleft}{\kern0pt}is{\isacharunderscore}{\kern0pt}odd\ {\isasymcirc}\isactrlsub c\ nth{\isacharunderscore}{\kern0pt}odd{\isacharparenright}{\kern0pt}\ {\isasymcirc}\isactrlsub c\ zero\ {\isacharequal}{\kern0pt}\ is{\isacharunderscore}{\kern0pt}odd\ {\isasymcirc}\isactrlsub c\ nth{\isacharunderscore}{\kern0pt}odd\ {\isasymcirc}\isactrlsub c\ zero{\isachardoublequoteclose}\isanewline
\ \ \ \ \ \ \isacommand{by}\isamarkupfalse%
\ {\isacharparenleft}{\kern0pt}typecheck{\isacharunderscore}{\kern0pt}cfuncs{\isacharcomma}{\kern0pt}\ simp\ add{\isacharcolon}{\kern0pt}\ comp{\isacharunderscore}{\kern0pt}associative{\isadigit{2}}{\isacharparenright}{\kern0pt}\isanewline
\ \ \ \ \isacommand{also}\isamarkupfalse%
\ \isacommand{have}\isamarkupfalse%
\ {\isachardoublequoteopen}{\isachardot}{\kern0pt}{\isachardot}{\kern0pt}{\isachardot}{\kern0pt}\ {\isacharequal}{\kern0pt}\ {\isasymt}{\isachardoublequoteclose}\isanewline
\ \ \ \ \ \ \isacommand{using}\isamarkupfalse%
\ comp{\isacharunderscore}{\kern0pt}associative{\isadigit{2}}\ is{\isacharunderscore}{\kern0pt}even{\isacharunderscore}{\kern0pt}not{\isacharunderscore}{\kern0pt}is{\isacharunderscore}{\kern0pt}odd\ is{\isacharunderscore}{\kern0pt}even{\isacharunderscore}{\kern0pt}zero\ is{\isacharunderscore}{\kern0pt}odd{\isacharunderscore}{\kern0pt}def{\isadigit{2}}\ nth{\isacharunderscore}{\kern0pt}odd{\isacharunderscore}{\kern0pt}def{\isadigit{2}}\ successor{\isacharunderscore}{\kern0pt}type\ zero{\isacharunderscore}{\kern0pt}type\ \isacommand{by}\isamarkupfalse%
\ auto\isanewline
\ \ \ \ \isacommand{also}\isamarkupfalse%
\ \isacommand{have}\isamarkupfalse%
\ {\isachardoublequoteopen}{\isachardot}{\kern0pt}{\isachardot}{\kern0pt}{\isachardot}{\kern0pt}\ {\isacharequal}{\kern0pt}\ {\isacharparenleft}{\kern0pt}{\isasymt}\ {\isasymcirc}\isactrlsub c\ {\isasymbeta}\isactrlbsub {\isasymnat}\isactrlsub c\isactrlesub {\isacharparenright}{\kern0pt}\ {\isasymcirc}\isactrlsub c\ zero{\isachardoublequoteclose}\isanewline
\ \ \ \ \ \ \isacommand{by}\isamarkupfalse%
\ {\isacharparenleft}{\kern0pt}typecheck{\isacharunderscore}{\kern0pt}cfuncs{\isacharcomma}{\kern0pt}\ metis\ comp{\isacharunderscore}{\kern0pt}associative{\isadigit{2}}\ is{\isacharunderscore}{\kern0pt}even{\isacharunderscore}{\kern0pt}nth{\isacharunderscore}{\kern0pt}even{\isacharunderscore}{\kern0pt}true\ is{\isacharunderscore}{\kern0pt}even{\isacharunderscore}{\kern0pt}type\ is{\isacharunderscore}{\kern0pt}even{\isacharunderscore}{\kern0pt}zero\ nth{\isacharunderscore}{\kern0pt}even{\isacharunderscore}{\kern0pt}def{\isadigit{2}}{\isacharparenright}{\kern0pt}\isanewline
\ \ \ \ \isacommand{then}\isamarkupfalse%
\ \isacommand{show}\isamarkupfalse%
\ {\isacharquery}{\kern0pt}thesis\isanewline
\ \ \ \ \ \ \isacommand{using}\isamarkupfalse%
\ calculation\ \isacommand{by}\isamarkupfalse%
\ auto\isanewline
\ \ \isacommand{qed}\isamarkupfalse%
\isanewline
\isanewline
\ \ \isacommand{show}\isamarkupfalse%
\ {\isachardoublequoteopen}{\isacharparenleft}{\kern0pt}is{\isacharunderscore}{\kern0pt}odd\ {\isasymcirc}\isactrlsub c\ nth{\isacharunderscore}{\kern0pt}odd{\isacharparenright}{\kern0pt}\ {\isasymcirc}\isactrlsub c\ successor\ {\isacharequal}{\kern0pt}\ id\isactrlsub c\ {\isasymOmega}\ {\isasymcirc}\isactrlsub c\ is{\isacharunderscore}{\kern0pt}odd\ {\isasymcirc}\isactrlsub c\ nth{\isacharunderscore}{\kern0pt}odd{\isachardoublequoteclose}\isanewline
\ \ \isacommand{proof}\isamarkupfalse%
\ {\isacharminus}{\kern0pt}\isanewline
\ \ \ \ \isacommand{have}\isamarkupfalse%
\ {\isachardoublequoteopen}{\isacharparenleft}{\kern0pt}is{\isacharunderscore}{\kern0pt}odd\ {\isasymcirc}\isactrlsub c\ nth{\isacharunderscore}{\kern0pt}odd{\isacharparenright}{\kern0pt}\ {\isasymcirc}\isactrlsub c\ successor\ {\isacharequal}{\kern0pt}\ is{\isacharunderscore}{\kern0pt}odd\ {\isasymcirc}\isactrlsub c\ nth{\isacharunderscore}{\kern0pt}odd\ {\isasymcirc}\isactrlsub c\ successor{\isachardoublequoteclose}\isanewline
\ \ \ \ \ \ \isacommand{by}\isamarkupfalse%
\ {\isacharparenleft}{\kern0pt}typecheck{\isacharunderscore}{\kern0pt}cfuncs{\isacharcomma}{\kern0pt}\ simp\ add{\isacharcolon}{\kern0pt}\ comp{\isacharunderscore}{\kern0pt}associative{\isadigit{2}}{\isacharparenright}{\kern0pt}\isanewline
\ \ \ \ \isacommand{also}\isamarkupfalse%
\ \isacommand{have}\isamarkupfalse%
\ {\isachardoublequoteopen}{\isachardot}{\kern0pt}{\isachardot}{\kern0pt}{\isachardot}{\kern0pt}\ {\isacharequal}{\kern0pt}\ is{\isacharunderscore}{\kern0pt}odd\ {\isasymcirc}\isactrlsub c\ successor\ {\isasymcirc}\isactrlsub c\ successor\ {\isasymcirc}\isactrlsub c\ nth{\isacharunderscore}{\kern0pt}odd{\isachardoublequoteclose}\isanewline
\ \ \ \ \ \ \isacommand{by}\isamarkupfalse%
\ {\isacharparenleft}{\kern0pt}simp\ add{\isacharcolon}{\kern0pt}\ nth{\isacharunderscore}{\kern0pt}odd{\isacharunderscore}{\kern0pt}successor{\isadigit{2}}{\isacharparenright}{\kern0pt}\isanewline
\ \ \ \ \isacommand{also}\isamarkupfalse%
\ \isacommand{have}\isamarkupfalse%
\ {\isachardoublequoteopen}{\isachardot}{\kern0pt}{\isachardot}{\kern0pt}{\isachardot}{\kern0pt}\ {\isacharequal}{\kern0pt}\ {\isacharparenleft}{\kern0pt}{\isacharparenleft}{\kern0pt}is{\isacharunderscore}{\kern0pt}odd\ {\isasymcirc}\isactrlsub c\ successor{\isacharparenright}{\kern0pt}\ {\isasymcirc}\isactrlsub c\ successor{\isacharparenright}{\kern0pt}\ {\isasymcirc}\isactrlsub c\ nth{\isacharunderscore}{\kern0pt}odd{\isachardoublequoteclose}\isanewline
\ \ \ \ \ \ \isacommand{by}\isamarkupfalse%
\ {\isacharparenleft}{\kern0pt}typecheck{\isacharunderscore}{\kern0pt}cfuncs{\isacharcomma}{\kern0pt}\ smt\ comp{\isacharunderscore}{\kern0pt}associative{\isadigit{2}}{\isacharparenright}{\kern0pt}\isanewline
\ \ \ \ \isacommand{also}\isamarkupfalse%
\ \isacommand{have}\isamarkupfalse%
\ {\isachardoublequoteopen}{\isachardot}{\kern0pt}{\isachardot}{\kern0pt}{\isachardot}{\kern0pt}\ {\isacharequal}{\kern0pt}\ \ is{\isacharunderscore}{\kern0pt}odd\ {\isasymcirc}\isactrlsub c\ nth{\isacharunderscore}{\kern0pt}odd{\isachardoublequoteclose}\isanewline
\ \ \ \ \ \ \isacommand{using}\isamarkupfalse%
\ is{\isacharunderscore}{\kern0pt}even{\isacharunderscore}{\kern0pt}def{\isadigit{2}}\ is{\isacharunderscore}{\kern0pt}even{\isacharunderscore}{\kern0pt}not{\isacharunderscore}{\kern0pt}is{\isacharunderscore}{\kern0pt}odd\ is{\isacharunderscore}{\kern0pt}odd{\isacharunderscore}{\kern0pt}def{\isadigit{2}}\ is{\isacharunderscore}{\kern0pt}odd{\isacharunderscore}{\kern0pt}not{\isacharunderscore}{\kern0pt}is{\isacharunderscore}{\kern0pt}even\ \isacommand{by}\isamarkupfalse%
\ {\isacharparenleft}{\kern0pt}typecheck{\isacharunderscore}{\kern0pt}cfuncs{\isacharcomma}{\kern0pt}\ auto{\isacharparenright}{\kern0pt}\isanewline
\ \ \ \ \isacommand{also}\isamarkupfalse%
\ \isacommand{have}\isamarkupfalse%
\ {\isachardoublequoteopen}{\isachardot}{\kern0pt}{\isachardot}{\kern0pt}{\isachardot}{\kern0pt}\ {\isacharequal}{\kern0pt}\ id\ {\isasymOmega}\ {\isasymcirc}\isactrlsub c\ is{\isacharunderscore}{\kern0pt}odd\ {\isasymcirc}\isactrlsub c\ nth{\isacharunderscore}{\kern0pt}odd{\isachardoublequoteclose}\isanewline
\ \ \ \ \ \ \isacommand{by}\isamarkupfalse%
\ {\isacharparenleft}{\kern0pt}typecheck{\isacharunderscore}{\kern0pt}cfuncs{\isacharcomma}{\kern0pt}\ simp\ add{\isacharcolon}{\kern0pt}\ id{\isacharunderscore}{\kern0pt}left{\isacharunderscore}{\kern0pt}unit{\isadigit{2}}{\isacharparenright}{\kern0pt}\isanewline
\ \ \ \ \isacommand{then}\isamarkupfalse%
\ \isacommand{show}\isamarkupfalse%
\ {\isacharquery}{\kern0pt}thesis\isanewline
\ \ \ \ \ \ \isacommand{using}\isamarkupfalse%
\ calculation\ \isacommand{by}\isamarkupfalse%
\ auto\isanewline
\ \ \isacommand{qed}\isamarkupfalse%
\isanewline
\isanewline
\ \ \isacommand{show}\isamarkupfalse%
\ {\isachardoublequoteopen}{\isacharparenleft}{\kern0pt}{\isasymt}\ {\isasymcirc}\isactrlsub c\ {\isasymbeta}\isactrlbsub {\isasymnat}\isactrlsub c\isactrlesub {\isacharparenright}{\kern0pt}\ {\isasymcirc}\isactrlsub c\ successor\ {\isacharequal}{\kern0pt}\ id\isactrlsub c\ {\isasymOmega}\ {\isasymcirc}\isactrlsub c\ {\isasymt}\ {\isasymcirc}\isactrlsub c\ {\isasymbeta}\isactrlbsub {\isasymnat}\isactrlsub c\isactrlesub {\isachardoublequoteclose}\isanewline
\ \ \ \ \isacommand{by}\isamarkupfalse%
\ {\isacharparenleft}{\kern0pt}typecheck{\isacharunderscore}{\kern0pt}cfuncs{\isacharcomma}{\kern0pt}\ smt\ comp{\isacharunderscore}{\kern0pt}associative{\isadigit{2}}\ id{\isacharunderscore}{\kern0pt}left{\isacharunderscore}{\kern0pt}unit{\isadigit{2}}\ terminal{\isacharunderscore}{\kern0pt}func{\isacharunderscore}{\kern0pt}comp{\isacharparenright}{\kern0pt}\isanewline
\isacommand{qed}\isamarkupfalse%
%
\endisatagproof
{\isafoldproof}%
%
\isadelimproof
\isanewline
%
\endisadelimproof
\isanewline
\isacommand{lemma}\isamarkupfalse%
\ is{\isacharunderscore}{\kern0pt}odd{\isacharunderscore}{\kern0pt}nth{\isacharunderscore}{\kern0pt}even{\isacharunderscore}{\kern0pt}false{\isacharcolon}{\kern0pt}\isanewline
\ \ {\isachardoublequoteopen}is{\isacharunderscore}{\kern0pt}odd\ {\isasymcirc}\isactrlsub c\ nth{\isacharunderscore}{\kern0pt}even\ {\isacharequal}{\kern0pt}\ {\isasymf}\ {\isasymcirc}\isactrlsub c\ {\isasymbeta}\isactrlbsub {\isasymnat}\isactrlsub c\isactrlesub {\isachardoublequoteclose}\isanewline
%
\isadelimproof
\ \ %
\endisadelimproof
%
\isatagproof
\isacommand{by}\isamarkupfalse%
\ {\isacharparenleft}{\kern0pt}smt\ NOT{\isacharunderscore}{\kern0pt}true{\isacharunderscore}{\kern0pt}is{\isacharunderscore}{\kern0pt}false\ NOT{\isacharunderscore}{\kern0pt}type\ comp{\isacharunderscore}{\kern0pt}associative{\isadigit{2}}\ is{\isacharunderscore}{\kern0pt}even{\isacharunderscore}{\kern0pt}def{\isadigit{2}}\ is{\isacharunderscore}{\kern0pt}even{\isacharunderscore}{\kern0pt}nth{\isacharunderscore}{\kern0pt}even{\isacharunderscore}{\kern0pt}true\isanewline
\ \ \ \ \ \ is{\isacharunderscore}{\kern0pt}odd{\isacharunderscore}{\kern0pt}not{\isacharunderscore}{\kern0pt}is{\isacharunderscore}{\kern0pt}even\ nth{\isacharunderscore}{\kern0pt}even{\isacharunderscore}{\kern0pt}def{\isadigit{2}}\ terminal{\isacharunderscore}{\kern0pt}func{\isacharunderscore}{\kern0pt}type\ true{\isacharunderscore}{\kern0pt}func{\isacharunderscore}{\kern0pt}type{\isacharparenright}{\kern0pt}%
\endisatagproof
{\isafoldproof}%
%
\isadelimproof
\isanewline
%
\endisadelimproof
\isanewline
\isacommand{lemma}\isamarkupfalse%
\ is{\isacharunderscore}{\kern0pt}even{\isacharunderscore}{\kern0pt}nth{\isacharunderscore}{\kern0pt}odd{\isacharunderscore}{\kern0pt}false{\isacharcolon}{\kern0pt}\isanewline
\ \ {\isachardoublequoteopen}is{\isacharunderscore}{\kern0pt}even\ {\isasymcirc}\isactrlsub c\ nth{\isacharunderscore}{\kern0pt}odd\ {\isacharequal}{\kern0pt}\ {\isasymf}\ {\isasymcirc}\isactrlsub c\ {\isasymbeta}\isactrlbsub {\isasymnat}\isactrlsub c\isactrlesub {\isachardoublequoteclose}\isanewline
%
\isadelimproof
\ \ %
\endisadelimproof
%
\isatagproof
\isacommand{by}\isamarkupfalse%
\ {\isacharparenleft}{\kern0pt}smt\ NOT{\isacharunderscore}{\kern0pt}true{\isacharunderscore}{\kern0pt}is{\isacharunderscore}{\kern0pt}false\ NOT{\isacharunderscore}{\kern0pt}type\ comp{\isacharunderscore}{\kern0pt}associative{\isadigit{2}}\ is{\isacharunderscore}{\kern0pt}odd{\isacharunderscore}{\kern0pt}def{\isadigit{2}}\ is{\isacharunderscore}{\kern0pt}odd{\isacharunderscore}{\kern0pt}nth{\isacharunderscore}{\kern0pt}odd{\isacharunderscore}{\kern0pt}true\isanewline
\ \ \ \ \ \ is{\isacharunderscore}{\kern0pt}even{\isacharunderscore}{\kern0pt}not{\isacharunderscore}{\kern0pt}is{\isacharunderscore}{\kern0pt}odd\ nth{\isacharunderscore}{\kern0pt}odd{\isacharunderscore}{\kern0pt}def{\isadigit{2}}\ terminal{\isacharunderscore}{\kern0pt}func{\isacharunderscore}{\kern0pt}type\ true{\isacharunderscore}{\kern0pt}func{\isacharunderscore}{\kern0pt}type{\isacharparenright}{\kern0pt}%
\endisatagproof
{\isafoldproof}%
%
\isadelimproof
\isanewline
%
\endisadelimproof
\isanewline
\isacommand{lemma}\isamarkupfalse%
\ EXISTS{\isacharunderscore}{\kern0pt}zero{\isacharunderscore}{\kern0pt}nth{\isacharunderscore}{\kern0pt}even{\isacharcolon}{\kern0pt}\isanewline
\ \ {\isachardoublequoteopen}{\isacharparenleft}{\kern0pt}EXISTS\ {\isasymnat}\isactrlsub c\ {\isasymcirc}\isactrlsub c\ {\isacharparenleft}{\kern0pt}eq{\isacharunderscore}{\kern0pt}pred\ {\isasymnat}\isactrlsub c\ {\isasymcirc}\isactrlsub c\ nth{\isacharunderscore}{\kern0pt}even\ {\isasymtimes}\isactrlsub f\ id\isactrlsub c\ {\isasymnat}\isactrlsub c{\isacharparenright}{\kern0pt}\isactrlsup {\isasymsharp}{\isacharparenright}{\kern0pt}\ {\isasymcirc}\isactrlsub c\ zero\ {\isacharequal}{\kern0pt}\ {\isasymt}{\isachardoublequoteclose}\isanewline
%
\isadelimproof
%
\endisadelimproof
%
\isatagproof
\isacommand{proof}\isamarkupfalse%
\ {\isacharminus}{\kern0pt}\isanewline
\ \ \isacommand{have}\isamarkupfalse%
\ \ {\isachardoublequoteopen}{\isacharparenleft}{\kern0pt}EXISTS\ {\isasymnat}\isactrlsub c\ {\isasymcirc}\isactrlsub c\ {\isacharparenleft}{\kern0pt}eq{\isacharunderscore}{\kern0pt}pred\ {\isasymnat}\isactrlsub c\ {\isasymcirc}\isactrlsub c\ nth{\isacharunderscore}{\kern0pt}even\ {\isasymtimes}\isactrlsub f\ id\isactrlsub c\ {\isasymnat}\isactrlsub c{\isacharparenright}{\kern0pt}\isactrlsup {\isasymsharp}{\isacharparenright}{\kern0pt}\ {\isasymcirc}\isactrlsub c\ zero\isanewline
\ \ \ \ \ \ {\isacharequal}{\kern0pt}\ EXISTS\ {\isasymnat}\isactrlsub c\ {\isasymcirc}\isactrlsub c\ {\isacharparenleft}{\kern0pt}eq{\isacharunderscore}{\kern0pt}pred\ {\isasymnat}\isactrlsub c\ {\isasymcirc}\isactrlsub c\ nth{\isacharunderscore}{\kern0pt}even\ {\isasymtimes}\isactrlsub f\ id\isactrlsub c\ {\isasymnat}\isactrlsub c{\isacharparenright}{\kern0pt}\isactrlsup {\isasymsharp}\ {\isasymcirc}\isactrlsub c\ zero{\isachardoublequoteclose}\isanewline
\ \ \ \ \isacommand{by}\isamarkupfalse%
\ {\isacharparenleft}{\kern0pt}typecheck{\isacharunderscore}{\kern0pt}cfuncs{\isacharcomma}{\kern0pt}\ simp\ add{\isacharcolon}{\kern0pt}\ comp{\isacharunderscore}{\kern0pt}associative{\isadigit{2}}{\isacharparenright}{\kern0pt}\isanewline
\ \ \isacommand{also}\isamarkupfalse%
\ \isacommand{have}\isamarkupfalse%
\ {\isachardoublequoteopen}{\isachardot}{\kern0pt}{\isachardot}{\kern0pt}{\isachardot}{\kern0pt}\ {\isacharequal}{\kern0pt}\ EXISTS\ {\isasymnat}\isactrlsub c\ {\isasymcirc}\isactrlsub c\ {\isacharparenleft}{\kern0pt}eq{\isacharunderscore}{\kern0pt}pred\ {\isasymnat}\isactrlsub c\ {\isasymcirc}\isactrlsub c\ {\isacharparenleft}{\kern0pt}nth{\isacharunderscore}{\kern0pt}even\ {\isasymtimes}\isactrlsub f\ id\isactrlsub c\ {\isasymnat}\isactrlsub c{\isacharparenright}{\kern0pt}\ {\isasymcirc}\isactrlsub c\ {\isacharparenleft}{\kern0pt}id\isactrlsub c\ {\isasymnat}\isactrlsub c\ {\isasymtimes}\isactrlsub f\ zero{\isacharparenright}{\kern0pt}{\isacharparenright}{\kern0pt}\isactrlsup {\isasymsharp}{\isachardoublequoteclose}\isanewline
\ \ \ \ \isacommand{by}\isamarkupfalse%
\ {\isacharparenleft}{\kern0pt}typecheck{\isacharunderscore}{\kern0pt}cfuncs{\isacharcomma}{\kern0pt}\ simp\ add{\isacharcolon}{\kern0pt}\ comp{\isacharunderscore}{\kern0pt}associative{\isadigit{2}}\ sharp{\isacharunderscore}{\kern0pt}comp{\isacharparenright}{\kern0pt}\isanewline
\ \ \isacommand{also}\isamarkupfalse%
\ \isacommand{have}\isamarkupfalse%
\ {\isachardoublequoteopen}{\isachardot}{\kern0pt}{\isachardot}{\kern0pt}{\isachardot}{\kern0pt}\ {\isacharequal}{\kern0pt}\ EXISTS\ {\isasymnat}\isactrlsub c\ {\isasymcirc}\isactrlsub c\ {\isacharparenleft}{\kern0pt}eq{\isacharunderscore}{\kern0pt}pred\ {\isasymnat}\isactrlsub c\ {\isasymcirc}\isactrlsub c\ {\isacharparenleft}{\kern0pt}nth{\isacharunderscore}{\kern0pt}even\ {\isasymtimes}\isactrlsub f\ zero{\isacharparenright}{\kern0pt}{\isacharparenright}{\kern0pt}\isactrlsup {\isasymsharp}{\isachardoublequoteclose}\isanewline
\ \ \ \ \isacommand{by}\isamarkupfalse%
\ {\isacharparenleft}{\kern0pt}typecheck{\isacharunderscore}{\kern0pt}cfuncs{\isacharcomma}{\kern0pt}\ simp\ add{\isacharcolon}{\kern0pt}\ cfunc{\isacharunderscore}{\kern0pt}cross{\isacharunderscore}{\kern0pt}prod{\isacharunderscore}{\kern0pt}comp{\isacharunderscore}{\kern0pt}cfunc{\isacharunderscore}{\kern0pt}cross{\isacharunderscore}{\kern0pt}prod\ id{\isacharunderscore}{\kern0pt}left{\isacharunderscore}{\kern0pt}unit{\isadigit{2}}\ id{\isacharunderscore}{\kern0pt}right{\isacharunderscore}{\kern0pt}unit{\isadigit{2}}{\isacharparenright}{\kern0pt}\isanewline
\ \ \isacommand{also}\isamarkupfalse%
\ \isacommand{have}\isamarkupfalse%
\ {\isachardoublequoteopen}{\isachardot}{\kern0pt}{\isachardot}{\kern0pt}{\isachardot}{\kern0pt}\ {\isacharequal}{\kern0pt}\ EXISTS\ {\isasymnat}\isactrlsub c\ {\isasymcirc}\isactrlsub c\ {\isacharparenleft}{\kern0pt}eq{\isacharunderscore}{\kern0pt}pred\ {\isasymnat}\isactrlsub c\ {\isasymcirc}\isactrlsub c\ {\isasymlangle}nth{\isacharunderscore}{\kern0pt}even\ {\isasymcirc}\isactrlsub c\ left{\isacharunderscore}{\kern0pt}cart{\isacharunderscore}{\kern0pt}proj\ {\isasymnat}\isactrlsub c\ {\isasymone}{\isacharcomma}{\kern0pt}\ zero\ {\isasymcirc}\isactrlsub c\ {\isasymbeta}\isactrlbsub {\isasymnat}\isactrlsub c\ {\isasymtimes}\isactrlsub c\ {\isasymone}\isactrlesub {\isasymrangle}\ {\isacharparenright}{\kern0pt}\isactrlsup {\isasymsharp}{\isachardoublequoteclose}\isanewline
\ \ \ \ \isacommand{by}\isamarkupfalse%
\ {\isacharparenleft}{\kern0pt}typecheck{\isacharunderscore}{\kern0pt}cfuncs{\isacharcomma}{\kern0pt}\ metis\ cfunc{\isacharunderscore}{\kern0pt}cross{\isacharunderscore}{\kern0pt}prod{\isacharunderscore}{\kern0pt}def\ cfunc{\isacharunderscore}{\kern0pt}type{\isacharunderscore}{\kern0pt}def\ right{\isacharunderscore}{\kern0pt}cart{\isacharunderscore}{\kern0pt}proj{\isacharunderscore}{\kern0pt}type\ terminal{\isacharunderscore}{\kern0pt}func{\isacharunderscore}{\kern0pt}unique{\isacharparenright}{\kern0pt}\isanewline
\ \ \isacommand{also}\isamarkupfalse%
\ \isacommand{have}\isamarkupfalse%
\ {\isachardoublequoteopen}{\isachardot}{\kern0pt}{\isachardot}{\kern0pt}{\isachardot}{\kern0pt}\ {\isacharequal}{\kern0pt}\ EXISTS\ {\isasymnat}\isactrlsub c\ {\isasymcirc}\isactrlsub c\ {\isacharparenleft}{\kern0pt}eq{\isacharunderscore}{\kern0pt}pred\ {\isasymnat}\isactrlsub c\ {\isasymcirc}\isactrlsub c\ {\isasymlangle}nth{\isacharunderscore}{\kern0pt}even\ {\isasymcirc}\isactrlsub c\ left{\isacharunderscore}{\kern0pt}cart{\isacharunderscore}{\kern0pt}proj\ {\isasymnat}\isactrlsub c\ {\isasymone}{\isacharcomma}{\kern0pt}\ {\isacharparenleft}{\kern0pt}zero\ {\isasymcirc}\isactrlsub c\ {\isasymbeta}\isactrlbsub {\isasymnat}\isactrlsub c\isactrlesub {\isacharparenright}{\kern0pt}\ {\isasymcirc}\isactrlsub c\ left{\isacharunderscore}{\kern0pt}cart{\isacharunderscore}{\kern0pt}proj\ {\isasymnat}\isactrlsub c\ {\isasymone}{\isasymrangle}\ {\isacharparenright}{\kern0pt}\isactrlsup {\isasymsharp}{\isachardoublequoteclose}\isanewline
\ \ \ \ \isacommand{by}\isamarkupfalse%
\ {\isacharparenleft}{\kern0pt}typecheck{\isacharunderscore}{\kern0pt}cfuncs{\isacharcomma}{\kern0pt}\ smt\ comp{\isacharunderscore}{\kern0pt}associative{\isadigit{2}}\ terminal{\isacharunderscore}{\kern0pt}func{\isacharunderscore}{\kern0pt}comp{\isacharparenright}{\kern0pt}\isanewline
\ \ \isacommand{also}\isamarkupfalse%
\ \isacommand{have}\isamarkupfalse%
\ {\isachardoublequoteopen}{\isachardot}{\kern0pt}{\isachardot}{\kern0pt}{\isachardot}{\kern0pt}\ {\isacharequal}{\kern0pt}\ EXISTS\ {\isasymnat}\isactrlsub c\ {\isasymcirc}\isactrlsub c\ {\isacharparenleft}{\kern0pt}{\isacharparenleft}{\kern0pt}eq{\isacharunderscore}{\kern0pt}pred\ {\isasymnat}\isactrlsub c\ {\isasymcirc}\isactrlsub c\ {\isasymlangle}nth{\isacharunderscore}{\kern0pt}even{\isacharcomma}{\kern0pt}\ zero\ {\isasymcirc}\isactrlsub c\ {\isasymbeta}\isactrlbsub {\isasymnat}\isactrlsub c\isactrlesub {\isasymrangle}{\isacharparenright}{\kern0pt}\ {\isasymcirc}\isactrlsub c\ left{\isacharunderscore}{\kern0pt}cart{\isacharunderscore}{\kern0pt}proj\ {\isasymnat}\isactrlsub c\ {\isasymone}{\isacharparenright}{\kern0pt}\isactrlsup {\isasymsharp}{\isachardoublequoteclose}\isanewline
\ \ \ \ \isacommand{by}\isamarkupfalse%
\ {\isacharparenleft}{\kern0pt}typecheck{\isacharunderscore}{\kern0pt}cfuncs{\isacharcomma}{\kern0pt}\ smt\ cfunc{\isacharunderscore}{\kern0pt}prod{\isacharunderscore}{\kern0pt}comp\ comp{\isacharunderscore}{\kern0pt}associative{\isadigit{2}}{\isacharparenright}{\kern0pt}\isanewline
\ \ \isacommand{also}\isamarkupfalse%
\ \isacommand{have}\isamarkupfalse%
\ {\isachardoublequoteopen}{\isachardot}{\kern0pt}{\isachardot}{\kern0pt}{\isachardot}{\kern0pt}\ {\isacharequal}{\kern0pt}\ {\isasymt}{\isachardoublequoteclose}\isanewline
\ \ \isacommand{proof}\isamarkupfalse%
\ {\isacharparenleft}{\kern0pt}rule\ exists{\isacharunderscore}{\kern0pt}true{\isacharunderscore}{\kern0pt}implies{\isacharunderscore}{\kern0pt}EXISTS{\isacharunderscore}{\kern0pt}true{\isacharparenright}{\kern0pt}\isanewline
\ \ \ \ \isacommand{show}\isamarkupfalse%
\ {\isachardoublequoteopen}eq{\isacharunderscore}{\kern0pt}pred\ {\isasymnat}\isactrlsub c\ {\isasymcirc}\isactrlsub c\ {\isasymlangle}nth{\isacharunderscore}{\kern0pt}even{\isacharcomma}{\kern0pt}zero\ {\isasymcirc}\isactrlsub c\ {\isasymbeta}\isactrlbsub {\isasymnat}\isactrlsub c\isactrlesub {\isasymrangle}\ {\isacharcolon}{\kern0pt}\ {\isasymnat}\isactrlsub c\ {\isasymrightarrow}\ {\isasymOmega}{\isachardoublequoteclose}\isanewline
\ \ \ \ \ \ \isacommand{by}\isamarkupfalse%
\ typecheck{\isacharunderscore}{\kern0pt}cfuncs\isanewline
\ \ \ \ \isacommand{show}\isamarkupfalse%
\ {\isachardoublequoteopen}{\isasymexists}x{\isachardot}{\kern0pt}\ x\ {\isasymin}\isactrlsub c\ {\isasymnat}\isactrlsub c\ {\isasymand}\ {\isacharparenleft}{\kern0pt}eq{\isacharunderscore}{\kern0pt}pred\ {\isasymnat}\isactrlsub c\ {\isasymcirc}\isactrlsub c\ {\isasymlangle}nth{\isacharunderscore}{\kern0pt}even{\isacharcomma}{\kern0pt}zero\ {\isasymcirc}\isactrlsub c\ {\isasymbeta}\isactrlbsub {\isasymnat}\isactrlsub c\isactrlesub {\isasymrangle}{\isacharparenright}{\kern0pt}\ {\isasymcirc}\isactrlsub c\ x\ {\isacharequal}{\kern0pt}\ {\isasymt}{\isachardoublequoteclose}\isanewline
\ \ \ \ \isacommand{proof}\isamarkupfalse%
\ {\isacharparenleft}{\kern0pt}typecheck{\isacharunderscore}{\kern0pt}cfuncs{\isacharcomma}{\kern0pt}\ intro\ exI{\isacharbrackleft}{\kern0pt}\isakeyword{where}\ x{\isacharequal}{\kern0pt}{\isachardoublequoteopen}zero{\isachardoublequoteclose}{\isacharbrackright}{\kern0pt}{\isacharcomma}{\kern0pt}\ clarify{\isacharparenright}{\kern0pt}\isanewline
\ \ \ \ \ \ \isacommand{have}\isamarkupfalse%
\ {\isachardoublequoteopen}{\isacharparenleft}{\kern0pt}eq{\isacharunderscore}{\kern0pt}pred\ {\isasymnat}\isactrlsub c\ {\isasymcirc}\isactrlsub c\ {\isasymlangle}nth{\isacharunderscore}{\kern0pt}even{\isacharcomma}{\kern0pt}zero\ {\isasymcirc}\isactrlsub c\ {\isasymbeta}\isactrlbsub {\isasymnat}\isactrlsub c\isactrlesub {\isasymrangle}{\isacharparenright}{\kern0pt}\ {\isasymcirc}\isactrlsub c\ zero\isanewline
\ \ \ \ \ \ \ \ {\isacharequal}{\kern0pt}\ eq{\isacharunderscore}{\kern0pt}pred\ {\isasymnat}\isactrlsub c\ {\isasymcirc}\isactrlsub c\ {\isasymlangle}nth{\isacharunderscore}{\kern0pt}even{\isacharcomma}{\kern0pt}zero\ {\isasymcirc}\isactrlsub c\ {\isasymbeta}\isactrlbsub {\isasymnat}\isactrlsub c\isactrlesub {\isasymrangle}\ {\isasymcirc}\isactrlsub c\ zero{\isachardoublequoteclose}\isanewline
\ \ \ \ \ \ \ \ \isacommand{by}\isamarkupfalse%
\ {\isacharparenleft}{\kern0pt}typecheck{\isacharunderscore}{\kern0pt}cfuncs{\isacharcomma}{\kern0pt}\ simp\ add{\isacharcolon}{\kern0pt}\ comp{\isacharunderscore}{\kern0pt}associative{\isadigit{2}}{\isacharparenright}{\kern0pt}\isanewline
\ \ \ \ \ \ \isacommand{also}\isamarkupfalse%
\ \isacommand{have}\isamarkupfalse%
\ {\isachardoublequoteopen}{\isachardot}{\kern0pt}{\isachardot}{\kern0pt}{\isachardot}{\kern0pt}\ {\isacharequal}{\kern0pt}\ eq{\isacharunderscore}{\kern0pt}pred\ {\isasymnat}\isactrlsub c\ {\isasymcirc}\isactrlsub c\ {\isasymlangle}nth{\isacharunderscore}{\kern0pt}even\ {\isasymcirc}\isactrlsub c\ zero{\isacharcomma}{\kern0pt}\ zero{\isasymrangle}{\isachardoublequoteclose}\isanewline
\ \ \ \ \ \ \ \ \isacommand{by}\isamarkupfalse%
\ {\isacharparenleft}{\kern0pt}typecheck{\isacharunderscore}{\kern0pt}cfuncs{\isacharcomma}{\kern0pt}\ smt\ {\isacharparenleft}{\kern0pt}z{\isadigit{3}}{\isacharparenright}{\kern0pt}\ cfunc{\isacharunderscore}{\kern0pt}prod{\isacharunderscore}{\kern0pt}comp\ comp{\isacharunderscore}{\kern0pt}associative{\isadigit{2}}\ id{\isacharunderscore}{\kern0pt}right{\isacharunderscore}{\kern0pt}unit{\isadigit{2}}\ terminal{\isacharunderscore}{\kern0pt}func{\isacharunderscore}{\kern0pt}comp{\isacharunderscore}{\kern0pt}elem{\isacharparenright}{\kern0pt}\isanewline
\ \ \ \ \ \ \isacommand{also}\isamarkupfalse%
\ \isacommand{have}\isamarkupfalse%
\ {\isachardoublequoteopen}{\isachardot}{\kern0pt}{\isachardot}{\kern0pt}{\isachardot}{\kern0pt}\ {\isacharequal}{\kern0pt}\ {\isasymt}{\isachardoublequoteclose}\isanewline
\ \ \ \ \ \ \ \ \isacommand{using}\isamarkupfalse%
\ eq{\isacharunderscore}{\kern0pt}pred{\isacharunderscore}{\kern0pt}iff{\isacharunderscore}{\kern0pt}eq\ nth{\isacharunderscore}{\kern0pt}even{\isacharunderscore}{\kern0pt}zero\ \isacommand{by}\isamarkupfalse%
\ {\isacharparenleft}{\kern0pt}typecheck{\isacharunderscore}{\kern0pt}cfuncs{\isacharcomma}{\kern0pt}\ blast{\isacharparenright}{\kern0pt}\isanewline
\ \ \ \ \ \ \isacommand{then}\isamarkupfalse%
\ \isacommand{show}\isamarkupfalse%
\ {\isachardoublequoteopen}{\isacharparenleft}{\kern0pt}eq{\isacharunderscore}{\kern0pt}pred\ {\isasymnat}\isactrlsub c\ {\isasymcirc}\isactrlsub c\ {\isasymlangle}nth{\isacharunderscore}{\kern0pt}even{\isacharcomma}{\kern0pt}zero\ {\isasymcirc}\isactrlsub c\ {\isasymbeta}\isactrlbsub {\isasymnat}\isactrlsub c\isactrlesub {\isasymrangle}{\isacharparenright}{\kern0pt}\ {\isasymcirc}\isactrlsub c\ zero\ {\isacharequal}{\kern0pt}\ {\isasymt}{\isachardoublequoteclose}\isanewline
\ \ \ \ \ \ \ \ \isacommand{using}\isamarkupfalse%
\ calculation\ \isacommand{by}\isamarkupfalse%
\ auto\isanewline
\ \ \ \ \isacommand{qed}\isamarkupfalse%
\isanewline
\ \ \isacommand{qed}\isamarkupfalse%
\isanewline
\ \ \isacommand{then}\isamarkupfalse%
\ \isacommand{show}\isamarkupfalse%
\ {\isacharquery}{\kern0pt}thesis\isanewline
\ \ \ \ \isacommand{using}\isamarkupfalse%
\ calculation\ \isacommand{by}\isamarkupfalse%
\ auto\isanewline
\isacommand{qed}\isamarkupfalse%
%
\endisatagproof
{\isafoldproof}%
%
\isadelimproof
\isanewline
%
\endisadelimproof
\isanewline
\isacommand{lemma}\isamarkupfalse%
\ not{\isacharunderscore}{\kern0pt}EXISTS{\isacharunderscore}{\kern0pt}zero{\isacharunderscore}{\kern0pt}nth{\isacharunderscore}{\kern0pt}odd{\isacharcolon}{\kern0pt}\isanewline
\ \ {\isachardoublequoteopen}{\isacharparenleft}{\kern0pt}EXISTS\ {\isasymnat}\isactrlsub c\ {\isasymcirc}\isactrlsub c\ {\isacharparenleft}{\kern0pt}eq{\isacharunderscore}{\kern0pt}pred\ {\isasymnat}\isactrlsub c\ {\isasymcirc}\isactrlsub c\ nth{\isacharunderscore}{\kern0pt}odd\ {\isasymtimes}\isactrlsub f\ id\isactrlsub c\ {\isasymnat}\isactrlsub c{\isacharparenright}{\kern0pt}\isactrlsup {\isasymsharp}{\isacharparenright}{\kern0pt}\ {\isasymcirc}\isactrlsub c\ zero\ {\isacharequal}{\kern0pt}\ {\isasymf}{\isachardoublequoteclose}\isanewline
%
\isadelimproof
%
\endisadelimproof
%
\isatagproof
\isacommand{proof}\isamarkupfalse%
\ {\isacharminus}{\kern0pt}\isanewline
\ \ \isacommand{have}\isamarkupfalse%
\ \ {\isachardoublequoteopen}{\isacharparenleft}{\kern0pt}EXISTS\ {\isasymnat}\isactrlsub c\ {\isasymcirc}\isactrlsub c\ {\isacharparenleft}{\kern0pt}eq{\isacharunderscore}{\kern0pt}pred\ {\isasymnat}\isactrlsub c\ {\isasymcirc}\isactrlsub c\ nth{\isacharunderscore}{\kern0pt}odd\ {\isasymtimes}\isactrlsub f\ id\isactrlsub c\ {\isasymnat}\isactrlsub c{\isacharparenright}{\kern0pt}\isactrlsup {\isasymsharp}{\isacharparenright}{\kern0pt}\ {\isasymcirc}\isactrlsub c\ zero\ {\isacharequal}{\kern0pt}\ EXISTS\ {\isasymnat}\isactrlsub c\ {\isasymcirc}\isactrlsub c\ {\isacharparenleft}{\kern0pt}eq{\isacharunderscore}{\kern0pt}pred\ {\isasymnat}\isactrlsub c\ {\isasymcirc}\isactrlsub c\ nth{\isacharunderscore}{\kern0pt}odd\ {\isasymtimes}\isactrlsub f\ id\isactrlsub c\ {\isasymnat}\isactrlsub c{\isacharparenright}{\kern0pt}\isactrlsup {\isasymsharp}\ {\isasymcirc}\isactrlsub c\ zero{\isachardoublequoteclose}\isanewline
\ \ \ \ \isacommand{by}\isamarkupfalse%
\ {\isacharparenleft}{\kern0pt}typecheck{\isacharunderscore}{\kern0pt}cfuncs{\isacharcomma}{\kern0pt}\ simp\ add{\isacharcolon}{\kern0pt}\ comp{\isacharunderscore}{\kern0pt}associative{\isadigit{2}}{\isacharparenright}{\kern0pt}\isanewline
\ \ \isacommand{also}\isamarkupfalse%
\ \isacommand{have}\isamarkupfalse%
\ {\isachardoublequoteopen}{\isachardot}{\kern0pt}{\isachardot}{\kern0pt}{\isachardot}{\kern0pt}\ {\isacharequal}{\kern0pt}\ EXISTS\ {\isasymnat}\isactrlsub c\ {\isasymcirc}\isactrlsub c\ {\isacharparenleft}{\kern0pt}eq{\isacharunderscore}{\kern0pt}pred\ {\isasymnat}\isactrlsub c\ {\isasymcirc}\isactrlsub c\ {\isacharparenleft}{\kern0pt}nth{\isacharunderscore}{\kern0pt}odd\ {\isasymtimes}\isactrlsub f\ id\isactrlsub c\ {\isasymnat}\isactrlsub c{\isacharparenright}{\kern0pt}\ {\isasymcirc}\isactrlsub c\ {\isacharparenleft}{\kern0pt}id\isactrlsub c\ {\isasymnat}\isactrlsub c\ {\isasymtimes}\isactrlsub f\ zero{\isacharparenright}{\kern0pt}{\isacharparenright}{\kern0pt}\isactrlsup {\isasymsharp}{\isachardoublequoteclose}\isanewline
\ \ \ \ \isacommand{by}\isamarkupfalse%
\ {\isacharparenleft}{\kern0pt}typecheck{\isacharunderscore}{\kern0pt}cfuncs{\isacharcomma}{\kern0pt}\ simp\ add{\isacharcolon}{\kern0pt}\ comp{\isacharunderscore}{\kern0pt}associative{\isadigit{2}}\ sharp{\isacharunderscore}{\kern0pt}comp{\isacharparenright}{\kern0pt}\isanewline
\ \ \isacommand{also}\isamarkupfalse%
\ \isacommand{have}\isamarkupfalse%
\ {\isachardoublequoteopen}{\isachardot}{\kern0pt}{\isachardot}{\kern0pt}{\isachardot}{\kern0pt}\ {\isacharequal}{\kern0pt}\ EXISTS\ {\isasymnat}\isactrlsub c\ {\isasymcirc}\isactrlsub c\ {\isacharparenleft}{\kern0pt}eq{\isacharunderscore}{\kern0pt}pred\ {\isasymnat}\isactrlsub c\ {\isasymcirc}\isactrlsub c\ {\isacharparenleft}{\kern0pt}nth{\isacharunderscore}{\kern0pt}odd\ {\isasymtimes}\isactrlsub f\ zero{\isacharparenright}{\kern0pt}{\isacharparenright}{\kern0pt}\isactrlsup {\isasymsharp}{\isachardoublequoteclose}\isanewline
\ \ \ \ \isacommand{by}\isamarkupfalse%
\ {\isacharparenleft}{\kern0pt}typecheck{\isacharunderscore}{\kern0pt}cfuncs{\isacharcomma}{\kern0pt}\ simp\ add{\isacharcolon}{\kern0pt}\ cfunc{\isacharunderscore}{\kern0pt}cross{\isacharunderscore}{\kern0pt}prod{\isacharunderscore}{\kern0pt}comp{\isacharunderscore}{\kern0pt}cfunc{\isacharunderscore}{\kern0pt}cross{\isacharunderscore}{\kern0pt}prod\ id{\isacharunderscore}{\kern0pt}left{\isacharunderscore}{\kern0pt}unit{\isadigit{2}}\ id{\isacharunderscore}{\kern0pt}right{\isacharunderscore}{\kern0pt}unit{\isadigit{2}}{\isacharparenright}{\kern0pt}\isanewline
\ \ \isacommand{also}\isamarkupfalse%
\ \isacommand{have}\isamarkupfalse%
\ {\isachardoublequoteopen}{\isachardot}{\kern0pt}{\isachardot}{\kern0pt}{\isachardot}{\kern0pt}\ {\isacharequal}{\kern0pt}\ EXISTS\ {\isasymnat}\isactrlsub c\ {\isasymcirc}\isactrlsub c\ {\isacharparenleft}{\kern0pt}eq{\isacharunderscore}{\kern0pt}pred\ {\isasymnat}\isactrlsub c\ {\isasymcirc}\isactrlsub c\ {\isasymlangle}nth{\isacharunderscore}{\kern0pt}odd\ {\isasymcirc}\isactrlsub c\ left{\isacharunderscore}{\kern0pt}cart{\isacharunderscore}{\kern0pt}proj\ {\isasymnat}\isactrlsub c\ {\isasymone}{\isacharcomma}{\kern0pt}\ zero\ {\isasymcirc}\isactrlsub c\ {\isasymbeta}\isactrlbsub {\isasymnat}\isactrlsub c\ {\isasymtimes}\isactrlsub c\ {\isasymone}\isactrlesub {\isasymrangle}\ {\isacharparenright}{\kern0pt}\isactrlsup {\isasymsharp}{\isachardoublequoteclose}\isanewline
\ \ \ \ \isacommand{by}\isamarkupfalse%
\ {\isacharparenleft}{\kern0pt}typecheck{\isacharunderscore}{\kern0pt}cfuncs{\isacharcomma}{\kern0pt}\ metis\ cfunc{\isacharunderscore}{\kern0pt}cross{\isacharunderscore}{\kern0pt}prod{\isacharunderscore}{\kern0pt}def\ cfunc{\isacharunderscore}{\kern0pt}type{\isacharunderscore}{\kern0pt}def\ right{\isacharunderscore}{\kern0pt}cart{\isacharunderscore}{\kern0pt}proj{\isacharunderscore}{\kern0pt}type\ terminal{\isacharunderscore}{\kern0pt}func{\isacharunderscore}{\kern0pt}unique{\isacharparenright}{\kern0pt}\isanewline
\ \ \isacommand{also}\isamarkupfalse%
\ \isacommand{have}\isamarkupfalse%
\ {\isachardoublequoteopen}{\isachardot}{\kern0pt}{\isachardot}{\kern0pt}{\isachardot}{\kern0pt}\ {\isacharequal}{\kern0pt}\ EXISTS\ {\isasymnat}\isactrlsub c\ {\isasymcirc}\isactrlsub c\ {\isacharparenleft}{\kern0pt}eq{\isacharunderscore}{\kern0pt}pred\ {\isasymnat}\isactrlsub c\ {\isasymcirc}\isactrlsub c\ {\isasymlangle}nth{\isacharunderscore}{\kern0pt}odd\ {\isasymcirc}\isactrlsub c\ left{\isacharunderscore}{\kern0pt}cart{\isacharunderscore}{\kern0pt}proj\ {\isasymnat}\isactrlsub c\ {\isasymone}{\isacharcomma}{\kern0pt}\ {\isacharparenleft}{\kern0pt}zero\ {\isasymcirc}\isactrlsub c\ {\isasymbeta}\isactrlbsub {\isasymnat}\isactrlsub c\isactrlesub {\isacharparenright}{\kern0pt}\ {\isasymcirc}\isactrlsub c\ left{\isacharunderscore}{\kern0pt}cart{\isacharunderscore}{\kern0pt}proj\ {\isasymnat}\isactrlsub c\ {\isasymone}{\isasymrangle}\ {\isacharparenright}{\kern0pt}\isactrlsup {\isasymsharp}{\isachardoublequoteclose}\isanewline
\ \ \ \ \isacommand{by}\isamarkupfalse%
\ {\isacharparenleft}{\kern0pt}typecheck{\isacharunderscore}{\kern0pt}cfuncs{\isacharcomma}{\kern0pt}\ smt\ comp{\isacharunderscore}{\kern0pt}associative{\isadigit{2}}\ terminal{\isacharunderscore}{\kern0pt}func{\isacharunderscore}{\kern0pt}comp{\isacharparenright}{\kern0pt}\isanewline
\ \ \isacommand{also}\isamarkupfalse%
\ \isacommand{have}\isamarkupfalse%
\ {\isachardoublequoteopen}{\isachardot}{\kern0pt}{\isachardot}{\kern0pt}{\isachardot}{\kern0pt}\ {\isacharequal}{\kern0pt}\ EXISTS\ {\isasymnat}\isactrlsub c\ {\isasymcirc}\isactrlsub c\ {\isacharparenleft}{\kern0pt}{\isacharparenleft}{\kern0pt}eq{\isacharunderscore}{\kern0pt}pred\ {\isasymnat}\isactrlsub c\ {\isasymcirc}\isactrlsub c\ {\isasymlangle}nth{\isacharunderscore}{\kern0pt}odd{\isacharcomma}{\kern0pt}\ zero\ {\isasymcirc}\isactrlsub c\ {\isasymbeta}\isactrlbsub {\isasymnat}\isactrlsub c\isactrlesub {\isasymrangle}{\isacharparenright}{\kern0pt}\ {\isasymcirc}\isactrlsub c\ left{\isacharunderscore}{\kern0pt}cart{\isacharunderscore}{\kern0pt}proj\ {\isasymnat}\isactrlsub c\ {\isasymone}{\isacharparenright}{\kern0pt}\isactrlsup {\isasymsharp}{\isachardoublequoteclose}\isanewline
\ \ \ \ \isacommand{by}\isamarkupfalse%
\ {\isacharparenleft}{\kern0pt}typecheck{\isacharunderscore}{\kern0pt}cfuncs{\isacharcomma}{\kern0pt}\ smt\ cfunc{\isacharunderscore}{\kern0pt}prod{\isacharunderscore}{\kern0pt}comp\ comp{\isacharunderscore}{\kern0pt}associative{\isadigit{2}}{\isacharparenright}{\kern0pt}\isanewline
\ \ \isacommand{also}\isamarkupfalse%
\ \isacommand{have}\isamarkupfalse%
\ {\isachardoublequoteopen}{\isachardot}{\kern0pt}{\isachardot}{\kern0pt}{\isachardot}{\kern0pt}\ {\isacharequal}{\kern0pt}\ {\isasymf}{\isachardoublequoteclose}\isanewline
\ \ \isacommand{proof}\isamarkupfalse%
\ {\isacharminus}{\kern0pt}\isanewline
\ \ \ \ \isacommand{have}\isamarkupfalse%
\ {\isachardoublequoteopen}{\isasymnexists}\ x{\isachardot}{\kern0pt}\ x\ {\isasymin}\isactrlsub c\ {\isasymnat}\isactrlsub c\ {\isasymand}\ {\isacharparenleft}{\kern0pt}eq{\isacharunderscore}{\kern0pt}pred\ {\isasymnat}\isactrlsub c\ {\isasymcirc}\isactrlsub c\ {\isasymlangle}nth{\isacharunderscore}{\kern0pt}odd{\isacharcomma}{\kern0pt}\ zero\ {\isasymcirc}\isactrlsub c\ {\isasymbeta}\isactrlbsub {\isasymnat}\isactrlsub c\isactrlesub {\isasymrangle}{\isacharparenright}{\kern0pt}\ {\isasymcirc}\isactrlsub c\ x\ {\isacharequal}{\kern0pt}\ {\isasymt}{\isachardoublequoteclose}\isanewline
\ \ \ \ \isacommand{proof}\isamarkupfalse%
\ clarify\isanewline
\ \ \ \ \ \ \isacommand{fix}\isamarkupfalse%
\ x\isanewline
\ \ \ \ \ \ \isacommand{assume}\isamarkupfalse%
\ x{\isacharunderscore}{\kern0pt}type{\isacharbrackleft}{\kern0pt}type{\isacharunderscore}{\kern0pt}rule{\isacharbrackright}{\kern0pt}{\isacharcolon}{\kern0pt}\ {\isachardoublequoteopen}x\ {\isasymin}\isactrlsub c\ {\isasymnat}\isactrlsub c{\isachardoublequoteclose}\isanewline
\ \ \ \ \ \ \isacommand{assume}\isamarkupfalse%
\ {\isachardoublequoteopen}{\isacharparenleft}{\kern0pt}eq{\isacharunderscore}{\kern0pt}pred\ {\isasymnat}\isactrlsub c\ {\isasymcirc}\isactrlsub c\ {\isasymlangle}nth{\isacharunderscore}{\kern0pt}odd{\isacharcomma}{\kern0pt}zero\ {\isasymcirc}\isactrlsub c\ {\isasymbeta}\isactrlbsub {\isasymnat}\isactrlsub c\isactrlesub {\isasymrangle}{\isacharparenright}{\kern0pt}\ {\isasymcirc}\isactrlsub c\ x\ {\isacharequal}{\kern0pt}\ {\isasymt}{\isachardoublequoteclose}\isanewline
\ \ \ \ \ \ \isacommand{then}\isamarkupfalse%
\ \isacommand{have}\isamarkupfalse%
\ {\isachardoublequoteopen}eq{\isacharunderscore}{\kern0pt}pred\ {\isasymnat}\isactrlsub c\ {\isasymcirc}\isactrlsub c\ {\isasymlangle}nth{\isacharunderscore}{\kern0pt}odd{\isacharcomma}{\kern0pt}\ zero\ {\isasymcirc}\isactrlsub c\ {\isasymbeta}\isactrlbsub {\isasymnat}\isactrlsub c\isactrlesub {\isasymrangle}\ {\isasymcirc}\isactrlsub c\ x\ {\isacharequal}{\kern0pt}\ {\isasymt}{\isachardoublequoteclose}\isanewline
\ \ \ \ \ \ \ \ \isacommand{by}\isamarkupfalse%
\ {\isacharparenleft}{\kern0pt}typecheck{\isacharunderscore}{\kern0pt}cfuncs{\isacharcomma}{\kern0pt}\ simp\ add{\isacharcolon}{\kern0pt}\ comp{\isacharunderscore}{\kern0pt}associative{\isadigit{2}}{\isacharparenright}{\kern0pt}\isanewline
\ \ \ \ \ \ \isacommand{then}\isamarkupfalse%
\ \isacommand{have}\isamarkupfalse%
\ {\isachardoublequoteopen}eq{\isacharunderscore}{\kern0pt}pred\ {\isasymnat}\isactrlsub c\ {\isasymcirc}\isactrlsub c\ {\isasymlangle}nth{\isacharunderscore}{\kern0pt}odd\ {\isasymcirc}\isactrlsub c\ x{\isacharcomma}{\kern0pt}\ zero\ {\isasymcirc}\isactrlsub c\ {\isasymbeta}\isactrlbsub {\isasymnat}\isactrlsub c\isactrlesub \ {\isasymcirc}\isactrlsub c\ x{\isasymrangle}\ {\isacharequal}{\kern0pt}\ {\isasymt}{\isachardoublequoteclose}\isanewline
\ \ \ \ \ \ \ \ \isacommand{by}\isamarkupfalse%
\ {\isacharparenleft}{\kern0pt}typecheck{\isacharunderscore}{\kern0pt}cfuncs{\isacharunderscore}{\kern0pt}prems{\isacharcomma}{\kern0pt}\ auto\ simp\ add{\isacharcolon}{\kern0pt}\ cfunc{\isacharunderscore}{\kern0pt}prod{\isacharunderscore}{\kern0pt}comp\ comp{\isacharunderscore}{\kern0pt}associative{\isadigit{2}}{\isacharparenright}{\kern0pt}\isanewline
\ \ \ \ \ \ \isacommand{then}\isamarkupfalse%
\ \isacommand{have}\isamarkupfalse%
\ {\isachardoublequoteopen}eq{\isacharunderscore}{\kern0pt}pred\ {\isasymnat}\isactrlsub c\ {\isasymcirc}\isactrlsub c\ {\isasymlangle}nth{\isacharunderscore}{\kern0pt}odd\ {\isasymcirc}\isactrlsub c\ x{\isacharcomma}{\kern0pt}\ zero{\isasymrangle}\ {\isacharequal}{\kern0pt}\ {\isasymt}{\isachardoublequoteclose}\isanewline
\ \ \ \ \ \ \ \ \isacommand{by}\isamarkupfalse%
\ {\isacharparenleft}{\kern0pt}typecheck{\isacharunderscore}{\kern0pt}cfuncs{\isacharunderscore}{\kern0pt}prems{\isacharcomma}{\kern0pt}\ metis\ cfunc{\isacharunderscore}{\kern0pt}type{\isacharunderscore}{\kern0pt}def\ id{\isacharunderscore}{\kern0pt}right{\isacharunderscore}{\kern0pt}unit\ id{\isacharunderscore}{\kern0pt}type\ one{\isacharunderscore}{\kern0pt}unique{\isacharunderscore}{\kern0pt}element{\isacharparenright}{\kern0pt}\isanewline
\ \ \ \ \ \ \isacommand{then}\isamarkupfalse%
\ \isacommand{have}\isamarkupfalse%
\ {\isachardoublequoteopen}nth{\isacharunderscore}{\kern0pt}odd\ {\isasymcirc}\isactrlsub c\ x\ {\isacharequal}{\kern0pt}\ zero{\isachardoublequoteclose}\isanewline
\ \ \ \ \ \ \ \ \isacommand{using}\isamarkupfalse%
\ eq{\isacharunderscore}{\kern0pt}pred{\isacharunderscore}{\kern0pt}iff{\isacharunderscore}{\kern0pt}eq\ \isacommand{by}\isamarkupfalse%
\ {\isacharparenleft}{\kern0pt}typecheck{\isacharunderscore}{\kern0pt}cfuncs{\isacharunderscore}{\kern0pt}prems{\isacharcomma}{\kern0pt}\ blast{\isacharparenright}{\kern0pt}\isanewline
\ \ \ \ \ \ \isacommand{then}\isamarkupfalse%
\ \isacommand{show}\isamarkupfalse%
\ False\isanewline
\ \ \ \ \ \ \ \ \isacommand{by}\isamarkupfalse%
\ {\isacharparenleft}{\kern0pt}typecheck{\isacharunderscore}{\kern0pt}cfuncs{\isacharunderscore}{\kern0pt}prems{\isacharcomma}{\kern0pt}\ smt\ comp{\isacharunderscore}{\kern0pt}associative{\isadigit{2}}\ comp{\isacharunderscore}{\kern0pt}type\ nth{\isacharunderscore}{\kern0pt}even{\isacharunderscore}{\kern0pt}def{\isadigit{2}}\ nth{\isacharunderscore}{\kern0pt}odd{\isacharunderscore}{\kern0pt}is{\isacharunderscore}{\kern0pt}succ{\isacharunderscore}{\kern0pt}nth{\isacharunderscore}{\kern0pt}even\ successor{\isacharunderscore}{\kern0pt}type\ zero{\isacharunderscore}{\kern0pt}is{\isacharunderscore}{\kern0pt}not{\isacharunderscore}{\kern0pt}successor{\isacharparenright}{\kern0pt}\isanewline
\ \ \ \ \isacommand{qed}\isamarkupfalse%
\isanewline
\ \ \ \ \isacommand{then}\isamarkupfalse%
\ \isacommand{have}\isamarkupfalse%
\ {\isachardoublequoteopen}EXISTS\ {\isasymnat}\isactrlsub c\ {\isasymcirc}\isactrlsub c\ {\isacharparenleft}{\kern0pt}{\isacharparenleft}{\kern0pt}eq{\isacharunderscore}{\kern0pt}pred\ {\isasymnat}\isactrlsub c\ {\isasymcirc}\isactrlsub c\ {\isasymlangle}nth{\isacharunderscore}{\kern0pt}odd{\isacharcomma}{\kern0pt}zero\ {\isasymcirc}\isactrlsub c\ {\isasymbeta}\isactrlbsub {\isasymnat}\isactrlsub c\isactrlesub {\isasymrangle}{\isacharparenright}{\kern0pt}\ {\isasymcirc}\isactrlsub c\ left{\isacharunderscore}{\kern0pt}cart{\isacharunderscore}{\kern0pt}proj\ {\isasymnat}\isactrlsub c\ {\isasymone}{\isacharparenright}{\kern0pt}\isactrlsup {\isasymsharp}\ {\isasymnoteq}\ {\isasymt}{\isachardoublequoteclose}\isanewline
\ \ \ \ \ \ \isacommand{using}\isamarkupfalse%
\ EXISTS{\isacharunderscore}{\kern0pt}true{\isacharunderscore}{\kern0pt}implies{\isacharunderscore}{\kern0pt}exists{\isacharunderscore}{\kern0pt}true\ \isacommand{by}\isamarkupfalse%
\ {\isacharparenleft}{\kern0pt}typecheck{\isacharunderscore}{\kern0pt}cfuncs{\isacharcomma}{\kern0pt}\ blast{\isacharparenright}{\kern0pt}\isanewline
\ \ \ \ \isacommand{then}\isamarkupfalse%
\ \isacommand{show}\isamarkupfalse%
\ {\isachardoublequoteopen}EXISTS\ {\isasymnat}\isactrlsub c\ {\isasymcirc}\isactrlsub c\ {\isacharparenleft}{\kern0pt}{\isacharparenleft}{\kern0pt}eq{\isacharunderscore}{\kern0pt}pred\ {\isasymnat}\isactrlsub c\ {\isasymcirc}\isactrlsub c\ {\isasymlangle}nth{\isacharunderscore}{\kern0pt}odd{\isacharcomma}{\kern0pt}zero\ {\isasymcirc}\isactrlsub c\ {\isasymbeta}\isactrlbsub {\isasymnat}\isactrlsub c\isactrlesub {\isasymrangle}{\isacharparenright}{\kern0pt}\ {\isasymcirc}\isactrlsub c\ left{\isacharunderscore}{\kern0pt}cart{\isacharunderscore}{\kern0pt}proj\ {\isasymnat}\isactrlsub c\ {\isasymone}{\isacharparenright}{\kern0pt}\isactrlsup {\isasymsharp}\ {\isacharequal}{\kern0pt}\ {\isasymf}{\isachardoublequoteclose}\isanewline
\ \ \ \ \ \ \isacommand{using}\isamarkupfalse%
\ true{\isacharunderscore}{\kern0pt}false{\isacharunderscore}{\kern0pt}only{\isacharunderscore}{\kern0pt}truth{\isacharunderscore}{\kern0pt}values\ \isacommand{by}\isamarkupfalse%
\ {\isacharparenleft}{\kern0pt}typecheck{\isacharunderscore}{\kern0pt}cfuncs{\isacharcomma}{\kern0pt}\ blast{\isacharparenright}{\kern0pt}\isanewline
\ \ \isacommand{qed}\isamarkupfalse%
\isanewline
\ \ \isacommand{then}\isamarkupfalse%
\ \isacommand{show}\isamarkupfalse%
\ {\isacharquery}{\kern0pt}thesis\isanewline
\ \ \ \ \isacommand{using}\isamarkupfalse%
\ calculation\ \isacommand{by}\isamarkupfalse%
\ auto\isanewline
\isacommand{qed}\isamarkupfalse%
%
\endisatagproof
{\isafoldproof}%
%
\isadelimproof
%
\endisadelimproof
%
\isadelimdocument
%
\endisadelimdocument
%
\isatagdocument
%
\isamarkupsubsection{Natural Number Halving%
}
\isamarkuptrue%
%
\endisatagdocument
{\isafolddocument}%
%
\isadelimdocument
%
\endisadelimdocument
\isacommand{definition}\isamarkupfalse%
\ halve{\isacharunderscore}{\kern0pt}with{\isacharunderscore}{\kern0pt}parity\ {\isacharcolon}{\kern0pt}{\isacharcolon}{\kern0pt}\ {\isachardoublequoteopen}cfunc{\isachardoublequoteclose}\ \isakeyword{where}\isanewline
\ \ {\isachardoublequoteopen}halve{\isacharunderscore}{\kern0pt}with{\isacharunderscore}{\kern0pt}parity\ {\isacharequal}{\kern0pt}\ {\isacharparenleft}{\kern0pt}THE\ u{\isachardot}{\kern0pt}\ u{\isacharcolon}{\kern0pt}\ {\isasymnat}\isactrlsub c\ {\isasymrightarrow}\ {\isasymnat}\isactrlsub c\ {\isasymCoprod}\ {\isasymnat}\isactrlsub c\ {\isasymand}\ \isanewline
\ \ \ \ u\ {\isasymcirc}\isactrlsub c\ zero\ {\isacharequal}{\kern0pt}\ left{\isacharunderscore}{\kern0pt}coproj\ {\isasymnat}\isactrlsub c\ {\isasymnat}\isactrlsub c\ {\isasymcirc}\isactrlsub c\ zero\ {\isasymand}\isanewline
\ \ \ \ {\isacharparenleft}{\kern0pt}right{\isacharunderscore}{\kern0pt}coproj\ {\isasymnat}\isactrlsub c\ {\isasymnat}\isactrlsub c\ {\isasymamalg}\ {\isacharparenleft}{\kern0pt}left{\isacharunderscore}{\kern0pt}coproj\ {\isasymnat}\isactrlsub c\ {\isasymnat}\isactrlsub c\ {\isasymcirc}\isactrlsub c\ successor{\isacharparenright}{\kern0pt}{\isacharparenright}{\kern0pt}\ {\isasymcirc}\isactrlsub c\ u\ {\isacharequal}{\kern0pt}\ u\ {\isasymcirc}\isactrlsub c\ successor{\isacharparenright}{\kern0pt}{\isachardoublequoteclose}\isanewline
\isanewline
\isacommand{lemma}\isamarkupfalse%
\ halve{\isacharunderscore}{\kern0pt}with{\isacharunderscore}{\kern0pt}parity{\isacharunderscore}{\kern0pt}def{\isadigit{2}}{\isacharcolon}{\kern0pt}\isanewline
\ \ {\isachardoublequoteopen}halve{\isacharunderscore}{\kern0pt}with{\isacharunderscore}{\kern0pt}parity\ {\isacharcolon}{\kern0pt}\ {\isasymnat}\isactrlsub c\ {\isasymrightarrow}\ {\isasymnat}\isactrlsub c\ {\isasymCoprod}\ {\isasymnat}\isactrlsub c\ {\isasymand}\ \isanewline
\ \ \ \ halve{\isacharunderscore}{\kern0pt}with{\isacharunderscore}{\kern0pt}parity\ {\isasymcirc}\isactrlsub c\ zero\ {\isacharequal}{\kern0pt}\ left{\isacharunderscore}{\kern0pt}coproj\ {\isasymnat}\isactrlsub c\ {\isasymnat}\isactrlsub c\ {\isasymcirc}\isactrlsub c\ zero\ {\isasymand}\isanewline
\ \ \ \ {\isacharparenleft}{\kern0pt}right{\isacharunderscore}{\kern0pt}coproj\ {\isasymnat}\isactrlsub c\ {\isasymnat}\isactrlsub c\ {\isasymamalg}\ {\isacharparenleft}{\kern0pt}left{\isacharunderscore}{\kern0pt}coproj\ {\isasymnat}\isactrlsub c\ {\isasymnat}\isactrlsub c\ {\isasymcirc}\isactrlsub c\ successor{\isacharparenright}{\kern0pt}{\isacharparenright}{\kern0pt}\ {\isasymcirc}\isactrlsub c\ halve{\isacharunderscore}{\kern0pt}with{\isacharunderscore}{\kern0pt}parity\ {\isacharequal}{\kern0pt}\ halve{\isacharunderscore}{\kern0pt}with{\isacharunderscore}{\kern0pt}parity\ {\isasymcirc}\isactrlsub c\ successor{\isachardoublequoteclose}\isanewline
%
\isadelimproof
\ \ %
\endisadelimproof
%
\isatagproof
\isacommand{unfolding}\isamarkupfalse%
\ halve{\isacharunderscore}{\kern0pt}with{\isacharunderscore}{\kern0pt}parity{\isacharunderscore}{\kern0pt}def\ \isacommand{by}\isamarkupfalse%
\ {\isacharparenleft}{\kern0pt}rule\ theI{\isacharprime}{\kern0pt}{\isacharcomma}{\kern0pt}\ etcs{\isacharunderscore}{\kern0pt}rule\ natural{\isacharunderscore}{\kern0pt}number{\isacharunderscore}{\kern0pt}object{\isacharunderscore}{\kern0pt}property{\isadigit{2}}{\isacharparenright}{\kern0pt}%
\endisatagproof
{\isafoldproof}%
%
\isadelimproof
\isanewline
%
\endisadelimproof
\isanewline
\isacommand{lemma}\isamarkupfalse%
\ halve{\isacharunderscore}{\kern0pt}with{\isacharunderscore}{\kern0pt}parity{\isacharunderscore}{\kern0pt}type{\isacharbrackleft}{\kern0pt}type{\isacharunderscore}{\kern0pt}rule{\isacharbrackright}{\kern0pt}{\isacharcolon}{\kern0pt}\isanewline
\ \ {\isachardoublequoteopen}halve{\isacharunderscore}{\kern0pt}with{\isacharunderscore}{\kern0pt}parity\ {\isacharcolon}{\kern0pt}\ {\isasymnat}\isactrlsub c\ {\isasymrightarrow}\ {\isasymnat}\isactrlsub c\ {\isasymCoprod}\ {\isasymnat}\isactrlsub c{\isachardoublequoteclose}\isanewline
%
\isadelimproof
\ \ %
\endisadelimproof
%
\isatagproof
\isacommand{by}\isamarkupfalse%
\ {\isacharparenleft}{\kern0pt}simp\ add{\isacharcolon}{\kern0pt}\ halve{\isacharunderscore}{\kern0pt}with{\isacharunderscore}{\kern0pt}parity{\isacharunderscore}{\kern0pt}def{\isadigit{2}}{\isacharparenright}{\kern0pt}%
\endisatagproof
{\isafoldproof}%
%
\isadelimproof
\isanewline
%
\endisadelimproof
\isanewline
\isacommand{lemma}\isamarkupfalse%
\ halve{\isacharunderscore}{\kern0pt}with{\isacharunderscore}{\kern0pt}parity{\isacharunderscore}{\kern0pt}zero{\isacharcolon}{\kern0pt}\isanewline
\ \ {\isachardoublequoteopen}halve{\isacharunderscore}{\kern0pt}with{\isacharunderscore}{\kern0pt}parity\ {\isasymcirc}\isactrlsub c\ zero\ {\isacharequal}{\kern0pt}\ left{\isacharunderscore}{\kern0pt}coproj\ {\isasymnat}\isactrlsub c\ {\isasymnat}\isactrlsub c\ {\isasymcirc}\isactrlsub c\ zero{\isachardoublequoteclose}\isanewline
%
\isadelimproof
\ \ %
\endisadelimproof
%
\isatagproof
\isacommand{by}\isamarkupfalse%
\ {\isacharparenleft}{\kern0pt}simp\ add{\isacharcolon}{\kern0pt}\ halve{\isacharunderscore}{\kern0pt}with{\isacharunderscore}{\kern0pt}parity{\isacharunderscore}{\kern0pt}def{\isadigit{2}}{\isacharparenright}{\kern0pt}%
\endisatagproof
{\isafoldproof}%
%
\isadelimproof
\isanewline
%
\endisadelimproof
\isanewline
\isacommand{lemma}\isamarkupfalse%
\ halve{\isacharunderscore}{\kern0pt}with{\isacharunderscore}{\kern0pt}parity{\isacharunderscore}{\kern0pt}successor{\isacharcolon}{\kern0pt}\isanewline
\ \ {\isachardoublequoteopen}{\isacharparenleft}{\kern0pt}right{\isacharunderscore}{\kern0pt}coproj\ {\isasymnat}\isactrlsub c\ {\isasymnat}\isactrlsub c\ {\isasymamalg}\ {\isacharparenleft}{\kern0pt}left{\isacharunderscore}{\kern0pt}coproj\ {\isasymnat}\isactrlsub c\ {\isasymnat}\isactrlsub c\ {\isasymcirc}\isactrlsub c\ successor{\isacharparenright}{\kern0pt}{\isacharparenright}{\kern0pt}\ {\isasymcirc}\isactrlsub c\ halve{\isacharunderscore}{\kern0pt}with{\isacharunderscore}{\kern0pt}parity\ {\isacharequal}{\kern0pt}\ halve{\isacharunderscore}{\kern0pt}with{\isacharunderscore}{\kern0pt}parity\ {\isasymcirc}\isactrlsub c\ successor{\isachardoublequoteclose}\isanewline
%
\isadelimproof
\ \ %
\endisadelimproof
%
\isatagproof
\isacommand{by}\isamarkupfalse%
\ {\isacharparenleft}{\kern0pt}simp\ add{\isacharcolon}{\kern0pt}\ halve{\isacharunderscore}{\kern0pt}with{\isacharunderscore}{\kern0pt}parity{\isacharunderscore}{\kern0pt}def{\isadigit{2}}{\isacharparenright}{\kern0pt}%
\endisatagproof
{\isafoldproof}%
%
\isadelimproof
\isanewline
%
\endisadelimproof
\isanewline
\isacommand{lemma}\isamarkupfalse%
\ halve{\isacharunderscore}{\kern0pt}with{\isacharunderscore}{\kern0pt}parity{\isacharunderscore}{\kern0pt}nth{\isacharunderscore}{\kern0pt}even{\isacharcolon}{\kern0pt}\isanewline
\ \ {\isachardoublequoteopen}halve{\isacharunderscore}{\kern0pt}with{\isacharunderscore}{\kern0pt}parity\ {\isasymcirc}\isactrlsub c\ nth{\isacharunderscore}{\kern0pt}even\ {\isacharequal}{\kern0pt}\ left{\isacharunderscore}{\kern0pt}coproj\ {\isasymnat}\isactrlsub c\ {\isasymnat}\isactrlsub c{\isachardoublequoteclose}\isanewline
%
\isadelimproof
%
\endisadelimproof
%
\isatagproof
\isacommand{proof}\isamarkupfalse%
\ {\isacharparenleft}{\kern0pt}etcs{\isacharunderscore}{\kern0pt}rule\ natural{\isacharunderscore}{\kern0pt}number{\isacharunderscore}{\kern0pt}object{\isacharunderscore}{\kern0pt}func{\isacharunderscore}{\kern0pt}unique{\isacharbrackleft}{\kern0pt}\isakeyword{where}\ X{\isacharequal}{\kern0pt}{\isachardoublequoteopen}{\isasymnat}\isactrlsub c\ {\isasymCoprod}\ {\isasymnat}\isactrlsub c{\isachardoublequoteclose}{\isacharcomma}{\kern0pt}\ \isakeyword{where}\ f{\isacharequal}{\kern0pt}{\isachardoublequoteopen}{\isacharparenleft}{\kern0pt}left{\isacharunderscore}{\kern0pt}coproj\ {\isasymnat}\isactrlsub c\ {\isasymnat}\isactrlsub c\ {\isasymcirc}\isactrlsub c\ successor{\isacharparenright}{\kern0pt}\ {\isasymamalg}\ {\isacharparenleft}{\kern0pt}right{\isacharunderscore}{\kern0pt}coproj\ {\isasymnat}\isactrlsub c\ {\isasymnat}\isactrlsub c\ {\isasymcirc}\isactrlsub c\ successor{\isacharparenright}{\kern0pt}{\isachardoublequoteclose}{\isacharbrackright}{\kern0pt}{\isacharparenright}{\kern0pt}\isanewline
\ \ \isacommand{show}\isamarkupfalse%
\ {\isachardoublequoteopen}{\isacharparenleft}{\kern0pt}halve{\isacharunderscore}{\kern0pt}with{\isacharunderscore}{\kern0pt}parity\ {\isasymcirc}\isactrlsub c\ nth{\isacharunderscore}{\kern0pt}even{\isacharparenright}{\kern0pt}\ {\isasymcirc}\isactrlsub c\ zero\ {\isacharequal}{\kern0pt}\ left{\isacharunderscore}{\kern0pt}coproj\ {\isasymnat}\isactrlsub c\ {\isasymnat}\isactrlsub c\ {\isasymcirc}\isactrlsub c\ zero{\isachardoublequoteclose}\isanewline
\ \ \isacommand{proof}\isamarkupfalse%
\ {\isacharminus}{\kern0pt}\isanewline
\ \ \ \ \isacommand{have}\isamarkupfalse%
\ {\isachardoublequoteopen}{\isacharparenleft}{\kern0pt}halve{\isacharunderscore}{\kern0pt}with{\isacharunderscore}{\kern0pt}parity\ {\isasymcirc}\isactrlsub c\ nth{\isacharunderscore}{\kern0pt}even{\isacharparenright}{\kern0pt}\ {\isasymcirc}\isactrlsub c\ zero\ {\isacharequal}{\kern0pt}\ halve{\isacharunderscore}{\kern0pt}with{\isacharunderscore}{\kern0pt}parity\ {\isasymcirc}\isactrlsub c\ nth{\isacharunderscore}{\kern0pt}even\ {\isasymcirc}\isactrlsub c\ zero{\isachardoublequoteclose}\isanewline
\ \ \ \ \ \ \isacommand{by}\isamarkupfalse%
\ {\isacharparenleft}{\kern0pt}typecheck{\isacharunderscore}{\kern0pt}cfuncs{\isacharcomma}{\kern0pt}\ simp\ add{\isacharcolon}{\kern0pt}\ comp{\isacharunderscore}{\kern0pt}associative{\isadigit{2}}{\isacharparenright}{\kern0pt}\isanewline
\ \ \ \ \isacommand{also}\isamarkupfalse%
\ \isacommand{have}\isamarkupfalse%
\ {\isachardoublequoteopen}{\isachardot}{\kern0pt}{\isachardot}{\kern0pt}{\isachardot}{\kern0pt}\ {\isacharequal}{\kern0pt}\ halve{\isacharunderscore}{\kern0pt}with{\isacharunderscore}{\kern0pt}parity\ {\isasymcirc}\isactrlsub c\ zero{\isachardoublequoteclose}\isanewline
\ \ \ \ \ \ \isacommand{by}\isamarkupfalse%
\ {\isacharparenleft}{\kern0pt}simp\ add{\isacharcolon}{\kern0pt}\ nth{\isacharunderscore}{\kern0pt}even{\isacharunderscore}{\kern0pt}zero{\isacharparenright}{\kern0pt}\isanewline
\ \ \ \ \isacommand{also}\isamarkupfalse%
\ \isacommand{have}\isamarkupfalse%
\ {\isachardoublequoteopen}{\isachardot}{\kern0pt}{\isachardot}{\kern0pt}{\isachardot}{\kern0pt}\ {\isacharequal}{\kern0pt}\ left{\isacharunderscore}{\kern0pt}coproj\ {\isasymnat}\isactrlsub c\ {\isasymnat}\isactrlsub c\ {\isasymcirc}\isactrlsub c\ zero{\isachardoublequoteclose}\isanewline
\ \ \ \ \ \ \isacommand{by}\isamarkupfalse%
\ {\isacharparenleft}{\kern0pt}simp\ add{\isacharcolon}{\kern0pt}\ halve{\isacharunderscore}{\kern0pt}with{\isacharunderscore}{\kern0pt}parity{\isacharunderscore}{\kern0pt}zero{\isacharparenright}{\kern0pt}\isanewline
\ \ \ \ \isacommand{then}\isamarkupfalse%
\ \isacommand{show}\isamarkupfalse%
\ {\isacharquery}{\kern0pt}thesis\isanewline
\ \ \ \ \ \ \isacommand{using}\isamarkupfalse%
\ calculation\ \isacommand{by}\isamarkupfalse%
\ auto\isanewline
\ \ \isacommand{qed}\isamarkupfalse%
\isanewline
\isanewline
\ \ \isacommand{show}\isamarkupfalse%
\ {\isachardoublequoteopen}{\isacharparenleft}{\kern0pt}halve{\isacharunderscore}{\kern0pt}with{\isacharunderscore}{\kern0pt}parity\ {\isasymcirc}\isactrlsub c\ nth{\isacharunderscore}{\kern0pt}even{\isacharparenright}{\kern0pt}\ {\isasymcirc}\isactrlsub c\ successor\ {\isacharequal}{\kern0pt}\isanewline
\ \ \ \ \ \ {\isacharparenleft}{\kern0pt}{\isacharparenleft}{\kern0pt}left{\isacharunderscore}{\kern0pt}coproj\ {\isasymnat}\isactrlsub c\ {\isasymnat}\isactrlsub c\ {\isasymcirc}\isactrlsub c\ successor{\isacharparenright}{\kern0pt}\ {\isasymamalg}\ {\isacharparenleft}{\kern0pt}right{\isacharunderscore}{\kern0pt}coproj\ {\isasymnat}\isactrlsub c\ {\isasymnat}\isactrlsub c\ {\isasymcirc}\isactrlsub c\ successor{\isacharparenright}{\kern0pt}{\isacharparenright}{\kern0pt}\ {\isasymcirc}\isactrlsub c\ halve{\isacharunderscore}{\kern0pt}with{\isacharunderscore}{\kern0pt}parity\ {\isasymcirc}\isactrlsub c\ nth{\isacharunderscore}{\kern0pt}even{\isachardoublequoteclose}\isanewline
\ \ \isacommand{proof}\isamarkupfalse%
\ {\isacharminus}{\kern0pt}\isanewline
\ \ \ \ \isacommand{have}\isamarkupfalse%
\ {\isachardoublequoteopen}{\isacharparenleft}{\kern0pt}halve{\isacharunderscore}{\kern0pt}with{\isacharunderscore}{\kern0pt}parity\ {\isasymcirc}\isactrlsub c\ nth{\isacharunderscore}{\kern0pt}even{\isacharparenright}{\kern0pt}\ {\isasymcirc}\isactrlsub c\ successor\ {\isacharequal}{\kern0pt}\ halve{\isacharunderscore}{\kern0pt}with{\isacharunderscore}{\kern0pt}parity\ {\isasymcirc}\isactrlsub c\ nth{\isacharunderscore}{\kern0pt}even\ {\isasymcirc}\isactrlsub c\ successor{\isachardoublequoteclose}\isanewline
\ \ \ \ \ \ \isacommand{by}\isamarkupfalse%
\ {\isacharparenleft}{\kern0pt}typecheck{\isacharunderscore}{\kern0pt}cfuncs{\isacharcomma}{\kern0pt}\ simp\ add{\isacharcolon}{\kern0pt}\ comp{\isacharunderscore}{\kern0pt}associative{\isadigit{2}}{\isacharparenright}{\kern0pt}\isanewline
\ \ \ \ \isacommand{also}\isamarkupfalse%
\ \isacommand{have}\isamarkupfalse%
\ {\isachardoublequoteopen}{\isachardot}{\kern0pt}{\isachardot}{\kern0pt}{\isachardot}{\kern0pt}\ {\isacharequal}{\kern0pt}\ halve{\isacharunderscore}{\kern0pt}with{\isacharunderscore}{\kern0pt}parity\ {\isasymcirc}\isactrlsub c\ {\isacharparenleft}{\kern0pt}successor\ {\isasymcirc}\isactrlsub c\ successor{\isacharparenright}{\kern0pt}\ {\isasymcirc}\isactrlsub c\ nth{\isacharunderscore}{\kern0pt}even{\isachardoublequoteclose}\isanewline
\ \ \ \ \ \ \isacommand{by}\isamarkupfalse%
\ {\isacharparenleft}{\kern0pt}simp\ add{\isacharcolon}{\kern0pt}\ nth{\isacharunderscore}{\kern0pt}even{\isacharunderscore}{\kern0pt}successor{\isacharparenright}{\kern0pt}\isanewline
\ \ \ \ \isacommand{also}\isamarkupfalse%
\ \isacommand{have}\isamarkupfalse%
\ {\isachardoublequoteopen}{\isachardot}{\kern0pt}{\isachardot}{\kern0pt}{\isachardot}{\kern0pt}\ {\isacharequal}{\kern0pt}\ {\isacharparenleft}{\kern0pt}{\isacharparenleft}{\kern0pt}halve{\isacharunderscore}{\kern0pt}with{\isacharunderscore}{\kern0pt}parity\ {\isasymcirc}\isactrlsub c\ successor{\isacharparenright}{\kern0pt}\ {\isasymcirc}\isactrlsub c\ successor{\isacharparenright}{\kern0pt}\ {\isasymcirc}\isactrlsub c\ nth{\isacharunderscore}{\kern0pt}even{\isachardoublequoteclose}\isanewline
\ \ \ \ \ \ \isacommand{by}\isamarkupfalse%
\ {\isacharparenleft}{\kern0pt}typecheck{\isacharunderscore}{\kern0pt}cfuncs{\isacharcomma}{\kern0pt}\ simp\ add{\isacharcolon}{\kern0pt}\ comp{\isacharunderscore}{\kern0pt}associative{\isadigit{2}}{\isacharparenright}{\kern0pt}\isanewline
\ \ \ \ \isacommand{also}\isamarkupfalse%
\ \isacommand{have}\isamarkupfalse%
\ {\isachardoublequoteopen}{\isachardot}{\kern0pt}{\isachardot}{\kern0pt}{\isachardot}{\kern0pt}\ {\isacharequal}{\kern0pt}\ {\isacharparenleft}{\kern0pt}{\isacharparenleft}{\kern0pt}{\isacharparenleft}{\kern0pt}right{\isacharunderscore}{\kern0pt}coproj\ {\isasymnat}\isactrlsub c\ {\isasymnat}\isactrlsub c\ {\isasymamalg}\ {\isacharparenleft}{\kern0pt}left{\isacharunderscore}{\kern0pt}coproj\ {\isasymnat}\isactrlsub c\ {\isasymnat}\isactrlsub c\ {\isasymcirc}\isactrlsub c\ successor{\isacharparenright}{\kern0pt}{\isacharparenright}{\kern0pt}\ {\isasymcirc}\isactrlsub c\ halve{\isacharunderscore}{\kern0pt}with{\isacharunderscore}{\kern0pt}parity{\isacharparenright}{\kern0pt}\ {\isasymcirc}\isactrlsub c\ successor{\isacharparenright}{\kern0pt}\ {\isasymcirc}\isactrlsub c\ nth{\isacharunderscore}{\kern0pt}even{\isachardoublequoteclose}\isanewline
\ \ \ \ \ \ \isacommand{by}\isamarkupfalse%
\ {\isacharparenleft}{\kern0pt}simp\ add{\isacharcolon}{\kern0pt}\ halve{\isacharunderscore}{\kern0pt}with{\isacharunderscore}{\kern0pt}parity{\isacharunderscore}{\kern0pt}def{\isadigit{2}}{\isacharparenright}{\kern0pt}\isanewline
\ \ \ \ \isacommand{also}\isamarkupfalse%
\ \isacommand{have}\isamarkupfalse%
\ {\isachardoublequoteopen}{\isachardot}{\kern0pt}{\isachardot}{\kern0pt}{\isachardot}{\kern0pt}\ {\isacharequal}{\kern0pt}\ {\isacharparenleft}{\kern0pt}right{\isacharunderscore}{\kern0pt}coproj\ {\isasymnat}\isactrlsub c\ {\isasymnat}\isactrlsub c\ {\isasymamalg}\ {\isacharparenleft}{\kern0pt}left{\isacharunderscore}{\kern0pt}coproj\ {\isasymnat}\isactrlsub c\ {\isasymnat}\isactrlsub c\ {\isasymcirc}\isactrlsub c\ successor{\isacharparenright}{\kern0pt}{\isacharparenright}{\kern0pt}\isanewline
\ \ \ \ \ \ \ \ {\isasymcirc}\isactrlsub c\ {\isacharparenleft}{\kern0pt}halve{\isacharunderscore}{\kern0pt}with{\isacharunderscore}{\kern0pt}parity\ {\isasymcirc}\isactrlsub c\ successor{\isacharparenright}{\kern0pt}\ {\isasymcirc}\isactrlsub c\ nth{\isacharunderscore}{\kern0pt}even{\isachardoublequoteclose}\isanewline
\ \ \ \ \ \ \isacommand{by}\isamarkupfalse%
\ {\isacharparenleft}{\kern0pt}typecheck{\isacharunderscore}{\kern0pt}cfuncs{\isacharcomma}{\kern0pt}\ simp\ add{\isacharcolon}{\kern0pt}\ comp{\isacharunderscore}{\kern0pt}associative{\isadigit{2}}{\isacharparenright}{\kern0pt}\isanewline
\ \ \ \ \isacommand{also}\isamarkupfalse%
\ \isacommand{have}\isamarkupfalse%
\ {\isachardoublequoteopen}{\isachardot}{\kern0pt}{\isachardot}{\kern0pt}{\isachardot}{\kern0pt}\ {\isacharequal}{\kern0pt}\ {\isacharparenleft}{\kern0pt}right{\isacharunderscore}{\kern0pt}coproj\ {\isasymnat}\isactrlsub c\ {\isasymnat}\isactrlsub c\ {\isasymamalg}\ {\isacharparenleft}{\kern0pt}left{\isacharunderscore}{\kern0pt}coproj\ {\isasymnat}\isactrlsub c\ {\isasymnat}\isactrlsub c\ {\isasymcirc}\isactrlsub c\ successor{\isacharparenright}{\kern0pt}{\isacharparenright}{\kern0pt}\isanewline
\ \ \ \ \ \ \ \ {\isasymcirc}\isactrlsub c\ {\isacharparenleft}{\kern0pt}{\isacharparenleft}{\kern0pt}right{\isacharunderscore}{\kern0pt}coproj\ {\isasymnat}\isactrlsub c\ {\isasymnat}\isactrlsub c\ {\isasymamalg}\ {\isacharparenleft}{\kern0pt}left{\isacharunderscore}{\kern0pt}coproj\ {\isasymnat}\isactrlsub c\ {\isasymnat}\isactrlsub c\ {\isasymcirc}\isactrlsub c\ successor{\isacharparenright}{\kern0pt}{\isacharparenright}{\kern0pt}\ {\isasymcirc}\isactrlsub c\ halve{\isacharunderscore}{\kern0pt}with{\isacharunderscore}{\kern0pt}parity{\isacharparenright}{\kern0pt}\ {\isasymcirc}\isactrlsub c\ nth{\isacharunderscore}{\kern0pt}even{\isachardoublequoteclose}\isanewline
\ \ \ \ \ \ \isacommand{by}\isamarkupfalse%
\ {\isacharparenleft}{\kern0pt}simp\ add{\isacharcolon}{\kern0pt}\ halve{\isacharunderscore}{\kern0pt}with{\isacharunderscore}{\kern0pt}parity{\isacharunderscore}{\kern0pt}def{\isadigit{2}}{\isacharparenright}{\kern0pt}\isanewline
\ \ \ \ \isacommand{also}\isamarkupfalse%
\ \isacommand{have}\isamarkupfalse%
\ {\isachardoublequoteopen}{\isachardot}{\kern0pt}{\isachardot}{\kern0pt}{\isachardot}{\kern0pt}\ {\isacharequal}{\kern0pt}\ {\isacharparenleft}{\kern0pt}{\isacharparenleft}{\kern0pt}right{\isacharunderscore}{\kern0pt}coproj\ {\isasymnat}\isactrlsub c\ {\isasymnat}\isactrlsub c\ {\isasymamalg}\ {\isacharparenleft}{\kern0pt}left{\isacharunderscore}{\kern0pt}coproj\ {\isasymnat}\isactrlsub c\ {\isasymnat}\isactrlsub c\ {\isasymcirc}\isactrlsub c\ successor{\isacharparenright}{\kern0pt}{\isacharparenright}{\kern0pt}\isanewline
\ \ \ \ \ \ \ \ {\isasymcirc}\isactrlsub c\ {\isacharparenleft}{\kern0pt}right{\isacharunderscore}{\kern0pt}coproj\ {\isasymnat}\isactrlsub c\ {\isasymnat}\isactrlsub c\ {\isasymamalg}\ {\isacharparenleft}{\kern0pt}left{\isacharunderscore}{\kern0pt}coproj\ {\isasymnat}\isactrlsub c\ {\isasymnat}\isactrlsub c\ {\isasymcirc}\isactrlsub c\ successor{\isacharparenright}{\kern0pt}{\isacharparenright}{\kern0pt}{\isacharparenright}{\kern0pt}\isanewline
\ \ \ \ \ \ \ \ {\isasymcirc}\isactrlsub c\ halve{\isacharunderscore}{\kern0pt}with{\isacharunderscore}{\kern0pt}parity\ {\isasymcirc}\isactrlsub c\ nth{\isacharunderscore}{\kern0pt}even{\isachardoublequoteclose}\isanewline
\ \ \ \ \ \ \isacommand{by}\isamarkupfalse%
\ {\isacharparenleft}{\kern0pt}typecheck{\isacharunderscore}{\kern0pt}cfuncs{\isacharcomma}{\kern0pt}\ simp\ add{\isacharcolon}{\kern0pt}\ comp{\isacharunderscore}{\kern0pt}associative{\isadigit{2}}{\isacharparenright}{\kern0pt}\isanewline
\ \ \ \ \isacommand{also}\isamarkupfalse%
\ \isacommand{have}\isamarkupfalse%
\ {\isachardoublequoteopen}{\isachardot}{\kern0pt}{\isachardot}{\kern0pt}{\isachardot}{\kern0pt}\ {\isacharequal}{\kern0pt}\ {\isacharparenleft}{\kern0pt}{\isacharparenleft}{\kern0pt}left{\isacharunderscore}{\kern0pt}coproj\ {\isasymnat}\isactrlsub c\ {\isasymnat}\isactrlsub c\ {\isasymcirc}\isactrlsub c\ successor{\isacharparenright}{\kern0pt}\ {\isasymamalg}\ {\isacharparenleft}{\kern0pt}right{\isacharunderscore}{\kern0pt}coproj\ {\isasymnat}\isactrlsub c\ {\isasymnat}\isactrlsub c\ {\isasymcirc}\isactrlsub c\ successor{\isacharparenright}{\kern0pt}{\isacharparenright}{\kern0pt}\isanewline
\ \ \ \ \ \ \ \ {\isasymcirc}\isactrlsub c\ halve{\isacharunderscore}{\kern0pt}with{\isacharunderscore}{\kern0pt}parity\ {\isasymcirc}\isactrlsub c\ nth{\isacharunderscore}{\kern0pt}even{\isachardoublequoteclose}\isanewline
\ \ \ \ \ \ \isacommand{by}\isamarkupfalse%
\ {\isacharparenleft}{\kern0pt}typecheck{\isacharunderscore}{\kern0pt}cfuncs{\isacharcomma}{\kern0pt}\ smt\ cfunc{\isacharunderscore}{\kern0pt}coprod{\isacharunderscore}{\kern0pt}comp\ comp{\isacharunderscore}{\kern0pt}associative{\isadigit{2}}\ left{\isacharunderscore}{\kern0pt}coproj{\isacharunderscore}{\kern0pt}cfunc{\isacharunderscore}{\kern0pt}coprod\ right{\isacharunderscore}{\kern0pt}coproj{\isacharunderscore}{\kern0pt}cfunc{\isacharunderscore}{\kern0pt}coprod{\isacharparenright}{\kern0pt}\isanewline
\ \ \ \ \isacommand{then}\isamarkupfalse%
\ \isacommand{show}\isamarkupfalse%
\ {\isacharquery}{\kern0pt}thesis\isanewline
\ \ \ \ \ \ \isacommand{using}\isamarkupfalse%
\ calculation\ \isacommand{by}\isamarkupfalse%
\ auto\isanewline
\ \ \isacommand{qed}\isamarkupfalse%
\isanewline
\isanewline
\ \ \isacommand{show}\isamarkupfalse%
\ {\isachardoublequoteopen}left{\isacharunderscore}{\kern0pt}coproj\ {\isasymnat}\isactrlsub c\ {\isasymnat}\isactrlsub c\ {\isasymcirc}\isactrlsub c\ successor\ {\isacharequal}{\kern0pt}\isanewline
\ \ \ \ {\isacharparenleft}{\kern0pt}left{\isacharunderscore}{\kern0pt}coproj\ {\isasymnat}\isactrlsub c\ {\isasymnat}\isactrlsub c\ {\isasymcirc}\isactrlsub c\ successor{\isacharparenright}{\kern0pt}\ {\isasymamalg}\ {\isacharparenleft}{\kern0pt}right{\isacharunderscore}{\kern0pt}coproj\ {\isasymnat}\isactrlsub c\ {\isasymnat}\isactrlsub c\ {\isasymcirc}\isactrlsub c\ successor{\isacharparenright}{\kern0pt}\ {\isasymcirc}\isactrlsub c\ left{\isacharunderscore}{\kern0pt}coproj\ {\isasymnat}\isactrlsub c\ {\isasymnat}\isactrlsub c{\isachardoublequoteclose}\isanewline
\ \ \ \ \isacommand{by}\isamarkupfalse%
\ {\isacharparenleft}{\kern0pt}typecheck{\isacharunderscore}{\kern0pt}cfuncs{\isacharcomma}{\kern0pt}\ simp\ add{\isacharcolon}{\kern0pt}\ left{\isacharunderscore}{\kern0pt}coproj{\isacharunderscore}{\kern0pt}cfunc{\isacharunderscore}{\kern0pt}coprod{\isacharparenright}{\kern0pt}\isanewline
\isacommand{qed}\isamarkupfalse%
%
\endisatagproof
{\isafoldproof}%
%
\isadelimproof
\isanewline
%
\endisadelimproof
\isanewline
\isacommand{lemma}\isamarkupfalse%
\ halve{\isacharunderscore}{\kern0pt}with{\isacharunderscore}{\kern0pt}parity{\isacharunderscore}{\kern0pt}nth{\isacharunderscore}{\kern0pt}odd{\isacharcolon}{\kern0pt}\isanewline
\ \ {\isachardoublequoteopen}halve{\isacharunderscore}{\kern0pt}with{\isacharunderscore}{\kern0pt}parity\ {\isasymcirc}\isactrlsub c\ nth{\isacharunderscore}{\kern0pt}odd\ {\isacharequal}{\kern0pt}\ right{\isacharunderscore}{\kern0pt}coproj\ {\isasymnat}\isactrlsub c\ {\isasymnat}\isactrlsub c{\isachardoublequoteclose}\isanewline
%
\isadelimproof
%
\endisadelimproof
%
\isatagproof
\isacommand{proof}\isamarkupfalse%
\ {\isacharparenleft}{\kern0pt}etcs{\isacharunderscore}{\kern0pt}rule\ natural{\isacharunderscore}{\kern0pt}number{\isacharunderscore}{\kern0pt}object{\isacharunderscore}{\kern0pt}func{\isacharunderscore}{\kern0pt}unique{\isacharbrackleft}{\kern0pt}\isakeyword{where}\ X{\isacharequal}{\kern0pt}{\isachardoublequoteopen}{\isasymnat}\isactrlsub c\ {\isasymCoprod}\ {\isasymnat}\isactrlsub c{\isachardoublequoteclose}{\isacharcomma}{\kern0pt}\ \isakeyword{where}\ f{\isacharequal}{\kern0pt}{\isachardoublequoteopen}{\isacharparenleft}{\kern0pt}left{\isacharunderscore}{\kern0pt}coproj\ {\isasymnat}\isactrlsub c\ {\isasymnat}\isactrlsub c\ {\isasymcirc}\isactrlsub c\ successor{\isacharparenright}{\kern0pt}\ {\isasymamalg}\ {\isacharparenleft}{\kern0pt}right{\isacharunderscore}{\kern0pt}coproj\ {\isasymnat}\isactrlsub c\ {\isasymnat}\isactrlsub c\ {\isasymcirc}\isactrlsub c\ successor{\isacharparenright}{\kern0pt}{\isachardoublequoteclose}{\isacharbrackright}{\kern0pt}{\isacharparenright}{\kern0pt}\isanewline
\ \ \isacommand{show}\isamarkupfalse%
\ {\isachardoublequoteopen}{\isacharparenleft}{\kern0pt}halve{\isacharunderscore}{\kern0pt}with{\isacharunderscore}{\kern0pt}parity\ {\isasymcirc}\isactrlsub c\ nth{\isacharunderscore}{\kern0pt}odd{\isacharparenright}{\kern0pt}\ {\isasymcirc}\isactrlsub c\ zero\ {\isacharequal}{\kern0pt}\ right{\isacharunderscore}{\kern0pt}coproj\ {\isasymnat}\isactrlsub c\ {\isasymnat}\isactrlsub c\ {\isasymcirc}\isactrlsub c\ zero{\isachardoublequoteclose}\isanewline
\ \ \isacommand{proof}\isamarkupfalse%
\ {\isacharminus}{\kern0pt}\isanewline
\ \ \ \ \isacommand{have}\isamarkupfalse%
\ {\isachardoublequoteopen}{\isacharparenleft}{\kern0pt}halve{\isacharunderscore}{\kern0pt}with{\isacharunderscore}{\kern0pt}parity\ {\isasymcirc}\isactrlsub c\ nth{\isacharunderscore}{\kern0pt}odd{\isacharparenright}{\kern0pt}\ {\isasymcirc}\isactrlsub c\ zero\ {\isacharequal}{\kern0pt}\ halve{\isacharunderscore}{\kern0pt}with{\isacharunderscore}{\kern0pt}parity\ {\isasymcirc}\isactrlsub c\ nth{\isacharunderscore}{\kern0pt}odd\ {\isasymcirc}\isactrlsub c\ zero{\isachardoublequoteclose}\isanewline
\ \ \ \ \ \ \isacommand{by}\isamarkupfalse%
\ {\isacharparenleft}{\kern0pt}typecheck{\isacharunderscore}{\kern0pt}cfuncs{\isacharcomma}{\kern0pt}\ simp\ add{\isacharcolon}{\kern0pt}\ comp{\isacharunderscore}{\kern0pt}associative{\isadigit{2}}{\isacharparenright}{\kern0pt}\isanewline
\ \ \ \ \isacommand{also}\isamarkupfalse%
\ \isacommand{have}\isamarkupfalse%
\ {\isachardoublequoteopen}{\isachardot}{\kern0pt}{\isachardot}{\kern0pt}{\isachardot}{\kern0pt}\ {\isacharequal}{\kern0pt}\ halve{\isacharunderscore}{\kern0pt}with{\isacharunderscore}{\kern0pt}parity\ {\isasymcirc}\isactrlsub c\ successor\ {\isasymcirc}\isactrlsub c\ zero{\isachardoublequoteclose}\isanewline
\ \ \ \ \ \ \isacommand{by}\isamarkupfalse%
\ {\isacharparenleft}{\kern0pt}simp\ add{\isacharcolon}{\kern0pt}\ nth{\isacharunderscore}{\kern0pt}odd{\isacharunderscore}{\kern0pt}def{\isadigit{2}}{\isacharparenright}{\kern0pt}\isanewline
\ \ \ \ \isacommand{also}\isamarkupfalse%
\ \isacommand{have}\isamarkupfalse%
\ {\isachardoublequoteopen}{\isachardot}{\kern0pt}{\isachardot}{\kern0pt}{\isachardot}{\kern0pt}\ {\isacharequal}{\kern0pt}\ {\isacharparenleft}{\kern0pt}halve{\isacharunderscore}{\kern0pt}with{\isacharunderscore}{\kern0pt}parity\ {\isasymcirc}\isactrlsub c\ successor{\isacharparenright}{\kern0pt}\ {\isasymcirc}\isactrlsub c\ zero{\isachardoublequoteclose}\isanewline
\ \ \ \ \ \ \isacommand{by}\isamarkupfalse%
\ {\isacharparenleft}{\kern0pt}typecheck{\isacharunderscore}{\kern0pt}cfuncs{\isacharcomma}{\kern0pt}\ simp\ add{\isacharcolon}{\kern0pt}\ comp{\isacharunderscore}{\kern0pt}associative{\isadigit{2}}{\isacharparenright}{\kern0pt}\isanewline
\ \ \ \ \isacommand{also}\isamarkupfalse%
\ \isacommand{have}\isamarkupfalse%
\ {\isachardoublequoteopen}{\isachardot}{\kern0pt}{\isachardot}{\kern0pt}{\isachardot}{\kern0pt}\ {\isacharequal}{\kern0pt}\ {\isacharparenleft}{\kern0pt}right{\isacharunderscore}{\kern0pt}coproj\ {\isasymnat}\isactrlsub c\ {\isasymnat}\isactrlsub c\ {\isasymamalg}\ {\isacharparenleft}{\kern0pt}left{\isacharunderscore}{\kern0pt}coproj\ {\isasymnat}\isactrlsub c\ {\isasymnat}\isactrlsub c\ {\isasymcirc}\isactrlsub c\ successor{\isacharparenright}{\kern0pt}\ {\isasymcirc}\isactrlsub c\ halve{\isacharunderscore}{\kern0pt}with{\isacharunderscore}{\kern0pt}parity{\isacharparenright}{\kern0pt}\ {\isasymcirc}\isactrlsub c\ zero{\isachardoublequoteclose}\isanewline
\ \ \ \ \ \ \isacommand{by}\isamarkupfalse%
\ {\isacharparenleft}{\kern0pt}simp\ add{\isacharcolon}{\kern0pt}\ halve{\isacharunderscore}{\kern0pt}with{\isacharunderscore}{\kern0pt}parity{\isacharunderscore}{\kern0pt}def{\isadigit{2}}{\isacharparenright}{\kern0pt}\isanewline
\ \ \ \ \isacommand{also}\isamarkupfalse%
\ \isacommand{have}\isamarkupfalse%
\ {\isachardoublequoteopen}{\isachardot}{\kern0pt}{\isachardot}{\kern0pt}{\isachardot}{\kern0pt}\ {\isacharequal}{\kern0pt}\ right{\isacharunderscore}{\kern0pt}coproj\ {\isasymnat}\isactrlsub c\ {\isasymnat}\isactrlsub c\ {\isasymamalg}\ {\isacharparenleft}{\kern0pt}left{\isacharunderscore}{\kern0pt}coproj\ {\isasymnat}\isactrlsub c\ {\isasymnat}\isactrlsub c\ {\isasymcirc}\isactrlsub c\ successor{\isacharparenright}{\kern0pt}\ {\isasymcirc}\isactrlsub c\ halve{\isacharunderscore}{\kern0pt}with{\isacharunderscore}{\kern0pt}parity\ {\isasymcirc}\isactrlsub c\ zero{\isachardoublequoteclose}\isanewline
\ \ \ \ \ \ \isacommand{by}\isamarkupfalse%
\ {\isacharparenleft}{\kern0pt}typecheck{\isacharunderscore}{\kern0pt}cfuncs{\isacharcomma}{\kern0pt}\ simp\ add{\isacharcolon}{\kern0pt}\ comp{\isacharunderscore}{\kern0pt}associative{\isadigit{2}}{\isacharparenright}{\kern0pt}\isanewline
\ \ \ \ \isacommand{also}\isamarkupfalse%
\ \isacommand{have}\isamarkupfalse%
\ {\isachardoublequoteopen}{\isachardot}{\kern0pt}{\isachardot}{\kern0pt}{\isachardot}{\kern0pt}\ {\isacharequal}{\kern0pt}\ right{\isacharunderscore}{\kern0pt}coproj\ {\isasymnat}\isactrlsub c\ {\isasymnat}\isactrlsub c\ {\isasymamalg}\ {\isacharparenleft}{\kern0pt}left{\isacharunderscore}{\kern0pt}coproj\ {\isasymnat}\isactrlsub c\ {\isasymnat}\isactrlsub c\ {\isasymcirc}\isactrlsub c\ successor{\isacharparenright}{\kern0pt}\ {\isasymcirc}\isactrlsub c\ left{\isacharunderscore}{\kern0pt}coproj\ {\isasymnat}\isactrlsub c\ {\isasymnat}\isactrlsub c\ {\isasymcirc}\isactrlsub c\ zero{\isachardoublequoteclose}\isanewline
\ \ \ \ \ \ \isacommand{by}\isamarkupfalse%
\ {\isacharparenleft}{\kern0pt}simp\ add{\isacharcolon}{\kern0pt}\ halve{\isacharunderscore}{\kern0pt}with{\isacharunderscore}{\kern0pt}parity{\isacharunderscore}{\kern0pt}def{\isadigit{2}}{\isacharparenright}{\kern0pt}\isanewline
\ \ \ \ \isacommand{also}\isamarkupfalse%
\ \isacommand{have}\isamarkupfalse%
\ {\isachardoublequoteopen}{\isachardot}{\kern0pt}{\isachardot}{\kern0pt}{\isachardot}{\kern0pt}\ {\isacharequal}{\kern0pt}\ {\isacharparenleft}{\kern0pt}right{\isacharunderscore}{\kern0pt}coproj\ {\isasymnat}\isactrlsub c\ {\isasymnat}\isactrlsub c\ {\isasymamalg}\ {\isacharparenleft}{\kern0pt}left{\isacharunderscore}{\kern0pt}coproj\ {\isasymnat}\isactrlsub c\ {\isasymnat}\isactrlsub c\ {\isasymcirc}\isactrlsub c\ successor{\isacharparenright}{\kern0pt}\ {\isasymcirc}\isactrlsub c\ left{\isacharunderscore}{\kern0pt}coproj\ {\isasymnat}\isactrlsub c\ {\isasymnat}\isactrlsub c{\isacharparenright}{\kern0pt}\ {\isasymcirc}\isactrlsub c\ zero{\isachardoublequoteclose}\isanewline
\ \ \ \ \ \ \isacommand{by}\isamarkupfalse%
\ {\isacharparenleft}{\kern0pt}typecheck{\isacharunderscore}{\kern0pt}cfuncs{\isacharcomma}{\kern0pt}\ simp\ add{\isacharcolon}{\kern0pt}\ comp{\isacharunderscore}{\kern0pt}associative{\isadigit{2}}{\isacharparenright}{\kern0pt}\isanewline
\ \ \ \ \isacommand{also}\isamarkupfalse%
\ \isacommand{have}\isamarkupfalse%
\ {\isachardoublequoteopen}{\isachardot}{\kern0pt}{\isachardot}{\kern0pt}{\isachardot}{\kern0pt}\ {\isacharequal}{\kern0pt}\ right{\isacharunderscore}{\kern0pt}coproj\ {\isasymnat}\isactrlsub c\ {\isasymnat}\isactrlsub c\ {\isasymcirc}\isactrlsub c\ zero{\isachardoublequoteclose}\isanewline
\ \ \ \ \ \ \isacommand{by}\isamarkupfalse%
\ {\isacharparenleft}{\kern0pt}typecheck{\isacharunderscore}{\kern0pt}cfuncs{\isacharcomma}{\kern0pt}\ simp\ add{\isacharcolon}{\kern0pt}\ left{\isacharunderscore}{\kern0pt}coproj{\isacharunderscore}{\kern0pt}cfunc{\isacharunderscore}{\kern0pt}coprod{\isacharparenright}{\kern0pt}\isanewline
\ \ \ \ \isacommand{then}\isamarkupfalse%
\ \isacommand{show}\isamarkupfalse%
\ {\isacharquery}{\kern0pt}thesis\isanewline
\ \ \ \ \ \ \isacommand{using}\isamarkupfalse%
\ calculation\ \isacommand{by}\isamarkupfalse%
\ auto\isanewline
\ \ \isacommand{qed}\isamarkupfalse%
\isanewline
\isanewline
\ \ \isacommand{show}\isamarkupfalse%
\ {\isachardoublequoteopen}{\isacharparenleft}{\kern0pt}halve{\isacharunderscore}{\kern0pt}with{\isacharunderscore}{\kern0pt}parity\ {\isasymcirc}\isactrlsub c\ nth{\isacharunderscore}{\kern0pt}odd{\isacharparenright}{\kern0pt}\ {\isasymcirc}\isactrlsub c\ successor\ {\isacharequal}{\kern0pt}\isanewline
\ \ \ \ \ \ {\isacharparenleft}{\kern0pt}left{\isacharunderscore}{\kern0pt}coproj\ {\isasymnat}\isactrlsub c\ {\isasymnat}\isactrlsub c\ {\isasymcirc}\isactrlsub c\ successor{\isacharparenright}{\kern0pt}\ {\isasymamalg}\ {\isacharparenleft}{\kern0pt}right{\isacharunderscore}{\kern0pt}coproj\ {\isasymnat}\isactrlsub c\ {\isasymnat}\isactrlsub c\ {\isasymcirc}\isactrlsub c\ successor{\isacharparenright}{\kern0pt}\ {\isasymcirc}\isactrlsub c\ halve{\isacharunderscore}{\kern0pt}with{\isacharunderscore}{\kern0pt}parity\ {\isasymcirc}\isactrlsub c\ nth{\isacharunderscore}{\kern0pt}odd{\isachardoublequoteclose}\isanewline
\ \ \isacommand{proof}\isamarkupfalse%
\ {\isacharminus}{\kern0pt}\isanewline
\ \ \ \ \isacommand{have}\isamarkupfalse%
\ {\isachardoublequoteopen}{\isacharparenleft}{\kern0pt}halve{\isacharunderscore}{\kern0pt}with{\isacharunderscore}{\kern0pt}parity\ {\isasymcirc}\isactrlsub c\ nth{\isacharunderscore}{\kern0pt}odd{\isacharparenright}{\kern0pt}\ {\isasymcirc}\isactrlsub c\ successor\ {\isacharequal}{\kern0pt}\ halve{\isacharunderscore}{\kern0pt}with{\isacharunderscore}{\kern0pt}parity\ {\isasymcirc}\isactrlsub c\ nth{\isacharunderscore}{\kern0pt}odd\ {\isasymcirc}\isactrlsub c\ successor{\isachardoublequoteclose}\isanewline
\ \ \ \ \ \ \isacommand{by}\isamarkupfalse%
\ {\isacharparenleft}{\kern0pt}typecheck{\isacharunderscore}{\kern0pt}cfuncs{\isacharcomma}{\kern0pt}\ simp\ add{\isacharcolon}{\kern0pt}\ comp{\isacharunderscore}{\kern0pt}associative{\isadigit{2}}{\isacharparenright}{\kern0pt}\isanewline
\ \ \ \ \isacommand{also}\isamarkupfalse%
\ \isacommand{have}\isamarkupfalse%
\ {\isachardoublequoteopen}{\isachardot}{\kern0pt}{\isachardot}{\kern0pt}{\isachardot}{\kern0pt}\ {\isacharequal}{\kern0pt}\ halve{\isacharunderscore}{\kern0pt}with{\isacharunderscore}{\kern0pt}parity\ {\isasymcirc}\isactrlsub c\ {\isacharparenleft}{\kern0pt}successor\ {\isasymcirc}\isactrlsub c\ successor{\isacharparenright}{\kern0pt}\ {\isasymcirc}\isactrlsub c\ nth{\isacharunderscore}{\kern0pt}odd{\isachardoublequoteclose}\isanewline
\ \ \ \ \ \ \isacommand{by}\isamarkupfalse%
\ {\isacharparenleft}{\kern0pt}simp\ add{\isacharcolon}{\kern0pt}\ nth{\isacharunderscore}{\kern0pt}odd{\isacharunderscore}{\kern0pt}successor{\isacharparenright}{\kern0pt}\isanewline
\ \ \ \ \isacommand{also}\isamarkupfalse%
\ \isacommand{have}\isamarkupfalse%
\ {\isachardoublequoteopen}{\isachardot}{\kern0pt}{\isachardot}{\kern0pt}{\isachardot}{\kern0pt}\ {\isacharequal}{\kern0pt}\ {\isacharparenleft}{\kern0pt}{\isacharparenleft}{\kern0pt}halve{\isacharunderscore}{\kern0pt}with{\isacharunderscore}{\kern0pt}parity\ {\isasymcirc}\isactrlsub c\ successor{\isacharparenright}{\kern0pt}\ {\isasymcirc}\isactrlsub c\ successor{\isacharparenright}{\kern0pt}\ {\isasymcirc}\isactrlsub c\ nth{\isacharunderscore}{\kern0pt}odd{\isachardoublequoteclose}\isanewline
\ \ \ \ \ \ \isacommand{by}\isamarkupfalse%
\ {\isacharparenleft}{\kern0pt}typecheck{\isacharunderscore}{\kern0pt}cfuncs{\isacharcomma}{\kern0pt}\ simp\ add{\isacharcolon}{\kern0pt}\ comp{\isacharunderscore}{\kern0pt}associative{\isadigit{2}}{\isacharparenright}{\kern0pt}\isanewline
\ \ \ \ \isacommand{also}\isamarkupfalse%
\ \isacommand{have}\isamarkupfalse%
\ {\isachardoublequoteopen}{\isachardot}{\kern0pt}{\isachardot}{\kern0pt}{\isachardot}{\kern0pt}\ {\isacharequal}{\kern0pt}\ {\isacharparenleft}{\kern0pt}{\isacharparenleft}{\kern0pt}right{\isacharunderscore}{\kern0pt}coproj\ {\isasymnat}\isactrlsub c\ {\isasymnat}\isactrlsub c\ {\isasymamalg}\ {\isacharparenleft}{\kern0pt}left{\isacharunderscore}{\kern0pt}coproj\ {\isasymnat}\isactrlsub c\ {\isasymnat}\isactrlsub c\ {\isasymcirc}\isactrlsub c\ successor{\isacharparenright}{\kern0pt}\ {\isasymcirc}\isactrlsub c\ halve{\isacharunderscore}{\kern0pt}with{\isacharunderscore}{\kern0pt}parity{\isacharparenright}{\kern0pt}\ \isanewline
\ \ \ \ \ \ \ \ {\isasymcirc}\isactrlsub c\ successor{\isacharparenright}{\kern0pt}\ {\isasymcirc}\isactrlsub c\ nth{\isacharunderscore}{\kern0pt}odd{\isachardoublequoteclose}\isanewline
\ \ \ \ \ \ \isacommand{by}\isamarkupfalse%
\ {\isacharparenleft}{\kern0pt}simp\ add{\isacharcolon}{\kern0pt}\ halve{\isacharunderscore}{\kern0pt}with{\isacharunderscore}{\kern0pt}parity{\isacharunderscore}{\kern0pt}successor{\isacharparenright}{\kern0pt}\isanewline
\ \ \ \ \isacommand{also}\isamarkupfalse%
\ \isacommand{have}\isamarkupfalse%
\ {\isachardoublequoteopen}{\isachardot}{\kern0pt}{\isachardot}{\kern0pt}{\isachardot}{\kern0pt}\ {\isacharequal}{\kern0pt}\ {\isacharparenleft}{\kern0pt}right{\isacharunderscore}{\kern0pt}coproj\ {\isasymnat}\isactrlsub c\ {\isasymnat}\isactrlsub c\ {\isasymamalg}\ {\isacharparenleft}{\kern0pt}left{\isacharunderscore}{\kern0pt}coproj\ {\isasymnat}\isactrlsub c\ {\isasymnat}\isactrlsub c\ {\isasymcirc}\isactrlsub c\ successor{\isacharparenright}{\kern0pt}\isanewline
\ \ \ \ \ \ \ \ {\isasymcirc}\isactrlsub c\ {\isacharparenleft}{\kern0pt}halve{\isacharunderscore}{\kern0pt}with{\isacharunderscore}{\kern0pt}parity\ {\isasymcirc}\isactrlsub c\ successor{\isacharparenright}{\kern0pt}{\isacharparenright}{\kern0pt}\ {\isasymcirc}\isactrlsub c\ nth{\isacharunderscore}{\kern0pt}odd{\isachardoublequoteclose}\isanewline
\ \ \ \ \ \ \isacommand{by}\isamarkupfalse%
\ {\isacharparenleft}{\kern0pt}typecheck{\isacharunderscore}{\kern0pt}cfuncs{\isacharcomma}{\kern0pt}\ simp\ add{\isacharcolon}{\kern0pt}\ comp{\isacharunderscore}{\kern0pt}associative{\isadigit{2}}{\isacharparenright}{\kern0pt}\isanewline
\ \ \ \ \isacommand{also}\isamarkupfalse%
\ \isacommand{have}\isamarkupfalse%
\ {\isachardoublequoteopen}{\isachardot}{\kern0pt}{\isachardot}{\kern0pt}{\isachardot}{\kern0pt}\ {\isacharequal}{\kern0pt}\ {\isacharparenleft}{\kern0pt}right{\isacharunderscore}{\kern0pt}coproj\ {\isasymnat}\isactrlsub c\ {\isasymnat}\isactrlsub c\ {\isasymamalg}\ {\isacharparenleft}{\kern0pt}left{\isacharunderscore}{\kern0pt}coproj\ {\isasymnat}\isactrlsub c\ {\isasymnat}\isactrlsub c\ {\isasymcirc}\isactrlsub c\ successor{\isacharparenright}{\kern0pt}\isanewline
\ \ \ \ \ \ \ \ {\isasymcirc}\isactrlsub c\ {\isacharparenleft}{\kern0pt}right{\isacharunderscore}{\kern0pt}coproj\ {\isasymnat}\isactrlsub c\ {\isasymnat}\isactrlsub c\ {\isasymamalg}\ {\isacharparenleft}{\kern0pt}left{\isacharunderscore}{\kern0pt}coproj\ {\isasymnat}\isactrlsub c\ {\isasymnat}\isactrlsub c\ {\isasymcirc}\isactrlsub c\ successor{\isacharparenright}{\kern0pt}\ {\isasymcirc}\isactrlsub c\ halve{\isacharunderscore}{\kern0pt}with{\isacharunderscore}{\kern0pt}parity{\isacharparenright}{\kern0pt}{\isacharparenright}{\kern0pt}\ {\isasymcirc}\isactrlsub c\ nth{\isacharunderscore}{\kern0pt}odd{\isachardoublequoteclose}\isanewline
\ \ \ \ \ \ \isacommand{by}\isamarkupfalse%
\ {\isacharparenleft}{\kern0pt}simp\ add{\isacharcolon}{\kern0pt}\ halve{\isacharunderscore}{\kern0pt}with{\isacharunderscore}{\kern0pt}parity{\isacharunderscore}{\kern0pt}successor{\isacharparenright}{\kern0pt}\isanewline
\ \ \ \ \isacommand{also}\isamarkupfalse%
\ \isacommand{have}\isamarkupfalse%
\ {\isachardoublequoteopen}{\isachardot}{\kern0pt}{\isachardot}{\kern0pt}{\isachardot}{\kern0pt}\ {\isacharequal}{\kern0pt}\ {\isacharparenleft}{\kern0pt}right{\isacharunderscore}{\kern0pt}coproj\ {\isasymnat}\isactrlsub c\ {\isasymnat}\isactrlsub c\ {\isasymamalg}\ {\isacharparenleft}{\kern0pt}left{\isacharunderscore}{\kern0pt}coproj\ {\isasymnat}\isactrlsub c\ {\isasymnat}\isactrlsub c\ {\isasymcirc}\isactrlsub c\ successor{\isacharparenright}{\kern0pt}\isanewline
\ \ \ \ \ \ \ \ {\isasymcirc}\isactrlsub c\ right{\isacharunderscore}{\kern0pt}coproj\ {\isasymnat}\isactrlsub c\ {\isasymnat}\isactrlsub c\ {\isasymamalg}\ {\isacharparenleft}{\kern0pt}left{\isacharunderscore}{\kern0pt}coproj\ {\isasymnat}\isactrlsub c\ {\isasymnat}\isactrlsub c\ {\isasymcirc}\isactrlsub c\ successor{\isacharparenright}{\kern0pt}{\isacharparenright}{\kern0pt}\ {\isasymcirc}\isactrlsub c\ halve{\isacharunderscore}{\kern0pt}with{\isacharunderscore}{\kern0pt}parity\ {\isasymcirc}\isactrlsub c\ nth{\isacharunderscore}{\kern0pt}odd{\isachardoublequoteclose}\isanewline
\ \ \ \ \ \ \isacommand{by}\isamarkupfalse%
\ {\isacharparenleft}{\kern0pt}typecheck{\isacharunderscore}{\kern0pt}cfuncs{\isacharcomma}{\kern0pt}\ simp\ add{\isacharcolon}{\kern0pt}\ comp{\isacharunderscore}{\kern0pt}associative{\isadigit{2}}{\isacharparenright}{\kern0pt}\isanewline
\ \ \ \ \isacommand{also}\isamarkupfalse%
\ \isacommand{have}\isamarkupfalse%
\ {\isachardoublequoteopen}{\isachardot}{\kern0pt}{\isachardot}{\kern0pt}{\isachardot}{\kern0pt}\ {\isacharequal}{\kern0pt}\ {\isacharparenleft}{\kern0pt}{\isacharparenleft}{\kern0pt}left{\isacharunderscore}{\kern0pt}coproj\ {\isasymnat}\isactrlsub c\ {\isasymnat}\isactrlsub c\ {\isasymcirc}\isactrlsub c\ successor{\isacharparenright}{\kern0pt}\ {\isasymamalg}\ {\isacharparenleft}{\kern0pt}right{\isacharunderscore}{\kern0pt}coproj\ {\isasymnat}\isactrlsub c\ {\isasymnat}\isactrlsub c\ {\isasymcirc}\isactrlsub c\ successor{\isacharparenright}{\kern0pt}{\isacharparenright}{\kern0pt}\ {\isasymcirc}\isactrlsub c\ halve{\isacharunderscore}{\kern0pt}with{\isacharunderscore}{\kern0pt}parity\ {\isasymcirc}\isactrlsub c\ nth{\isacharunderscore}{\kern0pt}odd{\isachardoublequoteclose}\isanewline
\ \ \ \ \ \ \isacommand{by}\isamarkupfalse%
\ {\isacharparenleft}{\kern0pt}typecheck{\isacharunderscore}{\kern0pt}cfuncs{\isacharcomma}{\kern0pt}\ smt\ cfunc{\isacharunderscore}{\kern0pt}coprod{\isacharunderscore}{\kern0pt}comp\ comp{\isacharunderscore}{\kern0pt}associative{\isadigit{2}}\ left{\isacharunderscore}{\kern0pt}coproj{\isacharunderscore}{\kern0pt}cfunc{\isacharunderscore}{\kern0pt}coprod\ right{\isacharunderscore}{\kern0pt}coproj{\isacharunderscore}{\kern0pt}cfunc{\isacharunderscore}{\kern0pt}coprod{\isacharparenright}{\kern0pt}\isanewline
\ \ \ \ \isacommand{then}\isamarkupfalse%
\ \isacommand{show}\isamarkupfalse%
\ {\isacharquery}{\kern0pt}thesis\isanewline
\ \ \ \ \ \ \isacommand{using}\isamarkupfalse%
\ calculation\ \isacommand{by}\isamarkupfalse%
\ auto\isanewline
\ \ \isacommand{qed}\isamarkupfalse%
\isanewline
\isanewline
\ \ \isacommand{show}\isamarkupfalse%
\ {\isachardoublequoteopen}right{\isacharunderscore}{\kern0pt}coproj\ {\isasymnat}\isactrlsub c\ {\isasymnat}\isactrlsub c\ {\isasymcirc}\isactrlsub c\ successor\ {\isacharequal}{\kern0pt}\isanewline
\ \ \ \ \ \ {\isacharparenleft}{\kern0pt}left{\isacharunderscore}{\kern0pt}coproj\ {\isasymnat}\isactrlsub c\ {\isasymnat}\isactrlsub c\ {\isasymcirc}\isactrlsub c\ successor{\isacharparenright}{\kern0pt}\ {\isasymamalg}\ {\isacharparenleft}{\kern0pt}right{\isacharunderscore}{\kern0pt}coproj\ {\isasymnat}\isactrlsub c\ {\isasymnat}\isactrlsub c\ {\isasymcirc}\isactrlsub c\ successor{\isacharparenright}{\kern0pt}\ {\isasymcirc}\isactrlsub c\ right{\isacharunderscore}{\kern0pt}coproj\ {\isasymnat}\isactrlsub c\ {\isasymnat}\isactrlsub c{\isachardoublequoteclose}\isanewline
\ \ \ \ \isacommand{by}\isamarkupfalse%
\ {\isacharparenleft}{\kern0pt}typecheck{\isacharunderscore}{\kern0pt}cfuncs{\isacharcomma}{\kern0pt}\ simp\ add{\isacharcolon}{\kern0pt}\ right{\isacharunderscore}{\kern0pt}coproj{\isacharunderscore}{\kern0pt}cfunc{\isacharunderscore}{\kern0pt}coprod{\isacharparenright}{\kern0pt}\isanewline
\isacommand{qed}\isamarkupfalse%
%
\endisatagproof
{\isafoldproof}%
%
\isadelimproof
\isanewline
%
\endisadelimproof
\isanewline
\isacommand{lemma}\isamarkupfalse%
\ nth{\isacharunderscore}{\kern0pt}even{\isacharunderscore}{\kern0pt}nth{\isacharunderscore}{\kern0pt}odd{\isacharunderscore}{\kern0pt}halve{\isacharunderscore}{\kern0pt}with{\isacharunderscore}{\kern0pt}parity{\isacharcolon}{\kern0pt}\isanewline
\ \ {\isachardoublequoteopen}{\isacharparenleft}{\kern0pt}nth{\isacharunderscore}{\kern0pt}even\ {\isasymamalg}\ nth{\isacharunderscore}{\kern0pt}odd{\isacharparenright}{\kern0pt}\ {\isasymcirc}\isactrlsub c\ halve{\isacharunderscore}{\kern0pt}with{\isacharunderscore}{\kern0pt}parity\ {\isacharequal}{\kern0pt}\ id\ {\isasymnat}\isactrlsub c{\isachardoublequoteclose}\isanewline
%
\isadelimproof
%
\endisadelimproof
%
\isatagproof
\isacommand{proof}\isamarkupfalse%
\ {\isacharparenleft}{\kern0pt}etcs{\isacharunderscore}{\kern0pt}rule\ natural{\isacharunderscore}{\kern0pt}number{\isacharunderscore}{\kern0pt}object{\isacharunderscore}{\kern0pt}func{\isacharunderscore}{\kern0pt}unique{\isacharbrackleft}{\kern0pt}\isakeyword{where}\ X{\isacharequal}{\kern0pt}{\isachardoublequoteopen}{\isasymnat}\isactrlsub c{\isachardoublequoteclose}{\isacharcomma}{\kern0pt}\ \isakeyword{where}\ f{\isacharequal}{\kern0pt}{\isachardoublequoteopen}successor{\isachardoublequoteclose}{\isacharbrackright}{\kern0pt}{\isacharparenright}{\kern0pt}\isanewline
\ \ \isacommand{show}\isamarkupfalse%
\ {\isachardoublequoteopen}{\isacharparenleft}{\kern0pt}nth{\isacharunderscore}{\kern0pt}even\ {\isasymamalg}\ nth{\isacharunderscore}{\kern0pt}odd\ {\isasymcirc}\isactrlsub c\ halve{\isacharunderscore}{\kern0pt}with{\isacharunderscore}{\kern0pt}parity{\isacharparenright}{\kern0pt}\ {\isasymcirc}\isactrlsub c\ zero\ {\isacharequal}{\kern0pt}\ id\isactrlsub c\ {\isasymnat}\isactrlsub c\ {\isasymcirc}\isactrlsub c\ zero{\isachardoublequoteclose}\isanewline
\ \ \isacommand{proof}\isamarkupfalse%
\ {\isacharminus}{\kern0pt}\isanewline
\ \ \ \ \isacommand{have}\isamarkupfalse%
\ {\isachardoublequoteopen}{\isacharparenleft}{\kern0pt}nth{\isacharunderscore}{\kern0pt}even\ {\isasymamalg}\ nth{\isacharunderscore}{\kern0pt}odd\ {\isasymcirc}\isactrlsub c\ halve{\isacharunderscore}{\kern0pt}with{\isacharunderscore}{\kern0pt}parity{\isacharparenright}{\kern0pt}\ {\isasymcirc}\isactrlsub c\ zero\ {\isacharequal}{\kern0pt}\ nth{\isacharunderscore}{\kern0pt}even\ {\isasymamalg}\ nth{\isacharunderscore}{\kern0pt}odd\ {\isasymcirc}\isactrlsub c\ halve{\isacharunderscore}{\kern0pt}with{\isacharunderscore}{\kern0pt}parity\ {\isasymcirc}\isactrlsub c\ zero{\isachardoublequoteclose}\isanewline
\ \ \ \ \ \ \isacommand{by}\isamarkupfalse%
\ {\isacharparenleft}{\kern0pt}typecheck{\isacharunderscore}{\kern0pt}cfuncs{\isacharcomma}{\kern0pt}\ simp\ add{\isacharcolon}{\kern0pt}\ comp{\isacharunderscore}{\kern0pt}associative{\isadigit{2}}{\isacharparenright}{\kern0pt}\isanewline
\ \ \ \ \isacommand{also}\isamarkupfalse%
\ \isacommand{have}\isamarkupfalse%
\ {\isachardoublequoteopen}{\isachardot}{\kern0pt}{\isachardot}{\kern0pt}{\isachardot}{\kern0pt}\ {\isacharequal}{\kern0pt}\ nth{\isacharunderscore}{\kern0pt}even\ {\isasymamalg}\ nth{\isacharunderscore}{\kern0pt}odd\ {\isasymcirc}\isactrlsub c\ left{\isacharunderscore}{\kern0pt}coproj\ {\isasymnat}\isactrlsub c\ {\isasymnat}\isactrlsub c\ {\isasymcirc}\isactrlsub c\ zero{\isachardoublequoteclose}\isanewline
\ \ \ \ \ \ \isacommand{by}\isamarkupfalse%
\ {\isacharparenleft}{\kern0pt}simp\ add{\isacharcolon}{\kern0pt}\ halve{\isacharunderscore}{\kern0pt}with{\isacharunderscore}{\kern0pt}parity{\isacharunderscore}{\kern0pt}zero{\isacharparenright}{\kern0pt}\isanewline
\ \ \ \ \isacommand{also}\isamarkupfalse%
\ \isacommand{have}\isamarkupfalse%
\ {\isachardoublequoteopen}{\isachardot}{\kern0pt}{\isachardot}{\kern0pt}{\isachardot}{\kern0pt}\ {\isacharequal}{\kern0pt}\ {\isacharparenleft}{\kern0pt}nth{\isacharunderscore}{\kern0pt}even\ {\isasymamalg}\ nth{\isacharunderscore}{\kern0pt}odd\ {\isasymcirc}\isactrlsub c\ left{\isacharunderscore}{\kern0pt}coproj\ {\isasymnat}\isactrlsub c\ {\isasymnat}\isactrlsub c{\isacharparenright}{\kern0pt}\ {\isasymcirc}\isactrlsub c\ zero{\isachardoublequoteclose}\isanewline
\ \ \ \ \ \ \isacommand{by}\isamarkupfalse%
\ {\isacharparenleft}{\kern0pt}typecheck{\isacharunderscore}{\kern0pt}cfuncs{\isacharcomma}{\kern0pt}\ simp\ add{\isacharcolon}{\kern0pt}\ comp{\isacharunderscore}{\kern0pt}associative{\isadigit{2}}{\isacharparenright}{\kern0pt}\isanewline
\ \ \ \ \isacommand{also}\isamarkupfalse%
\ \isacommand{have}\isamarkupfalse%
\ {\isachardoublequoteopen}{\isachardot}{\kern0pt}{\isachardot}{\kern0pt}{\isachardot}{\kern0pt}\ {\isacharequal}{\kern0pt}\ nth{\isacharunderscore}{\kern0pt}even\ {\isasymcirc}\isactrlsub c\ zero{\isachardoublequoteclose}\isanewline
\ \ \ \ \ \ \isacommand{by}\isamarkupfalse%
\ {\isacharparenleft}{\kern0pt}typecheck{\isacharunderscore}{\kern0pt}cfuncs{\isacharcomma}{\kern0pt}\ simp\ add{\isacharcolon}{\kern0pt}\ left{\isacharunderscore}{\kern0pt}coproj{\isacharunderscore}{\kern0pt}cfunc{\isacharunderscore}{\kern0pt}coprod{\isacharparenright}{\kern0pt}\isanewline
\ \ \ \ \isacommand{also}\isamarkupfalse%
\ \isacommand{have}\isamarkupfalse%
\ {\isachardoublequoteopen}{\isachardot}{\kern0pt}{\isachardot}{\kern0pt}{\isachardot}{\kern0pt}\ {\isacharequal}{\kern0pt}\ id\isactrlsub c\ {\isasymnat}\isactrlsub c\ {\isasymcirc}\isactrlsub c\ zero{\isachardoublequoteclose}\isanewline
\ \ \ \ \ \ \isacommand{using}\isamarkupfalse%
\ id{\isacharunderscore}{\kern0pt}left{\isacharunderscore}{\kern0pt}unit{\isadigit{2}}\ nth{\isacharunderscore}{\kern0pt}even{\isacharunderscore}{\kern0pt}def{\isadigit{2}}\ zero{\isacharunderscore}{\kern0pt}type\ \isacommand{by}\isamarkupfalse%
\ auto\isanewline
\ \ \ \ \isacommand{then}\isamarkupfalse%
\ \isacommand{show}\isamarkupfalse%
\ {\isacharquery}{\kern0pt}thesis\isanewline
\ \ \ \ \ \ \isacommand{using}\isamarkupfalse%
\ calculation\ \isacommand{by}\isamarkupfalse%
\ auto\isanewline
\ \ \isacommand{qed}\isamarkupfalse%
\isanewline
\isanewline
\ \ \isacommand{show}\isamarkupfalse%
\ {\isachardoublequoteopen}{\isacharparenleft}{\kern0pt}nth{\isacharunderscore}{\kern0pt}even\ {\isasymamalg}\ nth{\isacharunderscore}{\kern0pt}odd\ {\isasymcirc}\isactrlsub c\ halve{\isacharunderscore}{\kern0pt}with{\isacharunderscore}{\kern0pt}parity{\isacharparenright}{\kern0pt}\ {\isasymcirc}\isactrlsub c\ successor\ {\isacharequal}{\kern0pt}\isanewline
\ \ \ \ successor\ {\isasymcirc}\isactrlsub c\ nth{\isacharunderscore}{\kern0pt}even\ {\isasymamalg}\ nth{\isacharunderscore}{\kern0pt}odd\ {\isasymcirc}\isactrlsub c\ halve{\isacharunderscore}{\kern0pt}with{\isacharunderscore}{\kern0pt}parity{\isachardoublequoteclose}\isanewline
\ \ \isacommand{proof}\isamarkupfalse%
\ {\isacharminus}{\kern0pt}\isanewline
\ \ \ \ \isacommand{have}\isamarkupfalse%
\ {\isachardoublequoteopen}{\isacharparenleft}{\kern0pt}nth{\isacharunderscore}{\kern0pt}even\ {\isasymamalg}\ nth{\isacharunderscore}{\kern0pt}odd\ {\isasymcirc}\isactrlsub c\ halve{\isacharunderscore}{\kern0pt}with{\isacharunderscore}{\kern0pt}parity{\isacharparenright}{\kern0pt}\ {\isasymcirc}\isactrlsub c\ successor\ {\isacharequal}{\kern0pt}\ nth{\isacharunderscore}{\kern0pt}even\ {\isasymamalg}\ nth{\isacharunderscore}{\kern0pt}odd\ {\isasymcirc}\isactrlsub c\ halve{\isacharunderscore}{\kern0pt}with{\isacharunderscore}{\kern0pt}parity\ {\isasymcirc}\isactrlsub c\ successor{\isachardoublequoteclose}\isanewline
\ \ \ \ \ \ \isacommand{by}\isamarkupfalse%
\ {\isacharparenleft}{\kern0pt}typecheck{\isacharunderscore}{\kern0pt}cfuncs{\isacharcomma}{\kern0pt}\ simp\ add{\isacharcolon}{\kern0pt}\ comp{\isacharunderscore}{\kern0pt}associative{\isadigit{2}}{\isacharparenright}{\kern0pt}\isanewline
\ \ \ \ \isacommand{also}\isamarkupfalse%
\ \isacommand{have}\isamarkupfalse%
\ {\isachardoublequoteopen}{\isachardot}{\kern0pt}{\isachardot}{\kern0pt}{\isachardot}{\kern0pt}\ {\isacharequal}{\kern0pt}\ nth{\isacharunderscore}{\kern0pt}even\ {\isasymamalg}\ nth{\isacharunderscore}{\kern0pt}odd\ {\isasymcirc}\isactrlsub c\ right{\isacharunderscore}{\kern0pt}coproj\ {\isasymnat}\isactrlsub c\ {\isasymnat}\isactrlsub c\ {\isasymamalg}\ {\isacharparenleft}{\kern0pt}left{\isacharunderscore}{\kern0pt}coproj\ {\isasymnat}\isactrlsub c\ {\isasymnat}\isactrlsub c\ {\isasymcirc}\isactrlsub c\ successor{\isacharparenright}{\kern0pt}\ {\isasymcirc}\isactrlsub c\ halve{\isacharunderscore}{\kern0pt}with{\isacharunderscore}{\kern0pt}parity{\isachardoublequoteclose}\isanewline
\ \ \ \ \ \ \isacommand{by}\isamarkupfalse%
\ {\isacharparenleft}{\kern0pt}simp\ add{\isacharcolon}{\kern0pt}\ halve{\isacharunderscore}{\kern0pt}with{\isacharunderscore}{\kern0pt}parity{\isacharunderscore}{\kern0pt}successor{\isacharparenright}{\kern0pt}\isanewline
\ \ \ \ \isacommand{also}\isamarkupfalse%
\ \isacommand{have}\isamarkupfalse%
\ {\isachardoublequoteopen}{\isachardot}{\kern0pt}{\isachardot}{\kern0pt}{\isachardot}{\kern0pt}\ {\isacharequal}{\kern0pt}\ {\isacharparenleft}{\kern0pt}nth{\isacharunderscore}{\kern0pt}even\ {\isasymamalg}\ nth{\isacharunderscore}{\kern0pt}odd\ {\isasymcirc}\isactrlsub c\ right{\isacharunderscore}{\kern0pt}coproj\ {\isasymnat}\isactrlsub c\ {\isasymnat}\isactrlsub c\ {\isasymamalg}\ {\isacharparenleft}{\kern0pt}left{\isacharunderscore}{\kern0pt}coproj\ {\isasymnat}\isactrlsub c\ {\isasymnat}\isactrlsub c\ {\isasymcirc}\isactrlsub c\ successor{\isacharparenright}{\kern0pt}{\isacharparenright}{\kern0pt}\ {\isasymcirc}\isactrlsub c\ halve{\isacharunderscore}{\kern0pt}with{\isacharunderscore}{\kern0pt}parity{\isachardoublequoteclose}\isanewline
\ \ \ \ \ \ \isacommand{by}\isamarkupfalse%
\ {\isacharparenleft}{\kern0pt}typecheck{\isacharunderscore}{\kern0pt}cfuncs{\isacharcomma}{\kern0pt}\ simp\ add{\isacharcolon}{\kern0pt}\ comp{\isacharunderscore}{\kern0pt}associative{\isadigit{2}}{\isacharparenright}{\kern0pt}\isanewline
\ \ \ \ \isacommand{also}\isamarkupfalse%
\ \isacommand{have}\isamarkupfalse%
\ {\isachardoublequoteopen}{\isachardot}{\kern0pt}{\isachardot}{\kern0pt}{\isachardot}{\kern0pt}\ {\isacharequal}{\kern0pt}\ nth{\isacharunderscore}{\kern0pt}odd\ {\isasymamalg}\ {\isacharparenleft}{\kern0pt}nth{\isacharunderscore}{\kern0pt}even\ {\isasymcirc}\isactrlsub c\ successor{\isacharparenright}{\kern0pt}\ {\isasymcirc}\isactrlsub c\ halve{\isacharunderscore}{\kern0pt}with{\isacharunderscore}{\kern0pt}parity{\isachardoublequoteclose}\isanewline
\ \ \ \ \ \ \isacommand{by}\isamarkupfalse%
\ {\isacharparenleft}{\kern0pt}typecheck{\isacharunderscore}{\kern0pt}cfuncs{\isacharcomma}{\kern0pt}\ smt\ cfunc{\isacharunderscore}{\kern0pt}coprod{\isacharunderscore}{\kern0pt}comp\ comp{\isacharunderscore}{\kern0pt}associative{\isadigit{2}}\ left{\isacharunderscore}{\kern0pt}coproj{\isacharunderscore}{\kern0pt}cfunc{\isacharunderscore}{\kern0pt}coprod\ right{\isacharunderscore}{\kern0pt}coproj{\isacharunderscore}{\kern0pt}cfunc{\isacharunderscore}{\kern0pt}coprod{\isacharparenright}{\kern0pt}\isanewline
\ \ \ \ \isacommand{also}\isamarkupfalse%
\ \isacommand{have}\isamarkupfalse%
\ {\isachardoublequoteopen}{\isachardot}{\kern0pt}{\isachardot}{\kern0pt}{\isachardot}{\kern0pt}\ {\isacharequal}{\kern0pt}\ {\isacharparenleft}{\kern0pt}successor\ {\isasymcirc}\isactrlsub c\ nth{\isacharunderscore}{\kern0pt}even{\isacharparenright}{\kern0pt}\ {\isasymamalg}\ {\isacharparenleft}{\kern0pt}{\isacharparenleft}{\kern0pt}successor\ {\isasymcirc}\isactrlsub c\ successor{\isacharparenright}{\kern0pt}\ {\isasymcirc}\isactrlsub c\ nth{\isacharunderscore}{\kern0pt}even{\isacharparenright}{\kern0pt}\ {\isasymcirc}\isactrlsub c\ halve{\isacharunderscore}{\kern0pt}with{\isacharunderscore}{\kern0pt}parity{\isachardoublequoteclose}\isanewline
\ \ \ \ \ \ \isacommand{by}\isamarkupfalse%
\ {\isacharparenleft}{\kern0pt}simp\ add{\isacharcolon}{\kern0pt}\ nth{\isacharunderscore}{\kern0pt}even{\isacharunderscore}{\kern0pt}successor\ nth{\isacharunderscore}{\kern0pt}odd{\isacharunderscore}{\kern0pt}is{\isacharunderscore}{\kern0pt}succ{\isacharunderscore}{\kern0pt}nth{\isacharunderscore}{\kern0pt}even{\isacharparenright}{\kern0pt}\isanewline
\ \ \ \ \isacommand{also}\isamarkupfalse%
\ \isacommand{have}\isamarkupfalse%
\ {\isachardoublequoteopen}{\isachardot}{\kern0pt}{\isachardot}{\kern0pt}{\isachardot}{\kern0pt}\ {\isacharequal}{\kern0pt}\ {\isacharparenleft}{\kern0pt}successor\ {\isasymcirc}\isactrlsub c\ nth{\isacharunderscore}{\kern0pt}even{\isacharparenright}{\kern0pt}\ {\isasymamalg}\ {\isacharparenleft}{\kern0pt}successor\ {\isasymcirc}\isactrlsub c\ successor\ {\isasymcirc}\isactrlsub c\ nth{\isacharunderscore}{\kern0pt}even{\isacharparenright}{\kern0pt}\ {\isasymcirc}\isactrlsub c\ halve{\isacharunderscore}{\kern0pt}with{\isacharunderscore}{\kern0pt}parity{\isachardoublequoteclose}\isanewline
\ \ \ \ \ \ \isacommand{by}\isamarkupfalse%
\ {\isacharparenleft}{\kern0pt}typecheck{\isacharunderscore}{\kern0pt}cfuncs{\isacharcomma}{\kern0pt}\ simp\ add{\isacharcolon}{\kern0pt}\ comp{\isacharunderscore}{\kern0pt}associative{\isadigit{2}}{\isacharparenright}{\kern0pt}\isanewline
\ \ \ \ \isacommand{also}\isamarkupfalse%
\ \isacommand{have}\isamarkupfalse%
\ {\isachardoublequoteopen}{\isachardot}{\kern0pt}{\isachardot}{\kern0pt}{\isachardot}{\kern0pt}\ {\isacharequal}{\kern0pt}\ {\isacharparenleft}{\kern0pt}successor\ {\isasymcirc}\isactrlsub c\ nth{\isacharunderscore}{\kern0pt}even{\isacharparenright}{\kern0pt}\ {\isasymamalg}\ {\isacharparenleft}{\kern0pt}successor\ {\isasymcirc}\isactrlsub c\ nth{\isacharunderscore}{\kern0pt}odd{\isacharparenright}{\kern0pt}\ {\isasymcirc}\isactrlsub c\ halve{\isacharunderscore}{\kern0pt}with{\isacharunderscore}{\kern0pt}parity{\isachardoublequoteclose}\isanewline
\ \ \ \ \ \ \isacommand{by}\isamarkupfalse%
\ {\isacharparenleft}{\kern0pt}simp\ add{\isacharcolon}{\kern0pt}\ nth{\isacharunderscore}{\kern0pt}odd{\isacharunderscore}{\kern0pt}is{\isacharunderscore}{\kern0pt}succ{\isacharunderscore}{\kern0pt}nth{\isacharunderscore}{\kern0pt}even{\isacharparenright}{\kern0pt}\isanewline
\ \ \ \ \isacommand{also}\isamarkupfalse%
\ \isacommand{have}\isamarkupfalse%
\ {\isachardoublequoteopen}{\isachardot}{\kern0pt}{\isachardot}{\kern0pt}{\isachardot}{\kern0pt}\ {\isacharequal}{\kern0pt}\ successor\ {\isasymcirc}\isactrlsub c\ nth{\isacharunderscore}{\kern0pt}even\ {\isasymamalg}\ nth{\isacharunderscore}{\kern0pt}odd\ {\isasymcirc}\isactrlsub c\ halve{\isacharunderscore}{\kern0pt}with{\isacharunderscore}{\kern0pt}parity{\isachardoublequoteclose}\isanewline
\ \ \ \ \ \ \isacommand{by}\isamarkupfalse%
\ {\isacharparenleft}{\kern0pt}typecheck{\isacharunderscore}{\kern0pt}cfuncs{\isacharcomma}{\kern0pt}\ simp\ add{\isacharcolon}{\kern0pt}\ cfunc{\isacharunderscore}{\kern0pt}coprod{\isacharunderscore}{\kern0pt}comp\ comp{\isacharunderscore}{\kern0pt}associative{\isadigit{2}}{\isacharparenright}{\kern0pt}\isanewline
\ \ \ \ \isacommand{then}\isamarkupfalse%
\ \isacommand{show}\isamarkupfalse%
\ {\isacharquery}{\kern0pt}thesis\isanewline
\ \ \ \ \ \ \isacommand{using}\isamarkupfalse%
\ calculation\ \isacommand{by}\isamarkupfalse%
\ auto\isanewline
\ \ \isacommand{qed}\isamarkupfalse%
\isanewline
\isanewline
\ \ \isacommand{show}\isamarkupfalse%
\ {\isachardoublequoteopen}id\isactrlsub c\ {\isasymnat}\isactrlsub c\ {\isasymcirc}\isactrlsub c\ successor\ {\isacharequal}{\kern0pt}\ successor\ {\isasymcirc}\isactrlsub c\ id\isactrlsub c\ {\isasymnat}\isactrlsub c{\isachardoublequoteclose}\isanewline
\ \ \ \ \isacommand{using}\isamarkupfalse%
\ id{\isacharunderscore}{\kern0pt}left{\isacharunderscore}{\kern0pt}unit{\isadigit{2}}\ id{\isacharunderscore}{\kern0pt}right{\isacharunderscore}{\kern0pt}unit{\isadigit{2}}\ successor{\isacharunderscore}{\kern0pt}type\ \isacommand{by}\isamarkupfalse%
\ auto\isanewline
\isacommand{qed}\isamarkupfalse%
%
\endisatagproof
{\isafoldproof}%
%
\isadelimproof
\isanewline
%
\endisadelimproof
\isanewline
\isacommand{lemma}\isamarkupfalse%
\ halve{\isacharunderscore}{\kern0pt}with{\isacharunderscore}{\kern0pt}parity{\isacharunderscore}{\kern0pt}nth{\isacharunderscore}{\kern0pt}even{\isacharunderscore}{\kern0pt}nth{\isacharunderscore}{\kern0pt}odd{\isacharcolon}{\kern0pt}\isanewline
\ \ {\isachardoublequoteopen}halve{\isacharunderscore}{\kern0pt}with{\isacharunderscore}{\kern0pt}parity\ {\isasymcirc}\isactrlsub c\ {\isacharparenleft}{\kern0pt}nth{\isacharunderscore}{\kern0pt}even\ {\isasymamalg}\ nth{\isacharunderscore}{\kern0pt}odd{\isacharparenright}{\kern0pt}\ {\isacharequal}{\kern0pt}\ id\ {\isacharparenleft}{\kern0pt}{\isasymnat}\isactrlsub c\ {\isasymCoprod}\ {\isasymnat}\isactrlsub c{\isacharparenright}{\kern0pt}{\isachardoublequoteclose}\isanewline
%
\isadelimproof
\ \ %
\endisadelimproof
%
\isatagproof
\isacommand{by}\isamarkupfalse%
\ {\isacharparenleft}{\kern0pt}typecheck{\isacharunderscore}{\kern0pt}cfuncs{\isacharcomma}{\kern0pt}\ smt\ cfunc{\isacharunderscore}{\kern0pt}coprod{\isacharunderscore}{\kern0pt}comp\ halve{\isacharunderscore}{\kern0pt}with{\isacharunderscore}{\kern0pt}parity{\isacharunderscore}{\kern0pt}nth{\isacharunderscore}{\kern0pt}even\ halve{\isacharunderscore}{\kern0pt}with{\isacharunderscore}{\kern0pt}parity{\isacharunderscore}{\kern0pt}nth{\isacharunderscore}{\kern0pt}odd\ id{\isacharunderscore}{\kern0pt}coprod{\isacharparenright}{\kern0pt}%
\endisatagproof
{\isafoldproof}%
%
\isadelimproof
\isanewline
%
\endisadelimproof
\isanewline
\isacommand{lemma}\isamarkupfalse%
\ even{\isacharunderscore}{\kern0pt}odd{\isacharunderscore}{\kern0pt}iso{\isacharcolon}{\kern0pt}\isanewline
\ \ {\isachardoublequoteopen}isomorphism\ {\isacharparenleft}{\kern0pt}nth{\isacharunderscore}{\kern0pt}even\ {\isasymamalg}\ nth{\isacharunderscore}{\kern0pt}odd{\isacharparenright}{\kern0pt}{\isachardoublequoteclose}\isanewline
%
\isadelimproof
\ \ %
\endisadelimproof
%
\isatagproof
\isacommand{unfolding}\isamarkupfalse%
\ isomorphism{\isacharunderscore}{\kern0pt}def\isanewline
\isacommand{proof}\isamarkupfalse%
\ {\isacharparenleft}{\kern0pt}intro\ exI{\isacharbrackleft}{\kern0pt}\isakeyword{where}\ x{\isacharequal}{\kern0pt}halve{\isacharunderscore}{\kern0pt}with{\isacharunderscore}{\kern0pt}parity{\isacharbrackright}{\kern0pt}{\isacharcomma}{\kern0pt}\ safe{\isacharparenright}{\kern0pt}\isanewline
\ \ \isacommand{show}\isamarkupfalse%
\ {\isachardoublequoteopen}domain\ halve{\isacharunderscore}{\kern0pt}with{\isacharunderscore}{\kern0pt}parity\ {\isacharequal}{\kern0pt}\ codomain\ {\isacharparenleft}{\kern0pt}nth{\isacharunderscore}{\kern0pt}even\ {\isasymamalg}\ nth{\isacharunderscore}{\kern0pt}odd{\isacharparenright}{\kern0pt}{\isachardoublequoteclose}\isanewline
\ \ \ \ \isacommand{by}\isamarkupfalse%
\ {\isacharparenleft}{\kern0pt}typecheck{\isacharunderscore}{\kern0pt}cfuncs{\isacharcomma}{\kern0pt}\ unfold\ cfunc{\isacharunderscore}{\kern0pt}type{\isacharunderscore}{\kern0pt}def{\isacharcomma}{\kern0pt}\ auto{\isacharparenright}{\kern0pt}\isanewline
\ \ \isacommand{show}\isamarkupfalse%
\ {\isachardoublequoteopen}codomain\ halve{\isacharunderscore}{\kern0pt}with{\isacharunderscore}{\kern0pt}parity\ {\isacharequal}{\kern0pt}\ domain\ {\isacharparenleft}{\kern0pt}nth{\isacharunderscore}{\kern0pt}even\ {\isasymamalg}\ nth{\isacharunderscore}{\kern0pt}odd{\isacharparenright}{\kern0pt}{\isachardoublequoteclose}\isanewline
\ \ \ \ \isacommand{by}\isamarkupfalse%
\ {\isacharparenleft}{\kern0pt}typecheck{\isacharunderscore}{\kern0pt}cfuncs{\isacharcomma}{\kern0pt}\ unfold\ cfunc{\isacharunderscore}{\kern0pt}type{\isacharunderscore}{\kern0pt}def{\isacharcomma}{\kern0pt}\ auto{\isacharparenright}{\kern0pt}\isanewline
\ \ \isacommand{show}\isamarkupfalse%
\ {\isachardoublequoteopen}halve{\isacharunderscore}{\kern0pt}with{\isacharunderscore}{\kern0pt}parity\ {\isasymcirc}\isactrlsub c\ nth{\isacharunderscore}{\kern0pt}even\ {\isasymamalg}\ nth{\isacharunderscore}{\kern0pt}odd\ {\isacharequal}{\kern0pt}\ id\isactrlsub c\ {\isacharparenleft}{\kern0pt}domain\ {\isacharparenleft}{\kern0pt}nth{\isacharunderscore}{\kern0pt}even\ {\isasymamalg}\ nth{\isacharunderscore}{\kern0pt}odd{\isacharparenright}{\kern0pt}{\isacharparenright}{\kern0pt}{\isachardoublequoteclose}\isanewline
\ \ \ \ \isacommand{by}\isamarkupfalse%
\ {\isacharparenleft}{\kern0pt}typecheck{\isacharunderscore}{\kern0pt}cfuncs{\isacharcomma}{\kern0pt}\ unfold\ cfunc{\isacharunderscore}{\kern0pt}type{\isacharunderscore}{\kern0pt}def{\isacharcomma}{\kern0pt}\ auto\ simp\ add{\isacharcolon}{\kern0pt}\ halve{\isacharunderscore}{\kern0pt}with{\isacharunderscore}{\kern0pt}parity{\isacharunderscore}{\kern0pt}nth{\isacharunderscore}{\kern0pt}even{\isacharunderscore}{\kern0pt}nth{\isacharunderscore}{\kern0pt}odd{\isacharparenright}{\kern0pt}\isanewline
\ \ \isacommand{show}\isamarkupfalse%
\ {\isachardoublequoteopen}nth{\isacharunderscore}{\kern0pt}even\ {\isasymamalg}\ nth{\isacharunderscore}{\kern0pt}odd\ {\isasymcirc}\isactrlsub c\ halve{\isacharunderscore}{\kern0pt}with{\isacharunderscore}{\kern0pt}parity\ {\isacharequal}{\kern0pt}\ id\isactrlsub c\ {\isacharparenleft}{\kern0pt}domain\ halve{\isacharunderscore}{\kern0pt}with{\isacharunderscore}{\kern0pt}parity{\isacharparenright}{\kern0pt}{\isachardoublequoteclose}\isanewline
\ \ \ \ \isacommand{by}\isamarkupfalse%
\ {\isacharparenleft}{\kern0pt}typecheck{\isacharunderscore}{\kern0pt}cfuncs{\isacharcomma}{\kern0pt}\ unfold\ cfunc{\isacharunderscore}{\kern0pt}type{\isacharunderscore}{\kern0pt}def{\isacharcomma}{\kern0pt}\ auto\ simp\ add{\isacharcolon}{\kern0pt}\ nth{\isacharunderscore}{\kern0pt}even{\isacharunderscore}{\kern0pt}nth{\isacharunderscore}{\kern0pt}odd{\isacharunderscore}{\kern0pt}halve{\isacharunderscore}{\kern0pt}with{\isacharunderscore}{\kern0pt}parity{\isacharparenright}{\kern0pt}\isanewline
\isacommand{qed}\isamarkupfalse%
%
\endisatagproof
{\isafoldproof}%
%
\isadelimproof
\isanewline
%
\endisadelimproof
\isanewline
\isacommand{lemma}\isamarkupfalse%
\ halve{\isacharunderscore}{\kern0pt}with{\isacharunderscore}{\kern0pt}parity{\isacharunderscore}{\kern0pt}iso{\isacharcolon}{\kern0pt}\isanewline
\ \ {\isachardoublequoteopen}isomorphism\ halve{\isacharunderscore}{\kern0pt}with{\isacharunderscore}{\kern0pt}parity{\isachardoublequoteclose}\isanewline
%
\isadelimproof
\ \ \ %
\endisadelimproof
%
\isatagproof
\isacommand{unfolding}\isamarkupfalse%
\ isomorphism{\isacharunderscore}{\kern0pt}def\isanewline
\isacommand{proof}\isamarkupfalse%
\ {\isacharparenleft}{\kern0pt}intro\ exI{\isacharbrackleft}{\kern0pt}\isakeyword{where}\ x{\isacharequal}{\kern0pt}{\isachardoublequoteopen}nth{\isacharunderscore}{\kern0pt}even\ {\isasymamalg}\ nth{\isacharunderscore}{\kern0pt}odd{\isachardoublequoteclose}{\isacharbrackright}{\kern0pt}{\isacharcomma}{\kern0pt}\ safe{\isacharparenright}{\kern0pt}\isanewline
\ \ \isacommand{show}\isamarkupfalse%
\ {\isachardoublequoteopen}domain\ {\isacharparenleft}{\kern0pt}nth{\isacharunderscore}{\kern0pt}even\ {\isasymamalg}\ nth{\isacharunderscore}{\kern0pt}odd{\isacharparenright}{\kern0pt}\ {\isacharequal}{\kern0pt}\ codomain\ halve{\isacharunderscore}{\kern0pt}with{\isacharunderscore}{\kern0pt}parity{\isachardoublequoteclose}\isanewline
\ \ \ \ \isacommand{by}\isamarkupfalse%
\ {\isacharparenleft}{\kern0pt}typecheck{\isacharunderscore}{\kern0pt}cfuncs{\isacharcomma}{\kern0pt}\ unfold\ cfunc{\isacharunderscore}{\kern0pt}type{\isacharunderscore}{\kern0pt}def{\isacharcomma}{\kern0pt}\ auto{\isacharparenright}{\kern0pt}\isanewline
\ \ \isacommand{show}\isamarkupfalse%
\ {\isachardoublequoteopen}codomain\ {\isacharparenleft}{\kern0pt}nth{\isacharunderscore}{\kern0pt}even\ {\isasymamalg}\ nth{\isacharunderscore}{\kern0pt}odd{\isacharparenright}{\kern0pt}\ {\isacharequal}{\kern0pt}\ domain\ halve{\isacharunderscore}{\kern0pt}with{\isacharunderscore}{\kern0pt}parity{\isachardoublequoteclose}\isanewline
\ \ \ \ \isacommand{by}\isamarkupfalse%
\ {\isacharparenleft}{\kern0pt}typecheck{\isacharunderscore}{\kern0pt}cfuncs{\isacharcomma}{\kern0pt}\ unfold\ cfunc{\isacharunderscore}{\kern0pt}type{\isacharunderscore}{\kern0pt}def{\isacharcomma}{\kern0pt}\ auto{\isacharparenright}{\kern0pt}\isanewline
\ \ \isacommand{show}\isamarkupfalse%
\ {\isachardoublequoteopen}nth{\isacharunderscore}{\kern0pt}even\ {\isasymamalg}\ nth{\isacharunderscore}{\kern0pt}odd\ {\isasymcirc}\isactrlsub c\ halve{\isacharunderscore}{\kern0pt}with{\isacharunderscore}{\kern0pt}parity\ {\isacharequal}{\kern0pt}\ id\isactrlsub c\ {\isacharparenleft}{\kern0pt}domain\ halve{\isacharunderscore}{\kern0pt}with{\isacharunderscore}{\kern0pt}parity{\isacharparenright}{\kern0pt}{\isachardoublequoteclose}\isanewline
\ \ \ \ \isacommand{by}\isamarkupfalse%
\ {\isacharparenleft}{\kern0pt}typecheck{\isacharunderscore}{\kern0pt}cfuncs{\isacharcomma}{\kern0pt}\ unfold\ cfunc{\isacharunderscore}{\kern0pt}type{\isacharunderscore}{\kern0pt}def{\isacharcomma}{\kern0pt}\ auto\ simp\ add{\isacharcolon}{\kern0pt}\ nth{\isacharunderscore}{\kern0pt}even{\isacharunderscore}{\kern0pt}nth{\isacharunderscore}{\kern0pt}odd{\isacharunderscore}{\kern0pt}halve{\isacharunderscore}{\kern0pt}with{\isacharunderscore}{\kern0pt}parity{\isacharparenright}{\kern0pt}\isanewline
\ \ \isacommand{show}\isamarkupfalse%
\ {\isachardoublequoteopen}halve{\isacharunderscore}{\kern0pt}with{\isacharunderscore}{\kern0pt}parity\ {\isasymcirc}\isactrlsub c\ nth{\isacharunderscore}{\kern0pt}even\ {\isasymamalg}\ nth{\isacharunderscore}{\kern0pt}odd\ {\isacharequal}{\kern0pt}\ id\isactrlsub c\ {\isacharparenleft}{\kern0pt}domain\ {\isacharparenleft}{\kern0pt}nth{\isacharunderscore}{\kern0pt}even\ {\isasymamalg}\ nth{\isacharunderscore}{\kern0pt}odd{\isacharparenright}{\kern0pt}{\isacharparenright}{\kern0pt}{\isachardoublequoteclose}\isanewline
\ \ \ \ \isacommand{by}\isamarkupfalse%
\ {\isacharparenleft}{\kern0pt}typecheck{\isacharunderscore}{\kern0pt}cfuncs{\isacharcomma}{\kern0pt}\ unfold\ cfunc{\isacharunderscore}{\kern0pt}type{\isacharunderscore}{\kern0pt}def{\isacharcomma}{\kern0pt}\ auto\ simp\ add{\isacharcolon}{\kern0pt}\ halve{\isacharunderscore}{\kern0pt}with{\isacharunderscore}{\kern0pt}parity{\isacharunderscore}{\kern0pt}nth{\isacharunderscore}{\kern0pt}even{\isacharunderscore}{\kern0pt}nth{\isacharunderscore}{\kern0pt}odd{\isacharparenright}{\kern0pt}\isanewline
\isacommand{qed}\isamarkupfalse%
%
\endisatagproof
{\isafoldproof}%
%
\isadelimproof
\isanewline
%
\endisadelimproof
\isanewline
\isacommand{definition}\isamarkupfalse%
\ halve\ {\isacharcolon}{\kern0pt}{\isacharcolon}{\kern0pt}\ {\isachardoublequoteopen}cfunc{\isachardoublequoteclose}\ \isakeyword{where}\isanewline
\ \ {\isachardoublequoteopen}halve\ {\isacharequal}{\kern0pt}\ {\isacharparenleft}{\kern0pt}id\ {\isasymnat}\isactrlsub c\ {\isasymamalg}\ id\ {\isasymnat}\isactrlsub c{\isacharparenright}{\kern0pt}\ {\isasymcirc}\isactrlsub c\ halve{\isacharunderscore}{\kern0pt}with{\isacharunderscore}{\kern0pt}parity{\isachardoublequoteclose}\isanewline
\isanewline
\isacommand{lemma}\isamarkupfalse%
\ halve{\isacharunderscore}{\kern0pt}type{\isacharbrackleft}{\kern0pt}type{\isacharunderscore}{\kern0pt}rule{\isacharbrackright}{\kern0pt}{\isacharcolon}{\kern0pt}\isanewline
\ \ {\isachardoublequoteopen}halve\ {\isacharcolon}{\kern0pt}\ {\isasymnat}\isactrlsub c\ {\isasymrightarrow}\ {\isasymnat}\isactrlsub c{\isachardoublequoteclose}\isanewline
%
\isadelimproof
\ \ %
\endisadelimproof
%
\isatagproof
\isacommand{unfolding}\isamarkupfalse%
\ halve{\isacharunderscore}{\kern0pt}def\ \isacommand{by}\isamarkupfalse%
\ typecheck{\isacharunderscore}{\kern0pt}cfuncs%
\endisatagproof
{\isafoldproof}%
%
\isadelimproof
\isanewline
%
\endisadelimproof
\isanewline
\isacommand{lemma}\isamarkupfalse%
\ halve{\isacharunderscore}{\kern0pt}nth{\isacharunderscore}{\kern0pt}even{\isacharcolon}{\kern0pt}\isanewline
\ \ {\isachardoublequoteopen}halve\ {\isasymcirc}\isactrlsub c\ nth{\isacharunderscore}{\kern0pt}even\ {\isacharequal}{\kern0pt}\ id\ {\isasymnat}\isactrlsub c{\isachardoublequoteclose}\isanewline
%
\isadelimproof
\ \ %
\endisadelimproof
%
\isatagproof
\isacommand{unfolding}\isamarkupfalse%
\ halve{\isacharunderscore}{\kern0pt}def\ \isacommand{by}\isamarkupfalse%
\ {\isacharparenleft}{\kern0pt}typecheck{\isacharunderscore}{\kern0pt}cfuncs{\isacharcomma}{\kern0pt}\ smt\ comp{\isacharunderscore}{\kern0pt}associative{\isadigit{2}}\ halve{\isacharunderscore}{\kern0pt}with{\isacharunderscore}{\kern0pt}parity{\isacharunderscore}{\kern0pt}nth{\isacharunderscore}{\kern0pt}even\ left{\isacharunderscore}{\kern0pt}coproj{\isacharunderscore}{\kern0pt}cfunc{\isacharunderscore}{\kern0pt}coprod{\isacharparenright}{\kern0pt}%
\endisatagproof
{\isafoldproof}%
%
\isadelimproof
\isanewline
%
\endisadelimproof
\isanewline
\isacommand{lemma}\isamarkupfalse%
\ halve{\isacharunderscore}{\kern0pt}nth{\isacharunderscore}{\kern0pt}odd{\isacharcolon}{\kern0pt}\isanewline
\ \ {\isachardoublequoteopen}halve\ {\isasymcirc}\isactrlsub c\ nth{\isacharunderscore}{\kern0pt}odd\ {\isacharequal}{\kern0pt}\ id\ {\isasymnat}\isactrlsub c{\isachardoublequoteclose}\isanewline
%
\isadelimproof
\ \ %
\endisadelimproof
%
\isatagproof
\isacommand{unfolding}\isamarkupfalse%
\ halve{\isacharunderscore}{\kern0pt}def\ \isacommand{by}\isamarkupfalse%
\ {\isacharparenleft}{\kern0pt}typecheck{\isacharunderscore}{\kern0pt}cfuncs{\isacharcomma}{\kern0pt}\ smt\ comp{\isacharunderscore}{\kern0pt}associative{\isadigit{2}}\ halve{\isacharunderscore}{\kern0pt}with{\isacharunderscore}{\kern0pt}parity{\isacharunderscore}{\kern0pt}nth{\isacharunderscore}{\kern0pt}odd\ right{\isacharunderscore}{\kern0pt}coproj{\isacharunderscore}{\kern0pt}cfunc{\isacharunderscore}{\kern0pt}coprod{\isacharparenright}{\kern0pt}%
\endisatagproof
{\isafoldproof}%
%
\isadelimproof
\isanewline
%
\endisadelimproof
\isanewline
\isacommand{lemma}\isamarkupfalse%
\ is{\isacharunderscore}{\kern0pt}even{\isacharunderscore}{\kern0pt}def{\isadigit{3}}{\isacharcolon}{\kern0pt}\isanewline
\ \ {\isachardoublequoteopen}is{\isacharunderscore}{\kern0pt}even\ {\isacharequal}{\kern0pt}\ {\isacharparenleft}{\kern0pt}{\isacharparenleft}{\kern0pt}{\isasymt}\ {\isasymcirc}\isactrlsub c\ {\isasymbeta}\isactrlbsub {\isasymnat}\isactrlsub c\isactrlesub {\isacharparenright}{\kern0pt}\ {\isasymamalg}\ {\isacharparenleft}{\kern0pt}{\isasymf}\ {\isasymcirc}\isactrlsub c\ {\isasymbeta}\isactrlbsub {\isasymnat}\isactrlsub c\isactrlesub {\isacharparenright}{\kern0pt}{\isacharparenright}{\kern0pt}\ {\isasymcirc}\isactrlsub c\ halve{\isacharunderscore}{\kern0pt}with{\isacharunderscore}{\kern0pt}parity{\isachardoublequoteclose}\isanewline
%
\isadelimproof
%
\endisadelimproof
%
\isatagproof
\isacommand{proof}\isamarkupfalse%
\ {\isacharparenleft}{\kern0pt}etcs{\isacharunderscore}{\kern0pt}rule\ natural{\isacharunderscore}{\kern0pt}number{\isacharunderscore}{\kern0pt}object{\isacharunderscore}{\kern0pt}func{\isacharunderscore}{\kern0pt}unique{\isacharbrackleft}{\kern0pt}\isakeyword{where}\ X{\isacharequal}{\kern0pt}{\isasymOmega}{\isacharcomma}{\kern0pt}\ \isakeyword{where}\ f{\isacharequal}{\kern0pt}NOT{\isacharbrackright}{\kern0pt}{\isacharparenright}{\kern0pt}\isanewline
\ \ \isacommand{show}\isamarkupfalse%
\ {\isachardoublequoteopen}is{\isacharunderscore}{\kern0pt}even\ {\isasymcirc}\isactrlsub c\ zero\ {\isacharequal}{\kern0pt}\ {\isacharparenleft}{\kern0pt}{\isacharparenleft}{\kern0pt}{\isasymt}\ {\isasymcirc}\isactrlsub c\ {\isasymbeta}\isactrlbsub {\isasymnat}\isactrlsub c\isactrlesub {\isacharparenright}{\kern0pt}\ {\isasymamalg}\ {\isacharparenleft}{\kern0pt}{\isasymf}\ {\isasymcirc}\isactrlsub c\ {\isasymbeta}\isactrlbsub {\isasymnat}\isactrlsub c\isactrlesub {\isacharparenright}{\kern0pt}\ {\isasymcirc}\isactrlsub c\ halve{\isacharunderscore}{\kern0pt}with{\isacharunderscore}{\kern0pt}parity{\isacharparenright}{\kern0pt}\ {\isasymcirc}\isactrlsub c\ zero{\isachardoublequoteclose}\isanewline
\ \ \isacommand{proof}\isamarkupfalse%
\ {\isacharminus}{\kern0pt}\isanewline
\ \ \ \ \isacommand{have}\isamarkupfalse%
\ {\isachardoublequoteopen}{\isacharparenleft}{\kern0pt}{\isacharparenleft}{\kern0pt}{\isasymt}\ {\isasymcirc}\isactrlsub c\ {\isasymbeta}\isactrlbsub {\isasymnat}\isactrlsub c\isactrlesub {\isacharparenright}{\kern0pt}\ {\isasymamalg}\ {\isacharparenleft}{\kern0pt}{\isasymf}\ {\isasymcirc}\isactrlsub c\ {\isasymbeta}\isactrlbsub {\isasymnat}\isactrlsub c\isactrlesub {\isacharparenright}{\kern0pt}\ {\isasymcirc}\isactrlsub c\ halve{\isacharunderscore}{\kern0pt}with{\isacharunderscore}{\kern0pt}parity{\isacharparenright}{\kern0pt}\ {\isasymcirc}\isactrlsub c\ zero\isanewline
\ \ \ \ \ \ {\isacharequal}{\kern0pt}\ {\isacharparenleft}{\kern0pt}{\isasymt}\ {\isasymcirc}\isactrlsub c\ {\isasymbeta}\isactrlbsub {\isasymnat}\isactrlsub c\isactrlesub {\isacharparenright}{\kern0pt}\ {\isasymamalg}\ {\isacharparenleft}{\kern0pt}{\isasymf}\ {\isasymcirc}\isactrlsub c\ {\isasymbeta}\isactrlbsub {\isasymnat}\isactrlsub c\isactrlesub {\isacharparenright}{\kern0pt}\ {\isasymcirc}\isactrlsub c\ left{\isacharunderscore}{\kern0pt}coproj\ {\isasymnat}\isactrlsub c\ {\isasymnat}\isactrlsub c\ {\isasymcirc}\isactrlsub c\ zero{\isachardoublequoteclose}\isanewline
\ \ \ \ \ \ \isacommand{by}\isamarkupfalse%
\ {\isacharparenleft}{\kern0pt}typecheck{\isacharunderscore}{\kern0pt}cfuncs{\isacharcomma}{\kern0pt}\ metis\ cfunc{\isacharunderscore}{\kern0pt}type{\isacharunderscore}{\kern0pt}def\ comp{\isacharunderscore}{\kern0pt}associative\ halve{\isacharunderscore}{\kern0pt}with{\isacharunderscore}{\kern0pt}parity{\isacharunderscore}{\kern0pt}zero{\isacharparenright}{\kern0pt}\isanewline
\ \ \ \ \isacommand{also}\isamarkupfalse%
\ \isacommand{have}\isamarkupfalse%
\ {\isachardoublequoteopen}{\isachardot}{\kern0pt}{\isachardot}{\kern0pt}{\isachardot}{\kern0pt}\ {\isacharequal}{\kern0pt}\ {\isacharparenleft}{\kern0pt}{\isasymt}\ {\isasymcirc}\isactrlsub c\ {\isasymbeta}\isactrlbsub {\isasymnat}\isactrlsub c\isactrlesub {\isacharparenright}{\kern0pt}\ {\isasymcirc}\isactrlsub c\ zero{\isachardoublequoteclose}\isanewline
\ \ \ \ \ \ \isacommand{by}\isamarkupfalse%
\ {\isacharparenleft}{\kern0pt}typecheck{\isacharunderscore}{\kern0pt}cfuncs{\isacharcomma}{\kern0pt}\ simp\ add{\isacharcolon}{\kern0pt}\ comp{\isacharunderscore}{\kern0pt}associative{\isadigit{2}}\ left{\isacharunderscore}{\kern0pt}coproj{\isacharunderscore}{\kern0pt}cfunc{\isacharunderscore}{\kern0pt}coprod{\isacharparenright}{\kern0pt}\isanewline
\ \ \ \ \isacommand{also}\isamarkupfalse%
\ \isacommand{have}\isamarkupfalse%
\ {\isachardoublequoteopen}{\isachardot}{\kern0pt}{\isachardot}{\kern0pt}{\isachardot}{\kern0pt}\ {\isacharequal}{\kern0pt}\ {\isasymt}{\isachardoublequoteclose}\isanewline
\ \ \ \ \ \ \isacommand{using}\isamarkupfalse%
\ comp{\isacharunderscore}{\kern0pt}associative{\isadigit{2}}\ is{\isacharunderscore}{\kern0pt}even{\isacharunderscore}{\kern0pt}def{\isadigit{2}}\ is{\isacharunderscore}{\kern0pt}even{\isacharunderscore}{\kern0pt}nth{\isacharunderscore}{\kern0pt}even{\isacharunderscore}{\kern0pt}true\ nth{\isacharunderscore}{\kern0pt}even{\isacharunderscore}{\kern0pt}def{\isadigit{2}}\ \isacommand{by}\isamarkupfalse%
\ {\isacharparenleft}{\kern0pt}typecheck{\isacharunderscore}{\kern0pt}cfuncs{\isacharcomma}{\kern0pt}\ force{\isacharparenright}{\kern0pt}\isanewline
\ \ \ \ \isacommand{also}\isamarkupfalse%
\ \isacommand{have}\isamarkupfalse%
\ {\isachardoublequoteopen}{\isachardot}{\kern0pt}{\isachardot}{\kern0pt}{\isachardot}{\kern0pt}\ {\isacharequal}{\kern0pt}\ is{\isacharunderscore}{\kern0pt}even\ {\isasymcirc}\isactrlsub c\ zero{\isachardoublequoteclose}\isanewline
\ \ \ \ \ \ \isacommand{by}\isamarkupfalse%
\ {\isacharparenleft}{\kern0pt}simp\ add{\isacharcolon}{\kern0pt}\ is{\isacharunderscore}{\kern0pt}even{\isacharunderscore}{\kern0pt}zero{\isacharparenright}{\kern0pt}\isanewline
\ \ \ \ \isacommand{then}\isamarkupfalse%
\ \isacommand{show}\isamarkupfalse%
\ {\isacharquery}{\kern0pt}thesis\isanewline
\ \ \ \ \ \ \isacommand{using}\isamarkupfalse%
\ calculation\ \isacommand{by}\isamarkupfalse%
\ auto\isanewline
\ \ \isacommand{qed}\isamarkupfalse%
\isanewline
\isanewline
\ \ \isacommand{show}\isamarkupfalse%
\ {\isachardoublequoteopen}is{\isacharunderscore}{\kern0pt}even\ {\isasymcirc}\isactrlsub c\ successor\ {\isacharequal}{\kern0pt}\ NOT\ {\isasymcirc}\isactrlsub c\ is{\isacharunderscore}{\kern0pt}even{\isachardoublequoteclose}\isanewline
\ \ \ \ \isacommand{by}\isamarkupfalse%
\ {\isacharparenleft}{\kern0pt}simp\ add{\isacharcolon}{\kern0pt}\ is{\isacharunderscore}{\kern0pt}even{\isacharunderscore}{\kern0pt}successor{\isacharparenright}{\kern0pt}\isanewline
\isanewline
\ \ \isacommand{show}\isamarkupfalse%
\ {\isachardoublequoteopen}{\isacharparenleft}{\kern0pt}{\isacharparenleft}{\kern0pt}{\isasymt}\ {\isasymcirc}\isactrlsub c\ {\isasymbeta}\isactrlbsub {\isasymnat}\isactrlsub c\isactrlesub {\isacharparenright}{\kern0pt}\ {\isasymamalg}\ {\isacharparenleft}{\kern0pt}{\isasymf}\ {\isasymcirc}\isactrlsub c\ {\isasymbeta}\isactrlbsub {\isasymnat}\isactrlsub c\isactrlesub {\isacharparenright}{\kern0pt}\ {\isasymcirc}\isactrlsub c\ halve{\isacharunderscore}{\kern0pt}with{\isacharunderscore}{\kern0pt}parity{\isacharparenright}{\kern0pt}\ {\isasymcirc}\isactrlsub c\ successor\ {\isacharequal}{\kern0pt}\isanewline
\ \ \ \ NOT\ {\isasymcirc}\isactrlsub c\ {\isacharparenleft}{\kern0pt}{\isasymt}\ {\isasymcirc}\isactrlsub c\ {\isasymbeta}\isactrlbsub {\isasymnat}\isactrlsub c\isactrlesub {\isacharparenright}{\kern0pt}\ {\isasymamalg}\ {\isacharparenleft}{\kern0pt}{\isasymf}\ {\isasymcirc}\isactrlsub c\ {\isasymbeta}\isactrlbsub {\isasymnat}\isactrlsub c\isactrlesub {\isacharparenright}{\kern0pt}\ {\isasymcirc}\isactrlsub c\ halve{\isacharunderscore}{\kern0pt}with{\isacharunderscore}{\kern0pt}parity{\isachardoublequoteclose}\isanewline
\ \ \isacommand{proof}\isamarkupfalse%
\ {\isacharminus}{\kern0pt}\isanewline
\ \ \ \ \isacommand{have}\isamarkupfalse%
\ {\isachardoublequoteopen}{\isacharparenleft}{\kern0pt}{\isacharparenleft}{\kern0pt}{\isasymt}\ {\isasymcirc}\isactrlsub c\ {\isasymbeta}\isactrlbsub {\isasymnat}\isactrlsub c\isactrlesub {\isacharparenright}{\kern0pt}\ {\isasymamalg}\ {\isacharparenleft}{\kern0pt}{\isasymf}\ {\isasymcirc}\isactrlsub c\ {\isasymbeta}\isactrlbsub {\isasymnat}\isactrlsub c\isactrlesub {\isacharparenright}{\kern0pt}\ {\isasymcirc}\isactrlsub c\ halve{\isacharunderscore}{\kern0pt}with{\isacharunderscore}{\kern0pt}parity{\isacharparenright}{\kern0pt}\ {\isasymcirc}\isactrlsub c\ successor\isanewline
\ \ \ \ \ \ {\isacharequal}{\kern0pt}\ {\isacharparenleft}{\kern0pt}{\isasymt}\ {\isasymcirc}\isactrlsub c\ {\isasymbeta}\isactrlbsub {\isasymnat}\isactrlsub c\isactrlesub {\isacharparenright}{\kern0pt}\ {\isasymamalg}\ {\isacharparenleft}{\kern0pt}{\isasymf}\ {\isasymcirc}\isactrlsub c\ {\isasymbeta}\isactrlbsub {\isasymnat}\isactrlsub c\isactrlesub {\isacharparenright}{\kern0pt}\ {\isasymcirc}\isactrlsub c\ {\isacharparenleft}{\kern0pt}right{\isacharunderscore}{\kern0pt}coproj\ {\isasymnat}\isactrlsub c\ {\isasymnat}\isactrlsub c\ {\isasymamalg}\ {\isacharparenleft}{\kern0pt}left{\isacharunderscore}{\kern0pt}coproj\ {\isasymnat}\isactrlsub c\ {\isasymnat}\isactrlsub c\ {\isasymcirc}\isactrlsub c\ successor{\isacharparenright}{\kern0pt}{\isacharparenright}{\kern0pt}\ {\isasymcirc}\isactrlsub c\ halve{\isacharunderscore}{\kern0pt}with{\isacharunderscore}{\kern0pt}parity{\isachardoublequoteclose}\isanewline
\ \ \ \ \ \ \isacommand{by}\isamarkupfalse%
\ {\isacharparenleft}{\kern0pt}typecheck{\isacharunderscore}{\kern0pt}cfuncs{\isacharcomma}{\kern0pt}\ simp\ add{\isacharcolon}{\kern0pt}\ comp{\isacharunderscore}{\kern0pt}associative{\isadigit{2}}\ halve{\isacharunderscore}{\kern0pt}with{\isacharunderscore}{\kern0pt}parity{\isacharunderscore}{\kern0pt}successor{\isacharparenright}{\kern0pt}\isanewline
\ \ \ \ \isacommand{also}\isamarkupfalse%
\ \isacommand{have}\isamarkupfalse%
\ {\isachardoublequoteopen}{\isachardot}{\kern0pt}{\isachardot}{\kern0pt}{\isachardot}{\kern0pt}\ {\isacharequal}{\kern0pt}\ \isanewline
\ \ \ \ \ \ \ \ {\isacharparenleft}{\kern0pt}{\isacharparenleft}{\kern0pt}{\isacharparenleft}{\kern0pt}{\isasymt}\ {\isasymcirc}\isactrlsub c\ {\isasymbeta}\isactrlbsub {\isasymnat}\isactrlsub c\isactrlesub {\isacharparenright}{\kern0pt}\ {\isasymamalg}\ {\isacharparenleft}{\kern0pt}{\isasymf}\ {\isasymcirc}\isactrlsub c\ {\isasymbeta}\isactrlbsub {\isasymnat}\isactrlsub c\isactrlesub {\isacharparenright}{\kern0pt}\ {\isasymcirc}\isactrlsub c\ right{\isacharunderscore}{\kern0pt}coproj\ {\isasymnat}\isactrlsub c\ {\isasymnat}\isactrlsub c{\isacharparenright}{\kern0pt}\isanewline
\ \ \ \ \ \ \ \ \ \ {\isasymamalg}\ \isanewline
\ \ \ \ \ \ \ \ {\isacharparenleft}{\kern0pt}{\isacharparenleft}{\kern0pt}{\isasymt}\ {\isasymcirc}\isactrlsub c\ {\isasymbeta}\isactrlbsub {\isasymnat}\isactrlsub c\isactrlesub {\isacharparenright}{\kern0pt}\ {\isasymamalg}\ {\isacharparenleft}{\kern0pt}{\isasymf}\ {\isasymcirc}\isactrlsub c\ {\isasymbeta}\isactrlbsub {\isasymnat}\isactrlsub c\isactrlesub {\isacharparenright}{\kern0pt}\ {\isasymcirc}\isactrlsub c\ left{\isacharunderscore}{\kern0pt}coproj\ {\isasymnat}\isactrlsub c\ {\isasymnat}\isactrlsub c\ {\isasymcirc}\isactrlsub c\ successor{\isacharparenright}{\kern0pt}{\isacharparenright}{\kern0pt}\isanewline
\ \ \ \ \ \ \ \ \ \ {\isasymcirc}\isactrlsub c\ halve{\isacharunderscore}{\kern0pt}with{\isacharunderscore}{\kern0pt}parity{\isachardoublequoteclose}\isanewline
\ \ \ \ \ \ \isacommand{by}\isamarkupfalse%
\ {\isacharparenleft}{\kern0pt}typecheck{\isacharunderscore}{\kern0pt}cfuncs{\isacharcomma}{\kern0pt}\ smt\ cfunc{\isacharunderscore}{\kern0pt}coprod{\isacharunderscore}{\kern0pt}comp\ comp{\isacharunderscore}{\kern0pt}associative{\isadigit{2}}{\isacharparenright}{\kern0pt}\isanewline
\ \ \ \ \isacommand{also}\isamarkupfalse%
\ \isacommand{have}\isamarkupfalse%
\ {\isachardoublequoteopen}{\isachardot}{\kern0pt}{\isachardot}{\kern0pt}{\isachardot}{\kern0pt}\ {\isacharequal}{\kern0pt}\ {\isacharparenleft}{\kern0pt}{\isacharparenleft}{\kern0pt}{\isasymf}\ {\isasymcirc}\isactrlsub c\ {\isasymbeta}\isactrlbsub {\isasymnat}\isactrlsub c\isactrlesub {\isacharparenright}{\kern0pt}\ {\isasymamalg}\ {\isacharparenleft}{\kern0pt}{\isasymt}\ {\isasymcirc}\isactrlsub c\ {\isasymbeta}\isactrlbsub {\isasymnat}\isactrlsub c\isactrlesub \ {\isasymcirc}\isactrlsub c\ successor{\isacharparenright}{\kern0pt}{\isacharparenright}{\kern0pt}\ {\isasymcirc}\isactrlsub c\ halve{\isacharunderscore}{\kern0pt}with{\isacharunderscore}{\kern0pt}parity{\isachardoublequoteclose}\isanewline
\ \ \ \ \ \ \isacommand{by}\isamarkupfalse%
\ {\isacharparenleft}{\kern0pt}typecheck{\isacharunderscore}{\kern0pt}cfuncs{\isacharcomma}{\kern0pt}\ simp\ add{\isacharcolon}{\kern0pt}\ comp{\isacharunderscore}{\kern0pt}associative{\isadigit{2}}\ left{\isacharunderscore}{\kern0pt}coproj{\isacharunderscore}{\kern0pt}cfunc{\isacharunderscore}{\kern0pt}coprod\ right{\isacharunderscore}{\kern0pt}coproj{\isacharunderscore}{\kern0pt}cfunc{\isacharunderscore}{\kern0pt}coprod{\isacharparenright}{\kern0pt}\isanewline
\ \ \ \ \isacommand{also}\isamarkupfalse%
\ \isacommand{have}\isamarkupfalse%
\ {\isachardoublequoteopen}{\isachardot}{\kern0pt}{\isachardot}{\kern0pt}{\isachardot}{\kern0pt}\ {\isacharequal}{\kern0pt}\ {\isacharparenleft}{\kern0pt}{\isacharparenleft}{\kern0pt}NOT\ {\isasymcirc}\isactrlsub c\ {\isasymt}\ {\isasymcirc}\isactrlsub c\ {\isasymbeta}\isactrlbsub {\isasymnat}\isactrlsub c\isactrlesub {\isacharparenright}{\kern0pt}\ {\isasymamalg}\ {\isacharparenleft}{\kern0pt}NOT\ {\isasymcirc}\isactrlsub c\ {\isasymf}\ {\isasymcirc}\isactrlsub c\ {\isasymbeta}\isactrlbsub {\isasymnat}\isactrlsub c\isactrlesub \ {\isasymcirc}\isactrlsub c\ successor{\isacharparenright}{\kern0pt}{\isacharparenright}{\kern0pt}\ {\isasymcirc}\isactrlsub c\ halve{\isacharunderscore}{\kern0pt}with{\isacharunderscore}{\kern0pt}parity{\isachardoublequoteclose}\isanewline
\ \ \ \ \ \ \isacommand{by}\isamarkupfalse%
\ {\isacharparenleft}{\kern0pt}typecheck{\isacharunderscore}{\kern0pt}cfuncs{\isacharcomma}{\kern0pt}\ simp\ add{\isacharcolon}{\kern0pt}\ NOT{\isacharunderscore}{\kern0pt}false{\isacharunderscore}{\kern0pt}is{\isacharunderscore}{\kern0pt}true\ NOT{\isacharunderscore}{\kern0pt}true{\isacharunderscore}{\kern0pt}is{\isacharunderscore}{\kern0pt}false\ comp{\isacharunderscore}{\kern0pt}associative{\isadigit{2}}{\isacharparenright}{\kern0pt}\isanewline
\ \ \ \ \isacommand{also}\isamarkupfalse%
\ \isacommand{have}\isamarkupfalse%
\ {\isachardoublequoteopen}{\isachardot}{\kern0pt}{\isachardot}{\kern0pt}{\isachardot}{\kern0pt}\ {\isacharequal}{\kern0pt}\ NOT\ {\isasymcirc}\isactrlsub c\ {\isacharparenleft}{\kern0pt}{\isasymt}\ {\isasymcirc}\isactrlsub c\ {\isasymbeta}\isactrlbsub {\isasymnat}\isactrlsub c\isactrlesub {\isacharparenright}{\kern0pt}\ {\isasymamalg}\ {\isacharparenleft}{\kern0pt}{\isasymf}\ {\isasymcirc}\isactrlsub c\ {\isasymbeta}\isactrlbsub {\isasymnat}\isactrlsub c\isactrlesub {\isacharparenright}{\kern0pt}\ {\isasymcirc}\isactrlsub c\ halve{\isacharunderscore}{\kern0pt}with{\isacharunderscore}{\kern0pt}parity{\isachardoublequoteclose}\isanewline
\ \ \ \ \ \ \isacommand{by}\isamarkupfalse%
\ {\isacharparenleft}{\kern0pt}typecheck{\isacharunderscore}{\kern0pt}cfuncs{\isacharcomma}{\kern0pt}\ smt\ cfunc{\isacharunderscore}{\kern0pt}coprod{\isacharunderscore}{\kern0pt}comp\ comp{\isacharunderscore}{\kern0pt}associative{\isadigit{2}}\ terminal{\isacharunderscore}{\kern0pt}func{\isacharunderscore}{\kern0pt}unique{\isacharparenright}{\kern0pt}\isanewline
\ \ \ \ \isacommand{then}\isamarkupfalse%
\ \isacommand{show}\isamarkupfalse%
\ {\isacharquery}{\kern0pt}thesis\isanewline
\ \ \ \ \ \ \isacommand{using}\isamarkupfalse%
\ calculation\ \isacommand{by}\isamarkupfalse%
\ auto\isanewline
\ \ \isacommand{qed}\isamarkupfalse%
\isanewline
\isacommand{qed}\isamarkupfalse%
%
\endisatagproof
{\isafoldproof}%
%
\isadelimproof
\isanewline
%
\endisadelimproof
\isanewline
\isacommand{lemma}\isamarkupfalse%
\ is{\isacharunderscore}{\kern0pt}odd{\isacharunderscore}{\kern0pt}def{\isadigit{3}}{\isacharcolon}{\kern0pt}\isanewline
\ \ {\isachardoublequoteopen}is{\isacharunderscore}{\kern0pt}odd\ {\isacharequal}{\kern0pt}\ {\isacharparenleft}{\kern0pt}{\isacharparenleft}{\kern0pt}{\isasymf}\ {\isasymcirc}\isactrlsub c\ {\isasymbeta}\isactrlbsub {\isasymnat}\isactrlsub c\isactrlesub {\isacharparenright}{\kern0pt}\ {\isasymamalg}\ {\isacharparenleft}{\kern0pt}{\isasymt}\ {\isasymcirc}\isactrlsub c\ {\isasymbeta}\isactrlbsub {\isasymnat}\isactrlsub c\isactrlesub {\isacharparenright}{\kern0pt}{\isacharparenright}{\kern0pt}\ {\isasymcirc}\isactrlsub c\ halve{\isacharunderscore}{\kern0pt}with{\isacharunderscore}{\kern0pt}parity{\isachardoublequoteclose}\isanewline
%
\isadelimproof
%
\endisadelimproof
%
\isatagproof
\isacommand{proof}\isamarkupfalse%
\ {\isacharparenleft}{\kern0pt}etcs{\isacharunderscore}{\kern0pt}rule\ natural{\isacharunderscore}{\kern0pt}number{\isacharunderscore}{\kern0pt}object{\isacharunderscore}{\kern0pt}func{\isacharunderscore}{\kern0pt}unique{\isacharbrackleft}{\kern0pt}\isakeyword{where}\ X{\isacharequal}{\kern0pt}{\isasymOmega}{\isacharcomma}{\kern0pt}\ \isakeyword{where}\ f{\isacharequal}{\kern0pt}NOT{\isacharbrackright}{\kern0pt}{\isacharparenright}{\kern0pt}\isanewline
\ \ \isacommand{show}\isamarkupfalse%
\ {\isachardoublequoteopen}is{\isacharunderscore}{\kern0pt}odd\ {\isasymcirc}\isactrlsub c\ zero\ {\isacharequal}{\kern0pt}\ {\isacharparenleft}{\kern0pt}{\isacharparenleft}{\kern0pt}{\isasymf}\ {\isasymcirc}\isactrlsub c\ {\isasymbeta}\isactrlbsub {\isasymnat}\isactrlsub c\isactrlesub {\isacharparenright}{\kern0pt}\ {\isasymamalg}\ {\isacharparenleft}{\kern0pt}{\isasymt}\ {\isasymcirc}\isactrlsub c\ {\isasymbeta}\isactrlbsub {\isasymnat}\isactrlsub c\isactrlesub {\isacharparenright}{\kern0pt}\ {\isasymcirc}\isactrlsub c\ halve{\isacharunderscore}{\kern0pt}with{\isacharunderscore}{\kern0pt}parity{\isacharparenright}{\kern0pt}\ {\isasymcirc}\isactrlsub c\ zero{\isachardoublequoteclose}\isanewline
\ \ \isacommand{proof}\isamarkupfalse%
\ {\isacharminus}{\kern0pt}\isanewline
\ \ \ \ \isacommand{have}\isamarkupfalse%
\ {\isachardoublequoteopen}{\isacharparenleft}{\kern0pt}{\isacharparenleft}{\kern0pt}{\isasymf}\ {\isasymcirc}\isactrlsub c\ {\isasymbeta}\isactrlbsub {\isasymnat}\isactrlsub c\isactrlesub {\isacharparenright}{\kern0pt}\ {\isasymamalg}\ {\isacharparenleft}{\kern0pt}{\isasymt}\ {\isasymcirc}\isactrlsub c\ {\isasymbeta}\isactrlbsub {\isasymnat}\isactrlsub c\isactrlesub {\isacharparenright}{\kern0pt}\ {\isasymcirc}\isactrlsub c\ halve{\isacharunderscore}{\kern0pt}with{\isacharunderscore}{\kern0pt}parity{\isacharparenright}{\kern0pt}\ {\isasymcirc}\isactrlsub c\ zero\isanewline
\ \ \ \ \ \ {\isacharequal}{\kern0pt}\ {\isacharparenleft}{\kern0pt}{\isasymf}\ {\isasymcirc}\isactrlsub c\ {\isasymbeta}\isactrlbsub {\isasymnat}\isactrlsub c\isactrlesub {\isacharparenright}{\kern0pt}\ {\isasymamalg}\ {\isacharparenleft}{\kern0pt}{\isasymt}\ {\isasymcirc}\isactrlsub c\ {\isasymbeta}\isactrlbsub {\isasymnat}\isactrlsub c\isactrlesub {\isacharparenright}{\kern0pt}\ {\isasymcirc}\isactrlsub c\ left{\isacharunderscore}{\kern0pt}coproj\ {\isasymnat}\isactrlsub c\ {\isasymnat}\isactrlsub c\ {\isasymcirc}\isactrlsub c\ zero{\isachardoublequoteclose}\isanewline
\ \ \ \ \ \ \isacommand{by}\isamarkupfalse%
\ {\isacharparenleft}{\kern0pt}typecheck{\isacharunderscore}{\kern0pt}cfuncs{\isacharcomma}{\kern0pt}\ metis\ cfunc{\isacharunderscore}{\kern0pt}type{\isacharunderscore}{\kern0pt}def\ comp{\isacharunderscore}{\kern0pt}associative\ halve{\isacharunderscore}{\kern0pt}with{\isacharunderscore}{\kern0pt}parity{\isacharunderscore}{\kern0pt}zero{\isacharparenright}{\kern0pt}\isanewline
\ \ \ \ \isacommand{also}\isamarkupfalse%
\ \isacommand{have}\isamarkupfalse%
\ {\isachardoublequoteopen}{\isachardot}{\kern0pt}{\isachardot}{\kern0pt}{\isachardot}{\kern0pt}\ {\isacharequal}{\kern0pt}\ {\isacharparenleft}{\kern0pt}{\isasymf}\ {\isasymcirc}\isactrlsub c\ {\isasymbeta}\isactrlbsub {\isasymnat}\isactrlsub c\isactrlesub {\isacharparenright}{\kern0pt}\ {\isasymcirc}\isactrlsub c\ zero{\isachardoublequoteclose}\isanewline
\ \ \ \ \ \ \isacommand{by}\isamarkupfalse%
\ {\isacharparenleft}{\kern0pt}typecheck{\isacharunderscore}{\kern0pt}cfuncs{\isacharcomma}{\kern0pt}\ simp\ add{\isacharcolon}{\kern0pt}\ comp{\isacharunderscore}{\kern0pt}associative{\isadigit{2}}\ left{\isacharunderscore}{\kern0pt}coproj{\isacharunderscore}{\kern0pt}cfunc{\isacharunderscore}{\kern0pt}coprod{\isacharparenright}{\kern0pt}\isanewline
\ \ \ \ \isacommand{also}\isamarkupfalse%
\ \isacommand{have}\isamarkupfalse%
\ {\isachardoublequoteopen}{\isachardot}{\kern0pt}{\isachardot}{\kern0pt}{\isachardot}{\kern0pt}\ {\isacharequal}{\kern0pt}\ {\isasymf}{\isachardoublequoteclose}\isanewline
\ \ \ \ \ \ \isacommand{using}\isamarkupfalse%
\ comp{\isacharunderscore}{\kern0pt}associative{\isadigit{2}}\ is{\isacharunderscore}{\kern0pt}odd{\isacharunderscore}{\kern0pt}nth{\isacharunderscore}{\kern0pt}even{\isacharunderscore}{\kern0pt}false\ is{\isacharunderscore}{\kern0pt}odd{\isacharunderscore}{\kern0pt}type\ is{\isacharunderscore}{\kern0pt}odd{\isacharunderscore}{\kern0pt}zero\ nth{\isacharunderscore}{\kern0pt}even{\isacharunderscore}{\kern0pt}def{\isadigit{2}}\ \isacommand{by}\isamarkupfalse%
\ {\isacharparenleft}{\kern0pt}typecheck{\isacharunderscore}{\kern0pt}cfuncs{\isacharcomma}{\kern0pt}\ force{\isacharparenright}{\kern0pt}\isanewline
\ \ \ \ \isacommand{also}\isamarkupfalse%
\ \isacommand{have}\isamarkupfalse%
\ {\isachardoublequoteopen}{\isachardot}{\kern0pt}{\isachardot}{\kern0pt}{\isachardot}{\kern0pt}\ {\isacharequal}{\kern0pt}\ is{\isacharunderscore}{\kern0pt}odd\ {\isasymcirc}\isactrlsub c\ zero{\isachardoublequoteclose}\isanewline
\ \ \ \ \ \ \isacommand{by}\isamarkupfalse%
\ {\isacharparenleft}{\kern0pt}simp\ add{\isacharcolon}{\kern0pt}\ is{\isacharunderscore}{\kern0pt}odd{\isacharunderscore}{\kern0pt}def{\isadigit{2}}{\isacharparenright}{\kern0pt}\isanewline
\ \ \ \ \isacommand{then}\isamarkupfalse%
\ \isacommand{show}\isamarkupfalse%
\ {\isacharquery}{\kern0pt}thesis\isanewline
\ \ \ \ \ \ \isacommand{using}\isamarkupfalse%
\ calculation\ \isacommand{by}\isamarkupfalse%
\ auto\isanewline
\ \ \isacommand{qed}\isamarkupfalse%
\isanewline
\isanewline
\ \ \isacommand{show}\isamarkupfalse%
\ {\isachardoublequoteopen}is{\isacharunderscore}{\kern0pt}odd\ {\isasymcirc}\isactrlsub c\ successor\ {\isacharequal}{\kern0pt}\ NOT\ {\isasymcirc}\isactrlsub c\ is{\isacharunderscore}{\kern0pt}odd{\isachardoublequoteclose}\isanewline
\ \ \ \ \isacommand{by}\isamarkupfalse%
\ {\isacharparenleft}{\kern0pt}simp\ add{\isacharcolon}{\kern0pt}\ is{\isacharunderscore}{\kern0pt}odd{\isacharunderscore}{\kern0pt}successor{\isacharparenright}{\kern0pt}\isanewline
\isanewline
\ \ \isacommand{show}\isamarkupfalse%
\ {\isachardoublequoteopen}{\isacharparenleft}{\kern0pt}{\isacharparenleft}{\kern0pt}{\isasymf}\ {\isasymcirc}\isactrlsub c\ {\isasymbeta}\isactrlbsub {\isasymnat}\isactrlsub c\isactrlesub {\isacharparenright}{\kern0pt}\ {\isasymamalg}\ {\isacharparenleft}{\kern0pt}{\isasymt}\ {\isasymcirc}\isactrlsub c\ {\isasymbeta}\isactrlbsub {\isasymnat}\isactrlsub c\isactrlesub {\isacharparenright}{\kern0pt}\ {\isasymcirc}\isactrlsub c\ halve{\isacharunderscore}{\kern0pt}with{\isacharunderscore}{\kern0pt}parity{\isacharparenright}{\kern0pt}\ {\isasymcirc}\isactrlsub c\ successor\ {\isacharequal}{\kern0pt}\isanewline
\ \ \ \ NOT\ {\isasymcirc}\isactrlsub c\ {\isacharparenleft}{\kern0pt}{\isasymf}\ {\isasymcirc}\isactrlsub c\ {\isasymbeta}\isactrlbsub {\isasymnat}\isactrlsub c\isactrlesub {\isacharparenright}{\kern0pt}\ {\isasymamalg}\ {\isacharparenleft}{\kern0pt}{\isasymt}\ {\isasymcirc}\isactrlsub c\ {\isasymbeta}\isactrlbsub {\isasymnat}\isactrlsub c\isactrlesub {\isacharparenright}{\kern0pt}\ {\isasymcirc}\isactrlsub c\ halve{\isacharunderscore}{\kern0pt}with{\isacharunderscore}{\kern0pt}parity{\isachardoublequoteclose}\isanewline
\ \ \isacommand{proof}\isamarkupfalse%
\ {\isacharminus}{\kern0pt}\isanewline
\ \ \ \ \isacommand{have}\isamarkupfalse%
\ {\isachardoublequoteopen}{\isacharparenleft}{\kern0pt}{\isacharparenleft}{\kern0pt}{\isasymf}\ {\isasymcirc}\isactrlsub c\ {\isasymbeta}\isactrlbsub {\isasymnat}\isactrlsub c\isactrlesub {\isacharparenright}{\kern0pt}\ {\isasymamalg}\ {\isacharparenleft}{\kern0pt}{\isasymt}\ {\isasymcirc}\isactrlsub c\ {\isasymbeta}\isactrlbsub {\isasymnat}\isactrlsub c\isactrlesub {\isacharparenright}{\kern0pt}\ {\isasymcirc}\isactrlsub c\ halve{\isacharunderscore}{\kern0pt}with{\isacharunderscore}{\kern0pt}parity{\isacharparenright}{\kern0pt}\ {\isasymcirc}\isactrlsub c\ successor\isanewline
\ \ \ \ \ \ {\isacharequal}{\kern0pt}\ {\isacharparenleft}{\kern0pt}{\isasymf}\ {\isasymcirc}\isactrlsub c\ {\isasymbeta}\isactrlbsub {\isasymnat}\isactrlsub c\isactrlesub {\isacharparenright}{\kern0pt}\ {\isasymamalg}\ {\isacharparenleft}{\kern0pt}{\isasymt}\ {\isasymcirc}\isactrlsub c\ {\isasymbeta}\isactrlbsub {\isasymnat}\isactrlsub c\isactrlesub {\isacharparenright}{\kern0pt}\ {\isasymcirc}\isactrlsub c\ {\isacharparenleft}{\kern0pt}right{\isacharunderscore}{\kern0pt}coproj\ {\isasymnat}\isactrlsub c\ {\isasymnat}\isactrlsub c\ {\isasymamalg}\ {\isacharparenleft}{\kern0pt}left{\isacharunderscore}{\kern0pt}coproj\ {\isasymnat}\isactrlsub c\ {\isasymnat}\isactrlsub c\ {\isasymcirc}\isactrlsub c\ successor{\isacharparenright}{\kern0pt}{\isacharparenright}{\kern0pt}\ {\isasymcirc}\isactrlsub c\ halve{\isacharunderscore}{\kern0pt}with{\isacharunderscore}{\kern0pt}parity{\isachardoublequoteclose}\isanewline
\ \ \ \ \ \ \isacommand{by}\isamarkupfalse%
\ {\isacharparenleft}{\kern0pt}typecheck{\isacharunderscore}{\kern0pt}cfuncs{\isacharcomma}{\kern0pt}\ simp\ add{\isacharcolon}{\kern0pt}\ comp{\isacharunderscore}{\kern0pt}associative{\isadigit{2}}\ halve{\isacharunderscore}{\kern0pt}with{\isacharunderscore}{\kern0pt}parity{\isacharunderscore}{\kern0pt}successor{\isacharparenright}{\kern0pt}\isanewline
\ \ \ \ \isacommand{also}\isamarkupfalse%
\ \isacommand{have}\isamarkupfalse%
\ {\isachardoublequoteopen}{\isachardot}{\kern0pt}{\isachardot}{\kern0pt}{\isachardot}{\kern0pt}\ {\isacharequal}{\kern0pt}\ \isanewline
\ \ \ \ \ \ \ \ {\isacharparenleft}{\kern0pt}{\isacharparenleft}{\kern0pt}{\isacharparenleft}{\kern0pt}{\isasymf}\ {\isasymcirc}\isactrlsub c\ {\isasymbeta}\isactrlbsub {\isasymnat}\isactrlsub c\isactrlesub {\isacharparenright}{\kern0pt}\ {\isasymamalg}\ {\isacharparenleft}{\kern0pt}{\isasymt}\ {\isasymcirc}\isactrlsub c\ {\isasymbeta}\isactrlbsub {\isasymnat}\isactrlsub c\isactrlesub {\isacharparenright}{\kern0pt}\ {\isasymcirc}\isactrlsub c\ right{\isacharunderscore}{\kern0pt}coproj\ {\isasymnat}\isactrlsub c\ {\isasymnat}\isactrlsub c{\isacharparenright}{\kern0pt}\isanewline
\ \ \ \ \ \ \ \ \ \ {\isasymamalg}\ \isanewline
\ \ \ \ \ \ \ \ {\isacharparenleft}{\kern0pt}{\isacharparenleft}{\kern0pt}{\isasymf}\ {\isasymcirc}\isactrlsub c\ {\isasymbeta}\isactrlbsub {\isasymnat}\isactrlsub c\isactrlesub {\isacharparenright}{\kern0pt}\ {\isasymamalg}\ {\isacharparenleft}{\kern0pt}{\isasymt}\ {\isasymcirc}\isactrlsub c\ {\isasymbeta}\isactrlbsub {\isasymnat}\isactrlsub c\isactrlesub {\isacharparenright}{\kern0pt}\ {\isasymcirc}\isactrlsub c\ left{\isacharunderscore}{\kern0pt}coproj\ {\isasymnat}\isactrlsub c\ {\isasymnat}\isactrlsub c\ {\isasymcirc}\isactrlsub c\ successor{\isacharparenright}{\kern0pt}{\isacharparenright}{\kern0pt}\isanewline
\ \ \ \ \ \ \ \ \ \ {\isasymcirc}\isactrlsub c\ halve{\isacharunderscore}{\kern0pt}with{\isacharunderscore}{\kern0pt}parity{\isachardoublequoteclose}\isanewline
\ \ \ \ \ \ \isacommand{by}\isamarkupfalse%
\ {\isacharparenleft}{\kern0pt}typecheck{\isacharunderscore}{\kern0pt}cfuncs{\isacharcomma}{\kern0pt}\ smt\ cfunc{\isacharunderscore}{\kern0pt}coprod{\isacharunderscore}{\kern0pt}comp\ comp{\isacharunderscore}{\kern0pt}associative{\isadigit{2}}{\isacharparenright}{\kern0pt}\isanewline
\ \ \ \ \isacommand{also}\isamarkupfalse%
\ \isacommand{have}\isamarkupfalse%
\ {\isachardoublequoteopen}{\isachardot}{\kern0pt}{\isachardot}{\kern0pt}{\isachardot}{\kern0pt}\ {\isacharequal}{\kern0pt}\ {\isacharparenleft}{\kern0pt}{\isacharparenleft}{\kern0pt}{\isasymt}\ {\isasymcirc}\isactrlsub c\ {\isasymbeta}\isactrlbsub {\isasymnat}\isactrlsub c\isactrlesub {\isacharparenright}{\kern0pt}\ {\isasymamalg}\ {\isacharparenleft}{\kern0pt}{\isasymf}\ {\isasymcirc}\isactrlsub c\ {\isasymbeta}\isactrlbsub {\isasymnat}\isactrlsub c\isactrlesub \ {\isasymcirc}\isactrlsub c\ successor{\isacharparenright}{\kern0pt}{\isacharparenright}{\kern0pt}\ {\isasymcirc}\isactrlsub c\ halve{\isacharunderscore}{\kern0pt}with{\isacharunderscore}{\kern0pt}parity{\isachardoublequoteclose}\isanewline
\ \ \ \ \ \ \isacommand{by}\isamarkupfalse%
\ {\isacharparenleft}{\kern0pt}typecheck{\isacharunderscore}{\kern0pt}cfuncs{\isacharcomma}{\kern0pt}\ simp\ add{\isacharcolon}{\kern0pt}\ comp{\isacharunderscore}{\kern0pt}associative{\isadigit{2}}\ left{\isacharunderscore}{\kern0pt}coproj{\isacharunderscore}{\kern0pt}cfunc{\isacharunderscore}{\kern0pt}coprod\ right{\isacharunderscore}{\kern0pt}coproj{\isacharunderscore}{\kern0pt}cfunc{\isacharunderscore}{\kern0pt}coprod{\isacharparenright}{\kern0pt}\isanewline
\ \ \ \ \isacommand{also}\isamarkupfalse%
\ \isacommand{have}\isamarkupfalse%
\ {\isachardoublequoteopen}{\isachardot}{\kern0pt}{\isachardot}{\kern0pt}{\isachardot}{\kern0pt}\ {\isacharequal}{\kern0pt}\ {\isacharparenleft}{\kern0pt}{\isacharparenleft}{\kern0pt}NOT\ {\isasymcirc}\isactrlsub c\ {\isasymf}\ {\isasymcirc}\isactrlsub c\ {\isasymbeta}\isactrlbsub {\isasymnat}\isactrlsub c\isactrlesub {\isacharparenright}{\kern0pt}\ {\isasymamalg}\ {\isacharparenleft}{\kern0pt}NOT\ {\isasymcirc}\isactrlsub c\ {\isasymt}\ {\isasymcirc}\isactrlsub c\ {\isasymbeta}\isactrlbsub {\isasymnat}\isactrlsub c\isactrlesub \ {\isasymcirc}\isactrlsub c\ successor{\isacharparenright}{\kern0pt}{\isacharparenright}{\kern0pt}\ {\isasymcirc}\isactrlsub c\ halve{\isacharunderscore}{\kern0pt}with{\isacharunderscore}{\kern0pt}parity{\isachardoublequoteclose}\isanewline
\ \ \ \ \ \ \isacommand{by}\isamarkupfalse%
\ {\isacharparenleft}{\kern0pt}typecheck{\isacharunderscore}{\kern0pt}cfuncs{\isacharcomma}{\kern0pt}\ simp\ add{\isacharcolon}{\kern0pt}\ NOT{\isacharunderscore}{\kern0pt}false{\isacharunderscore}{\kern0pt}is{\isacharunderscore}{\kern0pt}true\ NOT{\isacharunderscore}{\kern0pt}true{\isacharunderscore}{\kern0pt}is{\isacharunderscore}{\kern0pt}false\ comp{\isacharunderscore}{\kern0pt}associative{\isadigit{2}}{\isacharparenright}{\kern0pt}\isanewline
\ \ \ \ \isacommand{also}\isamarkupfalse%
\ \isacommand{have}\isamarkupfalse%
\ {\isachardoublequoteopen}{\isachardot}{\kern0pt}{\isachardot}{\kern0pt}{\isachardot}{\kern0pt}\ {\isacharequal}{\kern0pt}\ NOT\ {\isasymcirc}\isactrlsub c\ {\isacharparenleft}{\kern0pt}{\isasymf}\ {\isasymcirc}\isactrlsub c\ {\isasymbeta}\isactrlbsub {\isasymnat}\isactrlsub c\isactrlesub {\isacharparenright}{\kern0pt}\ {\isasymamalg}\ {\isacharparenleft}{\kern0pt}{\isasymt}\ {\isasymcirc}\isactrlsub c\ {\isasymbeta}\isactrlbsub {\isasymnat}\isactrlsub c\isactrlesub {\isacharparenright}{\kern0pt}\ {\isasymcirc}\isactrlsub c\ halve{\isacharunderscore}{\kern0pt}with{\isacharunderscore}{\kern0pt}parity{\isachardoublequoteclose}\isanewline
\ \ \ \ \ \ \isacommand{by}\isamarkupfalse%
\ {\isacharparenleft}{\kern0pt}typecheck{\isacharunderscore}{\kern0pt}cfuncs{\isacharcomma}{\kern0pt}\ smt\ cfunc{\isacharunderscore}{\kern0pt}coprod{\isacharunderscore}{\kern0pt}comp\ comp{\isacharunderscore}{\kern0pt}associative{\isadigit{2}}\ terminal{\isacharunderscore}{\kern0pt}func{\isacharunderscore}{\kern0pt}unique{\isacharparenright}{\kern0pt}\isanewline
\ \ \ \ \isacommand{then}\isamarkupfalse%
\ \isacommand{show}\isamarkupfalse%
\ {\isacharquery}{\kern0pt}thesis\isanewline
\ \ \ \ \ \ \isacommand{using}\isamarkupfalse%
\ calculation\ \isacommand{by}\isamarkupfalse%
\ auto\isanewline
\ \ \isacommand{qed}\isamarkupfalse%
\isanewline
\isacommand{qed}\isamarkupfalse%
%
\endisatagproof
{\isafoldproof}%
%
\isadelimproof
\isanewline
%
\endisadelimproof
\isanewline
\isacommand{lemma}\isamarkupfalse%
\ nth{\isacharunderscore}{\kern0pt}even{\isacharunderscore}{\kern0pt}or{\isacharunderscore}{\kern0pt}nth{\isacharunderscore}{\kern0pt}odd{\isacharcolon}{\kern0pt}\isanewline
\ \ \isakeyword{assumes}\ {\isachardoublequoteopen}n\ {\isasymin}\isactrlsub c\ {\isasymnat}\isactrlsub c{\isachardoublequoteclose}\isanewline
\ \ \isakeyword{shows}\ {\isachardoublequoteopen}{\isacharparenleft}{\kern0pt}{\isasymexists}\ m{\isachardot}{\kern0pt}\ m\ {\isasymin}\isactrlsub c\ {\isasymnat}\isactrlsub c\ {\isasymand}\ nth{\isacharunderscore}{\kern0pt}even\ {\isasymcirc}\isactrlsub c\ m\ {\isacharequal}{\kern0pt}\ n{\isacharparenright}{\kern0pt}\ {\isasymor}\ {\isacharparenleft}{\kern0pt}{\isasymexists}\ m{\isachardot}{\kern0pt}\ m\ {\isasymin}\isactrlsub c\ {\isasymnat}\isactrlsub c\ {\isasymand}\ nth{\isacharunderscore}{\kern0pt}odd\ {\isasymcirc}\isactrlsub c\ m\ {\isacharequal}{\kern0pt}\ n{\isacharparenright}{\kern0pt}{\isachardoublequoteclose}\isanewline
%
\isadelimproof
%
\endisadelimproof
%
\isatagproof
\isacommand{proof}\isamarkupfalse%
\ {\isacharminus}{\kern0pt}\isanewline
\ \ \isacommand{have}\isamarkupfalse%
\ {\isachardoublequoteopen}{\isacharparenleft}{\kern0pt}{\isasymexists}m{\isachardot}{\kern0pt}\ m\ {\isasymin}\isactrlsub c\ {\isasymnat}\isactrlsub c\ {\isasymand}\ halve{\isacharunderscore}{\kern0pt}with{\isacharunderscore}{\kern0pt}parity\ {\isasymcirc}\isactrlsub c\ n\ {\isacharequal}{\kern0pt}\ left{\isacharunderscore}{\kern0pt}coproj\ {\isasymnat}\isactrlsub c\ {\isasymnat}\isactrlsub c\ {\isasymcirc}\isactrlsub c\ m{\isacharparenright}{\kern0pt}\isanewline
\ \ \ \ \ \ {\isasymor}\ {\isacharparenleft}{\kern0pt}{\isasymexists}m{\isachardot}{\kern0pt}\ m\ {\isasymin}\isactrlsub c\ {\isasymnat}\isactrlsub c\ {\isasymand}\ halve{\isacharunderscore}{\kern0pt}with{\isacharunderscore}{\kern0pt}parity\ {\isasymcirc}\isactrlsub c\ n\ {\isacharequal}{\kern0pt}\ right{\isacharunderscore}{\kern0pt}coproj\ {\isasymnat}\isactrlsub c\ {\isasymnat}\isactrlsub c\ {\isasymcirc}\isactrlsub c\ m{\isacharparenright}{\kern0pt}{\isachardoublequoteclose}\isanewline
\ \ \ \ \isacommand{by}\isamarkupfalse%
\ {\isacharparenleft}{\kern0pt}rule\ coprojs{\isacharunderscore}{\kern0pt}jointly{\isacharunderscore}{\kern0pt}surj{\isacharcomma}{\kern0pt}\ insert\ assms{\isacharcomma}{\kern0pt}\ typecheck{\isacharunderscore}{\kern0pt}cfuncs{\isacharparenright}{\kern0pt}\isanewline
\ \ \isacommand{then}\isamarkupfalse%
\ \isacommand{show}\isamarkupfalse%
\ {\isacharquery}{\kern0pt}thesis\isanewline
\ \ \isacommand{proof}\isamarkupfalse%
\ \isanewline
\ \ \ \ \isacommand{assume}\isamarkupfalse%
\ {\isachardoublequoteopen}{\isasymexists}m{\isachardot}{\kern0pt}\ m\ {\isasymin}\isactrlsub c\ {\isasymnat}\isactrlsub c\ {\isasymand}\ halve{\isacharunderscore}{\kern0pt}with{\isacharunderscore}{\kern0pt}parity\ {\isasymcirc}\isactrlsub c\ n\ {\isacharequal}{\kern0pt}\ left{\isacharunderscore}{\kern0pt}coproj\ {\isasymnat}\isactrlsub c\ {\isasymnat}\isactrlsub c\ {\isasymcirc}\isactrlsub c\ m{\isachardoublequoteclose}\isanewline
\ \ \ \ \isacommand{then}\isamarkupfalse%
\ \isacommand{obtain}\isamarkupfalse%
\ m\ \isakeyword{where}\ m{\isacharunderscore}{\kern0pt}type{\isacharcolon}{\kern0pt}\ {\isachardoublequoteopen}m\ {\isasymin}\isactrlsub c\ {\isasymnat}\isactrlsub c{\isachardoublequoteclose}\ \isakeyword{and}\ m{\isacharunderscore}{\kern0pt}def{\isacharcolon}{\kern0pt}\ {\isachardoublequoteopen}halve{\isacharunderscore}{\kern0pt}with{\isacharunderscore}{\kern0pt}parity\ {\isasymcirc}\isactrlsub c\ n\ {\isacharequal}{\kern0pt}\ left{\isacharunderscore}{\kern0pt}coproj\ {\isasymnat}\isactrlsub c\ {\isasymnat}\isactrlsub c\ {\isasymcirc}\isactrlsub c\ m{\isachardoublequoteclose}\isanewline
\ \ \ \ \ \ \isacommand{by}\isamarkupfalse%
\ auto\isanewline
\ \ \ \ \isacommand{then}\isamarkupfalse%
\ \isacommand{have}\isamarkupfalse%
\ {\isachardoublequoteopen}{\isacharparenleft}{\kern0pt}{\isacharparenleft}{\kern0pt}nth{\isacharunderscore}{\kern0pt}even\ {\isasymamalg}\ nth{\isacharunderscore}{\kern0pt}odd{\isacharparenright}{\kern0pt}\ {\isasymcirc}\isactrlsub c\ halve{\isacharunderscore}{\kern0pt}with{\isacharunderscore}{\kern0pt}parity{\isacharparenright}{\kern0pt}\ {\isasymcirc}\isactrlsub c\ n\ {\isacharequal}{\kern0pt}\ {\isacharparenleft}{\kern0pt}{\isacharparenleft}{\kern0pt}nth{\isacharunderscore}{\kern0pt}even\ {\isasymamalg}\ nth{\isacharunderscore}{\kern0pt}odd{\isacharparenright}{\kern0pt}\ {\isasymcirc}\isactrlsub c\ left{\isacharunderscore}{\kern0pt}coproj\ {\isasymnat}\isactrlsub c\ {\isasymnat}\isactrlsub c{\isacharparenright}{\kern0pt}\ {\isasymcirc}\isactrlsub c\ m{\isachardoublequoteclose}\isanewline
\ \ \ \ \ \ \isacommand{by}\isamarkupfalse%
\ {\isacharparenleft}{\kern0pt}typecheck{\isacharunderscore}{\kern0pt}cfuncs{\isacharcomma}{\kern0pt}\ smt\ assms\ comp{\isacharunderscore}{\kern0pt}associative{\isadigit{2}}{\isacharparenright}{\kern0pt}\isanewline
\ \ \ \ \isacommand{then}\isamarkupfalse%
\ \isacommand{have}\isamarkupfalse%
\ {\isachardoublequoteopen}n\ {\isacharequal}{\kern0pt}\ nth{\isacharunderscore}{\kern0pt}even\ {\isasymcirc}\isactrlsub c\ m{\isachardoublequoteclose}\isanewline
\ \ \ \ \ \ \isacommand{using}\isamarkupfalse%
\ assms\ \isacommand{by}\isamarkupfalse%
\ {\isacharparenleft}{\kern0pt}typecheck{\isacharunderscore}{\kern0pt}cfuncs{\isacharunderscore}{\kern0pt}prems{\isacharcomma}{\kern0pt}\ smt\ comp{\isacharunderscore}{\kern0pt}associative{\isadigit{2}}\ halve{\isacharunderscore}{\kern0pt}with{\isacharunderscore}{\kern0pt}parity{\isacharunderscore}{\kern0pt}nth{\isacharunderscore}{\kern0pt}even\ id{\isacharunderscore}{\kern0pt}left{\isacharunderscore}{\kern0pt}unit{\isadigit{2}}\ nth{\isacharunderscore}{\kern0pt}even{\isacharunderscore}{\kern0pt}nth{\isacharunderscore}{\kern0pt}odd{\isacharunderscore}{\kern0pt}halve{\isacharunderscore}{\kern0pt}with{\isacharunderscore}{\kern0pt}parity{\isacharparenright}{\kern0pt}\isanewline
\ \ \ \ \isacommand{then}\isamarkupfalse%
\ \isacommand{have}\isamarkupfalse%
\ {\isachardoublequoteopen}{\isasymexists}m{\isachardot}{\kern0pt}\ m\ {\isasymin}\isactrlsub c\ {\isasymnat}\isactrlsub c\ {\isasymand}\ nth{\isacharunderscore}{\kern0pt}even\ {\isasymcirc}\isactrlsub c\ m\ {\isacharequal}{\kern0pt}\ n{\isachardoublequoteclose}\isanewline
\ \ \ \ \ \ \isacommand{using}\isamarkupfalse%
\ m{\isacharunderscore}{\kern0pt}type\ \isacommand{by}\isamarkupfalse%
\ auto\isanewline
\ \ \ \ \isacommand{then}\isamarkupfalse%
\ \isacommand{show}\isamarkupfalse%
\ {\isacharquery}{\kern0pt}thesis\isanewline
\ \ \ \ \ \ \isacommand{by}\isamarkupfalse%
\ simp\isanewline
\ \ \isacommand{next}\isamarkupfalse%
\isanewline
\ \ \ \ \isacommand{assume}\isamarkupfalse%
\ {\isachardoublequoteopen}{\isasymexists}m{\isachardot}{\kern0pt}\ m\ {\isasymin}\isactrlsub c\ {\isasymnat}\isactrlsub c\ {\isasymand}\ halve{\isacharunderscore}{\kern0pt}with{\isacharunderscore}{\kern0pt}parity\ {\isasymcirc}\isactrlsub c\ n\ {\isacharequal}{\kern0pt}\ right{\isacharunderscore}{\kern0pt}coproj\ {\isasymnat}\isactrlsub c\ {\isasymnat}\isactrlsub c\ {\isasymcirc}\isactrlsub c\ m{\isachardoublequoteclose}\isanewline
\ \ \ \ \isacommand{then}\isamarkupfalse%
\ \isacommand{obtain}\isamarkupfalse%
\ m\ \isakeyword{where}\ m{\isacharunderscore}{\kern0pt}type{\isacharcolon}{\kern0pt}\ {\isachardoublequoteopen}m\ {\isasymin}\isactrlsub c\ {\isasymnat}\isactrlsub c{\isachardoublequoteclose}\ \isakeyword{and}\ m{\isacharunderscore}{\kern0pt}def{\isacharcolon}{\kern0pt}\ {\isachardoublequoteopen}halve{\isacharunderscore}{\kern0pt}with{\isacharunderscore}{\kern0pt}parity\ {\isasymcirc}\isactrlsub c\ n\ {\isacharequal}{\kern0pt}\ right{\isacharunderscore}{\kern0pt}coproj\ {\isasymnat}\isactrlsub c\ {\isasymnat}\isactrlsub c\ {\isasymcirc}\isactrlsub c\ m{\isachardoublequoteclose}\isanewline
\ \ \ \ \ \ \isacommand{by}\isamarkupfalse%
\ auto\isanewline
\ \ \ \ \isacommand{then}\isamarkupfalse%
\ \isacommand{have}\isamarkupfalse%
\ {\isachardoublequoteopen}{\isacharparenleft}{\kern0pt}{\isacharparenleft}{\kern0pt}nth{\isacharunderscore}{\kern0pt}even\ {\isasymamalg}\ nth{\isacharunderscore}{\kern0pt}odd{\isacharparenright}{\kern0pt}\ {\isasymcirc}\isactrlsub c\ halve{\isacharunderscore}{\kern0pt}with{\isacharunderscore}{\kern0pt}parity{\isacharparenright}{\kern0pt}\ {\isasymcirc}\isactrlsub c\ n\ {\isacharequal}{\kern0pt}\ {\isacharparenleft}{\kern0pt}{\isacharparenleft}{\kern0pt}nth{\isacharunderscore}{\kern0pt}even\ {\isasymamalg}\ nth{\isacharunderscore}{\kern0pt}odd{\isacharparenright}{\kern0pt}\ {\isasymcirc}\isactrlsub c\ right{\isacharunderscore}{\kern0pt}coproj\ {\isasymnat}\isactrlsub c\ {\isasymnat}\isactrlsub c{\isacharparenright}{\kern0pt}\ {\isasymcirc}\isactrlsub c\ m{\isachardoublequoteclose}\isanewline
\ \ \ \ \ \ \isacommand{by}\isamarkupfalse%
\ {\isacharparenleft}{\kern0pt}typecheck{\isacharunderscore}{\kern0pt}cfuncs{\isacharcomma}{\kern0pt}\ smt\ assms\ comp{\isacharunderscore}{\kern0pt}associative{\isadigit{2}}{\isacharparenright}{\kern0pt}\isanewline
\ \ \ \ \isacommand{then}\isamarkupfalse%
\ \isacommand{have}\isamarkupfalse%
\ {\isachardoublequoteopen}n\ {\isacharequal}{\kern0pt}\ nth{\isacharunderscore}{\kern0pt}odd\ {\isasymcirc}\isactrlsub c\ m{\isachardoublequoteclose}\isanewline
\ \ \ \ \ \ \isacommand{using}\isamarkupfalse%
\ assms\ \isacommand{by}\isamarkupfalse%
\ {\isacharparenleft}{\kern0pt}typecheck{\isacharunderscore}{\kern0pt}cfuncs{\isacharunderscore}{\kern0pt}prems{\isacharcomma}{\kern0pt}\ smt\ comp{\isacharunderscore}{\kern0pt}associative{\isadigit{2}}\ halve{\isacharunderscore}{\kern0pt}with{\isacharunderscore}{\kern0pt}parity{\isacharunderscore}{\kern0pt}nth{\isacharunderscore}{\kern0pt}odd\ id{\isacharunderscore}{\kern0pt}left{\isacharunderscore}{\kern0pt}unit{\isadigit{2}}\ nth{\isacharunderscore}{\kern0pt}even{\isacharunderscore}{\kern0pt}nth{\isacharunderscore}{\kern0pt}odd{\isacharunderscore}{\kern0pt}halve{\isacharunderscore}{\kern0pt}with{\isacharunderscore}{\kern0pt}parity{\isacharparenright}{\kern0pt}\isanewline
\ \ \ \ \isacommand{then}\isamarkupfalse%
\ \isacommand{show}\isamarkupfalse%
\ {\isacharquery}{\kern0pt}thesis\isanewline
\ \ \ \ \ \ \isacommand{using}\isamarkupfalse%
\ m{\isacharunderscore}{\kern0pt}type\ \isacommand{by}\isamarkupfalse%
\ auto\isanewline
\ \ \isacommand{qed}\isamarkupfalse%
\isanewline
\isacommand{qed}\isamarkupfalse%
%
\endisatagproof
{\isafoldproof}%
%
\isadelimproof
\isanewline
%
\endisadelimproof
\isanewline
\isacommand{lemma}\isamarkupfalse%
\ is{\isacharunderscore}{\kern0pt}even{\isacharunderscore}{\kern0pt}exists{\isacharunderscore}{\kern0pt}nth{\isacharunderscore}{\kern0pt}even{\isacharcolon}{\kern0pt}\isanewline
\ \ \isakeyword{assumes}\ {\isachardoublequoteopen}is{\isacharunderscore}{\kern0pt}even\ {\isasymcirc}\isactrlsub c\ n\ {\isacharequal}{\kern0pt}\ {\isasymt}{\isachardoublequoteclose}\ \isakeyword{and}\ n{\isacharunderscore}{\kern0pt}type{\isacharbrackleft}{\kern0pt}type{\isacharunderscore}{\kern0pt}rule{\isacharbrackright}{\kern0pt}{\isacharcolon}{\kern0pt}\ {\isachardoublequoteopen}n\ {\isasymin}\isactrlsub c\ {\isasymnat}\isactrlsub c{\isachardoublequoteclose}\isanewline
\ \ \isakeyword{shows}\ {\isachardoublequoteopen}{\isasymexists}m{\isachardot}{\kern0pt}\ m\ {\isasymin}\isactrlsub c\ {\isasymnat}\isactrlsub c\ {\isasymand}\ n\ {\isacharequal}{\kern0pt}\ nth{\isacharunderscore}{\kern0pt}even\ {\isasymcirc}\isactrlsub c\ m{\isachardoublequoteclose}\isanewline
%
\isadelimproof
%
\endisadelimproof
%
\isatagproof
\isacommand{proof}\isamarkupfalse%
\ {\isacharparenleft}{\kern0pt}rule\ ccontr{\isacharparenright}{\kern0pt}\isanewline
\ \ \isacommand{assume}\isamarkupfalse%
\ {\isachardoublequoteopen}{\isasymnexists}m{\isachardot}{\kern0pt}\ m\ {\isasymin}\isactrlsub c\ {\isasymnat}\isactrlsub c\ {\isasymand}\ n\ {\isacharequal}{\kern0pt}\ nth{\isacharunderscore}{\kern0pt}even\ {\isasymcirc}\isactrlsub c\ m{\isachardoublequoteclose}\isanewline
\ \ \isacommand{then}\isamarkupfalse%
\ \isacommand{obtain}\isamarkupfalse%
\ m\ \isakeyword{where}\ m{\isacharunderscore}{\kern0pt}type{\isacharbrackleft}{\kern0pt}type{\isacharunderscore}{\kern0pt}rule{\isacharbrackright}{\kern0pt}{\isacharcolon}{\kern0pt}\ {\isachardoublequoteopen}m\ {\isasymin}\isactrlsub c\ {\isasymnat}\isactrlsub c{\isachardoublequoteclose}\ \isakeyword{and}\ n{\isacharunderscore}{\kern0pt}def{\isacharcolon}{\kern0pt}\ {\isachardoublequoteopen}n\ {\isacharequal}{\kern0pt}\ nth{\isacharunderscore}{\kern0pt}odd\ {\isasymcirc}\isactrlsub c\ m{\isachardoublequoteclose}\isanewline
\ \ \ \ \isacommand{using}\isamarkupfalse%
\ n{\isacharunderscore}{\kern0pt}type\ nth{\isacharunderscore}{\kern0pt}even{\isacharunderscore}{\kern0pt}or{\isacharunderscore}{\kern0pt}nth{\isacharunderscore}{\kern0pt}odd\ \isacommand{by}\isamarkupfalse%
\ blast\isanewline
\ \ \isacommand{then}\isamarkupfalse%
\ \isacommand{have}\isamarkupfalse%
\ {\isachardoublequoteopen}is{\isacharunderscore}{\kern0pt}even\ {\isasymcirc}\isactrlsub c\ nth{\isacharunderscore}{\kern0pt}odd\ {\isasymcirc}\isactrlsub c\ m\ {\isacharequal}{\kern0pt}\ {\isasymt}{\isachardoublequoteclose}\isanewline
\ \ \ \ \isacommand{using}\isamarkupfalse%
\ assms{\isacharparenleft}{\kern0pt}{\isadigit{1}}{\isacharparenright}{\kern0pt}\ \isacommand{by}\isamarkupfalse%
\ blast\isanewline
\ \ \isacommand{then}\isamarkupfalse%
\ \isacommand{have}\isamarkupfalse%
\ {\isachardoublequoteopen}is{\isacharunderscore}{\kern0pt}odd\ {\isasymcirc}\isactrlsub c\ nth{\isacharunderscore}{\kern0pt}odd\ {\isasymcirc}\isactrlsub c\ m\ {\isacharequal}{\kern0pt}\ {\isasymf}{\isachardoublequoteclose}\isanewline
\ \ \ \ \isacommand{using}\isamarkupfalse%
\ NOT{\isacharunderscore}{\kern0pt}true{\isacharunderscore}{\kern0pt}is{\isacharunderscore}{\kern0pt}false\ NOT{\isacharunderscore}{\kern0pt}type\ comp{\isacharunderscore}{\kern0pt}associative{\isadigit{2}}\ is{\isacharunderscore}{\kern0pt}even{\isacharunderscore}{\kern0pt}def{\isadigit{2}}\ is{\isacharunderscore}{\kern0pt}odd{\isacharunderscore}{\kern0pt}not{\isacharunderscore}{\kern0pt}is{\isacharunderscore}{\kern0pt}even\ n{\isacharunderscore}{\kern0pt}def\ n{\isacharunderscore}{\kern0pt}type\ \isacommand{by}\isamarkupfalse%
\ fastforce\isanewline
\ \ \isacommand{then}\isamarkupfalse%
\ \isacommand{have}\isamarkupfalse%
\ {\isachardoublequoteopen}{\isasymt}\ {\isasymcirc}\isactrlsub c\ {\isasymbeta}\isactrlbsub {\isasymnat}\isactrlsub c\isactrlesub \ {\isasymcirc}\isactrlsub c\ m\ {\isacharequal}{\kern0pt}\ {\isasymf}{\isachardoublequoteclose}\isanewline
\ \ \ \ \isacommand{by}\isamarkupfalse%
\ {\isacharparenleft}{\kern0pt}typecheck{\isacharunderscore}{\kern0pt}cfuncs{\isacharunderscore}{\kern0pt}prems{\isacharcomma}{\kern0pt}\ smt\ comp{\isacharunderscore}{\kern0pt}associative{\isadigit{2}}\ is{\isacharunderscore}{\kern0pt}odd{\isacharunderscore}{\kern0pt}nth{\isacharunderscore}{\kern0pt}odd{\isacharunderscore}{\kern0pt}true\ terminal{\isacharunderscore}{\kern0pt}func{\isacharunderscore}{\kern0pt}type\ true{\isacharunderscore}{\kern0pt}func{\isacharunderscore}{\kern0pt}type{\isacharparenright}{\kern0pt}\isanewline
\ \ \isacommand{then}\isamarkupfalse%
\ \isacommand{have}\isamarkupfalse%
\ {\isachardoublequoteopen}{\isasymt}\ {\isacharequal}{\kern0pt}\ {\isasymf}{\isachardoublequoteclose}\isanewline
\ \ \ \ \isacommand{by}\isamarkupfalse%
\ {\isacharparenleft}{\kern0pt}typecheck{\isacharunderscore}{\kern0pt}cfuncs{\isacharunderscore}{\kern0pt}prems{\isacharcomma}{\kern0pt}\ metis\ id{\isacharunderscore}{\kern0pt}right{\isacharunderscore}{\kern0pt}unit{\isadigit{2}}\ id{\isacharunderscore}{\kern0pt}type\ one{\isacharunderscore}{\kern0pt}unique{\isacharunderscore}{\kern0pt}element{\isacharparenright}{\kern0pt}\isanewline
\ \ \isacommand{then}\isamarkupfalse%
\ \isacommand{show}\isamarkupfalse%
\ False\isanewline
\ \ \ \ \isacommand{using}\isamarkupfalse%
\ true{\isacharunderscore}{\kern0pt}false{\isacharunderscore}{\kern0pt}distinct\ \isacommand{by}\isamarkupfalse%
\ auto\isanewline
\isacommand{qed}\isamarkupfalse%
%
\endisatagproof
{\isafoldproof}%
%
\isadelimproof
\isanewline
%
\endisadelimproof
\isanewline
\isacommand{lemma}\isamarkupfalse%
\ is{\isacharunderscore}{\kern0pt}odd{\isacharunderscore}{\kern0pt}exists{\isacharunderscore}{\kern0pt}nth{\isacharunderscore}{\kern0pt}odd{\isacharcolon}{\kern0pt}\isanewline
\ \ \isakeyword{assumes}\ {\isachardoublequoteopen}is{\isacharunderscore}{\kern0pt}odd\ {\isasymcirc}\isactrlsub c\ n\ {\isacharequal}{\kern0pt}\ {\isasymt}{\isachardoublequoteclose}\ \isakeyword{and}\ n{\isacharunderscore}{\kern0pt}type{\isacharbrackleft}{\kern0pt}type{\isacharunderscore}{\kern0pt}rule{\isacharbrackright}{\kern0pt}{\isacharcolon}{\kern0pt}\ {\isachardoublequoteopen}n\ {\isasymin}\isactrlsub c\ {\isasymnat}\isactrlsub c{\isachardoublequoteclose}\isanewline
\ \ \isakeyword{shows}\ {\isachardoublequoteopen}{\isasymexists}m{\isachardot}{\kern0pt}\ m\ {\isasymin}\isactrlsub c\ {\isasymnat}\isactrlsub c\ {\isasymand}\ n\ {\isacharequal}{\kern0pt}\ nth{\isacharunderscore}{\kern0pt}odd\ {\isasymcirc}\isactrlsub c\ m{\isachardoublequoteclose}\isanewline
%
\isadelimproof
%
\endisadelimproof
%
\isatagproof
\isacommand{proof}\isamarkupfalse%
\ {\isacharparenleft}{\kern0pt}rule\ ccontr{\isacharparenright}{\kern0pt}\isanewline
\ \ \isacommand{assume}\isamarkupfalse%
\ {\isachardoublequoteopen}{\isasymnexists}m{\isachardot}{\kern0pt}\ m\ {\isasymin}\isactrlsub c\ {\isasymnat}\isactrlsub c\ {\isasymand}\ n\ {\isacharequal}{\kern0pt}\ nth{\isacharunderscore}{\kern0pt}odd\ {\isasymcirc}\isactrlsub c\ m{\isachardoublequoteclose}\isanewline
\ \ \isacommand{then}\isamarkupfalse%
\ \isacommand{obtain}\isamarkupfalse%
\ m\ \isakeyword{where}\ m{\isacharunderscore}{\kern0pt}type{\isacharbrackleft}{\kern0pt}type{\isacharunderscore}{\kern0pt}rule{\isacharbrackright}{\kern0pt}{\isacharcolon}{\kern0pt}\ {\isachardoublequoteopen}m\ {\isasymin}\isactrlsub c\ {\isasymnat}\isactrlsub c{\isachardoublequoteclose}\ \isakeyword{and}\ n{\isacharunderscore}{\kern0pt}def{\isacharcolon}{\kern0pt}\ {\isachardoublequoteopen}n\ {\isacharequal}{\kern0pt}\ nth{\isacharunderscore}{\kern0pt}even\ {\isasymcirc}\isactrlsub c\ m{\isachardoublequoteclose}\isanewline
\ \ \ \ \isacommand{using}\isamarkupfalse%
\ n{\isacharunderscore}{\kern0pt}type\ nth{\isacharunderscore}{\kern0pt}even{\isacharunderscore}{\kern0pt}or{\isacharunderscore}{\kern0pt}nth{\isacharunderscore}{\kern0pt}odd\ \isacommand{by}\isamarkupfalse%
\ blast\isanewline
\ \ \isacommand{then}\isamarkupfalse%
\ \isacommand{have}\isamarkupfalse%
\ {\isachardoublequoteopen}is{\isacharunderscore}{\kern0pt}odd\ {\isasymcirc}\isactrlsub c\ nth{\isacharunderscore}{\kern0pt}even\ {\isasymcirc}\isactrlsub c\ m\ {\isacharequal}{\kern0pt}\ {\isasymt}{\isachardoublequoteclose}\isanewline
\ \ \ \ \isacommand{using}\isamarkupfalse%
\ assms{\isacharparenleft}{\kern0pt}{\isadigit{1}}{\isacharparenright}{\kern0pt}\ \isacommand{by}\isamarkupfalse%
\ blast\isanewline
\ \ \isacommand{then}\isamarkupfalse%
\ \isacommand{have}\isamarkupfalse%
\ {\isachardoublequoteopen}is{\isacharunderscore}{\kern0pt}even\ {\isasymcirc}\isactrlsub c\ nth{\isacharunderscore}{\kern0pt}even\ {\isasymcirc}\isactrlsub c\ m\ {\isacharequal}{\kern0pt}\ {\isasymf}{\isachardoublequoteclose}\isanewline
\ \ \ \ \isacommand{using}\isamarkupfalse%
\ NOT{\isacharunderscore}{\kern0pt}true{\isacharunderscore}{\kern0pt}is{\isacharunderscore}{\kern0pt}false\ NOT{\isacharunderscore}{\kern0pt}type\ comp{\isacharunderscore}{\kern0pt}associative{\isadigit{2}}\ is{\isacharunderscore}{\kern0pt}even{\isacharunderscore}{\kern0pt}not{\isacharunderscore}{\kern0pt}is{\isacharunderscore}{\kern0pt}odd\ is{\isacharunderscore}{\kern0pt}odd{\isacharunderscore}{\kern0pt}def{\isadigit{2}}\ n{\isacharunderscore}{\kern0pt}def\ n{\isacharunderscore}{\kern0pt}type\ \isacommand{by}\isamarkupfalse%
\ fastforce\isanewline
\ \ \isacommand{then}\isamarkupfalse%
\ \isacommand{have}\isamarkupfalse%
\ {\isachardoublequoteopen}{\isasymt}\ {\isasymcirc}\isactrlsub c\ {\isasymbeta}\isactrlbsub {\isasymnat}\isactrlsub c\isactrlesub \ {\isasymcirc}\isactrlsub c\ m\ {\isacharequal}{\kern0pt}\ {\isasymf}{\isachardoublequoteclose}\isanewline
\ \ \ \ \isacommand{by}\isamarkupfalse%
\ {\isacharparenleft}{\kern0pt}typecheck{\isacharunderscore}{\kern0pt}cfuncs{\isacharunderscore}{\kern0pt}prems{\isacharcomma}{\kern0pt}\ smt\ comp{\isacharunderscore}{\kern0pt}associative{\isadigit{2}}\ is{\isacharunderscore}{\kern0pt}even{\isacharunderscore}{\kern0pt}nth{\isacharunderscore}{\kern0pt}even{\isacharunderscore}{\kern0pt}true\ terminal{\isacharunderscore}{\kern0pt}func{\isacharunderscore}{\kern0pt}type\ true{\isacharunderscore}{\kern0pt}func{\isacharunderscore}{\kern0pt}type{\isacharparenright}{\kern0pt}\isanewline
\ \ \isacommand{then}\isamarkupfalse%
\ \isacommand{have}\isamarkupfalse%
\ {\isachardoublequoteopen}{\isasymt}\ {\isacharequal}{\kern0pt}\ {\isasymf}{\isachardoublequoteclose}\isanewline
\ \ \ \ \isacommand{by}\isamarkupfalse%
\ {\isacharparenleft}{\kern0pt}typecheck{\isacharunderscore}{\kern0pt}cfuncs{\isacharunderscore}{\kern0pt}prems{\isacharcomma}{\kern0pt}\ metis\ id{\isacharunderscore}{\kern0pt}right{\isacharunderscore}{\kern0pt}unit{\isadigit{2}}\ id{\isacharunderscore}{\kern0pt}type\ one{\isacharunderscore}{\kern0pt}unique{\isacharunderscore}{\kern0pt}element{\isacharparenright}{\kern0pt}\isanewline
\ \ \isacommand{then}\isamarkupfalse%
\ \isacommand{show}\isamarkupfalse%
\ False\isanewline
\ \ \ \ \isacommand{using}\isamarkupfalse%
\ true{\isacharunderscore}{\kern0pt}false{\isacharunderscore}{\kern0pt}distinct\ \isacommand{by}\isamarkupfalse%
\ auto\isanewline
\isacommand{qed}\isamarkupfalse%
%
\endisatagproof
{\isafoldproof}%
%
\isadelimproof
\isanewline
%
\endisadelimproof
%
\isadelimtheory
\isanewline
%
\endisadelimtheory
%
\isatagtheory
\isacommand{end}\isamarkupfalse%
%
\endisatagtheory
{\isafoldtheory}%
%
\isadelimtheory
%
\endisadelimtheory
%
\end{isabellebody}%
\endinput
%:%file=~/ETCS/Category_Set/Nat_Parity.thy%:%
%:%11=1%:%
%:%27=3%:%
%:%28=3%:%
%:%29=4%:%
%:%30=5%:%
%:%44=7%:%
%:%54=9%:%
%:%55=9%:%
%:%56=10%:%
%:%58=12%:%
%:%59=13%:%
%:%60=14%:%
%:%61=14%:%
%:%62=15%:%
%:%65=16%:%
%:%69=16%:%
%:%70=16%:%
%:%71=16%:%
%:%76=16%:%
%:%79=17%:%
%:%80=18%:%
%:%81=18%:%
%:%82=19%:%
%:%85=20%:%
%:%89=20%:%
%:%90=20%:%
%:%95=20%:%
%:%98=21%:%
%:%99=22%:%
%:%100=22%:%
%:%101=23%:%
%:%104=24%:%
%:%108=24%:%
%:%109=24%:%
%:%114=24%:%
%:%117=25%:%
%:%118=26%:%
%:%119=26%:%
%:%120=27%:%
%:%123=28%:%
%:%127=28%:%
%:%128=28%:%
%:%133=28%:%
%:%136=29%:%
%:%137=30%:%
%:%138=30%:%
%:%139=31%:%
%:%142=32%:%
%:%146=32%:%
%:%147=32%:%
%:%148=32%:%
%:%162=34%:%
%:%172=36%:%
%:%173=36%:%
%:%174=37%:%
%:%176=39%:%
%:%177=40%:%
%:%178=41%:%
%:%179=41%:%
%:%180=42%:%
%:%183=43%:%
%:%187=43%:%
%:%188=43%:%
%:%189=43%:%
%:%194=43%:%
%:%197=44%:%
%:%198=45%:%
%:%199=45%:%
%:%200=46%:%
%:%203=47%:%
%:%207=47%:%
%:%208=47%:%
%:%213=47%:%
%:%216=48%:%
%:%217=49%:%
%:%218=49%:%
%:%219=50%:%
%:%222=51%:%
%:%226=51%:%
%:%227=51%:%
%:%232=51%:%
%:%235=52%:%
%:%236=53%:%
%:%237=53%:%
%:%238=54%:%
%:%241=55%:%
%:%245=55%:%
%:%246=55%:%
%:%251=55%:%
%:%254=56%:%
%:%255=57%:%
%:%256=57%:%
%:%257=58%:%
%:%260=59%:%
%:%264=59%:%
%:%265=59%:%
%:%266=59%:%
%:%271=59%:%
%:%274=60%:%
%:%275=61%:%
%:%276=61%:%
%:%277=62%:%
%:%284=63%:%
%:%285=63%:%
%:%286=64%:%
%:%287=64%:%
%:%288=65%:%
%:%289=65%:%
%:%290=66%:%
%:%291=66%:%
%:%292=67%:%
%:%293=67%:%
%:%294=68%:%
%:%295=68%:%
%:%296=68%:%
%:%297=69%:%
%:%298=69%:%
%:%299=69%:%
%:%300=70%:%
%:%301=70%:%
%:%302=70%:%
%:%303=71%:%
%:%304=71%:%
%:%305=71%:%
%:%306=72%:%
%:%307=72%:%
%:%308=73%:%
%:%309=74%:%
%:%310=74%:%
%:%311=75%:%
%:%312=75%:%
%:%313=76%:%
%:%314=77%:%
%:%315=77%:%
%:%316=78%:%
%:%317=78%:%
%:%318=79%:%
%:%319=79%:%
%:%320=80%:%
%:%321=80%:%
%:%322=81%:%
%:%323=81%:%
%:%324=81%:%
%:%325=82%:%
%:%326=82%:%
%:%327=83%:%
%:%328=83%:%
%:%329=83%:%
%:%330=84%:%
%:%331=84%:%
%:%332=85%:%
%:%333=85%:%
%:%334=85%:%
%:%335=86%:%
%:%336=86%:%
%:%337=86%:%
%:%338=87%:%
%:%339=87%:%
%:%340=88%:%
%:%346=88%:%
%:%349=89%:%
%:%350=90%:%
%:%351=90%:%
%:%352=91%:%
%:%359=92%:%
%:%360=92%:%
%:%361=93%:%
%:%362=93%:%
%:%363=94%:%
%:%364=94%:%
%:%365=95%:%
%:%366=95%:%
%:%367=96%:%
%:%368=96%:%
%:%369=96%:%
%:%370=97%:%
%:%371=97%:%
%:%372=97%:%
%:%373=98%:%
%:%374=98%:%
%:%375=98%:%
%:%376=99%:%
%:%377=99%:%
%:%378=99%:%
%:%379=100%:%
%:%380=100%:%
%:%381=100%:%
%:%382=101%:%
%:%383=101%:%
%:%384=102%:%
%:%385=103%:%
%:%386=103%:%
%:%387=104%:%
%:%388=104%:%
%:%389=105%:%
%:%390=105%:%
%:%391=105%:%
%:%392=106%:%
%:%393=106%:%
%:%394=106%:%
%:%395=107%:%
%:%410=109%:%
%:%420=111%:%
%:%421=111%:%
%:%422=112%:%
%:%423=113%:%
%:%424=114%:%
%:%425=114%:%
%:%426=115%:%
%:%429=116%:%
%:%433=116%:%
%:%434=116%:%
%:%435=116%:%
%:%440=116%:%
%:%443=117%:%
%:%444=118%:%
%:%445=118%:%
%:%446=119%:%
%:%449=120%:%
%:%453=120%:%
%:%454=120%:%
%:%459=120%:%
%:%462=121%:%
%:%463=122%:%
%:%464=122%:%
%:%465=123%:%
%:%468=124%:%
%:%472=124%:%
%:%473=124%:%
%:%478=124%:%
%:%481=125%:%
%:%482=126%:%
%:%483=126%:%
%:%484=127%:%
%:%487=128%:%
%:%491=128%:%
%:%492=128%:%
%:%506=130%:%
%:%516=132%:%
%:%517=132%:%
%:%518=133%:%
%:%519=134%:%
%:%520=135%:%
%:%521=135%:%
%:%522=136%:%
%:%525=137%:%
%:%529=137%:%
%:%530=137%:%
%:%531=137%:%
%:%536=137%:%
%:%539=138%:%
%:%540=139%:%
%:%541=139%:%
%:%542=140%:%
%:%545=141%:%
%:%549=141%:%
%:%550=141%:%
%:%555=141%:%
%:%558=142%:%
%:%559=143%:%
%:%560=143%:%
%:%561=144%:%
%:%564=145%:%
%:%568=145%:%
%:%569=145%:%
%:%574=145%:%
%:%577=146%:%
%:%578=147%:%
%:%579=147%:%
%:%580=148%:%
%:%583=149%:%
%:%587=149%:%
%:%588=149%:%
%:%593=149%:%
%:%596=150%:%
%:%597=151%:%
%:%598=151%:%
%:%599=152%:%
%:%606=153%:%
%:%607=153%:%
%:%608=154%:%
%:%609=154%:%
%:%610=155%:%
%:%611=155%:%
%:%612=156%:%
%:%613=157%:%
%:%614=157%:%
%:%615=158%:%
%:%616=158%:%
%:%617=159%:%
%:%618=160%:%
%:%619=160%:%
%:%620=161%:%
%:%621=161%:%
%:%622=162%:%
%:%628=162%:%
%:%631=163%:%
%:%632=164%:%
%:%633=164%:%
%:%634=165%:%
%:%641=166%:%
%:%642=166%:%
%:%643=167%:%
%:%644=167%:%
%:%645=168%:%
%:%646=168%:%
%:%647=169%:%
%:%648=170%:%
%:%649=170%:%
%:%650=171%:%
%:%651=171%:%
%:%652=172%:%
%:%653=173%:%
%:%654=173%:%
%:%655=174%:%
%:%656=174%:%
%:%657=175%:%
%:%663=175%:%
%:%666=176%:%
%:%667=177%:%
%:%668=177%:%
%:%669=178%:%
%:%670=179%:%
%:%673=180%:%
%:%677=180%:%
%:%678=180%:%
%:%679=180%:%
%:%684=180%:%
%:%687=181%:%
%:%688=182%:%
%:%689=182%:%
%:%690=183%:%
%:%691=184%:%
%:%694=185%:%
%:%698=185%:%
%:%699=185%:%
%:%704=185%:%
%:%707=186%:%
%:%708=187%:%
%:%709=187%:%
%:%710=188%:%
%:%717=189%:%
%:%718=189%:%
%:%719=190%:%
%:%720=190%:%
%:%721=191%:%
%:%722=191%:%
%:%723=192%:%
%:%724=192%:%
%:%725=193%:%
%:%726=193%:%
%:%727=194%:%
%:%728=194%:%
%:%729=194%:%
%:%730=195%:%
%:%731=195%:%
%:%732=196%:%
%:%733=196%:%
%:%734=196%:%
%:%735=197%:%
%:%736=197%:%
%:%737=198%:%
%:%738=198%:%
%:%739=198%:%
%:%740=199%:%
%:%741=199%:%
%:%742=199%:%
%:%743=200%:%
%:%744=200%:%
%:%745=201%:%
%:%746=202%:%
%:%747=202%:%
%:%748=203%:%
%:%749=203%:%
%:%750=204%:%
%:%751=204%:%
%:%752=205%:%
%:%753=205%:%
%:%754=206%:%
%:%755=206%:%
%:%756=206%:%
%:%757=207%:%
%:%758=207%:%
%:%759=208%:%
%:%760=208%:%
%:%761=208%:%
%:%762=209%:%
%:%763=209%:%
%:%764=210%:%
%:%765=210%:%
%:%766=210%:%
%:%767=211%:%
%:%768=211%:%
%:%769=211%:%
%:%770=212%:%
%:%771=212%:%
%:%772=212%:%
%:%773=213%:%
%:%774=213%:%
%:%775=214%:%
%:%776=214%:%
%:%777=214%:%
%:%778=215%:%
%:%779=215%:%
%:%780=215%:%
%:%781=216%:%
%:%782=216%:%
%:%783=217%:%
%:%784=218%:%
%:%785=218%:%
%:%786=219%:%
%:%787=219%:%
%:%788=220%:%
%:%794=220%:%
%:%797=221%:%
%:%798=222%:%
%:%799=222%:%
%:%800=223%:%
%:%807=224%:%
%:%808=224%:%
%:%809=225%:%
%:%810=225%:%
%:%811=226%:%
%:%812=226%:%
%:%813=227%:%
%:%814=227%:%
%:%815=228%:%
%:%816=228%:%
%:%817=229%:%
%:%818=229%:%
%:%819=229%:%
%:%820=230%:%
%:%821=230%:%
%:%822=230%:%
%:%823=231%:%
%:%824=231%:%
%:%825=231%:%
%:%826=232%:%
%:%827=232%:%
%:%828=233%:%
%:%829=233%:%
%:%830=233%:%
%:%831=234%:%
%:%832=234%:%
%:%833=234%:%
%:%834=235%:%
%:%835=235%:%
%:%836=236%:%
%:%837=237%:%
%:%838=237%:%
%:%839=238%:%
%:%840=238%:%
%:%841=239%:%
%:%842=239%:%
%:%843=240%:%
%:%844=240%:%
%:%845=241%:%
%:%846=241%:%
%:%847=241%:%
%:%848=242%:%
%:%849=242%:%
%:%850=243%:%
%:%851=243%:%
%:%852=243%:%
%:%853=244%:%
%:%854=244%:%
%:%855=245%:%
%:%856=245%:%
%:%857=245%:%
%:%858=246%:%
%:%859=246%:%
%:%860=246%:%
%:%861=247%:%
%:%862=247%:%
%:%863=247%:%
%:%864=248%:%
%:%865=248%:%
%:%866=249%:%
%:%867=249%:%
%:%868=249%:%
%:%869=250%:%
%:%870=250%:%
%:%871=250%:%
%:%872=251%:%
%:%873=251%:%
%:%874=252%:%
%:%875=253%:%
%:%876=253%:%
%:%877=254%:%
%:%878=254%:%
%:%879=255%:%
%:%885=255%:%
%:%888=256%:%
%:%889=257%:%
%:%890=257%:%
%:%891=258%:%
%:%894=259%:%
%:%898=259%:%
%:%899=259%:%
%:%900=260%:%
%:%905=260%:%
%:%908=261%:%
%:%909=262%:%
%:%910=262%:%
%:%911=263%:%
%:%914=264%:%
%:%918=264%:%
%:%919=264%:%
%:%920=265%:%
%:%925=265%:%
%:%928=266%:%
%:%929=267%:%
%:%930=267%:%
%:%931=268%:%
%:%938=269%:%
%:%939=269%:%
%:%940=270%:%
%:%941=270%:%
%:%942=271%:%
%:%943=272%:%
%:%944=272%:%
%:%945=273%:%
%:%946=273%:%
%:%947=273%:%
%:%948=274%:%
%:%949=274%:%
%:%950=275%:%
%:%951=275%:%
%:%952=275%:%
%:%953=276%:%
%:%954=276%:%
%:%955=277%:%
%:%956=277%:%
%:%957=277%:%
%:%958=278%:%
%:%959=278%:%
%:%960=279%:%
%:%961=279%:%
%:%962=279%:%
%:%963=280%:%
%:%964=280%:%
%:%965=281%:%
%:%966=281%:%
%:%967=281%:%
%:%968=282%:%
%:%969=282%:%
%:%970=283%:%
%:%971=283%:%
%:%972=283%:%
%:%973=284%:%
%:%974=284%:%
%:%975=285%:%
%:%976=285%:%
%:%977=286%:%
%:%978=286%:%
%:%979=287%:%
%:%980=287%:%
%:%981=288%:%
%:%982=288%:%
%:%983=289%:%
%:%984=289%:%
%:%985=290%:%
%:%986=291%:%
%:%987=291%:%
%:%988=292%:%
%:%989=292%:%
%:%990=292%:%
%:%991=293%:%
%:%992=293%:%
%:%993=294%:%
%:%994=294%:%
%:%995=294%:%
%:%996=295%:%
%:%997=295%:%
%:%998=295%:%
%:%999=296%:%
%:%1000=296%:%
%:%1001=296%:%
%:%1002=297%:%
%:%1003=297%:%
%:%1004=297%:%
%:%1005=298%:%
%:%1006=298%:%
%:%1007=299%:%
%:%1008=299%:%
%:%1009=300%:%
%:%1010=300%:%
%:%1011=300%:%
%:%1012=301%:%
%:%1013=301%:%
%:%1014=301%:%
%:%1015=302%:%
%:%1021=302%:%
%:%1024=303%:%
%:%1025=304%:%
%:%1026=304%:%
%:%1027=305%:%
%:%1034=306%:%
%:%1035=306%:%
%:%1036=307%:%
%:%1037=307%:%
%:%1038=308%:%
%:%1039=308%:%
%:%1040=309%:%
%:%1041=309%:%
%:%1042=309%:%
%:%1043=310%:%
%:%1044=310%:%
%:%1045=311%:%
%:%1046=311%:%
%:%1047=311%:%
%:%1048=312%:%
%:%1049=312%:%
%:%1050=313%:%
%:%1051=313%:%
%:%1052=313%:%
%:%1053=314%:%
%:%1054=314%:%
%:%1055=315%:%
%:%1056=315%:%
%:%1057=315%:%
%:%1058=316%:%
%:%1059=316%:%
%:%1060=317%:%
%:%1061=317%:%
%:%1062=317%:%
%:%1063=318%:%
%:%1064=318%:%
%:%1065=319%:%
%:%1066=319%:%
%:%1067=319%:%
%:%1068=320%:%
%:%1069=320%:%
%:%1070=321%:%
%:%1071=321%:%
%:%1072=322%:%
%:%1073=322%:%
%:%1074=323%:%
%:%1075=323%:%
%:%1076=324%:%
%:%1077=324%:%
%:%1078=325%:%
%:%1079=325%:%
%:%1080=326%:%
%:%1081=326%:%
%:%1082=326%:%
%:%1083=327%:%
%:%1084=327%:%
%:%1085=328%:%
%:%1086=328%:%
%:%1087=328%:%
%:%1088=329%:%
%:%1089=329%:%
%:%1090=330%:%
%:%1091=330%:%
%:%1092=330%:%
%:%1093=331%:%
%:%1094=331%:%
%:%1095=332%:%
%:%1096=332%:%
%:%1097=332%:%
%:%1098=333%:%
%:%1099=333%:%
%:%1100=333%:%
%:%1101=334%:%
%:%1102=334%:%
%:%1103=334%:%
%:%1104=335%:%
%:%1105=335%:%
%:%1106=336%:%
%:%1107=336%:%
%:%1108=337%:%
%:%1109=337%:%
%:%1110=337%:%
%:%1111=338%:%
%:%1112=338%:%
%:%1113=338%:%
%:%1114=339%:%
%:%1115=339%:%
%:%1116=339%:%
%:%1117=340%:%
%:%1118=340%:%
%:%1119=340%:%
%:%1120=341%:%
%:%1121=341%:%
%:%1122=342%:%
%:%1123=342%:%
%:%1124=342%:%
%:%1125=343%:%
%:%1126=343%:%
%:%1127=343%:%
%:%1128=344%:%
%:%1143=346%:%
%:%1153=348%:%
%:%1154=348%:%
%:%1155=349%:%
%:%1157=351%:%
%:%1158=352%:%
%:%1159=353%:%
%:%1160=353%:%
%:%1161=354%:%
%:%1163=356%:%
%:%1166=357%:%
%:%1170=357%:%
%:%1171=357%:%
%:%1172=357%:%
%:%1177=357%:%
%:%1180=358%:%
%:%1181=359%:%
%:%1182=359%:%
%:%1183=360%:%
%:%1186=361%:%
%:%1190=361%:%
%:%1191=361%:%
%:%1196=361%:%
%:%1199=362%:%
%:%1200=363%:%
%:%1201=363%:%
%:%1202=364%:%
%:%1205=365%:%
%:%1209=365%:%
%:%1210=365%:%
%:%1215=365%:%
%:%1218=366%:%
%:%1219=367%:%
%:%1220=367%:%
%:%1221=368%:%
%:%1224=369%:%
%:%1228=369%:%
%:%1229=369%:%
%:%1234=369%:%
%:%1237=370%:%
%:%1238=371%:%
%:%1239=371%:%
%:%1240=372%:%
%:%1247=373%:%
%:%1248=373%:%
%:%1249=374%:%
%:%1250=374%:%
%:%1251=375%:%
%:%1252=375%:%
%:%1253=376%:%
%:%1254=376%:%
%:%1255=377%:%
%:%1256=377%:%
%:%1257=378%:%
%:%1258=378%:%
%:%1259=378%:%
%:%1260=379%:%
%:%1261=379%:%
%:%1262=380%:%
%:%1263=380%:%
%:%1264=380%:%
%:%1265=381%:%
%:%1266=381%:%
%:%1267=382%:%
%:%1268=382%:%
%:%1269=382%:%
%:%1270=383%:%
%:%1271=383%:%
%:%1272=383%:%
%:%1273=384%:%
%:%1274=384%:%
%:%1275=385%:%
%:%1276=386%:%
%:%1277=386%:%
%:%1278=387%:%
%:%1279=388%:%
%:%1280=388%:%
%:%1281=389%:%
%:%1282=389%:%
%:%1283=390%:%
%:%1284=390%:%
%:%1285=391%:%
%:%1286=391%:%
%:%1287=391%:%
%:%1288=392%:%
%:%1289=392%:%
%:%1290=393%:%
%:%1291=393%:%
%:%1292=393%:%
%:%1293=394%:%
%:%1294=394%:%
%:%1295=395%:%
%:%1296=395%:%
%:%1297=395%:%
%:%1298=396%:%
%:%1299=396%:%
%:%1300=397%:%
%:%1301=397%:%
%:%1302=397%:%
%:%1303=398%:%
%:%1304=399%:%
%:%1305=399%:%
%:%1306=400%:%
%:%1307=400%:%
%:%1308=400%:%
%:%1309=401%:%
%:%1310=402%:%
%:%1311=402%:%
%:%1312=403%:%
%:%1313=403%:%
%:%1314=403%:%
%:%1316=405%:%
%:%1317=406%:%
%:%1318=406%:%
%:%1319=407%:%
%:%1320=407%:%
%:%1321=407%:%
%:%1322=408%:%
%:%1323=409%:%
%:%1324=409%:%
%:%1325=410%:%
%:%1326=410%:%
%:%1327=410%:%
%:%1328=411%:%
%:%1329=411%:%
%:%1330=411%:%
%:%1331=412%:%
%:%1332=412%:%
%:%1333=413%:%
%:%1334=414%:%
%:%1335=414%:%
%:%1336=415%:%
%:%1337=416%:%
%:%1338=416%:%
%:%1339=417%:%
%:%1345=417%:%
%:%1348=418%:%
%:%1349=419%:%
%:%1350=419%:%
%:%1351=420%:%
%:%1358=421%:%
%:%1359=421%:%
%:%1360=422%:%
%:%1361=422%:%
%:%1362=423%:%
%:%1363=423%:%
%:%1364=424%:%
%:%1365=424%:%
%:%1366=425%:%
%:%1367=425%:%
%:%1368=426%:%
%:%1369=426%:%
%:%1370=426%:%
%:%1371=427%:%
%:%1372=427%:%
%:%1373=428%:%
%:%1374=428%:%
%:%1375=428%:%
%:%1376=429%:%
%:%1377=429%:%
%:%1378=430%:%
%:%1379=430%:%
%:%1380=430%:%
%:%1381=431%:%
%:%1382=431%:%
%:%1383=432%:%
%:%1384=432%:%
%:%1385=432%:%
%:%1386=433%:%
%:%1387=433%:%
%:%1388=434%:%
%:%1389=434%:%
%:%1390=434%:%
%:%1391=435%:%
%:%1392=435%:%
%:%1393=436%:%
%:%1394=436%:%
%:%1395=436%:%
%:%1396=437%:%
%:%1397=437%:%
%:%1398=438%:%
%:%1399=438%:%
%:%1400=438%:%
%:%1401=439%:%
%:%1402=439%:%
%:%1403=440%:%
%:%1404=440%:%
%:%1405=440%:%
%:%1406=441%:%
%:%1407=441%:%
%:%1408=441%:%
%:%1409=442%:%
%:%1410=442%:%
%:%1411=443%:%
%:%1412=444%:%
%:%1413=444%:%
%:%1414=445%:%
%:%1415=446%:%
%:%1416=446%:%
%:%1417=447%:%
%:%1418=447%:%
%:%1419=448%:%
%:%1420=448%:%
%:%1421=449%:%
%:%1422=449%:%
%:%1423=449%:%
%:%1424=450%:%
%:%1425=450%:%
%:%1426=451%:%
%:%1427=451%:%
%:%1428=451%:%
%:%1429=452%:%
%:%1430=452%:%
%:%1431=453%:%
%:%1432=453%:%
%:%1433=453%:%
%:%1434=454%:%
%:%1435=455%:%
%:%1436=455%:%
%:%1437=456%:%
%:%1438=456%:%
%:%1439=456%:%
%:%1440=457%:%
%:%1441=458%:%
%:%1442=458%:%
%:%1443=459%:%
%:%1444=459%:%
%:%1445=459%:%
%:%1446=460%:%
%:%1447=461%:%
%:%1448=461%:%
%:%1449=462%:%
%:%1450=462%:%
%:%1451=462%:%
%:%1452=463%:%
%:%1453=464%:%
%:%1454=464%:%
%:%1455=465%:%
%:%1456=465%:%
%:%1457=465%:%
%:%1458=466%:%
%:%1459=466%:%
%:%1460=467%:%
%:%1461=467%:%
%:%1462=467%:%
%:%1463=468%:%
%:%1464=468%:%
%:%1465=468%:%
%:%1466=469%:%
%:%1467=469%:%
%:%1468=470%:%
%:%1469=471%:%
%:%1470=471%:%
%:%1471=472%:%
%:%1472=473%:%
%:%1473=473%:%
%:%1474=474%:%
%:%1480=474%:%
%:%1483=475%:%
%:%1484=476%:%
%:%1485=476%:%
%:%1486=477%:%
%:%1493=478%:%
%:%1494=478%:%
%:%1495=479%:%
%:%1496=479%:%
%:%1497=480%:%
%:%1498=480%:%
%:%1499=481%:%
%:%1500=481%:%
%:%1501=482%:%
%:%1502=482%:%
%:%1503=483%:%
%:%1504=483%:%
%:%1505=483%:%
%:%1506=484%:%
%:%1507=484%:%
%:%1508=485%:%
%:%1509=485%:%
%:%1510=485%:%
%:%1511=486%:%
%:%1512=486%:%
%:%1513=487%:%
%:%1514=487%:%
%:%1515=487%:%
%:%1516=488%:%
%:%1517=488%:%
%:%1518=489%:%
%:%1519=489%:%
%:%1520=489%:%
%:%1521=490%:%
%:%1522=490%:%
%:%1523=490%:%
%:%1524=491%:%
%:%1525=491%:%
%:%1526=491%:%
%:%1527=492%:%
%:%1528=492%:%
%:%1529=492%:%
%:%1530=493%:%
%:%1531=493%:%
%:%1532=494%:%
%:%1533=495%:%
%:%1534=495%:%
%:%1535=496%:%
%:%1536=497%:%
%:%1537=497%:%
%:%1538=498%:%
%:%1539=498%:%
%:%1540=499%:%
%:%1541=499%:%
%:%1542=500%:%
%:%1543=500%:%
%:%1544=500%:%
%:%1545=501%:%
%:%1546=501%:%
%:%1547=502%:%
%:%1548=502%:%
%:%1549=502%:%
%:%1550=503%:%
%:%1551=503%:%
%:%1552=504%:%
%:%1553=504%:%
%:%1554=504%:%
%:%1555=505%:%
%:%1556=505%:%
%:%1557=506%:%
%:%1558=506%:%
%:%1559=506%:%
%:%1560=507%:%
%:%1561=507%:%
%:%1562=508%:%
%:%1563=508%:%
%:%1564=508%:%
%:%1565=509%:%
%:%1566=509%:%
%:%1567=510%:%
%:%1568=510%:%
%:%1569=510%:%
%:%1570=511%:%
%:%1571=511%:%
%:%1572=512%:%
%:%1573=512%:%
%:%1574=512%:%
%:%1575=513%:%
%:%1576=513%:%
%:%1577=514%:%
%:%1578=514%:%
%:%1579=514%:%
%:%1580=515%:%
%:%1581=515%:%
%:%1582=515%:%
%:%1583=516%:%
%:%1584=516%:%
%:%1585=517%:%
%:%1586=518%:%
%:%1587=518%:%
%:%1588=519%:%
%:%1589=519%:%
%:%1590=519%:%
%:%1591=520%:%
%:%1597=520%:%
%:%1600=521%:%
%:%1601=522%:%
%:%1602=522%:%
%:%1603=523%:%
%:%1606=524%:%
%:%1610=524%:%
%:%1611=524%:%
%:%1616=524%:%
%:%1619=525%:%
%:%1620=526%:%
%:%1621=526%:%
%:%1622=527%:%
%:%1625=528%:%
%:%1629=528%:%
%:%1630=528%:%
%:%1631=529%:%
%:%1632=529%:%
%:%1633=530%:%
%:%1634=530%:%
%:%1635=531%:%
%:%1636=531%:%
%:%1637=532%:%
%:%1638=532%:%
%:%1639=533%:%
%:%1640=533%:%
%:%1641=534%:%
%:%1642=534%:%
%:%1643=535%:%
%:%1644=535%:%
%:%1645=536%:%
%:%1646=536%:%
%:%1647=537%:%
%:%1648=537%:%
%:%1649=538%:%
%:%1655=538%:%
%:%1658=539%:%
%:%1659=540%:%
%:%1660=540%:%
%:%1661=541%:%
%:%1664=542%:%
%:%1668=542%:%
%:%1669=542%:%
%:%1670=543%:%
%:%1671=543%:%
%:%1672=544%:%
%:%1673=544%:%
%:%1674=545%:%
%:%1675=545%:%
%:%1676=546%:%
%:%1677=546%:%
%:%1678=547%:%
%:%1679=547%:%
%:%1680=548%:%
%:%1681=548%:%
%:%1682=549%:%
%:%1683=549%:%
%:%1684=550%:%
%:%1685=550%:%
%:%1686=551%:%
%:%1687=551%:%
%:%1688=552%:%
%:%1694=552%:%
%:%1697=553%:%
%:%1698=554%:%
%:%1699=554%:%
%:%1700=555%:%
%:%1701=556%:%
%:%1702=557%:%
%:%1703=557%:%
%:%1704=558%:%
%:%1707=559%:%
%:%1711=559%:%
%:%1712=559%:%
%:%1713=559%:%
%:%1718=559%:%
%:%1721=560%:%
%:%1722=561%:%
%:%1723=561%:%
%:%1724=562%:%
%:%1727=563%:%
%:%1731=563%:%
%:%1732=563%:%
%:%1733=563%:%
%:%1738=563%:%
%:%1741=564%:%
%:%1742=565%:%
%:%1743=565%:%
%:%1744=566%:%
%:%1747=567%:%
%:%1751=567%:%
%:%1752=567%:%
%:%1753=567%:%
%:%1758=567%:%
%:%1761=568%:%
%:%1762=569%:%
%:%1763=569%:%
%:%1764=570%:%
%:%1771=571%:%
%:%1772=571%:%
%:%1773=572%:%
%:%1774=572%:%
%:%1775=573%:%
%:%1776=573%:%
%:%1777=574%:%
%:%1778=574%:%
%:%1779=575%:%
%:%1780=576%:%
%:%1781=576%:%
%:%1782=577%:%
%:%1783=577%:%
%:%1784=577%:%
%:%1785=578%:%
%:%1786=578%:%
%:%1787=579%:%
%:%1788=579%:%
%:%1789=579%:%
%:%1790=580%:%
%:%1791=580%:%
%:%1792=580%:%
%:%1793=581%:%
%:%1794=581%:%
%:%1795=581%:%
%:%1796=582%:%
%:%1797=582%:%
%:%1798=583%:%
%:%1799=583%:%
%:%1800=583%:%
%:%1801=584%:%
%:%1802=584%:%
%:%1803=584%:%
%:%1804=585%:%
%:%1805=585%:%
%:%1806=586%:%
%:%1807=587%:%
%:%1808=587%:%
%:%1809=588%:%
%:%1810=588%:%
%:%1811=589%:%
%:%1812=590%:%
%:%1813=590%:%
%:%1814=591%:%
%:%1815=592%:%
%:%1816=592%:%
%:%1817=593%:%
%:%1818=593%:%
%:%1819=594%:%
%:%1820=595%:%
%:%1821=595%:%
%:%1822=596%:%
%:%1823=596%:%
%:%1824=596%:%
%:%1828=600%:%
%:%1829=601%:%
%:%1830=601%:%
%:%1831=602%:%
%:%1832=602%:%
%:%1833=602%:%
%:%1834=603%:%
%:%1835=603%:%
%:%1836=604%:%
%:%1837=604%:%
%:%1838=604%:%
%:%1839=605%:%
%:%1840=605%:%
%:%1841=606%:%
%:%1842=606%:%
%:%1843=606%:%
%:%1844=607%:%
%:%1845=607%:%
%:%1846=608%:%
%:%1847=608%:%
%:%1848=608%:%
%:%1849=609%:%
%:%1850=609%:%
%:%1851=609%:%
%:%1852=610%:%
%:%1853=610%:%
%:%1854=611%:%
%:%1860=611%:%
%:%1863=612%:%
%:%1864=613%:%
%:%1865=613%:%
%:%1866=614%:%
%:%1873=615%:%
%:%1874=615%:%
%:%1875=616%:%
%:%1876=616%:%
%:%1877=617%:%
%:%1878=617%:%
%:%1879=618%:%
%:%1880=618%:%
%:%1881=619%:%
%:%1882=620%:%
%:%1883=620%:%
%:%1884=621%:%
%:%1885=621%:%
%:%1886=621%:%
%:%1887=622%:%
%:%1888=622%:%
%:%1889=623%:%
%:%1890=623%:%
%:%1891=623%:%
%:%1892=624%:%
%:%1893=624%:%
%:%1894=624%:%
%:%1895=625%:%
%:%1896=625%:%
%:%1897=625%:%
%:%1898=626%:%
%:%1899=626%:%
%:%1900=627%:%
%:%1901=627%:%
%:%1902=627%:%
%:%1903=628%:%
%:%1904=628%:%
%:%1905=628%:%
%:%1906=629%:%
%:%1907=629%:%
%:%1908=630%:%
%:%1909=631%:%
%:%1910=631%:%
%:%1911=632%:%
%:%1912=632%:%
%:%1913=633%:%
%:%1914=634%:%
%:%1915=634%:%
%:%1916=635%:%
%:%1917=636%:%
%:%1918=636%:%
%:%1919=637%:%
%:%1920=637%:%
%:%1921=638%:%
%:%1922=639%:%
%:%1923=639%:%
%:%1924=640%:%
%:%1925=640%:%
%:%1926=640%:%
%:%1930=644%:%
%:%1931=645%:%
%:%1932=645%:%
%:%1933=646%:%
%:%1934=646%:%
%:%1935=646%:%
%:%1936=647%:%
%:%1937=647%:%
%:%1938=648%:%
%:%1939=648%:%
%:%1940=648%:%
%:%1941=649%:%
%:%1942=649%:%
%:%1943=650%:%
%:%1944=650%:%
%:%1945=650%:%
%:%1946=651%:%
%:%1947=651%:%
%:%1948=652%:%
%:%1949=652%:%
%:%1950=652%:%
%:%1951=653%:%
%:%1952=653%:%
%:%1953=653%:%
%:%1954=654%:%
%:%1955=654%:%
%:%1956=655%:%
%:%1962=655%:%
%:%1965=656%:%
%:%1966=657%:%
%:%1967=657%:%
%:%1968=658%:%
%:%1969=659%:%
%:%1976=660%:%
%:%1977=660%:%
%:%1978=661%:%
%:%1979=661%:%
%:%1980=662%:%
%:%1981=663%:%
%:%1982=663%:%
%:%1983=664%:%
%:%1984=664%:%
%:%1985=664%:%
%:%1986=665%:%
%:%1987=665%:%
%:%1988=666%:%
%:%1989=666%:%
%:%1990=667%:%
%:%1991=667%:%
%:%1992=667%:%
%:%1993=668%:%
%:%1994=668%:%
%:%1995=669%:%
%:%1996=669%:%
%:%1997=669%:%
%:%1998=670%:%
%:%1999=670%:%
%:%2000=671%:%
%:%2001=671%:%
%:%2002=671%:%
%:%2003=672%:%
%:%2004=672%:%
%:%2005=672%:%
%:%2006=673%:%
%:%2007=673%:%
%:%2008=673%:%
%:%2009=674%:%
%:%2010=674%:%
%:%2011=674%:%
%:%2012=675%:%
%:%2013=675%:%
%:%2014=675%:%
%:%2015=676%:%
%:%2016=676%:%
%:%2017=677%:%
%:%2018=677%:%
%:%2019=678%:%
%:%2020=678%:%
%:%2021=679%:%
%:%2022=679%:%
%:%2023=679%:%
%:%2024=680%:%
%:%2025=680%:%
%:%2026=681%:%
%:%2027=681%:%
%:%2028=681%:%
%:%2029=682%:%
%:%2030=682%:%
%:%2031=683%:%
%:%2032=683%:%
%:%2033=683%:%
%:%2034=684%:%
%:%2035=684%:%
%:%2036=684%:%
%:%2037=685%:%
%:%2038=685%:%
%:%2039=685%:%
%:%2040=686%:%
%:%2041=686%:%
%:%2042=686%:%
%:%2043=687%:%
%:%2044=687%:%
%:%2045=688%:%
%:%2051=688%:%
%:%2054=689%:%
%:%2055=690%:%
%:%2056=690%:%
%:%2057=691%:%
%:%2058=692%:%
%:%2065=693%:%
%:%2066=693%:%
%:%2067=694%:%
%:%2068=694%:%
%:%2069=695%:%
%:%2070=695%:%
%:%2071=695%:%
%:%2072=696%:%
%:%2073=696%:%
%:%2074=696%:%
%:%2075=697%:%
%:%2076=697%:%
%:%2077=697%:%
%:%2078=698%:%
%:%2079=698%:%
%:%2080=698%:%
%:%2081=699%:%
%:%2082=699%:%
%:%2083=699%:%
%:%2084=700%:%
%:%2085=700%:%
%:%2086=700%:%
%:%2087=701%:%
%:%2088=701%:%
%:%2089=701%:%
%:%2090=702%:%
%:%2091=702%:%
%:%2092=703%:%
%:%2093=703%:%
%:%2094=703%:%
%:%2095=704%:%
%:%2096=704%:%
%:%2097=705%:%
%:%2098=705%:%
%:%2099=705%:%
%:%2100=706%:%
%:%2101=706%:%
%:%2102=706%:%
%:%2103=707%:%
%:%2109=707%:%
%:%2112=708%:%
%:%2113=709%:%
%:%2114=709%:%
%:%2115=710%:%
%:%2116=711%:%
%:%2123=712%:%
%:%2124=712%:%
%:%2125=713%:%
%:%2126=713%:%
%:%2127=714%:%
%:%2128=714%:%
%:%2129=714%:%
%:%2130=715%:%
%:%2131=715%:%
%:%2132=715%:%
%:%2133=716%:%
%:%2134=716%:%
%:%2135=716%:%
%:%2136=717%:%
%:%2137=717%:%
%:%2138=717%:%
%:%2139=718%:%
%:%2140=718%:%
%:%2141=718%:%
%:%2142=719%:%
%:%2143=719%:%
%:%2144=719%:%
%:%2145=720%:%
%:%2146=720%:%
%:%2147=720%:%
%:%2148=721%:%
%:%2149=721%:%
%:%2150=722%:%
%:%2151=722%:%
%:%2152=722%:%
%:%2153=723%:%
%:%2154=723%:%
%:%2155=724%:%
%:%2156=724%:%
%:%2157=724%:%
%:%2158=725%:%
%:%2159=725%:%
%:%2160=725%:%
%:%2161=726%:%
%:%2167=726%:%
%:%2172=727%:%
%:%2177=728%:%

%
\begin{isabellebody}%
\setisabellecontext{Cardinality}%
%
\isadelimdocument
%
\endisadelimdocument
%
\isatagdocument
%
\isamarkupsection{Cardinality and Finiteness%
}
\isamarkuptrue%
%
\endisatagdocument
{\isafolddocument}%
%
\isadelimdocument
%
\endisadelimdocument
%
\isadelimtheory
%
\endisadelimtheory
%
\isatagtheory
\isacommand{theory}\isamarkupfalse%
\ Cardinality\isanewline
\ \ \isakeyword{imports}\ Exponential{\isacharunderscore}{\kern0pt}Objects\isanewline
\isakeyword{begin}%
\endisatagtheory
{\isafoldtheory}%
%
\isadelimtheory
%
\endisadelimtheory
%
\begin{isamarkuptext}%
The definitions below correspond to Definition 2.6.1 in Halvorson.%
\end{isamarkuptext}\isamarkuptrue%
\isacommand{definition}\isamarkupfalse%
\ is{\isacharunderscore}{\kern0pt}finite\ {\isacharcolon}{\kern0pt}{\isacharcolon}{\kern0pt}\ {\isachardoublequoteopen}cset\ {\isasymRightarrow}\ bool{\isachardoublequoteclose}\ \ \isakeyword{where}\isanewline
\ \ \ {\isachardoublequoteopen}is{\isacharunderscore}{\kern0pt}finite\ X\ {\isasymlongleftrightarrow}\ {\isacharparenleft}{\kern0pt}{\isasymforall}m{\isachardot}{\kern0pt}\ {\isacharparenleft}{\kern0pt}m\ {\isacharcolon}{\kern0pt}\ X\ {\isasymrightarrow}\ X\ {\isasymand}\ monomorphism\ m{\isacharparenright}{\kern0pt}\ {\isasymlongrightarrow}\ isomorphism\ m{\isacharparenright}{\kern0pt}{\isachardoublequoteclose}\isanewline
\isanewline
\isacommand{definition}\isamarkupfalse%
\ is{\isacharunderscore}{\kern0pt}infinite\ {\isacharcolon}{\kern0pt}{\isacharcolon}{\kern0pt}\ {\isachardoublequoteopen}cset\ {\isasymRightarrow}\ bool{\isachardoublequoteclose}\ \ \isakeyword{where}\isanewline
\ \ \ {\isachardoublequoteopen}is{\isacharunderscore}{\kern0pt}infinite\ X\ {\isasymlongleftrightarrow}\ {\isacharparenleft}{\kern0pt}{\isasymexists}\ m{\isachardot}{\kern0pt}\ m\ {\isacharcolon}{\kern0pt}\ X\ {\isasymrightarrow}\ X\ {\isasymand}\ monomorphism\ m\ {\isasymand}\ {\isasymnot}surjective\ m{\isacharparenright}{\kern0pt}{\isachardoublequoteclose}\isanewline
\isanewline
\isacommand{lemma}\isamarkupfalse%
\ either{\isacharunderscore}{\kern0pt}finite{\isacharunderscore}{\kern0pt}or{\isacharunderscore}{\kern0pt}infinite{\isacharcolon}{\kern0pt}\isanewline
\ \ {\isachardoublequoteopen}is{\isacharunderscore}{\kern0pt}finite\ X\ {\isasymor}\ is{\isacharunderscore}{\kern0pt}infinite\ X{\isachardoublequoteclose}\isanewline
%
\isadelimproof
\ \ %
\endisadelimproof
%
\isatagproof
\isacommand{using}\isamarkupfalse%
\ epi{\isacharunderscore}{\kern0pt}mon{\isacharunderscore}{\kern0pt}is{\isacharunderscore}{\kern0pt}iso\ is{\isacharunderscore}{\kern0pt}finite{\isacharunderscore}{\kern0pt}def\ is{\isacharunderscore}{\kern0pt}infinite{\isacharunderscore}{\kern0pt}def\ surjective{\isacharunderscore}{\kern0pt}is{\isacharunderscore}{\kern0pt}epimorphism\ \isacommand{by}\isamarkupfalse%
\ blast%
\endisatagproof
{\isafoldproof}%
%
\isadelimproof
%
\endisadelimproof
%
\begin{isamarkuptext}%
The definition below corresponds to Definition 2.6.2 in Halvorson.%
\end{isamarkuptext}\isamarkuptrue%
\isacommand{definition}\isamarkupfalse%
\ is{\isacharunderscore}{\kern0pt}smaller{\isacharunderscore}{\kern0pt}than\ {\isacharcolon}{\kern0pt}{\isacharcolon}{\kern0pt}\ {\isachardoublequoteopen}cset\ {\isasymRightarrow}\ cset\ {\isasymRightarrow}\ bool{\isachardoublequoteclose}\ {\isacharparenleft}{\kern0pt}\isakeyword{infix}\ {\isachardoublequoteopen}{\isasymle}\isactrlsub c{\isachardoublequoteclose}\ {\isadigit{5}}{\isadigit{0}}{\isacharparenright}{\kern0pt}\ \isakeyword{where}\isanewline
\ \ \ {\isachardoublequoteopen}X\ {\isasymle}\isactrlsub c\ Y\ {\isasymlongleftrightarrow}\ {\isacharparenleft}{\kern0pt}{\isasymexists}\ m{\isachardot}{\kern0pt}\ m\ {\isacharcolon}{\kern0pt}\ X\ {\isasymrightarrow}\ Y\ {\isasymand}\ monomorphism\ m{\isacharparenright}{\kern0pt}{\isachardoublequoteclose}%
\begin{isamarkuptext}%
The purpose of the following lemma is simply to unify the two notations used in the book.%
\end{isamarkuptext}\isamarkuptrue%
\isacommand{lemma}\isamarkupfalse%
\ subobject{\isacharunderscore}{\kern0pt}iff{\isacharunderscore}{\kern0pt}smaller{\isacharunderscore}{\kern0pt}than{\isacharcolon}{\kern0pt}\isanewline
\ \ {\isachardoublequoteopen}{\isacharparenleft}{\kern0pt}X\ {\isasymle}\isactrlsub c\ Y{\isacharparenright}{\kern0pt}\ {\isacharequal}{\kern0pt}\ {\isacharparenleft}{\kern0pt}{\isasymexists}m{\isachardot}{\kern0pt}\ {\isacharparenleft}{\kern0pt}X{\isacharcomma}{\kern0pt}m{\isacharparenright}{\kern0pt}\ {\isasymsubseteq}\isactrlsub c\ Y{\isacharparenright}{\kern0pt}{\isachardoublequoteclose}\isanewline
%
\isadelimproof
\ \ %
\endisadelimproof
%
\isatagproof
\isacommand{using}\isamarkupfalse%
\ is{\isacharunderscore}{\kern0pt}smaller{\isacharunderscore}{\kern0pt}than{\isacharunderscore}{\kern0pt}def\ subobject{\isacharunderscore}{\kern0pt}of{\isacharunderscore}{\kern0pt}def{\isadigit{2}}\ \isacommand{by}\isamarkupfalse%
\ auto%
\endisatagproof
{\isafoldproof}%
%
\isadelimproof
\isanewline
%
\endisadelimproof
\isanewline
\isacommand{lemma}\isamarkupfalse%
\ set{\isacharunderscore}{\kern0pt}card{\isacharunderscore}{\kern0pt}transitive{\isacharcolon}{\kern0pt}\isanewline
\ \ \isakeyword{assumes}\ {\isachardoublequoteopen}A\ {\isasymle}\isactrlsub c\ B{\isachardoublequoteclose}\isanewline
\ \ \isakeyword{assumes}\ {\isachardoublequoteopen}B\ {\isasymle}\isactrlsub c\ C{\isachardoublequoteclose}\isanewline
\ \ \isakeyword{shows}\ \ \ {\isachardoublequoteopen}A\ {\isasymle}\isactrlsub c\ C{\isachardoublequoteclose}\isanewline
%
\isadelimproof
\ \ %
\endisadelimproof
%
\isatagproof
\isacommand{by}\isamarkupfalse%
\ {\isacharparenleft}{\kern0pt}typecheck{\isacharunderscore}{\kern0pt}cfuncs{\isacharcomma}{\kern0pt}\ metis\ {\isacharparenleft}{\kern0pt}full{\isacharunderscore}{\kern0pt}types{\isacharparenright}{\kern0pt}\ assms\ cfunc{\isacharunderscore}{\kern0pt}type{\isacharunderscore}{\kern0pt}def\ comp{\isacharunderscore}{\kern0pt}type\ composition{\isacharunderscore}{\kern0pt}of{\isacharunderscore}{\kern0pt}monic{\isacharunderscore}{\kern0pt}pair{\isacharunderscore}{\kern0pt}is{\isacharunderscore}{\kern0pt}monic\ is{\isacharunderscore}{\kern0pt}smaller{\isacharunderscore}{\kern0pt}than{\isacharunderscore}{\kern0pt}def{\isacharparenright}{\kern0pt}%
\endisatagproof
{\isafoldproof}%
%
\isadelimproof
\isanewline
%
\endisadelimproof
\isanewline
\isacommand{lemma}\isamarkupfalse%
\ all{\isacharunderscore}{\kern0pt}emptysets{\isacharunderscore}{\kern0pt}are{\isacharunderscore}{\kern0pt}finite{\isacharcolon}{\kern0pt}\isanewline
\ \ \isakeyword{assumes}\ {\isachardoublequoteopen}is{\isacharunderscore}{\kern0pt}empty\ X{\isachardoublequoteclose}\isanewline
\ \ \isakeyword{shows}\ {\isachardoublequoteopen}is{\isacharunderscore}{\kern0pt}finite\ X{\isachardoublequoteclose}\isanewline
%
\isadelimproof
\ \ %
\endisadelimproof
%
\isatagproof
\isacommand{by}\isamarkupfalse%
\ {\isacharparenleft}{\kern0pt}metis\ assms\ epi{\isacharunderscore}{\kern0pt}mon{\isacharunderscore}{\kern0pt}is{\isacharunderscore}{\kern0pt}iso\ epimorphism{\isacharunderscore}{\kern0pt}def{\isadigit{3}}\ is{\isacharunderscore}{\kern0pt}finite{\isacharunderscore}{\kern0pt}def\ is{\isacharunderscore}{\kern0pt}empty{\isacharunderscore}{\kern0pt}def\ one{\isacharunderscore}{\kern0pt}separator{\isacharparenright}{\kern0pt}%
\endisatagproof
{\isafoldproof}%
%
\isadelimproof
\isanewline
%
\endisadelimproof
\isanewline
\isacommand{lemma}\isamarkupfalse%
\ emptyset{\isacharunderscore}{\kern0pt}is{\isacharunderscore}{\kern0pt}smallest{\isacharunderscore}{\kern0pt}set{\isacharcolon}{\kern0pt}\isanewline
\ \ {\isachardoublequoteopen}{\isasymemptyset}\ {\isasymle}\isactrlsub c\ X{\isachardoublequoteclose}\isanewline
%
\isadelimproof
\ \ %
\endisadelimproof
%
\isatagproof
\isacommand{using}\isamarkupfalse%
\ empty{\isacharunderscore}{\kern0pt}subset\ is{\isacharunderscore}{\kern0pt}smaller{\isacharunderscore}{\kern0pt}than{\isacharunderscore}{\kern0pt}def\ subobject{\isacharunderscore}{\kern0pt}of{\isacharunderscore}{\kern0pt}def{\isadigit{2}}\ \isacommand{by}\isamarkupfalse%
\ auto%
\endisatagproof
{\isafoldproof}%
%
\isadelimproof
\isanewline
%
\endisadelimproof
\isanewline
\isacommand{lemma}\isamarkupfalse%
\ truth{\isacharunderscore}{\kern0pt}set{\isacharunderscore}{\kern0pt}is{\isacharunderscore}{\kern0pt}finite{\isacharcolon}{\kern0pt}\isanewline
\ \ {\isachardoublequoteopen}is{\isacharunderscore}{\kern0pt}finite\ {\isasymOmega}{\isachardoublequoteclose}\isanewline
%
\isadelimproof
\ \ %
\endisadelimproof
%
\isatagproof
\isacommand{unfolding}\isamarkupfalse%
\ is{\isacharunderscore}{\kern0pt}finite{\isacharunderscore}{\kern0pt}def\isanewline
\isacommand{proof}\isamarkupfalse%
{\isacharparenleft}{\kern0pt}clarify{\isacharparenright}{\kern0pt}\isanewline
\ \ \isacommand{fix}\isamarkupfalse%
\ m\ \isanewline
\ \ \isacommand{assume}\isamarkupfalse%
\ m{\isacharunderscore}{\kern0pt}type{\isacharbrackleft}{\kern0pt}type{\isacharunderscore}{\kern0pt}rule{\isacharbrackright}{\kern0pt}{\isacharcolon}{\kern0pt}\ {\isachardoublequoteopen}m\ {\isacharcolon}{\kern0pt}\ {\isasymOmega}\ {\isasymrightarrow}\ {\isasymOmega}{\isachardoublequoteclose}\isanewline
\ \ \isacommand{assume}\isamarkupfalse%
\ m{\isacharunderscore}{\kern0pt}mono{\isacharcolon}{\kern0pt}\ {\isachardoublequoteopen}monomorphism\ m{\isachardoublequoteclose}\isanewline
\ \ \isacommand{have}\isamarkupfalse%
\ {\isachardoublequoteopen}surjective\ m{\isachardoublequoteclose}\isanewline
\ \ \ \ \isacommand{unfolding}\isamarkupfalse%
\ surjective{\isacharunderscore}{\kern0pt}def\isanewline
\ \ \isacommand{proof}\isamarkupfalse%
{\isacharparenleft}{\kern0pt}clarify{\isacharparenright}{\kern0pt}\isanewline
\ \ \ \ \isacommand{fix}\isamarkupfalse%
\ y\isanewline
\ \ \ \ \isacommand{assume}\isamarkupfalse%
\ {\isachardoublequoteopen}y\ {\isasymin}\isactrlsub c\ codomain\ m{\isachardoublequoteclose}\ \isanewline
\ \ \ \ \isacommand{then}\isamarkupfalse%
\ \isacommand{have}\isamarkupfalse%
\ {\isachardoublequoteopen}y\ {\isasymin}\isactrlsub c\ {\isasymOmega}{\isachardoublequoteclose}\isanewline
\ \ \ \ \ \ \isacommand{using}\isamarkupfalse%
\ cfunc{\isacharunderscore}{\kern0pt}type{\isacharunderscore}{\kern0pt}def\ m{\isacharunderscore}{\kern0pt}type\ \isacommand{by}\isamarkupfalse%
\ force\isanewline
\ \ \ \ \isacommand{then}\isamarkupfalse%
\ \isacommand{show}\isamarkupfalse%
\ {\isachardoublequoteopen}{\isasymexists}x{\isachardot}{\kern0pt}\ x\ {\isasymin}\isactrlsub c\ domain\ m\ {\isasymand}\ m\ {\isasymcirc}\isactrlsub c\ x\ {\isacharequal}{\kern0pt}\ y{\isachardoublequoteclose}\isanewline
\ \ \ \ \ \ \isacommand{by}\isamarkupfalse%
\ {\isacharparenleft}{\kern0pt}smt\ {\isacharparenleft}{\kern0pt}verit{\isacharcomma}{\kern0pt}\ del{\isacharunderscore}{\kern0pt}insts{\isacharparenright}{\kern0pt}\ cfunc{\isacharunderscore}{\kern0pt}type{\isacharunderscore}{\kern0pt}def\ codomain{\isacharunderscore}{\kern0pt}comp\ domain{\isacharunderscore}{\kern0pt}comp\ injective{\isacharunderscore}{\kern0pt}def\ m{\isacharunderscore}{\kern0pt}mono\ m{\isacharunderscore}{\kern0pt}type\ monomorphism{\isacharunderscore}{\kern0pt}imp{\isacharunderscore}{\kern0pt}injective\ true{\isacharunderscore}{\kern0pt}false{\isacharunderscore}{\kern0pt}only{\isacharunderscore}{\kern0pt}truth{\isacharunderscore}{\kern0pt}values{\isacharparenright}{\kern0pt}\isanewline
\ \ \isacommand{qed}\isamarkupfalse%
\isanewline
\ \ \isacommand{then}\isamarkupfalse%
\ \isacommand{show}\isamarkupfalse%
\ {\isachardoublequoteopen}isomorphism\ m{\isachardoublequoteclose}\isanewline
\ \ \ \ \isacommand{by}\isamarkupfalse%
\ {\isacharparenleft}{\kern0pt}simp\ add{\isacharcolon}{\kern0pt}\ epi{\isacharunderscore}{\kern0pt}mon{\isacharunderscore}{\kern0pt}is{\isacharunderscore}{\kern0pt}iso\ m{\isacharunderscore}{\kern0pt}mono\ surjective{\isacharunderscore}{\kern0pt}is{\isacharunderscore}{\kern0pt}epimorphism{\isacharparenright}{\kern0pt}\isanewline
\isacommand{qed}\isamarkupfalse%
%
\endisatagproof
{\isafoldproof}%
%
\isadelimproof
\isanewline
%
\endisadelimproof
\isanewline
\isacommand{lemma}\isamarkupfalse%
\ smaller{\isacharunderscore}{\kern0pt}than{\isacharunderscore}{\kern0pt}finite{\isacharunderscore}{\kern0pt}is{\isacharunderscore}{\kern0pt}finite{\isacharcolon}{\kern0pt}\isanewline
\ \ \isakeyword{assumes}\ {\isachardoublequoteopen}X\ {\isasymle}\isactrlsub c\ Y{\isachardoublequoteclose}\ {\isachardoublequoteopen}is{\isacharunderscore}{\kern0pt}finite\ Y{\isachardoublequoteclose}\ \isanewline
\ \ \isakeyword{shows}\ {\isachardoublequoteopen}is{\isacharunderscore}{\kern0pt}finite\ X{\isachardoublequoteclose}\isanewline
%
\isadelimproof
\ \ %
\endisadelimproof
%
\isatagproof
\isacommand{unfolding}\isamarkupfalse%
\ is{\isacharunderscore}{\kern0pt}finite{\isacharunderscore}{\kern0pt}def\isanewline
\isacommand{proof}\isamarkupfalse%
{\isacharparenleft}{\kern0pt}clarify{\isacharparenright}{\kern0pt}\isanewline
\ \ \isacommand{fix}\isamarkupfalse%
\ x\isanewline
\ \ \isacommand{assume}\isamarkupfalse%
\ x{\isacharunderscore}{\kern0pt}type{\isacharcolon}{\kern0pt}\ {\isachardoublequoteopen}x\ {\isacharcolon}{\kern0pt}\ X\ {\isasymrightarrow}\ X{\isachardoublequoteclose}\isanewline
\ \ \isacommand{assume}\isamarkupfalse%
\ x{\isacharunderscore}{\kern0pt}mono{\isacharcolon}{\kern0pt}\ {\isachardoublequoteopen}monomorphism\ x{\isachardoublequoteclose}\isanewline
\isanewline
\ \ \isacommand{obtain}\isamarkupfalse%
\ m\ \isakeyword{where}\ m{\isacharunderscore}{\kern0pt}def{\isacharcolon}{\kern0pt}\ {\isachardoublequoteopen}m{\isacharcolon}{\kern0pt}\ X\ {\isasymrightarrow}\ Y\ {\isasymand}\ monomorphism\ m{\isachardoublequoteclose}\isanewline
\ \ \ \ \isacommand{using}\isamarkupfalse%
\ assms{\isacharparenleft}{\kern0pt}{\isadigit{1}}{\isacharparenright}{\kern0pt}\ is{\isacharunderscore}{\kern0pt}smaller{\isacharunderscore}{\kern0pt}than{\isacharunderscore}{\kern0pt}def\ \isacommand{by}\isamarkupfalse%
\ blast\isanewline
\ \ \isacommand{obtain}\isamarkupfalse%
\ {\isasymphi}\ \isakeyword{where}\ {\isasymphi}{\isacharunderscore}{\kern0pt}def{\isacharcolon}{\kern0pt}\ {\isachardoublequoteopen}{\isasymphi}\ {\isacharequal}{\kern0pt}\ into{\isacharunderscore}{\kern0pt}super\ m\ {\isasymcirc}\isactrlsub c\ {\isacharparenleft}{\kern0pt}x\ {\isasymbowtie}\isactrlsub f\ id{\isacharparenleft}{\kern0pt}Y\ {\isasymsetminus}\ {\isacharparenleft}{\kern0pt}X{\isacharcomma}{\kern0pt}m{\isacharparenright}{\kern0pt}{\isacharparenright}{\kern0pt}{\isacharparenright}{\kern0pt}\ {\isasymcirc}\isactrlsub c\ try{\isacharunderscore}{\kern0pt}cast\ m{\isachardoublequoteclose}\ \isanewline
\ \ \ \ \isacommand{by}\isamarkupfalse%
\ auto\isanewline
\isanewline
\ \ \isacommand{have}\isamarkupfalse%
\ {\isasymphi}{\isacharunderscore}{\kern0pt}type{\isacharcolon}{\kern0pt}\ {\isachardoublequoteopen}{\isasymphi}\ {\isacharcolon}{\kern0pt}\ Y\ {\isasymrightarrow}\ Y{\isachardoublequoteclose}\isanewline
\ \ \ \ \isacommand{unfolding}\isamarkupfalse%
\ {\isasymphi}{\isacharunderscore}{\kern0pt}def\isanewline
\ \ \ \ \isacommand{using}\isamarkupfalse%
\ x{\isacharunderscore}{\kern0pt}type\ m{\isacharunderscore}{\kern0pt}def\ \isacommand{by}\isamarkupfalse%
\ {\isacharparenleft}{\kern0pt}typecheck{\isacharunderscore}{\kern0pt}cfuncs{\isacharcomma}{\kern0pt}\ blast{\isacharparenright}{\kern0pt}\isanewline
\isanewline
\ \ \isacommand{have}\isamarkupfalse%
\ {\isachardoublequoteopen}injective{\isacharparenleft}{\kern0pt}x\ {\isasymbowtie}\isactrlsub f\ id{\isacharparenleft}{\kern0pt}Y\ {\isasymsetminus}\ {\isacharparenleft}{\kern0pt}X{\isacharcomma}{\kern0pt}m{\isacharparenright}{\kern0pt}{\isacharparenright}{\kern0pt}{\isacharparenright}{\kern0pt}{\isachardoublequoteclose}\isanewline
\ \ \ \ \isacommand{using}\isamarkupfalse%
\ cfunc{\isacharunderscore}{\kern0pt}bowtieprod{\isacharunderscore}{\kern0pt}inj\ id{\isacharunderscore}{\kern0pt}isomorphism\ id{\isacharunderscore}{\kern0pt}type\ iso{\isacharunderscore}{\kern0pt}imp{\isacharunderscore}{\kern0pt}epi{\isacharunderscore}{\kern0pt}and{\isacharunderscore}{\kern0pt}monic\ monomorphism{\isacharunderscore}{\kern0pt}imp{\isacharunderscore}{\kern0pt}injective\ x{\isacharunderscore}{\kern0pt}mono\ x{\isacharunderscore}{\kern0pt}type\ \isacommand{by}\isamarkupfalse%
\ blast\isanewline
\ \ \isacommand{then}\isamarkupfalse%
\ \isacommand{have}\isamarkupfalse%
\ mono{\isadigit{1}}{\isacharcolon}{\kern0pt}\ {\isachardoublequoteopen}monomorphism{\isacharparenleft}{\kern0pt}x\ {\isasymbowtie}\isactrlsub f\ id{\isacharparenleft}{\kern0pt}Y\ {\isasymsetminus}\ {\isacharparenleft}{\kern0pt}X{\isacharcomma}{\kern0pt}m{\isacharparenright}{\kern0pt}{\isacharparenright}{\kern0pt}{\isacharparenright}{\kern0pt}{\isachardoublequoteclose}\isanewline
\ \ \ \ \isacommand{using}\isamarkupfalse%
\ injective{\isacharunderscore}{\kern0pt}imp{\isacharunderscore}{\kern0pt}monomorphism\ \isacommand{by}\isamarkupfalse%
\ auto\isanewline
\ \ \isacommand{have}\isamarkupfalse%
\ mono{\isadigit{2}}{\isacharcolon}{\kern0pt}\ {\isachardoublequoteopen}monomorphism{\isacharparenleft}{\kern0pt}try{\isacharunderscore}{\kern0pt}cast\ m{\isacharparenright}{\kern0pt}{\isachardoublequoteclose}\isanewline
\ \ \ \ \isacommand{using}\isamarkupfalse%
\ m{\isacharunderscore}{\kern0pt}def\ try{\isacharunderscore}{\kern0pt}cast{\isacharunderscore}{\kern0pt}mono\ \isacommand{by}\isamarkupfalse%
\ blast\isanewline
\ \ \isacommand{have}\isamarkupfalse%
\ mono{\isadigit{3}}{\isacharcolon}{\kern0pt}\ {\isachardoublequoteopen}monomorphism{\isacharparenleft}{\kern0pt}{\isacharparenleft}{\kern0pt}x\ {\isasymbowtie}\isactrlsub f\ id{\isacharparenleft}{\kern0pt}Y\ {\isasymsetminus}\ {\isacharparenleft}{\kern0pt}X{\isacharcomma}{\kern0pt}m{\isacharparenright}{\kern0pt}{\isacharparenright}{\kern0pt}{\isacharparenright}{\kern0pt}\ {\isasymcirc}\isactrlsub c\ try{\isacharunderscore}{\kern0pt}cast\ m{\isacharparenright}{\kern0pt}{\isachardoublequoteclose}\isanewline
\ \ \ \ \isacommand{using}\isamarkupfalse%
\ cfunc{\isacharunderscore}{\kern0pt}type{\isacharunderscore}{\kern0pt}def\ composition{\isacharunderscore}{\kern0pt}of{\isacharunderscore}{\kern0pt}monic{\isacharunderscore}{\kern0pt}pair{\isacharunderscore}{\kern0pt}is{\isacharunderscore}{\kern0pt}monic\ m{\isacharunderscore}{\kern0pt}def\ mono{\isadigit{1}}\ mono{\isadigit{2}}\ x{\isacharunderscore}{\kern0pt}type\ \isacommand{by}\isamarkupfalse%
\ {\isacharparenleft}{\kern0pt}typecheck{\isacharunderscore}{\kern0pt}cfuncs{\isacharcomma}{\kern0pt}\ auto{\isacharparenright}{\kern0pt}\isanewline
\ \ \isacommand{then}\isamarkupfalse%
\ \isacommand{have}\isamarkupfalse%
\ {\isasymphi}{\isacharunderscore}{\kern0pt}mono{\isacharcolon}{\kern0pt}\ {\isachardoublequoteopen}monomorphism\ {\isasymphi}{\isachardoublequoteclose}\ \isanewline
\ \ \ \ \isacommand{unfolding}\isamarkupfalse%
\ {\isasymphi}{\isacharunderscore}{\kern0pt}def\isanewline
\ \ \ \ \isacommand{using}\isamarkupfalse%
\ cfunc{\isacharunderscore}{\kern0pt}type{\isacharunderscore}{\kern0pt}def\ composition{\isacharunderscore}{\kern0pt}of{\isacharunderscore}{\kern0pt}monic{\isacharunderscore}{\kern0pt}pair{\isacharunderscore}{\kern0pt}is{\isacharunderscore}{\kern0pt}monic\ \isanewline
\ \ \ \ \ \ \ \ \ \ into{\isacharunderscore}{\kern0pt}super{\isacharunderscore}{\kern0pt}mono\ m{\isacharunderscore}{\kern0pt}def\ mono{\isadigit{3}}\ x{\isacharunderscore}{\kern0pt}type\ \isacommand{by}\isamarkupfalse%
\ {\isacharparenleft}{\kern0pt}typecheck{\isacharunderscore}{\kern0pt}cfuncs{\isacharcomma}{\kern0pt}auto{\isacharparenright}{\kern0pt}\isanewline
\ \ \isacommand{then}\isamarkupfalse%
\ \isacommand{have}\isamarkupfalse%
\ {\isachardoublequoteopen}isomorphism\ {\isasymphi}{\isachardoublequoteclose}\ \isanewline
\ \ \ \ \isacommand{using}\isamarkupfalse%
\ {\isasymphi}{\isacharunderscore}{\kern0pt}def\ {\isasymphi}{\isacharunderscore}{\kern0pt}type\ assms{\isacharparenleft}{\kern0pt}{\isadigit{2}}{\isacharparenright}{\kern0pt}\ is{\isacharunderscore}{\kern0pt}finite{\isacharunderscore}{\kern0pt}def\ \isacommand{by}\isamarkupfalse%
\ blast\isanewline
\ \ \isacommand{have}\isamarkupfalse%
\ iso{\isacharunderscore}{\kern0pt}x{\isacharunderscore}{\kern0pt}bowtie{\isacharunderscore}{\kern0pt}id{\isacharcolon}{\kern0pt}\ {\isachardoublequoteopen}isomorphism{\isacharparenleft}{\kern0pt}x\ {\isasymbowtie}\isactrlsub f\ id{\isacharparenleft}{\kern0pt}Y\ {\isasymsetminus}\ {\isacharparenleft}{\kern0pt}X{\isacharcomma}{\kern0pt}m{\isacharparenright}{\kern0pt}{\isacharparenright}{\kern0pt}{\isacharparenright}{\kern0pt}{\isachardoublequoteclose}\isanewline
\ \ \ \ \isacommand{by}\isamarkupfalse%
\ {\isacharparenleft}{\kern0pt}typecheck{\isacharunderscore}{\kern0pt}cfuncs{\isacharcomma}{\kern0pt}\ smt\ {\isacartoucheopen}isomorphism\ {\isasymphi}{\isacartoucheclose}\ {\isasymphi}{\isacharunderscore}{\kern0pt}def\ comp{\isacharunderscore}{\kern0pt}associative{\isadigit{2}}\ id{\isacharunderscore}{\kern0pt}left{\isacharunderscore}{\kern0pt}unit{\isadigit{2}}\ into{\isacharunderscore}{\kern0pt}super{\isacharunderscore}{\kern0pt}iso\ into{\isacharunderscore}{\kern0pt}super{\isacharunderscore}{\kern0pt}try{\isacharunderscore}{\kern0pt}cast\ into{\isacharunderscore}{\kern0pt}super{\isacharunderscore}{\kern0pt}type\ isomorphism{\isacharunderscore}{\kern0pt}sandwich\ m{\isacharunderscore}{\kern0pt}def\ try{\isacharunderscore}{\kern0pt}cast{\isacharunderscore}{\kern0pt}type\ x{\isacharunderscore}{\kern0pt}type{\isacharparenright}{\kern0pt}\isanewline
\ \ \isacommand{have}\isamarkupfalse%
\ {\isachardoublequoteopen}left{\isacharunderscore}{\kern0pt}coproj\ X\ {\isacharparenleft}{\kern0pt}Y\ {\isasymsetminus}\ {\isacharparenleft}{\kern0pt}X{\isacharcomma}{\kern0pt}m{\isacharparenright}{\kern0pt}{\isacharparenright}{\kern0pt}\ {\isasymcirc}\isactrlsub c\ x\ {\isacharequal}{\kern0pt}\ {\isacharparenleft}{\kern0pt}x\ {\isasymbowtie}\isactrlsub f\ id{\isacharparenleft}{\kern0pt}Y\ {\isasymsetminus}\ {\isacharparenleft}{\kern0pt}X{\isacharcomma}{\kern0pt}m{\isacharparenright}{\kern0pt}{\isacharparenright}{\kern0pt}{\isacharparenright}{\kern0pt}\ {\isasymcirc}\isactrlsub c\ left{\isacharunderscore}{\kern0pt}coproj\ X\ {\isacharparenleft}{\kern0pt}Y\ {\isasymsetminus}\ {\isacharparenleft}{\kern0pt}X{\isacharcomma}{\kern0pt}m{\isacharparenright}{\kern0pt}{\isacharparenright}{\kern0pt}{\isachardoublequoteclose}\isanewline
\ \ \ \ \isacommand{using}\isamarkupfalse%
\ x{\isacharunderscore}{\kern0pt}type\ \ \isanewline
\ \ \ \ \isacommand{by}\isamarkupfalse%
\ {\isacharparenleft}{\kern0pt}typecheck{\isacharunderscore}{\kern0pt}cfuncs{\isacharcomma}{\kern0pt}\ simp\ add{\isacharcolon}{\kern0pt}\ left{\isacharunderscore}{\kern0pt}coproj{\isacharunderscore}{\kern0pt}cfunc{\isacharunderscore}{\kern0pt}bowtie{\isacharunderscore}{\kern0pt}prod{\isacharparenright}{\kern0pt}\isanewline
\ \ \isacommand{have}\isamarkupfalse%
\ {\isachardoublequoteopen}epimorphism{\isacharparenleft}{\kern0pt}x\ {\isasymbowtie}\isactrlsub f\ id{\isacharparenleft}{\kern0pt}Y\ {\isasymsetminus}\ {\isacharparenleft}{\kern0pt}X{\isacharcomma}{\kern0pt}m{\isacharparenright}{\kern0pt}{\isacharparenright}{\kern0pt}{\isacharparenright}{\kern0pt}{\isachardoublequoteclose}\isanewline
\ \ \ \ \isacommand{using}\isamarkupfalse%
\ iso{\isacharunderscore}{\kern0pt}imp{\isacharunderscore}{\kern0pt}epi{\isacharunderscore}{\kern0pt}and{\isacharunderscore}{\kern0pt}monic\ iso{\isacharunderscore}{\kern0pt}x{\isacharunderscore}{\kern0pt}bowtie{\isacharunderscore}{\kern0pt}id\ \isacommand{by}\isamarkupfalse%
\ blast\isanewline
\ \ \isacommand{then}\isamarkupfalse%
\ \isacommand{have}\isamarkupfalse%
\ {\isachardoublequoteopen}surjective{\isacharparenleft}{\kern0pt}x\ {\isasymbowtie}\isactrlsub f\ id{\isacharparenleft}{\kern0pt}Y\ {\isasymsetminus}\ {\isacharparenleft}{\kern0pt}X{\isacharcomma}{\kern0pt}m{\isacharparenright}{\kern0pt}{\isacharparenright}{\kern0pt}{\isacharparenright}{\kern0pt}{\isachardoublequoteclose}\isanewline
\ \ \ \ \isacommand{using}\isamarkupfalse%
\ \ epi{\isacharunderscore}{\kern0pt}is{\isacharunderscore}{\kern0pt}surj\ x{\isacharunderscore}{\kern0pt}type\ \isacommand{by}\isamarkupfalse%
\ {\isacharparenleft}{\kern0pt}typecheck{\isacharunderscore}{\kern0pt}cfuncs{\isacharcomma}{\kern0pt}\ blast{\isacharparenright}{\kern0pt}\isanewline
\ \ \isacommand{then}\isamarkupfalse%
\ \isacommand{have}\isamarkupfalse%
\ {\isachardoublequoteopen}epimorphism\ x{\isachardoublequoteclose}\isanewline
\ \ \ \ \isacommand{using}\isamarkupfalse%
\ x{\isacharunderscore}{\kern0pt}type\ cfunc{\isacharunderscore}{\kern0pt}bowtieprod{\isacharunderscore}{\kern0pt}surj{\isacharunderscore}{\kern0pt}converse\ id{\isacharunderscore}{\kern0pt}type\ surjective{\isacharunderscore}{\kern0pt}is{\isacharunderscore}{\kern0pt}epimorphism\ \isacommand{by}\isamarkupfalse%
\ blast\isanewline
\ \ \isacommand{then}\isamarkupfalse%
\ \isacommand{show}\isamarkupfalse%
\ {\isachardoublequoteopen}isomorphism\ x{\isachardoublequoteclose}\isanewline
\ \ \ \ \isacommand{by}\isamarkupfalse%
\ {\isacharparenleft}{\kern0pt}simp\ add{\isacharcolon}{\kern0pt}\ epi{\isacharunderscore}{\kern0pt}mon{\isacharunderscore}{\kern0pt}is{\isacharunderscore}{\kern0pt}iso\ x{\isacharunderscore}{\kern0pt}mono{\isacharparenright}{\kern0pt}\isanewline
\isacommand{qed}\isamarkupfalse%
%
\endisatagproof
{\isafoldproof}%
%
\isadelimproof
\isanewline
%
\endisadelimproof
\isanewline
\isacommand{lemma}\isamarkupfalse%
\ larger{\isacharunderscore}{\kern0pt}than{\isacharunderscore}{\kern0pt}infinite{\isacharunderscore}{\kern0pt}is{\isacharunderscore}{\kern0pt}infinite{\isacharcolon}{\kern0pt}\isanewline
\ \ \isakeyword{assumes}\ {\isachardoublequoteopen}X\ {\isasymle}\isactrlsub c\ Y{\isachardoublequoteclose}\ {\isachardoublequoteopen}is{\isacharunderscore}{\kern0pt}infinite\ X{\isachardoublequoteclose}\ \isanewline
\ \ \isakeyword{shows}\ {\isachardoublequoteopen}is{\isacharunderscore}{\kern0pt}infinite\ Y{\isachardoublequoteclose}\isanewline
%
\isadelimproof
\ \ %
\endisadelimproof
%
\isatagproof
\isacommand{using}\isamarkupfalse%
\ assms\ either{\isacharunderscore}{\kern0pt}finite{\isacharunderscore}{\kern0pt}or{\isacharunderscore}{\kern0pt}infinite\ epi{\isacharunderscore}{\kern0pt}is{\isacharunderscore}{\kern0pt}surj\ is{\isacharunderscore}{\kern0pt}finite{\isacharunderscore}{\kern0pt}def\ is{\isacharunderscore}{\kern0pt}infinite{\isacharunderscore}{\kern0pt}def\isanewline
\ \ \ \ iso{\isacharunderscore}{\kern0pt}imp{\isacharunderscore}{\kern0pt}epi{\isacharunderscore}{\kern0pt}and{\isacharunderscore}{\kern0pt}monic\ smaller{\isacharunderscore}{\kern0pt}than{\isacharunderscore}{\kern0pt}finite{\isacharunderscore}{\kern0pt}is{\isacharunderscore}{\kern0pt}finite\ \isacommand{by}\isamarkupfalse%
\ blast%
\endisatagproof
{\isafoldproof}%
%
\isadelimproof
\isanewline
%
\endisadelimproof
\isanewline
\isacommand{lemma}\isamarkupfalse%
\ iso{\isacharunderscore}{\kern0pt}pres{\isacharunderscore}{\kern0pt}finite{\isacharcolon}{\kern0pt}\isanewline
\ \ \isakeyword{assumes}\ {\isachardoublequoteopen}X\ {\isasymcong}\ Y{\isachardoublequoteclose}\isanewline
\ \ \isakeyword{assumes}\ {\isachardoublequoteopen}is{\isacharunderscore}{\kern0pt}finite\ X{\isachardoublequoteclose}\isanewline
\ \ \isakeyword{shows}\ {\isachardoublequoteopen}is{\isacharunderscore}{\kern0pt}finite\ Y{\isachardoublequoteclose}\isanewline
%
\isadelimproof
\ \ %
\endisadelimproof
%
\isatagproof
\isacommand{using}\isamarkupfalse%
\ assms\ is{\isacharunderscore}{\kern0pt}isomorphic{\isacharunderscore}{\kern0pt}def\ is{\isacharunderscore}{\kern0pt}smaller{\isacharunderscore}{\kern0pt}than{\isacharunderscore}{\kern0pt}def\ iso{\isacharunderscore}{\kern0pt}imp{\isacharunderscore}{\kern0pt}epi{\isacharunderscore}{\kern0pt}and{\isacharunderscore}{\kern0pt}monic\ isomorphic{\isacharunderscore}{\kern0pt}is{\isacharunderscore}{\kern0pt}symmetric\ smaller{\isacharunderscore}{\kern0pt}than{\isacharunderscore}{\kern0pt}finite{\isacharunderscore}{\kern0pt}is{\isacharunderscore}{\kern0pt}finite\ \isacommand{by}\isamarkupfalse%
\ blast%
\endisatagproof
{\isafoldproof}%
%
\isadelimproof
\isanewline
%
\endisadelimproof
\isanewline
\isacommand{lemma}\isamarkupfalse%
\ not{\isacharunderscore}{\kern0pt}finite{\isacharunderscore}{\kern0pt}and{\isacharunderscore}{\kern0pt}infinite{\isacharcolon}{\kern0pt}\isanewline
\ \ {\isachardoublequoteopen}{\isasymnot}{\isacharparenleft}{\kern0pt}is{\isacharunderscore}{\kern0pt}finite\ X\ {\isasymand}\ is{\isacharunderscore}{\kern0pt}infinite\ X{\isacharparenright}{\kern0pt}{\isachardoublequoteclose}\isanewline
%
\isadelimproof
\ \ %
\endisadelimproof
%
\isatagproof
\isacommand{using}\isamarkupfalse%
\ epi{\isacharunderscore}{\kern0pt}is{\isacharunderscore}{\kern0pt}surj\ is{\isacharunderscore}{\kern0pt}finite{\isacharunderscore}{\kern0pt}def\ is{\isacharunderscore}{\kern0pt}infinite{\isacharunderscore}{\kern0pt}def\ iso{\isacharunderscore}{\kern0pt}imp{\isacharunderscore}{\kern0pt}epi{\isacharunderscore}{\kern0pt}and{\isacharunderscore}{\kern0pt}monic\ \isacommand{by}\isamarkupfalse%
\ blast%
\endisatagproof
{\isafoldproof}%
%
\isadelimproof
\isanewline
%
\endisadelimproof
\isanewline
\isacommand{lemma}\isamarkupfalse%
\ iso{\isacharunderscore}{\kern0pt}pres{\isacharunderscore}{\kern0pt}infinite{\isacharcolon}{\kern0pt}\isanewline
\ \ \isakeyword{assumes}\ {\isachardoublequoteopen}X\ {\isasymcong}\ Y{\isachardoublequoteclose}\isanewline
\ \ \isakeyword{assumes}\ {\isachardoublequoteopen}is{\isacharunderscore}{\kern0pt}infinite\ X{\isachardoublequoteclose}\isanewline
\ \ \isakeyword{shows}\ {\isachardoublequoteopen}is{\isacharunderscore}{\kern0pt}infinite\ Y{\isachardoublequoteclose}\isanewline
%
\isadelimproof
\ \ %
\endisadelimproof
%
\isatagproof
\isacommand{using}\isamarkupfalse%
\ assms\ either{\isacharunderscore}{\kern0pt}finite{\isacharunderscore}{\kern0pt}or{\isacharunderscore}{\kern0pt}infinite\ not{\isacharunderscore}{\kern0pt}finite{\isacharunderscore}{\kern0pt}and{\isacharunderscore}{\kern0pt}infinite\ iso{\isacharunderscore}{\kern0pt}pres{\isacharunderscore}{\kern0pt}finite\ isomorphic{\isacharunderscore}{\kern0pt}is{\isacharunderscore}{\kern0pt}symmetric\ \isacommand{by}\isamarkupfalse%
\ blast%
\endisatagproof
{\isafoldproof}%
%
\isadelimproof
\isanewline
%
\endisadelimproof
\isanewline
\isacommand{lemma}\isamarkupfalse%
\ size{\isacharunderscore}{\kern0pt}{\isadigit{2}}{\isacharunderscore}{\kern0pt}sets{\isacharcolon}{\kern0pt}\isanewline
{\isachardoublequoteopen}{\isacharparenleft}{\kern0pt}X\ {\isasymcong}\ {\isasymOmega}{\isacharparenright}{\kern0pt}\ {\isacharequal}{\kern0pt}\ {\isacharparenleft}{\kern0pt}{\isasymexists}\ x{\isadigit{1}}{\isachardot}{\kern0pt}\ {\isasymexists}\ x{\isadigit{2}}{\isachardot}{\kern0pt}\ x{\isadigit{1}}\ {\isasymin}\isactrlsub c\ X\ {\isasymand}\ x{\isadigit{2}}\ {\isasymin}\isactrlsub c\ X\ {\isasymand}\ x{\isadigit{1}}\ {\isasymnoteq}\ x{\isadigit{2}}\ {\isasymand}\ {\isacharparenleft}{\kern0pt}{\isasymforall}x{\isachardot}{\kern0pt}\ x\ {\isasymin}\isactrlsub c\ X\ {\isasymlongrightarrow}\ x\ {\isacharequal}{\kern0pt}\ x{\isadigit{1}}\ {\isasymor}\ x\ {\isacharequal}{\kern0pt}\ x{\isadigit{2}}{\isacharparenright}{\kern0pt}{\isacharparenright}{\kern0pt}{\isachardoublequoteclose}\isanewline
%
\isadelimproof
%
\endisadelimproof
%
\isatagproof
\isacommand{proof}\isamarkupfalse%
\ \isanewline
\ \ \isacommand{assume}\isamarkupfalse%
\ {\isachardoublequoteopen}X\ {\isasymcong}\ {\isasymOmega}{\isachardoublequoteclose}\isanewline
\ \ \isacommand{then}\isamarkupfalse%
\ \isacommand{obtain}\isamarkupfalse%
\ {\isasymphi}\ \isakeyword{where}\ {\isasymphi}{\isacharunderscore}{\kern0pt}type{\isacharbrackleft}{\kern0pt}type{\isacharunderscore}{\kern0pt}rule{\isacharbrackright}{\kern0pt}{\isacharcolon}{\kern0pt}\ {\isachardoublequoteopen}{\isasymphi}\ {\isacharcolon}{\kern0pt}\ X\ {\isasymrightarrow}\ {\isasymOmega}{\isachardoublequoteclose}\ \isakeyword{and}\ {\isasymphi}{\isacharunderscore}{\kern0pt}iso{\isacharcolon}{\kern0pt}\ {\isachardoublequoteopen}isomorphism\ {\isasymphi}{\isachardoublequoteclose}\isanewline
\ \ \ \ \isacommand{using}\isamarkupfalse%
\ is{\isacharunderscore}{\kern0pt}isomorphic{\isacharunderscore}{\kern0pt}def\ \isacommand{by}\isamarkupfalse%
\ blast\isanewline
\ \ \isacommand{obtain}\isamarkupfalse%
\ x{\isadigit{1}}\ x{\isadigit{2}}\ \isakeyword{where}\ x{\isadigit{1}}{\isacharunderscore}{\kern0pt}type{\isacharbrackleft}{\kern0pt}type{\isacharunderscore}{\kern0pt}rule{\isacharbrackright}{\kern0pt}{\isacharcolon}{\kern0pt}\ {\isachardoublequoteopen}x{\isadigit{1}}\ {\isasymin}\isactrlsub c\ X{\isachardoublequoteclose}\ \isakeyword{and}\ x{\isadigit{1}}{\isacharunderscore}{\kern0pt}def{\isacharcolon}{\kern0pt}\ {\isachardoublequoteopen}{\isasymphi}\ {\isasymcirc}\isactrlsub c\ x{\isadigit{1}}\ {\isacharequal}{\kern0pt}\ {\isasymt}{\isachardoublequoteclose}\ \isakeyword{and}\isanewline
\ \ \ \ \ \ \ \ \ \ \ \ \ \ \ \ \ \ \ \ \ x{\isadigit{2}}{\isacharunderscore}{\kern0pt}type{\isacharbrackleft}{\kern0pt}type{\isacharunderscore}{\kern0pt}rule{\isacharbrackright}{\kern0pt}{\isacharcolon}{\kern0pt}\ {\isachardoublequoteopen}x{\isadigit{2}}\ {\isasymin}\isactrlsub c\ X{\isachardoublequoteclose}\ \isakeyword{and}\ x{\isadigit{2}}{\isacharunderscore}{\kern0pt}def{\isacharcolon}{\kern0pt}\ {\isachardoublequoteopen}{\isasymphi}\ {\isasymcirc}\isactrlsub c\ x{\isadigit{2}}\ {\isacharequal}{\kern0pt}\ {\isasymf}{\isachardoublequoteclose}\ \isakeyword{and}\isanewline
\ \ \ \ \ \ \ \ \ \ \ \ \ \ \ \ \ \ \ \ \ distinct{\isacharcolon}{\kern0pt}\ {\isachardoublequoteopen}x{\isadigit{1}}\ {\isasymnoteq}\ x{\isadigit{2}}{\isachardoublequoteclose}\isanewline
\ \ \ \ \isacommand{by}\isamarkupfalse%
\ {\isacharparenleft}{\kern0pt}typecheck{\isacharunderscore}{\kern0pt}cfuncs{\isacharcomma}{\kern0pt}\ smt\ {\isacharparenleft}{\kern0pt}z{\isadigit{3}}{\isacharparenright}{\kern0pt}\ {\isasymphi}{\isacharunderscore}{\kern0pt}iso\ cfunc{\isacharunderscore}{\kern0pt}type{\isacharunderscore}{\kern0pt}def\ comp{\isacharunderscore}{\kern0pt}associative\ comp{\isacharunderscore}{\kern0pt}type\ id{\isacharunderscore}{\kern0pt}left{\isacharunderscore}{\kern0pt}unit{\isadigit{2}}\ isomorphism{\isacharunderscore}{\kern0pt}def\ true{\isacharunderscore}{\kern0pt}false{\isacharunderscore}{\kern0pt}distinct{\isacharparenright}{\kern0pt}\isanewline
\ \ \isacommand{then}\isamarkupfalse%
\ \isacommand{show}\isamarkupfalse%
\ \ {\isachardoublequoteopen}{\isasymexists}x{\isadigit{1}}\ x{\isadigit{2}}{\isachardot}{\kern0pt}\ x{\isadigit{1}}\ {\isasymin}\isactrlsub c\ X\ {\isasymand}\ x{\isadigit{2}}\ {\isasymin}\isactrlsub c\ X\ {\isasymand}\ x{\isadigit{1}}\ {\isasymnoteq}\ x{\isadigit{2}}\ {\isasymand}\ {\isacharparenleft}{\kern0pt}{\isasymforall}x{\isachardot}{\kern0pt}\ x\ {\isasymin}\isactrlsub c\ X\ {\isasymlongrightarrow}\ x\ {\isacharequal}{\kern0pt}\ x{\isadigit{1}}\ {\isasymor}\ x\ {\isacharequal}{\kern0pt}\ x{\isadigit{2}}{\isacharparenright}{\kern0pt}{\isachardoublequoteclose}\isanewline
\ \ \ \ \isacommand{by}\isamarkupfalse%
\ {\isacharparenleft}{\kern0pt}smt\ {\isacharparenleft}{\kern0pt}verit{\isacharcomma}{\kern0pt}\ best{\isacharparenright}{\kern0pt}\ \ {\isasymphi}{\isacharunderscore}{\kern0pt}iso\ {\isasymphi}{\isacharunderscore}{\kern0pt}type\ cfunc{\isacharunderscore}{\kern0pt}type{\isacharunderscore}{\kern0pt}def\ comp{\isacharunderscore}{\kern0pt}associative{\isadigit{2}}\ comp{\isacharunderscore}{\kern0pt}type\ id{\isacharunderscore}{\kern0pt}left{\isacharunderscore}{\kern0pt}unit{\isadigit{2}}\ isomorphism{\isacharunderscore}{\kern0pt}def\ true{\isacharunderscore}{\kern0pt}false{\isacharunderscore}{\kern0pt}only{\isacharunderscore}{\kern0pt}truth{\isacharunderscore}{\kern0pt}values{\isacharparenright}{\kern0pt}\isanewline
\isacommand{next}\isamarkupfalse%
\isanewline
\ \ \isacommand{assume}\isamarkupfalse%
\ exactly{\isacharunderscore}{\kern0pt}two{\isacharcolon}{\kern0pt}\ {\isachardoublequoteopen}{\isasymexists}x{\isadigit{1}}\ x{\isadigit{2}}{\isachardot}{\kern0pt}\ x{\isadigit{1}}\ {\isasymin}\isactrlsub c\ X\ {\isasymand}\ x{\isadigit{2}}\ {\isasymin}\isactrlsub c\ X\ {\isasymand}\ x{\isadigit{1}}\ {\isasymnoteq}\ x{\isadigit{2}}\ {\isasymand}\ {\isacharparenleft}{\kern0pt}{\isasymforall}x{\isachardot}{\kern0pt}\ x\ {\isasymin}\isactrlsub c\ X\ {\isasymlongrightarrow}\ x\ {\isacharequal}{\kern0pt}\ x{\isadigit{1}}\ {\isasymor}\ x\ {\isacharequal}{\kern0pt}\ x{\isadigit{2}}{\isacharparenright}{\kern0pt}{\isachardoublequoteclose}\isanewline
\ \ \isacommand{then}\isamarkupfalse%
\ \isacommand{obtain}\isamarkupfalse%
\ x{\isadigit{1}}\ x{\isadigit{2}}\ \ \isakeyword{where}\ x{\isadigit{1}}{\isacharunderscore}{\kern0pt}type{\isacharbrackleft}{\kern0pt}type{\isacharunderscore}{\kern0pt}rule{\isacharbrackright}{\kern0pt}{\isacharcolon}{\kern0pt}\ {\isachardoublequoteopen}x{\isadigit{1}}\ {\isasymin}\isactrlsub c\ X{\isachardoublequoteclose}\ \isakeyword{and}\ x{\isadigit{2}}{\isacharunderscore}{\kern0pt}type{\isacharbrackleft}{\kern0pt}type{\isacharunderscore}{\kern0pt}rule{\isacharbrackright}{\kern0pt}{\isacharcolon}{\kern0pt}\ {\isachardoublequoteopen}x{\isadigit{2}}\ {\isasymin}\isactrlsub c\ X{\isachardoublequoteclose}\ \isakeyword{and}\ distinct{\isacharcolon}{\kern0pt}\ {\isachardoublequoteopen}x{\isadigit{1}}\ {\isasymnoteq}\ x{\isadigit{2}}{\isachardoublequoteclose}\isanewline
\ \ \ \ \isacommand{by}\isamarkupfalse%
\ force\isanewline
\ \ \isacommand{have}\isamarkupfalse%
\ iso{\isacharunderscore}{\kern0pt}type{\isacharcolon}{\kern0pt}\ {\isachardoublequoteopen}{\isacharparenleft}{\kern0pt}{\isacharparenleft}{\kern0pt}x{\isadigit{1}}\ {\isasymamalg}\ x{\isadigit{2}}{\isacharparenright}{\kern0pt}\ {\isasymcirc}\isactrlsub c\ case{\isacharunderscore}{\kern0pt}bool{\isacharparenright}{\kern0pt}\ {\isacharcolon}{\kern0pt}\ {\isasymOmega}\ {\isasymrightarrow}\ X{\isachardoublequoteclose}\isanewline
\ \ \ \ \isacommand{by}\isamarkupfalse%
\ typecheck{\isacharunderscore}{\kern0pt}cfuncs\isanewline
\ \ \isacommand{have}\isamarkupfalse%
\ surj{\isacharcolon}{\kern0pt}\ {\isachardoublequoteopen}surjective\ {\isacharparenleft}{\kern0pt}{\isacharparenleft}{\kern0pt}x{\isadigit{1}}\ {\isasymamalg}\ x{\isadigit{2}}{\isacharparenright}{\kern0pt}\ {\isasymcirc}\isactrlsub c\ case{\isacharunderscore}{\kern0pt}bool{\isacharparenright}{\kern0pt}{\isachardoublequoteclose}\isanewline
\ \ \ \ \isacommand{by}\isamarkupfalse%
\ {\isacharparenleft}{\kern0pt}typecheck{\isacharunderscore}{\kern0pt}cfuncs{\isacharcomma}{\kern0pt}\ smt\ {\isacharparenleft}{\kern0pt}verit{\isacharcomma}{\kern0pt}\ best{\isacharparenright}{\kern0pt}\ exactly{\isacharunderscore}{\kern0pt}two\ cfunc{\isacharunderscore}{\kern0pt}type{\isacharunderscore}{\kern0pt}def\ coprod{\isacharunderscore}{\kern0pt}case{\isacharunderscore}{\kern0pt}bool{\isacharunderscore}{\kern0pt}false\isanewline
\ \ \ \ \ \ \ \ \ \ \ \ \ \ \ \ coprod{\isacharunderscore}{\kern0pt}case{\isacharunderscore}{\kern0pt}bool{\isacharunderscore}{\kern0pt}true\ distinct\ false{\isacharunderscore}{\kern0pt}func{\isacharunderscore}{\kern0pt}type\ surjective{\isacharunderscore}{\kern0pt}def\ true{\isacharunderscore}{\kern0pt}func{\isacharunderscore}{\kern0pt}type{\isacharparenright}{\kern0pt}\isanewline
\ \ \isacommand{have}\isamarkupfalse%
\ inj{\isacharcolon}{\kern0pt}\ {\isachardoublequoteopen}injective\ {\isacharparenleft}{\kern0pt}{\isacharparenleft}{\kern0pt}x{\isadigit{1}}\ {\isasymamalg}\ x{\isadigit{2}}{\isacharparenright}{\kern0pt}\ {\isasymcirc}\isactrlsub c\ case{\isacharunderscore}{\kern0pt}bool{\isacharparenright}{\kern0pt}{\isachardoublequoteclose}\isanewline
\ \ \ \ \isacommand{by}\isamarkupfalse%
\ {\isacharparenleft}{\kern0pt}typecheck{\isacharunderscore}{\kern0pt}cfuncs{\isacharcomma}{\kern0pt}\ smt\ {\isacharparenleft}{\kern0pt}verit{\isacharcomma}{\kern0pt}\ ccfv{\isacharunderscore}{\kern0pt}SIG{\isacharparenright}{\kern0pt}\ distinct\ case{\isacharunderscore}{\kern0pt}bool{\isacharunderscore}{\kern0pt}true{\isacharunderscore}{\kern0pt}and{\isacharunderscore}{\kern0pt}false\ comp{\isacharunderscore}{\kern0pt}associative{\isadigit{2}}\ \isanewline
\ \ \ \ \ \ \ \ coprod{\isacharunderscore}{\kern0pt}case{\isacharunderscore}{\kern0pt}bool{\isacharunderscore}{\kern0pt}false\ injective{\isacharunderscore}{\kern0pt}def{\isadigit{2}}\ left{\isacharunderscore}{\kern0pt}coproj{\isacharunderscore}{\kern0pt}cfunc{\isacharunderscore}{\kern0pt}coprod\ true{\isacharunderscore}{\kern0pt}false{\isacharunderscore}{\kern0pt}only{\isacharunderscore}{\kern0pt}truth{\isacharunderscore}{\kern0pt}values{\isacharparenright}{\kern0pt}\isanewline
\ \ \isacommand{then}\isamarkupfalse%
\ \isacommand{have}\isamarkupfalse%
\ {\isachardoublequoteopen}isomorphism\ {\isacharparenleft}{\kern0pt}{\isacharparenleft}{\kern0pt}x{\isadigit{1}}\ {\isasymamalg}\ x{\isadigit{2}}{\isacharparenright}{\kern0pt}\ {\isasymcirc}\isactrlsub c\ case{\isacharunderscore}{\kern0pt}bool{\isacharparenright}{\kern0pt}{\isachardoublequoteclose}\isanewline
\ \ \ \ \isacommand{by}\isamarkupfalse%
\ {\isacharparenleft}{\kern0pt}meson\ epi{\isacharunderscore}{\kern0pt}mon{\isacharunderscore}{\kern0pt}is{\isacharunderscore}{\kern0pt}iso\ injective{\isacharunderscore}{\kern0pt}imp{\isacharunderscore}{\kern0pt}monomorphism\ singletonI\ surj\ surjective{\isacharunderscore}{\kern0pt}is{\isacharunderscore}{\kern0pt}epimorphism{\isacharparenright}{\kern0pt}\isanewline
\ \ \isacommand{then}\isamarkupfalse%
\ \isacommand{show}\isamarkupfalse%
\ {\isachardoublequoteopen}X\ {\isasymcong}\ {\isasymOmega}{\isachardoublequoteclose}\isanewline
\ \ \ \ \isacommand{using}\isamarkupfalse%
\ is{\isacharunderscore}{\kern0pt}isomorphic{\isacharunderscore}{\kern0pt}def\ iso{\isacharunderscore}{\kern0pt}type\ isomorphic{\isacharunderscore}{\kern0pt}is{\isacharunderscore}{\kern0pt}symmetric\ \isacommand{by}\isamarkupfalse%
\ blast\isanewline
\isacommand{qed}\isamarkupfalse%
%
\endisatagproof
{\isafoldproof}%
%
\isadelimproof
\isanewline
%
\endisadelimproof
\isanewline
\isacommand{lemma}\isamarkupfalse%
\ size{\isacharunderscore}{\kern0pt}{\isadigit{2}}plus{\isacharunderscore}{\kern0pt}sets{\isacharcolon}{\kern0pt}\isanewline
\ \ {\isachardoublequoteopen}{\isacharparenleft}{\kern0pt}{\isasymOmega}\ {\isasymle}\isactrlsub c\ X{\isacharparenright}{\kern0pt}\ {\isacharequal}{\kern0pt}\ {\isacharparenleft}{\kern0pt}{\isasymexists}\ x{\isadigit{1}}{\isachardot}{\kern0pt}\ {\isasymexists}\ x{\isadigit{2}}{\isachardot}{\kern0pt}\ x{\isadigit{1}}\ {\isasymin}\isactrlsub c\ X\ {\isasymand}\ x{\isadigit{2}}\ {\isasymin}\isactrlsub c\ X\ {\isasymand}\ x{\isadigit{1}}\ {\isasymnoteq}\ x{\isadigit{2}}{\isacharparenright}{\kern0pt}{\isachardoublequoteclose}\isanewline
%
\isadelimproof
%
\endisadelimproof
%
\isatagproof
\isacommand{proof}\isamarkupfalse%
\ standard\isanewline
\ \ \isacommand{show}\isamarkupfalse%
\ {\isachardoublequoteopen}{\isasymOmega}\ {\isasymle}\isactrlsub c\ X\ {\isasymLongrightarrow}\ {\isasymexists}x{\isadigit{1}}\ x{\isadigit{2}}{\isachardot}{\kern0pt}\ x{\isadigit{1}}\ {\isasymin}\isactrlsub c\ X\ {\isasymand}\ x{\isadigit{2}}\ {\isasymin}\isactrlsub c\ X\ {\isasymand}\ x{\isadigit{1}}\ {\isasymnoteq}\ x{\isadigit{2}}{\isachardoublequoteclose}\isanewline
\ \ \ \ \isacommand{by}\isamarkupfalse%
\ {\isacharparenleft}{\kern0pt}meson\ comp{\isacharunderscore}{\kern0pt}type\ false{\isacharunderscore}{\kern0pt}func{\isacharunderscore}{\kern0pt}type\ is{\isacharunderscore}{\kern0pt}smaller{\isacharunderscore}{\kern0pt}than{\isacharunderscore}{\kern0pt}def\ monomorphism{\isacharunderscore}{\kern0pt}def{\isadigit{3}}\ true{\isacharunderscore}{\kern0pt}false{\isacharunderscore}{\kern0pt}distinct\ true{\isacharunderscore}{\kern0pt}func{\isacharunderscore}{\kern0pt}type{\isacharparenright}{\kern0pt}\isanewline
\isacommand{next}\isamarkupfalse%
\isanewline
\ \ \isacommand{assume}\isamarkupfalse%
\ {\isachardoublequoteopen}{\isasymexists}x{\isadigit{1}}\ x{\isadigit{2}}{\isachardot}{\kern0pt}\ x{\isadigit{1}}\ {\isasymin}\isactrlsub c\ X\ {\isasymand}\ x{\isadigit{2}}\ {\isasymin}\isactrlsub c\ X\ {\isasymand}\ x{\isadigit{1}}\ {\isasymnoteq}\ x{\isadigit{2}}{\isachardoublequoteclose}\isanewline
\ \ \isacommand{then}\isamarkupfalse%
\ \isacommand{obtain}\isamarkupfalse%
\ x{\isadigit{1}}\ x{\isadigit{2}}\ \isakeyword{where}\ x{\isadigit{1}}{\isacharunderscore}{\kern0pt}type{\isacharbrackleft}{\kern0pt}type{\isacharunderscore}{\kern0pt}rule{\isacharbrackright}{\kern0pt}{\isacharcolon}{\kern0pt}\ {\isachardoublequoteopen}x{\isadigit{1}}\ {\isasymin}\isactrlsub c\ X{\isachardoublequoteclose}\ \isakeyword{and}\isanewline
\ \ \ \ \ \ \ \ \ \ \ \ \ \ \ \ \ \ \ \ \ x{\isadigit{2}}{\isacharunderscore}{\kern0pt}type{\isacharbrackleft}{\kern0pt}type{\isacharunderscore}{\kern0pt}rule{\isacharbrackright}{\kern0pt}{\isacharcolon}{\kern0pt}\ {\isachardoublequoteopen}x{\isadigit{2}}\ {\isasymin}\isactrlsub c\ X{\isachardoublequoteclose}\ \isakeyword{and}\isanewline
\ \ \ \ \ \ \ \ \ \ \ \ \ \ \ \ \ \ \ \ \ \ \ \ \ \ \ \ \ \ \ distinct{\isacharcolon}{\kern0pt}\ {\isachardoublequoteopen}x{\isadigit{1}}\ {\isasymnoteq}\ x{\isadigit{2}}{\isachardoublequoteclose}\isanewline
\ \ \ \ \isacommand{by}\isamarkupfalse%
\ blast\ \ \isanewline
\ \ \isacommand{have}\isamarkupfalse%
\ mono{\isacharunderscore}{\kern0pt}type{\isacharcolon}{\kern0pt}\ {\isachardoublequoteopen}{\isacharparenleft}{\kern0pt}{\isacharparenleft}{\kern0pt}x{\isadigit{1}}\ {\isasymamalg}\ x{\isadigit{2}}{\isacharparenright}{\kern0pt}\ {\isasymcirc}\isactrlsub c\ case{\isacharunderscore}{\kern0pt}bool{\isacharparenright}{\kern0pt}\ {\isacharcolon}{\kern0pt}\ {\isasymOmega}\ {\isasymrightarrow}\ X{\isachardoublequoteclose}\isanewline
\ \ \ \ \isacommand{by}\isamarkupfalse%
\ typecheck{\isacharunderscore}{\kern0pt}cfuncs\isanewline
\ \ \isacommand{have}\isamarkupfalse%
\ inj{\isacharcolon}{\kern0pt}\ {\isachardoublequoteopen}injective\ {\isacharparenleft}{\kern0pt}{\isacharparenleft}{\kern0pt}x{\isadigit{1}}\ {\isasymamalg}\ x{\isadigit{2}}{\isacharparenright}{\kern0pt}\ {\isasymcirc}\isactrlsub c\ case{\isacharunderscore}{\kern0pt}bool{\isacharparenright}{\kern0pt}{\isachardoublequoteclose}\isanewline
\ \ \ \ \isacommand{by}\isamarkupfalse%
\ {\isacharparenleft}{\kern0pt}typecheck{\isacharunderscore}{\kern0pt}cfuncs{\isacharcomma}{\kern0pt}\ smt\ {\isacharparenleft}{\kern0pt}verit{\isacharcomma}{\kern0pt}\ ccfv{\isacharunderscore}{\kern0pt}SIG{\isacharparenright}{\kern0pt}\ distinct\ case{\isacharunderscore}{\kern0pt}bool{\isacharunderscore}{\kern0pt}true{\isacharunderscore}{\kern0pt}and{\isacharunderscore}{\kern0pt}false\ comp{\isacharunderscore}{\kern0pt}associative{\isadigit{2}}\ \isanewline
\ \ \ \ \ \ \ \ coprod{\isacharunderscore}{\kern0pt}case{\isacharunderscore}{\kern0pt}bool{\isacharunderscore}{\kern0pt}false\ injective{\isacharunderscore}{\kern0pt}def{\isadigit{2}}\ left{\isacharunderscore}{\kern0pt}coproj{\isacharunderscore}{\kern0pt}cfunc{\isacharunderscore}{\kern0pt}coprod\ true{\isacharunderscore}{\kern0pt}false{\isacharunderscore}{\kern0pt}only{\isacharunderscore}{\kern0pt}truth{\isacharunderscore}{\kern0pt}values{\isacharparenright}{\kern0pt}\ \ \ \ \isanewline
\ \ \isacommand{then}\isamarkupfalse%
\ \isacommand{show}\isamarkupfalse%
\ {\isachardoublequoteopen}{\isasymOmega}\ {\isasymle}\isactrlsub c\ X{\isachardoublequoteclose}\isanewline
\ \ \ \ \isacommand{using}\isamarkupfalse%
\ injective{\isacharunderscore}{\kern0pt}imp{\isacharunderscore}{\kern0pt}monomorphism\ is{\isacharunderscore}{\kern0pt}smaller{\isacharunderscore}{\kern0pt}than{\isacharunderscore}{\kern0pt}def\ mono{\isacharunderscore}{\kern0pt}type\ \isacommand{by}\isamarkupfalse%
\ blast\isanewline
\isacommand{qed}\isamarkupfalse%
%
\endisatagproof
{\isafoldproof}%
%
\isadelimproof
\isanewline
%
\endisadelimproof
\isanewline
\isacommand{lemma}\isamarkupfalse%
\ not{\isacharunderscore}{\kern0pt}init{\isacharunderscore}{\kern0pt}not{\isacharunderscore}{\kern0pt}term{\isacharcolon}{\kern0pt}\isanewline
\ \ {\isachardoublequoteopen}{\isacharparenleft}{\kern0pt}{\isasymnot}{\isacharparenleft}{\kern0pt}initial{\isacharunderscore}{\kern0pt}object\ X{\isacharparenright}{\kern0pt}\ {\isasymand}\ {\isasymnot}{\isacharparenleft}{\kern0pt}terminal{\isacharunderscore}{\kern0pt}object\ X{\isacharparenright}{\kern0pt}{\isacharparenright}{\kern0pt}\ {\isacharequal}{\kern0pt}\ {\isacharparenleft}{\kern0pt}{\isasymexists}\ x{\isadigit{1}}{\isachardot}{\kern0pt}\ {\isasymexists}\ x{\isadigit{2}}{\isachardot}{\kern0pt}\ x{\isadigit{1}}\ {\isasymin}\isactrlsub c\ X\ {\isasymand}\ x{\isadigit{2}}\ {\isasymin}\isactrlsub c\ X\ {\isasymand}\ x{\isadigit{1}}\ {\isasymnoteq}\ x{\isadigit{2}}{\isacharparenright}{\kern0pt}{\isachardoublequoteclose}\isanewline
%
\isadelimproof
\ \ %
\endisadelimproof
%
\isatagproof
\isacommand{by}\isamarkupfalse%
\ {\isacharparenleft}{\kern0pt}metis\ is{\isacharunderscore}{\kern0pt}empty{\isacharunderscore}{\kern0pt}def\ initial{\isacharunderscore}{\kern0pt}iso{\isacharunderscore}{\kern0pt}empty\ iso{\isacharunderscore}{\kern0pt}empty{\isacharunderscore}{\kern0pt}initial\ iso{\isacharunderscore}{\kern0pt}to{\isadigit{1}}{\isacharunderscore}{\kern0pt}is{\isacharunderscore}{\kern0pt}term\ no{\isacharunderscore}{\kern0pt}el{\isacharunderscore}{\kern0pt}iff{\isacharunderscore}{\kern0pt}iso{\isacharunderscore}{\kern0pt}empty\ single{\isacharunderscore}{\kern0pt}elem{\isacharunderscore}{\kern0pt}iso{\isacharunderscore}{\kern0pt}one\ terminal{\isacharunderscore}{\kern0pt}object{\isacharunderscore}{\kern0pt}def{\isacharparenright}{\kern0pt}%
\endisatagproof
{\isafoldproof}%
%
\isadelimproof
\isanewline
%
\endisadelimproof
\isanewline
\isacommand{lemma}\isamarkupfalse%
\ sets{\isacharunderscore}{\kern0pt}size{\isacharunderscore}{\kern0pt}{\isadigit{3}}{\isacharunderscore}{\kern0pt}plus{\isacharcolon}{\kern0pt}\isanewline
\ \ {\isachardoublequoteopen}{\isacharparenleft}{\kern0pt}{\isasymnot}{\isacharparenleft}{\kern0pt}initial{\isacharunderscore}{\kern0pt}object\ X{\isacharparenright}{\kern0pt}\ {\isasymand}\ {\isasymnot}{\isacharparenleft}{\kern0pt}terminal{\isacharunderscore}{\kern0pt}object\ X{\isacharparenright}{\kern0pt}\ {\isasymand}\ {\isasymnot}{\isacharparenleft}{\kern0pt}X\ {\isasymcong}\ {\isasymOmega}{\isacharparenright}{\kern0pt}{\isacharparenright}{\kern0pt}\ {\isacharequal}{\kern0pt}\ {\isacharparenleft}{\kern0pt}{\isasymexists}\ x{\isadigit{1}}{\isachardot}{\kern0pt}\ {\isasymexists}\ x{\isadigit{2}}{\isachardot}{\kern0pt}\ \ {\isasymexists}\ x{\isadigit{3}}{\isachardot}{\kern0pt}\ x{\isadigit{1}}\ {\isasymin}\isactrlsub c\ X\ {\isasymand}\ x{\isadigit{2}}\ {\isasymin}\isactrlsub c\ X\ {\isasymand}\ x{\isadigit{3}}\ {\isasymin}\isactrlsub c\ X\ {\isasymand}\ x{\isadigit{1}}\ {\isasymnoteq}\ x{\isadigit{2}}\ {\isasymand}\ x{\isadigit{2}}\ {\isasymnoteq}\ x{\isadigit{3}}\ {\isasymand}\ x{\isadigit{1}}\ {\isasymnoteq}\ x{\isadigit{3}}{\isacharparenright}{\kern0pt}{\isachardoublequoteclose}\isanewline
%
\isadelimproof
\ \ %
\endisadelimproof
%
\isatagproof
\isacommand{by}\isamarkupfalse%
\ {\isacharparenleft}{\kern0pt}metis\ not{\isacharunderscore}{\kern0pt}init{\isacharunderscore}{\kern0pt}not{\isacharunderscore}{\kern0pt}term\ size{\isacharunderscore}{\kern0pt}{\isadigit{2}}{\isacharunderscore}{\kern0pt}sets{\isacharparenright}{\kern0pt}%
\endisatagproof
{\isafoldproof}%
%
\isadelimproof
%
\endisadelimproof
%
\begin{isamarkuptext}%
The next two lemmas below correspond to Proposition 2.6.3 in Halvorson.%
\end{isamarkuptext}\isamarkuptrue%
\isacommand{lemma}\isamarkupfalse%
\ smaller{\isacharunderscore}{\kern0pt}than{\isacharunderscore}{\kern0pt}coproduct{\isadigit{1}}{\isacharcolon}{\kern0pt}\isanewline
\ \ {\isachardoublequoteopen}X\ {\isasymle}\isactrlsub c\ X\ {\isasymCoprod}\ Y{\isachardoublequoteclose}\isanewline
%
\isadelimproof
\ \ %
\endisadelimproof
%
\isatagproof
\isacommand{using}\isamarkupfalse%
\ is{\isacharunderscore}{\kern0pt}smaller{\isacharunderscore}{\kern0pt}than{\isacharunderscore}{\kern0pt}def\ left{\isacharunderscore}{\kern0pt}coproj{\isacharunderscore}{\kern0pt}are{\isacharunderscore}{\kern0pt}monomorphisms\ left{\isacharunderscore}{\kern0pt}proj{\isacharunderscore}{\kern0pt}type\ \isacommand{by}\isamarkupfalse%
\ blast%
\endisatagproof
{\isafoldproof}%
%
\isadelimproof
\isanewline
%
\endisadelimproof
\isanewline
\isacommand{lemma}\isamarkupfalse%
\ \ smaller{\isacharunderscore}{\kern0pt}than{\isacharunderscore}{\kern0pt}coproduct{\isadigit{2}}{\isacharcolon}{\kern0pt}\isanewline
\ \ {\isachardoublequoteopen}X\ {\isasymle}\isactrlsub c\ Y\ {\isasymCoprod}\ X{\isachardoublequoteclose}\isanewline
%
\isadelimproof
\ \ %
\endisadelimproof
%
\isatagproof
\isacommand{using}\isamarkupfalse%
\ is{\isacharunderscore}{\kern0pt}smaller{\isacharunderscore}{\kern0pt}than{\isacharunderscore}{\kern0pt}def\ right{\isacharunderscore}{\kern0pt}coproj{\isacharunderscore}{\kern0pt}are{\isacharunderscore}{\kern0pt}monomorphisms\ right{\isacharunderscore}{\kern0pt}proj{\isacharunderscore}{\kern0pt}type\ \isacommand{by}\isamarkupfalse%
\ blast%
\endisatagproof
{\isafoldproof}%
%
\isadelimproof
%
\endisadelimproof
%
\begin{isamarkuptext}%
The next two lemmas below correspond to Proposition 2.6.4 in Halvorson.%
\end{isamarkuptext}\isamarkuptrue%
\isacommand{lemma}\isamarkupfalse%
\ smaller{\isacharunderscore}{\kern0pt}than{\isacharunderscore}{\kern0pt}product{\isadigit{1}}{\isacharcolon}{\kern0pt}\isanewline
\ \ \isakeyword{assumes}\ {\isachardoublequoteopen}nonempty\ Y{\isachardoublequoteclose}\isanewline
\ \ \isakeyword{shows}\ {\isachardoublequoteopen}X\ {\isasymle}\isactrlsub c\ X\ {\isasymtimes}\isactrlsub c\ Y{\isachardoublequoteclose}\isanewline
%
\isadelimproof
\ \ %
\endisadelimproof
%
\isatagproof
\isacommand{unfolding}\isamarkupfalse%
\ is{\isacharunderscore}{\kern0pt}smaller{\isacharunderscore}{\kern0pt}than{\isacharunderscore}{\kern0pt}def\ \ \isanewline
\isacommand{proof}\isamarkupfalse%
\ {\isacharminus}{\kern0pt}\isanewline
\ \ \isacommand{obtain}\isamarkupfalse%
\ y\ \isakeyword{where}\ y{\isacharunderscore}{\kern0pt}type{\isacharcolon}{\kern0pt}\ {\isachardoublequoteopen}y\ {\isasymin}\isactrlsub c\ Y{\isachardoublequoteclose}\isanewline
\ \ \isacommand{using}\isamarkupfalse%
\ assms\ nonempty{\isacharunderscore}{\kern0pt}def\ \isacommand{by}\isamarkupfalse%
\ blast\isanewline
\ \ \isacommand{have}\isamarkupfalse%
\ map{\isacharunderscore}{\kern0pt}type{\isacharcolon}{\kern0pt}\ {\isachardoublequoteopen}{\isasymlangle}id{\isacharparenleft}{\kern0pt}X{\isacharparenright}{\kern0pt}{\isacharcomma}{\kern0pt}y\ {\isasymcirc}\isactrlsub c\ {\isasymbeta}\isactrlbsub X\isactrlesub {\isasymrangle}\ {\isacharcolon}{\kern0pt}\ X\ {\isasymrightarrow}\ X\ {\isasymtimes}\isactrlsub c\ Y{\isachardoublequoteclose}\isanewline
\ \ \ \isacommand{using}\isamarkupfalse%
\ y{\isacharunderscore}{\kern0pt}type\ cfunc{\isacharunderscore}{\kern0pt}prod{\isacharunderscore}{\kern0pt}type\ cfunc{\isacharunderscore}{\kern0pt}type{\isacharunderscore}{\kern0pt}def\ codomain{\isacharunderscore}{\kern0pt}comp\ domain{\isacharunderscore}{\kern0pt}comp\ id{\isacharunderscore}{\kern0pt}type\ terminal{\isacharunderscore}{\kern0pt}func{\isacharunderscore}{\kern0pt}type\ \isacommand{by}\isamarkupfalse%
\ auto\isanewline
\ \ \isacommand{have}\isamarkupfalse%
\ mono{\isacharcolon}{\kern0pt}\ {\isachardoublequoteopen}monomorphism{\isacharparenleft}{\kern0pt}{\isasymlangle}id\ X{\isacharcomma}{\kern0pt}\ y\ {\isasymcirc}\isactrlsub c\ {\isasymbeta}\isactrlbsub X\isactrlesub {\isasymrangle}{\isacharparenright}{\kern0pt}{\isachardoublequoteclose}\isanewline
\ \ \ \ \isacommand{using}\isamarkupfalse%
\ map{\isacharunderscore}{\kern0pt}type\isanewline
\ \ \isacommand{proof}\isamarkupfalse%
\ {\isacharparenleft}{\kern0pt}unfold\ monomorphism{\isacharunderscore}{\kern0pt}def{\isadigit{3}}{\isacharcomma}{\kern0pt}\ clarify{\isacharparenright}{\kern0pt}\isanewline
\ \ \ \ \isacommand{fix}\isamarkupfalse%
\ g\ h\ A\isanewline
\ \ \ \ \isacommand{assume}\isamarkupfalse%
\ g{\isacharunderscore}{\kern0pt}h{\isacharunderscore}{\kern0pt}types{\isacharcolon}{\kern0pt}\ {\isachardoublequoteopen}g\ {\isacharcolon}{\kern0pt}\ A\ {\isasymrightarrow}\ X{\isachardoublequoteclose}\ {\isachardoublequoteopen}h\ {\isacharcolon}{\kern0pt}\ A\ {\isasymrightarrow}\ X{\isachardoublequoteclose}\isanewline
\ \ \ \ \isanewline
\ \ \ \ \isacommand{assume}\isamarkupfalse%
\ {\isachardoublequoteopen}{\isasymlangle}id\isactrlsub c\ X{\isacharcomma}{\kern0pt}y\ {\isasymcirc}\isactrlsub c\ {\isasymbeta}\isactrlbsub X\isactrlesub {\isasymrangle}\ {\isasymcirc}\isactrlsub c\ g\ {\isacharequal}{\kern0pt}\ {\isasymlangle}id\isactrlsub c\ X{\isacharcomma}{\kern0pt}y\ {\isasymcirc}\isactrlsub c\ {\isasymbeta}\isactrlbsub X\isactrlesub {\isasymrangle}\ {\isasymcirc}\isactrlsub c\ h{\isachardoublequoteclose}\isanewline
\ \ \ \ \isacommand{then}\isamarkupfalse%
\ \isacommand{have}\isamarkupfalse%
\ {\isachardoublequoteopen}{\isasymlangle}id\isactrlsub c\ X\ {\isasymcirc}\isactrlsub c\ g{\isacharcomma}{\kern0pt}\ y\ {\isasymcirc}\isactrlsub c\ {\isasymbeta}\isactrlbsub X\isactrlesub \ {\isasymcirc}\isactrlsub c\ g{\isasymrangle}\ \ {\isacharequal}{\kern0pt}\ {\isasymlangle}id\isactrlsub c\ X\ {\isasymcirc}\isactrlsub c\ h{\isacharcomma}{\kern0pt}\ y\ {\isasymcirc}\isactrlsub c\ {\isasymbeta}\isactrlbsub X\isactrlesub \ {\isasymcirc}\isactrlsub c\ h{\isasymrangle}{\isachardoublequoteclose}\isanewline
\ \ \ \ \ \ \isacommand{using}\isamarkupfalse%
\ y{\isacharunderscore}{\kern0pt}type\ g{\isacharunderscore}{\kern0pt}h{\isacharunderscore}{\kern0pt}types\ \isacommand{by}\isamarkupfalse%
\ {\isacharparenleft}{\kern0pt}typecheck{\isacharunderscore}{\kern0pt}cfuncs{\isacharcomma}{\kern0pt}\ smt\ cfunc{\isacharunderscore}{\kern0pt}prod{\isacharunderscore}{\kern0pt}comp\ comp{\isacharunderscore}{\kern0pt}associative{\isadigit{2}}\ comp{\isacharunderscore}{\kern0pt}type{\isacharparenright}{\kern0pt}\isanewline
\ \ \ \ \isacommand{then}\isamarkupfalse%
\ \isacommand{have}\isamarkupfalse%
\ {\isachardoublequoteopen}{\isasymlangle}g{\isacharcomma}{\kern0pt}\ y\ {\isasymcirc}\isactrlsub c\ {\isasymbeta}\isactrlbsub A\isactrlesub {\isasymrangle}\ \ {\isacharequal}{\kern0pt}\ {\isasymlangle}h{\isacharcomma}{\kern0pt}\ y\ {\isasymcirc}\isactrlsub c\ {\isasymbeta}\isactrlbsub A\isactrlesub {\isasymrangle}{\isachardoublequoteclose}\isanewline
\ \ \ \ \ \ \isacommand{using}\isamarkupfalse%
\ y{\isacharunderscore}{\kern0pt}type\ g{\isacharunderscore}{\kern0pt}h{\isacharunderscore}{\kern0pt}types\ id{\isacharunderscore}{\kern0pt}left{\isacharunderscore}{\kern0pt}unit{\isadigit{2}}\ terminal{\isacharunderscore}{\kern0pt}func{\isacharunderscore}{\kern0pt}comp\ \isacommand{by}\isamarkupfalse%
\ {\isacharparenleft}{\kern0pt}typecheck{\isacharunderscore}{\kern0pt}cfuncs{\isacharcomma}{\kern0pt}\ auto{\isacharparenright}{\kern0pt}\isanewline
\ \ \ \ \isacommand{then}\isamarkupfalse%
\ \isacommand{show}\isamarkupfalse%
\ {\isachardoublequoteopen}g\ {\isacharequal}{\kern0pt}\ h{\isachardoublequoteclose}\isanewline
\ \ \ \ \ \ \isacommand{using}\isamarkupfalse%
\ g{\isacharunderscore}{\kern0pt}h{\isacharunderscore}{\kern0pt}types\ y{\isacharunderscore}{\kern0pt}type\isanewline
\ \ \ \ \ \ \isacommand{by}\isamarkupfalse%
\ {\isacharparenleft}{\kern0pt}metis\ {\isacharparenleft}{\kern0pt}full{\isacharunderscore}{\kern0pt}types{\isacharparenright}{\kern0pt}\ comp{\isacharunderscore}{\kern0pt}type\ left{\isacharunderscore}{\kern0pt}cart{\isacharunderscore}{\kern0pt}proj{\isacharunderscore}{\kern0pt}cfunc{\isacharunderscore}{\kern0pt}prod\ terminal{\isacharunderscore}{\kern0pt}func{\isacharunderscore}{\kern0pt}type{\isacharparenright}{\kern0pt}\isanewline
\ \ \isacommand{qed}\isamarkupfalse%
\isanewline
\ \ \isacommand{show}\isamarkupfalse%
\ {\isachardoublequoteopen}{\isasymexists}m{\isachardot}{\kern0pt}\ m\ {\isacharcolon}{\kern0pt}\ X\ {\isasymrightarrow}\ X\ {\isasymtimes}\isactrlsub c\ Y\ {\isasymand}\ monomorphism\ m{\isachardoublequoteclose}\isanewline
\ \ \ \ \isacommand{using}\isamarkupfalse%
\ mono\ map{\isacharunderscore}{\kern0pt}type\ \isacommand{by}\isamarkupfalse%
\ auto\isanewline
\isacommand{qed}\isamarkupfalse%
%
\endisatagproof
{\isafoldproof}%
%
\isadelimproof
\isanewline
%
\endisadelimproof
\isanewline
\isacommand{lemma}\isamarkupfalse%
\ smaller{\isacharunderscore}{\kern0pt}than{\isacharunderscore}{\kern0pt}product{\isadigit{2}}{\isacharcolon}{\kern0pt}\isanewline
\ \ \isakeyword{assumes}\ {\isachardoublequoteopen}nonempty\ Y{\isachardoublequoteclose}\isanewline
\ \ \isakeyword{shows}\ {\isachardoublequoteopen}X\ {\isasymle}\isactrlsub c\ Y\ {\isasymtimes}\isactrlsub c\ X{\isachardoublequoteclose}\isanewline
%
\isadelimproof
\ \ %
\endisadelimproof
%
\isatagproof
\isacommand{unfolding}\isamarkupfalse%
\ is{\isacharunderscore}{\kern0pt}smaller{\isacharunderscore}{\kern0pt}than{\isacharunderscore}{\kern0pt}def\ \ \isanewline
\isacommand{proof}\isamarkupfalse%
\ {\isacharminus}{\kern0pt}\ \isanewline
\ \ \isacommand{have}\isamarkupfalse%
\ {\isachardoublequoteopen}X\ {\isasymle}\isactrlsub c\ X\ {\isasymtimes}\isactrlsub c\ Y{\isachardoublequoteclose}\isanewline
\ \ \ \ \isacommand{by}\isamarkupfalse%
\ {\isacharparenleft}{\kern0pt}simp\ add{\isacharcolon}{\kern0pt}\ assms\ smaller{\isacharunderscore}{\kern0pt}than{\isacharunderscore}{\kern0pt}product{\isadigit{1}}{\isacharparenright}{\kern0pt}\isanewline
\ \ \isacommand{then}\isamarkupfalse%
\ \isacommand{obtain}\isamarkupfalse%
\ m\ \isakeyword{where}\ m{\isacharunderscore}{\kern0pt}def{\isacharcolon}{\kern0pt}\ {\isachardoublequoteopen}m\ {\isacharcolon}{\kern0pt}\ X\ {\isasymrightarrow}\ X\ {\isasymtimes}\isactrlsub c\ Y\ {\isasymand}\ monomorphism\ m{\isachardoublequoteclose}\isanewline
\ \ \ \ \isacommand{using}\isamarkupfalse%
\ is{\isacharunderscore}{\kern0pt}smaller{\isacharunderscore}{\kern0pt}than{\isacharunderscore}{\kern0pt}def\ \isacommand{by}\isamarkupfalse%
\ blast\isanewline
\ \ \isacommand{obtain}\isamarkupfalse%
\ i\ \ \isakeyword{where}\ {\isachardoublequoteopen}i\ {\isacharcolon}{\kern0pt}\ {\isacharparenleft}{\kern0pt}X\ {\isasymtimes}\isactrlsub c\ Y{\isacharparenright}{\kern0pt}\ {\isasymrightarrow}\ {\isacharparenleft}{\kern0pt}Y\ {\isasymtimes}\isactrlsub c\ X{\isacharparenright}{\kern0pt}\ {\isasymand}\ isomorphism\ i{\isachardoublequoteclose}\isanewline
\ \ \ \ \isacommand{using}\isamarkupfalse%
\ is{\isacharunderscore}{\kern0pt}isomorphic{\isacharunderscore}{\kern0pt}def\ product{\isacharunderscore}{\kern0pt}commutes\ \isacommand{by}\isamarkupfalse%
\ blast\isanewline
\ \ \isacommand{then}\isamarkupfalse%
\ \isacommand{have}\isamarkupfalse%
\ {\isachardoublequoteopen}i\ {\isasymcirc}\isactrlsub c\ m\ {\isacharcolon}{\kern0pt}\ X\ {\isasymrightarrow}\ \ {\isacharparenleft}{\kern0pt}Y\ {\isasymtimes}\isactrlsub c\ X{\isacharparenright}{\kern0pt}\ {\isasymand}\ monomorphism{\isacharparenleft}{\kern0pt}i\ {\isasymcirc}\isactrlsub c\ m{\isacharparenright}{\kern0pt}{\isachardoublequoteclose}\isanewline
\ \ \ \ \isacommand{using}\isamarkupfalse%
\ cfunc{\isacharunderscore}{\kern0pt}type{\isacharunderscore}{\kern0pt}def\ comp{\isacharunderscore}{\kern0pt}type\ composition{\isacharunderscore}{\kern0pt}of{\isacharunderscore}{\kern0pt}monic{\isacharunderscore}{\kern0pt}pair{\isacharunderscore}{\kern0pt}is{\isacharunderscore}{\kern0pt}monic\ iso{\isacharunderscore}{\kern0pt}imp{\isacharunderscore}{\kern0pt}epi{\isacharunderscore}{\kern0pt}and{\isacharunderscore}{\kern0pt}monic\ m{\isacharunderscore}{\kern0pt}def\ \isacommand{by}\isamarkupfalse%
\ auto\isanewline
\ \ \isacommand{then}\isamarkupfalse%
\ \isacommand{show}\isamarkupfalse%
\ {\isachardoublequoteopen}{\isasymexists}m{\isachardot}{\kern0pt}\ m\ {\isacharcolon}{\kern0pt}\ X\ {\isasymrightarrow}\ Y\ {\isasymtimes}\isactrlsub c\ X\ {\isasymand}\ monomorphism\ m{\isachardoublequoteclose}\isanewline
\ \ \ \ \isacommand{by}\isamarkupfalse%
\ blast\isanewline
\isacommand{qed}\isamarkupfalse%
%
\endisatagproof
{\isafoldproof}%
%
\isadelimproof
\isanewline
%
\endisadelimproof
\isanewline
\isacommand{lemma}\isamarkupfalse%
\ coprod{\isacharunderscore}{\kern0pt}leq{\isacharunderscore}{\kern0pt}product{\isacharcolon}{\kern0pt}\isanewline
\ \ \isakeyword{assumes}\ X{\isacharunderscore}{\kern0pt}not{\isacharunderscore}{\kern0pt}init{\isacharcolon}{\kern0pt}\ {\isachardoublequoteopen}{\isasymnot}{\isacharparenleft}{\kern0pt}initial{\isacharunderscore}{\kern0pt}object{\isacharparenleft}{\kern0pt}X{\isacharparenright}{\kern0pt}{\isacharparenright}{\kern0pt}{\isachardoublequoteclose}\ \isanewline
\ \ \isakeyword{assumes}\ Y{\isacharunderscore}{\kern0pt}not{\isacharunderscore}{\kern0pt}init{\isacharcolon}{\kern0pt}\ {\isachardoublequoteopen}{\isasymnot}{\isacharparenleft}{\kern0pt}initial{\isacharunderscore}{\kern0pt}object{\isacharparenleft}{\kern0pt}Y{\isacharparenright}{\kern0pt}{\isacharparenright}{\kern0pt}{\isachardoublequoteclose}\ \isanewline
\ \ \isakeyword{assumes}\ X{\isacharunderscore}{\kern0pt}not{\isacharunderscore}{\kern0pt}term{\isacharcolon}{\kern0pt}\ {\isachardoublequoteopen}{\isasymnot}{\isacharparenleft}{\kern0pt}terminal{\isacharunderscore}{\kern0pt}object{\isacharparenleft}{\kern0pt}X{\isacharparenright}{\kern0pt}{\isacharparenright}{\kern0pt}{\isachardoublequoteclose}\isanewline
\ \ \isakeyword{assumes}\ Y{\isacharunderscore}{\kern0pt}not{\isacharunderscore}{\kern0pt}term{\isacharcolon}{\kern0pt}\ {\isachardoublequoteopen}{\isasymnot}{\isacharparenleft}{\kern0pt}terminal{\isacharunderscore}{\kern0pt}object{\isacharparenleft}{\kern0pt}Y{\isacharparenright}{\kern0pt}{\isacharparenright}{\kern0pt}{\isachardoublequoteclose}\isanewline
\ \ \isakeyword{shows}\ {\isachardoublequoteopen}X\ {\isasymCoprod}\ Y\ {\isasymle}\isactrlsub c\ X\ {\isasymtimes}\isactrlsub c\ Y{\isachardoublequoteclose}\isanewline
%
\isadelimproof
%
\endisadelimproof
%
\isatagproof
\isacommand{proof}\isamarkupfalse%
\ {\isacharminus}{\kern0pt}\ \isanewline
\ \ \isacommand{obtain}\isamarkupfalse%
\ x{\isadigit{1}}\ x{\isadigit{2}}\ \isakeyword{where}\ x{\isadigit{1}}x{\isadigit{2}}{\isacharunderscore}{\kern0pt}def{\isacharbrackleft}{\kern0pt}type{\isacharunderscore}{\kern0pt}rule{\isacharbrackright}{\kern0pt}{\isacharcolon}{\kern0pt}\ \ {\isachardoublequoteopen}{\isacharparenleft}{\kern0pt}x{\isadigit{1}}\ {\isasymin}\isactrlsub c\ X{\isacharparenright}{\kern0pt}{\isachardoublequoteclose}\ {\isachardoublequoteopen}{\isacharparenleft}{\kern0pt}x{\isadigit{2}}\ {\isasymin}\isactrlsub c\ X{\isacharparenright}{\kern0pt}{\isachardoublequoteclose}\ {\isachardoublequoteopen}{\isacharparenleft}{\kern0pt}x{\isadigit{1}}\ {\isasymnoteq}\ x{\isadigit{2}}{\isacharparenright}{\kern0pt}{\isachardoublequoteclose}\isanewline
\ \ \ \ \isacommand{using}\isamarkupfalse%
\ is{\isacharunderscore}{\kern0pt}empty{\isacharunderscore}{\kern0pt}def\ X{\isacharunderscore}{\kern0pt}not{\isacharunderscore}{\kern0pt}init\ X{\isacharunderscore}{\kern0pt}not{\isacharunderscore}{\kern0pt}term\ iso{\isacharunderscore}{\kern0pt}empty{\isacharunderscore}{\kern0pt}initial\ iso{\isacharunderscore}{\kern0pt}to{\isadigit{1}}{\isacharunderscore}{\kern0pt}is{\isacharunderscore}{\kern0pt}term\ no{\isacharunderscore}{\kern0pt}el{\isacharunderscore}{\kern0pt}iff{\isacharunderscore}{\kern0pt}iso{\isacharunderscore}{\kern0pt}empty\ single{\isacharunderscore}{\kern0pt}elem{\isacharunderscore}{\kern0pt}iso{\isacharunderscore}{\kern0pt}one\ \isacommand{by}\isamarkupfalse%
\ blast\isanewline
\ \ \isacommand{obtain}\isamarkupfalse%
\ y{\isadigit{1}}\ y{\isadigit{2}}\ \isakeyword{where}\ y{\isadigit{1}}y{\isadigit{2}}{\isacharunderscore}{\kern0pt}def{\isacharbrackleft}{\kern0pt}type{\isacharunderscore}{\kern0pt}rule{\isacharbrackright}{\kern0pt}{\isacharcolon}{\kern0pt}\ \ {\isachardoublequoteopen}{\isacharparenleft}{\kern0pt}y{\isadigit{1}}\ {\isasymin}\isactrlsub c\ Y{\isacharparenright}{\kern0pt}{\isachardoublequoteclose}\ {\isachardoublequoteopen}{\isacharparenleft}{\kern0pt}y{\isadigit{2}}\ {\isasymin}\isactrlsub c\ Y{\isacharparenright}{\kern0pt}{\isachardoublequoteclose}\ {\isachardoublequoteopen}{\isacharparenleft}{\kern0pt}y{\isadigit{1}}\ {\isasymnoteq}\ y{\isadigit{2}}{\isacharparenright}{\kern0pt}{\isachardoublequoteclose}\isanewline
\ \ \ \ \isacommand{using}\isamarkupfalse%
\ is{\isacharunderscore}{\kern0pt}empty{\isacharunderscore}{\kern0pt}def\ Y{\isacharunderscore}{\kern0pt}not{\isacharunderscore}{\kern0pt}init\ Y{\isacharunderscore}{\kern0pt}not{\isacharunderscore}{\kern0pt}term\ iso{\isacharunderscore}{\kern0pt}empty{\isacharunderscore}{\kern0pt}initial\ iso{\isacharunderscore}{\kern0pt}to{\isadigit{1}}{\isacharunderscore}{\kern0pt}is{\isacharunderscore}{\kern0pt}term\ no{\isacharunderscore}{\kern0pt}el{\isacharunderscore}{\kern0pt}iff{\isacharunderscore}{\kern0pt}iso{\isacharunderscore}{\kern0pt}empty\ single{\isacharunderscore}{\kern0pt}elem{\isacharunderscore}{\kern0pt}iso{\isacharunderscore}{\kern0pt}one\ \isacommand{by}\isamarkupfalse%
\ blast\isanewline
\ \ \isacommand{then}\isamarkupfalse%
\ \isacommand{have}\isamarkupfalse%
\ y{\isadigit{1}}{\isacharunderscore}{\kern0pt}mono{\isacharbrackleft}{\kern0pt}type{\isacharunderscore}{\kern0pt}rule{\isacharbrackright}{\kern0pt}{\isacharcolon}{\kern0pt}\ {\isachardoublequoteopen}monomorphism{\isacharparenleft}{\kern0pt}y{\isadigit{1}}{\isacharparenright}{\kern0pt}{\isachardoublequoteclose}\isanewline
\ \ \ \ \isacommand{using}\isamarkupfalse%
\ element{\isacharunderscore}{\kern0pt}monomorphism\ \isacommand{by}\isamarkupfalse%
\ blast\isanewline
\ \ \isacommand{obtain}\isamarkupfalse%
\ m\ \isakeyword{where}\ m{\isacharunderscore}{\kern0pt}def{\isacharcolon}{\kern0pt}\ {\isachardoublequoteopen}m\ {\isacharequal}{\kern0pt}\ {\isasymlangle}id{\isacharparenleft}{\kern0pt}X{\isacharparenright}{\kern0pt}{\isacharcomma}{\kern0pt}\ y{\isadigit{1}}\ {\isasymcirc}\isactrlsub c\ {\isasymbeta}\isactrlbsub X\isactrlesub {\isasymrangle}\ {\isasymamalg}\ {\isacharparenleft}{\kern0pt}{\isacharparenleft}{\kern0pt}{\isasymlangle}x{\isadigit{2}}{\isacharcomma}{\kern0pt}\ y{\isadigit{2}}{\isasymrangle}\ {\isasymamalg}\ {\isasymlangle}x{\isadigit{1}}\ {\isasymcirc}\isactrlsub c\ {\isasymbeta}\isactrlbsub Y\ {\isasymsetminus}\ {\isacharparenleft}{\kern0pt}{\isasymone}{\isacharcomma}{\kern0pt}y{\isadigit{1}}{\isacharparenright}{\kern0pt}\isactrlesub {\isacharcomma}{\kern0pt}\ y{\isadigit{1}}\isactrlsup c{\isasymrangle}{\isacharparenright}{\kern0pt}\ {\isasymcirc}\isactrlsub c\ \ try{\isacharunderscore}{\kern0pt}cast\ y{\isadigit{1}}{\isacharparenright}{\kern0pt}{\isachardoublequoteclose}\isanewline
\ \ \ \ \isacommand{by}\isamarkupfalse%
\ simp\isanewline
\ \ \isacommand{have}\isamarkupfalse%
\ type{\isadigit{1}}{\isacharcolon}{\kern0pt}\ {\isachardoublequoteopen}{\isasymlangle}id{\isacharparenleft}{\kern0pt}X{\isacharparenright}{\kern0pt}{\isacharcomma}{\kern0pt}\ y{\isadigit{1}}\ {\isasymcirc}\isactrlsub c\ {\isasymbeta}\isactrlbsub X\isactrlesub {\isasymrangle}\ {\isacharcolon}{\kern0pt}\ X\ {\isasymrightarrow}\ {\isacharparenleft}{\kern0pt}X\ {\isasymtimes}\isactrlsub c\ Y{\isacharparenright}{\kern0pt}{\isachardoublequoteclose}\isanewline
\ \ \ \ \isacommand{by}\isamarkupfalse%
\ {\isacharparenleft}{\kern0pt}meson\ cfunc{\isacharunderscore}{\kern0pt}prod{\isacharunderscore}{\kern0pt}type\ comp{\isacharunderscore}{\kern0pt}type\ id{\isacharunderscore}{\kern0pt}type\ terminal{\isacharunderscore}{\kern0pt}func{\isacharunderscore}{\kern0pt}type\ y{\isadigit{1}}y{\isadigit{2}}{\isacharunderscore}{\kern0pt}def{\isacharparenright}{\kern0pt}\isanewline
\ \ \isacommand{have}\isamarkupfalse%
\ trycast{\isacharunderscore}{\kern0pt}y{\isadigit{1}}{\isacharunderscore}{\kern0pt}type{\isacharcolon}{\kern0pt}\ {\isachardoublequoteopen}try{\isacharunderscore}{\kern0pt}cast\ y{\isadigit{1}}\ {\isacharcolon}{\kern0pt}\ Y\ {\isasymrightarrow}\ {\isasymone}\ {\isasymCoprod}\ {\isacharparenleft}{\kern0pt}Y\ {\isasymsetminus}\ {\isacharparenleft}{\kern0pt}{\isasymone}{\isacharcomma}{\kern0pt}y{\isadigit{1}}{\isacharparenright}{\kern0pt}{\isacharparenright}{\kern0pt}{\isachardoublequoteclose}\isanewline
\ \ \ \ \isacommand{by}\isamarkupfalse%
\ {\isacharparenleft}{\kern0pt}meson\ element{\isacharunderscore}{\kern0pt}monomorphism\ try{\isacharunderscore}{\kern0pt}cast{\isacharunderscore}{\kern0pt}type\ y{\isadigit{1}}y{\isadigit{2}}{\isacharunderscore}{\kern0pt}def{\isacharparenright}{\kern0pt}\isanewline
\ \ \isacommand{have}\isamarkupfalse%
\ y{\isadigit{1}}{\isacharprime}{\kern0pt}{\isacharunderscore}{\kern0pt}type{\isacharbrackleft}{\kern0pt}type{\isacharunderscore}{\kern0pt}rule{\isacharbrackright}{\kern0pt}{\isacharcolon}{\kern0pt}\ {\isachardoublequoteopen}y{\isadigit{1}}\isactrlsup c\ {\isacharcolon}{\kern0pt}\ Y\ {\isasymsetminus}\ {\isacharparenleft}{\kern0pt}{\isasymone}{\isacharcomma}{\kern0pt}y{\isadigit{1}}{\isacharparenright}{\kern0pt}\ {\isasymrightarrow}\ Y{\isachardoublequoteclose}\isanewline
\ \ \ \ \isacommand{using}\isamarkupfalse%
\ complement{\isacharunderscore}{\kern0pt}morphism{\isacharunderscore}{\kern0pt}type\ one{\isacharunderscore}{\kern0pt}terminal{\isacharunderscore}{\kern0pt}object\ terminal{\isacharunderscore}{\kern0pt}el{\isacharunderscore}{\kern0pt}monomorphism\ y{\isadigit{1}}y{\isadigit{2}}{\isacharunderscore}{\kern0pt}def\ \isacommand{by}\isamarkupfalse%
\ blast\isanewline
\ \ \isacommand{have}\isamarkupfalse%
\ type{\isadigit{4}}{\isacharcolon}{\kern0pt}\ {\isachardoublequoteopen}{\isasymlangle}x{\isadigit{1}}\ {\isasymcirc}\isactrlsub c\ {\isasymbeta}\isactrlbsub Y\ {\isasymsetminus}\ {\isacharparenleft}{\kern0pt}{\isasymone}{\isacharcomma}{\kern0pt}y{\isadigit{1}}{\isacharparenright}{\kern0pt}\isactrlesub {\isacharcomma}{\kern0pt}\ y{\isadigit{1}}\isactrlsup c{\isasymrangle}\ {\isacharcolon}{\kern0pt}\ Y\ {\isasymsetminus}\ {\isacharparenleft}{\kern0pt}{\isasymone}{\isacharcomma}{\kern0pt}y{\isadigit{1}}{\isacharparenright}{\kern0pt}\ {\isasymrightarrow}\ {\isacharparenleft}{\kern0pt}X\ {\isasymtimes}\isactrlsub c\ Y{\isacharparenright}{\kern0pt}{\isachardoublequoteclose}\isanewline
\ \ \ \ \isacommand{using}\isamarkupfalse%
\ cfunc{\isacharunderscore}{\kern0pt}prod{\isacharunderscore}{\kern0pt}type\ comp{\isacharunderscore}{\kern0pt}type\ terminal{\isacharunderscore}{\kern0pt}func{\isacharunderscore}{\kern0pt}type\ x{\isadigit{1}}x{\isadigit{2}}{\isacharunderscore}{\kern0pt}def\ y{\isadigit{1}}{\isacharprime}{\kern0pt}{\isacharunderscore}{\kern0pt}type\ \isacommand{by}\isamarkupfalse%
\ blast\isanewline
\ \ \isacommand{have}\isamarkupfalse%
\ type{\isadigit{5}}{\isacharcolon}{\kern0pt}\ {\isachardoublequoteopen}{\isasymlangle}x{\isadigit{2}}{\isacharcomma}{\kern0pt}\ y{\isadigit{2}}{\isasymrangle}\ {\isasymin}\isactrlsub c\ {\isacharparenleft}{\kern0pt}X\ {\isasymtimes}\isactrlsub c\ Y{\isacharparenright}{\kern0pt}{\isachardoublequoteclose}\isanewline
\ \ \ \ \isacommand{by}\isamarkupfalse%
\ {\isacharparenleft}{\kern0pt}simp\ add{\isacharcolon}{\kern0pt}\ cfunc{\isacharunderscore}{\kern0pt}prod{\isacharunderscore}{\kern0pt}type\ x{\isadigit{1}}x{\isadigit{2}}{\isacharunderscore}{\kern0pt}def\ y{\isadigit{1}}y{\isadigit{2}}{\isacharunderscore}{\kern0pt}def{\isacharparenright}{\kern0pt}\isanewline
\ \ \isacommand{then}\isamarkupfalse%
\ \isacommand{have}\isamarkupfalse%
\ type{\isadigit{6}}{\isacharcolon}{\kern0pt}\ {\isachardoublequoteopen}{\isasymlangle}x{\isadigit{2}}{\isacharcomma}{\kern0pt}\ y{\isadigit{2}}{\isasymrangle}\ {\isasymamalg}\ {\isasymlangle}x{\isadigit{1}}\ {\isasymcirc}\isactrlsub c\ {\isasymbeta}\isactrlbsub Y\ {\isasymsetminus}\ {\isacharparenleft}{\kern0pt}{\isasymone}{\isacharcomma}{\kern0pt}y{\isadigit{1}}{\isacharparenright}{\kern0pt}\isactrlesub {\isacharcomma}{\kern0pt}\ y{\isadigit{1}}\isactrlsup c{\isasymrangle}\ {\isacharcolon}{\kern0pt}{\isacharparenleft}{\kern0pt}{\isasymone}\ {\isasymCoprod}\ {\isacharparenleft}{\kern0pt}Y\ {\isasymsetminus}\ {\isacharparenleft}{\kern0pt}{\isasymone}{\isacharcomma}{\kern0pt}y{\isadigit{1}}{\isacharparenright}{\kern0pt}{\isacharparenright}{\kern0pt}{\isacharparenright}{\kern0pt}\ {\isasymrightarrow}\ {\isacharparenleft}{\kern0pt}X\ {\isasymtimes}\isactrlsub c\ Y{\isacharparenright}{\kern0pt}{\isachardoublequoteclose}\isanewline
\ \ \ \ \isacommand{using}\isamarkupfalse%
\ cfunc{\isacharunderscore}{\kern0pt}coprod{\isacharunderscore}{\kern0pt}type\ type{\isadigit{4}}\ \isacommand{by}\isamarkupfalse%
\ blast\isanewline
\ \ \isacommand{then}\isamarkupfalse%
\ \isacommand{have}\isamarkupfalse%
\ type{\isadigit{7}}{\isacharcolon}{\kern0pt}\ {\isachardoublequoteopen}{\isacharparenleft}{\kern0pt}{\isacharparenleft}{\kern0pt}{\isasymlangle}x{\isadigit{2}}{\isacharcomma}{\kern0pt}\ y{\isadigit{2}}{\isasymrangle}\ {\isasymamalg}\ {\isasymlangle}x{\isadigit{1}}\ {\isasymcirc}\isactrlsub c\ {\isasymbeta}\isactrlbsub Y\ {\isasymsetminus}\ {\isacharparenleft}{\kern0pt}{\isasymone}{\isacharcomma}{\kern0pt}y{\isadigit{1}}{\isacharparenright}{\kern0pt}\isactrlesub {\isacharcomma}{\kern0pt}\ y{\isadigit{1}}\isactrlsup c{\isasymrangle}{\isacharparenright}{\kern0pt}\ {\isasymcirc}\isactrlsub c\ \ try{\isacharunderscore}{\kern0pt}cast\ y{\isadigit{1}}{\isacharparenright}{\kern0pt}\ {\isacharcolon}{\kern0pt}\ Y\ {\isasymrightarrow}\ {\isacharparenleft}{\kern0pt}X\ {\isasymtimes}\isactrlsub c\ Y{\isacharparenright}{\kern0pt}{\isachardoublequoteclose}\isanewline
\ \ \ \ \isacommand{using}\isamarkupfalse%
\ comp{\isacharunderscore}{\kern0pt}type\ trycast{\isacharunderscore}{\kern0pt}y{\isadigit{1}}{\isacharunderscore}{\kern0pt}type\ \isacommand{by}\isamarkupfalse%
\ blast\isanewline
\ \ \isacommand{then}\isamarkupfalse%
\ \isacommand{have}\isamarkupfalse%
\ m{\isacharunderscore}{\kern0pt}type{\isacharcolon}{\kern0pt}\ {\isachardoublequoteopen}m\ {\isacharcolon}{\kern0pt}\ X\ \ {\isasymCoprod}\ Y\ {\isasymrightarrow}\ {\isacharparenleft}{\kern0pt}X\ {\isasymtimes}\isactrlsub c\ Y{\isacharparenright}{\kern0pt}{\isachardoublequoteclose}\isanewline
\ \ \ \ \isacommand{by}\isamarkupfalse%
\ {\isacharparenleft}{\kern0pt}simp\ add{\isacharcolon}{\kern0pt}\ cfunc{\isacharunderscore}{\kern0pt}coprod{\isacharunderscore}{\kern0pt}type\ m{\isacharunderscore}{\kern0pt}def\ type{\isadigit{1}}{\isacharparenright}{\kern0pt}\isanewline
\isanewline
\ \ \isacommand{have}\isamarkupfalse%
\ relative{\isacharcolon}{\kern0pt}\ {\isachardoublequoteopen}{\isasymAnd}y{\isachardot}{\kern0pt}\ y\ {\isasymin}\isactrlsub c\ Y\ {\isasymLongrightarrow}\ {\isacharparenleft}{\kern0pt}y\ {\isasymin}\isactrlbsub Y\isactrlesub \ {\isacharparenleft}{\kern0pt}{\isasymone}{\isacharcomma}{\kern0pt}\ y{\isadigit{1}}{\isacharparenright}{\kern0pt}{\isacharparenright}{\kern0pt}\ {\isacharequal}{\kern0pt}\ {\isacharparenleft}{\kern0pt}y\ {\isacharequal}{\kern0pt}\ y{\isadigit{1}}{\isacharparenright}{\kern0pt}{\isachardoublequoteclose}\isanewline
\ \ \isacommand{proof}\isamarkupfalse%
{\isacharparenleft}{\kern0pt}safe{\isacharparenright}{\kern0pt}\isanewline
\ \ \ \ \isacommand{fix}\isamarkupfalse%
\ y\ \isanewline
\ \ \ \ \isacommand{assume}\isamarkupfalse%
\ y{\isacharunderscore}{\kern0pt}type{\isacharcolon}{\kern0pt}\ {\isachardoublequoteopen}y\ {\isasymin}\isactrlsub c\ Y{\isachardoublequoteclose}\isanewline
\ \ \ \ \isacommand{show}\isamarkupfalse%
\ {\isachardoublequoteopen}y\ {\isasymin}\isactrlbsub Y\isactrlesub \ {\isacharparenleft}{\kern0pt}{\isasymone}{\isacharcomma}{\kern0pt}\ y{\isadigit{1}}{\isacharparenright}{\kern0pt}\ {\isasymLongrightarrow}\ y\ {\isacharequal}{\kern0pt}\ y{\isadigit{1}}{\isachardoublequoteclose}\isanewline
\ \ \ \ \ \ \isacommand{by}\isamarkupfalse%
\ {\isacharparenleft}{\kern0pt}metis\ cfunc{\isacharunderscore}{\kern0pt}type{\isacharunderscore}{\kern0pt}def\ factors{\isacharunderscore}{\kern0pt}through{\isacharunderscore}{\kern0pt}def\ id{\isacharunderscore}{\kern0pt}right{\isacharunderscore}{\kern0pt}unit{\isadigit{2}}\ id{\isacharunderscore}{\kern0pt}type\ one{\isacharunderscore}{\kern0pt}unique{\isacharunderscore}{\kern0pt}element\ relative{\isacharunderscore}{\kern0pt}member{\isacharunderscore}{\kern0pt}def{\isadigit{2}}{\isacharparenright}{\kern0pt}\isanewline
\ \ \isacommand{next}\isamarkupfalse%
\ \isanewline
\ \ \ \ \isacommand{show}\isamarkupfalse%
\ {\isachardoublequoteopen}y{\isadigit{1}}\ {\isasymin}\isactrlsub c\ Y\ {\isasymLongrightarrow}\ y{\isadigit{1}}\ {\isasymin}\isactrlbsub Y\isactrlesub \ {\isacharparenleft}{\kern0pt}{\isasymone}{\isacharcomma}{\kern0pt}\ y{\isadigit{1}}{\isacharparenright}{\kern0pt}{\isachardoublequoteclose}\isanewline
\ \ \ \ \ \ \isacommand{by}\isamarkupfalse%
\ {\isacharparenleft}{\kern0pt}metis\ cfunc{\isacharunderscore}{\kern0pt}type{\isacharunderscore}{\kern0pt}def\ factors{\isacharunderscore}{\kern0pt}through{\isacharunderscore}{\kern0pt}def\ id{\isacharunderscore}{\kern0pt}right{\isacharunderscore}{\kern0pt}unit{\isadigit{2}}\ id{\isacharunderscore}{\kern0pt}type\ relative{\isacharunderscore}{\kern0pt}member{\isacharunderscore}{\kern0pt}def{\isadigit{2}}\ y{\isadigit{1}}{\isacharunderscore}{\kern0pt}mono{\isacharparenright}{\kern0pt}\isanewline
\ \ \isacommand{qed}\isamarkupfalse%
\isanewline
\isanewline
\isanewline
\ \ \isacommand{have}\isamarkupfalse%
\ {\isachardoublequoteopen}injective{\isacharparenleft}{\kern0pt}m{\isacharparenright}{\kern0pt}{\isachardoublequoteclose}\isanewline
\ \ \ \ \isacommand{unfolding}\isamarkupfalse%
\ injective{\isacharunderscore}{\kern0pt}def\isanewline
\ \ \isacommand{proof}\isamarkupfalse%
{\isacharparenleft}{\kern0pt}clarify{\isacharparenright}{\kern0pt}\isanewline
\ \ \ \ \isacommand{fix}\isamarkupfalse%
\ a\ b\ \isanewline
\ \ \ \ \isacommand{assume}\isamarkupfalse%
\ {\isachardoublequoteopen}a\ {\isasymin}\isactrlsub c\ domain\ m{\isachardoublequoteclose}\ {\isachardoublequoteopen}b\ {\isasymin}\isactrlsub c\ domain\ m{\isachardoublequoteclose}\isanewline
\ \ \ \ \isacommand{then}\isamarkupfalse%
\ \isacommand{have}\isamarkupfalse%
\ a{\isacharunderscore}{\kern0pt}type{\isacharbrackleft}{\kern0pt}type{\isacharunderscore}{\kern0pt}rule{\isacharbrackright}{\kern0pt}{\isacharcolon}{\kern0pt}\ {\isachardoublequoteopen}a\ {\isasymin}\isactrlsub c\ X\ \ {\isasymCoprod}\ Y{\isachardoublequoteclose}\ \isakeyword{and}\ b{\isacharunderscore}{\kern0pt}type{\isacharbrackleft}{\kern0pt}type{\isacharunderscore}{\kern0pt}rule{\isacharbrackright}{\kern0pt}{\isacharcolon}{\kern0pt}\ {\isachardoublequoteopen}b\ {\isasymin}\isactrlsub c\ X\ \ {\isasymCoprod}\ Y{\isachardoublequoteclose}\isanewline
\ \ \ \ \ \ \isacommand{using}\isamarkupfalse%
\ m{\isacharunderscore}{\kern0pt}type\ \isacommand{unfolding}\isamarkupfalse%
\ cfunc{\isacharunderscore}{\kern0pt}type{\isacharunderscore}{\kern0pt}def\ \isacommand{by}\isamarkupfalse%
\ auto\isanewline
\ \ \ \ \isacommand{assume}\isamarkupfalse%
\ eqs{\isacharcolon}{\kern0pt}\ {\isachardoublequoteopen}m\ {\isasymcirc}\isactrlsub c\ a\ {\isacharequal}{\kern0pt}\ m\ {\isasymcirc}\isactrlsub c\ b{\isachardoublequoteclose}\isanewline
\isanewline
\ \ \ \ \ \ \isacommand{have}\isamarkupfalse%
\ m{\isacharunderscore}{\kern0pt}leftproj{\isacharunderscore}{\kern0pt}l{\isacharunderscore}{\kern0pt}equals{\isacharcolon}{\kern0pt}\ {\isachardoublequoteopen}{\isasymAnd}\ l{\isachardot}{\kern0pt}\ l\ \ {\isasymin}\isactrlsub c\ X\ {\isasymLongrightarrow}\ m\ {\isasymcirc}\isactrlsub c\ left{\isacharunderscore}{\kern0pt}coproj\ X\ Y\ {\isasymcirc}\isactrlsub c\ l\ {\isacharequal}{\kern0pt}\ {\isasymlangle}l{\isacharcomma}{\kern0pt}\ y{\isadigit{1}}{\isasymrangle}{\isachardoublequoteclose}\isanewline
\ \ \ \ \ \ \isacommand{proof}\isamarkupfalse%
{\isacharminus}{\kern0pt}\isanewline
\ \ \ \ \ \ \ \ \isacommand{fix}\isamarkupfalse%
\ l\ \isanewline
\ \ \ \ \ \ \ \ \isacommand{assume}\isamarkupfalse%
\ l{\isacharunderscore}{\kern0pt}type{\isacharcolon}{\kern0pt}\ {\isachardoublequoteopen}l\ {\isasymin}\isactrlsub c\ X{\isachardoublequoteclose}\isanewline
\ \ \ \ \ \ \ \ \isacommand{have}\isamarkupfalse%
\ {\isachardoublequoteopen}m\ {\isasymcirc}\isactrlsub c\ left{\isacharunderscore}{\kern0pt}coproj\ X\ Y\ {\isasymcirc}\isactrlsub c\ l\ {\isacharequal}{\kern0pt}\ {\isacharparenleft}{\kern0pt}{\isasymlangle}id{\isacharparenleft}{\kern0pt}X{\isacharparenright}{\kern0pt}{\isacharcomma}{\kern0pt}\ y{\isadigit{1}}\ {\isasymcirc}\isactrlsub c\ {\isasymbeta}\isactrlbsub X\isactrlesub {\isasymrangle}\ {\isasymamalg}\ {\isacharparenleft}{\kern0pt}{\isacharparenleft}{\kern0pt}{\isasymlangle}x{\isadigit{2}}{\isacharcomma}{\kern0pt}\ y{\isadigit{2}}{\isasymrangle}\ {\isasymamalg}\ {\isasymlangle}x{\isadigit{1}}\ {\isasymcirc}\isactrlsub c\ {\isasymbeta}\isactrlbsub Y\ {\isasymsetminus}\ {\isacharparenleft}{\kern0pt}{\isasymone}{\isacharcomma}{\kern0pt}y{\isadigit{1}}{\isacharparenright}{\kern0pt}\isactrlesub {\isacharcomma}{\kern0pt}\ y{\isadigit{1}}\isactrlsup c{\isasymrangle}{\isacharparenright}{\kern0pt}\ {\isasymcirc}\isactrlsub c\ \ try{\isacharunderscore}{\kern0pt}cast\ y{\isadigit{1}}{\isacharparenright}{\kern0pt}{\isacharparenright}{\kern0pt}\ {\isasymcirc}\isactrlsub c\ left{\isacharunderscore}{\kern0pt}coproj\ X\ Y\ {\isasymcirc}\isactrlsub c\ l{\isachardoublequoteclose}\isanewline
\ \ \ \ \ \ \ \ \ \ \isacommand{by}\isamarkupfalse%
\ {\isacharparenleft}{\kern0pt}simp\ add{\isacharcolon}{\kern0pt}\ m{\isacharunderscore}{\kern0pt}def{\isacharparenright}{\kern0pt}\isanewline
\ \ \ \ \ \ \ \ \isacommand{also}\isamarkupfalse%
\ \isacommand{have}\isamarkupfalse%
\ {\isachardoublequoteopen}{\isachardot}{\kern0pt}{\isachardot}{\kern0pt}{\isachardot}{\kern0pt}\ {\isacharequal}{\kern0pt}\ {\isacharparenleft}{\kern0pt}{\isasymlangle}id{\isacharparenleft}{\kern0pt}X{\isacharparenright}{\kern0pt}{\isacharcomma}{\kern0pt}\ y{\isadigit{1}}\ {\isasymcirc}\isactrlsub c\ {\isasymbeta}\isactrlbsub X\isactrlesub {\isasymrangle}\ {\isasymamalg}\ {\isacharparenleft}{\kern0pt}{\isacharparenleft}{\kern0pt}{\isasymlangle}x{\isadigit{2}}{\isacharcomma}{\kern0pt}\ y{\isadigit{2}}{\isasymrangle}\ {\isasymamalg}\ {\isasymlangle}x{\isadigit{1}}\ {\isasymcirc}\isactrlsub c\ {\isasymbeta}\isactrlbsub Y\ {\isasymsetminus}\ {\isacharparenleft}{\kern0pt}{\isasymone}{\isacharcomma}{\kern0pt}y{\isadigit{1}}{\isacharparenright}{\kern0pt}\isactrlesub {\isacharcomma}{\kern0pt}\ y{\isadigit{1}}\isactrlsup c{\isasymrangle}{\isacharparenright}{\kern0pt}\ {\isasymcirc}\isactrlsub c\ \ try{\isacharunderscore}{\kern0pt}cast\ y{\isadigit{1}}{\isacharparenright}{\kern0pt}\ {\isasymcirc}\isactrlsub c\ left{\isacharunderscore}{\kern0pt}coproj\ X\ Y{\isacharparenright}{\kern0pt}\ {\isasymcirc}\isactrlsub c\ l{\isachardoublequoteclose}\isanewline
\ \ \ \ \ \ \ \ \ \ \isacommand{using}\isamarkupfalse%
\ comp{\isacharunderscore}{\kern0pt}associative{\isadigit{2}}\ l{\isacharunderscore}{\kern0pt}type\ \isacommand{by}\isamarkupfalse%
\ {\isacharparenleft}{\kern0pt}typecheck{\isacharunderscore}{\kern0pt}cfuncs{\isacharcomma}{\kern0pt}\ blast{\isacharparenright}{\kern0pt}\isanewline
\ \ \ \ \ \ \ \ \isacommand{also}\isamarkupfalse%
\ \isacommand{have}\isamarkupfalse%
\ {\isachardoublequoteopen}{\isachardot}{\kern0pt}{\isachardot}{\kern0pt}{\isachardot}{\kern0pt}\ {\isacharequal}{\kern0pt}\ {\isasymlangle}id{\isacharparenleft}{\kern0pt}X{\isacharparenright}{\kern0pt}{\isacharcomma}{\kern0pt}\ y{\isadigit{1}}\ {\isasymcirc}\isactrlsub c\ {\isasymbeta}\isactrlbsub X\isactrlesub {\isasymrangle}\ {\isasymcirc}\isactrlsub c\ l{\isachardoublequoteclose}\isanewline
\ \ \ \ \ \ \ \ \ \ \isacommand{by}\isamarkupfalse%
\ {\isacharparenleft}{\kern0pt}typecheck{\isacharunderscore}{\kern0pt}cfuncs{\isacharcomma}{\kern0pt}\ simp\ add{\isacharcolon}{\kern0pt}\ left{\isacharunderscore}{\kern0pt}coproj{\isacharunderscore}{\kern0pt}cfunc{\isacharunderscore}{\kern0pt}coprod{\isacharparenright}{\kern0pt}\isanewline
\ \ \ \ \ \ \ \ \isacommand{also}\isamarkupfalse%
\ \isacommand{have}\isamarkupfalse%
\ {\isachardoublequoteopen}{\isachardot}{\kern0pt}{\isachardot}{\kern0pt}{\isachardot}{\kern0pt}\ {\isacharequal}{\kern0pt}\ {\isasymlangle}id{\isacharparenleft}{\kern0pt}X{\isacharparenright}{\kern0pt}{\isasymcirc}\isactrlsub c\ l\ {\isacharcomma}{\kern0pt}\ {\isacharparenleft}{\kern0pt}y{\isadigit{1}}\ {\isasymcirc}\isactrlsub c\ {\isasymbeta}\isactrlbsub X\isactrlesub {\isacharparenright}{\kern0pt}\ {\isasymcirc}\isactrlsub c\ l{\isasymrangle}{\isachardoublequoteclose}\isanewline
\ \ \ \ \ \ \ \ \ \ \isacommand{using}\isamarkupfalse%
\ l{\isacharunderscore}{\kern0pt}type\ cfunc{\isacharunderscore}{\kern0pt}prod{\isacharunderscore}{\kern0pt}comp\ \isacommand{by}\isamarkupfalse%
\ {\isacharparenleft}{\kern0pt}typecheck{\isacharunderscore}{\kern0pt}cfuncs{\isacharcomma}{\kern0pt}\ auto{\isacharparenright}{\kern0pt}\isanewline
\ \ \ \ \ \ \ \ \isacommand{also}\isamarkupfalse%
\ \isacommand{have}\isamarkupfalse%
\ {\isachardoublequoteopen}{\isachardot}{\kern0pt}{\isachardot}{\kern0pt}{\isachardot}{\kern0pt}\ {\isacharequal}{\kern0pt}\ {\isasymlangle}l\ {\isacharcomma}{\kern0pt}\ y{\isadigit{1}}\ {\isasymcirc}\isactrlsub c\ {\isasymbeta}\isactrlbsub X\isactrlesub \ {\isasymcirc}\isactrlsub c\ l{\isasymrangle}{\isachardoublequoteclose}\isanewline
\ \ \ \ \ \ \ \ \ \ \isacommand{using}\isamarkupfalse%
\ l{\isacharunderscore}{\kern0pt}type\ comp{\isacharunderscore}{\kern0pt}associative{\isadigit{2}}\ id{\isacharunderscore}{\kern0pt}left{\isacharunderscore}{\kern0pt}unit{\isadigit{2}}\ \isacommand{by}\isamarkupfalse%
\ {\isacharparenleft}{\kern0pt}typecheck{\isacharunderscore}{\kern0pt}cfuncs{\isacharcomma}{\kern0pt}\ auto{\isacharparenright}{\kern0pt}\isanewline
\ \ \ \ \ \ \ \ \isacommand{also}\isamarkupfalse%
\ \isacommand{have}\isamarkupfalse%
\ {\isachardoublequoteopen}{\isachardot}{\kern0pt}{\isachardot}{\kern0pt}{\isachardot}{\kern0pt}\ {\isacharequal}{\kern0pt}\ {\isasymlangle}l\ {\isacharcomma}{\kern0pt}\ y{\isadigit{1}}{\isasymrangle}{\isachardoublequoteclose}\isanewline
\ \ \ \ \ \ \ \ \ \ \isacommand{using}\isamarkupfalse%
\ l{\isacharunderscore}{\kern0pt}type\ \isacommand{by}\isamarkupfalse%
\ {\isacharparenleft}{\kern0pt}typecheck{\isacharunderscore}{\kern0pt}cfuncs{\isacharcomma}{\kern0pt}metis\ id{\isacharunderscore}{\kern0pt}right{\isacharunderscore}{\kern0pt}unit{\isadigit{2}}\ id{\isacharunderscore}{\kern0pt}type\ one{\isacharunderscore}{\kern0pt}unique{\isacharunderscore}{\kern0pt}element{\isacharparenright}{\kern0pt}\isanewline
\ \ \ \ \ \ \ \ \isacommand{then}\isamarkupfalse%
\ \isacommand{show}\isamarkupfalse%
\ {\isachardoublequoteopen}m\ {\isasymcirc}\isactrlsub c\ left{\isacharunderscore}{\kern0pt}coproj\ X\ Y\ {\isasymcirc}\isactrlsub c\ l\ {\isacharequal}{\kern0pt}\ {\isasymlangle}l{\isacharcomma}{\kern0pt}y{\isadigit{1}}{\isasymrangle}{\isachardoublequoteclose}\isanewline
\ \ \ \ \ \ \ \ \ \ \isacommand{by}\isamarkupfalse%
\ {\isacharparenleft}{\kern0pt}simp\ add{\isacharcolon}{\kern0pt}\ calculation{\isacharparenright}{\kern0pt}\isanewline
\ \ \ \ \ \ \isacommand{qed}\isamarkupfalse%
\isanewline
\isanewline
\ \ \ \ \ \ \isacommand{have}\isamarkupfalse%
\ m{\isacharunderscore}{\kern0pt}rightproj{\isacharunderscore}{\kern0pt}y{\isadigit{1}}{\isacharunderscore}{\kern0pt}equals{\isacharcolon}{\kern0pt}\ {\isachardoublequoteopen}m\ {\isasymcirc}\isactrlsub c\ right{\isacharunderscore}{\kern0pt}coproj\ X\ Y\ {\isasymcirc}\isactrlsub c\ y{\isadigit{1}}\ {\isacharequal}{\kern0pt}\ {\isasymlangle}x{\isadigit{2}}{\isacharcomma}{\kern0pt}\ y{\isadigit{2}}{\isasymrangle}{\isachardoublequoteclose}\isanewline
\ \ \ \ \ \ \isacommand{proof}\isamarkupfalse%
\ {\isacharminus}{\kern0pt}\ \isanewline
\ \ \ \ \ \ \ \ \isacommand{have}\isamarkupfalse%
\ {\isachardoublequoteopen}m\ {\isasymcirc}\isactrlsub c\ right{\isacharunderscore}{\kern0pt}coproj\ X\ Y\ {\isasymcirc}\isactrlsub c\ y{\isadigit{1}}\ {\isacharequal}{\kern0pt}\ {\isacharparenleft}{\kern0pt}m\ {\isasymcirc}\isactrlsub c\ right{\isacharunderscore}{\kern0pt}coproj\ X\ Y{\isacharparenright}{\kern0pt}\ {\isasymcirc}\isactrlsub c\ y{\isadigit{1}}{\isachardoublequoteclose}\isanewline
\ \ \ \ \ \ \ \ \ \ \isacommand{using}\isamarkupfalse%
\ \ comp{\isacharunderscore}{\kern0pt}associative{\isadigit{2}}\ m{\isacharunderscore}{\kern0pt}type\ \isacommand{by}\isamarkupfalse%
\ {\isacharparenleft}{\kern0pt}typecheck{\isacharunderscore}{\kern0pt}cfuncs{\isacharcomma}{\kern0pt}\ auto{\isacharparenright}{\kern0pt}\isanewline
\ \ \ \ \ \ \ \ \isacommand{also}\isamarkupfalse%
\ \isacommand{have}\isamarkupfalse%
\ {\isachardoublequoteopen}{\isachardot}{\kern0pt}{\isachardot}{\kern0pt}{\isachardot}{\kern0pt}\ {\isacharequal}{\kern0pt}\ {\isacharparenleft}{\kern0pt}{\isacharparenleft}{\kern0pt}{\isasymlangle}x{\isadigit{2}}{\isacharcomma}{\kern0pt}\ y{\isadigit{2}}{\isasymrangle}\ {\isasymamalg}\ {\isasymlangle}x{\isadigit{1}}\ {\isasymcirc}\isactrlsub c\ {\isasymbeta}\isactrlbsub Y\ {\isasymsetminus}\ {\isacharparenleft}{\kern0pt}{\isasymone}{\isacharcomma}{\kern0pt}y{\isadigit{1}}{\isacharparenright}{\kern0pt}\isactrlesub {\isacharcomma}{\kern0pt}\ y{\isadigit{1}}\isactrlsup c{\isasymrangle}{\isacharparenright}{\kern0pt}\ {\isasymcirc}\isactrlsub c\ \ try{\isacharunderscore}{\kern0pt}cast\ y{\isadigit{1}}{\isacharparenright}{\kern0pt}\ {\isasymcirc}\isactrlsub c\ y{\isadigit{1}}{\isachardoublequoteclose}\isanewline
\ \ \ \ \ \ \ \ \ \ \isacommand{using}\isamarkupfalse%
\ m{\isacharunderscore}{\kern0pt}def\ right{\isacharunderscore}{\kern0pt}coproj{\isacharunderscore}{\kern0pt}cfunc{\isacharunderscore}{\kern0pt}coprod\ type{\isadigit{1}}\ \isacommand{by}\isamarkupfalse%
\ {\isacharparenleft}{\kern0pt}typecheck{\isacharunderscore}{\kern0pt}cfuncs{\isacharcomma}{\kern0pt}\ auto{\isacharparenright}{\kern0pt}\isanewline
\ \ \ \ \ \ \ \ \isacommand{also}\isamarkupfalse%
\ \isacommand{have}\isamarkupfalse%
\ {\isachardoublequoteopen}{\isachardot}{\kern0pt}{\isachardot}{\kern0pt}{\isachardot}{\kern0pt}\ {\isacharequal}{\kern0pt}\ {\isacharparenleft}{\kern0pt}{\isasymlangle}x{\isadigit{2}}{\isacharcomma}{\kern0pt}\ y{\isadigit{2}}{\isasymrangle}\ {\isasymamalg}\ {\isasymlangle}x{\isadigit{1}}\ {\isasymcirc}\isactrlsub c\ {\isasymbeta}\isactrlbsub Y\ {\isasymsetminus}\ {\isacharparenleft}{\kern0pt}{\isasymone}{\isacharcomma}{\kern0pt}y{\isadigit{1}}{\isacharparenright}{\kern0pt}\isactrlesub {\isacharcomma}{\kern0pt}\ y{\isadigit{1}}\isactrlsup c{\isasymrangle}{\isacharparenright}{\kern0pt}\ {\isasymcirc}\isactrlsub c\ \ try{\isacharunderscore}{\kern0pt}cast\ y{\isadigit{1}}\ {\isasymcirc}\isactrlsub c\ y{\isadigit{1}}{\isachardoublequoteclose}\isanewline
\ \ \ \ \ \ \ \ \ \ \isacommand{using}\isamarkupfalse%
\ \ comp{\isacharunderscore}{\kern0pt}associative{\isadigit{2}}\ \isacommand{by}\isamarkupfalse%
\ {\isacharparenleft}{\kern0pt}typecheck{\isacharunderscore}{\kern0pt}cfuncs{\isacharcomma}{\kern0pt}\ auto{\isacharparenright}{\kern0pt}\isanewline
\ \ \ \ \ \ \ \ \isacommand{also}\isamarkupfalse%
\ \isacommand{have}\isamarkupfalse%
\ {\isachardoublequoteopen}{\isachardot}{\kern0pt}{\isachardot}{\kern0pt}{\isachardot}{\kern0pt}\ {\isacharequal}{\kern0pt}\ {\isacharparenleft}{\kern0pt}{\isasymlangle}x{\isadigit{2}}{\isacharcomma}{\kern0pt}\ y{\isadigit{2}}{\isasymrangle}\ {\isasymamalg}\ {\isasymlangle}x{\isadigit{1}}\ {\isasymcirc}\isactrlsub c\ {\isasymbeta}\isactrlbsub Y\ {\isasymsetminus}\ {\isacharparenleft}{\kern0pt}{\isasymone}{\isacharcomma}{\kern0pt}y{\isadigit{1}}{\isacharparenright}{\kern0pt}\isactrlesub {\isacharcomma}{\kern0pt}\ y{\isadigit{1}}\isactrlsup c{\isasymrangle}{\isacharparenright}{\kern0pt}\ {\isasymcirc}\isactrlsub c\ left{\isacharunderscore}{\kern0pt}coproj\ {\isasymone}\ {\isacharparenleft}{\kern0pt}Y\ {\isasymsetminus}\ {\isacharparenleft}{\kern0pt}{\isasymone}{\isacharcomma}{\kern0pt}y{\isadigit{1}}{\isacharparenright}{\kern0pt}{\isacharparenright}{\kern0pt}{\isachardoublequoteclose}\isanewline
\ \ \ \ \ \ \ \ \ \ \isacommand{using}\isamarkupfalse%
\ \ try{\isacharunderscore}{\kern0pt}cast{\isacharunderscore}{\kern0pt}m{\isacharunderscore}{\kern0pt}m\ y{\isadigit{1}}{\isacharunderscore}{\kern0pt}mono\ y{\isadigit{1}}y{\isadigit{2}}{\isacharunderscore}{\kern0pt}def{\isacharparenleft}{\kern0pt}{\isadigit{1}}{\isacharparenright}{\kern0pt}\ \isacommand{by}\isamarkupfalse%
\ auto\isanewline
\ \ \ \ \ \ \ \ \isacommand{also}\isamarkupfalse%
\ \isacommand{have}\isamarkupfalse%
\ {\isachardoublequoteopen}{\isachardot}{\kern0pt}{\isachardot}{\kern0pt}{\isachardot}{\kern0pt}\ {\isacharequal}{\kern0pt}\ \ {\isasymlangle}x{\isadigit{2}}{\isacharcomma}{\kern0pt}\ y{\isadigit{2}}{\isasymrangle}{\isachardoublequoteclose}\isanewline
\ \ \ \ \ \ \ \ \ \ \isacommand{using}\isamarkupfalse%
\ left{\isacharunderscore}{\kern0pt}coproj{\isacharunderscore}{\kern0pt}cfunc{\isacharunderscore}{\kern0pt}coprod\ type{\isadigit{4}}\ type{\isadigit{5}}\ \isacommand{by}\isamarkupfalse%
\ blast\isanewline
\ \ \ \ \ \ \ \ \isacommand{then}\isamarkupfalse%
\ \isacommand{show}\isamarkupfalse%
\ {\isacharquery}{\kern0pt}thesis\ \isacommand{using}\isamarkupfalse%
\ calculation\ \isacommand{by}\isamarkupfalse%
\ auto\isanewline
\ \ \ \ \ \ \isacommand{qed}\isamarkupfalse%
\isanewline
\isanewline
\ \ \ \ \ \ \isacommand{have}\isamarkupfalse%
\ m{\isacharunderscore}{\kern0pt}rightproj{\isacharunderscore}{\kern0pt}not{\isacharunderscore}{\kern0pt}y{\isadigit{1}}{\isacharunderscore}{\kern0pt}equals{\isacharcolon}{\kern0pt}\ {\isachardoublequoteopen}{\isasymAnd}\ r{\isachardot}{\kern0pt}\ r\ \ {\isasymin}\isactrlsub c\ Y\ {\isasymand}\ r\ {\isasymnoteq}\ y{\isadigit{1}}\ {\isasymLongrightarrow}\isanewline
\ \ \ \ \ \ \ \ \ \ \ \ {\isasymexists}k{\isachardot}{\kern0pt}\ k\ {\isasymin}\isactrlsub c\ Y\ {\isasymsetminus}\ {\isacharparenleft}{\kern0pt}{\isasymone}{\isacharcomma}{\kern0pt}y{\isadigit{1}}{\isacharparenright}{\kern0pt}\ {\isasymand}\ try{\isacharunderscore}{\kern0pt}cast\ y{\isadigit{1}}\ {\isasymcirc}\isactrlsub c\ r\ {\isacharequal}{\kern0pt}\ right{\isacharunderscore}{\kern0pt}coproj\ {\isasymone}\ {\isacharparenleft}{\kern0pt}Y\ {\isasymsetminus}\ {\isacharparenleft}{\kern0pt}{\isasymone}{\isacharcomma}{\kern0pt}y{\isadigit{1}}{\isacharparenright}{\kern0pt}{\isacharparenright}{\kern0pt}\ {\isasymcirc}\isactrlsub c\ k\ {\isasymand}\ \isanewline
\ \ \ \ \ \ \ \ \ \ \ \ m\ {\isasymcirc}\isactrlsub c\ right{\isacharunderscore}{\kern0pt}coproj\ X\ Y\ {\isasymcirc}\isactrlsub c\ r\ {\isacharequal}{\kern0pt}\ {\isasymlangle}x{\isadigit{1}}{\isacharcomma}{\kern0pt}\ y{\isadigit{1}}\isactrlsup c\ {\isasymcirc}\isactrlsub c\ k{\isasymrangle}{\isachardoublequoteclose}\isanewline
\ \ \ \ \ \ \isacommand{proof}\isamarkupfalse%
\ clarify\isanewline
\ \ \ \ \ \ \ \ \isacommand{fix}\isamarkupfalse%
\ r\ \isanewline
\ \ \ \ \ \ \ \ \isacommand{assume}\isamarkupfalse%
\ r{\isacharunderscore}{\kern0pt}type{\isacharcolon}{\kern0pt}\ {\isachardoublequoteopen}r\ {\isasymin}\isactrlsub c\ Y{\isachardoublequoteclose}\isanewline
\ \ \ \ \ \ \ \ \isacommand{assume}\isamarkupfalse%
\ r{\isacharunderscore}{\kern0pt}not{\isacharunderscore}{\kern0pt}y{\isadigit{1}}{\isacharcolon}{\kern0pt}\ {\isachardoublequoteopen}r\ {\isasymnoteq}\ y{\isadigit{1}}{\isachardoublequoteclose}\isanewline
\ \ \ \ \ \ \ \ \isacommand{then}\isamarkupfalse%
\ \isacommand{obtain}\isamarkupfalse%
\ k\ \isakeyword{where}\ k{\isacharunderscore}{\kern0pt}def{\isacharcolon}{\kern0pt}\ {\isachardoublequoteopen}k\ {\isasymin}\isactrlsub c\ Y\ {\isasymsetminus}\ {\isacharparenleft}{\kern0pt}{\isasymone}{\isacharcomma}{\kern0pt}y{\isadigit{1}}{\isacharparenright}{\kern0pt}\ {\isasymand}\ try{\isacharunderscore}{\kern0pt}cast\ y{\isadigit{1}}\ {\isasymcirc}\isactrlsub c\ r\ {\isacharequal}{\kern0pt}\ right{\isacharunderscore}{\kern0pt}coproj\ {\isasymone}\ {\isacharparenleft}{\kern0pt}Y\ {\isasymsetminus}\ {\isacharparenleft}{\kern0pt}{\isasymone}{\isacharcomma}{\kern0pt}y{\isadigit{1}}{\isacharparenright}{\kern0pt}{\isacharparenright}{\kern0pt}\ {\isasymcirc}\isactrlsub c\ k{\isachardoublequoteclose}\isanewline
\ \ \ \ \ \ \ \ \ \ \isacommand{using}\isamarkupfalse%
\ r{\isacharunderscore}{\kern0pt}type\ relative\ try{\isacharunderscore}{\kern0pt}cast{\isacharunderscore}{\kern0pt}not{\isacharunderscore}{\kern0pt}in{\isacharunderscore}{\kern0pt}X\ y{\isadigit{1}}{\isacharunderscore}{\kern0pt}mono\ y{\isadigit{1}}y{\isadigit{2}}{\isacharunderscore}{\kern0pt}def{\isacharparenleft}{\kern0pt}{\isadigit{1}}{\isacharparenright}{\kern0pt}\ \isacommand{by}\isamarkupfalse%
\ blast\isanewline
\ \ \ \ \ \ \ \ \isacommand{have}\isamarkupfalse%
\ m{\isacharunderscore}{\kern0pt}rightproj{\isacharunderscore}{\kern0pt}l{\isacharunderscore}{\kern0pt}equals{\isacharcolon}{\kern0pt}\ {\isachardoublequoteopen}m\ {\isasymcirc}\isactrlsub c\ right{\isacharunderscore}{\kern0pt}coproj\ X\ Y\ {\isasymcirc}\isactrlsub c\ r\ {\isacharequal}{\kern0pt}\ {\isasymlangle}x{\isadigit{1}}{\isacharcomma}{\kern0pt}\ y{\isadigit{1}}\isactrlsup c\ {\isasymcirc}\isactrlsub c\ k{\isasymrangle}{\isachardoublequoteclose}\isanewline
\ \ \ \ \ \ \ \ \ \ \ \ \ \isanewline
\ \ \ \ \ \ \ \ \isacommand{proof}\isamarkupfalse%
\ {\isacharminus}{\kern0pt}\isanewline
\ \ \ \ \ \ \ \ \ \ \isacommand{have}\isamarkupfalse%
\ {\isachardoublequoteopen}m\ {\isasymcirc}\isactrlsub c\ right{\isacharunderscore}{\kern0pt}coproj\ X\ Y\ {\isasymcirc}\isactrlsub c\ r\ {\isacharequal}{\kern0pt}\ {\isacharparenleft}{\kern0pt}m\ {\isasymcirc}\isactrlsub c\ right{\isacharunderscore}{\kern0pt}coproj\ X\ Y{\isacharparenright}{\kern0pt}\ {\isasymcirc}\isactrlsub c\ r{\isachardoublequoteclose}\isanewline
\ \ \ \ \ \ \ \ \ \ \ \ \isacommand{using}\isamarkupfalse%
\ r{\isacharunderscore}{\kern0pt}type\ comp{\isacharunderscore}{\kern0pt}associative{\isadigit{2}}\ m{\isacharunderscore}{\kern0pt}type\ \isacommand{by}\isamarkupfalse%
\ {\isacharparenleft}{\kern0pt}typecheck{\isacharunderscore}{\kern0pt}cfuncs{\isacharcomma}{\kern0pt}\ auto{\isacharparenright}{\kern0pt}\isanewline
\ \ \ \ \ \ \ \ \ \ \isacommand{also}\isamarkupfalse%
\ \isacommand{have}\isamarkupfalse%
\ {\isachardoublequoteopen}{\isachardot}{\kern0pt}{\isachardot}{\kern0pt}{\isachardot}{\kern0pt}\ {\isacharequal}{\kern0pt}\ {\isacharparenleft}{\kern0pt}{\isacharparenleft}{\kern0pt}{\isasymlangle}x{\isadigit{2}}{\isacharcomma}{\kern0pt}\ y{\isadigit{2}}{\isasymrangle}\ {\isasymamalg}\ {\isasymlangle}x{\isadigit{1}}\ {\isasymcirc}\isactrlsub c\ {\isasymbeta}\isactrlbsub Y\ {\isasymsetminus}\ {\isacharparenleft}{\kern0pt}{\isasymone}{\isacharcomma}{\kern0pt}y{\isadigit{1}}{\isacharparenright}{\kern0pt}\isactrlesub {\isacharcomma}{\kern0pt}\ y{\isadigit{1}}\isactrlsup c{\isasymrangle}{\isacharparenright}{\kern0pt}\ {\isasymcirc}\isactrlsub c\ \ try{\isacharunderscore}{\kern0pt}cast\ y{\isadigit{1}}{\isacharparenright}{\kern0pt}\ {\isasymcirc}\isactrlsub c\ r{\isachardoublequoteclose}\isanewline
\ \ \ \ \ \ \ \ \ \ \ \ \isacommand{using}\isamarkupfalse%
\ m{\isacharunderscore}{\kern0pt}def\ right{\isacharunderscore}{\kern0pt}coproj{\isacharunderscore}{\kern0pt}cfunc{\isacharunderscore}{\kern0pt}coprod\ type{\isadigit{1}}\ \isacommand{by}\isamarkupfalse%
\ {\isacharparenleft}{\kern0pt}typecheck{\isacharunderscore}{\kern0pt}cfuncs{\isacharcomma}{\kern0pt}\ auto{\isacharparenright}{\kern0pt}\isanewline
\ \ \ \ \ \ \ \ \ \ \isacommand{also}\isamarkupfalse%
\ \isacommand{have}\isamarkupfalse%
\ {\isachardoublequoteopen}{\isachardot}{\kern0pt}{\isachardot}{\kern0pt}{\isachardot}{\kern0pt}\ {\isacharequal}{\kern0pt}\ {\isacharparenleft}{\kern0pt}{\isasymlangle}x{\isadigit{2}}{\isacharcomma}{\kern0pt}\ y{\isadigit{2}}{\isasymrangle}\ {\isasymamalg}\ {\isasymlangle}x{\isadigit{1}}\ {\isasymcirc}\isactrlsub c\ {\isasymbeta}\isactrlbsub Y\ {\isasymsetminus}\ {\isacharparenleft}{\kern0pt}{\isasymone}{\isacharcomma}{\kern0pt}y{\isadigit{1}}{\isacharparenright}{\kern0pt}\isactrlesub {\isacharcomma}{\kern0pt}\ y{\isadigit{1}}\isactrlsup c{\isasymrangle}{\isacharparenright}{\kern0pt}\ {\isasymcirc}\isactrlsub c\ \ {\isacharparenleft}{\kern0pt}try{\isacharunderscore}{\kern0pt}cast\ y{\isadigit{1}}\ {\isasymcirc}\isactrlsub c\ r{\isacharparenright}{\kern0pt}{\isachardoublequoteclose}\isanewline
\ \ \ \ \ \ \ \ \ \ \ \ \isacommand{using}\isamarkupfalse%
\ r{\isacharunderscore}{\kern0pt}type\ comp{\isacharunderscore}{\kern0pt}associative{\isadigit{2}}\ \isacommand{by}\isamarkupfalse%
\ {\isacharparenleft}{\kern0pt}typecheck{\isacharunderscore}{\kern0pt}cfuncs{\isacharcomma}{\kern0pt}\ auto{\isacharparenright}{\kern0pt}\isanewline
\ \ \ \ \ \ \ \ \ \ \isacommand{also}\isamarkupfalse%
\ \isacommand{have}\isamarkupfalse%
\ {\isachardoublequoteopen}{\isachardot}{\kern0pt}{\isachardot}{\kern0pt}{\isachardot}{\kern0pt}\ {\isacharequal}{\kern0pt}\ {\isacharparenleft}{\kern0pt}{\isasymlangle}x{\isadigit{2}}{\isacharcomma}{\kern0pt}\ y{\isadigit{2}}{\isasymrangle}\ {\isasymamalg}\ {\isasymlangle}x{\isadigit{1}}\ {\isasymcirc}\isactrlsub c\ {\isasymbeta}\isactrlbsub Y\ {\isasymsetminus}\ {\isacharparenleft}{\kern0pt}{\isasymone}{\isacharcomma}{\kern0pt}y{\isadigit{1}}{\isacharparenright}{\kern0pt}\isactrlesub {\isacharcomma}{\kern0pt}\ y{\isadigit{1}}\isactrlsup c{\isasymrangle}{\isacharparenright}{\kern0pt}\ {\isasymcirc}\isactrlsub c\ {\isacharparenleft}{\kern0pt}right{\isacharunderscore}{\kern0pt}coproj\ {\isasymone}\ {\isacharparenleft}{\kern0pt}Y\ {\isasymsetminus}\ {\isacharparenleft}{\kern0pt}{\isasymone}{\isacharcomma}{\kern0pt}y{\isadigit{1}}{\isacharparenright}{\kern0pt}{\isacharparenright}{\kern0pt}\ {\isasymcirc}\isactrlsub c\ k{\isacharparenright}{\kern0pt}{\isachardoublequoteclose}\isanewline
\ \ \ \ \ \ \ \ \ \ \ \ \isacommand{using}\isamarkupfalse%
\ k{\isacharunderscore}{\kern0pt}def\ \isacommand{by}\isamarkupfalse%
\ auto\isanewline
\ \ \ \ \ \ \ \ \ \ \isacommand{also}\isamarkupfalse%
\ \isacommand{have}\isamarkupfalse%
\ {\isachardoublequoteopen}{\isachardot}{\kern0pt}{\isachardot}{\kern0pt}{\isachardot}{\kern0pt}\ {\isacharequal}{\kern0pt}\ {\isacharparenleft}{\kern0pt}{\isacharparenleft}{\kern0pt}{\isasymlangle}x{\isadigit{2}}{\isacharcomma}{\kern0pt}\ y{\isadigit{2}}{\isasymrangle}\ {\isasymamalg}\ {\isasymlangle}x{\isadigit{1}}\ {\isasymcirc}\isactrlsub c\ {\isasymbeta}\isactrlbsub Y\ {\isasymsetminus}\ {\isacharparenleft}{\kern0pt}{\isasymone}{\isacharcomma}{\kern0pt}y{\isadigit{1}}{\isacharparenright}{\kern0pt}\isactrlesub {\isacharcomma}{\kern0pt}\ y{\isadigit{1}}\isactrlsup c{\isasymrangle}{\isacharparenright}{\kern0pt}\ {\isasymcirc}\isactrlsub c\ right{\isacharunderscore}{\kern0pt}coproj\ {\isasymone}\ {\isacharparenleft}{\kern0pt}Y\ {\isasymsetminus}\ {\isacharparenleft}{\kern0pt}{\isasymone}{\isacharcomma}{\kern0pt}y{\isadigit{1}}{\isacharparenright}{\kern0pt}{\isacharparenright}{\kern0pt}{\isacharparenright}{\kern0pt}\ {\isasymcirc}\isactrlsub c\ k{\isachardoublequoteclose}\isanewline
\ \ \ \ \ \ \ \ \ \ \ \ \isacommand{using}\isamarkupfalse%
\ comp{\isacharunderscore}{\kern0pt}associative{\isadigit{2}}\ k{\isacharunderscore}{\kern0pt}def\ \isacommand{by}\isamarkupfalse%
\ {\isacharparenleft}{\kern0pt}typecheck{\isacharunderscore}{\kern0pt}cfuncs{\isacharcomma}{\kern0pt}\ blast{\isacharparenright}{\kern0pt}\isanewline
\ \ \ \ \ \ \ \ \ \ \isacommand{also}\isamarkupfalse%
\ \isacommand{have}\isamarkupfalse%
\ {\isachardoublequoteopen}{\isachardot}{\kern0pt}{\isachardot}{\kern0pt}{\isachardot}{\kern0pt}\ {\isacharequal}{\kern0pt}\ \ {\isasymlangle}x{\isadigit{1}}\ {\isasymcirc}\isactrlsub c\ {\isasymbeta}\isactrlbsub Y\ {\isasymsetminus}\ {\isacharparenleft}{\kern0pt}{\isasymone}{\isacharcomma}{\kern0pt}y{\isadigit{1}}{\isacharparenright}{\kern0pt}\isactrlesub {\isacharcomma}{\kern0pt}\ y{\isadigit{1}}\isactrlsup c{\isasymrangle}\ {\isasymcirc}\isactrlsub c\ k{\isachardoublequoteclose}\isanewline
\ \ \ \ \ \ \ \ \ \ \ \ \isacommand{using}\isamarkupfalse%
\ right{\isacharunderscore}{\kern0pt}coproj{\isacharunderscore}{\kern0pt}cfunc{\isacharunderscore}{\kern0pt}coprod\ type{\isadigit{4}}\ type{\isadigit{5}}\ \isacommand{by}\isamarkupfalse%
\ auto\isanewline
\ \ \ \ \ \ \ \ \ \ \isacommand{also}\isamarkupfalse%
\ \isacommand{have}\isamarkupfalse%
\ {\isachardoublequoteopen}{\isachardot}{\kern0pt}{\isachardot}{\kern0pt}{\isachardot}{\kern0pt}\ {\isacharequal}{\kern0pt}\ \ {\isasymlangle}x{\isadigit{1}}\ {\isasymcirc}\isactrlsub c\ {\isasymbeta}\isactrlbsub Y\ {\isasymsetminus}\ {\isacharparenleft}{\kern0pt}{\isasymone}{\isacharcomma}{\kern0pt}y{\isadigit{1}}{\isacharparenright}{\kern0pt}\isactrlesub \ {\isasymcirc}\isactrlsub c\ k{\isacharcomma}{\kern0pt}\ y{\isadigit{1}}\isactrlsup c\ {\isasymcirc}\isactrlsub c\ k\ {\isasymrangle}{\isachardoublequoteclose}\isanewline
\ \ \ \ \ \ \ \ \ \ \ \ \isacommand{using}\isamarkupfalse%
\ cfunc{\isacharunderscore}{\kern0pt}prod{\isacharunderscore}{\kern0pt}comp\ comp{\isacharunderscore}{\kern0pt}associative{\isadigit{2}}\ k{\isacharunderscore}{\kern0pt}def\ \isacommand{by}\isamarkupfalse%
\ {\isacharparenleft}{\kern0pt}typecheck{\isacharunderscore}{\kern0pt}cfuncs{\isacharcomma}{\kern0pt}\ auto{\isacharparenright}{\kern0pt}\isanewline
\ \ \ \ \ \ \ \ \ \ \isacommand{also}\isamarkupfalse%
\ \isacommand{have}\isamarkupfalse%
\ {\isachardoublequoteopen}{\isachardot}{\kern0pt}{\isachardot}{\kern0pt}{\isachardot}{\kern0pt}\ {\isacharequal}{\kern0pt}\ \ {\isasymlangle}x{\isadigit{1}}{\isacharcomma}{\kern0pt}\ y{\isadigit{1}}\isactrlsup c\ {\isasymcirc}\isactrlsub c\ k{\isasymrangle}{\isachardoublequoteclose}\isanewline
\ \ \ \ \ \ \ \ \ \ \ \ \isacommand{by}\isamarkupfalse%
\ {\isacharparenleft}{\kern0pt}metis\ id{\isacharunderscore}{\kern0pt}right{\isacharunderscore}{\kern0pt}unit{\isadigit{2}}\ id{\isacharunderscore}{\kern0pt}type\ k{\isacharunderscore}{\kern0pt}def\ one{\isacharunderscore}{\kern0pt}unique{\isacharunderscore}{\kern0pt}element\ terminal{\isacharunderscore}{\kern0pt}func{\isacharunderscore}{\kern0pt}comp\ terminal{\isacharunderscore}{\kern0pt}func{\isacharunderscore}{\kern0pt}type\ x{\isadigit{1}}x{\isadigit{2}}{\isacharunderscore}{\kern0pt}def{\isacharparenleft}{\kern0pt}{\isadigit{1}}{\isacharparenright}{\kern0pt}{\isacharparenright}{\kern0pt}\isanewline
\ \ \ \ \ \ \ \ \ \ \isacommand{then}\isamarkupfalse%
\ \isacommand{show}\isamarkupfalse%
\ {\isacharquery}{\kern0pt}thesis\ \isacommand{using}\isamarkupfalse%
\ calculation\ \isacommand{by}\isamarkupfalse%
\ auto\isanewline
\ \ \ \ \ \ \ \ \isacommand{qed}\isamarkupfalse%
\isanewline
\ \ \ \ \ \ \ \ \isacommand{then}\isamarkupfalse%
\ \isacommand{show}\isamarkupfalse%
\ {\isachardoublequoteopen}{\isasymexists}k{\isachardot}{\kern0pt}\ k\ {\isasymin}\isactrlsub c\ Y\ {\isasymsetminus}\ {\isacharparenleft}{\kern0pt}{\isasymone}{\isacharcomma}{\kern0pt}\ y{\isadigit{1}}{\isacharparenright}{\kern0pt}\ {\isasymand}\isanewline
\ \ \ \ \ \ \ \ \ \ try{\isacharunderscore}{\kern0pt}cast\ y{\isadigit{1}}\ {\isasymcirc}\isactrlsub c\ r\ {\isacharequal}{\kern0pt}\ right{\isacharunderscore}{\kern0pt}coproj\ {\isasymone}\ {\isacharparenleft}{\kern0pt}Y\ {\isasymsetminus}\ {\isacharparenleft}{\kern0pt}{\isasymone}{\isacharcomma}{\kern0pt}\ y{\isadigit{1}}{\isacharparenright}{\kern0pt}{\isacharparenright}{\kern0pt}\ {\isasymcirc}\isactrlsub c\ k\ {\isasymand}\ \isanewline
\ \ \ \ \ \ \ \ \ \ m\ {\isasymcirc}\isactrlsub c\ right{\isacharunderscore}{\kern0pt}coproj\ X\ Y\ {\isasymcirc}\isactrlsub c\ r\ {\isacharequal}{\kern0pt}\ {\isasymlangle}x{\isadigit{1}}{\isacharcomma}{\kern0pt}y{\isadigit{1}}\isactrlsup c\ {\isasymcirc}\isactrlsub c\ k{\isasymrangle}{\isachardoublequoteclose}\isanewline
\ \ \ \ \ \ \ \ \ \ \ \ \ \ \isacommand{using}\isamarkupfalse%
\ k{\isacharunderscore}{\kern0pt}def\ \isacommand{by}\isamarkupfalse%
\ blast\isanewline
\ \ \ \ \isacommand{qed}\isamarkupfalse%
\isanewline
\ \isanewline
\ \ \ \ \isacommand{show}\isamarkupfalse%
\ {\isachardoublequoteopen}a\ {\isacharequal}{\kern0pt}\ b{\isachardoublequoteclose}\isanewline
\ \ \ \ \isacommand{proof}\isamarkupfalse%
{\isacharparenleft}{\kern0pt}cases\ {\isachardoublequoteopen}{\isasymexists}x{\isachardot}{\kern0pt}\ a\ {\isacharequal}{\kern0pt}\ left{\isacharunderscore}{\kern0pt}coproj\ X\ Y\ {\isasymcirc}\isactrlsub c\ x\ \ {\isasymand}\ x\ {\isasymin}\isactrlsub c\ X{\isachardoublequoteclose}{\isacharparenright}{\kern0pt}\isanewline
\ \ \ \ \ \ \isacommand{assume}\isamarkupfalse%
\ {\isachardoublequoteopen}{\isasymexists}x{\isachardot}{\kern0pt}\ a\ {\isacharequal}{\kern0pt}\ left{\isacharunderscore}{\kern0pt}coproj\ X\ Y\ {\isasymcirc}\isactrlsub c\ x\ \ {\isasymand}\ x\ {\isasymin}\isactrlsub c\ X{\isachardoublequoteclose}\isanewline
\ \ \ \ \ \ \isacommand{then}\isamarkupfalse%
\ \isacommand{obtain}\isamarkupfalse%
\ x\ \isakeyword{where}\ x{\isacharunderscore}{\kern0pt}def{\isacharcolon}{\kern0pt}\ {\isachardoublequoteopen}a\ {\isacharequal}{\kern0pt}\ left{\isacharunderscore}{\kern0pt}coproj\ X\ Y\ {\isasymcirc}\isactrlsub c\ x\ \ {\isasymand}\ x\ {\isasymin}\isactrlsub c\ X{\isachardoublequoteclose}\isanewline
\ \ \ \ \ \ \ \ \isacommand{by}\isamarkupfalse%
\ auto\isanewline
\ \ \ \ \ \ \isacommand{then}\isamarkupfalse%
\ \isacommand{have}\isamarkupfalse%
\ m{\isacharunderscore}{\kern0pt}proj{\isacharunderscore}{\kern0pt}a{\isacharcolon}{\kern0pt}\ {\isachardoublequoteopen}m\ {\isasymcirc}\isactrlsub c\ left{\isacharunderscore}{\kern0pt}coproj\ X\ Y\ {\isasymcirc}\isactrlsub c\ x\ {\isacharequal}{\kern0pt}\ {\isasymlangle}x{\isacharcomma}{\kern0pt}\ y{\isadigit{1}}{\isasymrangle}{\isachardoublequoteclose}\isanewline
\ \ \ \ \ \ \ \ \isacommand{using}\isamarkupfalse%
\ m{\isacharunderscore}{\kern0pt}leftproj{\isacharunderscore}{\kern0pt}l{\isacharunderscore}{\kern0pt}equals\ \isacommand{by}\isamarkupfalse%
\ {\isacharparenleft}{\kern0pt}simp\ add{\isacharcolon}{\kern0pt}\ x{\isacharunderscore}{\kern0pt}def{\isacharparenright}{\kern0pt}\isanewline
\ \ \ \ \ \ \isacommand{show}\isamarkupfalse%
\ {\isachardoublequoteopen}a\ {\isacharequal}{\kern0pt}\ b{\isachardoublequoteclose}\isanewline
\ \ \ \ \ \ \isacommand{proof}\isamarkupfalse%
{\isacharparenleft}{\kern0pt}cases\ {\isachardoublequoteopen}{\isasymexists}c{\isachardot}{\kern0pt}\ b\ {\isacharequal}{\kern0pt}\ left{\isacharunderscore}{\kern0pt}coproj\ X\ Y\ {\isasymcirc}\isactrlsub c\ c\ \ {\isasymand}\ c\ {\isasymin}\isactrlsub c\ X{\isachardoublequoteclose}{\isacharparenright}{\kern0pt}\isanewline
\ \ \ \ \ \ \ \ \isacommand{assume}\isamarkupfalse%
\ {\isachardoublequoteopen}{\isasymexists}c{\isachardot}{\kern0pt}\ b\ {\isacharequal}{\kern0pt}\ left{\isacharunderscore}{\kern0pt}coproj\ X\ Y\ {\isasymcirc}\isactrlsub c\ c\ {\isasymand}\ c\ {\isasymin}\isactrlsub c\ X{\isachardoublequoteclose}\isanewline
\ \ \ \ \ \ \ \ \isacommand{then}\isamarkupfalse%
\ \isacommand{obtain}\isamarkupfalse%
\ c\ \isakeyword{where}\ c{\isacharunderscore}{\kern0pt}def{\isacharcolon}{\kern0pt}\ {\isachardoublequoteopen}b\ {\isacharequal}{\kern0pt}\ left{\isacharunderscore}{\kern0pt}coproj\ X\ Y\ {\isasymcirc}\isactrlsub c\ c\ \ {\isasymand}\ c\ {\isasymin}\isactrlsub c\ X{\isachardoublequoteclose}\isanewline
\ \ \ \ \ \ \ \ \ \ \isacommand{by}\isamarkupfalse%
\ auto\isanewline
\ \ \ \ \ \ \ \ \isacommand{then}\isamarkupfalse%
\ \isacommand{have}\isamarkupfalse%
\ {\isachardoublequoteopen}m\ {\isasymcirc}\isactrlsub c\ left{\isacharunderscore}{\kern0pt}coproj\ X\ Y\ {\isasymcirc}\isactrlsub c\ c\ {\isacharequal}{\kern0pt}\ {\isasymlangle}c{\isacharcomma}{\kern0pt}\ y{\isadigit{1}}{\isasymrangle}{\isachardoublequoteclose}\isanewline
\ \ \ \ \ \ \ \ \ \ \isacommand{by}\isamarkupfalse%
\ {\isacharparenleft}{\kern0pt}simp\ add{\isacharcolon}{\kern0pt}\ m{\isacharunderscore}{\kern0pt}leftproj{\isacharunderscore}{\kern0pt}l{\isacharunderscore}{\kern0pt}equals{\isacharparenright}{\kern0pt}\isanewline
\ \ \ \ \ \ \ \ \isacommand{then}\isamarkupfalse%
\ \isacommand{show}\isamarkupfalse%
\ {\isacharquery}{\kern0pt}thesis\isanewline
\ \ \ \ \ \ \ \ \ \ \isacommand{using}\isamarkupfalse%
\ c{\isacharunderscore}{\kern0pt}def\ element{\isacharunderscore}{\kern0pt}pair{\isacharunderscore}{\kern0pt}eq\ eqs\ m{\isacharunderscore}{\kern0pt}proj{\isacharunderscore}{\kern0pt}a\ x{\isacharunderscore}{\kern0pt}def\ y{\isadigit{1}}y{\isadigit{2}}{\isacharunderscore}{\kern0pt}def{\isacharparenleft}{\kern0pt}{\isadigit{1}}{\isacharparenright}{\kern0pt}\ \isacommand{by}\isamarkupfalse%
\ auto\isanewline
\ \ \ \ \ \ \isacommand{next}\isamarkupfalse%
\isanewline
\ \ \ \ \ \ \ \ \isacommand{assume}\isamarkupfalse%
\ {\isachardoublequoteopen}{\isasymnexists}c{\isachardot}{\kern0pt}\ b\ {\isacharequal}{\kern0pt}\ left{\isacharunderscore}{\kern0pt}coproj\ X\ Y\ {\isasymcirc}\isactrlsub c\ c\ {\isasymand}\ c\ {\isasymin}\isactrlsub c\ X{\isachardoublequoteclose}\isanewline
\ \ \ \ \ \ \ \ \isacommand{then}\isamarkupfalse%
\ \isacommand{obtain}\isamarkupfalse%
\ c\ \isakeyword{where}\ c{\isacharunderscore}{\kern0pt}def{\isacharcolon}{\kern0pt}\ {\isachardoublequoteopen}b\ {\isacharequal}{\kern0pt}\ right{\isacharunderscore}{\kern0pt}coproj\ X\ Y\ {\isasymcirc}\isactrlsub c\ c\ \ {\isasymand}\ c\ {\isasymin}\isactrlsub c\ Y{\isachardoublequoteclose}\isanewline
\ \ \ \ \ \ \ \ \ \ \isacommand{using}\isamarkupfalse%
\ b{\isacharunderscore}{\kern0pt}type\ coprojs{\isacharunderscore}{\kern0pt}jointly{\isacharunderscore}{\kern0pt}surj\ \isacommand{by}\isamarkupfalse%
\ blast\isanewline
\ \ \ \ \ \ \ \ \isacommand{show}\isamarkupfalse%
\ {\isachardoublequoteopen}a\ {\isacharequal}{\kern0pt}\ b{\isachardoublequoteclose}\isanewline
\ \ \ \ \ \ \ \ \isacommand{proof}\isamarkupfalse%
{\isacharparenleft}{\kern0pt}cases\ {\isachardoublequoteopen}c\ {\isacharequal}{\kern0pt}\ y{\isadigit{1}}{\isachardoublequoteclose}{\isacharparenright}{\kern0pt}\isanewline
\ \ \ \ \ \ \ \ \ \ \isacommand{assume}\isamarkupfalse%
\ {\isachardoublequoteopen}c\ {\isacharequal}{\kern0pt}\ y{\isadigit{1}}{\isachardoublequoteclose}\ \ \ \ \ \ \ \isanewline
\ \ \ \ \ \ \ \ \ \ \isacommand{have}\isamarkupfalse%
\ m{\isacharunderscore}{\kern0pt}rightproj{\isacharunderscore}{\kern0pt}l{\isacharunderscore}{\kern0pt}equals{\isacharcolon}{\kern0pt}\ {\isachardoublequoteopen}m\ {\isasymcirc}\isactrlsub c\ right{\isacharunderscore}{\kern0pt}coproj\ X\ Y\ {\isasymcirc}\isactrlsub c\ c\ {\isacharequal}{\kern0pt}\ {\isasymlangle}x{\isadigit{2}}{\isacharcomma}{\kern0pt}\ y{\isadigit{2}}{\isasymrangle}{\isachardoublequoteclose}\isanewline
\ \ \ \ \ \ \ \ \ \ \ \ \isacommand{by}\isamarkupfalse%
\ {\isacharparenleft}{\kern0pt}simp\ add{\isacharcolon}{\kern0pt}\ {\isacartoucheopen}c\ {\isacharequal}{\kern0pt}\ y{\isadigit{1}}{\isacartoucheclose}\ m{\isacharunderscore}{\kern0pt}rightproj{\isacharunderscore}{\kern0pt}y{\isadigit{1}}{\isacharunderscore}{\kern0pt}equals{\isacharparenright}{\kern0pt}\ \ \ \ \ \ \ \isanewline
\ \ \ \ \ \ \ \ \ \ \isacommand{then}\isamarkupfalse%
\ \isacommand{show}\isamarkupfalse%
\ {\isacharquery}{\kern0pt}thesis\isanewline
\ \ \ \ \ \ \ \ \ \ \ \ \isacommand{using}\isamarkupfalse%
\ {\isacartoucheopen}c\ {\isacharequal}{\kern0pt}\ y{\isadigit{1}}{\isacartoucheclose}\ c{\isacharunderscore}{\kern0pt}def\ cart{\isacharunderscore}{\kern0pt}prod{\isacharunderscore}{\kern0pt}eq{\isadigit{2}}\ eqs\ m{\isacharunderscore}{\kern0pt}proj{\isacharunderscore}{\kern0pt}a\ x{\isadigit{1}}x{\isadigit{2}}{\isacharunderscore}{\kern0pt}def{\isacharparenleft}{\kern0pt}{\isadigit{2}}{\isacharparenright}{\kern0pt}\ x{\isacharunderscore}{\kern0pt}def\ y{\isadigit{1}}y{\isadigit{2}}{\isacharunderscore}{\kern0pt}def{\isacharparenleft}{\kern0pt}{\isadigit{2}}{\isacharparenright}{\kern0pt}\ y{\isadigit{1}}y{\isadigit{2}}{\isacharunderscore}{\kern0pt}def{\isacharparenleft}{\kern0pt}{\isadigit{3}}{\isacharparenright}{\kern0pt}\ \isacommand{by}\isamarkupfalse%
\ auto\isanewline
\ \ \ \ \ \ \ \ \isacommand{next}\isamarkupfalse%
\isanewline
\ \ \ \ \ \ \ \ \ \ \isacommand{assume}\isamarkupfalse%
\ {\isachardoublequoteopen}c\ {\isasymnoteq}\ y{\isadigit{1}}{\isachardoublequoteclose}\ \ \ \ \ \ \ \isanewline
\ \ \ \ \ \ \ \ \ \ \isacommand{then}\isamarkupfalse%
\ \isacommand{obtain}\isamarkupfalse%
\ k\ \isakeyword{where}\ k{\isacharunderscore}{\kern0pt}def{\isacharcolon}{\kern0pt}\ \ {\isachardoublequoteopen}m\ {\isasymcirc}\isactrlsub c\ right{\isacharunderscore}{\kern0pt}coproj\ X\ Y\ {\isasymcirc}\isactrlsub c\ c\ {\isacharequal}{\kern0pt}\ {\isasymlangle}x{\isadigit{1}}{\isacharcomma}{\kern0pt}\ y{\isadigit{1}}\isactrlsup c\ {\isasymcirc}\isactrlsub c\ k{\isasymrangle}{\isachardoublequoteclose}\isanewline
\ \ \ \ \ \ \ \ \ \ \ \ \isacommand{using}\isamarkupfalse%
\ c{\isacharunderscore}{\kern0pt}def\ m{\isacharunderscore}{\kern0pt}rightproj{\isacharunderscore}{\kern0pt}not{\isacharunderscore}{\kern0pt}y{\isadigit{1}}{\isacharunderscore}{\kern0pt}equals\ \isacommand{by}\isamarkupfalse%
\ blast\ \ \ \ \ \ \ \ \ \ \ \ \ \ \ \ \ \ \ \ \ \isanewline
\ \ \ \ \ \ \ \ \ \ \isacommand{then}\isamarkupfalse%
\ \isacommand{have}\isamarkupfalse%
\ {\isachardoublequoteopen}{\isasymlangle}x{\isacharcomma}{\kern0pt}\ y{\isadigit{1}}{\isasymrangle}\ {\isacharequal}{\kern0pt}\ {\isasymlangle}x{\isadigit{1}}{\isacharcomma}{\kern0pt}\ y{\isadigit{1}}\isactrlsup c\ {\isasymcirc}\isactrlsub c\ k{\isasymrangle}{\isachardoublequoteclose}\isanewline
\ \ \ \ \ \ \ \ \ \ \ \ \isacommand{using}\isamarkupfalse%
\ c{\isacharunderscore}{\kern0pt}def\ eqs\ m{\isacharunderscore}{\kern0pt}proj{\isacharunderscore}{\kern0pt}a\ x{\isacharunderscore}{\kern0pt}def\ \isacommand{by}\isamarkupfalse%
\ auto\isanewline
\ \ \ \ \ \ \ \ \ \ \isacommand{then}\isamarkupfalse%
\ \isacommand{have}\isamarkupfalse%
\ {\isachardoublequoteopen}{\isacharparenleft}{\kern0pt}x\ {\isacharequal}{\kern0pt}\ x{\isadigit{1}}{\isacharparenright}{\kern0pt}\ {\isasymand}\ {\isacharparenleft}{\kern0pt}y{\isadigit{1}}\ {\isacharequal}{\kern0pt}\ y{\isadigit{1}}\isactrlsup c\ {\isasymcirc}\isactrlsub c\ k{\isacharparenright}{\kern0pt}{\isachardoublequoteclose}\isanewline
\ \ \ \ \ \ \ \ \ \ \ \ \isacommand{by}\isamarkupfalse%
\ {\isacharparenleft}{\kern0pt}smt\ {\isacartoucheopen}c\ {\isasymnoteq}\ y{\isadigit{1}}{\isacartoucheclose}\ c{\isacharunderscore}{\kern0pt}def\ cfunc{\isacharunderscore}{\kern0pt}type{\isacharunderscore}{\kern0pt}def\ comp{\isacharunderscore}{\kern0pt}associative\ comp{\isacharunderscore}{\kern0pt}type\ element{\isacharunderscore}{\kern0pt}pair{\isacharunderscore}{\kern0pt}eq\ k{\isacharunderscore}{\kern0pt}def\ m{\isacharunderscore}{\kern0pt}rightproj{\isacharunderscore}{\kern0pt}not{\isacharunderscore}{\kern0pt}y{\isadigit{1}}{\isacharunderscore}{\kern0pt}equals\ monomorphism{\isacharunderscore}{\kern0pt}def{\isadigit{3}}\ try{\isacharunderscore}{\kern0pt}cast{\isacharunderscore}{\kern0pt}m{\isacharunderscore}{\kern0pt}m{\isacharprime}{\kern0pt}\ try{\isacharunderscore}{\kern0pt}cast{\isacharunderscore}{\kern0pt}mono\ trycast{\isacharunderscore}{\kern0pt}y{\isadigit{1}}{\isacharunderscore}{\kern0pt}type\ x{\isadigit{1}}x{\isadigit{2}}{\isacharunderscore}{\kern0pt}def{\isacharparenleft}{\kern0pt}{\isadigit{1}}{\isacharparenright}{\kern0pt}\ x{\isacharunderscore}{\kern0pt}def\ y{\isadigit{1}}{\isacharprime}{\kern0pt}{\isacharunderscore}{\kern0pt}type\ y{\isadigit{1}}{\isacharunderscore}{\kern0pt}mono\ y{\isadigit{1}}y{\isadigit{2}}{\isacharunderscore}{\kern0pt}def{\isacharparenleft}{\kern0pt}{\isadigit{1}}{\isacharparenright}{\kern0pt}{\isacharparenright}{\kern0pt}\isanewline
\ \ \ \ \ \ \ \ \ \ \isacommand{then}\isamarkupfalse%
\ \isacommand{have}\isamarkupfalse%
\ False\isanewline
\ \ \ \ \ \ \ \ \ \ \ \ \isacommand{by}\isamarkupfalse%
\ {\isacharparenleft}{\kern0pt}smt\ {\isacartoucheopen}c\ {\isasymnoteq}\ y{\isadigit{1}}{\isacartoucheclose}\ \ c{\isacharunderscore}{\kern0pt}def\ comp{\isacharunderscore}{\kern0pt}type\ complement{\isacharunderscore}{\kern0pt}disjoint\ element{\isacharunderscore}{\kern0pt}pair{\isacharunderscore}{\kern0pt}eq\ id{\isacharunderscore}{\kern0pt}right{\isacharunderscore}{\kern0pt}unit{\isadigit{2}}\ id{\isacharunderscore}{\kern0pt}type\ k{\isacharunderscore}{\kern0pt}def\ m{\isacharunderscore}{\kern0pt}rightproj{\isacharunderscore}{\kern0pt}not{\isacharunderscore}{\kern0pt}y{\isadigit{1}}{\isacharunderscore}{\kern0pt}equals\ x{\isacharunderscore}{\kern0pt}def\ y{\isadigit{1}}{\isacharprime}{\kern0pt}{\isacharunderscore}{\kern0pt}type\ y{\isadigit{1}}{\isacharunderscore}{\kern0pt}mono\ y{\isadigit{1}}y{\isadigit{2}}{\isacharunderscore}{\kern0pt}def{\isacharparenleft}{\kern0pt}{\isadigit{1}}{\isacharparenright}{\kern0pt}{\isacharparenright}{\kern0pt}\isanewline
\ \ \ \ \ \ \ \ \ \ \isacommand{then}\isamarkupfalse%
\ \isacommand{show}\isamarkupfalse%
\ {\isacharquery}{\kern0pt}thesis\ \isacommand{by}\isamarkupfalse%
\ auto\isanewline
\ \ \ \ \ \ \ \ \isacommand{qed}\isamarkupfalse%
\isanewline
\ \ \ \ \ \ \isacommand{qed}\isamarkupfalse%
\isanewline
\ \ \ \ \isacommand{next}\isamarkupfalse%
\ \isanewline
\ \ \ \ \ \ \isacommand{assume}\isamarkupfalse%
\ {\isachardoublequoteopen}{\isasymnexists}x{\isachardot}{\kern0pt}\ a\ {\isacharequal}{\kern0pt}\ left{\isacharunderscore}{\kern0pt}coproj\ X\ Y\ {\isasymcirc}\isactrlsub c\ x\ {\isasymand}\ x\ {\isasymin}\isactrlsub c\ X{\isachardoublequoteclose}\isanewline
\ \ \ \ \ \ \isacommand{then}\isamarkupfalse%
\ \isacommand{obtain}\isamarkupfalse%
\ y\ \isakeyword{where}\ y{\isacharunderscore}{\kern0pt}def{\isacharcolon}{\kern0pt}\ {\isachardoublequoteopen}a\ {\isacharequal}{\kern0pt}\ right{\isacharunderscore}{\kern0pt}coproj\ X\ Y\ {\isasymcirc}\isactrlsub c\ y\ {\isasymand}\ y\ {\isasymin}\isactrlsub c\ Y{\isachardoublequoteclose}\isanewline
\ \ \ \ \ \ \ \ \isacommand{using}\isamarkupfalse%
\ a{\isacharunderscore}{\kern0pt}type\ coprojs{\isacharunderscore}{\kern0pt}jointly{\isacharunderscore}{\kern0pt}surj\ \isacommand{by}\isamarkupfalse%
\ blast\isanewline
\ \ \ \ \ \ \isacommand{show}\isamarkupfalse%
\ {\isachardoublequoteopen}a\ {\isacharequal}{\kern0pt}\ b{\isachardoublequoteclose}\isanewline
\ \ \ \ \ \ \isacommand{proof}\isamarkupfalse%
{\isacharparenleft}{\kern0pt}cases\ {\isachardoublequoteopen}y\ {\isacharequal}{\kern0pt}\ y{\isadigit{1}}{\isachardoublequoteclose}{\isacharparenright}{\kern0pt}\isanewline
\ \ \ \ \ \ \ \ \isacommand{assume}\isamarkupfalse%
\ {\isachardoublequoteopen}y\ {\isacharequal}{\kern0pt}\ y{\isadigit{1}}{\isachardoublequoteclose}\isanewline
\ \ \ \ \ \ \ \ \isacommand{then}\isamarkupfalse%
\ \ \isacommand{have}\isamarkupfalse%
\ m{\isacharunderscore}{\kern0pt}rightproj{\isacharunderscore}{\kern0pt}y{\isacharunderscore}{\kern0pt}equals{\isacharcolon}{\kern0pt}\ {\isachardoublequoteopen}m\ {\isasymcirc}\isactrlsub c\ right{\isacharunderscore}{\kern0pt}coproj\ X\ Y\ {\isasymcirc}\isactrlsub c\ y\ {\isacharequal}{\kern0pt}\ {\isasymlangle}x{\isadigit{2}}{\isacharcomma}{\kern0pt}\ y{\isadigit{2}}{\isasymrangle}{\isachardoublequoteclose}\isanewline
\ \ \ \ \ \ \ \ \ \ \isacommand{using}\isamarkupfalse%
\ m{\isacharunderscore}{\kern0pt}rightproj{\isacharunderscore}{\kern0pt}y{\isadigit{1}}{\isacharunderscore}{\kern0pt}equals\ \isacommand{by}\isamarkupfalse%
\ blast\isanewline
\ \ \ \ \ \ \ \ \isacommand{then}\isamarkupfalse%
\ \isacommand{have}\isamarkupfalse%
\ {\isachardoublequoteopen}m\ {\isasymcirc}\isactrlsub c\ a\ \ {\isacharequal}{\kern0pt}\ {\isasymlangle}x{\isadigit{2}}{\isacharcomma}{\kern0pt}\ y{\isadigit{2}}{\isasymrangle}{\isachardoublequoteclose}\isanewline
\ \ \ \ \ \ \ \ \ \ \isacommand{using}\isamarkupfalse%
\ y{\isacharunderscore}{\kern0pt}def\ \isacommand{by}\isamarkupfalse%
\ blast\isanewline
\ \ \ \ \ \ \ \ \isacommand{show}\isamarkupfalse%
\ {\isachardoublequoteopen}a\ {\isacharequal}{\kern0pt}\ b{\isachardoublequoteclose}\isanewline
\ \ \ \ \ \ \ \ \isacommand{proof}\isamarkupfalse%
{\isacharparenleft}{\kern0pt}cases\ {\isachardoublequoteopen}{\isasymexists}c{\isachardot}{\kern0pt}\ b\ {\isacharequal}{\kern0pt}\ left{\isacharunderscore}{\kern0pt}coproj\ X\ Y\ {\isasymcirc}\isactrlsub c\ c\ \ {\isasymand}\ c\ {\isasymin}\isactrlsub c\ X{\isachardoublequoteclose}{\isacharparenright}{\kern0pt}\isanewline
\ \ \ \ \ \ \ \ \ \ \isacommand{assume}\isamarkupfalse%
\ {\isachardoublequoteopen}{\isasymexists}c{\isachardot}{\kern0pt}\ b\ {\isacharequal}{\kern0pt}\ left{\isacharunderscore}{\kern0pt}coproj\ X\ Y\ {\isasymcirc}\isactrlsub c\ c\ {\isasymand}\ c\ {\isasymin}\isactrlsub c\ X{\isachardoublequoteclose}\isanewline
\ \ \ \ \ \ \ \ \ \ \isacommand{then}\isamarkupfalse%
\ \isacommand{obtain}\isamarkupfalse%
\ c\ \isakeyword{where}\ c{\isacharunderscore}{\kern0pt}def{\isacharcolon}{\kern0pt}\ {\isachardoublequoteopen}b\ {\isacharequal}{\kern0pt}\ left{\isacharunderscore}{\kern0pt}coproj\ X\ Y\ {\isasymcirc}\isactrlsub c\ c\ {\isasymand}\ c\ {\isasymin}\isactrlsub c\ X{\isachardoublequoteclose}\isanewline
\ \ \ \ \ \ \ \ \ \ \ \ \isacommand{by}\isamarkupfalse%
\ blast\isanewline
\ \ \ \ \ \ \ \ \ \ \isacommand{then}\isamarkupfalse%
\ \isacommand{show}\isamarkupfalse%
\ {\isachardoublequoteopen}a\ {\isacharequal}{\kern0pt}\ b{\isachardoublequoteclose}\isanewline
\ \ \ \ \ \ \ \ \ \ \ \ \isacommand{using}\isamarkupfalse%
\ cart{\isacharunderscore}{\kern0pt}prod{\isacharunderscore}{\kern0pt}eq{\isadigit{2}}\ eqs\ m{\isacharunderscore}{\kern0pt}leftproj{\isacharunderscore}{\kern0pt}l{\isacharunderscore}{\kern0pt}equals\ m{\isacharunderscore}{\kern0pt}rightproj{\isacharunderscore}{\kern0pt}y{\isacharunderscore}{\kern0pt}equals\ x{\isadigit{1}}x{\isadigit{2}}{\isacharunderscore}{\kern0pt}def{\isacharparenleft}{\kern0pt}{\isadigit{2}}{\isacharparenright}{\kern0pt}\ y{\isadigit{1}}y{\isadigit{2}}{\isacharunderscore}{\kern0pt}def\ y{\isacharunderscore}{\kern0pt}def\ \isacommand{by}\isamarkupfalse%
\ auto\isanewline
\ \ \ \ \ \ \ \ \isacommand{next}\isamarkupfalse%
\isanewline
\ \ \ \ \ \ \ \ \ \ \isacommand{assume}\isamarkupfalse%
\ {\isachardoublequoteopen}{\isasymnexists}c{\isachardot}{\kern0pt}\ b\ {\isacharequal}{\kern0pt}\ left{\isacharunderscore}{\kern0pt}coproj\ X\ Y\ {\isasymcirc}\isactrlsub c\ c\ {\isasymand}\ c\ {\isasymin}\isactrlsub c\ X{\isachardoublequoteclose}\isanewline
\ \ \ \ \ \ \ \ \ \ \isacommand{then}\isamarkupfalse%
\ \isacommand{obtain}\isamarkupfalse%
\ c\ \isakeyword{where}\ c{\isacharunderscore}{\kern0pt}def{\isacharcolon}{\kern0pt}\ {\isachardoublequoteopen}b\ {\isacharequal}{\kern0pt}\ right{\isacharunderscore}{\kern0pt}coproj\ X\ Y\ {\isasymcirc}\isactrlsub c\ c\ {\isasymand}\ c\ {\isasymin}\isactrlsub c\ Y{\isachardoublequoteclose}\isanewline
\ \ \ \ \ \ \ \ \ \ \ \ \isacommand{using}\isamarkupfalse%
\ b{\isacharunderscore}{\kern0pt}type\ coprojs{\isacharunderscore}{\kern0pt}jointly{\isacharunderscore}{\kern0pt}surj\ \isacommand{by}\isamarkupfalse%
\ blast\isanewline
\ \ \ \ \ \ \ \ \ \ \isacommand{show}\isamarkupfalse%
\ {\isachardoublequoteopen}a\ {\isacharequal}{\kern0pt}\ b{\isachardoublequoteclose}\isanewline
\ \ \ \ \ \ \ \ \ \ \isacommand{proof}\isamarkupfalse%
{\isacharparenleft}{\kern0pt}cases\ {\isachardoublequoteopen}c\ {\isacharequal}{\kern0pt}\ y{\isachardoublequoteclose}{\isacharparenright}{\kern0pt}\isanewline
\ \ \ \ \ \ \ \ \ \ \ \ \isacommand{assume}\isamarkupfalse%
\ {\isachardoublequoteopen}c\ {\isacharequal}{\kern0pt}\ y{\isachardoublequoteclose}\isanewline
\ \ \ \ \ \ \ \ \ \ \ \ \isacommand{show}\isamarkupfalse%
\ {\isachardoublequoteopen}a\ {\isacharequal}{\kern0pt}\ b{\isachardoublequoteclose}\isanewline
\ \ \ \ \ \ \ \ \ \ \ \ \ \ \isacommand{by}\isamarkupfalse%
\ {\isacharparenleft}{\kern0pt}simp\ add{\isacharcolon}{\kern0pt}\ {\isacartoucheopen}c\ {\isacharequal}{\kern0pt}\ y{\isacartoucheclose}\ c{\isacharunderscore}{\kern0pt}def\ y{\isacharunderscore}{\kern0pt}def{\isacharparenright}{\kern0pt}\isanewline
\ \ \ \ \ \ \ \ \ \ \isacommand{next}\isamarkupfalse%
\isanewline
\ \ \ \ \ \ \ \ \ \ \ \ \isacommand{assume}\isamarkupfalse%
\ {\isachardoublequoteopen}c\ {\isasymnoteq}\ y{\isachardoublequoteclose}\isanewline
\ \ \ \ \ \ \ \ \ \ \ \ \isacommand{then}\isamarkupfalse%
\ \isacommand{have}\isamarkupfalse%
\ {\isachardoublequoteopen}c\ {\isasymnoteq}\ y{\isadigit{1}}{\isachardoublequoteclose}\isanewline
\ \ \ \ \ \ \ \ \ \ \ \ \ \ \isacommand{by}\isamarkupfalse%
\ {\isacharparenleft}{\kern0pt}simp\ add{\isacharcolon}{\kern0pt}\ {\isacartoucheopen}y\ {\isacharequal}{\kern0pt}\ y{\isadigit{1}}{\isacartoucheclose}{\isacharparenright}{\kern0pt}\isanewline
\ \ \ \ \ \ \ \ \ \ \ \ \isacommand{then}\isamarkupfalse%
\ \isacommand{obtain}\isamarkupfalse%
\ k\ \isakeyword{where}\ k{\isacharunderscore}{\kern0pt}def{\isacharcolon}{\kern0pt}\ {\isachardoublequoteopen}k\ {\isasymin}\isactrlsub c\ Y\ {\isasymsetminus}\ {\isacharparenleft}{\kern0pt}{\isasymone}{\isacharcomma}{\kern0pt}y{\isadigit{1}}{\isacharparenright}{\kern0pt}\ {\isasymand}\ try{\isacharunderscore}{\kern0pt}cast\ y{\isadigit{1}}\ {\isasymcirc}\isactrlsub c\ c\ {\isacharequal}{\kern0pt}\ right{\isacharunderscore}{\kern0pt}coproj\ {\isasymone}\ {\isacharparenleft}{\kern0pt}Y\ {\isasymsetminus}\ {\isacharparenleft}{\kern0pt}{\isasymone}{\isacharcomma}{\kern0pt}y{\isadigit{1}}{\isacharparenright}{\kern0pt}{\isacharparenright}{\kern0pt}\ {\isasymcirc}\isactrlsub c\ k\ {\isasymand}\ \isanewline
\ \ \ \ \ \ \ \ \ \ \ \ \ \ \ \ \ m\ {\isasymcirc}\isactrlsub c\ right{\isacharunderscore}{\kern0pt}coproj\ X\ Y\ {\isasymcirc}\isactrlsub c\ c\ {\isacharequal}{\kern0pt}\ {\isasymlangle}x{\isadigit{1}}{\isacharcomma}{\kern0pt}\ y{\isadigit{1}}\isactrlsup c\ {\isasymcirc}\isactrlsub c\ k{\isasymrangle}{\isachardoublequoteclose}\isanewline
\ \ \ \ \ \ \ \ \ \ \ \ \ \ \isacommand{using}\isamarkupfalse%
\ c{\isacharunderscore}{\kern0pt}def\ m{\isacharunderscore}{\kern0pt}rightproj{\isacharunderscore}{\kern0pt}not{\isacharunderscore}{\kern0pt}y{\isadigit{1}}{\isacharunderscore}{\kern0pt}equals\ \isacommand{by}\isamarkupfalse%
\ blast\isanewline
\ \ \ \ \ \ \ \ \ \ \ \ \isacommand{then}\isamarkupfalse%
\ \isacommand{have}\isamarkupfalse%
\ {\isachardoublequoteopen}{\isasymlangle}x{\isadigit{2}}{\isacharcomma}{\kern0pt}\ y{\isadigit{2}}{\isasymrangle}\ {\isacharequal}{\kern0pt}\ {\isasymlangle}x{\isadigit{1}}{\isacharcomma}{\kern0pt}\ y{\isadigit{1}}\isactrlsup c\ {\isasymcirc}\isactrlsub c\ k{\isasymrangle}{\isachardoublequoteclose}\isanewline
\ \ \ \ \ \ \ \ \ \ \ \ \ \ \isacommand{using}\isamarkupfalse%
\ {\isacartoucheopen}m\ {\isasymcirc}\isactrlsub c\ a\ {\isacharequal}{\kern0pt}\ {\isasymlangle}x{\isadigit{2}}{\isacharcomma}{\kern0pt}y{\isadigit{2}}{\isasymrangle}{\isacartoucheclose}\ c{\isacharunderscore}{\kern0pt}def\ eqs\ \isacommand{by}\isamarkupfalse%
\ auto\isanewline
\ \ \ \ \ \ \ \ \ \ \ \ \isacommand{then}\isamarkupfalse%
\ \isacommand{have}\isamarkupfalse%
\ False\isanewline
\ \ \ \ \ \ \ \ \ \ \ \ \ \ \isacommand{using}\isamarkupfalse%
\ comp{\isacharunderscore}{\kern0pt}type\ element{\isacharunderscore}{\kern0pt}pair{\isacharunderscore}{\kern0pt}eq\ k{\isacharunderscore}{\kern0pt}def\ x{\isadigit{1}}x{\isadigit{2}}{\isacharunderscore}{\kern0pt}def\ y{\isadigit{1}}{\isacharprime}{\kern0pt}{\isacharunderscore}{\kern0pt}type\ y{\isadigit{1}}y{\isadigit{2}}{\isacharunderscore}{\kern0pt}def{\isacharparenleft}{\kern0pt}{\isadigit{2}}{\isacharparenright}{\kern0pt}\ \isacommand{by}\isamarkupfalse%
\ auto\isanewline
\ \ \ \ \ \ \ \ \ \ \ \ \isacommand{then}\isamarkupfalse%
\ \isacommand{show}\isamarkupfalse%
\ {\isacharquery}{\kern0pt}thesis\isanewline
\ \ \ \ \ \ \ \ \ \ \ \ \ \ \isacommand{by}\isamarkupfalse%
\ simp\isanewline
\ \ \ \ \ \ \ \ \ \ \isacommand{qed}\isamarkupfalse%
\isanewline
\ \ \ \ \ \ \ \ \isacommand{qed}\isamarkupfalse%
\isanewline
\ \ \ \ \ \ \isacommand{next}\isamarkupfalse%
\isanewline
\ \ \ \ \ \ \ \ \isacommand{assume}\isamarkupfalse%
\ {\isachardoublequoteopen}y\ {\isasymnoteq}\ y{\isadigit{1}}{\isachardoublequoteclose}\isanewline
\ \ \ \ \ \ \ \ \isacommand{then}\isamarkupfalse%
\ \isacommand{obtain}\isamarkupfalse%
\ k\ \isakeyword{where}\ k{\isacharunderscore}{\kern0pt}def{\isacharcolon}{\kern0pt}\ {\isachardoublequoteopen}k\ {\isasymin}\isactrlsub c\ Y\ {\isasymsetminus}\ {\isacharparenleft}{\kern0pt}{\isasymone}{\isacharcomma}{\kern0pt}y{\isadigit{1}}{\isacharparenright}{\kern0pt}\ {\isasymand}\ try{\isacharunderscore}{\kern0pt}cast\ y{\isadigit{1}}\ {\isasymcirc}\isactrlsub c\ y\ {\isacharequal}{\kern0pt}\ right{\isacharunderscore}{\kern0pt}coproj\ {\isasymone}\ {\isacharparenleft}{\kern0pt}Y\ {\isasymsetminus}\ {\isacharparenleft}{\kern0pt}{\isasymone}{\isacharcomma}{\kern0pt}y{\isadigit{1}}{\isacharparenright}{\kern0pt}{\isacharparenright}{\kern0pt}\ {\isasymcirc}\isactrlsub c\ k\ {\isasymand}\ \isanewline
\ \ \ \ \ \ \ \ \ \ m\ {\isasymcirc}\isactrlsub c\ right{\isacharunderscore}{\kern0pt}coproj\ X\ Y\ {\isasymcirc}\isactrlsub c\ y\ {\isacharequal}{\kern0pt}\ {\isasymlangle}x{\isadigit{1}}{\isacharcomma}{\kern0pt}\ y{\isadigit{1}}\isactrlsup c\ {\isasymcirc}\isactrlsub c\ k{\isasymrangle}{\isachardoublequoteclose}\isanewline
\ \ \ \ \ \ \ \ \ \ \isacommand{using}\isamarkupfalse%
\ m{\isacharunderscore}{\kern0pt}rightproj{\isacharunderscore}{\kern0pt}not{\isacharunderscore}{\kern0pt}y{\isadigit{1}}{\isacharunderscore}{\kern0pt}equals\ y{\isacharunderscore}{\kern0pt}def\ \isacommand{by}\isamarkupfalse%
\ blast\ \ \isanewline
\ \ \ \ \ \ \ \ \isacommand{then}\isamarkupfalse%
\ \isacommand{have}\isamarkupfalse%
\ {\isachardoublequoteopen}m\ {\isasymcirc}\isactrlsub c\ a\ \ {\isacharequal}{\kern0pt}\ {\isasymlangle}x{\isadigit{1}}{\isacharcomma}{\kern0pt}\ y{\isadigit{1}}\isactrlsup c\ {\isasymcirc}\isactrlsub c\ k{\isasymrangle}{\isachardoublequoteclose}\isanewline
\ \ \ \ \ \ \ \ \ \ \isacommand{using}\isamarkupfalse%
\ y{\isacharunderscore}{\kern0pt}def\ \isacommand{by}\isamarkupfalse%
\ blast\isanewline
\ \ \ \ \ \ \ \ \isacommand{show}\isamarkupfalse%
\ {\isachardoublequoteopen}a\ {\isacharequal}{\kern0pt}\ b{\isachardoublequoteclose}\isanewline
\ \ \ \ \ \ \ \ \isacommand{proof}\isamarkupfalse%
{\isacharparenleft}{\kern0pt}cases\ {\isachardoublequoteopen}{\isasymexists}c{\isachardot}{\kern0pt}\ b\ {\isacharequal}{\kern0pt}\ right{\isacharunderscore}{\kern0pt}coproj\ X\ Y\ {\isasymcirc}\isactrlsub c\ c\ \ {\isasymand}\ c\ {\isasymin}\isactrlsub c\ Y{\isachardoublequoteclose}{\isacharparenright}{\kern0pt}\isanewline
\ \ \ \ \ \ \ \ \ \ \isacommand{assume}\isamarkupfalse%
\ {\isachardoublequoteopen}{\isasymexists}c{\isachardot}{\kern0pt}\ b\ {\isacharequal}{\kern0pt}\ right{\isacharunderscore}{\kern0pt}coproj\ X\ Y\ {\isasymcirc}\isactrlsub c\ c\ \ {\isasymand}\ c\ {\isasymin}\isactrlsub c\ Y{\isachardoublequoteclose}\isanewline
\ \ \ \ \ \ \ \ \ \ \isacommand{then}\isamarkupfalse%
\ \isacommand{obtain}\isamarkupfalse%
\ c\ \isakeyword{where}\ c{\isacharunderscore}{\kern0pt}def{\isacharcolon}{\kern0pt}\ {\isachardoublequoteopen}b\ {\isacharequal}{\kern0pt}\ right{\isacharunderscore}{\kern0pt}coproj\ X\ Y\ {\isasymcirc}\isactrlsub c\ c\ {\isasymand}\ c\ {\isasymin}\isactrlsub c\ Y{\isachardoublequoteclose}\isanewline
\ \ \ \ \ \ \ \ \ \ \ \ \isacommand{by}\isamarkupfalse%
\ blast\ \ \isanewline
\ \ \ \ \ \ \ \ \ \ \isacommand{show}\isamarkupfalse%
\ {\isachardoublequoteopen}a\ {\isacharequal}{\kern0pt}\ b{\isachardoublequoteclose}\isanewline
\ \ \ \ \ \ \ \ \ \ \isacommand{proof}\isamarkupfalse%
{\isacharparenleft}{\kern0pt}cases\ {\isachardoublequoteopen}c\ {\isacharequal}{\kern0pt}\ y{\isadigit{1}}{\isachardoublequoteclose}{\isacharparenright}{\kern0pt}\isanewline
\ \ \ \ \ \ \ \ \ \ \ \ \isacommand{assume}\isamarkupfalse%
\ {\isachardoublequoteopen}c\ {\isacharequal}{\kern0pt}\ y{\isadigit{1}}{\isachardoublequoteclose}\ \isanewline
\ \ \ \ \ \ \ \ \ \ \ \ \isacommand{show}\isamarkupfalse%
\ {\isachardoublequoteopen}a\ {\isacharequal}{\kern0pt}\ b{\isachardoublequoteclose}\isanewline
\ \ \ \ \ \ \ \ \ \ \ \ \ \ \isacommand{proof}\isamarkupfalse%
\ {\isacharminus}{\kern0pt}\isanewline
\ \ \ \ \ \ \ \ \ \ \ \ \ \ \ \ \isacommand{obtain}\isamarkupfalse%
\ cc\ {\isacharcolon}{\kern0pt}{\isacharcolon}{\kern0pt}\ cfunc\ \isakeyword{where}\isanewline
\ \ \ \ \ \ \ \ \ \ \ \ \ \ \ \ \ \ f{\isadigit{1}}{\isacharcolon}{\kern0pt}\ {\isachardoublequoteopen}cc\ {\isasymin}\isactrlsub c\ Y\ {\isasymsetminus}\ {\isacharparenleft}{\kern0pt}{\isasymone}{\isacharcomma}{\kern0pt}\ y{\isadigit{1}}{\isacharparenright}{\kern0pt}\ {\isasymand}\ try{\isacharunderscore}{\kern0pt}cast\ y{\isadigit{1}}\ {\isasymcirc}\isactrlsub c\ y\ {\isacharequal}{\kern0pt}\ right{\isacharunderscore}{\kern0pt}coproj\ {\isasymone}\ {\isacharparenleft}{\kern0pt}Y\ {\isasymsetminus}\ {\isacharparenleft}{\kern0pt}{\isasymone}{\isacharcomma}{\kern0pt}\ y{\isadigit{1}}{\isacharparenright}{\kern0pt}{\isacharparenright}{\kern0pt}\ {\isasymcirc}\isactrlsub c\ cc\ {\isasymand}\ m\ {\isasymcirc}\isactrlsub c\ right{\isacharunderscore}{\kern0pt}coproj\ X\ Y\ {\isasymcirc}\isactrlsub c\ y\ {\isacharequal}{\kern0pt}\ {\isasymlangle}x{\isadigit{1}}{\isacharcomma}{\kern0pt}y{\isadigit{1}}\isactrlsup c\ {\isasymcirc}\isactrlsub c\ cc{\isasymrangle}{\isachardoublequoteclose}\isanewline
\ \ \ \ \ \ \ \ \ \ \ \ \ \ \ \ \ \ \isacommand{using}\isamarkupfalse%
\ {\isacartoucheopen}{\isasymAnd}thesis{\isachardot}{\kern0pt}\ {\isacharparenleft}{\kern0pt}{\isasymAnd}k{\isachardot}{\kern0pt}\ k\ {\isasymin}\isactrlsub c\ Y\ {\isasymsetminus}\ {\isacharparenleft}{\kern0pt}{\isasymone}{\isacharcomma}{\kern0pt}\ y{\isadigit{1}}{\isacharparenright}{\kern0pt}\ {\isasymand}\ try{\isacharunderscore}{\kern0pt}cast\ y{\isadigit{1}}\ {\isasymcirc}\isactrlsub c\ y\ {\isacharequal}{\kern0pt}\ right{\isacharunderscore}{\kern0pt}coproj\ {\isasymone}\ {\isacharparenleft}{\kern0pt}Y\ {\isasymsetminus}\ {\isacharparenleft}{\kern0pt}{\isasymone}{\isacharcomma}{\kern0pt}\ y{\isadigit{1}}{\isacharparenright}{\kern0pt}{\isacharparenright}{\kern0pt}\ {\isasymcirc}\isactrlsub c\ k\ {\isasymand}\ m\ {\isasymcirc}\isactrlsub c\ right{\isacharunderscore}{\kern0pt}coproj\ X\ Y\ {\isasymcirc}\isactrlsub c\ y\ {\isacharequal}{\kern0pt}\ {\isasymlangle}x{\isadigit{1}}{\isacharcomma}{\kern0pt}y{\isadigit{1}}\isactrlsup c\ {\isasymcirc}\isactrlsub c\ k{\isasymrangle}\ {\isasymLongrightarrow}\ thesis{\isacharparenright}{\kern0pt}\ {\isasymLongrightarrow}\ thesis{\isacartoucheclose}\ \isacommand{by}\isamarkupfalse%
\ blast\isanewline
\ \ \ \ \ \ \ \ \ \ \ \ \ \ \ \ \isacommand{have}\isamarkupfalse%
\ {\isachardoublequoteopen}{\isasymlangle}x{\isadigit{2}}{\isacharcomma}{\kern0pt}y{\isadigit{2}}{\isasymrangle}\ {\isacharequal}{\kern0pt}\ m\ {\isasymcirc}\isactrlsub c\ a{\isachardoublequoteclose}\isanewline
\ \ \ \ \ \ \ \ \ \ \ \ \ \ \isacommand{using}\isamarkupfalse%
\ {\isacartoucheopen}c\ {\isacharequal}{\kern0pt}\ y{\isadigit{1}}{\isacartoucheclose}\ c{\isacharunderscore}{\kern0pt}def\ eqs\ m{\isacharunderscore}{\kern0pt}rightproj{\isacharunderscore}{\kern0pt}y{\isadigit{1}}{\isacharunderscore}{\kern0pt}equals\ \isacommand{by}\isamarkupfalse%
\ presburger\isanewline
\ \ \ \ \ \ \ \ \ \ \ \ \ \ \isacommand{then}\isamarkupfalse%
\ \isacommand{show}\isamarkupfalse%
\ {\isacharquery}{\kern0pt}thesis\isanewline
\ \ \ \ \ \ \ \ \ \ \ \ \ \ \isacommand{using}\isamarkupfalse%
\ f{\isadigit{1}}\ cart{\isacharunderscore}{\kern0pt}prod{\isacharunderscore}{\kern0pt}eq{\isadigit{2}}\ comp{\isacharunderscore}{\kern0pt}type\ x{\isadigit{1}}x{\isadigit{2}}{\isacharunderscore}{\kern0pt}def\ y{\isadigit{1}}{\isacharprime}{\kern0pt}{\isacharunderscore}{\kern0pt}type\ y{\isadigit{1}}y{\isadigit{2}}{\isacharunderscore}{\kern0pt}def{\isacharparenleft}{\kern0pt}{\isadigit{2}}{\isacharparenright}{\kern0pt}\ y{\isacharunderscore}{\kern0pt}def\ \isacommand{by}\isamarkupfalse%
\ force\isanewline
\ \ \ \ \ \ \ \ \ \ \ \ \ \ \isacommand{qed}\isamarkupfalse%
\isanewline
\ \ \ \ \ \ \ \ \ \ \isacommand{next}\isamarkupfalse%
\isanewline
\ \ \ \ \ \ \ \ \ \ \ \ \ \ \isacommand{assume}\isamarkupfalse%
\ {\isachardoublequoteopen}c\ {\isasymnoteq}\ y{\isadigit{1}}{\isachardoublequoteclose}\ \ \ \ \ \ \ \ \ \ \ \ \ \ \isanewline
\ \ \ \ \ \ \ \ \ \ \ \ \ \ \isacommand{then}\isamarkupfalse%
\ \isacommand{obtain}\isamarkupfalse%
\ k{\isacharprime}{\kern0pt}\ \isakeyword{where}\ k{\isacharprime}{\kern0pt}{\isacharunderscore}{\kern0pt}def{\isacharcolon}{\kern0pt}\ {\isachardoublequoteopen}k{\isacharprime}{\kern0pt}\ {\isasymin}\isactrlsub c\ Y\ {\isasymsetminus}\ {\isacharparenleft}{\kern0pt}{\isasymone}{\isacharcomma}{\kern0pt}y{\isadigit{1}}{\isacharparenright}{\kern0pt}\ {\isasymand}\ try{\isacharunderscore}{\kern0pt}cast\ y{\isadigit{1}}\ {\isasymcirc}\isactrlsub c\ c\ {\isacharequal}{\kern0pt}\ right{\isacharunderscore}{\kern0pt}coproj\ {\isasymone}\ {\isacharparenleft}{\kern0pt}Y\ {\isasymsetminus}\ {\isacharparenleft}{\kern0pt}{\isasymone}{\isacharcomma}{\kern0pt}y{\isadigit{1}}{\isacharparenright}{\kern0pt}{\isacharparenright}{\kern0pt}\ {\isasymcirc}\isactrlsub c\ k{\isacharprime}{\kern0pt}\ {\isasymand}\ \isanewline
\ \ \ \ \ \ \ \ \ \ \ \ \ \ m\ {\isasymcirc}\isactrlsub c\ right{\isacharunderscore}{\kern0pt}coproj\ X\ Y\ {\isasymcirc}\isactrlsub c\ c\ {\isacharequal}{\kern0pt}\ {\isasymlangle}x{\isadigit{1}}{\isacharcomma}{\kern0pt}\ y{\isadigit{1}}\isactrlsup c\ {\isasymcirc}\isactrlsub c\ k{\isacharprime}{\kern0pt}{\isasymrangle}{\isachardoublequoteclose}\isanewline
\ \ \ \ \ \ \ \ \ \ \ \ \ \ \ \ \isacommand{using}\isamarkupfalse%
\ c{\isacharunderscore}{\kern0pt}def\ m{\isacharunderscore}{\kern0pt}rightproj{\isacharunderscore}{\kern0pt}not{\isacharunderscore}{\kern0pt}y{\isadigit{1}}{\isacharunderscore}{\kern0pt}equals\ \isacommand{by}\isamarkupfalse%
\ blast\isanewline
\ \ \ \ \ \ \ \ \ \ \ \ \ \ \isacommand{then}\isamarkupfalse%
\ \isacommand{have}\isamarkupfalse%
\ {\isachardoublequoteopen}{\isasymlangle}x{\isadigit{1}}{\isacharcomma}{\kern0pt}\ y{\isadigit{1}}\isactrlsup c\ {\isasymcirc}\isactrlsub c\ k{\isacharprime}{\kern0pt}{\isasymrangle}\ {\isacharequal}{\kern0pt}\ {\isasymlangle}x{\isadigit{1}}{\isacharcomma}{\kern0pt}\ y{\isadigit{1}}\isactrlsup c\ {\isasymcirc}\isactrlsub c\ k{\isasymrangle}{\isachardoublequoteclose}\isanewline
\ \ \ \ \ \ \ \ \ \ \ \ \ \ \ \ \isacommand{using}\isamarkupfalse%
\ c{\isacharunderscore}{\kern0pt}def\ eqs\ k{\isacharunderscore}{\kern0pt}def\ y{\isacharunderscore}{\kern0pt}def\ \isacommand{by}\isamarkupfalse%
\ auto\isanewline
\ \ \ \ \ \ \ \ \ \ \ \ \ \ \isacommand{then}\isamarkupfalse%
\ \isacommand{have}\isamarkupfalse%
\ {\isachardoublequoteopen}{\isacharparenleft}{\kern0pt}x{\isadigit{1}}\ {\isacharequal}{\kern0pt}\ x{\isadigit{1}}{\isacharparenright}{\kern0pt}\ {\isasymand}\ {\isacharparenleft}{\kern0pt}y{\isadigit{1}}\isactrlsup c\ {\isasymcirc}\isactrlsub c\ k{\isacharprime}{\kern0pt}\ {\isacharequal}{\kern0pt}\ y{\isadigit{1}}\isactrlsup c\ {\isasymcirc}\isactrlsub c\ k{\isacharparenright}{\kern0pt}{\isachardoublequoteclose}\isanewline
\ \ \ \ \ \ \ \ \ \ \ \ \ \ \ \ \isacommand{using}\isamarkupfalse%
\ \ element{\isacharunderscore}{\kern0pt}pair{\isacharunderscore}{\kern0pt}eq\ k{\isacharprime}{\kern0pt}{\isacharunderscore}{\kern0pt}def\ k{\isacharunderscore}{\kern0pt}def\ \isacommand{by}\isamarkupfalse%
\ {\isacharparenleft}{\kern0pt}typecheck{\isacharunderscore}{\kern0pt}cfuncs{\isacharcomma}{\kern0pt}\ blast{\isacharparenright}{\kern0pt}\isanewline
\ \ \ \ \ \ \ \ \ \ \ \ \ \ \isacommand{then}\isamarkupfalse%
\ \isacommand{have}\isamarkupfalse%
\ {\isachardoublequoteopen}k{\isacharprime}{\kern0pt}\ {\isacharequal}{\kern0pt}\ k{\isachardoublequoteclose}\isanewline
\ \ \ \ \ \ \ \ \ \ \ \ \ \ \ \ \isacommand{by}\isamarkupfalse%
\ {\isacharparenleft}{\kern0pt}metis\ cfunc{\isacharunderscore}{\kern0pt}type{\isacharunderscore}{\kern0pt}def\ complement{\isacharunderscore}{\kern0pt}morphism{\isacharunderscore}{\kern0pt}mono\ k{\isacharprime}{\kern0pt}{\isacharunderscore}{\kern0pt}def\ k{\isacharunderscore}{\kern0pt}def\ monomorphism{\isacharunderscore}{\kern0pt}def\ y{\isadigit{1}}{\isacharprime}{\kern0pt}{\isacharunderscore}{\kern0pt}type\ y{\isadigit{1}}{\isacharunderscore}{\kern0pt}mono{\isacharparenright}{\kern0pt}\isanewline
\ \ \ \ \ \ \ \ \ \ \ \ \ \ \isacommand{then}\isamarkupfalse%
\ \isacommand{have}\isamarkupfalse%
\ {\isachardoublequoteopen}c\ {\isacharequal}{\kern0pt}\ y{\isachardoublequoteclose}\isanewline
\ \ \ \ \ \ \ \ \ \ \ \ \ \ \ \ \isacommand{by}\isamarkupfalse%
\ {\isacharparenleft}{\kern0pt}metis\ c{\isacharunderscore}{\kern0pt}def\ cfunc{\isacharunderscore}{\kern0pt}type{\isacharunderscore}{\kern0pt}def\ k{\isacharprime}{\kern0pt}{\isacharunderscore}{\kern0pt}def\ k{\isacharunderscore}{\kern0pt}def\ monomorphism{\isacharunderscore}{\kern0pt}def\ try{\isacharunderscore}{\kern0pt}cast{\isacharunderscore}{\kern0pt}mono\ trycast{\isacharunderscore}{\kern0pt}y{\isadigit{1}}{\isacharunderscore}{\kern0pt}type\ y{\isadigit{1}}{\isacharunderscore}{\kern0pt}mono\ y{\isacharunderscore}{\kern0pt}def{\isacharparenright}{\kern0pt}\isanewline
\ \ \ \ \ \ \ \ \ \ \ \ \ \ \isacommand{then}\isamarkupfalse%
\ \isacommand{show}\isamarkupfalse%
\ {\isachardoublequoteopen}a\ {\isacharequal}{\kern0pt}\ b{\isachardoublequoteclose}\isanewline
\ \ \ \ \ \ \ \ \ \ \ \ \ \ \ \ \isacommand{by}\isamarkupfalse%
\ {\isacharparenleft}{\kern0pt}simp\ add{\isacharcolon}{\kern0pt}\ c{\isacharunderscore}{\kern0pt}def\ y{\isacharunderscore}{\kern0pt}def{\isacharparenright}{\kern0pt}\isanewline
\ \ \ \ \ \ \ \ \ \ \isacommand{qed}\isamarkupfalse%
\isanewline
\ \ \ \ \ \ \ \ \isacommand{next}\isamarkupfalse%
\isanewline
\ \ \ \ \ \ \ \ \ \ \ \ \isacommand{assume}\isamarkupfalse%
\ {\isachardoublequoteopen}{\isasymnexists}c{\isachardot}{\kern0pt}\ b\ {\isacharequal}{\kern0pt}\ right{\isacharunderscore}{\kern0pt}coproj\ X\ Y\ {\isasymcirc}\isactrlsub c\ c\ {\isasymand}\ c\ {\isasymin}\isactrlsub c\ Y{\isachardoublequoteclose}\isanewline
\ \ \ \ \ \ \ \ \ \ \ \ \isacommand{then}\isamarkupfalse%
\ \isacommand{obtain}\isamarkupfalse%
\ c\ \isakeyword{where}\ c{\isacharunderscore}{\kern0pt}def{\isacharcolon}{\kern0pt}\ \ {\isachardoublequoteopen}b\ {\isacharequal}{\kern0pt}\ left{\isacharunderscore}{\kern0pt}coproj\ X\ Y\ {\isasymcirc}\isactrlsub c\ c\ {\isasymand}\ c\ {\isasymin}\isactrlsub c\ X{\isachardoublequoteclose}\isanewline
\ \ \ \ \ \ \ \ \ \ \ \ \ \ \isacommand{using}\isamarkupfalse%
\ b{\isacharunderscore}{\kern0pt}type\ coprojs{\isacharunderscore}{\kern0pt}jointly{\isacharunderscore}{\kern0pt}surj\ \isacommand{by}\isamarkupfalse%
\ blast\isanewline
\ \ \ \ \ \ \ \ \ \ \ \ \isacommand{then}\isamarkupfalse%
\ \isacommand{have}\isamarkupfalse%
\ \ {\isachardoublequoteopen}m\ {\isasymcirc}\isactrlsub c\ left{\isacharunderscore}{\kern0pt}coproj\ X\ Y\ {\isasymcirc}\isactrlsub c\ c\ {\isacharequal}{\kern0pt}\ {\isasymlangle}c{\isacharcomma}{\kern0pt}\ y{\isadigit{1}}{\isasymrangle}{\isachardoublequoteclose}\isanewline
\ \ \ \ \ \ \ \ \ \ \ \ \ \ \isacommand{by}\isamarkupfalse%
\ {\isacharparenleft}{\kern0pt}simp\ add{\isacharcolon}{\kern0pt}\ m{\isacharunderscore}{\kern0pt}leftproj{\isacharunderscore}{\kern0pt}l{\isacharunderscore}{\kern0pt}equals{\isacharparenright}{\kern0pt}\ \ \ \ \ \ \isanewline
\ \ \ \ \ \ \ \ \ \ \ \ \isacommand{then}\isamarkupfalse%
\ \isacommand{have}\isamarkupfalse%
\ {\isachardoublequoteopen}{\isasymlangle}c{\isacharcomma}{\kern0pt}\ y{\isadigit{1}}{\isasymrangle}\ {\isacharequal}{\kern0pt}\ {\isasymlangle}x{\isadigit{1}}{\isacharcomma}{\kern0pt}\ y{\isadigit{1}}\isactrlsup c\ {\isasymcirc}\isactrlsub c\ k{\isasymrangle}{\isachardoublequoteclose}\isanewline
\ \ \ \ \ \ \ \ \ \ \ \ \ \ \ \ \isacommand{using}\isamarkupfalse%
\ {\isacartoucheopen}m\ {\isasymcirc}\isactrlsub c\ a\ {\isacharequal}{\kern0pt}\ {\isasymlangle}x{\isadigit{1}}{\isacharcomma}{\kern0pt}y{\isadigit{1}}\isactrlsup c\ {\isasymcirc}\isactrlsub c\ k{\isasymrangle}{\isacartoucheclose}\ {\isacartoucheopen}m\ {\isasymcirc}\isactrlsub c\ left{\isacharunderscore}{\kern0pt}coproj\ X\ Y\ {\isasymcirc}\isactrlsub c\ c\ {\isacharequal}{\kern0pt}\ {\isasymlangle}c{\isacharcomma}{\kern0pt}y{\isadigit{1}}{\isasymrangle}{\isacartoucheclose}\ c{\isacharunderscore}{\kern0pt}def\ eqs\ \isacommand{by}\isamarkupfalse%
\ auto\ \ \ \ \ \ \isanewline
\ \ \ \ \ \ \ \ \ \ \ \ \isacommand{then}\isamarkupfalse%
\ \isacommand{have}\isamarkupfalse%
\ {\isachardoublequoteopen}{\isacharparenleft}{\kern0pt}c\ {\isacharequal}{\kern0pt}\ x{\isadigit{1}}{\isacharparenright}{\kern0pt}\ {\isasymand}\ {\isacharparenleft}{\kern0pt}y{\isadigit{1}}\ {\isacharequal}{\kern0pt}\ y{\isadigit{1}}\isactrlsup c\ {\isasymcirc}\isactrlsub c\ k{\isacharparenright}{\kern0pt}{\isachardoublequoteclose}\isanewline
\ \ \ \ \ \ \ \ \ \ \ \ \ \ \isacommand{using}\isamarkupfalse%
\ c{\isacharunderscore}{\kern0pt}def\ cart{\isacharunderscore}{\kern0pt}prod{\isacharunderscore}{\kern0pt}eq{\isadigit{2}}\ comp{\isacharunderscore}{\kern0pt}type\ k{\isacharunderscore}{\kern0pt}def\ x{\isadigit{1}}x{\isadigit{2}}{\isacharunderscore}{\kern0pt}def{\isacharparenleft}{\kern0pt}{\isadigit{1}}{\isacharparenright}{\kern0pt}\ y{\isadigit{1}}{\isacharprime}{\kern0pt}{\isacharunderscore}{\kern0pt}type\ y{\isadigit{1}}y{\isadigit{2}}{\isacharunderscore}{\kern0pt}def{\isacharparenleft}{\kern0pt}{\isadigit{1}}{\isacharparenright}{\kern0pt}\ \isacommand{by}\isamarkupfalse%
\ auto\ \isanewline
\ \ \ \ \ \ \ \ \ \ \ \ \isacommand{then}\isamarkupfalse%
\ \isacommand{have}\isamarkupfalse%
\ False\isanewline
\ \ \ \ \ \ \ \ \ \ \ \ \ \ \isacommand{by}\isamarkupfalse%
\ {\isacharparenleft}{\kern0pt}metis\ cfunc{\isacharunderscore}{\kern0pt}type{\isacharunderscore}{\kern0pt}def\ complement{\isacharunderscore}{\kern0pt}disjoint\ id{\isacharunderscore}{\kern0pt}right{\isacharunderscore}{\kern0pt}unit\ id{\isacharunderscore}{\kern0pt}type\ k{\isacharunderscore}{\kern0pt}def\ y{\isadigit{1}}{\isacharunderscore}{\kern0pt}mono\ y{\isadigit{1}}y{\isadigit{2}}{\isacharunderscore}{\kern0pt}def{\isacharparenleft}{\kern0pt}{\isadigit{1}}{\isacharparenright}{\kern0pt}{\isacharparenright}{\kern0pt}\isanewline
\ \ \ \ \ \ \ \ \ \ \ \ \isacommand{then}\isamarkupfalse%
\ \isacommand{show}\isamarkupfalse%
\ {\isacharquery}{\kern0pt}thesis\isanewline
\ \ \ \ \ \ \ \ \ \ \ \ \ \ \isacommand{by}\isamarkupfalse%
\ simp\isanewline
\ \ \ \ \ \ \ \ \isacommand{qed}\isamarkupfalse%
\isanewline
\ \ \ \ \ \ \isacommand{qed}\isamarkupfalse%
\isanewline
\ \ \ \ \isacommand{qed}\isamarkupfalse%
\isanewline
\ \ \isacommand{qed}\isamarkupfalse%
\isanewline
\ \ \isacommand{then}\isamarkupfalse%
\ \isacommand{have}\isamarkupfalse%
\ {\isachardoublequoteopen}monomorphism\ m{\isachardoublequoteclose}\isanewline
\ \ \ \ \isacommand{using}\isamarkupfalse%
\ injective{\isacharunderscore}{\kern0pt}imp{\isacharunderscore}{\kern0pt}monomorphism\ \isacommand{by}\isamarkupfalse%
\ auto\ \isanewline
\ \ \isacommand{then}\isamarkupfalse%
\ \isacommand{show}\isamarkupfalse%
\ {\isacharquery}{\kern0pt}thesis\isanewline
\ \ \ \ \isacommand{using}\isamarkupfalse%
\ is{\isacharunderscore}{\kern0pt}smaller{\isacharunderscore}{\kern0pt}than{\isacharunderscore}{\kern0pt}def\ m{\isacharunderscore}{\kern0pt}type\ \isacommand{by}\isamarkupfalse%
\ blast\isanewline
\isacommand{qed}\isamarkupfalse%
%
\endisatagproof
{\isafoldproof}%
%
\isadelimproof
\isanewline
%
\endisadelimproof
\isanewline
\isacommand{lemma}\isamarkupfalse%
\ prod{\isacharunderscore}{\kern0pt}leq{\isacharunderscore}{\kern0pt}exp{\isacharcolon}{\kern0pt}\isanewline
\ \ \isakeyword{assumes}\ {\isachardoublequoteopen}{\isasymnot}\ terminal{\isacharunderscore}{\kern0pt}object\ Y{\isachardoublequoteclose}\isanewline
\ \ \isakeyword{shows}\ {\isachardoublequoteopen}X\ {\isasymtimes}\isactrlsub c\ Y\ {\isasymle}\isactrlsub c\ Y\isactrlbsup X\isactrlesup {\isachardoublequoteclose}\isanewline
%
\isadelimproof
%
\endisadelimproof
%
\isatagproof
\isacommand{proof}\isamarkupfalse%
{\isacharparenleft}{\kern0pt}cases\ {\isachardoublequoteopen}initial{\isacharunderscore}{\kern0pt}object\ Y{\isachardoublequoteclose}{\isacharparenright}{\kern0pt}\isanewline
\ \ \isacommand{show}\isamarkupfalse%
\ {\isachardoublequoteopen}initial{\isacharunderscore}{\kern0pt}object\ Y\ {\isasymLongrightarrow}\ X\ {\isasymtimes}\isactrlsub c\ Y\ {\isasymle}\isactrlsub c\ Y\isactrlbsup X\isactrlesup {\isachardoublequoteclose}\isanewline
\ \ \ \ \isacommand{by}\isamarkupfalse%
\ {\isacharparenleft}{\kern0pt}metis\ X{\isacharunderscore}{\kern0pt}prod{\isacharunderscore}{\kern0pt}empty\ initial{\isacharunderscore}{\kern0pt}iso{\isacharunderscore}{\kern0pt}empty\ initial{\isacharunderscore}{\kern0pt}maps{\isacharunderscore}{\kern0pt}mono\ initial{\isacharunderscore}{\kern0pt}object{\isacharunderscore}{\kern0pt}def\ is{\isacharunderscore}{\kern0pt}smaller{\isacharunderscore}{\kern0pt}than{\isacharunderscore}{\kern0pt}def\ iso{\isacharunderscore}{\kern0pt}empty{\isacharunderscore}{\kern0pt}initial\ isomorphic{\isacharunderscore}{\kern0pt}is{\isacharunderscore}{\kern0pt}reflexive\ isomorphic{\isacharunderscore}{\kern0pt}is{\isacharunderscore}{\kern0pt}transitive\ prod{\isacharunderscore}{\kern0pt}pres{\isacharunderscore}{\kern0pt}iso{\isacharparenright}{\kern0pt}\isanewline
\isacommand{next}\isamarkupfalse%
\isanewline
\ \ \isacommand{assume}\isamarkupfalse%
\ {\isachardoublequoteopen}{\isasymnot}\ initial{\isacharunderscore}{\kern0pt}object\ Y{\isachardoublequoteclose}\isanewline
\ \ \isacommand{then}\isamarkupfalse%
\ \isacommand{obtain}\isamarkupfalse%
\ y{\isadigit{1}}\ y{\isadigit{2}}\ \isakeyword{where}\ y{\isadigit{1}}{\isacharunderscore}{\kern0pt}type{\isacharbrackleft}{\kern0pt}type{\isacharunderscore}{\kern0pt}rule{\isacharbrackright}{\kern0pt}{\isacharcolon}{\kern0pt}\ {\isachardoublequoteopen}y{\isadigit{1}}\ {\isasymin}\isactrlsub c\ Y{\isachardoublequoteclose}\ \isakeyword{and}\ y{\isadigit{2}}{\isacharunderscore}{\kern0pt}type{\isacharbrackleft}{\kern0pt}type{\isacharunderscore}{\kern0pt}rule{\isacharbrackright}{\kern0pt}{\isacharcolon}{\kern0pt}\ {\isachardoublequoteopen}y{\isadigit{2}}\ {\isasymin}\isactrlsub c\ Y{\isachardoublequoteclose}\ \isakeyword{and}\ y{\isadigit{1}}{\isacharunderscore}{\kern0pt}not{\isacharunderscore}{\kern0pt}y{\isadigit{2}}{\isacharcolon}{\kern0pt}\ {\isachardoublequoteopen}y{\isadigit{1}}\ {\isasymnoteq}\ y{\isadigit{2}}{\isachardoublequoteclose}\isanewline
\ \ \ \ \isacommand{using}\isamarkupfalse%
\ assms\ not{\isacharunderscore}{\kern0pt}init{\isacharunderscore}{\kern0pt}not{\isacharunderscore}{\kern0pt}term\ \isacommand{by}\isamarkupfalse%
\ blast\isanewline
\ \ \isacommand{show}\isamarkupfalse%
\ {\isachardoublequoteopen}X\ {\isasymtimes}\isactrlsub c\ Y\ {\isasymle}\isactrlsub c\ Y\isactrlbsup X\isactrlesup {\isachardoublequoteclose}\isanewline
\ \ \isacommand{proof}\isamarkupfalse%
{\isacharparenleft}{\kern0pt}cases\ {\isachardoublequoteopen}X\ {\isasymcong}\ {\isasymOmega}{\isachardoublequoteclose}{\isacharparenright}{\kern0pt}\isanewline
\ \ \ \ \ \ \isacommand{assume}\isamarkupfalse%
\ {\isachardoublequoteopen}X\ {\isasymcong}\ {\isasymOmega}{\isachardoublequoteclose}\isanewline
\ \ \ \ \ \ \isacommand{have}\isamarkupfalse%
\ {\isachardoublequoteopen}{\isasymOmega}\ \ {\isasymle}\isactrlsub c\ \ Y{\isachardoublequoteclose}\isanewline
\ \ \ \ \ \ \ \ \ \isacommand{using}\isamarkupfalse%
\ {\isacartoucheopen}{\isasymnot}\ initial{\isacharunderscore}{\kern0pt}object\ Y{\isacartoucheclose}\ assms\ not{\isacharunderscore}{\kern0pt}init{\isacharunderscore}{\kern0pt}not{\isacharunderscore}{\kern0pt}term\ size{\isacharunderscore}{\kern0pt}{\isadigit{2}}plus{\isacharunderscore}{\kern0pt}sets\ \isacommand{by}\isamarkupfalse%
\ blast\isanewline
\ \ \ \ \ \ \isacommand{then}\isamarkupfalse%
\ \isacommand{obtain}\isamarkupfalse%
\ m\ \isakeyword{where}\ m{\isacharunderscore}{\kern0pt}type{\isacharbrackleft}{\kern0pt}type{\isacharunderscore}{\kern0pt}rule{\isacharbrackright}{\kern0pt}{\isacharcolon}{\kern0pt}\ {\isachardoublequoteopen}m\ {\isacharcolon}{\kern0pt}\ {\isasymOmega}\ \ {\isasymrightarrow}\ \ Y{\isachardoublequoteclose}\ \isakeyword{and}\ m{\isacharunderscore}{\kern0pt}mono{\isacharcolon}{\kern0pt}\ {\isachardoublequoteopen}monomorphism\ m{\isachardoublequoteclose}\isanewline
\ \ \ \ \ \ \ \ \isacommand{using}\isamarkupfalse%
\ is{\isacharunderscore}{\kern0pt}smaller{\isacharunderscore}{\kern0pt}than{\isacharunderscore}{\kern0pt}def\ \isacommand{by}\isamarkupfalse%
\ blast\isanewline
\ \ \ \ \ \ \isacommand{then}\isamarkupfalse%
\ \isacommand{have}\isamarkupfalse%
\ m{\isacharunderscore}{\kern0pt}id{\isacharunderscore}{\kern0pt}type{\isacharbrackleft}{\kern0pt}type{\isacharunderscore}{\kern0pt}rule{\isacharbrackright}{\kern0pt}{\isacharcolon}{\kern0pt}\ {\isachardoublequoteopen}m\ {\isasymtimes}\isactrlsub f\ id{\isacharparenleft}{\kern0pt}Y{\isacharparenright}{\kern0pt}\ {\isacharcolon}{\kern0pt}\ {\isasymOmega}\ {\isasymtimes}\isactrlsub c\ Y\ {\isasymrightarrow}\ Y\ {\isasymtimes}\isactrlsub c\ Y{\isachardoublequoteclose}\isanewline
\ \ \ \ \ \ \ \ \isacommand{by}\isamarkupfalse%
\ typecheck{\isacharunderscore}{\kern0pt}cfuncs\isanewline
\ \ \ \ \ \ \isacommand{have}\isamarkupfalse%
\ m{\isacharunderscore}{\kern0pt}id{\isacharunderscore}{\kern0pt}mono{\isacharcolon}{\kern0pt}\ {\isachardoublequoteopen}monomorphism\ {\isacharparenleft}{\kern0pt}m\ {\isasymtimes}\isactrlsub f\ id{\isacharparenleft}{\kern0pt}Y{\isacharparenright}{\kern0pt}{\isacharparenright}{\kern0pt}{\isachardoublequoteclose}\isanewline
\ \ \ \ \ \ \ \ \isacommand{by}\isamarkupfalse%
\ {\isacharparenleft}{\kern0pt}typecheck{\isacharunderscore}{\kern0pt}cfuncs{\isacharcomma}{\kern0pt}\ simp\ add{\isacharcolon}{\kern0pt}\ cfunc{\isacharunderscore}{\kern0pt}cross{\isacharunderscore}{\kern0pt}prod{\isacharunderscore}{\kern0pt}mono\ id{\isacharunderscore}{\kern0pt}isomorphism\ iso{\isacharunderscore}{\kern0pt}imp{\isacharunderscore}{\kern0pt}epi{\isacharunderscore}{\kern0pt}and{\isacharunderscore}{\kern0pt}monic\ m{\isacharunderscore}{\kern0pt}mono{\isacharparenright}{\kern0pt}\ \ \isanewline
\ \ \ \ \ \ \isacommand{obtain}\isamarkupfalse%
\ n\ \isakeyword{where}\ n{\isacharunderscore}{\kern0pt}type{\isacharbrackleft}{\kern0pt}type{\isacharunderscore}{\kern0pt}rule{\isacharbrackright}{\kern0pt}{\isacharcolon}{\kern0pt}\ {\isachardoublequoteopen}n\ {\isacharcolon}{\kern0pt}\ Y\ {\isasymtimes}\isactrlsub c\ Y\ \ {\isasymrightarrow}\ \ Y\isactrlbsup {\isasymOmega}\isactrlesup {\isachardoublequoteclose}\ \isakeyword{and}\ n{\isacharunderscore}{\kern0pt}mono{\isacharcolon}{\kern0pt}\ {\isachardoublequoteopen}monomorphism\ n{\isachardoublequoteclose}\isanewline
\ \ \ \ \ \ \ \ \isacommand{using}\isamarkupfalse%
\ is{\isacharunderscore}{\kern0pt}isomorphic{\isacharunderscore}{\kern0pt}def\ iso{\isacharunderscore}{\kern0pt}imp{\isacharunderscore}{\kern0pt}epi{\isacharunderscore}{\kern0pt}and{\isacharunderscore}{\kern0pt}monic\ isomorphic{\isacharunderscore}{\kern0pt}is{\isacharunderscore}{\kern0pt}symmetric\ sets{\isacharunderscore}{\kern0pt}squared\ \isacommand{by}\isamarkupfalse%
\ blast\isanewline
\ \ \ \ \ \ \isacommand{obtain}\isamarkupfalse%
\ r\ \isakeyword{where}\ r{\isacharunderscore}{\kern0pt}type{\isacharbrackleft}{\kern0pt}type{\isacharunderscore}{\kern0pt}rule{\isacharbrackright}{\kern0pt}{\isacharcolon}{\kern0pt}\ {\isachardoublequoteopen}r\ {\isacharcolon}{\kern0pt}\ Y\isactrlbsup {\isasymOmega}\isactrlesup \ \ {\isasymrightarrow}\ \ Y\isactrlbsup X\isactrlesup {\isachardoublequoteclose}\ \isakeyword{and}\ r{\isacharunderscore}{\kern0pt}mono{\isacharcolon}{\kern0pt}\ {\isachardoublequoteopen}monomorphism\ r{\isachardoublequoteclose}\isanewline
\ \ \ \ \ \ \ \ \isacommand{by}\isamarkupfalse%
\ {\isacharparenleft}{\kern0pt}meson\ {\isacartoucheopen}X\ {\isasymcong}\ {\isasymOmega}{\isacartoucheclose}\ exp{\isacharunderscore}{\kern0pt}pres{\isacharunderscore}{\kern0pt}iso{\isacharunderscore}{\kern0pt}right\ is{\isacharunderscore}{\kern0pt}isomorphic{\isacharunderscore}{\kern0pt}def\ iso{\isacharunderscore}{\kern0pt}imp{\isacharunderscore}{\kern0pt}epi{\isacharunderscore}{\kern0pt}and{\isacharunderscore}{\kern0pt}monic\ isomorphic{\isacharunderscore}{\kern0pt}is{\isacharunderscore}{\kern0pt}symmetric{\isacharparenright}{\kern0pt}\isanewline
\ \ \ \ \ \ \isacommand{obtain}\isamarkupfalse%
\ q\ \isakeyword{where}\ q{\isacharunderscore}{\kern0pt}type{\isacharbrackleft}{\kern0pt}type{\isacharunderscore}{\kern0pt}rule{\isacharbrackright}{\kern0pt}{\isacharcolon}{\kern0pt}\ {\isachardoublequoteopen}q\ {\isacharcolon}{\kern0pt}\ X\ {\isasymtimes}\isactrlsub c\ Y\ \ {\isasymrightarrow}\ \ {\isasymOmega}\ {\isasymtimes}\isactrlsub c\ Y{\isachardoublequoteclose}\ \isakeyword{and}\ q{\isacharunderscore}{\kern0pt}mono{\isacharcolon}{\kern0pt}\ {\isachardoublequoteopen}monomorphism\ q{\isachardoublequoteclose}\isanewline
\ \ \ \ \ \ \ \ \isacommand{by}\isamarkupfalse%
\ {\isacharparenleft}{\kern0pt}meson\ {\isacartoucheopen}X\ {\isasymcong}\ {\isasymOmega}{\isacartoucheclose}\ id{\isacharunderscore}{\kern0pt}isomorphism\ id{\isacharunderscore}{\kern0pt}type\ is{\isacharunderscore}{\kern0pt}isomorphic{\isacharunderscore}{\kern0pt}def\ iso{\isacharunderscore}{\kern0pt}imp{\isacharunderscore}{\kern0pt}epi{\isacharunderscore}{\kern0pt}and{\isacharunderscore}{\kern0pt}monic\ prod{\isacharunderscore}{\kern0pt}pres{\isacharunderscore}{\kern0pt}iso{\isacharparenright}{\kern0pt}\ \isanewline
\ \ \ \ \ \ \isacommand{have}\isamarkupfalse%
\ rnmq{\isacharunderscore}{\kern0pt}type{\isacharbrackleft}{\kern0pt}type{\isacharunderscore}{\kern0pt}rule{\isacharbrackright}{\kern0pt}{\isacharcolon}{\kern0pt}\ {\isachardoublequoteopen}r\ {\isasymcirc}\isactrlsub c\ n\ {\isasymcirc}\isactrlsub c\ {\isacharparenleft}{\kern0pt}m\ {\isasymtimes}\isactrlsub f\ id{\isacharparenleft}{\kern0pt}Y{\isacharparenright}{\kern0pt}{\isacharparenright}{\kern0pt}\ {\isasymcirc}\isactrlsub c\ q\ {\isacharcolon}{\kern0pt}\ X\ {\isasymtimes}\isactrlsub c\ Y\ {\isasymrightarrow}\ Y\isactrlbsup X\isactrlesup {\isachardoublequoteclose}\isanewline
\ \ \ \ \ \ \ \ \isacommand{by}\isamarkupfalse%
\ typecheck{\isacharunderscore}{\kern0pt}cfuncs\isanewline
\ \ \ \ \ \ \isacommand{have}\isamarkupfalse%
\ {\isachardoublequoteopen}monomorphism{\isacharparenleft}{\kern0pt}r\ {\isasymcirc}\isactrlsub c\ n\ {\isasymcirc}\isactrlsub c\ {\isacharparenleft}{\kern0pt}m\ {\isasymtimes}\isactrlsub f\ id{\isacharparenleft}{\kern0pt}Y{\isacharparenright}{\kern0pt}{\isacharparenright}{\kern0pt}\ {\isasymcirc}\isactrlsub c\ q{\isacharparenright}{\kern0pt}{\isachardoublequoteclose}\isanewline
\ \ \ \ \ \ \ \ \isacommand{by}\isamarkupfalse%
\ {\isacharparenleft}{\kern0pt}typecheck{\isacharunderscore}{\kern0pt}cfuncs{\isacharcomma}{\kern0pt}\ simp\ add{\isacharcolon}{\kern0pt}\ cfunc{\isacharunderscore}{\kern0pt}type{\isacharunderscore}{\kern0pt}def\ composition{\isacharunderscore}{\kern0pt}of{\isacharunderscore}{\kern0pt}monic{\isacharunderscore}{\kern0pt}pair{\isacharunderscore}{\kern0pt}is{\isacharunderscore}{\kern0pt}monic\ m{\isacharunderscore}{\kern0pt}id{\isacharunderscore}{\kern0pt}mono\ n{\isacharunderscore}{\kern0pt}mono\ q{\isacharunderscore}{\kern0pt}mono\ r{\isacharunderscore}{\kern0pt}mono{\isacharparenright}{\kern0pt}\isanewline
\ \ \ \ \ \ \isacommand{then}\isamarkupfalse%
\ \isacommand{show}\isamarkupfalse%
\ {\isacharquery}{\kern0pt}thesis\isanewline
\ \ \ \ \ \ \ \ \isacommand{by}\isamarkupfalse%
\ {\isacharparenleft}{\kern0pt}meson\ is{\isacharunderscore}{\kern0pt}smaller{\isacharunderscore}{\kern0pt}than{\isacharunderscore}{\kern0pt}def\ rnmq{\isacharunderscore}{\kern0pt}type{\isacharparenright}{\kern0pt}\isanewline
\ \ \ \ \isacommand{next}\isamarkupfalse%
\isanewline
\ \ \ \ \ \ \isacommand{assume}\isamarkupfalse%
\ {\isachardoublequoteopen}{\isasymnot}\ X\ {\isasymcong}\ {\isasymOmega}{\isachardoublequoteclose}\isanewline
\ \ \ \ \ \ \isacommand{show}\isamarkupfalse%
\ {\isachardoublequoteopen}X\ {\isasymtimes}\isactrlsub c\ Y\ {\isasymle}\isactrlsub c\ Y\isactrlbsup X\isactrlesup {\isachardoublequoteclose}\isanewline
\ \ \ \ \ \ \isacommand{proof}\isamarkupfalse%
{\isacharparenleft}{\kern0pt}cases\ {\isachardoublequoteopen}initial{\isacharunderscore}{\kern0pt}object\ X{\isachardoublequoteclose}{\isacharparenright}{\kern0pt}\isanewline
\ \ \ \ \ \ \ \ \isacommand{show}\isamarkupfalse%
\ {\isachardoublequoteopen}initial{\isacharunderscore}{\kern0pt}object\ X\ {\isasymLongrightarrow}\ X\ {\isasymtimes}\isactrlsub c\ Y\ {\isasymle}\isactrlsub c\ Y\isactrlbsup X\isactrlesup {\isachardoublequoteclose}\isanewline
\ \ \ \ \ \ \ \ \ \ \isacommand{by}\isamarkupfalse%
\ {\isacharparenleft}{\kern0pt}metis\ is{\isacharunderscore}{\kern0pt}empty{\isacharunderscore}{\kern0pt}def\ initial{\isacharunderscore}{\kern0pt}iso{\isacharunderscore}{\kern0pt}empty\ initial{\isacharunderscore}{\kern0pt}maps{\isacharunderscore}{\kern0pt}mono\ initial{\isacharunderscore}{\kern0pt}object{\isacharunderscore}{\kern0pt}def\ \isanewline
\ \ \ \ \ \ \ \ \ \ \ \ \ \ is{\isacharunderscore}{\kern0pt}smaller{\isacharunderscore}{\kern0pt}than{\isacharunderscore}{\kern0pt}def\ isomorphic{\isacharunderscore}{\kern0pt}is{\isacharunderscore}{\kern0pt}transitive\ no{\isacharunderscore}{\kern0pt}el{\isacharunderscore}{\kern0pt}iff{\isacharunderscore}{\kern0pt}iso{\isacharunderscore}{\kern0pt}empty\isanewline
\ \ \ \ \ \ \ \ \ \ \ \ \ \ not{\isacharunderscore}{\kern0pt}init{\isacharunderscore}{\kern0pt}not{\isacharunderscore}{\kern0pt}term\ prod{\isacharunderscore}{\kern0pt}with{\isacharunderscore}{\kern0pt}empty{\isacharunderscore}{\kern0pt}is{\isacharunderscore}{\kern0pt}empty{\isadigit{2}}\ product{\isacharunderscore}{\kern0pt}commutes\ terminal{\isacharunderscore}{\kern0pt}object{\isacharunderscore}{\kern0pt}def{\isacharparenright}{\kern0pt}\isanewline
\ \ \ \ \ \ \isacommand{next}\isamarkupfalse%
\isanewline
\ \ \ \ \ \ \isacommand{assume}\isamarkupfalse%
\ {\isachardoublequoteopen}{\isasymnot}\ initial{\isacharunderscore}{\kern0pt}object\ X{\isachardoublequoteclose}\isanewline
\ \ \ \ \ \ \isacommand{show}\isamarkupfalse%
\ {\isachardoublequoteopen}X\ {\isasymtimes}\isactrlsub c\ Y\ {\isasymle}\isactrlsub c\ Y\isactrlbsup X\isactrlesup {\isachardoublequoteclose}\isanewline
\ \ \ \ \ \ \isacommand{proof}\isamarkupfalse%
{\isacharparenleft}{\kern0pt}cases\ {\isachardoublequoteopen}terminal{\isacharunderscore}{\kern0pt}object\ X{\isachardoublequoteclose}{\isacharparenright}{\kern0pt}\isanewline
\ \ \ \ \ \ \ \ \isacommand{assume}\isamarkupfalse%
\ {\isachardoublequoteopen}terminal{\isacharunderscore}{\kern0pt}object\ X{\isachardoublequoteclose}\isanewline
\ \ \ \ \ \ \ \ \isacommand{then}\isamarkupfalse%
\ \isacommand{have}\isamarkupfalse%
\ {\isachardoublequoteopen}X\ {\isasymcong}\ {\isasymone}{\isachardoublequoteclose}\isanewline
\ \ \ \ \ \ \ \ \ \ \isacommand{by}\isamarkupfalse%
\ {\isacharparenleft}{\kern0pt}simp\ add{\isacharcolon}{\kern0pt}\ one{\isacharunderscore}{\kern0pt}terminal{\isacharunderscore}{\kern0pt}object\ terminal{\isacharunderscore}{\kern0pt}objects{\isacharunderscore}{\kern0pt}isomorphic{\isacharparenright}{\kern0pt}\isanewline
\ \ \ \ \ \ \ \ \isacommand{have}\isamarkupfalse%
\ {\isachardoublequoteopen}X\ {\isasymtimes}\isactrlsub c\ Y\ {\isasymcong}\ Y{\isachardoublequoteclose}\isanewline
\ \ \ \ \ \ \ \ \ \ \isacommand{by}\isamarkupfalse%
\ {\isacharparenleft}{\kern0pt}simp\ add{\isacharcolon}{\kern0pt}\ {\isacartoucheopen}terminal{\isacharunderscore}{\kern0pt}object\ X{\isacartoucheclose}\ prod{\isacharunderscore}{\kern0pt}with{\isacharunderscore}{\kern0pt}term{\isacharunderscore}{\kern0pt}obj{\isadigit{1}}{\isacharparenright}{\kern0pt}\isanewline
\ \ \ \ \ \ \ \ \isacommand{then}\isamarkupfalse%
\ \isacommand{have}\isamarkupfalse%
\ {\isachardoublequoteopen}X\ {\isasymtimes}\isactrlsub c\ Y\ {\isasymcong}\ Y\isactrlbsup X\isactrlesup {\isachardoublequoteclose}\isanewline
\ \ \ \ \ \ \ \ \ \ \isacommand{by}\isamarkupfalse%
\ {\isacharparenleft}{\kern0pt}meson\ {\isacartoucheopen}X\ {\isasymcong}\ {\isasymone}{\isacartoucheclose}\ exp{\isacharunderscore}{\kern0pt}pres{\isacharunderscore}{\kern0pt}iso{\isacharunderscore}{\kern0pt}right\ exp{\isacharunderscore}{\kern0pt}set{\isacharunderscore}{\kern0pt}inj\ isomorphic{\isacharunderscore}{\kern0pt}is{\isacharunderscore}{\kern0pt}symmetric\ isomorphic{\isacharunderscore}{\kern0pt}is{\isacharunderscore}{\kern0pt}transitive\ exp{\isacharunderscore}{\kern0pt}one{\isacharparenright}{\kern0pt}\isanewline
\ \ \ \ \ \ \ \ \isacommand{then}\isamarkupfalse%
\ \isacommand{show}\isamarkupfalse%
\ {\isacharquery}{\kern0pt}thesis\isanewline
\ \ \ \ \ \ \ \ \ \ \isacommand{using}\isamarkupfalse%
\ is{\isacharunderscore}{\kern0pt}isomorphic{\isacharunderscore}{\kern0pt}def\ is{\isacharunderscore}{\kern0pt}smaller{\isacharunderscore}{\kern0pt}than{\isacharunderscore}{\kern0pt}def\ iso{\isacharunderscore}{\kern0pt}imp{\isacharunderscore}{\kern0pt}epi{\isacharunderscore}{\kern0pt}and{\isacharunderscore}{\kern0pt}monic\ \isacommand{by}\isamarkupfalse%
\ blast\isanewline
\ \ \ \ \ \ \isacommand{next}\isamarkupfalse%
\isanewline
\ \ \ \ \ \ \ \ \isacommand{assume}\isamarkupfalse%
\ {\isachardoublequoteopen}{\isasymnot}\ terminal{\isacharunderscore}{\kern0pt}object\ X{\isachardoublequoteclose}\isanewline
\isanewline
\ \ \ \ \ \ \ \ \isacommand{obtain}\isamarkupfalse%
\ into\ \isakeyword{where}\ into{\isacharunderscore}{\kern0pt}def{\isacharcolon}{\kern0pt}\ {\isachardoublequoteopen}into\ {\isacharequal}{\kern0pt}\ {\isacharparenleft}{\kern0pt}left{\isacharunderscore}{\kern0pt}cart{\isacharunderscore}{\kern0pt}proj\ Y\ {\isasymone}\ {\isasymamalg}\ {\isacharparenleft}{\kern0pt}{\isacharparenleft}{\kern0pt}y{\isadigit{2}}\ {\isasymamalg}\ y{\isadigit{1}}{\isacharparenright}{\kern0pt}\ {\isasymcirc}\isactrlsub c\ case{\isacharunderscore}{\kern0pt}bool\ {\isasymcirc}\isactrlsub c\ eq{\isacharunderscore}{\kern0pt}pred\ Y\ {\isasymcirc}\isactrlsub c\ {\isacharparenleft}{\kern0pt}id\ Y\ {\isasymtimes}\isactrlsub f\ y{\isadigit{1}}{\isacharparenright}{\kern0pt}{\isacharparenright}{\kern0pt}{\isacharparenright}{\kern0pt}\ \isanewline
\ \ \ \ \ \ \ \ \ \ \ \ \ \ \ \ \ \ \ \ \ \ \ \ \ \ \ \ \ \ \ {\isasymcirc}\isactrlsub c\ dist{\isacharunderscore}{\kern0pt}prod{\isacharunderscore}{\kern0pt}coprod{\isacharunderscore}{\kern0pt}left\ Y\ {\isasymone}\ {\isasymone}\ {\isasymcirc}\isactrlsub c\ {\isacharparenleft}{\kern0pt}id\ Y\ {\isasymtimes}\isactrlsub f\ case{\isacharunderscore}{\kern0pt}bool{\isacharparenright}{\kern0pt}\ {\isasymcirc}\isactrlsub c\ {\isacharparenleft}{\kern0pt}id\ Y\ {\isasymtimes}\isactrlsub f\ eq{\isacharunderscore}{\kern0pt}pred\ X{\isacharparenright}{\kern0pt}\ {\isachardoublequoteclose}\isanewline
\ \ \ \ \ \ \ \ \ \ \isacommand{by}\isamarkupfalse%
\ simp\isanewline
\ \ \ \ \ \ \ \ \isacommand{then}\isamarkupfalse%
\ \isacommand{have}\isamarkupfalse%
\ into{\isacharunderscore}{\kern0pt}type{\isacharbrackleft}{\kern0pt}type{\isacharunderscore}{\kern0pt}rule{\isacharbrackright}{\kern0pt}{\isacharcolon}{\kern0pt}\ {\isachardoublequoteopen}into\ {\isacharcolon}{\kern0pt}\ Y\ {\isasymtimes}\isactrlsub c\ {\isacharparenleft}{\kern0pt}X\ {\isasymtimes}\isactrlsub c\ X{\isacharparenright}{\kern0pt}\ {\isasymrightarrow}\ Y{\isachardoublequoteclose}\isanewline
\ \ \ \ \ \ \ \ \ \ \isacommand{by}\isamarkupfalse%
\ {\isacharparenleft}{\kern0pt}simp{\isacharcomma}{\kern0pt}\ typecheck{\isacharunderscore}{\kern0pt}cfuncs{\isacharparenright}{\kern0pt}\isanewline
\ \ \ \isanewline
\isanewline
\ \ \ \ \ \ \ \ \isacommand{obtain}\isamarkupfalse%
\ {\isasymTheta}\ \isakeyword{where}\ {\isasymTheta}{\isacharunderscore}{\kern0pt}def{\isacharcolon}{\kern0pt}\ {\isachardoublequoteopen}{\isasymTheta}\ {\isacharequal}{\kern0pt}\ {\isacharparenleft}{\kern0pt}into\ {\isasymcirc}\isactrlsub c\ associate{\isacharunderscore}{\kern0pt}right\ Y\ X\ X\ {\isasymcirc}\isactrlsub c\ swap\ X\ {\isacharparenleft}{\kern0pt}Y\ {\isasymtimes}\isactrlsub c\ X{\isacharparenright}{\kern0pt}{\isacharparenright}{\kern0pt}\isactrlsup {\isasymsharp}\ {\isasymcirc}\isactrlsub c\ swap\ X\ Y{\isachardoublequoteclose}\isanewline
\ \ \ \ \ \ \ \ \ \ \isacommand{by}\isamarkupfalse%
\ auto\isanewline
\ \ \isanewline
\ \ \ \ \ \ \ \ \isacommand{have}\isamarkupfalse%
\ {\isasymTheta}{\isacharunderscore}{\kern0pt}type{\isacharbrackleft}{\kern0pt}type{\isacharunderscore}{\kern0pt}rule{\isacharbrackright}{\kern0pt}{\isacharcolon}{\kern0pt}\ {\isachardoublequoteopen}{\isasymTheta}\ {\isacharcolon}{\kern0pt}\ X\ {\isasymtimes}\isactrlsub c\ Y\ {\isasymrightarrow}\ Y\isactrlbsup X\isactrlesup {\isachardoublequoteclose}\isanewline
\ \ \ \ \ \ \ \ \ \ \isacommand{unfolding}\isamarkupfalse%
\ {\isasymTheta}{\isacharunderscore}{\kern0pt}def\ \isacommand{by}\isamarkupfalse%
\ typecheck{\isacharunderscore}{\kern0pt}cfuncs\isanewline
\isanewline
\ \ \ \ \ \ \ \ \isacommand{have}\isamarkupfalse%
\ f{\isadigit{0}}{\isacharcolon}{\kern0pt}\ {\isachardoublequoteopen}{\isasymAnd}x{\isachardot}{\kern0pt}\ {\isasymAnd}\ y{\isachardot}{\kern0pt}\ {\isasymAnd}\ z{\isachardot}{\kern0pt}\ x\ {\isasymin}\isactrlsub c\ X\ {\isasymand}\ y\ {\isasymin}\isactrlsub c\ Y\ {\isasymand}\ z\ {\isasymin}\isactrlsub c\ X\ {\isasymLongrightarrow}\ {\isacharparenleft}{\kern0pt}{\isasymTheta}\ {\isasymcirc}\isactrlsub c\ {\isasymlangle}x{\isacharcomma}{\kern0pt}\ y{\isasymrangle}{\isacharparenright}{\kern0pt}\isactrlsup {\isasymflat}\ {\isasymcirc}\isactrlsub c\ {\isasymlangle}id\ X{\isacharcomma}{\kern0pt}\ {\isasymbeta}\isactrlbsub X\isactrlesub {\isasymrangle}\ {\isasymcirc}\isactrlsub c\ z\ {\isacharequal}{\kern0pt}\ into\ {\isasymcirc}\isactrlsub c\ \ \ {\isasymlangle}y{\isacharcomma}{\kern0pt}\ {\isasymlangle}x{\isacharcomma}{\kern0pt}\ z{\isasymrangle}{\isasymrangle}{\isachardoublequoteclose}\isanewline
\ \ \ \ \ \ \ \ \isacommand{proof}\isamarkupfalse%
{\isacharparenleft}{\kern0pt}clarify{\isacharparenright}{\kern0pt}\isanewline
\ \ \ \ \ \ \ \ \ \ \isacommand{fix}\isamarkupfalse%
\ x\ y\ z\isanewline
\ \ \ \ \ \ \ \ \ \ \isacommand{assume}\isamarkupfalse%
\ x{\isacharunderscore}{\kern0pt}type{\isacharbrackleft}{\kern0pt}type{\isacharunderscore}{\kern0pt}rule{\isacharbrackright}{\kern0pt}{\isacharcolon}{\kern0pt}\ {\isachardoublequoteopen}x\ {\isasymin}\isactrlsub c\ X{\isachardoublequoteclose}\isanewline
\ \ \ \ \ \ \ \ \ \ \isacommand{assume}\isamarkupfalse%
\ y{\isacharunderscore}{\kern0pt}type{\isacharbrackleft}{\kern0pt}type{\isacharunderscore}{\kern0pt}rule{\isacharbrackright}{\kern0pt}{\isacharcolon}{\kern0pt}\ {\isachardoublequoteopen}y\ {\isasymin}\isactrlsub c\ Y{\isachardoublequoteclose}\isanewline
\ \ \ \ \ \ \ \ \ \ \isacommand{assume}\isamarkupfalse%
\ z{\isacharunderscore}{\kern0pt}type{\isacharbrackleft}{\kern0pt}type{\isacharunderscore}{\kern0pt}rule{\isacharbrackright}{\kern0pt}{\isacharcolon}{\kern0pt}\ {\isachardoublequoteopen}z\ {\isasymin}\isactrlsub c\ X{\isachardoublequoteclose}\isanewline
\ \ \ \ \ \ \ \ \ \ \isacommand{show}\isamarkupfalse%
\ {\isachardoublequoteopen}{\isacharparenleft}{\kern0pt}{\isasymTheta}\ {\isasymcirc}\isactrlsub c\ {\isasymlangle}x{\isacharcomma}{\kern0pt}y{\isasymrangle}{\isacharparenright}{\kern0pt}\isactrlsup {\isasymflat}\ {\isasymcirc}\isactrlsub c\ {\isasymlangle}id\isactrlsub c\ X{\isacharcomma}{\kern0pt}{\isasymbeta}\isactrlbsub X\isactrlesub {\isasymrangle}\ {\isasymcirc}\isactrlsub c\ z\ {\isacharequal}{\kern0pt}\ into\ {\isasymcirc}\isactrlsub c\ {\isasymlangle}y{\isacharcomma}{\kern0pt}{\isasymlangle}x{\isacharcomma}{\kern0pt}z{\isasymrangle}{\isasymrangle}{\isachardoublequoteclose}\isanewline
\ \ \ \ \ \ \ \ \ \ \isacommand{proof}\isamarkupfalse%
\ {\isacharminus}{\kern0pt}\ \isanewline
\ \ \ \ \ \ \ \ \ \ \ \ \isacommand{have}\isamarkupfalse%
\ {\isachardoublequoteopen}{\isacharparenleft}{\kern0pt}{\isasymTheta}\ {\isasymcirc}\isactrlsub c\ {\isasymlangle}x{\isacharcomma}{\kern0pt}y{\isasymrangle}{\isacharparenright}{\kern0pt}\isactrlsup {\isasymflat}\ {\isasymcirc}\isactrlsub c\ {\isasymlangle}id\isactrlsub c\ X{\isacharcomma}{\kern0pt}{\isasymbeta}\isactrlbsub X\isactrlesub {\isasymrangle}\ {\isasymcirc}\isactrlsub c\ z\ {\isacharequal}{\kern0pt}\ {\isacharparenleft}{\kern0pt}{\isasymTheta}\ {\isasymcirc}\isactrlsub c\ {\isasymlangle}x{\isacharcomma}{\kern0pt}y{\isasymrangle}{\isacharparenright}{\kern0pt}\isactrlsup {\isasymflat}\ {\isasymcirc}\isactrlsub c\ {\isasymlangle}id\isactrlsub c\ X\ {\isasymcirc}\isactrlsub c\ z{\isacharcomma}{\kern0pt}{\isasymbeta}\isactrlbsub X\isactrlesub \ {\isasymcirc}\isactrlsub c\ z{\isasymrangle}{\isachardoublequoteclose}\isanewline
\ \ \ \ \ \ \ \ \ \ \ \ \ \ \isacommand{by}\isamarkupfalse%
\ {\isacharparenleft}{\kern0pt}typecheck{\isacharunderscore}{\kern0pt}cfuncs{\isacharcomma}{\kern0pt}\ simp\ add{\isacharcolon}{\kern0pt}\ cfunc{\isacharunderscore}{\kern0pt}prod{\isacharunderscore}{\kern0pt}comp{\isacharparenright}{\kern0pt}\isanewline
\ \ \ \ \ \ \ \ \ \ \ \ \isacommand{also}\isamarkupfalse%
\ \isacommand{have}\isamarkupfalse%
\ {\isachardoublequoteopen}{\isachardot}{\kern0pt}{\isachardot}{\kern0pt}{\isachardot}{\kern0pt}\ {\isacharequal}{\kern0pt}\ {\isacharparenleft}{\kern0pt}{\isasymTheta}\ {\isasymcirc}\isactrlsub c\ {\isasymlangle}x{\isacharcomma}{\kern0pt}y{\isasymrangle}{\isacharparenright}{\kern0pt}\isactrlsup {\isasymflat}\ {\isasymcirc}\isactrlsub c\ {\isasymlangle}z{\isacharcomma}{\kern0pt}id\ {\isasymone}{\isasymrangle}{\isachardoublequoteclose}\isanewline
\ \ \ \ \ \ \ \ \ \ \ \ \ \ \isacommand{by}\isamarkupfalse%
\ {\isacharparenleft}{\kern0pt}typecheck{\isacharunderscore}{\kern0pt}cfuncs{\isacharcomma}{\kern0pt}\ metis\ id{\isacharunderscore}{\kern0pt}left{\isacharunderscore}{\kern0pt}unit{\isadigit{2}}\ one{\isacharunderscore}{\kern0pt}unique{\isacharunderscore}{\kern0pt}element{\isacharparenright}{\kern0pt}\isanewline
\ \ \ \ \ \ \ \ \ \ \ \ \isacommand{also}\isamarkupfalse%
\ \isacommand{have}\isamarkupfalse%
\ {\isachardoublequoteopen}{\isachardot}{\kern0pt}{\isachardot}{\kern0pt}{\isachardot}{\kern0pt}\ {\isacharequal}{\kern0pt}\ {\isacharparenleft}{\kern0pt}{\isasymTheta}\isactrlsup {\isasymflat}\ {\isasymcirc}\isactrlsub c\ {\isacharparenleft}{\kern0pt}id{\isacharparenleft}{\kern0pt}X{\isacharparenright}{\kern0pt}\ {\isasymtimes}\isactrlsub f\ {\isasymlangle}x{\isacharcomma}{\kern0pt}y{\isasymrangle}{\isacharparenright}{\kern0pt}{\isacharparenright}{\kern0pt}\ {\isasymcirc}\isactrlsub c\ {\isasymlangle}z{\isacharcomma}{\kern0pt}id\ {\isasymone}{\isasymrangle}{\isachardoublequoteclose}\isanewline
\ \ \ \ \ \ \ \ \ \ \ \ \ \ \isacommand{using}\isamarkupfalse%
\ inv{\isacharunderscore}{\kern0pt}transpose{\isacharunderscore}{\kern0pt}of{\isacharunderscore}{\kern0pt}composition\ \isacommand{by}\isamarkupfalse%
\ {\isacharparenleft}{\kern0pt}typecheck{\isacharunderscore}{\kern0pt}cfuncs{\isacharcomma}{\kern0pt}\ presburger{\isacharparenright}{\kern0pt}\isanewline
\ \ \ \ \ \ \ \ \ \ \ \ \isacommand{also}\isamarkupfalse%
\ \isacommand{have}\isamarkupfalse%
\ {\isachardoublequoteopen}{\isachardot}{\kern0pt}{\isachardot}{\kern0pt}{\isachardot}{\kern0pt}\ {\isacharequal}{\kern0pt}\ {\isasymTheta}\isactrlsup {\isasymflat}\ {\isasymcirc}\isactrlsub c\ {\isacharparenleft}{\kern0pt}id{\isacharparenleft}{\kern0pt}X{\isacharparenright}{\kern0pt}\ {\isasymtimes}\isactrlsub f\ {\isasymlangle}x{\isacharcomma}{\kern0pt}y{\isasymrangle}{\isacharparenright}{\kern0pt}\ {\isasymcirc}\isactrlsub c\ {\isasymlangle}z{\isacharcomma}{\kern0pt}id\ {\isasymone}{\isasymrangle}{\isachardoublequoteclose}\isanewline
\ \ \ \ \ \ \ \ \ \ \ \ \ \ \isacommand{using}\isamarkupfalse%
\ comp{\isacharunderscore}{\kern0pt}associative{\isadigit{2}}\ \isacommand{by}\isamarkupfalse%
\ {\isacharparenleft}{\kern0pt}typecheck{\isacharunderscore}{\kern0pt}cfuncs{\isacharcomma}{\kern0pt}\ auto{\isacharparenright}{\kern0pt}\isanewline
\ \ \ \ \ \ \ \ \ \ \ \ \isacommand{also}\isamarkupfalse%
\ \isacommand{have}\isamarkupfalse%
\ {\isachardoublequoteopen}{\isachardot}{\kern0pt}{\isachardot}{\kern0pt}{\isachardot}{\kern0pt}\ {\isacharequal}{\kern0pt}\ {\isasymTheta}\isactrlsup {\isasymflat}\ {\isasymcirc}\isactrlsub c\ {\isasymlangle}id{\isacharparenleft}{\kern0pt}X{\isacharparenright}{\kern0pt}\ {\isasymcirc}\isactrlsub c\ \ z{\isacharcomma}{\kern0pt}\ {\isasymlangle}x{\isacharcomma}{\kern0pt}y{\isasymrangle}\ {\isasymcirc}\isactrlsub c\ \ id\ {\isasymone}{\isasymrangle}{\isachardoublequoteclose}\isanewline
\ \ \ \ \ \ \ \ \ \ \ \ \ \ \isacommand{by}\isamarkupfalse%
\ {\isacharparenleft}{\kern0pt}typecheck{\isacharunderscore}{\kern0pt}cfuncs{\isacharcomma}{\kern0pt}\ simp\ add{\isacharcolon}{\kern0pt}\ cfunc{\isacharunderscore}{\kern0pt}cross{\isacharunderscore}{\kern0pt}prod{\isacharunderscore}{\kern0pt}comp{\isacharunderscore}{\kern0pt}cfunc{\isacharunderscore}{\kern0pt}prod{\isacharparenright}{\kern0pt}\isanewline
\ \ \ \ \ \ \ \ \ \ \ \ \isacommand{also}\isamarkupfalse%
\ \isacommand{have}\isamarkupfalse%
\ {\isachardoublequoteopen}{\isachardot}{\kern0pt}{\isachardot}{\kern0pt}{\isachardot}{\kern0pt}\ {\isacharequal}{\kern0pt}\ {\isasymTheta}\isactrlsup {\isasymflat}\ {\isasymcirc}\isactrlsub c\ {\isasymlangle}z{\isacharcomma}{\kern0pt}{\isasymlangle}x{\isacharcomma}{\kern0pt}y{\isasymrangle}{\isasymrangle}{\isachardoublequoteclose}\isanewline
\ \ \ \ \ \ \ \ \ \ \ \ \ \ \isacommand{by}\isamarkupfalse%
\ {\isacharparenleft}{\kern0pt}typecheck{\isacharunderscore}{\kern0pt}cfuncs{\isacharcomma}{\kern0pt}\ simp\ add{\isacharcolon}{\kern0pt}\ id{\isacharunderscore}{\kern0pt}left{\isacharunderscore}{\kern0pt}unit{\isadigit{2}}\ id{\isacharunderscore}{\kern0pt}right{\isacharunderscore}{\kern0pt}unit{\isadigit{2}}{\isacharparenright}{\kern0pt}\isanewline
\ \ \ \ \ \ \ \ \ \ \ \ \isacommand{also}\isamarkupfalse%
\ \isacommand{have}\isamarkupfalse%
\ {\isachardoublequoteopen}{\isachardot}{\kern0pt}{\isachardot}{\kern0pt}{\isachardot}{\kern0pt}\ {\isacharequal}{\kern0pt}\ {\isacharparenleft}{\kern0pt}{\isacharparenleft}{\kern0pt}into\ {\isasymcirc}\isactrlsub c\ associate{\isacharunderscore}{\kern0pt}right\ Y\ X\ X\ {\isasymcirc}\isactrlsub c\ swap\ X\ {\isacharparenleft}{\kern0pt}Y\ {\isasymtimes}\isactrlsub c\ X{\isacharparenright}{\kern0pt}{\isacharparenright}{\kern0pt}\isactrlsup {\isasymsharp}\ {\isasymcirc}\isactrlsub c\ swap\ X\ Y{\isacharparenright}{\kern0pt}\isactrlsup {\isasymflat}\ {\isasymcirc}\isactrlsub c\ {\isasymlangle}z{\isacharcomma}{\kern0pt}{\isasymlangle}x{\isacharcomma}{\kern0pt}y{\isasymrangle}{\isasymrangle}{\isachardoublequoteclose}\isanewline
\ \ \ \ \ \ \ \ \ \ \ \ \ \ \isacommand{by}\isamarkupfalse%
\ {\isacharparenleft}{\kern0pt}simp\ add{\isacharcolon}{\kern0pt}\ {\isasymTheta}{\isacharunderscore}{\kern0pt}def{\isacharparenright}{\kern0pt}\isanewline
\ \ \ \ \ \ \ \ \ \ \ \ \isacommand{also}\isamarkupfalse%
\ \isacommand{have}\isamarkupfalse%
\ {\isachardoublequoteopen}{\isachardot}{\kern0pt}{\isachardot}{\kern0pt}{\isachardot}{\kern0pt}\ {\isacharequal}{\kern0pt}\ {\isacharparenleft}{\kern0pt}{\isacharparenleft}{\kern0pt}into\ {\isasymcirc}\isactrlsub c\ associate{\isacharunderscore}{\kern0pt}right\ Y\ X\ X\ {\isasymcirc}\isactrlsub c\ swap\ X\ {\isacharparenleft}{\kern0pt}Y\ {\isasymtimes}\isactrlsub c\ X{\isacharparenright}{\kern0pt}{\isacharparenright}{\kern0pt}\isactrlsup {\isasymsharp}\isactrlsup {\isasymflat}\ {\isasymcirc}\isactrlsub c\ {\isacharparenleft}{\kern0pt}id\ X\ {\isasymtimes}\isactrlsub f\ swap\ X\ Y{\isacharparenright}{\kern0pt}{\isacharparenright}{\kern0pt}\ {\isasymcirc}\isactrlsub c\ {\isasymlangle}z{\isacharcomma}{\kern0pt}{\isasymlangle}x{\isacharcomma}{\kern0pt}y{\isasymrangle}{\isasymrangle}{\isachardoublequoteclose}\isanewline
\ \ \ \ \ \ \ \ \ \ \ \ \ \ \isacommand{using}\isamarkupfalse%
\ inv{\isacharunderscore}{\kern0pt}transpose{\isacharunderscore}{\kern0pt}of{\isacharunderscore}{\kern0pt}composition\ \isacommand{by}\isamarkupfalse%
\ {\isacharparenleft}{\kern0pt}typecheck{\isacharunderscore}{\kern0pt}cfuncs{\isacharcomma}{\kern0pt}\ presburger{\isacharparenright}{\kern0pt}\isanewline
\ \ \ \ \ \ \ \ \ \ \ \ \isacommand{also}\isamarkupfalse%
\ \isacommand{have}\isamarkupfalse%
\ {\isachardoublequoteopen}{\isachardot}{\kern0pt}{\isachardot}{\kern0pt}{\isachardot}{\kern0pt}\ {\isacharequal}{\kern0pt}\ {\isacharparenleft}{\kern0pt}into\ {\isasymcirc}\isactrlsub c\ associate{\isacharunderscore}{\kern0pt}right\ Y\ X\ X\ {\isasymcirc}\isactrlsub c\ swap\ X\ {\isacharparenleft}{\kern0pt}Y\ {\isasymtimes}\isactrlsub c\ X{\isacharparenright}{\kern0pt}{\isacharparenright}{\kern0pt}\ {\isasymcirc}\isactrlsub c\ \ {\isacharparenleft}{\kern0pt}id\ X\ {\isasymtimes}\isactrlsub f\ swap\ X\ Y{\isacharparenright}{\kern0pt}\ {\isasymcirc}\isactrlsub c\ {\isasymlangle}z{\isacharcomma}{\kern0pt}{\isasymlangle}x{\isacharcomma}{\kern0pt}y{\isasymrangle}{\isasymrangle}{\isachardoublequoteclose}\isanewline
\ \ \ \ \ \ \ \ \ \ \ \ \ \ \isacommand{by}\isamarkupfalse%
\ {\isacharparenleft}{\kern0pt}typecheck{\isacharunderscore}{\kern0pt}cfuncs{\isacharcomma}{\kern0pt}\ simp\ add{\isacharcolon}{\kern0pt}\ comp{\isacharunderscore}{\kern0pt}associative{\isadigit{2}}\ inv{\isacharunderscore}{\kern0pt}transpose{\isacharunderscore}{\kern0pt}func{\isacharunderscore}{\kern0pt}def{\isadigit{3}}\ transpose{\isacharunderscore}{\kern0pt}func{\isacharunderscore}{\kern0pt}def{\isacharparenright}{\kern0pt}\isanewline
\ \ \ \ \ \ \ \ \ \ \ \ \isacommand{also}\isamarkupfalse%
\ \isacommand{have}\isamarkupfalse%
\ {\isachardoublequoteopen}{\isachardot}{\kern0pt}{\isachardot}{\kern0pt}{\isachardot}{\kern0pt}\ {\isacharequal}{\kern0pt}\ {\isacharparenleft}{\kern0pt}into\ {\isasymcirc}\isactrlsub c\ associate{\isacharunderscore}{\kern0pt}right\ Y\ X\ X\ {\isasymcirc}\isactrlsub c\ swap\ X\ {\isacharparenleft}{\kern0pt}Y\ {\isasymtimes}\isactrlsub c\ X{\isacharparenright}{\kern0pt}{\isacharparenright}{\kern0pt}\ {\isasymcirc}\isactrlsub c\ \ {\isasymlangle}id\ X\ {\isasymcirc}\isactrlsub c\ z{\isacharcomma}{\kern0pt}\ swap\ X\ Y\ {\isasymcirc}\isactrlsub c\ {\isasymlangle}x{\isacharcomma}{\kern0pt}y{\isasymrangle}{\isasymrangle}{\isachardoublequoteclose}\isanewline
\ \ \ \ \ \ \ \ \ \ \ \ \ \ \isacommand{by}\isamarkupfalse%
\ {\isacharparenleft}{\kern0pt}typecheck{\isacharunderscore}{\kern0pt}cfuncs{\isacharcomma}{\kern0pt}\ simp\ add{\isacharcolon}{\kern0pt}\ cfunc{\isacharunderscore}{\kern0pt}cross{\isacharunderscore}{\kern0pt}prod{\isacharunderscore}{\kern0pt}comp{\isacharunderscore}{\kern0pt}cfunc{\isacharunderscore}{\kern0pt}prod{\isacharparenright}{\kern0pt}\isanewline
\ \ \ \ \ \ \ \ \ \ \ \ \isacommand{also}\isamarkupfalse%
\ \isacommand{have}\isamarkupfalse%
\ {\isachardoublequoteopen}{\isachardot}{\kern0pt}{\isachardot}{\kern0pt}{\isachardot}{\kern0pt}\ {\isacharequal}{\kern0pt}\ {\isacharparenleft}{\kern0pt}into\ {\isasymcirc}\isactrlsub c\ associate{\isacharunderscore}{\kern0pt}right\ Y\ X\ X\ {\isasymcirc}\isactrlsub c\ swap\ X\ {\isacharparenleft}{\kern0pt}Y\ {\isasymtimes}\isactrlsub c\ X{\isacharparenright}{\kern0pt}{\isacharparenright}{\kern0pt}\ {\isasymcirc}\isactrlsub c\ \ {\isasymlangle}z{\isacharcomma}{\kern0pt}\ {\isasymlangle}y{\isacharcomma}{\kern0pt}x{\isasymrangle}{\isasymrangle}{\isachardoublequoteclose}\isanewline
\ \ \ \ \ \ \ \ \ \ \ \ \ \ \isacommand{using}\isamarkupfalse%
\ id{\isacharunderscore}{\kern0pt}left{\isacharunderscore}{\kern0pt}unit{\isadigit{2}}\ swap{\isacharunderscore}{\kern0pt}ap\ \isacommand{by}\isamarkupfalse%
\ {\isacharparenleft}{\kern0pt}typecheck{\isacharunderscore}{\kern0pt}cfuncs{\isacharcomma}{\kern0pt}\ presburger{\isacharparenright}{\kern0pt}\isanewline
\ \ \ \ \ \ \ \ \ \ \ \ \isacommand{also}\isamarkupfalse%
\ \isacommand{have}\isamarkupfalse%
\ {\isachardoublequoteopen}{\isachardot}{\kern0pt}{\isachardot}{\kern0pt}{\isachardot}{\kern0pt}\ {\isacharequal}{\kern0pt}\ into\ {\isasymcirc}\isactrlsub c\ associate{\isacharunderscore}{\kern0pt}right\ Y\ X\ X\ {\isasymcirc}\isactrlsub c\ swap\ X\ {\isacharparenleft}{\kern0pt}Y\ {\isasymtimes}\isactrlsub c\ X{\isacharparenright}{\kern0pt}\ {\isasymcirc}\isactrlsub c\ \ {\isasymlangle}z{\isacharcomma}{\kern0pt}\ {\isasymlangle}y{\isacharcomma}{\kern0pt}x{\isasymrangle}{\isasymrangle}{\isachardoublequoteclose}\isanewline
\ \ \ \ \ \ \ \ \ \ \ \ \ \ \isacommand{by}\isamarkupfalse%
\ {\isacharparenleft}{\kern0pt}typecheck{\isacharunderscore}{\kern0pt}cfuncs{\isacharcomma}{\kern0pt}\ metis\ cfunc{\isacharunderscore}{\kern0pt}type{\isacharunderscore}{\kern0pt}def\ comp{\isacharunderscore}{\kern0pt}associative{\isacharparenright}{\kern0pt}\isanewline
\ \ \ \ \ \ \ \ \ \ \ \ \isacommand{also}\isamarkupfalse%
\ \isacommand{have}\isamarkupfalse%
\ {\isachardoublequoteopen}{\isachardot}{\kern0pt}{\isachardot}{\kern0pt}{\isachardot}{\kern0pt}\ {\isacharequal}{\kern0pt}\ into\ {\isasymcirc}\isactrlsub c\ associate{\isacharunderscore}{\kern0pt}right\ Y\ X\ X\ {\isasymcirc}\isactrlsub c\ \ \ {\isasymlangle}{\isasymlangle}y{\isacharcomma}{\kern0pt}x{\isasymrangle}{\isacharcomma}{\kern0pt}\ z{\isasymrangle}{\isachardoublequoteclose}\isanewline
\ \ \ \ \ \ \ \ \ \ \ \ \ \ \isacommand{using}\isamarkupfalse%
\ swap{\isacharunderscore}{\kern0pt}ap\ \isacommand{by}\isamarkupfalse%
\ {\isacharparenleft}{\kern0pt}typecheck{\isacharunderscore}{\kern0pt}cfuncs{\isacharcomma}{\kern0pt}\ presburger{\isacharparenright}{\kern0pt}\isanewline
\ \ \ \ \ \ \ \ \ \ \ \ \isacommand{also}\isamarkupfalse%
\ \isacommand{have}\isamarkupfalse%
\ {\isachardoublequoteopen}{\isachardot}{\kern0pt}{\isachardot}{\kern0pt}{\isachardot}{\kern0pt}\ {\isacharequal}{\kern0pt}\ into\ {\isasymcirc}\isactrlsub c\ \ \ {\isasymlangle}y{\isacharcomma}{\kern0pt}\ {\isasymlangle}x{\isacharcomma}{\kern0pt}\ z{\isasymrangle}{\isasymrangle}{\isachardoublequoteclose}\isanewline
\ \ \ \ \ \ \ \ \ \ \ \ \ \ \isacommand{using}\isamarkupfalse%
\ associate{\isacharunderscore}{\kern0pt}right{\isacharunderscore}{\kern0pt}ap\ \isacommand{by}\isamarkupfalse%
\ {\isacharparenleft}{\kern0pt}typecheck{\isacharunderscore}{\kern0pt}cfuncs{\isacharcomma}{\kern0pt}\ presburger{\isacharparenright}{\kern0pt}\isanewline
\ \ \ \ \ \ \ \ \ \ \ \ \isacommand{then}\isamarkupfalse%
\ \isacommand{show}\isamarkupfalse%
\ {\isacharquery}{\kern0pt}thesis\isanewline
\ \ \ \ \ \ \ \ \ \ \ \ \ \ \isacommand{using}\isamarkupfalse%
\ calculation\ \isacommand{by}\isamarkupfalse%
\ presburger\isanewline
\ \ \ \ \ \ \ \ \ \ \isacommand{qed}\isamarkupfalse%
\isanewline
\ \ \ \ \ \ \ \ \isacommand{qed}\isamarkupfalse%
\isanewline
\ \ \isanewline
\ \ \ \ \ \ \ \ \isacommand{have}\isamarkupfalse%
\ f{\isadigit{1}}{\isacharcolon}{\kern0pt}\ {\isachardoublequoteopen}{\isasymAnd}x\ y{\isachardot}{\kern0pt}\ x\ {\isasymin}\isactrlsub c\ X\ {\isasymLongrightarrow}\ y\ {\isasymin}\isactrlsub c\ Y\ \ {\isasymLongrightarrow}\ {\isacharparenleft}{\kern0pt}{\isasymTheta}\ {\isasymcirc}\isactrlsub c\ {\isasymlangle}x{\isacharcomma}{\kern0pt}\ y{\isasymrangle}{\isacharparenright}{\kern0pt}\isactrlsup {\isasymflat}\ {\isasymcirc}\isactrlsub c\ {\isasymlangle}id\ X{\isacharcomma}{\kern0pt}\ {\isasymbeta}\isactrlbsub X\isactrlesub {\isasymrangle}\ {\isasymcirc}\isactrlsub c\ x\ {\isacharequal}{\kern0pt}\ y{\isachardoublequoteclose}\isanewline
\ \ \ \ \ \ \ \ \isacommand{proof}\isamarkupfalse%
\ {\isacharminus}{\kern0pt}\ \isanewline
\ \ \ \ \ \ \ \ \ \ \isacommand{fix}\isamarkupfalse%
\ x\ y\ \isanewline
\ \ \ \ \ \ \ \ \ \ \isacommand{assume}\isamarkupfalse%
\ x{\isacharunderscore}{\kern0pt}type{\isacharbrackleft}{\kern0pt}type{\isacharunderscore}{\kern0pt}rule{\isacharbrackright}{\kern0pt}{\isacharcolon}{\kern0pt}\ {\isachardoublequoteopen}x\ {\isasymin}\isactrlsub c\ X{\isachardoublequoteclose}\isanewline
\ \ \ \ \ \ \ \ \ \ \isacommand{assume}\isamarkupfalse%
\ y{\isacharunderscore}{\kern0pt}type{\isacharbrackleft}{\kern0pt}type{\isacharunderscore}{\kern0pt}rule{\isacharbrackright}{\kern0pt}{\isacharcolon}{\kern0pt}\ {\isachardoublequoteopen}y\ {\isasymin}\isactrlsub c\ Y{\isachardoublequoteclose}\isanewline
\ \ \ \ \ \ \ \ \ \ \isacommand{have}\isamarkupfalse%
\ {\isachardoublequoteopen}{\isacharparenleft}{\kern0pt}{\isasymTheta}\ {\isasymcirc}\isactrlsub c\ {\isasymlangle}x{\isacharcomma}{\kern0pt}\ y{\isasymrangle}{\isacharparenright}{\kern0pt}\isactrlsup {\isasymflat}\ {\isasymcirc}\isactrlsub c\ {\isasymlangle}id\ X{\isacharcomma}{\kern0pt}\ {\isasymbeta}\isactrlbsub X\isactrlesub {\isasymrangle}\ {\isasymcirc}\isactrlsub c\ x\ {\isacharequal}{\kern0pt}\ into\ {\isasymcirc}\isactrlsub c\ \ \ {\isasymlangle}y{\isacharcomma}{\kern0pt}\ {\isasymlangle}x{\isacharcomma}{\kern0pt}\ x{\isasymrangle}{\isasymrangle}{\isachardoublequoteclose}\isanewline
\ \ \ \ \ \ \ \ \ \ \ \ \isacommand{by}\isamarkupfalse%
\ {\isacharparenleft}{\kern0pt}simp\ add{\isacharcolon}{\kern0pt}\ f{\isadigit{0}}\ x{\isacharunderscore}{\kern0pt}type\ y{\isacharunderscore}{\kern0pt}type{\isacharparenright}{\kern0pt}\isanewline
\ \ \ \ \ \ \ \ \ \ \isacommand{also}\isamarkupfalse%
\ \isacommand{have}\isamarkupfalse%
\ {\isachardoublequoteopen}{\isachardot}{\kern0pt}{\isachardot}{\kern0pt}{\isachardot}{\kern0pt}\ {\isacharequal}{\kern0pt}\ {\isacharparenleft}{\kern0pt}left{\isacharunderscore}{\kern0pt}cart{\isacharunderscore}{\kern0pt}proj\ Y\ {\isasymone}\ {\isasymamalg}\ {\isacharparenleft}{\kern0pt}{\isacharparenleft}{\kern0pt}y{\isadigit{2}}\ {\isasymamalg}\ y{\isadigit{1}}{\isacharparenright}{\kern0pt}\ {\isasymcirc}\isactrlsub c\ case{\isacharunderscore}{\kern0pt}bool\ {\isasymcirc}\isactrlsub c\ eq{\isacharunderscore}{\kern0pt}pred\ Y\ {\isasymcirc}\isactrlsub c\ {\isacharparenleft}{\kern0pt}id\ Y\ {\isasymtimes}\isactrlsub f\ y{\isadigit{1}}{\isacharparenright}{\kern0pt}{\isacharparenright}{\kern0pt}{\isacharparenright}{\kern0pt}\isanewline
\ \ \ \ \ \ \ \ \ \ \ \ \ \ \ \ \ \ \ \ \ \ \ \ \ \ \ \ \ \ \ \ \ {\isasymcirc}\isactrlsub c\ dist{\isacharunderscore}{\kern0pt}prod{\isacharunderscore}{\kern0pt}coprod{\isacharunderscore}{\kern0pt}left\ Y\ {\isasymone}\ {\isasymone}\ {\isasymcirc}\isactrlsub c\ {\isacharparenleft}{\kern0pt}id\ Y\ {\isasymtimes}\isactrlsub f\ case{\isacharunderscore}{\kern0pt}bool{\isacharparenright}{\kern0pt}\ {\isasymcirc}\isactrlsub c\ {\isacharparenleft}{\kern0pt}id\ Y\ {\isasymtimes}\isactrlsub f\ eq{\isacharunderscore}{\kern0pt}pred\ X{\isacharparenright}{\kern0pt}\ {\isasymcirc}\isactrlsub c\ \ \ {\isasymlangle}y{\isacharcomma}{\kern0pt}\ {\isasymlangle}x{\isacharcomma}{\kern0pt}\ x{\isasymrangle}{\isasymrangle}{\isachardoublequoteclose}\isanewline
\ \ \ \ \ \ \ \ \ \ \ \ \isacommand{using}\isamarkupfalse%
\ cfunc{\isacharunderscore}{\kern0pt}type{\isacharunderscore}{\kern0pt}def\ comp{\isacharunderscore}{\kern0pt}associative\ comp{\isacharunderscore}{\kern0pt}type\ into{\isacharunderscore}{\kern0pt}def\ \isacommand{by}\isamarkupfalse%
\ {\isacharparenleft}{\kern0pt}typecheck{\isacharunderscore}{\kern0pt}cfuncs{\isacharcomma}{\kern0pt}\ fastforce{\isacharparenright}{\kern0pt}\isanewline
\ \ \ \ \ \ \ \ \ \ \isacommand{also}\isamarkupfalse%
\ \isacommand{have}\isamarkupfalse%
\ {\isachardoublequoteopen}{\isachardot}{\kern0pt}{\isachardot}{\kern0pt}{\isachardot}{\kern0pt}\ {\isacharequal}{\kern0pt}\ {\isacharparenleft}{\kern0pt}left{\isacharunderscore}{\kern0pt}cart{\isacharunderscore}{\kern0pt}proj\ Y\ {\isasymone}\ {\isasymamalg}\ {\isacharparenleft}{\kern0pt}{\isacharparenleft}{\kern0pt}y{\isadigit{2}}\ {\isasymamalg}\ y{\isadigit{1}}{\isacharparenright}{\kern0pt}\ {\isasymcirc}\isactrlsub c\ case{\isacharunderscore}{\kern0pt}bool\ {\isasymcirc}\isactrlsub c\ eq{\isacharunderscore}{\kern0pt}pred\ Y\ {\isasymcirc}\isactrlsub c\ {\isacharparenleft}{\kern0pt}id\ Y\ {\isasymtimes}\isactrlsub f\ y{\isadigit{1}}{\isacharparenright}{\kern0pt}{\isacharparenright}{\kern0pt}{\isacharparenright}{\kern0pt}\isanewline
\ \ \ \ \ \ \ \ \ \ \ \ \ \ \ \ \ \ \ \ \ \ \ \ \ \ \ \ \ \ \ \ \ {\isasymcirc}\isactrlsub c\ dist{\isacharunderscore}{\kern0pt}prod{\isacharunderscore}{\kern0pt}coprod{\isacharunderscore}{\kern0pt}left\ Y\ {\isasymone}\ {\isasymone}\ {\isasymcirc}\isactrlsub c\ {\isacharparenleft}{\kern0pt}id\ Y\ {\isasymtimes}\isactrlsub f\ case{\isacharunderscore}{\kern0pt}bool{\isacharparenright}{\kern0pt}\ {\isasymcirc}\isactrlsub c\ \ {\isasymlangle}id\ Y\ {\isasymcirc}\isactrlsub c\ y{\isacharcomma}{\kern0pt}\ eq{\isacharunderscore}{\kern0pt}pred\ X\ {\isasymcirc}\isactrlsub c\ \ {\isasymlangle}x{\isacharcomma}{\kern0pt}\ x{\isasymrangle}{\isasymrangle}{\isachardoublequoteclose}\isanewline
\ \ \ \ \ \ \ \ \ \ \ \ \isacommand{by}\isamarkupfalse%
\ {\isacharparenleft}{\kern0pt}typecheck{\isacharunderscore}{\kern0pt}cfuncs{\isacharcomma}{\kern0pt}\ simp\ add{\isacharcolon}{\kern0pt}\ cfunc{\isacharunderscore}{\kern0pt}cross{\isacharunderscore}{\kern0pt}prod{\isacharunderscore}{\kern0pt}comp{\isacharunderscore}{\kern0pt}cfunc{\isacharunderscore}{\kern0pt}prod{\isacharparenright}{\kern0pt}\isanewline
\ \ \ \ \ \ \ \ \ \isacommand{also}\isamarkupfalse%
\ \isacommand{have}\isamarkupfalse%
\ {\isachardoublequoteopen}{\isachardot}{\kern0pt}{\isachardot}{\kern0pt}{\isachardot}{\kern0pt}\ {\isacharequal}{\kern0pt}\ {\isacharparenleft}{\kern0pt}left{\isacharunderscore}{\kern0pt}cart{\isacharunderscore}{\kern0pt}proj\ Y\ {\isasymone}\ {\isasymamalg}\ {\isacharparenleft}{\kern0pt}{\isacharparenleft}{\kern0pt}y{\isadigit{2}}\ {\isasymamalg}\ y{\isadigit{1}}{\isacharparenright}{\kern0pt}\ {\isasymcirc}\isactrlsub c\ case{\isacharunderscore}{\kern0pt}bool\ {\isasymcirc}\isactrlsub c\ eq{\isacharunderscore}{\kern0pt}pred\ Y\ {\isasymcirc}\isactrlsub c\ {\isacharparenleft}{\kern0pt}id\ Y\ {\isasymtimes}\isactrlsub f\ y{\isadigit{1}}{\isacharparenright}{\kern0pt}{\isacharparenright}{\kern0pt}{\isacharparenright}{\kern0pt}\ \isanewline
\ \ \ \ \ \ \ \ \ \ \ \ \ \ \ \ \ \ \ \ \ \ \ \ \ \ \ \ \ \ \ \ \ {\isasymcirc}\isactrlsub c\ dist{\isacharunderscore}{\kern0pt}prod{\isacharunderscore}{\kern0pt}coprod{\isacharunderscore}{\kern0pt}left\ Y\ {\isasymone}\ {\isasymone}\ {\isasymcirc}\isactrlsub c\ {\isacharparenleft}{\kern0pt}id\ Y\ {\isasymtimes}\isactrlsub f\ case{\isacharunderscore}{\kern0pt}bool{\isacharparenright}{\kern0pt}\ {\isasymcirc}\isactrlsub c\ \ {\isasymlangle}y{\isacharcomma}{\kern0pt}\ {\isasymt}{\isasymrangle}{\isachardoublequoteclose}\isanewline
\ \ \ \ \ \ \ \ \ \ \ \ \isacommand{by}\isamarkupfalse%
\ {\isacharparenleft}{\kern0pt}typecheck{\isacharunderscore}{\kern0pt}cfuncs{\isacharcomma}{\kern0pt}\ metis\ eq{\isacharunderscore}{\kern0pt}pred{\isacharunderscore}{\kern0pt}iff{\isacharunderscore}{\kern0pt}eq\ id{\isacharunderscore}{\kern0pt}left{\isacharunderscore}{\kern0pt}unit{\isadigit{2}}{\isacharparenright}{\kern0pt}\isanewline
\ \ \ \ \ \ \ \ \ \ \isacommand{also}\isamarkupfalse%
\ \isacommand{have}\isamarkupfalse%
\ {\isachardoublequoteopen}{\isachardot}{\kern0pt}{\isachardot}{\kern0pt}{\isachardot}{\kern0pt}\ {\isacharequal}{\kern0pt}\ {\isacharparenleft}{\kern0pt}left{\isacharunderscore}{\kern0pt}cart{\isacharunderscore}{\kern0pt}proj\ Y\ {\isasymone}\ {\isasymamalg}\ {\isacharparenleft}{\kern0pt}{\isacharparenleft}{\kern0pt}y{\isadigit{2}}\ {\isasymamalg}\ y{\isadigit{1}}{\isacharparenright}{\kern0pt}\ {\isasymcirc}\isactrlsub c\ case{\isacharunderscore}{\kern0pt}bool\ {\isasymcirc}\isactrlsub c\ eq{\isacharunderscore}{\kern0pt}pred\ Y\ {\isasymcirc}\isactrlsub c\ {\isacharparenleft}{\kern0pt}id\ Y\ {\isasymtimes}\isactrlsub f\ y{\isadigit{1}}{\isacharparenright}{\kern0pt}{\isacharparenright}{\kern0pt}{\isacharparenright}{\kern0pt}\ \isanewline
\ \ \ \ \ \ \ \ \ \ \ \ \ \ \ \ \ \ \ \ \ \ \ \ \ \ \ \ \ \ \ \ \ {\isasymcirc}\isactrlsub c\ dist{\isacharunderscore}{\kern0pt}prod{\isacharunderscore}{\kern0pt}coprod{\isacharunderscore}{\kern0pt}left\ Y\ {\isasymone}\ {\isasymone}\ \ {\isasymcirc}\isactrlsub c\ \ {\isasymlangle}y{\isacharcomma}{\kern0pt}\ left{\isacharunderscore}{\kern0pt}coproj\ {\isasymone}\ {\isasymone}{\isasymrangle}{\isachardoublequoteclose}\isanewline
\ \ \ \ \ \ \ \ \ \ \ \ \isacommand{by}\isamarkupfalse%
\ {\isacharparenleft}{\kern0pt}typecheck{\isacharunderscore}{\kern0pt}cfuncs{\isacharcomma}{\kern0pt}\ simp\ add{\isacharcolon}{\kern0pt}\ case{\isacharunderscore}{\kern0pt}bool{\isacharunderscore}{\kern0pt}true\ cfunc{\isacharunderscore}{\kern0pt}cross{\isacharunderscore}{\kern0pt}prod{\isacharunderscore}{\kern0pt}comp{\isacharunderscore}{\kern0pt}cfunc{\isacharunderscore}{\kern0pt}prod\ id{\isacharunderscore}{\kern0pt}left{\isacharunderscore}{\kern0pt}unit{\isadigit{2}}{\isacharparenright}{\kern0pt}\isanewline
\ \ \ \ \ \ \ \ \ \ \isacommand{also}\isamarkupfalse%
\ \isacommand{have}\isamarkupfalse%
\ {\isachardoublequoteopen}{\isachardot}{\kern0pt}{\isachardot}{\kern0pt}{\isachardot}{\kern0pt}\ {\isacharequal}{\kern0pt}\ {\isacharparenleft}{\kern0pt}left{\isacharunderscore}{\kern0pt}cart{\isacharunderscore}{\kern0pt}proj\ Y\ {\isasymone}\ {\isasymamalg}\ {\isacharparenleft}{\kern0pt}{\isacharparenleft}{\kern0pt}y{\isadigit{2}}\ {\isasymamalg}\ y{\isadigit{1}}{\isacharparenright}{\kern0pt}\ {\isasymcirc}\isactrlsub c\ case{\isacharunderscore}{\kern0pt}bool\ {\isasymcirc}\isactrlsub c\ eq{\isacharunderscore}{\kern0pt}pred\ Y\ {\isasymcirc}\isactrlsub c\ {\isacharparenleft}{\kern0pt}id\ Y\ {\isasymtimes}\isactrlsub f\ y{\isadigit{1}}{\isacharparenright}{\kern0pt}{\isacharparenright}{\kern0pt}{\isacharparenright}{\kern0pt}\ \isanewline
\ \ \ \ \ \ \ \ \ \ \ \ \ \ \ \ \ \ \ \ \ \ \ \ \ \ \ \ \ \ \ \ \ {\isasymcirc}\isactrlsub c\ dist{\isacharunderscore}{\kern0pt}prod{\isacharunderscore}{\kern0pt}coprod{\isacharunderscore}{\kern0pt}left\ Y\ {\isasymone}\ {\isasymone}\ \ {\isasymcirc}\isactrlsub c\ \ {\isasymlangle}y{\isacharcomma}{\kern0pt}\ left{\isacharunderscore}{\kern0pt}coproj\ {\isasymone}\ {\isasymone}\ {\isasymcirc}\isactrlsub c\ id\ {\isasymone}{\isasymrangle}{\isachardoublequoteclose}\isanewline
\ \ \ \ \ \ \ \ \ \ \ \ \isacommand{by}\isamarkupfalse%
\ {\isacharparenleft}{\kern0pt}typecheck{\isacharunderscore}{\kern0pt}cfuncs{\isacharcomma}{\kern0pt}\ metis\ id{\isacharunderscore}{\kern0pt}right{\isacharunderscore}{\kern0pt}unit{\isadigit{2}}{\isacharparenright}{\kern0pt}\isanewline
\ \ \ \ \ \ \ \ \ \ \isacommand{also}\isamarkupfalse%
\ \isacommand{have}\isamarkupfalse%
\ {\isachardoublequoteopen}{\isachardot}{\kern0pt}{\isachardot}{\kern0pt}{\isachardot}{\kern0pt}\ {\isacharequal}{\kern0pt}\ {\isacharparenleft}{\kern0pt}left{\isacharunderscore}{\kern0pt}cart{\isacharunderscore}{\kern0pt}proj\ Y\ {\isasymone}\ {\isasymamalg}\ {\isacharparenleft}{\kern0pt}{\isacharparenleft}{\kern0pt}y{\isadigit{2}}\ {\isasymamalg}\ y{\isadigit{1}}{\isacharparenright}{\kern0pt}\ {\isasymcirc}\isactrlsub c\ case{\isacharunderscore}{\kern0pt}bool\ {\isasymcirc}\isactrlsub c\ eq{\isacharunderscore}{\kern0pt}pred\ Y\ {\isasymcirc}\isactrlsub c\ {\isacharparenleft}{\kern0pt}id\ Y\ {\isasymtimes}\isactrlsub f\ y{\isadigit{1}}{\isacharparenright}{\kern0pt}{\isacharparenright}{\kern0pt}{\isacharparenright}{\kern0pt}\ \isanewline
\ \ \ \ \ \ \ \ \ \ \ \ \ \ \ \ \ \ \ \ \ \ \ \ \ \ \ \ \ \ \ \ \ {\isasymcirc}\isactrlsub c\ left{\isacharunderscore}{\kern0pt}coproj\ {\isacharparenleft}{\kern0pt}Y\ {\isasymtimes}\isactrlsub c\ {\isasymone}{\isacharparenright}{\kern0pt}\ {\isacharparenleft}{\kern0pt}Y\ {\isasymtimes}\isactrlsub c\ {\isasymone}{\isacharparenright}{\kern0pt}\ {\isasymcirc}\isactrlsub c\ {\isasymlangle}y{\isacharcomma}{\kern0pt}id\ {\isasymone}{\isasymrangle}{\isachardoublequoteclose}\isanewline
\ \ \ \ \ \ \ \ \ \ \ \ \isacommand{using}\isamarkupfalse%
\ dist{\isacharunderscore}{\kern0pt}prod{\isacharunderscore}{\kern0pt}coprod{\isacharunderscore}{\kern0pt}left{\isacharunderscore}{\kern0pt}ap{\isacharunderscore}{\kern0pt}left\ \isacommand{by}\isamarkupfalse%
\ {\isacharparenleft}{\kern0pt}typecheck{\isacharunderscore}{\kern0pt}cfuncs{\isacharcomma}{\kern0pt}\ auto{\isacharparenright}{\kern0pt}\isanewline
\ \ \ \ \ \ \ \ \ \ \isacommand{also}\isamarkupfalse%
\ \isacommand{have}\isamarkupfalse%
\ {\isachardoublequoteopen}{\isachardot}{\kern0pt}{\isachardot}{\kern0pt}{\isachardot}{\kern0pt}\ {\isacharequal}{\kern0pt}\ {\isacharparenleft}{\kern0pt}{\isacharparenleft}{\kern0pt}left{\isacharunderscore}{\kern0pt}cart{\isacharunderscore}{\kern0pt}proj\ Y\ {\isasymone}\ {\isasymamalg}\ {\isacharparenleft}{\kern0pt}{\isacharparenleft}{\kern0pt}y{\isadigit{2}}\ {\isasymamalg}\ y{\isadigit{1}}{\isacharparenright}{\kern0pt}\ {\isasymcirc}\isactrlsub c\ case{\isacharunderscore}{\kern0pt}bool\ {\isasymcirc}\isactrlsub c\ eq{\isacharunderscore}{\kern0pt}pred\ Y\ {\isasymcirc}\isactrlsub c\ {\isacharparenleft}{\kern0pt}id\ Y\ {\isasymtimes}\isactrlsub f\ y{\isadigit{1}}{\isacharparenright}{\kern0pt}{\isacharparenright}{\kern0pt}{\isacharparenright}{\kern0pt}\ \isanewline
\ \ \ \ \ \ \ \ \ \ \ \ \ \ \ \ \ \ \ \ \ \ \ \ \ \ \ \ \ \ \ \ \ {\isasymcirc}\isactrlsub c\ left{\isacharunderscore}{\kern0pt}coproj\ {\isacharparenleft}{\kern0pt}Y\ {\isasymtimes}\isactrlsub c\ {\isasymone}{\isacharparenright}{\kern0pt}\ {\isacharparenleft}{\kern0pt}Y\ {\isasymtimes}\isactrlsub c\ {\isasymone}{\isacharparenright}{\kern0pt}{\isacharparenright}{\kern0pt}\ {\isasymcirc}\isactrlsub c\ {\isasymlangle}y{\isacharcomma}{\kern0pt}id\ {\isasymone}{\isasymrangle}{\isachardoublequoteclose}\isanewline
\ \ \ \ \ \ \ \ \ \ \ \ \isacommand{by}\isamarkupfalse%
\ {\isacharparenleft}{\kern0pt}typecheck{\isacharunderscore}{\kern0pt}cfuncs{\isacharcomma}{\kern0pt}\ meson\ comp{\isacharunderscore}{\kern0pt}associative{\isadigit{2}}{\isacharparenright}{\kern0pt}\isanewline
\ \ \ \ \ \ \ \ \ \ \isacommand{also}\isamarkupfalse%
\ \isacommand{have}\isamarkupfalse%
\ {\isachardoublequoteopen}{\isachardot}{\kern0pt}{\isachardot}{\kern0pt}{\isachardot}{\kern0pt}\ {\isacharequal}{\kern0pt}\ left{\isacharunderscore}{\kern0pt}cart{\isacharunderscore}{\kern0pt}proj\ Y\ {\isasymone}\ {\isasymcirc}\isactrlsub c\ {\isasymlangle}y{\isacharcomma}{\kern0pt}id\ {\isasymone}{\isasymrangle}{\isachardoublequoteclose}\isanewline
\ \ \ \ \ \ \ \ \ \ \ \ \isacommand{using}\isamarkupfalse%
\ left{\isacharunderscore}{\kern0pt}coproj{\isacharunderscore}{\kern0pt}cfunc{\isacharunderscore}{\kern0pt}coprod\ \isacommand{by}\isamarkupfalse%
\ {\isacharparenleft}{\kern0pt}typecheck{\isacharunderscore}{\kern0pt}cfuncs{\isacharcomma}{\kern0pt}\ presburger{\isacharparenright}{\kern0pt}\isanewline
\ \ \ \ \ \ \ \ \ \ \isacommand{also}\isamarkupfalse%
\ \isacommand{have}\isamarkupfalse%
\ {\isachardoublequoteopen}{\isachardot}{\kern0pt}{\isachardot}{\kern0pt}{\isachardot}{\kern0pt}\ {\isacharequal}{\kern0pt}\ y{\isachardoublequoteclose}\isanewline
\ \ \ \ \ \ \ \ \ \ \ \ \isacommand{by}\isamarkupfalse%
\ {\isacharparenleft}{\kern0pt}typecheck{\isacharunderscore}{\kern0pt}cfuncs{\isacharcomma}{\kern0pt}\ simp\ add{\isacharcolon}{\kern0pt}\ left{\isacharunderscore}{\kern0pt}cart{\isacharunderscore}{\kern0pt}proj{\isacharunderscore}{\kern0pt}cfunc{\isacharunderscore}{\kern0pt}prod{\isacharparenright}{\kern0pt}\isanewline
\ \ \ \ \ \ \ \ \ \ \isacommand{then}\isamarkupfalse%
\ \isacommand{show}\isamarkupfalse%
\ {\isachardoublequoteopen}{\isacharparenleft}{\kern0pt}{\isasymTheta}\ {\isasymcirc}\isactrlsub c\ {\isasymlangle}x{\isacharcomma}{\kern0pt}\ y{\isasymrangle}{\isacharparenright}{\kern0pt}\isactrlsup {\isasymflat}\ {\isasymcirc}\isactrlsub c\ {\isasymlangle}id\ X{\isacharcomma}{\kern0pt}\ {\isasymbeta}\isactrlbsub X\isactrlesub {\isasymrangle}\ {\isasymcirc}\isactrlsub c\ x\ {\isacharequal}{\kern0pt}\ y{\isachardoublequoteclose}\isanewline
\ \ \ \ \ \ \ \ \ \ \ \ \isacommand{by}\isamarkupfalse%
\ {\isacharparenleft}{\kern0pt}simp\ add{\isacharcolon}{\kern0pt}\ calculation\ into{\isacharunderscore}{\kern0pt}def{\isacharparenright}{\kern0pt}\isanewline
\ \ \ \ \ \ \ \ \isacommand{qed}\isamarkupfalse%
\isanewline
\ \ \isanewline
\ \ \ \ \ \ \ \ \isacommand{have}\isamarkupfalse%
\ f{\isadigit{2}}{\isacharcolon}{\kern0pt}\ {\isachardoublequoteopen}{\isasymAnd}x\ y\ z{\isachardot}{\kern0pt}\ x\ {\isasymin}\isactrlsub c\ X\ {\isasymLongrightarrow}\ y\ {\isasymin}\isactrlsub c\ Y\ \ {\isasymLongrightarrow}\ \ z\ {\isasymin}\isactrlsub c\ X\ {\isasymLongrightarrow}\ z\ {\isasymnoteq}\ x\ {\isasymLongrightarrow}\ y\ {\isasymnoteq}\ y{\isadigit{1}}\ {\isasymLongrightarrow}\ {\isacharparenleft}{\kern0pt}{\isasymTheta}\ {\isasymcirc}\isactrlsub c\ {\isasymlangle}x{\isacharcomma}{\kern0pt}\ y{\isasymrangle}{\isacharparenright}{\kern0pt}\isactrlsup {\isasymflat}\ {\isasymcirc}\isactrlsub c\ {\isasymlangle}id\ X{\isacharcomma}{\kern0pt}\ {\isasymbeta}\isactrlbsub X\isactrlesub {\isasymrangle}\ {\isasymcirc}\isactrlsub c\ z\ {\isacharequal}{\kern0pt}\ y{\isadigit{1}}{\isachardoublequoteclose}\isanewline
\ \ \ \ \ \ \ \ \isacommand{proof}\isamarkupfalse%
\ {\isacharminus}{\kern0pt}\ \isanewline
\ \ \ \ \ \ \ \ \ \ \isacommand{fix}\isamarkupfalse%
\ x\ y\ z\isanewline
\ \ \ \ \ \ \ \ \ \ \isacommand{assume}\isamarkupfalse%
\ x{\isacharunderscore}{\kern0pt}type{\isacharbrackleft}{\kern0pt}type{\isacharunderscore}{\kern0pt}rule{\isacharbrackright}{\kern0pt}{\isacharcolon}{\kern0pt}\ {\isachardoublequoteopen}x\ {\isasymin}\isactrlsub c\ X{\isachardoublequoteclose}\isanewline
\ \ \ \ \ \ \ \ \ \ \isacommand{assume}\isamarkupfalse%
\ y{\isacharunderscore}{\kern0pt}type{\isacharbrackleft}{\kern0pt}type{\isacharunderscore}{\kern0pt}rule{\isacharbrackright}{\kern0pt}{\isacharcolon}{\kern0pt}\ {\isachardoublequoteopen}y\ {\isasymin}\isactrlsub c\ Y{\isachardoublequoteclose}\isanewline
\ \ \ \ \ \ \ \ \ \ \isacommand{assume}\isamarkupfalse%
\ z{\isacharunderscore}{\kern0pt}type{\isacharbrackleft}{\kern0pt}type{\isacharunderscore}{\kern0pt}rule{\isacharbrackright}{\kern0pt}{\isacharcolon}{\kern0pt}\ {\isachardoublequoteopen}z\ {\isasymin}\isactrlsub c\ X{\isachardoublequoteclose}\isanewline
\ \ \ \ \ \ \ \ \ \ \isacommand{assume}\isamarkupfalse%
\ {\isachardoublequoteopen}z\ {\isasymnoteq}\ x{\isachardoublequoteclose}\isanewline
\ \ \ \ \ \ \ \ \ \ \isacommand{assume}\isamarkupfalse%
\ {\isachardoublequoteopen}y\ {\isasymnoteq}\ y{\isadigit{1}}{\isachardoublequoteclose}\isanewline
\ \ \ \ \ \ \ \ \ \ \isacommand{have}\isamarkupfalse%
\ {\isachardoublequoteopen}{\isacharparenleft}{\kern0pt}{\isasymTheta}\ {\isasymcirc}\isactrlsub c\ {\isasymlangle}x{\isacharcomma}{\kern0pt}\ y{\isasymrangle}{\isacharparenright}{\kern0pt}\isactrlsup {\isasymflat}\ {\isasymcirc}\isactrlsub c\ {\isasymlangle}id\ X{\isacharcomma}{\kern0pt}\ {\isasymbeta}\isactrlbsub X\isactrlesub {\isasymrangle}\ {\isasymcirc}\isactrlsub c\ z\ {\isacharequal}{\kern0pt}\ into\ {\isasymcirc}\isactrlsub c\ \ \ {\isasymlangle}y{\isacharcomma}{\kern0pt}\ {\isasymlangle}x{\isacharcomma}{\kern0pt}\ z{\isasymrangle}{\isasymrangle}{\isachardoublequoteclose}\isanewline
\ \ \ \ \ \ \ \ \ \ \ \ \isacommand{by}\isamarkupfalse%
\ {\isacharparenleft}{\kern0pt}simp\ add{\isacharcolon}{\kern0pt}\ f{\isadigit{0}}\ x{\isacharunderscore}{\kern0pt}type\ y{\isacharunderscore}{\kern0pt}type\ z{\isacharunderscore}{\kern0pt}type{\isacharparenright}{\kern0pt}\isanewline
\ \ \ \ \ \ \ \ \ \ \isacommand{also}\isamarkupfalse%
\ \isacommand{have}\isamarkupfalse%
\ {\isachardoublequoteopen}{\isachardot}{\kern0pt}{\isachardot}{\kern0pt}{\isachardot}{\kern0pt}\ {\isacharequal}{\kern0pt}\ {\isacharparenleft}{\kern0pt}left{\isacharunderscore}{\kern0pt}cart{\isacharunderscore}{\kern0pt}proj\ Y\ {\isasymone}\ {\isasymamalg}\ {\isacharparenleft}{\kern0pt}{\isacharparenleft}{\kern0pt}y{\isadigit{2}}\ {\isasymamalg}\ y{\isadigit{1}}{\isacharparenright}{\kern0pt}\ {\isasymcirc}\isactrlsub c\ case{\isacharunderscore}{\kern0pt}bool\ {\isasymcirc}\isactrlsub c\ eq{\isacharunderscore}{\kern0pt}pred\ Y\ {\isasymcirc}\isactrlsub c\ {\isacharparenleft}{\kern0pt}id\ Y\ {\isasymtimes}\isactrlsub f\ y{\isadigit{1}}{\isacharparenright}{\kern0pt}{\isacharparenright}{\kern0pt}{\isacharparenright}{\kern0pt}\isanewline
\ \ \ \ \ \ \ \ \ \ \ \ \ \ \ \ \ \ \ \ \ \ \ \ \ \ \ \ \ \ \ \ \ {\isasymcirc}\isactrlsub c\ dist{\isacharunderscore}{\kern0pt}prod{\isacharunderscore}{\kern0pt}coprod{\isacharunderscore}{\kern0pt}left\ Y\ {\isasymone}\ {\isasymone}\ {\isasymcirc}\isactrlsub c\ {\isacharparenleft}{\kern0pt}id\ Y\ {\isasymtimes}\isactrlsub f\ case{\isacharunderscore}{\kern0pt}bool{\isacharparenright}{\kern0pt}\ {\isasymcirc}\isactrlsub c\ {\isacharparenleft}{\kern0pt}id\ Y\ {\isasymtimes}\isactrlsub f\ eq{\isacharunderscore}{\kern0pt}pred\ X{\isacharparenright}{\kern0pt}\ {\isasymcirc}\isactrlsub c\ \ \ {\isasymlangle}y{\isacharcomma}{\kern0pt}\ {\isasymlangle}x{\isacharcomma}{\kern0pt}\ z{\isasymrangle}{\isasymrangle}{\isachardoublequoteclose}\isanewline
\ \ \ \ \ \ \ \ \ \ \ \ \isacommand{using}\isamarkupfalse%
\ cfunc{\isacharunderscore}{\kern0pt}type{\isacharunderscore}{\kern0pt}def\ comp{\isacharunderscore}{\kern0pt}associative\ comp{\isacharunderscore}{\kern0pt}type\ into{\isacharunderscore}{\kern0pt}def\ \isacommand{by}\isamarkupfalse%
\ {\isacharparenleft}{\kern0pt}typecheck{\isacharunderscore}{\kern0pt}cfuncs{\isacharcomma}{\kern0pt}\ fastforce{\isacharparenright}{\kern0pt}\isanewline
\ \ \ \ \ \ \ \ \ \ \isacommand{also}\isamarkupfalse%
\ \isacommand{have}\isamarkupfalse%
\ {\isachardoublequoteopen}{\isachardot}{\kern0pt}{\isachardot}{\kern0pt}{\isachardot}{\kern0pt}\ {\isacharequal}{\kern0pt}\ {\isacharparenleft}{\kern0pt}left{\isacharunderscore}{\kern0pt}cart{\isacharunderscore}{\kern0pt}proj\ Y\ {\isasymone}\ {\isasymamalg}\ {\isacharparenleft}{\kern0pt}{\isacharparenleft}{\kern0pt}y{\isadigit{2}}\ {\isasymamalg}\ y{\isadigit{1}}{\isacharparenright}{\kern0pt}\ {\isasymcirc}\isactrlsub c\ case{\isacharunderscore}{\kern0pt}bool\ {\isasymcirc}\isactrlsub c\ eq{\isacharunderscore}{\kern0pt}pred\ Y\ {\isasymcirc}\isactrlsub c\ {\isacharparenleft}{\kern0pt}id\ Y\ {\isasymtimes}\isactrlsub f\ y{\isadigit{1}}{\isacharparenright}{\kern0pt}{\isacharparenright}{\kern0pt}{\isacharparenright}{\kern0pt}\isanewline
\ \ \ \ \ \ \ \ \ \ \ \ \ \ \ \ \ \ \ \ \ \ \ \ \ \ \ \ \ \ \ \ \ {\isasymcirc}\isactrlsub c\ dist{\isacharunderscore}{\kern0pt}prod{\isacharunderscore}{\kern0pt}coprod{\isacharunderscore}{\kern0pt}left\ Y\ {\isasymone}\ {\isasymone}\ {\isasymcirc}\isactrlsub c\ {\isacharparenleft}{\kern0pt}id\ Y\ {\isasymtimes}\isactrlsub f\ case{\isacharunderscore}{\kern0pt}bool{\isacharparenright}{\kern0pt}\ {\isasymcirc}\isactrlsub c\ \ {\isasymlangle}id\ Y\ {\isasymcirc}\isactrlsub c\ y{\isacharcomma}{\kern0pt}\ eq{\isacharunderscore}{\kern0pt}pred\ X\ {\isasymcirc}\isactrlsub c\ \ {\isasymlangle}x{\isacharcomma}{\kern0pt}\ z{\isasymrangle}{\isasymrangle}{\isachardoublequoteclose}\isanewline
\ \ \ \ \ \ \ \ \ \ \ \ \isacommand{by}\isamarkupfalse%
\ {\isacharparenleft}{\kern0pt}typecheck{\isacharunderscore}{\kern0pt}cfuncs{\isacharcomma}{\kern0pt}\ simp\ add{\isacharcolon}{\kern0pt}\ cfunc{\isacharunderscore}{\kern0pt}cross{\isacharunderscore}{\kern0pt}prod{\isacharunderscore}{\kern0pt}comp{\isacharunderscore}{\kern0pt}cfunc{\isacharunderscore}{\kern0pt}prod{\isacharparenright}{\kern0pt}\isanewline
\ \ \ \ \ \ \ \ \ \ \isacommand{also}\isamarkupfalse%
\ \isacommand{have}\isamarkupfalse%
\ {\isachardoublequoteopen}{\isachardot}{\kern0pt}{\isachardot}{\kern0pt}{\isachardot}{\kern0pt}\ {\isacharequal}{\kern0pt}\ {\isacharparenleft}{\kern0pt}left{\isacharunderscore}{\kern0pt}cart{\isacharunderscore}{\kern0pt}proj\ Y\ {\isasymone}\ {\isasymamalg}\ {\isacharparenleft}{\kern0pt}{\isacharparenleft}{\kern0pt}y{\isadigit{2}}\ {\isasymamalg}\ y{\isadigit{1}}{\isacharparenright}{\kern0pt}\ {\isasymcirc}\isactrlsub c\ case{\isacharunderscore}{\kern0pt}bool\ {\isasymcirc}\isactrlsub c\ eq{\isacharunderscore}{\kern0pt}pred\ Y\ {\isasymcirc}\isactrlsub c\ {\isacharparenleft}{\kern0pt}id\ Y\ {\isasymtimes}\isactrlsub f\ y{\isadigit{1}}{\isacharparenright}{\kern0pt}{\isacharparenright}{\kern0pt}{\isacharparenright}{\kern0pt}\ \isanewline
\ \ \ \ \ \ \ \ \ \ \ \ \ \ \ \ \ \ \ \ \ \ \ \ \ \ \ \ \ \ \ \ \ {\isasymcirc}\isactrlsub c\ dist{\isacharunderscore}{\kern0pt}prod{\isacharunderscore}{\kern0pt}coprod{\isacharunderscore}{\kern0pt}left\ Y\ {\isasymone}\ {\isasymone}\ {\isasymcirc}\isactrlsub c\ {\isacharparenleft}{\kern0pt}id\ Y\ {\isasymtimes}\isactrlsub f\ case{\isacharunderscore}{\kern0pt}bool{\isacharparenright}{\kern0pt}\ {\isasymcirc}\isactrlsub c\ \ {\isasymlangle}y{\isacharcomma}{\kern0pt}\ {\isasymf}{\isasymrangle}{\isachardoublequoteclose}\isanewline
\ \ \ \ \ \ \ \ \ \ \ \ \isacommand{by}\isamarkupfalse%
\ {\isacharparenleft}{\kern0pt}typecheck{\isacharunderscore}{\kern0pt}cfuncs{\isacharcomma}{\kern0pt}\ metis\ {\isacartoucheopen}z\ {\isasymnoteq}\ x{\isacartoucheclose}\ eq{\isacharunderscore}{\kern0pt}pred{\isacharunderscore}{\kern0pt}iff{\isacharunderscore}{\kern0pt}eq{\isacharunderscore}{\kern0pt}conv\ id{\isacharunderscore}{\kern0pt}left{\isacharunderscore}{\kern0pt}unit{\isadigit{2}}{\isacharparenright}{\kern0pt}\isanewline
\ \ \ \ \ \ \ \ \ \ \isacommand{also}\isamarkupfalse%
\ \isacommand{have}\isamarkupfalse%
\ {\isachardoublequoteopen}{\isachardot}{\kern0pt}{\isachardot}{\kern0pt}{\isachardot}{\kern0pt}\ {\isacharequal}{\kern0pt}\ {\isacharparenleft}{\kern0pt}left{\isacharunderscore}{\kern0pt}cart{\isacharunderscore}{\kern0pt}proj\ Y\ {\isasymone}\ {\isasymamalg}\ {\isacharparenleft}{\kern0pt}{\isacharparenleft}{\kern0pt}y{\isadigit{2}}\ {\isasymamalg}\ y{\isadigit{1}}{\isacharparenright}{\kern0pt}\ {\isasymcirc}\isactrlsub c\ case{\isacharunderscore}{\kern0pt}bool\ {\isasymcirc}\isactrlsub c\ eq{\isacharunderscore}{\kern0pt}pred\ Y\ {\isasymcirc}\isactrlsub c\ {\isacharparenleft}{\kern0pt}id\ Y\ {\isasymtimes}\isactrlsub f\ y{\isadigit{1}}{\isacharparenright}{\kern0pt}{\isacharparenright}{\kern0pt}{\isacharparenright}{\kern0pt}\ \isanewline
\ \ \ \ \ \ \ \ \ \ \ \ \ \ \ \ \ \ \ \ \ \ \ \ \ \ \ \ \ \ \ \ \ {\isasymcirc}\isactrlsub c\ dist{\isacharunderscore}{\kern0pt}prod{\isacharunderscore}{\kern0pt}coprod{\isacharunderscore}{\kern0pt}left\ Y\ {\isasymone}\ {\isasymone}\ \ {\isasymcirc}\isactrlsub c\ \ {\isasymlangle}y{\isacharcomma}{\kern0pt}\ right{\isacharunderscore}{\kern0pt}coproj\ {\isasymone}\ {\isasymone}{\isasymrangle}{\isachardoublequoteclose}\isanewline
\ \ \ \ \ \ \ \ \ \ \ \ \isacommand{by}\isamarkupfalse%
\ {\isacharparenleft}{\kern0pt}typecheck{\isacharunderscore}{\kern0pt}cfuncs{\isacharcomma}{\kern0pt}\ simp\ add{\isacharcolon}{\kern0pt}\ case{\isacharunderscore}{\kern0pt}bool{\isacharunderscore}{\kern0pt}false\ cfunc{\isacharunderscore}{\kern0pt}cross{\isacharunderscore}{\kern0pt}prod{\isacharunderscore}{\kern0pt}comp{\isacharunderscore}{\kern0pt}cfunc{\isacharunderscore}{\kern0pt}prod\ id{\isacharunderscore}{\kern0pt}left{\isacharunderscore}{\kern0pt}unit{\isadigit{2}}{\isacharparenright}{\kern0pt}\isanewline
\ \ \ \ \ \ \ \ \ \ \isacommand{also}\isamarkupfalse%
\ \isacommand{have}\isamarkupfalse%
\ {\isachardoublequoteopen}{\isachardot}{\kern0pt}{\isachardot}{\kern0pt}{\isachardot}{\kern0pt}\ {\isacharequal}{\kern0pt}\ {\isacharparenleft}{\kern0pt}left{\isacharunderscore}{\kern0pt}cart{\isacharunderscore}{\kern0pt}proj\ Y\ {\isasymone}\ {\isasymamalg}\ {\isacharparenleft}{\kern0pt}{\isacharparenleft}{\kern0pt}y{\isadigit{2}}\ {\isasymamalg}\ y{\isadigit{1}}{\isacharparenright}{\kern0pt}\ {\isasymcirc}\isactrlsub c\ case{\isacharunderscore}{\kern0pt}bool\ {\isasymcirc}\isactrlsub c\ eq{\isacharunderscore}{\kern0pt}pred\ Y\ {\isasymcirc}\isactrlsub c\ {\isacharparenleft}{\kern0pt}id\ Y\ {\isasymtimes}\isactrlsub f\ y{\isadigit{1}}{\isacharparenright}{\kern0pt}{\isacharparenright}{\kern0pt}{\isacharparenright}{\kern0pt}\isanewline
\ \ \ \ \ \ \ \ \ \ \ \ \ \ \ \ \ \ \ \ \ \ \ \ \ \ \ \ \ \ \ \ \ {\isasymcirc}\isactrlsub c\ dist{\isacharunderscore}{\kern0pt}prod{\isacharunderscore}{\kern0pt}coprod{\isacharunderscore}{\kern0pt}left\ Y\ {\isasymone}\ {\isasymone}\ \ {\isasymcirc}\isactrlsub c\ \ {\isasymlangle}y{\isacharcomma}{\kern0pt}\ right{\isacharunderscore}{\kern0pt}coproj\ {\isasymone}\ {\isasymone}\ {\isasymcirc}\isactrlsub c\ id\ {\isasymone}{\isasymrangle}{\isachardoublequoteclose}\isanewline
\ \ \ \ \ \ \ \ \ \ \ \ \isacommand{by}\isamarkupfalse%
\ {\isacharparenleft}{\kern0pt}typecheck{\isacharunderscore}{\kern0pt}cfuncs{\isacharcomma}{\kern0pt}\ simp\ add{\isacharcolon}{\kern0pt}\ id{\isacharunderscore}{\kern0pt}right{\isacharunderscore}{\kern0pt}unit{\isadigit{2}}{\isacharparenright}{\kern0pt}\isanewline
\ \ \ \ \ \ \ \ \ \ \isacommand{also}\isamarkupfalse%
\ \isacommand{have}\isamarkupfalse%
\ {\isachardoublequoteopen}{\isachardot}{\kern0pt}{\isachardot}{\kern0pt}{\isachardot}{\kern0pt}\ {\isacharequal}{\kern0pt}\ {\isacharparenleft}{\kern0pt}left{\isacharunderscore}{\kern0pt}cart{\isacharunderscore}{\kern0pt}proj\ Y\ {\isasymone}\ {\isasymamalg}\ {\isacharparenleft}{\kern0pt}{\isacharparenleft}{\kern0pt}y{\isadigit{2}}\ {\isasymamalg}\ y{\isadigit{1}}{\isacharparenright}{\kern0pt}\ {\isasymcirc}\isactrlsub c\ case{\isacharunderscore}{\kern0pt}bool\ {\isasymcirc}\isactrlsub c\ eq{\isacharunderscore}{\kern0pt}pred\ Y\ {\isasymcirc}\isactrlsub c\ {\isacharparenleft}{\kern0pt}id\ Y\ {\isasymtimes}\isactrlsub f\ y{\isadigit{1}}{\isacharparenright}{\kern0pt}{\isacharparenright}{\kern0pt}{\isacharparenright}{\kern0pt}\isanewline
\ \ \ \ \ \ \ \ \ \ \ \ \ \ \ \ \ \ \ \ \ \ \ \ \ \ \ \ \ \ \ \ \ {\isasymcirc}\isactrlsub c\ right{\isacharunderscore}{\kern0pt}coproj\ {\isacharparenleft}{\kern0pt}Y\ {\isasymtimes}\isactrlsub c\ {\isasymone}{\isacharparenright}{\kern0pt}\ {\isacharparenleft}{\kern0pt}Y\ {\isasymtimes}\isactrlsub c\ {\isasymone}{\isacharparenright}{\kern0pt}\ {\isasymcirc}\isactrlsub c\ {\isasymlangle}y{\isacharcomma}{\kern0pt}id\ {\isasymone}{\isasymrangle}{\isachardoublequoteclose}\isanewline
\ \ \ \ \ \ \ \ \ \ \ \ \isacommand{using}\isamarkupfalse%
\ dist{\isacharunderscore}{\kern0pt}prod{\isacharunderscore}{\kern0pt}coprod{\isacharunderscore}{\kern0pt}left{\isacharunderscore}{\kern0pt}ap{\isacharunderscore}{\kern0pt}right\ \isacommand{by}\isamarkupfalse%
\ {\isacharparenleft}{\kern0pt}typecheck{\isacharunderscore}{\kern0pt}cfuncs{\isacharcomma}{\kern0pt}\ auto{\isacharparenright}{\kern0pt}\isanewline
\ \ \ \ \ \ \ \ \ \ \isacommand{also}\isamarkupfalse%
\ \isacommand{have}\isamarkupfalse%
\ {\isachardoublequoteopen}{\isachardot}{\kern0pt}{\isachardot}{\kern0pt}{\isachardot}{\kern0pt}\ {\isacharequal}{\kern0pt}\ {\isacharparenleft}{\kern0pt}{\isacharparenleft}{\kern0pt}left{\isacharunderscore}{\kern0pt}cart{\isacharunderscore}{\kern0pt}proj\ Y\ {\isasymone}\ {\isasymamalg}\ {\isacharparenleft}{\kern0pt}{\isacharparenleft}{\kern0pt}y{\isadigit{2}}\ {\isasymamalg}\ y{\isadigit{1}}{\isacharparenright}{\kern0pt}\ {\isasymcirc}\isactrlsub c\ case{\isacharunderscore}{\kern0pt}bool\ {\isasymcirc}\isactrlsub c\ eq{\isacharunderscore}{\kern0pt}pred\ Y\ {\isasymcirc}\isactrlsub c\ {\isacharparenleft}{\kern0pt}id\ Y\ {\isasymtimes}\isactrlsub f\ y{\isadigit{1}}{\isacharparenright}{\kern0pt}{\isacharparenright}{\kern0pt}{\isacharparenright}{\kern0pt}\ \isanewline
\ \ \ \ \ \ \ \ \ \ \ \ \ \ \ \ \ \ \ \ \ \ \ \ \ \ \ \ \ \ \ \ \ {\isasymcirc}\isactrlsub c\ right{\isacharunderscore}{\kern0pt}coproj\ {\isacharparenleft}{\kern0pt}Y\ {\isasymtimes}\isactrlsub c\ {\isasymone}{\isacharparenright}{\kern0pt}\ {\isacharparenleft}{\kern0pt}Y\ {\isasymtimes}\isactrlsub c\ {\isasymone}{\isacharparenright}{\kern0pt}{\isacharparenright}{\kern0pt}\ {\isasymcirc}\isactrlsub c\ {\isasymlangle}y{\isacharcomma}{\kern0pt}id\ {\isasymone}{\isasymrangle}{\isachardoublequoteclose}\isanewline
\ \ \ \ \ \ \ \ \ \ \ \ \isacommand{by}\isamarkupfalse%
\ {\isacharparenleft}{\kern0pt}typecheck{\isacharunderscore}{\kern0pt}cfuncs{\isacharcomma}{\kern0pt}\ meson\ comp{\isacharunderscore}{\kern0pt}associative{\isadigit{2}}{\isacharparenright}{\kern0pt}\isanewline
\ \ \ \ \ \ \ \ \ \ \isacommand{also}\isamarkupfalse%
\ \isacommand{have}\isamarkupfalse%
\ {\isachardoublequoteopen}{\isachardot}{\kern0pt}{\isachardot}{\kern0pt}{\isachardot}{\kern0pt}\ {\isacharequal}{\kern0pt}\ {\isacharparenleft}{\kern0pt}{\isacharparenleft}{\kern0pt}y{\isadigit{2}}\ {\isasymamalg}\ y{\isadigit{1}}{\isacharparenright}{\kern0pt}\ {\isasymcirc}\isactrlsub c\ case{\isacharunderscore}{\kern0pt}bool\ {\isasymcirc}\isactrlsub c\ eq{\isacharunderscore}{\kern0pt}pred\ Y\ {\isasymcirc}\isactrlsub c\ {\isacharparenleft}{\kern0pt}id\ Y\ {\isasymtimes}\isactrlsub f\ y{\isadigit{1}}{\isacharparenright}{\kern0pt}{\isacharparenright}{\kern0pt}\ {\isasymcirc}\isactrlsub c\ {\isasymlangle}y{\isacharcomma}{\kern0pt}id\ {\isasymone}{\isasymrangle}{\isachardoublequoteclose}\isanewline
\ \ \ \ \ \ \ \ \ \ \ \ \isacommand{using}\isamarkupfalse%
\ right{\isacharunderscore}{\kern0pt}coproj{\isacharunderscore}{\kern0pt}cfunc{\isacharunderscore}{\kern0pt}coprod\ \isacommand{by}\isamarkupfalse%
\ {\isacharparenleft}{\kern0pt}typecheck{\isacharunderscore}{\kern0pt}cfuncs{\isacharcomma}{\kern0pt}\ auto{\isacharparenright}{\kern0pt}\isanewline
\ \ \ \ \ \ \ \ \ \ \isacommand{also}\isamarkupfalse%
\ \isacommand{have}\isamarkupfalse%
\ {\isachardoublequoteopen}{\isachardot}{\kern0pt}{\isachardot}{\kern0pt}{\isachardot}{\kern0pt}\ {\isacharequal}{\kern0pt}\ {\isacharparenleft}{\kern0pt}y{\isadigit{2}}\ {\isasymamalg}\ y{\isadigit{1}}{\isacharparenright}{\kern0pt}\ {\isasymcirc}\isactrlsub c\ case{\isacharunderscore}{\kern0pt}bool\ {\isasymcirc}\isactrlsub c\ eq{\isacharunderscore}{\kern0pt}pred\ Y\ {\isasymcirc}\isactrlsub c\ {\isacharparenleft}{\kern0pt}id\ Y\ {\isasymtimes}\isactrlsub f\ y{\isadigit{1}}{\isacharparenright}{\kern0pt}\ {\isasymcirc}\isactrlsub c\ {\isasymlangle}y{\isacharcomma}{\kern0pt}id\ {\isasymone}{\isasymrangle}{\isachardoublequoteclose}\isanewline
\ \ \ \ \ \ \ \ \ \ \ \ \isacommand{using}\isamarkupfalse%
\ comp{\isacharunderscore}{\kern0pt}associative{\isadigit{2}}\ \isacommand{by}\isamarkupfalse%
\ {\isacharparenleft}{\kern0pt}typecheck{\isacharunderscore}{\kern0pt}cfuncs{\isacharcomma}{\kern0pt}\ force{\isacharparenright}{\kern0pt}\isanewline
\ \ \ \ \ \ \ \ \ \ \isacommand{also}\isamarkupfalse%
\ \isacommand{have}\isamarkupfalse%
\ {\isachardoublequoteopen}{\isachardot}{\kern0pt}{\isachardot}{\kern0pt}{\isachardot}{\kern0pt}\ {\isacharequal}{\kern0pt}\ {\isacharparenleft}{\kern0pt}y{\isadigit{2}}\ {\isasymamalg}\ y{\isadigit{1}}{\isacharparenright}{\kern0pt}\ {\isasymcirc}\isactrlsub c\ case{\isacharunderscore}{\kern0pt}bool\ {\isasymcirc}\isactrlsub c\ eq{\isacharunderscore}{\kern0pt}pred\ Y\ \ {\isasymcirc}\isactrlsub c\ {\isasymlangle}y{\isacharcomma}{\kern0pt}y{\isadigit{1}}{\isasymrangle}{\isachardoublequoteclose}\isanewline
\ \ \ \ \ \ \ \ \ \ \ \ \isacommand{by}\isamarkupfalse%
\ {\isacharparenleft}{\kern0pt}typecheck{\isacharunderscore}{\kern0pt}cfuncs{\isacharcomma}{\kern0pt}\ simp\ add{\isacharcolon}{\kern0pt}\ cfunc{\isacharunderscore}{\kern0pt}cross{\isacharunderscore}{\kern0pt}prod{\isacharunderscore}{\kern0pt}comp{\isacharunderscore}{\kern0pt}cfunc{\isacharunderscore}{\kern0pt}prod\ id{\isacharunderscore}{\kern0pt}left{\isacharunderscore}{\kern0pt}unit{\isadigit{2}}\ id{\isacharunderscore}{\kern0pt}right{\isacharunderscore}{\kern0pt}unit{\isadigit{2}}{\isacharparenright}{\kern0pt}\isanewline
\ \ \ \ \ \ \ \ \ \ \isacommand{also}\isamarkupfalse%
\ \isacommand{have}\isamarkupfalse%
\ {\isachardoublequoteopen}{\isachardot}{\kern0pt}{\isachardot}{\kern0pt}{\isachardot}{\kern0pt}\ {\isacharequal}{\kern0pt}\ {\isacharparenleft}{\kern0pt}y{\isadigit{2}}\ {\isasymamalg}\ y{\isadigit{1}}{\isacharparenright}{\kern0pt}\ {\isasymcirc}\isactrlsub c\ case{\isacharunderscore}{\kern0pt}bool\ {\isasymcirc}\isactrlsub c\ {\isasymf}{\isachardoublequoteclose}\isanewline
\ \ \ \ \ \ \ \ \ \ \ \ \isacommand{by}\isamarkupfalse%
\ {\isacharparenleft}{\kern0pt}typecheck{\isacharunderscore}{\kern0pt}cfuncs{\isacharcomma}{\kern0pt}\ metis\ {\isacartoucheopen}y\ {\isasymnoteq}\ y{\isadigit{1}}{\isacartoucheclose}\ eq{\isacharunderscore}{\kern0pt}pred{\isacharunderscore}{\kern0pt}iff{\isacharunderscore}{\kern0pt}eq{\isacharunderscore}{\kern0pt}conv{\isacharparenright}{\kern0pt}\isanewline
\ \ \ \ \ \ \ \ \ \ \isacommand{also}\isamarkupfalse%
\ \isacommand{have}\isamarkupfalse%
\ {\isachardoublequoteopen}{\isachardot}{\kern0pt}{\isachardot}{\kern0pt}{\isachardot}{\kern0pt}\ {\isacharequal}{\kern0pt}\ y{\isadigit{1}}{\isachardoublequoteclose}\isanewline
\ \ \ \ \ \ \ \ \ \ \ \ \isacommand{using}\isamarkupfalse%
\ case{\isacharunderscore}{\kern0pt}bool{\isacharunderscore}{\kern0pt}false\ right{\isacharunderscore}{\kern0pt}coproj{\isacharunderscore}{\kern0pt}cfunc{\isacharunderscore}{\kern0pt}coprod\ \isacommand{by}\isamarkupfalse%
\ {\isacharparenleft}{\kern0pt}typecheck{\isacharunderscore}{\kern0pt}cfuncs{\isacharcomma}{\kern0pt}\ presburger{\isacharparenright}{\kern0pt}\isanewline
\ \ \ \ \ \ \ \ \ \ \isacommand{then}\isamarkupfalse%
\ \isacommand{show}\isamarkupfalse%
\ {\isachardoublequoteopen}{\isacharparenleft}{\kern0pt}{\isasymTheta}\ {\isasymcirc}\isactrlsub c\ {\isasymlangle}x{\isacharcomma}{\kern0pt}\ y{\isasymrangle}{\isacharparenright}{\kern0pt}\isactrlsup {\isasymflat}\ {\isasymcirc}\isactrlsub c\ {\isasymlangle}id\ X{\isacharcomma}{\kern0pt}\ {\isasymbeta}\isactrlbsub X\isactrlesub {\isasymrangle}\ {\isasymcirc}\isactrlsub c\ z\ {\isacharequal}{\kern0pt}\ y{\isadigit{1}}{\isachardoublequoteclose}\isanewline
\ \ \ \ \ \ \ \ \ \ \ \ \isacommand{by}\isamarkupfalse%
\ {\isacharparenleft}{\kern0pt}simp\ add{\isacharcolon}{\kern0pt}\ calculation{\isacharparenright}{\kern0pt}\isanewline
\ \ \ \ \ \ \ \ \isacommand{qed}\isamarkupfalse%
\isanewline
\ \ \ \ \ \ \ \ \isanewline
\ \ \ \ \ \ \ \ \isacommand{have}\isamarkupfalse%
\ f{\isadigit{3}}{\isacharcolon}{\kern0pt}\ {\isachardoublequoteopen}{\isasymAnd}x\ z{\isachardot}{\kern0pt}\ x\ {\isasymin}\isactrlsub c\ X\ {\isasymLongrightarrow}\ \ z\ {\isasymin}\isactrlsub c\ X\ {\isasymLongrightarrow}\ z\ {\isasymnoteq}\ x\ {\isasymLongrightarrow}\ \ {\isacharparenleft}{\kern0pt}{\isasymTheta}\ {\isasymcirc}\isactrlsub c\ {\isasymlangle}x{\isacharcomma}{\kern0pt}\ y{\isadigit{1}}{\isasymrangle}{\isacharparenright}{\kern0pt}\isactrlsup {\isasymflat}\ {\isasymcirc}\isactrlsub c\ {\isasymlangle}id\ X{\isacharcomma}{\kern0pt}\ {\isasymbeta}\isactrlbsub X\isactrlesub {\isasymrangle}\ {\isasymcirc}\isactrlsub c\ z\ {\isacharequal}{\kern0pt}\ y{\isadigit{2}}{\isachardoublequoteclose}\isanewline
\ \ \ \ \ \ \ \ \isacommand{proof}\isamarkupfalse%
\ {\isacharminus}{\kern0pt}\ \isanewline
\ \ \ \ \ \ \ \ \ \ \isacommand{fix}\isamarkupfalse%
\ x\ y\ z\isanewline
\ \ \ \ \ \ \ \ \ \ \isacommand{assume}\isamarkupfalse%
\ x{\isacharunderscore}{\kern0pt}type{\isacharbrackleft}{\kern0pt}type{\isacharunderscore}{\kern0pt}rule{\isacharbrackright}{\kern0pt}{\isacharcolon}{\kern0pt}\ {\isachardoublequoteopen}x\ {\isasymin}\isactrlsub c\ X{\isachardoublequoteclose}\isanewline
\ \ \ \ \ \ \ \ \ \ \isacommand{assume}\isamarkupfalse%
\ z{\isacharunderscore}{\kern0pt}type{\isacharbrackleft}{\kern0pt}type{\isacharunderscore}{\kern0pt}rule{\isacharbrackright}{\kern0pt}{\isacharcolon}{\kern0pt}\ {\isachardoublequoteopen}z\ {\isasymin}\isactrlsub c\ X{\isachardoublequoteclose}\isanewline
\ \ \ \ \ \ \ \ \ \ \isacommand{assume}\isamarkupfalse%
\ {\isachardoublequoteopen}z\ {\isasymnoteq}\ x{\isachardoublequoteclose}\isanewline
\ \ \ \ \ \ \ \ \ \ \isacommand{have}\isamarkupfalse%
\ {\isachardoublequoteopen}{\isacharparenleft}{\kern0pt}{\isasymTheta}\ {\isasymcirc}\isactrlsub c\ {\isasymlangle}x{\isacharcomma}{\kern0pt}\ y{\isadigit{1}}{\isasymrangle}{\isacharparenright}{\kern0pt}\isactrlsup {\isasymflat}\ {\isasymcirc}\isactrlsub c\ {\isasymlangle}id\ X{\isacharcomma}{\kern0pt}\ {\isasymbeta}\isactrlbsub X\isactrlesub {\isasymrangle}\ {\isasymcirc}\isactrlsub c\ z\ {\isacharequal}{\kern0pt}\ into\ {\isasymcirc}\isactrlsub c\ \ \ {\isasymlangle}y{\isadigit{1}}{\isacharcomma}{\kern0pt}\ {\isasymlangle}x{\isacharcomma}{\kern0pt}\ z{\isasymrangle}{\isasymrangle}{\isachardoublequoteclose}\isanewline
\ \ \ \ \ \ \ \ \ \ \ \ \isacommand{by}\isamarkupfalse%
\ {\isacharparenleft}{\kern0pt}simp\ add{\isacharcolon}{\kern0pt}\ f{\isadigit{0}}\ x{\isacharunderscore}{\kern0pt}type\ y{\isadigit{1}}{\isacharunderscore}{\kern0pt}type\ z{\isacharunderscore}{\kern0pt}type{\isacharparenright}{\kern0pt}\isanewline
\ \ \ \ \ \ \ \ \ \ \isacommand{also}\isamarkupfalse%
\ \isacommand{have}\isamarkupfalse%
\ {\isachardoublequoteopen}{\isachardot}{\kern0pt}{\isachardot}{\kern0pt}{\isachardot}{\kern0pt}\ {\isacharequal}{\kern0pt}\ {\isacharparenleft}{\kern0pt}left{\isacharunderscore}{\kern0pt}cart{\isacharunderscore}{\kern0pt}proj\ Y\ {\isasymone}\ {\isasymamalg}\ {\isacharparenleft}{\kern0pt}{\isacharparenleft}{\kern0pt}y{\isadigit{2}}\ {\isasymamalg}\ y{\isadigit{1}}{\isacharparenright}{\kern0pt}\ {\isasymcirc}\isactrlsub c\ case{\isacharunderscore}{\kern0pt}bool\ {\isasymcirc}\isactrlsub c\ eq{\isacharunderscore}{\kern0pt}pred\ Y\ {\isasymcirc}\isactrlsub c\ {\isacharparenleft}{\kern0pt}id\ Y\ {\isasymtimes}\isactrlsub f\ y{\isadigit{1}}{\isacharparenright}{\kern0pt}{\isacharparenright}{\kern0pt}{\isacharparenright}{\kern0pt}\isanewline
\ \ \ \ \ \ \ \ \ \ \ \ \ \ \ \ \ \ \ \ \ \ \ \ \ \ \ \ \ \ \ \ \ {\isasymcirc}\isactrlsub c\ dist{\isacharunderscore}{\kern0pt}prod{\isacharunderscore}{\kern0pt}coprod{\isacharunderscore}{\kern0pt}left\ Y\ {\isasymone}\ {\isasymone}\ {\isasymcirc}\isactrlsub c\ {\isacharparenleft}{\kern0pt}id\ Y\ {\isasymtimes}\isactrlsub f\ case{\isacharunderscore}{\kern0pt}bool{\isacharparenright}{\kern0pt}\ {\isasymcirc}\isactrlsub c\ {\isacharparenleft}{\kern0pt}id\ Y\ {\isasymtimes}\isactrlsub f\ eq{\isacharunderscore}{\kern0pt}pred\ X{\isacharparenright}{\kern0pt}\ {\isasymcirc}\isactrlsub c\ \ \ {\isasymlangle}y{\isadigit{1}}{\isacharcomma}{\kern0pt}\ {\isasymlangle}x{\isacharcomma}{\kern0pt}\ z{\isasymrangle}{\isasymrangle}{\isachardoublequoteclose}\isanewline
\ \ \ \ \ \ \ \ \ \ \ \ \isacommand{using}\isamarkupfalse%
\ cfunc{\isacharunderscore}{\kern0pt}type{\isacharunderscore}{\kern0pt}def\ comp{\isacharunderscore}{\kern0pt}associative\ comp{\isacharunderscore}{\kern0pt}type\ into{\isacharunderscore}{\kern0pt}def\ \isacommand{by}\isamarkupfalse%
\ {\isacharparenleft}{\kern0pt}typecheck{\isacharunderscore}{\kern0pt}cfuncs{\isacharcomma}{\kern0pt}\ fastforce{\isacharparenright}{\kern0pt}\isanewline
\ \ \ \ \ \ \ \ \ \ \isacommand{also}\isamarkupfalse%
\ \isacommand{have}\isamarkupfalse%
\ {\isachardoublequoteopen}{\isachardot}{\kern0pt}{\isachardot}{\kern0pt}{\isachardot}{\kern0pt}\ {\isacharequal}{\kern0pt}\ {\isacharparenleft}{\kern0pt}left{\isacharunderscore}{\kern0pt}cart{\isacharunderscore}{\kern0pt}proj\ Y\ {\isasymone}\ {\isasymamalg}\ {\isacharparenleft}{\kern0pt}{\isacharparenleft}{\kern0pt}y{\isadigit{2}}\ {\isasymamalg}\ y{\isadigit{1}}{\isacharparenright}{\kern0pt}\ {\isasymcirc}\isactrlsub c\ case{\isacharunderscore}{\kern0pt}bool\ {\isasymcirc}\isactrlsub c\ eq{\isacharunderscore}{\kern0pt}pred\ Y\ {\isasymcirc}\isactrlsub c\ {\isacharparenleft}{\kern0pt}id\ Y\ {\isasymtimes}\isactrlsub f\ y{\isadigit{1}}{\isacharparenright}{\kern0pt}{\isacharparenright}{\kern0pt}{\isacharparenright}{\kern0pt}\isanewline
\ \ \ \ \ \ \ \ \ \ \ \ \ \ \ \ \ \ \ \ \ \ \ \ \ \ \ \ \ \ \ \ \ {\isasymcirc}\isactrlsub c\ dist{\isacharunderscore}{\kern0pt}prod{\isacharunderscore}{\kern0pt}coprod{\isacharunderscore}{\kern0pt}left\ Y\ {\isasymone}\ {\isasymone}\ {\isasymcirc}\isactrlsub c\ {\isacharparenleft}{\kern0pt}id\ Y\ {\isasymtimes}\isactrlsub f\ case{\isacharunderscore}{\kern0pt}bool{\isacharparenright}{\kern0pt}\ {\isasymcirc}\isactrlsub c\ \ {\isasymlangle}id\ Y\ {\isasymcirc}\isactrlsub c\ y{\isadigit{1}}{\isacharcomma}{\kern0pt}\ eq{\isacharunderscore}{\kern0pt}pred\ X\ {\isasymcirc}\isactrlsub c\ \ {\isasymlangle}x{\isacharcomma}{\kern0pt}\ z{\isasymrangle}{\isasymrangle}{\isachardoublequoteclose}\isanewline
\ \ \ \ \ \ \ \ \ \ \ \ \isacommand{by}\isamarkupfalse%
\ {\isacharparenleft}{\kern0pt}typecheck{\isacharunderscore}{\kern0pt}cfuncs{\isacharcomma}{\kern0pt}\ simp\ add{\isacharcolon}{\kern0pt}\ cfunc{\isacharunderscore}{\kern0pt}cross{\isacharunderscore}{\kern0pt}prod{\isacharunderscore}{\kern0pt}comp{\isacharunderscore}{\kern0pt}cfunc{\isacharunderscore}{\kern0pt}prod{\isacharparenright}{\kern0pt}\isanewline
\ \ \ \ \ \ \ \ \ \ \isacommand{also}\isamarkupfalse%
\ \isacommand{have}\isamarkupfalse%
\ {\isachardoublequoteopen}{\isachardot}{\kern0pt}{\isachardot}{\kern0pt}{\isachardot}{\kern0pt}\ {\isacharequal}{\kern0pt}\ {\isacharparenleft}{\kern0pt}left{\isacharunderscore}{\kern0pt}cart{\isacharunderscore}{\kern0pt}proj\ Y\ {\isasymone}\ {\isasymamalg}\ {\isacharparenleft}{\kern0pt}{\isacharparenleft}{\kern0pt}y{\isadigit{2}}\ {\isasymamalg}\ y{\isadigit{1}}{\isacharparenright}{\kern0pt}\ {\isasymcirc}\isactrlsub c\ case{\isacharunderscore}{\kern0pt}bool\ {\isasymcirc}\isactrlsub c\ eq{\isacharunderscore}{\kern0pt}pred\ Y\ {\isasymcirc}\isactrlsub c\ {\isacharparenleft}{\kern0pt}id\ Y\ {\isasymtimes}\isactrlsub f\ y{\isadigit{1}}{\isacharparenright}{\kern0pt}{\isacharparenright}{\kern0pt}{\isacharparenright}{\kern0pt}\ \isanewline
\ \ \ \ \ \ \ \ \ \ \ \ \ \ \ \ \ \ \ \ \ \ \ \ \ \ \ \ \ \ \ \ \ {\isasymcirc}\isactrlsub c\ dist{\isacharunderscore}{\kern0pt}prod{\isacharunderscore}{\kern0pt}coprod{\isacharunderscore}{\kern0pt}left\ Y\ {\isasymone}\ {\isasymone}\ {\isasymcirc}\isactrlsub c\ {\isacharparenleft}{\kern0pt}id\ Y\ {\isasymtimes}\isactrlsub f\ case{\isacharunderscore}{\kern0pt}bool{\isacharparenright}{\kern0pt}\ {\isasymcirc}\isactrlsub c\ \ {\isasymlangle}y{\isadigit{1}}{\isacharcomma}{\kern0pt}\ {\isasymf}{\isasymrangle}{\isachardoublequoteclose}\isanewline
\ \ \ \ \ \ \ \ \ \ \ \ \isacommand{by}\isamarkupfalse%
\ {\isacharparenleft}{\kern0pt}typecheck{\isacharunderscore}{\kern0pt}cfuncs{\isacharcomma}{\kern0pt}\ metis\ {\isacartoucheopen}z\ {\isasymnoteq}\ x{\isacartoucheclose}\ eq{\isacharunderscore}{\kern0pt}pred{\isacharunderscore}{\kern0pt}iff{\isacharunderscore}{\kern0pt}eq{\isacharunderscore}{\kern0pt}conv\ id{\isacharunderscore}{\kern0pt}left{\isacharunderscore}{\kern0pt}unit{\isadigit{2}}{\isacharparenright}{\kern0pt}\isanewline
\ \ \ \ \ \ \ \ \ \ \isacommand{also}\isamarkupfalse%
\ \isacommand{have}\isamarkupfalse%
\ {\isachardoublequoteopen}{\isachardot}{\kern0pt}{\isachardot}{\kern0pt}{\isachardot}{\kern0pt}\ {\isacharequal}{\kern0pt}\ {\isacharparenleft}{\kern0pt}left{\isacharunderscore}{\kern0pt}cart{\isacharunderscore}{\kern0pt}proj\ Y\ {\isasymone}\ {\isasymamalg}\ {\isacharparenleft}{\kern0pt}{\isacharparenleft}{\kern0pt}y{\isadigit{2}}\ {\isasymamalg}\ y{\isadigit{1}}{\isacharparenright}{\kern0pt}\ {\isasymcirc}\isactrlsub c\ case{\isacharunderscore}{\kern0pt}bool\ {\isasymcirc}\isactrlsub c\ eq{\isacharunderscore}{\kern0pt}pred\ Y\ {\isasymcirc}\isactrlsub c\ {\isacharparenleft}{\kern0pt}id\ Y\ {\isasymtimes}\isactrlsub f\ y{\isadigit{1}}{\isacharparenright}{\kern0pt}{\isacharparenright}{\kern0pt}{\isacharparenright}{\kern0pt}\ \isanewline
\ \ \ \ \ \ \ \ \ \ \ \ \ \ \ \ \ \ \ \ \ \ \ \ \ \ \ \ \ \ \ \ \ {\isasymcirc}\isactrlsub c\ dist{\isacharunderscore}{\kern0pt}prod{\isacharunderscore}{\kern0pt}coprod{\isacharunderscore}{\kern0pt}left\ Y\ {\isasymone}\ {\isasymone}\ \ {\isasymcirc}\isactrlsub c\ \ {\isasymlangle}y{\isadigit{1}}{\isacharcomma}{\kern0pt}\ right{\isacharunderscore}{\kern0pt}coproj\ {\isasymone}\ {\isasymone}{\isasymrangle}{\isachardoublequoteclose}\isanewline
\ \ \ \ \ \ \ \ \ \ \ \ \isacommand{by}\isamarkupfalse%
\ {\isacharparenleft}{\kern0pt}typecheck{\isacharunderscore}{\kern0pt}cfuncs{\isacharcomma}{\kern0pt}\ simp\ add{\isacharcolon}{\kern0pt}\ case{\isacharunderscore}{\kern0pt}bool{\isacharunderscore}{\kern0pt}false\ cfunc{\isacharunderscore}{\kern0pt}cross{\isacharunderscore}{\kern0pt}prod{\isacharunderscore}{\kern0pt}comp{\isacharunderscore}{\kern0pt}cfunc{\isacharunderscore}{\kern0pt}prod\ id{\isacharunderscore}{\kern0pt}left{\isacharunderscore}{\kern0pt}unit{\isadigit{2}}{\isacharparenright}{\kern0pt}\isanewline
\ \ \ \ \ \ \ \ \ \ \isacommand{also}\isamarkupfalse%
\ \isacommand{have}\isamarkupfalse%
\ {\isachardoublequoteopen}{\isachardot}{\kern0pt}{\isachardot}{\kern0pt}{\isachardot}{\kern0pt}\ {\isacharequal}{\kern0pt}\ {\isacharparenleft}{\kern0pt}left{\isacharunderscore}{\kern0pt}cart{\isacharunderscore}{\kern0pt}proj\ Y\ {\isasymone}\ {\isasymamalg}\ {\isacharparenleft}{\kern0pt}{\isacharparenleft}{\kern0pt}y{\isadigit{2}}\ {\isasymamalg}\ y{\isadigit{1}}{\isacharparenright}{\kern0pt}\ {\isasymcirc}\isactrlsub c\ case{\isacharunderscore}{\kern0pt}bool\ {\isasymcirc}\isactrlsub c\ eq{\isacharunderscore}{\kern0pt}pred\ Y\ {\isasymcirc}\isactrlsub c\ {\isacharparenleft}{\kern0pt}id\ Y\ {\isasymtimes}\isactrlsub f\ y{\isadigit{1}}{\isacharparenright}{\kern0pt}{\isacharparenright}{\kern0pt}{\isacharparenright}{\kern0pt}\isanewline
\ \ \ \ \ \ \ \ \ \ \ \ \ \ \ \ \ \ \ \ \ \ \ \ \ \ \ \ \ \ \ \ \ {\isasymcirc}\isactrlsub c\ dist{\isacharunderscore}{\kern0pt}prod{\isacharunderscore}{\kern0pt}coprod{\isacharunderscore}{\kern0pt}left\ Y\ {\isasymone}\ {\isasymone}\ \ {\isasymcirc}\isactrlsub c\ \ {\isasymlangle}y{\isadigit{1}}{\isacharcomma}{\kern0pt}\ right{\isacharunderscore}{\kern0pt}coproj\ {\isasymone}\ {\isasymone}\ {\isasymcirc}\isactrlsub c\ id\ {\isasymone}{\isasymrangle}{\isachardoublequoteclose}\isanewline
\ \ \ \ \ \ \ \ \ \ \ \ \isacommand{by}\isamarkupfalse%
\ {\isacharparenleft}{\kern0pt}typecheck{\isacharunderscore}{\kern0pt}cfuncs{\isacharcomma}{\kern0pt}\ simp\ add{\isacharcolon}{\kern0pt}\ id{\isacharunderscore}{\kern0pt}right{\isacharunderscore}{\kern0pt}unit{\isadigit{2}}{\isacharparenright}{\kern0pt}\isanewline
\ \ \ \ \ \ \ \ \ \ \isacommand{also}\isamarkupfalse%
\ \isacommand{have}\isamarkupfalse%
\ {\isachardoublequoteopen}{\isachardot}{\kern0pt}{\isachardot}{\kern0pt}{\isachardot}{\kern0pt}\ {\isacharequal}{\kern0pt}\ {\isacharparenleft}{\kern0pt}left{\isacharunderscore}{\kern0pt}cart{\isacharunderscore}{\kern0pt}proj\ Y\ {\isasymone}\ {\isasymamalg}\ {\isacharparenleft}{\kern0pt}{\isacharparenleft}{\kern0pt}y{\isadigit{2}}\ {\isasymamalg}\ y{\isadigit{1}}{\isacharparenright}{\kern0pt}\ {\isasymcirc}\isactrlsub c\ case{\isacharunderscore}{\kern0pt}bool\ {\isasymcirc}\isactrlsub c\ eq{\isacharunderscore}{\kern0pt}pred\ Y\ {\isasymcirc}\isactrlsub c\ {\isacharparenleft}{\kern0pt}id\ Y\ {\isasymtimes}\isactrlsub f\ y{\isadigit{1}}{\isacharparenright}{\kern0pt}{\isacharparenright}{\kern0pt}{\isacharparenright}{\kern0pt}\isanewline
\ \ \ \ \ \ \ \ \ \ \ \ \ \ \ \ \ \ \ \ \ \ \ \ \ \ \ \ \ \ \ \ \ {\isasymcirc}\isactrlsub c\ right{\isacharunderscore}{\kern0pt}coproj\ {\isacharparenleft}{\kern0pt}Y\ {\isasymtimes}\isactrlsub c\ {\isasymone}{\isacharparenright}{\kern0pt}\ {\isacharparenleft}{\kern0pt}Y\ {\isasymtimes}\isactrlsub c\ {\isasymone}{\isacharparenright}{\kern0pt}\ {\isasymcirc}\isactrlsub c\ {\isasymlangle}y{\isadigit{1}}{\isacharcomma}{\kern0pt}id\ {\isasymone}{\isasymrangle}{\isachardoublequoteclose}\isanewline
\ \ \ \ \ \ \ \ \ \ \ \ \isacommand{using}\isamarkupfalse%
\ dist{\isacharunderscore}{\kern0pt}prod{\isacharunderscore}{\kern0pt}coprod{\isacharunderscore}{\kern0pt}left{\isacharunderscore}{\kern0pt}ap{\isacharunderscore}{\kern0pt}right\ \isacommand{by}\isamarkupfalse%
\ {\isacharparenleft}{\kern0pt}typecheck{\isacharunderscore}{\kern0pt}cfuncs{\isacharcomma}{\kern0pt}\ auto{\isacharparenright}{\kern0pt}\isanewline
\ \ \ \ \ \ \ \ \ \ \isacommand{also}\isamarkupfalse%
\ \isacommand{have}\isamarkupfalse%
\ {\isachardoublequoteopen}{\isachardot}{\kern0pt}{\isachardot}{\kern0pt}{\isachardot}{\kern0pt}\ {\isacharequal}{\kern0pt}\ {\isacharparenleft}{\kern0pt}{\isacharparenleft}{\kern0pt}left{\isacharunderscore}{\kern0pt}cart{\isacharunderscore}{\kern0pt}proj\ Y\ {\isasymone}\ {\isasymamalg}\ {\isacharparenleft}{\kern0pt}{\isacharparenleft}{\kern0pt}y{\isadigit{2}}\ {\isasymamalg}\ y{\isadigit{1}}{\isacharparenright}{\kern0pt}\ {\isasymcirc}\isactrlsub c\ case{\isacharunderscore}{\kern0pt}bool\ {\isasymcirc}\isactrlsub c\ eq{\isacharunderscore}{\kern0pt}pred\ Y\ {\isasymcirc}\isactrlsub c\ {\isacharparenleft}{\kern0pt}id\ Y\ {\isasymtimes}\isactrlsub f\ y{\isadigit{1}}{\isacharparenright}{\kern0pt}{\isacharparenright}{\kern0pt}{\isacharparenright}{\kern0pt}\ \isanewline
\ \ \ \ \ \ \ \ \ \ \ \ \ \ \ \ \ \ \ \ \ \ \ \ \ \ \ \ \ \ \ \ \ {\isasymcirc}\isactrlsub c\ right{\isacharunderscore}{\kern0pt}coproj\ {\isacharparenleft}{\kern0pt}Y\ {\isasymtimes}\isactrlsub c\ {\isasymone}{\isacharparenright}{\kern0pt}\ {\isacharparenleft}{\kern0pt}Y\ {\isasymtimes}\isactrlsub c\ {\isasymone}{\isacharparenright}{\kern0pt}{\isacharparenright}{\kern0pt}\ {\isasymcirc}\isactrlsub c\ {\isasymlangle}y{\isadigit{1}}{\isacharcomma}{\kern0pt}id\ {\isasymone}{\isasymrangle}{\isachardoublequoteclose}\isanewline
\ \ \ \ \ \ \ \ \ \ \ \ \isacommand{by}\isamarkupfalse%
\ {\isacharparenleft}{\kern0pt}typecheck{\isacharunderscore}{\kern0pt}cfuncs{\isacharcomma}{\kern0pt}\ meson\ comp{\isacharunderscore}{\kern0pt}associative{\isadigit{2}}{\isacharparenright}{\kern0pt}\isanewline
\ \ \ \ \ \ \ \ \ \ \isacommand{also}\isamarkupfalse%
\ \isacommand{have}\isamarkupfalse%
\ {\isachardoublequoteopen}{\isachardot}{\kern0pt}{\isachardot}{\kern0pt}{\isachardot}{\kern0pt}\ {\isacharequal}{\kern0pt}\ {\isacharparenleft}{\kern0pt}{\isacharparenleft}{\kern0pt}y{\isadigit{2}}\ {\isasymamalg}\ y{\isadigit{1}}{\isacharparenright}{\kern0pt}\ {\isasymcirc}\isactrlsub c\ case{\isacharunderscore}{\kern0pt}bool\ {\isasymcirc}\isactrlsub c\ eq{\isacharunderscore}{\kern0pt}pred\ Y\ {\isasymcirc}\isactrlsub c\ {\isacharparenleft}{\kern0pt}id\ Y\ {\isasymtimes}\isactrlsub f\ y{\isadigit{1}}{\isacharparenright}{\kern0pt}{\isacharparenright}{\kern0pt}\ {\isasymcirc}\isactrlsub c\ {\isasymlangle}y{\isadigit{1}}{\isacharcomma}{\kern0pt}id\ {\isasymone}{\isasymrangle}{\isachardoublequoteclose}\isanewline
\ \ \ \ \ \ \ \ \ \ \ \ \isacommand{using}\isamarkupfalse%
\ right{\isacharunderscore}{\kern0pt}coproj{\isacharunderscore}{\kern0pt}cfunc{\isacharunderscore}{\kern0pt}coprod\ \isacommand{by}\isamarkupfalse%
\ {\isacharparenleft}{\kern0pt}typecheck{\isacharunderscore}{\kern0pt}cfuncs{\isacharcomma}{\kern0pt}\ auto{\isacharparenright}{\kern0pt}\isanewline
\ \ \ \ \ \ \ \ \ \ \isacommand{also}\isamarkupfalse%
\ \isacommand{have}\isamarkupfalse%
\ {\isachardoublequoteopen}{\isachardot}{\kern0pt}{\isachardot}{\kern0pt}{\isachardot}{\kern0pt}\ {\isacharequal}{\kern0pt}\ {\isacharparenleft}{\kern0pt}y{\isadigit{2}}\ {\isasymamalg}\ y{\isadigit{1}}{\isacharparenright}{\kern0pt}\ {\isasymcirc}\isactrlsub c\ case{\isacharunderscore}{\kern0pt}bool\ {\isasymcirc}\isactrlsub c\ eq{\isacharunderscore}{\kern0pt}pred\ Y\ {\isasymcirc}\isactrlsub c\ {\isacharparenleft}{\kern0pt}id\ Y\ {\isasymtimes}\isactrlsub f\ y{\isadigit{1}}{\isacharparenright}{\kern0pt}\ {\isasymcirc}\isactrlsub c\ {\isasymlangle}y{\isadigit{1}}{\isacharcomma}{\kern0pt}id\ {\isasymone}{\isasymrangle}{\isachardoublequoteclose}\isanewline
\ \ \ \ \ \ \ \ \ \ \ \ \isacommand{using}\isamarkupfalse%
\ comp{\isacharunderscore}{\kern0pt}associative{\isadigit{2}}\ \isacommand{by}\isamarkupfalse%
\ {\isacharparenleft}{\kern0pt}typecheck{\isacharunderscore}{\kern0pt}cfuncs{\isacharcomma}{\kern0pt}\ force{\isacharparenright}{\kern0pt}\isanewline
\ \ \ \ \ \ \ \ \ \ \isacommand{also}\isamarkupfalse%
\ \isacommand{have}\isamarkupfalse%
\ {\isachardoublequoteopen}{\isachardot}{\kern0pt}{\isachardot}{\kern0pt}{\isachardot}{\kern0pt}\ {\isacharequal}{\kern0pt}\ {\isacharparenleft}{\kern0pt}y{\isadigit{2}}\ {\isasymamalg}\ y{\isadigit{1}}{\isacharparenright}{\kern0pt}\ {\isasymcirc}\isactrlsub c\ case{\isacharunderscore}{\kern0pt}bool\ {\isasymcirc}\isactrlsub c\ eq{\isacharunderscore}{\kern0pt}pred\ Y\ \ {\isasymcirc}\isactrlsub c\ {\isasymlangle}y{\isadigit{1}}{\isacharcomma}{\kern0pt}y{\isadigit{1}}{\isasymrangle}{\isachardoublequoteclose}\isanewline
\ \ \ \ \ \ \ \ \ \ \ \ \isacommand{by}\isamarkupfalse%
\ {\isacharparenleft}{\kern0pt}typecheck{\isacharunderscore}{\kern0pt}cfuncs{\isacharcomma}{\kern0pt}\ simp\ add{\isacharcolon}{\kern0pt}\ cfunc{\isacharunderscore}{\kern0pt}cross{\isacharunderscore}{\kern0pt}prod{\isacharunderscore}{\kern0pt}comp{\isacharunderscore}{\kern0pt}cfunc{\isacharunderscore}{\kern0pt}prod\ id{\isacharunderscore}{\kern0pt}left{\isacharunderscore}{\kern0pt}unit{\isadigit{2}}\ id{\isacharunderscore}{\kern0pt}right{\isacharunderscore}{\kern0pt}unit{\isadigit{2}}{\isacharparenright}{\kern0pt}\isanewline
\ \ \ \ \ \ \ \ \ \ \isacommand{also}\isamarkupfalse%
\ \isacommand{have}\isamarkupfalse%
\ {\isachardoublequoteopen}{\isachardot}{\kern0pt}{\isachardot}{\kern0pt}{\isachardot}{\kern0pt}\ {\isacharequal}{\kern0pt}\ {\isacharparenleft}{\kern0pt}y{\isadigit{2}}\ {\isasymamalg}\ y{\isadigit{1}}{\isacharparenright}{\kern0pt}\ {\isasymcirc}\isactrlsub c\ case{\isacharunderscore}{\kern0pt}bool\ {\isasymcirc}\isactrlsub c\ {\isasymt}{\isachardoublequoteclose}\isanewline
\ \ \ \ \ \ \ \ \ \ \ \ \isacommand{by}\isamarkupfalse%
\ {\isacharparenleft}{\kern0pt}typecheck{\isacharunderscore}{\kern0pt}cfuncs{\isacharcomma}{\kern0pt}\ metis\ eq{\isacharunderscore}{\kern0pt}pred{\isacharunderscore}{\kern0pt}iff{\isacharunderscore}{\kern0pt}eq{\isacharparenright}{\kern0pt}\isanewline
\ \ \ \ \ \ \ \ \ \ \isacommand{also}\isamarkupfalse%
\ \isacommand{have}\isamarkupfalse%
\ {\isachardoublequoteopen}{\isachardot}{\kern0pt}{\isachardot}{\kern0pt}{\isachardot}{\kern0pt}\ {\isacharequal}{\kern0pt}\ y{\isadigit{2}}{\isachardoublequoteclose}\isanewline
\ \ \ \ \ \ \ \ \ \ \ \ \isacommand{using}\isamarkupfalse%
\ case{\isacharunderscore}{\kern0pt}bool{\isacharunderscore}{\kern0pt}true\ left{\isacharunderscore}{\kern0pt}coproj{\isacharunderscore}{\kern0pt}cfunc{\isacharunderscore}{\kern0pt}coprod\ \isacommand{by}\isamarkupfalse%
\ {\isacharparenleft}{\kern0pt}typecheck{\isacharunderscore}{\kern0pt}cfuncs{\isacharcomma}{\kern0pt}\ presburger{\isacharparenright}{\kern0pt}\isanewline
\ \ \ \ \ \ \ \ \ \ \isacommand{then}\isamarkupfalse%
\ \isacommand{show}\isamarkupfalse%
\ {\isachardoublequoteopen}{\isacharparenleft}{\kern0pt}{\isasymTheta}\ {\isasymcirc}\isactrlsub c\ {\isasymlangle}x{\isacharcomma}{\kern0pt}\ y{\isadigit{1}}{\isasymrangle}{\isacharparenright}{\kern0pt}\isactrlsup {\isasymflat}\ {\isasymcirc}\isactrlsub c\ {\isasymlangle}id\ X{\isacharcomma}{\kern0pt}\ {\isasymbeta}\isactrlbsub X\isactrlesub {\isasymrangle}\ {\isasymcirc}\isactrlsub c\ z\ {\isacharequal}{\kern0pt}\ y{\isadigit{2}}{\isachardoublequoteclose}\isanewline
\ \ \ \ \ \ \ \ \ \ \ \ \isacommand{by}\isamarkupfalse%
\ {\isacharparenleft}{\kern0pt}simp\ add{\isacharcolon}{\kern0pt}\ calculation{\isacharparenright}{\kern0pt}\isanewline
\ \ \ \ \ \ \ \ \isacommand{qed}\isamarkupfalse%
\isanewline
\ \ \isanewline
\ \ \ \ \ \isacommand{have}\isamarkupfalse%
\ {\isasymTheta}{\isacharunderscore}{\kern0pt}injective{\isacharcolon}{\kern0pt}\ {\isachardoublequoteopen}injective{\isacharparenleft}{\kern0pt}{\isasymTheta}{\isacharparenright}{\kern0pt}{\isachardoublequoteclose}\isanewline
\ \ \ \ \ \ \ \isacommand{unfolding}\isamarkupfalse%
\ injective{\isacharunderscore}{\kern0pt}def\isanewline
\ \ \ \ \ \isacommand{proof}\isamarkupfalse%
{\isacharparenleft}{\kern0pt}clarify{\isacharparenright}{\kern0pt}\isanewline
\ \ \ \ \ \ \ \isacommand{fix}\isamarkupfalse%
\ xy\ st\isanewline
\ \ \ \ \ \ \ \isacommand{assume}\isamarkupfalse%
\ xy{\isacharunderscore}{\kern0pt}type{\isacharbrackleft}{\kern0pt}type{\isacharunderscore}{\kern0pt}rule{\isacharbrackright}{\kern0pt}{\isacharcolon}{\kern0pt}\ {\isachardoublequoteopen}xy\ {\isasymin}\isactrlsub c\ domain\ {\isasymTheta}{\isachardoublequoteclose}\isanewline
\ \ \ \ \ \ \ \isacommand{assume}\isamarkupfalse%
\ st{\isacharunderscore}{\kern0pt}type{\isacharbrackleft}{\kern0pt}type{\isacharunderscore}{\kern0pt}rule{\isacharbrackright}{\kern0pt}{\isacharcolon}{\kern0pt}\ {\isachardoublequoteopen}st\ {\isasymin}\isactrlsub c\ domain\ {\isasymTheta}{\isachardoublequoteclose}\isanewline
\ \ \ \ \ \ \ \isacommand{assume}\isamarkupfalse%
\ equals{\isacharcolon}{\kern0pt}\ {\isachardoublequoteopen}{\isasymTheta}\ {\isasymcirc}\isactrlsub c\ xy\ {\isacharequal}{\kern0pt}\ {\isasymTheta}\ {\isasymcirc}\isactrlsub c\ st{\isachardoublequoteclose}\isanewline
\ \ \ \ \ \ \ \isacommand{obtain}\isamarkupfalse%
\ x\ y\ \isakeyword{where}\ x{\isacharunderscore}{\kern0pt}type{\isacharbrackleft}{\kern0pt}type{\isacharunderscore}{\kern0pt}rule{\isacharbrackright}{\kern0pt}{\isacharcolon}{\kern0pt}\ {\isachardoublequoteopen}x\ {\isasymin}\isactrlsub c\ X{\isachardoublequoteclose}\ \isakeyword{and}\ y{\isacharunderscore}{\kern0pt}type{\isacharbrackleft}{\kern0pt}type{\isacharunderscore}{\kern0pt}rule{\isacharbrackright}{\kern0pt}{\isacharcolon}{\kern0pt}\ {\isachardoublequoteopen}y\ {\isasymin}\isactrlsub c\ Y{\isachardoublequoteclose}\ \isakeyword{and}\ xy{\isacharunderscore}{\kern0pt}def{\isacharcolon}{\kern0pt}\ {\isachardoublequoteopen}xy\ {\isacharequal}{\kern0pt}\ {\isasymlangle}x{\isacharcomma}{\kern0pt}y{\isasymrangle}{\isachardoublequoteclose}\isanewline
\ \ \ \ \ \ \ \ \ \isacommand{by}\isamarkupfalse%
\ {\isacharparenleft}{\kern0pt}metis\ {\isasymTheta}{\isacharunderscore}{\kern0pt}type\ cart{\isacharunderscore}{\kern0pt}prod{\isacharunderscore}{\kern0pt}decomp\ cfunc{\isacharunderscore}{\kern0pt}type{\isacharunderscore}{\kern0pt}def\ xy{\isacharunderscore}{\kern0pt}type{\isacharparenright}{\kern0pt}\isanewline
\ \ \ \ \ \ \ \isacommand{obtain}\isamarkupfalse%
\ s\ t\ \isakeyword{where}\ s{\isacharunderscore}{\kern0pt}type{\isacharbrackleft}{\kern0pt}type{\isacharunderscore}{\kern0pt}rule{\isacharbrackright}{\kern0pt}{\isacharcolon}{\kern0pt}\ {\isachardoublequoteopen}s\ {\isasymin}\isactrlsub c\ X{\isachardoublequoteclose}\ \isakeyword{and}\ t{\isacharunderscore}{\kern0pt}type{\isacharbrackleft}{\kern0pt}type{\isacharunderscore}{\kern0pt}rule{\isacharbrackright}{\kern0pt}{\isacharcolon}{\kern0pt}\ {\isachardoublequoteopen}t\ {\isasymin}\isactrlsub c\ Y{\isachardoublequoteclose}\ \isakeyword{and}\ st{\isacharunderscore}{\kern0pt}def{\isacharcolon}{\kern0pt}\ {\isachardoublequoteopen}st\ {\isacharequal}{\kern0pt}\ {\isasymlangle}s{\isacharcomma}{\kern0pt}t{\isasymrangle}{\isachardoublequoteclose}\isanewline
\ \ \ \ \ \ \ \ \ \isacommand{by}\isamarkupfalse%
\ {\isacharparenleft}{\kern0pt}metis\ {\isasymTheta}{\isacharunderscore}{\kern0pt}type\ cart{\isacharunderscore}{\kern0pt}prod{\isacharunderscore}{\kern0pt}decomp\ cfunc{\isacharunderscore}{\kern0pt}type{\isacharunderscore}{\kern0pt}def\ st{\isacharunderscore}{\kern0pt}type{\isacharparenright}{\kern0pt}\ \ \ \isanewline
\ \ \ \ \ \ \ \isacommand{have}\isamarkupfalse%
\ equals{\isadigit{2}}{\isacharcolon}{\kern0pt}\ {\isachardoublequoteopen}{\isasymTheta}\ {\isasymcirc}\isactrlsub c\ {\isasymlangle}x{\isacharcomma}{\kern0pt}y{\isasymrangle}\ {\isacharequal}{\kern0pt}\ {\isasymTheta}\ {\isasymcirc}\isactrlsub c\ {\isasymlangle}s{\isacharcomma}{\kern0pt}t{\isasymrangle}{\isachardoublequoteclose}\isanewline
\ \ \ \ \ \ \ \ \ \isacommand{using}\isamarkupfalse%
\ equals\ st{\isacharunderscore}{\kern0pt}def\ xy{\isacharunderscore}{\kern0pt}def\ \isacommand{by}\isamarkupfalse%
\ auto\isanewline
\ \ \ \ \ \ \ \isacommand{have}\isamarkupfalse%
\ {\isachardoublequoteopen}{\isasymlangle}x{\isacharcomma}{\kern0pt}y{\isasymrangle}\ {\isacharequal}{\kern0pt}\ {\isasymlangle}s{\isacharcomma}{\kern0pt}t{\isasymrangle}{\isachardoublequoteclose}\isanewline
\ \ \ \ \ \ \ \isacommand{proof}\isamarkupfalse%
{\isacharparenleft}{\kern0pt}cases\ {\isachardoublequoteopen}y\ {\isacharequal}{\kern0pt}\ y{\isadigit{1}}{\isachardoublequoteclose}{\isacharparenright}{\kern0pt}\ \ \isanewline
\ \ \ \ \ \ \ \ \ \isacommand{assume}\isamarkupfalse%
\ {\isachardoublequoteopen}y\ {\isacharequal}{\kern0pt}\ y{\isadigit{1}}{\isachardoublequoteclose}\isanewline
\ \ \ \ \ \ \ \ \ \isacommand{show}\isamarkupfalse%
\ {\isachardoublequoteopen}{\isasymlangle}x{\isacharcomma}{\kern0pt}y{\isasymrangle}\ {\isacharequal}{\kern0pt}\ {\isasymlangle}s{\isacharcomma}{\kern0pt}t{\isasymrangle}{\isachardoublequoteclose}\isanewline
\ \ \ \ \ \ \ \ \ \isacommand{proof}\isamarkupfalse%
{\isacharparenleft}{\kern0pt}cases\ {\isachardoublequoteopen}t\ {\isacharequal}{\kern0pt}\ y{\isadigit{1}}{\isachardoublequoteclose}{\isacharparenright}{\kern0pt}\isanewline
\ \ \ \ \ \ \ \ \ \ \ \isacommand{show}\isamarkupfalse%
\ {\isachardoublequoteopen}t\ {\isacharequal}{\kern0pt}\ y{\isadigit{1}}\ {\isasymLongrightarrow}\ {\isasymlangle}x{\isacharcomma}{\kern0pt}y{\isasymrangle}\ {\isacharequal}{\kern0pt}\ {\isasymlangle}s{\isacharcomma}{\kern0pt}t{\isasymrangle}{\isachardoublequoteclose}\isanewline
\ \ \ \ \ \ \ \ \ \ \ \ \ \isacommand{by}\isamarkupfalse%
\ {\isacharparenleft}{\kern0pt}typecheck{\isacharunderscore}{\kern0pt}cfuncs{\isacharcomma}{\kern0pt}\ metis\ {\isacartoucheopen}y\ {\isacharequal}{\kern0pt}\ y{\isadigit{1}}{\isacartoucheclose}\ equals\ f{\isadigit{1}}\ f{\isadigit{3}}\ st{\isacharunderscore}{\kern0pt}def\ xy{\isacharunderscore}{\kern0pt}def\ y{\isadigit{1}}{\isacharunderscore}{\kern0pt}not{\isacharunderscore}{\kern0pt}y{\isadigit{2}}{\isacharparenright}{\kern0pt}\isanewline
\ \ \ \ \ \ \ \ \ \isacommand{next}\isamarkupfalse%
\isanewline
\ \ \ \ \ \ \ \ \ \ \ \isacommand{assume}\isamarkupfalse%
\ {\isachardoublequoteopen}t\ {\isasymnoteq}\ y{\isadigit{1}}{\isachardoublequoteclose}\isanewline
\ \ \ \ \ \ \ \ \ \ \ \isacommand{show}\isamarkupfalse%
\ {\isachardoublequoteopen}{\isasymlangle}x{\isacharcomma}{\kern0pt}y{\isasymrangle}\ {\isacharequal}{\kern0pt}\ {\isasymlangle}s{\isacharcomma}{\kern0pt}t{\isasymrangle}{\isachardoublequoteclose}\isanewline
\ \ \ \ \ \ \ \ \ \ \ \isacommand{proof}\isamarkupfalse%
{\isacharparenleft}{\kern0pt}cases\ {\isachardoublequoteopen}s\ {\isacharequal}{\kern0pt}\ x{\isachardoublequoteclose}{\isacharparenright}{\kern0pt}\isanewline
\ \ \ \ \ \ \ \ \ \ \ \ \ \isacommand{show}\isamarkupfalse%
\ {\isachardoublequoteopen}s\ {\isacharequal}{\kern0pt}\ x\ {\isasymLongrightarrow}\ {\isasymlangle}x{\isacharcomma}{\kern0pt}y{\isasymrangle}\ {\isacharequal}{\kern0pt}\ {\isasymlangle}s{\isacharcomma}{\kern0pt}t{\isasymrangle}{\isachardoublequoteclose}\isanewline
\ \ \ \ \ \ \ \ \ \ \ \ \ \ \ \isacommand{by}\isamarkupfalse%
\ {\isacharparenleft}{\kern0pt}typecheck{\isacharunderscore}{\kern0pt}cfuncs{\isacharcomma}{\kern0pt}\ metis\ equals{\isadigit{2}}\ f{\isadigit{1}}{\isacharparenright}{\kern0pt}\isanewline
\ \ \ \ \ \ \ \ \ \ \ \isacommand{next}\isamarkupfalse%
\isanewline
\ \ \ \ \ \ \ \ \ \ \ \ \ \isacommand{assume}\isamarkupfalse%
\ {\isachardoublequoteopen}s\ {\isasymnoteq}\ x{\isachardoublequoteclose}\ \ \isanewline
\ \ \ \ \ \ \ \ \ \ \ \ \ \isacommand{obtain}\isamarkupfalse%
\ z\ \isakeyword{where}\ z{\isacharunderscore}{\kern0pt}type{\isacharbrackleft}{\kern0pt}type{\isacharunderscore}{\kern0pt}rule{\isacharbrackright}{\kern0pt}{\isacharcolon}{\kern0pt}\ {\isachardoublequoteopen}z\ {\isasymin}\isactrlsub c\ X{\isachardoublequoteclose}\ \isakeyword{and}\ z{\isacharunderscore}{\kern0pt}not{\isacharunderscore}{\kern0pt}x{\isacharcolon}{\kern0pt}\ {\isachardoublequoteopen}z\ {\isasymnoteq}\ x{\isachardoublequoteclose}\ \isakeyword{and}\ z{\isacharunderscore}{\kern0pt}not{\isacharunderscore}{\kern0pt}s{\isacharcolon}{\kern0pt}\ {\isachardoublequoteopen}z\ {\isasymnoteq}\ s{\isachardoublequoteclose}\isanewline
\ \ \ \ \ \ \ \ \ \ \ \ \ \ \ \isacommand{by}\isamarkupfalse%
\ {\isacharparenleft}{\kern0pt}metis\ {\isacartoucheopen}{\isasymnot}\ X\ {\isasymcong}\ {\isasymOmega}{\isacartoucheclose}\ {\isacartoucheopen}{\isasymnot}\ initial{\isacharunderscore}{\kern0pt}object\ X{\isacartoucheclose}\ {\isacartoucheopen}{\isasymnot}\ terminal{\isacharunderscore}{\kern0pt}object\ X{\isacartoucheclose}\ sets{\isacharunderscore}{\kern0pt}size{\isacharunderscore}{\kern0pt}{\isadigit{3}}{\isacharunderscore}{\kern0pt}plus{\isacharparenright}{\kern0pt}\isanewline
\ \ \ \ \ \ \ \ \ \ \ \ \ \isacommand{have}\isamarkupfalse%
\ t{\isacharunderscore}{\kern0pt}sz{\isacharcolon}{\kern0pt}\ {\isachardoublequoteopen}{\isacharparenleft}{\kern0pt}{\isasymTheta}\ {\isasymcirc}\isactrlsub c\ {\isasymlangle}s{\isacharcomma}{\kern0pt}\ t{\isasymrangle}{\isacharparenright}{\kern0pt}\isactrlsup {\isasymflat}\ {\isasymcirc}\isactrlsub c\ {\isasymlangle}id\ X{\isacharcomma}{\kern0pt}\ {\isasymbeta}\isactrlbsub X\isactrlesub {\isasymrangle}\ {\isasymcirc}\isactrlsub c\ z\ {\isacharequal}{\kern0pt}\ y{\isadigit{1}}{\isachardoublequoteclose}\isanewline
\ \ \ \ \ \ \ \ \ \ \ \ \ \ \ \isacommand{by}\isamarkupfalse%
\ {\isacharparenleft}{\kern0pt}simp\ add{\isacharcolon}{\kern0pt}\ {\isacartoucheopen}t\ {\isasymnoteq}\ y{\isadigit{1}}{\isacartoucheclose}\ f{\isadigit{2}}\ s{\isacharunderscore}{\kern0pt}type\ t{\isacharunderscore}{\kern0pt}type\ z{\isacharunderscore}{\kern0pt}not{\isacharunderscore}{\kern0pt}s\ z{\isacharunderscore}{\kern0pt}type{\isacharparenright}{\kern0pt}\isanewline
\ \ \ \ \ \ \ \ \ \ \ \ \ \isacommand{have}\isamarkupfalse%
\ y{\isacharunderscore}{\kern0pt}xz{\isacharcolon}{\kern0pt}\ {\isachardoublequoteopen}{\isacharparenleft}{\kern0pt}{\isasymTheta}\ {\isasymcirc}\isactrlsub c\ {\isasymlangle}x{\isacharcomma}{\kern0pt}\ y{\isasymrangle}{\isacharparenright}{\kern0pt}\isactrlsup {\isasymflat}\ {\isasymcirc}\isactrlsub c\ {\isasymlangle}id\ X{\isacharcomma}{\kern0pt}\ {\isasymbeta}\isactrlbsub X\isactrlesub {\isasymrangle}\ {\isasymcirc}\isactrlsub c\ z\ {\isacharequal}{\kern0pt}\ y{\isadigit{2}}{\isachardoublequoteclose}\isanewline
\ \ \ \ \ \ \ \ \ \ \ \ \ \ \ \isacommand{by}\isamarkupfalse%
\ {\isacharparenleft}{\kern0pt}simp\ add{\isacharcolon}{\kern0pt}\ {\isacartoucheopen}y\ {\isacharequal}{\kern0pt}\ y{\isadigit{1}}{\isacartoucheclose}\ f{\isadigit{3}}\ x{\isacharunderscore}{\kern0pt}type\ z{\isacharunderscore}{\kern0pt}not{\isacharunderscore}{\kern0pt}x\ z{\isacharunderscore}{\kern0pt}type{\isacharparenright}{\kern0pt}\ \ \ \ \isanewline
\ \ \ \ \ \ \ \ \ \ \ \ \ \isacommand{then}\isamarkupfalse%
\ \isacommand{have}\isamarkupfalse%
\ {\isachardoublequoteopen}y{\isadigit{1}}\ {\isacharequal}{\kern0pt}\ y{\isadigit{2}}{\isachardoublequoteclose}\isanewline
\ \ \ \ \ \ \ \ \ \ \ \ \ \ \ \isacommand{using}\isamarkupfalse%
\ equals{\isadigit{2}}\ t{\isacharunderscore}{\kern0pt}sz\ \isacommand{by}\isamarkupfalse%
\ auto\isanewline
\ \ \ \ \ \ \ \ \ \ \ \ \ \isacommand{then}\isamarkupfalse%
\ \isacommand{have}\isamarkupfalse%
\ False\isanewline
\ \ \ \ \ \ \ \ \ \ \ \ \ \ \ \isacommand{using}\isamarkupfalse%
\ y{\isadigit{1}}{\isacharunderscore}{\kern0pt}not{\isacharunderscore}{\kern0pt}y{\isadigit{2}}\ \isacommand{by}\isamarkupfalse%
\ auto\isanewline
\ \ \ \ \ \ \ \ \ \ \ \ \ \isacommand{then}\isamarkupfalse%
\ \isacommand{show}\isamarkupfalse%
\ {\isachardoublequoteopen}{\isasymlangle}x{\isacharcomma}{\kern0pt}y{\isasymrangle}\ {\isacharequal}{\kern0pt}\ {\isasymlangle}s{\isacharcomma}{\kern0pt}t{\isasymrangle}{\isachardoublequoteclose}\isanewline
\ \ \ \ \ \ \ \ \ \ \ \ \ \ \ \isacommand{by}\isamarkupfalse%
\ simp\isanewline
\ \ \ \ \ \ \ \ \ \ \ \isacommand{qed}\isamarkupfalse%
\isanewline
\ \ \ \ \ \ \ \ \ \isacommand{qed}\isamarkupfalse%
\isanewline
\ \ \ \ \ \ \ \isacommand{next}\isamarkupfalse%
\isanewline
\ \ \ \ \ \ \ \ \ \isacommand{assume}\isamarkupfalse%
\ {\isachardoublequoteopen}y\ {\isasymnoteq}\ y{\isadigit{1}}{\isachardoublequoteclose}\isanewline
\ \ \ \ \ \ \ \ \ \isacommand{show}\isamarkupfalse%
\ {\isachardoublequoteopen}{\isasymlangle}x{\isacharcomma}{\kern0pt}y{\isasymrangle}\ {\isacharequal}{\kern0pt}\ {\isasymlangle}s{\isacharcomma}{\kern0pt}t{\isasymrangle}{\isachardoublequoteclose}\isanewline
\ \ \ \ \ \ \ \ \ \isacommand{proof}\isamarkupfalse%
{\isacharparenleft}{\kern0pt}cases\ {\isachardoublequoteopen}y\ {\isacharequal}{\kern0pt}\ y{\isadigit{2}}{\isachardoublequoteclose}{\isacharparenright}{\kern0pt}\isanewline
\ \ \ \ \ \ \ \ \ \ \ \isacommand{assume}\isamarkupfalse%
\ {\isachardoublequoteopen}y\ {\isacharequal}{\kern0pt}\ y{\isadigit{2}}{\isachardoublequoteclose}\isanewline
\ \ \ \ \ \ \ \ \ \ \ \isacommand{show}\isamarkupfalse%
\ {\isachardoublequoteopen}{\isasymlangle}x{\isacharcomma}{\kern0pt}y{\isasymrangle}\ {\isacharequal}{\kern0pt}\ {\isasymlangle}s{\isacharcomma}{\kern0pt}t{\isasymrangle}{\isachardoublequoteclose}\isanewline
\ \ \ \ \ \ \ \ \ \ \ \isacommand{proof}\isamarkupfalse%
{\isacharparenleft}{\kern0pt}cases\ {\isachardoublequoteopen}t\ {\isacharequal}{\kern0pt}\ y{\isadigit{2}}{\isachardoublequoteclose}{\isacharcomma}{\kern0pt}\ clarify{\isacharparenright}{\kern0pt}\isanewline
\ \ \ \ \ \ \ \ \ \ \ \ \ \isacommand{show}\isamarkupfalse%
\ {\isachardoublequoteopen}t\ {\isacharequal}{\kern0pt}\ y{\isadigit{2}}\ {\isasymLongrightarrow}\ {\isasymlangle}x{\isacharcomma}{\kern0pt}y{\isasymrangle}\ {\isacharequal}{\kern0pt}\ {\isasymlangle}s{\isacharcomma}{\kern0pt}y{\isadigit{2}}{\isasymrangle}{\isachardoublequoteclose}\isanewline
\ \ \ \ \ \ \ \ \ \ \ \ \ \ \ \isacommand{by}\isamarkupfalse%
\ {\isacharparenleft}{\kern0pt}typecheck{\isacharunderscore}{\kern0pt}cfuncs{\isacharcomma}{\kern0pt}\ metis\ {\isacartoucheopen}y\ {\isacharequal}{\kern0pt}\ y{\isadigit{2}}{\isacartoucheclose}\ {\isacartoucheopen}y\ {\isasymnoteq}\ y{\isadigit{1}}{\isacartoucheclose}\ equals\ f{\isadigit{1}}\ f{\isadigit{2}}\ st{\isacharunderscore}{\kern0pt}def\ xy{\isacharunderscore}{\kern0pt}def{\isacharparenright}{\kern0pt}\isanewline
\ \ \ \ \ \ \ \ \ \ \ \isacommand{next}\isamarkupfalse%
\isanewline
\ \ \ \ \ \ \ \ \ \ \ \ \ \isacommand{assume}\isamarkupfalse%
\ {\isachardoublequoteopen}t\ {\isasymnoteq}\ y{\isadigit{2}}{\isachardoublequoteclose}\isanewline
\ \ \ \ \ \ \ \ \ \ \ \ \ \isacommand{show}\isamarkupfalse%
\ {\isachardoublequoteopen}{\isasymlangle}x{\isacharcomma}{\kern0pt}y{\isasymrangle}\ {\isacharequal}{\kern0pt}\ {\isasymlangle}s{\isacharcomma}{\kern0pt}t{\isasymrangle}{\isachardoublequoteclose}\isanewline
\ \ \ \ \ \ \ \ \ \ \ \ \ \isacommand{proof}\isamarkupfalse%
{\isacharparenleft}{\kern0pt}cases\ {\isachardoublequoteopen}x\ {\isacharequal}{\kern0pt}\ s{\isachardoublequoteclose}{\isacharcomma}{\kern0pt}\ clarify{\isacharparenright}{\kern0pt}\isanewline
\ \ \ \ \ \ \ \ \ \ \ \ \ \ \ \isacommand{show}\isamarkupfalse%
\ {\isachardoublequoteopen}x\ {\isacharequal}{\kern0pt}\ s\ {\isasymLongrightarrow}\ {\isasymlangle}s{\isacharcomma}{\kern0pt}y{\isasymrangle}\ {\isacharequal}{\kern0pt}\ {\isasymlangle}s{\isacharcomma}{\kern0pt}t{\isasymrangle}{\isachardoublequoteclose}\isanewline
\ \ \ \ \ \ \ \ \ \ \ \ \ \ \ \ \ \isacommand{by}\isamarkupfalse%
\ {\isacharparenleft}{\kern0pt}metis\ equals{\isadigit{2}}\ f{\isadigit{1}}\ s{\isacharunderscore}{\kern0pt}type\ t{\isacharunderscore}{\kern0pt}type\ y{\isacharunderscore}{\kern0pt}type{\isacharparenright}{\kern0pt}\isanewline
\ \ \ \ \ \ \ \ \ \ \ \ \ \isacommand{next}\isamarkupfalse%
\isanewline
\ \ \ \ \ \ \ \ \ \ \ \ \ \ \ \isacommand{assume}\isamarkupfalse%
\ {\isachardoublequoteopen}x\ {\isasymnoteq}\ s{\isachardoublequoteclose}\isanewline
\ \ \ \ \ \ \ \ \ \ \ \ \ \ \ \isacommand{show}\isamarkupfalse%
\ {\isachardoublequoteopen}{\isasymlangle}x{\isacharcomma}{\kern0pt}y{\isasymrangle}\ {\isacharequal}{\kern0pt}\ {\isasymlangle}s{\isacharcomma}{\kern0pt}t{\isasymrangle}{\isachardoublequoteclose}\isanewline
\ \ \ \ \ \ \ \ \ \ \ \ \ \ \ \isacommand{proof}\isamarkupfalse%
{\isacharparenleft}{\kern0pt}cases\ {\isachardoublequoteopen}t\ {\isacharequal}{\kern0pt}\ y{\isadigit{1}}{\isachardoublequoteclose}{\isacharcomma}{\kern0pt}clarify{\isacharparenright}{\kern0pt}\isanewline
\ \ \ \ \ \ \ \ \ \ \ \ \ \ \ \ \ \isacommand{show}\isamarkupfalse%
\ {\isachardoublequoteopen}t\ {\isacharequal}{\kern0pt}\ y{\isadigit{1}}\ {\isasymLongrightarrow}\ {\isasymlangle}x{\isacharcomma}{\kern0pt}y{\isasymrangle}\ {\isacharequal}{\kern0pt}\ {\isasymlangle}s{\isacharcomma}{\kern0pt}y{\isadigit{1}}{\isasymrangle}{\isachardoublequoteclose}\isanewline
\ \ \ \ \ \ \ \ \ \ \ \ \ \ \ \ \ \ \ \isacommand{by}\isamarkupfalse%
\ {\isacharparenleft}{\kern0pt}metis\ {\isacartoucheopen}{\isasymnot}\ X\ {\isasymcong}\ {\isasymOmega}{\isacartoucheclose}\ {\isacartoucheopen}{\isasymnot}\ initial{\isacharunderscore}{\kern0pt}object\ X{\isacartoucheclose}\ {\isacartoucheopen}{\isasymnot}\ terminal{\isacharunderscore}{\kern0pt}object\ X{\isacartoucheclose}\ {\isacartoucheopen}y\ {\isacharequal}{\kern0pt}\ y{\isadigit{2}}{\isacartoucheclose}\ {\isacartoucheopen}y\ {\isasymnoteq}\ y{\isadigit{1}}{\isacartoucheclose}\ equals\ f{\isadigit{2}}\ f{\isadigit{3}}\ s{\isacharunderscore}{\kern0pt}type\ sets{\isacharunderscore}{\kern0pt}size{\isacharunderscore}{\kern0pt}{\isadigit{3}}{\isacharunderscore}{\kern0pt}plus\ st{\isacharunderscore}{\kern0pt}def\ x{\isacharunderscore}{\kern0pt}type\ xy{\isacharunderscore}{\kern0pt}def\ y{\isadigit{2}}{\isacharunderscore}{\kern0pt}type{\isacharparenright}{\kern0pt}\isanewline
\ \ \ \ \ \ \ \ \ \ \ \ \ \ \ \isacommand{next}\isamarkupfalse%
\isanewline
\ \ \ \ \ \ \ \ \ \ \ \ \ \ \ \ \ \isacommand{assume}\isamarkupfalse%
\ {\isachardoublequoteopen}t\ {\isasymnoteq}\ y{\isadigit{1}}{\isachardoublequoteclose}\isanewline
\ \ \ \ \ \ \ \ \ \ \ \ \ \ \ \ \ \isacommand{show}\isamarkupfalse%
\ {\isachardoublequoteopen}{\isasymlangle}x{\isacharcomma}{\kern0pt}y{\isasymrangle}\ {\isacharequal}{\kern0pt}\ {\isasymlangle}s{\isacharcomma}{\kern0pt}t{\isasymrangle}{\isachardoublequoteclose}\isanewline
\ \ \ \ \ \ \ \ \ \ \ \ \ \ \ \ \ \ \ \isacommand{by}\isamarkupfalse%
\ {\isacharparenleft}{\kern0pt}typecheck{\isacharunderscore}{\kern0pt}cfuncs{\isacharcomma}{\kern0pt}\ metis\ {\isacartoucheopen}t\ {\isasymnoteq}\ y{\isadigit{1}}{\isacartoucheclose}\ {\isacartoucheopen}y\ {\isasymnoteq}\ y{\isadigit{1}}{\isacartoucheclose}\ equals\ f{\isadigit{1}}\ f{\isadigit{2}}\ st{\isacharunderscore}{\kern0pt}def\ xy{\isacharunderscore}{\kern0pt}def{\isacharparenright}{\kern0pt}\isanewline
\ \ \ \ \ \ \ \ \ \ \ \ \ \ \ \isacommand{qed}\isamarkupfalse%
\isanewline
\ \ \ \ \ \ \ \ \ \ \ \ \ \isacommand{qed}\isamarkupfalse%
\isanewline
\ \ \ \ \ \ \ \ \ \ \ \isacommand{qed}\isamarkupfalse%
\isanewline
\ \ \ \ \ \ \ \ \ \isacommand{next}\isamarkupfalse%
\isanewline
\ \ \ \ \ \ \ \ \ \ \ \isacommand{assume}\isamarkupfalse%
\ {\isachardoublequoteopen}y\ {\isasymnoteq}\ y{\isadigit{2}}{\isachardoublequoteclose}\isanewline
\ \ \ \ \ \ \ \ \ \ \ \isacommand{show}\isamarkupfalse%
\ {\isachardoublequoteopen}{\isasymlangle}x{\isacharcomma}{\kern0pt}y{\isasymrangle}\ {\isacharequal}{\kern0pt}\ {\isasymlangle}s{\isacharcomma}{\kern0pt}t{\isasymrangle}{\isachardoublequoteclose}\isanewline
\ \ \ \ \ \ \ \ \ \ \ \isacommand{proof}\isamarkupfalse%
{\isacharparenleft}{\kern0pt}cases\ {\isachardoublequoteopen}s\ {\isacharequal}{\kern0pt}\ x{\isachardoublequoteclose}{\isacharcomma}{\kern0pt}\ clarify{\isacharparenright}{\kern0pt}\isanewline
\ \ \ \ \ \ \ \ \ \ \ \ \ \isacommand{show}\isamarkupfalse%
\ {\isachardoublequoteopen}s\ {\isacharequal}{\kern0pt}\ x\ {\isasymLongrightarrow}\ {\isasymlangle}x{\isacharcomma}{\kern0pt}y{\isasymrangle}\ {\isacharequal}{\kern0pt}\ {\isasymlangle}x{\isacharcomma}{\kern0pt}t{\isasymrangle}{\isachardoublequoteclose}\isanewline
\ \ \ \ \ \ \ \ \ \ \ \ \ \ \ \isacommand{by}\isamarkupfalse%
\ {\isacharparenleft}{\kern0pt}metis\ equals{\isadigit{2}}\ f{\isadigit{1}}\ t{\isacharunderscore}{\kern0pt}type\ x{\isacharunderscore}{\kern0pt}type\ y{\isacharunderscore}{\kern0pt}type{\isacharparenright}{\kern0pt}\isanewline
\ \ \ \ \ \ \ \ \ \ \ \ \ \isacommand{show}\isamarkupfalse%
\ {\isachardoublequoteopen}s\ {\isasymnoteq}\ x\ {\isasymLongrightarrow}\ {\isasymlangle}x{\isacharcomma}{\kern0pt}y{\isasymrangle}\ {\isacharequal}{\kern0pt}\ {\isasymlangle}s{\isacharcomma}{\kern0pt}t{\isasymrangle}{\isachardoublequoteclose}\isanewline
\ \ \ \ \ \ \ \ \ \ \ \ \ \ \ \isacommand{by}\isamarkupfalse%
\ {\isacharparenleft}{\kern0pt}metis\ {\isacartoucheopen}y\ {\isasymnoteq}\ y{\isadigit{1}}{\isacartoucheclose}\ {\isacartoucheopen}y\ {\isasymnoteq}\ y{\isadigit{2}}{\isacartoucheclose}\ equals\ f{\isadigit{1}}\ f{\isadigit{2}}\ f{\isadigit{3}}\ s{\isacharunderscore}{\kern0pt}type\ st{\isacharunderscore}{\kern0pt}def\ t{\isacharunderscore}{\kern0pt}type\ x{\isacharunderscore}{\kern0pt}type\ xy{\isacharunderscore}{\kern0pt}def\ y{\isacharunderscore}{\kern0pt}type{\isacharparenright}{\kern0pt}\isanewline
\ \ \ \ \ \ \ \ \ \ \ \isacommand{qed}\isamarkupfalse%
\isanewline
\ \ \ \ \ \ \ \ \ \isacommand{qed}\isamarkupfalse%
\isanewline
\ \ \ \ \ \ \ \isacommand{qed}\isamarkupfalse%
\isanewline
\ \ \ \ \ \isacommand{then}\isamarkupfalse%
\ \isacommand{show}\isamarkupfalse%
\ {\isachardoublequoteopen}xy\ {\isacharequal}{\kern0pt}\ st{\isachardoublequoteclose}\isanewline
\ \ \ \ \ \ \ \isacommand{by}\isamarkupfalse%
\ {\isacharparenleft}{\kern0pt}typecheck{\isacharunderscore}{\kern0pt}cfuncs{\isacharcomma}{\kern0pt}\ simp\ add{\isacharcolon}{\kern0pt}\ \ st{\isacharunderscore}{\kern0pt}def\ xy{\isacharunderscore}{\kern0pt}def{\isacharparenright}{\kern0pt}\isanewline
\ \ \ \isacommand{qed}\isamarkupfalse%
\isanewline
\ \ \ \ \ \ \isacommand{then}\isamarkupfalse%
\ \isacommand{show}\isamarkupfalse%
\ {\isacharquery}{\kern0pt}thesis\isanewline
\ \ \ \ \ \ \ \ \isacommand{using}\isamarkupfalse%
\ {\isasymTheta}{\isacharunderscore}{\kern0pt}type\ injective{\isacharunderscore}{\kern0pt}imp{\isacharunderscore}{\kern0pt}monomorphism\ is{\isacharunderscore}{\kern0pt}smaller{\isacharunderscore}{\kern0pt}than{\isacharunderscore}{\kern0pt}def\ \isacommand{by}\isamarkupfalse%
\ blast\isanewline
\ \ \ \ \isacommand{qed}\isamarkupfalse%
\isanewline
\ \ \isacommand{qed}\isamarkupfalse%
\ \ \isanewline
\ \isacommand{qed}\isamarkupfalse%
\isanewline
\isacommand{qed}\isamarkupfalse%
%
\endisatagproof
{\isafoldproof}%
%
\isadelimproof
\isanewline
%
\endisadelimproof
\isanewline
\isacommand{lemma}\isamarkupfalse%
\ Y{\isacharunderscore}{\kern0pt}nonempty{\isacharunderscore}{\kern0pt}then{\isacharunderscore}{\kern0pt}X{\isacharunderscore}{\kern0pt}le{\isacharunderscore}{\kern0pt}XtoY{\isacharcolon}{\kern0pt}\isanewline
\ \ \isakeyword{assumes}\ {\isachardoublequoteopen}nonempty\ Y{\isachardoublequoteclose}\isanewline
\ \ \isakeyword{shows}\ {\isachardoublequoteopen}X\ {\isasymle}\isactrlsub c\ X\isactrlbsup Y\isactrlesup {\isachardoublequoteclose}\isanewline
%
\isadelimproof
%
\endisadelimproof
%
\isatagproof
\isacommand{proof}\isamarkupfalse%
\ {\isacharminus}{\kern0pt}\ \isanewline
\ \ \isacommand{obtain}\isamarkupfalse%
\ f\ \isakeyword{where}\ f{\isacharunderscore}{\kern0pt}def{\isacharcolon}{\kern0pt}\ {\isachardoublequoteopen}f\ {\isacharequal}{\kern0pt}\ {\isacharparenleft}{\kern0pt}right{\isacharunderscore}{\kern0pt}cart{\isacharunderscore}{\kern0pt}proj\ Y\ X{\isacharparenright}{\kern0pt}\isactrlsup {\isasymsharp}{\isachardoublequoteclose}\isanewline
\ \ \ \ \isacommand{by}\isamarkupfalse%
\ blast\isanewline
\ \ \isacommand{then}\isamarkupfalse%
\ \isacommand{have}\isamarkupfalse%
\ f{\isacharunderscore}{\kern0pt}type{\isacharcolon}{\kern0pt}\ {\isachardoublequoteopen}f\ {\isacharcolon}{\kern0pt}\ X\ {\isasymrightarrow}\ X\isactrlbsup Y\isactrlesup {\isachardoublequoteclose}\isanewline
\ \ \ \ \isacommand{by}\isamarkupfalse%
\ {\isacharparenleft}{\kern0pt}simp\ add{\isacharcolon}{\kern0pt}\ right{\isacharunderscore}{\kern0pt}cart{\isacharunderscore}{\kern0pt}proj{\isacharunderscore}{\kern0pt}type\ transpose{\isacharunderscore}{\kern0pt}func{\isacharunderscore}{\kern0pt}type{\isacharparenright}{\kern0pt}\isanewline
\ \ \isacommand{have}\isamarkupfalse%
\ mono{\isacharunderscore}{\kern0pt}f{\isacharcolon}{\kern0pt}\ {\isachardoublequoteopen}injective{\isacharparenleft}{\kern0pt}f{\isacharparenright}{\kern0pt}{\isachardoublequoteclose}\isanewline
\ \ \ \ \isacommand{unfolding}\isamarkupfalse%
\ injective{\isacharunderscore}{\kern0pt}def\isanewline
\ \ \isacommand{proof}\isamarkupfalse%
{\isacharparenleft}{\kern0pt}clarify{\isacharparenright}{\kern0pt}\isanewline
\ \ \ \ \isacommand{fix}\isamarkupfalse%
\ x\ y\ \isanewline
\ \ \ \ \isacommand{assume}\isamarkupfalse%
\ x{\isacharunderscore}{\kern0pt}type{\isacharcolon}{\kern0pt}\ {\isachardoublequoteopen}x\ {\isasymin}\isactrlsub c\ domain\ f{\isachardoublequoteclose}\isanewline
\ \ \ \ \isacommand{assume}\isamarkupfalse%
\ y{\isacharunderscore}{\kern0pt}type{\isacharcolon}{\kern0pt}\ {\isachardoublequoteopen}y\ {\isasymin}\isactrlsub c\ domain\ f{\isachardoublequoteclose}\isanewline
\ \ \ \ \isacommand{assume}\isamarkupfalse%
\ equals{\isacharcolon}{\kern0pt}\ {\isachardoublequoteopen}f\ {\isasymcirc}\isactrlsub c\ x\ {\isacharequal}{\kern0pt}\ f\ {\isasymcirc}\isactrlsub c\ y{\isachardoublequoteclose}\isanewline
\ \ \ \ \isacommand{have}\isamarkupfalse%
\ x{\isacharunderscore}{\kern0pt}type{\isadigit{2}}\ {\isacharcolon}{\kern0pt}\ {\isachardoublequoteopen}x\ {\isasymin}\isactrlsub c\ X{\isachardoublequoteclose}\isanewline
\ \ \ \ \ \ \isacommand{using}\isamarkupfalse%
\ cfunc{\isacharunderscore}{\kern0pt}type{\isacharunderscore}{\kern0pt}def\ f{\isacharunderscore}{\kern0pt}type\ x{\isacharunderscore}{\kern0pt}type\ \isacommand{by}\isamarkupfalse%
\ auto\isanewline
\ \ \ \ \isacommand{have}\isamarkupfalse%
\ y{\isacharunderscore}{\kern0pt}type{\isadigit{2}}\ {\isacharcolon}{\kern0pt}\ {\isachardoublequoteopen}y\ {\isasymin}\isactrlsub c\ X{\isachardoublequoteclose}\isanewline
\ \ \ \ \ \ \isacommand{using}\isamarkupfalse%
\ cfunc{\isacharunderscore}{\kern0pt}type{\isacharunderscore}{\kern0pt}def\ f{\isacharunderscore}{\kern0pt}type\ y{\isacharunderscore}{\kern0pt}type\ \isacommand{by}\isamarkupfalse%
\ auto\isanewline
\ \ \ \ \isacommand{have}\isamarkupfalse%
\ {\isachardoublequoteopen}x\ {\isasymcirc}\isactrlsub c\ {\isacharparenleft}{\kern0pt}right{\isacharunderscore}{\kern0pt}cart{\isacharunderscore}{\kern0pt}proj\ Y\ {\isasymone}{\isacharparenright}{\kern0pt}\ {\isacharequal}{\kern0pt}\ {\isacharparenleft}{\kern0pt}right{\isacharunderscore}{\kern0pt}cart{\isacharunderscore}{\kern0pt}proj\ Y\ X{\isacharparenright}{\kern0pt}\ {\isasymcirc}\isactrlsub c\ {\isacharparenleft}{\kern0pt}id{\isacharparenleft}{\kern0pt}Y{\isacharparenright}{\kern0pt}\ {\isasymtimes}\isactrlsub f\ x{\isacharparenright}{\kern0pt}{\isachardoublequoteclose}\isanewline
\ \ \ \ \ \ \isacommand{using}\isamarkupfalse%
\ right{\isacharunderscore}{\kern0pt}cart{\isacharunderscore}{\kern0pt}proj{\isacharunderscore}{\kern0pt}cfunc{\isacharunderscore}{\kern0pt}cross{\isacharunderscore}{\kern0pt}prod\ x{\isacharunderscore}{\kern0pt}type{\isadigit{2}}\ \isacommand{by}\isamarkupfalse%
\ {\isacharparenleft}{\kern0pt}typecheck{\isacharunderscore}{\kern0pt}cfuncs{\isacharcomma}{\kern0pt}\ auto{\isacharparenright}{\kern0pt}\isanewline
\ \ \ \ \isacommand{also}\isamarkupfalse%
\ \isacommand{have}\isamarkupfalse%
\ {\isachardoublequoteopen}{\isachardot}{\kern0pt}{\isachardot}{\kern0pt}{\isachardot}{\kern0pt}\ {\isacharequal}{\kern0pt}\ {\isacharparenleft}{\kern0pt}{\isacharparenleft}{\kern0pt}eval{\isacharunderscore}{\kern0pt}func\ X\ Y{\isacharparenright}{\kern0pt}\ {\isasymcirc}\isactrlsub c\ {\isacharparenleft}{\kern0pt}id{\isacharparenleft}{\kern0pt}Y{\isacharparenright}{\kern0pt}\ {\isasymtimes}\isactrlsub f\ f{\isacharparenright}{\kern0pt}{\isacharparenright}{\kern0pt}\ {\isasymcirc}\isactrlsub c\ {\isacharparenleft}{\kern0pt}id{\isacharparenleft}{\kern0pt}Y{\isacharparenright}{\kern0pt}\ {\isasymtimes}\isactrlsub f\ x{\isacharparenright}{\kern0pt}{\isachardoublequoteclose}\isanewline
\ \ \ \ \ \ \isacommand{by}\isamarkupfalse%
\ {\isacharparenleft}{\kern0pt}typecheck{\isacharunderscore}{\kern0pt}cfuncs{\isacharcomma}{\kern0pt}\ simp\ add{\isacharcolon}{\kern0pt}\ f{\isacharunderscore}{\kern0pt}def\ transpose{\isacharunderscore}{\kern0pt}func{\isacharunderscore}{\kern0pt}def{\isacharparenright}{\kern0pt}\isanewline
\ \ \ \ \isacommand{also}\isamarkupfalse%
\ \isacommand{have}\isamarkupfalse%
\ {\isachardoublequoteopen}{\isachardot}{\kern0pt}{\isachardot}{\kern0pt}{\isachardot}{\kern0pt}\ {\isacharequal}{\kern0pt}\ {\isacharparenleft}{\kern0pt}eval{\isacharunderscore}{\kern0pt}func\ X\ Y{\isacharparenright}{\kern0pt}\ {\isasymcirc}\isactrlsub c\ {\isacharparenleft}{\kern0pt}{\isacharparenleft}{\kern0pt}id{\isacharparenleft}{\kern0pt}Y{\isacharparenright}{\kern0pt}\ {\isasymtimes}\isactrlsub f\ f{\isacharparenright}{\kern0pt}\ {\isasymcirc}\isactrlsub c\ {\isacharparenleft}{\kern0pt}id{\isacharparenleft}{\kern0pt}Y{\isacharparenright}{\kern0pt}\ {\isasymtimes}\isactrlsub f\ x{\isacharparenright}{\kern0pt}{\isacharparenright}{\kern0pt}{\isachardoublequoteclose}\isanewline
\ \ \ \ \ \ \isacommand{using}\isamarkupfalse%
\ comp{\isacharunderscore}{\kern0pt}associative{\isadigit{2}}\ f{\isacharunderscore}{\kern0pt}type\ x{\isacharunderscore}{\kern0pt}type{\isadigit{2}}\ \isacommand{by}\isamarkupfalse%
\ {\isacharparenleft}{\kern0pt}typecheck{\isacharunderscore}{\kern0pt}cfuncs{\isacharcomma}{\kern0pt}\ fastforce{\isacharparenright}{\kern0pt}\isanewline
\ \ \ \ \isacommand{also}\isamarkupfalse%
\ \isacommand{have}\isamarkupfalse%
\ {\isachardoublequoteopen}{\isachardot}{\kern0pt}{\isachardot}{\kern0pt}{\isachardot}{\kern0pt}\ {\isacharequal}{\kern0pt}\ {\isacharparenleft}{\kern0pt}eval{\isacharunderscore}{\kern0pt}func\ X\ Y{\isacharparenright}{\kern0pt}\ {\isasymcirc}\isactrlsub c\ {\isacharparenleft}{\kern0pt}id{\isacharparenleft}{\kern0pt}Y{\isacharparenright}{\kern0pt}\ {\isasymtimes}\isactrlsub f\ {\isacharparenleft}{\kern0pt}f\ {\isasymcirc}\isactrlsub c\ x{\isacharparenright}{\kern0pt}{\isacharparenright}{\kern0pt}{\isachardoublequoteclose}\isanewline
\ \ \ \ \ \ \isacommand{using}\isamarkupfalse%
\ f{\isacharunderscore}{\kern0pt}type\ identity{\isacharunderscore}{\kern0pt}distributes{\isacharunderscore}{\kern0pt}across{\isacharunderscore}{\kern0pt}composition\ x{\isacharunderscore}{\kern0pt}type{\isadigit{2}}\ \isacommand{by}\isamarkupfalse%
\ auto\isanewline
\ \ \ \ \isacommand{also}\isamarkupfalse%
\ \isacommand{have}\isamarkupfalse%
\ {\isachardoublequoteopen}{\isachardot}{\kern0pt}{\isachardot}{\kern0pt}{\isachardot}{\kern0pt}\ {\isacharequal}{\kern0pt}\ {\isacharparenleft}{\kern0pt}eval{\isacharunderscore}{\kern0pt}func\ X\ Y{\isacharparenright}{\kern0pt}\ {\isasymcirc}\isactrlsub c\ {\isacharparenleft}{\kern0pt}id{\isacharparenleft}{\kern0pt}Y{\isacharparenright}{\kern0pt}\ {\isasymtimes}\isactrlsub f\ {\isacharparenleft}{\kern0pt}f\ {\isasymcirc}\isactrlsub c\ y{\isacharparenright}{\kern0pt}{\isacharparenright}{\kern0pt}{\isachardoublequoteclose}\isanewline
\ \ \ \ \ \ \isacommand{by}\isamarkupfalse%
\ {\isacharparenleft}{\kern0pt}simp\ add{\isacharcolon}{\kern0pt}\ equals{\isacharparenright}{\kern0pt}\isanewline
\ \ \ \ \isacommand{also}\isamarkupfalse%
\ \isacommand{have}\isamarkupfalse%
\ {\isachardoublequoteopen}{\isachardot}{\kern0pt}{\isachardot}{\kern0pt}{\isachardot}{\kern0pt}\ {\isacharequal}{\kern0pt}\ {\isacharparenleft}{\kern0pt}eval{\isacharunderscore}{\kern0pt}func\ X\ Y{\isacharparenright}{\kern0pt}\ {\isasymcirc}\isactrlsub c\ {\isacharparenleft}{\kern0pt}{\isacharparenleft}{\kern0pt}id{\isacharparenleft}{\kern0pt}Y{\isacharparenright}{\kern0pt}\ {\isasymtimes}\isactrlsub f\ f{\isacharparenright}{\kern0pt}\ {\isasymcirc}\isactrlsub c\ {\isacharparenleft}{\kern0pt}id{\isacharparenleft}{\kern0pt}Y{\isacharparenright}{\kern0pt}\ {\isasymtimes}\isactrlsub f\ y{\isacharparenright}{\kern0pt}{\isacharparenright}{\kern0pt}{\isachardoublequoteclose}\isanewline
\ \ \ \ \ \ \isacommand{using}\isamarkupfalse%
\ f{\isacharunderscore}{\kern0pt}type\ identity{\isacharunderscore}{\kern0pt}distributes{\isacharunderscore}{\kern0pt}across{\isacharunderscore}{\kern0pt}composition\ y{\isacharunderscore}{\kern0pt}type{\isadigit{2}}\ \isacommand{by}\isamarkupfalse%
\ auto\isanewline
\ \ \ \ \isacommand{also}\isamarkupfalse%
\ \isacommand{have}\isamarkupfalse%
\ {\isachardoublequoteopen}{\isachardot}{\kern0pt}{\isachardot}{\kern0pt}{\isachardot}{\kern0pt}\ {\isacharequal}{\kern0pt}\ {\isacharparenleft}{\kern0pt}{\isacharparenleft}{\kern0pt}eval{\isacharunderscore}{\kern0pt}func\ X\ Y{\isacharparenright}{\kern0pt}\ {\isasymcirc}\isactrlsub c\ {\isacharparenleft}{\kern0pt}id{\isacharparenleft}{\kern0pt}Y{\isacharparenright}{\kern0pt}\ {\isasymtimes}\isactrlsub f\ f{\isacharparenright}{\kern0pt}{\isacharparenright}{\kern0pt}\ {\isasymcirc}\isactrlsub c\ {\isacharparenleft}{\kern0pt}id{\isacharparenleft}{\kern0pt}Y{\isacharparenright}{\kern0pt}\ {\isasymtimes}\isactrlsub f\ y{\isacharparenright}{\kern0pt}{\isachardoublequoteclose}\isanewline
\ \ \ \ \ \ \isacommand{using}\isamarkupfalse%
\ comp{\isacharunderscore}{\kern0pt}associative{\isadigit{2}}\ f{\isacharunderscore}{\kern0pt}type\ y{\isacharunderscore}{\kern0pt}type{\isadigit{2}}\ \isacommand{by}\isamarkupfalse%
\ {\isacharparenleft}{\kern0pt}typecheck{\isacharunderscore}{\kern0pt}cfuncs{\isacharcomma}{\kern0pt}\ fastforce{\isacharparenright}{\kern0pt}\isanewline
\ \ \ \ \isacommand{also}\isamarkupfalse%
\ \isacommand{have}\isamarkupfalse%
\ {\isachardoublequoteopen}{\isachardot}{\kern0pt}{\isachardot}{\kern0pt}{\isachardot}{\kern0pt}\ {\isacharequal}{\kern0pt}\ {\isacharparenleft}{\kern0pt}right{\isacharunderscore}{\kern0pt}cart{\isacharunderscore}{\kern0pt}proj\ Y\ X{\isacharparenright}{\kern0pt}\ {\isasymcirc}\isactrlsub c\ {\isacharparenleft}{\kern0pt}id{\isacharparenleft}{\kern0pt}Y{\isacharparenright}{\kern0pt}\ {\isasymtimes}\isactrlsub f\ y{\isacharparenright}{\kern0pt}{\isachardoublequoteclose}\isanewline
\ \ \ \ \ \ \isacommand{by}\isamarkupfalse%
\ {\isacharparenleft}{\kern0pt}typecheck{\isacharunderscore}{\kern0pt}cfuncs{\isacharcomma}{\kern0pt}\ simp\ add{\isacharcolon}{\kern0pt}\ f{\isacharunderscore}{\kern0pt}def\ transpose{\isacharunderscore}{\kern0pt}func{\isacharunderscore}{\kern0pt}def{\isacharparenright}{\kern0pt}\isanewline
\ \ \ \ \isacommand{also}\isamarkupfalse%
\ \isacommand{have}\isamarkupfalse%
\ {\isachardoublequoteopen}{\isachardot}{\kern0pt}{\isachardot}{\kern0pt}{\isachardot}{\kern0pt}\ {\isacharequal}{\kern0pt}\ y\ {\isasymcirc}\isactrlsub c\ {\isacharparenleft}{\kern0pt}right{\isacharunderscore}{\kern0pt}cart{\isacharunderscore}{\kern0pt}proj\ Y\ {\isasymone}{\isacharparenright}{\kern0pt}{\isachardoublequoteclose}\isanewline
\ \ \ \ \ \ \isacommand{using}\isamarkupfalse%
\ right{\isacharunderscore}{\kern0pt}cart{\isacharunderscore}{\kern0pt}proj{\isacharunderscore}{\kern0pt}cfunc{\isacharunderscore}{\kern0pt}cross{\isacharunderscore}{\kern0pt}prod\ y{\isacharunderscore}{\kern0pt}type{\isadigit{2}}\ \isacommand{by}\isamarkupfalse%
\ {\isacharparenleft}{\kern0pt}typecheck{\isacharunderscore}{\kern0pt}cfuncs{\isacharcomma}{\kern0pt}\ auto{\isacharparenright}{\kern0pt}\isanewline
\ \ \ \ \isacommand{then}\isamarkupfalse%
\ \isacommand{show}\isamarkupfalse%
\ {\isachardoublequoteopen}x\ {\isacharequal}{\kern0pt}\ y{\isachardoublequoteclose}\isanewline
\ \ \ \ \ \ \isacommand{using}\isamarkupfalse%
\ \ assms\ calculation\ epimorphism{\isacharunderscore}{\kern0pt}def{\isadigit{3}}\ nonempty{\isacharunderscore}{\kern0pt}left{\isacharunderscore}{\kern0pt}imp{\isacharunderscore}{\kern0pt}right{\isacharunderscore}{\kern0pt}proj{\isacharunderscore}{\kern0pt}epimorphism\ right{\isacharunderscore}{\kern0pt}cart{\isacharunderscore}{\kern0pt}proj{\isacharunderscore}{\kern0pt}type\ x{\isacharunderscore}{\kern0pt}type{\isadigit{2}}\ y{\isacharunderscore}{\kern0pt}type{\isadigit{2}}\ \isacommand{by}\isamarkupfalse%
\ fastforce\isanewline
\ \ \isacommand{qed}\isamarkupfalse%
\isanewline
\ \ \isacommand{then}\isamarkupfalse%
\ \isacommand{show}\isamarkupfalse%
\ {\isachardoublequoteopen}X\ {\isasymle}\isactrlsub c\ X\isactrlbsup Y\isactrlesup {\isachardoublequoteclose}\isanewline
\ \ \ \ \isacommand{using}\isamarkupfalse%
\ f{\isacharunderscore}{\kern0pt}type\ injective{\isacharunderscore}{\kern0pt}imp{\isacharunderscore}{\kern0pt}monomorphism\ is{\isacharunderscore}{\kern0pt}smaller{\isacharunderscore}{\kern0pt}than{\isacharunderscore}{\kern0pt}def\ \isacommand{by}\isamarkupfalse%
\ blast\isanewline
\isacommand{qed}\isamarkupfalse%
%
\endisatagproof
{\isafoldproof}%
%
\isadelimproof
\isanewline
%
\endisadelimproof
\isanewline
\isacommand{lemma}\isamarkupfalse%
\ non{\isacharunderscore}{\kern0pt}init{\isacharunderscore}{\kern0pt}non{\isacharunderscore}{\kern0pt}ter{\isacharunderscore}{\kern0pt}sets{\isacharcolon}{\kern0pt}\isanewline
\ \ \isakeyword{assumes}\ {\isachardoublequoteopen}{\isasymnot}{\isacharparenleft}{\kern0pt}terminal{\isacharunderscore}{\kern0pt}object\ X{\isacharparenright}{\kern0pt}{\isachardoublequoteclose}\isanewline
\ \ \isakeyword{assumes}\ {\isachardoublequoteopen}{\isasymnot}{\isacharparenleft}{\kern0pt}initial{\isacharunderscore}{\kern0pt}object\ X{\isacharparenright}{\kern0pt}{\isachardoublequoteclose}\isanewline
\ \ \isakeyword{shows}\ {\isachardoublequoteopen}{\isasymOmega}\ {\isasymle}\isactrlsub c\ X{\isachardoublequoteclose}\ \isanewline
%
\isadelimproof
%
\endisadelimproof
%
\isatagproof
\isacommand{proof}\isamarkupfalse%
\ {\isacharminus}{\kern0pt}\ \isanewline
\ \ \isacommand{obtain}\isamarkupfalse%
\ x{\isadigit{1}}\ \isakeyword{and}\ x{\isadigit{2}}\ \isakeyword{where}\ x{\isadigit{1}}{\isacharunderscore}{\kern0pt}type{\isacharbrackleft}{\kern0pt}type{\isacharunderscore}{\kern0pt}rule{\isacharbrackright}{\kern0pt}{\isacharcolon}{\kern0pt}\ {\isachardoublequoteopen}x{\isadigit{1}}\ {\isasymin}\isactrlsub c\ X{\isachardoublequoteclose}\ \isakeyword{and}\ \isanewline
\ \ \ \ \ \ \ \ \ \ \ \ \ \ \ \ \ \ \ \ \ \ \ \ \ x{\isadigit{2}}{\isacharunderscore}{\kern0pt}type{\isacharbrackleft}{\kern0pt}type{\isacharunderscore}{\kern0pt}rule{\isacharbrackright}{\kern0pt}{\isacharcolon}{\kern0pt}\ {\isachardoublequoteopen}x{\isadigit{2}}\ {\isasymin}\isactrlsub c\ X{\isachardoublequoteclose}\ \isakeyword{and}\isanewline
\ \ \ \ \ \ \ \ \ \ \ \ \ \ \ \ \ \ \ \ \ \ \ \ \ \ \ \ \ \ \ \ \ \ \ distinct{\isacharcolon}{\kern0pt}\ {\isachardoublequoteopen}x{\isadigit{1}}\ {\isasymnoteq}\ x{\isadigit{2}}{\isachardoublequoteclose}\isanewline
\ \ \ \ \isacommand{using}\isamarkupfalse%
\ is{\isacharunderscore}{\kern0pt}empty{\isacharunderscore}{\kern0pt}def\ assms\ iso{\isacharunderscore}{\kern0pt}empty{\isacharunderscore}{\kern0pt}initial\ iso{\isacharunderscore}{\kern0pt}to{\isadigit{1}}{\isacharunderscore}{\kern0pt}is{\isacharunderscore}{\kern0pt}term\ no{\isacharunderscore}{\kern0pt}el{\isacharunderscore}{\kern0pt}iff{\isacharunderscore}{\kern0pt}iso{\isacharunderscore}{\kern0pt}empty\ single{\isacharunderscore}{\kern0pt}elem{\isacharunderscore}{\kern0pt}iso{\isacharunderscore}{\kern0pt}one\ \isacommand{by}\isamarkupfalse%
\ blast\isanewline
\ \ \isacommand{then}\isamarkupfalse%
\ \isacommand{have}\isamarkupfalse%
\ map{\isacharunderscore}{\kern0pt}type{\isacharcolon}{\kern0pt}\ {\isachardoublequoteopen}{\isacharparenleft}{\kern0pt}x{\isadigit{1}}\ {\isasymamalg}\ x{\isadigit{2}}{\isacharparenright}{\kern0pt}\ {\isasymcirc}\isactrlsub c\ case{\isacharunderscore}{\kern0pt}bool\ \ \ {\isacharcolon}{\kern0pt}\ {\isasymOmega}\ {\isasymrightarrow}\ X{\isachardoublequoteclose}\isanewline
\ \ \ \ \isacommand{by}\isamarkupfalse%
\ typecheck{\isacharunderscore}{\kern0pt}cfuncs\isanewline
\ \ \isacommand{have}\isamarkupfalse%
\ injective{\isacharcolon}{\kern0pt}\ {\isachardoublequoteopen}injective{\isacharparenleft}{\kern0pt}{\isacharparenleft}{\kern0pt}x{\isadigit{1}}\ {\isasymamalg}\ x{\isadigit{2}}{\isacharparenright}{\kern0pt}\ {\isasymcirc}\isactrlsub c\ case{\isacharunderscore}{\kern0pt}bool{\isacharparenright}{\kern0pt}{\isachardoublequoteclose}\isanewline
\ \ \ \ \isacommand{unfolding}\isamarkupfalse%
\ injective{\isacharunderscore}{\kern0pt}def\isanewline
\ \ \isacommand{proof}\isamarkupfalse%
{\isacharparenleft}{\kern0pt}clarify{\isacharparenright}{\kern0pt}\isanewline
\ \ \ \ \isacommand{fix}\isamarkupfalse%
\ {\isasymomega}{\isadigit{1}}\ {\isasymomega}{\isadigit{2}}\ \isanewline
\ \ \ \ \isacommand{assume}\isamarkupfalse%
\ {\isachardoublequoteopen}{\isasymomega}{\isadigit{1}}\ {\isasymin}\isactrlsub c\ domain\ {\isacharparenleft}{\kern0pt}x{\isadigit{1}}\ {\isasymamalg}\ x{\isadigit{2}}\ {\isasymcirc}\isactrlsub c\ case{\isacharunderscore}{\kern0pt}bool{\isacharparenright}{\kern0pt}{\isachardoublequoteclose}\isanewline
\ \ \ \ \isacommand{then}\isamarkupfalse%
\ \isacommand{have}\isamarkupfalse%
\ {\isasymomega}{\isadigit{1}}{\isacharunderscore}{\kern0pt}type{\isacharbrackleft}{\kern0pt}type{\isacharunderscore}{\kern0pt}rule{\isacharbrackright}{\kern0pt}{\isacharcolon}{\kern0pt}\ {\isachardoublequoteopen}{\isasymomega}{\isadigit{1}}\ {\isasymin}\isactrlsub c\ {\isasymOmega}{\isachardoublequoteclose}\isanewline
\ \ \ \ \ \ \isacommand{using}\isamarkupfalse%
\ cfunc{\isacharunderscore}{\kern0pt}type{\isacharunderscore}{\kern0pt}def\ map{\isacharunderscore}{\kern0pt}type\ \isacommand{by}\isamarkupfalse%
\ auto\isanewline
\ \ \ \ \isacommand{assume}\isamarkupfalse%
\ {\isachardoublequoteopen}{\isasymomega}{\isadigit{2}}\ {\isasymin}\isactrlsub c\ domain\ {\isacharparenleft}{\kern0pt}x{\isadigit{1}}\ {\isasymamalg}\ x{\isadigit{2}}\ {\isasymcirc}\isactrlsub c\ case{\isacharunderscore}{\kern0pt}bool{\isacharparenright}{\kern0pt}{\isachardoublequoteclose}\isanewline
\ \ \ \ \isacommand{then}\isamarkupfalse%
\ \isacommand{have}\isamarkupfalse%
\ {\isasymomega}{\isadigit{2}}{\isacharunderscore}{\kern0pt}type{\isacharbrackleft}{\kern0pt}type{\isacharunderscore}{\kern0pt}rule{\isacharbrackright}{\kern0pt}{\isacharcolon}{\kern0pt}\ {\isachardoublequoteopen}{\isasymomega}{\isadigit{2}}\ {\isasymin}\isactrlsub c\ {\isasymOmega}{\isachardoublequoteclose}\isanewline
\ \ \ \ \ \ \isacommand{using}\isamarkupfalse%
\ cfunc{\isacharunderscore}{\kern0pt}type{\isacharunderscore}{\kern0pt}def\ map{\isacharunderscore}{\kern0pt}type\ \isacommand{by}\isamarkupfalse%
\ auto\isanewline
\ \ \ \ \isanewline
\ \ \ \ \isacommand{assume}\isamarkupfalse%
\ equals{\isacharcolon}{\kern0pt}\ {\isachardoublequoteopen}{\isacharparenleft}{\kern0pt}x{\isadigit{1}}\ {\isasymamalg}\ x{\isadigit{2}}\ {\isasymcirc}\isactrlsub c\ case{\isacharunderscore}{\kern0pt}bool{\isacharparenright}{\kern0pt}\ {\isasymcirc}\isactrlsub c\ {\isasymomega}{\isadigit{1}}\ {\isacharequal}{\kern0pt}\ {\isacharparenleft}{\kern0pt}x{\isadigit{1}}\ {\isasymamalg}\ x{\isadigit{2}}\ {\isasymcirc}\isactrlsub c\ case{\isacharunderscore}{\kern0pt}bool{\isacharparenright}{\kern0pt}\ {\isasymcirc}\isactrlsub c\ {\isasymomega}{\isadigit{2}}{\isachardoublequoteclose}\isanewline
\ \ \ \ \isacommand{show}\isamarkupfalse%
\ {\isachardoublequoteopen}{\isasymomega}{\isadigit{1}}\ {\isacharequal}{\kern0pt}\ {\isasymomega}{\isadigit{2}}{\isachardoublequoteclose}\isanewline
\ \ \ \ \isacommand{proof}\isamarkupfalse%
{\isacharparenleft}{\kern0pt}cases\ {\isachardoublequoteopen}{\isasymomega}{\isadigit{1}}\ {\isacharequal}{\kern0pt}\ {\isasymt}{\isachardoublequoteclose}{\isacharcomma}{\kern0pt}\ clarify{\isacharparenright}{\kern0pt}\isanewline
\ \ \ \ \ \ \isacommand{assume}\isamarkupfalse%
\ {\isachardoublequoteopen}{\isasymomega}{\isadigit{1}}\ {\isacharequal}{\kern0pt}\ {\isasymt}{\isachardoublequoteclose}\isanewline
\ \ \ \ \ \ \isacommand{show}\isamarkupfalse%
\ {\isachardoublequoteopen}{\isasymt}\ {\isacharequal}{\kern0pt}\ {\isasymomega}{\isadigit{2}}{\isachardoublequoteclose}\isanewline
\ \ \ \ \ \ \isacommand{proof}\isamarkupfalse%
{\isacharparenleft}{\kern0pt}rule\ ccontr{\isacharparenright}{\kern0pt}\isanewline
\ \ \ \ \ \ \ \ \isacommand{assume}\isamarkupfalse%
\ {\isachardoublequoteopen}\ {\isasymt}\ {\isasymnoteq}\ {\isasymomega}{\isadigit{2}}{\isachardoublequoteclose}\isanewline
\ \ \ \ \ \ \ \ \isacommand{then}\isamarkupfalse%
\ \isacommand{have}\isamarkupfalse%
\ {\isachardoublequoteopen}{\isasymf}\ {\isacharequal}{\kern0pt}\ {\isasymomega}{\isadigit{2}}{\isachardoublequoteclose}\isanewline
\ \ \ \ \ \ \ \ \ \ \isacommand{using}\isamarkupfalse%
\ {\isacartoucheopen}{\isasymt}\ {\isasymnoteq}\ {\isasymomega}{\isadigit{2}}{\isacartoucheclose}\ true{\isacharunderscore}{\kern0pt}false{\isacharunderscore}{\kern0pt}only{\isacharunderscore}{\kern0pt}truth{\isacharunderscore}{\kern0pt}values\ \isacommand{by}\isamarkupfalse%
\ {\isacharparenleft}{\kern0pt}typecheck{\isacharunderscore}{\kern0pt}cfuncs{\isacharcomma}{\kern0pt}\ blast{\isacharparenright}{\kern0pt}\isanewline
\ \ \ \ \ \ \ \ \isacommand{then}\isamarkupfalse%
\ \isacommand{have}\isamarkupfalse%
\ RHS{\isacharcolon}{\kern0pt}\ {\isachardoublequoteopen}{\isacharparenleft}{\kern0pt}x{\isadigit{1}}\ {\isasymamalg}\ x{\isadigit{2}}\ {\isasymcirc}\isactrlsub c\ case{\isacharunderscore}{\kern0pt}bool{\isacharparenright}{\kern0pt}\ {\isasymcirc}\isactrlsub c\ {\isasymomega}{\isadigit{2}}\ {\isacharequal}{\kern0pt}\ x{\isadigit{2}}{\isachardoublequoteclose}\isanewline
\ \ \ \ \ \ \ \ \ \ \isacommand{by}\isamarkupfalse%
\ {\isacharparenleft}{\kern0pt}meson\ coprod{\isacharunderscore}{\kern0pt}case{\isacharunderscore}{\kern0pt}bool{\isacharunderscore}{\kern0pt}false\ x{\isadigit{1}}{\isacharunderscore}{\kern0pt}type\ x{\isadigit{2}}{\isacharunderscore}{\kern0pt}type{\isacharparenright}{\kern0pt}\isanewline
\ \ \ \ \ \ \ \ \isacommand{have}\isamarkupfalse%
\ {\isachardoublequoteopen}{\isacharparenleft}{\kern0pt}x{\isadigit{1}}\ {\isasymamalg}\ x{\isadigit{2}}\ {\isasymcirc}\isactrlsub c\ case{\isacharunderscore}{\kern0pt}bool{\isacharparenright}{\kern0pt}\ {\isasymcirc}\isactrlsub c\ {\isasymomega}{\isadigit{1}}\ {\isacharequal}{\kern0pt}\ x{\isadigit{1}}{\isachardoublequoteclose}\isanewline
\ \ \ \ \ \ \ \ \ \ \isacommand{using}\isamarkupfalse%
\ {\isacartoucheopen}{\isasymomega}{\isadigit{1}}\ {\isacharequal}{\kern0pt}\ {\isasymt}{\isacartoucheclose}\ coprod{\isacharunderscore}{\kern0pt}case{\isacharunderscore}{\kern0pt}bool{\isacharunderscore}{\kern0pt}true\ x{\isadigit{1}}{\isacharunderscore}{\kern0pt}type\ x{\isadigit{2}}{\isacharunderscore}{\kern0pt}type\ \isacommand{by}\isamarkupfalse%
\ blast\isanewline
\ \ \ \ \ \ \ \ \isacommand{then}\isamarkupfalse%
\ \isacommand{show}\isamarkupfalse%
\ False\isanewline
\ \ \ \ \ \ \ \ \ \ \isacommand{using}\isamarkupfalse%
\ RHS\ distinct\ equals\ \isacommand{by}\isamarkupfalse%
\ force\isanewline
\ \ \ \ \ \ \isacommand{qed}\isamarkupfalse%
\isanewline
\ \ \ \ \isacommand{next}\isamarkupfalse%
\isanewline
\ \ \ \ \ \ \isacommand{assume}\isamarkupfalse%
\ {\isachardoublequoteopen}{\isasymomega}{\isadigit{1}}\ {\isasymnoteq}\ {\isasymt}{\isachardoublequoteclose}\isanewline
\ \ \ \ \ \ \isacommand{then}\isamarkupfalse%
\ \isacommand{have}\isamarkupfalse%
\ {\isachardoublequoteopen}{\isasymomega}{\isadigit{1}}\ {\isacharequal}{\kern0pt}\ {\isasymf}{\isachardoublequoteclose}\isanewline
\ \ \ \ \ \ \ \ \isacommand{using}\isamarkupfalse%
\ \ true{\isacharunderscore}{\kern0pt}false{\isacharunderscore}{\kern0pt}only{\isacharunderscore}{\kern0pt}truth{\isacharunderscore}{\kern0pt}values\ \isacommand{by}\isamarkupfalse%
\ {\isacharparenleft}{\kern0pt}typecheck{\isacharunderscore}{\kern0pt}cfuncs{\isacharcomma}{\kern0pt}\ blast{\isacharparenright}{\kern0pt}\isanewline
\ \ \ \ \ \ \isacommand{have}\isamarkupfalse%
\ {\isachardoublequoteopen}{\isasymomega}{\isadigit{2}}\ {\isacharequal}{\kern0pt}\ {\isasymf}{\isachardoublequoteclose}\isanewline
\ \ \ \ \ \ \isacommand{proof}\isamarkupfalse%
{\isacharparenleft}{\kern0pt}rule\ ccontr{\isacharparenright}{\kern0pt}\isanewline
\ \ \ \ \ \ \ \ \isacommand{assume}\isamarkupfalse%
\ {\isachardoublequoteopen}{\isasymomega}{\isadigit{2}}\ {\isasymnoteq}\ {\isasymf}{\isachardoublequoteclose}\isanewline
\ \ \ \ \ \ \ \ \isacommand{then}\isamarkupfalse%
\ \isacommand{have}\isamarkupfalse%
\ {\isachardoublequoteopen}{\isasymomega}{\isadigit{2}}\ {\isacharequal}{\kern0pt}\ {\isasymt}{\isachardoublequoteclose}\isanewline
\ \ \ \ \ \ \ \ \ \ \isacommand{using}\isamarkupfalse%
\ \ true{\isacharunderscore}{\kern0pt}false{\isacharunderscore}{\kern0pt}only{\isacharunderscore}{\kern0pt}truth{\isacharunderscore}{\kern0pt}values\ \isacommand{by}\isamarkupfalse%
\ {\isacharparenleft}{\kern0pt}typecheck{\isacharunderscore}{\kern0pt}cfuncs{\isacharcomma}{\kern0pt}\ blast{\isacharparenright}{\kern0pt}\isanewline
\ \ \ \ \ \ \ \ \isacommand{then}\isamarkupfalse%
\ \isacommand{have}\isamarkupfalse%
\ RHS{\isacharcolon}{\kern0pt}\ {\isachardoublequoteopen}{\isacharparenleft}{\kern0pt}x{\isadigit{1}}\ {\isasymamalg}\ x{\isadigit{2}}\ {\isasymcirc}\isactrlsub c\ case{\isacharunderscore}{\kern0pt}bool{\isacharparenright}{\kern0pt}\ {\isasymcirc}\isactrlsub c\ {\isasymomega}{\isadigit{2}}\ {\isacharequal}{\kern0pt}\ x{\isadigit{2}}{\isachardoublequoteclose}\isanewline
\ \ \ \ \ \ \ \ \ \ \isacommand{using}\isamarkupfalse%
\ {\isacartoucheopen}{\isasymomega}{\isadigit{1}}\ {\isacharequal}{\kern0pt}\ {\isasymf}{\isacartoucheclose}\ coprod{\isacharunderscore}{\kern0pt}case{\isacharunderscore}{\kern0pt}bool{\isacharunderscore}{\kern0pt}false\ equals\ x{\isadigit{1}}{\isacharunderscore}{\kern0pt}type\ x{\isadigit{2}}{\isacharunderscore}{\kern0pt}type\ \isacommand{by}\isamarkupfalse%
\ auto\isanewline
\ \ \ \ \ \ \ \ \isacommand{have}\isamarkupfalse%
\ {\isachardoublequoteopen}{\isacharparenleft}{\kern0pt}x{\isadigit{1}}\ {\isasymamalg}\ x{\isadigit{2}}\ {\isasymcirc}\isactrlsub c\ case{\isacharunderscore}{\kern0pt}bool{\isacharparenright}{\kern0pt}\ {\isasymcirc}\isactrlsub c\ {\isasymomega}{\isadigit{1}}\ {\isacharequal}{\kern0pt}\ x{\isadigit{1}}{\isachardoublequoteclose}\isanewline
\ \ \ \ \ \ \ \ \ \ \isacommand{using}\isamarkupfalse%
\ {\isacartoucheopen}{\isasymomega}{\isadigit{2}}\ {\isacharequal}{\kern0pt}\ {\isasymt}{\isacartoucheclose}\ coprod{\isacharunderscore}{\kern0pt}case{\isacharunderscore}{\kern0pt}bool{\isacharunderscore}{\kern0pt}true\ equals\ x{\isadigit{1}}{\isacharunderscore}{\kern0pt}type\ x{\isadigit{2}}{\isacharunderscore}{\kern0pt}type\ \isacommand{by}\isamarkupfalse%
\ presburger\isanewline
\ \ \ \ \ \ \ \ \isacommand{then}\isamarkupfalse%
\ \isacommand{show}\isamarkupfalse%
\ False\isanewline
\ \ \ \ \ \ \ \ \ \ \isacommand{using}\isamarkupfalse%
\ RHS\ distinct\ equals\ \isacommand{by}\isamarkupfalse%
\ auto\isanewline
\ \ \ \ \ \ \isacommand{qed}\isamarkupfalse%
\isanewline
\ \ \ \ \ \ \isacommand{show}\isamarkupfalse%
\ {\isachardoublequoteopen}{\isasymomega}{\isadigit{1}}\ {\isacharequal}{\kern0pt}\ {\isasymomega}{\isadigit{2}}{\isachardoublequoteclose}\isanewline
\ \ \ \ \ \ \ \ \isacommand{by}\isamarkupfalse%
\ {\isacharparenleft}{\kern0pt}simp\ add{\isacharcolon}{\kern0pt}\ {\isacartoucheopen}{\isasymomega}{\isadigit{1}}\ {\isacharequal}{\kern0pt}\ {\isasymf}{\isacartoucheclose}\ {\isacartoucheopen}{\isasymomega}{\isadigit{2}}\ {\isacharequal}{\kern0pt}\ {\isasymf}{\isacartoucheclose}{\isacharparenright}{\kern0pt}\isanewline
\ \ \ \ \isacommand{qed}\isamarkupfalse%
\isanewline
\ \ \isacommand{qed}\isamarkupfalse%
\isanewline
\ \ \isacommand{then}\isamarkupfalse%
\ \isacommand{have}\isamarkupfalse%
\ {\isachardoublequoteopen}monomorphism{\isacharparenleft}{\kern0pt}{\isacharparenleft}{\kern0pt}x{\isadigit{1}}\ {\isasymamalg}\ x{\isadigit{2}}{\isacharparenright}{\kern0pt}\ {\isasymcirc}\isactrlsub c\ case{\isacharunderscore}{\kern0pt}bool{\isacharparenright}{\kern0pt}{\isachardoublequoteclose}\isanewline
\ \ \ \ \isacommand{using}\isamarkupfalse%
\ injective{\isacharunderscore}{\kern0pt}imp{\isacharunderscore}{\kern0pt}monomorphism\ \isacommand{by}\isamarkupfalse%
\ auto\isanewline
\ \ \isacommand{then}\isamarkupfalse%
\ \isacommand{show}\isamarkupfalse%
\ {\isachardoublequoteopen}{\isasymOmega}\ {\isasymle}\isactrlsub c\ X{\isachardoublequoteclose}\isanewline
\ \ \ \ \isacommand{using}\isamarkupfalse%
\ \ is{\isacharunderscore}{\kern0pt}smaller{\isacharunderscore}{\kern0pt}than{\isacharunderscore}{\kern0pt}def\ map{\isacharunderscore}{\kern0pt}type\ \isacommand{by}\isamarkupfalse%
\ blast\isanewline
\isacommand{qed}\isamarkupfalse%
%
\endisatagproof
{\isafoldproof}%
%
\isadelimproof
\isanewline
%
\endisadelimproof
\isanewline
\isacommand{lemma}\isamarkupfalse%
\ exp{\isacharunderscore}{\kern0pt}preserves{\isacharunderscore}{\kern0pt}card{\isadigit{1}}{\isacharcolon}{\kern0pt}\isanewline
\ \ \isakeyword{assumes}\ {\isachardoublequoteopen}A\ {\isasymle}\isactrlsub c\ B{\isachardoublequoteclose}\isanewline
\ \ \isakeyword{assumes}\ {\isachardoublequoteopen}nonempty\ X{\isachardoublequoteclose}\ \ \ \isanewline
\ \ \isakeyword{shows}\ {\isachardoublequoteopen}X\isactrlbsup A\isactrlesup \ {\isasymle}\isactrlsub c\ X\isactrlbsup B\isactrlesup {\isachardoublequoteclose}\isanewline
%
\isadelimproof
\ \ %
\endisadelimproof
%
\isatagproof
\isacommand{unfolding}\isamarkupfalse%
\ is{\isacharunderscore}{\kern0pt}smaller{\isacharunderscore}{\kern0pt}than{\isacharunderscore}{\kern0pt}def\isanewline
\isacommand{proof}\isamarkupfalse%
\ {\isacharminus}{\kern0pt}\isanewline
\ \ \isacommand{obtain}\isamarkupfalse%
\ x\ \isakeyword{where}\ x{\isacharunderscore}{\kern0pt}type{\isacharbrackleft}{\kern0pt}type{\isacharunderscore}{\kern0pt}rule{\isacharbrackright}{\kern0pt}{\isacharcolon}{\kern0pt}\ {\isachardoublequoteopen}x\ {\isasymin}\isactrlsub c\ X{\isachardoublequoteclose}\isanewline
\ \ \ \ \isacommand{using}\isamarkupfalse%
\ assms{\isacharparenleft}{\kern0pt}{\isadigit{2}}{\isacharparenright}{\kern0pt}\ \isacommand{unfolding}\isamarkupfalse%
\ nonempty{\isacharunderscore}{\kern0pt}def\ \isacommand{by}\isamarkupfalse%
\ auto\isanewline
\ \ \isacommand{obtain}\isamarkupfalse%
\ m\ \isakeyword{where}\ m{\isacharunderscore}{\kern0pt}def{\isacharbrackleft}{\kern0pt}type{\isacharunderscore}{\kern0pt}rule{\isacharbrackright}{\kern0pt}{\isacharcolon}{\kern0pt}\ {\isachardoublequoteopen}m\ {\isacharcolon}{\kern0pt}\ A\ {\isasymrightarrow}\ B{\isachardoublequoteclose}\ {\isachardoublequoteopen}monomorphism\ m{\isachardoublequoteclose}\isanewline
\ \ \ \ \isacommand{using}\isamarkupfalse%
\ assms{\isacharparenleft}{\kern0pt}{\isadigit{1}}{\isacharparenright}{\kern0pt}\ \isacommand{unfolding}\isamarkupfalse%
\ is{\isacharunderscore}{\kern0pt}smaller{\isacharunderscore}{\kern0pt}than{\isacharunderscore}{\kern0pt}def\ \isacommand{by}\isamarkupfalse%
\ auto\isanewline
\ \ \isacommand{show}\isamarkupfalse%
\ {\isachardoublequoteopen}{\isasymexists}m{\isachardot}{\kern0pt}\ m\ {\isacharcolon}{\kern0pt}\ X\isactrlbsup A\isactrlesup \ {\isasymrightarrow}\ X\isactrlbsup B\isactrlesup \ {\isasymand}\ monomorphism\ m{\isachardoublequoteclose}\isanewline
\ \ \isacommand{proof}\isamarkupfalse%
\ {\isacharparenleft}{\kern0pt}intro\ exI{\isacharbrackleft}{\kern0pt}\isakeyword{where}\ x{\isacharequal}{\kern0pt}{\isachardoublequoteopen}{\isacharparenleft}{\kern0pt}{\isacharparenleft}{\kern0pt}{\isacharparenleft}{\kern0pt}eval{\isacharunderscore}{\kern0pt}func\ X\ A\ {\isasymcirc}\isactrlsub c\ swap\ {\isacharparenleft}{\kern0pt}X\isactrlbsup A\isactrlesup {\isacharparenright}{\kern0pt}\ A{\isacharparenright}{\kern0pt}\ {\isasymamalg}\ {\isacharparenleft}{\kern0pt}x\ {\isasymcirc}\isactrlsub c\ {\isasymbeta}\isactrlbsub X\isactrlbsup A\isactrlesup \ {\isasymtimes}\isactrlsub c\ {\isacharparenleft}{\kern0pt}B\ {\isasymsetminus}\ {\isacharparenleft}{\kern0pt}A{\isacharcomma}{\kern0pt}\ m{\isacharparenright}{\kern0pt}{\isacharparenright}{\kern0pt}\isactrlesub {\isacharparenright}{\kern0pt}{\isacharparenright}{\kern0pt}\isanewline
\ \ \ \ {\isasymcirc}\isactrlsub c\ dist{\isacharunderscore}{\kern0pt}prod{\isacharunderscore}{\kern0pt}coprod{\isacharunderscore}{\kern0pt}left\ {\isacharparenleft}{\kern0pt}X\isactrlbsup A\isactrlesup {\isacharparenright}{\kern0pt}\ A\ {\isacharparenleft}{\kern0pt}B\ {\isasymsetminus}\ {\isacharparenleft}{\kern0pt}A{\isacharcomma}{\kern0pt}\ m{\isacharparenright}{\kern0pt}{\isacharparenright}{\kern0pt}\ \isanewline
\ \ \ \ {\isasymcirc}\isactrlsub c\ swap\ {\isacharparenleft}{\kern0pt}A\ {\isasymCoprod}\ {\isacharparenleft}{\kern0pt}B\ {\isasymsetminus}\ {\isacharparenleft}{\kern0pt}A{\isacharcomma}{\kern0pt}\ m{\isacharparenright}{\kern0pt}{\isacharparenright}{\kern0pt}{\isacharparenright}{\kern0pt}\ {\isacharparenleft}{\kern0pt}X\isactrlbsup A\isactrlesup {\isacharparenright}{\kern0pt}\ {\isasymcirc}\isactrlsub c\ {\isacharparenleft}{\kern0pt}try{\isacharunderscore}{\kern0pt}cast\ m\ {\isasymtimes}\isactrlsub f\ id\ {\isacharparenleft}{\kern0pt}X\isactrlbsup A\isactrlesup {\isacharparenright}{\kern0pt}{\isacharparenright}{\kern0pt}{\isacharparenright}{\kern0pt}\isactrlsup {\isasymsharp}{\isachardoublequoteclose}{\isacharbrackright}{\kern0pt}{\isacharcomma}{\kern0pt}\ safe{\isacharparenright}{\kern0pt}\isanewline
\isanewline
\ \ \ \ \isacommand{show}\isamarkupfalse%
\ {\isachardoublequoteopen}{\isacharparenleft}{\kern0pt}{\isacharparenleft}{\kern0pt}eval{\isacharunderscore}{\kern0pt}func\ X\ A\ {\isasymcirc}\isactrlsub c\ swap\ {\isacharparenleft}{\kern0pt}X\isactrlbsup A\isactrlesup {\isacharparenright}{\kern0pt}\ A{\isacharparenright}{\kern0pt}\ {\isasymamalg}\ {\isacharparenleft}{\kern0pt}x\ {\isasymcirc}\isactrlsub c\ {\isasymbeta}\isactrlbsub X\isactrlbsup A\isactrlesup \ {\isasymtimes}\isactrlsub c\ {\isacharparenleft}{\kern0pt}B\ {\isasymsetminus}\ {\isacharparenleft}{\kern0pt}A{\isacharcomma}{\kern0pt}\ m{\isacharparenright}{\kern0pt}{\isacharparenright}{\kern0pt}\isactrlesub {\isacharparenright}{\kern0pt}\ {\isasymcirc}\isactrlsub c\ dist{\isacharunderscore}{\kern0pt}prod{\isacharunderscore}{\kern0pt}coprod{\isacharunderscore}{\kern0pt}left\ {\isacharparenleft}{\kern0pt}X\isactrlbsup A\isactrlesup {\isacharparenright}{\kern0pt}\ A\ {\isacharparenleft}{\kern0pt}B\ {\isasymsetminus}\ {\isacharparenleft}{\kern0pt}A{\isacharcomma}{\kern0pt}\ m{\isacharparenright}{\kern0pt}{\isacharparenright}{\kern0pt}\ {\isasymcirc}\isactrlsub c\ swap\ {\isacharparenleft}{\kern0pt}A\ {\isasymCoprod}\ {\isacharparenleft}{\kern0pt}B\ {\isasymsetminus}\ {\isacharparenleft}{\kern0pt}A{\isacharcomma}{\kern0pt}\ m{\isacharparenright}{\kern0pt}{\isacharparenright}{\kern0pt}{\isacharparenright}{\kern0pt}\ {\isacharparenleft}{\kern0pt}X\isactrlbsup A\isactrlesup {\isacharparenright}{\kern0pt}\ {\isasymcirc}\isactrlsub c\ try{\isacharunderscore}{\kern0pt}cast\ m\ {\isasymtimes}\isactrlsub f\ id\isactrlsub c\ {\isacharparenleft}{\kern0pt}X\isactrlbsup A\isactrlesup {\isacharparenright}{\kern0pt}{\isacharparenright}{\kern0pt}\isactrlsup {\isasymsharp}\ {\isacharcolon}{\kern0pt}\ X\isactrlbsup A\isactrlesup \ {\isasymrightarrow}\ X\isactrlbsup B\isactrlesup {\isachardoublequoteclose}\isanewline
\ \ \ \ \ \ \isacommand{by}\isamarkupfalse%
\ \ typecheck{\isacharunderscore}{\kern0pt}cfuncs\isanewline
\ \ \ \ \isacommand{then}\isamarkupfalse%
\ \isacommand{show}\isamarkupfalse%
\ {\isachardoublequoteopen}monomorphism\isanewline
\ \ \ \ \ \ {\isacharparenleft}{\kern0pt}{\isacharparenleft}{\kern0pt}{\isacharparenleft}{\kern0pt}eval{\isacharunderscore}{\kern0pt}func\ X\ A\ {\isasymcirc}\isactrlsub c\ swap\ {\isacharparenleft}{\kern0pt}X\isactrlbsup A\isactrlesup {\isacharparenright}{\kern0pt}\ A{\isacharparenright}{\kern0pt}\ {\isasymamalg}\ {\isacharparenleft}{\kern0pt}x\ {\isasymcirc}\isactrlsub c\ {\isasymbeta}\isactrlbsub X\isactrlbsup A\isactrlesup \ {\isasymtimes}\isactrlsub c\ {\isacharparenleft}{\kern0pt}B\ {\isasymsetminus}\ {\isacharparenleft}{\kern0pt}A{\isacharcomma}{\kern0pt}\ m{\isacharparenright}{\kern0pt}{\isacharparenright}{\kern0pt}\isactrlesub {\isacharparenright}{\kern0pt}\ {\isasymcirc}\isactrlsub c\isanewline
\ \ \ \ \ \ \ \ dist{\isacharunderscore}{\kern0pt}prod{\isacharunderscore}{\kern0pt}coprod{\isacharunderscore}{\kern0pt}left\ {\isacharparenleft}{\kern0pt}X\isactrlbsup A\isactrlesup {\isacharparenright}{\kern0pt}\ A\ {\isacharparenleft}{\kern0pt}B\ {\isasymsetminus}\ {\isacharparenleft}{\kern0pt}A{\isacharcomma}{\kern0pt}\ m{\isacharparenright}{\kern0pt}{\isacharparenright}{\kern0pt}\ {\isasymcirc}\isactrlsub c\isanewline
\ \ \ \ \ \ \ \ swap\ {\isacharparenleft}{\kern0pt}A\ {\isasymCoprod}\ {\isacharparenleft}{\kern0pt}B\ {\isasymsetminus}\ {\isacharparenleft}{\kern0pt}A{\isacharcomma}{\kern0pt}\ m{\isacharparenright}{\kern0pt}{\isacharparenright}{\kern0pt}{\isacharparenright}{\kern0pt}\ {\isacharparenleft}{\kern0pt}X\isactrlbsup A\isactrlesup {\isacharparenright}{\kern0pt}\ {\isasymcirc}\isactrlsub c\ try{\isacharunderscore}{\kern0pt}cast\ m\ {\isasymtimes}\isactrlsub f\ id\isactrlsub c\ {\isacharparenleft}{\kern0pt}X\isactrlbsup A\isactrlesup {\isacharparenright}{\kern0pt}{\isacharparenright}{\kern0pt}\isactrlsup {\isasymsharp}{\isacharparenright}{\kern0pt}{\isachardoublequoteclose}\isanewline
\ \ \ \ \isacommand{proof}\isamarkupfalse%
\ {\isacharparenleft}{\kern0pt}unfold\ monomorphism{\isacharunderscore}{\kern0pt}def{\isadigit{3}}{\isacharcomma}{\kern0pt}\ clarify{\isacharparenright}{\kern0pt}\isanewline
\ \ \ \ \ \ \isacommand{fix}\isamarkupfalse%
\ g\ h\ Z\isanewline
\ \ \ \ \ \ \isacommand{assume}\isamarkupfalse%
\ g{\isacharunderscore}{\kern0pt}type{\isacharbrackleft}{\kern0pt}type{\isacharunderscore}{\kern0pt}rule{\isacharbrackright}{\kern0pt}{\isacharcolon}{\kern0pt}\ {\isachardoublequoteopen}g\ {\isacharcolon}{\kern0pt}\ Z\ {\isasymrightarrow}\ X\isactrlbsup A\isactrlesup {\isachardoublequoteclose}\isanewline
\ \ \ \ \ \ \isacommand{assume}\isamarkupfalse%
\ h{\isacharunderscore}{\kern0pt}type{\isacharbrackleft}{\kern0pt}type{\isacharunderscore}{\kern0pt}rule{\isacharbrackright}{\kern0pt}{\isacharcolon}{\kern0pt}\ {\isachardoublequoteopen}h\ {\isacharcolon}{\kern0pt}\ Z\ {\isasymrightarrow}\ X\isactrlbsup A\isactrlesup {\isachardoublequoteclose}\isanewline
\ \ \ \ \ \ \isacommand{assume}\isamarkupfalse%
\ eq{\isacharcolon}{\kern0pt}\ {\isachardoublequoteopen}{\isacharparenleft}{\kern0pt}{\isacharparenleft}{\kern0pt}eval{\isacharunderscore}{\kern0pt}func\ X\ A\ {\isasymcirc}\isactrlsub c\ swap\ {\isacharparenleft}{\kern0pt}X\isactrlbsup A\isactrlesup {\isacharparenright}{\kern0pt}\ A{\isacharparenright}{\kern0pt}\ {\isasymamalg}\ {\isacharparenleft}{\kern0pt}x\ {\isasymcirc}\isactrlsub c\ {\isasymbeta}\isactrlbsub X\isactrlbsup A\isactrlesup \ {\isasymtimes}\isactrlsub c\ {\isacharparenleft}{\kern0pt}B\ {\isasymsetminus}\ {\isacharparenleft}{\kern0pt}A{\isacharcomma}{\kern0pt}\ m{\isacharparenright}{\kern0pt}{\isacharparenright}{\kern0pt}\isactrlesub {\isacharparenright}{\kern0pt}\ {\isasymcirc}\isactrlsub c\isanewline
\ \ \ \ \ \ \ \ \ \ dist{\isacharunderscore}{\kern0pt}prod{\isacharunderscore}{\kern0pt}coprod{\isacharunderscore}{\kern0pt}left\ {\isacharparenleft}{\kern0pt}X\isactrlbsup A\isactrlesup {\isacharparenright}{\kern0pt}\ A\ {\isacharparenleft}{\kern0pt}B\ {\isasymsetminus}\ {\isacharparenleft}{\kern0pt}A{\isacharcomma}{\kern0pt}\ m{\isacharparenright}{\kern0pt}{\isacharparenright}{\kern0pt}\ {\isasymcirc}\isactrlsub c\isanewline
\ \ \ \ \ \ \ \ \ \ swap\ {\isacharparenleft}{\kern0pt}A\ {\isasymCoprod}\ {\isacharparenleft}{\kern0pt}B\ {\isasymsetminus}\ {\isacharparenleft}{\kern0pt}A{\isacharcomma}{\kern0pt}\ m{\isacharparenright}{\kern0pt}{\isacharparenright}{\kern0pt}{\isacharparenright}{\kern0pt}\ {\isacharparenleft}{\kern0pt}X\isactrlbsup A\isactrlesup {\isacharparenright}{\kern0pt}\ {\isasymcirc}\isactrlsub c\ try{\isacharunderscore}{\kern0pt}cast\ m\ {\isasymtimes}\isactrlsub f\ id\isactrlsub c\ {\isacharparenleft}{\kern0pt}X\isactrlbsup A\isactrlesup {\isacharparenright}{\kern0pt}{\isacharparenright}{\kern0pt}\isactrlsup {\isasymsharp}\ {\isasymcirc}\isactrlsub c\ g\isanewline
\ \ \ \ \ \ \ \ {\isacharequal}{\kern0pt}\isanewline
\ \ \ \ \ \ \ \ \ \ {\isacharparenleft}{\kern0pt}{\isacharparenleft}{\kern0pt}eval{\isacharunderscore}{\kern0pt}func\ X\ A\ {\isasymcirc}\isactrlsub c\ swap\ {\isacharparenleft}{\kern0pt}X\isactrlbsup A\isactrlesup {\isacharparenright}{\kern0pt}\ A{\isacharparenright}{\kern0pt}\ {\isasymamalg}\ {\isacharparenleft}{\kern0pt}x\ {\isasymcirc}\isactrlsub c\ {\isasymbeta}\isactrlbsub X\isactrlbsup A\isactrlesup \ {\isasymtimes}\isactrlsub c\ {\isacharparenleft}{\kern0pt}B\ {\isasymsetminus}\ {\isacharparenleft}{\kern0pt}A{\isacharcomma}{\kern0pt}\ m{\isacharparenright}{\kern0pt}{\isacharparenright}{\kern0pt}\isactrlesub {\isacharparenright}{\kern0pt}\ {\isasymcirc}\isactrlsub c\isanewline
\ \ \ \ \ \ \ \ \ \ dist{\isacharunderscore}{\kern0pt}prod{\isacharunderscore}{\kern0pt}coprod{\isacharunderscore}{\kern0pt}left\ {\isacharparenleft}{\kern0pt}X\isactrlbsup A\isactrlesup {\isacharparenright}{\kern0pt}\ A\ {\isacharparenleft}{\kern0pt}B\ {\isasymsetminus}\ {\isacharparenleft}{\kern0pt}A{\isacharcomma}{\kern0pt}\ m{\isacharparenright}{\kern0pt}{\isacharparenright}{\kern0pt}\ {\isasymcirc}\isactrlsub c\isanewline
\ \ \ \ \ \ \ \ \ \ swap\ {\isacharparenleft}{\kern0pt}A\ {\isasymCoprod}\ {\isacharparenleft}{\kern0pt}B\ {\isasymsetminus}\ {\isacharparenleft}{\kern0pt}A{\isacharcomma}{\kern0pt}\ m{\isacharparenright}{\kern0pt}{\isacharparenright}{\kern0pt}{\isacharparenright}{\kern0pt}\ {\isacharparenleft}{\kern0pt}X\isactrlbsup A\isactrlesup {\isacharparenright}{\kern0pt}\ {\isasymcirc}\isactrlsub c\ try{\isacharunderscore}{\kern0pt}cast\ m\ {\isasymtimes}\isactrlsub f\ id\isactrlsub c\ {\isacharparenleft}{\kern0pt}X\isactrlbsup A\isactrlesup {\isacharparenright}{\kern0pt}{\isacharparenright}{\kern0pt}\isactrlsup {\isasymsharp}\ {\isasymcirc}\isactrlsub c\ h{\isachardoublequoteclose}\isanewline
\isanewline
\ \ \ \ \ \ \isacommand{show}\isamarkupfalse%
\ {\isachardoublequoteopen}g\ {\isacharequal}{\kern0pt}\ h{\isachardoublequoteclose}\isanewline
\ \ \ \ \ \ \isacommand{proof}\isamarkupfalse%
\ {\isacharparenleft}{\kern0pt}typecheck{\isacharunderscore}{\kern0pt}cfuncs{\isacharcomma}{\kern0pt}\ rule\ same{\isacharunderscore}{\kern0pt}evals{\isacharunderscore}{\kern0pt}equal{\isacharbrackleft}{\kern0pt}\isakeyword{where}\ Z{\isacharequal}{\kern0pt}Z{\isacharcomma}{\kern0pt}\ \isakeyword{where}\ A{\isacharequal}{\kern0pt}A{\isacharcomma}{\kern0pt}\ \isakeyword{where}\ X{\isacharequal}{\kern0pt}X{\isacharbrackright}{\kern0pt}{\isacharcomma}{\kern0pt}\ clarify{\isacharparenright}{\kern0pt}\isanewline
\ \ \ \ \ \ \ \ \isacommand{show}\isamarkupfalse%
\ {\isachardoublequoteopen}eval{\isacharunderscore}{\kern0pt}func\ X\ A\ {\isasymcirc}\isactrlsub c\ id\isactrlsub c\ A\ {\isasymtimes}\isactrlsub f\ g\ {\isacharequal}{\kern0pt}\ eval{\isacharunderscore}{\kern0pt}func\ X\ A\ {\isasymcirc}\isactrlsub c\ id\isactrlsub c\ A\ {\isasymtimes}\isactrlsub f\ h{\isachardoublequoteclose}\isanewline
\ \ \ \ \ \ \ \ \isacommand{proof}\isamarkupfalse%
\ {\isacharparenleft}{\kern0pt}typecheck{\isacharunderscore}{\kern0pt}cfuncs{\isacharcomma}{\kern0pt}\ rule\ one{\isacharunderscore}{\kern0pt}separator{\isacharbrackleft}{\kern0pt}\isakeyword{where}\ X{\isacharequal}{\kern0pt}{\isachardoublequoteopen}A\ {\isasymtimes}\isactrlsub c\ Z{\isachardoublequoteclose}{\isacharcomma}{\kern0pt}\ \isakeyword{where}\ Y{\isacharequal}{\kern0pt}{\isachardoublequoteopen}X{\isachardoublequoteclose}{\isacharbrackright}{\kern0pt}{\isacharcomma}{\kern0pt}\ clarify{\isacharparenright}{\kern0pt}\isanewline
\ \ \ \ \ \ \ \ \ \ \isacommand{fix}\isamarkupfalse%
\ az\isanewline
\ \ \ \ \ \ \ \ \ \ \isacommand{assume}\isamarkupfalse%
\ az{\isacharunderscore}{\kern0pt}type{\isacharbrackleft}{\kern0pt}type{\isacharunderscore}{\kern0pt}rule{\isacharbrackright}{\kern0pt}{\isacharcolon}{\kern0pt}\ {\isachardoublequoteopen}az\ {\isasymin}\isactrlsub c\ A\ {\isasymtimes}\isactrlsub c\ Z{\isachardoublequoteclose}\isanewline
\isanewline
\ \ \ \ \ \ \ \ \ \ \isacommand{obtain}\isamarkupfalse%
\ a\ z\ \isakeyword{where}\ az{\isacharunderscore}{\kern0pt}types{\isacharbrackleft}{\kern0pt}type{\isacharunderscore}{\kern0pt}rule{\isacharbrackright}{\kern0pt}{\isacharcolon}{\kern0pt}\ {\isachardoublequoteopen}a\ {\isasymin}\isactrlsub c\ A{\isachardoublequoteclose}\ {\isachardoublequoteopen}z\ {\isasymin}\isactrlsub c\ Z{\isachardoublequoteclose}\ \isakeyword{and}\ az{\isacharunderscore}{\kern0pt}def{\isacharcolon}{\kern0pt}\ {\isachardoublequoteopen}az\ {\isacharequal}{\kern0pt}\ {\isasymlangle}a{\isacharcomma}{\kern0pt}z{\isasymrangle}{\isachardoublequoteclose}\isanewline
\ \ \ \ \ \ \ \ \ \ \ \ \isacommand{using}\isamarkupfalse%
\ cart{\isacharunderscore}{\kern0pt}prod{\isacharunderscore}{\kern0pt}decomp\ az{\isacharunderscore}{\kern0pt}type\ \isacommand{by}\isamarkupfalse%
\ blast\isanewline
\isanewline
\ \ \ \ \ \ \ \ \ \ \isacommand{have}\isamarkupfalse%
\ {\isachardoublequoteopen}{\isacharparenleft}{\kern0pt}eval{\isacharunderscore}{\kern0pt}func\ X\ B{\isacharparenright}{\kern0pt}\ {\isasymcirc}\isactrlsub c\ {\isacharparenleft}{\kern0pt}id\ B\ {\isasymtimes}\isactrlsub f\ {\isacharparenleft}{\kern0pt}{\isacharparenleft}{\kern0pt}{\isacharparenleft}{\kern0pt}eval{\isacharunderscore}{\kern0pt}func\ X\ A\ {\isasymcirc}\isactrlsub c\ swap\ {\isacharparenleft}{\kern0pt}X\isactrlbsup A\isactrlesup {\isacharparenright}{\kern0pt}\ A{\isacharparenright}{\kern0pt}\ {\isasymamalg}\ {\isacharparenleft}{\kern0pt}x\ {\isasymcirc}\isactrlsub c\ {\isasymbeta}\isactrlbsub X\isactrlbsup A\isactrlesup \ {\isasymtimes}\isactrlsub c\ {\isacharparenleft}{\kern0pt}B\ {\isasymsetminus}\ {\isacharparenleft}{\kern0pt}A{\isacharcomma}{\kern0pt}\ m{\isacharparenright}{\kern0pt}{\isacharparenright}{\kern0pt}\isactrlesub {\isacharparenright}{\kern0pt}\ {\isasymcirc}\isactrlsub c\isanewline
\ \ \ \ \ \ \ \ \ \ \ \ dist{\isacharunderscore}{\kern0pt}prod{\isacharunderscore}{\kern0pt}coprod{\isacharunderscore}{\kern0pt}left\ {\isacharparenleft}{\kern0pt}X\isactrlbsup A\isactrlesup {\isacharparenright}{\kern0pt}\ A\ {\isacharparenleft}{\kern0pt}B\ {\isasymsetminus}\ {\isacharparenleft}{\kern0pt}A{\isacharcomma}{\kern0pt}\ m{\isacharparenright}{\kern0pt}{\isacharparenright}{\kern0pt}\ {\isasymcirc}\isactrlsub c\isanewline
\ \ \ \ \ \ \ \ \ \ \ \ swap\ {\isacharparenleft}{\kern0pt}A\ {\isasymCoprod}\ {\isacharparenleft}{\kern0pt}B\ {\isasymsetminus}\ {\isacharparenleft}{\kern0pt}A{\isacharcomma}{\kern0pt}\ m{\isacharparenright}{\kern0pt}{\isacharparenright}{\kern0pt}{\isacharparenright}{\kern0pt}\ {\isacharparenleft}{\kern0pt}X\isactrlbsup A\isactrlesup {\isacharparenright}{\kern0pt}\ {\isasymcirc}\isactrlsub c\ try{\isacharunderscore}{\kern0pt}cast\ m\ {\isasymtimes}\isactrlsub f\ id\isactrlsub c\ {\isacharparenleft}{\kern0pt}X\isactrlbsup A\isactrlesup {\isacharparenright}{\kern0pt}{\isacharparenright}{\kern0pt}\isactrlsup {\isasymsharp}\ {\isasymcirc}\isactrlsub c\ g{\isacharparenright}{\kern0pt}{\isacharparenright}{\kern0pt}\ {\isacharequal}{\kern0pt}\ \isanewline
\ \ \ \ \ \ \ \ \ \ {\isacharparenleft}{\kern0pt}eval{\isacharunderscore}{\kern0pt}func\ X\ B{\isacharparenright}{\kern0pt}\ {\isasymcirc}\isactrlsub c\ {\isacharparenleft}{\kern0pt}id\ B\ {\isasymtimes}\isactrlsub f\ {\isacharparenleft}{\kern0pt}{\isacharparenleft}{\kern0pt}{\isacharparenleft}{\kern0pt}eval{\isacharunderscore}{\kern0pt}func\ X\ A\ {\isasymcirc}\isactrlsub c\ swap\ {\isacharparenleft}{\kern0pt}X\isactrlbsup A\isactrlesup {\isacharparenright}{\kern0pt}\ A{\isacharparenright}{\kern0pt}\ {\isasymamalg}\ {\isacharparenleft}{\kern0pt}x\ {\isasymcirc}\isactrlsub c\ {\isasymbeta}\isactrlbsub X\isactrlbsup A\isactrlesup \ {\isasymtimes}\isactrlsub c\ {\isacharparenleft}{\kern0pt}B\ {\isasymsetminus}\ {\isacharparenleft}{\kern0pt}A{\isacharcomma}{\kern0pt}\ m{\isacharparenright}{\kern0pt}{\isacharparenright}{\kern0pt}\isactrlesub {\isacharparenright}{\kern0pt}\ {\isasymcirc}\isactrlsub c\isanewline
\ \ \ \ \ \ \ \ \ \ \ \ dist{\isacharunderscore}{\kern0pt}prod{\isacharunderscore}{\kern0pt}coprod{\isacharunderscore}{\kern0pt}left\ {\isacharparenleft}{\kern0pt}X\isactrlbsup A\isactrlesup {\isacharparenright}{\kern0pt}\ A\ {\isacharparenleft}{\kern0pt}B\ {\isasymsetminus}\ {\isacharparenleft}{\kern0pt}A{\isacharcomma}{\kern0pt}\ m{\isacharparenright}{\kern0pt}{\isacharparenright}{\kern0pt}\ {\isasymcirc}\isactrlsub c\isanewline
\ \ \ \ \ \ \ \ \ \ \ \ swap\ {\isacharparenleft}{\kern0pt}A\ {\isasymCoprod}\ {\isacharparenleft}{\kern0pt}B\ {\isasymsetminus}\ {\isacharparenleft}{\kern0pt}A{\isacharcomma}{\kern0pt}\ m{\isacharparenright}{\kern0pt}{\isacharparenright}{\kern0pt}{\isacharparenright}{\kern0pt}\ {\isacharparenleft}{\kern0pt}X\isactrlbsup A\isactrlesup {\isacharparenright}{\kern0pt}\ {\isasymcirc}\isactrlsub c\ try{\isacharunderscore}{\kern0pt}cast\ m\ {\isasymtimes}\isactrlsub f\ id\isactrlsub c\ {\isacharparenleft}{\kern0pt}X\isactrlbsup A\isactrlesup {\isacharparenright}{\kern0pt}{\isacharparenright}{\kern0pt}\isactrlsup {\isasymsharp}\ {\isasymcirc}\isactrlsub c\ h{\isacharparenright}{\kern0pt}{\isacharparenright}{\kern0pt}{\isachardoublequoteclose}\isanewline
\ \ \ \ \ \ \ \ \ \ \ \ \isacommand{using}\isamarkupfalse%
\ eq\ \isacommand{by}\isamarkupfalse%
\ simp\isanewline
\ \ \ \ \ \ \ \ \ \ \isacommand{then}\isamarkupfalse%
\ \isacommand{have}\isamarkupfalse%
\ {\isachardoublequoteopen}{\isacharparenleft}{\kern0pt}eval{\isacharunderscore}{\kern0pt}func\ X\ B{\isacharparenright}{\kern0pt}{\isasymcirc}\isactrlsub c\ {\isacharparenleft}{\kern0pt}id\ B\ {\isasymtimes}\isactrlsub f\ {\isacharparenleft}{\kern0pt}{\isacharparenleft}{\kern0pt}{\isacharparenleft}{\kern0pt}eval{\isacharunderscore}{\kern0pt}func\ X\ A\ {\isasymcirc}\isactrlsub c\ swap\ {\isacharparenleft}{\kern0pt}X\isactrlbsup A\isactrlesup {\isacharparenright}{\kern0pt}\ A{\isacharparenright}{\kern0pt}\ {\isasymamalg}\ {\isacharparenleft}{\kern0pt}x\ {\isasymcirc}\isactrlsub c\ {\isasymbeta}\isactrlbsub X\isactrlbsup A\isactrlesup \ {\isasymtimes}\isactrlsub c\ {\isacharparenleft}{\kern0pt}B\ {\isasymsetminus}\ {\isacharparenleft}{\kern0pt}A{\isacharcomma}{\kern0pt}\ m{\isacharparenright}{\kern0pt}{\isacharparenright}{\kern0pt}\isactrlesub {\isacharparenright}{\kern0pt}\ {\isasymcirc}\isactrlsub c\isanewline
\ \ \ \ \ \ \ \ \ \ \ \ dist{\isacharunderscore}{\kern0pt}prod{\isacharunderscore}{\kern0pt}coprod{\isacharunderscore}{\kern0pt}left\ {\isacharparenleft}{\kern0pt}X\isactrlbsup A\isactrlesup {\isacharparenright}{\kern0pt}\ A\ {\isacharparenleft}{\kern0pt}B\ {\isasymsetminus}\ {\isacharparenleft}{\kern0pt}A{\isacharcomma}{\kern0pt}\ m{\isacharparenright}{\kern0pt}{\isacharparenright}{\kern0pt}\ {\isasymcirc}\isactrlsub c\isanewline
\ \ \ \ \ \ \ \ \ \ \ \ swap\ {\isacharparenleft}{\kern0pt}A\ {\isasymCoprod}\ {\isacharparenleft}{\kern0pt}B\ {\isasymsetminus}\ {\isacharparenleft}{\kern0pt}A{\isacharcomma}{\kern0pt}\ m{\isacharparenright}{\kern0pt}{\isacharparenright}{\kern0pt}{\isacharparenright}{\kern0pt}\ {\isacharparenleft}{\kern0pt}X\isactrlbsup A\isactrlesup {\isacharparenright}{\kern0pt}\ {\isasymcirc}\isactrlsub c\ try{\isacharunderscore}{\kern0pt}cast\ m\ {\isasymtimes}\isactrlsub f\ id\isactrlsub c\ {\isacharparenleft}{\kern0pt}X\isactrlbsup A\isactrlesup {\isacharparenright}{\kern0pt}{\isacharparenright}{\kern0pt}\isactrlsup {\isasymsharp}{\isacharparenright}{\kern0pt}{\isacharparenright}{\kern0pt}\ {\isasymcirc}\isactrlsub c\ {\isacharparenleft}{\kern0pt}id\ B\ {\isasymtimes}\isactrlsub f\ \ g{\isacharparenright}{\kern0pt}\ {\isacharequal}{\kern0pt}\ \isanewline
\ \ \ \ \ \ \ \ \ \ {\isacharparenleft}{\kern0pt}eval{\isacharunderscore}{\kern0pt}func\ X\ B{\isacharparenright}{\kern0pt}{\isasymcirc}\isactrlsub c\ {\isacharparenleft}{\kern0pt}id\ B\ {\isasymtimes}\isactrlsub f\ {\isacharparenleft}{\kern0pt}{\isacharparenleft}{\kern0pt}{\isacharparenleft}{\kern0pt}eval{\isacharunderscore}{\kern0pt}func\ X\ A\ {\isasymcirc}\isactrlsub c\ swap\ {\isacharparenleft}{\kern0pt}X\isactrlbsup A\isactrlesup {\isacharparenright}{\kern0pt}\ A{\isacharparenright}{\kern0pt}\ {\isasymamalg}\ {\isacharparenleft}{\kern0pt}x\ {\isasymcirc}\isactrlsub c\ {\isasymbeta}\isactrlbsub X\isactrlbsup A\isactrlesup \ {\isasymtimes}\isactrlsub c\ {\isacharparenleft}{\kern0pt}B\ {\isasymsetminus}\ {\isacharparenleft}{\kern0pt}A{\isacharcomma}{\kern0pt}\ m{\isacharparenright}{\kern0pt}{\isacharparenright}{\kern0pt}\isactrlesub {\isacharparenright}{\kern0pt}\ {\isasymcirc}\isactrlsub c\isanewline
\ \ \ \ \ \ \ \ \ \ \ \ dist{\isacharunderscore}{\kern0pt}prod{\isacharunderscore}{\kern0pt}coprod{\isacharunderscore}{\kern0pt}left\ {\isacharparenleft}{\kern0pt}X\isactrlbsup A\isactrlesup {\isacharparenright}{\kern0pt}\ A\ {\isacharparenleft}{\kern0pt}B\ {\isasymsetminus}\ {\isacharparenleft}{\kern0pt}A{\isacharcomma}{\kern0pt}\ m{\isacharparenright}{\kern0pt}{\isacharparenright}{\kern0pt}\ {\isasymcirc}\isactrlsub c\isanewline
\ \ \ \ \ \ \ \ \ \ \ \ swap\ {\isacharparenleft}{\kern0pt}A\ {\isasymCoprod}\ {\isacharparenleft}{\kern0pt}B\ {\isasymsetminus}\ {\isacharparenleft}{\kern0pt}A{\isacharcomma}{\kern0pt}\ m{\isacharparenright}{\kern0pt}{\isacharparenright}{\kern0pt}{\isacharparenright}{\kern0pt}\ {\isacharparenleft}{\kern0pt}X\isactrlbsup A\isactrlesup {\isacharparenright}{\kern0pt}\ {\isasymcirc}\isactrlsub c\ try{\isacharunderscore}{\kern0pt}cast\ m\ {\isasymtimes}\isactrlsub f\ id\isactrlsub c\ {\isacharparenleft}{\kern0pt}X\isactrlbsup A\isactrlesup {\isacharparenright}{\kern0pt}{\isacharparenright}{\kern0pt}\isactrlsup {\isasymsharp}{\isacharparenright}{\kern0pt}{\isacharparenright}{\kern0pt}\ {\isasymcirc}\isactrlsub c\ {\isacharparenleft}{\kern0pt}id\ B\ {\isasymtimes}\isactrlsub f\ \ h{\isacharparenright}{\kern0pt}{\isachardoublequoteclose}\isanewline
\ \ \ \ \ \ \ \ \ \ \ \ \isacommand{using}\isamarkupfalse%
\ identity{\isacharunderscore}{\kern0pt}distributes{\isacharunderscore}{\kern0pt}across{\isacharunderscore}{\kern0pt}composition\ \isacommand{by}\isamarkupfalse%
\ {\isacharparenleft}{\kern0pt}typecheck{\isacharunderscore}{\kern0pt}cfuncs{\isacharcomma}{\kern0pt}\ auto{\isacharparenright}{\kern0pt}\isanewline
\ \ \ \ \ \ \ \ \ \ \isacommand{then}\isamarkupfalse%
\ \isacommand{have}\isamarkupfalse%
\ {\isachardoublequoteopen}{\isacharparenleft}{\kern0pt}{\isacharparenleft}{\kern0pt}eval{\isacharunderscore}{\kern0pt}func\ X\ B{\isacharparenright}{\kern0pt}{\isasymcirc}\isactrlsub c\ {\isacharparenleft}{\kern0pt}id\ B\ {\isasymtimes}\isactrlsub f\ {\isacharparenleft}{\kern0pt}{\isacharparenleft}{\kern0pt}{\isacharparenleft}{\kern0pt}eval{\isacharunderscore}{\kern0pt}func\ X\ A\ {\isasymcirc}\isactrlsub c\ swap\ {\isacharparenleft}{\kern0pt}X\isactrlbsup A\isactrlesup {\isacharparenright}{\kern0pt}\ A{\isacharparenright}{\kern0pt}\ {\isasymamalg}\ {\isacharparenleft}{\kern0pt}x\ {\isasymcirc}\isactrlsub c\ {\isasymbeta}\isactrlbsub X\isactrlbsup A\isactrlesup \ {\isasymtimes}\isactrlsub c\ {\isacharparenleft}{\kern0pt}B\ {\isasymsetminus}\ {\isacharparenleft}{\kern0pt}A{\isacharcomma}{\kern0pt}\ m{\isacharparenright}{\kern0pt}{\isacharparenright}{\kern0pt}\isactrlesub {\isacharparenright}{\kern0pt}\ {\isasymcirc}\isactrlsub c\isanewline
\ \ \ \ \ \ \ \ \ \ \ \ dist{\isacharunderscore}{\kern0pt}prod{\isacharunderscore}{\kern0pt}coprod{\isacharunderscore}{\kern0pt}left\ {\isacharparenleft}{\kern0pt}X\isactrlbsup A\isactrlesup {\isacharparenright}{\kern0pt}\ A\ {\isacharparenleft}{\kern0pt}B\ {\isasymsetminus}\ {\isacharparenleft}{\kern0pt}A{\isacharcomma}{\kern0pt}\ m{\isacharparenright}{\kern0pt}{\isacharparenright}{\kern0pt}\ {\isasymcirc}\isactrlsub c\isanewline
\ \ \ \ \ \ \ \ \ \ \ \ swap\ {\isacharparenleft}{\kern0pt}A\ {\isasymCoprod}\ {\isacharparenleft}{\kern0pt}B\ {\isasymsetminus}\ {\isacharparenleft}{\kern0pt}A{\isacharcomma}{\kern0pt}\ m{\isacharparenright}{\kern0pt}{\isacharparenright}{\kern0pt}{\isacharparenright}{\kern0pt}\ {\isacharparenleft}{\kern0pt}X\isactrlbsup A\isactrlesup {\isacharparenright}{\kern0pt}\ {\isasymcirc}\isactrlsub c\ try{\isacharunderscore}{\kern0pt}cast\ m\ {\isasymtimes}\isactrlsub f\ id\isactrlsub c\ {\isacharparenleft}{\kern0pt}X\isactrlbsup A\isactrlesup {\isacharparenright}{\kern0pt}{\isacharparenright}{\kern0pt}\isactrlsup {\isasymsharp}{\isacharparenright}{\kern0pt}{\isacharparenright}{\kern0pt}{\isacharparenright}{\kern0pt}\ {\isasymcirc}\isactrlsub c\ {\isacharparenleft}{\kern0pt}id\ B\ {\isasymtimes}\isactrlsub f\ \ g{\isacharparenright}{\kern0pt}\ {\isacharequal}{\kern0pt}\ \isanewline
\ \ \ \ \ \ \ \ \ \ {\isacharparenleft}{\kern0pt}{\isacharparenleft}{\kern0pt}eval{\isacharunderscore}{\kern0pt}func\ X\ B{\isacharparenright}{\kern0pt}{\isasymcirc}\isactrlsub c\ {\isacharparenleft}{\kern0pt}id\ B\ {\isasymtimes}\isactrlsub f\ {\isacharparenleft}{\kern0pt}{\isacharparenleft}{\kern0pt}{\isacharparenleft}{\kern0pt}eval{\isacharunderscore}{\kern0pt}func\ X\ A\ {\isasymcirc}\isactrlsub c\ swap\ {\isacharparenleft}{\kern0pt}X\isactrlbsup A\isactrlesup {\isacharparenright}{\kern0pt}\ A{\isacharparenright}{\kern0pt}\ {\isasymamalg}\ {\isacharparenleft}{\kern0pt}x\ {\isasymcirc}\isactrlsub c\ {\isasymbeta}\isactrlbsub X\isactrlbsup A\isactrlesup \ {\isasymtimes}\isactrlsub c\ {\isacharparenleft}{\kern0pt}B\ {\isasymsetminus}\ {\isacharparenleft}{\kern0pt}A{\isacharcomma}{\kern0pt}\ m{\isacharparenright}{\kern0pt}{\isacharparenright}{\kern0pt}\isactrlesub {\isacharparenright}{\kern0pt}\ {\isasymcirc}\isactrlsub c\isanewline
\ \ \ \ \ \ \ \ \ \ \ \ dist{\isacharunderscore}{\kern0pt}prod{\isacharunderscore}{\kern0pt}coprod{\isacharunderscore}{\kern0pt}left\ {\isacharparenleft}{\kern0pt}X\isactrlbsup A\isactrlesup {\isacharparenright}{\kern0pt}\ A\ {\isacharparenleft}{\kern0pt}B\ {\isasymsetminus}\ {\isacharparenleft}{\kern0pt}A{\isacharcomma}{\kern0pt}\ m{\isacharparenright}{\kern0pt}{\isacharparenright}{\kern0pt}\ {\isasymcirc}\isactrlsub c\isanewline
\ \ \ \ \ \ \ \ \ \ \ \ swap\ {\isacharparenleft}{\kern0pt}A\ {\isasymCoprod}\ {\isacharparenleft}{\kern0pt}B\ {\isasymsetminus}\ {\isacharparenleft}{\kern0pt}A{\isacharcomma}{\kern0pt}\ m{\isacharparenright}{\kern0pt}{\isacharparenright}{\kern0pt}{\isacharparenright}{\kern0pt}\ {\isacharparenleft}{\kern0pt}X\isactrlbsup A\isactrlesup {\isacharparenright}{\kern0pt}\ {\isasymcirc}\isactrlsub c\ try{\isacharunderscore}{\kern0pt}cast\ m\ {\isasymtimes}\isactrlsub f\ id\isactrlsub c\ {\isacharparenleft}{\kern0pt}X\isactrlbsup A\isactrlesup {\isacharparenright}{\kern0pt}{\isacharparenright}{\kern0pt}\isactrlsup {\isasymsharp}{\isacharparenright}{\kern0pt}{\isacharparenright}{\kern0pt}{\isacharparenright}{\kern0pt}\ {\isasymcirc}\isactrlsub c\ {\isacharparenleft}{\kern0pt}id\ B\ {\isasymtimes}\isactrlsub f\ \ h{\isacharparenright}{\kern0pt}{\isachardoublequoteclose}\isanewline
\ \ \ \ \ \ \ \ \ \ \ \ \isacommand{by}\isamarkupfalse%
\ {\isacharparenleft}{\kern0pt}typecheck{\isacharunderscore}{\kern0pt}cfuncs{\isacharcomma}{\kern0pt}\ smt\ eq\ inv{\isacharunderscore}{\kern0pt}transpose{\isacharunderscore}{\kern0pt}func{\isacharunderscore}{\kern0pt}def{\isadigit{3}}\ inv{\isacharunderscore}{\kern0pt}transpose{\isacharunderscore}{\kern0pt}of{\isacharunderscore}{\kern0pt}composition{\isacharparenright}{\kern0pt}\isanewline
\ \ \ \ \ \ \ \ \ \ \isacommand{then}\isamarkupfalse%
\ \isacommand{have}\isamarkupfalse%
\ {\isachardoublequoteopen}{\isacharparenleft}{\kern0pt}{\isacharparenleft}{\kern0pt}eval{\isacharunderscore}{\kern0pt}func\ X\ A\ {\isasymcirc}\isactrlsub c\ swap\ {\isacharparenleft}{\kern0pt}X\isactrlbsup A\isactrlesup {\isacharparenright}{\kern0pt}\ A{\isacharparenright}{\kern0pt}\ {\isasymamalg}\ {\isacharparenleft}{\kern0pt}x\ {\isasymcirc}\isactrlsub c\ {\isasymbeta}\isactrlbsub X\isactrlbsup A\isactrlesup \ {\isasymtimes}\isactrlsub c\ {\isacharparenleft}{\kern0pt}B\ {\isasymsetminus}\ {\isacharparenleft}{\kern0pt}A{\isacharcomma}{\kern0pt}\ m{\isacharparenright}{\kern0pt}{\isacharparenright}{\kern0pt}\isactrlesub {\isacharparenright}{\kern0pt}\ {\isasymcirc}\isactrlsub c\isanewline
\ \ \ \ \ \ \ \ \ \ \ \ dist{\isacharunderscore}{\kern0pt}prod{\isacharunderscore}{\kern0pt}coprod{\isacharunderscore}{\kern0pt}left\ {\isacharparenleft}{\kern0pt}X\isactrlbsup A\isactrlesup {\isacharparenright}{\kern0pt}\ A\ {\isacharparenleft}{\kern0pt}B\ {\isasymsetminus}\ {\isacharparenleft}{\kern0pt}A{\isacharcomma}{\kern0pt}\ m{\isacharparenright}{\kern0pt}{\isacharparenright}{\kern0pt}\ {\isasymcirc}\isactrlsub c\isanewline
\ \ \ \ \ \ \ \ \ \ \ \ swap\ {\isacharparenleft}{\kern0pt}A\ {\isasymCoprod}\ {\isacharparenleft}{\kern0pt}B\ {\isasymsetminus}\ {\isacharparenleft}{\kern0pt}A{\isacharcomma}{\kern0pt}\ m{\isacharparenright}{\kern0pt}{\isacharparenright}{\kern0pt}{\isacharparenright}{\kern0pt}\ {\isacharparenleft}{\kern0pt}X\isactrlbsup A\isactrlesup {\isacharparenright}{\kern0pt}\ {\isasymcirc}\isactrlsub c\ try{\isacharunderscore}{\kern0pt}cast\ m\ {\isasymtimes}\isactrlsub f\ id\isactrlsub c\ {\isacharparenleft}{\kern0pt}X\isactrlbsup A\isactrlesup {\isacharparenright}{\kern0pt}{\isacharparenright}{\kern0pt}\ {\isasymcirc}\isactrlsub c\ {\isacharparenleft}{\kern0pt}id\ B\ {\isasymtimes}\isactrlsub f\ \ g{\isacharparenright}{\kern0pt}\isanewline
\ \ \ \ \ \ \ \ \ \ {\isacharequal}{\kern0pt}\ {\isacharparenleft}{\kern0pt}{\isacharparenleft}{\kern0pt}eval{\isacharunderscore}{\kern0pt}func\ X\ A\ {\isasymcirc}\isactrlsub c\ swap\ {\isacharparenleft}{\kern0pt}X\isactrlbsup A\isactrlesup {\isacharparenright}{\kern0pt}\ A{\isacharparenright}{\kern0pt}\ {\isasymamalg}\ {\isacharparenleft}{\kern0pt}x\ {\isasymcirc}\isactrlsub c\ {\isasymbeta}\isactrlbsub X\isactrlbsup A\isactrlesup \ {\isasymtimes}\isactrlsub c\ {\isacharparenleft}{\kern0pt}B\ {\isasymsetminus}\ {\isacharparenleft}{\kern0pt}A{\isacharcomma}{\kern0pt}\ m{\isacharparenright}{\kern0pt}{\isacharparenright}{\kern0pt}\isactrlesub {\isacharparenright}{\kern0pt}\ {\isasymcirc}\isactrlsub c\isanewline
\ \ \ \ \ \ \ \ \ \ \ \ dist{\isacharunderscore}{\kern0pt}prod{\isacharunderscore}{\kern0pt}coprod{\isacharunderscore}{\kern0pt}left\ {\isacharparenleft}{\kern0pt}X\isactrlbsup A\isactrlesup {\isacharparenright}{\kern0pt}\ A\ {\isacharparenleft}{\kern0pt}B\ {\isasymsetminus}\ {\isacharparenleft}{\kern0pt}A{\isacharcomma}{\kern0pt}\ m{\isacharparenright}{\kern0pt}{\isacharparenright}{\kern0pt}\ {\isasymcirc}\isactrlsub c\isanewline
\ \ \ \ \ \ \ \ \ \ \ \ swap\ {\isacharparenleft}{\kern0pt}A\ {\isasymCoprod}\ {\isacharparenleft}{\kern0pt}B\ {\isasymsetminus}\ {\isacharparenleft}{\kern0pt}A{\isacharcomma}{\kern0pt}\ m{\isacharparenright}{\kern0pt}{\isacharparenright}{\kern0pt}{\isacharparenright}{\kern0pt}\ {\isacharparenleft}{\kern0pt}X\isactrlbsup A\isactrlesup {\isacharparenright}{\kern0pt}\ {\isasymcirc}\isactrlsub c\ try{\isacharunderscore}{\kern0pt}cast\ m\ {\isasymtimes}\isactrlsub f\ id\isactrlsub c\ {\isacharparenleft}{\kern0pt}X\isactrlbsup A\isactrlesup {\isacharparenright}{\kern0pt}{\isacharparenright}{\kern0pt}\ {\isasymcirc}\isactrlsub c\ {\isacharparenleft}{\kern0pt}id\ B\ {\isasymtimes}\isactrlsub f\ \ h{\isacharparenright}{\kern0pt}{\isachardoublequoteclose}\isanewline
\ \ \ \ \ \ \ \ \ \ \ \ \isacommand{using}\isamarkupfalse%
\ \ \ transpose{\isacharunderscore}{\kern0pt}func{\isacharunderscore}{\kern0pt}def\ \isacommand{by}\isamarkupfalse%
\ {\isacharparenleft}{\kern0pt}typecheck{\isacharunderscore}{\kern0pt}cfuncs{\isacharcomma}{\kern0pt}auto{\isacharparenright}{\kern0pt}\isanewline
\ \ \ \ \ \ \ \ \ \ \isacommand{then}\isamarkupfalse%
\ \isacommand{have}\isamarkupfalse%
\ {\isachardoublequoteopen}{\isacharparenleft}{\kern0pt}{\isacharparenleft}{\kern0pt}{\isacharparenleft}{\kern0pt}eval{\isacharunderscore}{\kern0pt}func\ X\ A\ {\isasymcirc}\isactrlsub c\ swap\ {\isacharparenleft}{\kern0pt}X\isactrlbsup A\isactrlesup {\isacharparenright}{\kern0pt}\ A{\isacharparenright}{\kern0pt}\ {\isasymamalg}\ {\isacharparenleft}{\kern0pt}x\ {\isasymcirc}\isactrlsub c\ {\isasymbeta}\isactrlbsub X\isactrlbsup A\isactrlesup \ {\isasymtimes}\isactrlsub c\ {\isacharparenleft}{\kern0pt}B\ {\isasymsetminus}\ {\isacharparenleft}{\kern0pt}A{\isacharcomma}{\kern0pt}\ m{\isacharparenright}{\kern0pt}{\isacharparenright}{\kern0pt}\isactrlesub {\isacharparenright}{\kern0pt}\ {\isasymcirc}\isactrlsub c\isanewline
\ \ \ \ \ \ \ \ \ \ \ \ dist{\isacharunderscore}{\kern0pt}prod{\isacharunderscore}{\kern0pt}coprod{\isacharunderscore}{\kern0pt}left\ {\isacharparenleft}{\kern0pt}X\isactrlbsup A\isactrlesup {\isacharparenright}{\kern0pt}\ A\ {\isacharparenleft}{\kern0pt}B\ {\isasymsetminus}\ {\isacharparenleft}{\kern0pt}A{\isacharcomma}{\kern0pt}\ m{\isacharparenright}{\kern0pt}{\isacharparenright}{\kern0pt}\ {\isasymcirc}\isactrlsub c\isanewline
\ \ \ \ \ \ \ \ \ \ \ \ swap\ {\isacharparenleft}{\kern0pt}A\ {\isasymCoprod}\ {\isacharparenleft}{\kern0pt}B\ {\isasymsetminus}\ {\isacharparenleft}{\kern0pt}A{\isacharcomma}{\kern0pt}\ m{\isacharparenright}{\kern0pt}{\isacharparenright}{\kern0pt}{\isacharparenright}{\kern0pt}\ {\isacharparenleft}{\kern0pt}X\isactrlbsup A\isactrlesup {\isacharparenright}{\kern0pt}\ {\isasymcirc}\isactrlsub c\ try{\isacharunderscore}{\kern0pt}cast\ m\ {\isasymtimes}\isactrlsub f\ id\isactrlsub c\ {\isacharparenleft}{\kern0pt}X\isactrlbsup A\isactrlesup {\isacharparenright}{\kern0pt}{\isacharparenright}{\kern0pt}\ {\isasymcirc}\isactrlsub c\ {\isacharparenleft}{\kern0pt}id\ B\ {\isasymtimes}\isactrlsub f\ \ g{\isacharparenright}{\kern0pt}{\isacharparenright}{\kern0pt}\ {\isasymcirc}\isactrlsub c\ {\isasymlangle}m\ {\isasymcirc}\isactrlsub c\ a{\isacharcomma}{\kern0pt}\ z{\isasymrangle}\isanewline
\ \ \ \ \ \ \ \ \ \ {\isacharequal}{\kern0pt}\ {\isacharparenleft}{\kern0pt}{\isacharparenleft}{\kern0pt}{\isacharparenleft}{\kern0pt}eval{\isacharunderscore}{\kern0pt}func\ X\ A\ {\isasymcirc}\isactrlsub c\ swap\ {\isacharparenleft}{\kern0pt}X\isactrlbsup A\isactrlesup {\isacharparenright}{\kern0pt}\ A{\isacharparenright}{\kern0pt}\ {\isasymamalg}\ {\isacharparenleft}{\kern0pt}x\ {\isasymcirc}\isactrlsub c\ {\isasymbeta}\isactrlbsub X\isactrlbsup A\isactrlesup \ {\isasymtimes}\isactrlsub c\ {\isacharparenleft}{\kern0pt}B\ {\isasymsetminus}\ {\isacharparenleft}{\kern0pt}A{\isacharcomma}{\kern0pt}\ m{\isacharparenright}{\kern0pt}{\isacharparenright}{\kern0pt}\isactrlesub {\isacharparenright}{\kern0pt}\ {\isasymcirc}\isactrlsub c\isanewline
\ \ \ \ \ \ \ \ \ \ \ \ dist{\isacharunderscore}{\kern0pt}prod{\isacharunderscore}{\kern0pt}coprod{\isacharunderscore}{\kern0pt}left\ {\isacharparenleft}{\kern0pt}X\isactrlbsup A\isactrlesup {\isacharparenright}{\kern0pt}\ A\ {\isacharparenleft}{\kern0pt}B\ {\isasymsetminus}\ {\isacharparenleft}{\kern0pt}A{\isacharcomma}{\kern0pt}\ m{\isacharparenright}{\kern0pt}{\isacharparenright}{\kern0pt}\ {\isasymcirc}\isactrlsub c\isanewline
\ \ \ \ \ \ \ \ \ \ \ \ swap\ {\isacharparenleft}{\kern0pt}A\ {\isasymCoprod}\ {\isacharparenleft}{\kern0pt}B\ {\isasymsetminus}\ {\isacharparenleft}{\kern0pt}A{\isacharcomma}{\kern0pt}\ m{\isacharparenright}{\kern0pt}{\isacharparenright}{\kern0pt}{\isacharparenright}{\kern0pt}\ {\isacharparenleft}{\kern0pt}X\isactrlbsup A\isactrlesup {\isacharparenright}{\kern0pt}\ {\isasymcirc}\isactrlsub c\ try{\isacharunderscore}{\kern0pt}cast\ m\ {\isasymtimes}\isactrlsub f\ id\isactrlsub c\ {\isacharparenleft}{\kern0pt}X\isactrlbsup A\isactrlesup {\isacharparenright}{\kern0pt}{\isacharparenright}{\kern0pt}\ {\isasymcirc}\isactrlsub c\ {\isacharparenleft}{\kern0pt}id\ B\ {\isasymtimes}\isactrlsub f\ \ h{\isacharparenright}{\kern0pt}{\isacharparenright}{\kern0pt}\ {\isasymcirc}\isactrlsub c\ {\isasymlangle}m\ {\isasymcirc}\isactrlsub c\ a{\isacharcomma}{\kern0pt}\ z{\isasymrangle}{\isachardoublequoteclose}\isanewline
\ \ \ \ \ \ \ \ \ \ \ \ \isacommand{by}\isamarkupfalse%
\ auto\isanewline
\ \ \ \ \ \ \ \ \ \ \isacommand{then}\isamarkupfalse%
\ \isacommand{have}\isamarkupfalse%
\ {\isachardoublequoteopen}{\isacharparenleft}{\kern0pt}{\isacharparenleft}{\kern0pt}eval{\isacharunderscore}{\kern0pt}func\ X\ A\ {\isasymcirc}\isactrlsub c\ swap\ {\isacharparenleft}{\kern0pt}X\isactrlbsup A\isactrlesup {\isacharparenright}{\kern0pt}\ A{\isacharparenright}{\kern0pt}\ {\isasymamalg}\ {\isacharparenleft}{\kern0pt}x\ {\isasymcirc}\isactrlsub c\ {\isasymbeta}\isactrlbsub X\isactrlbsup A\isactrlesup \ {\isasymtimes}\isactrlsub c\ {\isacharparenleft}{\kern0pt}B\ {\isasymsetminus}\ {\isacharparenleft}{\kern0pt}A{\isacharcomma}{\kern0pt}\ m{\isacharparenright}{\kern0pt}{\isacharparenright}{\kern0pt}\isactrlesub {\isacharparenright}{\kern0pt}\ {\isasymcirc}\isactrlsub c\isanewline
\ \ \ \ \ \ \ \ \ \ \ \ dist{\isacharunderscore}{\kern0pt}prod{\isacharunderscore}{\kern0pt}coprod{\isacharunderscore}{\kern0pt}left\ {\isacharparenleft}{\kern0pt}X\isactrlbsup A\isactrlesup {\isacharparenright}{\kern0pt}\ A\ {\isacharparenleft}{\kern0pt}B\ {\isasymsetminus}\ {\isacharparenleft}{\kern0pt}A{\isacharcomma}{\kern0pt}\ m{\isacharparenright}{\kern0pt}{\isacharparenright}{\kern0pt}\ {\isasymcirc}\isactrlsub c\isanewline
\ \ \ \ \ \ \ \ \ \ \ \ swap\ {\isacharparenleft}{\kern0pt}A\ {\isasymCoprod}\ {\isacharparenleft}{\kern0pt}B\ {\isasymsetminus}\ {\isacharparenleft}{\kern0pt}A{\isacharcomma}{\kern0pt}\ m{\isacharparenright}{\kern0pt}{\isacharparenright}{\kern0pt}{\isacharparenright}{\kern0pt}\ {\isacharparenleft}{\kern0pt}X\isactrlbsup A\isactrlesup {\isacharparenright}{\kern0pt}\ {\isasymcirc}\isactrlsub c\ try{\isacharunderscore}{\kern0pt}cast\ m\ {\isasymtimes}\isactrlsub f\ id\isactrlsub c\ {\isacharparenleft}{\kern0pt}X\isactrlbsup A\isactrlesup {\isacharparenright}{\kern0pt}{\isacharparenright}{\kern0pt}\ {\isasymcirc}\isactrlsub c\ {\isacharparenleft}{\kern0pt}id\ B\ {\isasymtimes}\isactrlsub f\ \ g{\isacharparenright}{\kern0pt}\ {\isasymcirc}\isactrlsub c\ {\isasymlangle}m\ {\isasymcirc}\isactrlsub c\ a{\isacharcomma}{\kern0pt}\ z{\isasymrangle}\isanewline
\ \ \ \ \ \ \ \ \ \ {\isacharequal}{\kern0pt}\ {\isacharparenleft}{\kern0pt}{\isacharparenleft}{\kern0pt}eval{\isacharunderscore}{\kern0pt}func\ X\ A\ {\isasymcirc}\isactrlsub c\ swap\ {\isacharparenleft}{\kern0pt}X\isactrlbsup A\isactrlesup {\isacharparenright}{\kern0pt}\ A{\isacharparenright}{\kern0pt}\ {\isasymamalg}\ {\isacharparenleft}{\kern0pt}x\ {\isasymcirc}\isactrlsub c\ {\isasymbeta}\isactrlbsub X\isactrlbsup A\isactrlesup \ {\isasymtimes}\isactrlsub c\ {\isacharparenleft}{\kern0pt}B\ {\isasymsetminus}\ {\isacharparenleft}{\kern0pt}A{\isacharcomma}{\kern0pt}\ m{\isacharparenright}{\kern0pt}{\isacharparenright}{\kern0pt}\isactrlesub {\isacharparenright}{\kern0pt}\ {\isasymcirc}\isactrlsub c\isanewline
\ \ \ \ \ \ \ \ \ \ \ \ dist{\isacharunderscore}{\kern0pt}prod{\isacharunderscore}{\kern0pt}coprod{\isacharunderscore}{\kern0pt}left\ {\isacharparenleft}{\kern0pt}X\isactrlbsup A\isactrlesup {\isacharparenright}{\kern0pt}\ A\ {\isacharparenleft}{\kern0pt}B\ {\isasymsetminus}\ {\isacharparenleft}{\kern0pt}A{\isacharcomma}{\kern0pt}\ m{\isacharparenright}{\kern0pt}{\isacharparenright}{\kern0pt}\ {\isasymcirc}\isactrlsub c\isanewline
\ \ \ \ \ \ \ \ \ \ \ \ swap\ {\isacharparenleft}{\kern0pt}A\ {\isasymCoprod}\ {\isacharparenleft}{\kern0pt}B\ {\isasymsetminus}\ {\isacharparenleft}{\kern0pt}A{\isacharcomma}{\kern0pt}\ m{\isacharparenright}{\kern0pt}{\isacharparenright}{\kern0pt}{\isacharparenright}{\kern0pt}\ {\isacharparenleft}{\kern0pt}X\isactrlbsup A\isactrlesup {\isacharparenright}{\kern0pt}\ {\isasymcirc}\isactrlsub c\ try{\isacharunderscore}{\kern0pt}cast\ m\ {\isasymtimes}\isactrlsub f\ id\isactrlsub c\ {\isacharparenleft}{\kern0pt}X\isactrlbsup A\isactrlesup {\isacharparenright}{\kern0pt}{\isacharparenright}{\kern0pt}\ {\isasymcirc}\isactrlsub c\ {\isacharparenleft}{\kern0pt}id\ B\ {\isasymtimes}\isactrlsub f\ \ h{\isacharparenright}{\kern0pt}\ {\isasymcirc}\isactrlsub c\ {\isasymlangle}m\ {\isasymcirc}\isactrlsub c\ a{\isacharcomma}{\kern0pt}\ z{\isasymrangle}{\isachardoublequoteclose}\isanewline
\ \ \ \ \ \ \ \ \ \ \ \ \isacommand{by}\isamarkupfalse%
\ {\isacharparenleft}{\kern0pt}typecheck{\isacharunderscore}{\kern0pt}cfuncs{\isacharcomma}{\kern0pt}\ auto\ simp\ add{\isacharcolon}{\kern0pt}\ comp{\isacharunderscore}{\kern0pt}associative{\isadigit{2}}{\isacharparenright}{\kern0pt}\isanewline
\ \ \ \ \ \ \ \ \ \ \isacommand{then}\isamarkupfalse%
\ \isacommand{have}\isamarkupfalse%
\ {\isachardoublequoteopen}{\isacharparenleft}{\kern0pt}{\isacharparenleft}{\kern0pt}eval{\isacharunderscore}{\kern0pt}func\ X\ A\ {\isasymcirc}\isactrlsub c\ swap\ {\isacharparenleft}{\kern0pt}X\isactrlbsup A\isactrlesup {\isacharparenright}{\kern0pt}\ A{\isacharparenright}{\kern0pt}\ {\isasymamalg}\ {\isacharparenleft}{\kern0pt}x\ {\isasymcirc}\isactrlsub c\ {\isasymbeta}\isactrlbsub X\isactrlbsup A\isactrlesup \ {\isasymtimes}\isactrlsub c\ {\isacharparenleft}{\kern0pt}B\ {\isasymsetminus}\ {\isacharparenleft}{\kern0pt}A{\isacharcomma}{\kern0pt}\ m{\isacharparenright}{\kern0pt}{\isacharparenright}{\kern0pt}\isactrlesub {\isacharparenright}{\kern0pt}\ {\isasymcirc}\isactrlsub c\isanewline
\ \ \ \ \ \ \ \ \ \ \ \ dist{\isacharunderscore}{\kern0pt}prod{\isacharunderscore}{\kern0pt}coprod{\isacharunderscore}{\kern0pt}left\ {\isacharparenleft}{\kern0pt}X\isactrlbsup A\isactrlesup {\isacharparenright}{\kern0pt}\ A\ {\isacharparenleft}{\kern0pt}B\ {\isasymsetminus}\ {\isacharparenleft}{\kern0pt}A{\isacharcomma}{\kern0pt}\ m{\isacharparenright}{\kern0pt}{\isacharparenright}{\kern0pt}\ {\isasymcirc}\isactrlsub c\isanewline
\ \ \ \ \ \ \ \ \ \ \ \ swap\ {\isacharparenleft}{\kern0pt}A\ {\isasymCoprod}\ {\isacharparenleft}{\kern0pt}B\ {\isasymsetminus}\ {\isacharparenleft}{\kern0pt}A{\isacharcomma}{\kern0pt}\ m{\isacharparenright}{\kern0pt}{\isacharparenright}{\kern0pt}{\isacharparenright}{\kern0pt}\ {\isacharparenleft}{\kern0pt}X\isactrlbsup A\isactrlesup {\isacharparenright}{\kern0pt}\ {\isasymcirc}\isactrlsub c\ try{\isacharunderscore}{\kern0pt}cast\ m\ {\isasymtimes}\isactrlsub f\ id\isactrlsub c\ {\isacharparenleft}{\kern0pt}X\isactrlbsup A\isactrlesup {\isacharparenright}{\kern0pt}{\isacharparenright}{\kern0pt}\ {\isasymcirc}\isactrlsub c\ {\isasymlangle}m\ {\isasymcirc}\isactrlsub c\ a{\isacharcomma}{\kern0pt}\ g\ {\isasymcirc}\isactrlsub c\ z{\isasymrangle}\isanewline
\ \ \ \ \ \ \ \ \ \ {\isacharequal}{\kern0pt}\ {\isacharparenleft}{\kern0pt}{\isacharparenleft}{\kern0pt}eval{\isacharunderscore}{\kern0pt}func\ X\ A\ {\isasymcirc}\isactrlsub c\ swap\ {\isacharparenleft}{\kern0pt}X\isactrlbsup A\isactrlesup {\isacharparenright}{\kern0pt}\ A{\isacharparenright}{\kern0pt}\ {\isasymamalg}\ {\isacharparenleft}{\kern0pt}x\ {\isasymcirc}\isactrlsub c\ {\isasymbeta}\isactrlbsub X\isactrlbsup A\isactrlesup \ {\isasymtimes}\isactrlsub c\ {\isacharparenleft}{\kern0pt}B\ {\isasymsetminus}\ {\isacharparenleft}{\kern0pt}A{\isacharcomma}{\kern0pt}\ m{\isacharparenright}{\kern0pt}{\isacharparenright}{\kern0pt}\isactrlesub {\isacharparenright}{\kern0pt}\ {\isasymcirc}\isactrlsub c\isanewline
\ \ \ \ \ \ \ \ \ \ \ \ dist{\isacharunderscore}{\kern0pt}prod{\isacharunderscore}{\kern0pt}coprod{\isacharunderscore}{\kern0pt}left\ {\isacharparenleft}{\kern0pt}X\isactrlbsup A\isactrlesup {\isacharparenright}{\kern0pt}\ A\ {\isacharparenleft}{\kern0pt}B\ {\isasymsetminus}\ {\isacharparenleft}{\kern0pt}A{\isacharcomma}{\kern0pt}\ m{\isacharparenright}{\kern0pt}{\isacharparenright}{\kern0pt}\ {\isasymcirc}\isactrlsub c\isanewline
\ \ \ \ \ \ \ \ \ \ \ \ swap\ {\isacharparenleft}{\kern0pt}A\ {\isasymCoprod}\ {\isacharparenleft}{\kern0pt}B\ {\isasymsetminus}\ {\isacharparenleft}{\kern0pt}A{\isacharcomma}{\kern0pt}\ m{\isacharparenright}{\kern0pt}{\isacharparenright}{\kern0pt}{\isacharparenright}{\kern0pt}\ {\isacharparenleft}{\kern0pt}X\isactrlbsup A\isactrlesup {\isacharparenright}{\kern0pt}\ {\isasymcirc}\isactrlsub c\ try{\isacharunderscore}{\kern0pt}cast\ m\ {\isasymtimes}\isactrlsub f\ id\isactrlsub c\ {\isacharparenleft}{\kern0pt}X\isactrlbsup A\isactrlesup {\isacharparenright}{\kern0pt}{\isacharparenright}{\kern0pt}\ {\isasymcirc}\isactrlsub c\ {\isasymlangle}m\ {\isasymcirc}\isactrlsub c\ a{\isacharcomma}{\kern0pt}\ h\ {\isasymcirc}\isactrlsub c\ z{\isasymrangle}{\isachardoublequoteclose}\isanewline
\ \ \ \ \ \ \ \ \ \ \ \ \isacommand{by}\isamarkupfalse%
\ {\isacharparenleft}{\kern0pt}typecheck{\isacharunderscore}{\kern0pt}cfuncs{\isacharcomma}{\kern0pt}\ smt\ cfunc{\isacharunderscore}{\kern0pt}cross{\isacharunderscore}{\kern0pt}prod{\isacharunderscore}{\kern0pt}comp{\isacharunderscore}{\kern0pt}cfunc{\isacharunderscore}{\kern0pt}prod\ id{\isacharunderscore}{\kern0pt}left{\isacharunderscore}{\kern0pt}unit{\isadigit{2}}\ id{\isacharunderscore}{\kern0pt}type{\isacharparenright}{\kern0pt}\isanewline
\ \ \ \ \ \ \ \ \ \ \isacommand{then}\isamarkupfalse%
\ \isacommand{have}\isamarkupfalse%
\ {\isachardoublequoteopen}{\isacharparenleft}{\kern0pt}eval{\isacharunderscore}{\kern0pt}func\ X\ A\ {\isasymcirc}\isactrlsub c\ swap\ {\isacharparenleft}{\kern0pt}X\isactrlbsup A\isactrlesup {\isacharparenright}{\kern0pt}\ A{\isacharparenright}{\kern0pt}\ {\isasymamalg}\ {\isacharparenleft}{\kern0pt}x\ {\isasymcirc}\isactrlsub c\ {\isasymbeta}\isactrlbsub X\isactrlbsup A\isactrlesup \ {\isasymtimes}\isactrlsub c\ {\isacharparenleft}{\kern0pt}B\ {\isasymsetminus}\ {\isacharparenleft}{\kern0pt}A{\isacharcomma}{\kern0pt}\ m{\isacharparenright}{\kern0pt}{\isacharparenright}{\kern0pt}\isactrlesub {\isacharparenright}{\kern0pt}\ {\isasymcirc}\isactrlsub c\isanewline
\ \ \ \ \ \ \ \ \ \ \ \ dist{\isacharunderscore}{\kern0pt}prod{\isacharunderscore}{\kern0pt}coprod{\isacharunderscore}{\kern0pt}left\ {\isacharparenleft}{\kern0pt}X\isactrlbsup A\isactrlesup {\isacharparenright}{\kern0pt}\ A\ {\isacharparenleft}{\kern0pt}B\ {\isasymsetminus}\ {\isacharparenleft}{\kern0pt}A{\isacharcomma}{\kern0pt}\ m{\isacharparenright}{\kern0pt}{\isacharparenright}{\kern0pt}\ {\isasymcirc}\isactrlsub c\isanewline
\ \ \ \ \ \ \ \ \ \ \ \ swap\ {\isacharparenleft}{\kern0pt}A\ {\isasymCoprod}\ {\isacharparenleft}{\kern0pt}B\ {\isasymsetminus}\ {\isacharparenleft}{\kern0pt}A{\isacharcomma}{\kern0pt}\ m{\isacharparenright}{\kern0pt}{\isacharparenright}{\kern0pt}{\isacharparenright}{\kern0pt}\ {\isacharparenleft}{\kern0pt}X\isactrlbsup A\isactrlesup {\isacharparenright}{\kern0pt}\ {\isasymcirc}\isactrlsub c\ {\isacharparenleft}{\kern0pt}try{\isacharunderscore}{\kern0pt}cast\ m\ {\isasymtimes}\isactrlsub f\ id\isactrlsub c\ {\isacharparenleft}{\kern0pt}X\isactrlbsup A\isactrlesup {\isacharparenright}{\kern0pt}{\isacharparenright}{\kern0pt}\ {\isasymcirc}\isactrlsub c\ {\isasymlangle}m\ {\isasymcirc}\isactrlsub c\ a{\isacharcomma}{\kern0pt}\ g\ {\isasymcirc}\isactrlsub c\ z{\isasymrangle}\isanewline
\ \ \ \ \ \ \ \ \ \ {\isacharequal}{\kern0pt}\ {\isacharparenleft}{\kern0pt}eval{\isacharunderscore}{\kern0pt}func\ X\ A\ {\isasymcirc}\isactrlsub c\ swap\ {\isacharparenleft}{\kern0pt}X\isactrlbsup A\isactrlesup {\isacharparenright}{\kern0pt}\ A{\isacharparenright}{\kern0pt}\ {\isasymamalg}\ {\isacharparenleft}{\kern0pt}x\ {\isasymcirc}\isactrlsub c\ {\isasymbeta}\isactrlbsub X\isactrlbsup A\isactrlesup \ {\isasymtimes}\isactrlsub c\ {\isacharparenleft}{\kern0pt}B\ {\isasymsetminus}\ {\isacharparenleft}{\kern0pt}A{\isacharcomma}{\kern0pt}\ m{\isacharparenright}{\kern0pt}{\isacharparenright}{\kern0pt}\isactrlesub {\isacharparenright}{\kern0pt}\ {\isasymcirc}\isactrlsub c\isanewline
\ \ \ \ \ \ \ \ \ \ \ \ dist{\isacharunderscore}{\kern0pt}prod{\isacharunderscore}{\kern0pt}coprod{\isacharunderscore}{\kern0pt}left\ {\isacharparenleft}{\kern0pt}X\isactrlbsup A\isactrlesup {\isacharparenright}{\kern0pt}\ A\ {\isacharparenleft}{\kern0pt}B\ {\isasymsetminus}\ {\isacharparenleft}{\kern0pt}A{\isacharcomma}{\kern0pt}\ m{\isacharparenright}{\kern0pt}{\isacharparenright}{\kern0pt}\ {\isasymcirc}\isactrlsub c\isanewline
\ \ \ \ \ \ \ \ \ \ \ \ swap\ {\isacharparenleft}{\kern0pt}A\ {\isasymCoprod}\ {\isacharparenleft}{\kern0pt}B\ {\isasymsetminus}\ {\isacharparenleft}{\kern0pt}A{\isacharcomma}{\kern0pt}\ m{\isacharparenright}{\kern0pt}{\isacharparenright}{\kern0pt}{\isacharparenright}{\kern0pt}\ {\isacharparenleft}{\kern0pt}X\isactrlbsup A\isactrlesup {\isacharparenright}{\kern0pt}\ {\isasymcirc}\isactrlsub c\ {\isacharparenleft}{\kern0pt}try{\isacharunderscore}{\kern0pt}cast\ m\ {\isasymtimes}\isactrlsub f\ id\isactrlsub c\ {\isacharparenleft}{\kern0pt}X\isactrlbsup A\isactrlesup {\isacharparenright}{\kern0pt}{\isacharparenright}{\kern0pt}\ {\isasymcirc}\isactrlsub c\ {\isasymlangle}m\ {\isasymcirc}\isactrlsub c\ a{\isacharcomma}{\kern0pt}\ h\ {\isasymcirc}\isactrlsub c\ z{\isasymrangle}{\isachardoublequoteclose}\isanewline
\ \ \ \ \ \ \ \ \ \ \ \ \isacommand{by}\isamarkupfalse%
\ {\isacharparenleft}{\kern0pt}typecheck{\isacharunderscore}{\kern0pt}cfuncs{\isacharunderscore}{\kern0pt}prems{\isacharcomma}{\kern0pt}\ smt\ comp{\isacharunderscore}{\kern0pt}associative{\isadigit{2}}{\isacharparenright}{\kern0pt}\isanewline
\ \ \ \ \ \ \ \ \ \ \isacommand{then}\isamarkupfalse%
\ \isacommand{have}\isamarkupfalse%
\ {\isachardoublequoteopen}{\isacharparenleft}{\kern0pt}eval{\isacharunderscore}{\kern0pt}func\ X\ A\ {\isasymcirc}\isactrlsub c\ swap\ {\isacharparenleft}{\kern0pt}X\isactrlbsup A\isactrlesup {\isacharparenright}{\kern0pt}\ A{\isacharparenright}{\kern0pt}\ {\isasymamalg}\ {\isacharparenleft}{\kern0pt}x\ {\isasymcirc}\isactrlsub c\ {\isasymbeta}\isactrlbsub X\isactrlbsup A\isactrlesup \ {\isasymtimes}\isactrlsub c\ {\isacharparenleft}{\kern0pt}B\ {\isasymsetminus}\ {\isacharparenleft}{\kern0pt}A{\isacharcomma}{\kern0pt}\ m{\isacharparenright}{\kern0pt}{\isacharparenright}{\kern0pt}\isactrlesub {\isacharparenright}{\kern0pt}\ {\isasymcirc}\isactrlsub c\isanewline
\ \ \ \ \ \ \ \ \ \ \ \ dist{\isacharunderscore}{\kern0pt}prod{\isacharunderscore}{\kern0pt}coprod{\isacharunderscore}{\kern0pt}left\ {\isacharparenleft}{\kern0pt}X\isactrlbsup A\isactrlesup {\isacharparenright}{\kern0pt}\ A\ {\isacharparenleft}{\kern0pt}B\ {\isasymsetminus}\ {\isacharparenleft}{\kern0pt}A{\isacharcomma}{\kern0pt}\ m{\isacharparenright}{\kern0pt}{\isacharparenright}{\kern0pt}\ {\isasymcirc}\isactrlsub c\isanewline
\ \ \ \ \ \ \ \ \ \ \ \ swap\ {\isacharparenleft}{\kern0pt}A\ {\isasymCoprod}\ {\isacharparenleft}{\kern0pt}B\ {\isasymsetminus}\ {\isacharparenleft}{\kern0pt}A{\isacharcomma}{\kern0pt}\ m{\isacharparenright}{\kern0pt}{\isacharparenright}{\kern0pt}{\isacharparenright}{\kern0pt}\ {\isacharparenleft}{\kern0pt}X\isactrlbsup A\isactrlesup {\isacharparenright}{\kern0pt}\ {\isasymcirc}\isactrlsub c\ {\isasymlangle}try{\isacharunderscore}{\kern0pt}cast\ m\ {\isasymcirc}\isactrlsub c\ m\ {\isasymcirc}\isactrlsub c\ a{\isacharcomma}{\kern0pt}\ g\ {\isasymcirc}\isactrlsub c\ z{\isasymrangle}\isanewline
\ \ \ \ \ \ \ \ \ \ {\isacharequal}{\kern0pt}\ {\isacharparenleft}{\kern0pt}eval{\isacharunderscore}{\kern0pt}func\ X\ A\ {\isasymcirc}\isactrlsub c\ swap\ {\isacharparenleft}{\kern0pt}X\isactrlbsup A\isactrlesup {\isacharparenright}{\kern0pt}\ A{\isacharparenright}{\kern0pt}\ {\isasymamalg}\ {\isacharparenleft}{\kern0pt}x\ {\isasymcirc}\isactrlsub c\ {\isasymbeta}\isactrlbsub X\isactrlbsup A\isactrlesup \ {\isasymtimes}\isactrlsub c\ {\isacharparenleft}{\kern0pt}B\ {\isasymsetminus}\ {\isacharparenleft}{\kern0pt}A{\isacharcomma}{\kern0pt}\ m{\isacharparenright}{\kern0pt}{\isacharparenright}{\kern0pt}\isactrlesub {\isacharparenright}{\kern0pt}\ {\isasymcirc}\isactrlsub c\isanewline
\ \ \ \ \ \ \ \ \ \ \ \ dist{\isacharunderscore}{\kern0pt}prod{\isacharunderscore}{\kern0pt}coprod{\isacharunderscore}{\kern0pt}left\ {\isacharparenleft}{\kern0pt}X\isactrlbsup A\isactrlesup {\isacharparenright}{\kern0pt}\ A\ {\isacharparenleft}{\kern0pt}B\ {\isasymsetminus}\ {\isacharparenleft}{\kern0pt}A{\isacharcomma}{\kern0pt}\ m{\isacharparenright}{\kern0pt}{\isacharparenright}{\kern0pt}\ {\isasymcirc}\isactrlsub c\isanewline
\ \ \ \ \ \ \ \ \ \ \ \ swap\ {\isacharparenleft}{\kern0pt}A\ {\isasymCoprod}\ {\isacharparenleft}{\kern0pt}B\ {\isasymsetminus}\ {\isacharparenleft}{\kern0pt}A{\isacharcomma}{\kern0pt}\ m{\isacharparenright}{\kern0pt}{\isacharparenright}{\kern0pt}{\isacharparenright}{\kern0pt}\ {\isacharparenleft}{\kern0pt}X\isactrlbsup A\isactrlesup {\isacharparenright}{\kern0pt}\ {\isasymcirc}\isactrlsub c\ {\isasymlangle}try{\isacharunderscore}{\kern0pt}cast\ m\ {\isasymcirc}\isactrlsub c\ m\ {\isasymcirc}\isactrlsub c\ a{\isacharcomma}{\kern0pt}\ h\ {\isasymcirc}\isactrlsub c\ z{\isasymrangle}{\isachardoublequoteclose}\isanewline
\ \ \ \ \ \ \ \ \ \ \ \ \isacommand{using}\isamarkupfalse%
\ cfunc{\isacharunderscore}{\kern0pt}cross{\isacharunderscore}{\kern0pt}prod{\isacharunderscore}{\kern0pt}comp{\isacharunderscore}{\kern0pt}cfunc{\isacharunderscore}{\kern0pt}prod\ id{\isacharunderscore}{\kern0pt}left{\isacharunderscore}{\kern0pt}unit{\isadigit{2}}\ \isacommand{by}\isamarkupfalse%
\ {\isacharparenleft}{\kern0pt}typecheck{\isacharunderscore}{\kern0pt}cfuncs{\isacharunderscore}{\kern0pt}prems{\isacharcomma}{\kern0pt}\ smt{\isacharparenright}{\kern0pt}\isanewline
\ \ \ \ \ \ \ \ \ \ \isacommand{then}\isamarkupfalse%
\ \isacommand{have}\isamarkupfalse%
\ {\isachardoublequoteopen}{\isacharparenleft}{\kern0pt}eval{\isacharunderscore}{\kern0pt}func\ X\ A\ {\isasymcirc}\isactrlsub c\ swap\ {\isacharparenleft}{\kern0pt}X\isactrlbsup A\isactrlesup {\isacharparenright}{\kern0pt}\ A{\isacharparenright}{\kern0pt}\ {\isasymamalg}\ {\isacharparenleft}{\kern0pt}x\ {\isasymcirc}\isactrlsub c\ {\isasymbeta}\isactrlbsub X\isactrlbsup A\isactrlesup \ {\isasymtimes}\isactrlsub c\ {\isacharparenleft}{\kern0pt}B\ {\isasymsetminus}\ {\isacharparenleft}{\kern0pt}A{\isacharcomma}{\kern0pt}\ m{\isacharparenright}{\kern0pt}{\isacharparenright}{\kern0pt}\isactrlesub {\isacharparenright}{\kern0pt}\ {\isasymcirc}\isactrlsub c\isanewline
\ \ \ \ \ \ \ \ \ \ \ \ dist{\isacharunderscore}{\kern0pt}prod{\isacharunderscore}{\kern0pt}coprod{\isacharunderscore}{\kern0pt}left\ {\isacharparenleft}{\kern0pt}X\isactrlbsup A\isactrlesup {\isacharparenright}{\kern0pt}\ A\ {\isacharparenleft}{\kern0pt}B\ {\isasymsetminus}\ {\isacharparenleft}{\kern0pt}A{\isacharcomma}{\kern0pt}\ m{\isacharparenright}{\kern0pt}{\isacharparenright}{\kern0pt}\ {\isasymcirc}\isactrlsub c\isanewline
\ \ \ \ \ \ \ \ \ \ \ \ swap\ {\isacharparenleft}{\kern0pt}A\ {\isasymCoprod}\ {\isacharparenleft}{\kern0pt}B\ {\isasymsetminus}\ {\isacharparenleft}{\kern0pt}A{\isacharcomma}{\kern0pt}\ m{\isacharparenright}{\kern0pt}{\isacharparenright}{\kern0pt}{\isacharparenright}{\kern0pt}\ {\isacharparenleft}{\kern0pt}X\isactrlbsup A\isactrlesup {\isacharparenright}{\kern0pt}\ {\isasymcirc}\isactrlsub c\ {\isasymlangle}{\isacharparenleft}{\kern0pt}try{\isacharunderscore}{\kern0pt}cast\ m\ {\isasymcirc}\isactrlsub c\ m{\isacharparenright}{\kern0pt}\ {\isasymcirc}\isactrlsub c\ a{\isacharcomma}{\kern0pt}\ g\ {\isasymcirc}\isactrlsub c\ z{\isasymrangle}\isanewline
\ \ \ \ \ \ \ \ \ \ {\isacharequal}{\kern0pt}\ {\isacharparenleft}{\kern0pt}eval{\isacharunderscore}{\kern0pt}func\ X\ A\ {\isasymcirc}\isactrlsub c\ swap\ {\isacharparenleft}{\kern0pt}X\isactrlbsup A\isactrlesup {\isacharparenright}{\kern0pt}\ A{\isacharparenright}{\kern0pt}\ {\isasymamalg}\ {\isacharparenleft}{\kern0pt}x\ {\isasymcirc}\isactrlsub c\ {\isasymbeta}\isactrlbsub X\isactrlbsup A\isactrlesup \ {\isasymtimes}\isactrlsub c\ {\isacharparenleft}{\kern0pt}B\ {\isasymsetminus}\ {\isacharparenleft}{\kern0pt}A{\isacharcomma}{\kern0pt}\ m{\isacharparenright}{\kern0pt}{\isacharparenright}{\kern0pt}\isactrlesub {\isacharparenright}{\kern0pt}\ {\isasymcirc}\isactrlsub c\isanewline
\ \ \ \ \ \ \ \ \ \ \ \ dist{\isacharunderscore}{\kern0pt}prod{\isacharunderscore}{\kern0pt}coprod{\isacharunderscore}{\kern0pt}left\ {\isacharparenleft}{\kern0pt}X\isactrlbsup A\isactrlesup {\isacharparenright}{\kern0pt}\ A\ {\isacharparenleft}{\kern0pt}B\ {\isasymsetminus}\ {\isacharparenleft}{\kern0pt}A{\isacharcomma}{\kern0pt}\ m{\isacharparenright}{\kern0pt}{\isacharparenright}{\kern0pt}\ {\isasymcirc}\isactrlsub c\isanewline
\ \ \ \ \ \ \ \ \ \ \ \ swap\ {\isacharparenleft}{\kern0pt}A\ {\isasymCoprod}\ {\isacharparenleft}{\kern0pt}B\ {\isasymsetminus}\ {\isacharparenleft}{\kern0pt}A{\isacharcomma}{\kern0pt}\ m{\isacharparenright}{\kern0pt}{\isacharparenright}{\kern0pt}{\isacharparenright}{\kern0pt}\ {\isacharparenleft}{\kern0pt}X\isactrlbsup A\isactrlesup {\isacharparenright}{\kern0pt}\ {\isasymcirc}\isactrlsub c\ {\isasymlangle}{\isacharparenleft}{\kern0pt}try{\isacharunderscore}{\kern0pt}cast\ m\ {\isasymcirc}\isactrlsub c\ m{\isacharparenright}{\kern0pt}\ {\isasymcirc}\isactrlsub c\ a{\isacharcomma}{\kern0pt}\ h\ {\isasymcirc}\isactrlsub c\ z{\isasymrangle}{\isachardoublequoteclose}\isanewline
\ \ \ \ \ \ \ \ \ \ \ \ \isacommand{by}\isamarkupfalse%
\ {\isacharparenleft}{\kern0pt}typecheck{\isacharunderscore}{\kern0pt}cfuncs{\isacharcomma}{\kern0pt}\ auto\ simp\ add{\isacharcolon}{\kern0pt}\ comp{\isacharunderscore}{\kern0pt}associative{\isadigit{2}}{\isacharparenright}{\kern0pt}\isanewline
\ \ \ \ \ \ \ \ \ \ \isacommand{then}\isamarkupfalse%
\ \isacommand{have}\isamarkupfalse%
\ {\isachardoublequoteopen}{\isacharparenleft}{\kern0pt}eval{\isacharunderscore}{\kern0pt}func\ X\ A\ {\isasymcirc}\isactrlsub c\ swap\ {\isacharparenleft}{\kern0pt}X\isactrlbsup A\isactrlesup {\isacharparenright}{\kern0pt}\ A{\isacharparenright}{\kern0pt}\ {\isasymamalg}\ {\isacharparenleft}{\kern0pt}x\ {\isasymcirc}\isactrlsub c\ {\isasymbeta}\isactrlbsub X\isactrlbsup A\isactrlesup \ {\isasymtimes}\isactrlsub c\ {\isacharparenleft}{\kern0pt}B\ {\isasymsetminus}\ {\isacharparenleft}{\kern0pt}A{\isacharcomma}{\kern0pt}\ m{\isacharparenright}{\kern0pt}{\isacharparenright}{\kern0pt}\isactrlesub {\isacharparenright}{\kern0pt}\ {\isasymcirc}\isactrlsub c\isanewline
\ \ \ \ \ \ \ \ \ \ \ \ dist{\isacharunderscore}{\kern0pt}prod{\isacharunderscore}{\kern0pt}coprod{\isacharunderscore}{\kern0pt}left\ {\isacharparenleft}{\kern0pt}X\isactrlbsup A\isactrlesup {\isacharparenright}{\kern0pt}\ A\ {\isacharparenleft}{\kern0pt}B\ {\isasymsetminus}\ {\isacharparenleft}{\kern0pt}A{\isacharcomma}{\kern0pt}\ m{\isacharparenright}{\kern0pt}{\isacharparenright}{\kern0pt}\ {\isasymcirc}\isactrlsub c\isanewline
\ \ \ \ \ \ \ \ \ \ \ \ swap\ {\isacharparenleft}{\kern0pt}A\ {\isasymCoprod}\ {\isacharparenleft}{\kern0pt}B\ {\isasymsetminus}\ {\isacharparenleft}{\kern0pt}A{\isacharcomma}{\kern0pt}\ m{\isacharparenright}{\kern0pt}{\isacharparenright}{\kern0pt}{\isacharparenright}{\kern0pt}\ {\isacharparenleft}{\kern0pt}X\isactrlbsup A\isactrlesup {\isacharparenright}{\kern0pt}\ {\isasymcirc}\isactrlsub c\ {\isasymlangle}left{\isacharunderscore}{\kern0pt}coproj\ A\ {\isacharparenleft}{\kern0pt}B\ {\isasymsetminus}\ {\isacharparenleft}{\kern0pt}A{\isacharcomma}{\kern0pt}m{\isacharparenright}{\kern0pt}{\isacharparenright}{\kern0pt}\ {\isasymcirc}\isactrlsub c\ a{\isacharcomma}{\kern0pt}\ g\ {\isasymcirc}\isactrlsub c\ z{\isasymrangle}\isanewline
\ \ \ \ \ \ \ \ \ \ {\isacharequal}{\kern0pt}\ {\isacharparenleft}{\kern0pt}eval{\isacharunderscore}{\kern0pt}func\ X\ A\ {\isasymcirc}\isactrlsub c\ swap\ {\isacharparenleft}{\kern0pt}X\isactrlbsup A\isactrlesup {\isacharparenright}{\kern0pt}\ A{\isacharparenright}{\kern0pt}\ {\isasymamalg}\ {\isacharparenleft}{\kern0pt}x\ {\isasymcirc}\isactrlsub c\ {\isasymbeta}\isactrlbsub X\isactrlbsup A\isactrlesup \ {\isasymtimes}\isactrlsub c\ {\isacharparenleft}{\kern0pt}B\ {\isasymsetminus}\ {\isacharparenleft}{\kern0pt}A{\isacharcomma}{\kern0pt}\ m{\isacharparenright}{\kern0pt}{\isacharparenright}{\kern0pt}\isactrlesub {\isacharparenright}{\kern0pt}\ {\isasymcirc}\isactrlsub c\isanewline
\ \ \ \ \ \ \ \ \ \ \ \ dist{\isacharunderscore}{\kern0pt}prod{\isacharunderscore}{\kern0pt}coprod{\isacharunderscore}{\kern0pt}left\ {\isacharparenleft}{\kern0pt}X\isactrlbsup A\isactrlesup {\isacharparenright}{\kern0pt}\ A\ {\isacharparenleft}{\kern0pt}B\ {\isasymsetminus}\ {\isacharparenleft}{\kern0pt}A{\isacharcomma}{\kern0pt}\ m{\isacharparenright}{\kern0pt}{\isacharparenright}{\kern0pt}\ {\isasymcirc}\isactrlsub c\isanewline
\ \ \ \ \ \ \ \ \ \ \ \ swap\ {\isacharparenleft}{\kern0pt}A\ {\isasymCoprod}\ {\isacharparenleft}{\kern0pt}B\ {\isasymsetminus}\ {\isacharparenleft}{\kern0pt}A{\isacharcomma}{\kern0pt}\ m{\isacharparenright}{\kern0pt}{\isacharparenright}{\kern0pt}{\isacharparenright}{\kern0pt}\ {\isacharparenleft}{\kern0pt}X\isactrlbsup A\isactrlesup {\isacharparenright}{\kern0pt}\ {\isasymcirc}\isactrlsub c\ {\isasymlangle}left{\isacharunderscore}{\kern0pt}coproj\ A\ {\isacharparenleft}{\kern0pt}B\ {\isasymsetminus}\ {\isacharparenleft}{\kern0pt}A{\isacharcomma}{\kern0pt}m{\isacharparenright}{\kern0pt}{\isacharparenright}{\kern0pt}\ {\isasymcirc}\isactrlsub c\ a{\isacharcomma}{\kern0pt}\ h\ {\isasymcirc}\isactrlsub c\ z{\isasymrangle}{\isachardoublequoteclose}\isanewline
\ \ \ \ \ \ \ \ \ \ \ \ \isacommand{using}\isamarkupfalse%
\ m{\isacharunderscore}{\kern0pt}def{\isacharparenleft}{\kern0pt}{\isadigit{2}}{\isacharparenright}{\kern0pt}\ try{\isacharunderscore}{\kern0pt}cast{\isacharunderscore}{\kern0pt}m{\isacharunderscore}{\kern0pt}m\ \isacommand{by}\isamarkupfalse%
\ {\isacharparenleft}{\kern0pt}typecheck{\isacharunderscore}{\kern0pt}cfuncs{\isacharcomma}{\kern0pt}\ auto{\isacharparenright}{\kern0pt}\isanewline
\ \ \ \ \ \ \ \ \ \ \isacommand{then}\isamarkupfalse%
\ \isacommand{have}\isamarkupfalse%
\ {\isachardoublequoteopen}{\isacharparenleft}{\kern0pt}eval{\isacharunderscore}{\kern0pt}func\ X\ A\ {\isasymcirc}\isactrlsub c\ swap\ {\isacharparenleft}{\kern0pt}X\isactrlbsup A\isactrlesup {\isacharparenright}{\kern0pt}\ A{\isacharparenright}{\kern0pt}\ {\isasymamalg}\ {\isacharparenleft}{\kern0pt}x\ {\isasymcirc}\isactrlsub c\ {\isasymbeta}\isactrlbsub X\isactrlbsup A\isactrlesup \ {\isasymtimes}\isactrlsub c\ {\isacharparenleft}{\kern0pt}B\ {\isasymsetminus}\ {\isacharparenleft}{\kern0pt}A{\isacharcomma}{\kern0pt}\ m{\isacharparenright}{\kern0pt}{\isacharparenright}{\kern0pt}\isactrlesub {\isacharparenright}{\kern0pt}\ {\isasymcirc}\isactrlsub c\isanewline
\ \ \ \ \ \ \ \ \ \ \ \ dist{\isacharunderscore}{\kern0pt}prod{\isacharunderscore}{\kern0pt}coprod{\isacharunderscore}{\kern0pt}left\ {\isacharparenleft}{\kern0pt}X\isactrlbsup A\isactrlesup {\isacharparenright}{\kern0pt}\ A\ {\isacharparenleft}{\kern0pt}B\ {\isasymsetminus}\ {\isacharparenleft}{\kern0pt}A{\isacharcomma}{\kern0pt}\ m{\isacharparenright}{\kern0pt}{\isacharparenright}{\kern0pt}\ {\isasymcirc}\isactrlsub c\ {\isasymlangle}g\ {\isasymcirc}\isactrlsub c\ z{\isacharcomma}{\kern0pt}\ left{\isacharunderscore}{\kern0pt}coproj\ A\ {\isacharparenleft}{\kern0pt}B\ {\isasymsetminus}\ {\isacharparenleft}{\kern0pt}A{\isacharcomma}{\kern0pt}m{\isacharparenright}{\kern0pt}{\isacharparenright}{\kern0pt}\ {\isasymcirc}\isactrlsub c\ a{\isasymrangle}\isanewline
\ \ \ \ \ \ \ \ \ \ {\isacharequal}{\kern0pt}\ {\isacharparenleft}{\kern0pt}eval{\isacharunderscore}{\kern0pt}func\ X\ A\ {\isasymcirc}\isactrlsub c\ swap\ {\isacharparenleft}{\kern0pt}X\isactrlbsup A\isactrlesup {\isacharparenright}{\kern0pt}\ A{\isacharparenright}{\kern0pt}\ {\isasymamalg}\ {\isacharparenleft}{\kern0pt}x\ {\isasymcirc}\isactrlsub c\ {\isasymbeta}\isactrlbsub X\isactrlbsup A\isactrlesup \ {\isasymtimes}\isactrlsub c\ {\isacharparenleft}{\kern0pt}B\ {\isasymsetminus}\ {\isacharparenleft}{\kern0pt}A{\isacharcomma}{\kern0pt}\ m{\isacharparenright}{\kern0pt}{\isacharparenright}{\kern0pt}\isactrlesub {\isacharparenright}{\kern0pt}\ {\isasymcirc}\isactrlsub c\isanewline
\ \ \ \ \ \ \ \ \ \ \ \ dist{\isacharunderscore}{\kern0pt}prod{\isacharunderscore}{\kern0pt}coprod{\isacharunderscore}{\kern0pt}left\ {\isacharparenleft}{\kern0pt}X\isactrlbsup A\isactrlesup {\isacharparenright}{\kern0pt}\ A\ {\isacharparenleft}{\kern0pt}B\ {\isasymsetminus}\ {\isacharparenleft}{\kern0pt}A{\isacharcomma}{\kern0pt}\ m{\isacharparenright}{\kern0pt}{\isacharparenright}{\kern0pt}\ {\isasymcirc}\isactrlsub c\ {\isasymlangle}h\ {\isasymcirc}\isactrlsub c\ z{\isacharcomma}{\kern0pt}\ left{\isacharunderscore}{\kern0pt}coproj\ A\ {\isacharparenleft}{\kern0pt}B\ {\isasymsetminus}\ {\isacharparenleft}{\kern0pt}A{\isacharcomma}{\kern0pt}m{\isacharparenright}{\kern0pt}{\isacharparenright}{\kern0pt}\ {\isasymcirc}\isactrlsub c\ a{\isasymrangle}{\isachardoublequoteclose}\isanewline
\ \ \ \ \ \ \ \ \ \ \ \ \isacommand{using}\isamarkupfalse%
\ swap{\isacharunderscore}{\kern0pt}ap\ \isacommand{by}\isamarkupfalse%
\ {\isacharparenleft}{\kern0pt}typecheck{\isacharunderscore}{\kern0pt}cfuncs{\isacharcomma}{\kern0pt}\ auto{\isacharparenright}{\kern0pt}\isanewline
\ \ \ \ \ \ \ \ \ \ \isacommand{then}\isamarkupfalse%
\ \isacommand{have}\isamarkupfalse%
\ {\isachardoublequoteopen}{\isacharparenleft}{\kern0pt}eval{\isacharunderscore}{\kern0pt}func\ X\ A\ {\isasymcirc}\isactrlsub c\ swap\ {\isacharparenleft}{\kern0pt}X\isactrlbsup A\isactrlesup {\isacharparenright}{\kern0pt}\ A{\isacharparenright}{\kern0pt}\ {\isasymamalg}\ {\isacharparenleft}{\kern0pt}x\ {\isasymcirc}\isactrlsub c\ {\isasymbeta}\isactrlbsub X\isactrlbsup A\isactrlesup \ {\isasymtimes}\isactrlsub c\ {\isacharparenleft}{\kern0pt}B\ {\isasymsetminus}\ {\isacharparenleft}{\kern0pt}A{\isacharcomma}{\kern0pt}\ m{\isacharparenright}{\kern0pt}{\isacharparenright}{\kern0pt}\isactrlesub {\isacharparenright}{\kern0pt}\ {\isasymcirc}\isactrlsub c\isanewline
\ \ \ \ \ \ \ \ \ \ \ \ left{\isacharunderscore}{\kern0pt}coproj\ {\isacharparenleft}{\kern0pt}X\isactrlbsup A\isactrlesup {\isasymtimes}\isactrlsub cA{\isacharparenright}{\kern0pt}\ {\isacharparenleft}{\kern0pt}X\isactrlbsup A\isactrlesup {\isasymtimes}\isactrlsub c{\isacharparenleft}{\kern0pt}B\ {\isasymsetminus}\ {\isacharparenleft}{\kern0pt}A{\isacharcomma}{\kern0pt}m{\isacharparenright}{\kern0pt}{\isacharparenright}{\kern0pt}{\isacharparenright}{\kern0pt}\ {\isasymcirc}\isactrlsub c\ {\isasymlangle}g\ {\isasymcirc}\isactrlsub c\ z{\isacharcomma}{\kern0pt}\ a{\isasymrangle}\isanewline
\ \ \ \ \ \ \ \ \ \ {\isacharequal}{\kern0pt}\ {\isacharparenleft}{\kern0pt}eval{\isacharunderscore}{\kern0pt}func\ X\ A\ {\isasymcirc}\isactrlsub c\ swap\ {\isacharparenleft}{\kern0pt}X\isactrlbsup A\isactrlesup {\isacharparenright}{\kern0pt}\ A{\isacharparenright}{\kern0pt}\ {\isasymamalg}\ {\isacharparenleft}{\kern0pt}x\ {\isasymcirc}\isactrlsub c\ {\isasymbeta}\isactrlbsub X\isactrlbsup A\isactrlesup \ {\isasymtimes}\isactrlsub c\ {\isacharparenleft}{\kern0pt}B\ {\isasymsetminus}\ {\isacharparenleft}{\kern0pt}A{\isacharcomma}{\kern0pt}\ m{\isacharparenright}{\kern0pt}{\isacharparenright}{\kern0pt}\isactrlesub {\isacharparenright}{\kern0pt}\ {\isasymcirc}\isactrlsub c\isanewline
\ \ \ \ \ \ \ \ \ \ \ \ left{\isacharunderscore}{\kern0pt}coproj\ {\isacharparenleft}{\kern0pt}X\isactrlbsup A\isactrlesup {\isasymtimes}\isactrlsub cA{\isacharparenright}{\kern0pt}\ {\isacharparenleft}{\kern0pt}X\isactrlbsup A\isactrlesup {\isasymtimes}\isactrlsub c{\isacharparenleft}{\kern0pt}B\ {\isasymsetminus}\ {\isacharparenleft}{\kern0pt}A{\isacharcomma}{\kern0pt}m{\isacharparenright}{\kern0pt}{\isacharparenright}{\kern0pt}{\isacharparenright}{\kern0pt}\ {\isasymcirc}\isactrlsub c\ {\isasymlangle}h\ {\isasymcirc}\isactrlsub c\ z{\isacharcomma}{\kern0pt}a{\isasymrangle}{\isachardoublequoteclose}\isanewline
\ \ \ \ \ \ \ \ \ \ \ \ \isacommand{using}\isamarkupfalse%
\ dist{\isacharunderscore}{\kern0pt}prod{\isacharunderscore}{\kern0pt}coprod{\isacharunderscore}{\kern0pt}left{\isacharunderscore}{\kern0pt}ap{\isacharunderscore}{\kern0pt}left\ \isacommand{by}\isamarkupfalse%
\ {\isacharparenleft}{\kern0pt}typecheck{\isacharunderscore}{\kern0pt}cfuncs{\isacharcomma}{\kern0pt}\ auto{\isacharparenright}{\kern0pt}\isanewline
\ \ \ \ \ \ \ \ \ \ \isacommand{then}\isamarkupfalse%
\ \isacommand{have}\isamarkupfalse%
\ {\isachardoublequoteopen}{\isacharparenleft}{\kern0pt}{\isacharparenleft}{\kern0pt}eval{\isacharunderscore}{\kern0pt}func\ X\ A\ {\isasymcirc}\isactrlsub c\ swap\ {\isacharparenleft}{\kern0pt}X\isactrlbsup A\isactrlesup {\isacharparenright}{\kern0pt}\ A{\isacharparenright}{\kern0pt}\ {\isasymamalg}\ {\isacharparenleft}{\kern0pt}x\ {\isasymcirc}\isactrlsub c\ {\isasymbeta}\isactrlbsub X\isactrlbsup A\isactrlesup \ {\isasymtimes}\isactrlsub c\ {\isacharparenleft}{\kern0pt}B\ {\isasymsetminus}\ {\isacharparenleft}{\kern0pt}A{\isacharcomma}{\kern0pt}\ m{\isacharparenright}{\kern0pt}{\isacharparenright}{\kern0pt}\isactrlesub {\isacharparenright}{\kern0pt}\ {\isasymcirc}\isactrlsub c\isanewline
\ \ \ \ \ \ \ \ \ \ \ \ left{\isacharunderscore}{\kern0pt}coproj\ {\isacharparenleft}{\kern0pt}X\isactrlbsup A\isactrlesup {\isasymtimes}\isactrlsub cA{\isacharparenright}{\kern0pt}\ {\isacharparenleft}{\kern0pt}X\isactrlbsup A\isactrlesup {\isasymtimes}\isactrlsub c{\isacharparenleft}{\kern0pt}B\ {\isasymsetminus}\ {\isacharparenleft}{\kern0pt}A{\isacharcomma}{\kern0pt}m{\isacharparenright}{\kern0pt}{\isacharparenright}{\kern0pt}{\isacharparenright}{\kern0pt}{\isacharparenright}{\kern0pt}\ {\isasymcirc}\isactrlsub c\ {\isasymlangle}g\ {\isasymcirc}\isactrlsub c\ z{\isacharcomma}{\kern0pt}\ a{\isasymrangle}\isanewline
\ \ \ \ \ \ \ \ \ \ {\isacharequal}{\kern0pt}\ {\isacharparenleft}{\kern0pt}{\isacharparenleft}{\kern0pt}eval{\isacharunderscore}{\kern0pt}func\ X\ A\ {\isasymcirc}\isactrlsub c\ swap\ {\isacharparenleft}{\kern0pt}X\isactrlbsup A\isactrlesup {\isacharparenright}{\kern0pt}\ A{\isacharparenright}{\kern0pt}\ {\isasymamalg}\ {\isacharparenleft}{\kern0pt}x\ {\isasymcirc}\isactrlsub c\ {\isasymbeta}\isactrlbsub X\isactrlbsup A\isactrlesup \ {\isasymtimes}\isactrlsub c\ {\isacharparenleft}{\kern0pt}B\ {\isasymsetminus}\ {\isacharparenleft}{\kern0pt}A{\isacharcomma}{\kern0pt}\ m{\isacharparenright}{\kern0pt}{\isacharparenright}{\kern0pt}\isactrlesub {\isacharparenright}{\kern0pt}\ {\isasymcirc}\isactrlsub c\isanewline
\ \ \ \ \ \ \ \ \ \ \ \ left{\isacharunderscore}{\kern0pt}coproj\ {\isacharparenleft}{\kern0pt}X\isactrlbsup A\isactrlesup {\isasymtimes}\isactrlsub cA{\isacharparenright}{\kern0pt}\ {\isacharparenleft}{\kern0pt}X\isactrlbsup A\isactrlesup {\isasymtimes}\isactrlsub c{\isacharparenleft}{\kern0pt}B\ {\isasymsetminus}\ {\isacharparenleft}{\kern0pt}A{\isacharcomma}{\kern0pt}m{\isacharparenright}{\kern0pt}{\isacharparenright}{\kern0pt}{\isacharparenright}{\kern0pt}{\isacharparenright}{\kern0pt}\ {\isasymcirc}\isactrlsub c\ {\isasymlangle}h\ {\isasymcirc}\isactrlsub c\ z{\isacharcomma}{\kern0pt}a{\isasymrangle}{\isachardoublequoteclose}\isanewline
\ \ \ \ \ \ \ \ \ \ \ \ \isacommand{by}\isamarkupfalse%
\ {\isacharparenleft}{\kern0pt}typecheck{\isacharunderscore}{\kern0pt}cfuncs{\isacharunderscore}{\kern0pt}prems{\isacharcomma}{\kern0pt}\ auto\ simp\ add{\isacharcolon}{\kern0pt}\ comp{\isacharunderscore}{\kern0pt}associative{\isadigit{2}}{\isacharparenright}{\kern0pt}\isanewline
\ \ \ \ \ \ \ \ \ \ \isacommand{then}\isamarkupfalse%
\ \isacommand{have}\isamarkupfalse%
\ {\isachardoublequoteopen}{\isacharparenleft}{\kern0pt}eval{\isacharunderscore}{\kern0pt}func\ X\ A\ {\isasymcirc}\isactrlsub c\ swap\ {\isacharparenleft}{\kern0pt}X\isactrlbsup A\isactrlesup {\isacharparenright}{\kern0pt}\ A{\isacharparenright}{\kern0pt}\ {\isasymcirc}\isactrlsub c\ {\isasymlangle}g\ {\isasymcirc}\isactrlsub c\ z{\isacharcomma}{\kern0pt}\ a{\isasymrangle}\isanewline
\ \ \ \ \ \ \ \ \ \ \ \ {\isacharequal}{\kern0pt}\ {\isacharparenleft}{\kern0pt}eval{\isacharunderscore}{\kern0pt}func\ X\ A\ {\isasymcirc}\isactrlsub c\ swap\ {\isacharparenleft}{\kern0pt}X\isactrlbsup A\isactrlesup {\isacharparenright}{\kern0pt}\ A{\isacharparenright}{\kern0pt}\ {\isasymcirc}\isactrlsub c\ {\isasymlangle}h\ {\isasymcirc}\isactrlsub c\ z{\isacharcomma}{\kern0pt}a{\isasymrangle}{\isachardoublequoteclose}\isanewline
\ \ \ \ \ \ \ \ \ \ \ \ \isacommand{by}\isamarkupfalse%
\ {\isacharparenleft}{\kern0pt}typecheck{\isacharunderscore}{\kern0pt}cfuncs{\isacharunderscore}{\kern0pt}prems{\isacharcomma}{\kern0pt}\ auto\ simp\ add{\isacharcolon}{\kern0pt}\ left{\isacharunderscore}{\kern0pt}coproj{\isacharunderscore}{\kern0pt}cfunc{\isacharunderscore}{\kern0pt}coprod{\isacharparenright}{\kern0pt}\isanewline
\ \ \ \ \ \ \ \ \ \ \isacommand{then}\isamarkupfalse%
\ \isacommand{have}\isamarkupfalse%
\ {\isachardoublequoteopen}eval{\isacharunderscore}{\kern0pt}func\ X\ A\ {\isasymcirc}\isactrlsub c\ swap\ {\isacharparenleft}{\kern0pt}X\isactrlbsup A\isactrlesup {\isacharparenright}{\kern0pt}\ A\ {\isasymcirc}\isactrlsub c\ {\isasymlangle}g\ {\isasymcirc}\isactrlsub c\ z{\isacharcomma}{\kern0pt}\ a{\isasymrangle}\isanewline
\ \ \ \ \ \ \ \ \ \ \ \ {\isacharequal}{\kern0pt}\ eval{\isacharunderscore}{\kern0pt}func\ X\ A\ {\isasymcirc}\isactrlsub c\ swap\ {\isacharparenleft}{\kern0pt}X\isactrlbsup A\isactrlesup {\isacharparenright}{\kern0pt}\ A\ {\isasymcirc}\isactrlsub c\ {\isasymlangle}h\ {\isasymcirc}\isactrlsub c\ z{\isacharcomma}{\kern0pt}a{\isasymrangle}{\isachardoublequoteclose}\isanewline
\ \ \ \ \ \ \ \ \ \ \ \ \isacommand{by}\isamarkupfalse%
\ {\isacharparenleft}{\kern0pt}typecheck{\isacharunderscore}{\kern0pt}cfuncs{\isacharunderscore}{\kern0pt}prems{\isacharcomma}{\kern0pt}\ auto\ simp\ add{\isacharcolon}{\kern0pt}\ comp{\isacharunderscore}{\kern0pt}associative{\isadigit{2}}{\isacharparenright}{\kern0pt}\isanewline
\ \ \ \ \ \ \ \ \ \ \isacommand{then}\isamarkupfalse%
\ \isacommand{have}\isamarkupfalse%
\ {\isachardoublequoteopen}eval{\isacharunderscore}{\kern0pt}func\ X\ A\ {\isasymcirc}\isactrlsub c\ {\isasymlangle}a{\isacharcomma}{\kern0pt}\ g\ {\isasymcirc}\isactrlsub c\ z{\isasymrangle}\ {\isacharequal}{\kern0pt}\ eval{\isacharunderscore}{\kern0pt}func\ X\ A\ {\isasymcirc}\isactrlsub c\ {\isasymlangle}a{\isacharcomma}{\kern0pt}\ h\ {\isasymcirc}\isactrlsub c\ z{\isasymrangle}{\isachardoublequoteclose}\isanewline
\ \ \ \ \ \ \ \ \ \ \ \ \isacommand{by}\isamarkupfalse%
\ {\isacharparenleft}{\kern0pt}typecheck{\isacharunderscore}{\kern0pt}cfuncs{\isacharunderscore}{\kern0pt}prems{\isacharcomma}{\kern0pt}\ auto\ simp\ add{\isacharcolon}{\kern0pt}\ swap{\isacharunderscore}{\kern0pt}ap{\isacharparenright}{\kern0pt}\isanewline
\ \ \ \ \ \ \ \ \ \ \isacommand{then}\isamarkupfalse%
\ \isacommand{have}\isamarkupfalse%
\ {\isachardoublequoteopen}eval{\isacharunderscore}{\kern0pt}func\ X\ A\ {\isasymcirc}\isactrlsub c\ {\isacharparenleft}{\kern0pt}id\ A\ {\isasymtimes}\isactrlsub f\ g{\isacharparenright}{\kern0pt}\ {\isasymcirc}\isactrlsub c\ {\isasymlangle}a{\isacharcomma}{\kern0pt}\ z{\isasymrangle}\ {\isacharequal}{\kern0pt}\ eval{\isacharunderscore}{\kern0pt}func\ X\ A\ {\isasymcirc}\isactrlsub c\ {\isacharparenleft}{\kern0pt}id\ A\ {\isasymtimes}\isactrlsub f\ h{\isacharparenright}{\kern0pt}\ {\isasymcirc}\isactrlsub c\ {\isasymlangle}a{\isacharcomma}{\kern0pt}\ z{\isasymrangle}{\isachardoublequoteclose}\isanewline
\ \ \ \ \ \ \ \ \ \ \ \ \isacommand{by}\isamarkupfalse%
\ {\isacharparenleft}{\kern0pt}typecheck{\isacharunderscore}{\kern0pt}cfuncs{\isacharcomma}{\kern0pt}\ simp\ add{\isacharcolon}{\kern0pt}\ cfunc{\isacharunderscore}{\kern0pt}cross{\isacharunderscore}{\kern0pt}prod{\isacharunderscore}{\kern0pt}comp{\isacharunderscore}{\kern0pt}cfunc{\isacharunderscore}{\kern0pt}prod\ id{\isacharunderscore}{\kern0pt}left{\isacharunderscore}{\kern0pt}unit{\isadigit{2}}{\isacharparenright}{\kern0pt}\isanewline
\ \ \ \ \ \ \ \ \ \ \isacommand{then}\isamarkupfalse%
\ \isacommand{show}\isamarkupfalse%
\ {\isachardoublequoteopen}{\isacharparenleft}{\kern0pt}eval{\isacharunderscore}{\kern0pt}func\ X\ A\ {\isasymcirc}\isactrlsub c\ id\isactrlsub c\ A\ {\isasymtimes}\isactrlsub f\ g{\isacharparenright}{\kern0pt}\ {\isasymcirc}\isactrlsub c\ az\ {\isacharequal}{\kern0pt}\ {\isacharparenleft}{\kern0pt}eval{\isacharunderscore}{\kern0pt}func\ X\ A\ {\isasymcirc}\isactrlsub c\ id\isactrlsub c\ A\ {\isasymtimes}\isactrlsub f\ h{\isacharparenright}{\kern0pt}\ {\isasymcirc}\isactrlsub c\ az{\isachardoublequoteclose}\isanewline
\ \ \ \ \ \ \ \ \ \ \ \ \isacommand{unfolding}\isamarkupfalse%
\ az{\isacharunderscore}{\kern0pt}def\ \isacommand{by}\isamarkupfalse%
\ {\isacharparenleft}{\kern0pt}typecheck{\isacharunderscore}{\kern0pt}cfuncs{\isacharunderscore}{\kern0pt}prems{\isacharcomma}{\kern0pt}\ auto\ simp\ add{\isacharcolon}{\kern0pt}\ comp{\isacharunderscore}{\kern0pt}associative{\isadigit{2}}{\isacharparenright}{\kern0pt}\isanewline
\ \ \ \ \ \ \ \ \isacommand{qed}\isamarkupfalse%
\isanewline
\ \ \ \ \ \ \isacommand{qed}\isamarkupfalse%
\isanewline
\ \ \ \ \isacommand{qed}\isamarkupfalse%
\isanewline
\ \ \isacommand{qed}\isamarkupfalse%
\isanewline
\isacommand{qed}\isamarkupfalse%
%
\endisatagproof
{\isafoldproof}%
%
\isadelimproof
\isanewline
%
\endisadelimproof
\isanewline
\isacommand{lemma}\isamarkupfalse%
\ exp{\isacharunderscore}{\kern0pt}preserves{\isacharunderscore}{\kern0pt}card{\isadigit{2}}{\isacharcolon}{\kern0pt}\isanewline
\ \ \isakeyword{assumes}\ {\isachardoublequoteopen}A\ {\isasymle}\isactrlsub c\ B{\isachardoublequoteclose}\isanewline
\ \ \isakeyword{shows}\ {\isachardoublequoteopen}A\isactrlbsup X\isactrlesup \ {\isasymle}\isactrlsub c\ B\isactrlbsup X\isactrlesup {\isachardoublequoteclose}\isanewline
%
\isadelimproof
\ \ %
\endisadelimproof
%
\isatagproof
\isacommand{unfolding}\isamarkupfalse%
\ is{\isacharunderscore}{\kern0pt}smaller{\isacharunderscore}{\kern0pt}than{\isacharunderscore}{\kern0pt}def\isanewline
\isacommand{proof}\isamarkupfalse%
\ {\isacharminus}{\kern0pt}\isanewline
\ \ \isacommand{obtain}\isamarkupfalse%
\ m\ \isakeyword{where}\ m{\isacharunderscore}{\kern0pt}def{\isacharbrackleft}{\kern0pt}type{\isacharunderscore}{\kern0pt}rule{\isacharbrackright}{\kern0pt}{\isacharcolon}{\kern0pt}\ {\isachardoublequoteopen}m\ {\isacharcolon}{\kern0pt}\ A\ {\isasymrightarrow}\ B{\isachardoublequoteclose}\ {\isachardoublequoteopen}monomorphism\ m{\isachardoublequoteclose}\isanewline
\ \ \ \ \ \ \ \ \isacommand{using}\isamarkupfalse%
\ assms\ \isacommand{unfolding}\isamarkupfalse%
\ is{\isacharunderscore}{\kern0pt}smaller{\isacharunderscore}{\kern0pt}than{\isacharunderscore}{\kern0pt}def\ \isacommand{by}\isamarkupfalse%
\ auto\isanewline
\ \ \isacommand{show}\isamarkupfalse%
\ {\isachardoublequoteopen}{\isasymexists}m{\isachardot}{\kern0pt}\ m\ {\isacharcolon}{\kern0pt}\ A\isactrlbsup X\isactrlesup \ {\isasymrightarrow}\ B\isactrlbsup X\isactrlesup \ {\isasymand}\ monomorphism\ m{\isachardoublequoteclose}\isanewline
\ \ \isacommand{proof}\isamarkupfalse%
\ {\isacharparenleft}{\kern0pt}intro\ exI{\isacharbrackleft}{\kern0pt}\isakeyword{where}\ x{\isacharequal}{\kern0pt}{\isachardoublequoteopen}{\isacharparenleft}{\kern0pt}m\ {\isasymcirc}\isactrlsub c\ eval{\isacharunderscore}{\kern0pt}func\ A\ X{\isacharparenright}{\kern0pt}\isactrlsup {\isasymsharp}{\isachardoublequoteclose}{\isacharbrackright}{\kern0pt}{\isacharcomma}{\kern0pt}\ safe{\isacharparenright}{\kern0pt}\isanewline
\ \ \ \ \isacommand{show}\isamarkupfalse%
\ {\isachardoublequoteopen}{\isacharparenleft}{\kern0pt}m\ {\isasymcirc}\isactrlsub c\ eval{\isacharunderscore}{\kern0pt}func\ A\ X{\isacharparenright}{\kern0pt}\isactrlsup {\isasymsharp}\ {\isacharcolon}{\kern0pt}\ A\isactrlbsup X\isactrlesup \ {\isasymrightarrow}\ B\isactrlbsup X\isactrlesup {\isachardoublequoteclose}\isanewline
\ \ \ \ \ \ \isacommand{by}\isamarkupfalse%
\ typecheck{\isacharunderscore}{\kern0pt}cfuncs\isanewline
\ \ \ \ \isacommand{then}\isamarkupfalse%
\ \isacommand{show}\isamarkupfalse%
\ {\isachardoublequoteopen}monomorphism{\isacharparenleft}{\kern0pt}{\isacharparenleft}{\kern0pt}m\ {\isasymcirc}\isactrlsub c\ eval{\isacharunderscore}{\kern0pt}func\ A\ X{\isacharparenright}{\kern0pt}\isactrlsup {\isasymsharp}{\isacharparenright}{\kern0pt}{\isachardoublequoteclose}\isanewline
\ \ \ \ \isacommand{proof}\isamarkupfalse%
\ {\isacharparenleft}{\kern0pt}unfold\ monomorphism{\isacharunderscore}{\kern0pt}def{\isadigit{3}}{\isacharcomma}{\kern0pt}\ clarify{\isacharparenright}{\kern0pt}\isanewline
\ \ \ \ \ \ \isacommand{fix}\isamarkupfalse%
\ g\ h\ Z\isanewline
\ \ \ \ \ \ \isacommand{assume}\isamarkupfalse%
\ g{\isacharunderscore}{\kern0pt}type{\isacharbrackleft}{\kern0pt}type{\isacharunderscore}{\kern0pt}rule{\isacharbrackright}{\kern0pt}{\isacharcolon}{\kern0pt}\ {\isachardoublequoteopen}g\ {\isacharcolon}{\kern0pt}\ Z\ {\isasymrightarrow}\ A\isactrlbsup X\isactrlesup {\isachardoublequoteclose}\isanewline
\ \ \ \ \ \ \isacommand{assume}\isamarkupfalse%
\ h{\isacharunderscore}{\kern0pt}type{\isacharbrackleft}{\kern0pt}type{\isacharunderscore}{\kern0pt}rule{\isacharbrackright}{\kern0pt}{\isacharcolon}{\kern0pt}\ {\isachardoublequoteopen}h\ {\isacharcolon}{\kern0pt}\ Z\ {\isasymrightarrow}\ A\isactrlbsup X\isactrlesup {\isachardoublequoteclose}\isanewline
\isanewline
\ \ \ \ \ \ \isacommand{assume}\isamarkupfalse%
\ eq{\isacharcolon}{\kern0pt}\ {\isachardoublequoteopen}{\isacharparenleft}{\kern0pt}m\ {\isasymcirc}\isactrlsub c\ eval{\isacharunderscore}{\kern0pt}func\ A\ X{\isacharparenright}{\kern0pt}\isactrlsup {\isasymsharp}\ {\isasymcirc}\isactrlsub c\ g\ {\isacharequal}{\kern0pt}\ {\isacharparenleft}{\kern0pt}m\ {\isasymcirc}\isactrlsub c\ eval{\isacharunderscore}{\kern0pt}func\ A\ X{\isacharparenright}{\kern0pt}\isactrlsup {\isasymsharp}\ {\isasymcirc}\isactrlsub c\ h{\isachardoublequoteclose}\isanewline
\ \ \ \ \ \ \isacommand{show}\isamarkupfalse%
\ {\isachardoublequoteopen}g\ {\isacharequal}{\kern0pt}\ h{\isachardoublequoteclose}\isanewline
\ \ \ \ \ \ \isacommand{proof}\isamarkupfalse%
\ {\isacharparenleft}{\kern0pt}typecheck{\isacharunderscore}{\kern0pt}cfuncs{\isacharcomma}{\kern0pt}\ rule\ same{\isacharunderscore}{\kern0pt}evals{\isacharunderscore}{\kern0pt}equal{\isacharbrackleft}{\kern0pt}\isakeyword{where}\ Z{\isacharequal}{\kern0pt}Z{\isacharcomma}{\kern0pt}\ \isakeyword{where}\ A{\isacharequal}{\kern0pt}X{\isacharcomma}{\kern0pt}\ \isakeyword{where}\ X{\isacharequal}{\kern0pt}A{\isacharbrackright}{\kern0pt}{\isacharcomma}{\kern0pt}\ clarify{\isacharparenright}{\kern0pt}\isanewline
\ \ \ \ \ \ \ \ \ \ \isacommand{have}\isamarkupfalse%
\ {\isachardoublequoteopen}{\isacharparenleft}{\kern0pt}{\isacharparenleft}{\kern0pt}eval{\isacharunderscore}{\kern0pt}func\ B\ X{\isacharparenright}{\kern0pt}\ {\isasymcirc}\isactrlsub c\ {\isacharparenleft}{\kern0pt}id\ X\ {\isasymtimes}\isactrlsub f\ {\isacharparenleft}{\kern0pt}m\ {\isasymcirc}\isactrlsub c\ eval{\isacharunderscore}{\kern0pt}func\ A\ X{\isacharparenright}{\kern0pt}\isactrlsup {\isasymsharp}{\isacharparenright}{\kern0pt}{\isacharparenright}{\kern0pt}\ {\isasymcirc}\isactrlsub c\ {\isacharparenleft}{\kern0pt}id\ X\ {\isasymtimes}\isactrlsub f\ g{\isacharparenright}{\kern0pt}\ \ {\isacharequal}{\kern0pt}\ \isanewline
\ \ \ \ \ \ \ \ \ \ \ \ \ \ \ \ {\isacharparenleft}{\kern0pt}{\isacharparenleft}{\kern0pt}eval{\isacharunderscore}{\kern0pt}func\ B\ X{\isacharparenright}{\kern0pt}\ {\isasymcirc}\isactrlsub c\ {\isacharparenleft}{\kern0pt}id\ X\ {\isasymtimes}\isactrlsub f\ {\isacharparenleft}{\kern0pt}m\ {\isasymcirc}\isactrlsub c\ eval{\isacharunderscore}{\kern0pt}func\ A\ X{\isacharparenright}{\kern0pt}\isactrlsup {\isasymsharp}{\isacharparenright}{\kern0pt}{\isacharparenright}{\kern0pt}\ {\isasymcirc}\isactrlsub c\ {\isacharparenleft}{\kern0pt}id\ X\ {\isasymtimes}\isactrlsub f\ h{\isacharparenright}{\kern0pt}{\isachardoublequoteclose}\isanewline
\ \ \ \ \ \ \ \ \ \ \ \ \isacommand{by}\isamarkupfalse%
\ {\isacharparenleft}{\kern0pt}typecheck{\isacharunderscore}{\kern0pt}cfuncs{\isacharcomma}{\kern0pt}\ smt\ comp{\isacharunderscore}{\kern0pt}associative{\isadigit{2}}\ eq\ inv{\isacharunderscore}{\kern0pt}transpose{\isacharunderscore}{\kern0pt}func{\isacharunderscore}{\kern0pt}def{\isadigit{3}}\ inv{\isacharunderscore}{\kern0pt}transpose{\isacharunderscore}{\kern0pt}of{\isacharunderscore}{\kern0pt}composition{\isacharparenright}{\kern0pt}\isanewline
\ \ \ \ \ \ \ \ \ \ \isacommand{then}\isamarkupfalse%
\ \isacommand{have}\isamarkupfalse%
\ {\isachardoublequoteopen}{\isacharparenleft}{\kern0pt}m\ {\isasymcirc}\isactrlsub c\ eval{\isacharunderscore}{\kern0pt}func\ A\ X{\isacharparenright}{\kern0pt}\ {\isasymcirc}\isactrlsub c\ {\isacharparenleft}{\kern0pt}id\ X\ {\isasymtimes}\isactrlsub f\ g{\isacharparenright}{\kern0pt}\ \ {\isacharequal}{\kern0pt}\ {\isacharparenleft}{\kern0pt}m\ {\isasymcirc}\isactrlsub c\ eval{\isacharunderscore}{\kern0pt}func\ A\ X{\isacharparenright}{\kern0pt}\ {\isasymcirc}\isactrlsub c\ {\isacharparenleft}{\kern0pt}id\ X\ {\isasymtimes}\isactrlsub f\ h{\isacharparenright}{\kern0pt}{\isachardoublequoteclose}\isanewline
\ \ \ \ \ \ \ \ \ \ \ \ \isacommand{by}\isamarkupfalse%
\ {\isacharparenleft}{\kern0pt}smt\ comp{\isacharunderscore}{\kern0pt}type\ eval{\isacharunderscore}{\kern0pt}func{\isacharunderscore}{\kern0pt}type\ m{\isacharunderscore}{\kern0pt}def{\isacharparenleft}{\kern0pt}{\isadigit{1}}{\isacharparenright}{\kern0pt}\ transpose{\isacharunderscore}{\kern0pt}func{\isacharunderscore}{\kern0pt}def{\isacharparenright}{\kern0pt}\isanewline
\ \ \ \ \ \ \ \ \ \ \isacommand{then}\isamarkupfalse%
\ \isacommand{have}\isamarkupfalse%
\ {\isachardoublequoteopen}m\ {\isasymcirc}\isactrlsub c\ {\isacharparenleft}{\kern0pt}eval{\isacharunderscore}{\kern0pt}func\ A\ X\ {\isasymcirc}\isactrlsub c\ {\isacharparenleft}{\kern0pt}id\ X\ {\isasymtimes}\isactrlsub f\ g{\isacharparenright}{\kern0pt}{\isacharparenright}{\kern0pt}\ \ {\isacharequal}{\kern0pt}\ m\ {\isasymcirc}\isactrlsub c\ {\isacharparenleft}{\kern0pt}eval{\isacharunderscore}{\kern0pt}func\ A\ X\ {\isasymcirc}\isactrlsub c\ {\isacharparenleft}{\kern0pt}id\ X\ {\isasymtimes}\isactrlsub f\ h{\isacharparenright}{\kern0pt}{\isacharparenright}{\kern0pt}{\isachardoublequoteclose}\isanewline
\ \ \ \ \ \ \ \ \ \ \ \ \isacommand{by}\isamarkupfalse%
\ {\isacharparenleft}{\kern0pt}typecheck{\isacharunderscore}{\kern0pt}cfuncs{\isacharcomma}{\kern0pt}\ smt\ comp{\isacharunderscore}{\kern0pt}associative{\isadigit{2}}{\isacharparenright}{\kern0pt}\isanewline
\ \ \ \ \ \ \ \ \ \ \isacommand{then}\isamarkupfalse%
\ \isacommand{have}\isamarkupfalse%
\ {\isachardoublequoteopen}eval{\isacharunderscore}{\kern0pt}func\ A\ X\ {\isasymcirc}\isactrlsub c\ {\isacharparenleft}{\kern0pt}id\ X\ {\isasymtimes}\isactrlsub f\ g{\isacharparenright}{\kern0pt}\ \ {\isacharequal}{\kern0pt}\ eval{\isacharunderscore}{\kern0pt}func\ A\ X\ {\isasymcirc}\isactrlsub c\ {\isacharparenleft}{\kern0pt}id\ X\ {\isasymtimes}\isactrlsub f\ h{\isacharparenright}{\kern0pt}{\isachardoublequoteclose}\isanewline
\ \ \ \ \ \ \ \ \ \ \ \ \isacommand{using}\isamarkupfalse%
\ m{\isacharunderscore}{\kern0pt}def\ monomorphism{\isacharunderscore}{\kern0pt}def{\isadigit{3}}\ \isacommand{by}\isamarkupfalse%
\ {\isacharparenleft}{\kern0pt}typecheck{\isacharunderscore}{\kern0pt}cfuncs{\isacharcomma}{\kern0pt}\ blast{\isacharparenright}{\kern0pt}\isanewline
\ \ \ \ \ \ \ \ \ \ \isacommand{then}\isamarkupfalse%
\ \isacommand{show}\isamarkupfalse%
\ {\isachardoublequoteopen}{\isacharparenleft}{\kern0pt}eval{\isacharunderscore}{\kern0pt}func\ A\ X\ {\isasymcirc}\isactrlsub c\ {\isacharparenleft}{\kern0pt}id\ X\ {\isasymtimes}\isactrlsub f\ g{\isacharparenright}{\kern0pt}{\isacharparenright}{\kern0pt}\ \ {\isacharequal}{\kern0pt}\ {\isacharparenleft}{\kern0pt}eval{\isacharunderscore}{\kern0pt}func\ A\ X\ {\isasymcirc}\isactrlsub c\ {\isacharparenleft}{\kern0pt}id\ X\ {\isasymtimes}\isactrlsub f\ h{\isacharparenright}{\kern0pt}{\isacharparenright}{\kern0pt}{\isachardoublequoteclose}\isanewline
\ \ \ \ \ \ \ \ \ \ \ \ \isacommand{by}\isamarkupfalse%
\ {\isacharparenleft}{\kern0pt}typecheck{\isacharunderscore}{\kern0pt}cfuncs{\isacharcomma}{\kern0pt}\ smt\ comp{\isacharunderscore}{\kern0pt}associative{\isadigit{2}}{\isacharparenright}{\kern0pt}\isanewline
\ \ \ \ \ \ \isacommand{qed}\isamarkupfalse%
\isanewline
\ \ \ \ \isacommand{qed}\isamarkupfalse%
\isanewline
\ \ \isacommand{qed}\isamarkupfalse%
\isanewline
\isacommand{qed}\isamarkupfalse%
%
\endisatagproof
{\isafoldproof}%
%
\isadelimproof
\isanewline
%
\endisadelimproof
\isanewline
\isacommand{lemma}\isamarkupfalse%
\ exp{\isacharunderscore}{\kern0pt}preserves{\isacharunderscore}{\kern0pt}card{\isadigit{3}}{\isacharcolon}{\kern0pt}\isanewline
\ \ \isakeyword{assumes}\ {\isachardoublequoteopen}A\ {\isasymle}\isactrlsub c\ B{\isachardoublequoteclose}\isanewline
\ \ \isakeyword{assumes}\ {\isachardoublequoteopen}X\ {\isasymle}\isactrlsub c\ Y{\isachardoublequoteclose}\isanewline
\ \ \isakeyword{assumes}\ {\isachardoublequoteopen}nonempty{\isacharparenleft}{\kern0pt}X{\isacharparenright}{\kern0pt}{\isachardoublequoteclose}\isanewline
\ \ \isakeyword{shows}\ {\isachardoublequoteopen}X\isactrlbsup A\isactrlesup \ {\isasymle}\isactrlsub c\ Y\isactrlbsup B\isactrlesup {\isachardoublequoteclose}\isanewline
%
\isadelimproof
%
\endisadelimproof
%
\isatagproof
\isacommand{proof}\isamarkupfalse%
\ {\isacharminus}{\kern0pt}\ \isanewline
\ \ \isacommand{have}\isamarkupfalse%
\ leq{\isadigit{1}}{\isacharcolon}{\kern0pt}\ {\isachardoublequoteopen}X\isactrlbsup A\isactrlesup \ {\isasymle}\isactrlsub c\ X\isactrlbsup B\isactrlesup {\isachardoublequoteclose}\isanewline
\ \ \ \ \isacommand{by}\isamarkupfalse%
\ {\isacharparenleft}{\kern0pt}simp\ add{\isacharcolon}{\kern0pt}\ assms{\isacharparenleft}{\kern0pt}{\isadigit{1}}{\isacharcomma}{\kern0pt}{\isadigit{3}}{\isacharparenright}{\kern0pt}\ exp{\isacharunderscore}{\kern0pt}preserves{\isacharunderscore}{\kern0pt}card{\isadigit{1}}{\isacharparenright}{\kern0pt}\ \ \ \ \isanewline
\ \ \isacommand{have}\isamarkupfalse%
\ leq{\isadigit{2}}{\isacharcolon}{\kern0pt}\ {\isachardoublequoteopen}X\isactrlbsup B\isactrlesup \ {\isasymle}\isactrlsub c\ Y\isactrlbsup B\isactrlesup {\isachardoublequoteclose}\isanewline
\ \ \ \ \isacommand{by}\isamarkupfalse%
\ {\isacharparenleft}{\kern0pt}simp\ add{\isacharcolon}{\kern0pt}\ assms{\isacharparenleft}{\kern0pt}{\isadigit{2}}{\isacharparenright}{\kern0pt}\ exp{\isacharunderscore}{\kern0pt}preserves{\isacharunderscore}{\kern0pt}card{\isadigit{2}}{\isacharparenright}{\kern0pt}\isanewline
\ \ \isacommand{show}\isamarkupfalse%
\ {\isachardoublequoteopen}X\isactrlbsup A\isactrlesup \ {\isasymle}\isactrlsub c\ Y\isactrlbsup B\isactrlesup {\isachardoublequoteclose}\isanewline
\ \ \ \ \isacommand{using}\isamarkupfalse%
\ leq{\isadigit{1}}\ leq{\isadigit{2}}\ set{\isacharunderscore}{\kern0pt}card{\isacharunderscore}{\kern0pt}transitive\ \isacommand{by}\isamarkupfalse%
\ blast\isanewline
\isacommand{qed}\isamarkupfalse%
%
\endisatagproof
{\isafoldproof}%
%
\isadelimproof
\isanewline
%
\endisadelimproof
%
\isadelimtheory
\isanewline
%
\endisadelimtheory
%
\isatagtheory
\isacommand{end}\isamarkupfalse%
%
\endisatagtheory
{\isafoldtheory}%
%
\isadelimtheory
%
\endisadelimtheory
%
\end{isabellebody}%
\endinput
%:%file=~/ETCS/Category_Set/Cardinality.thy%:%
%:%11=1%:%
%:%27=3%:%
%:%28=3%:%
%:%29=4%:%
%:%30=5%:%
%:%39=7%:%
%:%41=8%:%
%:%42=8%:%
%:%43=9%:%
%:%44=10%:%
%:%45=11%:%
%:%46=11%:%
%:%47=12%:%
%:%48=13%:%
%:%49=14%:%
%:%50=14%:%
%:%51=15%:%
%:%54=16%:%
%:%58=16%:%
%:%59=16%:%
%:%60=16%:%
%:%69=18%:%
%:%71=19%:%
%:%72=19%:%
%:%73=20%:%
%:%75=22%:%
%:%77=23%:%
%:%78=23%:%
%:%79=24%:%
%:%82=25%:%
%:%86=25%:%
%:%87=25%:%
%:%88=25%:%
%:%93=25%:%
%:%96=26%:%
%:%97=27%:%
%:%98=27%:%
%:%99=28%:%
%:%100=29%:%
%:%101=30%:%
%:%104=31%:%
%:%108=31%:%
%:%109=31%:%
%:%114=31%:%
%:%117=32%:%
%:%118=33%:%
%:%119=33%:%
%:%120=34%:%
%:%121=35%:%
%:%124=36%:%
%:%128=36%:%
%:%129=36%:%
%:%134=36%:%
%:%137=37%:%
%:%138=38%:%
%:%139=38%:%
%:%140=39%:%
%:%143=40%:%
%:%147=40%:%
%:%148=40%:%
%:%149=40%:%
%:%154=40%:%
%:%157=41%:%
%:%158=42%:%
%:%159=42%:%
%:%160=43%:%
%:%163=44%:%
%:%167=44%:%
%:%168=44%:%
%:%169=45%:%
%:%170=45%:%
%:%171=46%:%
%:%172=46%:%
%:%173=47%:%
%:%174=47%:%
%:%175=48%:%
%:%176=48%:%
%:%177=49%:%
%:%178=49%:%
%:%179=50%:%
%:%180=50%:%
%:%181=51%:%
%:%182=51%:%
%:%183=52%:%
%:%184=52%:%
%:%185=53%:%
%:%186=53%:%
%:%187=54%:%
%:%188=54%:%
%:%189=54%:%
%:%190=55%:%
%:%191=55%:%
%:%192=55%:%
%:%193=56%:%
%:%194=56%:%
%:%195=56%:%
%:%196=57%:%
%:%197=57%:%
%:%198=58%:%
%:%199=58%:%
%:%200=59%:%
%:%201=59%:%
%:%202=59%:%
%:%203=60%:%
%:%204=60%:%
%:%205=61%:%
%:%211=61%:%
%:%214=62%:%
%:%215=63%:%
%:%216=63%:%
%:%217=64%:%
%:%218=65%:%
%:%221=66%:%
%:%225=66%:%
%:%226=66%:%
%:%227=67%:%
%:%228=67%:%
%:%229=68%:%
%:%230=68%:%
%:%231=69%:%
%:%232=69%:%
%:%233=70%:%
%:%234=70%:%
%:%235=71%:%
%:%236=72%:%
%:%237=72%:%
%:%238=73%:%
%:%239=73%:%
%:%240=73%:%
%:%241=74%:%
%:%242=74%:%
%:%243=75%:%
%:%244=75%:%
%:%245=76%:%
%:%246=77%:%
%:%247=77%:%
%:%248=78%:%
%:%249=78%:%
%:%250=79%:%
%:%251=79%:%
%:%252=79%:%
%:%253=80%:%
%:%254=81%:%
%:%255=81%:%
%:%256=82%:%
%:%257=82%:%
%:%258=82%:%
%:%259=83%:%
%:%260=83%:%
%:%261=83%:%
%:%262=84%:%
%:%263=84%:%
%:%264=84%:%
%:%265=85%:%
%:%266=85%:%
%:%267=86%:%
%:%268=86%:%
%:%269=86%:%
%:%270=87%:%
%:%271=87%:%
%:%272=88%:%
%:%273=88%:%
%:%274=88%:%
%:%275=89%:%
%:%276=89%:%
%:%277=89%:%
%:%278=90%:%
%:%279=90%:%
%:%280=91%:%
%:%281=91%:%
%:%282=92%:%
%:%283=92%:%
%:%284=93%:%
%:%285=93%:%
%:%286=93%:%
%:%287=94%:%
%:%288=94%:%
%:%289=94%:%
%:%290=95%:%
%:%291=95%:%
%:%292=96%:%
%:%293=96%:%
%:%294=97%:%
%:%295=97%:%
%:%296=98%:%
%:%297=98%:%
%:%298=99%:%
%:%299=99%:%
%:%300=100%:%
%:%301=100%:%
%:%302=101%:%
%:%303=101%:%
%:%304=101%:%
%:%305=102%:%
%:%306=102%:%
%:%307=102%:%
%:%308=103%:%
%:%309=103%:%
%:%310=103%:%
%:%311=104%:%
%:%312=104%:%
%:%313=104%:%
%:%314=105%:%
%:%315=105%:%
%:%316=105%:%
%:%317=106%:%
%:%318=106%:%
%:%319=106%:%
%:%320=107%:%
%:%321=107%:%
%:%322=108%:%
%:%328=108%:%
%:%331=109%:%
%:%332=110%:%
%:%333=110%:%
%:%334=111%:%
%:%335=112%:%
%:%338=113%:%
%:%342=113%:%
%:%343=113%:%
%:%344=114%:%
%:%345=114%:%
%:%350=114%:%
%:%353=115%:%
%:%354=116%:%
%:%355=116%:%
%:%356=117%:%
%:%357=118%:%
%:%358=119%:%
%:%361=120%:%
%:%365=120%:%
%:%366=120%:%
%:%367=120%:%
%:%372=120%:%
%:%375=121%:%
%:%376=122%:%
%:%377=122%:%
%:%378=123%:%
%:%381=124%:%
%:%385=124%:%
%:%386=124%:%
%:%387=124%:%
%:%392=124%:%
%:%395=125%:%
%:%396=126%:%
%:%397=126%:%
%:%398=127%:%
%:%399=128%:%
%:%400=129%:%
%:%403=130%:%
%:%407=130%:%
%:%408=130%:%
%:%409=130%:%
%:%414=130%:%
%:%417=131%:%
%:%418=132%:%
%:%419=132%:%
%:%420=133%:%
%:%427=134%:%
%:%428=134%:%
%:%429=135%:%
%:%430=135%:%
%:%431=136%:%
%:%432=136%:%
%:%433=136%:%
%:%434=137%:%
%:%435=137%:%
%:%436=137%:%
%:%437=138%:%
%:%438=138%:%
%:%439=139%:%
%:%440=140%:%
%:%441=141%:%
%:%442=141%:%
%:%443=142%:%
%:%444=142%:%
%:%445=142%:%
%:%446=143%:%
%:%447=143%:%
%:%448=144%:%
%:%449=144%:%
%:%450=145%:%
%:%451=145%:%
%:%452=146%:%
%:%453=146%:%
%:%454=146%:%
%:%455=147%:%
%:%456=147%:%
%:%457=148%:%
%:%458=148%:%
%:%459=149%:%
%:%460=149%:%
%:%461=150%:%
%:%462=150%:%
%:%463=151%:%
%:%464=151%:%
%:%465=152%:%
%:%466=153%:%
%:%467=153%:%
%:%468=154%:%
%:%469=154%:%
%:%470=155%:%
%:%471=156%:%
%:%472=156%:%
%:%473=156%:%
%:%474=157%:%
%:%475=157%:%
%:%476=158%:%
%:%477=158%:%
%:%478=158%:%
%:%479=159%:%
%:%480=159%:%
%:%481=159%:%
%:%482=160%:%
%:%488=160%:%
%:%491=161%:%
%:%492=162%:%
%:%493=162%:%
%:%494=163%:%
%:%501=164%:%
%:%502=164%:%
%:%503=165%:%
%:%504=165%:%
%:%505=166%:%
%:%506=166%:%
%:%507=167%:%
%:%508=167%:%
%:%509=168%:%
%:%510=168%:%
%:%511=169%:%
%:%512=169%:%
%:%513=169%:%
%:%514=170%:%
%:%515=171%:%
%:%516=172%:%
%:%517=172%:%
%:%518=173%:%
%:%519=173%:%
%:%520=174%:%
%:%521=174%:%
%:%522=175%:%
%:%523=175%:%
%:%524=176%:%
%:%525=176%:%
%:%526=177%:%
%:%527=178%:%
%:%528=178%:%
%:%529=178%:%
%:%530=179%:%
%:%531=179%:%
%:%532=179%:%
%:%533=180%:%
%:%539=180%:%
%:%542=181%:%
%:%543=182%:%
%:%544=182%:%
%:%545=183%:%
%:%548=184%:%
%:%552=184%:%
%:%553=184%:%
%:%558=184%:%
%:%561=185%:%
%:%562=186%:%
%:%563=186%:%
%:%564=187%:%
%:%567=188%:%
%:%571=188%:%
%:%572=188%:%
%:%581=190%:%
%:%583=191%:%
%:%584=191%:%
%:%585=192%:%
%:%588=193%:%
%:%592=193%:%
%:%593=193%:%
%:%594=193%:%
%:%599=193%:%
%:%602=194%:%
%:%603=195%:%
%:%604=195%:%
%:%605=196%:%
%:%608=197%:%
%:%612=197%:%
%:%613=197%:%
%:%614=197%:%
%:%623=199%:%
%:%625=200%:%
%:%626=200%:%
%:%627=201%:%
%:%628=202%:%
%:%631=203%:%
%:%635=203%:%
%:%636=203%:%
%:%637=204%:%
%:%638=204%:%
%:%639=205%:%
%:%640=205%:%
%:%641=206%:%
%:%642=206%:%
%:%643=206%:%
%:%644=207%:%
%:%645=207%:%
%:%646=208%:%
%:%647=208%:%
%:%648=208%:%
%:%649=209%:%
%:%650=209%:%
%:%651=210%:%
%:%652=210%:%
%:%653=211%:%
%:%654=211%:%
%:%655=212%:%
%:%656=212%:%
%:%657=213%:%
%:%658=213%:%
%:%659=214%:%
%:%660=215%:%
%:%661=215%:%
%:%662=216%:%
%:%663=216%:%
%:%664=216%:%
%:%665=217%:%
%:%666=217%:%
%:%667=217%:%
%:%668=218%:%
%:%669=218%:%
%:%670=218%:%
%:%671=219%:%
%:%672=219%:%
%:%673=219%:%
%:%674=220%:%
%:%675=220%:%
%:%676=220%:%
%:%677=221%:%
%:%678=221%:%
%:%679=222%:%
%:%680=222%:%
%:%681=223%:%
%:%682=223%:%
%:%683=224%:%
%:%684=224%:%
%:%685=225%:%
%:%686=225%:%
%:%687=225%:%
%:%688=226%:%
%:%694=226%:%
%:%697=227%:%
%:%698=228%:%
%:%699=228%:%
%:%700=229%:%
%:%701=230%:%
%:%704=231%:%
%:%708=231%:%
%:%709=231%:%
%:%710=232%:%
%:%711=232%:%
%:%712=233%:%
%:%713=233%:%
%:%714=234%:%
%:%715=234%:%
%:%716=235%:%
%:%717=235%:%
%:%718=235%:%
%:%719=236%:%
%:%720=236%:%
%:%721=236%:%
%:%722=237%:%
%:%723=237%:%
%:%724=238%:%
%:%725=238%:%
%:%726=238%:%
%:%727=239%:%
%:%728=239%:%
%:%729=239%:%
%:%730=240%:%
%:%731=240%:%
%:%732=240%:%
%:%733=241%:%
%:%734=241%:%
%:%735=241%:%
%:%736=242%:%
%:%737=242%:%
%:%738=243%:%
%:%744=243%:%
%:%747=244%:%
%:%748=245%:%
%:%749=245%:%
%:%750=246%:%
%:%751=247%:%
%:%752=248%:%
%:%753=249%:%
%:%754=250%:%
%:%761=251%:%
%:%762=251%:%
%:%763=252%:%
%:%764=252%:%
%:%765=253%:%
%:%766=253%:%
%:%767=253%:%
%:%768=254%:%
%:%769=254%:%
%:%770=255%:%
%:%771=255%:%
%:%772=255%:%
%:%773=256%:%
%:%774=256%:%
%:%775=256%:%
%:%776=257%:%
%:%777=257%:%
%:%778=257%:%
%:%779=258%:%
%:%780=258%:%
%:%781=259%:%
%:%782=259%:%
%:%783=260%:%
%:%784=260%:%
%:%785=261%:%
%:%786=261%:%
%:%787=262%:%
%:%788=262%:%
%:%789=263%:%
%:%790=263%:%
%:%791=264%:%
%:%792=264%:%
%:%793=265%:%
%:%794=265%:%
%:%795=265%:%
%:%796=266%:%
%:%797=266%:%
%:%798=267%:%
%:%799=267%:%
%:%800=267%:%
%:%801=268%:%
%:%802=268%:%
%:%803=269%:%
%:%804=269%:%
%:%805=270%:%
%:%806=270%:%
%:%807=270%:%
%:%808=271%:%
%:%809=271%:%
%:%810=271%:%
%:%811=272%:%
%:%812=272%:%
%:%813=272%:%
%:%814=273%:%
%:%815=273%:%
%:%816=273%:%
%:%817=274%:%
%:%818=274%:%
%:%819=274%:%
%:%820=275%:%
%:%821=275%:%
%:%822=276%:%
%:%823=277%:%
%:%824=277%:%
%:%825=278%:%
%:%826=278%:%
%:%827=279%:%
%:%828=279%:%
%:%829=280%:%
%:%830=280%:%
%:%831=281%:%
%:%832=281%:%
%:%833=282%:%
%:%834=282%:%
%:%835=283%:%
%:%836=283%:%
%:%837=284%:%
%:%838=284%:%
%:%839=285%:%
%:%840=285%:%
%:%841=286%:%
%:%842=286%:%
%:%843=287%:%
%:%844=288%:%
%:%845=289%:%
%:%846=289%:%
%:%847=290%:%
%:%848=290%:%
%:%849=291%:%
%:%850=291%:%
%:%851=292%:%
%:%852=292%:%
%:%853=293%:%
%:%854=293%:%
%:%855=294%:%
%:%856=294%:%
%:%857=294%:%
%:%858=295%:%
%:%859=295%:%
%:%860=295%:%
%:%861=295%:%
%:%862=296%:%
%:%863=296%:%
%:%864=297%:%
%:%865=298%:%
%:%866=298%:%
%:%867=299%:%
%:%868=299%:%
%:%869=300%:%
%:%870=300%:%
%:%871=301%:%
%:%872=301%:%
%:%873=302%:%
%:%874=302%:%
%:%875=303%:%
%:%876=303%:%
%:%877=304%:%
%:%878=304%:%
%:%879=304%:%
%:%880=305%:%
%:%881=305%:%
%:%882=305%:%
%:%883=306%:%
%:%884=306%:%
%:%885=306%:%
%:%886=307%:%
%:%887=307%:%
%:%888=308%:%
%:%889=308%:%
%:%890=308%:%
%:%891=309%:%
%:%892=309%:%
%:%893=309%:%
%:%894=310%:%
%:%895=310%:%
%:%896=310%:%
%:%897=311%:%
%:%898=311%:%
%:%899=311%:%
%:%900=312%:%
%:%901=312%:%
%:%902=312%:%
%:%903=313%:%
%:%904=313%:%
%:%905=313%:%
%:%906=314%:%
%:%907=314%:%
%:%908=314%:%
%:%909=315%:%
%:%910=315%:%
%:%911=316%:%
%:%912=316%:%
%:%913=317%:%
%:%914=318%:%
%:%915=318%:%
%:%916=319%:%
%:%917=319%:%
%:%918=320%:%
%:%919=320%:%
%:%920=321%:%
%:%921=321%:%
%:%922=321%:%
%:%923=322%:%
%:%924=322%:%
%:%925=322%:%
%:%926=323%:%
%:%927=323%:%
%:%928=323%:%
%:%929=324%:%
%:%930=324%:%
%:%931=324%:%
%:%932=325%:%
%:%933=325%:%
%:%934=325%:%
%:%935=326%:%
%:%936=326%:%
%:%937=326%:%
%:%938=327%:%
%:%939=327%:%
%:%940=327%:%
%:%941=328%:%
%:%942=328%:%
%:%943=328%:%
%:%944=329%:%
%:%945=329%:%
%:%946=329%:%
%:%947=330%:%
%:%948=330%:%
%:%949=330%:%
%:%950=330%:%
%:%951=330%:%
%:%952=331%:%
%:%953=331%:%
%:%954=332%:%
%:%955=333%:%
%:%956=333%:%
%:%958=335%:%
%:%959=336%:%
%:%960=336%:%
%:%961=337%:%
%:%962=337%:%
%:%963=338%:%
%:%964=338%:%
%:%965=339%:%
%:%966=339%:%
%:%967=340%:%
%:%968=340%:%
%:%969=340%:%
%:%970=341%:%
%:%971=341%:%
%:%972=341%:%
%:%973=342%:%
%:%974=342%:%
%:%975=343%:%
%:%976=344%:%
%:%977=344%:%
%:%978=345%:%
%:%979=345%:%
%:%980=346%:%
%:%981=346%:%
%:%982=346%:%
%:%983=347%:%
%:%984=347%:%
%:%985=347%:%
%:%986=348%:%
%:%987=348%:%
%:%988=348%:%
%:%989=349%:%
%:%990=349%:%
%:%991=349%:%
%:%992=350%:%
%:%993=350%:%
%:%994=350%:%
%:%995=351%:%
%:%996=351%:%
%:%997=351%:%
%:%998=352%:%
%:%999=352%:%
%:%1000=352%:%
%:%1001=353%:%
%:%1002=353%:%
%:%1003=353%:%
%:%1004=354%:%
%:%1005=354%:%
%:%1006=354%:%
%:%1007=355%:%
%:%1008=355%:%
%:%1009=355%:%
%:%1010=356%:%
%:%1011=356%:%
%:%1012=356%:%
%:%1013=357%:%
%:%1014=357%:%
%:%1015=357%:%
%:%1016=358%:%
%:%1017=358%:%
%:%1018=358%:%
%:%1019=359%:%
%:%1020=359%:%
%:%1021=359%:%
%:%1022=360%:%
%:%1023=360%:%
%:%1024=361%:%
%:%1025=361%:%
%:%1026=361%:%
%:%1027=361%:%
%:%1028=361%:%
%:%1029=362%:%
%:%1030=362%:%
%:%1031=363%:%
%:%1032=363%:%
%:%1033=363%:%
%:%1035=365%:%
%:%1036=366%:%
%:%1037=366%:%
%:%1038=366%:%
%:%1039=367%:%
%:%1040=367%:%
%:%1041=368%:%
%:%1042=369%:%
%:%1043=369%:%
%:%1044=370%:%
%:%1045=370%:%
%:%1046=371%:%
%:%1047=371%:%
%:%1048=372%:%
%:%1049=372%:%
%:%1050=372%:%
%:%1051=373%:%
%:%1052=373%:%
%:%1053=374%:%
%:%1054=374%:%
%:%1055=374%:%
%:%1056=375%:%
%:%1057=375%:%
%:%1058=375%:%
%:%1059=376%:%
%:%1060=376%:%
%:%1061=377%:%
%:%1062=377%:%
%:%1063=378%:%
%:%1064=378%:%
%:%1065=379%:%
%:%1066=379%:%
%:%1067=379%:%
%:%1068=380%:%
%:%1069=380%:%
%:%1070=381%:%
%:%1071=381%:%
%:%1072=381%:%
%:%1073=382%:%
%:%1074=382%:%
%:%1075=383%:%
%:%1076=383%:%
%:%1077=383%:%
%:%1078=384%:%
%:%1079=384%:%
%:%1080=384%:%
%:%1081=385%:%
%:%1082=385%:%
%:%1083=386%:%
%:%1084=386%:%
%:%1085=387%:%
%:%1086=387%:%
%:%1087=387%:%
%:%1088=388%:%
%:%1089=388%:%
%:%1090=388%:%
%:%1091=389%:%
%:%1092=389%:%
%:%1093=390%:%
%:%1094=390%:%
%:%1095=391%:%
%:%1096=391%:%
%:%1097=392%:%
%:%1098=392%:%
%:%1099=393%:%
%:%1100=393%:%
%:%1101=394%:%
%:%1102=394%:%
%:%1103=394%:%
%:%1104=395%:%
%:%1105=395%:%
%:%1106=395%:%
%:%1107=396%:%
%:%1108=396%:%
%:%1109=397%:%
%:%1110=397%:%
%:%1111=398%:%
%:%1112=398%:%
%:%1113=398%:%
%:%1114=399%:%
%:%1115=399%:%
%:%1116=399%:%
%:%1117=400%:%
%:%1118=400%:%
%:%1119=400%:%
%:%1120=401%:%
%:%1121=401%:%
%:%1122=401%:%
%:%1123=402%:%
%:%1124=402%:%
%:%1125=402%:%
%:%1126=403%:%
%:%1127=403%:%
%:%1128=404%:%
%:%1129=404%:%
%:%1130=404%:%
%:%1131=405%:%
%:%1132=405%:%
%:%1133=406%:%
%:%1134=406%:%
%:%1135=406%:%
%:%1136=406%:%
%:%1137=407%:%
%:%1138=407%:%
%:%1139=408%:%
%:%1140=408%:%
%:%1141=409%:%
%:%1142=409%:%
%:%1143=410%:%
%:%1144=410%:%
%:%1145=411%:%
%:%1146=411%:%
%:%1147=411%:%
%:%1148=412%:%
%:%1149=412%:%
%:%1150=412%:%
%:%1151=413%:%
%:%1152=413%:%
%:%1153=414%:%
%:%1154=414%:%
%:%1155=415%:%
%:%1156=415%:%
%:%1157=416%:%
%:%1158=416%:%
%:%1159=416%:%
%:%1160=417%:%
%:%1161=417%:%
%:%1162=417%:%
%:%1163=418%:%
%:%1164=418%:%
%:%1165=418%:%
%:%1166=419%:%
%:%1167=419%:%
%:%1168=419%:%
%:%1169=420%:%
%:%1170=420%:%
%:%1171=421%:%
%:%1172=421%:%
%:%1173=422%:%
%:%1174=422%:%
%:%1175=423%:%
%:%1176=423%:%
%:%1177=423%:%
%:%1178=424%:%
%:%1179=424%:%
%:%1180=425%:%
%:%1181=425%:%
%:%1182=425%:%
%:%1183=426%:%
%:%1184=426%:%
%:%1185=426%:%
%:%1186=427%:%
%:%1187=427%:%
%:%1188=428%:%
%:%1189=428%:%
%:%1190=429%:%
%:%1191=429%:%
%:%1192=429%:%
%:%1193=430%:%
%:%1194=430%:%
%:%1195=430%:%
%:%1196=431%:%
%:%1197=431%:%
%:%1198=432%:%
%:%1199=432%:%
%:%1200=433%:%
%:%1201=433%:%
%:%1202=434%:%
%:%1203=434%:%
%:%1204=435%:%
%:%1205=435%:%
%:%1206=436%:%
%:%1207=436%:%
%:%1208=437%:%
%:%1209=437%:%
%:%1210=438%:%
%:%1211=438%:%
%:%1212=438%:%
%:%1213=439%:%
%:%1214=439%:%
%:%1215=440%:%
%:%1216=440%:%
%:%1217=440%:%
%:%1218=441%:%
%:%1219=442%:%
%:%1220=442%:%
%:%1221=442%:%
%:%1222=443%:%
%:%1223=443%:%
%:%1224=443%:%
%:%1225=444%:%
%:%1226=444%:%
%:%1227=444%:%
%:%1228=445%:%
%:%1229=445%:%
%:%1230=445%:%
%:%1231=446%:%
%:%1232=446%:%
%:%1233=446%:%
%:%1234=447%:%
%:%1235=447%:%
%:%1236=447%:%
%:%1237=448%:%
%:%1238=448%:%
%:%1239=449%:%
%:%1240=449%:%
%:%1241=450%:%
%:%1242=450%:%
%:%1243=451%:%
%:%1244=451%:%
%:%1245=452%:%
%:%1246=452%:%
%:%1247=453%:%
%:%1248=453%:%
%:%1249=453%:%
%:%1250=454%:%
%:%1251=455%:%
%:%1252=455%:%
%:%1253=455%:%
%:%1254=456%:%
%:%1255=456%:%
%:%1256=456%:%
%:%1257=457%:%
%:%1258=457%:%
%:%1259=457%:%
%:%1260=458%:%
%:%1261=458%:%
%:%1262=459%:%
%:%1263=459%:%
%:%1264=460%:%
%:%1265=460%:%
%:%1266=461%:%
%:%1267=461%:%
%:%1268=461%:%
%:%1269=462%:%
%:%1270=462%:%
%:%1271=463%:%
%:%1272=463%:%
%:%1273=464%:%
%:%1274=464%:%
%:%1275=465%:%
%:%1276=465%:%
%:%1277=466%:%
%:%1278=466%:%
%:%1279=467%:%
%:%1280=467%:%
%:%1281=468%:%
%:%1282=468%:%
%:%1283=469%:%
%:%1284=470%:%
%:%1285=470%:%
%:%1286=470%:%
%:%1287=471%:%
%:%1288=471%:%
%:%1289=472%:%
%:%1290=472%:%
%:%1291=472%:%
%:%1292=473%:%
%:%1293=473%:%
%:%1294=473%:%
%:%1295=474%:%
%:%1296=474%:%
%:%1297=474%:%
%:%1298=475%:%
%:%1299=475%:%
%:%1300=476%:%
%:%1301=476%:%
%:%1302=477%:%
%:%1303=477%:%
%:%1304=478%:%
%:%1305=478%:%
%:%1306=478%:%
%:%1307=479%:%
%:%1308=480%:%
%:%1309=480%:%
%:%1310=480%:%
%:%1311=481%:%
%:%1312=481%:%
%:%1313=481%:%
%:%1314=482%:%
%:%1315=482%:%
%:%1316=482%:%
%:%1317=483%:%
%:%1318=483%:%
%:%1319=483%:%
%:%1320=484%:%
%:%1321=484%:%
%:%1322=484%:%
%:%1323=485%:%
%:%1324=485%:%
%:%1325=485%:%
%:%1326=486%:%
%:%1327=486%:%
%:%1328=487%:%
%:%1329=487%:%
%:%1330=487%:%
%:%1331=488%:%
%:%1332=488%:%
%:%1333=489%:%
%:%1334=489%:%
%:%1335=489%:%
%:%1336=490%:%
%:%1337=490%:%
%:%1338=491%:%
%:%1339=491%:%
%:%1340=492%:%
%:%1341=492%:%
%:%1342=493%:%
%:%1343=493%:%
%:%1344=494%:%
%:%1345=494%:%
%:%1346=494%:%
%:%1347=495%:%
%:%1348=495%:%
%:%1349=495%:%
%:%1350=496%:%
%:%1351=496%:%
%:%1352=496%:%
%:%1353=497%:%
%:%1354=497%:%
%:%1355=498%:%
%:%1356=498%:%
%:%1357=498%:%
%:%1358=499%:%
%:%1359=499%:%
%:%1360=499%:%
%:%1361=500%:%
%:%1362=500%:%
%:%1363=500%:%
%:%1364=501%:%
%:%1365=501%:%
%:%1366=501%:%
%:%1367=502%:%
%:%1368=502%:%
%:%1369=502%:%
%:%1370=503%:%
%:%1371=503%:%
%:%1372=504%:%
%:%1373=504%:%
%:%1374=504%:%
%:%1375=505%:%
%:%1376=505%:%
%:%1377=506%:%
%:%1378=506%:%
%:%1379=507%:%
%:%1380=507%:%
%:%1381=508%:%
%:%1382=508%:%
%:%1383=509%:%
%:%1384=509%:%
%:%1385=510%:%
%:%1386=510%:%
%:%1387=510%:%
%:%1388=511%:%
%:%1389=511%:%
%:%1390=511%:%
%:%1391=512%:%
%:%1392=512%:%
%:%1393=512%:%
%:%1394=513%:%
%:%1395=513%:%
%:%1396=513%:%
%:%1397=514%:%
%:%1403=514%:%
%:%1406=515%:%
%:%1407=516%:%
%:%1408=516%:%
%:%1409=517%:%
%:%1410=518%:%
%:%1417=519%:%
%:%1418=519%:%
%:%1419=520%:%
%:%1420=520%:%
%:%1421=521%:%
%:%1422=521%:%
%:%1423=522%:%
%:%1424=522%:%
%:%1425=523%:%
%:%1426=523%:%
%:%1427=524%:%
%:%1428=524%:%
%:%1429=524%:%
%:%1430=525%:%
%:%1431=525%:%
%:%1432=525%:%
%:%1433=526%:%
%:%1434=526%:%
%:%1435=527%:%
%:%1436=527%:%
%:%1437=528%:%
%:%1438=528%:%
%:%1439=529%:%
%:%1440=529%:%
%:%1441=530%:%
%:%1442=530%:%
%:%1443=530%:%
%:%1444=531%:%
%:%1445=531%:%
%:%1446=531%:%
%:%1447=532%:%
%:%1448=532%:%
%:%1449=532%:%
%:%1450=533%:%
%:%1451=533%:%
%:%1452=533%:%
%:%1453=534%:%
%:%1454=534%:%
%:%1455=535%:%
%:%1456=535%:%
%:%1457=536%:%
%:%1458=536%:%
%:%1459=537%:%
%:%1460=537%:%
%:%1461=538%:%
%:%1462=538%:%
%:%1463=538%:%
%:%1464=539%:%
%:%1465=539%:%
%:%1466=540%:%
%:%1467=540%:%
%:%1468=541%:%
%:%1469=541%:%
%:%1470=542%:%
%:%1471=542%:%
%:%1472=543%:%
%:%1473=543%:%
%:%1474=544%:%
%:%1475=544%:%
%:%1476=545%:%
%:%1477=545%:%
%:%1478=546%:%
%:%1479=546%:%
%:%1480=547%:%
%:%1481=547%:%
%:%1482=547%:%
%:%1483=548%:%
%:%1484=548%:%
%:%1485=549%:%
%:%1486=549%:%
%:%1487=550%:%
%:%1488=550%:%
%:%1489=551%:%
%:%1490=551%:%
%:%1491=552%:%
%:%1492=552%:%
%:%1493=553%:%
%:%1494=553%:%
%:%1495=554%:%
%:%1496=554%:%
%:%1497=555%:%
%:%1498=556%:%
%:%1499=557%:%
%:%1500=557%:%
%:%1501=558%:%
%:%1502=558%:%
%:%1503=559%:%
%:%1504=559%:%
%:%1505=560%:%
%:%1506=560%:%
%:%1507=561%:%
%:%1508=561%:%
%:%1509=562%:%
%:%1510=562%:%
%:%1511=562%:%
%:%1512=563%:%
%:%1513=563%:%
%:%1514=564%:%
%:%1515=564%:%
%:%1516=565%:%
%:%1517=565%:%
%:%1518=566%:%
%:%1519=566%:%
%:%1520=566%:%
%:%1521=567%:%
%:%1522=567%:%
%:%1523=568%:%
%:%1524=568%:%
%:%1525=568%:%
%:%1526=569%:%
%:%1527=569%:%
%:%1528=569%:%
%:%1529=570%:%
%:%1530=570%:%
%:%1531=571%:%
%:%1532=571%:%
%:%1533=572%:%
%:%1534=573%:%
%:%1535=573%:%
%:%1536=574%:%
%:%1537=575%:%
%:%1538=575%:%
%:%1539=576%:%
%:%1540=576%:%
%:%1541=576%:%
%:%1542=577%:%
%:%1543=577%:%
%:%1544=578%:%
%:%1545=579%:%
%:%1546=580%:%
%:%1547=580%:%
%:%1548=581%:%
%:%1549=581%:%
%:%1550=582%:%
%:%1551=583%:%
%:%1552=583%:%
%:%1553=584%:%
%:%1554=584%:%
%:%1555=584%:%
%:%1556=585%:%
%:%1557=586%:%
%:%1558=586%:%
%:%1559=587%:%
%:%1560=587%:%
%:%1561=588%:%
%:%1562=588%:%
%:%1563=589%:%
%:%1564=589%:%
%:%1565=590%:%
%:%1566=590%:%
%:%1567=591%:%
%:%1568=591%:%
%:%1569=592%:%
%:%1570=592%:%
%:%1571=593%:%
%:%1572=593%:%
%:%1573=594%:%
%:%1574=594%:%
%:%1575=595%:%
%:%1576=595%:%
%:%1577=596%:%
%:%1578=596%:%
%:%1579=596%:%
%:%1580=597%:%
%:%1581=597%:%
%:%1582=598%:%
%:%1583=598%:%
%:%1584=598%:%
%:%1585=599%:%
%:%1586=599%:%
%:%1587=599%:%
%:%1588=600%:%
%:%1589=600%:%
%:%1590=600%:%
%:%1591=601%:%
%:%1592=601%:%
%:%1593=601%:%
%:%1594=602%:%
%:%1595=602%:%
%:%1596=602%:%
%:%1597=603%:%
%:%1598=603%:%
%:%1599=604%:%
%:%1600=604%:%
%:%1601=604%:%
%:%1602=605%:%
%:%1603=605%:%
%:%1604=606%:%
%:%1605=606%:%
%:%1606=606%:%
%:%1607=607%:%
%:%1608=607%:%
%:%1609=608%:%
%:%1610=608%:%
%:%1611=608%:%
%:%1612=609%:%
%:%1613=609%:%
%:%1614=609%:%
%:%1615=610%:%
%:%1616=610%:%
%:%1617=610%:%
%:%1618=611%:%
%:%1619=611%:%
%:%1620=612%:%
%:%1621=612%:%
%:%1622=612%:%
%:%1623=613%:%
%:%1624=613%:%
%:%1625=614%:%
%:%1626=614%:%
%:%1627=614%:%
%:%1628=615%:%
%:%1629=615%:%
%:%1630=615%:%
%:%1631=616%:%
%:%1632=616%:%
%:%1633=616%:%
%:%1634=617%:%
%:%1635=617%:%
%:%1636=618%:%
%:%1637=618%:%
%:%1638=618%:%
%:%1639=619%:%
%:%1640=619%:%
%:%1641=619%:%
%:%1642=620%:%
%:%1643=620%:%
%:%1644=620%:%
%:%1645=621%:%
%:%1646=621%:%
%:%1647=621%:%
%:%1648=622%:%
%:%1649=622%:%
%:%1650=622%:%
%:%1651=623%:%
%:%1652=623%:%
%:%1653=623%:%
%:%1654=624%:%
%:%1655=624%:%
%:%1656=625%:%
%:%1657=625%:%
%:%1658=626%:%
%:%1659=627%:%
%:%1660=627%:%
%:%1661=628%:%
%:%1662=628%:%
%:%1663=629%:%
%:%1664=629%:%
%:%1665=630%:%
%:%1666=630%:%
%:%1667=631%:%
%:%1668=631%:%
%:%1669=632%:%
%:%1670=632%:%
%:%1671=633%:%
%:%1672=633%:%
%:%1673=634%:%
%:%1674=634%:%
%:%1675=634%:%
%:%1676=635%:%
%:%1677=636%:%
%:%1678=636%:%
%:%1679=636%:%
%:%1680=637%:%
%:%1681=637%:%
%:%1682=637%:%
%:%1683=638%:%
%:%1684=639%:%
%:%1685=639%:%
%:%1686=640%:%
%:%1687=640%:%
%:%1688=640%:%
%:%1689=641%:%
%:%1690=642%:%
%:%1691=642%:%
%:%1692=643%:%
%:%1693=643%:%
%:%1694=643%:%
%:%1695=644%:%
%:%1696=645%:%
%:%1697=645%:%
%:%1698=646%:%
%:%1699=646%:%
%:%1700=646%:%
%:%1701=647%:%
%:%1702=648%:%
%:%1703=648%:%
%:%1704=649%:%
%:%1705=649%:%
%:%1706=649%:%
%:%1707=650%:%
%:%1708=651%:%
%:%1709=651%:%
%:%1710=651%:%
%:%1711=652%:%
%:%1712=652%:%
%:%1713=652%:%
%:%1714=653%:%
%:%1715=654%:%
%:%1716=654%:%
%:%1717=655%:%
%:%1718=655%:%
%:%1719=655%:%
%:%1720=656%:%
%:%1721=656%:%
%:%1722=656%:%
%:%1723=657%:%
%:%1724=657%:%
%:%1725=657%:%
%:%1726=658%:%
%:%1727=658%:%
%:%1728=659%:%
%:%1729=659%:%
%:%1730=659%:%
%:%1731=660%:%
%:%1732=660%:%
%:%1733=661%:%
%:%1734=661%:%
%:%1735=662%:%
%:%1736=663%:%
%:%1737=663%:%
%:%1738=664%:%
%:%1739=664%:%
%:%1740=665%:%
%:%1741=665%:%
%:%1742=666%:%
%:%1743=666%:%
%:%1744=667%:%
%:%1745=667%:%
%:%1746=668%:%
%:%1747=668%:%
%:%1748=669%:%
%:%1749=669%:%
%:%1750=670%:%
%:%1751=670%:%
%:%1752=671%:%
%:%1753=671%:%
%:%1754=672%:%
%:%1755=672%:%
%:%1756=673%:%
%:%1757=673%:%
%:%1758=673%:%
%:%1759=674%:%
%:%1760=675%:%
%:%1761=675%:%
%:%1762=675%:%
%:%1763=676%:%
%:%1764=676%:%
%:%1765=676%:%
%:%1766=677%:%
%:%1767=678%:%
%:%1768=678%:%
%:%1769=679%:%
%:%1770=679%:%
%:%1771=679%:%
%:%1772=680%:%
%:%1773=681%:%
%:%1774=681%:%
%:%1775=682%:%
%:%1776=682%:%
%:%1777=682%:%
%:%1778=683%:%
%:%1779=684%:%
%:%1780=684%:%
%:%1781=685%:%
%:%1782=685%:%
%:%1783=685%:%
%:%1784=686%:%
%:%1785=687%:%
%:%1786=687%:%
%:%1787=688%:%
%:%1788=688%:%
%:%1789=688%:%
%:%1790=689%:%
%:%1791=690%:%
%:%1792=690%:%
%:%1793=690%:%
%:%1794=691%:%
%:%1795=691%:%
%:%1796=691%:%
%:%1797=692%:%
%:%1798=693%:%
%:%1799=693%:%
%:%1800=694%:%
%:%1801=694%:%
%:%1802=694%:%
%:%1803=695%:%
%:%1804=695%:%
%:%1805=695%:%
%:%1806=696%:%
%:%1807=696%:%
%:%1808=696%:%
%:%1809=697%:%
%:%1810=697%:%
%:%1811=697%:%
%:%1812=698%:%
%:%1813=698%:%
%:%1814=698%:%
%:%1815=699%:%
%:%1816=699%:%
%:%1817=700%:%
%:%1818=700%:%
%:%1819=700%:%
%:%1820=701%:%
%:%1821=701%:%
%:%1822=702%:%
%:%1823=702%:%
%:%1824=702%:%
%:%1825=703%:%
%:%1826=703%:%
%:%1827=703%:%
%:%1828=704%:%
%:%1829=704%:%
%:%1830=704%:%
%:%1831=705%:%
%:%1832=705%:%
%:%1833=706%:%
%:%1834=706%:%
%:%1835=707%:%
%:%1836=708%:%
%:%1837=708%:%
%:%1838=709%:%
%:%1839=709%:%
%:%1840=710%:%
%:%1841=710%:%
%:%1842=711%:%
%:%1843=711%:%
%:%1844=712%:%
%:%1845=712%:%
%:%1846=713%:%
%:%1847=713%:%
%:%1848=714%:%
%:%1849=714%:%
%:%1850=715%:%
%:%1851=715%:%
%:%1852=716%:%
%:%1853=716%:%
%:%1854=716%:%
%:%1855=717%:%
%:%1856=718%:%
%:%1857=718%:%
%:%1858=718%:%
%:%1859=719%:%
%:%1860=719%:%
%:%1861=719%:%
%:%1862=720%:%
%:%1863=721%:%
%:%1864=721%:%
%:%1865=722%:%
%:%1866=722%:%
%:%1867=722%:%
%:%1868=723%:%
%:%1869=724%:%
%:%1870=724%:%
%:%1871=725%:%
%:%1872=725%:%
%:%1873=725%:%
%:%1874=726%:%
%:%1875=727%:%
%:%1876=727%:%
%:%1877=728%:%
%:%1878=728%:%
%:%1879=728%:%
%:%1880=729%:%
%:%1881=730%:%
%:%1882=730%:%
%:%1883=731%:%
%:%1884=731%:%
%:%1885=731%:%
%:%1886=732%:%
%:%1887=733%:%
%:%1888=733%:%
%:%1889=733%:%
%:%1890=734%:%
%:%1891=734%:%
%:%1892=734%:%
%:%1893=735%:%
%:%1894=736%:%
%:%1895=736%:%
%:%1896=737%:%
%:%1897=737%:%
%:%1898=737%:%
%:%1899=738%:%
%:%1900=738%:%
%:%1901=738%:%
%:%1902=739%:%
%:%1903=739%:%
%:%1904=739%:%
%:%1905=740%:%
%:%1906=740%:%
%:%1907=740%:%
%:%1908=741%:%
%:%1909=741%:%
%:%1910=741%:%
%:%1911=742%:%
%:%1912=742%:%
%:%1913=743%:%
%:%1914=743%:%
%:%1915=743%:%
%:%1916=744%:%
%:%1917=744%:%
%:%1918=745%:%
%:%1919=745%:%
%:%1920=745%:%
%:%1921=746%:%
%:%1922=746%:%
%:%1923=746%:%
%:%1924=747%:%
%:%1925=747%:%
%:%1926=747%:%
%:%1927=748%:%
%:%1928=748%:%
%:%1929=749%:%
%:%1930=749%:%
%:%1931=750%:%
%:%1932=751%:%
%:%1933=751%:%
%:%1934=752%:%
%:%1935=752%:%
%:%1936=753%:%
%:%1937=753%:%
%:%1938=754%:%
%:%1939=754%:%
%:%1940=755%:%
%:%1941=755%:%
%:%1942=756%:%
%:%1943=756%:%
%:%1944=757%:%
%:%1945=757%:%
%:%1946=758%:%
%:%1947=758%:%
%:%1948=759%:%
%:%1949=759%:%
%:%1950=760%:%
%:%1951=760%:%
%:%1952=761%:%
%:%1953=761%:%
%:%1954=762%:%
%:%1955=762%:%
%:%1956=763%:%
%:%1957=763%:%
%:%1958=763%:%
%:%1959=764%:%
%:%1960=764%:%
%:%1961=765%:%
%:%1962=765%:%
%:%1963=766%:%
%:%1964=766%:%
%:%1965=767%:%
%:%1966=767%:%
%:%1967=768%:%
%:%1968=768%:%
%:%1969=769%:%
%:%1970=769%:%
%:%1971=770%:%
%:%1972=770%:%
%:%1973=771%:%
%:%1974=771%:%
%:%1975=772%:%
%:%1976=772%:%
%:%1977=773%:%
%:%1978=773%:%
%:%1979=774%:%
%:%1980=774%:%
%:%1981=775%:%
%:%1982=775%:%
%:%1983=776%:%
%:%1984=776%:%
%:%1985=777%:%
%:%1986=777%:%
%:%1987=778%:%
%:%1988=778%:%
%:%1989=779%:%
%:%1990=779%:%
%:%1991=780%:%
%:%1992=780%:%
%:%1993=781%:%
%:%1994=781%:%
%:%1995=782%:%
%:%1996=782%:%
%:%1997=783%:%
%:%1998=783%:%
%:%1999=784%:%
%:%2000=784%:%
%:%2001=785%:%
%:%2002=785%:%
%:%2003=785%:%
%:%2004=786%:%
%:%2005=786%:%
%:%2006=786%:%
%:%2007=787%:%
%:%2008=787%:%
%:%2009=787%:%
%:%2010=788%:%
%:%2011=788%:%
%:%2012=788%:%
%:%2013=789%:%
%:%2014=789%:%
%:%2015=789%:%
%:%2016=790%:%
%:%2017=790%:%
%:%2018=791%:%
%:%2019=791%:%
%:%2020=792%:%
%:%2021=792%:%
%:%2022=793%:%
%:%2023=793%:%
%:%2024=794%:%
%:%2025=794%:%
%:%2026=795%:%
%:%2027=795%:%
%:%2028=796%:%
%:%2029=796%:%
%:%2030=797%:%
%:%2031=797%:%
%:%2032=798%:%
%:%2033=798%:%
%:%2034=799%:%
%:%2035=799%:%
%:%2036=800%:%
%:%2037=800%:%
%:%2038=801%:%
%:%2039=801%:%
%:%2040=802%:%
%:%2041=802%:%
%:%2042=803%:%
%:%2043=803%:%
%:%2044=804%:%
%:%2045=804%:%
%:%2046=805%:%
%:%2047=805%:%
%:%2048=806%:%
%:%2049=806%:%
%:%2050=807%:%
%:%2051=807%:%
%:%2052=808%:%
%:%2053=808%:%
%:%2054=809%:%
%:%2055=809%:%
%:%2056=810%:%
%:%2057=810%:%
%:%2058=811%:%
%:%2059=811%:%
%:%2060=812%:%
%:%2061=812%:%
%:%2062=813%:%
%:%2063=813%:%
%:%2064=814%:%
%:%2065=814%:%
%:%2066=815%:%
%:%2067=815%:%
%:%2068=816%:%
%:%2069=816%:%
%:%2070=817%:%
%:%2071=817%:%
%:%2072=818%:%
%:%2073=818%:%
%:%2074=819%:%
%:%2075=819%:%
%:%2076=820%:%
%:%2077=820%:%
%:%2078=821%:%
%:%2079=821%:%
%:%2080=822%:%
%:%2081=822%:%
%:%2082=823%:%
%:%2083=823%:%
%:%2084=824%:%
%:%2085=824%:%
%:%2086=825%:%
%:%2087=825%:%
%:%2088=826%:%
%:%2089=826%:%
%:%2090=827%:%
%:%2091=827%:%
%:%2092=828%:%
%:%2093=828%:%
%:%2094=829%:%
%:%2095=829%:%
%:%2096=830%:%
%:%2097=830%:%
%:%2098=831%:%
%:%2099=831%:%
%:%2100=832%:%
%:%2101=832%:%
%:%2102=832%:%
%:%2103=833%:%
%:%2104=833%:%
%:%2105=834%:%
%:%2106=834%:%
%:%2107=835%:%
%:%2108=835%:%
%:%2109=835%:%
%:%2110=836%:%
%:%2111=836%:%
%:%2112=836%:%
%:%2113=837%:%
%:%2114=837%:%
%:%2115=838%:%
%:%2116=838%:%
%:%2117=839%:%
%:%2118=839%:%
%:%2119=840%:%
%:%2125=840%:%
%:%2128=841%:%
%:%2129=842%:%
%:%2130=842%:%
%:%2131=843%:%
%:%2132=844%:%
%:%2139=845%:%
%:%2140=845%:%
%:%2141=846%:%
%:%2142=846%:%
%:%2143=847%:%
%:%2144=847%:%
%:%2145=848%:%
%:%2146=848%:%
%:%2147=848%:%
%:%2148=849%:%
%:%2149=849%:%
%:%2150=850%:%
%:%2151=850%:%
%:%2152=851%:%
%:%2153=851%:%
%:%2154=852%:%
%:%2155=852%:%
%:%2156=853%:%
%:%2157=853%:%
%:%2158=854%:%
%:%2159=854%:%
%:%2160=855%:%
%:%2161=855%:%
%:%2162=856%:%
%:%2163=856%:%
%:%2164=857%:%
%:%2165=857%:%
%:%2166=858%:%
%:%2167=858%:%
%:%2168=858%:%
%:%2169=859%:%
%:%2170=859%:%
%:%2171=860%:%
%:%2172=860%:%
%:%2173=860%:%
%:%2174=861%:%
%:%2175=861%:%
%:%2176=862%:%
%:%2177=862%:%
%:%2178=862%:%
%:%2179=863%:%
%:%2180=863%:%
%:%2181=863%:%
%:%2182=864%:%
%:%2183=864%:%
%:%2184=865%:%
%:%2185=865%:%
%:%2186=865%:%
%:%2187=866%:%
%:%2188=866%:%
%:%2189=866%:%
%:%2190=867%:%
%:%2191=867%:%
%:%2192=867%:%
%:%2193=868%:%
%:%2194=868%:%
%:%2195=868%:%
%:%2196=869%:%
%:%2197=869%:%
%:%2198=869%:%
%:%2199=870%:%
%:%2200=870%:%
%:%2201=871%:%
%:%2202=871%:%
%:%2203=871%:%
%:%2204=872%:%
%:%2205=872%:%
%:%2206=872%:%
%:%2207=873%:%
%:%2208=873%:%
%:%2209=873%:%
%:%2210=874%:%
%:%2211=874%:%
%:%2212=874%:%
%:%2213=875%:%
%:%2214=875%:%
%:%2215=875%:%
%:%2216=876%:%
%:%2217=876%:%
%:%2218=877%:%
%:%2219=877%:%
%:%2220=877%:%
%:%2221=878%:%
%:%2222=878%:%
%:%2223=878%:%
%:%2224=879%:%
%:%2225=879%:%
%:%2226=879%:%
%:%2227=880%:%
%:%2228=880%:%
%:%2229=880%:%
%:%2230=881%:%
%:%2231=881%:%
%:%2232=882%:%
%:%2233=882%:%
%:%2234=882%:%
%:%2235=883%:%
%:%2236=883%:%
%:%2237=883%:%
%:%2238=884%:%
%:%2244=884%:%
%:%2247=885%:%
%:%2248=886%:%
%:%2249=886%:%
%:%2250=887%:%
%:%2251=888%:%
%:%2252=889%:%
%:%2259=890%:%
%:%2260=890%:%
%:%2261=891%:%
%:%2262=891%:%
%:%2263=892%:%
%:%2264=893%:%
%:%2265=894%:%
%:%2266=894%:%
%:%2267=894%:%
%:%2268=895%:%
%:%2269=895%:%
%:%2270=895%:%
%:%2271=896%:%
%:%2272=896%:%
%:%2273=897%:%
%:%2274=897%:%
%:%2275=898%:%
%:%2276=898%:%
%:%2277=899%:%
%:%2278=899%:%
%:%2279=900%:%
%:%2280=900%:%
%:%2281=901%:%
%:%2282=901%:%
%:%2283=902%:%
%:%2284=902%:%
%:%2285=902%:%
%:%2286=903%:%
%:%2287=903%:%
%:%2288=903%:%
%:%2289=904%:%
%:%2290=904%:%
%:%2291=905%:%
%:%2292=905%:%
%:%2293=905%:%
%:%2294=906%:%
%:%2295=906%:%
%:%2296=906%:%
%:%2297=907%:%
%:%2298=908%:%
%:%2299=908%:%
%:%2300=909%:%
%:%2301=909%:%
%:%2302=910%:%
%:%2303=910%:%
%:%2304=911%:%
%:%2305=911%:%
%:%2306=912%:%
%:%2307=912%:%
%:%2308=913%:%
%:%2309=913%:%
%:%2310=914%:%
%:%2311=914%:%
%:%2312=915%:%
%:%2313=915%:%
%:%2314=915%:%
%:%2315=916%:%
%:%2316=916%:%
%:%2317=916%:%
%:%2318=917%:%
%:%2319=917%:%
%:%2320=917%:%
%:%2321=918%:%
%:%2322=918%:%
%:%2323=919%:%
%:%2324=919%:%
%:%2325=920%:%
%:%2326=920%:%
%:%2327=920%:%
%:%2328=921%:%
%:%2329=921%:%
%:%2330=921%:%
%:%2331=922%:%
%:%2332=922%:%
%:%2333=922%:%
%:%2334=923%:%
%:%2335=923%:%
%:%2336=924%:%
%:%2337=924%:%
%:%2338=925%:%
%:%2339=925%:%
%:%2340=926%:%
%:%2341=926%:%
%:%2342=926%:%
%:%2343=927%:%
%:%2344=927%:%
%:%2345=927%:%
%:%2346=928%:%
%:%2347=928%:%
%:%2348=929%:%
%:%2349=929%:%
%:%2350=930%:%
%:%2351=930%:%
%:%2352=931%:%
%:%2353=931%:%
%:%2354=931%:%
%:%2355=932%:%
%:%2356=932%:%
%:%2357=932%:%
%:%2358=933%:%
%:%2359=933%:%
%:%2360=933%:%
%:%2361=934%:%
%:%2362=934%:%
%:%2363=934%:%
%:%2364=935%:%
%:%2365=935%:%
%:%2366=936%:%
%:%2367=936%:%
%:%2368=936%:%
%:%2369=937%:%
%:%2370=937%:%
%:%2371=937%:%
%:%2372=938%:%
%:%2373=938%:%
%:%2374=938%:%
%:%2375=939%:%
%:%2376=939%:%
%:%2377=940%:%
%:%2378=940%:%
%:%2379=941%:%
%:%2380=941%:%
%:%2381=942%:%
%:%2382=942%:%
%:%2383=943%:%
%:%2384=943%:%
%:%2385=944%:%
%:%2386=944%:%
%:%2387=944%:%
%:%2388=945%:%
%:%2389=945%:%
%:%2390=945%:%
%:%2391=946%:%
%:%2392=946%:%
%:%2393=946%:%
%:%2394=947%:%
%:%2395=947%:%
%:%2396=947%:%
%:%2397=948%:%
%:%2403=948%:%
%:%2406=949%:%
%:%2407=950%:%
%:%2408=950%:%
%:%2409=951%:%
%:%2410=952%:%
%:%2411=953%:%
%:%2414=954%:%
%:%2418=954%:%
%:%2419=954%:%
%:%2420=955%:%
%:%2421=955%:%
%:%2422=956%:%
%:%2423=956%:%
%:%2424=957%:%
%:%2425=957%:%
%:%2426=957%:%
%:%2427=957%:%
%:%2428=958%:%
%:%2429=958%:%
%:%2430=959%:%
%:%2431=959%:%
%:%2432=959%:%
%:%2433=959%:%
%:%2434=960%:%
%:%2435=960%:%
%:%2436=961%:%
%:%2437=961%:%
%:%2439=963%:%
%:%2440=964%:%
%:%2441=965%:%
%:%2442=965%:%
%:%2443=966%:%
%:%2444=966%:%
%:%2445=967%:%
%:%2446=967%:%
%:%2447=967%:%
%:%2450=970%:%
%:%2451=971%:%
%:%2452=971%:%
%:%2453=972%:%
%:%2454=972%:%
%:%2455=973%:%
%:%2456=973%:%
%:%2457=974%:%
%:%2458=974%:%
%:%2459=975%:%
%:%2460=975%:%
%:%2466=981%:%
%:%2467=982%:%
%:%2468=983%:%
%:%2469=983%:%
%:%2470=984%:%
%:%2471=984%:%
%:%2472=985%:%
%:%2473=985%:%
%:%2474=986%:%
%:%2475=986%:%
%:%2476=987%:%
%:%2477=987%:%
%:%2478=988%:%
%:%2479=988%:%
%:%2480=989%:%
%:%2481=990%:%
%:%2482=990%:%
%:%2483=991%:%
%:%2484=991%:%
%:%2485=991%:%
%:%2486=992%:%
%:%2487=993%:%
%:%2488=993%:%
%:%2493=998%:%
%:%2494=999%:%
%:%2495=999%:%
%:%2496=999%:%
%:%2497=1000%:%
%:%2498=1000%:%
%:%2499=1000%:%
%:%2504=1005%:%
%:%2505=1006%:%
%:%2506=1006%:%
%:%2507=1006%:%
%:%2508=1007%:%
%:%2509=1007%:%
%:%2510=1007%:%
%:%2515=1012%:%
%:%2516=1013%:%
%:%2517=1013%:%
%:%2518=1014%:%
%:%2519=1014%:%
%:%2520=1014%:%
%:%2525=1019%:%
%:%2526=1020%:%
%:%2527=1020%:%
%:%2528=1020%:%
%:%2529=1021%:%
%:%2530=1021%:%
%:%2531=1021%:%
%:%2536=1026%:%
%:%2537=1027%:%
%:%2538=1027%:%
%:%2539=1028%:%
%:%2540=1028%:%
%:%2541=1028%:%
%:%2546=1033%:%
%:%2547=1034%:%
%:%2548=1034%:%
%:%2549=1035%:%
%:%2550=1035%:%
%:%2551=1035%:%
%:%2556=1040%:%
%:%2557=1041%:%
%:%2558=1041%:%
%:%2559=1042%:%
%:%2560=1042%:%
%:%2561=1042%:%
%:%2566=1047%:%
%:%2567=1048%:%
%:%2568=1048%:%
%:%2569=1049%:%
%:%2570=1049%:%
%:%2571=1049%:%
%:%2576=1054%:%
%:%2577=1055%:%
%:%2578=1055%:%
%:%2579=1055%:%
%:%2580=1056%:%
%:%2581=1056%:%
%:%2582=1056%:%
%:%2587=1061%:%
%:%2588=1062%:%
%:%2589=1062%:%
%:%2590=1063%:%
%:%2591=1063%:%
%:%2592=1063%:%
%:%2597=1068%:%
%:%2598=1069%:%
%:%2599=1069%:%
%:%2600=1069%:%
%:%2601=1070%:%
%:%2602=1070%:%
%:%2603=1070%:%
%:%2606=1073%:%
%:%2607=1074%:%
%:%2608=1074%:%
%:%2609=1074%:%
%:%2610=1075%:%
%:%2611=1075%:%
%:%2612=1075%:%
%:%2615=1078%:%
%:%2616=1079%:%
%:%2617=1079%:%
%:%2618=1079%:%
%:%2619=1080%:%
%:%2620=1080%:%
%:%2621=1080%:%
%:%2624=1083%:%
%:%2625=1084%:%
%:%2626=1084%:%
%:%2627=1085%:%
%:%2628=1085%:%
%:%2629=1085%:%
%:%2630=1086%:%
%:%2631=1087%:%
%:%2632=1087%:%
%:%2633=1088%:%
%:%2634=1088%:%
%:%2635=1088%:%
%:%2636=1089%:%
%:%2637=1090%:%
%:%2638=1090%:%
%:%2639=1091%:%
%:%2640=1091%:%
%:%2641=1091%:%
%:%2642=1092%:%
%:%2643=1092%:%
%:%2644=1093%:%
%:%2645=1093%:%
%:%2646=1093%:%
%:%2647=1094%:%
%:%2648=1094%:%
%:%2649=1095%:%
%:%2650=1095%:%
%:%2651=1095%:%
%:%2652=1096%:%
%:%2653=1096%:%
%:%2654=1096%:%
%:%2655=1097%:%
%:%2656=1097%:%
%:%2657=1098%:%
%:%2658=1098%:%
%:%2659=1099%:%
%:%2660=1099%:%
%:%2661=1100%:%
%:%2662=1100%:%
%:%2663=1101%:%
%:%2669=1101%:%
%:%2672=1102%:%
%:%2673=1103%:%
%:%2674=1103%:%
%:%2675=1104%:%
%:%2676=1105%:%
%:%2679=1106%:%
%:%2683=1106%:%
%:%2684=1106%:%
%:%2685=1107%:%
%:%2686=1107%:%
%:%2687=1108%:%
%:%2688=1108%:%
%:%2689=1109%:%
%:%2690=1109%:%
%:%2691=1109%:%
%:%2692=1109%:%
%:%2693=1110%:%
%:%2694=1110%:%
%:%2695=1111%:%
%:%2696=1111%:%
%:%2697=1112%:%
%:%2698=1112%:%
%:%2699=1113%:%
%:%2700=1113%:%
%:%2701=1114%:%
%:%2702=1114%:%
%:%2703=1114%:%
%:%2704=1115%:%
%:%2705=1115%:%
%:%2706=1116%:%
%:%2707=1116%:%
%:%2708=1117%:%
%:%2709=1117%:%
%:%2710=1118%:%
%:%2711=1118%:%
%:%2712=1119%:%
%:%2713=1120%:%
%:%2714=1120%:%
%:%2715=1121%:%
%:%2716=1121%:%
%:%2717=1122%:%
%:%2718=1122%:%
%:%2719=1123%:%
%:%2720=1123%:%
%:%2721=1124%:%
%:%2722=1125%:%
%:%2723=1125%:%
%:%2724=1126%:%
%:%2725=1126%:%
%:%2726=1126%:%
%:%2727=1127%:%
%:%2728=1127%:%
%:%2729=1128%:%
%:%2730=1128%:%
%:%2731=1128%:%
%:%2732=1129%:%
%:%2733=1129%:%
%:%2734=1130%:%
%:%2735=1130%:%
%:%2736=1130%:%
%:%2737=1131%:%
%:%2738=1131%:%
%:%2739=1131%:%
%:%2740=1132%:%
%:%2741=1132%:%
%:%2742=1132%:%
%:%2743=1133%:%
%:%2744=1133%:%
%:%2745=1134%:%
%:%2746=1134%:%
%:%2747=1135%:%
%:%2748=1135%:%
%:%2749=1136%:%
%:%2750=1136%:%
%:%2751=1137%:%
%:%2757=1137%:%
%:%2760=1138%:%
%:%2761=1139%:%
%:%2762=1139%:%
%:%2763=1140%:%
%:%2764=1141%:%
%:%2765=1142%:%
%:%2766=1143%:%
%:%2773=1144%:%
%:%2774=1144%:%
%:%2775=1145%:%
%:%2776=1145%:%
%:%2777=1146%:%
%:%2778=1146%:%
%:%2779=1147%:%
%:%2780=1147%:%
%:%2781=1148%:%
%:%2782=1148%:%
%:%2783=1149%:%
%:%2784=1149%:%
%:%2785=1150%:%
%:%2786=1150%:%
%:%2787=1150%:%
%:%2788=1151%:%
%:%2794=1151%:%
%:%2799=1152%:%
%:%2804=1153%:%

%
\begin{isabellebody}%
\setisabellecontext{Countable}%
%
\isadelimtheory
%
\endisadelimtheory
%
\isatagtheory
\isacommand{theory}\isamarkupfalse%
\ Countable\isanewline
\ \ \isakeyword{imports}\ Nats\ Axiom{\isacharunderscore}{\kern0pt}Of{\isacharunderscore}{\kern0pt}Choice\ Nat{\isacharunderscore}{\kern0pt}Parity\ Cardinality\isanewline
\isakeyword{begin}%
\endisatagtheory
{\isafoldtheory}%
%
\isadelimtheory
%
\endisadelimtheory
%
\begin{isamarkuptext}%
The definition below corresponds to Definition 2.6.9 in Halvorson.%
\end{isamarkuptext}\isamarkuptrue%
\isacommand{definition}\isamarkupfalse%
\ epi{\isacharunderscore}{\kern0pt}countable\ {\isacharcolon}{\kern0pt}{\isacharcolon}{\kern0pt}\ {\isachardoublequoteopen}cset\ {\isasymRightarrow}\ bool{\isachardoublequoteclose}\ \isakeyword{where}\isanewline
\ \ {\isachardoublequoteopen}epi{\isacharunderscore}{\kern0pt}countable\ X\ {\isasymlongleftrightarrow}\ {\isacharparenleft}{\kern0pt}{\isasymexists}\ f{\isachardot}{\kern0pt}\ f\ {\isacharcolon}{\kern0pt}\ {\isasymnat}\isactrlsub c\ {\isasymrightarrow}\ X\ {\isasymand}\ epimorphism\ f{\isacharparenright}{\kern0pt}{\isachardoublequoteclose}\isanewline
\isanewline
\isacommand{lemma}\isamarkupfalse%
\ emptyset{\isacharunderscore}{\kern0pt}is{\isacharunderscore}{\kern0pt}not{\isacharunderscore}{\kern0pt}epi{\isacharunderscore}{\kern0pt}countable{\isacharcolon}{\kern0pt}\isanewline
\ \ {\isachardoublequoteopen}{\isasymnot}\ {\isacharparenleft}{\kern0pt}epi{\isacharunderscore}{\kern0pt}countable\ {\isasymemptyset}{\isacharparenright}{\kern0pt}{\isachardoublequoteclose}\isanewline
%
\isadelimproof
\ \ %
\endisadelimproof
%
\isatagproof
\isacommand{using}\isamarkupfalse%
\ comp{\isacharunderscore}{\kern0pt}type\ emptyset{\isacharunderscore}{\kern0pt}is{\isacharunderscore}{\kern0pt}empty\ epi{\isacharunderscore}{\kern0pt}countable{\isacharunderscore}{\kern0pt}def\ zero{\isacharunderscore}{\kern0pt}type\ \isacommand{by}\isamarkupfalse%
\ blast%
\endisatagproof
{\isafoldproof}%
%
\isadelimproof
%
\endisadelimproof
%
\begin{isamarkuptext}%
The fact that the empty set is not countable according to the definition from Halvorson
  (\isa{epi{\isacharunderscore}{\kern0pt}countable\ {\isacharquery}{\kern0pt}X\ {\isacharequal}{\kern0pt}\ {\isacharparenleft}{\kern0pt}{\isasymexists}f{\isachardot}{\kern0pt}\ f\ {\isacharcolon}{\kern0pt}\ {\isasymnat}\isactrlsub c\ {\isasymrightarrow}\ {\isacharquery}{\kern0pt}X\ {\isasymand}\ epimorphism\ f{\isacharparenright}{\kern0pt}}) motivated the following definition.%
\end{isamarkuptext}\isamarkuptrue%
\isacommand{definition}\isamarkupfalse%
\ countable\ {\isacharcolon}{\kern0pt}{\isacharcolon}{\kern0pt}\ {\isachardoublequoteopen}cset\ {\isasymRightarrow}\ bool{\isachardoublequoteclose}\ \isakeyword{where}\isanewline
\ \ {\isachardoublequoteopen}countable\ X\ {\isasymlongleftrightarrow}\ {\isacharparenleft}{\kern0pt}{\isasymexists}\ f{\isachardot}{\kern0pt}\ f\ {\isacharcolon}{\kern0pt}\ X\ {\isasymrightarrow}\ {\isasymnat}\isactrlsub c\ {\isasymand}\ monomorphism\ f{\isacharparenright}{\kern0pt}{\isachardoublequoteclose}\isanewline
\isanewline
\isacommand{lemma}\isamarkupfalse%
\ epi{\isacharunderscore}{\kern0pt}countable{\isacharunderscore}{\kern0pt}is{\isacharunderscore}{\kern0pt}countable{\isacharcolon}{\kern0pt}\ \isanewline
\ \ \isakeyword{assumes}\ {\isachardoublequoteopen}epi{\isacharunderscore}{\kern0pt}countable\ X{\isachardoublequoteclose}\isanewline
\ \ \isakeyword{shows}\ {\isachardoublequoteopen}countable\ X{\isachardoublequoteclose}\isanewline
%
\isadelimproof
\ \ %
\endisadelimproof
%
\isatagproof
\isacommand{using}\isamarkupfalse%
\ assms\ countable{\isacharunderscore}{\kern0pt}def\ epi{\isacharunderscore}{\kern0pt}countable{\isacharunderscore}{\kern0pt}def\ epis{\isacharunderscore}{\kern0pt}give{\isacharunderscore}{\kern0pt}monos\ \isacommand{by}\isamarkupfalse%
\ blast%
\endisatagproof
{\isafoldproof}%
%
\isadelimproof
\isanewline
%
\endisadelimproof
\isanewline
\isacommand{lemma}\isamarkupfalse%
\ emptyset{\isacharunderscore}{\kern0pt}is{\isacharunderscore}{\kern0pt}countable{\isacharcolon}{\kern0pt}\isanewline
\ \ {\isachardoublequoteopen}countable\ {\isasymemptyset}{\isachardoublequoteclose}\isanewline
%
\isadelimproof
\ \ %
\endisadelimproof
%
\isatagproof
\isacommand{using}\isamarkupfalse%
\ countable{\isacharunderscore}{\kern0pt}def\ empty{\isacharunderscore}{\kern0pt}subset\ subobject{\isacharunderscore}{\kern0pt}of{\isacharunderscore}{\kern0pt}def{\isadigit{2}}\ \isacommand{by}\isamarkupfalse%
\ blast%
\endisatagproof
{\isafoldproof}%
%
\isadelimproof
\isanewline
%
\endisadelimproof
\isanewline
\isacommand{lemma}\isamarkupfalse%
\ natural{\isacharunderscore}{\kern0pt}numbers{\isacharunderscore}{\kern0pt}are{\isacharunderscore}{\kern0pt}countably{\isacharunderscore}{\kern0pt}infinite{\isacharcolon}{\kern0pt}\isanewline
\ \ {\isachardoublequoteopen}{\isacharparenleft}{\kern0pt}countable\ {\isasymnat}\isactrlsub c{\isacharparenright}{\kern0pt}\ {\isasymand}\ {\isacharparenleft}{\kern0pt}is{\isacharunderscore}{\kern0pt}infinite\ {\isasymnat}\isactrlsub c{\isacharparenright}{\kern0pt}{\isachardoublequoteclose}\isanewline
%
\isadelimproof
\ \ %
\endisadelimproof
%
\isatagproof
\isacommand{by}\isamarkupfalse%
\ {\isacharparenleft}{\kern0pt}meson\ CollectI\ Peano{\isacharprime}{\kern0pt}s{\isacharunderscore}{\kern0pt}Axioms\ countable{\isacharunderscore}{\kern0pt}def\ injective{\isacharunderscore}{\kern0pt}imp{\isacharunderscore}{\kern0pt}monomorphism\ is{\isacharunderscore}{\kern0pt}infinite{\isacharunderscore}{\kern0pt}def\ successor{\isacharunderscore}{\kern0pt}type{\isacharparenright}{\kern0pt}%
\endisatagproof
{\isafoldproof}%
%
\isadelimproof
\isanewline
%
\endisadelimproof
\isanewline
\isacommand{lemma}\isamarkupfalse%
\ iso{\isacharunderscore}{\kern0pt}to{\isacharunderscore}{\kern0pt}N{\isacharunderscore}{\kern0pt}is{\isacharunderscore}{\kern0pt}countably{\isacharunderscore}{\kern0pt}infinite{\isacharcolon}{\kern0pt}\isanewline
\ \ \isakeyword{assumes}\ {\isachardoublequoteopen}X\ {\isasymcong}\ {\isasymnat}\isactrlsub c{\isachardoublequoteclose}\isanewline
\ \ \isakeyword{shows}\ {\isachardoublequoteopen}{\isacharparenleft}{\kern0pt}countable\ X{\isacharparenright}{\kern0pt}\ {\isasymand}\ {\isacharparenleft}{\kern0pt}is{\isacharunderscore}{\kern0pt}infinite\ X{\isacharparenright}{\kern0pt}{\isachardoublequoteclose}\isanewline
%
\isadelimproof
\ \ %
\endisadelimproof
%
\isatagproof
\isacommand{by}\isamarkupfalse%
\ {\isacharparenleft}{\kern0pt}meson\ assms\ countable{\isacharunderscore}{\kern0pt}def\ is{\isacharunderscore}{\kern0pt}isomorphic{\isacharunderscore}{\kern0pt}def\ is{\isacharunderscore}{\kern0pt}smaller{\isacharunderscore}{\kern0pt}than{\isacharunderscore}{\kern0pt}def\ iso{\isacharunderscore}{\kern0pt}imp{\isacharunderscore}{\kern0pt}epi{\isacharunderscore}{\kern0pt}and{\isacharunderscore}{\kern0pt}monic\ isomorphic{\isacharunderscore}{\kern0pt}is{\isacharunderscore}{\kern0pt}symmetric\ larger{\isacharunderscore}{\kern0pt}than{\isacharunderscore}{\kern0pt}infinite{\isacharunderscore}{\kern0pt}is{\isacharunderscore}{\kern0pt}infinite\ natural{\isacharunderscore}{\kern0pt}numbers{\isacharunderscore}{\kern0pt}are{\isacharunderscore}{\kern0pt}countably{\isacharunderscore}{\kern0pt}infinite{\isacharparenright}{\kern0pt}%
\endisatagproof
{\isafoldproof}%
%
\isadelimproof
\isanewline
%
\endisadelimproof
\isanewline
\isacommand{lemma}\isamarkupfalse%
\ smaller{\isacharunderscore}{\kern0pt}than{\isacharunderscore}{\kern0pt}countable{\isacharunderscore}{\kern0pt}is{\isacharunderscore}{\kern0pt}countable{\isacharcolon}{\kern0pt}\isanewline
\ \ \isakeyword{assumes}\ {\isachardoublequoteopen}X\ {\isasymle}\isactrlsub c\ Y{\isachardoublequoteclose}\ {\isachardoublequoteopen}countable\ Y{\isachardoublequoteclose}\isanewline
\ \ \isakeyword{shows}\ {\isachardoublequoteopen}countable\ X{\isachardoublequoteclose}\isanewline
%
\isadelimproof
\ \ %
\endisadelimproof
%
\isatagproof
\isacommand{by}\isamarkupfalse%
\ {\isacharparenleft}{\kern0pt}smt\ assms\ cfunc{\isacharunderscore}{\kern0pt}type{\isacharunderscore}{\kern0pt}def\ comp{\isacharunderscore}{\kern0pt}type\ composition{\isacharunderscore}{\kern0pt}of{\isacharunderscore}{\kern0pt}monic{\isacharunderscore}{\kern0pt}pair{\isacharunderscore}{\kern0pt}is{\isacharunderscore}{\kern0pt}monic\ countable{\isacharunderscore}{\kern0pt}def\ is{\isacharunderscore}{\kern0pt}smaller{\isacharunderscore}{\kern0pt}than{\isacharunderscore}{\kern0pt}def{\isacharparenright}{\kern0pt}%
\endisatagproof
{\isafoldproof}%
%
\isadelimproof
\isanewline
%
\endisadelimproof
\isanewline
\isacommand{lemma}\isamarkupfalse%
\ iso{\isacharunderscore}{\kern0pt}pres{\isacharunderscore}{\kern0pt}countable{\isacharcolon}{\kern0pt}\isanewline
\ \ \isakeyword{assumes}\ {\isachardoublequoteopen}X\ {\isasymcong}\ Y{\isachardoublequoteclose}\ {\isachardoublequoteopen}countable\ Y{\isachardoublequoteclose}\isanewline
\ \ \isakeyword{shows}\ {\isachardoublequoteopen}countable\ X{\isachardoublequoteclose}\isanewline
%
\isadelimproof
\ \ %
\endisadelimproof
%
\isatagproof
\isacommand{using}\isamarkupfalse%
\ assms\ is{\isacharunderscore}{\kern0pt}isomorphic{\isacharunderscore}{\kern0pt}def\ is{\isacharunderscore}{\kern0pt}smaller{\isacharunderscore}{\kern0pt}than{\isacharunderscore}{\kern0pt}def\ iso{\isacharunderscore}{\kern0pt}imp{\isacharunderscore}{\kern0pt}epi{\isacharunderscore}{\kern0pt}and{\isacharunderscore}{\kern0pt}monic\ smaller{\isacharunderscore}{\kern0pt}than{\isacharunderscore}{\kern0pt}countable{\isacharunderscore}{\kern0pt}is{\isacharunderscore}{\kern0pt}countable\ \isacommand{by}\isamarkupfalse%
\ blast%
\endisatagproof
{\isafoldproof}%
%
\isadelimproof
\isanewline
%
\endisadelimproof
\isanewline
\isacommand{lemma}\isamarkupfalse%
\ NuN{\isacharunderscore}{\kern0pt}is{\isacharunderscore}{\kern0pt}countable{\isacharcolon}{\kern0pt}\isanewline
\ \ {\isachardoublequoteopen}countable{\isacharparenleft}{\kern0pt}{\isasymnat}\isactrlsub c\ {\isasymCoprod}\ {\isasymnat}\isactrlsub c{\isacharparenright}{\kern0pt}{\isachardoublequoteclose}\isanewline
%
\isadelimproof
\ \ %
\endisadelimproof
%
\isatagproof
\isacommand{using}\isamarkupfalse%
\ countable{\isacharunderscore}{\kern0pt}def\ epis{\isacharunderscore}{\kern0pt}give{\isacharunderscore}{\kern0pt}monos\ halve{\isacharunderscore}{\kern0pt}with{\isacharunderscore}{\kern0pt}parity{\isacharunderscore}{\kern0pt}iso\ halve{\isacharunderscore}{\kern0pt}with{\isacharunderscore}{\kern0pt}parity{\isacharunderscore}{\kern0pt}type\ iso{\isacharunderscore}{\kern0pt}imp{\isacharunderscore}{\kern0pt}epi{\isacharunderscore}{\kern0pt}and{\isacharunderscore}{\kern0pt}monic\ \isacommand{by}\isamarkupfalse%
\ smt%
\endisatagproof
{\isafoldproof}%
%
\isadelimproof
%
\endisadelimproof
%
\begin{isamarkuptext}%
The lemma below corresponds to Exercise 2.6.11 in Halvorson.%
\end{isamarkuptext}\isamarkuptrue%
\isacommand{lemma}\isamarkupfalse%
\ coproduct{\isacharunderscore}{\kern0pt}of{\isacharunderscore}{\kern0pt}countables{\isacharunderscore}{\kern0pt}is{\isacharunderscore}{\kern0pt}countable{\isacharcolon}{\kern0pt}\isanewline
\ \ \isakeyword{assumes}\ {\isachardoublequoteopen}countable\ X{\isachardoublequoteclose}\ {\isachardoublequoteopen}countable\ Y{\isachardoublequoteclose}\isanewline
\ \ \isakeyword{shows}\ {\isachardoublequoteopen}countable{\isacharparenleft}{\kern0pt}X\ {\isasymCoprod}\ Y{\isacharparenright}{\kern0pt}{\isachardoublequoteclose}\isanewline
%
\isadelimproof
\ \ %
\endisadelimproof
%
\isatagproof
\isacommand{unfolding}\isamarkupfalse%
\ countable{\isacharunderscore}{\kern0pt}def\isanewline
\isacommand{proof}\isamarkupfalse%
{\isacharminus}{\kern0pt}\isanewline
\ \ \isacommand{obtain}\isamarkupfalse%
\ x\ \isakeyword{where}\ x{\isacharunderscore}{\kern0pt}def{\isacharcolon}{\kern0pt}\ \ {\isachardoublequoteopen}x\ {\isacharcolon}{\kern0pt}\ X\ \ {\isasymrightarrow}\ {\isasymnat}\isactrlsub c\ {\isasymand}\ monomorphism\ x{\isachardoublequoteclose}\isanewline
\ \ \ \ \isacommand{using}\isamarkupfalse%
\ assms{\isacharparenleft}{\kern0pt}{\isadigit{1}}{\isacharparenright}{\kern0pt}\ countable{\isacharunderscore}{\kern0pt}def\ \isacommand{by}\isamarkupfalse%
\ blast\isanewline
\ \ \isacommand{obtain}\isamarkupfalse%
\ y\ \isakeyword{where}\ y{\isacharunderscore}{\kern0pt}def{\isacharcolon}{\kern0pt}\ \ {\isachardoublequoteopen}y\ {\isacharcolon}{\kern0pt}\ Y\ \ {\isasymrightarrow}\ {\isasymnat}\isactrlsub c\ {\isasymand}\ monomorphism\ y{\isachardoublequoteclose}\isanewline
\ \ \ \ \isacommand{using}\isamarkupfalse%
\ assms{\isacharparenleft}{\kern0pt}{\isadigit{2}}{\isacharparenright}{\kern0pt}\ countable{\isacharunderscore}{\kern0pt}def\ \isacommand{by}\isamarkupfalse%
\ blast\isanewline
\ \ \isacommand{obtain}\isamarkupfalse%
\ n\ \isakeyword{where}\ n{\isacharunderscore}{\kern0pt}def{\isacharcolon}{\kern0pt}\ {\isachardoublequoteopen}\ n\ {\isacharcolon}{\kern0pt}\ {\isasymnat}\isactrlsub c\ {\isasymCoprod}\ {\isasymnat}\isactrlsub c\ {\isasymrightarrow}\ {\isasymnat}\isactrlsub c\ {\isasymand}\ monomorphism\ n{\isachardoublequoteclose}\isanewline
\ \ \ \ \isacommand{using}\isamarkupfalse%
\ NuN{\isacharunderscore}{\kern0pt}is{\isacharunderscore}{\kern0pt}countable\ countable{\isacharunderscore}{\kern0pt}def\ \isacommand{by}\isamarkupfalse%
\ blast\isanewline
\ \ \isacommand{have}\isamarkupfalse%
\ xy{\isacharunderscore}{\kern0pt}type{\isacharcolon}{\kern0pt}\ {\isachardoublequoteopen}x\ {\isasymbowtie}\isactrlsub f\ y\ {\isacharcolon}{\kern0pt}\ X\ {\isasymCoprod}\ Y\ {\isasymrightarrow}\ {\isasymnat}\isactrlsub c\ {\isasymCoprod}\ {\isasymnat}\isactrlsub c{\isachardoublequoteclose}\isanewline
\ \ \ \ \isacommand{using}\isamarkupfalse%
\ x{\isacharunderscore}{\kern0pt}def\ y{\isacharunderscore}{\kern0pt}def\ \isacommand{by}\isamarkupfalse%
\ {\isacharparenleft}{\kern0pt}typecheck{\isacharunderscore}{\kern0pt}cfuncs{\isacharcomma}{\kern0pt}\ auto{\isacharparenright}{\kern0pt}\isanewline
\ \ \isacommand{then}\isamarkupfalse%
\ \isacommand{have}\isamarkupfalse%
\ nxy{\isacharunderscore}{\kern0pt}type{\isacharcolon}{\kern0pt}\ {\isachardoublequoteopen}n\ {\isasymcirc}\isactrlsub c\ {\isacharparenleft}{\kern0pt}x\ {\isasymbowtie}\isactrlsub f\ y{\isacharparenright}{\kern0pt}\ {\isacharcolon}{\kern0pt}\ X\ {\isasymCoprod}\ Y\ {\isasymrightarrow}\ {\isasymnat}\isactrlsub c{\isachardoublequoteclose}\isanewline
\ \ \ \ \isacommand{using}\isamarkupfalse%
\ comp{\isacharunderscore}{\kern0pt}type\ n{\isacharunderscore}{\kern0pt}def\ \isacommand{by}\isamarkupfalse%
\ blast\isanewline
\ \ \isacommand{have}\isamarkupfalse%
\ {\isachardoublequoteopen}injective{\isacharparenleft}{\kern0pt}x\ {\isasymbowtie}\isactrlsub f\ y{\isacharparenright}{\kern0pt}{\isachardoublequoteclose}\isanewline
\ \ \ \ \isacommand{using}\isamarkupfalse%
\ cfunc{\isacharunderscore}{\kern0pt}bowtieprod{\isacharunderscore}{\kern0pt}inj\ monomorphism{\isacharunderscore}{\kern0pt}imp{\isacharunderscore}{\kern0pt}injective\ x{\isacharunderscore}{\kern0pt}def\ y{\isacharunderscore}{\kern0pt}def\ \isacommand{by}\isamarkupfalse%
\ blast\isanewline
\ \ \isacommand{then}\isamarkupfalse%
\ \isacommand{have}\isamarkupfalse%
\ {\isachardoublequoteopen}monomorphism{\isacharparenleft}{\kern0pt}x\ {\isasymbowtie}\isactrlsub f\ y{\isacharparenright}{\kern0pt}{\isachardoublequoteclose}\isanewline
\ \ \ \ \isacommand{using}\isamarkupfalse%
\ injective{\isacharunderscore}{\kern0pt}imp{\isacharunderscore}{\kern0pt}monomorphism\ \isacommand{by}\isamarkupfalse%
\ auto\isanewline
\ \ \isacommand{then}\isamarkupfalse%
\ \isacommand{have}\isamarkupfalse%
\ {\isachardoublequoteopen}monomorphism{\isacharparenleft}{\kern0pt}n\ {\isasymcirc}\isactrlsub c\ {\isacharparenleft}{\kern0pt}x\ {\isasymbowtie}\isactrlsub f\ y{\isacharparenright}{\kern0pt}{\isacharparenright}{\kern0pt}{\isachardoublequoteclose}\isanewline
\ \ \ \ \isacommand{using}\isamarkupfalse%
\ cfunc{\isacharunderscore}{\kern0pt}type{\isacharunderscore}{\kern0pt}def\ composition{\isacharunderscore}{\kern0pt}of{\isacharunderscore}{\kern0pt}monic{\isacharunderscore}{\kern0pt}pair{\isacharunderscore}{\kern0pt}is{\isacharunderscore}{\kern0pt}monic\ n{\isacharunderscore}{\kern0pt}def\ xy{\isacharunderscore}{\kern0pt}type\ \isacommand{by}\isamarkupfalse%
\ auto\isanewline
\ \ \isacommand{then}\isamarkupfalse%
\ \isacommand{show}\isamarkupfalse%
\ {\isachardoublequoteopen}{\isasymexists}f{\isachardot}{\kern0pt}\ f\ {\isacharcolon}{\kern0pt}\ X\ {\isasymCoprod}\ Y\ {\isasymrightarrow}\ {\isasymnat}\isactrlsub c\ {\isasymand}\ monomorphism\ f{\isachardoublequoteclose}\isanewline
\ \ \ \ \isacommand{using}\isamarkupfalse%
\ nxy{\isacharunderscore}{\kern0pt}type\ \isacommand{by}\isamarkupfalse%
\ blast\isanewline
\isacommand{qed}\isamarkupfalse%
%
\endisatagproof
{\isafoldproof}%
%
\isadelimproof
\isanewline
%
\endisadelimproof
%
\isadelimtheory
\isanewline
%
\endisadelimtheory
%
\isatagtheory
\isacommand{end}\isamarkupfalse%
%
\endisatagtheory
{\isafoldtheory}%
%
\isadelimtheory
%
\endisadelimtheory
%
\end{isabellebody}%
\endinput
%:%file=~/ETCS/HOL-ETCS/Countable.thy%:%
%:%10=1%:%
%:%11=1%:%
%:%12=2%:%
%:%13=3%:%
%:%22=5%:%
%:%24=6%:%
%:%25=6%:%
%:%26=7%:%
%:%27=8%:%
%:%28=9%:%
%:%29=9%:%
%:%30=10%:%
%:%33=11%:%
%:%37=11%:%
%:%38=11%:%
%:%39=11%:%
%:%48=13%:%
%:%49=14%:%
%:%51=15%:%
%:%52=15%:%
%:%53=16%:%
%:%54=17%:%
%:%55=18%:%
%:%56=18%:%
%:%57=19%:%
%:%58=20%:%
%:%61=21%:%
%:%65=21%:%
%:%66=21%:%
%:%67=21%:%
%:%72=21%:%
%:%75=22%:%
%:%76=23%:%
%:%77=23%:%
%:%78=24%:%
%:%81=25%:%
%:%85=25%:%
%:%86=25%:%
%:%87=25%:%
%:%92=25%:%
%:%95=26%:%
%:%96=27%:%
%:%97=27%:%
%:%98=28%:%
%:%101=29%:%
%:%105=29%:%
%:%106=29%:%
%:%111=29%:%
%:%114=30%:%
%:%115=31%:%
%:%116=31%:%
%:%117=32%:%
%:%118=33%:%
%:%121=34%:%
%:%125=34%:%
%:%126=34%:%
%:%131=34%:%
%:%134=35%:%
%:%135=36%:%
%:%136=36%:%
%:%137=37%:%
%:%138=38%:%
%:%141=39%:%
%:%145=39%:%
%:%146=39%:%
%:%151=39%:%
%:%154=40%:%
%:%155=41%:%
%:%156=41%:%
%:%157=42%:%
%:%158=43%:%
%:%161=44%:%
%:%165=44%:%
%:%166=44%:%
%:%167=44%:%
%:%172=44%:%
%:%175=45%:%
%:%176=46%:%
%:%177=46%:%
%:%178=47%:%
%:%181=48%:%
%:%185=48%:%
%:%186=48%:%
%:%187=48%:%
%:%196=50%:%
%:%198=51%:%
%:%199=51%:%
%:%200=52%:%
%:%201=53%:%
%:%204=54%:%
%:%208=54%:%
%:%209=54%:%
%:%210=55%:%
%:%211=55%:%
%:%212=56%:%
%:%213=56%:%
%:%214=57%:%
%:%215=57%:%
%:%216=57%:%
%:%217=58%:%
%:%218=58%:%
%:%219=59%:%
%:%220=59%:%
%:%221=59%:%
%:%222=60%:%
%:%223=60%:%
%:%224=61%:%
%:%225=61%:%
%:%226=61%:%
%:%227=62%:%
%:%228=62%:%
%:%229=63%:%
%:%230=63%:%
%:%231=63%:%
%:%232=64%:%
%:%233=64%:%
%:%234=64%:%
%:%235=65%:%
%:%236=65%:%
%:%237=65%:%
%:%238=66%:%
%:%239=66%:%
%:%240=67%:%
%:%241=67%:%
%:%242=67%:%
%:%243=68%:%
%:%244=68%:%
%:%245=68%:%
%:%246=69%:%
%:%247=69%:%
%:%248=69%:%
%:%249=70%:%
%:%250=70%:%
%:%251=70%:%
%:%252=71%:%
%:%253=71%:%
%:%254=71%:%
%:%255=72%:%
%:%256=72%:%
%:%257=72%:%
%:%258=73%:%
%:%259=73%:%
%:%260=73%:%
%:%261=74%:%
%:%267=74%:%
%:%272=75%:%
%:%277=76%:%

%
\begin{isabellebody}%
\setisabellecontext{Fixed{\isacharunderscore}{\kern0pt}Points}%
%
\isadelimdocument
%
\endisadelimdocument
%
\isatagdocument
%
\isamarkupsection{Fixed Points and Cantor's Theorems%
}
\isamarkuptrue%
%
\endisatagdocument
{\isafolddocument}%
%
\isadelimdocument
%
\endisadelimdocument
%
\isadelimtheory
%
\endisadelimtheory
%
\isatagtheory
\isacommand{theory}\isamarkupfalse%
\ Fixed{\isacharunderscore}{\kern0pt}Points\isanewline
\ \ \isakeyword{imports}\ Axiom{\isacharunderscore}{\kern0pt}Of{\isacharunderscore}{\kern0pt}Choice\ Pred{\isacharunderscore}{\kern0pt}Logic\ Cardinality\isanewline
\isakeyword{begin}%
\endisatagtheory
{\isafoldtheory}%
%
\isadelimtheory
%
\endisadelimtheory
%
\begin{isamarkuptext}%
The definitions below correspond to Definition 2.6.12 in Halvorson.%
\end{isamarkuptext}\isamarkuptrue%
\isacommand{definition}\isamarkupfalse%
\ fixed{\isacharunderscore}{\kern0pt}point\ {\isacharcolon}{\kern0pt}{\isacharcolon}{\kern0pt}\ {\isachardoublequoteopen}cfunc\ {\isasymRightarrow}\ cfunc\ {\isasymRightarrow}\ bool{\isachardoublequoteclose}\ \isakeyword{where}\isanewline
\ \ {\isachardoublequoteopen}fixed{\isacharunderscore}{\kern0pt}point\ a\ g\ {\isasymlongleftrightarrow}\ {\isacharparenleft}{\kern0pt}{\isasymexists}\ A{\isachardot}{\kern0pt}\ g\ {\isacharcolon}{\kern0pt}\ A\ {\isasymrightarrow}\ A\ {\isasymand}\ a\ {\isasymin}\isactrlsub c\ A\ {\isasymand}\ g\ {\isasymcirc}\isactrlsub c\ a\ {\isacharequal}{\kern0pt}\ a{\isacharparenright}{\kern0pt}{\isachardoublequoteclose}\isanewline
\isacommand{definition}\isamarkupfalse%
\ has{\isacharunderscore}{\kern0pt}fixed{\isacharunderscore}{\kern0pt}point\ {\isacharcolon}{\kern0pt}{\isacharcolon}{\kern0pt}\ {\isachardoublequoteopen}cfunc\ {\isasymRightarrow}\ bool{\isachardoublequoteclose}\ \isakeyword{where}\isanewline
\ \ {\isachardoublequoteopen}has{\isacharunderscore}{\kern0pt}fixed{\isacharunderscore}{\kern0pt}point\ g\ {\isasymlongleftrightarrow}\ {\isacharparenleft}{\kern0pt}{\isasymexists}\ a{\isachardot}{\kern0pt}\ fixed{\isacharunderscore}{\kern0pt}point\ a\ g{\isacharparenright}{\kern0pt}{\isachardoublequoteclose}\isanewline
\isacommand{definition}\isamarkupfalse%
\ fixed{\isacharunderscore}{\kern0pt}point{\isacharunderscore}{\kern0pt}property\ {\isacharcolon}{\kern0pt}{\isacharcolon}{\kern0pt}\ {\isachardoublequoteopen}cset\ {\isasymRightarrow}\ bool{\isachardoublequoteclose}\ \isakeyword{where}\isanewline
\ \ {\isachardoublequoteopen}fixed{\isacharunderscore}{\kern0pt}point{\isacharunderscore}{\kern0pt}property\ A\ {\isasymlongleftrightarrow}\ {\isacharparenleft}{\kern0pt}{\isasymforall}\ g{\isachardot}{\kern0pt}\ g\ {\isacharcolon}{\kern0pt}\ A\ {\isasymrightarrow}\ A\ {\isasymlongrightarrow}\ has{\isacharunderscore}{\kern0pt}fixed{\isacharunderscore}{\kern0pt}point\ g{\isacharparenright}{\kern0pt}{\isachardoublequoteclose}\isanewline
\isanewline
\isacommand{lemma}\isamarkupfalse%
\ fixed{\isacharunderscore}{\kern0pt}point{\isacharunderscore}{\kern0pt}def{\isadigit{2}}{\isacharcolon}{\kern0pt}\ \isanewline
\ \ \isakeyword{assumes}\ {\isachardoublequoteopen}g\ {\isacharcolon}{\kern0pt}\ A\ {\isasymrightarrow}\ A{\isachardoublequoteclose}\ {\isachardoublequoteopen}a\ {\isasymin}\isactrlsub c\ A{\isachardoublequoteclose}\isanewline
\ \ \isakeyword{shows}\ {\isachardoublequoteopen}fixed{\isacharunderscore}{\kern0pt}point\ a\ g\ {\isacharequal}{\kern0pt}\ {\isacharparenleft}{\kern0pt}g\ {\isasymcirc}\isactrlsub c\ a\ {\isacharequal}{\kern0pt}\ a{\isacharparenright}{\kern0pt}{\isachardoublequoteclose}\isanewline
%
\isadelimproof
\ \ %
\endisadelimproof
%
\isatagproof
\isacommand{unfolding}\isamarkupfalse%
\ fixed{\isacharunderscore}{\kern0pt}point{\isacharunderscore}{\kern0pt}def\ \isacommand{using}\isamarkupfalse%
\ assms\ \isacommand{by}\isamarkupfalse%
\ blast%
\endisatagproof
{\isafoldproof}%
%
\isadelimproof
%
\endisadelimproof
%
\begin{isamarkuptext}%
The lemma below corresponds to Theorem 2.6.13 in Halvorson.%
\end{isamarkuptext}\isamarkuptrue%
\isacommand{lemma}\isamarkupfalse%
\ Lawveres{\isacharunderscore}{\kern0pt}fixed{\isacharunderscore}{\kern0pt}point{\isacharunderscore}{\kern0pt}theorem{\isacharcolon}{\kern0pt}\isanewline
\ \ \isakeyword{assumes}\ p{\isacharunderscore}{\kern0pt}type{\isacharbrackleft}{\kern0pt}type{\isacharunderscore}{\kern0pt}rule{\isacharbrackright}{\kern0pt}{\isacharcolon}{\kern0pt}\ {\isachardoublequoteopen}p\ {\isacharcolon}{\kern0pt}\ X\ {\isasymrightarrow}\ A\isactrlbsup X\isactrlesup {\isachardoublequoteclose}\isanewline
\ \ \isakeyword{assumes}\ p{\isacharunderscore}{\kern0pt}surj{\isacharcolon}{\kern0pt}\ {\isachardoublequoteopen}surjective\ p{\isachardoublequoteclose}\isanewline
\ \ \isakeyword{shows}\ {\isachardoublequoteopen}fixed{\isacharunderscore}{\kern0pt}point{\isacharunderscore}{\kern0pt}property\ A{\isachardoublequoteclose}\isanewline
%
\isadelimproof
\ \ %
\endisadelimproof
%
\isatagproof
\isacommand{unfolding}\isamarkupfalse%
\ fixed{\isacharunderscore}{\kern0pt}point{\isacharunderscore}{\kern0pt}property{\isacharunderscore}{\kern0pt}def\ has{\isacharunderscore}{\kern0pt}fixed{\isacharunderscore}{\kern0pt}point{\isacharunderscore}{\kern0pt}def\isanewline
\isacommand{proof}\isamarkupfalse%
{\isacharparenleft}{\kern0pt}clarify{\isacharparenright}{\kern0pt}\ \isanewline
\ \ \isacommand{fix}\isamarkupfalse%
\ g\ \isanewline
\ \ \isacommand{assume}\isamarkupfalse%
\ g{\isacharunderscore}{\kern0pt}type{\isacharbrackleft}{\kern0pt}type{\isacharunderscore}{\kern0pt}rule{\isacharbrackright}{\kern0pt}{\isacharcolon}{\kern0pt}\ {\isachardoublequoteopen}g\ {\isacharcolon}{\kern0pt}\ A\ {\isasymrightarrow}\ A{\isachardoublequoteclose}\isanewline
\ \ \isacommand{obtain}\isamarkupfalse%
\ {\isasymphi}\ \isakeyword{where}\ {\isasymphi}{\isacharunderscore}{\kern0pt}def{\isacharcolon}{\kern0pt}\ {\isachardoublequoteopen}{\isasymphi}\ {\isacharequal}{\kern0pt}\ p\isactrlsup {\isasymflat}{\isachardoublequoteclose}\isanewline
\ \ \ \ \isacommand{by}\isamarkupfalse%
\ auto\isanewline
\ \ \isacommand{then}\isamarkupfalse%
\ \isacommand{have}\isamarkupfalse%
\ {\isasymphi}{\isacharunderscore}{\kern0pt}type{\isacharbrackleft}{\kern0pt}type{\isacharunderscore}{\kern0pt}rule{\isacharbrackright}{\kern0pt}{\isacharcolon}{\kern0pt}\ {\isachardoublequoteopen}{\isasymphi}\ {\isacharcolon}{\kern0pt}\ X\ {\isasymtimes}\isactrlsub c\ X\ {\isasymrightarrow}\ A{\isachardoublequoteclose}\isanewline
\ \ \ \ \isacommand{by}\isamarkupfalse%
\ {\isacharparenleft}{\kern0pt}simp\ add{\isacharcolon}{\kern0pt}\ flat{\isacharunderscore}{\kern0pt}type\ p{\isacharunderscore}{\kern0pt}type{\isacharparenright}{\kern0pt}\isanewline
\ \ \isacommand{obtain}\isamarkupfalse%
\ f\ \isakeyword{where}\ f{\isacharunderscore}{\kern0pt}def{\isacharcolon}{\kern0pt}\ {\isachardoublequoteopen}f\ {\isacharequal}{\kern0pt}\ g\ {\isasymcirc}\isactrlsub c\ {\isasymphi}\ {\isasymcirc}\isactrlsub c\ diagonal{\isacharparenleft}{\kern0pt}X{\isacharparenright}{\kern0pt}{\isachardoublequoteclose}\isanewline
\ \ \ \ \isacommand{by}\isamarkupfalse%
\ auto\isanewline
\ \ \isacommand{then}\isamarkupfalse%
\ \isacommand{have}\isamarkupfalse%
\ f{\isacharunderscore}{\kern0pt}type{\isacharbrackleft}{\kern0pt}type{\isacharunderscore}{\kern0pt}rule{\isacharbrackright}{\kern0pt}{\isacharcolon}{\kern0pt}{\isachardoublequoteopen}f\ {\isacharcolon}{\kern0pt}\ X\ {\isasymrightarrow}\ A{\isachardoublequoteclose}\isanewline
\ \ \ \ \isacommand{using}\isamarkupfalse%
\ {\isasymphi}{\isacharunderscore}{\kern0pt}type\ comp{\isacharunderscore}{\kern0pt}type\ diagonal{\isacharunderscore}{\kern0pt}type\ f{\isacharunderscore}{\kern0pt}def\ g{\isacharunderscore}{\kern0pt}type\ \isacommand{by}\isamarkupfalse%
\ blast\isanewline
\ \ \isacommand{obtain}\isamarkupfalse%
\ x{\isacharunderscore}{\kern0pt}f\ \isakeyword{where}\ x{\isacharunderscore}{\kern0pt}f{\isacharcolon}{\kern0pt}\ {\isachardoublequoteopen}metafunc\ f\ {\isacharequal}{\kern0pt}\ p\ {\isasymcirc}\isactrlsub c\ x{\isacharunderscore}{\kern0pt}f{\isachardoublequoteclose}\ \isakeyword{and}\ x{\isacharunderscore}{\kern0pt}f{\isacharunderscore}{\kern0pt}type{\isacharbrackleft}{\kern0pt}type{\isacharunderscore}{\kern0pt}rule{\isacharbrackright}{\kern0pt}{\isacharcolon}{\kern0pt}\ {\isachardoublequoteopen}x{\isacharunderscore}{\kern0pt}f\ {\isasymin}\isactrlsub c\ X{\isachardoublequoteclose}\isanewline
\ \ \ \ \isacommand{using}\isamarkupfalse%
\ assms\ \isacommand{by}\isamarkupfalse%
\ {\isacharparenleft}{\kern0pt}typecheck{\isacharunderscore}{\kern0pt}cfuncs{\isacharcomma}{\kern0pt}\ metis\ p{\isacharunderscore}{\kern0pt}surj\ surjective{\isacharunderscore}{\kern0pt}def{\isadigit{2}}{\isacharparenright}{\kern0pt}\isanewline
\ \ \isacommand{have}\isamarkupfalse%
\ {\isachardoublequoteopen}{\isasymphi}\isactrlbsub {\isacharbrackleft}{\kern0pt}{\isacharminus}{\kern0pt}{\isacharcomma}{\kern0pt}x{\isacharunderscore}{\kern0pt}f{\isacharbrackright}{\kern0pt}\isactrlesub \ {\isacharequal}{\kern0pt}\ f{\isachardoublequoteclose}\isanewline
\ \ \isacommand{proof}\isamarkupfalse%
{\isacharparenleft}{\kern0pt}etcs{\isacharunderscore}{\kern0pt}rule\ one{\isacharunderscore}{\kern0pt}separator{\isacharparenright}{\kern0pt}\isanewline
\ \ \ \ \isacommand{fix}\isamarkupfalse%
\ x\ \isanewline
\ \ \ \ \isacommand{assume}\isamarkupfalse%
\ x{\isacharunderscore}{\kern0pt}type{\isacharbrackleft}{\kern0pt}type{\isacharunderscore}{\kern0pt}rule{\isacharbrackright}{\kern0pt}{\isacharcolon}{\kern0pt}\ {\isachardoublequoteopen}x\ {\isasymin}\isactrlsub c\ X{\isachardoublequoteclose}\isanewline
\ \ \ \ \isacommand{have}\isamarkupfalse%
\ {\isachardoublequoteopen}{\isasymphi}\isactrlbsub {\isacharbrackleft}{\kern0pt}{\isacharminus}{\kern0pt}{\isacharcomma}{\kern0pt}x{\isacharunderscore}{\kern0pt}f{\isacharbrackright}{\kern0pt}\isactrlesub \ {\isasymcirc}\isactrlsub c\ x\ {\isacharequal}{\kern0pt}\ {\isasymphi}\ {\isasymcirc}\isactrlsub c\ {\isasymlangle}x{\isacharcomma}{\kern0pt}\ x{\isacharunderscore}{\kern0pt}f{\isasymrangle}{\isachardoublequoteclose}\isanewline
\ \ \ \ \ \ \isacommand{by}\isamarkupfalse%
\ {\isacharparenleft}{\kern0pt}typecheck{\isacharunderscore}{\kern0pt}cfuncs{\isacharcomma}{\kern0pt}\ meson\ right{\isacharunderscore}{\kern0pt}param{\isacharunderscore}{\kern0pt}on{\isacharunderscore}{\kern0pt}el\ x{\isacharunderscore}{\kern0pt}f{\isacharparenright}{\kern0pt}\isanewline
\ \ \ \ \isacommand{also}\isamarkupfalse%
\ \isacommand{have}\isamarkupfalse%
\ {\isachardoublequoteopen}{\isachardot}{\kern0pt}{\isachardot}{\kern0pt}{\isachardot}{\kern0pt}\ {\isacharequal}{\kern0pt}\ {\isacharparenleft}{\kern0pt}{\isacharparenleft}{\kern0pt}eval{\isacharunderscore}{\kern0pt}func\ A\ X{\isacharparenright}{\kern0pt}\ {\isasymcirc}\isactrlsub c\ {\isacharparenleft}{\kern0pt}id\ X\ {\isasymtimes}\isactrlsub f\ p{\isacharparenright}{\kern0pt}{\isacharparenright}{\kern0pt}\ {\isasymcirc}\isactrlsub c\ {\isasymlangle}x{\isacharcomma}{\kern0pt}\ x{\isacharunderscore}{\kern0pt}f{\isasymrangle}{\isachardoublequoteclose}\isanewline
\ \ \ \ \ \ \isacommand{using}\isamarkupfalse%
\ assms\ {\isasymphi}{\isacharunderscore}{\kern0pt}def\ inv{\isacharunderscore}{\kern0pt}transpose{\isacharunderscore}{\kern0pt}func{\isacharunderscore}{\kern0pt}def{\isadigit{3}}\ \isacommand{by}\isamarkupfalse%
\ auto\isanewline
\ \ \ \ \isacommand{also}\isamarkupfalse%
\ \isacommand{have}\isamarkupfalse%
\ {\isachardoublequoteopen}{\isachardot}{\kern0pt}{\isachardot}{\kern0pt}{\isachardot}{\kern0pt}\ {\isacharequal}{\kern0pt}\ {\isacharparenleft}{\kern0pt}eval{\isacharunderscore}{\kern0pt}func\ A\ X{\isacharparenright}{\kern0pt}\ {\isasymcirc}\isactrlsub c\ {\isacharparenleft}{\kern0pt}id\ X\ {\isasymtimes}\isactrlsub f\ p{\isacharparenright}{\kern0pt}\ {\isasymcirc}\isactrlsub c\ {\isasymlangle}x{\isacharcomma}{\kern0pt}\ x{\isacharunderscore}{\kern0pt}f{\isasymrangle}{\isachardoublequoteclose}\isanewline
\ \ \ \ \ \ \isacommand{by}\isamarkupfalse%
\ {\isacharparenleft}{\kern0pt}typecheck{\isacharunderscore}{\kern0pt}cfuncs{\isacharcomma}{\kern0pt}\ metis\ comp{\isacharunderscore}{\kern0pt}associative{\isadigit{2}}{\isacharparenright}{\kern0pt}\isanewline
\ \ \ \ \isacommand{also}\isamarkupfalse%
\ \isacommand{have}\isamarkupfalse%
\ {\isachardoublequoteopen}{\isachardot}{\kern0pt}{\isachardot}{\kern0pt}{\isachardot}{\kern0pt}\ {\isacharequal}{\kern0pt}\ {\isacharparenleft}{\kern0pt}eval{\isacharunderscore}{\kern0pt}func\ A\ X{\isacharparenright}{\kern0pt}\ {\isasymcirc}\isactrlsub c\ {\isasymlangle}id\ X\ \ {\isasymcirc}\isactrlsub c\ \ x{\isacharcomma}{\kern0pt}\ p\ {\isasymcirc}\isactrlsub c\ x{\isacharunderscore}{\kern0pt}f{\isasymrangle}{\isachardoublequoteclose}\isanewline
\ \ \ \ \ \ \isacommand{using}\isamarkupfalse%
\ cfunc{\isacharunderscore}{\kern0pt}cross{\isacharunderscore}{\kern0pt}prod{\isacharunderscore}{\kern0pt}comp{\isacharunderscore}{\kern0pt}cfunc{\isacharunderscore}{\kern0pt}prod\ x{\isacharunderscore}{\kern0pt}f\ \isacommand{by}\isamarkupfalse%
\ {\isacharparenleft}{\kern0pt}typecheck{\isacharunderscore}{\kern0pt}cfuncs{\isacharcomma}{\kern0pt}\ force{\isacharparenright}{\kern0pt}\isanewline
\ \ \ \ \isacommand{also}\isamarkupfalse%
\ \isacommand{have}\isamarkupfalse%
\ {\isachardoublequoteopen}{\isachardot}{\kern0pt}{\isachardot}{\kern0pt}{\isachardot}{\kern0pt}\ {\isacharequal}{\kern0pt}\ {\isacharparenleft}{\kern0pt}eval{\isacharunderscore}{\kern0pt}func\ A\ X{\isacharparenright}{\kern0pt}\ {\isasymcirc}\isactrlsub c\ {\isasymlangle}x{\isacharcomma}{\kern0pt}\ metafunc\ f{\isasymrangle}{\isachardoublequoteclose}\isanewline
\ \ \ \ \ \ \isacommand{using}\isamarkupfalse%
\ id{\isacharunderscore}{\kern0pt}left{\isacharunderscore}{\kern0pt}unit{\isadigit{2}}\ x{\isacharunderscore}{\kern0pt}f\ \isacommand{by}\isamarkupfalse%
\ {\isacharparenleft}{\kern0pt}typecheck{\isacharunderscore}{\kern0pt}cfuncs{\isacharcomma}{\kern0pt}\ auto{\isacharparenright}{\kern0pt}\isanewline
\ \ \ \ \isacommand{also}\isamarkupfalse%
\ \isacommand{have}\isamarkupfalse%
\ {\isachardoublequoteopen}{\isachardot}{\kern0pt}{\isachardot}{\kern0pt}{\isachardot}{\kern0pt}\ {\isacharequal}{\kern0pt}\ f\ {\isasymcirc}\isactrlsub c\ x{\isachardoublequoteclose}\isanewline
\ \ \ \ \ \ \isacommand{by}\isamarkupfalse%
\ {\isacharparenleft}{\kern0pt}simp\ add{\isacharcolon}{\kern0pt}\ eval{\isacharunderscore}{\kern0pt}lemma\ f{\isacharunderscore}{\kern0pt}type\ x{\isacharunderscore}{\kern0pt}type{\isacharparenright}{\kern0pt}\isanewline
\ \ \ \ \isacommand{finally}\isamarkupfalse%
\ \isacommand{show}\isamarkupfalse%
\ {\isachardoublequoteopen}{\isasymphi}\isactrlbsub {\isacharbrackleft}{\kern0pt}{\isacharminus}{\kern0pt}{\isacharcomma}{\kern0pt}x{\isacharunderscore}{\kern0pt}f{\isacharbrackright}{\kern0pt}\isactrlesub \ {\isasymcirc}\isactrlsub c\ x\ {\isacharequal}{\kern0pt}\ f\ {\isasymcirc}\isactrlsub c\ x{\isachardoublequoteclose}\isacommand{{\isachardot}{\kern0pt}}\isamarkupfalse%
\isanewline
\ \ \isacommand{qed}\isamarkupfalse%
\isanewline
\ \ \isacommand{then}\isamarkupfalse%
\ \isacommand{have}\isamarkupfalse%
\ {\isachardoublequoteopen}{\isasymphi}\isactrlbsub {\isacharbrackleft}{\kern0pt}{\isacharminus}{\kern0pt}{\isacharcomma}{\kern0pt}x{\isacharunderscore}{\kern0pt}f{\isacharbrackright}{\kern0pt}\isactrlesub \ {\isasymcirc}\isactrlsub c\ x{\isacharunderscore}{\kern0pt}f\ {\isacharequal}{\kern0pt}\ g\ {\isasymcirc}\isactrlsub c\ {\isasymphi}\ {\isasymcirc}\isactrlsub c\ diagonal{\isacharparenleft}{\kern0pt}X{\isacharparenright}{\kern0pt}\ {\isasymcirc}\isactrlsub c\ x{\isacharunderscore}{\kern0pt}f{\isachardoublequoteclose}\isanewline
\ \ \ \ \isacommand{by}\isamarkupfalse%
\ {\isacharparenleft}{\kern0pt}typecheck{\isacharunderscore}{\kern0pt}cfuncs{\isacharcomma}{\kern0pt}\ smt\ {\isacharparenleft}{\kern0pt}z{\isadigit{3}}{\isacharparenright}{\kern0pt}\ cfunc{\isacharunderscore}{\kern0pt}type{\isacharunderscore}{\kern0pt}def\ comp{\isacharunderscore}{\kern0pt}associative\ domain{\isacharunderscore}{\kern0pt}comp\ f{\isacharunderscore}{\kern0pt}def\ x{\isacharunderscore}{\kern0pt}f{\isacharparenright}{\kern0pt}\isanewline
\ \ \isacommand{then}\isamarkupfalse%
\ \isacommand{have}\isamarkupfalse%
\ {\isachardoublequoteopen}{\isasymphi}\ {\isasymcirc}\isactrlsub c\ {\isasymlangle}x{\isacharunderscore}{\kern0pt}f{\isacharcomma}{\kern0pt}\ x{\isacharunderscore}{\kern0pt}f{\isasymrangle}\ {\isacharequal}{\kern0pt}\ g\ {\isasymcirc}\isactrlsub c\ {\isasymphi}\ {\isasymcirc}\isactrlsub c\ {\isasymlangle}x{\isacharunderscore}{\kern0pt}f{\isacharcomma}{\kern0pt}\ x{\isacharunderscore}{\kern0pt}f{\isasymrangle}{\isachardoublequoteclose}\isanewline
\ \ \ \ \isacommand{using}\isamarkupfalse%
\ \ diag{\isacharunderscore}{\kern0pt}on{\isacharunderscore}{\kern0pt}elements\ right{\isacharunderscore}{\kern0pt}param{\isacharunderscore}{\kern0pt}on{\isacharunderscore}{\kern0pt}el\ x{\isacharunderscore}{\kern0pt}f\ \isacommand{by}\isamarkupfalse%
\ {\isacharparenleft}{\kern0pt}typecheck{\isacharunderscore}{\kern0pt}cfuncs{\isacharcomma}{\kern0pt}\ auto{\isacharparenright}{\kern0pt}\isanewline
\ \ \isacommand{then}\isamarkupfalse%
\ \isacommand{have}\isamarkupfalse%
\ {\isachardoublequoteopen}fixed{\isacharunderscore}{\kern0pt}point\ {\isacharparenleft}{\kern0pt}{\isasymphi}\ {\isasymcirc}\isactrlsub c\ {\isasymlangle}x{\isacharunderscore}{\kern0pt}f{\isacharcomma}{\kern0pt}\ x{\isacharunderscore}{\kern0pt}f{\isasymrangle}{\isacharparenright}{\kern0pt}\ g{\isachardoublequoteclose}\isanewline
\ \ \ \ \isacommand{using}\isamarkupfalse%
\ fixed{\isacharunderscore}{\kern0pt}point{\isacharunderscore}{\kern0pt}def{\isadigit{2}}\ \isacommand{by}\isamarkupfalse%
\ {\isacharparenleft}{\kern0pt}typecheck{\isacharunderscore}{\kern0pt}cfuncs{\isacharcomma}{\kern0pt}\ auto{\isacharparenright}{\kern0pt}\isanewline
\ \ \isacommand{then}\isamarkupfalse%
\ \isacommand{show}\isamarkupfalse%
\ {\isachardoublequoteopen}{\isasymexists}a{\isachardot}{\kern0pt}\ fixed{\isacharunderscore}{\kern0pt}point\ a\ g{\isachardoublequoteclose}\isanewline
\ \ \ \ \isacommand{using}\isamarkupfalse%
\ fixed{\isacharunderscore}{\kern0pt}point{\isacharunderscore}{\kern0pt}def\ \isacommand{by}\isamarkupfalse%
\ auto\isanewline
\isacommand{qed}\isamarkupfalse%
%
\endisatagproof
{\isafoldproof}%
%
\isadelimproof
%
\endisadelimproof
%
\begin{isamarkuptext}%
The theorem below corresponds to Theorem 2.6.14 in Halvorson.%
\end{isamarkuptext}\isamarkuptrue%
\isacommand{theorem}\isamarkupfalse%
\ Cantors{\isacharunderscore}{\kern0pt}Negative{\isacharunderscore}{\kern0pt}Theorem{\isacharcolon}{\kern0pt}\isanewline
\ \ {\isachardoublequoteopen}{\isasymnexists}\ s{\isachardot}{\kern0pt}\ s\ {\isacharcolon}{\kern0pt}\ X\ {\isasymrightarrow}\ {\isasymP}\ X\ {\isasymand}\ surjective\ s{\isachardoublequoteclose}\isanewline
%
\isadelimproof
%
\endisadelimproof
%
\isatagproof
\isacommand{proof}\isamarkupfalse%
{\isacharparenleft}{\kern0pt}rule\ ccontr{\isacharcomma}{\kern0pt}\ clarify{\isacharparenright}{\kern0pt}\isanewline
\ \ \isacommand{fix}\isamarkupfalse%
\ s\ \isanewline
\ \ \isacommand{assume}\isamarkupfalse%
\ s{\isacharunderscore}{\kern0pt}type{\isacharcolon}{\kern0pt}\ {\isachardoublequoteopen}s\ {\isacharcolon}{\kern0pt}\ X\ {\isasymrightarrow}\ {\isasymP}\ X{\isachardoublequoteclose}\isanewline
\ \ \isacommand{assume}\isamarkupfalse%
\ s{\isacharunderscore}{\kern0pt}surj{\isacharcolon}{\kern0pt}\ {\isachardoublequoteopen}surjective\ s{\isachardoublequoteclose}\isanewline
\ \ \isacommand{then}\isamarkupfalse%
\ \isacommand{have}\isamarkupfalse%
\ Omega{\isacharunderscore}{\kern0pt}has{\isacharunderscore}{\kern0pt}ffp{\isacharcolon}{\kern0pt}\ {\isachardoublequoteopen}fixed{\isacharunderscore}{\kern0pt}point{\isacharunderscore}{\kern0pt}property\ {\isasymOmega}{\isachardoublequoteclose}\isanewline
\ \ \ \ \isacommand{using}\isamarkupfalse%
\ Lawveres{\isacharunderscore}{\kern0pt}fixed{\isacharunderscore}{\kern0pt}point{\isacharunderscore}{\kern0pt}theorem\ powerset{\isacharunderscore}{\kern0pt}def\ s{\isacharunderscore}{\kern0pt}type\ \isacommand{by}\isamarkupfalse%
\ auto\isanewline
\ \ \isacommand{have}\isamarkupfalse%
\ Omega{\isacharunderscore}{\kern0pt}doesnt{\isacharunderscore}{\kern0pt}have{\isacharunderscore}{\kern0pt}ffp{\isacharcolon}{\kern0pt}\ {\isachardoublequoteopen}{\isasymnot}{\isacharparenleft}{\kern0pt}fixed{\isacharunderscore}{\kern0pt}point{\isacharunderscore}{\kern0pt}property\ {\isasymOmega}{\isacharparenright}{\kern0pt}{\isachardoublequoteclose}\isanewline
\ \ \ \ \isacommand{unfolding}\isamarkupfalse%
\ fixed{\isacharunderscore}{\kern0pt}point{\isacharunderscore}{\kern0pt}property{\isacharunderscore}{\kern0pt}def\ has{\isacharunderscore}{\kern0pt}fixed{\isacharunderscore}{\kern0pt}point{\isacharunderscore}{\kern0pt}def\ fixed{\isacharunderscore}{\kern0pt}point{\isacharunderscore}{\kern0pt}def\isanewline
\ \ \isacommand{proof}\isamarkupfalse%
\ \ \ \ \isanewline
\ \ \ \ \isacommand{assume}\isamarkupfalse%
\ BWOC{\isacharcolon}{\kern0pt}\ {\isachardoublequoteopen}{\isasymforall}g{\isachardot}{\kern0pt}\ g\ {\isacharcolon}{\kern0pt}\ {\isasymOmega}\ {\isasymrightarrow}\ {\isasymOmega}\ {\isasymlongrightarrow}\ {\isacharparenleft}{\kern0pt}{\isasymexists}a\ A{\isachardot}{\kern0pt}\ g\ {\isacharcolon}{\kern0pt}\ A\ {\isasymrightarrow}\ A\ {\isasymand}\ a\ {\isasymin}\isactrlsub c\ A\ {\isasymand}\ g\ {\isasymcirc}\isactrlsub c\ a\ {\isacharequal}{\kern0pt}\ a{\isacharparenright}{\kern0pt}{\isachardoublequoteclose}\isanewline
\ \ \ \ \isacommand{have}\isamarkupfalse%
\ \ {\isachardoublequoteopen}NOT\ {\isacharcolon}{\kern0pt}\ {\isasymOmega}\ {\isasymrightarrow}\ {\isasymOmega}\ {\isasymand}\ {\isacharparenleft}{\kern0pt}{\isasymforall}a{\isachardot}{\kern0pt}\ {\isasymforall}A{\isachardot}{\kern0pt}\ a\ {\isasymin}\isactrlsub c\ A\ {\isasymlongrightarrow}\ NOT\ {\isacharcolon}{\kern0pt}\ A\ {\isasymrightarrow}\ A\ {\isasymlongrightarrow}\ NOT\ {\isasymcirc}\isactrlsub c\ a\ {\isasymnoteq}\ a\ {\isasymor}\ {\isasymnot}\ a\ {\isasymin}\isactrlsub c\ {\isasymOmega}{\isacharparenright}{\kern0pt}{\isachardoublequoteclose}\isanewline
\ \ \ \ \ \ \isacommand{by}\isamarkupfalse%
\ {\isacharparenleft}{\kern0pt}typecheck{\isacharunderscore}{\kern0pt}cfuncs{\isacharcomma}{\kern0pt}\ metis\ AND{\isacharunderscore}{\kern0pt}complementary\ AND{\isacharunderscore}{\kern0pt}idempotent\ OR{\isacharunderscore}{\kern0pt}complementary\ OR{\isacharunderscore}{\kern0pt}idempotent\ true{\isacharunderscore}{\kern0pt}false{\isacharunderscore}{\kern0pt}distinct{\isacharparenright}{\kern0pt}\isanewline
\ \ \ \ \isacommand{then}\isamarkupfalse%
\ \isacommand{have}\isamarkupfalse%
\ {\isachardoublequoteopen}{\isasymexists}g{\isachardot}{\kern0pt}\ g\ {\isacharcolon}{\kern0pt}\ {\isasymOmega}\ {\isasymrightarrow}\ {\isasymOmega}\ {\isasymand}\ {\isacharparenleft}{\kern0pt}{\isasymforall}a{\isachardot}{\kern0pt}\ {\isasymforall}A{\isachardot}{\kern0pt}\ a\ {\isasymin}\isactrlsub c\ A\ {\isasymlongrightarrow}\ g\ {\isacharcolon}{\kern0pt}\ A\ {\isasymrightarrow}\ A\ {\isasymlongrightarrow}\ g\ {\isasymcirc}\isactrlsub c\ a\ {\isasymnoteq}\ a{\isacharparenright}{\kern0pt}{\isachardoublequoteclose}\isanewline
\ \ \ \ \ \ \isacommand{by}\isamarkupfalse%
\ {\isacharparenleft}{\kern0pt}metis\ cfunc{\isacharunderscore}{\kern0pt}type{\isacharunderscore}{\kern0pt}def{\isacharparenright}{\kern0pt}\isanewline
\ \ \ \ \isacommand{then}\isamarkupfalse%
\ \isacommand{show}\isamarkupfalse%
\ False\isanewline
\ \ \ \ \ \ \isacommand{using}\isamarkupfalse%
\ BWOC\ \isacommand{by}\isamarkupfalse%
\ presburger\isanewline
\ \ \isacommand{qed}\isamarkupfalse%
\isanewline
\ \ \isacommand{show}\isamarkupfalse%
\ False\isanewline
\ \ \ \ \isacommand{using}\isamarkupfalse%
\ Omega{\isacharunderscore}{\kern0pt}doesnt{\isacharunderscore}{\kern0pt}have{\isacharunderscore}{\kern0pt}ffp\ Omega{\isacharunderscore}{\kern0pt}has{\isacharunderscore}{\kern0pt}ffp\ \isacommand{by}\isamarkupfalse%
\ auto\isanewline
\isacommand{qed}\isamarkupfalse%
%
\endisatagproof
{\isafoldproof}%
%
\isadelimproof
%
\endisadelimproof
%
\begin{isamarkuptext}%
The theorem below corresponds to Exercise 2.6.15 in Halvorson.%
\end{isamarkuptext}\isamarkuptrue%
\isacommand{theorem}\isamarkupfalse%
\ Cantors{\isacharunderscore}{\kern0pt}Positive{\isacharunderscore}{\kern0pt}Theorem{\isacharcolon}{\kern0pt}\isanewline
\ \ {\isachardoublequoteopen}{\isasymexists}m{\isachardot}{\kern0pt}\ m\ {\isacharcolon}{\kern0pt}\ X\ {\isasymrightarrow}\ {\isasymOmega}\isactrlbsup X\isactrlesup \ {\isasymand}\ injective\ m{\isachardoublequoteclose}\isanewline
%
\isadelimproof
%
\endisadelimproof
%
\isatagproof
\isacommand{proof}\isamarkupfalse%
\ {\isacharminus}{\kern0pt}\ \isanewline
\ \ \isacommand{have}\isamarkupfalse%
\ eq{\isacharunderscore}{\kern0pt}pred{\isacharunderscore}{\kern0pt}sharp{\isacharunderscore}{\kern0pt}type{\isacharbrackleft}{\kern0pt}type{\isacharunderscore}{\kern0pt}rule{\isacharbrackright}{\kern0pt}{\isacharcolon}{\kern0pt}\ {\isachardoublequoteopen}eq{\isacharunderscore}{\kern0pt}pred\ X\isactrlsup {\isasymsharp}\ {\isacharcolon}{\kern0pt}\ X\ {\isasymrightarrow}\ \ {\isasymOmega}\isactrlbsup X\isactrlesup {\isachardoublequoteclose}\isanewline
\ \ \ \ \isacommand{by}\isamarkupfalse%
\ typecheck{\isacharunderscore}{\kern0pt}cfuncs\isanewline
\ \ \isacommand{have}\isamarkupfalse%
\ {\isachardoublequoteopen}injective{\isacharparenleft}{\kern0pt}eq{\isacharunderscore}{\kern0pt}pred\ X\isactrlsup {\isasymsharp}{\isacharparenright}{\kern0pt}{\isachardoublequoteclose}\isanewline
\ \ \ \ \isacommand{unfolding}\isamarkupfalse%
\ injective{\isacharunderscore}{\kern0pt}def\isanewline
\ \ \isacommand{proof}\isamarkupfalse%
\ {\isacharparenleft}{\kern0pt}clarify{\isacharparenright}{\kern0pt}\isanewline
\ \ \ \ \isacommand{fix}\isamarkupfalse%
\ x\ y\ \isanewline
\ \ \ \ \isacommand{assume}\isamarkupfalse%
\ {\isachardoublequoteopen}x\ {\isasymin}\isactrlsub c\ domain\ {\isacharparenleft}{\kern0pt}eq{\isacharunderscore}{\kern0pt}pred\ X\isactrlsup {\isasymsharp}{\isacharparenright}{\kern0pt}{\isachardoublequoteclose}\ \isacommand{then}\isamarkupfalse%
\ \isacommand{have}\isamarkupfalse%
\ x{\isacharunderscore}{\kern0pt}type{\isacharbrackleft}{\kern0pt}type{\isacharunderscore}{\kern0pt}rule{\isacharbrackright}{\kern0pt}{\isacharcolon}{\kern0pt}\ {\isachardoublequoteopen}x\ {\isasymin}\isactrlsub c\ X{\isachardoublequoteclose}\isanewline
\ \ \ \ \ \ \isacommand{using}\isamarkupfalse%
\ cfunc{\isacharunderscore}{\kern0pt}type{\isacharunderscore}{\kern0pt}def\ eq{\isacharunderscore}{\kern0pt}pred{\isacharunderscore}{\kern0pt}sharp{\isacharunderscore}{\kern0pt}type\ \isacommand{by}\isamarkupfalse%
\ auto\isanewline
\ \ \ \ \isacommand{assume}\isamarkupfalse%
\ {\isachardoublequoteopen}y\ {\isasymin}\isactrlsub c\ domain\ {\isacharparenleft}{\kern0pt}eq{\isacharunderscore}{\kern0pt}pred\ X\isactrlsup {\isasymsharp}{\isacharparenright}{\kern0pt}{\isachardoublequoteclose}\ \isacommand{then}\isamarkupfalse%
\ \isacommand{have}\isamarkupfalse%
\ y{\isacharunderscore}{\kern0pt}type{\isacharbrackleft}{\kern0pt}type{\isacharunderscore}{\kern0pt}rule{\isacharbrackright}{\kern0pt}{\isacharcolon}{\kern0pt}{\isachardoublequoteopen}y\ {\isasymin}\isactrlsub c\ X{\isachardoublequoteclose}\isanewline
\ \ \ \ \ \ \isacommand{using}\isamarkupfalse%
\ cfunc{\isacharunderscore}{\kern0pt}type{\isacharunderscore}{\kern0pt}def\ eq{\isacharunderscore}{\kern0pt}pred{\isacharunderscore}{\kern0pt}sharp{\isacharunderscore}{\kern0pt}type\ \isacommand{by}\isamarkupfalse%
\ auto\isanewline
\ \ \ \ \isacommand{assume}\isamarkupfalse%
\ eq{\isacharcolon}{\kern0pt}\ {\isachardoublequoteopen}eq{\isacharunderscore}{\kern0pt}pred\ X\isactrlsup {\isasymsharp}\ {\isasymcirc}\isactrlsub c\ x\ {\isacharequal}{\kern0pt}\ eq{\isacharunderscore}{\kern0pt}pred\ X\isactrlsup {\isasymsharp}\ {\isasymcirc}\isactrlsub c\ y{\isachardoublequoteclose}\isanewline
\ \ \ \ \isacommand{have}\isamarkupfalse%
\ {\isachardoublequoteopen}eq{\isacharunderscore}{\kern0pt}pred\ X\ {\isasymcirc}\isactrlsub c\ {\isasymlangle}x{\isacharcomma}{\kern0pt}\ x{\isasymrangle}\ {\isacharequal}{\kern0pt}\ eq{\isacharunderscore}{\kern0pt}pred\ X\ {\isasymcirc}\isactrlsub c\ {\isasymlangle}x{\isacharcomma}{\kern0pt}\ y{\isasymrangle}{\isachardoublequoteclose}\isanewline
\ \ \ \ \isacommand{proof}\isamarkupfalse%
\ {\isacharminus}{\kern0pt}\ \isanewline
\ \ \ \ \ \ \isacommand{have}\isamarkupfalse%
\ {\isachardoublequoteopen}eq{\isacharunderscore}{\kern0pt}pred\ X\ {\isasymcirc}\isactrlsub c\ {\isasymlangle}x{\isacharcomma}{\kern0pt}\ x{\isasymrangle}\ {\isacharequal}{\kern0pt}\ {\isacharparenleft}{\kern0pt}{\isacharparenleft}{\kern0pt}eval{\isacharunderscore}{\kern0pt}func\ {\isasymOmega}\ X{\isacharparenright}{\kern0pt}\ {\isasymcirc}\isactrlsub c\ {\isacharparenleft}{\kern0pt}id\ X\ {\isasymtimes}\isactrlsub f\ {\isacharparenleft}{\kern0pt}eq{\isacharunderscore}{\kern0pt}pred\ X\isactrlsup {\isasymsharp}{\isacharparenright}{\kern0pt}{\isacharparenright}{\kern0pt}\ {\isacharparenright}{\kern0pt}\ {\isasymcirc}\isactrlsub c\ {\isasymlangle}x{\isacharcomma}{\kern0pt}\ x{\isasymrangle}{\isachardoublequoteclose}\isanewline
\ \ \ \ \ \ \ \ \isacommand{using}\isamarkupfalse%
\ transpose{\isacharunderscore}{\kern0pt}func{\isacharunderscore}{\kern0pt}def\ \isacommand{by}\isamarkupfalse%
\ {\isacharparenleft}{\kern0pt}typecheck{\isacharunderscore}{\kern0pt}cfuncs{\isacharcomma}{\kern0pt}\ presburger{\isacharparenright}{\kern0pt}\isanewline
\ \ \ \ \ \ \isacommand{also}\isamarkupfalse%
\ \isacommand{have}\isamarkupfalse%
\ {\isachardoublequoteopen}{\isachardot}{\kern0pt}{\isachardot}{\kern0pt}{\isachardot}{\kern0pt}\ {\isacharequal}{\kern0pt}\ {\isacharparenleft}{\kern0pt}eval{\isacharunderscore}{\kern0pt}func\ {\isasymOmega}\ X{\isacharparenright}{\kern0pt}\ {\isasymcirc}\isactrlsub c\ {\isacharparenleft}{\kern0pt}id\ X\ {\isasymtimes}\isactrlsub f\ {\isacharparenleft}{\kern0pt}eq{\isacharunderscore}{\kern0pt}pred\ X\isactrlsup {\isasymsharp}{\isacharparenright}{\kern0pt}{\isacharparenright}{\kern0pt}\ {\isasymcirc}\isactrlsub c\ {\isasymlangle}x{\isacharcomma}{\kern0pt}\ x{\isasymrangle}{\isachardoublequoteclose}\isanewline
\ \ \ \ \ \ \ \ \isacommand{by}\isamarkupfalse%
\ {\isacharparenleft}{\kern0pt}typecheck{\isacharunderscore}{\kern0pt}cfuncs{\isacharcomma}{\kern0pt}\ simp\ add{\isacharcolon}{\kern0pt}\ comp{\isacharunderscore}{\kern0pt}associative{\isadigit{2}}{\isacharparenright}{\kern0pt}\isanewline
\ \ \ \ \ \ \isacommand{also}\isamarkupfalse%
\ \isacommand{have}\isamarkupfalse%
\ {\isachardoublequoteopen}{\isachardot}{\kern0pt}{\isachardot}{\kern0pt}{\isachardot}{\kern0pt}\ {\isacharequal}{\kern0pt}\ {\isacharparenleft}{\kern0pt}eval{\isacharunderscore}{\kern0pt}func\ {\isasymOmega}\ X{\isacharparenright}{\kern0pt}\ {\isasymcirc}\isactrlsub c\ {\isasymlangle}id\ X\ {\isasymcirc}\isactrlsub c\ x{\isacharcomma}{\kern0pt}\ {\isacharparenleft}{\kern0pt}eq{\isacharunderscore}{\kern0pt}pred\ X\isactrlsup {\isasymsharp}{\isacharparenright}{\kern0pt}\ {\isasymcirc}\isactrlsub c\ x{\isasymrangle}{\isachardoublequoteclose}\isanewline
\ \ \ \ \ \ \ \ \isacommand{using}\isamarkupfalse%
\ cfunc{\isacharunderscore}{\kern0pt}cross{\isacharunderscore}{\kern0pt}prod{\isacharunderscore}{\kern0pt}comp{\isacharunderscore}{\kern0pt}cfunc{\isacharunderscore}{\kern0pt}prod\ \isacommand{by}\isamarkupfalse%
\ {\isacharparenleft}{\kern0pt}typecheck{\isacharunderscore}{\kern0pt}cfuncs{\isacharcomma}{\kern0pt}\ force{\isacharparenright}{\kern0pt}\isanewline
\ \ \ \ \ \ \isacommand{also}\isamarkupfalse%
\ \isacommand{have}\isamarkupfalse%
\ {\isachardoublequoteopen}{\isachardot}{\kern0pt}{\isachardot}{\kern0pt}{\isachardot}{\kern0pt}\ {\isacharequal}{\kern0pt}\ {\isacharparenleft}{\kern0pt}eval{\isacharunderscore}{\kern0pt}func\ {\isasymOmega}\ X{\isacharparenright}{\kern0pt}\ {\isasymcirc}\isactrlsub c\ {\isasymlangle}id\ X\ {\isasymcirc}\isactrlsub c\ x{\isacharcomma}{\kern0pt}\ {\isacharparenleft}{\kern0pt}eq{\isacharunderscore}{\kern0pt}pred\ X\isactrlsup {\isasymsharp}{\isacharparenright}{\kern0pt}\ {\isasymcirc}\isactrlsub c\ y{\isasymrangle}{\isachardoublequoteclose}\isanewline
\ \ \ \ \ \ \ \ \isacommand{by}\isamarkupfalse%
\ {\isacharparenleft}{\kern0pt}simp\ add{\isacharcolon}{\kern0pt}\ eq{\isacharparenright}{\kern0pt}\isanewline
\ \ \ \ \ \ \isacommand{also}\isamarkupfalse%
\ \isacommand{have}\isamarkupfalse%
\ {\isachardoublequoteopen}{\isachardot}{\kern0pt}{\isachardot}{\kern0pt}{\isachardot}{\kern0pt}\ {\isacharequal}{\kern0pt}\ {\isacharparenleft}{\kern0pt}eval{\isacharunderscore}{\kern0pt}func\ {\isasymOmega}\ X{\isacharparenright}{\kern0pt}\ {\isasymcirc}\isactrlsub c\ {\isacharparenleft}{\kern0pt}id\ X\ {\isasymtimes}\isactrlsub f\ {\isacharparenleft}{\kern0pt}eq{\isacharunderscore}{\kern0pt}pred\ X\isactrlsup {\isasymsharp}{\isacharparenright}{\kern0pt}{\isacharparenright}{\kern0pt}\ {\isasymcirc}\isactrlsub c\ {\isasymlangle}x{\isacharcomma}{\kern0pt}\ y{\isasymrangle}{\isachardoublequoteclose}\isanewline
\ \ \ \ \ \ \ \ \isacommand{by}\isamarkupfalse%
\ {\isacharparenleft}{\kern0pt}typecheck{\isacharunderscore}{\kern0pt}cfuncs{\isacharcomma}{\kern0pt}\ simp\ add{\isacharcolon}{\kern0pt}\ cfunc{\isacharunderscore}{\kern0pt}cross{\isacharunderscore}{\kern0pt}prod{\isacharunderscore}{\kern0pt}comp{\isacharunderscore}{\kern0pt}cfunc{\isacharunderscore}{\kern0pt}prod{\isacharparenright}{\kern0pt}\isanewline
\ \ \ \ \ \ \isacommand{also}\isamarkupfalse%
\ \isacommand{have}\isamarkupfalse%
\ {\isachardoublequoteopen}{\isachardot}{\kern0pt}{\isachardot}{\kern0pt}{\isachardot}{\kern0pt}\ {\isacharequal}{\kern0pt}\ {\isacharparenleft}{\kern0pt}{\isacharparenleft}{\kern0pt}eval{\isacharunderscore}{\kern0pt}func\ {\isasymOmega}\ X{\isacharparenright}{\kern0pt}\ {\isasymcirc}\isactrlsub c\ {\isacharparenleft}{\kern0pt}id\ X\ {\isasymtimes}\isactrlsub f\ {\isacharparenleft}{\kern0pt}eq{\isacharunderscore}{\kern0pt}pred\ X\isactrlsup {\isasymsharp}{\isacharparenright}{\kern0pt}{\isacharparenright}{\kern0pt}\ {\isacharparenright}{\kern0pt}\ {\isasymcirc}\isactrlsub c\ {\isasymlangle}x{\isacharcomma}{\kern0pt}\ y{\isasymrangle}{\isachardoublequoteclose}\isanewline
\ \ \ \ \ \ \ \ \isacommand{using}\isamarkupfalse%
\ comp{\isacharunderscore}{\kern0pt}associative{\isadigit{2}}\ \isacommand{by}\isamarkupfalse%
\ {\isacharparenleft}{\kern0pt}typecheck{\isacharunderscore}{\kern0pt}cfuncs{\isacharcomma}{\kern0pt}\ blast{\isacharparenright}{\kern0pt}\isanewline
\ \ \ \ \ \ \isacommand{also}\isamarkupfalse%
\ \isacommand{have}\isamarkupfalse%
\ {\isachardoublequoteopen}{\isachardot}{\kern0pt}{\isachardot}{\kern0pt}{\isachardot}{\kern0pt}\ {\isacharequal}{\kern0pt}\ eq{\isacharunderscore}{\kern0pt}pred\ X\ {\isasymcirc}\isactrlsub c\ {\isasymlangle}x{\isacharcomma}{\kern0pt}\ y{\isasymrangle}{\isachardoublequoteclose}\isanewline
\ \ \ \ \ \ \ \ \isacommand{using}\isamarkupfalse%
\ transpose{\isacharunderscore}{\kern0pt}func{\isacharunderscore}{\kern0pt}def\ \isacommand{by}\isamarkupfalse%
\ {\isacharparenleft}{\kern0pt}typecheck{\isacharunderscore}{\kern0pt}cfuncs{\isacharcomma}{\kern0pt}\ presburger{\isacharparenright}{\kern0pt}\isanewline
\ \ \ \ \ \ \isacommand{finally}\isamarkupfalse%
\ \isacommand{show}\isamarkupfalse%
\ {\isacharquery}{\kern0pt}thesis\isacommand{{\isachardot}{\kern0pt}}\isamarkupfalse%
\isanewline
\ \ \ \ \isacommand{qed}\isamarkupfalse%
\isanewline
\ \ \ \ \isacommand{then}\isamarkupfalse%
\ \isacommand{show}\isamarkupfalse%
\ {\isachardoublequoteopen}x\ {\isacharequal}{\kern0pt}\ y{\isachardoublequoteclose}\isanewline
\ \ \ \ \ \ \isacommand{by}\isamarkupfalse%
\ {\isacharparenleft}{\kern0pt}metis\ eq{\isacharunderscore}{\kern0pt}pred{\isacharunderscore}{\kern0pt}iff{\isacharunderscore}{\kern0pt}eq\ x{\isacharunderscore}{\kern0pt}type\ y{\isacharunderscore}{\kern0pt}type{\isacharparenright}{\kern0pt}\isanewline
\ \ \isacommand{qed}\isamarkupfalse%
\isanewline
\ \ \isacommand{then}\isamarkupfalse%
\ \isacommand{show}\isamarkupfalse%
\ {\isachardoublequoteopen}{\isasymexists}m{\isachardot}{\kern0pt}\ m\ {\isacharcolon}{\kern0pt}\ X\ {\isasymrightarrow}\ {\isasymOmega}\isactrlbsup X\isactrlesup \ {\isasymand}\ injective\ m{\isachardoublequoteclose}\isanewline
\ \ \ \ \isacommand{using}\isamarkupfalse%
\ eq{\isacharunderscore}{\kern0pt}pred{\isacharunderscore}{\kern0pt}sharp{\isacharunderscore}{\kern0pt}type\ injective{\isacharunderscore}{\kern0pt}imp{\isacharunderscore}{\kern0pt}monomorphism\ \isacommand{by}\isamarkupfalse%
\ blast\isanewline
\isacommand{qed}\isamarkupfalse%
%
\endisatagproof
{\isafoldproof}%
%
\isadelimproof
%
\endisadelimproof
%
\begin{isamarkuptext}%
The corollary below corresponds to Corollary 2.6.16 in Halvorson.%
\end{isamarkuptext}\isamarkuptrue%
\isacommand{corollary}\isamarkupfalse%
\ \isanewline
\ \ {\isachardoublequoteopen}X\ {\isasymle}\isactrlsub c\ {\isasymP}\ X\ {\isasymand}\ {\isasymnot}\ {\isacharparenleft}{\kern0pt}X\ {\isasymcong}\ {\isasymP}\ X{\isacharparenright}{\kern0pt}{\isachardoublequoteclose}\isanewline
%
\isadelimproof
\ \ %
\endisadelimproof
%
\isatagproof
\isacommand{using}\isamarkupfalse%
\ Cantors{\isacharunderscore}{\kern0pt}Negative{\isacharunderscore}{\kern0pt}Theorem\ Cantors{\isacharunderscore}{\kern0pt}Positive{\isacharunderscore}{\kern0pt}Theorem\isanewline
\ \ \isacommand{unfolding}\isamarkupfalse%
\ is{\isacharunderscore}{\kern0pt}smaller{\isacharunderscore}{\kern0pt}than{\isacharunderscore}{\kern0pt}def\ is{\isacharunderscore}{\kern0pt}isomorphic{\isacharunderscore}{\kern0pt}def\ powerset{\isacharunderscore}{\kern0pt}def\isanewline
\ \ \isacommand{by}\isamarkupfalse%
\ {\isacharparenleft}{\kern0pt}metis\ epi{\isacharunderscore}{\kern0pt}is{\isacharunderscore}{\kern0pt}surj\ injective{\isacharunderscore}{\kern0pt}imp{\isacharunderscore}{\kern0pt}monomorphism\ iso{\isacharunderscore}{\kern0pt}imp{\isacharunderscore}{\kern0pt}epi{\isacharunderscore}{\kern0pt}and{\isacharunderscore}{\kern0pt}monic{\isacharparenright}{\kern0pt}%
\endisatagproof
{\isafoldproof}%
%
\isadelimproof
\isanewline
%
\endisadelimproof
\isanewline
\isacommand{corollary}\isamarkupfalse%
\ Generalized{\isacharunderscore}{\kern0pt}Cantors{\isacharunderscore}{\kern0pt}Positive{\isacharunderscore}{\kern0pt}Theorem{\isacharcolon}{\kern0pt}\isanewline
\ \ \isakeyword{assumes}\ {\isachardoublequoteopen}{\isasymnot}\ terminal{\isacharunderscore}{\kern0pt}object\ Y{\isachardoublequoteclose}\isanewline
\ \ \isakeyword{assumes}\ {\isachardoublequoteopen}{\isasymnot}\ initial{\isacharunderscore}{\kern0pt}object\ Y{\isachardoublequoteclose}\isanewline
\ \ \isakeyword{shows}\ {\isachardoublequoteopen}X\ \ {\isasymle}\isactrlsub c\ Y\isactrlbsup X\isactrlesup {\isachardoublequoteclose}\isanewline
%
\isadelimproof
%
\endisadelimproof
%
\isatagproof
\isacommand{proof}\isamarkupfalse%
\ {\isacharminus}{\kern0pt}\ \isanewline
\ \ \isacommand{have}\isamarkupfalse%
\ {\isachardoublequoteopen}{\isasymOmega}\ {\isasymle}\isactrlsub c\ Y{\isachardoublequoteclose}\isanewline
\ \ \ \ \isacommand{by}\isamarkupfalse%
\ {\isacharparenleft}{\kern0pt}simp\ add{\isacharcolon}{\kern0pt}\ assms\ non{\isacharunderscore}{\kern0pt}init{\isacharunderscore}{\kern0pt}non{\isacharunderscore}{\kern0pt}ter{\isacharunderscore}{\kern0pt}sets{\isacharparenright}{\kern0pt}\isanewline
\ \ \isacommand{then}\isamarkupfalse%
\ \isacommand{have}\isamarkupfalse%
\ fact{\isacharcolon}{\kern0pt}\ {\isachardoublequoteopen}{\isasymOmega}\isactrlbsup X\isactrlesup \ {\isasymle}\isactrlsub c\ \ Y\isactrlbsup X\isactrlesup {\isachardoublequoteclose}\isanewline
\ \ \ \ \isacommand{by}\isamarkupfalse%
\ {\isacharparenleft}{\kern0pt}simp\ add{\isacharcolon}{\kern0pt}\ exp{\isacharunderscore}{\kern0pt}preserves{\isacharunderscore}{\kern0pt}card{\isadigit{2}}{\isacharparenright}{\kern0pt}\isanewline
\ \ \isacommand{have}\isamarkupfalse%
\ {\isachardoublequoteopen}X\ {\isasymle}\isactrlsub c\ {\isasymOmega}\isactrlbsup X\isactrlesup {\isachardoublequoteclose}\isanewline
\ \ \ \ \isacommand{by}\isamarkupfalse%
\ {\isacharparenleft}{\kern0pt}meson\ Cantors{\isacharunderscore}{\kern0pt}Positive{\isacharunderscore}{\kern0pt}Theorem\ CollectI\ injective{\isacharunderscore}{\kern0pt}imp{\isacharunderscore}{\kern0pt}monomorphism\ is{\isacharunderscore}{\kern0pt}smaller{\isacharunderscore}{\kern0pt}than{\isacharunderscore}{\kern0pt}def{\isacharparenright}{\kern0pt}\isanewline
\ \ \isacommand{then}\isamarkupfalse%
\ \isacommand{show}\isamarkupfalse%
\ {\isacharquery}{\kern0pt}thesis\isanewline
\ \ \ \ \isacommand{using}\isamarkupfalse%
\ fact\ set{\isacharunderscore}{\kern0pt}card{\isacharunderscore}{\kern0pt}transitive\ \isacommand{by}\isamarkupfalse%
\ blast\isanewline
\isacommand{qed}\isamarkupfalse%
%
\endisatagproof
{\isafoldproof}%
%
\isadelimproof
\isanewline
%
\endisadelimproof
\isanewline
\isacommand{corollary}\isamarkupfalse%
\ Generalized{\isacharunderscore}{\kern0pt}Cantors{\isacharunderscore}{\kern0pt}Negative{\isacharunderscore}{\kern0pt}Theorem{\isacharcolon}{\kern0pt}\isanewline
\ \ \isakeyword{assumes}\ {\isachardoublequoteopen}{\isasymnot}\ initial{\isacharunderscore}{\kern0pt}object\ X{\isachardoublequoteclose}\isanewline
\ \ \isakeyword{assumes}\ {\isachardoublequoteopen}{\isasymnot}\ terminal{\isacharunderscore}{\kern0pt}object\ Y{\isachardoublequoteclose}\isanewline
\ \ \isakeyword{shows}\ {\isachardoublequoteopen}{\isasymnexists}\ s{\isachardot}{\kern0pt}\ s\ {\isacharcolon}{\kern0pt}\ X\ {\isasymrightarrow}\ Y\isactrlbsup X\isactrlesup \ {\isasymand}\ surjective\ s{\isachardoublequoteclose}\isanewline
%
\isadelimproof
%
\endisadelimproof
%
\isatagproof
\isacommand{proof}\isamarkupfalse%
{\isacharparenleft}{\kern0pt}rule\ ccontr{\isacharcomma}{\kern0pt}\ clarify{\isacharparenright}{\kern0pt}\ \isanewline
\ \ \isacommand{fix}\isamarkupfalse%
\ s\ \isanewline
\ \ \isacommand{assume}\isamarkupfalse%
\ s{\isacharunderscore}{\kern0pt}type{\isacharcolon}{\kern0pt}\ {\isachardoublequoteopen}s\ {\isacharcolon}{\kern0pt}\ X\ {\isasymrightarrow}\ Y\isactrlbsup X\isactrlesup {\isachardoublequoteclose}\isanewline
\ \ \isacommand{assume}\isamarkupfalse%
\ s{\isacharunderscore}{\kern0pt}surj{\isacharcolon}{\kern0pt}\ {\isachardoublequoteopen}surjective\ s{\isachardoublequoteclose}\isanewline
\ \ \isacommand{obtain}\isamarkupfalse%
\ m\ \isakeyword{where}\ m{\isacharunderscore}{\kern0pt}type{\isacharcolon}{\kern0pt}\ {\isachardoublequoteopen}m\ {\isacharcolon}{\kern0pt}\ Y\isactrlbsup X\isactrlesup \ {\isasymrightarrow}\ X{\isachardoublequoteclose}\ \isakeyword{and}\ m{\isacharunderscore}{\kern0pt}mono{\isacharcolon}{\kern0pt}\ {\isachardoublequoteopen}monomorphism{\isacharparenleft}{\kern0pt}m{\isacharparenright}{\kern0pt}{\isachardoublequoteclose}\isanewline
\ \ \ \ \isacommand{by}\isamarkupfalse%
\ {\isacharparenleft}{\kern0pt}meson\ epis{\isacharunderscore}{\kern0pt}give{\isacharunderscore}{\kern0pt}monos\ s{\isacharunderscore}{\kern0pt}surj\ s{\isacharunderscore}{\kern0pt}type\ surjective{\isacharunderscore}{\kern0pt}is{\isacharunderscore}{\kern0pt}epimorphism{\isacharparenright}{\kern0pt}\isanewline
\ \ \isacommand{have}\isamarkupfalse%
\ {\isachardoublequoteopen}nonempty\ X{\isachardoublequoteclose}\isanewline
\ \ \ \ \isacommand{using}\isamarkupfalse%
\ is{\isacharunderscore}{\kern0pt}empty{\isacharunderscore}{\kern0pt}def\ assms{\isacharparenleft}{\kern0pt}{\isadigit{1}}{\isacharparenright}{\kern0pt}\ iso{\isacharunderscore}{\kern0pt}empty{\isacharunderscore}{\kern0pt}initial\ no{\isacharunderscore}{\kern0pt}el{\isacharunderscore}{\kern0pt}iff{\isacharunderscore}{\kern0pt}iso{\isacharunderscore}{\kern0pt}empty\ nonempty{\isacharunderscore}{\kern0pt}def\ \isacommand{by}\isamarkupfalse%
\ blast\isanewline
\isanewline
\ \ \isacommand{then}\isamarkupfalse%
\ \isacommand{have}\isamarkupfalse%
\ nonempty{\isacharcolon}{\kern0pt}\ {\isachardoublequoteopen}nonempty\ {\isacharparenleft}{\kern0pt}{\isasymOmega}\isactrlbsup X\isactrlesup {\isacharparenright}{\kern0pt}{\isachardoublequoteclose}\isanewline
\ \ \ \ \isacommand{using}\isamarkupfalse%
\ nonempty{\isacharunderscore}{\kern0pt}def\ nonempty{\isacharunderscore}{\kern0pt}to{\isacharunderscore}{\kern0pt}nonempty\ true{\isacharunderscore}{\kern0pt}func{\isacharunderscore}{\kern0pt}type\ \isacommand{by}\isamarkupfalse%
\ blast\isanewline
\ \ \isacommand{show}\isamarkupfalse%
\ False\isanewline
\ \ \isacommand{proof}\isamarkupfalse%
{\isacharparenleft}{\kern0pt}cases\ {\isachardoublequoteopen}initial{\isacharunderscore}{\kern0pt}object\ Y{\isachardoublequoteclose}{\isacharparenright}{\kern0pt}\isanewline
\ \ \ \ \isacommand{assume}\isamarkupfalse%
\ {\isachardoublequoteopen}initial{\isacharunderscore}{\kern0pt}object\ Y{\isachardoublequoteclose}\isanewline
\ \ \ \ \isacommand{then}\isamarkupfalse%
\ \isacommand{have}\isamarkupfalse%
\ {\isachardoublequoteopen}Y\isactrlbsup X\isactrlesup \ {\isasymcong}\ {\isasymemptyset}{\isachardoublequoteclose}\isanewline
\ \ \ \ \ \ \isacommand{by}\isamarkupfalse%
\ {\isacharparenleft}{\kern0pt}simp\ add{\isacharcolon}{\kern0pt}\ {\isacartoucheopen}nonempty\ X{\isacartoucheclose}\ empty{\isacharunderscore}{\kern0pt}to{\isacharunderscore}{\kern0pt}nonempty\ initial{\isacharunderscore}{\kern0pt}iso{\isacharunderscore}{\kern0pt}empty\ no{\isacharunderscore}{\kern0pt}el{\isacharunderscore}{\kern0pt}iff{\isacharunderscore}{\kern0pt}iso{\isacharunderscore}{\kern0pt}empty{\isacharparenright}{\kern0pt}\ \ \ \ \ \ \isanewline
\ \ \ \ \isacommand{then}\isamarkupfalse%
\ \isacommand{show}\isamarkupfalse%
\ False\isanewline
\ \ \ \ \ \ \isacommand{by}\isamarkupfalse%
\ {\isacharparenleft}{\kern0pt}meson\ is{\isacharunderscore}{\kern0pt}empty{\isacharunderscore}{\kern0pt}def\ assms{\isacharparenleft}{\kern0pt}{\isadigit{1}}{\isacharparenright}{\kern0pt}\ comp{\isacharunderscore}{\kern0pt}type\ iso{\isacharunderscore}{\kern0pt}empty{\isacharunderscore}{\kern0pt}initial\ no{\isacharunderscore}{\kern0pt}el{\isacharunderscore}{\kern0pt}iff{\isacharunderscore}{\kern0pt}iso{\isacharunderscore}{\kern0pt}empty\ s{\isacharunderscore}{\kern0pt}type{\isacharparenright}{\kern0pt}\ \isanewline
\ \ \isacommand{next}\isamarkupfalse%
\isanewline
\ \ \ \ \isacommand{assume}\isamarkupfalse%
\ {\isachardoublequoteopen}{\isasymnot}\ initial{\isacharunderscore}{\kern0pt}object\ Y{\isachardoublequoteclose}\isanewline
\ \ \ \ \isacommand{then}\isamarkupfalse%
\ \isacommand{have}\isamarkupfalse%
\ {\isachardoublequoteopen}{\isasymOmega}\ {\isasymle}\isactrlsub c\ Y{\isachardoublequoteclose}\isanewline
\ \ \ \ \ \ \isacommand{by}\isamarkupfalse%
\ {\isacharparenleft}{\kern0pt}simp\ add{\isacharcolon}{\kern0pt}\ assms{\isacharparenleft}{\kern0pt}{\isadigit{2}}{\isacharparenright}{\kern0pt}\ non{\isacharunderscore}{\kern0pt}init{\isacharunderscore}{\kern0pt}non{\isacharunderscore}{\kern0pt}ter{\isacharunderscore}{\kern0pt}sets{\isacharparenright}{\kern0pt}\isanewline
\ \ \ \ \isacommand{then}\isamarkupfalse%
\ \isacommand{obtain}\isamarkupfalse%
\ n\ \isakeyword{where}\ n{\isacharunderscore}{\kern0pt}type{\isacharcolon}{\kern0pt}\ {\isachardoublequoteopen}n\ {\isacharcolon}{\kern0pt}\ {\isasymOmega}\isactrlbsup X\isactrlesup \ {\isasymrightarrow}\ Y\isactrlbsup X\isactrlesup {\isachardoublequoteclose}\ \isakeyword{and}\ n{\isacharunderscore}{\kern0pt}mono{\isacharcolon}{\kern0pt}\ {\isachardoublequoteopen}monomorphism{\isacharparenleft}{\kern0pt}n{\isacharparenright}{\kern0pt}{\isachardoublequoteclose}\isanewline
\ \ \ \ \ \ \isacommand{by}\isamarkupfalse%
\ {\isacharparenleft}{\kern0pt}meson\ exp{\isacharunderscore}{\kern0pt}preserves{\isacharunderscore}{\kern0pt}card{\isadigit{2}}\ is{\isacharunderscore}{\kern0pt}smaller{\isacharunderscore}{\kern0pt}than{\isacharunderscore}{\kern0pt}def{\isacharparenright}{\kern0pt}\isanewline
\ \ \ \ \isacommand{then}\isamarkupfalse%
\ \isacommand{have}\isamarkupfalse%
\ mn{\isacharunderscore}{\kern0pt}type{\isacharcolon}{\kern0pt}\ {\isachardoublequoteopen}m\ {\isasymcirc}\isactrlsub c\ n\ {\isacharcolon}{\kern0pt}\ \ {\isasymOmega}\isactrlbsup X\isactrlesup \ {\isasymrightarrow}\ X{\isachardoublequoteclose}\isanewline
\ \ \ \ \ \ \isacommand{by}\isamarkupfalse%
\ {\isacharparenleft}{\kern0pt}meson\ comp{\isacharunderscore}{\kern0pt}type\ m{\isacharunderscore}{\kern0pt}type{\isacharparenright}{\kern0pt}\isanewline
\ \ \ \ \isacommand{have}\isamarkupfalse%
\ mn{\isacharunderscore}{\kern0pt}mono{\isacharcolon}{\kern0pt}\ {\isachardoublequoteopen}monomorphism{\isacharparenleft}{\kern0pt}m\ {\isasymcirc}\isactrlsub c\ n{\isacharparenright}{\kern0pt}{\isachardoublequoteclose}\isanewline
\ \ \ \ \ \ \isacommand{using}\isamarkupfalse%
\ cfunc{\isacharunderscore}{\kern0pt}type{\isacharunderscore}{\kern0pt}def\ composition{\isacharunderscore}{\kern0pt}of{\isacharunderscore}{\kern0pt}monic{\isacharunderscore}{\kern0pt}pair{\isacharunderscore}{\kern0pt}is{\isacharunderscore}{\kern0pt}monic\ m{\isacharunderscore}{\kern0pt}mono\ m{\isacharunderscore}{\kern0pt}type\ n{\isacharunderscore}{\kern0pt}mono\ n{\isacharunderscore}{\kern0pt}type\ \isacommand{by}\isamarkupfalse%
\ presburger\isanewline
\ \ \ \ \isacommand{then}\isamarkupfalse%
\ \isacommand{have}\isamarkupfalse%
\ {\isachardoublequoteopen}{\isasymexists}g{\isachardot}{\kern0pt}\ g{\isacharcolon}{\kern0pt}\ X\ \ {\isasymrightarrow}\ {\isasymOmega}\isactrlbsup X\isactrlesup \ {\isasymand}\ epimorphism{\isacharparenleft}{\kern0pt}g{\isacharparenright}{\kern0pt}\ {\isasymand}\ g\ {\isasymcirc}\isactrlsub c\ {\isacharparenleft}{\kern0pt}m\ {\isasymcirc}\isactrlsub c\ n{\isacharparenright}{\kern0pt}\ {\isacharequal}{\kern0pt}\ id\ {\isacharparenleft}{\kern0pt}{\isasymOmega}\isactrlbsup X\isactrlesup {\isacharparenright}{\kern0pt}{\isachardoublequoteclose}\isanewline
\ \ \ \ \ \ \isacommand{by}\isamarkupfalse%
\ {\isacharparenleft}{\kern0pt}simp\ add{\isacharcolon}{\kern0pt}\ mn{\isacharunderscore}{\kern0pt}type\ monos{\isacharunderscore}{\kern0pt}give{\isacharunderscore}{\kern0pt}epis\ nonempty{\isacharparenright}{\kern0pt}\isanewline
\ \ \ \ \isacommand{then}\isamarkupfalse%
\ \isacommand{show}\isamarkupfalse%
\ False\isanewline
\ \ \ \ \ \ \isacommand{by}\isamarkupfalse%
\ {\isacharparenleft}{\kern0pt}metis\ Cantors{\isacharunderscore}{\kern0pt}Negative{\isacharunderscore}{\kern0pt}Theorem\ epi{\isacharunderscore}{\kern0pt}is{\isacharunderscore}{\kern0pt}surj\ powerset{\isacharunderscore}{\kern0pt}def{\isacharparenright}{\kern0pt}\isanewline
\ \ \isacommand{qed}\isamarkupfalse%
\isanewline
\isacommand{qed}\isamarkupfalse%
%
\endisatagproof
{\isafoldproof}%
%
\isadelimproof
\isanewline
%
\endisadelimproof
%
\isadelimtheory
\isanewline
%
\endisadelimtheory
%
\isatagtheory
\isacommand{end}\isamarkupfalse%
%
\endisatagtheory
{\isafoldtheory}%
%
\isadelimtheory
%
\endisadelimtheory
%
\end{isabellebody}%
\endinput
%:%file=~/ETCS/Category_Set/Fixed_Points.thy%:%
%:%11=1%:%
%:%27=3%:%
%:%28=3%:%
%:%29=4%:%
%:%30=5%:%
%:%39=7%:%
%:%41=8%:%
%:%42=8%:%
%:%43=9%:%
%:%44=10%:%
%:%45=10%:%
%:%46=11%:%
%:%47=12%:%
%:%48=12%:%
%:%49=13%:%
%:%50=14%:%
%:%51=15%:%
%:%52=15%:%
%:%53=16%:%
%:%54=17%:%
%:%57=18%:%
%:%61=18%:%
%:%62=18%:%
%:%63=18%:%
%:%64=18%:%
%:%73=20%:%
%:%75=21%:%
%:%76=21%:%
%:%77=22%:%
%:%78=23%:%
%:%79=24%:%
%:%82=25%:%
%:%86=25%:%
%:%87=25%:%
%:%88=26%:%
%:%89=26%:%
%:%90=27%:%
%:%91=27%:%
%:%92=28%:%
%:%93=28%:%
%:%94=29%:%
%:%95=29%:%
%:%96=30%:%
%:%97=30%:%
%:%98=31%:%
%:%99=31%:%
%:%100=31%:%
%:%101=32%:%
%:%102=32%:%
%:%103=33%:%
%:%104=33%:%
%:%105=34%:%
%:%106=34%:%
%:%107=35%:%
%:%108=35%:%
%:%109=35%:%
%:%110=36%:%
%:%111=36%:%
%:%112=36%:%
%:%113=37%:%
%:%114=37%:%
%:%115=38%:%
%:%116=38%:%
%:%117=38%:%
%:%118=39%:%
%:%119=39%:%
%:%120=40%:%
%:%121=40%:%
%:%122=41%:%
%:%123=41%:%
%:%124=42%:%
%:%125=42%:%
%:%126=43%:%
%:%127=43%:%
%:%128=44%:%
%:%129=44%:%
%:%130=45%:%
%:%131=45%:%
%:%132=45%:%
%:%133=46%:%
%:%134=46%:%
%:%135=46%:%
%:%136=47%:%
%:%137=47%:%
%:%138=47%:%
%:%139=48%:%
%:%140=48%:%
%:%141=49%:%
%:%142=49%:%
%:%143=49%:%
%:%144=50%:%
%:%145=50%:%
%:%146=50%:%
%:%147=51%:%
%:%148=51%:%
%:%149=51%:%
%:%150=52%:%
%:%151=52%:%
%:%152=52%:%
%:%153=53%:%
%:%154=53%:%
%:%155=53%:%
%:%156=54%:%
%:%157=54%:%
%:%158=55%:%
%:%159=55%:%
%:%160=55%:%
%:%161=55%:%
%:%162=56%:%
%:%163=56%:%
%:%164=57%:%
%:%165=57%:%
%:%166=57%:%
%:%167=58%:%
%:%168=58%:%
%:%169=59%:%
%:%170=59%:%
%:%171=59%:%
%:%172=60%:%
%:%173=60%:%
%:%174=60%:%
%:%175=61%:%
%:%176=61%:%
%:%177=61%:%
%:%178=62%:%
%:%179=62%:%
%:%180=62%:%
%:%181=63%:%
%:%182=63%:%
%:%183=63%:%
%:%184=64%:%
%:%185=64%:%
%:%186=64%:%
%:%187=65%:%
%:%197=67%:%
%:%199=68%:%
%:%200=68%:%
%:%201=69%:%
%:%208=70%:%
%:%209=70%:%
%:%210=71%:%
%:%211=71%:%
%:%212=72%:%
%:%213=72%:%
%:%214=73%:%
%:%215=73%:%
%:%216=74%:%
%:%217=74%:%
%:%218=74%:%
%:%219=75%:%
%:%220=75%:%
%:%221=75%:%
%:%222=76%:%
%:%223=76%:%
%:%224=77%:%
%:%225=77%:%
%:%226=78%:%
%:%227=78%:%
%:%228=79%:%
%:%229=79%:%
%:%230=80%:%
%:%231=80%:%
%:%232=81%:%
%:%233=81%:%
%:%234=82%:%
%:%235=82%:%
%:%236=82%:%
%:%237=83%:%
%:%238=83%:%
%:%239=84%:%
%:%240=84%:%
%:%241=84%:%
%:%242=85%:%
%:%243=85%:%
%:%244=85%:%
%:%245=86%:%
%:%246=86%:%
%:%247=87%:%
%:%248=87%:%
%:%249=88%:%
%:%250=88%:%
%:%251=88%:%
%:%252=89%:%
%:%262=91%:%
%:%264=92%:%
%:%265=92%:%
%:%266=93%:%
%:%273=94%:%
%:%274=94%:%
%:%275=95%:%
%:%276=95%:%
%:%277=96%:%
%:%278=96%:%
%:%279=97%:%
%:%280=97%:%
%:%281=98%:%
%:%282=98%:%
%:%283=99%:%
%:%284=99%:%
%:%285=100%:%
%:%286=100%:%
%:%287=101%:%
%:%288=101%:%
%:%289=101%:%
%:%290=101%:%
%:%291=102%:%
%:%292=102%:%
%:%293=102%:%
%:%294=103%:%
%:%295=103%:%
%:%296=103%:%
%:%297=103%:%
%:%298=104%:%
%:%299=104%:%
%:%300=104%:%
%:%301=105%:%
%:%302=105%:%
%:%303=106%:%
%:%304=106%:%
%:%305=107%:%
%:%306=107%:%
%:%307=108%:%
%:%308=108%:%
%:%309=109%:%
%:%310=109%:%
%:%311=109%:%
%:%312=110%:%
%:%313=110%:%
%:%314=110%:%
%:%315=111%:%
%:%316=111%:%
%:%317=112%:%
%:%318=112%:%
%:%319=112%:%
%:%320=113%:%
%:%321=113%:%
%:%322=113%:%
%:%323=114%:%
%:%324=114%:%
%:%325=114%:%
%:%326=115%:%
%:%327=115%:%
%:%328=116%:%
%:%329=116%:%
%:%330=116%:%
%:%331=117%:%
%:%332=117%:%
%:%333=118%:%
%:%334=118%:%
%:%335=118%:%
%:%336=119%:%
%:%337=119%:%
%:%338=119%:%
%:%339=120%:%
%:%340=120%:%
%:%341=120%:%
%:%342=121%:%
%:%343=121%:%
%:%344=121%:%
%:%345=122%:%
%:%346=122%:%
%:%347=122%:%
%:%348=122%:%
%:%349=123%:%
%:%350=123%:%
%:%351=124%:%
%:%352=124%:%
%:%353=124%:%
%:%354=125%:%
%:%355=125%:%
%:%356=126%:%
%:%357=126%:%
%:%358=127%:%
%:%359=127%:%
%:%360=127%:%
%:%361=128%:%
%:%362=128%:%
%:%363=128%:%
%:%364=129%:%
%:%374=131%:%
%:%376=132%:%
%:%377=132%:%
%:%378=133%:%
%:%381=134%:%
%:%385=134%:%
%:%386=134%:%
%:%387=135%:%
%:%388=135%:%
%:%389=136%:%
%:%390=136%:%
%:%395=136%:%
%:%398=137%:%
%:%399=138%:%
%:%400=138%:%
%:%401=139%:%
%:%402=140%:%
%:%403=141%:%
%:%410=142%:%
%:%411=142%:%
%:%412=143%:%
%:%413=143%:%
%:%414=144%:%
%:%415=144%:%
%:%416=145%:%
%:%417=145%:%
%:%418=145%:%
%:%419=146%:%
%:%420=146%:%
%:%421=147%:%
%:%422=147%:%
%:%423=148%:%
%:%424=148%:%
%:%425=149%:%
%:%426=149%:%
%:%427=149%:%
%:%428=150%:%
%:%429=150%:%
%:%430=150%:%
%:%431=151%:%
%:%437=151%:%
%:%440=152%:%
%:%441=153%:%
%:%442=153%:%
%:%443=154%:%
%:%444=155%:%
%:%445=156%:%
%:%452=157%:%
%:%453=157%:%
%:%454=158%:%
%:%455=158%:%
%:%456=159%:%
%:%457=159%:%
%:%458=160%:%
%:%459=160%:%
%:%460=161%:%
%:%461=161%:%
%:%462=162%:%
%:%463=162%:%
%:%464=163%:%
%:%465=163%:%
%:%466=164%:%
%:%467=164%:%
%:%468=164%:%
%:%469=165%:%
%:%470=166%:%
%:%471=166%:%
%:%472=166%:%
%:%473=167%:%
%:%474=167%:%
%:%475=167%:%
%:%476=168%:%
%:%477=168%:%
%:%478=169%:%
%:%479=169%:%
%:%480=170%:%
%:%481=170%:%
%:%482=171%:%
%:%483=171%:%
%:%484=171%:%
%:%485=172%:%
%:%486=172%:%
%:%487=173%:%
%:%488=173%:%
%:%489=173%:%
%:%490=174%:%
%:%491=174%:%
%:%492=175%:%
%:%493=175%:%
%:%494=176%:%
%:%495=176%:%
%:%496=177%:%
%:%497=177%:%
%:%498=177%:%
%:%499=178%:%
%:%500=178%:%
%:%501=179%:%
%:%502=179%:%
%:%503=179%:%
%:%504=180%:%
%:%505=180%:%
%:%506=181%:%
%:%507=181%:%
%:%508=181%:%
%:%509=182%:%
%:%510=182%:%
%:%511=183%:%
%:%512=183%:%
%:%513=184%:%
%:%514=184%:%
%:%515=184%:%
%:%516=185%:%
%:%517=185%:%
%:%518=185%:%
%:%519=186%:%
%:%520=186%:%
%:%521=187%:%
%:%522=187%:%
%:%523=187%:%
%:%524=188%:%
%:%525=188%:%
%:%526=189%:%
%:%527=189%:%
%:%528=190%:%
%:%534=190%:%
%:%539=191%:%
%:%544=192%:%

%
\begin{isabellebody}%
\setisabellecontext{ETCS}%
%
\isadelimtheory
%
\endisadelimtheory
%
\isatagtheory
\isacommand{theory}\isamarkupfalse%
\ ETCS\isanewline
\ \ \isakeyword{imports}\ Axiom{\isacharunderscore}{\kern0pt}Of{\isacharunderscore}{\kern0pt}Choice\ Nats\ Quant{\isacharunderscore}{\kern0pt}Logic\ Countable\ Fixed{\isacharunderscore}{\kern0pt}Points\isanewline
\isakeyword{begin}\isanewline
\isacommand{end}\isamarkupfalse%
%
\endisatagtheory
{\isafoldtheory}%
%
\isadelimtheory
%
\endisadelimtheory
%
\end{isabellebody}%
\endinput
%:%file=~/ETCS/HOL-ETCS/ETCS.thy%:%
%:%10=1%:%
%:%11=1%:%
%:%12=2%:%
%:%13=3%:%
%:%14=4%:%



\bibliographystyle{abbrv}
\bibliography{root}

\end{document}
